\documentclass{amsart}


%\documentclass[11pt]{article}
%\usepackage{fullpage}
\usepackage{amsmath,amscd}
\usepackage{amsthm}
\usepackage{amssymb}
\usepackage{setspace}
\usepackage{fancyhdr}
%\usepackage{tikz}
%\usepackage{tikz-cd}
%\usetikzlibrary{matrix,arrows,decorations.pathmorphing}
\usepackage{stmaryrd}
\usepackage{bbm}
\input xy
\xyoption{all}
\usepackage{color}


%%%%%%%%%%%%%%%%
\usepackage[normalem]{ulem}
%%%%%%%%%%%%%%%%


%EXTRACTING
 \usepackage{./extract}
 % ENDEXTRACTING
\usepackage{linelo}
\linenumbers
 
\newtheorem{thm}{{Theorem}}[section]
\newtheorem{cor}[thm]{{Corollary}}
\newtheorem{lemma}[thm]{{Lemma}}
\newtheorem{prop}[thm]{Proposition}
\newtheorem{conj}[thm]{{Conjecture}}
\newtheorem{example}[thm]{{Example}}

\theoremstyle{definition}
\newtheorem{defn}[thm]{Definition}
\newtheorem{rk}[thm]{{Remark}}
\newtheorem{notation}[thm]{{Notation}}


\renewcommand{\phi}{\varphi}
\newcommand{\mtrx}[4]{\left(\begin{array}{cc} #1 & #2 \\ #3 & #4 \\ \end{array}\right)}

%%%%%%%%%%%

\newcommand{\ide}{\mathbf{1}}
\newcommand{\bw}{\mathbf{w}}
\newcommand{\F}{\mathbb F}
\newcommand{\N}{\mathbb N}
\newcommand{\Q}{\mathbb Q}
\newcommand{\R}{\mathbf R}
\newcommand{\Z}{\mathbb Z}

%%%%%%%%%%
\newcommand{\NN}{\mathbb N}
\newcommand{\QQ}{\mathbb{Q}}
\newcommand{\ZZ}{\mathbb{Z}}
\newcommand{\VV}{\mathbb{V}}
\newcommand{\DD}{\mathbb{D}}
\newcommand{\BB}{\mathbb{B}}

\newcommand{\Zp}{\mathbb{Z}_p}
\newcommand{\Qp}{\mathbb{Q}_p}
\newcommand{\Cp}{\mathbb{C}_p}
\newcommand{\Fp}{\mathbb{F}_p}
\newcommand{\RR}{\mathbb{R}}
\newcommand{\FF}{\mathbb{F}}
\newcommand{\MM}{\mathbb{M}}
\newcommand{\LL}{\mathbb{L}}
\newcommand{\CC}{\mathbb{C}}
\newcommand{\HH}{\mathbb{H}}
\newcommand{\TT}{\mathbb{T}}
\newcommand{\GG}{\mathbb{G}}
\newcommand{\PP}{\mathbb{P}}

%%%%%%%%%%%%%%
\newcommand{\cR}{\mathcal{R}}
\newcommand{\cH}{\mathcal{H}}
\newcommand{\cW}{\mathcal{W}}
\newcommand{\cA}{\mathcal{A}}
\newcommand{\cZ}{\mathcal{Z}}
\newcommand{\cD}{\mathcal{D}}
\newcommand{\cL}{\mathcal{L}}
\newcommand{\cF}{\mathcal{F}}
\newcommand{\cO}{\mathcal{O}}
\newcommand{\cG}{\mathcal{G}}
\newcommand{\cM}{\mathcal{M}}
\newcommand{\cX}{\mathcal{X}}
\newcommand{\cP}{\mathcal{P}}


\newcommand{\cY}{\mathcal{Y}}

\newcommand{\cT}{\mathcal{T}}
\newcommand{\cE}{\mathcal{E}}
\newcommand{\cN}{\mathcal{N}}
\newcommand{\cU}{\mathcal{U}}
\newcommand{\cV}{\mathcal{V}}
\newcommand{\cI}{\mathcal{I}} 
\newcommand{\FL}{\mathcal{FL}}

%%%%%%%%%%%%%%
\newcommand{\fA}{\mathfrak{A}}
\newcommand{\fa}{\mathfrak{a}}
\newcommand{\fp}{\mathfrak{p}}
\newcommand{\fq}{\mathfrak{q}}
\newcommand{\fr}{\mathfrak{r}}
\newcommand{\fm}{\mathfrak{m}}
\newcommand{\fM}{\mathfrak{M}}
\newcommand{\fb}{\mathfrak{b}}
\newcommand{\fX}{\mathfrak{X}}
\newcommand{\fT}{\mathfrak{T}}
\newcommand{\fU}{\mathfrak{U}}
\newcommand{\fV}{\mathfrak{V}}
\newcommand{\fY}{\mathfrak{Y}}
\newcommand{\fc}{\mathfrak{c}}
\newcommand{\fn}{\mathfrak{n}}
\newcommand{\fC}{\mathfrak{C}}
\newcommand{\m}{\mathfrak{m}}


%%%%%%%%%%%%%%


\newcommand{\vp}{\varphi}
\newcommand{\ka}{\kappa}
\newcommand{\ra}{\rightarrow}
\newcommand{\lra}{\longrightarrow}
\newcommand{\thra}{\twoheadrightarrow}
\newcommand{\hra}{\hookrightarrow}
\newcommand{\hooklongrightarrow}{\lhook\joinrel\longrightarrow}
\newcommand{\twoheadlongrightarrow}{\relbar\joinrel\twoheadrightarrow}
\newcommand{\ur}{\mathrm{ur}}


%%%%%%%%%%%%%%
\DeclareMathOperator{\Spec}{Spec}
\DeclareMathOperator{\Sp}{Sp}
\DeclareMathOperator{\Sh}{Sh}

\DeclareMathOperator{\Spa}{Spa}
%\DeclareMathOperator{\ker}{ker}
\DeclareMathOperator{\Coker}{Coker}
\DeclareMathOperator{\Spf}{Spf}
\DeclareMathOperator{\End}{End}
\DeclareMathOperator{\Hom}{Hom}
\DeclareMathOperator{\Aut}{Aut}
\DeclareMathOperator{\Gal}{Gal}
\DeclareMathOperator{\Lie}{Lie}
\DeclareMathOperator{\Res}{Res}
\DeclareMathOperator{\Proj}{Proj}
\DeclareMathOperator{\Supp}{Supp}
\DeclareMathOperator{\Fil}{Fil}



%%%%%%%%%%
\newcommand{\bK}{\overline{K}}
\newcommand{\bk}{\overline{k}}

\newcommand{\bFpt}{\overline{\F}_p^{\times}}
\newcommand{\bFp}{\overline{\F}_p}
\newcommand{\bQp}{\overline{\Q}_p}
\newcommand{\bZp}{\overline{\Z}_p}
\newcommand{\xto}[1][]{\xrightarrow{#1}}
\newcommand{\simto}{%\stackrel{\sim}{\to}}%isomorphic to
\xto[\sim]} %isomorphic to;


\def\red#1{\textcolor{red}{#1}}
\def\blue#1{\textcolor{blue}{#1}}
\def\green#1{\textcolor{green}{#1}}




%%%%%%%%%%
\newcommand{\BBdrp}{\mathbb{B}_{\mathrm{dR}}^+}
\newcommand{\BBcrp}{\mathbb{B}_{\mathrm{cris}}^+}
\newcommand{\BBst}{\mathbb{B}_{\mathrm{st}}}
\newcommand{\BBstp}{\mathbb{B}_{\mathrm{st}}^+}
\newcommand{\BBlog}{\mathbb{B}_{\mathrm{log}}}
\newcommand{\BBlogp}{\mathbb{B}_{\mathrm{log}}^+}
\newcommand{\BBcr}{\mathbb{B}_{\mathrm{cris}}}
\newcommand{\AAcr}{\mathbb{A}_{\mathrm{cris}}}
\newcommand{\AAinf}{\mathbb{A}_{\mathrm{inf}}}
\newcommand{\BBinf}{\mathbb{B}_{\mathrm{inf}}}

\newcommand{\BBdr}{\mathbb{B}_{\mathrm{dR}}}
\newcommand{\OBdrp}{\mathcal{O}\mathbb{B}_{\mathrm{dR}}^+}
\newcommand{\OBdr}{\mathcal{O}\mathbb{B}_{\mathrm{dR}}}
\newcommand{\OBcr}{\mathcal{O}\mathbb{B}_{\mathrm{cris}}}
\newcommand{\OBcrp}{\mathcal{O}\mathbb{B}_{\mathrm{cris}}^+}

\newcommand{\proet}{\mathrm{pro\acute{e}t}}

\newcommand{\proket}{\mathrm{prok\acute{e}t}}
\newcommand{\fet}{\mathrm{f\acute{e}t}}
\newcommand{\ket}{\mathrm{k\acute{e}t}}
\newcommand{\fket}{\mathrm{fk\acute{e}t}}
\newcommand{\profket}{\mathrm{profk\acute{e}t}}

\newcommand{\profet}{\mathrm{prof\acute{e}t}}
\newcommand{\et}{\mathrm{\acute{e}t}}
\newcommand{\OX}{\mathcal{O}_X}
\newcommand{\OXp}{\mathcal{O}_X^+}
\newcommand{\hOX}{\widehat{\mathcal{O}}_X}
\newcommand{\ho}{\widehat{\otimes}}
\newcommand{\hOXp}{\widehat{\mathcal{O}}_X^+}
\newcommand{\pro}{\mathrm{pro}}
\newcommand{\dR}{\mathrm{dR}}
\newcommand{\cris}{\mathrm{cris}}
\newcommand{\ad}{\mathrm{ad}}
\newcommand{\an}{\mathrm{an}}
\newcommand{\cont}{\mathrm{cont}}
\newcommand{\HT}{\mathrm{HT}}

\numberwithin{equation}{subsection}
 
 \fancyhead[LE,RO]{\thepage}
 \raggedbottom

 \def\red#1{\textcolor{red}{#1}}

 
 \begin{document}
 \linenumbers

\author{Fucheng Tan, Jilong Tong}
\title{Crystalline comparison isomorphisms in $p$-adic Hodge theory: the absolutely unramified case}



\date{}

\maketitle


\markright{Crystalline comparison isomorphism in $p$-adic Hodge theory}


\thispagestyle{plain}


\begin{abstract}

We construct the crystalline comparison isomorphisms for proper smooth formal schemes over an absolutely unramified base. Such isomorphisms hold for \'etale cohomology with nontrivial coefficients, as well as  in the relative setting, i.e. for proper smooth morphisms of smooth formal schemes. The proof is formulated in terms of the pro-\'etale topos introduced by Scholze, and uses his primitive comparison theorem for the structure sheaf on the pro-\'etale site. Moreover, we need to prove the Poincar\'e lemma for crystalline period sheaves, for which we adapt the idea of Andreatta and Iovita. Another ingredient  for the proof is the geometric acyclicity of crystalline period sheaves, whose computation is due to Andreatta and Brinon.
 
% We also provide the crystalline comparison isomorphism for affine curves. 
\end{abstract}


\tableofcontents

\section*{Notation}        


\begin{itemize}

\item Let $p$ be a prime number.

% and

%\item $\varepsilon=(\varepsilon^{(n)})_{n\in \N}$ a choice of $p$-power roots of unity, such that $\varepsilon^{(0)}=1, \varepsilon^{(1)}\neq 1$, and $\varepsilon^{(n+1)}=(\varepsilon^{(n)})^p$ for $n\geq 1$.

\item Let $k$ be a $p$-adic field, i.e., a discretely valued complete nonarchimedean extension of $\Qp$, whose residue field $\kappa$ is a perfect field of characteristic $p$.  (We often assume $k$ to be  absolutely unramified in this paper.)  

\item Let  $\bk$ be a fixed algebraic closure of $k$.  Set $\Cp:=\widehat{\bk}$ the $p$-adic completion of $\bk$. The $p$-adic valuation $v$ on $\Cp$ is normalized so that $v(p)=1$. Write the absolute Galois group $\Gal(\bk/k)$ as $G_k$.


 
% \item Let $    \chi_{\rm cyc}: G_{k}\ra \Zp^{\time     \chi_{\rm cyc}ote the cyclotomic character. 

\item For a topological  ring $A$ which is complete with respect to $p$-adic topology, let $A\langle T_1,\ldots, T_d\rangle$ be the PD-envelope of the polynomial ring $A[T_1,\ldots, T_d]$ with respect to the ideal $(T_1,\ldots, T_d)\subset A[T_1,\ldots, T_d]$ (with the requirement that the PD-structure be compatible with the one on the ideal $(p)$) and then let $A\{\langle T_1,\ldots, T_d\rangle\}$ be its $p$-adic completion. 

\item We use the symbol $\simeq$ to denote canonical isomorphisms and sometimes use $\approx$ for almost isomorphisms (often with respect to the maximal ideal of $\cO_{\Cp}$).

 \end{itemize}

\section{Introduction}

Let $k$ be a $p$-adic field, that is a discretely valued complete nonarchimedean extension of $\Qp$ with a perfect  residue field   of characteristic $p$, which is absolutely unramified. 
Consider a rigid analytic variety over $k$, or more generally  an adic space $X$ over $\Spa(k,\cO_k)$ which admits a proper smooth  formal model $\cX$ over $\Spf{\cO_k}$, whose special fiber is denoted by $\cX_0$. Let $\LL$ be a lisse $\Z_p$-sheaf on $X_{\et}$. 
On one hand, we have the $p$-adic \'etale cohomology $H^i(X_{\bk}, \LL)$ which is a finitely generated $\Zp$-module carrying a continuous $G_k=\Gal(\bk/k)$-action. On the other hand,  one may consider the crystalline cohomology $H_{\cris}^i(\cX_0/\cO_k, \cE)$ with the coefficient $\cE$ being a filtered (convergent) $F$-isocrystal on $\cX_0/\cO_k$. At least in the case that $X$ comes from a scheme and the coefficients $\LL$ and $\cE$ are trivial, it was Grothendieck's problem of \emph{mysterious functor} to find a comparison between the two cohomology theories. This problem was later formulated as the \emph{crystalline conjecture} by Fontaine \cite{Fon82}. 

In the past decades, the crystalline conjecture was proved in various generalities,  by  Fontaine-Messing, Kato, Tsuji, Niziol,  Faltings, Andreatta-Iovita,  Beilinson and Bhatt. Among them, the first proof for the whole conjecture was given by Faltings \cite{Fal}. Along this line, Andreatta-Iovita introduced   the Poincar\'e lemma for the crystalline period sheaf $\BBcr$ on the Faltings site, a sheaf-theoretic   generalization of  Fontaine's period ring $B_{\cris}	$. Both the approach of Fontaine-Messing and that of Faltings-Andreatta-Iovita use an intermediate topology, namely the syntomic topology and the Faltings topology, respectively. The approach of Faltings-Andreatta-Iovita, however, has the advantage that it works for nontrivial coefficients $\LL$ and $\cE$.   

More recently, Scholze \cite{Sch13} introduced the pro-\'etale site $X_{\proet}$, which allows him to construct the de Rham comparison isomorphism for any proper smooth adic space over  a discretely valued complete nonarchimedean field over  $\Qp$, with coefficients being lisse $\Z_p$-sheaves on $X_{\proet}$. (The notion of lisse $\Zp$-sheaf on $X_{\et}$ and that   on $X_{\proet}$ are equivalent.) Moreover, his approach is direct and flexible enough to attack the relative version of the de Rham comparison isomorphism, i.e. the comparison for a proper smooth morphism between two smooth adic spaces. %We will indicate the similarity and difference between Schozle's method and that of Faltings-Andreatta-Iovita below. 


It seems  that to deal with nontrivial coefficients in a comparison isomorphism, one is forced to work over analytic bases. For the generality and some technical advantages provided by the pro-\'etale topology, we adapt Scholze's approach to give a proof of the crystalline conjecture for    proper smooth formal schemes over $\Spf{\cO_k}$, with nontrivial coefficients, in both  absolute and relative settings. We note that it is not hard to prove in our setup the crystalline conjecture for  formal schemes over arbitrary complete discretely valued rings.  Furthermore, in a sequel paper we shall construct crystalline comparison isomorphisms for arbitrary varieties over $p$-adic fields.

\smallskip

Let us explain our construction of crystalline comparison isomorphism (in the absolutely unramified case) in more details.
First of all, Scholze is able to prove the finiteness of the \'etale cohomology of  a proper smooth  adic space $X$ over $K=\widehat{\bk}$ with coefficient $\LL'$ being an $\FF_p$-local system. Consequently, he shows the following ``primitive comparison", an almost  (with respect to the maximal ideal of $\cO_{K}$) isomorphism
\[H^i(X_{\bk,\et}, \LL')\otimes_{\Fp} \cO_K/p\stackrel{\approx}{\lra} H^i(X_{\bk,\et}, \LL'\otimes_{\Fp} \cO_{X}^+/p).\] 
%which inductively gives the isomorphism \[H^i(X_{\bk,\et}, \LL)\otimes_{\Zp} B_{\rm dR}^+ \simto H^i(X_{\bk,\proet}, \LL\otimes \mathbb{B}_{\rm dR}^+)\] for $\LL$ a lisse $\Z_p$-sheaf on $X$, where $B_{\rm dR}^+$  is Fontaine's absolute period ring and $\BBdrp$ the de Rham period sheaf on the pro-e\'tale site $X_{\proet}$. 
With some more efforts, one can produce the primitive comparison isomorphism in the crystalline case: 
\begin{thm}[See Theorem \ref{inout}] For $\LL$ a lisse $\Zp$-sheaf on $X_{\et}$, we have a canonical isomorphism of $B_{\cris}$-modules
\begin{equation}H^i(X_{\bk,\et}, \LL)\otimes_{\Zp} B_{\cris} \simto H^i(X_{\bk,\proet}, \LL\otimes \mathbb{B}_{\cris}).\end{equation}
compatible with $G_k$-action, filtration, and Frobenius. 
\end{thm}
It seems to us that such a result alone may have interesting arithmetic applications, since it works for any lisse $\Zp$-sheaves, without the crystalline condition needed for comparison theorems. 

 Following Faltings,  we say a lisse $\ZZ_p$-sheaf $\LL$  on  the pro-\'etale site $X_{\proet}$  is \emph{crystalline} if there exists a filtered $F$-isocrystal $\cE$ on $\cX_0/\cO_k$ together with an isomorphism of $\cO\BBcr$-modules 
\begin{equation}\label{associated1}
\cE\otimes_{\cO_X^{\ur}} \cO\BBcr\simeq \LL\otimes_{\ZZ_p}\cO\BBcr
\end{equation}
which is compatible with connection, filtration and Frobenius. Here, $\cO_X^{\ur}$ is the pullback to $X_{\proet}$  of $\cO_{\cX_{\et}}$ and $\cO\BBcr$ is the crystalline period sheaf of $\cO_X^{\ur}$-module with connection $\nabla$ such that $\cO\BBcr^{\nabla=0}=\BBcr$.  When this holds, we say the lisse $\ZZ_p$-sheaf $\LL$ and the filtered $F$-isocrystal $\cE$ are \emph{associated}. 

We illustrate the  construction of  the crystalline comparison isomorphism briefly.
Firstly, we prove  a Poincar\'e lemma for the crystalline period sheaf $\mathbb{B}_{\cris}$ on  $X_{\proet}$. It follows from the Poincar\'e lemma (Proposition \ref{poincare}) that the natural morphism from $\BBcr$  to the de Rham complex $DR(\cO\BBcr)$ of $\cO\BBcr$ is a quasi-isomorphism, which is compatible with  filtration and Frobenius. 
When $\LL$ and $\cE$ are associated, the natural morphism\[
\LL\otimes_{\ZZ_p}DR(\cO\BBcr) \rightarrow DR(\cE)\otimes \cO\BBcr
\]
is an isomorphism compatible with Frobenius  and filtration. Therefore we find  a quasi-isomorphism
\begin{equation*}
\LL\otimes_{\ZZ_p}\BBcr\simeq   DR(\cE)\otimes\cO\BBcr.
\end{equation*}
From this we deduce  
\begin{equation}
R\Gamma(X_{\bk,\proet}, \LL\otimes_{\ZZ_p} \BBcr)\simto R\Gamma(X_{\bk,\proet}, DR(\cE)\otimes \cO\mathbb B_{\cris} ). 
\end{equation}
Via the natural  morphism of topoi $\overline{w}: X_{\bk, \proet}^{\sim}\ra \mathcal X_{\et}^{\sim}$, one has
\begin{equation}
R\Gamma(X_{\bk, \proet},DR(\cE)\otimes\cO\BBcr)\simeq R\Gamma(\cX_{\et},DR(\cE)\ho_{\cO_k} B_{\cris}))
\end{equation}
for which  we have used the fact that the natural morphism \[\cO_{\cX}\widehat{\otimes}_{\cO_k}B_{\cris}\to R\overline w_{\ast}\cO\BBcr\] is an isomorphism (compatible with extra structures), which is  a result of Andreatta-Brinon.

%On the other hand, as $A_{\cris}$ is flat over $\cO_k$ and as $\cX$ is proper over $\cO_k$, the following map is an isomorphism compatible with filtration and Galois action:
%\[R\Gamma(\cX_{\et},DR(\cE))\otimes_{\cO_k} A_{\cris} \simto  R\Gamma(\cX_{\et},DR(\cE)\widehat{\otimes}_{\cO_k}A_{\cris})\] and then
%\begin{equation}R\Gamma(\cX_{\et},DR(\cE))\otimes_{\cO_k} B_{\cris}\simto  R\Gamma(\cX_{\et}, DR(\cE)\widehat{\otimes}_kB_{\cris}).\end{equation}
Combining the isomorphisms above, we obtain the desired crystalline comparison isomorphism. 



\begin{thm}[See Theorem \ref{thm.comp}]\label{abs}  Let $\LL$ be a lisse $\ZZ_p$-sheaf on $X$ and $\mathcal E$ be a filtered $F$-isocrystal on $\cX_0/\cO_k$ which  are associated as in (\ref{associated1}). Then there is a natural isomorphism of $B_{\cris}$-modules
\[
H^i(X_{\bk,\et}, \LL)\otimes B_{\cris} \simto H_{\cris}^i(\cX_0/\cO_k,\mathcal E)\otimes_{k} B_{\cris}
\]which is compatible with $G_k$-action, filtration and Frobenius.
\end{thm}


After obtaining a refined version of the acyclicity of crystalline period sheaf $\cO\BBcr$ in \S \ref{acy}, we achieve the crystalline comparison in the relative setting, which reduces to Theorem \ref{abs} when $\cY=\Spf \cO_k$:

\begin{thm}[See Theorem \ref{thm.relativecomp}]
Let  $f\colon \cX\to \cY$ be a proper smooth morphism of smooth formal schemes over $\Spf \cO_k$, with $f_k\colon X\to Y$ the generic fiber  and  $f_{\cris}$  the morphism between the crystalline topoi. Let $\LL, \cE$ be as in Theorem \ref{abs}. Suppose that $R^if_{k*}\LL$ is a lisse  $\ZZ_p$-sheaf on $Y$. Then it is crystalline and is associated to the filtered $F$-isocrystal $R^if_{\cris \ast}\cE$. 
 \end{thm}

\bigskip


\textbf{Acknowledgments.}
The authors are deeply indebted to Andreatta, Iovita and Scholze for the works \cite{AI} and \cite{Sch13}. They wish to thank Kiran Kedlaya and Barry Mazur  for their interests in this project.   %They are grateful to Shanghai Center for Mathematical Sciences for the hospitality in the final stages of the project. 







\section{Crystalline period sheaves}

Let $k$ be a $p$-adic field with residue field $\kappa$.  Let $X$ be a locally noetherian adic space over $\mathrm{Spa}(k,\cO_k)$.  For the fundamentals on the pro-\'etale site $X_{\proet}$, we refer to \cite{Sch13}. 

The following terminology and notation will be used frequently throughout the paper. We shall fix once for all an algebraic closure $\bk$ of $k$, and consider $X_{\bk}:=X\times_{\Spa(k, \cO_k)}\Spa(\bk, \cO_{\bk})$ as an object of $X_{\proet}$. As in  \cite[Definition 4.3]{Sch13}, an object $U\in X_{\proet}$ lying above $X_{\bk}$ is called an \emph{affinoid perfectoid} (lying above $X_{\bk}$) if $U$ has a pro-\' etale presentation $U=\varprojlim U_i \to X$ by affinoids $U_i=\Spa(R_i,R_i^+)$ above $X_{\bk}$ such that, with $R^+$ the $p$-adic completion of $\varinjlim R_i^+$ and $R=R^+[1/p]$,  the pair $(R,R^+)$ is a perfectoid affinoid $(\widehat{\bk}, \cO_{\widehat{\bk}})$-algebra. Write $\widehat{U}=\Spa(R,R^+)$. By \cite[Proposition 4.8, Lemma 4.6]{Sch13}, the set of affinoid perfectoids lying above $X_{\bk}$ of $X_{\proet}$ forms a basis for the topology. 


\subsection{Period sheaves and their acyclicities} Following \cite{Sch13}, let 
\[
\nu\colon X_{\proet}^{\sim}\longrightarrow X_{\et}^{\sim}
\]
be the morphism of topoi, which, on the underlying sites, sends an \'etale morphism $U\to X$ to the pro-\'etale morphism from $U$ (viewed as a constant projective system) to $X$. Consider $\cO_X^+=\nu^{-1}\cO_{X_{\et}}^+$ and $\cO_X=\nu^{-1}\cO_{X_{\et}}$,  the (uncompleted) structural sheaves on $X_{\proet}$. More concretely, for $U=\varprojlim U_i$ a qcqs (quasi-compact and quasi-separated) object of $X_{\proet}$, one has  $\cO_X(U)=\varinjlim \cO_{X}(U_i)=\varinjlim \cO_{X_{\et}}(U_i)$ (\cite[Lemma 3.16]{Sch13}). Set
\[
\widehat{\cO_{X}^+}:=\varprojlim_{n} \cO_X^+/p^n,\quad \cO_X^{\flat +}=\varprojlim_{x\mapsto x^p}\cO_X^+/p\cO_X^+. 
\]



For $U=\varprojlim U_i\in X_{\proet}$ an affinoid perfectoid lying above $X_{\bk}$ with $\widehat{U}=\Spa(R,R^+)$, by \cite[Lemmas 4.10, 5.10]{Sch13}, we have 
\begin{equation}\label{eq.identification}
\widehat{\cO_X^+}(U) =R^+, \quad \textrm{and}\quad \cO_X^{\flat+}(U) = R^{\flat+}:=\varprojlim_{x\mapsto x^p}R^+/pR^+. 
\end{equation}

Denote 
 \[R^{\flat+}\ra R^+,\quad x=(x_0, x_1, \cdots)\mapsto x^{\sharp}:=\lim_{n\ra \infty}\widehat{x}_n^{p^n}, \]for $\widehat{x}_{n}$ any lifting from $R^{+}/p$ to $R^+$.
We have the multiplicative homeomorphism (induced by projection): \[\varprojlim_{x\mapsto x^p}R^+\simto R^{\flat+}, \quad  (x^{\sharp},(x^{1/p})^{\sharp}, \cdots)\mapsfrom x\]which can be extended to a ring homomorphism.  




Put $\mathbb A_{\inf}:=W(\cO_X^{\flat +})$ and $\BBinf=\AAinf[\frac{1}{p}]$. As $R^{\flat +}$ is a perfect ring, $\AAinf(U)=W(R^{\flat +})$ has no $p$-torsions. In particular, $\AAinf$ has no $p$-torsions and it is a subsheaf of $\BBinf$. Following Fontaine, define as in \cite[Definition 6.1]{Sch13} a natural  morphism 
\begin{equation}\label{theta}
\theta\colon \mathbb A_{\inf}\to \widehat{\cO_{X}^+}
\end{equation} which, on an affinoid perfectoid $U$ with $\widehat{U}=\Spa(R,R^+)$, is given by 
\begin{equation}\label{thetaU}
\theta(U)\colon \mathbb A_{\inf}(U)=W(R^{\flat +}) \longrightarrow \widehat{\cO_X^+}(U)=R^+, \quad  (x_0,x_1, \cdots) \mapsto \sum_{n=0}^{\infty}p^nx_{n,n}^{\sharp}
\end{equation}
with $x_n=(x_{n,i})_{i\in \N} \in R^{\flat +}=\varprojlim_{x\mapsto x^p} R^+/p$. % and $\widehat{x}_{n,n+m}$ any lifting from $R^{+}/p$ to $R^+$. 
 As $(R,R^+)$ is a perfectoid affinoid algebra, $\theta(U)$ is known to be surjective (cf. \cite[5.1.2]{Bri}). Therefore, $\theta$ is also surjective.  


\begin{defn}Let $X$ be a locally noetherian adic space over $\Spa(k, \cO_k)$ as above. Consider the following sheaves on $X_{\proet}$. 
\begin{enumerate}
\item Define $\mathbb A_{\cris}$ to be the $p$-adic completion of the PD-envelope $\mathbb A_{\cris}^0$ of $\mathbb A_{\inf}$ with respect to the ideal sheaf $\ker(\theta)\subset \mathbb A_{\inf}$, and define $\mathbb B_{\cris}^+:=\mathbb A_{\cris}[1/p]$. 


\item For $r\in \mathbb Z_{\geq 0}$, set $\Fil^r \AAcr^0:=\ker(\theta)^{[r]}\AAcr^0\subset \AAcr^0$ to be the $r$-th divided power ideal, and $\Fil^{-r}\AAcr^0=\AAcr^0$. So the family $\{\Fil^r\AAcr^0:r\in \mathbb Z\}$ gives a descending filtration of $\AAcr^0$. 
 
\item For $r\in  \ZZ$, define $\Fil^{r}\AAcr\subset \AAcr$ to be the image of the following morphism of sheaves (we shall see below that this map is actually injective):
\begin{equation}\label{eq.mor}
\varprojlim_n  (\Fil^r\AAcr^0)\slash p^n\longrightarrow \varprojlim_n \AAcr^0/p^n=\AAcr, 
\end{equation}
and define $\Fil^{r}\BBcrp=\Fil^r\AAcr[1/p]$. 
\end{enumerate}
\end{defn}


Let $p^{\flat}=(p_i)_{i\geq 0}$ be a family of elements of $\bk$ such that $p_0=p$ and that $p_{i+1}^p=p_i$ for any $i\geq 0$. 
Set \[\xi=[p^{\flat}]-p\in \mathbb A_{\inf}|_{X_{\bk}}.\]
\begin{lemma} \label{nonzerodivisor}
We have $\ker(\theta)|_{X_{\bk}}=(\xi)\subset \mathbb A_{\inf}|_{X_{\bk}}$. Furthermore, $\xi\in \AAinf|_{X_{\bk}}$ is not a zero-divisor. 
\end{lemma}

\begin{proof} As the set of affinoids perfectoids $U$ lying above $X_{\bk}$ forms a basis for the topology of $X_{\proet}/X_{\bk}$, we only need to check that, for any such  $U$, $\xi\in \AAinf(U)$ is not a zero-divisor and that the kernel of $\theta(U)\colon \AAinf(U)\to \widehat{\cO_X^+}(U)$ is generated by $\xi$. Write $\widehat{U}=\Spa(R,R^+)$. Then $\AAinf(U)=W(R^{\flat +})$ and $\widehat{O_X^{+}}(U)=R^+$, hence we reduce our statement to (the proof of) \cite[Lemma 6.3]{Sch13}. 
\end{proof}

\begin{cor}\label{cor.DespOfAcris0} (1) We have $
\mathbb A_{\cris}^0|_{X_{\bk}}=\mathbb A_{\inf}|_{X_{\bk}}\left[\xi^n/n!: n\in \mathbb N \right]\subset \mathbb{B}_{\inf}|_{X_{\bk}}$. Moreover, for $r\ge0$, $\Fil^r\AAcr^0|_{X_{\bk}}=\mathbb{A}_{\inf}|_{X_{\bk}}[\xi^{n}/n! : n\geq r]$ and $\mathrm{gr}^r \AAcr^0|_{X_{\bk}}\stackrel{\sim}{\longrightarrow}\widehat{\cO_X^+}|_{X_{\bk}}$. 


(2) The morphism \eqref{eq.mor} is injective, hence $\varprojlim_n\Fil^r\AAcr^0/p^n\stackrel{\sim}{\longrightarrow}\Fil^r\AAcr$. Moreover, for $r\geq 0$, $\mathrm{gr}^r \AAcr|_{X_{\bk}}\stackrel{\sim}{\longrightarrow}\widehat{\cO_X^+}|_{X_{\bk}}$.
\end{cor}

\begin{proof} The first two statements in (1) are clear. 
In particular, for $r\geq 0$ we have the following exact sequence
\begin{equation}\label{eq.sesFilAcris0}
\xymatrix{0\ar[r] &  \Fil^{r+1}\AAcr^0|_{X_{\bk}}\ar[r]& \Fil^r \AAcr^0|_{X_{\bk}}\ar[r] & \widehat{\cO_X^{+}}|_{X_{\bk}}\ar[r] & 0},
\end{equation}
where the second map sends $a\xi^{r}/r!$ to $\theta(a)$. This gives the last assertion   of (1). 

As $\widehat{\cO_X^+}$ has no $p$-torsions, an induction on $r$ shows that the cokernel of the inclusion $\Fil^r\AAcr^0\subset \AAcr^0 $ has no $p$-torsions. As a result, the morphism \eqref{eq.mor} is injective and $\Fil^r\AAcr$ is the $p$-adic completion of $\Fil^r\AAcr^0$. Since $\widehat{\cO_X^+}$ is $p$-adically complete, we deduce from \eqref{eq.sesFilAcris0} also the following short exact sequence after passing to $p$-adic completions:
\begin{equation}\label{eq.sesFilAcris}
\xymatrix{0\ar[r] &  \Fil^{r+1}\AAcr |_{X_{\bk}}\ar[r]& \Fil^r \AAcr|_{X_{\bk}}\ar[r] & \widehat{\cO_X^{+}}|_{X_{\bk}}\ar[r] & 0}
\end{equation}
giving the last part of (2). 
\end{proof}

Let $\epsilon=(\epsilon^{(i)})_{i\geq 0}$ be a sequence of elements of $\bk$ such that $\epsilon^{(0)}=1$, $\epsilon^{(1)}\neq 1$ and $(\epsilon^{(i+1)})^p=\epsilon^{(i)}$ for all $i\geq 0$. Then $1-[\epsilon]$ is a well-defined element of the restriction $\mathbb A_{\inf}|_{X_{\bk}}$ to $X_{\proet}/X_{\bk}$ of $\mathbb A_{\inf}$. Moreover $1-[\epsilon]\in \ker(\theta)|_{X_{\bk}}=\Fil^1\AAcr|_{X_{\bk}}$. Let
\begin{equation}\label{t}
t:=\log([\epsilon])=-\sum_{n=1}^{\infty}\frac{(1-[\epsilon])^n}{n},
\end{equation}
which is well-defined in $\AAcr|_{X_{\bk}}$ since $\Fil^1\AAcr$ is a PD-ideal. 


\begin{defn} Let $X$ be a locally noetherian adic space over $\Spa(k,\cO_k)$. Define $\BBcr=\BBcrp[1/t]$. For $r\in \mathbb Z$, set $\Fil^r\BBcr=\sum_{s\in \ZZ}t^{-s}\Fil^{r+s}\BBcrp\subset \BBcr$. (As the canonical element $t\in \AAcr$ exists locally, these definitions do make sense.)  
\end{defn}




Before investigating these period sheaves in details, we first study them  over a perfectoid affinoid $(\widehat{\bk}, \cO_{\widehat{\bk}})$-algebra $(R,R^+)$. Consider 
\[
\mathbb A_{\inf}(R,R^{+}):=W(R^{\flat +}),\quad \mathbb B_{\inf}(R,R^+):=\mathbb A_{\inf}(R,R^+)[1/p],
\]
and define the morphism 
\begin{equation}\label{theta2}
\theta_{(R,R^{+})}\colon \mathbb A_{\inf}(R,R^+)\longrightarrow R^+
\end{equation} 
in the same way as in (\ref{thetaU}). It is known to be surjective as $(R,R^+)$ is perfectoid. Furthermore, we have seen that the element $\xi\in \mathbb A_{\inf}(\widehat{\bk},\cO_{\widehat{\bk}})$ generates $\ker(\theta_{(R,R^+)})$ and is not a zero-divisor in $\mathbb A_{\inf}(R,R^+)$. Let $\mathbb{A}_{\cris}(R,R^{+})$ be the $p$-adic completion of the PD-envelope of $\mathbb A_{\inf}(R,R^+)$ with respect to the ideal $\ker(\theta_{(R,R^+)})$. So $\mathbb A_{\cris}(R,R^+)$ is the $p$-adic completion of $\mathbb A_{\cris}^{0}(R,R^+)$ with 
\[
\mathbb A_{\cris}^0(R,R^{+}):=\mathbb A_{\inf}(R,R^{+})\left[\frac{\xi^n}{n!} : n\in \mathbb N\right]\subset \mathbb B_{\inf}(R,R^{+}). 
\]%is the cokernel of the multiplication-by-$(X_0-\xi)$ on $\widetilde{\mathbb A_{\cris}^0}(R,R^+)$. 
For $r$ an integer, let $\Fil^r\AAcr^0(R,R^+)\subset \AAcr^0(R,R^+)$ be the $r$-th PD-ideal, \emph{i.e.,} the ideal generated by $\xi^n/n!$ for $n\geq \mathrm{max}\{r,0\}$.  Let $\Fil^r\AAcr(R,R^+)\subset \mathbb A_{\cris}(R,R^+)$ be the closure (for the $p$-adic topology) of $\Fil^r\AAcr^0(R,R^+)$ inside $\AAcr(R,R^+)$. Finally, put $\mathbb B_{\cris}^{+}(R,R^+):=\mathbb A_{\cris}(R,R^+)[1/p]$, $\mathbb B_{\cris}(R,R^{+}):=\mathbb B_{\cris}^{+}(R,R^+)[1/t]$, and for $r\in \mathbb Z$, set 
\begin{eqnarray*}
\Fil^r\BBcrp(R,R^+):=\Fil^ r\AAcr(R,R^+)[1/p] \textrm{ and} \\ \Fil^r\BBcr(R,R^+):=\sum_{s\in \ZZ}t^{-s}\Fil^{r+s}\BBcrp(R,R^+).
\end{eqnarray*}
 
In particular, taking $R^+=\cO_{\Cp}$ with $\Cp$ the $p$-adic completion of the fixed algebraic closure $\bk$ of $k$, we get Fontaine's rings $A_{\cris}$, $B_{\cris}^+$, $B_{\cris}$ as in \cite{Fon94}.  


 


\begin{lemma}\label{vanish}Let $X$ be a locally noetherian adic space over $(k,\cO_k)$. Let $U\in X_{\proet}$ be an affinoid perfectoid above $X_{\bk}$ with $\widehat{U}=\mathrm{Spa}(R,R^{+})$. Let $\mathcal F \in \{\AAcr^0, \AAcr\}$. Then for any $r\geq \mathbb Z$, we have a natural almost isomorphism $ \Fil^r \mathcal F(R, R^+)^a \stackrel{\approx}{\ra} \Fil^r\mathcal F(U)^a$,  and have $H^i(U,\Fil^r\mathcal F)^a=0$ for any $i>0$.
\end{lemma}

\begin{proof}  As $U$ is affinoid perfectoid, we have $\widehat{\cO_X^+}(U)=R^+$, $\cO_{X}^{\flat +}(U)=R^{\flat +}$ and $\theta(U)=\theta_{(R,R^+)}$. In particular, $\mathbb A_{\inf}(U)=\mathbb A_{\inf}(R,R^+)$, and the natural morphism $
\alpha\colon \mathbb A_{\inf}(R,R^+)=\mathbb A_{\inf}(U) \longrightarrow \mathbb A_{\cris}^0(U)$ sends $\ker(\theta_{(R,R^+)})$ into $\Fil^1\AAcr^0(U)$. 
%So $\alpha$ factors through $\AAcr^0(R,R^+)$. \footnote{Should be "extends to"}
As a consequence, we get a natural morphism $\AAcr^0(R,R^+)\ra \AAcr^0(U)$, inducing morphisms $\Fil^r\AAcr^0(R,R^+)\to \Fil^r\AAcr^0(U)$ between the filtrations. Passing to $p$-adic completions, we obtain the natural maps $\Fil^r\AAcr(R,R^+)\to \Fil^r\AAcr(U)$ for all $r\in \mathbb Z$. 

We need to show that the morphisms constructed above are almost isomorphisms. Recall that, as $U$ is affinoid perfectoid, $H^i(U,\AAinf)^a=0$ for $i>0$ (\cite[Theorem 6.5]{Sch13}). Consider $
\widetilde{\mathbb A_{\cris}^0}:=\frac{\mathbb A_{\inf}[X_i:i\in \mathbb N]}{(X_i^p-a_iX_{i+1}:i\in \mathbb N)}
$
with $a_i=\frac{p^{i+1}!}{(p^i!)^p}$, then one has the following short exact sequence
\[
\xymatrix{0\ar[r] & \widetilde{\mathbb A_{\cris}^{0}}\ar[r]^{\cdot (X_0-\xi)} & \widetilde{\mathbb A_{\cris}^0}\ar[r] &  \mathbb A_{\cris}^0\ar[r] & 0}.
\]
Since $\widetilde{\mathbb A_{\cris}^0}$ is a direct sum of $\mathbb A_{\inf}$'s as an abelian sheaf and $U$ is qcqs, we get $H^i\left(U,\widetilde{\mathbb A_{\cris}^0}\right)^a=0$ for $i>0$. Hence $H^i(U, \AAcr^ 0)^a=0$ for $i>0$ and the short exact sequence above stays almost exact after taking sections over $U$: \begin{equation}\label{eq.Acris0tildeU}
\xymatrix{0\ar[r] & \widetilde{\mathbb A_{\cris}^{0}}(U)^a\ar[r]^{\cdot (X_0-\xi)} & \widetilde{\mathbb A_{\cris}^0}(U)^a\ar[r] &  \mathbb A_{\cris}^0(U)^a\ar[r] & 0}.
\end{equation}
On the other hand, set $
\widetilde{\mathbb A_{\cris}^0}(R,R^+):=\frac{\mathbb A_{\inf}(R,R^{+})[X_i:i\in \mathbb N]}{(X_i^p-a_iX_{i+1}:i\in \mathbb N)}$
with $a_i=\frac{p^{i+1}!}{(p^i!)^p}$ as above, then the following similar sequence is exact:
\begin{equation}\label{eq.Acris0tildeR}
\xymatrix{0\ar[r] & \widetilde{\mathbb A_{\cris}^{0}}(R,R^+)^a\ar[r]^{\cdot (X_0-\xi)} & \widetilde{\mathbb A_{\cris}^0}(R,R^+)^a\ar[r] &  \mathbb A_{\cris}^0(R,R^+)^a\ar[r] & 0}.
\end{equation}
Since $U$ is qcqs, $\widetilde{\mathbb A_{\cris}^0}(U)=\widetilde{\mathbb A_{\cris}^0}(R,R^+)$. Combining \eqref{eq.Acris0tildeU} and \eqref{eq.Acris0tildeR}, we find $\mathbb A_{\cris}^0(R,R^+)^a\simto\mathbb A_{\cris}^0(U)^a$. This proves the statement for $\AAcr^0$. 

Next, as $\AAcr^0\subset\BBinf$ (Corollary \ref{cor.DespOfAcris0}), the sheaf $\AAcr^0$ has no $p$-torsions. So we get the following tautological exact sequence 
\[
0\longrightarrow \mathbb A_{\cris}^{0} \stackrel{p^n}{\longrightarrow} \mathbb A_{\cris}^{0} \longrightarrow \mathbb A_{\cris}^{0}/p^n\longrightarrow 0. 
\]
By what we have shown for $\AAcr^0$, we get from the exact sequence above that $
\left(\mathbb A_{\cris}^0(R,R^+)/p^n\right)^a\stackrel{\sim}{\to} \left(\mathbb A_{\cris}^0(U)/p^n\right)^a\stackrel{\sim}{\to} \left(\mathbb A_{\cris}^0/p^n\right)(U)^a, 
$
and $H^i(U,\mathbb A_{\cris}^0/p^n)^a=0$ for $ i>0$. Therefore the transition maps of the projective system $\{(\mathbb A_{\cris}^0/p^n)(U)\}_{n\geq 0}$ are almost surjective, thus $R^{1}\varprojlim \left((\mathbb A_{\cris}^0/p^n)(U)\right)^a=0$. So the projective system $\{\mathbb A_{\cris}^0/p^n\}$ verifies the assumptions of (the almost version of)\cite[Lemma 3.18]{Sch13}. As a result, we find 
\begin{equation}\label{Rjvanish}
R^j\varprojlim A_{\cris}^0/p^n=0, \quad \textrm{for any } j>0,
\end{equation} 
\[
\mathbb A_{\cris}(R,R^+)^a=\varprojlim (\mathbb A_{\cris}^0(R,R^+)/p^n)^a\stackrel{\sim}{\lra}\varprojlim \left(\left(\mathbb A_{\cris}^0/p^n\right)(U)\right)^a =\mathbb A_{\cris}(U)^a
,\] 
and
\[H^i(U,\mathbb A_{\cris})^a=H^i(U,\varprojlim_n \mathbb A_{\cris}^0/p^n)^a=0
\]
for $i>0$.  

 

To prove the statements for $\Fil^r\AAcr^0$ for $r\geq 0$, we shall use the exact sequence \eqref{eq.sesFilAcris0}. As $H^i(U, \widehat{\cO_X^+})^a=0$ for $i>0$ (\cite[Lemma 4.10]{Sch13}), we find $H^i(U,\Fil^r\AAcr^0)^a=0$ for all $i\geq 2$, and also the induced long exact sequence
\[
\begin{array}{c}
0\lra \Fil^{r+1}\AAcr^0(U)^a\lra  \Fil^{r}\AAcr^0(U)^a\lra \widehat{\cO_X^+} (U)^a\lra \\ H^{1}(U, \Fil^{r+1}\AAcr^0)^a\lra H^{1}(U, \Fil^{r}\AAcr^0)^a \lra 0. 
\end{array}
\]
On the other hand, we have the analogous exact sequence for $\Fil^r\AAcr^0(R,R^+)$: 
\[
0\lra \Fil^{r+1}\AAcr^0(R,R^+)\lra  \Fil^{r}\AAcr^0(R,R^+)\lra R^+\lra  0, 
\]
where the second morphism sends $a\xi^r/r!$ to $\theta_{(R,R^+)}(a)$. Together with the two exact sequences above and the vanishing $H^1(U, \AAcr^ 0)^a=0$, an induction on $r\geq 0$ shows $
\Fil^r\AAcr^0(R,R^+)^a\stackrel{\sim}{\to}\Fil^r\AAcr^0(U)^a$ and $H^1(U, \Fil^r\AAcr^0)^a=0. 
$
This gives the statement for $\Fil^r\AAcr^0$ for $r\in \mathbb Z$ (note that $\Fil^r \AAcr^0=\AAcr^0$ when $r\leq 0$). The statement for $\Fil^r\AAcr$ can be done in the same way; one starts with \eqref{eq.sesFilAcris} and uses the vanishing of $H^1(U,\mathbb A_{\cris})^a$. 
\end{proof}

\begin{cor}\label{cor.tNotZeroDiv} Keep the notation of Lemma \ref{vanish}.  
\begin{enumerate}
\item The morphism multiplication by $t$ on $\AAcr$ is almost injective. In particular, $t\in \BBcrp|_{X_{\bk}}$ is not a zero-divisor. 
\item Let $\mathcal F\in \{\BBcrp, \BBcr\}$. Then for any $r\in \mathbb Z$, there is a natural isomorphism $\Fil^r \mathcal  F(R,R^+)\stackrel{\sim}{\to} \mathcal \Fil^r\cF(U)$, and $H^i(U,\Fil^r \mathcal F)=0$ for $i\geq 1$.   
\end{enumerate}
\end{cor}

\begin{proof} (1) By Lemma \ref{vanish}, we are reduced to showing that $t\in \AAcr(R,R^+)$ is not a zero-divisor.  So we just need to apply \cite[Corollaire 6.2.2]{Bri}, whose proof works for any such perfectoid $(R,R^+)$.
%\footnote{Note for ourselves: By $\xi\BBdrp=([\epsilon]-1)\BBdrp$ and Lemma \ref{nonzerodivisor}, $\BBdrp$ has not $t$-torsions. (This is what  \cite[Corollaire 6.2.2]{Bri} uses. As explained in [Bri, 5.1.4], that  $\xi\BBdrp=([\epsilon]-1)\BBdrp$ follows from the classical case of Fontaine. In 2.17 of Fontaine's 1982 Annales paper,  he showed $t$ is a uniformizer of $B_{\rm dR}^+$.) Then use the inclusion $\BBcr^+\subset \BBdrp$ in Lemma \ref{BcrisBdR}. } 

(2) As $U$ is qcqs, we deduce from Lemma \ref{vanish} the statement for $\mathbb B_{\cris}^{+}$ and for $\Fil^r\BBcrp$ on inverting $p$. Similarly, inverting $t$ we get the statement for $\mathbb B_{\cris}$. Finally, as $
\Fil^r\BBcr|_{X_{\bk}}=\varinjlim_s t^{-s}\Fil^{r+s}\BBcrp|_{X_{\bk}}$ 
and as $t^{-s}\Fil^{r+s}\BBcrp|_{X_{\bk}}\simeq \Fil^{r+s}\BBcrp|_{X_{\bk}}$, passing to limits we get the statement for $\Fil^r\BBcr$.
\end{proof}

\iffalse
\begin{lemma}
Let $X$ be a locally noetherian adic space over a perfectoid field. Let $T$ be a basis of $X_{\proet}$ (for example, the affinoid perfectoids). For an abelian sheaf $\cF$ on $X_{\proet}$ which is $p$-torsion free and has the property that $H^i(U, \cF)^a=0$ for any $U\in T$, one has, denoting $\cF_n=\cF/p^n$ and  $\widehat{\cF}=\varprojlim\cF_n$, that 
\[\widehat{\cF}(V)^a\simeq \varprojlim(\cF_n(V))^a, \quad (R^i\varprojlim\cF_n)^a=0, \quad \forall i>0, V\in X_{\proet}\]and 
\[H^i(U, \widehat{\cF})^a=0, \quad \forall i>0, U\in T.\] 

\end{lemma}
\begin{proof}
By the $p$-torsion freeness,  we have the following short exact sequence 
\[
\xymatrix{0\ar[r]& \cF \ar[r]^{p^n}& \cF \ar[r]& \cF_n\ar[r] & 0. }
\]
The associated long exact sequence and the vanishing of $H^i(U, \cF)^a$ give that  $H^i(U, \cF_n)^a=0$ ($\forall  i>0, U\in T$), and 
\[
\cF_n(U)^a\simeq \cF(U)^a/p^n.
\]
The latter implies that the transition morphisms of the projective system $\{\cF_n(U)\}_{n\geq 0}$ are almost surjective, hence $R^{1}\varprojlim (\cF_n(U))^a=0$. So the projective system $\{\cF_n\}$ verifies the assumptions of (the almost version of) \cite[Lemma 3.18]{Sch13}, from which the conclusions follow.
\end{proof}
\fi

 \medskip

\subsection{Period sheaves with connections}  In this section, assume that the $p$-adic field $k$ is \emph{absolutely unramified}.  
Let $\cX$ be a smooth formal scheme over $\cO_k$. Set $X:=\cX_k$ the generic fiber of $\cX$, viewed as an adic space over $\Spa(k,\cO_k)$. For any \'etale morphism $\mathcal Y\to \cX$, by taking the generic fibers, we obtain an \'etale morphism $\mathcal Y_k\to X$ of adic spaces, hence an object of the pro-\'etale site $X_{\proet}$. In this way, we get a morphism of sites $\cX_{\et}\to X_{\proet}$, with the induced morphism of topoi
\[
w\colon X_{\proet}^{\sim}\longrightarrow \cX_{\et}^{\sim}. 
\]
Let $\cO_{\cX_{\et}}$ denote the structure sheaf of the \'etale site $\cX_{\et}$: for any \'etale morphism $\mathcal Y\to \cX$ of formal schemes over $\cO_k$, $\cO_{\cX_{\et}}(\mathcal Y)=\Gamma(\mathcal Y,\cO_{\mathcal Y})$. Define $\cO_{X}^{\ur+}:=w^{-1}\cO_{\cX_{\et}}$  and $\cO_{X}^{\ur}:=w^{-1 }\cO_{\cX_{\et}}[1/p]$.  Thus $\cO_{X}^{\ur+}$ is the associated sheaf of the presheaf $\widetilde{\cO_{X}^{\ur+}}$:
\[
X_{\proet} \ni U \mapsto \varinjlim_{(\mathcal Y,a)}\cO_{\cX_{\et}}(\mathcal Y)=:\widetilde{\cO_{X}^{\ur+}}(U),
\]
where the limit runs through all pairs $(\mathcal Y,a)$ with $\mathcal Y\in \cX_{\et}$ and $a\colon U\to \mathcal Y_k$ a morphism making the following diagram commutative \begin{equation}\label{eq.factorization}
\xymatrix{U\ar[r]\ar[rd]_a & X=\cX_k \\ & \mathcal Y_k.\ar[u]}
\end{equation}
The morphism $a\colon U\to \mathcal Y_k$ induces a map $\Gamma(\mathcal Y,\cO_{\mathcal Y})\to \cO_{X}(U)$. 
There is then a morphism of presheaves $\widetilde{\cO_{X}^{\ur+}}\to \cO_X^+$, whence a morphism of sheaves
\begin{equation}\label{eq.OurAlg}
\cO_{X}^{\ur+}\to \cO_{X}^{+}.
\end{equation} 
Recall $\AAinf:=W(\cO_{X}^{\flat+})$. Set $\cO\AAinf:= \cO_{X}^{\rm ur +}\otimes_{\cO_k}\mathbb A_{\inf}$ and 
\begin{equation}\label{eq.thetaX}
\theta_{X}\colon \cO\AAinf \longrightarrow  \widehat{\cO_{X}^{+}}
\end{equation}
to be the map induced from $\theta\colon \mathbb A_{\inf}\to \widehat{\cO_{X}^+}$ of \eqref{theta} by extension of scalars.

\begin{defn} Consider the following sheaves on $X_{\proet}$. 
\begin{enumerate}
\item Let $\cO\mathbb A_{\cris}$ be the $p$-adic completion of the PD-envelope $\cO\mathbb A_{\cris}^0$ of $\cO\mathbb A_{\inf}$ with respect to the ideal sheaf $\ker(\theta_X)\subset \cO\mathbb{A}_{\inf}$, $\cO\mathbb B_{\cris}^+:=\cO\mathbb A_{\cris}[1/p]$, and $\cO\mathbb B_{\cris}:=\cO\mathbb B_{\cris}^+[1/t]$ with $t=\log([\epsilon])$ defined in \eqref{t}. 
\item For $r\in \ZZ_{\geq 0}$, define $\Fil^{r}\cO\AAcr^0\subset \cO\AAcr^ 0$ to be the $r$-th PD-ideal $\ker(\theta_X)^{[r]}$, and $\Fil^r\cO\AAcr$  the image of the canonical map 
\[
\varprojlim \Fil^r\cO\AAcr^0/p^n \longrightarrow \varprojlim \cO\AAcr^0/p^n=\cO\AAcr.
\]
Also set $\Fil^{-r}\cO\AAcr=\cO\AAcr$ for $r>0$. 
\item For $r\in \mathbb Z$, set
\[
\Fil^r\cO\BBcrp:=\Fil^r\cO\AAcr[1/p] \quad \textrm{and} \quad \Fil^r\cO\BBcr:=\sum_{s\in \ZZ}t^{-s}\Fil^{r+s}\cO\BBcrp.\]
\end{enumerate}
\end{defn}


\begin{rk} As $t^{p}=p!\cdot t^{[p]}$ in $A_{\cris}:=\AAcr(\widehat{\bk},\cO_{\widehat{\bk}})$, one can also define $\Fil^r\cO\BBcr$ as $\sum_{s\in \mathbb N}t^{-s}\Fil^{r+s}\cO\AAcr$. A similar observation holds equally for $\Fil^r\BBcr$.
\end{rk}

\begin{rk}\label{rk.Fil1OAcris} \begin{enumerate}
\item As $\cX$ is flat over $\cO_k$, the structure sheaf $\cO_{\cX_{\et}}$ has no $p$-torsions. It follows that $\cO\AAinf$, $\cO\AAcr^0$ and $\cO\AAcr$ have no $p$-torsions. So $\cO\AAcr\subset \cO\BBcrp$.

\item The morphism $\theta_X$ of \eqref{eq.thetaX} extends to a surjective morphism from $\cO\AAcr^0$ to $\widehat{\cO_X^+}$ with kernel $\Fil^1\cO\AAcr^0$, hence also a morphism from $\cO\AAcr$ to $\widehat{\cO_X^+}$. Let us denote these two morphisms again by $\theta_X$. As $\mathrm{coker}(\Fil^1\cO\AAcr^0\hookrightarrow \cO\AAcr^0)\simeq \widehat{\cO_X^{+}}$ is $p$-adically complete and has no $p$-torsions, using the snake lemma and passing to limits one can deduce the following short exact sequence 
\[
0\longrightarrow \varprojlim_n (\Fil^1\cO\AAcr^0/p^n) \longrightarrow \cO\AAcr \stackrel{\theta_X}{\longrightarrow} \widehat{\cO_X^{+}} \longrightarrow 0. 
\]
In particular, $\Fil^1\cO\AAcr=\ker(\theta_X)$. 
\end{enumerate}
\end{rk}


 
In order to describe explicitly the sheaf $\cO\mathbb A_{\cris}$, we shall need some auxiliary sheaves on $X_{\proet}$.  

\begin{defn} Consider the following sheaves on $X_{\proet}$. 
\begin{enumerate}
\item Let $\AAcr\{\langle u_1,\ldots, u_d\rangle\}$ be the $p$-adic completion of the sheaf of PD polynomial rings $\AAcr^0\langle u_1,\ldots, u_d\rangle \subset \BBinf[u_1,\ldots, u_d]$. Set $\BBcrp\{\langle u_1,\ldots, u_d\rangle \}:=\AAcr\{\langle u_1,\ldots, u_d\rangle \}[1/p]$ and $\BBcr\{\langle u_1,\ldots, u_d\rangle \}:=\AAcr\{\langle u_1,\ldots, u_d\rangle\}[1/t]$.

\item For $r\in \mathbb Z$, let $\Fil^{r}\AAcr^0\langle u_1,\ldots, u_d\rangle\subset \AAcr^0\langle u_1,\ldots, u_d\rangle$ be the ideal sheaf given by  
\[
\Fil^r(\AAcr^0\langle u_1,\ldots, u_d\rangle):=\sum_{i_1,\ldots,i_d\geq 0}\Fil^{r-(i_1+\ldots+i_d)}\mathbb A_{\cris}^0\cdot u_1^{[i_1]}\cdots u_{d}^{[i_d]},
\]
and $\Fil^r(\mathbb A_{\cris}\{\langle u_1,\ldots, u_d\rangle\})\subset \AAcr\{\langle u_1,\ldots, u_d\rangle \}$ the image of the morphism
\[
\varprojlim_n \left(\Fil^r\AAcr^0\{\langle u_1,\ldots, u_d\rangle \}/p^n\right) \longrightarrow \AAcr\{\langle u_1,\ldots, u_d\rangle\}. 
\]
The family $\{\Fil^r(\AAcr\{\langle u_1,\ldots, u_d\rangle \}): r\in \mathbb Z\}$ gives a descending filtration of $\AAcr\{\langle u_1,\ldots, u_d\rangle\}$. Inverting $p$, we obtain $\Fil^r(\BBcrp\{\langle u_1,\ldots, u_d\rangle\})$. Set finally
\[
\Fil^r(\BBcr\{\langle u_1,\ldots, u_d\rangle \}):=\sum_{s\in \mathbb Z}t^{-s} \Fil^{r+s}(\BBcrp\{\langle u_1,\ldots, u_d\rangle \}).
\]  
\end{enumerate}
\end{defn}

The proof of the following lemma is similar to that of Lemma \ref{vanish}. We omit the details here. 
\begin{lemma}\label{lem1forOBcris} Let $V\in X_{\proet}$ be an affinoid perfectoid lying above $X_{\bk}$ with $\widehat{V}=\Spa(R,R^+)$. Then the following natural map is an almost isomorhpism 
\[
\AAcr(R,R^+)\{\langle u_1,\ldots, u_d\rangle \}\stackrel{\approx}{\longrightarrow} (\AAcr\{\langle u_1,\ldots, u_d\rangle \})(V). 
\]
Under this identification, $\Fil^r(\AAcr\{\langle u_1,\ldots, u_d\rangle \})(V)^a$ consists of series
\[
\sum_{i_1,\ldots, i_d\geq 0} a_{i_1,\ldots, i_d} u_{1}^{[i_1]}\cdots u_d^{[i_d]} \in \AAcr(R,R^+)\{\langle u_1,\ldots, u_d\rangle\}
\]
such that $a_{i_1,\ldots, i_d}\in \Fil^{r-(i_1+\ldots +i_d)}\AAcr(R,R^+)$ and $a_{i_1,\ldots, i_d}$ tends to $0$ for the $p$-adic topology when the sum $i_1+\ldots +i_d$ tends to infinity. Furthermore 
\[
H^ i(V, \Fil^r(\AAcr\{\langle u_1,\ldots, u_d\rangle\}))^a=0, \quad \textrm{for }i>0. 
\]
\end{lemma}

We want to describe $\cO\AAcr$ more explicitly . For this, assume there is an \'etale morphism $\cX\rightarrow \Spf(\cO_k\{T_1^{\pm 1},\ldots, T_{d}^{\pm 1}\})=:\mathcal T^d$ of formal schemes over $\cO_k$. Let $\TT^d$ denote the generic fiber of $\mathcal T^d$ and $\widetilde{\TT^d}$ be obtained from $\TT^d$ by adding a compatible system of $p^{n}$-th root of $T_i$ for $1\leq i\leq d$ and $n\geq 1$: 
\[
\widetilde{\TT^d}:=\Spa(k\{T_{1}^{\pm1/p^{\infty}},\ldots, T_d^{\pm 1/p^{\infty}}\},\cO_k\{T_{1}^{\pm1/p^{\infty}},\ldots, T_d^{\pm 1/p^{\infty}}\}).
\]
Set $\widetilde X:=X\times_{\TT^d}\widetilde{\TT^d}$. Let $T_i^{\flat}\in \cO_{X}^{\flat+}|_{\widetilde X}$ be the element $(T_i, T_i^{1/p},\ldots,T_i^{1/p^n},\ldots)$. Then $\theta_{X}(T_i\otimes 1-1\otimes [T_{i}^{\flat}])=0$, which allows us to define an $\mathbb A_{\cris}$-linear morphism
\begin{equation}\label{alpha}
\alpha\colon \mathbb A_{\cris}\{\langle u_1,\ldots, u_d\rangle\}|_{\widetilde{X}}\longrightarrow \cO\mathbb A_{\cris}|_{\widetilde X},\quad u_i\mapsto T_i\otimes 1-1\otimes [T_i^{\flat}]. 
\end{equation}
  
\begin{prop} \label{iso}The morphism $\alpha$ of \eqref{alpha} is an almost isomorphism. Furthermore, it respects the filtrations on both sides. 
\end{prop}

\begin{lemma}\label{algebra} Let $\bk$ be an algebraic closure of $k$. Then $\mathbb A_{\cris} \{\langle u_1,\ldots, u_d\rangle \}|_{\widetilde X_{\bk}}$ has a natural $\cO_X^{\ur+}|_{\widetilde X_{\bk}}$-algebra structure, sending $T_i$ to $u_i+[T_i^{\flat}]$, such that the composition 
\[
\cO_X^{\ur+}|_{\widetilde X_{\bk}}\longrightarrow \mathbb A_{\cris}\{\langle u_1,\ldots, u_d\rangle\}|_{\widetilde X_{\bk}}\stackrel{\theta'|_{\widetilde X_{\bk}}}{\longrightarrow} \widehat{\cO_{X}^{+}}|_{\widetilde X_{\bk}} 
\]
is the map \eqref{eq.OurAlg} composed with $\cO_{X}^{+}\to \widehat{\cO_{X}^+}$.  Here $\theta'\colon \AAcr\{\langle u_1,\ldots, u_d\rangle\}\to \widehat{\cO_X^+}$ is induced from the map $\mathbb A_{\cris}\stackrel{\theta}{\ra}  \widehat{\cO_{X}^{+}}$ by sending $U_i$'s to $0$.
\end{lemma}

\begin{proof} As $\cO_X^{\ur +}$ is the associated sheaf of $\widetilde{\cO_X^{\ur +}}$, and as the affinoids perfectoids lying above $\widetilde{X}_{\bk}$ form a basis for the topology of $X_{\proet}/\widetilde{X}_{\bk}$, we only need to define naturally, for any affinoid perfectoid $U\in X_{\proet}$ lying above $\widetilde{X}_{\bk}$, a morphism of rings:
\[
\widetilde{\cO_{X}^{\ur+}}(U)\longrightarrow (\AAcr^a\{\langle u_1,\ldots, u_d\rangle \})(U),
\]
sending $T_i$ to $u_i+[T_i^{\flat}]$. 

Write $\widehat{U}=\Spa(R,R^{+})$. Let $p^{\flat}\in \cO_{\widehat{\bk}}^{\flat}\subset R^{\flat +}$ be the element given by a compatible system of $p^n$-th roots of $p$. Then we have (see \cite[Proposition 6.1.2]{Bri})
\[
\frac{\mathbb A_{\cris}(R,R^{+})}{(p)}\simeq \frac{(R^{\flat+}/(p^{\flat})^p)[\delta_0,\delta_1,\ldots]}{(\delta_0^p,\delta_1^p,\ldots)},
\]
where $\delta_i$ is the image of $\gamma^{i+1}(\xi)$ with $\gamma: x\mapsto x^p/p$.
So 
\begin{eqnarray*}
\frac{\mathbb A_{\cris}(R,R^{+})\{\langle u_1,\ldots, u_d\rangle\}}{(p)} &\simeq & \frac{(\mathbb A_{\cris}(R,R^{+})/p)[u_i,u_{i,j}: 1\leq i\leq d, j=0,1,\ldots]}{(u_i^p,u_{i,j}^p)} \\ & \simeq & \frac{\frac{R^{\flat+}[u_1,\ldots, u_d]}{((p^{\flat})^p,u_1^p,\ldots, u_d^p)}[\delta_j, u_{i,j}:1\leq i\leq d, j=0,1,\ldots ]}{(\delta_j^p,u_{i,j}^p)}.
\end{eqnarray*}
Set $I:=(p^{\flat},u_1,\ldots, u_d)\subset R^{\flat+}[u_1,\ldots, u_d]/((p^{\flat})^p,u_1^p,\ldots, u_d^p)$: this is an nilpotent ideal with cokernel $R^{\flat+}/(p^{\flat})\simeq R^{+}/p$. Furthermore, there is a canonical morphism of schemes 
\[
\Spec\left(\frac{\mathbb A_{\cris}(R,R^{+})\{\langle u_1,\ldots, u_d\rangle\}}{(p)}\right)\longrightarrow \Spec\left(\frac{R^{\flat+}[u_1,\ldots, u_d]}{((p^{\flat})^p,u_1^p,\ldots, u_d^p)}\right)
\] 
induced from the natural inclusion
$\frac{R^{\flat+}[u_1,\ldots, u_d]}{((p^{\flat})^p,u_1^p,\ldots, u_d^p)}\subset \frac{\mathbb A_{\cris}(R,R^{+})\{\langle u_1,\ldots, u_d\rangle\}}{(p)}$.

Let $\mathcal Y$ be a formal scheme \'etale over $\cX$ together with a factorization $a\colon U\to \cY_k$ as in \eqref{eq.factorization}. We shall construct a natural morphism of $\cO_k$-algebras 
\begin{equation}\label{eq.cY}
\cO_{\cY}(\cY)\longrightarrow \left( \AAcr\{\langle u_1,\ldots, u_d\rangle \}\right) (U)
\end{equation}
sending $T_i$ to $u_i+[T_i^{\flat}]$.  Assume firstly that the image $a(U)$ is contained in the generic fiber $\Spa(A[1/p],A)$ of some affine open subset $\Spf(A)\subset \cY$. In particular, there exists a morphism of $\cO_k$-algebras $A\to R^+$, whence a morphism of $\kappa$-schemes 
\begin{equation}\label{eq.factorization1}
\Spec(R^+/p)\longrightarrow  \Spec(A/p)\longrightarrow \cY\otimes_{\cO_k} \kappa=:\cY_1.
\end{equation}
Composing $\cY\to \cX$ with the \'etale morphism $\cX\to \mathcal T^d$, we obtain an \'etale morphism $\mathcal Y\to \mathcal T^d$, hence an \'etale morphism $\mathcal Y_n\colon =\mathcal Y\otimes \cO_k/p^n\rightarrow \mathcal T^d\otimes \cO_k/p^n=:\mathcal T^{d,n}$ for each $n$. As $T_{i}^{\flat}+u_i\in R^{\flat+}[u_1,\ldots, u_d]/((p^{\flat})^p, u_1^p,\ldots, u_d^p)$ is invertible, one deduces  a map 
\[
\Spec(R^{\flat+}[u_1,\ldots, u_d]/((p^{\flat})^p,u_1^p,\ldots, u_d^p)) \longrightarrow \mathcal T^{d,1},\quad  T_i\mapsto T_{i}^{\flat}+u_i. 
\]
But we have the following commutative diagram 
\[
\xymatrix{\Spec(R^{+}/p)\ar@{=}[r] & \Spec(R^{\flat+}/(p^{\flat})) \ar@{^(->}[d]_{\textrm{nil-immersion}}\ar[r]^<<<<<<{\textrm{\eqref{eq.factorization1}}} & \mathcal Y_1\ar[d]^{\textrm{\'etale morphism}} \\ & \Spec\left(\frac{R^{\flat+}[u_1,\ldots, u_d]}{((p^{\flat})^p, u_1^p,\ldots, u_d^p)}\right) \ar[r]\ar@{.>}[ru]^{\exists}_{g} & \mathcal T^{d,1},}
\]
from which we deduce a morphism, denoted by $g_1$:
\[
g_1: \Spec\left(\frac{\mathbb A_{\cris}(R,R^+)\{\langle u_1,\ldots, u_d\rangle\}}{(p)}\right)\longrightarrow \Spec\left(\frac{R^{\flat+}[u_1,\ldots, u_d]}{((p^{\flat})^p, u_1^p,\ldots, u_d^p)}\right)\stackrel{g}{\longrightarrow} \mathcal Y_1. 
\]
Then we have the following commutative diagram 
\[
\xymatrix{\Spec\left(\frac{\mathbb A_{\cris}(R,R^{+})\{\langle u_1,\ldots, u_d\rangle\}}{(p)}\right)\ar[r]^<<<<<<{g_1}\ar@{^(->}[d]_{\textrm{nil-immersion}} & \mathcal Y_1\ar@^{^(->}[r] & \mathcal Y_n\ar[d]^{\textrm{\'etale morphism}} \\ \Spec\left(\frac{\mathbb A_{\cris}(R,R^+)\{\langle u_1,\ldots, u_d\rangle\}}{(p^n)}\right)\ar[rr]\ar@{.>}[urr]^{\exists}_{g_n} & & \mathcal T^{d,n}.}
\]with the bottom map $T_i\mapsto T_{i}^{\flat}+u_i$.
These $g_n$'s are compatible with each other, so that they give rise to a morphism $\Spf(\mathbb A_{\cris}(R,R^{+})\{\langle u_1,\ldots, u_d\rangle\})\rightarrow \mathcal Y$ of formal schemes over $\cO_k$, inducing a morphism of $\cO_k$-algebras 
\[
\cO_{\cY}(\cY)\rightarrow \AAcr(R,R^+)\{\langle u_1,\ldots, u_d\rangle \}\]
sending $T_i$ to $u_i+[T_i^{\flat}]$. Combining it with  the natural morphism 
\[
\AAcr(R,R^+)\{\langle u_1,\ldots, u_d\rangle \}\rightarrow \AAcr\{\langle u_1,\ldots, u_d\rangle\}(U)\] 
we obtain the desired map \eqref{eq.cY}. For the general case, \emph{i.e.,} without assuming that $a(U)$ is contained in the generic fiber of some affine open of $\cY$, cover $U$ by affinoids perfectoids $V_j$ such that each $a(V_j)$ is contained in the generic fiber of some affine open subset  of $\cY$. The construction above gives, for each $j$, a morphism of $\cO_k$-algebras
\[
\cO_{\cY}(\cY)\longrightarrow \left(\AAcr\{\langle u_1,\ldots, u_d\rangle\}\right)(V_j), \quad T_i\mapsto u_i+[T_i^{\flat}].
\]
As the construction is functorial on the affinoid perfectoid $V_j$, we deduce the morphism \eqref{eq.cY} in the general case using the sheaf property of $\AAcr\{\langle u_1,\ldots, u_d\rangle\}$. 


Finally, since $\widetilde{\cO_{X}^{\ur+}}(U) = \varinjlim \cO_{\mathcal Y}(\cY)$ with the limit runs through the diagrams \eqref{eq.factorization}, we get  a morphism of $\cO_k$-algebras
\[
\widetilde{\cO_{X}^{\ur+}}(U)\longrightarrow \mathbb A_{\cris}(U)\{\langle u_1,\ldots, u_d\rangle\}, \quad T_i\mapsto u_i+[T_i^{\flat}]
\]
which is functorial with respect to affinoid perfectoid $U\in X_{\proet}$ lying above $\widetilde{X}_{\bk}$.
Passing to the associated sheaf, we obtain finally a natural morphism of sheaves of $\cO_{k}$-algebras $\cO_{X}^{\ur}|_{\widetilde{X}_{\bk}}\to \mathbb A_{\cris}|_{\widetilde{X}_{\bk}}\{\langle u_1,\ldots, u_d\rangle\}$ sending $T_i$ to $u_i+[T_i^{\flat}]$. The last statement follows from the assignment $\theta'(U_i)=0$ and the fact that $\theta( [T_i^{\flat}])=T_i$.  
\end{proof}




\begin{proof}[Proof of Proposition~\ref{iso}] As $\widetilde X_{\bk}\to \widetilde X$ is a covering in the pro-\'etale site $X_{\proet}$, we only need to show that $\alpha|_{\widetilde{X}_{\bk}}$ is an almost isomorphism. 
By the lemma above, there exists a morphism of sheaves of $\cO_k$-algebras $
\cO_X^{\ur+}|_{\widetilde{X}_{\bk}}\rightarrow\mathbb A_{\cris}\{\langle u_1,\ldots, u_d\rangle\} |_{\widetilde{X}_{\bk}}
$
sending $T_i$ to $u_i+[T_{i}^{\flat}]$. By extension of scalars, we find the following morphism 
\[
\beta: \cO\AAinf|_{\widetilde{X}_{\bk}}=\left(\cO_{X}^{\ur+}\otimes_{\cO_k} \AAinf \right)|_{\widetilde X_{\bk}}\longrightarrow \mathbb A_{\cris}\{\langle u_1,\ldots, u_d\rangle\}|_{\widetilde X_{\bk}}
\]
which maps $T_i\otimes 1$ to $u_i+[T_{i}^{\flat}]$. Consider the composite (with $\theta'$ as in Lemma \ref{algebra})
\[
\theta'|_{\widetilde X_{\bk}}\circ \beta \colon \cO\AAinf|_{\widetilde X_{\bk}} \longrightarrow \mathbb A_{\cris}\{\langle u_1,\ldots, u_d\rangle\}|_{\widetilde X_{\bk}}{\longrightarrow} \widehat{\cO_{X}^{+}}|_{\widetilde{X}_{\bk}},
\] 
which is simply $\theta_X|_{\widetilde X_{\bk}}$ by Lemma \ref{algebra}. Therefore, $\beta \left(\ker(\theta_X|_{\widetilde X_{\bk}})\right)\subset  \ker( \theta'|_{\widetilde X_{\bk}})$. Since $\ker(\theta')$ has a PD-structure, the map $\beta$ extends to the PD-envelope $\cO\AAcr^0|_{\widetilde X_{\bk}}$ of the source. Furthermore, $\mathbb A_{\cris}\{\langle u_1,\ldots, u_d\rangle\}^a$ is by definition  $p$-adically complete: $\AAcr\{\langle u_1,\ldots, u_d\rangle\}^a\stackrel{\sim}{\to} \varprojlim_n (\AAcr\{\langle u_1,\ldots, u_d\rangle\}^a/p^n)$. The map $\beta$ in turn extends to the $p$-adic completion of $\cO\AAcr^{0,a}|_{\widetilde X_{\bk}}$.  So we obtain the following morphism, still denoted by $\beta$: 
\[
\beta \colon \cO\mathbb{A}_{\cris}^a|_{\widetilde X_{\bk}}\longrightarrow \mathbb A_{\cris}\{\langle u_1,\ldots, u_d\rangle\}^a|_{\widetilde X_{\bk}}, \quad T_i\otimes 1\mapsto u_i+[T_{i}^{\flat}].
\]
Then one shows that $\beta$ and $\alpha $ are inverse to each other, giving the first part of our proposition. 

It remains to check that $\alpha$ respects the filtrations. As $\theta_X|_{\widetilde X}\circ \alpha=\theta'|_{\widetilde X}$ and as $\alpha$ is an almost isomorphism, $\alpha$ induces an almost isomorphism $\alpha\colon \ker(\theta'|_{\widetilde X})\stackrel{\approx}{\to}\ker(\theta_X|_{\widetilde X})$. Therefore, since $\ker(\theta')=\Fil^1(\AAcr\{\langle u_1,\ldots, u_d\rangle\})$ and $\Fil^1\cO\AAcr=\ker(\theta_X)$ (Remark~\ref{rk.Fil1OAcris}), $\alpha$ gives an almost isomorphism 
\[
\alpha\colon \Fil^1 (\AAcr\{\langle u_1,\ldots, u_d\rangle\})^a|_{\widetilde X}\stackrel{\sim}{\longrightarrow}\Fil^1\cO\AAcr^a|_{\widetilde X}. 
\]
As $\Fil^1\cO\AAcr|_{\widetilde X}\subset \cO\AAcr|_{\widetilde X}$ is a PD-ideal, we can consider its $i$-th PD ideal subsheaf $\mathcal I^{[i]}\subset \cO\AAcr|_{\widetilde X}$. Using the almost isomorphism above and the explicit description in Lemma~\ref{lem1forOBcris}, one checks that the $p$-adic completion of $\mathcal I^{[i]}$ is (almost) equal to $\alpha(\Fil^i(\AAcr\{\langle u_1,\ldots, u_d\rangle\})|_{\widetilde X})$. As the image of the morphism 
\[
\Fil^i \cO\AAcr^0|_{\widetilde X}\longrightarrow \cO\AAcr|_{\widetilde X}
\]
is naturally contained in $\mathcal I^{[i]}$, on passing to $p$-adic completion, we obtain 
\[
\Fil^i\cO\AAcr^a|_{\widetilde X}\subset \alpha(\Fil^i\AAcr\{\langle u_1,\ldots, u_d\rangle\}^a|_{\widetilde X}).
 \]
On the other hand, we have the following commutative diagram 
\[
\xymatrix{\Fil^i (\AAcr\{\langle u_1,\ldots, u_d\rangle \})|_{\widetilde X} \ar[r]^<<<<<<<{\alpha} & \cO\AAcr|_{\widetilde X} \\ \varprojlim\frac{\Fil^i(\AAcr^0\langle u_1,\ldots, u_d\rangle)|_{\widetilde X}}{p^n}\ar[r]^<<<<<<{\alpha}\ar[u]^{\cong} & \varprojlim \frac{\Fil^i\cO\AAcr^0|_{\widetilde X}}{p^n}=\Fil^i\cO\AAcr^a|_{\widetilde X}.\ar[u]}
\]
Therefore $\alpha(\Fil^i\AAcr\{\langle u_1,\ldots, u_d\rangle\}|_{\widetilde X})\subset \Fil^i\cO\AAcr|_{\widetilde X}$, whence the equality
\[
\alpha(\Fil^i(\AAcr\{\langle u_1,\ldots, u_d\rangle\})^a|_{\widetilde X})=\Fil^i\cO\AAcr^a|_{\widetilde X}.
\] 
\end{proof}

\begin{cor}\label{BcrisIso} Keep the notation above. There are  natural filtered isomorphisms 
\[
\mathbb B_{\cris}^{+}|_{\widetilde{X}}\{ \langle u_1,\ldots, u_d \rangle\}\stackrel{\sim}{\longrightarrow} \cO\mathbb B_{\cris}^{+}|_{\widetilde X}, \quad \textrm{and}\quad \BBcr|_{\widetilde X}\{\langle u_1,\ldots, u_d \rangle \}\stackrel{\sim}{\longrightarrow} \cO\BBcr|_{\widetilde X}
\] 
both sending $u_i$ to $T_i\otimes 1-1\otimes [T_i^{\flat}]$. 
\end{cor} 

\begin{cor} Let $\cX$ be a smooth formal scheme over $\cO_k$. Then the morphism of multiplication by $t$ on $\cO\AAcr|_{X_{\bk}}$ is almost injective. In particular $t\in \cO\BBcrp|_{X_{\bk}}$ is not a zero-divisor and $\cO\BBcrp\subset \cO\BBcr$. 
\end{cor}

\begin{proof} This is a local question on $X$. Hence we may and do assume there is an \'etale morphism $X\to \Spf(\cO_k\{T_1^{\pm 1},\ldots,T_d^{\pm 1}\})$. Thus our corollary results from Proposition \ref{iso} and Corollary \ref{cor.tNotZeroDiv} (1). 
\end{proof}

An important feature of $\cO\mathbb A_{\cris}$ is that it has an $\AAcr$-linear connection on it. To see this, set $\Omega_{X/k}^{1,\ur+}:=w^{-1}\Omega^1_{\mathcal{X}_{\et}/\cO_k}$, which is  locally free of finite rank over $\cO_X^{\ur+ }$.  Let 
\[
\Omega^{i,\ur+}_{X/k}:=\wedge_{\cO_{X}^{\ur +}}^i \Omega_{X/k}^{1,\ur+}, \quad \textrm{and}\quad \Omega_{X/k}^{i,\ur}:=\Omega_{X/k}^{1,\ur +}[1/p] \quad \forall i\geq 0.
\]
Then $\cO\AAinf$ admits a unique $\AAinf$-linear connection
\[
\nabla\colon \cO\AAinf\longrightarrow \cO\AAinf\otimes_{\cO_X^{\ur+}}\Omega_{X/k}^{1,\ur+}
\] 
induced from the usual one on $\cO_{\cX_{\et}}$. This connection extends uniquely to $\cO\AAcr^0$ and to its completion
\[
\nabla\colon \cO\AAcr\longrightarrow \cO\AAcr\otimes_{\cO_X^{\ur+}}\Omega_{X/k}^{1,\ur+}
\]
This extension is $\AAcr$-linear. Inverting $p$ (resp. $t$), we get  also a $\BBcrp$-linear (resp. $\BBcr$-linear) connection on $\cO\BBcrp$ (resp. on $\cO\BBcr$):
\[
\nabla\colon \cO\BBcrp\longrightarrow \cO\BBcrp\otimes_{\cO_X^{\ur}}\Omega_{X/k}^{1,\ur},\quad  \textrm{and}\quad \nabla\colon \cO\BBcr\longrightarrow \cO\BBcr\otimes_{\cO_X^{\ur}}\Omega_{X/k}^{1,\ur}.
\]
From Proposition \ref{iso}, we obtain

\begin{cor}[Crystalline Poincar\'e lemma]\label{poincare}  Let $\cX$ be a smooth formal scheme of dimension $d$ over $\cO_k$. Then there is  an exact sequence   of pro-\'etale sheaves:
\[
0\to \mathbb B_{\cris}^+\to \cO\mathbb B_{\cris}^{+}\stackrel{\nabla}{\to }\cO\mathbb B_{\cris}^{+}\otimes_{\cO_{X}^{\ur}}\Omega^{1,\ur}_{X/k}\stackrel{\nabla}{\to}\ldots \stackrel{\nabla}{\to} \cO\mathbb B_{\cris}^{+}\otimes_{\cO_{X}^{\ur}}\Omega^{d,\ur}_{X/k} \to 0,
\]
which is strictly exact with respect to the filtration giving $\Omega^{i,\ur}_{X/k}$ degree $i$. In particular, the connection $\nabla$ is integrable and satisfies Griffiths transversality with respect to the filtration on $\cO\mathbb B_{\cris}^{+}$, i.e. $\nabla (\Fil^i\cO\mathbb B_{\cris}^{+})\subset \Fil^{i-1}\cO\mathbb B_{\cris}^{+}\otimes_{\cO_{X}^{\ur}}\Omega^{1,\ur}_{X/k}$. 
\end{cor}



\begin{proof} We just need to establish the almost version of our corollary for $\cO\AAcr$. This is a local question on $X$, hence we may and do assume there is an \'etale morphism from $\cX$ to $\Spf(\cO_k\{T_1^{\pm1},\ldots, T_d^{\pm 1}\})$. Then under the almost isomorphism \eqref{alpha} of Proposition \ref{iso}, $\Fil^i\cO\mathbb{A}_{\cris}^{a}|_{\widetilde X}$ is the $p$-adic completion of 
 \[
 \sum_{i_1,\ldots i_{d}\geq 0}\Fil^{i-(i_0+\ldots +i_{d })}\mathbb A_{\cris}^{a}|_{\widetilde X} u_1^{[i_1]}\cdots u_{d}^{[i_{d}]}
 \]
 with $T_i\otimes 1-1\otimes [T_i^{\flat}]$ sent to $u_i$. Moreover $\nabla(u_i^{[n]})=u_i^{[n-1]}\otimes d T_i$ for any $i, n\geq 1$, since the connection $\nabla$ on $\cO\mathbb A_{\cris}$ is $\mathbb A_{\cris}$-linear. The strict exactness and Griffiths transversality then follow. 
\end{proof}



Using Proposition~\ref{iso}, we can also establish an analogous acyclicity result for $\cO\AAcr$ as in Lemma \ref{vanish}. Let $\cU=\Spf(R^+)$ be an affine subset of $\cX$ admitting an \'etale morphism to $\mathcal T^d=\Spf(\cO_k\{T_1^{\pm 1},\ldots, T_d^{\pm 1}\})$. Let $U$ be the generic fiber, and set $
\widetilde{U}:=U\times_{\mathbb T^d}\widetilde{\mathbb T}^d$. Let $V$ be an affinoid perfectoid of $X_{\proet}$ lying above $\widetilde{U}_{\bk}$. Write $\widehat{V}=\Spa(S,S^+)$. Let $\cO\AAcr(S,S^+)$ be the $p$-adic completion of the PD-envelope $\cO\AAcr^0(S,S^+)$ of $R^+\otimes_{\cO_k}W(S^{\flat+})$ with respect to the kernel of the following morphism  induced from $\theta_{(S,S^+)}$ by extending scalars to $R^+$:
\[
\theta_{R^+}\colon R^{+}\otimes_{\cO_k}W(S^{\flat+})\longrightarrow S^+.
\]
Set $\cO\BBcrp(S,S^+):=\cO\AAcr(S,S^+)[1/p]$, $\cO\BBcr(S,S^+):=\cO\BBcrp(S,S^+)[1/t]$. For $r\in \mathbb Z$, define $\Fil^r\cO\AAcr(S,S^+)$ to be the closure inside $\cO\AAcr(S,S^+)$ for the $p$-adic topology of the $r$-th PD-ideal of $\cO\AAcr^0(S,S^+)$. Finally, set  $\Fil^r\cO\BBcrp(S,S^+):=\Fil^r\cO\AAcr(S,S^+)[1/p]$ and  $
\Fil^r\cO\BBcr(S,S^+):=\sum_{s\in \ZZ}t^{-s}\Fil^{r+s}\cO\BBcrp(S,S^+)$.

\begin{lemma} \label{2obcris} Keep the notation above. For any $\mathcal F\in \{\cO\AAcr, \cO\BBcrp, \cO\BBcr\}$ and any $r\in \mathbb Z$, there exists a  natural almost isomorphism $\Fil^r\mathcal F(S,S^+)\stackrel{\approx}{\to}\Fil^r\mathcal F(V)$. Moreover, $H^i(V,\Fil^r\mathcal F)^a=0$ whenever $i>0$. 
\end{lemma}

\begin{proof}  
Consider the following morphism, again denoted by $\alpha$ 
\[
\alpha\colon 
\AAcr(S,S^+)\{\langle u_1,\ldots, u_d\rangle\}\longrightarrow\cO\AAcr(S,S^+), \quad u_i\mapsto T_i\otimes 1-1\otimes [T_i^{\flat}].
\]
One checks similarly as in Proposition~\ref{iso} that this morphism is an isomorphism. Moreover, if we define $\Fil^r(\AAcr(S,S^+)\{\langle u_1,\ldots, u_d\rangle\})$ to be the $p$-adic completion of 
\[
\sum_{i_1,\ldots, i_d\geq 0} \Fil^{r-(i_1+\ldots +i_d)}\AAcr^0(S,S^+)u_{1}^{[i_1]}\cdots u_d^{[i_d]}\subset \AAcr(S,S^+)\{\langle u_1,\ldots, u_d\rangle \},
\] 
then $\alpha$ respects  the filtrations of both sides. The first part of our lemma can be deduced from the following commutative diagram 
\[
\xymatrix{\Fil^r(\AAcr(S,S^+)\{\langle u_1,\ldots, u_d\rangle\})\ar[r]^<<<<<{\approx}_<<<<<{\alpha}\ar[d]_{\approx} & \Fil^r\cO\AAcr(S,S^+) \ar[d]\\ \Fil^r(\AAcr\{\langle u_1,\ldots, u_d\rangle\})(V)\ar[r]^<<<<<<<{\approx}_<<<<<<<{\alpha} & \Fil^r\cO\AAcr(V)}
\] 
where the left vertical almost isomorphism comes from Lemma~\ref{lem1forOBcris}. Using the last part of Lemma~\ref{lem1forOBcris} and the almost isomorphism $\Fil^r(\AAcr\{\langle u_1,\ldots, u_d\rangle\})|_{\widetilde X}\stackrel{\approx}{\to}\Fil^r\cO\AAcr|_{\widetilde X}$ of Proposition \ref{iso}, we deduce $H^i(V, \Fil^r\cO\AAcr)^a=0$ for $i>0$. Inverting respectively $p$ and $t$, we obtain the statements for $\Fil^r\cO\BBcrp$ and for $\Fil^r\cO\BBcr$. 
\end{proof}






\subsection{Frobenius on crystalline period sheaves}\label{localfrob} 
We keep the notations in the previous \S. So $k$ is \emph{absolutely unramified} and $\cX$ is a smooth formal scheme of dimension $d$ over $\cO_k$. We want to endow Frobenius endomorphisms on the crystalline period sheaves. 

\medskip

On $\AAinf=W(\cO_X^{\flat+})$, we have the Frobenius map 
\[
\varphi\colon \AAinf\longrightarrow \AAinf, \quad (a_0, a_1,\ldots, a_n,\ldots)\mapsto (a_0^p,a_1^p,\ldots, a_n^p,\ldots). 
\]
Then for any $a\in \AAinf$, we have $\varphi(a)\equiv a^p \textrm{ mod }p$. Thus, $\varphi(\xi)=\xi^p+p \cdot b$ with $b\in \AAinf|_{X_{\bk}}$. In particular $\varphi(\xi)\in \AAcr^0|_{X_{\bk}}$ has all divided powers. As a consequence we obtain a Frobenius $\varphi$ on $\AAcr^0$ extending that on $\AAinf$. By continuity, $\varphi$ extends to $\AAcr$ and $\BBcrp$. Note that $\varphi(t)=\log([\epsilon^p])=pt$. Consequently $\varphi$ is extended to $\BBcr$ by setting $\varphi(\frac{1}{t})=\frac{1}{pt}$. 

\medskip 

To endow a Frobenius on $\cO\AAcr$, we first assume that the Frobenius of $\cX_0=\cX\otimes_{\cO_k}\kappa$ lifts to a morphism $\sigma$ on $\cX$, which is compatible with the Frobenius on $\cO_k$. Then for $\mathcal Y\in \cX_{\et}$, consider the following diagram:
\[
\xymatrix{\mathcal Y_{\kappa}\ar@{^(->}[rrr] & & &  \mathcal Y\ar[d]^{\textrm{\'etale}} \\ \mathcal Y_{\kappa}\ar@{^(->}[r]\ar[u]^{\textrm{absolute Frobenius}} & \mathcal Y \ar[r]\ar@{.>}[urr]^{\exists \sigma_{\cY}} & \cX\ar[r]^{\sigma} & \cX.}
\]
As the right vertical map is \'etale, there is a unique dotted morphism above making the diagram commute. When $\cY$ varies in $\cX_{\et}$, the $\sigma_{\cY}$'s give rise to a $\sigma$-semilinear endomorphism on $\cO_{\cX_{\et}}$  whence a $\sigma$-semilinear endomorphism $\varphi$ on $\cO_X^{\ur+}$. 

\begin{rk} In general $\cX$ does not admit a lifting of Frobenius. But as $\cX$ is smooth over $\cO_k$, for each open subset $\cU\subset \cX$ admitting an \'etale morphism $\cU\to \Spf(\cO_k\{T_1^{\pm 1},\ldots, T_d^{\pm 1}\})$, a similar argument as above shows that there exists a unique lifting of Frobenius on $\cU$ mapping $T_i$ to $T_i^p$. 
\end{rk}



We deduce from above a Frobenius on $\cO\mathbb A_{\inf}=\cO_{X}^{\ur+}\otimes_{\cO_k}\AAinf$ given by $\varphi\otimes \varphi$. Abusing notation, we will denote it again  by $\varphi$. A similar argument as in the previous paragraphs shows that $\varphi$ extends to $\cO\AAcr^0$, hence to $\cO\AAcr$ by continuity, and finally to $\cO\mathbb B_{\cris}^{+}$ and $\cO\BBcr$. Moreover, under the almost isomorphism \eqref{alpha}, the Frobenius on $\AAcr\{\langle u_1,\ldots, u_d\rangle \}\stackrel{\approx}{\to}\cO\AAcr$ sends $u_i$ to $\varphi(u_i)=\sigma(T_i)-[T_i^{\flat}]^p$. 


\begin{lemma}\label{FrobHorizontal} Assume as above that the Frobenius of $\cX_0=\cX\otimes_{\cO_k}\kappa$ lifts to a morphism $\sigma$ on $\cX$ compatible with the Frobenius on $\cO_k$. The Frobenius $\varphi$ on $\cO\BBcrp$ is horizontal with respect to the connection $\nabla \colon \cO\BBcrp\to \cO\BBcrp\otimes \Omega^{1,\ur}_{X/k}$. 
\end{lemma}

\begin{proof}We just need to check $\nabla\circ\varphi=(\varphi\otimes d\sigma)\circ\nabla$ on $\cO\mathbb{A}_{\cris}$ in the almost sense. It is enough to do this locally. Thus we may assume there exists an \'etale morphism $\cX\to \Spf(\cO_k\{T_1^{\pm 1},\ldots, T_d^{\pm 1}\})$. Recall the almost isomorphism \eqref{alpha}. By $\mathbb{A}_{\cris}$-linearity
% and using the almost isomorphism \eqref{isop}, 
it reduces to check the equality on the $U_i^{[n]}$.  We have 
\begin{eqnarray*}
(\nabla\circ\varphi)(u_i^{[n]})&=&\nabla(\varphi(u_i)^{[n]})= \varphi(u_i)^{[n-1]}\nabla(\varphi(u_i))
\end{eqnarray*}
Meanwhile, note that $\varphi(u_i)-\sigma(T_i)=-[T_i^{\flat}]^p\in \AAcr$, hence $\nabla(\varphi(u_i))=d\sigma(T_i)$. Thus
\begin{eqnarray*}
\left((\varphi\otimes d\sigma)\circ\nabla\right)(u_i^{[n]})
&=& (\varphi\otimes d\sigma)(u_i^{[n-1]}\otimes dT_i)\\ & =&\varphi(u_i^{[n-1]})\otimes d \sigma(T_i) \\ 
&=&(  \nabla\circ\varphi )(u_i^{[n]}),
\end{eqnarray*}
as desired.
\end{proof}








Clearly, the Frobenius on $\cO\BBcrp$ above depends on the initial lifting of Frobenius on $\cX$. For different choices of liftings of Frobenius on $\cX$, it is possible to compare explicitly the resulting Frobenius endomorphisms on $\cO\mathbb B_{\cris}^{+}$ with the help of the connection on it, at least when the formal scheme $\cX$ is \emph{small}, i.e. when  it admits an \'etale morphism to $\Spf(\cO_k\{T_1^{\pm 1}, \ldots, T_d^{\pm 1}\})$.

\begin{lemma} \label{sigma12}Assume there is an \'etale morphism $\cX\to \Spf(\cO_k\{T_1^{\pm 1}, \ldots, T_d^{\pm 1}\})$. Let $\sigma_1,\sigma_2$ be two Frobenius liftings on $\cX$, and let $\varphi_1$ and $\varphi_2$ be the induced Frobenius maps on $\cO\mathbb B_{\cris}^{+}$, respectively. Then for any quasi-compact $U\in X_{\proet}$, we have the following relation on $\cO\mathbb B_{\cris}^{+}(U)$:
\begin{equation}\label{phi2}
\varphi_2=\sum_{(n_1,\ldots, n_d)\in \NN^d}(\prod_{i=1}^d(\sigma_2(T_i)-\sigma_1(T_i))^{[n_i]})(\varphi_1\circ(\prod_{i=1}^dN_i^{n_i})) \end{equation}
where the $N_i$'s are the endomorphisms of $\cO \mathbb{B}_{\cris}^{+}$ such that $\nabla=\sum_{i=1}^d N_i\otimes d T_i$. 
\end{lemma}

\begin{proof} We only need to check the almost analogue for $\cO\AAcr$. To simplify the notations, we shall  use the multi-index: for $\underline{m}=(m_1,\ldots, u_d)\in \mathbb N^d$, set $\underline N^{\underline m}:=\prod_{i=1}^d N_i^{m_i}$ and $|\underline m|:=\sum_im_i $.  Let us remark first that for any $a\in \cO\AAcr(U)$ and any $r\in \mathbb N$, $\underline N^{\underline m}(a)\in p^r\cdot \cO\AAcr(U)$ when $|m|$ is sufficiently large. As $U$ is quasi-compact, we may and do assume $U$ is affinoid perfectoid with $\widehat{U}=\Spa(R,R^+)$. As $\cO\AAcr$ has no $p$-torsions, up to replacing $a$ by $p\cdot a$, we may and do assume that $a$ is of the form
\[
a=\sum_{\underline m\in \mathbb N^d} b_{\underline m} \cdot \underline{u}^{[\underline m]}, \quad b_{\underline m}\in \AAcr(R,R^+) \textrm{ and }\lim_{|m|\to \infty} b_{\underline m}=0.
\]
Here we have again used the almost isomorphism \eqref{alpha}. An easy calculation shows
\[
\underline N^{\underline n}(a)=\sum_{\underline m\geq \underline n}b_{\underline m}\underline{u}^{[\underline m-\underline n]}=\sum_{\underline m \in \mathbb N^d} b_{\underline m+\underline n} \underline{u}^ {[m]}.
\]
As the coefficient $b_{\underline m}$ tends to $0$ for the $p$-adic topology when $|m|$ goes to infinity, it follows that $\underline N^{\underline n}(a)\in p^r\cdot \cO\AAcr(U)$ when $|n|\gg 0$, as desired. Meanwhile, note that $\sigma_2(T_i)-\sigma_1(T_i)\in p\cO_X^{\ur+}(U)$, hence  their divided powers lie in $\cO_X^{\ur+}(U)$. Therefore the series of the righthand side of \eqref{phi2} applied to $a$ converges.

It remains to verify the formula \eqref{phi2} for any $a\in \cO\AAcr(U)$. Assume again $U$ is affinoid perfectoid. Since both sides of \eqref{phi2} are semilinear with respect to the Frobenius of $\AAcr$, it suffices to check the equality for $a=\underline{u}^{[\underline{m}]}$. In fact, we have 
\[
\begin{array}{cl}
& \left(\sum_{(n_1,\ldots, n_d)\in \NN^d}(\prod_{i=1}^d(\sigma_2(T_i)-\sigma_1(T_i))^{[n_i]})(\varphi_1\circ(\prod_{i=1}^dN_i^{n_i}))\right)(u^{[\underline{m}]}) \\ = & \sum_{\underline n\in \mathbb N^d} (\sigma_2(\underline T)-\sigma_1(\underline T))^{[\underline n]}(\varphi_1(\underline N^{\underline n}(\underline{u}^{[m]}))) \\ = & \sum_{\underline n\in \mathbb N^d} (\varphi_2(\underline{u})-\varphi_1(\underline{u}))^{[\underline n]}(\varphi_1(\underline N^{\underline n}(\underline{u}^{[m]}))) \\ = & \sum_{\underline n\in \mathbb N^d \text{ s.t. }\underline n\leq \underline m} (\varphi_2(\underline{u})-\varphi_1(\underline{u}))^{[\underline n]}\cdot \varphi_1(\underline{u})^{[\underline m-\underline n]} \\ =& (\varphi_2(\underline{u})-\varphi_1(\underline{u})+\varphi_1(\underline{u}))^{[\underline m]} \\ = & \varphi_2(u^{[\underline m]}). 
\end{array} 
\]
This finishes the proof. 
\end{proof}





\subsection{Comparison with de Rham period sheaves}

%The materials in this section will not be needed for the proof of the main theorem. We keep them for the fundamental roles they play in $p$-adic Hodge theory. 
Let $X$ be a locally noetherian adic space over $\mathrm{Spa}(k,\cO_k)$ and recall the map \eqref{theta}. Set  
$\BBdrp=\varprojlim\BBinf/(\ker\theta)^n$, $\BBdr=\BBdrp[1/t]$. For $r\in \mathbb Z$, let $\Fil^r\BBdr=(\ker\theta)^r\BBdrp$. By its very definition, the filtration on $\BBdr$ is decreasing, separated and exhaustive. Similarly, with $\cO\BBinf$ in place of $\BBinf$, we define $\cO\BBdrp$ and $\cO\BBdr$. Define $\Fil^r\cO\BBdrp=(\ker\theta_X)^r\cO\BBdrp$ and $\Fil^r\cO\BBdr=\sum_{s\in \ZZ}t^{-s}\Fil^{r+s}\cO\BBdrp$. The filtration on $\cO\BBdrp$ is decreasing, separated and exhaustive. Moreover, as in \cite[5.2.8, 5.2.9]{Bri}, one can also show that 
\[\cO\BBdrp\cap \Fil^r\cO\BBdr=\Fil^r\cO\BBdrp,\] which in particular implies that the filtration on $\cO\BBdr$ is also (decreasing and) separated and exhaustive.

In the rest of this subsection, assume $k/\Qp$ is absolutely unramified. 


\begin{lemma}\label{BcrisBdR} Let $\cX$ be a smooth formal scheme over $\cO_k$. 
\begin{enumerate}
\item 
There are natural injective morphisms \[\mathbb{B}_{\rm cris}^+\hookrightarrow \BBdr^+,\quad \cO\mathbb{B}_{\rm cris}^+\hookrightarrow \cO\BBdr^+.
\]
In the following, we will view $\mathbb B_{\cris}^{+}$ (resp. $\cO\mathbb B_{\cris}^{+}$) as a subring of $\mathbb B_{\dR}^{+}$ (resp. of $\cO\mathbb B_{\dR}^{+}$). 
\item For  any integer $i\geq 0$, one has  
\[
\Fil^i \mathbb B_{\cris}^{+}=\Fil^i\cO\mathbb B_{\dR}^{+}\cap \mathbb B_{\cris}^{+}\quad \Fil^i\cO\mathbb B_{\cris}^{+}=\Fil^i\cO\mathbb B_{\dR}^{+}\cap \cO\mathbb B_{\cris}^{+}. 
\]
In particular, the filtration $\{\Fil^i \BBcrp:i\in \mathbb Z\}$ (resp. $\{\Fil^i\cO\BBcrp:i\in \mathbb Z\}$) on $\BBcrp$ (resp. on $\cO\BBcrp$) is decreasing, separated and exhaustive. 
\item For $i\geq 0$, the following canonical morphisms are isomorphisms:
\[
\mathrm{gr}^i \mathbb B_{\cris}^{+}\simto\mathrm{gr}^i\mathbb B_{\dR}^{+}, \quad \mathrm{gr}^i\cO\mathbb B_{\cris}^{+}\simto\mathrm{gr}^i\cO\mathbb B_{\dR}^{+}.
\]
\end{enumerate}
\end{lemma}

\begin{proof} (1) 
We will first construct the two natural morphisms claimed in our lemma. Recall that we have the natural mophism $\theta\colon W(\cO_{X}^{\flat+})\to \widehat{\cO_X^+}$, and that $\mathbb B_{\dR}^{+}$ is a sheaf of $\Qp$-algebras. In particular, under the natural morphism 
\[
\mathbb A_{\inf}=W(\cO_{X}^{\flat+})\longrightarrow W(\cO_{X}^{\flat+})[1/p]\longrightarrow \mathbb B_{\dR}^{+},
\]
the ideal of $\mathbb B_{\dR}^{+}$ generated by the image of $\ker(\theta)$ has a PD-structure. Therefore, the composed morphism above extends to a unique morphism  $\AAcr^0\to \mathbb B_{\dR}^{+}$.  On the other hand, for each $n$, the quotient $\mathbb B_{\dR}^{+}/\Fil^n\mathbb B_{\dR}^{+}$ is $p$-adically complete, the composite 
\[
\mathbb A_{\cris}^{0}\longrightarrow \mathbb B^{+}_{\dR}\longrightarrow \mathbb B_{\dR}^{+}/\Fil^n \mathbb B_{\dR}^{+}
\]
factors through the $p$-adic completion $\mathbb A_{\cris}$ of $\mathbb A_{\cris}^{0}$, giving a morphism $\mathbb A_{\cris}\to \mathbb B_{\dR}^{+}/\Fil^n\mathbb B_{\dR}^+$. On passing to limit with respect to $n$, we get a morphism $\mathbb A_{\cris}\to \mathbb B_{\dR}^{+}$, whence the required natural morphism $\mathbb B_{\cris}^{+}\to \mathbb B_{\dR}^{+}$ by inverting $p\in \mathbb A_{\cris}$. The natural morphism from $\cO\mathbb B_{\cris}\to \cO\mathbb B_{\dR}^{+}$ is constructed in a similar way.

The two morphisms constructed above are compatible with the isomorphisms in Corollary \ref{BcrisIso} and its de Rham analogue \cite[Proposition 6.10]{Sch13}. To finish the proof of (1), we only need to show the morphism $\mathbb B_{\cris}^{+}\to \mathbb B_{\dR}^+$ constructed above is injective. Let $\bk$ be an algebraic closure of $k$. We only need to check that for any affinoid perfectoid $U$ lying above $X_{\bk}$, the induced map $\mathbb B_{\cris}^{+}(U)\to \mathbb B_{\dR}^+(U)$ is injective. Write $\widehat U=\Spa(R,R^+)$. Using the identification in Lemma \ref{vanish} together with its de Rham analogue (\cite[Theorem 6.5]{Sch13}), we are reduced to showing that the map $h\colon \mathbb B_{\cris}^{+}(R,R^+)\to \mathbb B_{\dR}^{+}(R,R^+)$ is injective, where $h$ is constructed analogously as the natural map $\mathbb B_{\cris}^+\to \mathbb B_{\dR}^+$ above. This is proved in \cite[Proposition 6.2.1]{Bri}, and we reproduce the proof here for the sake of completeness. 

As $\BBcrp(R,R^+)=\mathbb A_{\cris}(R,R^{+})[1/p]$, we only need to show the following composite $h'$ is injective: 
\[
\xymatrix{h'\colon \  \mathbb A_{\cris}(R,R^+)\ar@{^(->}[r] & \mathbb B_{\cris}^{+}(R,R^+)\ar[r]^{h} & \mathbb B_{\dR}^+(R,R^+)}.
\] 
For $n\geq 0$, let $J^{[n]}\subset \mathbb A_{\cris}(R,R^+)$ denote the closure for the $p$-adic topology of the ideal generated by $\xi^{[i]}=\xi^i/i! $ ($i\geq n$), with $\xi:=[p^{\flat}]-p\in \mathbb A_{\cris}$ generating the kernel of $\theta$. We claim first that ${h'}^{-1}(\Fil^n \mathbb B_{\dR}^{+}(R,R^+))=J^{[n]}$. Clearly only the inclusion ${h'}^{-1}(\Fil^n \mathbb B_{\dR}^{+}(R,R^+))\subset J^{[n]}$ requires  verification. The case $n=1$ is obvious from the definition. For general $n\geq 2$, we will proceed by induction. Let $x\in {h'}^{-1}(\Fil^n\mathbb B_{\dR}^{+}(R,R^+))$. By induction hypothesis, $x\in J^{[n-1]}$. Write $x=x_0\xi^{[n-1]}+x_1$ with $x_0\in W(R^{\flat+})$ and $x_1\in J^{[n]}$. Under the identification 
\[
\frac{\Fil^{n-1}\mathbb B_{\dR}^+(R,R^+)}{\Fil^{n}\mathbb B_{\dR}^+(R,R^+)}\simto R\cdot \xi^{n-1}, \quad \overline{a\xi ^{n-1}}\mapsto \theta(a)\xi^{n-1}, 
\]
the class of $h'(x)\in \Fil^{n-1}\mathbb B_{\dR}^+(R,R^+)$ corresponds to $\theta(x_0)\xi^{n-1}/(n-1)!$. As $h'(x)$ lies in $\Fil^n\mathbb B_{\dR}^+(R,R^+)$, it follows that $\theta(x_0)=0$. So $x_0\in J$ and thus $x=x_0\xi^{[n-1]}+x_1\in J^{[n]}$. Consequently $J^{[n]}=h^{-1}(\Fil^n\mathbb B_{\dR}^+(R,R^+))$.  

 
To conclude the proof of injectivity of $h'$, it remains to show $\bigcap_n J^{[n]}=0$ in $\mathbb A_{\cris}(R,R^{+})$. Recall from the proof of Lemma \ref{algebra} \[
\mathbb A_{\cris}(R,R^{+})/(p)\simeq  \frac{(R^{\flat+}/(p^{\flat})^p)[\delta_0,\delta_1,\ldots]}{(\delta_0^p,\delta_1^p,\ldots)}
\]for $\delta_i$ the image of $\gamma^{i+1}(\xi)$, where  $\gamma(x)=x^p/p$.
Under this isomorphism, for any $n\geq 1$, the image of $J^{[p^n]}$ in $\mathbb A_{\cris}(R,R^{+})/(p)$ is generated by $(\delta_i)_{i\geq n-1}$. This implies that $\bigcap_{i\in \ZZ_{\geq 1}}J^{[i]}\subset p\AAcr (R,R^+)$. Take $x=px'\in \bigcap_{i\geq 1}J^{[i]}$ with $x'\in \AAcr (R,R^+)$. Then $h'(x')\in \bigcap_i \Fil^i\mathbb B_{\dR}^{+}(R,R^+)$ as $p\in \mathbb B_{\cris}^{+}$ is invertible. Thus $x'\in \bigcap_i {h'}^{-1}(\Fil^i\mathbb B_{\dR}(R,R^+))=\bigcap_i J^{[i]}\subset p\mathbb A_{\cris}(R,R^+)$. Hence $x=px'\in p^2\mathbb A_{\cris}(R,R^+)$. Repeating this argument, one sees that $\bigcap_{i\in \ZZ_{\geq 1}}J^{[i]}\subset p^n\AAcr$ for any $n\geq 1$. So $\bigcap_{i\in \ZZ_{\geq 1}}J^{[i]}=\{0\}$ as $\mathbb A_{\cris}$ is $p$-adically separated. Thus we have constructed a natural injection $\mathbb{B}_{\rm cris}^+\hookrightarrow \BBdr^+$. The claim $\cO\mathbb{B}_{\rm cris}^+\hookrightarrow \cO\BBdr^+$ follows from this: this is a local question on $X_{\proet}$, thus we may assume that our formal scheme $\cX$ admits an \'etale map to $\Spf(\cO_k\{T_1^{\pm 1},\ldots, T_d^{\pm 1}\})$, and then conclude by Corollary~\ref{BcrisIso} and its de Rham analogue \cite[Proposition 6.10]{Sch13}.  



(2) To check $\Fil^i\BBcrp=\BBcrp \bigcap \Fil^i\mathbb{B}_{\rm dR}^+$, it suffices  to show $\Fil^i\mathbb{B}_{\cris}^{+}(U)=\mathbb B_{\cris}^{+}(U)\bigcap \Fil^i\mathbb B_{\dR}^{+}(U)$ for any affinoid perfectoid $U$ above $X_{\bk}$. Write $\widehat{U}=\Spa(R,R^{+})$. Under the identifications $\mathbb B_{\dR}^{+}(R,R^+)\simeq\mathbb B_{\dR}^{+}(U)$ and $\mathbb B_{\cris}^{+}(R,R^+)\simeq \mathbb B_{\cris}^{+}(U)$, we have $\Fil^i\mathbb B_{\cris}^{+}(U)=\Fil^i\mathbb B_{\cris}^{+}(R,R^+)$, and $\Fil^i\mathbb B_{\dR}^+(U)=\Fil^i\mathbb B_{\dR}^+(R,R^+)$ (Lemma \ref{vanish}). On the other hand, from the proof of (1), we have 
$J^{[i]}=h^{-1}(\Fil^i \mathbb B_{\dR}(R,R^+))=\mathbb A_{\cris}(R,R^+)\bigcap \Fil^i\mathbb B_{\dR}(R,R^{+})$, from which we deduce 
\[
\Fil^{i}\mathbb B^{+}_{\cris}(R,R^+)=\mathbb B_{\cris}^{+}(R,R^+)\bigcap \Fil^i\mathbb B_{\dR}^{+}(R,R^+),
\]
as desired. 



To show $\Fil^{i}\cO\mathbb B_{\cris}^{+}=\cO\mathbb B_{\cris}^{+}\bigcap \Fil^i\cO\mathbb B_{\dR}^{+}$, we only need to check $\Fil^{i}\cO\mathbb B_{\cris}^{+}\supset \cO\mathbb B_{\cris}^{+}\bigcap\Fil^i\cO\mathbb B_{\dR}^{+}$. For this  we may  again assume the formal scheme $\cX$ admits an \'etale map to $\Spf(\cO_k\{T_1^{\pm 1},\ldots, T_d^{\pm 1}\})$. Then we can conclude by Corollary~\ref{BcrisIso} and its de Rham analogue.  

(3) From (2), we know that the canonical morphism $\mathrm{gr}^i\mathbb B_{\cris}^{+}\to \mathrm{gr}^i\mathbb B_{\dR}^{+}$ is injective. Furthermore we have the identification 
\[
\mathrm{gr}^i\mathbb B_{\dR}^{+}|_{X_{\bk}}=\frac{\Fil^i\mathbb B_{\dR}^{+}|_{X_{\bk}}}{\Fil^{i+1}\mathbb B_{\dR}^{+}|_{X_{\bk}}} \simto \widehat{\cO_{X}}|_{X_{\bk}}\cdot \xi^{i}, \quad \overline{a\xi^i}\mapsto \theta(a)\xi^i. 
\]
As $\xi^i=i!\xi^{[i]}\in \Fil^i\mathbb B_{\cris}^{+}$, the injection $\mathrm{gr}^i\mathbb B_{\cris}^{+}\hookrightarrow \mathrm{gr}^i\mathbb B_{\dR}^{+}$ is also surjective. Thus $\mathrm{gr}^i\mathbb B_{\cris}^{+}\simto \mathrm{gr}^i\mathbb B_{\dR}^{+}$. Using Corollary~\ref{BcrisIso} and its de Rham analogue, we see that the latter isomorphism also implies that $\mathrm{gr}^i\cO\mathbb B_{\cris}^{ +}\simto\mathrm{gr}^i\cO\mathbb B_{\dR}^{+}$. 
\end{proof}



\begin{cor}\label{GradedOfBcris}Let $\cX$ be a smooth formal scheme over $\cO_k$. 
\begin{enumerate}
\item There are two natural injections 
\[
\mathbb B_{\cris}\hookrightarrow \mathbb B_{\dR}, \quad \cO\mathbb B_{\cris}\hookrightarrow \cO\mathbb B_{\dR}.
\]
\item %Define $\Fil^i \mathbb B_{\cris}=\mathbb B_{\cris}\bigcap \Fil^i\mathbb B_{\dR}$ and $\Fil^i \cO\mathbb B_{\cris}=\cO\mathbb B_{\cris}\bigcap \Fil^i\cO\mathbb B_{\dR}$ for $i\in \ZZ_{<0}$.
For any $i\in \mathbb Z$, we have $\Fil^i \mathbb B_{\cris}=\mathbb B_{\cris}\bigcap \Fil^i\mathbb B_{\dR}$ and $\Fil^i \cO\mathbb B_{\cris}=\cO\mathbb B_{\cris}\bigcap \Fil^i\cO\mathbb B_{\dR}$.

 Furthermore, $\mathrm{gr}^i\mathbb B_{\cris}\simto \mathrm{gr}^i\mathbb B_{\dR}$ and $\mathrm{gr}^i\cO\mathbb B_{\cris}\simto\mathrm{gr}^i\cO\mathbb B_{\dR}$. In particular, the filtration  $\{\Fil^i \mathbb B_{\cris}\}_{i\in \ZZ}$ (resp. $\{\Fil^i \cO\mathbb B_{\cris}\}_{i\in \ZZ}$) is decreasing, separated and exhaustive. \end{enumerate}
\end{cor}
 \begin{proof}
These follow from the previous lemma, by inverting $t$.
\end{proof}
 
 
 As a consequence, we can compute the cohomology of the graded quotients $\mathrm{gr}^{i}\mathcal F$ for $\mathcal F\in \{\BBcr^+, \BBcr,\cO\BBcrp, \cO\BBcr\}$: one just reduces to its de Rham analogue such as \cite[Proposition 6.16]{Sch13} etc. 



\begin{cor}\label{higherox}
Let $X$ be a smooth adic space over $\Spa(k,\cO_k)$ which admits a smooth formal model $\cX$ over $\cO_k$ (so that we can define $\cO\mathbb B_{\cris}$), then \[w_*\cO\BBcr\simeq\cO_{\cX_{\et}}[1/p].\]
\end{cor}

\begin{proof} 
Let $\nu: X_{\proet}^{\sim}\ra X_{\et}^{\sim}$ and $\nu' \colon X_{\et}^{\sim}\to \cX_{\et}^{\sim}$ the natural morphisms of topoi. Then $w=\nu'\circ \nu$. Therefore
\[
\cO_{\cX_{\et}}[1/p]\simto\nu_{\ast}'\cO_{X_{\et}}\simto \nu_{\ast}'\nu_{\ast}\cO_X=w_{\ast}\cO_{X}. 
\] By \cite[Corollary 6.19]{Sch13}, the natural morphism $\cO_{X_{\et}}\to \nu_{\ast}\cO\mathbb B_{\dR} $ is an isomorphism. Thus, $w_{\ast}\cO\mathbb B_{\dR}=\nu_{\ast}'(\nu_{\ast}\cO\mathbb B_{\dR})\simeq \nu_*'\cO_{X_{\et}}\simeq \cO_{\cX_{\et}}[1/p]$. 
On the other hand, we have the injection of $\cO_{\cX_{\et}}[1/p]$-algebras $w_{\ast}\cO\mathbb B_{\cris}\hookrightarrow w_{\ast}\cO\mathbb B_{\dR}$. Thereby $\cO_{\cX_{\et}}[1/p]\simto w_{\ast}\cO\BBcr$.  
\end{proof}








\section{Crystalline cohomology and pro-\'etale cohomology} 
 

In this section, we assume $k$ is \emph{absolutely unramified}. Let $\sigma$ denote the Frobenius on $\cO_k$ and on $k$, lifting the Frobenius of the residue field $\kappa$.  Note that the ideal $(p)\subset \cO_k$ is endowed naturally with a PD-structure and $\cO_k$ becomes a PD-ring in this way.  

\subsection{A reminder on convergent $F$-isocrystals} 
Let $\cX_0$ be a $\kappa$-scheme of finite type. %be a smooth formal scheme over $\cO_k$, with $X=\cX_k$ the generic fiber of $\cX$ in the sense of Huber, and $\cX_0$ its closed fiber. 
Let us begin with some general definitions about crystals on the small crystalline site $\left(\mathcal{X}_0/\cO_k\right)_{\cris}$ endowed with \'etale topology. For basics of crystals, we refer to \cite{Ber}, \cite{BO}. Recall that a \emph{crystal of $\cO_{\cX_0/\cO_k}$-modules} is an $\cO_{\cX_0/\cO_k}$-module $\mathbb E$ on $(\cX_0/\cO_k)_{\cris}$ such that (i) for any object $(U,T)\in (\cX_0/\cO_k)_{\cris}$, the restriction $\mathbb E_{T}$ of $\mathbb E$ to the \'etale site of $T$ is a coherent $\cO_T$-module; and (ii) for any morphism $u:(U',T')\ra (U,T)$ in $\left(\mathcal{X}_0/\cO_k\right)_{\cris}$, the canonical morphism $u^*\mathbb{E}_{T}\simto \mathbb{E}_{T'}$ is an isomorphism.



\begin{rk}\label{rk.Crystals} Write  $\cX_0$  the closed fiber of a smooth formal scheme $\cX$ over $\cO_k$. Then the category of crystals on $(\cX_0/\cO_k)_{\cris}$ is equivalent to that of coherent $\cO_{\cX}$-modules $\mathcal M$ equipped with an integrable and quasi-nilpotent connection $\nabla\colon \mathcal M\to \mathcal M\otimes_{\cO_{\cX}}\Omega^1_{\cX/\cO_k}$. Here the connection $\nabla$ is said to be \emph{quasi-nilpotent} if its reduction modulo $p$ is quasi-nilpotent in the sense of \cite[Definition 4.10]{BO}. The correspondence between these two categories is given as follows: for $\mathbb E$ a crystal on $\cX_0/\cO_k$, as $\cX_0\hookrightarrow \cX$ is a $p$-adic PD-thickening, we can evalue $\mathbb E$ at it: set $
\mathbb E_{\cX}:=\varprojlim_n \mathbb E_{\cX\otimes \cO_k/p^n}$. Let $\Delta_1\hookrightarrow \cX\times \cX$ be the PD-thickening of order $1$ of the diagonal embedding $\cX\hookrightarrow \cX\times \cX$. The two projections $p_i\colon \Delta_1\to \cX$ are PD-morphisms. So we have two isomorphisms $p_{i}^{\ast}\mathbb E_{\cX}\simto \mathbb E_{\Delta_1}:=\varprojlim_n \mathbb E_{\Delta_1\otimes \cO_k/p^n}$, whence a natural isomorphism $p_{2}^{\ast}\mathbb E_{\cX}\simto p_{1}^{\ast} \mathbb E_{\cX}$. The latter isomorphism gives a connection $\nabla\colon \mathbb E_{\cX}\rightarrow \mathbb E_{\cX}\otimes \Omega^1_{\cX/\cO_k}$ on $\mathbb E_{\cX}$. Together with a limit argument, that $\nabla$ is integrable and quasi-nilpotent is due to \cite[Theorem 6.6]{BO}. 
\end{rk}


 

The absolute Frobenius $F\colon \cX_0\to \cX_0$ is a morphism over the Frobenius $\sigma$ on $\cO_k$, hence it induces a morphism of topoi, still denoted by $F$:
\[
F \colon \left(\cX_0/\cO_k\right)_{\cris}^{\sim}\longrightarrow \left(\cX_0/\cO_k\right)_{\cris}^{\sim}.\]
An \emph{$F$-crystal} on $(\cX_0/\cO_k)_{\cris}$ is a crystal $\mathbb E$  equipped with a morphism $\varphi\colon F^{\ast}\mathbb E\ra \mathbb E$ of $\cO_{\cX_0/\cO_k}$-modules, which is nondegenerate, i.e. there exists a map $V:\mathbb{E}\ra F^{\ast}\mathbb E$ of $\cO_{\cX_0/\cO_k}$-modules such that $\varphi V=V\varphi=p^m$ for some $m\in \N$. In the following, we will denote by $F\textrm{-Cris}(\cX_0,\cO_k)$ the category of $F$-crystals on $\cX_0/\cO_k$. 

\medskip 
Before discussing isocrystals, let us first observe the following facts. 

\begin{rk}\label{rk.CohSheafOnAdicSpace} Let $X^{\mathrm{rig}}$ be a classical rigid analytic space over $k$, with associated adic space $X$. Using \cite[Theorem 9.1]{Sch13}, one sees that the notion of coherent $\cO_{X^{\mathrm{rig}}}$-modules on $X^{\mathrm{rig}}$ coincides with that of coherent $\cO_{X_{\an}}$-modules on $X_{\an}$, where $X_{\an}$ denote the site of open subsets of the adic space $X$. %Then the latter notion coincides also with that of locally free coherent $\cO_{X_{\an}}$-modules. %(resp. locally free coherent $\cO_X=\cO_{X_{\proet}}$-modules) on $X$; for these equivalences we have used \cite[Lemma 7.3]{Sch13}. 
\end{rk}


\begin{rk}\label{O[1/p]} Let $\cX$ be a smooth formal scheme over $\cO_k$, with $X$ its generic fiber in the sense of Huber. Let $\mathrm{Coh}(\cO_{\cX}[1/p])$ denote the category of coherent $\cO_{\cX}[1/p]$-modules on $\cX$, or equivalently, the full subcategory of the category of $\cO_{\cX}[1/p]$-modules on $\cX$ consisting of $\cO_{\cX}[1/p]$-modules which are isomorphic to $\mathcal M^+[1/p]$ for some coherent sheaf $\mathcal M^+$ on $\cX$. Denote also $\mathrm{Coh}(\cO_{X_{\an}})$ the category of coherent $\cO_{X_{\rm an}}$-modules on $X$. The analytification functor gives a fully faithful embedding $
\mathrm{Coh}(\cO_{\cX}[1/p]) \hookrightarrow \mathrm{Coh}(\cO_{X_{\an}})$.   %whose essential image consists of the coherent $\cO_{X_{\an}}$-modules on $X$ admitting an $\cO_{\cX}$-coherent formal model on $\cX$. %This functor is compatible with Raynaud's functor $\cX\mapsto X^{\rm rig}$ and the Huber's functor from rigid analytic spaces to adic spaces (See \cite{Hub94} and \cite[\S1.9]{Hub} for more details). 
Moreover, the essential image is stable under taking direct summands (as $\cO_{X_{\an}}$-modules). Indeed, let $\cE$ be a coherent $\cO_{X_{\an}}$-module admitting a coherent formal model $\cE^+$ over $\cX$, and $\cE'\subset \cE$ a direct summand. Let $f\colon \cE\to \cE$ be the idempotent corresponding to $\cE'$: write $\cE=\cE'\oplus \cE''$, then $f$ is the composite of the projection from $\cE$ to $\cE'$ followed by the inclusion $\cE'\subset \cE$. Therefore, there exists some $n\gg 0$ such that $p^n f$ comes from a morphism $f^{+}\colon \cE^+\to \cE^+$ of $\cO_{\cX}$-modules. Then the image of $f^+$ gives a formal model of $\cE'$ over $\cX$, as desired. 
\end{rk} 

Let $\cX_0$ be a $\ka$-scheme of finite type. Asumme that it can be embedded as a closed subscheme into a smooth formal scheme $\mathcal P$. Let $P$ be the associated adic space of $\mathcal P$ and $]\cX_0[_{\mathcal P}\subset P$ the pre-image of the closed subset $\cX_0\subset \mathcal P$ under the specialization map. Following \cite[2.3.2 (i)]{Ber} (with Remark \ref{rk.CohSheafOnAdicSpace} in mind), the \emph{realization on $\mathcal P$ of a convergent isocrystal} on $\cX_0/\cO_k$ is a coherent $\cO_{]\cX_0[_{\mathcal P}}$-module %(equivalently, $\cO_{X}=\cO_{X_{\proet}}$-module) 
$\mathcal E$ equipped with an integrable and \emph{convergent} connection $\nabla\colon \mathcal E\rightarrow \mathcal E\otimes_{\cO_{]\cX_0[_{\mathcal P}}} \Omega^{1}_{]\cX_0[_{\mathcal P}/k}$ (we refer to \cite[2.2.5]{Ber} for the definition of convergent connections). Being a coherent $\cO_{]\cX_0[_{\mathcal P}}$-module with integrable connection, $\cE$ is locally free of finite rank by \cite[2.2.3 (ii)]{Ber}. The category of realizations on $\cP$ of convergent isocrystals on $\cX_0/\cO_k$ is denoted by $\mathrm{Isoc}^{\dagger}(\cX_0/\cO_k,\cP)$, where the morphisms are morphisms of $\cO_{]\cX_0[_{\mathcal P}}$-modules which commute with connections. 

Let $\cX_0\hookrightarrow \cP' $ be a second embedding of $\cX_0$ into a smooth formal scheme $\cP' $ over $\cO_k$, and assume there exists a morphism $u\colon \cP'\to \cP $ of formal schemes inducing identity on $\cX_0$. The generic fiber of $u$ gives a morphism of adic spaces $u_k\colon ]\cX_0[_{\mathcal P'}\to ]\cX_0[_{\mathcal P}$, hence a natural functor 
\[
u_k^{\ast}\colon \mathrm{Isoc}^{\dagger}(\cX_0/\cO_k,\cP) \longrightarrow \mathrm{Isoc}^{\dagger}(\cX_0/\cO_k,\cP'), \quad (\cE,\nabla)\mapsto (u_k^{\ast}\cE, u_k^{\ast}\nabla).
\]
By \cite[2.3.2 (i)]{Ber}, the functor $u_k^{\ast}$ is an equivalence of categories. Furthermore, for a second morphism $v\colon \cP'\to\cP$ of formal schemes inducing identity on $\cX_0$, the two equivalence $u_k^{\ast},v_k^{\ast}$ are canonically isomorphic (\cite[2.2.17 (i)]{Ber}). Now the category of \emph{convergent isocrystal on $\cX_0/\cO_k$}, denoted by $\mathrm{Isoc}^{\dagger}(\cX_0/\cO_k)$, is defined as \[
\mathrm{Isoc}^{\dagger}(\cX_0/\cO_k):=\textrm{2-}\!\varinjlim_{\cP} \mathrm{Isoc}^{\dagger}(\cX_0/\cO_k,\cP),
\] where the limit runs through all smooth formal embedding $\cX_0\hookrightarrow \cP$ of $\cX_0$.   

\begin{rk} In general, $\cX_0$ does not necessarily admit a global formal embedding. In this case, the category of convergent isocrystals on $\cX_0/\cO_k$ can still be defined by a gluing argument (see \cite[2.3.2(iii)]{Ber}). But the definition recalled above will be enough for our purpose. 
\end{rk}


As for the category of crystals on $\cX_0/\cO_k$, the Frobenius morphism $F\colon \cX_0\to \cX_0$ induces a natural functor (see \cite[2.3.7]{Ber} for the construction):
\[
F^{\ast}\colon \mathrm{Isoc}^{\dagger}(\cX_0/\cO_k)\longrightarrow \mathrm{Isoc}^{\dagger}(\cX_0/\cO_k). 
\]

A \emph{convergent $F$-isocrystal} on $\cX_0/\cO_k$ is a convergent isocrystal $\mathcal E$ on $\cX_0/\cO_k$ equipped with an isomorphism  $F^*\mathcal E \simto \mathcal E$ in $\mathrm{Isoc}^{\dagger}(\cX_0/\cO_k)$. The category of convergent $F$-isocrystals on $\cX_0/\cO_k$ will be denoted in the following by $F\textrm{-Isoc}^{\dagger}(\cX_0/\cO_k)$.

\begin{rk} \label{rk.CrystalVSIsoc}The category $F\textrm{-Isoc}^{\dagger}(\cX_0/\cO_k)$ has as a full subcategory the isogeny category $F\textrm{-Cris}(\cX_0/\cO_k)\otimes\mathbb Q$ of $F$-crystals $\mathbb E$ on $\left(\mathcal{X}_0/\cO_k\right)_{\cris}$. To explain this, assume for simplicity that $\cX_0$ is the closed fiber of a smooth formal scheme $\cX$ over $\cO_k$. So $]\cX_0[_{\cX}=X$, the generic fiber of $\cX$. Let $(\cM,\nabla)$ be the $\cO_{\cX}$-module with integrable and quasi-nilpotent connection associated to the $F$-crystal $\mathbb E$ (Remark \ref{rk.Crystals}). Let $\mathbb E^{\an}:=\cM_{\nu}$ denote the generic fiber of $\cM$, which is a coherent (hence locally free by \cite[2.3.2 (ii)]{Ber}) $\cO_{X_{\an}}$-module equipped with an integrable connection $\nabla^{\rm an}\colon \mathbb E^{\rm an}\longrightarrow \mathbb E^{\rm an}\otimes \Omega^{1}_{X_{\an}/k}$, which is nothing but the generic fiber of $\nabla$. Because of the $F$-crystal structure on $\mathbb E$, the connection $\nabla^{\rm an}$ is necessarily convergent (\cite[2.4.1]{Ber}). In this way we obtain an $F$-isocrystal $\mathbb E^{\mathrm{an}}$ on $\cX_0/\cO_k$, whence a natural functor 
\begin{equation}\label{eq.AnFunctor}
(-)^{\rm an}\colon F\textrm{-Cris}(\cX_0/\cO_k) \otimes \mathbb Q\longrightarrow F\textrm{-Isoc}^{\dagger}(\cX_0/\cO_k), \quad \mathbb E\mapsto \mathbb E^{\rm an}.
\end{equation}
By \cite[2.4.2]{Ber}, this analytification functor is fully faithful, and for $\mathcal E$ a convergent $F$-isocrystal on $\cX_0/\cO_k$, there exists an integer $n\geq 0$ and an $F$-crystal $\mathbb E$ such that $\mathcal E\simto \mathbb E^{\rm  an}(n)$, where for $\mathcal F=(\mathcal F, \nabla, \varphi\colon F^{\ast}\mathcal F\simto \mathcal F)$ an $F$-isocrystal on $\cX_0/\cO_k$, $\mathcal F(n)$ denotes the Tate twist of $\mathcal F$, given by $(\mathcal F, \nabla, \frac{\varphi}{p^n}\colon F^{\ast}\mathcal F\simto \mathcal F)$  (\cite[2.3.8 (i)]{Ber}).  %Via this embedding, we may sometimes call the $\cO_{\cX}[1/p]$-module $\cM[1/p]$ a convergent $F$-isocrystal. 

%Recall the natural morphism $w: X_{\proet}\ra \cX_{\et}$ and the sub sheaf $\cO_{X}^{\ur}=w^{-1}\cO_{\cX}[1/p]$ of $\cO_X$. Thus, the $\cO_X$-module $\cE$ together with the connection $\nabla\colon \mathcal E\longrightarrow \mathcal E\otimes_{\cO_{X}} \Omega^{1}_{X/k}$ descends to $w^{-1}\cM$ over $\cO_{X}^{\ur}$, with the connection \[\nabla: w^{-1}\cM\ra w^{-1}\cM\otimes_{\cO_X^{\ur}}\Omega_X^{1,\ur}.\]
\end{rk}

Our next goal is to give a more explicit description of the Frobenii on convergent $F$-isocrystals on $\cX_0/\cO_k$. From now on, assume for simplicity that \emph{$\cX_0$ is the closed fiber of a smooth formal scheme $\cX$} and we shall identify the notion of convergent iscrystals on $\cX_0/\cO_k$ with its realizations on $\cX$. Let $X$ be the generic fiber of $\cX$. The proof of the following lemma is obvious.

\begin{lemma}\label{lem.FCrystal} Assume that the Frobenius $F\colon \cX_0\to \cX_0$  can be lifted to a morphism $\sigma \colon \cX\to \cX$ compatible with the Frobenius on $\cO_k$.  Still denote by $\sigma$ the endomorphism on $X$ induced by $\sigma$. Then there is an equivalence of categories between 
\begin{enumerate}
\item the category $F\textrm{-}\mathrm{Isoc}^{\dagger}(\cX_0/\cO_k)$ of convergent $F$-isocrystals on $\cX_0/\cO_k$; and 
\item the category $\mathbf{Mod}_{\cO_{X}}^{\sigma,\nabla}$ of  $\cO_{X_{\an}}$-vector bundles $\mathcal E$ equipped with an integrable and convergent connection $\nabla$ and an $\cO_{X_{\an}}$-linear horizontal isomorphism $\varphi\colon \sigma^*\cE\to \cE$. 
\end{enumerate} 
\end{lemma}

%\begin{proof} 
%This follows directly from the definitions and functoriality; see \cite[2.3.2 (iv)]{Ber}. 
%\end{proof}

 


Consider two liftings of Froebnius $\sigma_i$ ($i=1,2$) on $\cX$. By the lemma above, for $i=1,2$, both categories $\mathbf{Mod}_{\cO_{X}}^{\sigma_i,\nabla}$ are naturally equivalent to the category of convergent $F$-iscrystals on $\cX_0/\cO_k$:
\[
\xymatrix{\mathbf{Mod}_{\cO_{X}}^{\sigma_1,\nabla} & F\textrm{-}\mathrm{Isoc}^{\dagger}(\cX_0/\cO_k)\ar[r]^{\sim}\ar[l]_{\sim} & \mathbf{Mod}_{\cO_{X}}^{\sigma_2,\nabla}}.
\]
Therefore we deduce a natural equivalence of categories
\begin{equation}\label{eq.FunctorF}
F_{\sigma_1,\sigma_2}\colon \mathbf{Mod}_{\cO_{X}}^{\sigma_1,\nabla} \longrightarrow \mathbf{Mod}_{\cO_{X}}^{\sigma_2,\nabla}.
\end{equation}
When our formal scheme $\cX$ is small, we can explicitly describe this equivalence as follows.  Assume there is an \'etale morphism $\cX\to \mathcal T^d=\Spf(\cO_k\{T_1^{\pm 1},\ldots, T_d^{\pm 1}\})$. So $\Omega_{X_{\an}/k}^{1}$ is a free $\cO_{X_{\an}}$-module with a basis given by $dT_i$ ($i=1,\ldots, d$). In the following, for $\nabla$ a connection on an $\cO_{X_{\an}}$-module $\mathcal E$, let $N_i$ be the endomorphism of $\cE$ (as an abelian sheaf) such that  $\nabla=\sum_{i=1}^{d}N_i\otimes dT_i$. 



\begin{lemma}[see also \cite{Bri} 7.2.3] \label{fsigma12}

Assume $\cX=\Spf(A)$ is affine  and that there exists an \'etale morphism $\cX\to \mathcal T^d$ as above. For $(\mathcal E,\nabla, \varphi_1)\in \mathbf{Mod}_{\cO_{X}}^{\sigma_1,\nabla}$, with $(\mathcal E,\nabla, \varphi_2)$ the corresponding object of $\mathbf{Mod}_{\cO_{X}}^{\sigma_2,\nabla}$ under the functor $F_{\sigma_1,\sigma_2}$. Then on $\cE(X)$ we have 
\begin{equation}\label{eq.CompFrob}
\varphi_2=\sum_{(n_1,\ldots, n_d)\in \NN^d}\left(\prod_{i=1}^d(\sigma_2(T_i)-\sigma_1(T_i))^{[n_i]}\right)\left(\varphi_1\circ \left(\prod_{i=1}^dN_i^{n_i}\right)\right).
\end{equation}
Furthermore, $\varphi_1$ and $\varphi_2$ coincide on $\cE(X)^{\nabla=0}$. 
\end{lemma}


\begin{proof} To simplify the notations, we shall use the multi-index: so $\underline N^{\underline n}:=\prod_n N_i^{n_i} $ for $\underline{n}=(n_1,\ldots, n_d)\in \mathbb N^d$ etc. We observe first that the right hand side of \eqref{eq.CompFrob} converges. Indeed, by \cite[2.4.2]{Ber}, there exists some $n\in \mathbb N$ such that $\mathcal E(-n)$ lies in the essential image of the  functor \eqref{eq.AnFunctor}. In particular, this implies that there exists a coherent $\cO_{\cX}$-module $\cE^{+}$ equipped with a quasi-nilpotent connection $\nabla^{+}$ such that $(\cE,\nabla)$ is the generic fiber of $(\cE^{+},\nabla^{+})$. In particular, for any $e\in E^+:=\cE^+(\cX)$, $\underline N^{\underline n}(e)\in p\cdot E^{+}$ for all but finitely many $\underline n\in \mathbb N^d$. Furthermore, as $\sigma_1,\sigma_2$ are liftings of Frobenius, $\sigma_2(T_i)-\sigma_1(T_i)\in p\cdot A$. So the divided power $(\sigma_2(T_i)-\sigma_1(T_i))^{[n_i]}\in A$. Thereby the right hand side of \eqref{eq.CompFrob} applied to an element of $\cE(X)=E^{+}[1/p]$ converges for the cofinite filter on $\mathbb N^d$. 

Next we claim that the equality \eqref{eq.CompFrob} holds when $\cE=\cO_{X_{\an}}$ endowed with the natural structure of $F$-isocrystal. Indeed, necessarily $\varphi_i=\sigma_i$ in this case. Let $\sigma_2' $ be the endomorphism of $A$ defined by the right hand side of \eqref{eq.CompFrob}. Then both $\sigma_2, \sigma_2' $ are liftings of Frobenius on $A$, and it is elementary to check that they coincide on the $\cO_k$-subalgbra $\cO_k\{T_1^{\pm 1},\ldots, T_d^{\pm 1}\}$. As $A$ is \'etale over $\cO_k\{T_1^{\pm 1}, \ldots, T_d^{\pm 1}\}$, we deduce $\sigma_2=\sigma_2'$, giving our claim.  By consequence, in the general case, the right hand side of \eqref{eq.CompFrob} defines a $\sigma_2$-semilinear endomorphism $\varphi_2' $ on $\cE(X)$. In particular, we obtain a morphism of $\cO_{X_{\an}}$-modules $\sigma_{2}^{\ast}\cE\to \cE$, still denoted by $\varphi_2'$. One checks that $\varphi_2'\colon \sigma_2^{\ast}\cE\to \cE $ is horizontal, hence it is a morphism of convergent isocrystals.  

Now one needs to verify the equality \eqref{eq.CompFrob}.  Consider the fiber product $\cX\times \cX$ and its generic fiber $X\times X$. Let $U=]\cX_0[_{\cX\times \cX}\subset X\times X$ denote the pre-image of the closed subset $\delta(\cX_0)\subset \cX_0\times \cX_0$ under the specialization map, and $q_{i}\colon U\to X$ ($i=1,2$) the two projections. If we endow $\cX\times \cX$ with the Frobenius $\sigma:=\sigma_1\times \sigma_2$, then $\sigma_i\circ q_i=q_i\circ \sigma$. In particular, the pull-back $q_{1}^{\ast}\cE$ is endowed with a morphism 
\[
q_1^{\ast}\varphi_1\colon \sigma^{\ast}q_1^{\ast}\cE=q_1^{\ast}\sigma_1^{\ast}\cE\longrightarrow q_1^{\ast}\cE.  
\]
Similarly, $q_2^{\ast}\cE$ is a convergent isocrystal on $U$ endowed with a horizontal morphism $q_2^{\ast}\varphi_2' $. Consider the following $\cO_{U_{\an}}$-linear isomorphism induced by the connection on $\cE$:
\[
\eta\colon \cO_{U_{\an}} \otimes_{\cO_{X_{\an}}} \cE \stackrel{\sim}{\longrightarrow} \cE\otimes_{\cO_{X_{\an}}}\cO_{U_{\an}}, \quad 1\otimes m\mapsto \sum_{\underline{n}\in \mathbb N^d}\frac{\underline{N}^{\underline n}(m)\otimes \underline{\tau}^{\underline n}}{\underline n!},
\]
where $\underline{\tau}=(\tau_1,\ldots, \tau_d)$ with $\tau_i=1\otimes T_i-T_i\otimes 1$ (note that this formula makes sense as the connection on $\cE$ is convergent (see \cite[2.2.5]{Ber})). By a direct calculation, one checks the commutativity of the following diagram:
\[
\xymatrix{\sigma^{\ast}(\cO_{U_{\an}}\otimes \cE) \ar[rr]^{q_2^{\ast}\varphi_2'}\ar[d]_{\sigma^{\ast}(\eta)} &  & \cO_{U_{\an}}\otimes \cE \ar[d]^{\eta} \\ \sigma^{\ast}(\cE\otimes \cO_{U_{\an}})\ar[rr]_{q_1^{\ast}\varphi_1}&  &  \cE\otimes \cO_{U_{\an}}.}
\]  
Indeed, taking $m\in \mathcal E$, we find 
\[
\sum_{\underline{n}\in \mathbb N^d}\frac{\underline{N}^{\underline n}(\varphi_2'(m))\otimes \underline{\tau}^{\underline n}}{\underline n!}=\sum_{\underline{n}\in \mathbb N^d}\frac{\varphi_1(\underline{N}^{\underline n}(m))\otimes \widehat{\sigma}(\underline{\tau}^{\underline n})}{\underline n!}.
\]
Finally, let $\cI=(\tau_1,\ldots, \tau_d)$ be the ideal defining the closed immersion $X\hookrightarrow U$. Then modulo the ideal $\cI$ in the above equality, the left hand side becomes just $\varphi_2'(m)$. Moreover, as $\sigma(\tau_i)=\sigma_2(T_i)-\sigma_1(T_i)$, we find the following equality in $\cE$:
\[
\varphi_2(m)=\sum_{\underline{n}\in \mathbb N^d}\frac{\varphi_1(\underline{N}^{\underline n}(m))\cdot \prod_{i}(\sigma_2(T_i)-\sigma_1(T_i))^{n_i}}{\underline n!},
\]
as desired.

In particular, $\varphi_2' \colon \sigma_2^{\ast}\cE\to \cE$ is a horizontal isomorphism, hence $(\cE, \nabla, \varphi_2')$ is an $F$-isocrystal. Moreover $q_1^{\ast}(\cE,\nabla, \varphi_1)$ and $q_2^{\ast}(\cE,\nabla,\varphi_2')$ are isomorphic as realizations of $F$-isocrystals on the embedding $\cX_0\hookrightarrow \cX\times \cX$. Therefore, by definition \cite[2.3.2]{Ber}, $(\cE,\nabla,\varphi_1)$ and $(\cE,\nabla, \varphi_2')$ realize the same $F$-isocrystal on $\cX_0/\cO_k$. 


The last statement of our lemma is clear from the formula just proved as $\cE(X)^{\nabla=0}=\bigcap_{i=1}^{d} \ker(N_i)$.  
\end{proof}

 
More generally, i.e. without assuming the existence of Frobenius lifts to $\cX$,  for $(\cE,\nabla)$ an $\cO_{X_{\an}}$-module with integrable and convergent connection, \emph{a compatible system of Frobenii on $\cE$} consists of, for any open subset $\cU\subset \cX$ equipped with a lifting of Frobenius $\sigma_{\cU}$, a horizontal isomorphism $\varphi_{(\cU,\sigma_{\cU})}\colon \sigma_{\cU}^{\ast} \cE|_{\cU_k}\to \cE|_{\cU_k}$ satisfying the following condition: for $\cV\subset \cX$ another open subset equipped with a lifting of Frobenius $\sigma_{\cV}$, the functor 
\[
F_{\sigma_{\cU},\sigma_{\cV}} \colon \mathbf{Mod}_{\cO_{\cU_k \bigcap \cV_k}}^{\sigma_{\cU},\nabla} \longrightarrow \mathbf{Mod}_{\cO_{\cU_k\bigcap \cV_k}}^{\sigma_{\cV},\nabla}
\]
sends $(\cE|_{\cU_k\bigcap \cV_k}, \nabla, \varphi_{(\cU,\sigma_{\cU})}|_{\mathcal U_k\bigcap \cV_k})$ to $(\cE|_{\mathcal U_k\bigcap \cV_k}, \nabla, \varphi_{(\cV,\sigma_{\cV})}|_{\mathcal U_k\bigcap \cV_k})$. We denote a compatible system of Frobenii on $\cE$ by the symbol $\varphi$, when no confusion arises. 
Let $\mathbf{Mod}_{\cO_{X}}^{\sigma,\nabla}$ be the category of $\cO_{X_{\an}}$-vector bundles equipped with an integrable and convergent connection, and with a compatible system of Frobenii. The morphism in $\mathbf{Mod}_{\cO_{X}}^{\sigma,\nabla}$ are the morphisms of $\cO_{X_{\an}}$-modules which commute with the connections, and with the Frobenius morphisms on any open subset $\mathcal U\subset \cX$ equipped with a lifting of Frobenius. 

\begin{rk} Let $\cE$ be a convergent isocrystal on $\cX_0/\cO_k$. To define a compatible system of Frobenii on $\cE$, we only need to give, for a cover $\cX=\bigcup_i \cU_i$ of $\cX$ by open subsets $\cU_i$ equipped with a lifting of Frobenius $\sigma_{i}$, a family of Frobenius morphisms $\varphi_{i}\colon \sigma_i^{\ast}\cE|_{U_i} \stackrel{\sim}{\to} \cE|_{U_i}$ such that $\varphi_i|_{U_{i}\bigcap U_{j}}$ corresponds to $\varphi_j|_{U_{i}\bigcap U_{j}}$ under the functor $F_{\sigma_i,\sigma_j}\colon \mathbf{Mod}_{\cO_{U_{i}\bigcap U_{j}}}^{\sigma_i,\nabla}\to  \mathbf{Mod}_{\cO_{U_{i}\bigcap U_{j}}}^{\sigma_j,\nabla}$ (Here $U_{\bullet}:=\cU_{\bullet,k}$). Indeed, for $\cU$ any open subset equipped with a lifting of Frobenius $\sigma_{\cU}$, one can first use the functor $F_{\sigma_i,\sigma_{\cU}}$ of \eqref{eq.FunctorF} applied to $(\cE|_{U_i}, \nabla|_{U_i}, \varphi_i)|_{U_i\bigcap U}$ to obtain a horizontal isomorphism $\varphi_{\cU,i}\colon (\sigma_{\cU}^{\ast}(\cE|_{\cU}))|_{U_i\bigcap U} \to \cE|_{U_i\bigcap U}$. From the compatibility of the $\varphi_i$'s, we deduce $
\varphi_{\cU,i}|_{U\bigcap U_{i}\bigcap U_{j}}=\varphi_{\cU,j}|_{U\bigcap U_{i}\bigcap U_{j}}$. 
Consequently we can glue the $\varphi_{\cU,i}$'s ($i\in I$) to get a horizontal isomorphism $\varphi_{\cU}\colon \sigma_{\cU}^{\ast}(\cE|_{U})\to \cE|_{U}$. One checks that these $\varphi_{\cU}$'s give the desired compatible system of Frobenii on $\cE$. 
\end{rk}

Let $\cE$ be a convergent $F$-isocrystal on $\cX_0/\cO_k$. For $\cU\subset \cX$ an open subset equipped with a lifting of Frobenius $\sigma_{\cU}$, the restriction $\cE|_{\cU_k}$ gives rise to  a convergent $F$-isocrystal on $\cU_0/k$. Thus there exists a $\nabla$-horizontal isomorphism $\varphi_{(\cU, \sigma_{\cU})}\colon \sigma_{\cU}^{\ast}\cE|_{\cU_k}\to \cE|_{\cU_k}$. Varying $(\cU,\sigma_{\cU})$ we obtain a compatible system of Frobenii $\varphi$ on $\cE$. In this way, $(\cE,\nabla,\varphi)$ becomes an object of $\mathbf{Mod}_{\cO_{X}}^{\sigma, \nabla}$. Directly from the definition, we have the following 

\begin{cor} The natural functor $F\textrm{-}\mathrm{Isoc}^{\dagger}(\cX_0/\cO_k)\to \mathbf{Mod}_{\cO_{X}}^{\sigma,\nabla}$ is an equivalence of categories. 
\end{cor}


 In the following, denote by $\mathbf{FMod}_{\cO_{X}}^{\sigma,\nabla}$ the category of  quadruples $(\cE, \nabla, \varphi, \Fil^{\bullet}(\cE))$ with $(\cE, \nabla, \varphi)\in \mathbf{Mod}_{\cO_{X}}^{\sigma,\nabla}$ and  a decreasing, separated and exhaustive filtration $\Fil^{\bullet}(\cE)$ on $\cE$ by locally free direct summands, such that $\nabla$ satisfies Griffiths transversality with respect to $\Fil^{\bullet}(\cE)$, \emph{i.e.,} $
\nabla (\Fil^{i}(\cE))\subset \Fil^{i-1}(\cE)\otimes_{\cO_{X_{\an}}}\Omega_{X_{\an}/k}^{1}$.
The morphisms are the morphisms in $\mathbf{Mod}_{\cO_{X}}^{\sigma,\nabla}$ which respect the filtrations. We call the objects in $\mathbf{FMod}_{\cO_{X}}^{\sigma,\nabla}$ \emph{filtered (convergent) $F$-isocrystals} on $\cX_0/\cO_k$. By analogy with the category $F\textrm{-}\mathrm{Isoc}^{\dagger}(\cX_0/\cO_k)$ of $F$-isocrystals, we also denote the category of filtered $F$-isocrystals on $\cX_0/\cO_k$ by $FF\textrm{-}\mathrm{Isoc}^{\dagger}(\cX_0/\cO_k)$.

 
\medskip

%Denote by $FF\!-\!\mathrm{Cris}(\cX_0/\cO_k)$ the category of filtered $F$-crystals on  $\left(\mathcal{X}_0/\cO_k\right)_{\cris}$.\begin{prop} The natural functor $FF\!-\!\mathrm{Cris}(\cX_0/\cO_k)\to \mathbf{FMod}_{\cO_{X}}^{\sigma,\nabla}$ is an equivalence of categories. \end{prop}




\subsection{Lisse $\widehat{\ZZ}_p$-sheaves and filtered $F$-isocystals}

Let $\cX$ be a smooth formal scheme over $\cO_k$ with $X$ its generic fiber in the sense of Huber. Define $\widehat{\ZZ}_p:=\varprojlim \ZZ/p^n$ and $\widehat{\QQ}_p:=\widehat{\ZZ}_p[1/p]$ as sheaves on $X_{\proet}$. Recall that 
a \emph{lisse $\Zp$-sheaf} on $X_{\et}$ is an inverse system of sheaves of $\ZZ/p^n$-modules $\LL_{\bullet}=(\LL_n)_{n\in \mathbb{N}}$ on $X_{\et}$ such that each $\LL_n$ is locally a constant sheaf associated to a finitely generated $\ZZ/p^n$-modules, and such that the inverse system is isomorphic in the pro-category to an inverse system for which $\LL_{n+1}/p^n\simeq \LL_n$. A \emph{lisse $\widehat{\ZZ}_p$-sheaf} on $X_{\proet}$ is a sheaf of $\widehat{\ZZ}_p$-modules on $X_{\proet}$, which  is locally isomorphic to $\widehat{\ZZ}_p\otimes_{\Zp} M$ where $M$ is a finitely generated $\Zp$-module. By \cite[Proposition 8.2]{Sch13} these two notions are equivalent via the functor $\nu^{\ast}\colon X_{\et}^{\sim}\to X_{\proet}^{\sim}$. In the following, we use the natural  morphism of topoi $w\colon X_{\proet}^{\sim}\stackrel{\nu}{\ra} X_{\et}^{\sim}\rightarrow \cX_{\et}^{\sim}$ frequently. Before defining crystalline sheaves let us first make the following observation. %By \cite[Corollary 3.17]{Sch13} the functor $\nu$ induces a fully faithful embedding $\Sh(X_{\et})\hookrightarrow\Sh(X_{\proet})$.

\begin{rk}\label{rk.wE} (1) Let $\mathcal M$ be a crystal on $\cX_0/\cO_k$, viewed as a coherent $\cO_{\cX}$-module admitting an integrable connection. Then $w^{-1}\mathcal M$ is a coherent $\cO_X^{\ur+}$-module with an integrable connection $w^{-1}\mathcal M\to w^{-1}\mathcal{M}\otimes_{\cO_X^{\ur+}}\Omega_{X/k}^{1,\ur+}$. If furthermore $\cM$ is an $F$-crystal, then $w^{-1}\mathcal M$ inherits a system of Frobenii: for any open subset $\cU\subset \cX$ equipped with a lifting of Frobenius $\sigma_{\cU}$, there is naturally an endomorphism of $w^{-1}\cM|_{U}$ which is semilinear with respect to the Frobenius $w^{-1}\sigma_{\cU}$ on $\cO_X^{\ur +}|_{U}$ (here $U:=\cU_k$). Indeed, the Frobenius structure on $\cM$ gives a horizontal $\cO_{\cU}$-linear morphism $ \sigma_{\cU}^{\ast}\cM|_{\cU}\to \cM|_{\cU}$, or equivalently, a $\sigma_{\cU}$-semilinear morphism $\varphi_{\cU}\colon \cM|_{\cU}\to \cM|_{\cU}$ (as $\sigma_{\cU}$ is the identity map on the underlying topological space). So we obtain a natural endomorphism $w^{-1}\varphi_{\cU}$ of $w^{-1}\cM|_{U}$, which is $w^{-1}\sigma_{\cU}$-semilinear.   

\vskip 1mm

(2) Let $\cE$ be a convergent $F$-isocrystal on $\cX_0/\cO_k$. By Remark \ref{rk.CrystalVSIsoc}, there exists an $F$-crystal $\mathcal M$ on $\cX_0/\cO_k$ and $n\in \mathbb N$ such that $\cE\simeq \mathcal M^{\rm an}(n)$. By (1), $w^{-1}\cM$ is a coherent $\cO_{X}^{\ur+}$-module equipped with an integrable connection and a compatible system of Frobenii $\varphi$. Inverting $p$, we get an $\cO_{X}^{\ur}$-module $w^{-1}\cM[1/p]$ equipped with an integrable connection and a system of Frobenii $\varphi /p^n$, which does not depend on the choice of the formal model $\cM$ or the integer $n$. For this reason, abusing notation, let us denote $w^{-1}\cM[1/p]$ by $w^{-1}\cE$, which is equipped with an integrable connection and a system of Frobenii inherited from $\cE$. If furthermore $\cE$ has a descending filtration $\{\Fil^i \cE\}$ by locally direct summands, by Remark \ref{O[1/p]}, each $\Fil^i\cE$ has a coherent formal model $\cE_i^+$ on $\cX$. Then $\{w^{-1}\cE_i^{+}[1/p]\}$ gives a descending filtration (by locally direct summands) on $w^{-1}\cE$.  
\end{rk}



 
\begin{defn}\label{associated}
 We say a lisse $\widehat{\ZZ}_p$-sheaf $\LL$  on $X_{\proet}$  is \emph{crystalline} if there exists a filtered $F$-isocrystal $\cE$ together with an isomorphism of $\cO\BBcr$-modules 
\begin{equation}\label{eq.associated}
w^{-1}\cE\otimes_{\cO_X^{\ur}} \cO\BBcr\simeq \LL\otimes_{\widehat{\ZZ}_p}\cO\BBcr
\end{equation}
which is compatible with connection, filtration and Frobenius. In this case, we say that the lisse $\widehat{\ZZ}_p$-sheaf $\LL$ and the filtered $F$-isocrystal $\cE$ are \emph{associated}. 
\end{defn}

\begin{rk} \label{abuseem} The Frobenius compatibility of the isomorphism \eqref{eq.associated} means the following. Take any open subset $\mathcal U\subset \cX$ equipped with a lifting of Frobenius $\sigma \colon \mathcal U\to \mathcal U$. By the discussion in \S \ref{localfrob}, we know that $\cO\mathbb B_{\cris}|_{\mathcal U_k}$ is naturally endowed with a Frobenius $\varphi$. Meanwhile, as $\mathcal E$ is an $F$-isocrystal, by Remark \ref{rk.wE} $w^{-1}\cE|_{\cU_k}$ is endowed with a $w^{-1}\sigma$-semilinear Frobenius, still denoted by $\varphi$. Now the required Frobenius compatibility means that when restricted to any such $\mathcal U_k$, we have $\varphi\otimes \varphi=\mathrm{id}\otimes \varphi$ via the isomorphism \eqref{eq.associated}. 
\end{rk}



\begin{defn} For $\LL$ a lisse $\widehat{\ZZ}_p$-sheaf and $i\in \mathbb Z$, set 
\[
\mathbb D_{\cris}(\LL):=w_{\ast}(\LL\otimes_{\widehat{\ZZ}_p}\cO\BBcr), \quad \text{and} \quad \Fil^i\mathbb D_{\cris}(\LL):=w_{\ast}(\LL\otimes_{\widehat{\ZZ}_p}\Fil^i\cO\BBcr).
\]
All of them are $\cO_{\cX}[1/p]$-modules, and the $\Fil^i\mathbb D_{\cris}(\LL)$ give a separated exhaustive decreasing filtration on $\mathbb D_{\cris}(\LL)$ (as the same holds for the filtration on $\cO\BBcr$; see Corollary \ref{GradedOfBcris}). 
\end{defn}


Next we shall compare the notion of crystalline sheaves with other related notions considered in \cite[Chapitre 8]{Bri}, \cite{Fal} and \cite{Sch13}. We begin with the following characterization of crystalline sheaves, which is more closely related to the classical definition of crystalline representations by Fontaine (see also \cite[Chapitre 8]{Bri}). 


\begin{prop}\label{prop.CrysSheaf} Let $\LL$ be a lisse $\widehat{\Z}_p$-sheaf on $X_{\proet}$. Then $\LL$ is crystalline if and only if the following two conditions are verified:
\begin{enumerate}
\item the $\cO_{\cX}[1/p]$-modules $\mathbb D_{\cris}(\LL)$ and $\Fil^i\mathbb D_{\cris}(\LL)$ ($i\in \mathbb Z$) are all coherent.  
\item the adjunction morphism $w^{-1}\mathbb D_{\cris}(\LL)\otimes_{\cO_X^{\ur}} \cO\BBcr\to \LL\otimes_{\widehat{\ZZ}_p}\cO\BBcr$ is an isomorphism of $\cO\BBcr$-modules.%,  which is compatible with connection, Frobenius and filtration. %\footnote{\red{The last half sentence is newly added. AI did something on compatibility with filtration in their Lemma 3.8. I don't know their purpose.}}
\end{enumerate}
%Moreover, the (analytification of the) $\cO_{\cX}[1/p]$-module $\mathbb D_{\cris}(\LL)$ has a structure of filtered convergent $F$-isocrystal.
\end{prop}
 
Before proving this proposition, let us express locally the sheaf $\mathbb D_{\cris}(\LL)=w_{\ast}(\LL\otimes \cO\BBcr)$ as the Galois invariants of some Galois module. Consider $\cU=\Spf(R^+)\subset \cX$ a connected affine open subset admitting an \'etale map $\cU\to \Spf(\cO_k\{T_1^{\pm 1},\ldots, T_d^{\pm 1}\})$. Write $R=R^+[1/p]$ and denote $U$ the generic fiber of $\cU$. As $\cU$ is smooth and connected, $R^+$ is an integral domain. Fix an algebraic closure $\Omega$ of $\mathrm{Frac}(R)$, and let $\overline{R}^+$ be the union of finite and normal $R^+$-algebras $Q^+$ contained in $\Omega$ such that $Q^+[1/p]$ is \'etale over $R$. Write $\overline{R}=\overline{R}^+[1/p]$. Write $G_U:=\mathrm{Gal}(\overline{R}/R)$, which is nothing but the fundamental group of $U=\cU_k$. Let $U^{\mathrm{univ}}$ be the profinite \'etale cover of $U$ corresponding to $(\overline R, \overline R^+)$. One checks that $U^{\mathrm{univ}}$ is affinoid perfectoid (over the completion of $\overline k$). As $\LL$ is a lisse $\widehat{\ZZ}_p$-sheaf on $X$, its restriction to $U$ corresponds to a continuous $\Zp$-representation $V_{U}(\LL):=\LL(U^{\mathrm{univ}})$ of $G_U$. Write $\widehat{U^{\rm univ}}=\Spa(S,S^+)$, where $(S,S^+)$ is the $p$-adic completion of $(\overline R, \overline R^+)$.

\begin{lemma} \label{twodcris} Keep the notation above. Let $\LL$ be a lisse $\widehat{\ZZ}_p$-sheaf on $X$.
Then there exist natural isomorphisms of $R$-modules 
\[
%w_{\ast}(\LL\otimes \cO\BBcr)(\cU)=
\mathbb D_{\cris}(\LL)(\cU)\stackrel{\sim}{\longrightarrow} \left(V_{U}(\LL)\otimes_{\Zp}\cO\BBcr(S,S^+)\right)^{G_U}=:D_{\cris}(V_{U}(\LL))
\] 
and, for any $r\in \ZZ$,
\[
\left(\Fil^r \mathbb D_{\cris}(\LL)\right)(\cU)\stackrel{\sim}{\longrightarrow} \left(V_{U}(\LL)\otimes_{\Zp}\Fil^r\cO\BBcr(S,S^+)\right)^{G_U}.
\]
Moreover, the $R$-module $\mathbb D_{\cris}(\LL)(\cU)$ is projective of rank at most that of $V_{U}(\LL)\otimes \mathbb Q_p$.
\end{lemma}
\begin{proof} As $\LL$ is a lisse $\widehat{\ZZ}_p$-sheaf, it becomes constant restricted  to $U^{\mathrm{univ}}$. In other words, we have $
\LL|_{U^{\mathrm{univ}}}\simeq V_{U}(\LL)\otimes_{\Zp}\widehat{\ZZ}_p|_{U^{\mathrm{univ}}}$. 
For $i\geq 0$ an integer, we denote by $U^{\mathrm{univ},i}$ the (i+1)-fold product of $U^{\mathrm{univ}}$ over $U$. Then $
U^{\mathrm{univ},i}\simeq U^{\mathrm{univ}}\times G_U^i$, and it is again an affinoid perfectoid. By the use of Lemma \ref{2obcris}, we find $H^j(U^{\mathrm{univ},i},\LL\otimes_{\widehat{\ZZ}_p}\cO\BBcr)=0$ for $j>0$. Moreover 
\[
H^0(U^{\mathrm{univ},i},\LL\otimes_{\widehat{\ZZ}_p}\cO\BBcr)=\Hom_{\mathrm{cont}}\left(G_U^i, V_{U}(\LL)\otimes_{\Zp}\cO\AAcr(U^{\mathrm{univ}})\right)[1/t].
\]
Again, as $U^{\mathrm{univ}}$ is affinoid perfectoid, $\cO\BBcr(U^{\mathrm{univ}})\simeq \cO\BBcr(S,S^+)$ by Lemma \ref{2obcris}. Consider the Cartan-Leray spectral sequence (cf. \cite[V.3]{SGA4}) associated to the cover  $U^{\rm univ}\to U$: 
\[
E_{1}^{i,j}=H^j(U^{\mathrm{univ},i},\LL\otimes_{\widehat{\ZZ}_p}\cO\BBcr) \Longrightarrow H^{i+j}(U,\LL\otimes_{\widehat{\ZZ}_p}\cO\BBcr).
\]
As $E_{1}^{i,j}= 0$ for $j\geq 1$, we have $
E_{2}^{n,0}=H^n(U,\LL\otimes_{\widehat{\ZZ}_p}\cO\BBcr)$. 
Thus, we deduce a natural isomorphism  
\[
H^j(U,\LL\otimes_{\widehat{\ZZ}_p}\cO\BBcr)\stackrel{\sim}{\longrightarrow} H^j_{\mathrm{cont}}(G_U,V_{U}(\LL)\otimes_{\Zp}\cO\AAcr(S,S^+))[1/t] 
\]
where the right hand side is the continuous group cohomology. Taking $j=0$, we obtain our first assertion. The isomorphism concerning $\Fil^r\cO\BBcr$ can be proved exactly in the same way. The last assertion follows from the first isomorphism and  \cite[Proposition 8.3.1]{Bri}, which gives the assertion for the right hand side.
\end{proof}

The lemma above has the following two consequences.

\begin{cor}\label{easyfact} Let $\LL$ be a lisse $\widehat{\Z}_p$-sheaf on $X_{\proet}$, which satisfies the condition (1) of Proposition \ref{prop.CrysSheaf}. Let $\cU=\Spf(R^+)$ be a small connected affine open subset of $\cX$. Write $R=R^+[1/p]$ and $U=\cU_k$. Then for any $V\in X_{\proet}/U$, we have 
\[\mathbb D_{\cris}(\LL)(\cU)\otimes_{R}\cO\BBcr\left(V\right)\stackrel{\sim}{\rightarrow} (w^{-1}\mathbb D_{\cris}(\LL)\otimes_{\cO_X^{\ur}} \cO\BBcr)(V).\]
\end{cor}

\begin{proof}By Lemma \ref{twodcris}, the $R$-module $\mathbb D_{\cris}(\LL)(\cU)$ is projective of finite type over $R$, hence it is a direct summand of a finite free $R$-module.  As $\mathbb D_{\cris}(\LL)$ is coherent over $\cO_{\cX}[1/p]$ and as $\cU$ is affine, $\mathbb D_{\cris}(\LL)|_{\cU}$ is then a direct summand of a finite free $\cO_{\cX}[1/p]|_{\cU}$-module. The isomorphism in our corollary then follows, since we have similar isomorphism when $\mathbb D_{\cris}(\LL)|_{\cU}$ is replace by a free $\cO_{\cX}[1/p]|_{\cU}$-module. 
\end{proof}


\begin{cor}\label{CrysRepAff} Let $\LL$ be a lisse $\widehat{\ZZ}_p$-sheaf verifying the condition (1) of Proposition \ref{prop.CrysSheaf}. Then the condition (2) of Proposition \ref{prop.CrysSheaf} holds for $\LL$ if and only if for any small affine connected open subset $\cU\subset \cX$ (with $U:=\cU_k$), the $G_U$-representation $V_{U}(\LL)\otimes_{\Zp}\Qp$ is crystalline in the sense that the following natural morphism is an isomorphism (\cite[Chapitre 8]{Bri})
\[
D_{\cris}(V_U(\LL))\otimes_{R} \cO\BBcr(S,S^+) \stackrel{\sim}{\longrightarrow} V_U(\LL)\otimes_{\Zp}\cO\BBcr(S,S^+), 
\]
where $G_U, U^{\rm univ}, \widehat{U^{\rm univ}}=\Spa(S,S^+)$ are as in the paragraph before Lemma \ref{twodcris}.
%  the $G_R$-representation $V_{U}(\LL)\otimes_{\Zp}\Qp$ is crystalline in the sense of \cite[Chapitre 8]{Bri} and if the $\cO_{\cX}[1/p]$-module $\mathbb D_{\cris}(\LL)$ is coherent, then  the $\widehat{\ZZ}_p$-sheaf $\LL$ verifies the conditions (2) of Proposition \ref{prop.CrysSheaf}.
\end{cor}

\begin{proof} If $\LL$ satisfies in addition the condition (2) of Proposition \ref{prop.CrysSheaf}, combining with Corollary \ref{easyfact}, we find 
\begin{eqnarray*}
\mathbb D_{\cris}(\LL)(\cU)\otimes_{R}\cO\BBcr\left(U^{\rm univ}\right) &\stackrel{\sim}{\longrightarrow} & (w^{-1}\mathbb D_{\cris}(\LL)\otimes_{\cO_X^{\ur}}\cO\BBcr)(U^{\rm univ}) \\ & \stackrel{\sim}{\longrightarrow} & (\LL\otimes_{\widehat{\ZZ}_p}\cO\BBcr)(U^{\rm univ})\\ & =  & V_{U}(\LL)\otimes_{\Zp}\cO\BBcr(U^{\rm univ}).
\end{eqnarray*}
So, by Lemmas \ref{2obcris} and \ref{twodcris}, the $G_U$-representation $V_{U}(\LL)\otimes \mathbb Q_p$ is crystalline. 


Conversely, assume that for any small connected affine open subset $\cU=\Spf(R^+)$ of $\cX$, the $G_{U}$-representation $V_{U}(\LL)\otimes_{\Zp}\Qp$ is crystalline. Together with  Lemmas \ref{2obcris} and \ref{twodcris}, we get $
\mathbb{D}_{\cris}(\LL)(\cU)\otimes_{R}\cO\BBcr(U^{\rm univ})\stackrel{\sim}{\rightarrow} V_{U}(\LL)\otimes_{\Zp}\cO\BBcr(U^{\rm univ})$ and the similar isomorphism after replacing $U^{\rm univ}$ by any $V\in X_{\proet}/U^{\rm univ}$. Using Corollary \ref{easyfact}, we deduce $
(w^{-1}\mathbb D_{\cris}(\LL)\otimes_{\cO_{X}^{\rm ur}}\cO\BBcr)(V) \stackrel{\sim}{\rightarrow} (\LL\otimes_{\widehat{\Z}_p}\cO\BBcr)(V)$ for any $V\in X_{\proet}/U^{\rm univ}$, \emph{i.e.,} $(w^{-1}\mathbb D_{\cris}(\LL)\otimes_{\cO_{X}^{\rm ur}}\cO\BBcr)|_{U^{\rm univ}}\stackrel{\sim}{\to} (\LL\otimes_{\widehat{\Z}_p}\cO\BBcr)|_{U^{\rm univ}}$. When the small opens $\cU$'s run through a cover of $\cX$, the $U^{\rm univ}$'s form a cover of $X$ for the pro-\'etale topology. Therefore, $w^{-1}\mathbb D_{\cris}(\LL)\otimes \cO\BBcr\stackrel{\sim}{\to} \LL\otimes \cO\BBcr$, as desired. 
\end{proof}
%The compatibility  of the isomorphism above with connection and Frobenius  is clear. The compatibility with filtration  follows from \cite[Proposition 8.4.3]{Bri}, combined with Corollary \ref{GradedOfBcris} .  We thus obtain the conditions (1) and (2) of Proposition \ref{prop.CrysSheaf}, as the small affines $\cU\subset \cX$  form a basis of the Zariski topology of $\cX$.


%\begin{lemma} Let $R^+$ be a normal $p$-adically complete flat topologically finite type $\cO_k$-algebra. Let $R'$ be an $R$-submodule of $R^+[1/p]$ containing $R$. Assume $\bigcap_n p^n R' =0$. Then there exists an integer $N\in \mathbb N$ such that $p^N\cdot  R' \subset R$. In particular, $R'$ is of finite type over $R^+$. 
%\end{lemma}

%\begin{proof} As $R^+$ is normal, we only need to show that there exists some integer $N$ such that, for any prime ideal $\mathfrak p\subset R^+$, 
%\end{proof}




\begin{lemma}\label{IsoCrysOnDcris} Let $\LL$ be a lisse $\widehat{\Z}_p$-sheaf on $X$ satisfying the two conditions of Proposition \ref{prop.CrysSheaf}. Then (the analytification of) $\mathbb D_{\cris}(\LL)$ has a natural structure of filtered convergent $F$-isocrystal on $\cX_0/\cO_k$. 
\end{lemma}

\begin{proof} First of all, the $\Fil^i\mathbb D_{\cris}(\LL)$'s ($i\in \mathbb Z$) endow a separated exhaustive decreasing filtration on $\mathbb D_{\cris}(\LL)$ by Corollary \ref{GradedOfBcris}, and the connection on $\mathbb D_{\cris}(\LL)=w_{\ast}(\LL\otimes \cO\BBcr)$ can be given by the composite of
\begin{eqnarray*}
w_{\ast}(\LL\otimes \cO\BBcr)&\stackrel{w_{\ast}(\mathrm{id}\otimes \nabla)}{\longrightarrow} & w_{\ast}(\LL\otimes \cO\BBcr\otimes_{\cO_{X}^{\ur}}\Omega_{X/k}^{1,\ur}) \\ & \stackrel{\sim}{\longrightarrow} &  w_{\ast}(\LL\otimes \cO\BBcr)\otimes_{\cO_{\cX}[1/p]}\Omega^1_{\cX/\cO_k}[1/p] 
\end{eqnarray*}
where the last isomorphism is the projection formula. That the connection satisfies the Griffiths transversality with respect to the filtration $\Fil^{\bullet}\mathbb D_{\cris}$  follows from the analogous assertion for $\cO\BBcr$ (Proposition \ref{poincare}). 

Now consider the special case where $\cX=\Spf(R^+)$ is affine connected admitting an \'etale map $\cX\to \Spf(\cO_k\{T_1^{\pm 1}, \ldots, T_d^{\pm 1}\})$, such that $\cX$ is equipped with a lifting of Frobenius $\sigma$.  As in the paragraph before Lemma \ref{twodcris}, let $X^{\rm univ}$ be the univeral profinite \'etale cover of $X$ (which is an affinoid perfectoid). Write $\widehat{X^{\rm univ}}=\Spa(S,S^+)$ and $G_X$ the fundamental group of $X$. As $\cX$ is affine, the category $\mathrm{Coh}(\cO_{\cX}[1/p])$ is equivalent to the category of finite type $R$-modules (here $R:=R^{+}[1/p]$). Under this equivalence, $\mathbb D_{\cris}(\LL)$ corresponds to $D_{\cris}(V_{X}(\LL)):=(V_X(\LL)\otimes \cO\BBcr(S,S^+))^{G_X}$, denoted by $D$ for simplicity. So $D$ is a projective $R$-module of finite type (Lemma \ref{twodcris}) equipped with a connection $\nabla \colon D\to D\otimes \Omega^{1}_{R/k}$. Under the same equivalence, $\Fil^i\mathbb D_{\cris}(\LL)$ corresponds to $
\Fil^i D:=(V_{X}(\LL)\otimes \Fil^i \cO\BBcr(S,S^+))^{G_X}$, by Lemma \ref{twodcris} again. 
By the same proof as in  \cite[8.3.2]{Bri}, the graded quotient $\mathrm{gr}^i(D)$ is a projective module. In particular, $\Fil^i D\subset D$ is a direct summand. Therefore, each $\Fil^i\mathbb D_{\cris}(\LL)$ is a direct summand of $\mathbb D_{\cris}(\LL)$. Furthermore, since $\cX$ admits a lifting of Frobenius $\sigma$, we get from \S~\ref{localfrob} a $\sigma$-semilinear endomorphism $\varphi$ on $\cO\BBcr(X^{\rm univ})\simeq \cO\BBcr(S,S^+)$, whence a $\sigma$-semilinear endomorphism on $D$, still denoted by $\varphi$. From Lemma  \ref{FrobHorizontal}, One checks that the Frobenius $\varphi$ on $D$ is horizontal with respect to its connection. Thus $\mathbb D_{\cris}(\LL)$ is endowed with a horizontal $\sigma$-semilinear morphism $\mathbb D_{\cris}(\LL)\to \mathbb D_{\cris}(\LL)$, always denoted by $\varphi$ in the following. 

To finish the proof in the special case, one still needs to show that the triple $(\mathbb D_{\cris}(\LL),\nabla,\varphi)$ gives an $F$-isocrystal on $\cX_0/\cO_k$.  As $D$ is of finite type over $R$, there exists some $n\in \mathbb N$ such that $D=D^+[1/p]$ with $D^+:=(V_{X}(\LL)\otimes_{\Zp} t^{-n}\cO\AAcr(S,S^+))^{G_X}$.
The connection on $t^{-n}\cO\AAcr(S,S^+)$ induces a connection $\nabla^+ \colon D^+\to D^{+}\otimes_{R^+}\Omega^{1}_{R^+/\cO_k}$ on $D^+$, compatible with that of $D_{\cris}(V_{X}(\LL))$. Moreover, if we take $N_i$ to be  the endomorphism of $D^+$ so that $\nabla^+=\sum_{i=1}^{d} N_i\otimes dT_i$, then for any $a\in D^+$, $\underline N^{\underline m}(a)\in p\cdot D^+$ for all but finitely many $\underline m\in \mathbb N^d$ (as this holds for the connection on $t^{-n}\cO\AAcr$, seen in the proof of Lemma \ref{sigma12}). Similarly, the Frobenius on $\cO\BBcr(S,S^+)$ induces a map (note that the Frobenius on $\cO\BBcr(S,S^+)$ sends $t$ to $p\cdot t$)
\[
\varphi\colon D^+\longrightarrow (V_{X}(\LL)\otimes p^{-n} t^{-n}\cO\AAcr(S,S^+))^{G_X}.
\]
Thus $\psi:=p^n\varphi$ gives a well-defined $\sigma$-semilinear morphism on $D^+$. One checks that $\psi$ is horizontal with respect to the connection $\nabla^+$ on $D^+$ and it induces an $R^+$-linear isomorphism $\sigma^{\ast} D^+\stackrel{\sim}{\to} D^+$.  
As a result, the triple $(D^+,\nabla, \psi)$ will define an $F$-crystal on $\cU_0/\cO_k$, once we know $D^+$ is of finite type over $R^+$. The required finiteness of $D^+$ is explained in \cite[Proposition 3.6]{AI}, and for the sake of completeness we recall briefly their proof here. As $D$ is projective of finite type (Lemma \ref{twodcris}), it is a direct summand of a finite free $R$-module $T$. Let $T^+\subset T$ be a finite free $R^+$-submodule of $T$ such that $T^{+}[1/p]=T$. Then we have the inclusion $D\otimes_{R}\cO\BBcr(S,S^+)\hookrightarrow T^+\otimes_{R^+} \cO\BBcr(S,S^+).$
As $V_{X}(\LL)$ is of finite type over $\Zp$ and $\cO\BBcr(S,S^+)=\cO\AAcr(S,S^+)[1/t]$, there exists $m\in \mathbb N$ such that the $\cO\AAcr(S,S^+)$-submodule $V_{X}(\LL)\otimes t^{-n}\cO\AAcr(S,S^+)$ of $V_X(\LL)\otimes \cO\BBcr(S,S+)\simeq D\otimes \cO\BBcr(S,S^+)$ is contained in $T^{+}\otimes_{R^+}t^{-m}\cO\AAcr(S,S^+)$. By taking $G_{U}$-invariants and using the fact that $R^+$ is noetherian, we are reduced to showing that $R':=(t^{-m}\cO\AAcr(S,S^+))^{G_{X}}$ is of finite type over $R^+$. From the construction,  $R'$ is $p$-adically separated and $R^+\subset R' \subset R=(\cO\BBcr(S,S^+))^{G_X}$.  As $R^+$ is normal, we deduce $p^NR'\subset R^+ $ for some $N\in \mathbb N$. Thus $p^NR'$ and  hence $R'$ are of finite type over $R^+$. As a result, $(D^+,\nabla, \psi)$ defines an $F$-crystal $\mathcal D^+$ on $\cU_0/\cO_k$. As $D=D^+[1/p]$ and $\nabla=\nabla^+[1/p]$, the connection $\nabla$ on $\mathbb D_{\cris}(\LL)$ is convergent; this is standard and we refer to \cite[2.4.1]{Ber} for detail. Consequently, the triple $(\mathbb D_{\cris}(\LL), \nabla, \varphi )$ is an $F$-isocrystal on $\cX_0/\cO_k$, which is isomorphic to  $\mathcal D^{+,{\rm an}}(n)$. This finishes the proof in the special case. 


In the general case, consider a covering $\cX=\bigcup_i \cU_i$ of $\cX$ by connected small affine open subsets such that each $\cU_i$ admits a lifting of Frobenius $\sigma_i$ and an \'etale morphism to some torus over $\cO_k$. By the special case, we have seen that each $\Fil^i\mathbb D_{\cris}(\LL)\subset \mathbb D_{\cris}(\LL)$ is locally a direct summand, and that the connection on $\mathbb D_{\cris}(\LL)$ is convergent (\cite[2.2.8]{Ber}). Furthermore, each $\mathbb D_{\cris}(\LL)|_{\cU_i}$ is equipped with a Frobenius $\varphi_i$, and over $\cU_i\cap \cU_j$, the two Frobenii $\varphi_i,\varphi_j$ on $\mathbb D_{\cris}(\LL)|_{\cU_{i}\bigcap \cU_{j}}$ are related by the formula in Lemma \ref{fsigma12} as it is the case for $\varphi_i,\varphi_j$ on $\cO\BBcr|_{U_i\bigcap U_j}$ (Lemma \ref{sigma12}). So these local Frobenii glue together to give a compatible system of Frobenii $\varphi$ on $\mathbb D_{\cris}(\LL)$ and the analytification of the quadruple $(\mathbb D_{\cris}(\LL),\Fil^{\bullet}\mathbb D_{\cris}(\LL), \nabla, \varphi)$ is a filtered $F$-isocrystal on $\cX_0/\cO_k$, as wanted. 
\end{proof}


\begin{proof}[Proof of Proposition~\ref{prop.CrysSheaf}]  If a lisse $\widehat{\ZZ}_p$-sheaf $\LL$ on $X$ is associated to a filtered $F$-isocrystal $\cE$ on $X$, then we just have to show $\cE\simeq \DD_{\cris}(\LL)$. By assumption, we have $\mathbb L\otimes_{\widehat{\ZZ}_p}\cO\BBcr\simeq w^{-1}\cE\otimes_{\cO_X^{\ur}}\cO\BBcr$. Then
\[
w_*(\mathbb L\otimes_{\widehat{\ZZ}_p}\cO\BBcr)\simeq w_*(w^{-1}\cE\otimes_{\cO_{X}^{\ur}}\cO\BBcr)\simeq \cE\otimes_{\cO_{\cX_{\et}}[1/p]}w_*\cO\BBcr\simeq\cE
\] where the second isomorphism has used Remark \ref{abuseem}, and the last isomorphism is by the isomorphism  $w_*\cO\BBcr\simeq \cO_{\cX_{\et}}[1/p]$ from Corollary \ref{higherox}. %It is obvious\footnote{Conn and From seem ok. We seem to have $w_*(\Fil^i\cE)=\Fil^i\cE$. What is $w_*(\Fil^i\cO\BB_{\cris})$?}  that the isomorphisms above are compatible with connection, filtration and Frobenius.  Therefore, the two conditions (1) and (2) are satisfied. 

Conversely, let $\LL$ be a lisse $\widehat{\ZZ}_p$-sheaf verifying the two conditions of our proposition. By Lemma \ref{IsoCrysOnDcris}, $\mathbb D_{\cris}(\LL)$ is naturally a filtered $F$-isocrystal. To finish the proof, we need to show that the isomorphism in (2) is compatible with the extra structures. Only the compatibility with filtrations needs  verification. This is a local question, hence we shall assume $\cX=\Spf(R^+)$ is a small connected affine formal scheme. As $\Fil^i\mathbb D_{\cris}(\LL)$ is coherent over $\cO_{\cX}[1/p]$ and is a direct summand of $\mathbb D_{\cris}(\LL)$, the same proof as that of Corollary \ref{easyfact} gives  
\[
\Fil^i\mathbb D_{\cris}(\LL)(\cX)\otimes_R \Fil^j\cO\BBcr(V)\stackrel{\sim}{\longrightarrow} (w^{-1}\Fil^i \mathbb D_{\cris}(\LL)\otimes_{\cO_X^{\ur}} \Fil^j\cO\BBcr)(V)
\]
for any $V\in X_{\proet}$. Consequently, the isomorphism in Corollary \ref{easyfact} is strictly compatible with filtrations on both sides. Thus, we reduce to show that, for any affinoid perfectoid $V\in X_{\proet}/X^{\rm univ} $, the isomorphism $D_{\cris}(V_X(\LL))\otimes_R \cO\BBcr(V)\stackrel{\sim}{\to} V_{X}(\LL)\otimes \cO\BBcr(V)$ is strictly compatible with the filtrations, or equivalently, the induced morphisms between the graded quotients are isomorphisms:
\begin{equation}\label{eq.gradediso}
\oplus_{i+j=n}\left(\mathrm{gr}^iD_{\cris}(V_X(\LL))\otimes_R \mathrm{gr}^j\cO\BBcr(V)\right)\longrightarrow \LL\otimes \mathrm{gr}^n\cO\BBcr(V). 
\end{equation}
When $V=X^{\rm univ}$, this follows from \cite[8.4.3]{Bri}. For the general case, write $\widehat{X^{\rm univ}}=\Spa(S,S^+)$ and $\widehat{V}=\Spa(S_1,S_1^+)$. Then by \cite[Corollary 6.15]{Sch13} and Corollary \ref{GradedOfBcris}, we have $\mathrm{gr}^j\cO\BBcr(V)\simeq S_1\xi^j[U_1/\xi,\ldots, U_d/\xi]$. So the natural morphism $\mathrm{gr}^j\cO\BBcr(X^{\rm univ})\otimes_SS_1\stackrel{\sim}{\to}\mathrm{gr}^j\cO\BBcr(V) $ is an isomorphism. The required isomorphism \eqref{eq.gradediso} for general $V$ then follows from the special case for $X^{\rm univ}$.  
\end{proof}


 


Let $\mathrm{Lis}^{\cris}_{\widehat{\ZZ}_p}(X)$ denote the category of lisse crystalline $\widehat{\ZZ}_p$-sheaves on $X$, and $\mathrm{Lis}^{\cris}_{\widehat{\QQ}_p}(X)$ the corresponding isogeny category. The functor 
\[
\mathbb D_{\cris}\colon \mathrm{Lis}^{\cris}_{\widehat{\QQ}_p}(X)\longrightarrow FF\textrm{-Iso}^{\dagger}(\cX_0/\cO_k), \quad  \LL\mapsto \mathbb D_{\cris}(\LL)
\] 
allows us to relate $\mathrm{Lis}^{\cris}_{\widehat{\QQ}_p}(X)$ to the category $FF\textrm{-}\mathrm{Iso}^{\dagger}(\cX_0/\cO_k)$  of filtered convergent $F$-isocrystals on $\cX_0/\cO_k$, thanks to Proposition \ref{prop.CrysSheaf}.
A filtered $F$-isocrystal $\cE$ on $\cX_0/\cO_k$ is called \emph{admissible} if it lies in the essential image of the functor above. The full subcategory of admissible filtered $F$-isocrystals on $\cX_0/\cO_k$ will be denoted by $FF\textrm{-Iso}^{\dagger}(\cX_0/\cO_k)^{\textrm{adm}}$. 


 

\begin{thm}
The functor $\DD_{\cris}$ above induces an equivalence of categories 
\[
\mathbb D_{\cris}\colon \mathrm{Lis}^{\cris}_{\widehat{\QQ}_p}(X)\stackrel{\sim}{\longrightarrow} FF\textrm{-}\mathrm{Iso}^{\dagger}(\cX_0/\cO_k)^{\mathrm{adm}}. 
\] 
A quasi-inverse of $\mathbb D_{\cris}$ is given by 
\[
\VV_{\cris}: \cE\mapsto \Fil^0(w^{-1}\cE\otimes_{\cO_{X}^{\ur}}\cO\BBcr)^{\nabla=0, \varphi=1}
\] 
where $\varphi$ denotes the compatible system of Frobenii on $\cE$ as before.
\end{thm}

\begin{proof} Observe first that, for $\cE$ a filtered convergent $F$-isocrystal, the local Frobenii on $\cE^{\nabla=0}$ glue to give a unique $\sigma$-semilinear morphism on $\mathcal E^{\nabla=0}$ (Lemma \ref{fsigma12}). In particular, the abelian sheaf $\mathbb V_{\cris}(\cE)$ is well-defined. Assume moreover $\cE$ is admissible, and let $\LL$ be a lisse $\widehat{\Z}_p$-sheaf such that $\cE\simeq \mathbb D_{\cris}(\LL)$. So $\LL$ and $\cE$ are associated by Proposition \ref{prop.CrysSheaf}. Hence $
\LL\otimes_{\widehat{\ZZ}_p}\cO\BBcr\simeq w^{-1}\cE\otimes_{\cO_X^{\ur}}\cO\BBcr$, and we find  
\begin{eqnarray*}
\LL\otimes_{\widehat{\ZZ}_p}\widehat{\QQ}_p & \stackrel{\sim}{\longrightarrow} &  \LL\otimes_{\widehat{\ZZ}_p} \Fil^0(\cO\BBcr)^{\nabla=0, \varphi=1} \\ & \stackrel{\sim}{\longrightarrow} & \Fil^0(\LL\otimes_{\widehat{\ZZ}_p}\cO\BBcr)^{\nabla=0,\varphi=1} \\ & \stackrel{\sim}{\lra} & \Fil^0(w^{-1}\cE\otimes_{\cO_{X}^{\ur}}\cO\BBcr)^{\nabla=0,\varphi=1}
\\ & =& \mathbb V_{\cris}(\cE),
\end{eqnarray*}
where the first isomorphism following from the the fundamental exact sequence  (by Lemma \ref{vanish} and \cite[Corollary 6.2.19]{Bri}) 
\[
0\lra \Qp\lra \Fil^0\BBcr\stackrel{1-\varphi}{\lra}\BBcr\lra 0.
\]  
In particular, $\mathbb V_{\cris}(\cE)$ is the associated $\widehat{\mathbb Q}_p$-sheaf of a lisse $\widehat{\Z}_p$-sheaf. Thus $\mathbb V_{\cris}(\cE)\in \mathrm{Lis}^{\rm cris}_{\widehat{\mathbb Q_p}}(X)$ and the functor $\mathbb V_{\cris}$ is well-defined. Furthermore, as we can recover the the lisse $\widehat{\Z}_p$-sheaf up to isogeny, it follows that $\mathbb D_{\cris}$ is fully faithful, and a quasi-inverse on its essential image is given by $\mathbb V_{\cris}$. 
\end{proof}
 
\begin{rk} Using \cite[Theorem 8.5.2]{Bri}, one can show that the equivalence above is an equivalence of tannakian category. 
\end{rk}
 


Next we compare Definition \ref{associated} with the ``associatedness" defined in \cite{Fal}. Let $\cE$ be a filtered convergent $F$-isocrystal $\cE$ on $\cX_0/\cO_k$, and $\cM$ an $F$-crystal on $\cX_0/\cO_k$ such that $\cM^{\an}=\cE(-n)$ for some $n\in \N$ (see Remark \ref{rk.CrystalVSIsoc} for the notations). Let $\cU=\Spf(R^+)$ be a small connected affine open subset of $\cX$, equipped with a lifting of Frobenius $\sigma$. Write $U=\Spa(R,R^+)$ the generic fiber of $\cU$. As before, let $\overline R^+$ be the union of all finite normal $R^+$-algebras (contained in some fixed algebraic closure of $\mathrm{Frac}(R^+)$) which are \'etale over $R:=R^+[1/p]$, and $\overline R:=\overline R^+[1/p]$. Let $G_U:=\mathrm{Gal}(\overline R/R)$ and $(S,S^+)$ the $p$-adic completion of $(\overline R, \overline R^+)$. Then $(S,S^+)$ is an perfectoid affinoid algebra over the $p$-adic completion of $\bk$. So we can consider the period sheaf $\mathbb A_{\cris}(S,S^+)$. Moreover the composite of the following two natural morphisms
\begin{equation}\label{PDThickening}
\AAcr \left(S,S^+\right) \stackrel{\theta}{\longrightarrow} S^+ \stackrel{\textrm{can}}{\longrightarrow} S^+/pS^+,
\end{equation}
defines a $p$-adic PD-thickening of $\Spec(S^+/pS^+)$. Evaluate our $F$-crystal $\cM$ at it and write $\cM(\mathbb A_{\cris}(S,S^+))$ for the resulting finite type $\mathbb A_{\cris}(S,S^+)$-module. As an element of $G_U$ defines a morphism of the PD-thickening \eqref{PDThickening} in the big crystalline site of $\cX_0/\cO_k$ and $\cM$ is a crystal, $\cM(\mathbb A(S,S^+))$ is endowed naturally with an action of $G_U$. Similarly, the Frobenius on the crystal $\cM$ gives a Frobenius $\psi$ on $\cM(\AAcr(S,S^+))$. 
Set $\cE(\mathbb B_{\cris}(S,S^+)):=\cM(\mathbb A_{\cris}(S,S^+))[1/t]$, which is a $\mathbb B_{\cris}(S,S^+)$-module of finite type endowed with a Frobenius $\varphi=\psi/p^n$ and an action of $G_U$. 

On the other hand, as $\cU$ is small, there exists a morphism $\alpha\colon R^+\to \mathbb A_{\cris}(S,S^+)$ of $\cO_k$-algebras, whose composite with the projection $\AAcr(S,S^+)\to S^+$ is the inclusion $R^+\subset S^+$. For example, consider an \'etale morphism $\cU\to \Spf(\cO_k\{T_1^{\pm 1}, \ldots, T_d^{\pm 1}\})$. Let $(T_i^{1/p^{n}})$ be a compatible system of $p^n$-th roots of $T_i$ inside $\overline R^+\subset S^+$, and $T_i^{\flat}$ the corresponding element of $S^{\flat +}:=\varprojlim_{x\mapsto x^p} S^+/pS^+$. Then one can take $\alpha$ as the unique morphism of $\cO_k$-algebras $R^+\to \mathbb A_{\cris}(S,S^+)$ sending $T_i$ to $[T_i^{\flat}]$, such that its composite with the projection $\AAcr(S,S^+)\to S^+/pS^+$ is just the natural map $R^+\to S^+/pS^+$ (such a morphism exists as $R^+$ is \'etale over $\cO_k\{T_1^{\pm 1},\ldots, T_d^{\pm 1}\}$ and because of \eqref{PDThickening}; see the proof of Lemma \ref{algebra} for a similar situation). Now we fix  such a morphism $\alpha$. So we obtain a morphism of PD-thickenings from  $\cU_0\hookrightarrow \cU$ to the one defined by \eqref{PDThickening}. Consequently we get a natural isomorphism 
$
\mathcal M(\mathbb A_{\cris}(S,S^+))\simeq \mathcal M(\cU)\otimes_{R^+,\alpha}\mathbb A_{\cris}(S,S^+)$, whence 
\[
\mathcal E(\mathbb B_{\cris}(S,S^+))\simeq \mathcal E(\cU)\otimes_{R,\alpha}\mathbb B_{\cris}(S,S^+).
\]
Using this isomorphism, we define the filtration on $\cE(\BBcr(S,S^+))$ as the tensor product of the filtration on $\cE(\cU)$ and that on $\BBcr(S,S^+)$. 

\begin{rk}\label{rk.compfiltrations} It is well known that the filtration on $\cE(\BBcr(S,S^+))$ does not depend on the choice of $\alpha$. More precisely, let $\alpha'$ be a second morphism $R^+\to \AAcr(S,S^+)$ of $\cO_k$-algebras whose composite with $\AAcr(S,S^+)\to S^+$ is the inclusion $R^+\subset S^+$. Fix an \'etale morphism $\cU\to \Spf(\cO_k\{T_1^{\pm 1},\ldots, T_d^{\pm 1}\})$. Denote $\beta=(\alpha, \alpha'):R^+\otimes_{\cO_k}R^+\to \AAcr(S,S^+)$ and by the same notation the corresponding map on schemes, and write $p_1, p_2: \Spec R^+\times \Spec R^+\ra \Spec R^+$ the two projections. We have a canonical isomorphism $(p_2\circ \beta)^*\cE\simto (p_1\circ \beta)^*\cE$, as $\cE$ is  a crystal.  In terms of  the connection $\nabla$ on $\cE$, this gives (cf. \cite[2.2.4]{Ber}) the following $\BBcr(S,S^+)$-linear isomorphism 
\[
\eta\colon \cE(\cU)\otimes_{R,\alpha}\mathbb B_{\cris}(S,S^+) \longrightarrow \cE(\cU)\otimes_{R,\alpha' }\BBcr(S,S^+)
\]
sending $e\otimes 1$ to $\sum_{\underline n\in \mathbb N^d} \underline{N}^{\underline n}(e)\otimes (\alpha(\underline T)-\alpha'(\underline T))^{[\underline n]}$, with $\underline N$ the endomorphism of $\cE$ such that $\nabla=\underline N\otimes d\underline T$. Here we use the multi-index to simplify the notations, and note that $\alpha(T_i)-\alpha'(T_i)\in \Fil^1 \AAcr(S,S^+)$ hence the divided power $(\alpha(T_i)-\alpha'(T_i))^{[n_i]}$ is well-defined. Moreover, the series converge since the connection on $\cM$ is quasi-nilpotent. Now as the filtration on $\cE$ satisfies Griffiths transversality, the isomorphism $\eta$ is compatible with the tensor product filtrations on both sides. Since the inverse $\eta^{-1}$ can be described by a similar formula (one just switches $\alpha$ and $\alpha'$), it is also compatible with filtrations on both sides. Hence the isomorphism $\eta$ is strictly compatible with the filtrations, and the filtration on $\cE(\BBcr(S,S^+))$ does not depend on the choice of $\alpha$.  
\end{rk}



Let $\LL$ be a lisse $\widehat{\Z}_p$-sheaf on $X$, and write as before $V_U(\LL)$ the $\Zp$-representation of $G_U$ corresponding to the lisse sheaf $\LL|_U$. Following \cite{Fal}, we say  a filtered convergent $F$-isocrystal \emph{$\cE$ is associated to $\LL$ in the sense of Faltings} if, for all small open subset $\cU\subset \cX$, there is a functorial isomorphism:
\begin{equation}\label{eq.Faltings}
\cE\left(\BBcr\left(S,S^+\right)\right) \stackrel{\sim}{\longrightarrow}  V_{U}(\LL)\otimes_{\Qp}\BBcr\left(S,S^+\right)
\end{equation} 
which is compatible with filtration, $G_U$-action and Frobenius. 

\begin{prop}\label{prop.faltings} Keep the notation above. If $\cE$ is associated to $\LL$ in the sense of Faltings then $\LL$ is crystalline (not necessarily associated to $\cE$) and there is an isomorphism $\mathbb D_{\cris}(\LL)\simeq \cE$ compatible with filtration and Frobenius. 
Conversely,  if  $\LL$ is crystalline and if there is an isomorphism $\mathbb D_{\cris}(\LL)\simeq \cE$ of $\cO_{X^{\an}}$-modules compatible with filtration and Frobenius, then $\LL$ and $\cE$ are associated  in the sense of Faltings.
\end{prop} 



Before giving the proof of Proposition \ref{prop.faltings}, we observe first the following commutative diagram in which  the left vertical morphisms are all PD-morphisms:
\[
\xymatrix{R^+\ar[r]\ar[d]_{\mathrm{can}} & R^+/pR^+ \ar[d] \\ \cO\AAcr\left(S,S^+\right) \ar[r]^{\theta_R} & S^+/pS^+ \ar@{=}[d]\\ \AAcr\left(S,S^+\right)\ar[r]^{\theta} \ar[u]^{\mathrm{can}}& S^+/pS^+}.\]
Therefore, we have isomorphisms 
\begin{eqnarray*}
\cM(\cU)\otimes_{R^+}\cO\AAcr\left(S,S^+\right) & \stackrel{\sim}{\longrightarrow} & \cM\left(\cO\AAcr\left(S,S^+\right)\right) \\ & \stackrel{\sim}{\longleftarrow} &  \cM\left(\AAcr\left(S,S^+\right)\right)\otimes_{\AAcr\left(S,S^+\right)}\cO\AAcr\left(S,S^+\right),
\end{eqnarray*}
where the second term in the first row denotes the evaluation of the crystal $\cM$ at the PD-thickening defined by the PD-morphism $\theta_R$ in the commutative diagram above. Inverting $t$, we obtain a natural isomorphism
\begin{equation}\label{identification}
\cE(\cU)\otimes_{R}\cO\BBcr\left(S,S^+\right)  \stackrel{\sim}{\longrightarrow}   \cE\left(\BBcr\left(S,S^+\right)\right)\otimes\cO\BBcr\left(S,S^+\right),
\end{equation}
where the last tensor product is taken over $\BBcr(S,S^+) $. This isomorphism is clearly compatible with Galois action and Frobenius. By a similar argument as in Remark \ref{rk.compfiltrations} one checks that \eqref{identification} is also strictly compatible with the filtrations. Furthermore, using the identification 
\[
\AAcr\left(S,S^+\right)\{\langle u_1,\ldots, u_d\rangle \}\stackrel{\sim}{\longrightarrow} \cO\AAcr\left(S,S^+\right),\quad u_i\mapsto T_i\otimes 1-1\otimes [T_i^{\flat}]
\]
we obtain a section $s$ of the canonical map $\AAcr\left(S,S^+\right) \to \cO\AAcr\left(S,S^+\right) $:
\[
s\colon \cO\AAcr\left(S,S^+\right) \to \AAcr\left(S,S^+\right)\quad u_i\mapsto 0 
\] 
which is again a PD-morphism. Composing with the inclusion $R^+\subset \cO\AAcr(S,S^+)$, we get a morphism $\alpha_0\colon R^+\to \AAcr(S,S^+)$ whose composite with the projection $\AAcr(S,S^+)\to S^+$ is the inclusion $R^+\subset S^+$.  


\begin{proof}[Proof of Proposition \ref{prop.faltings}]Now assume that $\cE$ is associated with $\LL$ in the sense of \cite{Fal}. Extending scalars to $\cO\BBcr(S,S^+) $ of the isomorphism \eqref{eq.Faltings} and using the identification \eqref{identification}, we obtain a functorial isomorphism, compatible with filtration, $G_U$-action, and Frobenius:
\[
V_{U}(\LL)\otimes_{\Zp}\cO\BBcr\left(S,S^+\right) \stackrel{\sim}{\longrightarrow}\cE(\cU)\otimes_{R} \cO\BBcr\left(S,S^+\right). 
\]
Therefore, $V_{U}(\LL)\otimes_{\Zp}\Qp$ is a crystalline $G_U$-representation (Corollary \ref{CrysRepAff}), and we get by Lemma  \ref{twodcris} an isomorphism $\cE(\cU)\simto\mathbb D_{\cris}(\LL)(\cU)$ compatible with filtrations and Frobenius. As such small open subsets $\cU$ form a basis for the Zariski topology of $\cX$, we find an isomorphism $\cE\simto \mathbb D_{\cris}(\LL)$ compatible with filtrations and Frobenius, and that $\LL$ is crystalline in the sense of Definition \ref{associated} (Corollary \ref{CrysRepAff}).

Conversely, assume $\LL$ is crystalline with $\mathbb D_{\cris}(\LL)\cong \cE$ compatible with filtrations and Frobenius. As in the proof of Corollary \ref{CrysRepAff}, we have a functorial isomorphism 
\[
\cE(\cU)\otimes_R\cO\BBcr \left(S,S^+\right) \stackrel{\sim}{\longrightarrow}V_{U}(\LL)\otimes_{\Zp}\cO\BBcr \left(S,S^+\right) 
\]
which is compatible with filtration, Galois action and Frobenius. Pulling it back via the section $\cO\BBcr\left(S,S^+\right) \to \BBcr\left(S,S^+\right)$ obtained from $s$ by inverting $p$, we obtain a functorial isomorphism 
\[
\cE(\mathbb B_{\cris}(S,S^+))\simeq \cE(\cU)\otimes_{R,\alpha_0} \BBcr(S,S^+)\stackrel{\sim}{\longrightarrow} V_U(\LL)\otimes_{\Zp}\BBcr(S,S^+),
\]
which is again compatible with Galois action, Frobenius and filtrations. Therefore $\LL$ and $\cE$ are associated in the sense of Faltings. 
\end{proof}

Finally we compare Definition \ref{associated} with its de Rham analogue considered in \cite{Sch13}. 

\begin{prop}\label{prop.crisdr} Let $\LL$ be a lisse $\widehat{\Z}_p$-sheaf on $X$ and $\cE$ a filtered convergent $F$-isocrystal on $\cX_0/\cO_k$. Assume that $\LL$ and $\cE$ are associated as defined in Definition \ref{associated}, then $\LL$ is a de Rham in the sense of \cite[Definition 8.3]{Sch13} . More precisely, if we view $\cE$ as a filtered module with integrable connection on $X$ (namely we forget the Frobenius), there exists a natural filtered isomorphism that is compatible with connections:
\[
\LL\otimes_{\widehat{\Z}_p}\cO\BB_{\rm dR}\lra \cE\otimes_{\cO_X} \cO\BB_{\rm dR}
\]  
\end{prop}

\begin{proof} Let $\cU=\Spf(R^+)\subset \cX$ be a connected affine open subset, and denote $U$ (resp. $U^{\rm univ}$) the generic fiber of $\cU$ (resp. the universal \'etale cover of $U$). Let $V$ be a affinoid perfectoid lying above $U^{\rm univ}$. As $\LL$ and $\cE$ are associated, there exits a filtered isomorphism compatible with connections and Frobenius
\[
\LL\otimes_{\widehat{\Z}_p}\cO\BBcr\stackrel{\sim}{\lra} w^{-1}\cE\otimes_{\cO_{X}^{\rm ur}} \cO\BBcr.  
\] 
Evaluate this isomorphism at $V\in X_{\proet}$ and use the fact that the $R$-module $\cE(\cU)$ is projective (here $R:=R^{+}[1/p]$), we deduce a filtered isomorphism compatible with all extra structures:
\[
V_U(\LL)\otimes_{\Zp}\cO\BBcr(V)\stackrel{\sim}{\lra} \cE(U)\otimes_{R} \cO\BBcr(V).  
\] 
Taking tensor product $-\otimes_{\cO\BBcr(V)}\cO\BB_{\rm dR}(V)$ on both sides, we get a filtered isomorphism compatible with connection:
\[
V_U(\LL)\otimes_{\Zp}\cO\BB_{\rm dR}(V)\stackrel{\sim}{\lra} \cE(U)\otimes_{R} \cO\BB_{\rm dR}(V).  
\]
Again, as $\cE(U)$ is a projective $R$-module and as $\cE$ is coherent, the isomorphism above can be rewritten as 
\[
(\LL\otimes_{\widehat{\Z}_p}\cO\BB_{\rm dR})(V)\stackrel{\sim}{\lra} (\cE\otimes_{\cO_X}\cO\BB_{\rm dR})(V), 
\]
which is clearly functorial in $\cU$ and in $V$. Varying $\cU$ and $V$, we deduce that $\LL$ is de Rham, hence  our proposition.  
\end{proof}



\subsection{From pro-\'etale site to \'etale site} Let $\cX$ be a smooth formal scheme over $\cO_k$. %For  an $F$-crystal $\mathcal M$ on $\cX_0/\cO_k$ (resp. $F$-isocrystal $\mathcal E$ on $\cX_0/\cO_k$),  we abuse the notation and denote the associated  $\cO_{X}^{\ur+}$-module (resp. $\cO_{X}^{\ur}$-module) by the same symbol. 
For $\cO=\cO_{\cX}, \cO_{\cX}[1/p], \cO_X^{\ur+},\cO_X^{\ur}$ and a sheaf of $\cO$-modules $\cF$ with connection, we denote the de Rham complex of $\cF$ as:
\[
DR(\cF)=(0\ra \cF
\stackrel{\nabla}{\lra}\cF\otimes_{\cO} \Omega^1\stackrel{\nabla}{\lra}\cdots). 
\] 
Let $\overline{w}$ be the composite of natural morphisms of topoi (here we use the same notation to denote the object in $X_{\proet}^{\sim}$ represented by $X_{\bk}\in X_{\proet}$):
\[
X_{\proet}^{\sim}\slash X_{\bk} \lra X_{\proet}^{\sim}\stackrel{w}{\lra} \mathcal X_{\et}^{\sim}.
\] 
The following lemma is just a global reformulation of the main results of \cite{AB}. As we shall prove a more general result later (Lemma \ref{quasirelative}), let us omit the proof here. 


\begin{lemma}\label{quasi}
Let $\cX$ be smooth formal scheme over $\cO_k$. Then the natural morphism below is an isomorphism in the filtered derived category:
\[
\cO_{\cX}\widehat{\otimes}_{\cO_k}B_{\cris}\lra R\overline{w}_{*}(\cO\BBcr).
\]
Here $\cO_{\cX}\widehat{\otimes}_{\cO_k}B_{\cris}:=\left(\cO_{\cX}\widehat{\otimes}_{\cO_k}A_{\cris}\right)[1/t]$ with 
\[
\cO_{\cX}\widehat{\otimes}_{\cO_k}A_{\cris}:=\varprojlim_{n\in \mathbb N} \cO_{\cX}\otimes_{\cO_k}A_{\cris}/p^n, 
\]
and $\cO_{\cX}\widehat{\otimes}_{\cO_k}B_{\cris}$ is filtered by the subsheaves
\[
\cO\widehat{\otimes}_{\cO_k}\Fil^r B_{\cris}:=\varinjlim_{n\in \mathbb N} t^{-n}(\cO_{\cX}\widehat{\otimes}_{\cO_k} \Fil^{r+n}A_{\cris}), \quad r\in \mathbb Z. 
\]
%for any $j>0$ and $r\in \mathbb Z$, 
%(1) $R^j\overline{w}_{\ast}\cO\BBcr=R^j\overline{w}_{\ast}(\Fil^r\cO\BBcr)=0$; and  
%(2) the natural morphism $\cO_{\cX}\widehat{\otimes}_{\cO_k}\Fil^rB_{\cris}\to \overline{w}_{\ast}(\Fil^r\cO\BBcr)$ is an isomorphism, where $\cO_{\cX}\widehat{\otimes}_{\cO_k}\Fil^r B_{\cris}$ denotes $\varinjlim_{n\geq|r|} t^{-n} (\cO_{\cX}\widehat{\otimes}_{\cO_k} \Fil^{r+n} A_{\cris})$.
\end{lemma}





\begin{cor}\label{quasicor} Let $\cX$ be a smooth formal scheme over $\cO_k$. Let $\LL$ be a crystalline lisse $\widehat{\Z}_p$-sheaf associated with a filtered convergent $F$-isocrystal $\cE$. Then there exists a natural quasi-isomorphism in the filtered derived category 
\[
R\overline{w}_{\ast}(\LL\otimes_{\widehat{\Z}_p} \BBcr)\stackrel{\sim}{\lra} DR(\cE)\widehat{\otimes}_k B_{\cris}. 
\]
If moreover $\cX$ is endowed with a lifting of Frobenius $\sigma$, then the isomorphism above is also compatible with the Frobenii deduced from $\sigma$ on both sides. 
\end{cor}

\begin{proof}  
Using the Poincar\'e lemma (Corollary \ref{poincare}), we get first a quasi-isomorphism which is strictly compatible with filtrations:
\[
\LL\otimes \BBcr\stackrel{\sim}{\lra} \LL\otimes DR(\cO\BBcr)=DR(\LL\otimes \cO\BBcr). 
\]
As $\LL$ and $\cE$ are associated, there is a filtered isomorphism $\LL\otimes \cO\BBcr\simto w^{-1}\cE\otimes_{\cO_X^{\ur}} \cO\BBcr$ compatible with connection and Frobenius, from which we get the quasi-isomorphisms in the filtered derived category 
\begin{equation}\label{eq.quasicor1}
\LL\otimes \BBcr\stackrel{\sim}{\lra}
DR(\LL\otimes \cO\BBcr)\stackrel{\sim}{\lra} DR(w^{-1}\cE\otimes \cO\BBcr). 
\end{equation}
On the other hand, as $R^j\overline{w}_{\ast}\cO\BBcr=0$ for $j>0$ (Lemma \ref{quasi}), we obtain using projection formula that $R^j\overline{w}_{\ast}((w^{-1}\cE\otimes \cO\BBcr)|_{X_{\bk}})=\cE\otimes R^j\overline w_{\ast}\cO\BBcr=0$ (note that $\cE$ is locally a direct factor of a finite free $\cO_{\cX}[1/p]$-module, hence one can apply projection formula here). In particular, each component of $DR(w^{-1}\cE\otimes \cO\BBcr)$ is $\overline{w}_{\ast}$-acyclic. Therefore, 
\[
DR(\cE\otimes \overline{w}_{\ast}\cO\BBcr)\stackrel{\sim}{\lra}\overline w_{\ast}(DR(w^{-1}\cE\otimes \cO\BBcr))\stackrel{\sim}{\lra} R\overline{w}_{\ast} (DR(w^{-1}\cE\otimes \cO\BBcr)).
\]
Combining this with Lemma \ref{quasi}, we deduce the following quasi-isomorphisms in the filtered derived category
\begin{equation}\label{eq.quasicor2}
DR(\cE)\widehat{\otimes}_k B_{\cris}\stackrel{\sim}{\lra} DR(\cE\otimes\overline{w}_{\ast}\cO\BBcr )\stackrel{\sim}{\lra} R\overline{w}_{\ast} (DR(w^{-1}\cE\otimes \cO\BBcr)).
\end{equation}

The desired quasi-isomorphism follows from \eqref{eq.quasicor1} and \eqref{eq.quasicor2}. When furthermore $\cX$ admits a lifting of Frobenius $\sigma$, one checks easily  that both quasi-isomorphisms are compatible with Frobenius, hence the last part of our corollary. 
\end{proof}

\begin{rk} Recall that $G_k$ denotes the absolute Galois group of $k$. Each element of $G_k$ defines a morphism of $U_{\bk}$ in the pro-\'etale site $X_{\proet}$ for any $\cU\in \cX_{\proet}$ with $U:=\cU_k$. Therefore, the object $R\overline{w}_{\ast}(\LL\otimes \BBcr)$ comes with  a natural Galois action of $G_k$. With this Galois action, one checks that the quasi-isomorphism in Corollary \ref{quasicor} is also Galois equivariant. 
\end{rk}


Let $\cE$ be a filtered convergent $F$-isocrystal on $\cX_0/\cO_k$, and $\cM$ an $F$-crystal on $\cX_0/\cO_k$  (viewed as a coherent $\cO_{\cX}$-module equipped with an integrable connection) such that $\mathcal E\simeq \mathcal M ^{\rm  an}(n)$ for some $n\in \mathbb N$ (Remark \ref{rk.CrystalVSIsoc}). The crystalline cohomology group $H^{i}_{\cris}(\cX_0/\cO_k,\cM)$ is an $\cO_k$-module of finite type endowed with a Frobenius $\psi$. In the following, the crystalline cohomology (or more appropriately, the rigid cohomology) of the convergent $F$-isocrystal $\cE$ is defined as  
\[
H^{i}_{\cris}(\cX_0/\cO_k, \cE):=H^i_{\cris}(\cX_0/\cO_k, \cM)[1/p].
\]
It is a finite dimensional $k$-vector space equipped with the Frobenius $\psi/p^n$. Moreover, let $u=u_{\cX_0/\cO_k}$ be the following morphism of topoi
\[
(\cX_0/\cO_k)_{\cris}^{\sim}\lra \cX_{\et}^{\sim}
\]
such that $u_{\ast}(\mathcal F)(\cU)=H^0((\cU_0/\cO_k)_{\cris}^{\sim},\mathcal F)$ for $\cU\in \cX_{\et}$. With the \'etale topology replaced by the Zariski topology, this is precisely the morphism $u_{\cX_0/\hat{S}}$ (with $\hat{S}=\Spf(\cO_k)$) considered in \cite[Theorem 7.23]{BO}. By \emph{loc.cit.}, there exists a natural quasi-isomorphism in the derived category 
\begin{equation}\label{eq.derivedBO}
Ru_{\ast} \cM\stackrel{\sim}{\lra} DR(\cM),
\end{equation}
which induces a natural isomorphism $H^i_{\cris}(\cX_0/\cO_k,\cM)\simto \mathbb H^i(\cX,DR(\cM))$. Thereby 
\begin{equation}\label{eq.BO}
H^i_{\cris}(\cX_0/\cO_k,\cE)\stackrel{\sim}{\lra} \mathbb H^i(\cX,DR(\cE)).
\end{equation}
On the other hand,  the de Rham complex $DR(\cE)$ of $\cE$ is filtered by its subcomplexes 
\[
\Fil^r DR(\cE):=(\Fil^r \cE\stackrel{\nabla}{\lra} \Fil^{r-1}\cE\otimes \Omega_{X/k}^{1}\stackrel{\nabla}{\lra}\ldots).
\]
So the hypercohomology $\mathbb H^i(\cX,DR(\cE))$ has a descending filtration given by 
\[
\Fil^r \mathbb H^i(\cX,DR(\cE)):=\mathrm{Im}\left(\mathbb H^i(\cX,\Fil^r DR(\cE))\lra \mathbb H^i(\cX,DR(\cE))\right). 
\]
Consequently, through the isomorphism \eqref{eq.BO}, the $k$-space $H^i_{\cris}(\cX_0/\cO_k,\cE)$ is endowed naturally with a decreasing filtration.  

\begin{thm} \label{main1} Assume further that the smooth formal scheme $\cX$ is proper over $\cO_k$. Let $\mathcal E$ be a filtered convergent $F$-isocrystal on $\cX_0/\cO_k$ and $\LL$ a lisse $\widehat{\ZZ}_p$-sheaf on $X_{\proet}$. Assume that $\mathcal E$ and $\LL$ are associated. Then there is a natural filtered isomorphism of $B_{\cris}$-modules
\begin{equation}\label{eq.iso}
H^i(X_{\bk,\proet},\LL\otimes_{\widehat{\ZZ}_p}\BB_{\cris}) \stackrel{\sim}{\longrightarrow} H_{\cris}^i(\cX_0/\cO_k, \cE)\otimes_{k} B_{\cris}
\end{equation}
which is compatible with Frobenius and Galois action.
\end{thm}


\begin{proof} By Corollary \ref{quasicor}, we have the natural Galois equivariant quasi-isomorphism in the filtered derived category:
\[
R\Gamma(X_{\bk,\proet},\LL\otimes \BBcr)=R\Gamma(\cX,R\overline{w}_{\ast}(\LL\otimes \BBcr))\stackrel{\sim}{\lra}R\Gamma(\cX,DR(\cE)\widehat{\otimes}_k B_{\cris}). 
\]
We first claim that the following natural morphism in the filtered derived category is an isomorphism:
\begin{equation*}%\label{eq.iso2}
R\Gamma(\cX,DR(\cE))\otimes_{\cO_k} A_{\cris}\lra R\Gamma(\cX_{\et}, DR(\cE)\widehat{\otimes}_{\cO_k}A_{\cris}).
\end{equation*}
Indeed, as $A_{\cris}$ is flat over $\cO_k$, the morphism above is an isomorphism respecting the filtrations in the derived category. Thus to prove our claim, it suffices to check that the morphism above induces quasi-isomorphisms on gradeds. Further filtering the de Rham complex by its naive filtration, we are reduced to checking the following isomorphism for $\mathcal A$ a coherent $\cO_{\cX}$-module:
\[
R\Gamma(\cX,\mathcal A)\otimes_{\cO_k} \cO_{\mathbb C_p} \stackrel{\sim}{\lra} R\Gamma(\cX,\mathcal A\widehat{\otimes}_{\cO_k}\cO_{\mathbb C_p)}, 
\]
which holds because again $\cO_{\mathbb{C}_p}$ is flat over $\cO_k$. Consequently, inverting $t$ we obtain an isomorphism in the filtered derived category 
\begin{equation*}%\label{eq.iso2}
R\Gamma(\cX,DR(\cE))\otimes_{k} B_{\cris}\lra R\Gamma(\cX_{\et}, DR(\cE)\widehat{\otimes}_{k}B_{\cris}).
\end{equation*}
In this way, we get a Galois equivariant quasi-isomorphism in the filtered derived category
\[
R\Gamma(X_{\bk,\proet},\LL\otimes \BBcr) \stackrel{\sim}{\lra} R\Gamma(\cX,DR(\cE))\otimes_k B_{\cris}.
\] 
Combining it with \eqref{eq.BO}, we obtain the isomorphism \eqref{eq.iso} verifying the required properties except for the Frobenius compatibility. 

To check the Frobenius compatibility, we only need to check that the restriction to $H^i_{\cris}(\cX_0/\cO_k,\cE)\hookrightarrow H^i_{\cris}(\cX_0/\cO_k,\cE)\otimes_{k} B_{\cris}$ of the inverse of \eqref{eq.iso} is Frobenius-compatible. Let $\cM$ be an $F$-crystal on $\cX_0/\cO_k$ such that $\cE=\cM^{\an}(n)$. Via the identification $H^i_{\cris}(\cX_0/\cO_k,\cE)=H^i_{\cris}(\cX_0/\cO_k,\cM)[1/p]$, the restriction map in question is induced from the following composite of morphisms at the level of derived category:
\begin{eqnarray*}
Ru_{\ast} \cM \stackrel{\sim}{\lra} DR(\cM)\lra  DR(\cM)\widehat{\otimes}_{\cO_k}B_{\cris}\simeq DR(\cE)\widehat{\otimes}_{k}B_{\cris}
\stackrel{\sim}{\lra} R\overline{w}_{\ast} (\LL\otimes \BBcr),
\end{eqnarray*} 
where the first quasi-isomorphism is \eqref{eq.derivedBO}, and the last morphism is just the inverse in the derived category of the quasi-isomorphism in Corollary \ref{quasicor}. Let us denote by $\theta$ the composite of these morphisms. 
Let $\psi$ (resp. $\varphi$) be the induced Frobenius on $Ru_{\ast}\cM$ (resp. on $R\overline{w}_{\ast}(\LL\otimes \BBcr)$). One only needs to check that $\varphi \circ \theta=\frac{1}{p^n}\theta\circ \psi$. This can be checked locally on $\cX$. So let $\cU\subset \cX$ be a small open subset equipped with a lifting of Frobenius $\sigma$. Thus $\cM|_{\cU}$ (resp. $\cE|_{\cU}$) admits naturally a Frobenius, which we denote by $\psi_{\cU}$ (resp.  $\varphi_{\cU}$). Then all the morphisms above except for the identification in the middle are Frobenius-compatible (see Corollary \ref{quasicor} for the last quasi-isomorphism). But by definition, under the identification $\cM[1/p]|_{\cU}\simeq \cE|_{\cU}$, the Frobenius $\varphi_{\cU}$ on $\cE$ corresponds exactly to $\psi_{\cU}/p^n$ on $\cM[1/p]$. This gives the desired equality $\varphi\circ \theta=\frac{1}{p^n}\theta\circ \psi$ on $\cU$, from which  the  Frobenius compatibility in \eqref{eq.iso} follows. 
\end{proof}














 


 



\section{Primitive comparison on the pro-\'etale site}
Let $\cX$ be a proper smooth formal scheme over $\cO_k$, with $X$ (resp. $\cX_0$) its generic (resp. closed) fiber. Let $\LL$ be a lisse $\widehat{\ZZ}_p$-sheaf on $X_{\proet}$.
In this section, we will construct a primitive  comparison isomorphism for any lisse $\widehat{\ZZ}_p$-sheaf $\LL$ on the pro-\'etale site $X_{\proet}$ (Theorem \ref{inout}). In particular, this primitive comparison isomorphism also holds for non-crystalline  lisse $\widehat{\ZZ}_p$-sheaves, which may lead to interesting arithmetic applications. On the other hand, in the case that $\LL$ is crystalline, such a result and Theorem \ref{main1} together give rise to the crystalline comparison isomorphism between \'etale cohomology and crystalline cohomology. 

We shall begin with some preparations. The first lemma is well-known. 



\begin{lemma}\label{jannsen} Let $(\mathcal F_n)_{n\in \mathbb N}$ be a projective system of abelian sheaves on a site $T$, such that $R^j\varprojlim \mathcal F_n=0$ whenever $j>0$.  %Assume that there exists a base $B$ of $T$ such that for any $U\in B$, we have
%\begin{itemize}
%\item $R^1\varprojlim \mathcal F_n(U)=0$; and 
%\item $H^i(U,\mathcal F_n)=0$ for any $i>0$.
%\end{itemize} 
Then for any object $Y\in T$ and any $i\in \mathbb Z$, the following sequence is exact:
\[
0\lra R^1\varprojlim H^{i-1}(Y,\mathcal F_n)\lra H^i(Y,\varprojlim \mathcal F_n)\lra \varprojlim H^i(Y,\mathcal F_n)\lra 0. 
\]
\end{lemma}

\begin{proof} Let $\mathrm{Sh}$ (resp. $\mathrm{PreSh}$) denote the category of abelian sheaves (resp. abelian presheaves) on $T$, and $\mathrm{Sh}^{\mathbb N}$ (resp. $\mathrm{PreSh}^{\mathbb N}$) the category of projective systems of abelian sheaves (resp. of abelian presheaves) indexed by $\mathbb N$. Let $\mathrm{Ab}$ denote the category of abelian groups. Consider the functor 
\[
\tau \colon \mathrm{Sh}^{\mathbb N}\longrightarrow \mathrm{Ab}, \quad (\mathcal G_n) \mapsto \varprojlim \Gamma(Y,\mathcal G_n).
\]
Clearly $\tau$ is left exact, hence we can consider its right derived functors. Let us compute $R\tau (\mathcal F_n)$ in two different ways. 

Firstly, one can write $\tau$ as the composite of the following two functors 
\[
\xymatrix{\mathrm{Sh}^{\mathbb N}\ar[r]^{\varprojlim} &  \mathrm{Sh}\ar[r]^{\Gamma(U,-)}\ar[r] & \mathrm{Ab}}.
\]
Since the projective limit functor $\varprojlim \colon \mathrm{Sh}^{\mathbb N}\to \mathrm{Sh}$ admits an exact left adjoint given by constant projective system, it sends injectives to injectives. Thus we obtain a spectral sequence 
\[
E_2^{i,j}=H^i(Y,R^j\varprojlim \mathcal F_n)\Longrightarrow R^{i+j}\tau (\mathcal F_n). 
\]
By %\cite[Lemma 3.18]{Sch13}, 
assumption, $R^j\varprojlim \mathcal F_n=0$ whenver $j>0$. So the spectral sequence above degenerates at $E_2$ and we get $H^i(Y,\varprojlim \mathcal F_n)\simeq R^{i}\tau (\mathcal F_n)$. 

Secondly, one can equally decompose $\tau$ as follows:
\[
\xymatrix{\mathrm{Sh}^{\mathbb{N}}\ar[r]^{\alpha} & \mathrm{Ab}^{\mathbb N}\ar[r]^{\varprojlim} & \mathrm{Ab}},
\]
where the functor $\alpha$ sends $(\mathcal G_n)$ to the projective system of abelian groups $(\Gamma(Y,\mathcal G_n))$. Let $I_{\bullet}=(I_n)$ be an injective object of $\mathrm{Sh}^{\mathbb N}$. By \cite[Proposition 1.1]{Jan}, each component $I_n$ is an injective object of $\mathrm{Ab}$ and the transition maps $I_{n+1}\to I_n$ are \emph{split} surjective. Therefore, the transition maps of the projective system $\alpha(I_{\bullet})$ are also split surjective. In particular, the projective system $\alpha(I_{\bullet})$ is $\varprojlim$-acyclity. So we can consider the following spectral sequence 
\[
E_{2}^{i,j}=R^i\varprojlim H^j(Y,\mathcal F_n)\Longrightarrow R^{i+j}\tau (\mathcal F_n). 
\]
Since the category $\mathrm{Ab}$ satisfies the axiom (AB5*) of abelian categories (\emph{i.e.,} infinite products are exact), $R^{i}\varprojlim A_n=0$ ($i\notin \{0,1\}$) for any projective system of abelian groups $(A_n)_n$. So the previous spectral sequence degenerates at $E_2$, from which we deduce a natural short exact sequence for each $i\in \mathbb Z$
\[
0\lra R^1\varprojlim H^{i-1}(Y,\mathcal F_n)\lra R^i\tau (\mathcal F_n)\lra \varprojlim H^i(Y,\mathcal F_n)\lra 0. 
\]
As we have seen $H^i(Y, \varprojlim \mathcal F_n)\simeq R^i\tau (\mathcal F_n)$, we deduce the short exact sequence as asserted by our lemma.  
\end{proof}



%\[F: \mathcal{C}^{\N}\ra \mathrm{Ab}^{\N},  \quad (\cF_i,d_i)\ra (\Gamma(X,\cF_i),d_i)\]and the projective limit functor 
%\[G: \mathrm{Ab}^{\N}\ra \mathrm{Ab}, \quad (\cG_i,s_i)\ra \varprojlim \cG_i.\]
%Furthermore, as we consider only projective system indexed by $\N$, $R^jG: \mathrm{Ab}^{\N}\ra\mathrm{Ab}$ vanishes for $j\geq 2$; see the paragraph before the formula \cite[(1.4)]{Jan}.


%\begin{lemma}\label{318}

%Let $T$ be a site and $(\cF_i)_{i\in \N}$ an inverse system of abelian sheaves on $T$. Supospe there is a basis for the site $T$ such that for any $U$ in the basis 
%\[R^1\varprojlim(\cF_i(U))=0, \forall i\in \N, \quad H^j(U,\cF_i)=0, \forall j\geq 1.\]
%Then \[\forall j\geq 1,R^j\varprojlim\cF_i=0,\quad \varprojlim(\cF_i(V))=(\varprojlim\cF_i)(V) \text{ for any } V\in T, \]and for any $U$ in the basis, 
%\[H^j(U,\varprojlim\cF_i)=0, \forall j\geq 1.\]\end{lemma}

\begin{lemma}\label{lem.finiteness} Let $\LL$ be a lisse $\widehat{\ZZ}_p$-sheaf on $X_{\bk,\proet}$. Then for $i\in \mathbb Z$, $H^i(X_{\bk, \proet},\LL)$ is a $\Zp$-module of finite type, and $H^i(X_{\bk,\proet},\LL)=0$ whenever $i\notin [0,2 \dim(X)]$. 
\end{lemma}
\begin{proof} Since all the cohomology groups below are computed in the pro-\'etale site, we shall omit the subscript {\textquotedblleft pro\'et\textquotedblright} from the notations. 

Let $\LL_{\textrm{tor}}$ denote the torsion subsheaf of $\LL$. Then our lemma follows from the corresponding statements for $\LL_{\textrm{tor}}$ and for $\LL/\LL_{\textrm{tor}}$. Therefore, we may assume either $\LL$ is  torsion or is locally on $X_{\bk,\proet}$ free of finite rank over $\widehat{\ZZ}_p$. In the first case, we reduce immediately to the finiteness statement of Scholze (Theorem 5.1 of \cite{Sch13}). So it remains to consider the case when $\LL$ is locally on $X_{\bk,\proet}$ free of finite rank over $\widehat{\ZZ}_p$. Let $\LL_n:=\LL/p^n\LL$. We have the  tautological exact sequence 
\[
0\longrightarrow  \LL \stackrel{p^n}{\longrightarrow} \LL\longrightarrow \LL_n\longrightarrow 0,
\] 
inducing the following short exact sequence 
\begin{equation}\label{eq.exact0}
0\longrightarrow H^i(X_{\bk},\LL)/p^n \longrightarrow H^i(X_{\bk},\LL_n)\longrightarrow H^{i+1}(X_{\bk},\LL)[p^n]\longrightarrow 0. 
\end{equation}

We shall show that the following natural morphism is an isomorphism: 
\begin{equation}\label{exactlim}
H^i(X_{\bk},\LL)\lra \varprojlim H^i(X_{\bk},\LL_n).
\end{equation}
Indeed, as $\LL$ is locally free, using \cite[Proposition 8.2]{Sch13}, we find $R^j\varprojlim_n \LL_n=0$ for $j>0$. Moreover, as $H^{i-1}(X_{\bk},\LL_n)$ are finite $\mathbb Z/p^n\mathbb Z$-modules by \cite[Theorem 5.1]{Sch13}, $R^1\varprojlim H^{i-1}(X_{\bk},\LL_n)=0$. Consequently, by Lemma \ref{jannsen}, the morphism \eqref{exactlim} is an isomorphism. 

%whence the following short exact sequence of pro-\'etale sheaves (see the exact sequence (1.4) of \cite{Jan})
%\[
%0\lra \varprojlim \LL_n \lra \prod \LL_n\stackrel{\alpha}{\lra} \prod \LL_n\lra R^1\varprojlim \LL_n=0
%\]
%where the second map $\alpha$ sends $(x_n)_n$ to $(f_{n+1}(x_{n+1})-x_n)_n$, and $f_n: \LL_n\ra \LL_{n-1}$ is the transition map. The morphism $\alpha$ induces a map between the cohomology groups:
%\[
%\alpha'\colon 
%H^i(X_{\bk},\prod \LL_n)\ra H^i(X_{\bk},\prod \LL_n),
%\]
%whose cokernel is $R^1\varprojlim_n H^i(X_{\bk}, \LL_n)$ (see the exact sequence (1.4) of \cite{Jan}). As $H^i(X_{\bk},\LL_n)$ are  finite groups by \cite[Theorem 5.1]{Sch13}, the higher derived projective limit $R^1\varprojlim H^i(X_{\bk}, \LL_n)$ vanishes. So $\alpha'$ is surjective, and we deduce the short exact sequence for each $i$: 
%\[
%0\lra H^i(X_{\bk},\varprojlim \LL_n) \lra H^i(X_{\bk},\prod \LL_n)\stackrel{\alpha'}{\lra} H^i(X_{\bk}, \prod \LL_n)\ra 0.
%\]
%In other words, the morphism \eqref{exactlim} is an isomorphism. 


In particular, $H^i(X_{\bk},\LL)\stackrel{\sim}{\to} \varprojlim H^i(X_{\bk},\LL_n)$ is a pro-$p$ abelian group, hence it does not contain any element infinitely divisible by $p$. Thus, $\varprojlim  (H^i(X_{\bk},\LL)[p^n])=0$ (where the transition map is multiplication by $p$). From the exactness of \eqref{eq.exact0}, we then deduce a canonical isomorphism 
\[
\varprojlim \left(H^i(X_{\bk},\LL)/p^n\right)\simto \varprojlim H^{i}(X_{\bk},\LL_n).
\]
So $H^i(X_{\bk},\LL)\simto \varprojlim_n H^i(X_{\bk},\LL)/p^n$. Consequently the $\Zp$-module $H^i(X_{\bk},\LL)$ is $p$-adically complete, and it can be generated as a $\Zp$-module by a family of elements whose images in $H^i(X_{\bk},\LL)/p$ generate it as an $\Fp$-vector space. Since the latter is finite dimensional over $\Fp$, the $\Zp$-module $H^i(X_{\bk},\LL)$ is of finite type, as desired.   
\end{proof}




The primitive form of the comparison isomorphism on the pro-\'etale site is as follows.

\begin{thm} \label{inout}Let $\LL$ be a lisse $\widehat{\ZZ}_p$-sheaf on $X_{\proet}$. There is a canonical isomorphism of $B_{\cris}^+$-modules
\begin{equation}\label{eq.PrimitiveIso}
H^i(X_{\bk,\proet},\LL)\otimes_{\Zp}B_{\cris}^{+}\simto H^i(X_{\bk,\proet},\LL\otimes_{\widehat{\ZZ}_p}\mathbb B_{\cris}^{+})
\end{equation}
compatible with Galois action, filtration and Frobenius. 
\end{thm}


\begin{proof} 

In the following, all the cohomologies are computed on the pro-\'etale site, hence we  omit the subscript {\textquotedblleft \textrm{pro\'et}\textquotedblright} from the notations. 

The proof begins with the almost isomorphism in \cite[Theorem 8.4]{Sch13}: 
\begin{equation*}
H^i(X_{\bk},\LL)\otimes_{\Zp} A_{\inf} \stackrel{\approx}{\longrightarrow} H^i(X_{\bk},\LL\otimes_{\widehat{\ZZ}_p}\mathbb A_{\inf}). 
\end{equation*}
Therefore, setting 
\[
\widetilde{A_{\cris}^0}:=\frac{A_{\inf}[X_i:i\in \mathbb N]}{(X_i^p-a_iX_{i+1}:i\in \mathbb N)} \quad \textrm{and}\quad \widetilde{\mathbb A_{\cris}^0}:=\frac{\mathbb A_{\inf}[X_i:i\in \mathbb N]}{(X_i^p-a_iX_{i+1}:i\in \mathbb N)} 
\]
with $a_i=\frac{p^{i+1}!}{(p^i!)^p}$ and using the fact that $X_{\bk}$ is qcqs,  
we find the following almost isomorphism
\begin{equation}\label{eq.iso_Acristilde}
H^i(X_{\bk},\LL)\otimes_{\Zp} \widetilde{A_{\cris}^0}\stackrel{\approx}{\longrightarrow}  H^i\left(X_{\bk},\LL\otimes_{\widehat{\ZZ}_p}\widetilde{\mathbb A_{\cris}^0}\right). 
\end{equation}
On the other hand, from the tautological short exact sequence 
\[
0\longrightarrow \widetilde{A_{\cris}^0}\stackrel{X_0-\xi}{\longrightarrow} \widetilde{A_{\cris}^0}\longrightarrow A_{\cris}^0\longrightarrow 0
\]
and the fact that $A_{\cris}^0$ is flat over $\Zp$, we deduce a short exact sequence: 
\[
0\longrightarrow H^i(X_{\bk},\LL)\otimes_{\Zp}\widetilde{A_{\cris}^0}\stackrel{X_0-\xi}{\longrightarrow} H^i(X_{\bk},\LL)\otimes_{\Zp}\widetilde{A_{\cris}^0}\longrightarrow H^i(X_{\bk},\LL)\otimes_{\Zp} A_{\cris}^0\longrightarrow 0.
\]
As a result, together with the almost  isomorphism \eqref{eq.iso_Acristilde}, we see that the morphism 
\[
\alpha_i\colon  H^i\left(X_{\bk},\LL\otimes \widetilde{\mathbb A_{\cris}^0}\right)\lra H^i\left(X_{\bk},\LL\otimes \widetilde{\mathbb A_{\cris}^0}\right)
\]
induced by multiplication-by-$(X_0-\xi)$ is almost injective for all $i\in \mathbb Z$. Similarly, since $\AAcr^0$ is flat over $\widehat{\Z}_p$, the following sequence of abelian sheaves on $X_{\proet}/X_{\bk}$ is exact:
\[
0\longrightarrow (\LL\otimes_{\widehat{\ZZ}_p}\widetilde{\mathbb A_{\cris}^0})|_{X_{\bk}}\stackrel{X_0-\xi}{\longrightarrow}  (\LL\otimes_{\widehat{\ZZ}_p}\widetilde{\mathbb A_{\cris}^0})|_{X_{\bk}}
\longrightarrow  (\LL\otimes_{\widehat{\ZZ}_p}\mathbb A_{\cris}^{0})|_{X_{\bk}}
\longrightarrow 0,
\]
giving the associated long exact sequence on the cohomology: 
\[
\cdots \ra H^i(X_{\bk}, \LL\otimes_{\widehat{\ZZ}_p}\widetilde{\mathbb A_{\cris}^0})\stackrel{\alpha_i}{\longrightarrow} H^i(X_{\bk},  \LL\otimes_{\widehat{\ZZ}_p}\widetilde{\mathbb A_{\cris}^0})
\ra  H^i(X_{\bk}, \LL\otimes_{\widehat{\ZZ}_p}\mathbb A_{\cris}^{0})\ra \cdots. 
\]
As the $\alpha_i$'s are almost injective for all $i$, the previous long exact sequence splits into short \emph{almost} exact sequences
\[
0 \lra H^i(X_{\bk}, \LL\otimes_{\widehat{\ZZ}_p}\widetilde{\mathbb A_{\cris}^0})\stackrel{\alpha_i}{\longrightarrow} H^i(X_{\bk},  \LL\otimes_{\widehat{\ZZ}_p}\widetilde{\mathbb A_{\cris}^0})
\lra  H^i(X_{\bk}, \LL\otimes_{\widehat{\ZZ}_p}\mathbb A_{\cris}^{0})\lra  0. 
\]
Thus the following natural morphism is an almost isomorphism\begin{equation}\label{Acris0}
H^i(X_{\bk},\LL)\otimes_{\Zp} A_{\cris}^{0} \stackrel{\approx}{\longrightarrow }H^i(X_{\bk},\LL\otimes_{\widehat{\ZZ}_p}\mathbb A_{\cris}^{0}), \quad \forall i\geq 0.  
\end{equation}


To pass to $p$-adic completion, we remark that, by Lemma \ref{lem.finiteness}, the $\Zp$-module $H^i(X_{\bk},\LL)$ is of finite type and vanishes when $i\notin [0,2\dim X]$. Let $N$ be an integer such that the torsion part $\LL_{\rm tor}$ of $\LL$ and for all $i$ the torsion parts $H^i(X_{\bk},\LL)_{\textrm{tor}}$ of $H^i(X_{\bk},\LL)$ are annilated by $p^N$. For $n>N$ an integer, we claim that the following natural morphism has kernel and cokernel killed by $p^{2N}$: 
\begin{equation}\label{eq.iso_n}
H^i(X_{\bk},\LL)\otimes_{\Zp} (A_{\cris}^{0}/p^n) \longrightarrow H^i(X_{\bk},\LL\otimes_{\widehat{\ZZ}_p}\mathbb A_{\cris}^{0}/p^n), 
\end{equation}
or equivalently (via the isomorphism \eqref{Acris0}),
that the natural morphism below has kernel and cokernel killed by $p^{2N}$: 
\begin{equation}\label{eq.iso_n11}
H^i(X_{\bk},\LL\otimes \AAcr^0)/p^n \longrightarrow H^i(X_{\bk},\LL\otimes_{\widehat{\ZZ}_p}\mathbb A_{\cris}^{0}/p^n). 
\end{equation}
To see this, consider the following tautological exact sequence 
\[
\LL\otimes \mathbb A_{\cris}^{0}\stackrel{p^n}{\longrightarrow} \LL\otimes \mathbb A_{\cris}^0\longrightarrow \LL\otimes\mathbb A_{\cris}^0/p^n\longrightarrow 0.
\]
Let $\mathbb K_n:=\ker(\LL\otimes\AAcr^0\stackrel{p^n}{\to}\LL\otimes\AAcr^0)$. Then $\mathbb K_n$ is isomorphic to 
$\mathcal{T}or_{\Zp}^{1}(\LL,\AAcr^0/p^n)\simeq \mathcal{T}or_{\Zp}^{1}(\LL_{\rm tor},\AAcr^0/p^n)$, thus is killed by $p^N$. Let $\mathbb{I}_n:=p^n(\LL\otimes \AAcr^0)\subset \LL\otimes \AAcr^0$. So we have two short exact sequences:
\[
0\lra \mathbb K_n\lra \LL\otimes \AAcr^0\lra \mathbb I_n\lra 0,
\]
and 
\[
0\lra \mathbb I_n\lra \LL\otimes \AAcr^0 \lra \LL\otimes \AAcr^0/p^n\lra 0. 
\]
Taking cohomology one gets exact sequences
\[
\ldots\lra H^i(X_{\bk},\mathbb K_n)\lra H^i(X_{\bk},\LL\otimes\AAcr^0)\stackrel{\gamma_i}{\lra} H^i(X_{\bk},\mathbb I_n)\lra H^{i+1}(X_{\bk},\mathbb K_n)\lra \ldots 
\]
and 
\[
\ldots \lra H^i(X_{\bk},\mathbb I_n)\stackrel{\beta_i}{\lra} H^i(X_{\bk},\LL\otimes\AAcr^0)\lra H^i(X_{\bk},\LL\otimes\AAcr^0/p^n)\lra \ldots, 
\] 
which give rise to the exact sequence below:
\[
H^i(X_{\bk},\LL\otimes \mathbb A_{\cris}^0)/p^n\longrightarrow H^i(X_{\bk},\LL\otimes \mathbb A_{\cris}^0/p^n)\longrightarrow \ker(\beta_{i+1})\lra 0,
\] 
such that the kernel of the first morphism is killed by $p^N$. On the other hand, we have the following commutative diagram 
\[
\xymatrix{H^{i+1}(X_{\bk},\mathbb{I}_n)\ar[r]^<<<<<<<{\beta_{i+1}} & H^{i+1}(X_{\bk},\LL\otimes\AAcr/p^n) \\ H^{i+1}(X_{\bk},\LL\otimes \AAcr)\ar[u]^{\gamma_{i+1}}\ar[ur]_{p^{n}} & }.
\]
Note that $\ker(H^{i+1}(X_{\bk},\LL\otimes \AAcr^{0})\stackrel{p^n}{\to}H^{i+1}(X_{\bk},\LL\otimes \AAcr^0))$ is, via the almost isomorphism \eqref{Acris0} for $H^{i+1}$, almost isomorphic to $
\textrm{Tor}_{\Zp}^{1}(H^{i+1}(X_{\bk},\LL)_{\mathrm{tor}}, A_{\cris}^{0}/p^n)$, hence is killed by $p^N$. Moreover, $\mathrm{coker}(\gamma_{i+1})$ is contained in $H^{i+1}(X_{\bk},\mathbb K_n)$, thus is also killed by $p^N$. As a result, from the commutative diagram above we deduce that $\ker(\beta_{i+1})$ is killed by $p^{2N}$, giving our claim. 






%Since $H^i(X_{\bk},\LL)\otimes_{\Zp}  (A_{\cris}^0/p^n) \stackrel{\approx}{\to} H^i(X_{\bk},\LL\otimes \mathbb A_{\cris}^0)/p^n $ by \eqref{Acris0}, we deduce from the exact sequence above that the morphism \eqref{eq.iso_n} is almost injective. Moreover its cokernel is almost isomorphic to $H^{i+1}(X_{\bk},\LL\otimes A_{\cris}^0)[p^n]$, which, via the almost isomorphism \eqref{Acris0} for $H^{i+1}$, is almost isomorphic to 
%\[
%\left(H^{i+1}(X_{\bk},\LL)\otimes A_{\cris}^{0}\right)[p^n] \simeq \textrm{Tor}_{\Zp}^{1}(H^{i+1}(X_{\bk},\LL)_{\mathrm{tor}}, A_{\cris}^{0}/p^n).
%\]
%Therefore the cokernel of \eqref{eq.iso_n} are killed by $p^N$, as desired. 

Now we claim that the following canonical map 
\begin{equation}\label{hlim}
H^i(X_{\bk},\LL\otimes_{\widehat{\ZZ}_p} \mathbb A_{\cris})\longrightarrow \varprojlim H^i(X_{\bk},\LL\otimes_{\widehat{\ZZ}_p} \mathbb A_{\cris}/p^n)
\end{equation}
is almost surjective with kernel killed by $p^{2N}$. Since $\LL$ is a lisse $\widehat{\ZZ}_p$-sheaf, it is locally a direct sum of a finite free $\widehat{\Z}_p$-module and a finite product of copies of $\widehat{\Z}_p/p^m$'s. In particular, %we deduce first from Lemma \ref{vanish} that there exists a basis $\mathcal T$ of $X_{\proet}/X_{\bk}$, such that for each $U\in \mathcal T$ we have for all $n\in \mathbb N$, 
%\[
%H^i(U,\LL\otimes \mathbb A_{\cris}^a/p^n)=0 \quad \forall i>0,\quad  \textrm{and}\quad R^1\varprojlim \left((\LL\otimes \mathbb A_{\cris}^a/p^n)(U)\right)=0.
%\] 
%Thus by Lemma 3.18 of \cite{Sch13}, 
$R^j\varprojlim (\LL\otimes \mathbb A_{\cris}^{a}/p^n)=0$ whenever $j>0$ as the same holds for the projective system $\{\AAcr^a/p^n\}_{n}$ (recall \eqref{Rjvanish}). %\footnote{State it somewhere in 2.5} 
 As a result, by the almost version of Lemma \ref{jannsen}, 
we dispose the following almost short exact sequence for each $i$: 
\[
0\longrightarrow R^1\varprojlim H^ {i-1}(X_{\bk}, \LL\otimes \AAcr/p^n)\longrightarrow H^i(X_{\bk},\LL\otimes \AAcr)\longrightarrow \varprojlim H^{i}(X_{\bk},\LL\otimes \AAcr/p^n)\longrightarrow 0
\]
On the other hand, the morphisms \eqref{eq.iso_n} for all $n$ give rise to a morphism of projective systems whose kernel and cokernel are killed by $p^{2N}$:
\[
\{H^{i-1}(X_{\bk},\LL)\otimes_{\Zp} (A_{\cris}^{0}/p^n)\}_{n\geq 0} \to \{ H^{i-1}(X_{\bk},\LL\otimes_{\widehat{\ZZ}_p}\mathbb A_{\cris}^{0}/p^n)\}_{n\geq 0}
\]
Therefore $R^1\varprojlim_n  H^{i-1}(X_{\bk},\LL\otimes_{\widehat{\ZZ}_p}\mathbb A_{\cris}^{0}/p^n)$ is killed by $p^{2N}$. This concludes the proof of the claim.

Consequently, by the following commutative digram 
\[
\xymatrix{
H^i(X_{\bk},\LL)\otimes_{\Zp} A_{\cris}^a \ar[r]^{\textrm{can}} \ar[rd]_{\varprojlim \eqref{eq.iso_n}}& H^i(X_{\bk},\LL\otimes_{\widehat{\ZZ}_p}\mathbb A_{\cris}^{a})\ar[d]^{\eqref{hlim}} \\ & \varprojlim H^i(X_{\bk},\LL\otimes_{\widehat{\ZZ}_p} \mathbb A_{\cris}/p^n)}
\] 
we deduce that the kernel and the cokernel of the horizontal canonical morphism are killed by $p^{4N}$. On inverting $p$, we obtain the desired isomorphism \eqref{eq.PrimitiveIso}.  

We still need to check that \eqref{eq.PrimitiveIso} is compatible with the extra structures. Clearly only the strict compatibility with filtrations needs verification, and it suffices to check this on gradeds. So we reduce to showing that the natural morphism is an isomorphism:
\[
H^i(X_{\bk},\LL)\otimes_{\Zp} \mathbb C_p(j) \lra H^i(X_{\bk},\LL\otimes \widehat{\cO}_X(j)).
\]
Twisting, one reduces to $j=0$, which is given by the following lemma. 
\end{proof}







\begin{lemma}\label{primitiveofstructuresheaf} Let $\LL$ be a lisse $\widehat{\ZZ}_p$-sheaf on $X_{\bk,\proet}$. Then the following natural morphism is an isomorphism:
\[
H^i(X_{\bk,\proet}, \LL)\otimes_{\Zp}\mathbb C_p \stackrel{\approx}{ \lra} H^i\left(X_{\bk,\proet}, \LL\otimes_{\widehat{\Z}_p} \widehat{\cO}_X\right), 
\]
where $\widehat{\cO}_{X}$ is the completed structural sheaf of $X_{\bk,\proet}$ and $\mathbb C_p=\widehat{\bk}$. 
\end{lemma}

\begin{proof} The proof is similar to that of the first part of Theorem \ref{inout}. Let $\LL_n:=\LL/p^n\LL$. Using the (finite) filtration
$\{p^m\cdot \LL_n\}_{m}$ of $\LL_n$ and by induction on $m$, we get from \cite[Theorem 5.1]{Sch13} the following natural almost isomorphisms
\[
H^i(X_{\bk},\LL_n)\otimes_{\Zp}\cO_{\mathbb C_p}\stackrel{\approx}{\lra} H^i\left(X_{\bk},\LL_n\otimes_{\widehat{\Z}_p}\widehat{\cO}_{X}^{+}\right). 
\]
%or equivalently the following natural almost isomorphisms
%\[
%H^i(X_{\bk},\LL)\otimes_{\Zp}\cO_{\mathbb C_p}/p^n\stackrel{\approx}{\lra} H^i\left(X_{\bk},\LL\otimes_{\widehat{\Z}_p}\widehat{\cO}_{X}^{+}/p^n\right).
%\]
We need to look at the project limits (with respect to $n$) of both sides of the previous morphisms. Pick $N\in \mathbb N$ such that $p^N$ kills $\LL_{\rm tor}$ and $H^i(X_{\bk},\LL)_{\rm tor}$ for all $i\in \mathbb N$. For $n\geq N$, there is a tautological exact sequence 
\[
0\lra \LL_{\rm tor}\lra \LL\stackrel{p^n}{\lra} \LL \lra \LL_n\lra 0. 
\]
Splitting it into two short exact sequences and taking cohomology, we obtain exact sequences
\[
\ldots \lra H^i(X_{\bk},p^n\LL)\lra H^i(X_{\bk},\LL)\lra H^i(X_{\bk},\LL_n)\lra \ldots
\]
and 
\[
\ldots \lra H^i(X_{\bk},\LL_{\rm tor})\lra H^i(X_{\bk},\LL)\lra H^i(X_{\bk},p^n\LL)\lra \ldots,  
\]
from which we deduce the exact sequence below:
\[
H^i(X_{\bk},\LL)/p^n\lra H^i(X_{\bk},\LL_n)\lra \ker(H^{i+1}(X_{\bk},p^n\LL)\lra H^{i+1}(X_{\bk},\LL))\lra 0. 
\]
As in the proof of Theorem \ref{inout}, the kernel of the first morphism in the sequence above is killed by $p^N$ while the last abelian group is killed by $p^{2N}$. 
Consequently, the kernel and the cokernel of the following natural morphism are killed by $p^{2N}$:
\[
H^i(X_{\bk},\LL)\otimes \cO_{\mathbb C_p}/p^n\lra H^i(X_{\bk},\LL_n)\otimes \cO_{\mathbb C_p}. 
\] 
Passing to projective limits, we find a natural morphism with kernel and cokernel killed by $p^{2N}$:
\[
H^i(X_{\bk},\LL)\otimes \cO_{\mathbb C_p}\simeq \varprojlim \left(H^i(X_{\bk},\LL)\otimes \cO_{\mathbb C_p}/p^n\right)\lra \varprojlim \left(H^i(X_{\bk},\LL_n)\otimes \cO_{\mathbb C_p}\right), 
\]
and $R^1\varprojlim \left(H^i(X_{\bk},\LL_n)\otimes \cO_{\mathbb C_p}\right)\simeq R^1\varprojlim H^i(X_{\bk},\LL_n\otimes \widehat{\cO}_X^{+})$ are both killed by $p^{2N}$.    
On the other hand, as $\LL$ is a lisse $\widehat{\Z}_p$-sheaf, $R^j\varprojlim (\LL_n\otimes \widehat{\cO}_{X}^{+})=0$ for $j>0$ since the same holds for $\{\widehat{\cO}_{X}^+/p^n\}_n$; for this, apply \cite[Lemma 3.18]{Sch13} to Lemma 4.10 \emph{loc.cit.}.  %\footnote{State it somewhere in section 2} 
Hence by Lemma \ref{jannsen}, we deduce a short exact sequence 
\[
0\lra R^1\varprojlim H^{i-1}(X_{\bk},\LL_n\otimes \widehat{\cO}_X^+) \lra H^i(X_{\bk},\LL\otimes \widehat{\cO}_X^{+})\lra \varprojlim H^i(X_{\bk},\LL_n\otimes \widehat{\cO}_X^{+})\lra 0. 
\]
So we get the following commutative diagram 
\[
\xymatrix{H^i(X_{\bk},\LL)\otimes \cO_{\mathbb C_p}\ar[r]\ar[rd]_{\textrm{iso. up to }p^{2N}-\textrm{torsion}\ \ \ \ } &  H^i(X_{\bk},\LL\otimes\widehat{\cO}_{X}^+)\ar[d]^{\textrm{iso. up to }p^{2N}-\textrm{torsion}}.\\ & \varprojlim H^i(X_{\bk}, \LL_n\otimes \widehat{\cO}_{X}^{+})}
\]
In particular, the horizontal morphism is an isomorphism up to $p^{4N}$-torsions. On inverting $p$, we get our lemma. \end{proof}




Recall that the notion of lisse $\Zp$-sheaf on $X_{\et}$ and lisse $\widehat{\ZZ}_p$-sheaf on $X_{\proet}$ are equivalent. We finally deduce the following crystalline comparison theorem:
\begin{thm}\label{thm.comp}Let $\cX$ be a proper smooth formal scheme over $\cO_k$, with $X$ (resp. $\cX_0$) its generic (resp. closed) fiber. Let $\LL$ be a lisse $\widehat{\ZZ}_p$-sheaf on $X_{\proet}$. Assume that $\LL$ is associated to a filtered $F$-isocrystal $\mathcal E$ on $\cX_0/\cO_k$. Then there exists a canonical isomorphism of $B_{\cris}$-modules
\[
H^i(X_{\bk,\et},\LL)\otimes_{\Zp}B_{\cris}\simto H^i_{\cris}(\cX_0/\cO_k, \mathcal E)\otimes_{\cO_k}B_{\cris}
\]
compatible with Galois action, filtration and Frobenius.  
\end{thm}
\begin{proof}
 This is just the composition  of the isomorphisms in Themorem \ref{main1} and Theorem \ref{inout}.
\end{proof}


\section{Comparison isomorphism in the relative setting}\label{sec.relative}
 
 
Let $f\colon \cX\to \cY$ be a  smooth morphism between two smooth formal schemes over $\Spf(\cO_k)$ of relative dimension $d\geq 0$. The induced morphism between the generic fibers will be denoted by $f_k\colon X\to Y$. We shall denote by $w_{\cX}$ (resp. $w_{\cY}$) the natural morphism of topoi $X_{\proet}^{\sim}\to \cX_{\et}^{\sim}$ (resp. $Y_{\proet}^{\sim}\to \cY_{\et}^{\sim}$). By abuse of notation, the morphism of topoi $X_{\proet}^{\sim}\to Y_{\proet}^{\sim}$ will be still denoted by $f_k$. 
 
Let $\nabla_{X/Y}: \cO\BB_{\cris,X}^+\ra \cO\BB_{\cris,X}^+\otimes_{\cO_X^{\ur}}\Omega_{X/Y}^{1,\ur}$ be the natural relative derivation, where $\Omega_{X/Y}^{1,\ur}:=w_{\cX}^*\Omega_{\cX/\cY}^1$.
 
 \begin{prop}\label{relativepoincare} (1) (Relative Poincar\'e lemma) The following sequence of pro-\'etale sheaves is exact and strict  with respect to the filtration giving $\Omega^{i,\ur}_{X/Y}$ degree $i$: 
\[
\begin{array}{c}
0\lra \BB_{\cris,X}^+\widehat{\otimes}_{f_k^{-1}\BB_{\cris,Y}^+}f_k^{-1}\cO\BB_{\cris,Y}^+\ra\cO\BB_{\cris,X}^+\stackrel{\nabla_{X/Y}}{\lra}\cO\BB_{\cris,X}^+\otimes_{\cO_X^{\ur}}\Omega_{X/Y}^{1,\ur} \\
\stackrel{\nabla_{X/Y}}{\lra}\cdots\stackrel{\nabla_{X/Y}}{\lra}\cO\BB_{\cris,X}^+\otimes_{\cO_X^{\ur}}\Omega^{d,\ur}_{X/Y}\ra 0.
\end{array}
\]
Furthermore, the connection $\nabla_{X/Y}$ is integrable and  satisfies Griffiths transversality with respect to the filtration, i.e. $\nabla_{X/Y} (\Fil^i\cO\BB_{\cris,X}^+)\subset \Fil^{i-1}\cO\BB_{\cris,X}^+\otimes_{\cO_X^{\ur}}\Omega_{X/Y}^{1,\ur}$.

%(ii) The connection $\nabla_{X/Y}$ is convergent.

(2) Suppose the Frobenius on $\cX_0$ (resp.  $\cY_0$) lifts to a Frobenius $\sigma_X$ (resp. $\sigma_Y$) on the formal scheme $\cX$ (resp. $\cY$) and they commute with $f$. Then the induced Frobenius  $\varphi_{X}$ on $\cO\BB_{\cris,X}^+$ is horizontal with respect to $\nabla_{X/Y}$. 

\end{prop}
\begin{proof} The proof is routine (cf. Proposition \ref{iso}), so we omit the detail here.
\end{proof}
 
 
For the relative version of the crystalline comparison, we shall need the following primitive comparison in the relative setting.

\begin{prop}\label{part1relative} Let $f\colon \cX\to \cY$ be a proper smooth morphism between two smooth formal schemes over $\cO_k$. 
 Let $\LL$ be a lisse $\widehat{\ZZ}_p$-sheaf on $X_{\proet}$. Suppose that  $R^if_{k*}\LL$ is a lisse $\widehat{\ZZ}_p$-sheaf on $Y_{\proet}$ for all $i\geq 0$. Then the following canonical morphism is an isomorphism: 
\begin{equation}\label{eq.relativeprimitive}
(R^if_{k*}\LL)\otimes_{\widehat{\ZZ}_p}\BB_{\cris,Y}^+\simto R^if_{k*}(\LL\otimes_{\widehat{\ZZ}_p}\BB_{\cris,X}^+) 
\end{equation} 
which is compatible with filtration and Frobenius. 
 \end{prop}
 
\begin{proof}
%Note that we may replace $f_{k, \et}$ by $f_{k,\proet}$ by \cite[Corollary 3.17(ii)]{Sch13}, hence denote both of them simply by $f_{k}$. Also we may assume that our lisse $\widehat{\Z}_p$-sheaf is torsion-free.
Remark first that $R^if_{k*}\LL=0$ for $i>2d$ where $d$ denotes the relative dimension of $f$. To show this, write $\LL'$ and $\mathbb L''$  the $\mathbb F_p$-local system on $X_{\proet}$ defined by the following exact sequence 
\begin{equation}\label{eq.les4terms}
0\lra \LL' \lra \LL\stackrel{p}{\lra} \LL\lra \LL'' \lra 0. 
\end{equation}
We claim that $R^if_{k*}\mathbb L'=R^if_{k*}\mathbb L''=0$ for $i>2d$. Indeed, as $\LL'$ is an $\mathbb F_p$-local system of finite presentation, it comes from an $\mathbb{F}_p$-local system of finite presentation on $X_{\et}$, still denoted by $\LL'$. By \cite[Corollary 3.17(ii)]{Sch13}, we are reduced to showing that $R^if_{k~ \!\et *}\LL'=0$ for $i>2d$, which follows from \cite[Theorem 5.1]{Sch13} and \cite[2.6.1]{Hub} by taking fibers of $R^if_{k~ \!\et*}\LL'$ at geometric points of $Y$. Similarly $R^if_{k*}\LL''=0$ for $i>2d$. Then, splitting the exact sequence \eqref{eq.les4terms} into two short exact sequences as in the proof of Theorem \ref{inout} and applying the higher direct image functor $Rf_{k*}$, we deduce that the multiplication-by-$p$ morphism on $R^if_{k*}\LL$ is surjective for $i>2d$. But $R^if_{k*}\LL$ is a lisse $\widehat{\Z}_p$-sheaf on $Y_{\proet}$ by our assumption, necessarily $R^if_{k*}\LL=0$ for $i>2d$. Consequently, we can choose a sufficiently large integer $N\in \mathbb N$ such that $p^N$ kills the torsion part of $\LL$ and also the torsion part of $R^if_{k*}\LL$ for all $i\in \mathbb Z$.  
 
Then, it is shown in the proof of \cite[Theorem 8.8 (i)]{Sch13}, as a consequence of  the primitive comparison isomorphism in the relative setting (\cite[Corollary 5.11]{Sch13}), that the following canonical morphism is an almost isomorphism:
\begin{equation}\label{eq.RelAinf}
R^if_{k*}\LL\otimes_{\widehat{\ZZ}_p}\mathbb{A}_{\mathrm{inf},Y}\stackrel{\approx}{\longrightarrow} R^if_{k*}(\LL\otimes_{\widehat{\ZZ}_p}\mathbb{A}_{\mathrm{inf},X}).
\end{equation}
With this in hand, the proof of Theorem \ref{inout} applies and gives the result. Indeed, consider the PD-envelope $\mathbb{A}_{\cris, X}^ 0$ (resp. $\mathbb{A}_{\cris,Y}^ 0$) of $\mathbb{A}_{\inf,X}$ (resp. of $\mathbb A_{\inf,Y}$) with respect to the ideal $\ker(\theta_X\colon \mathbb A_{\inf,X}\to \widehat{\cO_X^ +})$ (resp. to the ideal $\ker(\theta_Y\colon \mathbb A_{\inf,Y}\to \widehat{\cO_Y^ +})$). Then $\mathbb A_{\cris,X}$ and $\mathbb A_{\cris, Y}$ are respectively the $p$-adic completions of $\mathbb A_{\cris,X}^ 0$ and $\mathbb A_{\cris,Y}^ 0$. As in the proof of Theorem \ref{inout}, 
we obtain from \eqref{eq.RelAinf} the following canonical almost isomorphism 
\begin{equation}\label{eq.RelAcris0}
R^if_{k*}\LL\otimes_{\widehat{\ZZ}_p}\mathbb{A}_{\mathrm{cris},Y}^0\stackrel{\approx}{\longrightarrow} R^if_{k*}(\LL\otimes_{\widehat{\ZZ}_p}\mathbb{A}_{\mathrm{cris},X}^0),
\end{equation}
from which we deduce that for each $n$ the kernel and the cokernel of the  natural morphism below are killed by $p^{2N}$: 
\begin{equation}\label{eq.RelAcris0pN}
R^if_{k*}\LL\otimes_{\widehat{\ZZ}_p}\mathbb{A}_{\mathrm{cris},Y}^0/p^n\lra  R^if_{k*}(\LL\otimes_{\widehat{\ZZ}_p}\mathbb{A}_{\mathrm{cris},X}^0/p^n).
\end{equation}
Recall from the proof of Theorem \ref{inout} that  $R^j\varprojlim (\LL\otimes \mathbb A_{\cris,X}^{a}/p^n)=0$ whenever $j>0$, hence a spectral sequence:
\[
E_{2}^{j,i}=R^j\varprojlim R^if_{k*}(\LL\otimes \mathbb A_{\cris, X}^{a}/p^n) \Longrightarrow R^{i+j}f_{k*}(\LL\otimes \mathbb A_{\cris, X}^a).
\]
As $R^if_{k*}\LL$ is a lisse $\widehat{\Z}_p$-sheaf, $R^j\varprojlim_n\left(R^if_{k*}(\LL)\otimes \mathbb{A}_{\cris,Y}^0/p^n\right)=0$ for $j>0$. It follows that $R^j\varprojlim_n R^if_{k*}(\LL\otimes \mathbb{A}_{\cris,X}^0/p^n)$ is killed by $p^{2N}$ for any $j>0$. In particular, $p^{2N}\cdot E_{\infty}^{j,i}=0$ whenever $j>0$. With $i$ fixed, let $\mathbb H:=R^if_{k*}(\LL\otimes \mathbb{A}_{\cris,X}^0)$. From the theory of spectral sequences, we know that $\mathbb H$ is endowed with a finite filtration
\[
0=F^{i+1}\mathbb H \subset F^i\mathbb H\subset \ldots F^1\mathbb H\subset F^0\mathbb H=\mathbb H,
\]
such that $F^q\mathbb H/F^{q+1}\mathbb H\simeq E_{\infty}^{q,i-q}$. Therefore, $F^1\mathbb H$ is killed by $p^{2Ni}$. On the other hand, $E_{2}^{0,i}$ has a filtration of length $i$
\[
E_{\infty}^{0,i}=E_{2+i}^{0,i}\subset E_{2+(i-1)}^{0,i} \subset \ldots \subset E_{2}^{0,i}=\varprojlim R^if_{k*}(\LL\otimes \mathbb A_{\cris,X}/p^n). 
\]
Since in general $E_{r+1}^{0,i}=\ker(d_r\colon E_r^{0,i}\to E_{r}^{r,i-r+1} )$, it follows that all the successive quotients of the filtration above are killed by $p^{2N}$. So the inclusion $E_{\infty}^{0,i}\subset E_2^{0,i}$ has cokernel killed by $p^{2Ni}$. To summarize, we have a commutative diagram 
\[
\xymatrix{R^if_{k*}(\LL\otimes \mathbb{A}_{\cris,X})\ar[r]\ar@{->>}[d]_{\textrm{kernel killed by }p^{2Ni}} & E_2^{0,i}=\varprojlim R^if_{k*}(\LL\otimes \mathbb A_{\cris,X}/p^n) \\ E_{\infty}^{0,i}\ar@{^(->}[ru]_{\ \ \textrm{cokernel killed by }p^{2Ni}} & }
\]
Hence the natural morphism   
\begin{equation}
R^{i}f_{k*}(\LL\otimes \mathbb A_{\cris, X}^a)\twoheadrightarrow \varprojlim R^if_{k*}(\LL\otimes \mathbb A_{\cris, X}^{a}/p^n)
\end{equation}
has kernel and cokernel killed by $p^{2Ni}$. Therefore we deduce like in the proof of Theorem \ref{inout} that the kernel and the cokernel of the following canonical morphism are killed by $p^{2N+2Ni}$:
\begin{equation}\label{eq.RelAcris}
R^if_{k*}\LL\otimes_{\widehat{\ZZ}_p}\mathbb{A}_{\mathrm{cris},Y}\longrightarrow R^if_{k*}(\LL\otimes_{\widehat{\ZZ}_p}\mathbb{A}_{\mathrm{cris},X}).
\end{equation}
Inverting $p$ in the above morphism, we obtain the desired isomorphism of our lemma. 

It remains to verify the compatibility of the isomorphism \eqref{eq.relativeprimitive} with the extra structures. It clearly respects  Frobenius structures. To check the (strict) compatibility with respect to filtrations, by taking grading quotients, we just need to show that for each $r\in \mathbb N$, the following natural morphism 
\[
Rf_{k*}\LL\otimes \widehat{\cO}_{Y}(r)\lra Rf_{k*}(\LL\otimes \widehat{\cO}_X(r))
\]
is an isomorphism: it is a local question, hence it suffices to show this after restricting the latter morphism to $Y_{\bk}$. As $\widehat{\cO}_X(r)|_{X_{\bk}}\simeq \widehat{\cO}_X|_{X_{\bk}}$ and $\widehat{\cO}_Y(r)|_{Y_{\bk}}\simeq \widehat{\cO}_Y|_{Y_{\bk}}$, we then reduce to the case where $r=0$. The proof of the latter statement is similar to that of Lemma \ref{primitiveofstructuresheaf}, so we omit the details here. 
\end{proof}


For a sheaf of $\cO_{\cX}$-modules $\cF$ with an $\cO_{\cY}$-linear connection $\nabla\colon \cF\to \cF\otimes\Omega^1_{\cX/\cY}$, we denote the de Rham complex of $\cF$ as:
\[
DR_{X/Y}(\cF):=(\ldots \lra 0\lra \cF
\stackrel{\nabla}{\lra}\cF\otimes_{\cO_{\cY}} \Omega^1_{\cX/\cY}\stackrel{\nabla}{\lra}\ldots). 
\]
The same rule applies if we consider an $\cO_{X}^{\rm un}$-module endowed with an $\cO_Y^{\rm un}$-linear connection etc.  


In the lemma below, assume $\cY=\Spf(A)$ is affine and is \'etale over a torus $\mathcal S=\Spf(\cO_k\{S_1^{\pm 1},\ldots, S_{\delta}^{\pm 1}\})$. For each $1\leq j\leq \delta$, let $(S_j^{1/p^n})_{n\in \mathbb N}$ be a compatible family of $p$-power roots of $S_j$. As in Proposition \ref{iso}, set 
\[
\widetilde{Y}:=\left(Y\times_{\mathcal S_k}\Spa\left(k\{S_1^{\pm 1/p^{n}},\ldots, S_{\delta}^{\pm 1/p^{n}}\},\cO_k\{S_1^{\pm 1/p^{n}},\ldots, S_{\delta}^{\pm 1/p^{n}}\}\right)\right)_{n\in \mathbb N}\in Y_{\proet}.  
\]



\begin{lemma}\label{quasirelative} Let $V\in Y_{\proet}$ be an affinoid perfectoid which is pro-\'etale over $\widetilde{Y}_{\bk}$, with $\widehat{V}=\Spa(R,R^+)$. Let $w_V$ be the composite of natural morphisms of topoi 
\[
w_V\colon X_{\proet}^{\sim}\slash X_{V} \lra X_{\proet}^{\sim}\stackrel{w}{\lra} \mathcal X_{\et}^{\sim}.
\]
Then  

(1) for any $j>0$ and $r\in \mathbb Z$, $R^jw_{V\ast}\cO\BBcr=R^jw_{V\ast}(\Fil^r\cO\BBcr)=0$; and  

(2) the natural morphisms 
\[
\cO_{\cX}\widehat{\otimes}_A\cO\BB_{\cris,Y}(V) \lra w_{V*}(\cO\BB_{\cris,X})
\]
and 
\[
\cO_{\cX}\widehat{\otimes}_{A}\Fil^r\cO\BB_{\cris,Y}(V)\to w_{V\ast}(\Fil^r\cO\BB_{\cris,X}) \quad \textrm{for all }r\in \mathbb Z
\]
are isomorphisms. Here  $\cO_{\cX}\widehat{\otimes}_A\cO\BB_{\cris, Y}(V):=\left(\cO_{\cX}\widehat{\otimes}_A\cO\mathbb{A}_{\cris,Y}(V)\right)[1/t]$ with 
\[
\cO_{\cX}\widehat{\otimes}_A\cO\mathbb{A}_{\cris,Y}(V):=\varprojlim \left(\cO_{\cX}\otimes_A\cO\mathbb{A}_{\cris,Y}(V)/p^n\right),
\]
and 
\[
\cO_{\cX}\widehat{\otimes}_{A}\Fil^r\cO\BB_{\cris,Y}(V):=\varinjlim_{n\in \mathbb N} t^{-n} \left(\cO_{\cX}\widehat{\otimes}_{A} \Fil^{r+n} \cO\mathbb{A}_{\cris,Y}(V)\right).
\]

In particular, if we filter $\cO_{\cX}\widehat{\otimes}_{A}\cO\BB_{\cris,Y}(V)$ using $\{\cO_{\cX}\widehat{\otimes}_A\Fil^r\cO\BB_{\cris,Y}(V)\}_{r\in \mathbb Z}$, the natural morphism 
\[
\cO_{\cX}\widehat{\otimes}_A\cO\BB_{\cris,Y}(V)\lra Rw_{V*}(\cO\BB_{\cris,X})
\]
is an isomorphism in the filtered derived category. 
\end{lemma}

\begin{proof} (1) Recall that for $j\geq 0$, $R^jw_{V*}\cO\BB_{\cris,X}$ is the associated sheaf on $\cX_{\et}$ of the presheaf sending $\cU\in \cX_{\et}$ to $
H^i(\cU_V,\cO\BB_{\cris,X})$, where $\cU_V:=\cU_k\times_X X_V$. Now we take $\mathcal U=\Spf(B) \in \cX_{\et}$ to be affine such that the composition of $\cU\to\cX$ together with $f\colon \cX\to \cY$ can be factored as
\[
\cU\lra \mathcal T\lra \cY,
\]
where the first morphism is \'etale and that $\mathcal T=\Spf(A\{T_1^{\pm 1},\ldots, T_d^{\pm 1}\})$ is a $d$-dimensional torus over $\cY$. Write $\mathcal{T}_V=\mathcal{T}_k\times_Y V$. Then $\mathcal T_V=\Spa(S,S^+)$ with $S^+=R\{T_1^{\pm 1},\ldots, T_d^{\pm 1}\}$ and $S=S^+[1/p]$. Write also $\mathcal U_V=\Spa(\widetilde{S},\widetilde S^+)$. For each $1\leq i\leq d$, let $(T_i^{1/p^{n}})_{n\in\mathbb N}$ be a compatible family of $p$-power root of $T_i$, and set 
\[
S_{\infty}^+=R^+\{T_1^{\pm 1/p^{\infty}},\ldots, T_d^{\pm 1/p^{\infty}}\}, \quad \widetilde{S}_{\infty}^+:=B\widehat{\otimes}_{A\{T_1^{\pm 1},\ldots, T_d^{\pm 1}\}}S_{\infty}^+,
\]
$S_{\infty}:=S_{\infty}^+[1/p]$ and $\widetilde{S}_{\infty}=\widetilde{S}_{\infty}^+[1/p]$. Then $(S_{\infty},S_{\infty}^+)$ and $(\widetilde{S}_{\infty},\widetilde{S}_{\infty}^+)$ are two affinoid perfectoid algebras over $(\widehat{\bk},\cO_{\widehat{\bk}})$. Let $\widetilde{\mathcal U_V}\in X_{\proet}$ (resp. $\widetilde{\mathcal T_V}\in \mathcal{T}_{k\ \!\proet}$) be the affinoid perfectoid corresponding to $(\widetilde{S}_{\infty},\widetilde{S}_{\infty}^+)$ (resp. to $(S_{\infty},S_{\infty}^+)$). So we have the following commutative diagram of ringed spaces
\[
\xymatrix{\widehat{\widetilde{\mathcal U_V}}=\Spf(\widetilde{S}_{\infty},\widetilde{S}_{\infty}^+)\ar[r]^{\Gamma} \ar[d]& \widehat{\mathcal U_V}=\Spa(\widetilde{S},\widetilde{S}^+)\ar[d]\ar[r] & \cU=\Spf(B)\ar[d]\\ \widehat{\widetilde{\mathcal{T}_V} }=\Spa(S_{\infty},S_{\infty}^+)\ar[r] & \widehat{\mathcal T_V}=\Spa(S,S^+)\ar[r]\ar[d] & \mathcal T\ar[d]\\ & \widehat{V}=\Spa(R,R^+)\ar[r] & \mathcal Y=\Spf(A)}.
\]
The morphism $\widetilde{\cU_V}\to \cU_V$ is a profinite Galois cover, with Galois group $\Gamma$ isomorphic to $\Zp(1)^d$. 


For $q\in \mathbb N$, let $\widetilde{\cU_V}^{q}$ be the $(q+1)$-fold fiber product of $\widetilde{\cU_{V}}$ over $\cU_{V}$. So $\widetilde{\cU_{V}}^{q}\simeq \widetilde{\cU_{V}}\times \Gamma^{q}$ is affinoid perfectoid. In particular, for $j>0$, $H^j(\widetilde{\cU_V}^{q},\cO\mathbb{A}_{\cris,X})^a= 0$ (Lemma \ref{vanish}) and 
\begin{equation}\label{eq.ContCoh}
H^0\left(\widetilde{\cU_{V}}^{q},\cO\mathbb{A}_{\cris,X}\right)=\mathrm{Hom}_{\mathrm{cont}}\left(\Gamma^{q}, \cO\mathbb{A}_{\cris,X}(\widetilde{\cU_V})\right).
\end{equation}
Consequently, the Cartan-Leray spectral sequence associated with the cover $\widetilde{\cU_{V}}\to \cU_{V}$ almost degenerates at $E_2$: 
\[
E_{1}^{q,j}=H^j\left(\widetilde{\cU_{V}}^{q},\cO\mathbb{A}_{\cris,X}\right) \Longrightarrow H^{q+j}(\cU_{V},\cO\mathbb{A}_{\cris,X}),
\]
giving an almost isomorphism $E_{2}^{q,0}\approx H^q(\cU_{V},\cO\mathbb{A}_{\cris,X})$ for each $q\in \mathbb Z$. 
Using \eqref{eq.ContCoh}, we see that $E_{2}^{q,0}$ is the continuous group cohomology $H^q(\Gamma, \cO\mathbb{A}_{\cris,X}(\widetilde{\cU_{V}}))$. On the other hand, $\cO\mathbb A_{\cris,X}(\widetilde{\cU_{V}})\approx \cO\mathbb A_{\cris}(\widetilde{S}_{\infty},\widetilde{S}_{\infty}^+)$ by Lemma \ref{2obcris}, so 
\begin{eqnarray*}
H^q(\cU_{V},\cO\BB_{\cris,X})& \simeq & H^q(\cU_{V},\cO\mathbb{A}_{\cris,X})[1/t] \\ & \simeq &  
H^q(\Gamma, \cO\mathbb A_{\cris,X}(\widetilde{\cU_V}))[1/t] \\ &  \simeq & H^q(\Gamma, \cO\mathbb A_{\cris}(\widetilde{S}_{\infty},\widetilde{S}_{\infty}^+))[1/t]. 
\end{eqnarray*}
By definition, the last cohomology group is precisely $H^q(\Gamma, \cO\BBcr(\widetilde{S}_{\infty},\widetilde{S}_{\infty}^+))$ computed in \S \ref{acy}. Namely, from Theorem \ref{withoutfil} we deduce $H^q(\cU_V,\cO\BB_{\cris,X})=0$ whenever $q>0$, and that the natural morphism 
\[
\cO_{\cX}(\cU)\widehat{\otimes}_A\cO\BBcr(V)=B\widehat{\otimes}_A\cO\BBcr(R,R^+)\lra H^0(\Gamma,\cO\BBcr(\widetilde{S}_{\infty},\widetilde{S}_{\infty}^+))=H^0(\cU_V,\cO\BB_{\cris,X})
\] 
is an isomorphism. Varying $\cU$ in $\cX_{\et}$ and passing to associated sheaf, we obtain
\[
\cO_{\cX}\widetilde{\otimes}_A\cO\mathbb{B}_{\cris,Y}(V)\stackrel{\sim}{\lra} w_{V*} \cO\BB_{\cris,X},
\]  
where by definition, $\cO_{\cX}\widetilde{\otimes}_A\cO\mathbb{B}_{\cris,Y}(V)$ is the associated sheaf of the presheaf over $\cX_{\et}$ sending $\cU\in \cX_{\et}$ to $\cO_{\cX}(\cU)\widehat{\otimes}_A\cO\BBcr(V)$.

To concludes the proof of (1) it remains to check that the canonical morphism below is an isomorhpism:
\begin{equation}\label{eq.desirediso}
\cO_{\cX}\widetilde{\otimes}_A\cO\BB_{\cris, Y}(V)\lra \cO_{\cX}\widehat{\otimes}_A\cO\BB_{\cris,Y}(V).
\end{equation}
Indeed, using $w_{V*}\cO\mathbb{A}_{\cris,X}=\varprojlim w_{V*}(\cO\mathbb{A}_{\cris,X}/p^n)$ and Proposition \ref{cor.cohomologyofobcris}, we deduce a natural morphism which is injective with cokernel killed by $(1-[\epsilon])^{2d}$:
\[
\cO_{\cX}\widehat{\otimes}_A\cO\mathbb{A}_{\cris,Y}(V)\lra w_{V*}\cO\mathbb{A}_{\cris,X}.
\]
Therefore inverting $t$, we get an isomorphism 
\[
\cO_{\cX}\widehat{\otimes}_A\cO\BB_{\cris,Y}(V)\stackrel{\sim}{\lra} w_{V*}\cO\mathbb{B}_{\cris,X}, 
\]
from which  the desired isomorphism \eqref{eq.desirediso} follows, since the objects on both sides of \emph{loc. cit.} are isomorphic in a natural way to $w_{V*}(\cO\BB_{\cris,X})$. 


(2) For $r\in \mathbb Z$, define $\cO_{\cX}\widetilde{\otimes}_A\Fil^r\cO\mathbb{A}_{\cris,Y}(V)$ and $\cO_{\cX}\widetilde{\otimes}_A\mathrm{gr}^r\cO\mathbb{A}_{\cris,Y}(V)$ in the same way as is done for $\cO_{\cX}\widetilde{\otimes}_A\BB_{\cris,Y}(V)$. Like the proof above, we have the natural isomorphism below
\[
\cO_{\cX}\widetilde{\otimes}_A\Fil^r \cO\BB_{\cris,Y}(V)\stackrel{\sim}{\lra} w_{V*}\Fil^r\cO\BB_{\cris,X}, 
\]
and we need to check that the following canonical morphism is an isomorphism:
\[
\cO_{\cX}\widetilde{\otimes}_A\Fil^r\cO\BB_{\cris,Y}(V)\lra \cO_{\cX}\widehat{\otimes}_A\Fil^r\cO\BB_{\cris,Y}(V).
\]
To see this, remark first that the morphism above is injective since both sides of the morphism above are naturally subsheaves of $\cO_{\cX}\widetilde{\otimes}_A\cO\BB_{\cris,Y}(V)\simeq \cO_{\cX}\widehat{\otimes}_A\cO\BB_{\cris,Y}$. For the surjectivity, by taking the graded quotients, we are reduced to showing that the natural morphism below is an isomorphism:
\begin{equation}\label{eq.requiredisoforOX}
\cO_{\cX}\widetilde{\otimes}_A\mathrm{gr}^r\cO\BB_{\cris,Y}(V)\lra  \cO_{\cX}\widehat{\otimes}_A\mathrm{gr}^r\cO\BB_{\cris,Y}(V).
\end{equation}
But $\mathrm{gr}^r\cO\mathbb{A}_{\cris,Y}(V)\approx \mathrm{gr}^r\cO\AAcr(R,R^+)$, while the latter is a free module over $R^+$. Furthermore $R^+$ is almost flat over $A$ by the almost purity theorem (recall that $V$ is pro-\'etale over $\widetilde{Y}_{\bk}$ which is an affinoid perfectoid over $(\widehat{\bk}, \cO_{\widehat{\bk}})$). So, for $\cU\in \cX_{\et}$ an affine formal scheme, 
\[
\cO_{\cX}(\cU)\widehat{\otimes}_A R^+=\varprojlim \left((\cO_{\cX}(\cU)/p^n)\otimes_A R^+\right)\approx \left(\varprojlim \left(\left(\cO_{\cX}/p^n\right)\otimes_A R^+\right)\right)(\cU).
\]
Inverting $p$ and passing to associated sheaves, we get the required isomorphism \eqref{eq.requiredisoforOX}, which concludes the proof of our lemma.  
\end{proof}



%\begin{cor}\label{quasicorrelative} We keep the assumptions of Lemma \ref{quasirelative}. Consider the lift of Frobenius $\sigma$ on $\widetilde{Y}$ sending $S_{j}^{1/p^n}$ to $S_j^{1/p^{n-1}}$ for all $1\leq j\leq \delta$ and for all $n\in \mathbb N$. (In particular, $V\in Y_{\proet}$ admits a lift of Frobenius). Let $\LL$ be a crystalline lisse $\widehat{\Z}_p$-sheaf on $X_{\proet}$, associated with a filtered convergent $F$-isocrystal $\cE$. Let $V\in Y_{\proet}$ be an affinoid perfectoid above $\widetilde{Y}_{\bk}$. Then there exists a natural quasi-isomorphism in the filtered derived category 
%\[
%Rw_{V\ast}(\LL\otimes_{\widehat{\Z}_p} \BB_{\cris,X}\widehat{\otimes} f_k^{-1}\cO\BB_{\cris,Y})\stackrel{\sim}{\lra} DR(\cE)\otimes_{\cO_{\cX}} \left(\cO_{\cX}\widehat{\otimes}_A \cO\BB_{\cris,Y}(V)\right). 
%\]
%If moreover $\cX$ is endowed with a lifting of Frobenius $\sigma$, then the isomorphism above is also compatible with the Frobenii deduced from $\sigma$ on both sides. 
%\end{cor}

%\begin{proof} The proof is identical to that of Corollary \ref{quasicor}: one uses just the relative Poincar\'e lemma (Proposition \ref{relativepoincare}) instead of its absolute analogue (Corollary \ref{poincare}). 
%\end{proof}

%\begin{rk} Let $G_k$ denote the absolute Galois group of $k$. Each element of $G_k$ defines a morphism of $U_{\bk}$ in the pro-\'etale site $X_{\proet}$ for any $\cU\in \cX_{\proet}$ with $U:=\cU_k$. Therefore, the object $R\overline{w}_{\ast}(\LL\otimes \BBcr)$ comes with  a natural Galois action of $G_k$. With this Galois action, one checks that the quasi-isomorphism in Corollary \ref{quasicor} is also Galois equivariant. 
%\end{rk}

\begin{lemma}\label{lem.techBC} Let $\cE$ be a filtered convergent $F$-isocrystal on $\cX$, $\cY=\Spf A$, and $V\in Y_{\proet}$ an affinoid perfectoid which is pro-\'etale over $\widetilde{Y}_{\bk}$. The canonical morphism 
\[
R\Gamma(\cX,DR_{X/Y}(\cE))\otimes_{A}\cO\BB_{\cris,Y}(V)\lra R\Gamma(\cX, DR_{X/Y}(\cE)\widehat{\otimes}_A\cO\BB_{\cris,Y}(V))
\]
is an isomorphism in the filtered derived category. 
%With the same assumptions in the previous corollary, then there exists an isomorphism in the filtered derived category
%\[
%R\Gamma\left(X_V, \LL\otimes_{\widehat{\Z}_p}\BB_{\cris,X}\widehat{\otimes}f_k^{-1}\cO\BB_{\cris,Y}\right) \stackrel{\sim}{\lra} R\Gamma(\cX,DR_{X/Y}(\cE))\otimes_A\cO\BB_{\cris,Y}(V).
%\]
\end{lemma}


\begin{proof} Write $\widehat{V}=\Spa(R,R^+)$. To show our lemma, remark first that the morphism above respects clearly the filtration on both sides. Secondly, since $\cO\mathbb{A}_{\cris,Y}(V)\approx \cO\mathbb{A}_{\cris}(R,R^+)$ while the latter is $\mathcal I^2$-flat over $A$ with $\mathcal I\subset \cO\mathbb{A}_{\cris}(R,R^+)$ the ideal generated by $([\epsilon]^{1/p^n}-1)_{n\in \mathbb N}$ (we refer to \cite[Section 6.3]{Bri} for this notion and the proof of this assertion), for any coherent sheaf $\mathcal F$ on $\cX$, the kernel and the cokernel of the morphism below induced by base change are killed by some (finite) power of $\mathcal I$:
\[
H^i(\cX, \mathcal F)\otimes_A \cO\mathbb{A}_{\cris}(R,R^+)\lra H^i(\cX,\mathcal F\widehat{\otimes}_A\cO\mathbb{A}_{\cris}(R,R^+)). 
\]
Inverting $t$, we get an isomorphism
\[
H^i(\cX, \mathcal F)\otimes_A \cO\mathbb{B}_{\cris,Y}(V)\lra H^i(\cX,\mathcal F\widehat{\otimes}_A\cO\mathbb{B}_{\cris,Y}(V)). 
\]
By some standard devissage, we deduce that the morphism in our lemma is an isomorphism in the derived category. To conclude,  we only need to check that the induced morphisms between the graded quotients are all quasi-isomorphisms. As in the proof of Corollary \ref{quasicor} we are reduced to checking the almost isomorphism below for any coherent sheaf $\mathcal F$ on $\cX$: 
\[
R\Gamma(\cX,\mathcal F)\otimes_A \mathrm{gr}^r\cO\mathbb{A}_{\cris,Y}(V) \lra R\Gamma(\cX,\mathcal F\widehat{\otimes}_{A}\mathrm{gr}^r\cO\mathbb{A}_{\cris,Y}(V)),
\]
which is clear since $ \mathrm{gr}^r\cO\mathbb{A}_{\cris,Y}(V)$ is free over $R^+$, which  is almost flat over $A$ by the almost purity theorem.  
%We claim first that this is an isomorphism in the derived category (i.e., we ignore first the filtration on both sides). Indeed, by considering a flat resolution of $\cO\mathbb{A}_{\cris,Y}(V)$ as $A$-module, we see that the natural morphism below is an isomorphism in the derived category
%\[
%R\Gamma(\cX, DR_{X/Y}(\cE))\otimes_A^{L}\cO\mathbb{A}_{\cris,Y}(V)\lra R\Gamma(\cX, DR_{X/Y}(\cE)\widehat{\otimes}_A \cO\mathbb{A}_{\cris,Y}(V))
%\]
\end{proof}

From now on, assume $f\colon \cX\to \cY$ is a proper smooth morphism (between smooth formal schemes) over $\cO_k$. Its closed fiber gives rise to a morphism between the crystalline topoi, 
\[
f_{\cris}\colon (\cX_0/\cO_k)_{\cris}^{\sim}\lra (\cY_0/\cO_k)_{\cris}^{\sim}. 
\]
Let $\cE$ be a filtered convergent $F$-isocrystal on $\cX_0/\cO_k$, and $\mathcal M$ an $F$-crystal on $\cX_0/\cO_k$ such that $\cE\simeq \mathcal M^{\rm an}(n)$ for some $n\in \mathbb N$ (cf. Remark \ref{rk.CrystalVSIsoc}). Then $\cM$ can be viewed naturally as a coherent $\cO_{\cX}$-module endowed with an integrable and quasi-nilpotent $\cO_k$-linear connection $\cM\to \cM\otimes \Omega^1_{\cX/\cO_k}$. 

In the following we consider the higher direct image $R^{i}f_{\cris*}\mathcal M$ of the crystal $\mathcal M$. One can determine the value of this abelian sheaf on $\cY_0/\cO_k$ at the $p$-adic PD-thickening $\cY_0\hookrightarrow \cY$ in terms of the relative de Rham complex $DR_{X/Y}(\mathcal M)$ of $\mathcal M$. To state this, take $\mathcal{V}=\Spf(A)$ an affine open subset of $\cY$, and put $\cX_A:=f^{-1}(\mathcal V)$. We consider $A$ as a PD-ring with the canonical divided power structure on $(p)\subset A$. In particular, we can consider the crystalline site $(\cX_{A,0}/A)_{\cris}$ of $\cX_{A,0}:=\cX\times_{\cY}\mathcal V_0$ relative to $A$. By \cite[Lemme 3.2.2]{Ber}, the latter can be identified naturally to the open subsite of $(\cX_0/\cO_k)$ whose objects are objets $(U,T)$ of $(\cX_0/\cO_k)_{\cris}$ such that $f(U)\subset \mathcal V_0$ and such that there exists a morphism $\alpha\colon T\to \mathcal V_n:=\mathcal V\otimes_A A/p^{n+1}$ for some $n\in \mathbb N$, making the square below commute 
\[
\xymatrix{U\ar@{^(->}[r]\ar[d]_{\rm can} & T\ar[d] \\ \mathcal{V}_0\ar@{^(->}[r] & \mathcal{V}_n}. 
\]
Using \cite[Corollaire 3.2.3]{Ber} and a limit argument, one finds a canonical identification 
\[
R^if_{\cris*}(\cM)(\mathcal{V}_0,\mathcal{V})\stackrel{\sim}{\lra} H^i((\cX_{A,0}/A)_{\cris},\cM)
\]
where we denote again by $\cM$ the restriction of $\cM$ to $(\cX_{A,0}/A)_{\cris}$. Let $u=u_{\cX_{A,0}/A}$ be the morphism of topoi
\[
(\cX_{A,0}/A)_{\cris}^{\sim}\lra \cX_{A~ \! \et}^{\sim}
\]
such that $u_{\ast}(\mathcal F)(\cU)=H^0((\cU_0/A)_{\cris}^{\sim},\mathcal F)$ for $\cU\in \cX_{A~\!\et}$. By \cite[Theorem 7.23]{BO}, there exists a natural quasi-isomorphism in the derived category 
\begin{equation}\label{eq.derivedBOrelative}
Ru_{\ast} \cM\stackrel{\sim}{\lra} DR_{X/Y}(\cM),  
\end{equation}
inducing an isomorphism $H^i_{\cris}(\cX_{A,0}/A,\cM)\simto \mathbb H^i(\cX_{A},DR_{X/Y}(\cM))$. Thereby 
\begin{equation}\label{eq.BOrelative}
H^i_{\cris}(\cX_{A,0}/A,\cM)\stackrel{\sim}{\lra} \mathbb H^i(\cX_A,DR_{X/Y}(\cM)).
\end{equation}
Passing to associated sheaves, we deduce that \[R^if_{\cris *}(\cM)_{\cY}=R^if_{*}(DR_{X/Y}(\cM)).\] On the other hand, as $f\colon \cX\to \cY$ is proper and smooth, $R^if_{*}(DR_{X/Y}(\cE))$, viewed as a coherent sheaf on the adic space $Y$, is the $i$-th relative convergent cohomology of $\cE$ with respect to the morphism $f_0\colon \cX_0\to \cY_0$. Thus, by \cite[Th\'eor\`eme 5]{Ber86} (see also \cite[Theorem 4.1.4]{Tsu}), if we invert $p$, the $\cO_{\cY}[1/p]$-module $R^if_{*}(DR_{X/Y}(\cE))\simeq R^if_{*}(DR_{X/Y}(\cM))[1/p]$, together with the Gauss-Manin connection and the natural Frobenius structure inherited from $R^if_{\cris*}(\cM)_{\cY}\simeq R^if_{*}(DR_{X/Y}(\cM))$, is a convergent $F$-isocrystal on $\cY_0/\cO_k$, denoted by $R^if_{\cris *}(\cE)$ in the following (this is an abuse of notation, a more appropriate notation should be $R^if_{0~ \! \mathrm{conv}*}(\cE)$). Using the filtration on $\cE$, one sees that $R^if_{\cris *}(\cE)$ has naturally a filtration, and it is well-known that this filtration satisfies Griffiths transversality with respect to the Gauss-Manin connection. 

\begin{prop} \label{main1relative} Let $\cX\to \cY$ be a proper smooth morphism between two smooth formal schemes over $\cO_k$. Let $\mathcal E$ be a filtered convergent $F$-isocrystal on $\cX_0/\cO_k$ and $\LL$ a lisse $\widehat{\ZZ}_p$-sheaf on $X_{\proet}$. Assume that $\mathcal E$ and $\LL$ are associated. Then there is a natural filtered isomorphism of $\cO\BB_{\cris,Y}$-modules
\begin{equation}\label{eq.isorelative}
R^if_{k*}(\LL\otimes_{\widehat{\ZZ}_p}\BB_{\cris,X}\widehat{\otimes}f_k^{-1}\cO\BB_{\cris,Y}) \stackrel{\sim}{\longrightarrow} w_{\cY}^{-1}(R^if_{\cris *}( \cE))\otimes\cO\BB_{\cris,Y}
\end{equation}
which is compatible with Frobenius and connection.
\end{prop}

\begin{proof} Using the relative Poincar\'e lemma (Proposition \ref{relativepoincare} (1)) and the fact that $\LL$ and $\cE$ are associated, we have the following filtered isomorphisms compatible with connection: 
\begin{equation}\label{eq.isosrelatifs}
\begin{array}{cl}
& R^if_{k*}(\LL\otimes \BB_{\cris,X}\widehat{\otimes}f_k^{-1}\cO\BB_{\cris,Y}) \\ \stackrel{\sim}{\lra} & R^if_{k*}(\LL\otimes DR_{X/Y}(\cO\BB_{\cris,X})) \\ \stackrel{\sim}{\lra} & R^if_{k*}(DR_{X/Y}(\LL\otimes \cO\BB_{\cris,X})) \\ \stackrel{\sim}{\lra} & R^if_{k*}(DR_{X/Y}(w_{\cX}^{-1}\cE\otimes \cO\BB_{\cris,X})) \\ \stackrel{\sim}{\lra} & R^if_{k*}(w_{\cX}^{-1}DR_{X/Y}(\cE)\otimes\cO\BB_{\cris,X}).\end{array}
\end{equation}
On the other hand, we have the morphism below given by adjunction which respects also the connections on both sides:
\begin{equation}\label{eq.toshowitisaniso}
w_{\cY}^{-1}R^if_{*}(DR_{X/Y}(\cE))\otimes \cO\BB_{\cris,Y}\lra R^if_{k*}(w_{\cX}^{-1}DR_{X/Y}(\cE)\widehat{\otimes}\cO\BB_{\cris,X}).
\end{equation}
We claim that the morphism \eqref{eq.toshowitisaniso} is a filtered isomorphism. This is a local question, we may and do assume first that $\cY=\Spf(A)$ is affine and is \'etale over some torus defined over $\cO_k$. Let $V\in Y_{\proet}$ be an affinoid perfectoid pro-\'etale over $\widetilde{Y}_{\bk}$. As $R^if_{*}(DR_{X/Y}(\cE))=R^if_{\cris*}(\cE)$ is a locally free $\cO_{\cY}[1/p]$-module over $\cY$ and as $\cO\BBcr(V)$ is flat over $A$ (\cite[Th\'eor\`eme 6.3.8]{Bri}),
\[
\left(w_{\cY}^{-1}R^if_{*}(DR_{X/Y}(\cE))\otimes \cO\BB_{\cris,Y}\right)(V)\simeq H^i(\cX,DR_{X/Y}(\cE))\otimes_A\cO\BB_{\cris,Y}(V). 
\]
So we only need to check that the natural morphism below is a filtered isomorphism
\[
H^i(\cX,DR_{X/Y}(\cE))\otimes_A\cO\BB_{\cris,Y}(V)\lra H^i(X_V,w_{\cX}^{-1}DR_{X/Y}(\cE)\widehat{\otimes}\cO\BB_{\cris,X}).
\]
By Lemma \ref{quasirelative} one has the following identifications that are strictly compatible with filtrations: 
\begin{eqnarray*}
H^i(X_V,w_{\cX}^{-1}DR_{X/Y}(\cE)\otimes \cO\BB_{\cris,X})&\simeq & H^i(\cX, Rw_{V*}(w_{\cX}^{-1}DR_{X/Y}(\cE)\otimes \cO\BB_{\cris,X}))\\ & \simeq &  H^i(\cX, DR_{X/Y}(\cE)\widehat{\otimes}_A\cO\BB_{\cris,Y}(V)).
\end{eqnarray*}
Thus we are reduced to proving that the canonical morphism below is a filtered isomorphism: 
\[
H^i(\cX,DR_{X/Y}(\cE))\otimes_A\cO\BB_{\cris,Y}(V)\lra H^i(\cX, DR_{X/Y}(\cE)\widehat{\otimes}_A\cO\BB_{\cris,Y}(V)),
\]
which follows from Lemma \ref{lem.techBC}. 

Composing the isomorphisms in \eqref{eq.isosrelatifs} with the inverse of \eqref{eq.toshowitisaniso}, we get the desired filtered isomorphism \eqref{eq.isorelative} that is compatible with connections on both sides. It remains to check the Frobenius compatibility of \eqref{eq.isorelative}. For this, we may and do assume again that $\cY=\Spf(A)$ is affine and is \'etale over some torus over $\cO_k$, and let $V\in Y_{\proet}$ some affinoid perfectoid pro-\'etale over $\widetilde{Y}$. In particular, $A$ admits a lifting of the Frobenius on $\cY_0$, denoted by $\sigma$.  Let $\mathcal M$ be an $F$-crystal on $\cX_0/\cO_k$ such that $\cE=\mathcal M^{\rm an}(n)$ for some $n\in \mathbb N$ (Remark \ref{rk.CrystalVSIsoc}).  Then the crystalline cohomology $H^i(\cX_0/A,\mathcal M)$ is endowed with a Frobenius which is $\sigma$-semilinear. We just need to check the Frobenius compatibility of composition of the maps below (here the last one is induced by the inverse of \eqref{eq.isorelative}): 
\[
\begin{array}{c}
H_{\cris}^i(\cX_0/A,\mathcal M)\lra H_{\cris}^i(\cX_0/A,\mathcal M)[1/p]\stackrel{\sim}{\lra} H^i(\cX,DR_{X/Y}(\cE))\lra \\ \left(w_{\cY}^{-1}(R^if_{\cris*}(\cE)\otimes \cO\BB_{\cris,Y}\right)(V)  \lra H^i(X_V, \LL\otimes \BB_{\cris,X}\widehat{\otimes}f_k^{-1}\cO\BB_{\cris,Y}),
\end{array}
\]
which can be done exactly in the same way as in the proof of Theorem \ref{main1}. 
\end{proof}




The relative crystalline comparison theorem then can be stated as follows:

\begin{thm}\label{thm.relativecomp} Let $\LL$ be a crystalline lisse  $\widehat{\ZZ}_p$-sheaf on $X$ associated to a filtered $F$-isocrystal  $\cE$ on $\cX_0/\cO_k$. Assume that, for any $i\in \mathbb Z$, $R^if_{k*}\LL$ is a lisse  $\widehat{\ZZ}_p$-sheaf on $Y$. Then $R^if_{k*}\LL$ is crystalline and  is associated to the filtered convergent $F$-isocrystal $R^if_{\cris\ast}\cE$.
\end{thm}


\begin{proof} Let us first observe that the filtration on $R^if_{\cris*}\cE$ is given by locally direct summands (so $R^if_{\cris *}\cE$ is indeed a filtered convergent $F$-isocrystal on $\cY_0/\cO_k$). To see this, one uses \cite[Theorem 8.8]{Sch13}: by Proposition \ref{prop.crisdr}, the lisse $\widehat{\Z}_p$-sheaf $\LL$ is de Rham with associated filtered $\cO_X$-module with integrable connection $\cE$. Therefore the Hodge-to-de Rham spectral sequence 
\[
E_1^{i,j}=R^{i+j}f_{*}\left(\mathrm{gr}^{i}\left(DR_{X/Y}(\cE)\right)\right)\Longrightarrow R^{i+j}f_{*}(DR_{X/Y}(\cE))
\]
degenerates at $E_1$. Moreover $E_{1}^{i,j}$, the relative Hodge cohomology of $\cE$ in \cite[Theorem 8.8]{Sch13}, is a locally free $\cO_Y$-module of finite rank for all $i,j$ by \emph{loc. cit}. Therefore the filtration on $R^if_{*}(DR_{X/Y}(\cE))=R^if_{\cris*}(\cE)$, which is the same as the one induced by the spectral sequence above, is given by locally direct summands.   




To complete the proof, we need to find filtered isomorphisms that are compatible with Frobenius and connections:
\[
R^if_{k*}(\LL)\otimes \cO\BB_{\cris,Y}\stackrel{\sim}{\lra} w_{\cY}^{-1}R^if_{\cris*}(\cE)\otimes \cO\BB_{\cris,Y}, \quad i\in \mathbb Z.
\]
By Proposition \ref{part1relative} and Proposition \ref{main1relative}, we only need to check that the natural morphism below is a filtered almost isomorphism compatible with Frobenius and connections:
\[
R^if_{k*}(\LL\otimes \mathbb{A}_{\cris,X})\widehat{\otimes}\cO\mathbb{A}_{\cris,Y} \lra R^if_{k*}(\LL\otimes \mathbb{A}_{\cris,X}\widehat{\otimes}f^{-1}_k \cO\mathbb{A}_{\cris,Y} ). 
\]
The proof of this is similar to that of Proposition \ref{part1relative}: one just remarks that for each $n\in \mathbb N$, $\cO\mathbb{A}_{\cris,Y}/p^n|_{\widetilde Y}\simeq \mathbb{A}_{\cris,Y}|_{\widetilde{Y}}\langle w_1,\ldots, w_{\delta}\rangle /p^n$ hence $\cO\mathbb{A}_{\cris,Y}|_{\widetilde Y}/p^n$ is free over $\mathbb{A}_{\cris,Y}/p^n$ with a basis given by the divided powers $w^{[\underline{\alpha}]}$ with $\underline{\alpha}\in \mathbb N^{\delta}$ (recall that for $1\leq j\leq \delta$, $w_j=S_j-[S_j^{\flat}]\in \cO\mathbb{A}_{\cris,Y}|_{\widetilde{Y}}$). 
%Using Proposition \ref{relativepoincare}, Proposition \ref{part1}, and Lemma \ref{relativeab}, this can be obtained as follows:
%\begin{eqnarray*}
%R^if_{k*}\LL\otimes_{\widehat{\ZZ}_p}\cO\BB_{\cris,Y} & \stackrel{\sim}{\longleftarrow}& R^if_{k*}\LL\otimes_{\widehat{\ZZ}_p}\BB_{\cris,Y}\otimes_{\BB_{\cris,Y}}\cO\BB_{\cris,Y} \\ & \stackrel{\sim}{\lra} & R^if_{k*}(\LL\otimes_{\widehat{\ZZ}_p}\BB_{\cris,X})\otimes_{\BB_{\cris, Y}}\cO\BB_{\cris,Y} \\ & \stackrel{\sim}{\lra}& R^if_{k*}(\LL\otimes_{\widehat{\ZZ}_p}\BB_{\cris,X}\otimes_{f_k^{-1}\BB_{\cris,Y}} f_k^{-1}\cO\BB_{\cris,Y}) \\ & \stackrel{\sim}{\lra} & R^if_{k*}(DR(\LL\otimes \cO\BB_{\cris,X})) \\ &\stackrel{\sim}{\lra}  & R^if_{k*}(DR(\cE)\otimes \cO\BB_{\cris,X}) \\ & \stackrel{\sim}{\longleftarrow} & R^if_{k*}(DR(\cE))\otimes\cO\BB_{\cris, Y} \\ &\stackrel{\sim}{\longleftarrow} & R^if_{0\ast}(\cE)\otimes \cO\BB_{\cris, Y}. 
%\end{eqnarray*}
%Note that the compatibility with filtration and connection is clear by the results mentioned above. The compatibility with Frobenius is clear for all but the maps $R^if_{k,*}(DR(\LL\otimes \cO\BB_{\cris,X})) \simeq R^if_{k,*}(DR(\cE)\otimes \cO\BB_{\cris,X})$ and $R^if_{k,*}(DR(\cE)\otimes \cO\BB_{\cris,X}) \stackrel{\sim}{\leftarrow}R^if_{k,*}(DR(\cE))\otimes\cO\BB_{\cris, Y}$. For the first map, the compatibility with Frobenius follows from the argument in \S~\ref{341}. For the second one, one recalls it from  Lemma \ref{relativeab}.
\end{proof}


\section{Appendix: geometric acyclicity of $\cO\BBcr$}\label{acy}

In this section, we extend the main results of \cite{AB} to the setting of perfectoids. The generalization is rather straightforward. Although one might see here certain difference from the arguments in \cite{AB}, the strategy and technique are entirely theirs.

\medskip

Let $f\colon \cX=\Spf(B)\to \cY=\Spf(A)$ be a smooth morphism between two smooth affine formal scheme over $\cO_k$. Write $X$ (resp. $Y$) the generic fiber of $\cX$ (resp. of $\cY$). By abuse of notation the morphism $X\to Y$ induced from $f$ is still denoted by $f$. 

Assume that $\cY$ is \'etale over the torus $\mathcal S:=\Spf(\cO_k\{S_1^{\pm 1}, \ldots, S_{\delta}^{\pm 1}\})$ defined over $\cO_k$ and that the morphism $f\colon \cX\to \cY$ can factor as 
\[
\cX \stackrel{\textrm{\'etale}}{\lra} \mathcal{T}\lra \cY,
\]
so that  $\mathcal T=\Spf(C)$ is a torus over $\cY$ and  the first morphism $\cX\to \mathcal T$ is \'etale. 
\medskip 

Write $C=A\{T_1^{\pm 1},\ldots, T_d^{\pm 1}\}$. For each $1\leq i\leq d$ (resp. each $1\leq j\leq \delta$), let $\{T_i^{1/p^n}\}_{n\in \mathbb N}$ (resp. $\{S_j^{1/p^n}\}_{n\in \mathbb N}$) be a compatible family of $p$-power roots of $T_i$ (resp. of $S_j$). As in Proposition \ref{iso}, we denote by $\widetilde Y$ the following fiber product over the generic fiber $\mathcal S_k$ of $\mathcal S$:
\[
Y\times_{\mathcal S_k}\Spa(k\{S_1^{\pm 1/p^{\infty}},\ldots, S_{\delta}^{\pm 1/p^{\infty}}\},\cO_k\{S_1^{\pm 1/p^{\infty}},\ldots, S_{\delta}^{\pm 1/p^{\infty}}\}).
\]
Let $V\in Y_{\proet}$ be an affinoid perfectoid over $\widetilde{Y}_{\bk}$ with $\widehat{V}=\Spa(R,R^+)$. Let $T_V=\Spa(S,S^+)$ be the base change $\cT_k\times_Y V$ and $X_V=\Spa(\widetilde{S},\widetilde{S}^+)$ the base change $X\times_Y V$. Thus $S^+=R^+\{T_1^{\pm 1}, \ldots, T_d^{\pm 1}\}$ and $S=S^+[1/p]$. Set 
\[
S_{\infty}^+=R^+\left\{T_1^{\pm 1/p^{\infty}},\cdots, T_d^{\pm 1/p^{\infty}}\right\}, \quad \widetilde{S}_{\infty}^{+}:=B\widehat{\otimes}_C S_{\infty}^+, 
\]
$S_{\infty}:=S_{\infty}^+[1/p]$ and $\widetilde{S}_{\infty}:=\widetilde{S}_{\infty}^+[1/p]$. Then $(S_{\infty},S_{\infty}^+)$ and $(\widetilde{S}_{\infty},\widetilde{S}_{\infty}^+)$ are affinoids perfectoids and 
\[
S^{\flat+}_{\infty}=R^{\flat+}\left\{(T_1^{\flat})^{\pm 1/p^{\infty}},\cdots, (T_d^{\flat})^{\pm 1/p^{\infty}}\right\},
\]
where $T_i^{\flat}:=(T_i, T_i^{1/p}, T_{i}^{1/p^2}, \ldots )\in S_{\infty}^{\flat +}$. The inclusions $S^+\subset S_{\infty}^+$ and $\widetilde S^+\subset \widetilde{S}_{\infty}^+$ define two profinite Galois covers. Their Galois groups are the same, denoted by $\Gamma$, which is a profinite group isomorphic to $\mathbb Z_p(1)^d$. One can summarize these notations in the following commutative diagramme 
\[
\xymatrix{\widetilde{S}_{\infty}^+ & \widetilde{S}^+\ar[l]_{\Gamma} & B\ar[l] \\ S_{\infty}^+\ar[u] & S^+\ar[u]\ar[l]_{\Gamma} & C\ar[u]_{\textrm{\'etale}}\ar[l] \\ R^+\{T_1^{\pm 1/p^{\infty}},\ldots, T_d^{\pm 1/p^{\infty}}\}\ar@{=}[u] & R^+\{T_1^{\pm 1},\ldots, T_d^{\pm 1}\}\ar@{=}[u]\ar[l]_{\Gamma} & A\{T_1^{\pm 1},\ldots, T_d^{\pm 1}\}\ar@{=}[u]\ar[l] \\ & R^+\ar[u]& A\ar[u]\ar[l]}
\]
The group $\Gamma$ acts naturally on the period ring $\cO\BBcr(\widetilde{S}_{\infty},\widetilde{S}_{\infty}^+)$ and on its filtration $\Fil^r\cO\BBcr(\widetilde{S}_{\infty},\widetilde{S}_{\infty}^+)$. The aim of this appendix is to compute the group cohomology
\[
H^q\left(\Gamma, \cO\BBcr\left(\widetilde{S}_{\infty},\widetilde{S}_{\infty}^+\right)\right):=H^q_{\rm cont}\left(\Gamma, \cO\AAcr\left(\widetilde{S}_{\infty},\widetilde{S}^+_{\infty}\right)\right)[1/t]
\]
and 
\[
H^q\left(\Gamma, \Fil^r\cO\BBcr\left(\widetilde{S}_{\infty},\widetilde{S}_{\infty}^+\right)\right):=\varinjlim_{n\geq |r|} H_{\rm cont}^q\left(\Gamma, \frac{1}{t^n}\Fil^{r+n}\cO\AAcr\left(\widetilde{S}_{\infty},\widetilde{S}_{\infty}^+\right)\right)
\]
for $q,r\in \mathbb Z$. 

In the following, we will omit systematically the subscript {\textquotedblleft cont\textquotedblright} whenever there is no confusion arising. Moreover, we shall use the multi-index to simplify the notation:  for example, for $\underline{a}=(a_1,\ldots, a_d)\in \mathbb Z[1/p]^d$, $T^{\underline{a}}:=T_1^{a_1}\cdot T_2^{a_2}\cdots T_d^{a_d}$. 

\subsection{Cohomology of $\cO\BBcr$}

We will first compute $
H^q(\Gamma, \cO\AAcr(S_{\infty},S_{\infty}^+)/p^n)$ up to $(1-[\epsilon])^{\infty}$-torsion 
for all $q,n\in \mathbb N$.
 

\begin{lemma}\label{lem.isotechnique}
For $n\in \Z_{\geq 1}$, there are natural isomorphisms
\[
\AAcr(R,R^+)/p^n\otimes_{W(R^{\flat+})/p^n}W(S_{\infty}^{\flat+})/p^n\stackrel{\sim}{\lra} \AAcr(S_{\infty},S_{\infty}^+)/p^n  
\]
and 
\[
\left(\AAcr(S_{\infty},S_{\infty}^+)/p^n \otimes \cO\AAcr(R,R^+)/p^n\right)\langle u_1,\ldots, u_d\rangle \stackrel{\sim}{\lra} \cO\AAcr(S_{\infty},S_{\infty}^+)/p^n,
\]
sending $u_i$ to $T_i-[T_i^{\flat}]$. Here the tensor product in the last isomorphism above is taken over $\AAcr(R,R^+)/p^n$. 

Moreover, the natural morphisms 
\[
\AAcr(R,R^+)/p^n\to \AAcr(S_{\infty},S_{\infty}^+)/p^n, \quad \cO\AAcr(R,R^+)/p^n\to \cO\AAcr(S_{\infty},S_{\infty}^+)/p^n
\] 
are both injective. 


\end{lemma}

\begin{proof}
 Recall $\xi=[p^{\flat}]-p$. We know that $\AAcr(S_{\infty},S_{\infty}^+)$ is the $p$-adic completion of 
\[
\AAcr^{0}(S_{\infty},S_{\infty}^+):=W(S_{\infty}^{\flat+})\left[\frac{\xi^m}{m!}|m=0,1,\cdots\right]=\frac{W(S_{\infty}^{\flat+})[X_0,X_1,\cdots]}{(m! X_m-\xi^m: m\in \mathbb Z_{\geq 0})}.
\]
Note that we have the same expression with $R$ in place of $S_{\infty}$. We then have 
\begin{eqnarray*}
\AAcr^{0}(S_{\infty},S_{\infty}^+) &\stackrel{\sim}{\longleftarrow} & W(S_{\infty}^{\flat+})\otimes_{W(R^{\flat+})}\frac{W(R^{\flat+})[X_0,X_1,\cdots]}{(m!X_m-\xi^m|m\in \Z_{\geq 0})}\\ & =& W(S_{\infty}^{\flat+})\otimes_{W(R^{\flat+})}\AAcr^{0}(R,R^+).
\end{eqnarray*}
The first isomorphism follows. 

Secondly, as $V$ lies above $\widetilde Y_{\bk}$, by Proposition \ref{iso} we have
\[
\AAcr(R,R^+)\{\langle w_1,\ldots, w_{\delta}\rangle \} \stackrel{\sim}{\lra}\cO\AAcr(R,R^+), \quad w_j\mapsto S_j-[S_j^{\flat}]
\]
where $S_j^{\flat}:=(S_j,S_j^{1/p},S_j^{1/p^2},\ldots)\in R^{\flat+}$. Similarly 
\[
\AAcr(S_{\infty},S_{\infty}^+)\{\langle u_1,\ldots, u_d,w_1,\ldots, w_{\delta}\rangle\} \stackrel{\sim}{\lra} \cO\AAcr(S_{\infty},S_{\infty}^+), \quad u_i \mapsto T_i-[T_i^{\flat}], \ w_j\mapsto S_j-[S_j^{\flat}].  
\]
Thus (the isomorphisms below are all the natural ones)
\begin{eqnarray*}
\frac{\cO\AAcr(S_{\infty},S_{\infty}^+)}{p^n} & \stackrel{\sim}{\longleftarrow} & \left(\frac{\AAcr(S_{\infty},S_{\infty}^+)}{p^n}\right)\langle u_1,\ldots, u_d,w_1,\ldots, w_{\delta}\rangle \\ & \stackrel{\sim}{\longleftarrow} & \frac{\AAcr (S_{\infty},S_{\infty}^+)}{p^n} \otimes_{\frac{\AAcr(R,R^+)}{p^n}} \left(\frac{\AAcr(R,R^+)}{p^n}\langle u_1,\ldots, u_d,w_1,\ldots, w_{\delta}\rangle\right) \\ & \stackrel{\sim}{\lra}& \frac{\AAcr(S_{\infty},S_{\infty}^+)}{p^n}\otimes_{\frac{\AAcr(R,R^+)}{p^n}}\left(\frac{\cO\AAcr(R,R^+)}{p^n}\langle u_1,\ldots, u_d\rangle\right) \\ 
& \stackrel{\sim}{\lra} & \left(\frac{\AAcr(S_{\infty},S_{\infty}^+)}{p^n}\otimes_{\frac{\AAcr(R,R^+)}{p^n}}\frac{\cO\AAcr(R,R^+)}{p^n}\right)\langle u_1,\ldots, u_d\rangle.
 \end{eqnarray*}
So our second isomorphism is obtained. 


Next we prove that the natural morphism $\AAcr(R,R^+)/p^n\to \AAcr(S_{\infty},S_{\infty}^+)/p^n$ is injective.  When $n=1$, we are reduced to showing the injectivity of 
 \[
\frac{(R^{\flat+}/(p^{\flat})^p)[X_1,X_2,\ldots ]}{(X_1^p,X_2^p,\ldots)} \lra \frac{(S^{\flat+}/(p^{\flat})^p)[X_1,X_2,\ldots ]}{(X_1^p,X_2^p,\ldots)}, 
\] 
or equivalently the injectivity of 
\[
R^{\flat+}/(p^{\flat})^p\to S^{\flat+}/(p^{\flat})^p=\left(R^{\flat+}/(p^{\flat})^p\right)\left[(T_1^{\flat})^{\pm 1/p^{\infty}},\ldots (T_d^{\flat})^{\pm 1/p^{\infty}}\right],
\] 
which is clear. The general case follows easily since $\AAcr(S_{\infty},S_{\infty}^+)$ is $p$-torsion free. One deduces also the injectivity of $\cO\AAcr(R,^+)/p^n\to \cO\AAcr(S_{\infty},S_{\infty}^+)/p^n$ by using the natural isomorphisms
$
\cO\AAcr(R,R^+)/p^n\simeq (\AAcr(R,R^+)/p^n)\langle w_1,\ldots, w_{\delta}\rangle$, and $\cO\AAcr(S_{\infty},S_{\infty}^+)/p^n\simeq (\AAcr(S_{\infty},S_{\infty}^+)/p^n)\langle u_1,\ldots, u_d,w_1,\ldots, w_{\delta}\rangle$. This concludes the proof of our lemma.  
\end{proof}
 

\begin{prop} \label{prop.freeness}
$ \AAcr(S_{\infty},S_{\infty}^+)/p^n$ is free over $\AAcr(R,R^+)/p^n$ with a basis give by $\{[T^{\flat}]^{\underline a}|\underline{a}\in \Z[1/p]^d\}$. 
\end{prop}
 
 \begin{proof}
By Lemma \ref{lem.isotechnique}, $\AAcr(S_{\infty},S_{\infty}^+)/p^n$ is generated over $\AAcr(R,R^+)/p^n$ by elements of the form $[x]$ with $x\in S_{\infty}^{\flat+}=R^{\flat+}\{(T_1^{\flat})^{\pm 1/p^{\infty}},\ldots, (T_d^{\flat})^{\pm 1/p^{\infty}}\}$. Write $B_n\subset \AAcr(S_{\infty},S_{\infty}^+)/p^n$ for the $\AAcr(R,R^+)/p^n$-submodule generated by elements of the form $[x]$ with $x\in \mathsf{S}:=R^{\flat+}\left[(T_1^{\flat})^{\pm 1/p^{\infty}},\ldots, (T_d^{\flat})^{\pm 1/p^{\infty}}\right]\subset S_{\infty}^{\flat +}$. We claim that $B_n=\AAcr(S_{\infty},S_{\infty}^+)/p^n$. 
 
 
Since $S_{\infty}^{\flat +}$ is the $p^{\flat}$-adic completion of $\mathsf{S}$, for each $x\in S^{\flat+}$ we can write $x=y_0+p^{\flat}x'$ with $x'\in \mathsf{S}$. Iteration yields 
\[
x=y_0+p^{\flat}y_1+\cdots +(p^{\flat})^{p-1}y_{p-1}+(p^{\flat})^px''
\]
with $y_i\in \mathsf{S}$ and $x''\in S_{\infty}^{\flat +}$. Then in $W(S_{\infty}^{\flat +})$: 
 \begin{eqnarray*}
[x]&\equiv & [y_0]+[p^{\flat}][y_1]+\cdots +[(p^{\flat})^{p-1}][y_{p-1}]+[(p^{\flat})^p][x'']  \quad \textrm{mod} \quad pW(S_{\infty}^{\flat+})\\
&\equiv & [y_0]+\xi [y_1]+\cdots +\xi^{p-1}[y_{p-1}]+\xi^p[x'']  \quad  \textrm{mod} \quad pW(S_{\infty}^{\flat+}).
\end{eqnarray*}
As $\xi\in \mathbb A_{\cris}(S_{\infty},S_{\infty}^+)$ has divided power, $\xi^p=p!\cdot \xi^{[p]}\in p\AAcr(S_{\infty},S_{\infty}^+)$. So we obtain  in $\mathbb A_{\cris}(S_{\infty},S_{\infty}^+)$
\[
[x]\equiv [y_0]+\xi [y_1]+\cdots +\xi^{p-1}[y_{p-1}] \quad  \textrm{mod} \quad p\AAcr(S_{\infty},S_{\infty}^+).
\]
 For any $\alpha\in \AAcr(S_{\infty},S_{\infty}^+)/p^n=\AAcr(R,R^+)/p^n\otimes_{W(R^{\flat+})/p^n}W(S_{\infty}^{\flat+})/p^n$, we may write 
 \[
 \alpha=\sum_{i=0}^m\lambda_i[x_i]+p\alpha', \quad x_i\in S_{\infty}^{\flat+},\lambda_i\in \AAcr(R,R^+)/p^n, \alpha'\in \AAcr(S_{\infty},S_{\infty}^+)/p^n.
 \]
 The observation above tells us that one can write 
 \[
 \alpha=\beta_0+p\alpha'',\quad \beta_0\in B_n, \alpha''\in \AAcr(S_{\infty},S_{\infty}^+)/p^n.
 \]
By iteration again, we find 
\[
\alpha=\beta_0+p\beta_1+\cdots p^{n-1}\beta_{n-1}+p^n\tilde{\alpha}, \quad \beta_0,\cdots, \beta_{n-1}\in B_n, \tilde{\alpha}\in \AAcr(S_{\infty},S_{\infty}^+)/p^n.
\]
Thus  
\[
\alpha=\beta_0+p\beta_1+\cdots p^{n-1}\beta_{n-1}\in B_n\subset \AAcr(S_{\infty},S_{\infty}^+)/p^n.
 \]
This shows the claim, i.e.  $\AAcr(S_{\infty},S_{\infty}^+)/p^n$ is generated over $\AAcr(R,R^+)/p^n$ by the elements of the form $[x]$ with $x\in \mathsf S=R^{\flat +}[(T_1^{\flat})^{\pm 1/p^{\infty}},\ldots, (T_d^{\flat})^{\pm 1/p^{\infty}}]\subset S_{\infty}^{\flat+}$. Furthermore, as for any $x,y\in S_{\infty}^{\flat +}$
\[
[x+y]\equiv [x]+[y] \quad \mathrm{mod} \quad pW(S_{\infty}^{\flat +}),
\]
a similar argument shows that $\AAcr(S_{\infty},S_{\infty}^+)/p^n$ is generated over $\AAcr(R,R^+)/p^n$ by the family of elements  $\{[T^{\flat}]^{\underline a}|\underline{a}\in \Z[1/p]^d\}$.
 
It remains to show  the freeness of the family $\{[T^{\flat}]^{\underline a}|\underline{a}\in \Z[1/p]^d\}$ over $\AAcr(R,R^+)/p^n$. For this, suppose there exist $\lambda_1,\cdots, \lambda_m\in \AAcr(R,R^+)$ and distinct elements $\underline{a}_1,\cdots,\underline{a}_m\in \Z[1/p]^d$ such that 
 \[
 \sum_{i=1}^m\lambda_i[T^{\flat}]^{\underline{a}_i}\in p^n\AAcr(S_{\infty},S_{\infty}^+).
\]
One needs to prove $\lambda_i\in p^n \mathbb A_{\cris}(R,R^+)$ for each $i$. Modulo $p$ we find in $\AAcr(S_{\infty},S_{\infty}^+)/p$ that
\[
\sum_{i=1}^m\overline{\lambda_i} \cdot (T^{\flat})^{\underline{a}_i}=0,
\] 
with $\overline{\lambda_i}\in \AAcr(R,R^+)/p$ the reduction modulo $p$ of $\lambda_i$. 
On the other hand, the family  of elements $\{(T^{\flat})^{\underline{a}}:\underline{a}\in \mathbb Z[1/p]^d\}$ in 
\begin{eqnarray*}
\AAcr(S_{\infty},S_{\infty}^+)/p & \simeq  & \frac{S^{\flat+}/((p^{\flat})^p)[\delta_2,\delta_3,\ldots]}{(\delta_2^p,\delta_3^p,\ldots )} \\ & \simeq & \frac{R^{\flat +}/((p^{\flat})^p)[(T_1^{\flat})^{\pm1/p^{\infty}},\ldots, (T_d^{\flat})^{\pm 1/p^{\infty}}, \delta_2, \delta_3,\ldots ]}{(\delta_2^p,\delta_3^p,\ldots )}
\end{eqnarray*}
is free over $\AAcr(R,R^+)/p\simeq \frac{R^{\flat+}/((p^{\flat})^p)[\delta_2,\delta_3,\ldots]}{(\delta_2^p,\delta_3^p,\ldots )}$. Therefore, $\overline{\lambda_i}=0$, or equivalently, $\lambda_i=p\lambda_i'$ for some $\lambda_i'\in  \AAcr(R,R^+)$. In particular,  
 \[
 \sum_{i=1}^{m} \lambda_i[T^{\flat}]^{\underline{a}_i}=
p\cdot \left (\sum_{i=1}^m\lambda_i'[T^{\flat}]^{\underline{a}_i}\right)\in p^n\AAcr(S_{\infty},S_{\infty}^+).
\]
 But $\AAcr(S_{\infty},S_{\infty}^+)$ is $p$-torsion free, which implies that  
\[
\sum_{i=1}^m\lambda_i'[T^{\flat}]^{\underline{a}_i}\in p^{n-1}\AAcr(S_{\infty},S_{\infty}^+).
\]
This way, we may find $\lambda_i=p^n\tilde{\lambda}_i$ for some $\tilde{\lambda}_i\in  \AAcr(R,R^+)$, which concludes the proof of the freeness.
 \end{proof}
 
 
 
 
 
 
Recall that $\Gamma$ is the Galois group of the profinite cover $(S_{\infty},S_{\infty}^+)$ of $(S,S^+)$. Let $\epsilon=(\epsilon^{(0)},\epsilon^{(1)},\ldots)\in \cO_{\widehat{\bk}}^{\flat}$ be a compatible system of $p$-power roots of unity such that $\epsilon^{(0)}=1$ and that $\epsilon^{(1)}\neq 1$. Let $\{\gamma_1,\ldots, \gamma_d\}$ be a family of generators such that for each $1\leq i\leq d$, $\gamma_i$ acts trivially on the variables $T_j$ for any index $j$ different from $i$ and that $\gamma_i(T_i^{\flat})=\epsilon T_i^{\flat}$. 

 
\begin{lemma}[\cite{AB} Lemme 11] Let $1\leq i\leq d$ be an integer. Then one has $\gamma_i([T_i^{\flat}]^{p^n})=[T_i^{\flat}]^{p^n}$ in $\cO\AAcr(S_{\infty},S_{\infty}^+)/p^n$. 
\end{lemma}

\begin{proof} By definition, $\gamma_i([T_i^{\flat}]^{p^n})=[\epsilon]^{p^n}[T_i^{\flat}]^{p^n}$ in $\cO\AAcr(S_{\infty},S_{\infty}^+)$. So our lemma follows from the fact that $[\epsilon]^{p^n}-1=\exp(p^n t)-1=\sum_{r\geq 1}p^{nr} t^{[r]}\in p^nA_{\cris}$. 
\end{proof}
 
 
Let $A_n$ be the $\cO\AAcr(R,R^+)$-subalgebra of $\cO\AAcr(S_{\infty},S_{\infty}^+)/p^n$ generated by $[T_i^{\flat}]^{\pm p^n}$ for $1\leq i\leq d$. The previous lemma shows that $\Gamma$ acts trivially on $A_n$. Furthermore, by the second isomorphism of Lemma \ref{lem.isotechnique} and by Proposition \ref{prop.freeness}, we have 
\[
\frac{\cO\AAcr(S_{\infty},S_{\infty}^+)}{p^n} \stackrel{\sim}{\lra} \left(\bigoplus_{\underline{a}\in \mathbb Z[1/p]^d\cap [0,p^n)^d} A_{n}[T^{\flat}]^{\underline a}\right) \langle u_1,\ldots, u_d\rangle,\quad T_i-[T_i^{\flat}]\mapsto u_i.  
\]
Transport the Galois action of $\Gamma$ on $\cO\mathbb A_{\cris}(S_{\infty},S_{\infty}^+)/p^n$ to the righthand side of this isomorphism. It follows that  
\[
\gamma_i(u_i)=u_i+(1-[\epsilon])[T_i^{\flat}].
\] 
Therefore, 
\[
\gamma_i(u_i^{[n]})=u_i^{[n]}+\sum_{j=1}^{n}[T_i^{\flat}]^j(1-[\epsilon])^{[j]}u_i^{[n-j]}. 
\]
For other index $j\neq i$, $\gamma_i(u_j)=u_j$ and hence $\gamma_i(u_j^{[n]})=u_j^{[n]}$ for any $n$. Set 
\[
\mathbf{X}_{n}:=\bigoplus_{\underline{a}\in (\mathbb Z[1/p]\cap [0,p^n))^d \setminus \mathbb Z^d} A_n[T^{\flat}]^{\underline a}, \quad \textrm{and}\quad \mathbf A_n:=\bigoplus_{\underline{a}\in \mathbb Z^d\cap [0,p^n)^d}A_n[T^{\flat}]^{\underline{a}}. 
\]
Then we have the following decomposition, which respects  the $\Gamma$-actions:
 \[
 \cO\AAcr(S_{\infty},S_{\infty}^+)/p^n= \mathbf{X}_n\langle u_1,\ldots, u_d\rangle\oplus \mathbf A_n\langle u_1,\ldots, u_d\rangle.
 \] 

\begin{prop}[\cite{AB} Proposition 16] \label{hqxnu}For any integer $q\geq 0$, $H^{q}(\Gamma,\mathbf X_n\langle u_1,\ldots, u_d\rangle)$ is killed by $(1-[\epsilon]^{1/p})^2$. 
\end{prop}

\begin{proof}  Let $A_n[T^{\flat}]^{\underline{a}}$ be a direct summand of $\mathbf X_n$ with $\underline{a}\in (\Z[1/p]\cap [0,p^n))^d\setminus \mathbb Z$. Write $\underline{a}=(a_1,\ldots, a_d)$. Clearly we may assume that $a_1\notin \mathbb Z$. 

We claim first that cohomology group $H^q(\Gamma_1, A_n[T^{\flat}]^{\underline{a}})$ is killed by $1-[\epsilon]^{1/p}$ for any $q$, where $\Gamma_1\subset \Gamma$ is the closed subgroup generated by $\gamma_1$. As $\Gamma_1$ is topologically cyclic with a generator $\gamma_1$, the desired cohomology $H^q(\Gamma_1,A_n\cdot [T^{\flat}]^{\underline{a}})$ is computed using the following complex 
\[
A_n\cdot [T^{\flat}]^{\underline a}\stackrel{\gamma_1-1}{\longrightarrow}A_n\cdot [T^{\flat}]^{\underline a}.
\] 
We need to show that the kernel and the cokernel of the previous map are both killed by $1-[\epsilon]^{1/p}$. We have $(\gamma_1-1)([T^{\flat}]^{\underline{a}})=([\epsilon]^{a_1}-1)[T^{\flat}]^{\underline{a}}$. Since $a_1\notin \mathbb Z$, its $p$-adic valuation is $\leq -1$. Therefore, $1-[\epsilon]^{1/p}$ is a multiple of $1-[\epsilon]^{a_1}$, which implies that $\mathrm{coker}(\gamma-1\colon \mathbf X_n\to \mathbf X_n)$ is killed by $1-[\epsilon]^{1/p}$. On the other hand, let $x=\lambda\cdot [T^{\flat}]^{\underline a}$ be an element in $\ker(\gamma_1-1\colon A_n\cdot [T^{\flat}]^{\underline{a}}\to A_n\cdot [T^{\flat}]^{\underline a})$. Then $(\gamma_1-1)(x)=\lambda([\epsilon]^{a_1}-1) [T^{\flat}]^a$ and necessarily $\lambda([\epsilon]^{a_1}-1)=0$. Since $1-[\epsilon]^{1/p}$ is a multiple of $[\epsilon]^{a_1}-1$, $(1-[\epsilon])\lambda=0$ and thus $(1-[\epsilon]^{1/p})x=0$, giving our claim. 

Now for each $1\leq i\leq d$, let 
\[
E^{(i)}:=\left\{\underline{a}=(a_1,\ldots, a_d)\in \left(\mathbb Z[1/p]\bigcap [0,p^n)\right)^d: a_1,\ldots,a_{i-1}\in \mathbb Z,a_i\notin \mathbb Z\right\}
\]
and
\[
\mathbf{X}_n^{(i)}:=\bigoplus_{\underline{a}\in E^{(i)}} A_n[T^{\flat}]^{\underline{a}}.
\]
Then we have the following decompositions which respect also the $\Gamma$-action: 
\[
\mathbf{X}_n=\bigoplus_{i=1}^d\mathbf{X}_n^{(i)}, \quad \mathbf X_n\langle u_1,\ldots, u_d\rangle =\bigoplus_{i=1}^d\mathbf{X}_n^{(i)}\langle u_1,\ldots, u_d\rangle. 
\]
So we are reduced to proving that for each $1\leq i\leq d$, $H^q(\Gamma, \mathbf{X}_n^{(i)}\langle u_1,\ldots, u_d\rangle)$ is killed by $(1-[\epsilon]^{1/p})^2$ for all $q\geq 0$. By what we have shown in the beginning of the proof, $H^q(\mathbb Z_p\gamma_i, \mathbf{X}_n^{(i)})$ is killed by $1-[\epsilon]^{1/p}$. As $\gamma_i$ acts trivially on $u_j$ for $j\neq i$, one computes as in \cite[Lemme 15]{AB} and then finds that $H^q(\mathbb Z_p\gamma_i, \mathbf{X}_n^{(i)}\langle u_1,\ldots, u_d\rangle)$ is killed by $1-[\epsilon]^{1/p}$. Finally, one uses the Hochschild-Serre spectral sequence to conclude that $H^q(\Gamma, \mathbf{X}_n^{(i)}\langle u_1\ldots, u_d\rangle)$ is killed by $(1-[\epsilon]^{1/p})^2$ for any $q\geq 0$, as desired.  
\end{proof}




The computation of $H^q(\Gamma, \mathbf A_n\langle u_1,\ldots, u_d\rangle )$ is more subtle. Note that we have the following decomposition 
\[
\mathbf A_n\langle u_1,\ldots, u_d\rangle =\bigotimes_{i=1}^d (\cO\AAcr(R,R^+)/p^n)\left[[T_i^{\flat}]^{\pm 1}\right]\langle u_i\rangle, 
\]
where the tensor products above are taken over $\cO\AAcr(R,R^+)/p^n$. 

We shall first treat the case where $d=1$. We set $T:=T_1,u:=u_1$ and $\gamma:=\gamma_1$. Let $\mathcal A_n^{(m)}$ be the $A_n$-submodule of $\cO\AAcr(S_{\infty},S_{\infty}^+)/p^n$ generated by the $u^{[m+a]}/[T^{\flat}]^a$'s with $m+a\geq 0$ and $0\leq a<p^n$. Then 
\[
\mathbf{A}_n\langle u\rangle =A_n\left[[T^{\flat}]\right]\langle u\rangle =\sum_{m> -p^n} \mathcal A_n^{(m)}. 
\] 
Consider the following complex:
\begin{equation}\label{eq.An<U>}
A_n \left[[T^{\flat}]\right]\langle u\rangle \stackrel{\gamma_-1}{\longrightarrow} A_n\left[[T^{\flat}]\right]\langle u\rangle, 
\end{equation}
which computes $H^q(\Gamma, \mathbf A_n\langle u\rangle )=H^q\left(\Gamma,A_n\left[[T^{\flat}]\right]\langle u\rangle \right)$.

\begin{lemma}[\cite{AB} Proposition 20] The cokernel of \eqref{eq.An<U>}, and hence $H^q\left(\Gamma, \mathbf A_n\langle u\rangle\right)$ for any $q>0$, are killed by $1-[\epsilon]$. 
\end{lemma}
\begin{proof} 


We will proceed by induction on $m>-p^n$ to show that $(1-[\epsilon])\mathcal A_n^{(m)}$ is contained in the image of \eqref{eq.An<U>}. Note first $\mathcal A_n^{(1-p^n)}=A_n\cdot \frac{1}{[T^{\flat}]^{p^n-1}}$, while 
\[
(\gamma-1)(U/[T^{\flat}]^{p^n})=(1-[\epsilon])\frac{1}{[T^{\flat}]^{p^n-1}}. 
\] 
Thus $(1-[\epsilon])\mathcal A_{n}^{(1-p^n)}$ is contained in the image of \eqref{eq.An<U>}. Let $m>-p^n$ be a fixed integer. Then $\mathcal A_n^{(m)}$ is generated over $A_n$ by $\frac{u^{[m+p^n-i]}}{[T^{\flat}]^{p^n-i}}$ for $i=1,\ldots, D:=\mathrm{min}(p^n,p^n+m)$. 
On the other hand, for $u^{[m+a]}/[T^{\flat}]^a\in A_n\left[[T^{\flat}]\right]\langle u\rangle $ with $\mathrm{max}(0,-m)\leq a\leq  p^n$, we have
\begin{eqnarray*}
[\epsilon]^{a}(\gamma-1)(u^{[m+a]}/[T^{\flat}]^{a})&=& \frac{(u+(1-[\epsilon])[T^{\flat}])^{[m+a]}-[\epsilon]^au^{[m+a]}}{[T^{\flat}]^a} \\ & =& (1-[\epsilon]^a)\frac{u^{[m+a]}}{[T^{\flat}]^a}+\sum_{\begin{subarray}{c}1\leq i\leq a \\ i\leq m+a\end{subarray}}(1-[\epsilon])^{[i]}\frac{u^{[m+a-i]}}{[T^{\flat}]^{a-i}} \\ & & +\sum_{\begin{subarray}{c}a+1\leq i\leq a+p^n\\ i\leq m+a\end{subarray}}(1-[\epsilon])^{[i]}[T^{\flat}]^{p^n}\frac{u^{[m+a-i]}}{[T^{\flat}]^{a-i+p^n}} \\ & & +\ldots
\end{eqnarray*}
As for $n\geq 2$, $(1-[\epsilon])^{[n]}$ is a multiple of $1-[\epsilon]$ (\cite[Lemme 18]{AB}), and all the terms from the second line of the last equality belong to $\mathcal A_{n}^{(m-rp^n)}$ for $r=1,2,\ldots$. By induction hypothesis, we may assume that all these terms belong to the image of \eqref{eq.An<U>}. Write 
\[
\alpha_j=\frac{1-[\epsilon]^j}{1-[\epsilon]}, \quad \textrm{and}\quad \beta_j=\frac{(1-[\epsilon])^{[j]}}{1-[\epsilon]}.
\] 
Modulo the image of \eqref{eq.An<U>}, we find for $\mathrm{max}(0,-m)\leq a<p^n$
\[
(1-[\epsilon])\left(\alpha_a\frac{U^{[m+a]}}{[T^{\flat}]^a}+\frac{U^{[m+a-1]}}{[T^{\flat}]^{a-1}}+\beta_2\frac{U^{[m+a-2]}}{[T^{\flat}]^{a-2}}+\ldots + \beta_{a-(p^n-D)}\frac{U^{[m+p^n-D]}}{[T^{\flat}]^{p^n-D}}\right)\equiv 0; 
\]
and for $a=p^n$, as $[\epsilon]^{p^n}=1$ in $A_{\cris}/p^n$, modulo the image of \eqref{eq.An<U>} we have 
\[
(1-[\epsilon])\left(\frac{u^{[m+p^n-1]}}{[T^{\flat}]^{p^n-1}}+\beta_2\frac{u^{[m+p^n-2]}}{[T^{\flat}]^{p^n-2}}+\ldots + \beta_{D}\frac{u^{[m+p^n-D]}}{[T^{\flat}]^{p^n-D}}\right)\equiv 0
\]
Therefore, combining these congruences and modulo the image of \eqref{eq.An<U>} we have 
\[
(1-[\epsilon])\left(\frac{u^{[m+p^n-1]}}{[T^{\flat}]^{p^n-1}},\ldots , \frac{u^{[m+p^n-D]}}{[T^{\flat}]^{p^n-D}}\right)\cdot M_{n}^{(m)}\equiv 0
\]
where $M_n^{(m)}$ is the matrix 
\[
\left(\begin{array}{ccccc}
1 & a_{p^n-1} & 0 & \ldots & 0 \\ \beta_2 & 1 & a_{p^n-2} &\ldots & 0 \\ \vdots & \vdots & \vdots &\ddots & \vdots \\ \beta_{D-1}& \beta_{D-2} &  \beta_{D-3} & \ldots & a_{p^n-D} \\ \beta_{D} & \beta_{D-1}&  \beta_{D-2} & \ldots & 1
\end{array}\right) \in \mathbf M_D(A_{\cris}/p^n). 
\]
One can check that this matrix is invertible (\cite[Lemme 19]{AB}), so modulo the image of \eqref{eq.An<U>} we have $(1-[\epsilon])\frac{u^{[m+p^n-i]}}{[T^{\flat}]^{p^n-i}}=0$ for $1\leq i\leq D$. In other words, $(1-[\epsilon])\mathcal A_{n}^{(m)}$ is contained in the image of \eqref{eq.An<U>}. 
\end{proof}



One still needs to compute $H^0\left(\Gamma, \mathbf A_n\langle u\rangle \right)$. One remarks first that we have the following isomorphism  
\[
\left(\cO\AAcr(R,R^+)/p^n\right)\left[T^{\pm 1}\right]\langle u\rangle \simto \mathbf{A}_n\langle u\rangle=(\cO\AAcr(R,R^+)/p^n) \left[[T^{\flat}]\right]\langle u\rangle  
\]
sending $T$ to $u+[T^{\flat}]$. Endow an action of $\Gamma$ on $(\cO\AAcr(R,R^+)/p^n)[T^{\pm 1}]\langle u\rangle$ via the isomorphism above. So $H^0(\Gamma,\mathbf A_n\langle u\rangle )$ is naturally isomorphic to the kernel of the morphism 
\begin{equation}\label{eq.gammaminus1}
\gamma-1\colon (\cO\AAcr (R,R^+)/p^n)[T^{\pm 1}]\langle u\rangle \longrightarrow (\cO\AAcr(R,R^+)/p^n)[T^{\pm 1}]\langle u\rangle,
\end{equation}
and there is a natural injection
\[
C\otimes_{A} \cO\AAcr (R,R^+)/p^n=(\cO\AAcr(R,R^+)/p^n)[T^{\pm 1}] \hra H^0(\Gamma,\mathbf A_n\langle u\rangle). 
\]


\begin{lemma}[\cite{AB} Lemme 29] The cokernel of the last map is killed by $1-[\epsilon]$. 
\end{lemma}

\begin{proof} We have $\gamma(u)=T-[\epsilon][T^{\flat}]=(1-[\epsilon])T+[\epsilon]u$. Therefore
\begin{eqnarray*}
(\gamma-1)(u^{[m]}) & =& \left(\left(1-[\epsilon]\right)T+[\epsilon] u\right)^{[m]}-u^{[m]} \\ & =& ([\epsilon]^m-1)u^{[m]} +\sum_{j=1}^{m}(1-[\epsilon])^{[j]}T^j[\epsilon]^{m-j}u^{[m-j]} \\ & =& (1-[\epsilon])\left(\mu_m u^{[m]}+\sum_{j=1}^{m}\beta_jT^j[\epsilon]^{m-j}u^{[m-j]} \right), 
\end{eqnarray*}
where $\mu_m=\frac{[\epsilon]^m-1}{1-[\epsilon]}$, $\beta_j=\frac{(1-[\epsilon])^{[j]}}{1-[\epsilon]}$ (so $\beta_1=1$). For $N>0$ an integer, consider the matrix
\[
G^{(N)}:=\left(\begin{array}{ccccc}
0 & T & T^2\beta_2 & \ldots &  T^N\beta_N\\ 0 & \mu_1 & T[\epsilon] &\ldots & T^{N-1}\beta_{N-1}[\epsilon] \\ \vdots & \vdots & \vdots &\ddots & \vdots \\ 0 & 0 &  0 & \ldots & T\beta_1[\epsilon]^{N-1} \\ 0 & 0 &  0 & \ldots & \mu_N
\end{array}\right) \in \mathbf M_{1+N}(A_{\cris,n}[T^{\pm 1}]). 
\]
Then we have 
\[
(\gamma-1)(1,u,u^{[2]},\ldots, u^{[N]})=(1-[\epsilon])(1,u,u^{[2]},\ldots, u^{[N]}) G^{(N)}.
\]
Write 
\[
G^{(N)}=\left(\begin{array}{cc} 0 & H^{(N)} \\ 0 & D\end{array}\right)
\]
with $D=(0,\ldots, 0, \mu_N)$ a matrix of type $1\times N$. On checks that the matrix $H^{(N)}\in \mathbf M_{N}(A_{\cris, n}[T^{\pm 1}])$ is invertible. 

Now let $\alpha\in (\cO\AAcr(R,R^+)/p^n)\left[T^{\pm 1}\right]\langle u\rangle $ be an element contained in the kernel of \eqref{eq.gammaminus1}. Write 
\[
\alpha=\sum_{s=0}^{N} a_s u^{[s]} \in \left(\left(\cO\AAcr(R,R^+)/p^n\right)\left[T^{\pm 1}\right]\right)\langle u\rangle, \quad a_s\in (\cO\AAcr(R,R^+)/p^n)[T^{\pm 1}].  
\]
As $\alpha$ is contained in the kernel of \eqref{eq.gammaminus1},  
\[
(\gamma-1)(1,u,\ldots, u^{[N]})(a_0,a_1,\ldots,a_N)^{t}=0.
\]
It follows that 
\[
(1-[\epsilon])(1,u,\ldots, u^{[N]})\left(\begin{array}{c}H^{(N)}\tilde{a} \\ \mu_N a_N\end{array}\right)=0
\]
with $\tilde{a}=(a_1,\ldots, a_N)$. As the family $\{1,u,\ldots, u^{[N]}\}$ is linearly independent, $(1-[\epsilon])H^{(N)}\tilde a=0$. Since $H^{(N)}$ is invertible, we deduce $(1-[\epsilon])\tilde{a}=0$. In other words, $(1-[\epsilon])a_i=0$ for $i\geq 1$ and the lemma follows. 
\end{proof}


Now we are ready to prove 



\begin{prop} \label{hqanu}
For any $d>0$, $n>0$ and $q>0$, $H^q\left(\Gamma, \mathbf A_n\langle u_1,\ldots, u_d\rangle \right)$ is killed by $(1-[\epsilon])^{2d-1}$. Moreover, the natural morphism 
\begin{equation}\label{eq.injmor}
C\otimes_A\cO\AAcr(R,R^+)/p^n \lra H^0(\Gamma, \mathbf A_n\langle u_1,\ldots, u_d\rangle ), \quad T_i\mapsto u_i+[T_i^{\flat}]
\end{equation}
is injective with cokernel killed by $(1-[\epsilon])^{2d-1}$. 
\end{prop}
\begin{proof} Recall that we have the  decomposition 
\[
\mathbf A_n\langle u_1,\ldots, u_d\rangle =\bigotimes_{i=1}^d (\cO\AAcr(R,R^+)/p^n)\left[[T_i^{\flat}]^{\pm 1}\right]\langle u_i\rangle. 
\]
We shall proceed by induction on $d$. The case $d=1$ comes from the previous two lemmas. For integer $d>1$, one uses Hochschild-Serre spectral sequence 
\[
E_{2}^{i,j}=H^i(\Gamma/\Gamma_1,H^j(\Gamma_1, \mathbf{A}_n\langle u_1,\ldots, u_d\rangle))\Longrightarrow H^{i+j}(\Gamma, \mathbf A_n\langle u_1,\ldots, u_d\rangle ). 
\]
Using the decomposition above, the group $H^j(\Gamma_1, \mathbf{A}_n\langle u_1,\ldots, u_d\rangle)$ is isomorphic to 
\begin{eqnarray*}
H^j\left(\Gamma_1,\cO\AAcr(R,R^+)/p^n\left[[T_1^{\flat}]^{\pm 1}\right]\langle u_1\rangle \right)\otimes \left(\otimes_{i=2}^{d}(\cO\AAcr(R,R^+)/p^n)\left[[T_i^{\flat}]^{\pm 1}\right]\langle u_i\rangle \right). \end{eqnarray*} 
So by the calculation done for the case $d=1$, we find that, up to $(1-[\epsilon])$-torsion, $H^j(\Gamma_1, A_n\langle u_1,\ldots, u_d\rangle)$ is zero when $j>0$, and is equal to 
\[
(\cO\AAcr(R,R+)/p^n)[T_1^{\pm 1}]\otimes \left(\otimes_{i=2}^{d}(\cO\AAcr(R,R^+)/p^n)\left[[T_i^{\flat}]^{\pm 1} \right] \langle u_i\rangle\right)
\]
when $j=0$. 
Thus, up to $(1-[\epsilon])$-torsion, $E_{2}^{i,j}=0$ when $j>0$ and $E_{2}^{i,0}$ is equal to 
\[
(\cO\AAcr(R,R+)/p^n)[T_1^{\pm 1}]\otimes H^i\left(\Gamma/\Gamma_1,\left(\otimes_{i=2}^{d}(\cO\AAcr(R,R^+)/p^n)\left[[T_i^{\flat}]^{\pm 1}\right]\langle u_i\rangle \right)\right).
\]
Using the induction hypothesis, we get that, up to $(1-[\epsilon])^{2(d-1)-1+1}$-torsion, $E_{2}^{i,0}=0$ when $i>0$ and 
\[
E_{2}^{0,0}=(\cO\AAcr(R,R^+)/p^n)[T_1^{\pm 1},\ldots, T_d^{\pm 1}]=C\otimes_A \cO\AAcr(R,R^+)/p^n.
\]
As $E_2^{i,j}=0$ for $j>1$, we have short exact sequence 
\[
0\lra E_{\infty}^{q,0}\lra H^q(\Gamma, \mathbf A_n\langle u_1,\ldots, u_d\rangle)\lra E_{\infty}^{q-1,1}\lra 0.
\]
By what we have shown above, $E_{\infty}^{q-1,1}$ is killed by $(1-[\epsilon])$ (as this is already the case for $E_2^{q-1,1}$), and $E_{\infty}^{q,0}$ is killed by $(1-[\epsilon])^{2d-2}$ for $q>0$ (as this is the case for $E_2^{q,0}$), thus $H^q(\Gamma, \mathbf A_n\langle u_1,\ldots, u_d\rangle)$ is killed by $(1-[\epsilon])^{2d-1}$. 
For $q=0$, $H^0(\Gamma, \mathbf A_n\langle u_1,\ldots, u_d\rangle)\simeq E_{\infty}^{0,0}=E_2^{0,0}$. So the cokernel of the natural injection 
\[
C\otimes_A\cO\AAcr(R,R^+)/p^n\lra H^0(\Gamma, \mathbf A_n\langle u_1,\ldots, u_d\rangle)
\] 
is killed by $(1-[\epsilon])^{2d-2}$, hence by $(1-[\epsilon])^{2d-1}$. 
\end{proof}
 
\begin{rk} With more efforts, one may prove that $H^q(\Gamma, \mathbf{A}_n\langle u_1,\ldots, u_d\rangle)$ is killed by $(1-[\epsilon])^d$ for $q>0$ (\cite[Proposition 21]{AB}), and that the cokernel of the morphism \eqref{eq.injmor} is killed by $(1-[\epsilon])^2$ (\cite[Proposition 30]{AB}). 
\end{rk} 
 

\begin{cor}\label{cor.cohomologyofobcris1} For any $n\geq 0$ and any $q>0$, $H^q(\Gamma, \cO\AAcr(S_{\infty},S_{\infty}^+)/p^n)$ is killed by $(1-[\epsilon])^{2d}$. Moreover, the natural morphism 
\[
C\widehat{\otimes}_A\cO\AAcr(R,R^+)/p^n\lra  H^0(\Gamma, \cO\AAcr(S_{\infty},S_{\infty}^+)/p^n)
\]
is injective, with cokernel killed by $(1-[\epsilon])^{2d}$. 
\end{cor}



Recall that we want to compute $H^q(\Gamma, \cO\BBcr(\widetilde{S}_{\infty},\widetilde{S}_{\infty}^+))$. For this, one needs

\begin{lemma} Keep the notations above and assume that the morphism $f\colon \cX\to \cY$ is \'etale; thus $V=\Spa(R,R^+)$ and $X\times_YV=\Spa(\widetilde{S},\widetilde{S}^+)=\Spa(\widetilde{S}_{\infty},\widetilde{S}_{\infty}^+)$.
The natural morphism 
\[
B\widehat{\otimes}_A\cO\AAcr(R,R^+)\lra \cO\AAcr(\widetilde{S}_{\infty},\widetilde{S}_{\infty}^+)
\]
is an isomorphism. 
\end{lemma}

\begin{proof}By Lemma \ref{2obcris} we are reduced to showing that the natural map (here $w_j=S_j-[S_j^{\flat}]$) 
\[
B\widehat{\otimes}_A \mathbb A_{\cris}(R,R^+)\{\langle w_1,\ldots, w_{\delta}\rangle \} \lra \mathbb A_{\cris}(\widetilde{S},\widetilde{S}^+)\{\langle w_1,\ldots, w_{\delta}\rangle\}
\]
is an isomorphism. Since both sides of the previous maps are $p$-adically complete and without $p$-torsion, we just need to check that its reduction modulo $p$
\[
B/p\otimes_{A/p}(\mathbb A_{\cris}(R,R^+)/p)\langle w_1,\ldots, w_{\delta}\rangle \lra (\mathbb A_{\cris}(\widetilde{S},\widetilde{S}^+)/p)\langle w_1,\ldots, w_{\delta}\rangle
\]
is an isomorphism. Note that we have the following expression 
\[
(\mathbb{A}_{\cris}(R,R^+)/p)\langle w_1,\ldots, w_{\delta}\rangle \simeq \frac{(R^{\flat +}/((p^{\flat})^p))[\delta_m,w_i, Z_{im}]_{1\leq i\leq \delta, m\in \mathbb N}}{(\delta_m^p,w_i^m, Z_{im}^p)_{1\leq i\leq \delta,m\in \mathbb N}},
\]
and the similar expression for $(\mathbb A_{\cris}(\widetilde{S},\widetilde{S}^+)/p)\langle w_1,\ldots, w_{\delta}\rangle$, where $\delta_m$ is the image of $\gamma^{m+1}(\xi)$ with $\gamma: x\mapsto x^p/p$.
One sees that both sides of the morphism above  are $p^{\flat}$-adically complete (in fact $p^{\flat}$ is nilpotent). Moreover $R^{\flat+}$ has no $p^{\flat}$-torsion. So we are reduced to showing that the morphism 
\[
B/p\otimes_{A/p} \frac{(R^{\flat +}/p^{\flat})[\delta_m,w_i,Z_{im}]_{1\leq i\leq \delta,m\in \mathbb N}}{(\delta_m^p,w_i^p, Z_{im}^p)_{1\leq i\leq \delta,m\in \mathbb N}}\lra \frac{(\widetilde{S}^{\flat +}/p^{\flat})[\delta_m,w_i,Z_{im}]_{1\leq i\leq \delta,m\in \mathbb N}}{(\delta_m^p,w_i^p, Z_{im}^p)_{1\leq i\leq \delta,m\in \mathbb N}}
\]
is an isomorphism. But $R^{\flat +}/p^{\flat}\simeq R^{+}/p$ and $\widetilde{S}^{\flat +}/p^{\flat}\simeq \widetilde{S}^+/p$, so we just need to show that the following morphism is an isomorphism:
\[
\alpha:B/p\otimes_{A/p} \frac{(R^{+}/p)[\delta_m,w_i,Z_{im}]_{1\leq i\leq \delta,m\in \mathbb N}}{(\delta_m^p,w_i^p, Z_{im}^p)_{1\leq i\leq \delta,m\in \mathbb N}}\lra \frac{(\widetilde{S}^{+}/p)[\delta_m,w_i,Z_{im}]_{1\leq i\leq \delta,m\in \mathbb N}}{(\delta_m^p,w_i^p, Z_{im}^p)_{1\leq i\leq \delta,m\in \mathbb N}}.
\]
To see this, we consider the following diagram 
\[
\xymatrix{&& \frac{\widetilde{S}^+/p[\delta_m,w_i,Z_{im}]_{1\leq i\leq \delta,m\in \mathbb N}}{(\delta_m^p,w_i^p, Z_{im}^p)_{1\leq i\leq \delta,m\in \mathbb N}} \\ B/p\ar@/^1pc/[urr]\ar[r] & B/p\otimes_{A/p} \frac{R^+/p[\delta_m,w_i,Z_{im}]_{1\leq i\leq \delta,m\in \mathbb N}}{(\delta_m^p,w_i^p, Z_{im}^p)_{1\leq i\leq \delta,m\in \mathbb N}}\ar[ur]^{\alpha}  & \\ A/p\ar[u]^{\textrm{\'etale}}\ar[r] & \frac{R^+/p[\delta_m,w_i,Z_{im}]_{1\leq i\leq \delta,m\in \mathbb N}}{(\delta_m^p,w_i^p, Z_{im}^p)_{1\leq i\leq \delta,m\in \mathbb N}} \ar@/_1pc/[uur]_{\textrm{\'etale}} \ar[u]^{\textrm{\'etale}}& }.
\]
It follows that $\alpha$ is \'etale. To see that $\alpha$ is an isomorphism, we just need to show that this is the case after modulo some nilpotent ideals of both sides of $\alpha$. Hence we are reduced to showing that the natural morphism 
\[
B/p\otimes_{A/p} R^+/p \lra \widetilde{S}^+/p
\] 
is an isomorphism, which is clear from the definition.
\end{proof}

Apply the previous lemma to the \'etale morphism $f\colon \cX\to \mathcal T$, we find a canonical $\Gamma$-equivariant isomorphism
\[
B\widehat{\otimes}_C\cO\AAcr(S_{\infty},S_{\infty}^+)\stackrel{\sim}{\lra} \cO\AAcr(\widetilde{S}_{\infty},\widetilde{S}_{\infty}^+).
\]
In particular, we find 
\[
H^q(\Gamma, B\widehat{\otimes}_C\cO\AAcr(S_{\infty},S_{\infty}^+))\stackrel{\sim}{\lra} H^q(\Gamma, \cO\AAcr(\widetilde{S}_{\infty},\widetilde{S}_{\infty}^+)).
\] 
Now consider the following spectral sequence 
\[
E_{2}^{i,j}=R^i\varprojlim_n H^j(\Gamma, B\otimes_C\cO\AAcr(S_{\infty},S_{\infty}^+)/p^n) \Longrightarrow H^{i+j}(\Gamma, B\widehat{\otimes}_C\cO\AAcr(S_{\infty},S_{\infty}^+))
\]
which induces a short exact sequence for each $i$:
\begin{equation}\label{eq.sesforRlim}
\begin{array}{c}
0\lra R^1\varprojlim_n H^{i-1}(\Gamma, B\otimes_C\cO\AAcr(S_{\infty},S_{\infty})/p^n)\lra H^{i}(\Gamma, B\widehat{\otimes}_C\cO\AAcr(S_{\infty},S_{\infty}^+)) \\ \lra \varprojlim_n H^{i}(\Gamma, B\otimes_C\cO\AAcr(S_{\infty},S_{\infty})/p^n) \lra 0. 
\end{array}
\end{equation}
As $B$ is flat over $C$, it can be written as a filtered limit of finite free $C$-modules, and as $\Gamma$ acts trivially on $B$, the following natural morphism is an isomorphism for each $i$: 
\[
B\otimes_CH^{i}(\Gamma, \cO\AAcr(S_{\infty},S_{\infty})/p^n) \stackrel{\sim}{\lra} H^{i}(\Gamma, B\otimes_C\cO\AAcr(S_{\infty},S_{\infty})/p^n).
\]
Therefore, for $i\geq 1$, $H^{i}(\Gamma, B\otimes_C\cO\AAcr(S_{\infty},S_{\infty}^+)/p^n)$ is killed by $(1-[\epsilon])^{2d}$ by Corollary \ref{cor.cohomologyofobcris1}. Moreover, by the same corollary, we know that the following morphism is injective with cokernel killed by $(1-[\epsilon])^{2d}$:
\[
C\otimes_A \cO\AAcr(R,R^+)/p^n\lra H^0(\Gamma, \cO\AAcr(S_{\infty},S_{\infty}^+)/p^n). 
\]
Thus the same holds if we apply the functor $B\otimes_C-$ to the morphism above  
\[
B\otimes_A \cO\AAcr(R,R^+)/p^n\lra B\otimes_CH^0(\Gamma, \cO\AAcr(S_{\infty},S_{\infty}^+)/p^n).
\]
Passing to limits we obtain an injective morphism 
\[
B\widehat{\otimes}_A \cO\AAcr(R,R^+)\lra \varprojlim_n \left(B\otimes_CH^0(\Gamma, \cO\AAcr(S_{\infty},S_{\infty}^+)/p^n)\right),
\]
whose cokernel is killed by $(1-[\epsilon])^{2d}$, and  that 
\[
R^1\varprojlim_n \left(B\otimes_C H^0(\Gamma, \cO\AAcr(S_{\infty},S_{\infty}^+)/p^n)\right)
\]
is killed by $(1-[\epsilon])^{2d}$. As a result, using the short exact sequence \eqref{eq.sesforRlim}, we deduce that for $i\geq 1$, $H^i(\Gamma, B\widehat{\otimes}_C\cO\AAcr(S_{\infty},S_{\infty}^+))\simeq H^i(\Gamma, \cO\AAcr(\widetilde{S}_{\infty},\widetilde{S}_{\infty}^+))$ is killed by $(1-[\epsilon])^{4d}$, and that the canonical morphism 
\[
B\widehat{\otimes}_A\cO\AAcr(R,R^+)\lra H^0(\Gamma, B\widehat{\otimes}_C\cO\AAcr(S_{\infty},S_{\infty}^+))\simeq H^0(\Gamma, \cO\AAcr(\widetilde{S}_{\infty},\widetilde{S}_{\infty}^+))
\]
is injective with cokernel killed by $(1-[\epsilon])^{2d}$. One can summarize the calculations above as follows:

\begin{prop}\label{cor.cohomologyofobcris} (i) For any $n\geq 0$ and $q>0$, $H^q(\Gamma, \cO\AAcr(\widetilde{S}_{\infty},\widetilde{S}_{\infty}^+)/p^n)$ is killed by $(1-[\epsilon])^{2d}$, and the natural morphism 
\[
B\otimes_A\cO\AAcr(R,R^+)/p^n\lra H^0(\Gamma, \cO\AAcr(\widetilde{S}_{\infty},\widetilde{S}_{\infty}^+))/p^n
\]
is injective with cokernel killed by $(1-[\epsilon])^{2d}$.   

%(ii) $R^1\varprojlim_n H^0\left(\Gamma, \cO\AAcr(S_{\infty},S_{\infty}^+)/p^n\right)$  is killed by $(1-[\epsilon])^{2d}$.


(ii) For any  $q>0$, $H^q(\Gamma, \cO\AAcr(S_{\infty},S_{\infty}^+))$ is killed by $(1-[\epsilon])^{4d}$ and the natural morphism 
\[
B\widehat{\otimes}_A\cO\AAcr(R,R^+)\to H^0(\Gamma, \cO\AAcr(\widetilde{S}_{\infty},\widetilde{S}_{\infty}^+))
\]
is injective, with cokernel killed by $(1-[\epsilon])^{2d}$. 
\end{prop}

From this proposition, we deduce the following 


\begin{thm}\label{withoutfil} Keep the notations above. Then $H^q(\Gamma,\cO\BBcr(\widetilde{S}_{\infty},\widetilde{S}_{\infty}^+))=0$ for $q\geq 1$, and the natural morphism 
\[
B\widehat{\otimes}_A\cO\BBcr(R,R^+)\lra H^0(\Gamma,\cO\BBcr(\widetilde{S}_{\infty},\widetilde{S}_{\infty}^+))
\]
is an isomorphism. 
\end{thm}
\begin{proof} By the previous proposition, we just need to remark that inverting $t$ is equivalent to inverting  $1-[\epsilon]$, as 
\[
t=\log([\epsilon])=-\sum_{n\geq 1}(n-1)!\cdot (1-[\epsilon])^{[n]}=-(1-[\epsilon])\sum_{n\geq 1}(n-1)!\cdot \frac{(1-[\epsilon])^{[n]}}{1-[\epsilon]}.
\]
Here, by \cite[Lemme 18]{AB}, $\frac{(1-[\epsilon])^{[n]}}{1-[\epsilon]}\in \ker(A_{\cris}\stackrel{\theta}{\to} \widehat{\cO_{\bk}} )$, hence the last summation above converges in $A_{\cris}$. 
\end{proof}


\subsection{Cohomology of $\Fil^r\cO\BBcr$}

We keep the notations at the beginning of the appendix. In this \S, for simplicity, we will denote $\cO\AAcr(\widetilde{S}_{\infty},\widetilde{S}_{\infty}^+)$ (resp. $\Fil^r\cO\AAcr(\widetilde{S}_{\infty},\widetilde{S}_{\infty}^+)$) by $\cO\AAcr$ (resp. $\Fil^r\cO\AAcr$). This part is entirely taken from \cite[\S 5]{AB}, which we keep here for the sake of completeness.

\begin{lemma}\label{lem.killedbylargepower} Let $q\in \mathbb N_{>0}$, and $n$ an integer $\geq 4d+r$. The multiplication-by-$t^n$ morphism of $H^q(\Gamma, \Fil^r\cO\AAcr)$ is trivial.    
\end{lemma}

\begin{proof} Let $\mathrm{gr}^r\cO\AAcr:=\Fil^r\cO\AAcr/\Fil^{r+1}\cO\AAcr$. As $\theta(1-[\epsilon])=0$, $\mathrm{gr}^r\cO\AAcr$ is killed by $1-[\epsilon]$. So using the tautological short exact sequence below
\[
0\lra \Fil^{r+1}\cO\AAcr\lra \Fil^r\cO\AAcr\lra \mathrm{gr}^r\cO\AAcr\lra 0 
\]
and by induction on the integer $r\geq 0$, one shows that $H^q(\Gamma, \Fil^r\cO\AAcr)$ is killed by $(1-[\epsilon])^{4d+r}$; the $r=0$ case being Proposition \ref{cor.cohomologyofobcris}(ii). So the multiplication-by-$t^n$ with $n\geq 4d+r$ is zero for $H^q(\Gamma,\cO\AAcr)$. 
\end{proof}

Recall that for $q\in \mathbb Z$, we have defined 
\[
H^q(\Gamma,\Fil^r\cO\BBcr(\widetilde{S}_{\infty},\widetilde{S}_{\infty}^+)):=\varinjlim_{n\in \mathbb N} H^q(\Gamma, t^{-n}\Fil^{r+n}\cO\AAcr),
\] 
or, in an equivalent way,
\[
H^q(\Gamma, \Fil^r\cO\BBcr(\widetilde{S}_{\infty},\widetilde{S}_{\infty}^+))=\varinjlim_{n\in \mathbb N} H^q(\Gamma, \Fil^{r+n}\cO\AAcr),
\]
where for each $n$, the transition map 
\[
H^q(\Gamma, \Fil^{r+n}\cO\AAcr)\to H^q(\Gamma, \Fil^{r+n+1}\cO\AAcr)
\]
is induced from the map 
\[
\Fil^{r+n}\cO\AAcr\to \Fil^{r+n+1}\cO\AAcr, \quad x\mapsto t\cdot x. 
\]

We have the following easy observation.
\begin{lemma}[\cite{AB} Lemme 33]\label{lem.QpSpace} For each $q\in \mathbb Z$, $H^q(\Gamma, \Fil^r\cO\BBcr(\widetilde{S}_{\infty},\widetilde{S}_{\infty}^+))$ is a vector space over $\Qp$. 
\end{lemma}


The first main result of this section is the following.

\begin{prop}[\cite{AB} Proposition 34] $H^q(\Gamma, \Fil^r\cO\BBcr(\widetilde{S}_{\infty},\widetilde{S}_{\infty}^+))=0$ for any $q>0$. 
\end{prop}


\begin{proof} By Lemma \ref{lem.QpSpace}, the cohomology group $H^q(\Gamma, \Fil^r\cO\BBcr(\widetilde{S}_{\infty},\widetilde{S}_{\infty}^+))$ is a vector space over $\Qp$. Hence to show the desired annihilation, we just need to show that for any $x\in H^q(\Gamma, t^{-n}\Fil^{r+n}\AAcr)$, there exists $m\gg n$ such that the image of $x$ in $H^q(\Gamma, t^{-m}\Fil^{r+m}\cO\AAcr)$ is $p$-torsion. In view of Lemma \ref{lem.killedbylargepower}, we are reduced to showing that the kernel of the map 
\[
H^q(\Gamma, \Fil^{r+1}\cO\AAcr)\to H^q(\Gamma,\Fil^r\cO\AAcr) 
\] 
is of $p$-torsion, or equivalently, the cokernel of the map 
\[
H^{q-1}(\Gamma, \Fil^r\cO\AAcr)\to H^{q-1}(\Gamma, \mathrm{gr}^r\cO\AAcr)
\]
is of $p$-torsion for any $q\geq 1$. This assertion is verified in the following lemma. 
\end{proof}


\begin{lemma}[\cite{AB} Lemme 35, Lemme 36] For each $q\in \mathbb N$, the cokernel of the map 
\begin{equation}\label{eq.imageptorsion}
H^{q}(\Gamma, \Fil^r\cO\AAcr)\to H^{q}(\Gamma, \mathrm{gr}^r\cO\AAcr)
\end{equation}
is of $p$-torsion. 
\end{lemma}


\begin{proof} Recall that $\tilde{\xi}:=\frac{[\epsilon]-1}{[\epsilon]^{1/p}-1}$ is a generator of $\ker(\theta\colon W(\widetilde{S}_{\infty}^{\flat +})\to \widetilde{S}_{\infty}^+)$. Moreover, there exists a canonical isomorphism 
\[
\cO\AAcr\simeq \AAcr(\widetilde{S}_{\infty},\widetilde{S}_{\infty}^+)\{\langle u_1,\ldots, u_d,w_1,\ldots w_{\delta}\rangle \}.
\]
For $\underline{n}=(n_0,n_1,\ldots, n_{d+\delta})\in \mathbb N^{1+d+\delta}$ a multi-index, we set 
\[
\underline{u}^{\underline{n}}:=u_1^{[n_1]}\cdots u_d^{[n_d]}, \quad \underline w^{\underline n}:=w_1^{[n_{d+1}]}\cdots w_{\delta}^{[n_{d+\delta}]}. 
\]
In particular $\mathrm{gr}^r\cO\AAcr$ is a free $\widetilde{S}_{\infty}^+$-module with a basis given by 
\[
\tilde{\xi}^{[n_0]}\cdot \underline{u}^{\underline n}\cdot \underline w^{\underline n}, \quad \textrm{where } \underline n\in \mathbb N^{1+d+\delta} \textrm{ such that }|\underline n|=r.
\]
Recall that $\widetilde{S}^+\subset \widetilde{S}_{\infty}^+$. Let 
\[
D_r:=\bigoplus_{|\underline n|=r} \widetilde{S}^+\tilde{\xi}^{[n_0]}\cdot \underline{u}^{[\underline n]}\cdot \underline{w}^{[\underline n]} \subset \mathrm{gr}^r\cO\AAcr.
\]
For each $1\leq i\leq d $, $\gamma_i(u_i)=T_i-[\epsilon][T_i^{\flat}]=u_i-([\epsilon]-1)[T_i^{\flat}]=u_i-([\epsilon]^{1/p}-1)\tilde{\xi}[T_i]$. Viewed as element in $\mathrm{gr}^r\cO\AAcr$, we have $\gamma_i(u_i)=u_i-(\epsilon^{(1)}-1)\tilde{\xi}T_i$. In particular, $D_r\subset \mathrm{gr}^r\cO\AAcr$ is stable under the action of $\Gamma$. Similarly, one checks that the $S^+$-submodule $T^{\underline{\alpha}}\cdot D_r\subset \mathrm{gr}^r\cO\AAcr$ is again stable under the action of $\Gamma$. 


We first claim that, for any $\underline{\alpha}\in \left(\mathbb Z[1/p]\bigcap [0,1)\right)^d$ such that $\alpha_i\neq 0$ for some $1\leq i\leq d$, the cohomology $H^q(\Gamma, T^{\underline{\alpha}}D_r/p^hT^{\underline{\alpha}}D_r)$ is killed by $(\epsilon^{(1)}-1)^2$ for any $q,h$. Using Hochschild-Serre spectral sequence, we are reduced to showing that the cohomology of the complex 
\[
\cdots \lra 0\lra T^{\underline{\alpha}}D_r/p^hT^{\underline{\alpha}}D_r \stackrel{\gamma_i-1}{\lra} T^{\underline{\alpha}}D_r/p^hT^{\underline{\alpha}}D_r\lra 0\lra \cdots
\] 
is killed by $\epsilon^{\alpha_i}-1$ (as $\alpha_i\neq 0$, $\alpha^{\alpha_i}-1| \epsilon^{(1)}-1$), or in an equivalent way, the cohomology of the complex 
\[
\cdots \lra 0\lra D_r/p^hD_r \stackrel{\epsilon^{\alpha_i}\gamma_i-1}{\lra} D_r/p^hD_r\lra 0\lra \cdots
\]
is killed by $\epsilon^{\alpha_i}-1$. By definition, 
\begin{eqnarray*}
& (\epsilon^{\alpha_i}\gamma_i-1)(\tilde{\xi}^{[n_0]}\underline{u}^{[\underline n]}\underline{w}^{[\underline n]}) \\ = & \epsilon^{\alpha_i}\tilde{\xi}^{[n_0]}(u_i-(\epsilon^{(1)}-1)\tilde{\xi}T_i)^{[n_i]}\cdot \prod_{j\neq i}u_j^{[n_j]} \cdot \underline{w}^{[\underline n]}-\tilde{\xi}^{[n_0]}\underline{u}^{[\underline n]}\underline{w}^{[\underline n]} \\ =& (\epsilon^{\alpha_i}-1)\tilde{\xi}^{[n_0]}\underline{u}^{[\underline n]}\underline{w}^{[\underline n]}+(\epsilon^{(1)}-1)\eta \\ = & (\epsilon^{\alpha_i}-1)(\tilde{\xi}^{[n_0]}\underline{u}^{[\underline n]}\underline{w}^{[\underline n]}+\eta') 
\end{eqnarray*}
where $\eta$ and $\eta'$ are linear combinations of elements of the form $\tilde{\xi}^{[m_0]}\underline{u}^{[\underline{m}]}\underline{w}^{[\underline m]}$ with $|\underline{m}|=r$ such that $m_i<n_i$. Note that we have the last equality because $(\epsilon^{\alpha_i}-1)|(\epsilon^{(1)}-1)$. So if we write down the matrix $M$ of $\epsilon^{\alpha}\gamma_i-1$ with respect to the basis $\{\tilde{\xi}^{n_0}\underline{u}^{[\underline n]}\underline w^{[\underline n]}: |n|=r\}$ (the elements of this basis is ordered by the increasing value of the number $n_i$), then $M=(\epsilon^{\alpha_i}-1)\cdot U$ with $U$ a unipotent matrix. In particular $U$ is invertible and hence the kernel and the cokernel of the map $\epsilon^{\alpha_i}\gamma_i-1\colon D_r/p^hD_r\to D_r/p^hD_r$ are killed by $\epsilon^{\alpha_i}-1$, and hence by $\epsilon^{(1)}-1$. 


Now set
\[
X:=\bigoplus_{\underline{\alpha}\in \left(\mathbb Z[1/p]\bigcap [0,1)\right)^d\setminus \{(0,\cdots, 0)\}} \widetilde{S}^+\cdot T^{\underline{\alpha}} \subset \widetilde{S}_{\infty}^+. 
\]
By what we have shown above, $H^q(\Gamma, X\otimes_{\widetilde{S}^+}D_r/p^h)$ is killed by $(\epsilon^{(1)}-1)^2$ for all $q,h$. Hence a standard use of spectral sequence involving the higher derived functor of $\varprojlim$ shows that $H^q(\Gamma, X\widehat{\otimes}_{\widetilde{S}^+} D_r)$ is killed by $(\epsilon^{(1)}-1)^{4}$ for all $q$. On the other hand, as $D_r$ is an $\widetilde{S}^+$-module of finite rank, $X\widehat{\otimes}_{\widetilde{S}^+}D_r=\widehat{X}\otimes_{\widetilde{S}^+}D_r$ and one checks easily that $\mathrm{gr}^r\cO\AAcr=D_r\bigoplus (\widehat{X}\otimes_{\widetilde{S}^+}D_r)$. Therefore, the canonical map 
\[
H^q(\Gamma, D_r)\lra H^q(\Gamma, \mathrm{gr}^r\cO\AAcr)
\]
is injective with cokernel killed by $(\epsilon^{(1)}-1)^4$, hence by some power of $p$.

\medskip


It remains to show that $H^q(\Gamma, D_r)\subset H^q(\Gamma, \mathrm{gr}^r\cO\AAcr)$ is contained in the image of \eqref{eq.imageptorsion}. To see this, we shall use a different basis for the free $\widetilde{S}^+$-module $D_r$. For each $1\leq i\leq d$, set 
\[
v_i=\log\left(\frac{[T_i^{\flat}]}{T_i}\right)=\sum_{m=1}^{\infty}(-1)^{m-1}(m-1)!\left(\frac{[T_i^{\flat}]}{T_i}-1\right)^{[m]}.
\]
Then $v_i\equiv -T_i^{-1}u_i \ \mathrm{mod}\ \Fil^2A_{\cris}$, and $\gamma_i(v_i)=v_i+t$. Moreover, as $t\equiv (\epsilon^{(1)}-1)\tilde{\xi} \ \Fil^2A_{\cris}$, we see that 
\[
M_r:=\bigoplus_{|\underline n|=r} \widetilde{S}^+(\epsilon^{(1)}-1)^{n_0}\tilde{\xi}^{[n_0]}\underline{u}^{[\underline n]}\underline w^{[\underline n]} = \bigoplus_{|\underline n|=r} \widetilde{S}^+t^{[n_0]}\underline v^{[\underline n]}\underline w^{[\underline n]} 
\] 
Let $M_r^0:=\bigoplus_{|\underline n|=r}\mathbb Z_p t^{[n_0]}\underline v^{[\underline n]}\underline w^{[\underline n]}$. Then $M_r\simeq \widetilde{S}^+\otimes_{\Zp}M_r^0$. In particular we find 
\[
H^q(\Gamma, M_r)\simeq \widetilde{S}^+\otimes_{\Zp} H^q(\Gamma, M_r^0). 
\]
On the other hand, let $\widetilde{M}_r$ (resp. $\widetilde M_r^0$) be the $B\widehat{\otimes}_A\cO\AAcr(R,R^+)$-submodule (resp. the $\Zp$-submodule) of $\Fil^r\cO\AAcr$ generated by $\{t^{[n_0]}\underline v^{[\underline n]}\underline w^{[\underline n]}\}_{|\underline n|=r}$. Since $\gamma_i(v_i)=v_i+t$, $\widetilde M_r$ and $\widetilde M_r^0$ are both stable under the action of $\Gamma$. Moreover, the natural morphism $\Fil^r\cO\AAcr \to \mathrm{gr}^{r}\cO\AAcr$ induces a surjective morphism $\widetilde M_r\thra M_r$. Now we have the following tautological commutative diagram 
\[
\xymatrix{(B\widehat{\otimes}_A\cO\AAcr(R,R^+))\otimes_{\Zp}H^q(\Gamma, \widetilde M_r^0)\ar[d]\ar[r]^<<<<{\textrm{sur.}} & \widetilde{S}^+\otimes_{\Zp}H^q(\Gamma, M_r^0) \ar[d]^{\simeq }\\ 
H^q(\Gamma,\widetilde M_r)\ar[r] \ar[d]& H^q(\Gamma,M_r)\ar[d]^{\beta} \\ H^q(\Gamma, \Fil^r\cO\AAcr)\ar[r]^{\eqref{eq.imageptorsion}}& H^q(\Gamma, \mathrm{gr}^r\cO\AAcr)}.
\]
But the morphism $\beta$ can be decomposed as the composite of two morphisms 
\[
H^q(\Gamma, M_r)\ra H^q(\Gamma, D_r)\hra H^q(\Gamma, \mathrm{gr}^r\cO\AAcr)
\]
which is killed by $(\epsilon^{(1)}-1)^{r+4}$. This concludes the proof of our lemma. 
\end{proof}

It remains to compute the $\Gamma$-invariants of $\Fil^r\cO\BBcr(S_{\infty},S_{\infty})$. For this, we shall first prove $H^0(\Gamma,\Fil^r\cO\AAcr)=B\widehat{\otimes}_A \Fil^r\cO\AAcr(R,R^+)$. 

Recall that $\tilde{\xi}:=\frac{[\epsilon]-1}{[\epsilon]^{1/p}-1}$. Put \[\eta:=\frac{([\epsilon]-1)^{p-1}}{p}\in A_{\cris}[p^{-1}].\] For each $x\in \ker(\theta\colon \cO\AAcr(\widetilde{S}_{\infty},\widetilde{S}_{\infty}^+)\to \widetilde{S}_{\infty}^+)$, set $\gamma(x):=x^p/p$. One checks that $\gamma(x)\in \cO\AAcr$ and $\theta(\gamma(x))=0$. 

We have the following observation.

\begin{lemma}[\cite{AB} Lemme 37] \label{lem.easyone} There exists $\lambda\in \tilde{\xi}W(\cO_{\widehat{\bk}^{\flat}})$ such that $\eta=(p-1)!\tilde{\xi}^{[p]}+\lambda$. In particular, $\eta\in \ker(\theta\colon A_{\cris}\to \cO_{\widehat{\bk}})$. Furthermore, $\gamma^{\nu}(\eta)\in p^{-r}\eta A_{\cris}$ for any integers $0\leq \nu\leq r$. 
\end{lemma}

\begin{lemma}[\cite{AB} page 1011] There exists isomorphism 
\[
\cO\AAcr/p\simeq \frac{(\widetilde{S}_{\infty}^{\flat+}/\tilde{\xi}^p)[\eta_m,w_j,Z_{j,m}, u_i, T_{i,m}]_{1\leq i,j\leq d; m\in \mathbb N}}{(\eta_m^p, w_j^p,Z_{j,m}^p,u_i^p,T_{i,m}^p)}
\]
where $\eta_m$ denotes the image of $\gamma^m(\eta)$ in $\cO\AAcr/p$. We have a similar assertion for $\cO\AAcr(R,R^+)/p$. 
\end{lemma}

\begin{proof} Recall that $\tilde{\xi}$ is a generator of the kernel $\ker(\theta\colon W(S_{\infty}^{\flat +})\to S_{\infty}^+)$, hence we have isomorphism 
\[
\cO\AAcr/p\simeq \frac{(S_{\infty}^{\flat+}/\tilde{\xi}^p)[\delta_m,w_j,Z_{j,m}, u_i, T_{i,m}]_{1\leq i,j\leq d; m\in \mathbb N}}{(\delta_m^p, w_j^p,Z_{j,m}^p,u_i^p,T_{i,m}^p)}
\]
where $\delta_m$ represents the image in $\cO\AAcr/p$ of the element $\gamma^{m+1}(\tilde \xi)$. As $\eta=(p-1)!\tilde{\xi}^{[p]}+\lambda$ for some $\lambda\in \tilde \xi W(\cO_{\widehat{\bk}}^{\flat})$, it follows that $\eta_m\in \frac{(S_{\infty}^{\flat+}/\tilde{\xi}^p)[\delta_\nu]_{0\leq \nu\leq m}}{(\delta_{\nu}^p)_{0\leq \nu\leq m}}\subset \cO\AAcr/p$, and $\eta_m^p=0$, giving a morphism of $S_{\infty}^{\flat+}/{\tilde \xi}^p$-algebras
\[
\alpha_m\colon 
\frac{(S_{\infty}^{\flat +}/\tilde{\xi}^p)[W_{\nu}]_{0\leq \nu\leq m}}{(W_{\nu}^p)_{0\leq \nu\leq m}} \lra \frac{(S_{\infty}^{\flat +}/\tilde{\xi}^p)[\delta_{\nu}]_{0\leq \nu\leq m}}{(\delta_{\nu}^p)_{0\leq \nu\leq m}}, \quad W_{\nu}\mapsto \eta_{\nu}. 
\]
On the other hand, for any $\nu\in \mathbb N$, 
\[
\gamma^{\nu}((p-1)!\tilde{\xi}^{[p]})=\gamma^{{\nu}}(\eta-\lambda)=\sum_{i=0}^{p^{\nu}}a_i\eta^{[p^{\nu}-i]}\lambda^{[i]}
\]
for some coefficients $a_i\in S_{\infty}^{\flat +}/\tilde{\xi}^p$,  and $\lambda^{[i]}\in \frac{(S_{\infty}^{\flat +}/\tilde{\xi}^p)[\delta_j]_{0\leq j\leq \nu-1}}{(\delta_j^p)_{0\leq j\leq \nu-1}}$ (as $i\leq p^{\nu}$). By induction on $\nu$, one checks that $\delta_{\nu}\in \mathrm{Im}(\alpha_m)$, hence $\alpha_m$ is an isomorphism for the reason of length. Taking direct limits, one deduces that the morphism of $S_{\infty}^{\flat +}/\tilde{\xi}^p$-algebras
\[
\alpha\colon 
\frac{(S_{\infty}^{\flat +}/\tilde{\xi}^p)[W_m]_{m\in \mathbb N}}{(W_m^p)_{m\in \mathbb N}} \lra \frac{(S_{\infty}^{\flat +}/\tilde{\xi}^p)[\delta_m]_{m\in \mathbb N}}{(\delta_m^p)_{m\in \mathbb N}}, \quad W_m\mapsto \eta_m 
\]
is an isomorphism, proving our lemma. 
\end{proof}



\begin{lemma}[\cite{AB} Lemme 38] \label{lem.intersection0}For any $r\in \mathbb N$, we have $(p^r\cO\AAcr)\bigcap \eta\cO\AAcr=\sum_{\nu=0}^r p^r \gamma^{\nu}(\eta)\cO\AAcr$. 
\end{lemma}

\begin{proof} The proof is done by induction on $r$. The case $r=0$ is trivial. So we may and do assume that $r>0$. Let $x\in p^r\cO\AAcr\bigcap \eta\cO\AAcr$. By induction hypothesis, one can write 
\[
x=\sum_{\nu=0}^{r-1}p^{r-1}\gamma^{\nu}(\eta)x_{\nu}, \quad \textrm{with }x_{\nu}\in \cO\AAcr. 
\]
Set $x':=\sum_{\nu=0}^{r-1}\gamma^{\nu}(\eta)x_{\nu}$. Then $x'\in p\cO\AAcr$ since $x\in p^r\cO\AAcr$. So in $\cO\AAcr/p$, we have 
\[
\sum_{\nu=0}^{r-1}\eta_{\nu} \overline{x_{\nu}}=0,
\]
where for $a\in \cO\AAcr$ we denote by $\overline{a}$ its reduction modulo $p$ and $\eta_{\nu}$ the image of $\gamma^{\nu}(\eta)$ in $\cO\AAcr/p$. Let 
\[
\Lambda:=\frac{(\widetilde{S}_{\infty}^{\flat+}/\tilde{\xi}^p)[\eta_h,w_j,Z_{jm},u_i,T_{im}]_{1\leq i\leq d,1\leq j\leq \delta,m\in \mathbb N, h\geq r}}{(\eta_h^p,w_j^p,Z_{jm}^p,u_i^p,T_{im}^p)_{1\leq i\leq d,1\leq j\leq \delta,m\in \mathbb N, h\geq r}}.
\]
Then $\cO\AAcr/p\simeq \Lambda[\eta_m]_{0\leq m\leq r-1}/(\eta_m^p)_{0\leq m\leq r-1}$ by the previous lemma. In particular, $\cO\AAcr/p$ is a free $\Lambda$-module with a basis given by $\{\prod_{m=0}^{r-1}\eta_m^{\alpha_m}: \underline{\alpha}\in \{0,\ldots, p-1\}^r\}$. Thus up to modifying the $x_{\nu}$'s for $0\leq \nu\leq r-2$, we may assume that $\overline{x_{\nu}}\in \Lambda [\eta_0,\ldots, \eta_{\nu}]/(\eta_0^p,\ldots,\eta_{\nu}^p)$. 

Next we claim that $\overline{x_{\nu}}\in \eta_{\nu}^{p-1}(\cO\AAcr/p)$, which is also done by induction on $\nu$. When $\nu=0$, as $\overline{x_0}\in \Lambda[\eta_0]/(\eta_0^p)$, we just need to show $\eta_0\overline{x_0}=0$. But 
\[
\eta_0\overline{x_0}=-\sum_{h=1}^{r-1}\eta_h \overline{x_h} \in \Lambda[\eta_0,\eta_1,\ldots, \eta_{r-1}]/(\eta_0^p,\eta_1^p,\ldots, \eta_{r-1}^{p}),
\]
we have necessarily $\eta_0\overline{x_0}=0$, giving our claim for $\nu=0$. Let now $\nu\in \{1,\ldots, r-1\}$ such that for any $0\leq h\leq \nu$, $\overline{x_h}\in \eta_h^{p-1}(\cO\AAcr/p)$. So $\eta_h\overline{x_h}=0$ for $0\leq h\leq \nu$ and 
\[
\eta_{\nu} \overline{x_{\nu}}=-\sum_{l=\nu+1}^{r-1}\eta_l\overline{x_l}. 
\]
So necessarily $\eta_{\nu}\overline{x_{\nu}}=0$, and hence $\overline{x_{\nu}}\in \eta_{\nu}^{p-1}(\cO\AAcr/p)$. This shows our claim for any $0\leq \nu\leq r-1$. 

As a result, for any $0\leq \nu\leq r-1$, we have $x_{\nu}\in \gamma^{\nu}(\eta)^{p-1}\cO\AAcr+p\cO\AAcr$, and hence
\[
p^{r-1}\gamma^{\nu}(\eta)x_{\nu}\in p^{r-1}\gamma^{\nu}(\eta)^p\cO\AAcr+p^r\gamma^{\nu}(\eta)\cO\AAcr. 
\]
But $p^{r-1}\gamma^{\nu}(\eta)^p=p^r\gamma^{\nu+1}(\eta)$, so we get finally  $x\in \sum_{\nu=0}^r p^r\gamma^{\nu}(\eta)\cO\AAcr$. 
\end{proof}



Recall that we have an injection $B\widehat{\otimes}_A\cO\AAcr(R,R^+)\subset \cO\AAcr$ (Proposition \ref{cor.cohomologyofobcris}). The key lemma is the following 

\begin{lemma}[\cite{AB} Lemme 39]\label{lem.intersectioninobcris} Inside $\cO\AAcr$, we have 
\[
(B\widehat{\otimes}_A\cO\AAcr(R,R^+))\bigcap \eta\cO\AAcr=\eta(B\widehat{\otimes}_A \cO\AAcr(R,R^+)). 
\]
\end{lemma}

\begin{proof} The proof of the lemma is divided into several steps. First we remark that, as $B\otimes_A \cO\AAcr(R,R^+)/p\hookrightarrow \cO\AAcr/p$, we have 
\[
(B\widehat{\otimes}_A\cO\AAcr(R,R^+))\bigcap p\cO\AAcr=p(B\widehat{\otimes}_A \cO\AAcr(R,R^+)).
\]


Secondly, we claim that, to show our lemma, it is enough to prove that 
\begin{equation}\label{eq.enough}
(B\widehat{\otimes}_A \cO\AAcr(R,R^+))\bigcap \eta\cO\AAcr\subset \eta(B\widehat{\otimes}_A\cO\AAcr(R,R^+))+p\eta\cO\AAcr. 
\end{equation}
Indeed, once this is done, we obtain 
\begin{eqnarray*}
& (B\widehat{\otimes}_A \cO\AAcr(R,R+))\bigcap \eta\cO\AAcr\\ \subset & \eta(B\widehat{\otimes}_A\cO\AAcr(R,R^+))+p\eta\cO\AAcr\bigcap (B\widehat{\otimes}_A\cO\AAcr(R,R^+)) \\ = & \eta(B\widehat{\otimes}_A\cO\AAcr(R,R^+))+p\left(\eta\cO\AAcr\bigcap (B\widehat{\otimes}_A\cO\AAcr(R,R^+))\right) \\ \subset & \eta(B\widehat{\otimes}_A\cO\AAcr(R,R^+))+p\left(\eta (B\widehat{\otimes}_A\cO\AAcr(R,R^+))+p\eta\cO\AAcr\right) \\ \subset & \eta(B\widehat{\otimes}_A\cO\AAcr(R,R^+))+p^2\eta\cO\AAcr.
\end{eqnarray*}
An iteration of this argument shows that 
\[
(B\widehat{\otimes}_A \cO\AAcr(R,R+))\bigcap \eta\cO\AAcr\subset \eta(B\widehat{\otimes}_A\cO\AAcr(R,R^+))+p^r\eta\cO\AAcr
\]
for any $r\in \mathbb N$. Thus $(B\widehat{\otimes}_A \cO\AAcr(R,R+))\bigcap \eta\cO\AAcr\subset \eta(B\widehat{\otimes}_A\cO\AAcr(R,R^+))$ as $\cO\AAcr$ and $B\widehat{\otimes}_A\cO\AAcr(R,R^+)$ are $p$-adically complete.  

\vskip 2mm

Next we claim that for any $r\in \mathbb N$ we have 
\begin{equation}\label{eq.enough0}
(B\widehat{\otimes}_A\cO\AAcr(R,R^+))\bigcap \eta\cO\AAcr\subset \eta(B\widehat{\otimes}_A \cO\AAcr(R,R^+))+p^r\cO\AAcr. 
\end{equation}
When $r=0$, this is trivial. Assume now the inclusion above holds for some $r\in \mathbb N$. So by Lemma \ref{lem.intersection0}
\begin{eqnarray*}
& (B\widehat{\otimes}_A \cO\AAcr(R,R^+))\bigcap \eta \cO\AAcr \\ \subset & \eta(B\widehat{\otimes}_A \cO\AAcr(R,R^+))+(B\widehat{\otimes}_A\cO\AAcr(R,R^+))\bigcap p^r\cO\AAcr\bigcap \eta\cO\AAcr \\ \subset &  \eta (B\widehat{\otimes}_A\cO\AAcr(R,R^+))+(B\widehat{\otimes}_A\cO\AAcr(R,R^+))\bigcap p^r\left(\sum_{\nu=0}^{r}\gamma^{\nu}(\eta)\cO\AAcr\right) \\ \subset & \eta (B\widehat{\otimes}_A\cO\AAcr(R,R^+))+p^r\left((B\widehat{\otimes}_A\cO\AAcr(R,R^+))\bigcap \left(\sum_{\nu=0}^{r}\gamma^{\nu}(\eta)\cO\AAcr\right) \right).
\end{eqnarray*}
So we are reduced to showing that 
\[
\left(\left(B\widehat{\otimes}_A\cO\AAcr\left(R,R^+\right)\right)\bigcap \left(\sum_{\nu=0}^{r}\gamma^{\nu}(\eta)\cO\AAcr\right) \right)\subset p^{-r}\eta(B\widehat{\otimes}_A\cO\AAcr(R,R^+))+p\cO\AAcr. 
\]
Put 
\[
\Lambda(R,R^+):=\frac{(R^{\flat+}/\tilde{\xi}^p)[w_j, Z_{jm}]_{1\leq j\leq \delta, m\in \mathbb N}}{(w_j^m,Z_{jm}^p)_{1\leq j\leq \delta, m\in \mathbb N}}\subset \cO\AAcr(R,R^+)/p,
\]
and 
\[
\Lambda(\widetilde{S}_{\infty},\widetilde{S}_{\infty}^+):=\frac{(\widetilde{S}_{\infty}^{\flat+}/\tilde{\xi}^p)[w_j, Z_{jm}]_{1\leq j\leq \delta, m\in \mathbb N}}{(w_j^m,Z_{jm}^p)_{1\leq j\leq \delta, m\in \mathbb N}}\subset \cO\AAcr/p.
\]
We claim that the natural morphism $A\to \cO\AAcr(R,R^+)/p$ (resp. $B\to \cO\AAcr/p$) factors through $\Lambda(R,R^+)$ (resp. through $\Lambda(\widetilde{S}_{\infty},\widetilde{S}_{\infty}^+)[u_i]_{1\leq i\leq d}/(u_i^p)_{1\leq i\leq d}$). Indeed, we consider  the reduction modulo $p$ of the morphism $\theta$
\[
\overline{\theta}\colon \cO\AAcr(R,R^+)/p\lra R^+/p. 
\]
Then all elements of $\ker(\overline{\theta})$ are nilpotent, and $\overline{\theta}(\Lambda(R,R^+))=R^+/p$. Recall that $A$ is \'etale over $\cO_k\{S_1^{\pm 1}, \ldots, S_{\delta}^{\pm 1}\}$ and the composite 
\[
\cO_k\{S_1^{\pm 1},\ldots, S_{\delta}^{\pm 1}\}\to A\to \cO\AAcr(R,R^+)/p 
\]
sends $S_j$ to the reduction modulo $p$ of $w_j+S_j^{\flat}\in \cO\AAcr(R,R^+)/p$, which  factors through $\Lambda(R,R^+)$ and hence so does the morphism $A\to \cO\AAcr(R,R^+)/p$ by the \'etaleness of the morphism $\cO_k\{S_1^{\pm 1}, \ldots, S_{\delta}^{\pm 1}\}\to A$. In a similar way, one sees that the morphism $B\to \cO\AAcr/p$ factors through $\Lambda(\widetilde{S}_{\infty},\widetilde{S}_{\infty}^+)[u_i]_{1\leq i\leq d}/(u_i^p)_{1\leq i\leq d}$. In particular, it makes sense to consider the map 
\[
B\otimes_A \frac{\Lambda(R,R^+)[\eta_m]_{m\in \mathbb N}}{(\eta_m^p)_{m\in \mathbb N}}\lra  \widetilde{\Lambda}:=\frac{\Lambda(\widetilde{S}_{\infty},\widetilde{S}_{\infty}^+)[u_i,\eta_m]_{1\leq i\leq d, m\in \mathbb N}}{(u_i^p,\eta_m^p)_{1\leq i\leq d, m\in \mathbb N}},
\]
which is injective. Now for $x\in (B\widehat{\otimes}_A\cO\AAcr(R,R^+))\bigcap \sum_{\nu=0}^r \gamma^{\nu}(\eta)\cO\AAcr$, its reduction $\overline{x}$ modulo $p$ lies in $\sum_{\nu=0}^r\eta_{\nu}\cO\AAcr/p$. On the other hand, as a module $\cO\AAcr/p$ is free over $\widetilde{\Lambda}$ and the inclusion $\widetilde{\Lambda}\subset \cO\AAcr/p$ is a direct factor of $\cO\AAcr/p$ as $\widetilde{\Lambda}$-modules, the fact that $\overline{x}$ appears in the image of the morphism above (note that $\cO\AAcr(R,R^+)/p=\Lambda(R,R^+)[\eta_m]_{m\in \mathbb N}/(\eta_m^p)_{m\in \mathbb N}$) implies that 
\[
\overline x\in \sum_{\nu=0}^{r}\eta_{\nu} \widetilde{\Lambda}.
\]
Thus 
\[
\overline x\in \sum_{\nu=0}^{r} \eta_{\nu}(B\widehat{\otimes}_A \Lambda(R,R^+))[\eta_m]_{m\in \mathbb N}/(\eta_m^p)_{m\in \mathbb N}
\]
as $(B\otimes_A \Lambda(R,R^+))[\eta_m]_{m\in \mathbb N}/(\eta_m^p)_{m\in \mathbb N}$ and $\widetilde{\Lambda}$ are free over $B\otimes_A\Lambda(R,R^+)$ and over $\Lambda(R,R^+)[u_i]_{1\leq i\leq d}/(u_i^p)_{1\leq i\leq d}$ respectively , with a basis given by $\{\prod_{m\in \mathbb N} \eta_m^{\alpha_m}:\underline{\alpha}\in \{0,\ldots, p-1\}^{(\mathbb N)}\}$. In other words, this shows 
\[
x\in \sum_{\nu=0}^{r}\gamma^{\nu}(\eta)(B\widehat{\otimes}_A\cO\AAcr(R,R^+))+p\cO\AAcr. 
\]
By Lemma \ref{lem.easyone}, $\gamma^{\nu}(\eta)\in p^{-r}\eta(B\widehat{\otimes}_A\cO\AAcr(R,R^+))$, we obtained finally 
\[
x\in p^{-r}\eta(B\widehat{\otimes}_A\cO\AAcr(R,R^+))+p\cO\AAcr,
\]
as desired. 

%\vskip 2mm

To complete the proof, it remains to show that the inclusions \eqref{eq.enough0} for all $r\in \mathbb N$ implies \eqref{eq.enough}.  For $\nu\geq -1$ an integer, set $\beta_{\nu}:=\prod_{i=0}^{\nu}\gamma^i(\eta)^{p-1}$. A direct calculation shows $\eta \beta_{\nu}=p^{\nu+1}\gamma^{\nu+1}(\eta)$ for any $\nu\geq -1$. Let 
\[
\Lambda:=\frac{\Lambda(\widetilde{S}_{\infty},\widetilde{S}_{\infty}^+)[u_i,T_{im}]_{1\leq i\leq d, m\in \mathbb N}}{(u_i^p,T_{im}^p)_{1\leq i\leq d, m\in \mathbb N}}.
\]
Then $\cO\AAcr/p=\Lambda[\eta_m]_{m\in \mathbb N}/(\eta_m^p)_{m\in \mathbb N}$. Recall also that 
\[
\cO\AAcr(R,R^+)=\Lambda(R,R^+)[\eta_m]_{m\in \mathbb N}/(\eta_m^p)_{m\in \mathbb N}.
\]
For $\nu\geq -1$, let $M_{\nu}$ (resp. $N_{\nu}$) be the sub-$\Lambda$-module (resp. sub-$B\otimes_A \Lambda(R,R^+)$-module) of $\cO\AAcr/p$ (resp. of $B\otimes_A \cO\AAcr(R,R^+)/p$) generated by the family 
\[
\left\{\prod_{m=0}^{\infty}\eta_m^{\alpha_m}: \underline{\alpha}\in \{0,\ldots, p-1\}^{(\mathbb N)}, \alpha_m=p-1 \textrm{ for }0\leq m\leq \nu, \alpha_{\nu+1}<p-1 \right\}.
\]
Clearly $N_{\nu}\subset M_{\nu}$, $\cO\AAcr/p=\oplus_{\nu\geq -1}M_{\nu}$, and $B\otimes_A\cO\AAcr(R,R^+)/p=\oplus_{\nu\geq -1}N_{\nu}$.  
Let $z\in (B\widehat{\otimes}_A\cO\AAcr(R,R^+))\bigcap \eta \cO\AAcr$, and write $z=\eta z' $ with $z'\in \cO\AAcr$. Let $N$ be an integer such that
\[
\overline{z'}\in \bigoplus_{\nu=-1}^N M_{\nu}.
\] 
By \eqref{eq.enough0} for $r=N+2$, one can write 
\[
z=\eta \cdot y + w, \quad \textrm{with}\quad y\in B\widehat{\otimes}_A\cO\AAcr(R,R^+) \textrm{ and }w\in p^{N+2}\cO\AAcr\bigcap \eta\cO\AAcr. 
\]
By Lemma \ref{lem.intersection0}, $w=p^{N+2}\sum_{i=0}^{N+2}\gamma^i(\eta)\alpha_i$ with $\alpha_i\in \cO\AAcr$. So we find 
\[
w=\sum_{i=0}^{N+2}p^{N+2-i}p^i\gamma^i(\eta)\alpha_i=\sum_{i=0}^{N+2}p^{N+2-i}\eta\beta_{i-1}\alpha_i. 
\]
Therefore $z'=y+\sum_{i=0}^{N+2}p^{N+2-i}\beta_{i-1}\alpha_i$ and thus $\overline{z'}=\overline y+\overline{\beta_{N+1}}\overline{\alpha_{N+2}}$. But by the definition of the integer $N$, 
\[
\overline{z'}\in \bigoplus_{\nu=-1}^{N}M_{\nu}
\]
while 
\[
\overline{y}\in \bigoplus_{\nu=-1}^{\infty}N_{\nu}\quad \textrm{and}\quad \overline{\beta_{N+1}}\overline{\alpha_{N+2}} \in \bigoplus_{\nu>N}M_{\nu}.
\]
So necessarily $\overline{\beta_{N+1}}\overline{\alpha_{N+2}}\in \oplus_{\nu>N}N_{\nu} \in B\otimes_A\cO\AAcr(R,R^+)/p$. In other words, we find $\beta_{N+1}\alpha_{N+2}\in B\widehat{\otimes}_A \cO\AAcr(R,R^+)+p\cO\AAcr$. So finally 
\[
z\in \eta(y+\beta_{N+1}\alpha_{N+2})\in \eta (B\widehat{\otimes}_A\cO\AAcr(R,R^+))+p\eta \cO\AAcr, 
\]
as required by \eqref{eq.enough}. This finishes the proof of this lemma. 
\end{proof}




\begin{cor}[\cite{AB} Corollaire 40]\label{cor.intersectioninobcris} Inside $\cO\AAcr$, we have 
\[
(B\widehat{\otimes}_A \cO\AAcr(R,R^+))\bigcap ([\epsilon]-1)^{p-1}\cO\AAcr=([\epsilon]-1)^{p-1}(B\widehat{\otimes}_A\cO\AAcr(R,R^+)).
\] 
\end{cor}

\begin{proof} Let $x\in (B\widehat{\otimes}_A\cO\AAcr(R,R^+))\bigcap ([\epsilon]-1)^{p-1}\cO\AAcr $. As $p\eta=([\epsilon]-1)^{p-1}$, $x\in (B\widehat{\otimes}_A\Fil^r\cO\AAcr(R,R^+))\bigcap p\cO\AAcr=p(B\widehat{\otimes}_A\cO\AAcr(R,R^+))$. Thus, by Lemma \ref{lem.intersectioninobcris}, we find $x/p\in (B\widehat{\otimes}_A\cO\AAcr(R,R^+))\bigcap \eta \cO\AAcr=\eta(B\widehat{\otimes}_A \cO\AAcr(R,R^+))$. Therefore $x\in p\eta (B\widehat{\otimes}_A\cO\AAcr(R,R^+))=([\epsilon]-1)^{p-1}(B\widehat{\otimes}_A \cO\AAcr(R,R^+))$.  
\end{proof}

Finally, we can assemble the previous results to get the main result about the $\Gamma$-invariants.

\begin{prop}[\cite{AB} Proposition 41] For each $r\in \mathbb N$, the natural injection 
\[
\iota_r\colon B\widehat{\otimes}_A\Fil^r\cO\AAcr(R,R^+)\lra H^0(\Gamma, \Fil^r\cO\AAcr)
\]
is an isomorphism. 
\end{prop}


\begin{proof} We shall begin with the case where $r=0$. By Proposition \ref{cor.cohomologyofobcris} (ii), the natural morphism $B\widehat{\otimes}_A\cO\AAcr(R,R^+)\to H^0(\Gamma, \cO\AAcr)$ is injective with cokernel killed by $(1-[\epsilon])^{2d}$. It remains to show that the latter map is also surjective. Let $x\in H^0(\Gamma, \cO\AAcr)$. So $(1-[\epsilon])^{2d}x\in B\widehat{\otimes}_A\cO\AAcr(R,R^+)$. In particular, 
\[
(1-[\epsilon])^{2d(p-1)}x\in \left(B\widehat{\otimes}_A\cO\AAcr(R,R^+)\right)\bigcap (1-[\epsilon])^{2d(p-1)}\cO\AAcr.
\]
By Corollary \ref{cor.intersectioninobcris}, the last intersection is just $(1-[\epsilon])^{2d(p-1)}(B\widehat{\otimes}_A\cO\AAcr(R,R^+))$. In particular, $x\in B\widehat{\otimes}_A\cO\AAcr(R,R^+)$; note that $1-[\epsilon]$ is a regular element. This concludes the proof of our proposition for $r=0$. Assume now $r>0$ and that the statement holds for $\Fil^{r-1}\cO\AAcr$. Then we have the following commutative diagram with exact rows 
\[
\xymatrix{0\ar[r] & B\widehat{\otimes}_A \Fil^r\cO\AAcr(R,R^+)\ar[r] \ar[d]^{\iota_r}& B\widehat{\otimes}_A \Fil^{r-1}\cO\AAcr(R,R^+)\ar[r]\ar[d]^{\iota_{r-1}} & B\widehat{\otimes}_A \mathrm{gr}^{r-1}\cO\AAcr(R,R^+)\ar[r]\ar@{^(->}[d] & 0 \\ 0\ar[r] & H^0(\Gamma, \Fil^{r}\cO\AAcr)\ar[r] & H^0(\Gamma, \Fil^{r-1}\cO\AAcr)\ar[r] & H^0(\Gamma, \mathrm{gr}^{r-1}\cO\AAcr) & } 
\]
One checks easily that the last vertical morphism is injective as $B\widehat{\otimes}_A R^+=\widetilde{S}^+\subset \widetilde{S}_{\infty}^+$. So by snake lemma,  that $\iota_{r-1}$ is an isomorphism implies that the morphism $\iota_r$ is also an isomorphism. This finishes the induction and hence the proof of our proposition.  
\end{proof}


\begin{cor}[\cite{AB} Corollaire 42] The natural morphism 
\[
B\widehat{\otimes}_A \Fil^r\cO\BBcr(R,R^+) \lra H^0(\Gamma, \Fil^r\cO\BBcr) 
\]
is an isomorphism, where 
\[
B\widehat{\otimes}_A\Fil^r\cO\BBcr(R,R^+):=\varinjlim_{n\geq 0} B\widehat{\otimes}_A \Fil^{r+n}\cO\AAcr(R,R^+)
\]
with transition maps given by multiplication by $t$. 
\end{cor}


%%%%%%%%%%%%%%%%%%%%%%%%%%%%%%%%%%%%
\begin{thebibliography}{54}

\bibitem[AB]{AB} F. Andreatta, O. Brinon: {\em Acyclicit\'e g\'eom\'etrique de $B_{\cris}$}, Comment. Math. Helv. \textbf{88} (2013), 965-1022.


\bibitem[AI]{AI}F. Andreatta, A. Iovita: {\em Comparison Isomorphisms for Smooth Formal Schemes},  Journal de l'Institut de Math\'ematiques de Jussieu (2013) \textbf{12}(1), 77-151.

\bibitem[Ber86]{Ber86} P. Berthelot, {\em G\'eom\'etrie rigide et cohomologie des vari\'et\'es alg\'ebriques de caract\'eristique $p$}, Bull. Soc. Math. France, M\'emoire 23 (1986), 7-32.
 
\bibitem[Ber96]{Ber} P. Berthelot, {\em Cohomologie rigide et cohomologie rigide \`a supports propres}, Premi\'ere partie, Preprint IRMAR \textbf{96-03} (1996), 91 pp.

 
\bibitem[BO]{BO}P. Berthelot, A. Ogus, {\em Notes on crystalline cohomology}, Princeton University Press, 1978.

\bibitem[Bri]{Bri} O. Brinon, {\em Repr\'esentations p-adiques cristallines et de de Rham dans le cas relatif}, M\'emoires de la SMF \textbf{112} (2008).
 
%\bibitem[Ete]{Ete} J.-Y. Etesse, {\em Images directes II: $F$-isocristaux convergents}, preprint available in arXiv: 0910.4430 
 
 
\bibitem[Fal]{Fal} G. Faltings, {\em Crystalline cohomology and $p$-adic Galois representations}, Algebraic analysis, geometry, and number theory, Johns Hopkins Univ. Press (1989), 25-80. 
 
% \bibitem[JP]{JP} J. de Jong, M. van der Put. {\em \'Etale cohomology of rigid analytic spaces}, Doc. Math. \textbf{1} (1996), 1-56.

\bibitem[Fon82]{Fon82} J.-M.Fontaine, {\em Sur certains types de repr\'esentations $p$-adiques du group de Galois d'un corps local; construction d'un anneau de Barsotti-Tate}, Ann. of Math. (2) \textbf{115} (1982), 529-577.


\bibitem[Fon94]{Fon94} J.-M. Fontaine, {\em Le corps des p\'eriodes $p$-adiques}, P\'eriodes $p$-adiques (Bures-sur-Yvette, 1988), Ast\'erisque \textbf{223} (1994), 59-111.

\bibitem[GR]{GR} O. Gabber, L. Ramero, {\em Almost ring theory},  LNM \textbf{1800}, Springer-Verlag, Berlin, 2003.


\bibitem[Hub]{Hub}  R. Huber, {\em \'Etale cohomology of rigid analytic varieties and adic spaces}, Aspects of Mathematics, \textbf{E 30}. Friedr. Vieweg \& Sohn, Braunschweig, 1996.
 
\bibitem[Hub94]{Hub94}  R. Huber, {\em A generalization of formal schemes and rigid analytic varieties}, Math. Z. \textbf{217} (1994), 513-551.


\bibitem[Jan]{Jan} U. Jannsen, {\em Continuous \'etale cohomology}, Math. Ann. \textbf{280}(1988), 207-245.
 
%\bibitem[Kie]{Kie} R. Kiehl,  {\em Der Endlichkeitssatz f\"ur eigentliche Abbildungen in der nichtarchimedischen Funktionentheorie}, Invent. Math. \textbf{2} (1967), 191-214.
 
\bibitem[Sch13]{Sch13} P. Scholze, {\em $p$-adic Hodge theory for rigid-analytic varieties}, Forum of Mathematics, Pi, \textbf{1} (2013), 1-77.


\bibitem[SGA4]{SGA4} M. Artin, A. Grothendieck, J.-L. Verdier, {\em Th\'eorie des topos et cohomologie \'etale des sch\'emas}, LNM \textbf{269}, \textbf{270}, \textbf{305}, Springer Verlang, 1972, 1973.
 
\bibitem[Tat]{Tat} J. Tate, {\em $p$-divisible groups}, Proc. Conf. Local Fields (Driebergen, 1966), 158-183, Springer, Berlin, 1967.

\bibitem[Tsu]{Tsu} N. Tsuzuki, {\em On base change theorem and coherence in rigid cohomology}, Kazuya Kato's fiftieth birthday, Doc. Math. 2003, Extra Vol., 891?918 

 \end{thebibliography} 

  
\vspace{\baselineskip}


 



Fucheng Tan



800 Dongchuan Rd

Department of Mathematics, Shanghai Jiao Tong University

Shanghai 200240, China

\medskip

and

\medskip

220 Handan Rd

Shanghai Center for Mathematical Sciences

Shanghai 200433, China

Email: fuchengtan@sjtu.edu.cn




\vspace{\baselineskip}
 \vspace{\baselineskip}
    
Jilong Tong

Institut de Math\'e�matiques de Bordeaux

Universit\'e� de Bordeaux

351, cours de la Lib\'e�ration

F-33405 Talence cedex, France

Email: jilong.tong@math.u-bordeaux1.fr
    
\end{document}