\documentclass[10pt,twoside]{amsart}

\usepackage{amsmath,amssymb
}
%,showkeys}
%\usepackage[active]{srcltx}
\usepackage{amsfonts}
\usepackage{amsthm,amscd}
%\usepackage[english]{babel}
%\usepackage{graphicx}
\usepackage{amsmath}
\usepackage{amssymb}
%\usepackage{psfrag}
\usepackage{amstext}
\usepackage{color}
\usepackage{latexsym}
\usepackage{amscd,graphics}
%\usepackage{verbatim}


%%%%%%%%%%%%%%%%

\begin{document}
\newcommand{\comments}[1]{\marginpar{\footnotesize #1}} %Comment
%\newcommand{\comments}[1]{}


\newtheorem{proposition}{Proposition}[section]
\newtheorem{lemma}[proposition]{Lemma}
\newtheorem{sublemma}[proposition]{Sublemma}
\newtheorem{theorem}[proposition]{Theorem}

\newtheorem{maintheorem}{Main Theorem}
\newtheorem{corollary}[proposition]{Corollary}

\newtheorem{ex}[proposition]{Example}

\theoremstyle{remark}

\newtheorem{remark}[proposition]{Remark}

\theoremstyle{definition}
\newtheorem{definition}[proposition]{Definition}
\newcommand{\ovB}{\bar{B}}
\def\Erg{\mathrm{Erg}\, }
\def\cone{\mathbf{C}} % Cone
\def\real{\mathbb{R}}
\def\sphere{\mathbf{S}^{d-1}}% Sphere
\def\integer{\mathbb{Z}}
\def\complex{\mathbb{C}}
\def\BBB{\mathbb{B}}
\def\supp{\mathrm{supp}}
\def\var{\mathrm{var}}
\def\sgn{\mathrm{sgn}}
\def\sp{\mathrm{sp}}
\def\id{\mathrm{id}}
\def\Imm{\mathrm{Image}}
\def\cc{\Subset}
\def\D{\mathrm {d}}
\def\I{i}
\def\E{e}
\def\Lip{\mathrm{Lip}}
\def\BB{\mathcal{B}}
\def\CC{\mathcal{C}}
\def\DD{\mathcal{D}}
\def\EE{\mathcal{E}}
\def\FF{\mathcal{F}}
\def\GG{\mathcal{G}}
\def\II{\mathcal{I}}
\def\JJ{\mathcal{J}}
\def\KK{\mathcal{K}}
\def\LL{\mathcal{L}}
\def\LLL{\mathbb{L}}
\def\MM{\mathcal{M}}
\def\NN{\mathcal{N}}
\def \OO{\mathcal {O}}
\def \PP{\mathcal {P}}
\def \QQ{\mathcal {Q}}
\def \RR{\mathcal {R}}
\def\SS{\mathcal{S}}
\def\TT{\mathcal{T}}
\def\UU{\mathcal{U}}
\def\VV{\mathcal{V}}
\def\WW{\mathcal{W}}
\def\YY{\mathcal{Y}}
\def\ZZ{\mathcal{Z}}
\def\FFF{\mathbb{F}}
\def\PPP{\mathbb{P}}




\title[The quest for the ultimate anisotropic Banach space]{
%\vspace*{-1.8cm}
C\MakeLowercase{orrections and
complements to:}\\
T\MakeLowercase{he quest for the ultimate anisotropic} 
B\MakeLowercase{anach space}}

\date{January 10, 2018}
\author{
\vspace*{-0.4cm}
Viviane Baladi} 
\thanks{I thank M. J\'ez\'equel, who found several  unclear points in \cite{Baladijoel} and helped to clarify them.}


%%%%%%%%%%%%%%%%%%%%%


\maketitle

%
\vspace*{-0.6cm}

We provide corrections and complements to  \cite[\S4]{Baladijoel}: Formulas (29) (and (42))
and, especially,  (31) must be amended, as explained below.
None of the main statements
are changed, except that the condition $-(r-1)<s<-t<0$
 must be replaced by  $t-(r-1)<s<-t<0$ in Lemma~4.2
and  Theorem 4.1. 


\section{Details for the leafwise Young inequality (38)}

To prove (38) on p. 542, notice that, for  $\ell\ge 1$ and $x\in \real^{d_s}$,
Fubini implies
\begin{align*}
&(\psi_\ell^{(d_s)})^{Op}\bigl[(\varphi \ast \hat \psi) \circ \pi_{\Gamma}^{-1}\bigr] (x) = 
[(\psi_\ell^{(d_s)})^{Op}\int \hat \psi (z) \varphi  (\pi_{\Gamma}^{-1}(\cdot)-z)  \mathrm{d} z ](x)\\
&\, \,\, = 
[(\psi_\ell^{(d_s)})^{Op}\int \hat \psi (z) \varphi  (\pi_{\Gamma-z}^{-1}(\cdot))   \mathrm{d} z ](x)=
\int_{\real^n}\hat \psi(z) (\psi_\ell^{(d_s)})^{Op} 
\bigl[\varphi \circ \pi_{\Gamma -z}^{-1} \bigr] ( x ) \mathrm{d}z\, .
\end{align*}
Since $\| \int_{\real^n} \hat \psi(z) \Phi_z(\cdot) \mathrm{d} z \|_{L_p(\real^{d_s})}
\le \|\hat \psi\|_{L_1(\real^d)} \sup_z \| \Phi_z(\cdot)\|_{L_p(\real^{d_s})}$ by the Minkowski integral inequality,
we find,
\begin{equation*}
\left\| (\psi_\ell^{(d_s)})^{Op}\bigl[(\varphi \ast \hat \psi) \circ \pi_{\Gamma}^{-1}\bigr] \right\|_{L_p(\real^{d_s})} 
\leqslant \|\hat \psi\|_{L_1(\real^d)} 
\sup_{\tilde{\Gamma} \in \mathcal{F} } 
\bigl\| (\psi_\ell^{(d_s)})^{Op} [\varphi \circ \pi_{\tilde{\Gamma}}^{-1} ] \bigr\|_{L_p(\real^{d_s})} \, .
\end{equation*} 

\section{Proof of Theorem 4.1}

To replace $\|f \circ F^{-1}\|_{C^{r-1}(F(\Gamma))}$
by $C\sup|f|$ in (42), one uses that the bound (43) also holds for 
$f_m(x)=\prod_{j=0}^{m-1} (g \chi_{j,\epsilon} / |\det DT|) (T^{-j}(x))
$
where $\chi_{j,\epsilon}$ is the characteristic function of a $d$-dimensional ball 
of fixed radius $\epsilon$. (The partitions of unity adapted to $T^m$ are of the
form $\prod_{j=0}^{m-1} \chi_{j,\epsilon} \circ T^{-j}$.)

The last sentence in the proof of Theorem 4.1 on p.~543 must be shortened
 to: ``We just mention here that, in the present case, the ``fragmentation lemma'' (used 
to expand along a partition of unity) is just the triangle inequality,
while the ``reconstitution lemma'' (used to regroup the terms from a partition
of unity)
 is the trivial inequality
$\sum| a_k e_k| \le (\sum|a_k|) \sup |e_k|$.''
Footnote 18. on the same page must be suppressed.


\iffalse
More precisely,  set $\tilde g=|g|$
if $\inf |g|>0$, and otherwise choose a $C^{r-1}$ function
 $\tilde g$ with $\tilde g(x)>|g(x)|$. Then,
 letting $\Omega_m$,  $\theta^{(m)}_{\overrightarrow{\eta}}$,
 and $E_{\overrightarrow{\eta}}$  be as in the proof of Thm 5.1 in [2, Chap 5], we must estimate
\begin{align}
\label{tobound}
\sum_{\overrightarrow{\eta} \in  \Omega_m}
 \,\,
\sum_{\omega'}
\|(\theta_\omega\theta^{(m)}_{\overrightarrow{\eta}}  g^{(m)}) \circ  \kappa_\omega^{-1} \cdot 
\bigl ( ( \theta_ {\omega'} \varphi)\circ
\kappa_{\omega'}^{-1} \circ 
T^{-m}_{\omega \omega'}|_{\kappa_\omega(E_{\overrightarrow{\eta}})} \bigr )\|_{\UU^{t,s}_1} \, .
\end{align}
By the improvements discussed above of the Lasota-Yorke Lemma~4.2,
setting 
\begin{equation*}
G^{(m)}(x)= |\tilde g^{(m)}(x)| |\det (DT^m|_{E^s})|(x) \, 
\mbox{ 
and }
\lambda^{(m)}(x)=  \lambda^{(t,s, m)}(x)\, 
\end{equation*}
\eqref{tobound} is bounded by the sum of
\begin{align}
\label{bdded}
C \sum_{\overrightarrow{\eta} \in  \Omega_m}
 \,\, \biggl (
\sup_{x \in E_{\overrightarrow{\eta}}} (|G^{(m)}(x)| \lambda^{(m)}(x))
\cdot  \sum_{\omega'}
\| ( \theta_ {\omega'} \varphi)\circ
\kappa_{\omega'}^{-1} \|_{\UU^{t,s}_1} \biggr )
\end{align}
with a compact term.
The above term is bounded by
\begin{align*}
C \biggl ( \sum_{\overrightarrow{\eta} \in  \Omega_m}
 \,
\sup_{E_{\overrightarrow{\eta}} }(|G^{(m)}(x)| \lambda^{(m)}(x))\biggr )
\cdot \sum_{\omega'}
  \| ( \theta_ {\omega'} \varphi)\circ
\kappa_{\omega'}^{-1} \|_{\BB^{t,s}} \, .
\end{align*}
Finally, we obtain
\begin{align}\label{eqn:MbQ'v}
\|\LL^m_g \varphi - \LL^{m}_{c}\varphi\|_{\BB^{t,s}}
&\le  C \cdot
\biggl ( \sum_{\overrightarrow{\eta} \in  \Omega_m}\, 
\sup_{E_{\overrightarrow{\eta}}} |G^{(m)}(x)| \lambda^{(t,s,m)}(x)
\biggr )
 \| \varphi\|_{\UU_1^{t,s}}\,  , \qquad 
\end{align}
where $\LL^m_{c}$ is a compact operator on $\UU^{t,s}_1$. 
To finish, apply Lemmas B.3 and B.6 from [2].
\fi

\section{Details for   (40) in Sublemma 4.4 and correcting (29)}

We explain how to   bound $\| f(\varphi \circ F)\|^s_{p,\Gamma}$, by duality,
giving the proof of  (40) on p. 543:
 Since $B^s_{p,\infty}$ is the dual of $b^{|s|}_{p',1}$ (with $1/p'=1-1/p$),   setting
  $F_\Gamma= \pi_{F(\Gamma)} \circ F\circ \pi_\Gamma^{-1}$, it suffices
 to estimate $\| ((fh) \circ F_\Gamma^{-1}) |\det( DF_\Gamma)^{-1}|\|_{B^{|s|}_{p',1}(\real^{d_s})}$
  for  $C^\infty$ functions $h$.
First note that $\| ((fh) \circ F_\Gamma^{-1}) |\det( DF_\Gamma)^{-1}|\|_{L_{p'}(\real^{d_s})}
\le \sup \frac{| f |}{|\det DF_\Gamma|^{1/p}}$.
 The $B^{|s|}_{p',1}(\real^{d_s})$ norm of $v$ is equivalent to\footnote{See e.g. \cite[\S 2.1]{RS}, with $\lceil x\rceil$  the smallest
 integer which is $\ge x$.}
 $$
 \|v\|_{W^{ \lceil |s|-1\rceil}_{p'}(\real^{d_s})}+
 \sum_{|\alpha| = \lceil |s|-1\rceil} \int_{\real^{d_s}} \frac{1}{|y|^{d_s} }
\frac{ \|Z(D^\alpha v,\cdot, y) \|_{L_{p'}(\real^{d_s})} }{|y|^{ |s|- \lceil |s|-1 \rceil}} \D y \, ,
 $$
 where $\|v\|_{W^k_{p'}}= \sum_{0\le |\alpha| \le k} \|D^\alpha v\|_{L_{p'}}$ and $Z(w,x,y)= w(x+y)+w(x-y)-2 w(x)$.
Thus,  
 since $\inf |DF_\Gamma|\ge \|F\|_-\ge 1$, 
and using 
the``Zygmund derivation'' in\footnote{See also \cite[(2.6)--(2.8)]{BJL}, writing
$|f_i'|_\delta=|f_i'||f_i'|_\delta/|f_i'|$ in \cite[(2.5)]{BJL}, and noting that
($|(F_\Gamma^{-1})'|_\delta= |F_\Gamma'|_\delta/|F_\Gamma'|^2$ so that $|(F_\Gamma^{-1})'|_\delta/|(F_\Gamma^{-1})'|=|F_\Gamma'|_\delta/|F_\Gamma'|$.} \cite[\S 2]{BJL} 
\begin{align*}
Z(f h, x,y)
&=f(x) Z(h, x,y)
+ h(x) Z(f, x,y)\\
&\qquad +  \cdot\Delta_+ (f,x,y) \Delta_+(h,x,y)
+ \cdot \Delta_-(f,x,y)\Delta_-(h,x,y) \, ,
\end{align*}
where  
$\Delta_+(\upsilon,x,y)=(\upsilon(x+y)-\upsilon(x))$
and $\Delta_-(\upsilon,x,y)=(\upsilon(x)-\upsilon(x-y))$,
%for the second term,
and recalling that for any noninteger $\sigma>0$
   \cite[Prop 2.1.2, Prop 2.2.1]{RS}
$$\|v\|_{\WW^{\sigma}_{p',p'}}\le C (p',\sigma) \|v\|_{B^{\sigma}_{p',p'}}\le  C^2 (p',|s|) \|v\|_{B^{\sigma}_{p',1}}\, ,
$$
(with $\WW^{\sigma}_{p',p'}$ the Slobodeckij norm),
we find   for any $\epsilon>0$ constants $C(\FF,)$ and $C(\FF,\epsilon)$
so that
 \begin{align*}
(\widetilde {40})\,\, 
&\| ((fh) \circ F_\Gamma^{-1}) |\det( DF_\Gamma)^{-1}|\|_{B^{|s|}_{p',1}(\real^{d_s})}\\
 &\,\, \le 
C(\FF)\sum_{j=0}^{\lceil |s|-1\rceil}
\sum_{\ell=0}^j
\frac{1}{\|F\|_-^\ell} \frac{1} {|\det DF_\Gamma|^{1/p}} \|h\|_{W^{\ell}_{p'}}\\
&\qquad\times
\sum_{i=0}^{j-\ell} \| f \circ F_\Gamma^{-1}\|_{C^i} 
\||\det (DF_\Gamma ^{-1})|\|_{C^{j-\ell-i}}
\|DF_\Gamma^{-1}\|_{C^{j-\ell-i}} \\
&\quad
+C(\FF)\sum_{\ell=1}^{\lceil |s|-1 \rceil}
\frac{1}{\|F\|_-^\ell} \frac{1} {|\det DF_\Gamma|^{1/p}} \|h\|_{B^{\ell}_{p',1}}\\
&\qquad\times
\sum_{i=0}^{\lceil |s|-1 \rceil-\ell} \| f \circ F_\Gamma^{-1}\|_{C^i} 
\||\det (DF_\Gamma ^{-1})|\|_{C^{\lceil |s|-1 \rceil-\ell-i}}
\|DF_\Gamma^{-1}\|_{C^{\lceil |s|-1 \rceil-\ell-i}}\\
&\quad
+C(\FF, \epsilon)\frac{\|DF_\Gamma^{-1}\|_{C^{\epsilon}}}{\|F\|_-^{|s|-2\epsilon}} \frac{\||\det (DF_\Gamma^{-1})|\|_{C^{\epsilon}}} {|\det DF_\Gamma|^{1/p}} \|h\|_{B^{|s|-\epsilon}_{p',1}}
%\\
%&\qquad\times
\| f \circ F_\Gamma^{-1}\|_{C^\epsilon} 
\\
&\quad
+C(\FF, \epsilon)\frac{1}{\|F\|_-^{|s|}} \frac{\sup|f|} {|\det DF_\Gamma|^{1/p}} \|h\|_{B^{|s|}_{p',1}} \, .
\end{align*}


Finally, using $\inf | \det (DF |_{(C'_+)^\perp})| \ge C  | \det (D (F |_{\Gamma})|$,  we get
 (40), up to slightly amending (29) as follows ($d_s$ and $p$ are fixed):
\begin{equation*}
\mbox{(29*)} \, \,
C(F, \Gamma, s)= 
C'(\FF)|s| \|D(F|_\Gamma)^{-1}\|_{C^{r-1}} \||\det (DF_\Gamma^{-1})|\|_{C^{r-1}}\, .
\end{equation*}
Similarly,   in (42) one should replace $\sup_\Gamma \|f \circ F^{-1}\|_{C^{r-1}(F(\Gamma))}$ 
by 
$$\sup_\Gamma \|f \circ F^{-1}\|_{C^{r-1}(F(\Gamma))} \||\det (D (F_{(C'_+)^\perp})^{-1})|\|_{C^{r-1}(F(\Gamma))}\, .
$$




\iffalse
(An alternative proof of the bound  can be obtained by exploiting
the decomposition into  compact plus bounded
operators from \cite{BT, BT1}: 
in particular the proof\footnote{See also  \cite[Lemma 7]{NS}
 using \cite[(15), comment after (19)]{BT}. This also works for $s<0$.} of \cite[(6.11)]{BT1}.)
\fi


\section{Bounding $\|H^{\ell, \tau}_{n,\sigma}(v)\|^s_{p,\Gamma}$ (proof of Lemma 4.2) --- Correcting (31)}

Since $\|\cdot \|^s_{p,\Gamma}$ is not an $L_p$ norm, (38) does not suffice
to
deduce from (54) a bound   on $\|H^{\ell, \tau}_{n,\sigma}(v)\|^s_{p,\Gamma}$.
For any compact $K\subset \real^d$ and any
$\delta >0$, there exists  $C_0\ge 2$ so that for  all  $C_0'\ge C_0$ 
there exists $\widetilde C_0$ so that for all
$v$ supported in $K$, 
%all   $\widetilde \Gamma$ and  $\ell$, 
% using $(\psi_\ell^{Op} v) \circ \pi_{\widetilde \Gamma}^{-1}= \sum_{j=0}^\infty (\psi_{j}^{d_s})^{Op}((\psi_\ell^{Op} v) \circ \pi_{\widetilde \Gamma}^{-1} )$,
\begin{align*}
 \| (\psi_\ell^{Op} v) \circ& \pi_{\widetilde \Gamma}^{-1} \|_{L_p(\real^{d_s})} \le
\widetilde C_0 2^{\ell(-s+\delta)} \sum_{j=0}^{\ell+[C_0']} 2^{j (s-\delta)}
\| (\psi_{j}^{d_s})^{Op}((\psi_\ell^{Op} v) \circ \pi_{\widetilde \Gamma}^{-1} )\|_{L_p(\real^{d_s})}\\
&\,\, + C_0
\sum_{j=\ell+[C_0']+1}^{\infty} 2^{-j r} \sum_{m=\ell-2}^{\ell+2} \sup_{\widehat \Gamma}
\| (\psi_{m}^{Op} v)\circ \pi_{\widehat \Gamma}^{-1} \|_{L_p(\real^{d_s})}  \, ,
\, \forall \ell\, , \forall \widetilde \Gamma \, .
\end{align*}
(This  is clear if $\widetilde \Gamma$ is\footnote{The second line is then not needed.} affine, otherwise, 
proceed as in  \cite[Lemma 3.5]{Ba}, using the $L_p$ version of the leafwise Young inequality
\cite[Lemma 4.2]{BT2},
to obtain the above estimate in the sum over $j > \ell+C_0'$, 
after decomposing 
$v= \sum_m \psi_m^{Op}v$ and using almost orthogonality.)
Therefore, since $|s|< r$ and $a \le b + \epsilon a$ implies
$a \le (1-\epsilon)^{-1}b$ if $a>0$, $b>0$, and $\epsilon <1$, for each $\delta>0$ there is $C$  so that
\begin{equation*}
(0*)\, \, \, \,
2^{\ell t} \| (\psi_\ell^{Op} v) \circ \pi_{\widetilde \Gamma}^{-1} \|_{L_p(\real^{d_s})} 
\le C 2^{\ell(-s+\delta)} \|v\|_{\UU_p^{t,s}} \quad
\forall \widetilde \Gamma\, , \, \forall \ell\, .
\end{equation*}
Then, applying  $\| \phi\|_{B^s_{p,\infty}} \le C \| \phi\|_{L_p}$
to $\phi= (H_{n,\sigma} ^{\ell ,\tau} (\sum_{i=-2}^2\psi_{\ell+i}^{Op} v))\circ \pi_{\Gamma}^{-1}$, 
and using (54) and the $L_p$ version of \cite[Lemma 4.2]{BT2}, one obtains
$$
2^{\ell t}\|H^{\ell, \tau}_{n,\sigma}( v)\|^s_{p,\Gamma}=2^{\ell t}\|H^{\ell, \tau}_{n,\sigma}(\sum_{i=-2}^2\psi_{\ell+i}^{Op} v)\|^s_{p,\Gamma}
\le C_{F,f} 2^{-(r-1)\max\{n ,\ell\}} 2^{(-s+\delta) \ell}  \| v\|_{\UU_p^{t,s}}\, .
$$
This replaces the stronger bound  stated two lines above (52) on p.~546 and gives
the following weakening of (31):
\begin{equation*}
\mbox{(31*)} \, \, \, \,  \|  (\phi - \RR_{n_0}) \MM_c \varphi \|_{\UU_p^{\cone'_\pm,t,s}}
\le  C_{F,f,\delta} 2^{-(r-1-2\delta-(t-s))n_0} \, .
\end{equation*} 
Therefore, one must replace
$-(r-1)<s<-t<0$ by  $t-(r-1)<s<-t<0$ in Lemma~4.2,
and thus in\footnote{The condition on $s$ and $t$ was not explicited in Theorem 4.1.} Theorem 4.1. 

\iffalse
Probably cannot be improved since the following argument gives the same factor:
To avoid this issue, one can replace the Besov space $B^s_p(\real^{d_s})$
by the Sobolev space $H^s_p(\real^{d_s})$ and use that, by duality, 
$\|\int_{\real^d} V_{n,\sigma}^{\ell,\tau}(x,y)\varphi(y) \mathrm{d} y\|_{H^s_p}$ coincides with
$$ \sup_{\|h \|_{ H^{|s|}_{p/(p-1)}}\le 1} 
\int_{\real^{d_s}} h(\pi_\Gamma^{-1} x) \int _{\real^d} V_{n,\sigma}^{\ell,\tau}(\pi_\Gamma^{-1}(x),y)\varphi(y) \mathrm{d} y \mathrm{d} x\, .
$$
Integrating  by parts $|s|$ times, first $V_{n,\sigma}^{\ell,\tau}(x,y)$ w.r. to $x$, giving $C 2^{ns}$,
and  then $\varphi$ w.r. to $y$ giving $C 2^{|s| \ell}$,
one should get $2^{|s|(\ell-n)} 2^{-(r-1)\max\{n,\ell\}}\le 2^{-(r-1+s)n}$, which is not better than (31*).
\fi


\section{Fixing the end of the proof of Sublemma 4.4}\label{five}

The formulas for some kernels in the proof
of Sublemma~4.4 (p.~548) are garbled. The corrections are detailed below.
The statement of the sublemma is unchanged.



Lines 9--14 and the footnote of p.~548  must be replaced by
``Recalling the  functions $b_m$ from (53), we claim that
 there exists a constant $C_0 >1$ depending only
on $C_\FF$ and $\cone_\pm$ 
so that, for any $\Gamma\in \FF(\cone_+)$ and all
$n$, $n_s$, the 
kernels $V^{n,-}_{n_s, \Gamma}(w,y)$ defined
for  $w \in \Gamma$ and $y\in \real^d$  by
\begin{align*}
\nonumber 
\int_{\real^d}  \FFF^{-1}(\psi_{\Theta',n,-})(-x)&(\phi \cdot \psi^{Op(\Gamma+x)}_{n_s} \tilde \varphi)(w+x)\D x
= \frac{1}{(2\pi)^{d+d_s}}\int_{\real^{d}} V^{n,-}_{n_s, \Gamma}(w,y)
\tilde \varphi(y) \, \D y
\end{align*}
satisfy,\footnote[21]{For the kernels $V^{n,+}_{n_s, \Gamma+x}(w,y)$ defined by replacing
$\psi_{\Theta',n,-}$ with $\psi_{\Theta,n,+}$, we only get $C_0>1$ so that
$| V^{n,+}_{n_s, \Gamma}(w,y) |\le C_0 2^{-(r-1)n}b_{n_s}(w-y)$ if  $C_0 2^{n_s} \ge  2^n$. In particular,  $V^{n,+}_{n_s, \Gamma}$ 
need not be small  if $n$ is big and  $n_s$ small.} 
$(60)\quad
|  V^{n,-}_{n_s, \Gamma}(w,y)  |
\le C_0 2^{-(r-1)n} b_{n_s}(w-y)
\mbox{ if }   C_0 2^{n_s} \le   2^n
\mbox{ or }   2^{n_s} \ge  C_0 2^n$.''

Replace lines 15--19 of p.~548 by:
``To prove (60), recall (16) and note that 
 \begin{align*}
V^{n,-}_{n_s, \Gamma}(w,y)=&\int_{\eta\in \real^{d},\,  x_-, \eta_s\in \real^{d_s}}
\E^{-\I x(y,x_-)\eta}  
\E^{\I( \pi_{\Gamma+x(y,x_-)}(w+x(y,x_-))-
z(y, x_-)) \eta_s}
\\
&\qquad  \times
\frac{ \phi(w+x(y,x_-))}{|\det D \YY_{\Gamma, x_-} (\YY_{\Gamma, x_-}^{-1}(y))|}
 \psi_{n_s}^{(d_s)}(\eta_s) \psi_{\Theta', n,-}(\eta)
   \D \eta \D \eta_s  \D x_- \, ,
\end{align*} 
using for each $x_-\in \real^{d_s}$  the $C^r$ change of variable $y=\YY_{\Gamma, x_-}(z,x_+):= \pi_{\Gamma+(x_-,x_+)}^{-1}(z)$
in $\real^d$, with $z\in \real^{d_s}$ and $x_+\in \real^ {d_u}$, 
setting also 
$$
x(y,x_-)= (x_-, \Pi_+(\YY^{-1}_{\Gamma, x_-}(y))) \, , \quad
z(y,x_-)= \pi_{\Gamma+x(y,x_-)} (y)\, ,
$$
where $\Pi_+:\real^{d_s+d_u}\to \real^{d_u}$ is defined by $\Pi_+(x_-, x_+)=x_+$.
Next just like in [10, 11]
(see also [Lemma 2.34, 2]), using that $\pi_{\Gamma+x}(w+x)=\pi_\Gamma(w)+x_-$
if $x=(x_-,x_+)\in \real^{d_s}\times \real^{d_u}$ and that $\Gamma \in \FF$, first
integrate by parts (see Appendix~3)
$(r-1)$ times
with respect to $x_-\in \real^{d_s}$ in the
formula for $V^{n,-}_{n_s, \Gamma}(w,y)$, and second, noticing that
$\|y-w\|> \epsilon$ implies that either 
$\|\pi_{\Gamma}(w)-\pi_{\Gamma+x(y,x_-)}(y)\|> \epsilon/(2C_0)$
or $\|\Pi_+(\YY^{-1}_{\Gamma, x_-}(y))\|>\epsilon/(2C_0)$,  integrate
by parts with respect the other variables as many times as necessary. It is an  enlightening exercise to prove (60) for affine
$\Gamma$.''




There is a minor typo in the left-hand side of (64) on p.~549, which should read:
\begin{equation*}
(64)\quad \|\int \FFF^{-1}(\psi_{\Theta',n,-})(-x)\cdot (\RR_{\tilde n_s, \Gamma+x} )(\tilde \varphi)(\cdot+x) \D x  \|^s_{p,\Gamma}
\le \sup_x C_1 2^{-(r-1)m_0} \|\tilde \varphi\|^s_{p, \Gamma+x} \, .
\end{equation*}

Finally,  (64) follows from (60) and the leafwise
 Young inequality (38).$\qed$


\section{Typos}

On p.~537, the condition $\real^{d_s}\times \{0\} \subset \cone_-$ must be replaced by 
``$\real^{d_s}\times \{0\}$ is included in $(\real^{d}\setminus \cone_+)\cup\{0\}$'' (thrice, including Defs 3.2--3.3).  Also, the
assumptions ensure that 
$\Pi_\Gamma$ is surjective. Same page, 6 lines after (17), the norms are equivalent uniformly in $\Gamma$, not equal, due to the
Jacobian. 
In Lemma 4.2, one must assume that $F$ can be extended by a bilipschitz regular cone hyperbolic diffeomorphism $\tilde F$ of $\real^d$,
with $\|\tilde F\|_+$, $1/\|\tilde F\|_{-}$, $1/\|\tilde F\|_{--}$ and $1/|\det (D\tilde F|_{(\cone'_+)^\perp})|$ controlled by twice the corresponding constants
for $F$.
In  lines 2--3 of p.~555, the sum is over all $\ell \ge 0$, and in  line 2, one of
the $(1+\|x_-\|)^{Q_1}$ must  be replaced by $(1+\|x_+\|)^{Q_2}$, while $\psi^{Op}_\ell \upsilon(x)$
should be replaced by $\|\psi^{Op}_\ell \upsilon\|_{L_\infty}$.


 \begin{thebibliography}{BT1}
 
% \bibitem{Bbook} V. Baladi,  Dynamical Zeta Functions and Dynamical Determinants for Hyperbolic Maps, to appear
% Springer Ergenisse (2018).
%{\tt https://webusers.imj-prg.fr/$\sim$viviane.baladi/baladi-zeta2016.pdf}

 \bibitem{Baladijoel} V. Baladi, {\it The quest for the ultimate anisotropic Banach space,} J. Stat. Phys.  
\textbf{166} (2017) 525--557 (in honour of Ruelle and Sinai). DOI: 10.1007/s10955-016-1663-0.


\bibitem{Ba} V. Baladi, \emph{Characteristic functions as bounded multipliers on anisotropic spaces,}
arXiv:1704.00157.
% (2017).



\bibitem{BJL} V. Baladi, Y. Jiang, and O.E. Lanford,
{\it Transfer operators acting on Zygmund functions,}
Trans. Amer. Math. Soc. {\bf 348} (1996) 1599--1615.



%\bibitem{BT1} V. Baladi and M. Tsujii,
%\emph{Anisotropic H\"older and Sobolev spaces for hyperbolic diffeomorphisms,}  Ann. Inst. Fourier \textbf{57} (2007) 127--154.

%\bibitem{BT}  V. Baladi and M. Tsujii,
%\emph{Spectra of differentiable hyperbolic maps} in ``Traces in number theory, geometry and quantum fields", 
%S. Albeverio, M. Marcolli, S. Paycha (eds), 
%Aspects of Mathematics E38, pp. 1--21, Vieweg 
%Verlag 
%(2008).

\bibitem{BT2}
V. Baladi and M.  Tsujii,  {\it Dynamical determinants and spectrum for hyperbolic diffeomorphisms,}
In 
Probabilistic and Geometric Structures in Dynamics, pp. 29--68,
Contemp. Math., \textbf{469}, Amer. Math. Soc.,  Providence, RI (2008).

 
%\bibitem{NS} Y. Nakano and S. Sakamoto, \emph{Spectra of expanding maps on Besov spaces,}
%arXiv:1710.09673.
% (2017).

\bibitem{RS} T. Runst and W.  Sickel, 
Sobolev spaces of Fractional Order, Nemytskij
Operators, and Nonlinear Partial Differential Equations,
Walter de Gruyter \& Co., Berlin (1996).


%\bibitem{Trie} H. Triebel,  Theory of Function Spaces II, Birkh\"auser, Basel (1992).


 \end{thebibliography}
\end{document}
