\chapter{The noncommutative geometry of reductive $p$-adic groups}

\section{Periodic cyclic homology}
\label{sec:3.1}


The results of Chapters 1 and 2 enable us to compare the $K$-theory and
the periodic cyclic homology of reductive $p$-adic groups. Before 
proceeding we recall that $HP_*$ is continuous in certain situations.
Here and in the next results we will freely use the notations from Chapter 2.

\begin{thm}\label{thm:3.1}
\begin{description}
\item[a)] Suppose that $A = \lim_{i \to \infty} A_i$ is an inductive limit of algebras
and that there exists $N \in \mh N$ such that $HH_n (A_i ) = 0 \, \forall n > N , 
\forall i$. Then $HP_* (A) \cong \lim_{i \to \infty} HP_* (A_i )$.
\item[b)] Suppose that  $B = \lim_{i \to \infty} B_i$ is a strict inductive limit of nuclear
Fr\'echet algebras and that there exists $N \in \mh N$ such that 
$HH_n (B_i ,\hot ) = 0 \, \forall n > N , \forall i$. 
Then $HP_* (B ,\hot_{\mh C} ) \cong \lim_{i \to \infty} HP_* (B_i ,\hot)$.
\end{description}
\end{thm}
\emph{Proof.}
According to \cite[Theorem 1.93]{Mey3} the completed bornological tensor product
agrees with Grothendieck's completed inductive tensor product for strict inductive limits 
of nuclear Fr\'echet spaces. This identifies b) with \cite[Theorem 3]{BrPl1}.
Part a) is just the simpler algebraic version of this result, which can also be found in
\cite[Proposition 2.2]{Nis} $\qquad \Box$
\\[2mm]
\begin{thm}\label{thm:3.2}
Let $\mf s \in \Omega (G)$ be a Bernstein component and let $K_{\mf s} \in \mr{CO}(G)$
be as in Proposition \textup{\ref{prop:2.1}.b.}
\begin{description}
\item[a)] The Chern character for $\mc S (G,K_{\mf s} )^{\mf s}$ induces an isomorphism
\[
K_* (C_r (G)^{\mf s} ) \otimes_{\mh Z} \mh C \to HP_* (\mc S (G)^{\mf s} ,\hot_{\mh C} ) \,.
\]
\item[b)] The direct sum of these maps, over all $\mf s \in \Omega (G)$, 
is a natural isomorphism
\[
K_* (C_r (G)) \otimes_{\mh Z} \mh C \to HP_* (\mc S (G),\hot_{\mh C} ) \,.
\]
\end{description}
\end{thm}
\emph{Proof.}
By \eqref{eq:2.11} and Theorem \ref{thm:1.6} the Chern character
\begin{equation}\label{eq:3.24}
ch \otimes \mr{id}_{\mh C} : K_* \big( \mc S (G,K_{\mf s} )^{\mf s} \big) \otimes_{\mh Z} \mh C
\to HP_* \big( \mc S (G,K_{\mf s} )^{\mf s} \big)
\end{equation}
is an isomorphism. Furthermore there are natural isomorphisms
\begin{equation}\label{eq:3.1}
\begin{aligned}
K_* (C_r^* (G)) \, &\cong K_* \Big( \varinjlim_{\mf S} 
\bigoplus_{\mf s \in \mf S} C_r^* (G)^{\mf s} \Big) \\
& \cong \bigoplus_{\mf s \in \Omega (G)} K_* \left( C_r^* (G)^{\mf s} \right) \\
& \cong \bigoplus_{\mf s \in \Omega (G)} \varinjlim_{K \in \mr{CO} (G)} 
K_* \left( C_r^* (G,K)^{\mf s} \right) \\
& \cong \bigoplus_{\mf s \in \Omega (G)} 
K_* \big( C_r^* (G,K_{\mf s} )^{\mf s} \big)\\
&\cong \bigoplus_{\mf s \in \Omega (G)} 
K_* \big( \mc S (G,K_{\mf s} )^{\mf s} \big) \,.
\end{aligned}
\end{equation}
Here we used respectively \eqref{eq:2.13}, \eqref{eq:2.14},
Proposition \ref{prop:2.1}.d and that $K_*$ is invariant for passing 
to holomorphically closed dense Fr\'echet subalgebras. Similarly,
\begin{equation}\label{eq:3.2}
\begin{aligned}
HP_* (\mc S (G), \hot_{\mh C} ) \: & \cong 
\varinjlim_{K \in \mr{CO} (G)} HP_* (\mc S (G,K), \hot )\\
&\cong \bigoplus_{\mf s \in \Omega (G)} \varinjlim_{K \in \mr{CO} (G)}
HP_* (\mc S (G,K)^{\mf s}, \hot )\\
& \cong \bigoplus_{\mf s \in \Omega (G)} 
HP_* (\mc S (G)^{\mf s}, \hot_{\mh C} )\\
&\cong \bigoplus_{\mf s \in \Omega (G)} 
HP_* (\mc S (G, K_{\mf s})^{\mf s}, \hot ) \,.
\end{aligned}
\end{equation}
For the first and third isomorphisms we need Theorem \ref{thm:3.1}.b,
which we may indeed apply due to Corollary \ref{cor:2.3}.b. The second and
fourth isomorphisms rely on Proposition \ref{prop:2.1}.c. 

Finally we combine the last lines of  \eqref{eq:3.1} and \eqref{eq:3.2} with \eqref{eq:3.24}.
$\qquad \Box$ \\[2mm]

Altogether this theorem involves quite a few steps, but one is always
guided by the general principle that algebras with the same spectrum should 
have closely related invariants. 

Next we will prove the comparison theorem for the periodic cyclic homology
of reductive $p$-adic groups. This result was suggested in 
\cite[Conjecture 8.9]{BHP2} and in \cite[Section 4]{ABP}. It is more difficult than 
Theorem \ref{thm:3.2}, precisely because the above principle does not apply.
We remark that Nistor \cite[Theorem 4.2]{Nis} already gave a rather explicit 
description of $HP_* (\mc H (G))$.

\begin{thm}\label{thm:3.3}
The inclusions $\mc H (G)^{\mf s} \to \mc S (G)^{\mf s}$ for $\mf s \in \Omega (G)$
induce isomorphisms
\begin{align*}
HP_* ( \mc H (G)^{\mf s} ) &\isom HP_* ( \mc S (G)^{\mf s}, \hot_{\mh C} ) \,, \\
HP_* ( \mc H (G) ) &\isom HP_* ( \mc S (G),\hot_{\mh C} ) \,.
\end{align*}
\end{thm}
\emph{Proof.}
It follows from Theorem \ref{thm:3.1}.a, Corollary \ref{cor:2.3}.a and Proposition 
\ref{prop:2.1}.b that 
\begin{equation}\label{eq:3.4}
\begin{aligned}
HP_* (\mc H (G) ) \: &\cong 
\bigoplus_{\mf s \in \Omega (G)} 
HP_* (\mc H (G)^{\mf s} )\\
&\cong \bigoplus_{\mf s \in \Omega (G)} \varinjlim_{K \in \mr{CO}(G)}
HP_* (\mc H (G,K)^{\mf s} )\\
&\cong \bigoplus_{\mf s \in \Omega (G)} 
HP_* (\mc H (G, K_{\mf s})^{\mf s} )\\
\end{aligned}
\end{equation}
Together with \eqref{eq:3.2} this reduces the proof to showing that for every
Bernstein component $\mf s \in \Omega (G)$ the inclusion
\begin{equation}
\mc H (G,K_{\mf s} )^{\mf s} \to \mc S (G,K_{\mf s} )^{\mf s} 
\end{equation}
induces an isomorphism on \pch \!. Applying Lemma \ref{lem:1.3} to the filtrations
from Lemma \ref{lem:2.14}, this will follow from the next result.

\begin{prop}\label{prop:3.8}
Let $\mc H_i^{\mf s}$ and $\mc S_i^{\mf s}$ be as in \eqref{eq:2.22}. The inclusion map
\[
\mc H_{i-1}^{\mf s} / \mc H_i^{\mf s} \to \mc S_{i-1}^{\mf s} / \mc S_i^{\mf s}
\]
induces an isomorphism on \pch \!.
\end{prop}
\emph{Proof.}
We would like to copy the proof of Theorem \ref{thm:1.7} with $\Gamma = \mc W_i$ and
\[
\begin{array}{lll@{,\qquad}lll}
A\al & = & \mc H^{\mf s}_{i-1} / \mc H^{\mf s}_i & X & = & \xnr  (M_i ) \,, \\
A\sm & = & \mc S^{\mf s}_{i-1} / \mc S^{\mf s}_i & X' & = & \xunr  (M_i ) \,.
\end{array}
\]
By Lemmas \ref{lem:2.16} and \ref{lem:2.14} the primitive ideal spectra fit into the framework
of Section \ref{sec:1.4}. On the smooth side everything works fine, but on the algebraic
side we have to take into account that in general
\begin{itemize}
\item the intertwiners $u_w \; (w \in \mc W_i )$ are only rational on $\xnr (M_i )$,
\item the center of $\mc H^{\mf s}_{i-1} / \mc H^{\mf s}_i$ is ``too small", in the sense that
it does not contain all polynomial functions on $\xnr (M_i ) / \mc W_i$.
\end{itemize}
In fact every element of 
\[
Z \big( \mc H^{\mf s}_{i-1} / \mc H^{\mf s}_i \big) \subset \mc O (\xnr (M_i ) / \mc W_i )
\]
necessarily vanishes on the singularities of the intertwiners, because these points of 
$\xnr (M_i )$ carry representations from Irr$ (G )_{(P_j ,A_j ,\omega_j )}$ with $j < i$.

Let $\mc L$ be the collection of all the irreducible components of all the $X^H$, with $H$ 
running over all subsets of $\mc W_i$. We note that these are all of the form $\chi T$, 
with $\chi \in \xunr (M_i )$ and $T$ an algebraic subtorus of $\xnr (M_i )$. 
We write $T_u = T \cap \xunr (M_i)$. Let $\mc L_p$ be the subset of $\mc L$ consisting
of elements of dimension $\leq p$, and define the $\mc W_i$-stable subvarieties
\[
\begin{array}{lll}
X_p & := & \bigcup_{\chi T \in \mc L_p} \, \chi T \,, \\
X'_p & := & X_p \cap \xunr (M_i ) = 
\bigcup_{\chi T \in \mc L_p} \, \chi T_u \,.
\end{array}
\]
Let $V$ be the finite dimensional vector space $I^G_{P_i} (E_i )^{K_{\mf s}}$ and
consider the ideals 
\[
\begin{array}{lll}
I_p & := & \{ h \in \mc H^{\mf s}_{i-1} / \mc H^{\mf s}_i : 
I(P_i ,A_i ,\omega_i ,\chi )(h) = 0 \; \forall \chi \in X_p \} \,, \\
J_p & := & \{ h \in \mc S^{\mf s}_{i-1} / \mc S^{\mf s}_i : 
I(P_i ,A_i ,\omega_i ,\chi )(h) = 0 \; \forall \chi \in X'_p \} \,.
\end{array}
\]
It follows from \eqref{eq:2.23} that 
\[
\begin{array}{lll}
J_{p-1} / J_p & = & C_0^\infty (X'_p , X'_{p-1} ; \mr{End}(V) )^{\mc W_i} \,, \\
I_{p-1} / I_p & \subset & \mc O_0 (X_p , X_{p-1} ; \mr{End}(V) )^{\mc W_i} \,.
\end{array}
\]
The Jacobson topology makes Prim$ (A_{alg} / I_p )$ into a complex scheme,
which need not be separable. Let 
\[
\theta_p : \mr{Prim}(A_{alg} / I_p ) \to X_p / \mc W_i
\]
be the natural morphism. Take $C \in \mc L_p$ and consider
$Y_C = \theta_p^{-1} ((\mc W_i C) / \mc W_i)$. From Lemma \ref{lem:2.16} we see that 
$Y_C$ is a disjoint union of copies of $(\mc W_i C) / \mc W_i $, of which certain subvarieties 
are given a higher multiplicity. Moreover, the only thing that can cause inseparability on
Prim$ (A\al / I_p )$ is a jump in the isotropy group of $\mc W_i$ acting on $C$. By construction
of $X_p$ this implies that all inseparable points lie in $\theta_p^{-1} (X_{p-1} / \mc W_i )$, 
so the corresponding primitive ideals contain $I_{p-1} / I_p$. 

We define $Y_p$ as the (maximal) separable quotient of Prim$ (A\al / I_p )$. From the above 
description we see that it is a complex algebraic variety on which $\theta_p$ is well-defined. 
Let $Z_p$ be its closed subvariety corresponding to $X_{p-1} / \mc W_i$. Then
\[
\mr{Prim}(I_{p-1} /I_p ) = Y_p \setminus Z_p \,,
\]
and this would agree with \eqref{eq:1.20} if all the intertwiners were polynomial. We have
\[
\begin{array}{lllll}
Y'_p & := & Y_p \cap \theta_p^{-1} (X'_p / \mc W_i ) & = & 
\text{Hausdorff quotient of Prim} (A\sm / J_p ) \,, \\
Z'_p & := & Z_p \cap \theta_p^{-1} (X'_p / \mc W_i ) & = & 
\{ J/ \! \sim \, \in Y'_p : Z(J_{p-1} / J_p ) \subset J \} \,, \\
Y'_p \setminus Z'_p & = & \mr{Prim}(J_{p-1} /J_p ) & = & \mr{Prim} \big( Z(J_{p-1} /J_p ) \big) \,.
\end{array}
\]
With these notations the singularities of the $u_w$ form a closed subvariety of $Y_p$, 
disjoint from $Y'_p$. From page \pageref{eq:1.21} we know that there are natural isomorphisms
\begin{equation}
HP_* (J_{p-1} / J_p ) \leftarrow HP_* \big( Z(J_{p-1} / J_p ) \big) =
HP_* \big( \tilde C_0^\infty (Y'_p ,Z'_p) \big) \cong H^{[*]} (Y'_p ,Z'_p ) \,.
\end{equation}
For our comparison we will construct something similar on the algebraic side.

\begin{lem}\label{lem:3.9}
There exists a natural map
\[
HP_* (I_{p-1} /I_p ) \to HP_* (\mc O_0 (Y_p ,Z_p )) \,. 
\]
\end{lem}
\emph{Proof.}
The main problem is the absence of a natural algebra homomorphism between $I_{p-1} / I_p$ and 
$\mc O_0 (Y_p ,Z_p)$. As a substitute we will use the generalized
trace map on the periodic cyclic bicomplex.

Let $C$ be an irreducible component of $X_p$, which is not contained in $X_{p-1}$. Pick
any point $\chi \in C \setminus (C \cap X_{p-1})$ and define
\[
\mc W_C := \{ w \in \mc W : w \chi = \chi \} \,.
\]
This depends only on $C$, not on the choice of $\chi$. Since $C$ is of the form ``algebraic
subtorus translated by a unitary element", $C \cap \xunr (M_i )$ is a (nonempty) real form 
of $C$. According to Theorem \ref{thm:2.4} the intertwiners $I (w,\omega \otimes \chi )$ with 
$w \in \mc W_C$ are regular on $C \cap \xunr (M_i )$, so they are regular on a nonempty 
Zariski-open subset of $C$. But $I(w ,\omega \otimes \chi ) \in GL ( \mr{End}_{\mh C}(V))$ 
has finite order for all $\chi \in C$, so it is in fact regular on the whole of $C$. In particular, 
for every $\chi \in C$ there is a canonical decomposition of $V$ into isotypical projective 
$\mc W_C$-representations:
\begin{equation}\label{eq:3.6}
I(P_i ,A_i ,\omega_i ,\chi )^{K_{\mf s}} = V = V_1^\chi \oplus \cdots \oplus V_{n_C}^\chi \,.
\end{equation}
Since $\chi \mapsto I(w,\omega \otimes \chi)$ is polynomial on $C$, the type of $V$ as a 
projective $\mc W_C$-representation is independent of $\chi \in C$. Moreover the corresponding 
projections $p(C,n,\chi) \in \mr{End}_{\mh C} (V)$ are polynomial in $\chi$.

For $\chi \in C \cap (X'_p \setminus X'_{p-1} )$ Theorem \ref{thm:2.7} assures that
\[
\mr{End}_{\mc W_C} \big( I(P_i ,A_i ,\omega_i ,\chi )^{K_{\mf s}} \big) = 
\{ I(P_i ,A_i ,\omega_i ,\chi) (h) : h \in \mc S^{\mf s}_{i-1} / \mc S^{\mf s}_i \} \,,
\]
so the summands of \eqref{eq:3.6} are in bijection with the irreducible constituents of the
$\mc H^{\mf s}_{i-1} / \mc H^{\mf s}_i$-module $I(P_i ,A_i ,\omega_i ,\chi )^{K_{\mf s}}$.
Together with Lemma \ref{lem:2.16} this shows that for every 
$\chi \in C \cap (X_p \setminus X_{p-1})$ and every 
$\pi \in \mr{JH}_{A\al} \big( I(P_i ,A_i ,\omega_i ,\chi )^{K_{\mf s}} \big)$ there is
a unique direct summand of \eqref{eq:3.6} in which $\pi$ appears. We remark that  
$\mr{JH}_G ( I(P_i ,A_i ,\omega_i ,\chi ) )$ may be larger, but the 
additional elements do not belong to Irr$ (G )_{(P_i ,A_i ,\omega_i )}$.

For any $w \in \mc W_i$ and any $n \leq n_C$ the intertwiner $I(w,\omega \otimes \chi)$ maps
$\mr{End}_{\mh C}(V_n^\chi )$ to $\mr{End}_{\mh C}(V_{n'}^{w \chi} )$, for some isotypical
component $V_{n'}^{w \chi}$ of the projective $(w \mc W_C w^{-1})$-representation 
$I(P_i ,A_i ,\omega_i ,w \chi )^{K_{\mf s}}$. Write 
\[
\tilde Y_p := {\ts \bigsqcup_C} \{ 1,\ldots ,n_C \} \times C \quad , \quad 
\tilde Z_p := {\ts \bigsqcup_C} \{ 1,\ldots ,n_C \} \times (C \cap X_{p-1}) \,,
\]
and let $\mc W_i$ act on these spaces by sending $(n,C, \chi)$ to $(n',w C ,w \chi)$.
The above amounts to a bijection
\begin{equation}\label{eq:3.3}
\mr{Prim}(I_{p-1}/I_p ) = Y_p \setminus Z_p \to (\tilde Y_p \setminus \tilde Z_p ) / \mc W_i \,.
\end{equation}
Furthermore we can make $\mc W_i$ act on the algebra 
\[
B_p := \mc O_0 (\tilde Y_p ,\tilde Z_p ) \otimes \mr{End}(V)
\]
by the natural combination of the actions on $I_{p-1}/I_p$ and on $\tilde Y_p$. There is a
morphism of finite type $Z \big( \mc H (G,K_{\mf s})^{\mf s} \big)$-algebras
\begin{align*}
& \phi : I_{p-1}/I_p \to B_p^{\mc W_i} \,, \\
& \phi (h) (C,n,\chi) = p(C,n,\chi ) \, I(P_i ,A_i ,\omega_i ,\chi )(h) \, p(C,n,\chi )\,.
\end{align*}
By \eqref{eq:3.3} the central subalgebra
\[
Z(B_p )^{\mc W_i} = \mc O_0 (\tilde Y_p ,\tilde Z_p )^{\mc W_i} \subset B_p^{\mc W_i}
\]
satisfies
\[
\mr{Prim} \big( Z(B_p )^{\mc W_i} \big) = (\tilde Y_p \setminus \tilde Z_p ) / \mc W_i
\cong \mr{Prim}(I_{p-1}/I_p ) \,.
\]
Now we are in the right position to apply the generalized trace map. 
Recall \cite[Section 1.2]{Lod} that, for any algebra $B$, this is a collection of linear maps 
\[
\mr{tr}_m : (B \otimes \mr{End} (V) )^{\otimes m} \to B^{\otimes m}
\]
which together form chain maps on the standard complexes computing Hochschild
and (periodic) cyclic homology. The tracial property can be formulated as
\begin{equation}\label{eq:3.5}
\mr{tr}_n \circ \mr{Ad}(b)^{\otimes n} = \mr{tr}_n
\end{equation}
for all invertible elements $b$ in the multiplier algebra of $B$. The induced maps on 
homology are natural and inverse to the maps induced by the inclusion
\begin{equation}\label{eq:3.11}
B \to B \otimes \mr{End}(V) \;:\; b \to e b e
\end{equation}
with $e \in \mr{End}(V)$ an idempotent of rank one. Consider the composition 
\[
\mr{tr}_m \circ \phi^{\otimes m} : (I_{p-1}/I_p )^{\otimes m} \to Z(B_p )^{\otimes m} \,.
\]
In view of \eqref{eq:3.5} and \eqref{eq:2.7} the image is contained in 
$\big( ( Z(B_p ) )^{\otimes m} \big)^{\mc W_i}$. The periodic cyclic bicomplex 
\cite[Section 5.1]{Lod} with terms $\big( ( Z(B_p ) )^{\otimes m} \big)^{\mc W_i}$ 
has homology $HP_* (Z(B_p ))^{\mc W_i}$, because $\mc W_i$ is finite and 
acts by algebra automorphisms. Thus we obtain a natural map
\begin{equation}\label{eq:3.25}
\tau := HP_* (\mr{tr} \circ \phi ) : HP_* (I_{p-1}/I_p ) \to HP_* (Z(B_p ))^{\mc W_i} \,.
\end{equation}
By Theorem \ref{thm:1.4}.a there are natural isomorphisms
\begin{equation}\label{eq:3.28}
\begin{split}
HP_* (Z(B_p ))^{\mc W_i} & \leftarrow HP_* \big( Z(B_p )^{\mc W_i} \big) = 
HP_* \big(  \mc O_0 (\tilde Y_p ,\tilde Z_p )^{\mc W_i} \big) \\
& = HP_* (\mc O_0 (Y_p ,Z_p )) \cong H^{[*]}(Y_p ,Z_p) \,.
\end{split}
\end{equation}
The combination of \eqref{eq:3.25} and \eqref{eq:3.28} yields the required map. 
$\qquad \Box$ 
\\[2mm]
\begin{lem}\label{lem:3.10}
The map 
\[
HP_* (I_{p-1} /I_p ) \to HP_* (\mc O_0 (Y_p ,Z_p ))
\]
constructed in Lemma \textup{\ref{lem:3.9}} is an isomorphism.
\end{lem}
\emph{Proof.} We will prove the equivalent statement that \eqref{eq:3.25} is bijective.

For $q \in \mh N$, let $Y_{p,q} \subset \mr{Prim}(I_{p-1}/I_p)$ be the collection of all primitive
ideals for which the corresponding irreducible $I_{p-1}/I_p $-module has dimension $\leq q$. 
According to \cite[p. 328]{KNS} these are Jacobson-closed subsets, and they give rise to 
the standard filtration
\begin{equation}\label{eq:3.26}
\begin{split}
& I_{p-1}/I_p = L_0 \supset L_1 \supset \cdots \supset L_{\dim V} \,, \\
& L_q := {\ts \bigcap_{J \in Y_{p,q}} } J \,. \\
\end{split}
\end{equation}
Since every irreducible $I_{p-1}/I_p$-module is a subquotient of $V \,, L_{\dim V}$ is the 
Jacobson radical of $I_{p-1}/I_p$. We would like to apply the results of \cite{KNS} to 
\eqref{eq:3.26}, but this is not directly possible because $I_{p-1}/I_p$ does not have a unit. 
Therefore we consider the unital finite type algebra
\[
\begin{array}{lll}
A & := & Z \big( \mc H (G,K_{\mf s})^{\mf s} \big) \oplus M_2 (I_{p-1}/I_p ) \,, \\
(f \oplus h) (f' \oplus h') & := & (ff' \oplus fh' + f' h + hh' ) \,.
\end{array}
\]
Since $\mc H_i^{\mf s}$ and $\mc H_{i-1}^{\mf s}$ are ideals in $\mc H (G,K_{\mf s})^{\mf s}$ and
every $I_p \subset \mc H^{\mf s}_{i-1} / \mc H^{\mf s}_i$ is defined in terms of $\mc W_i$-stable
subvarieties of $\xnr (M_i )$, this multiplication is well-defined.
The standard filtration of $A$ is
\[
A \supset M_2 (L_0) \supset M_2 (L_0 ) \supset M_2 (L_1) \supset M_2 (L_1) \supset \cdots \supset 
M_2 (L_{\dim V}) \,.
\]
According to \cite[Proposition 1 and Theorem 10]{KNS}, applied to the algebra $A$ and its ideal
$M_2 (L_0 ) = M_2 (I_{p-1}/I_p )$, there are natural isomorphisms
\begin{equation}\label{eq:3.27}
HP_* (M_2 (L_{q-1}) / M_2 (L_q) ) \cong H^{[*]} (Y_{p,q} \cup Z_p ,Y_{p,q-1} \cup Z_p ) \,.
\end{equation}
Let $\tilde Y_{p,q}$ be the inverse image of $Y_{p,q}$ under the map $\tilde Y_p \setminus \tilde Z_p
\to Y_p \setminus Z_p$ from \eqref{eq:3.3}. To \eqref{eq:3.26} corresponds the filtration
\begin{align*}
& Z(B_p) = F_0 \supset F_1 \supset \cdots \supset F_{\dim V} = 0 \,, \\
& F_q := \mc O_0 (\tilde Y_p , \tilde Y_{p,q} \cup Z_p ) \,.
\end{align*}
By construction \eqref{eq:3.25} induces a map
\[
\tau_q : HP_* (L_{q-1}/L_q ) \to HP_* (F_{q-1}/F_q )^{\mc W_i} \,,
\]
which is natural with respect to morphisms of the underlying varieties.
By Theorem \ref{thm:1.4}.a the right hand side is isomorphic to
\begin{equation}\label{eq:3.29}
\begin{split} HP_* (\mc O_0 (\tilde Y_{p,q} \cup \tilde Z_p 
,\tilde Y_{p,q-1} \cup \tilde Z_p ) )^{\mc W_i} 
& \cong HP_* (\mc O_0 (Y_{p,q} \cup Z_p ,Y_{p,q-1} \cup Z_p )) \\
& \cong H^{[*]} (Y_{p,q} \cup Z_p ,Y_{p,q-1} \cup Z_p ) \,.
\end{split}
\end{equation}
Since $\tau_q$ is natural, \eqref{eq:3.27} and \eqref{eq:3.29} show that it is an isomorphism.
Although $\tau$ and $\tau_q$ are not induced by algebra homomorphisms, they do come from maps of
the appropriate periodic cyclic bicomplexes, which is enough to ensure that they are compatible
with the connecting maps from \eqref{eq:1.19}. Hence we can use a variation on Lemma \ref{lem:1.3}, 
where the role of $HP_* (\phi )$ is played by $\tau_*$.
This leads to the conclusion that \eqref{eq:3.25} is an isomorphism. $\qquad \Box$
\\[2mm]

We return to the proof of Proposition \ref{prop:3.8}.
Lemmas \ref{lem:3.9} and \ref{lem:3.10} allow us to write down the following diagram:
\[
\begin{array}{ccccc}
HP_* (I_{p-1}/I_p ) & \rightarrow & HP_* \big( \mc O_0 (Y_p ,Z_p ) \big) 
& \cong & H^{[*]} (Y_p , Z_p ) \\
\downarrow {\scs (1)} & & \downarrow {\scs (2)} & & \downarrow {\scs (3)} \\
HP_* (J_{p-1}/J_p ) & \leftarrow & HP_* \big( \tilde C_0^\infty (Y'_p , Z'_p ) \big) 
& \cong &H^{[*]} (Y'_p ,Z'_p ) 
\end{array}
\]
By Lemma \ref{lem:2.16} $(Y'_p ,Z'_p )$ is a deformation retract of $(Y_p ,Z_p )$, so (3) and (2)
are isomorphisms. Unfortunately the diagram does not commute, because unlike \eqref{eq:3.11} 
$\tilde C_0^\infty (Y'_p , Z'_p ) \to J_{p-1} / J_p$ is not a ``rank one" inclusion. However, 
all the vector spaces in the diagram decompose as direct sums over the components of 
$Y_p \setminus Z_p$, which were labelled $(C,n)$ on page \pageref{eq:3.3}. For every such 
component the diagram commutes up to a scalar factor, namely the dimension of the corresponding 
module $V^x_n$ from \eqref{eq:3.6}. Therefore the diagram does show that the arrow (1) is an 
isomorphism. This concludes the proofs of Proposition \ref{prop:3.8} and Theorem \ref{thm:3.3}.
$\qquad \Box$ \\[2mm]

We remark that a slightly simpler version of the above proof also works for 
affine Hecke algebras and their Schwartz completions, see \cite{Sol2}. 
\vspace{4mm}




\section{Example: $SL_2 (\mh Q_p )$}
\label{sec:3.2}

To clarify the proof of Theorem \ref{thm:3.3} we show in some detail what it
involves in the simplest case. This section is partly based on the related 
calculations in \cite{BHP1} and \cite[Section 6.1]{Sol3}.

Let $p$ be an odd prime, $\mh Q_p$ the field of
$p$-adic numbers, $\mh Z_p$ the ring of $p$-adic integers and $p \mh Z_p$
its unique maximal ideal. We consider the reductive group
$G = SL_2 (\mh Q_p )$ with the maximal torus
$A = \big\{ \matje{a}{0}{0}{a^{-1}} \,: a \in \mh Q^\times_p \big\}$
and the minimal parabolic subgroup
$P = \big\{ \matje{a}{b}{0}{a^{-1}} \,: a \in \mh Q^\times_p\,, b \in \mh Q_p \big\} $.
We have 
\begin{align*}
& M = Z_G (A)  = A \,, \\
& W = N_G (A) / Z_G (A) = \big( A \cup \matje{0}{-1}{1}{0} A \big) / A \,.
\end{align*}
The Iwahori subgroup is
$K = \big\{ \matje{a}{b}{c}{d} \,: a,b,d \in \mh Z_p \,, c \in p \mh Z_p \big\} $.
In this situation $\mc H (G,K)$ is Morita equivalent to $\mc H (G )^{\mf s}$,
where $\mf s = [M,\sigma ] \in \Omega (G)$ is the Borel component, 
corresponding to the trivial representation $(\sigma ,E)$ of $M$. According to
\cite{IwMa} $\mc H (G,K)$ is isomorphic to the Iwahori--Hecke algebra 
$\mc H (A_1 ,p)$ of type $A_1$ with parameter $p$. Furthermore $\mc S (G,K)$
is isomorphic to the Schwartz completion $\mc S (A_1 ,p)$ of $\mc H (A_1 ,p)$,
see \cite{DeOp1}.

We identify $\xnr (M)$ with $\mh C^\times$ by evaluation at $\matje{p}{0}{0}{p^{-1}}$ 
For almost all $\chi \in \xnr (M)$ the $G$-representation $I(P,A,\sigma ,\chi )$ 
is irreducible, so the separated quotient of Prim$( \mc H (G )^{\mf s}$ is 
\[
\xnr (M) / W \cong \mh C^\times / (z \sim z^{-1}) \,.
\]
The $K$-invariant part $I' (\chi)$ of $I(P,A,\sigma ,\chi )$ is a two-dimensional 
$\mc H (G,K)$-module with underlying vector space $V_\chi = V := I_P^G (E)^K$.
The intertwining operator $u_w (\chi )$ has rank one if $\chi^2 = p^{\pm 1}$
and is invertible for all other $\chi \in \xnr (M)$. 
More precisely the homomorphism
\[
I' (\chi ) : \mc H (G,K) \to \mr{End}_{\mh C} (V_\chi )
\]
is surjective for the generic $\chi$ and has image conjugate to
$\{ \matje{a}{b}{0}{d} \,: a,b,d \in \mh C \}$ for special $\chi$. Therefore Prim$
(\mc H (G,K))$ has only two pairs of nonseparated points, at
\begin{equation}
W \chi = p^{\pm 1/2} \quad \text{and} \quad W \chi = -p^{\pm 1/2} \,.
\end{equation}
The $\mc H (G,K)$-modules $V_\chi$ with $\chi \in \xunr (M)$
extend continuously to $\mc S (G,K)$-modules. Besides that, $\mc S (G,K)$
admits precisely two inequivalent one-dimensional square-integrable modules,
namely the irreducible submodules of $V_{p^{1/2}}$ and of $V_{-p^{1/2}}$. Hence 
Prim$ (\mc S(G,K))$ consists of two isolated points (say $\delta_+$ and 
$\delta_-$) and a copy of $\xunr (M) / W \cong [-1,1]$.
In the filtrations 
\begin{align*}
&\mc H (G, K_{\mf s})^{\mf s} = \mc H^{\mf s}_0 \supset \mc H^{\mf s}_1 
\supset \mc H^{\mf s}_2 = 0 \,, \\ 
&\mc S (G, K_{\mf s})^{\mf s} \, = \,\mc S^{\mf s}_0 \supset \, \mc S^{\mf s}_1  
\supset \, \mc S^{\mf s}_2 = 0 \,,
\end{align*}
we have
\begin{align*}
& \mc S_0^{\mf s} \cong C^\infty (\xnr (M))^W \otimes 
\mr{End}_{\mh C} (V) \; \oplus \; \mh C \; \oplus \; \mh C \,,\\
& \mc S_1^{\mf s} \cong C^\infty (\xnr (M))^W \otimes 
\mr{End}_{\mh C} (V) \,,\\
& \mc H_1^{\mf s} = \ker (\delta_+ ) \cap \ker (\delta_- )  \subset
\mc O \big( \xnr (M) ;\mr{End}_{\mh C} (V) \big)^W \,, \\
& \mc H_0^{\mf s} / \mc H_1^{\mf s} \cong \mc S_0^{\mf s} / \mc S_1^{\mf s}
\cong \mr{End}_{\mh C}(\delta_+ ) \oplus \mr{End}_{\mh C}(\delta_- ) \cong  
\mh C \oplus \mh C \,.
\end{align*}
The tricky step is to see that $HP_* (\mc H_1^{\mf s}) \cong 
HP_* (\mc S_1^{\mf s})$. Clearly
\begin{align*}
& \mr{Prim}(\mc S_1^{\mf s}) \cong \xunr (M) /W \,, \\
& Z (\mc S_1^{\mf s}) \cong  C^\infty (\xunr (M))^W \,,\\
& HP_* (\mc S_1^{\mf s}) \cong HP_* (Z(\mc S_1^{\mf s})) \cong
H^{[*]}( \xunr (M) / W) \cong \check H^* ([-1,1] ; \mh C) \,.
\end{align*}
However the image of 
\[
I' \big( \pm p^{1/2} \big) : \mc H_1^{\mf s} \to \mr{End}_{\mh C} ( V_{\pm p^{1/2}} )
\]
is not $M_2 (\mh C )$, but it is conjugate to 
$\{ \matje{0}{b}{0}{d} \,: b,d \in \mh C \}$. Therefore 
\[
Z (\mc H_1^{\mf s}) \cong \mc O_0 \big(\xnr (M) / W , 
\{ p^{\pm 1/2} ,-p^{\pm 1/2} \} \big) \,,
\]
even though Prim$ (\mc H_1^{\mf s}) \cong \xnr (M) / W$. 
Consider the diagram
\[
\begin{array}{ccccc}
\ker I' \big( p^{1/2} \big) \cap \ker I' \big( -p^{1/2} \big) & \to & \mc H_1^{\mf s} & 
\to & \mr{End} ( V_{p^{1/2}}) \oplus \mr{End} (V_{-p^{1/2}} ) \\
\uparrow & & \downarrow \mr{tr} & & \uparrow \\
\! \mc O_0 \big(\xnr (M) / W , W \{ \pm p^{1/2} \} \big) & \to &
\mc O \big(\xnr (M) / W \big) & \to & \mc O \big( \{ p^{1/2} ,-p^{1/2} \} \big) 
\end{array}
\]
The upward arrows identify the centers of the respective algebras. These
morphisms are spectrum preserving, so they induce isomorphisms on
periodic cyclic homology. The downward arrow is the (generalized) trace map, 
which induces a map
\[
HP_* (\mr{tr}) : HP_* (\mc H_1^{\mf s}) \to HP_* (\mc O (\xnr (M) ) )^W
\cong HP_* (\mc O (\xnr (M)/W)) \,.
\]
Now we apply the functor $HP_*$ to the entire diagram, and we replace the
downward arrow by $\frac{1}{2} HP_* (\mr {tr})$. The resulting diagram 
commutes and shows that $HP_* (\mr{tr})$ is a natural isomorphism.
From the commutative diagram 
\[
\begin{array}{ccccc}
HP_* (\mc H_1^{\mf s}) & \xrightarrow{\frac{1}{2} HP_* (\mr{tr})} & 
HP_* (\mc O (\xnr (M) / W )) & \cong & H^{[*]} (\mh C^\times / (z \sim z^{-1})) \\
\downarrow & & \downarrow & & \downarrow \\
HP_* (\mc S_1^{\mf s}) & \longleftarrow & HP_* (C^\infty (\xunr (M)/W )) &
\cong & H^{[*]} (S^1 / (z \sim z^{-1} ))
\end{array}
\]
we see that the left vertical arrow is indeed an isomorphism.
From this we can derive a natural isomorphism
\[
HP_* (\mc H (G)^{\mf s}) \cong HP_* (\mc H (G,K)) \to 
HP_* (\mc S (G,K), \hot) \cong HP_* (\mc S (G)^{\mf s} ,\hot) \,.
\]
We remark that the Borel component is the most complicated
Bernstein component of $SL_2 (\mh Q_p )$. Indeed with one exception 
every other $\mf s \in \Omega (SL_2 (\mh Q_p ))$ has a trivial Weyl group, 
and therefore Prim$ \big( \mc H (SL_2 (\mh Q_p ))^{\mf s} \big)$ is 
homeomorphic to either $\mh C^\times$ or a point. Moreover both 
$\mc H (SL_2 (\mh Q_p ))^{\mf s}$ and $\mc S (SL_2 (\mh Q_p ))^{\mf s}$
are Morita equivalent to commutative algebras for such $\mf s$.
\vspace{4mm}




\section{Equivariant cosheaf homology}
\label{sec:3.3}

The most interesting applications of the results of Section \ref{sec:3.1} 
lie in their connection with the Baum--Connes conjecture. 
To make this relation precise we need several additional homology theories.
In particular, our forthcoming discussing will require some detailed knowledge
of equivariant cosheaf homology. Therefore we first provide an overview of this 
theory, which is mostly taken from \cite{BCH,HiNi}.

Let $G$ be a totally disconnected group and $\Sigma$ a polysimplicial
complex. We assume that $\Sigma$ is equipped with a polysimplicial $G$-action,
which is proper in the sense that the isotropy group $G_\sigma$ of any 
polysimplex $\sigma$ is compact and open. Let $\Sigma^p$ denote the collection 
of $p$-dimensional polysimplices of $\Sigma$, endowed with the discrete topology. 
Define the vector space 
\[
C_p (G ; \Sigma ) := {\ts \bigoplus_{\sigma \in \Sigma^p} } \, C_c^\infty (G_\sigma )
\]
where $C^\infty (X)$ denotes the set of locally constant complex valued functions
on a totally disconnected space $X$. We write the elements of $C_p (G;\Sigma )$
as formal sums $\sum_\sigma f_\sigma [\sigma ]$. If $\tau$ is a face of $\sigma$
then $G_\tau \supset G_\sigma$, so we may consider $f_\sigma$ as a locally constant
function on $G_\tau$. On every polysimplex we fix an orientation, and we identify
$[\bar \sigma]$ with $-[\sigma]$, where $\bar \sigma$ means $\sigma$ with the
opposite orientation. We write the simplicial boundary operator as
\[
\delta \sigma = {\ts \sum_{\tau \in \Sigma^{p-1}} } [\sigma : \tau] \, \tau \qquad 
\text{with} \qquad [\sigma : \tau ] \in \{ -1,0,1 \} \,.
\]
This gives a differential
\begin{align*}
& \delta_p : C_p (G ; \Sigma) \to C_{p-1} (G;\Sigma ) \,, \\
& \delta_p \big( f_\sigma [\sigma] \big) = {\ts \sum_{\tau \in \Sigma^{p-1}} }
[\sigma : \tau] f_\sigma [\tau] \,.
\end{align*}
We endow the differential complex $( C_* (G;\Sigma ) ,\delta_* )$ with
the $G$-action 
\[
g \cdot f_\sigma [\sigma ] = f_\sigma^g [g \sigma] \,,
\]
where $f_\sigma^g \in C_c^\infty (G_{g \sigma})$ is defined by
$f_\sigma^g (h) = f_\sigma (g^{-1} h g)$ and $g \sigma$ is endowed with the
orientation coming from our chosen orientation on $\sigma$. Clearly $\delta_*$ is 
$G$-equivariant, so it is well-defined on the space $C_* (G;\Sigma )_G$ of 
$G$-coinvariants. The equivariant cosheaf homology of $\Sigma$ is 
\begin{equation}
CH_n^G (\Sigma ) := H_n ( C_* (G;\Sigma )_G ,\delta_* ) \,.
\end{equation}
There is also a relative version of this theory. Let $\Sigma'$ be a $G$-stable
subcomplex of $\Sigma$. We define the relative equivariant cosheaf homology of
$(\Sigma ,\Sigma')$ as
\begin{equation}
CH_n^G (\Sigma ,\Sigma' ) := 
H_n ( C_* (G;\Sigma )_G / C_* (G;\Sigma')_G ,\delta_* ) \,.
\end{equation}
As usual there is a long exact sequence in homology:
\begin{equation}\label{eq:3.22}
\cdots \to CH_{n+1}^G (\Sigma ,\Sigma') \to CH_n^G (\Sigma' ) 
\to CH^G_n (\Sigma ) \to CH_n^G (\Sigma ,\Sigma') \to \cdots
\end{equation}
If $G$ acts freely on $\Sigma$ then $CH^G_n (\Sigma ,\Sigma')$ reduces
to the usual simplicial homology $H_n (\Sigma /G ,\Sigma' /G) $
with complex coefficients.

Higson and Nistor \cite{HiNi} introduced a natural map
\begin{equation}\label{eq:3.37}
CH_n (\Sigma ) \to HH_n (C_c^\infty (G)) \,,
\end{equation}
whose construction we recall in as much detail as we need. Let 
\[
\prefix{}{_n}{\hat \Sigma^p} := \{ (g_0 ,g_1 ,\ldots, g_n ,\sigma ) \in
G^{n+1} \times \Sigma^p : g_0 g_1 \cdots g_n \sigma = \sigma \}
\]
be the $n$th Brylinski space of $\Sigma^p$. By definition $\Sigma^p$ is
discrete, so $\prefix{}{_n}{\hat \Sigma^p}$ is a totally disconnected
space and $C_c^\infty \big( \prefix{}{_n}{\hat \Sigma^p} \big)$ is
defined. According to \cite[Section 4]{HiNi} there is an exact sequence
\begin{equation}\label{eq:3.15}
0 \leftarrow C_p (G;\Sigma )_G \leftarrow C_c^\infty \big( 
\prefix{}{_0}{\hat \Sigma^p} \big) \leftarrow C_c^\infty \big( 
\prefix{}{_1}{\hat \Sigma^p} \big) \leftarrow C_c^\infty \big(
\prefix{}{_2}{\hat \Sigma^p} \big) \leftarrow \cdots
\end{equation}
Consequently $CH_*^G (\Sigma )$ can be computed as the homology of a
double complex
\begin{equation}\label{eq:3.8}
\begin{array}{cccccc}
\downarrow & & \downarrow & &\downarrow \\
C_c^\infty \big( \prefix{}{_2}{\hat \Sigma^0} \big) & \leftarrow & C_c^\infty \big(
\prefix{}{_2}{\hat \Sigma^1} \big) & \leftarrow & C_c^\infty \big(
\prefix{}{_2}{\hat \Sigma^2} \big) &  \leftarrow \\
\downarrow & & \downarrow & &\downarrow \\
C_c^\infty \big( \prefix{}{_1}{\hat \Sigma^0} \big) & \leftarrow & C_c^\infty \big(
\prefix{}{_1}{\hat \Sigma^1} \big) & \leftarrow & C_c^\infty \big(
\prefix{}{_1}{\hat \Sigma^2} \big) &  \leftarrow \\
\downarrow & & \downarrow & &\downarrow \\
C_c^\infty \big( \prefix{}{_0}{\hat \Sigma^0} \big) & \leftarrow & C_c^\infty \big(
\prefix{}{_0}{\hat \Sigma^1} \big) & \leftarrow & C_c^\infty \big(
\prefix{}{_0}{\hat \Sigma^2} \big) &  \leftarrow 
\end{array}
\end{equation}
In this diagram the horizontal maps come from the boundary map $\partial$ on
$\Sigma$, while the vertical maps are essentially the differentials for a 
Hochschild complex. There are natural maps 
\begin{equation}\label{eq:3.16}
\begin{array}{lll}
C_c^\infty \big( \prefix{}{_n}{\hat \Sigma^0} \big) & \to & C_c^\infty (G^{n+1}) \\
f & \to & \big( (g_0 ,\ldots ,g_n) \mapsto \sum_{x \in \Sigma^0} f (g_0 ,\ldots ,g_n, x) \big)
\end{array}
\end{equation}
from the double complex \eqref{eq:3.8} to the standard Hochschild complex for
$C_c^\infty (G)$. Together \eqref{eq:3.15}, \eqref{eq:3.8} and \eqref{eq:3.16} 
yield the map \eqref{eq:3.37}.

Suppose now that $G$ acts properly on some affine building. The Hochschild homology 
of $C_c^\infty (G)$ admits a decomposition 
\[
HH_n (C_c^\infty (G)) = HH_n (C_c^\infty (G))_{\mr{ell}} \oplus HH_n (C_c^\infty (G))_{\mr{hyp}}
\]
into an elliptic and an hyperbolic part. Upon periodization the hyperbolic part
disappears and one finds \cite[Section 7]{HiNi}
\begin{equation}\label{eq:3.17}
HP_n (C_c^\infty (G)) = {\ts \bigoplus_{m \in \mh Z} } HH_{n + 2m} (C_c^\infty (G))_{\mr{ell}} \,.
\end{equation}
This and \eqref{eq:3.37} yield a map 
\begin{equation}\label{eq:3.12}
\mu_{HN} : CH_{[*]}^G (\Sigma ) \to HP_* (C_c^\infty (G)) \,.
\end{equation}
Let $h : X \to \Sigma$ be any morphism of proper $G$-simplicial complexes. 
From the explicit formula \eqref{eq:3.16} we see that $\mu_{HN}$ for $X$ factors as
\begin{equation}\label{eq:3.38}
CH_n^G (X) \xrightarrow{CH_n^G (h)} CH_n^G (\Sigma ) \to HH_n (C_c^\infty (G)) 
\to HP_n (C_c^\infty (G))\,.
\end{equation}
Higson and Nistor \cite{HiNi} showed that \eqref{eq:3.12} is an isomorphism if 
$\Sigma$ is an affine building. The special case where $G$ is a simple $p$-adic 
group was also proved by Schneider \cite{Schn}. 
\vspace{4mm}




\section{The Baum--Connes conjecture}
\label{sec:3.4}

Let $G$ be any locally compact group acting properly on a Hausdorff space $\Sigma$. 
A subspace $X \subset \Sigma$ is called $G$-compact if $X / G$ is compact. 
The equivariant $K$-homology of $\Sigma$ is defined as
\begin{equation}\label{eq:3.18}
K_*^G (\Sigma ) = \varinjlim KK_*^G (C_0 (X), \mh C )
\end{equation}
where $KK_*^G$ is Kasparov's equivariant $KK$-theory \cite{Kas}
and the limit runs over all $G$-compact subspaces $X$ of $\Sigma$.
The Baum--Connes conjecture asserts that the assembly map
\begin{equation}\label{eq:3.7}
\mu : K_*^G (\Sigma) \to K_* (C_r^* (G))
\end{equation}
is an isomorphism if $\Sigma$ is a classifying space for proper $G$-actions.
Building upon the work of Kasparov, Vincent Lafforgue proved this conjecture 
for many groups, including all locally compact groups 
that act properly isometrically on an affine building \cite{Laf}.

Now we specialize to a reductive $p$-adic group $G$. In this case the affine 
Bruhat--Tits building $\beta G$ is a classifying space for proper $G$-actions. 
We recall that $\beta G$ is a finite dimensional locally finite polysimplicial 
complex endowed with an isometric $G$-action such that $\beta G / G$ 
is compact and contractible.
For any $X$ as above there exists a continuous $G$-map $h : X \to \beta G$,
and it is unique up to homotopy. The assembly map 
$\mu : K_*^G (X) \to K_* (C_r^* (G)$ factors as
\begin{equation}\label{eq:3.36}
K_*^G (X) \xrightarrow{K_*^G (h)} K_*^G (\beta G) \xrightarrow{\mu} K_* (C_r^* (G)) \,.
\end{equation}
A natural receptacle for a Chern character from $K_*^G (X)$ is formed by 
\[
HL_*^G (X) := HL_*^G (C_0 (X), \mh C)
\]
where $HL_*^G$ denotes equivariant local cyclic homology, as defined
and studied by Voigt \cite{Voi3}.
With these notions we can state and prove a more precise
version of \cite[Proposition 9.4]{BHP2}. We note that a similar idea
was already used in \cite{BHP1} to prove the Baum--Connes 
conjecture for $G = GL_n (\mh F )$.

\begin{thm}\label{thm:3.4}
There exists a commutative diagram 
\[
\begin{array}{ccc}
K_*^G (\beta G) & \xrightarrow{\quad \mu \quad} & K_* (C_r^* (G)) \\
\downarrow ch & & \downarrow ch \\
HL_*^G (\beta G) & & HP_* (\mc S (G),\hot_{\mh C} ) \\
\downarrow & & \uparrow \\
CH_{[*]}^G (\beta G) & \xrightarrow{\quad \mu' \quad} & HP_* (\mc H (G)) 
\end{array}
\]
with the properties:
\begin{description}
\item[a)] Both Chern characters become isomorphisms after 
applying $\otimes_{\mh Z} \mh C$ to their domain.
\item[b)] The other maps are natural isomorphisms.
\end{description}
\end{thm}
\emph{Proof.} As mentioned before, Lafforgue \cite{Laf} showed that the 
assembly map $\mu$ is an isomorphism. The right column is taken 
care of by Theorems \ref{thm:3.2} and \ref{thm:3.3}. 

Let $\Sigma$ be a finite dimensional, locally finite $G$-compact 
$G$-simplicial complex.
It was proved in \cite[Proposition 10.4]{Voi3} that the inclusion map
$C_c^\infty (\Sigma) \to C_0 (\Sigma)$ induces an isomorphism
\[
HL^G_* (\Sigma) =  HL_*^G (C_0 (\Sigma),\mh C ) \to
HL_*^G (C_c^\infty (\Sigma),\mh C ) \,.
\]
Let $HP_*^G$ denote Voigt's equivariant periodic cyclic homology \cite[Section 3]{Voi2}. 
According to \cite[Section 6]{Voi4} there are natural isomorphisms
\[
\begin{array}{rcl}
HL_*^G (C_c^\infty (\Sigma),\mh C ) & \cong & HP_*^G (C_c^\infty (\Sigma),\mh C ) \,, \\
ch \,:\, KK_*^G (C_0 (\Sigma) ,\mh C ) \otimes_{\mh Z} \mh C & \to & 
 HL_*^G (C_0 (\Sigma) ,\mh C ) \,.
\end{array}
\]
We have to check that
\begin{equation}\label{eq:3.9}
HP_*^G (C_c^\infty (\Sigma) ,\mh C ) \cong CH_{[*]}^G (\Sigma) \,.
\end{equation}
Baum and Schneider \cite[Section 1.B]{BaSc} showed that cosheaf homology 
can be regarded as a special case of a bivariant (co)homology theory:
\begin{equation}\label{eq:3.10}
CH_n^G (\Sigma) \cong H^n_G (\Sigma, \mr{point}) \,.
\end{equation}
According to \cite{Voi2} the right hand side of \eqref{eq:3.10} is naturally
isomorphic to the left hand side of \eqref{eq:3.9}, so we get natural isomorphisms
\begin{equation}\label{eq:3.31}
K_*^G (\Sigma ) \otimes_{\mh Z} \mh C \to HL_*^G (\Sigma ) \to CH_{[*]}^G (\Sigma ) \,.
\end{equation}
The case $\Sigma = \beta G$ gives us the left column of the theorem. 
To complete the proof we define 
\begin{equation}\label{eq:3.13}
\mu' : CH^G_{[*]} (\beta G) \to HP_* (\mc H (G))
\end{equation}
as the unique map so that the diagram commutes. $\qquad \Box$
\\[2mm]

It is not immediately clear that \eqref{eq:3.13} and \eqref{eq:3.12}, for
$C_c^\infty (G) = \mc H (G)$ and $\Sigma = \beta G$, are the same map. 
We will prove this by reduction to the following simpler case.
Let $U \in$ CO$ (G)$ and consider the discrete proper homogeneous
$G$-space  $G/U$. By the universal property of $\beta G$ there exists
a continuous $G$-equivariant map $G/U \to \beta G$, and it is unique
up to homotopy. With a suitable simplicial subdivision of $\beta G$ we can
achieve that this is in fact a simplicial $G$-map.

\begin{lem}\label{lem:3.5} 
The following diagram commutes for elements in the upper left corner.
\[
\begin{array}{ccccc}
K_*^G (G/U) & \to & K_*^G (\beta G) & \xrightarrow{\mu} & K_* (C_r^* (G)) \\
\downarrow & &\downarrow & &\downarrow \\
CH^G_{[*]} (G/U) & \to & CH^G_{[*]} (\beta G) & 
\xrightarrow{\mu_{HN}} & HP_* (\mc H (G))
\end{array}
\]
\end{lem}
\emph{Proof.} 
The left hand square commutes by functoriality. It follows readily from the 
definitions that $K_*^G (G/U) \cong K_*^U (\mr{point})$. By the 
functoriality of the Baum-Connes assembly map there is a commutative diagram
\begin{equation}\label{eq:3.20}
\begin{array}{ccc}
K_*^U (\mr{point}) &\xrightarrow{\mu_U} & K_* (C_r^* (U)) \\
\downarrow & & \downarrow \\
K_*^G (\beta G) & \xrightarrow{\mu} & K_* (C_r^* (G)) \,.
\end{array}
\end{equation}
Since $U$ is compact and totally disconnected, both $K_*^U (\mr{point})$ 
and $K_* (C_r^* (U))$ are naturally isomorphic to the ring of smooth (virtual) 
representations $R (U)$, and $\mu_U$ corresponds to the composition 
of these isomorphisms. The right vertical map comes from the inclusion 
$C_r^* (U) \to C_r^* (G)$, so it sends a $U$-module $V$ to $\mr{Ind}_U^G (V)$.\\ 
Similarly there are a canonical isomorphism 
\[
CH_*^G (G/U) \cong CH_*^U (\mr{point})
\]
and a commutative diagram
\begin{equation}\label{eq:3.21}
\begin{array}{ccc}
CH_{[*]}^U (\mr{point}) &\xrightarrow{\mu_{HN}} & HP_* (C_c^\infty (U)) \\
\downarrow & & \downarrow \\
CH_{[*]}^G (\beta G) & \xrightarrow{\mu_{HN}} & HP_* (C_c^\infty (G)) \,.
\end{array}
\end{equation}
According to \cite[Section 4]{HiNi} we have $HP_1 (C_c^\infty (U)) = 0$ and
\[
HP_0 (C_c^\infty (U)) = HH_0 (C_c^\infty (U)) = C_c^\infty (U)_U \,.
\]
By definition, also
\[
CH_n^U (\mr{point}) = \left\{ \begin{array}{lll}
C_c^\infty (U)_U & \mr{if} & n = 0 \\
0 & \mr{if} & n > 0 \,.
\end{array} \right.
\]
A glance at the double complex \eqref{eq:3.8} shows that 
\[
\mu_{HN} : CH_0^U (\mr{point}) \to HP_0 (C_c^\infty (U))
\]
corresponds to the identity map under these identifications.
Furthermore $U$ is profinite, so
\[
C_c^\infty (U) = \varinjlim \mh C [F] ,
\]
where the limit runs over all finite quotient groups $F$ of $U$. Similarly we can write 
$C_r^* (U)$ as an inductive limit in the category of $C^*$-algebras. In this situation
both $K_*$ and $HP_*$ commute with $\varinjlim$, so we get a Chern character
\begin{equation}\label{eq:3.33}
K_* (C_r^* (U)) \cong \varinjlim K_* (\mh C [F]) \to 
 \varinjlim HP_* (\mh C [F]) \cong HP_* (C_c^\infty (U)) \,.
\end{equation}
Now the right hand square of the diagram
\begin{equation}\label{eq:3.19}
\begin{array}{ccccc}
K_*^U (\mr{point}) & \xrightarrow{\mu} & K_* (C_r^* (U)) &
\to & K_* (C_r^* (G)) \\
\downarrow & & \downarrow & & \downarrow \\
CH_{[*]}^U (\mr{point}) & \xrightarrow{\mu_{HN}} & HP_* (C_c^\infty (U)) &
\to & HP_* (\mc H (G))
\end{array}
\end{equation}
commutes by functoriality.
According to Voigt \cite[Proposition 13.5]{Voi3} the Chern character
\[
K_*^U (\mr{point}) \to HL_*^U (\mr{point})
\]
can be identified with the character map $R (U) \to C_c^\infty (U)^U$.
The isomorphism between $HL_*^U$(point) and $CH_*^U$(point)
then becomes the canonical map
\[
C_c^\infty (U)^U \to C_c^\infty (U)_U ,
\]
which is bijective because $U$ is compact. Since the Chern character for $\mh C [F]$ 
in \eqref{eq:3.33} may also be identified with the character map, we find that the 
left hand square of \eqref{eq:3.19} commutes. Together the commutative diagrams 
\eqref{eq:3.20}, \eqref{eq:3.21} and \eqref{eq:3.19} complete the proof. $\qquad \Box$
\\[2mm]
\begin{lem}\label{lem:3.6}
The maps $\mu_{HN}$ from \eqref{eq:3.12} and $\mu'$ from \eqref{eq:3.13} are the same.
\end{lem}
\emph{Proof.}
We have to show that the diagram
\begin{equation}\label{eq:3.14}
\begin{array}{ccc}
K_*^G (\Sigma) \otimes_{\mh Z} \mh C & \xrightarrow{\mu} & 
K_* (C_r^* (G)) \otimes_{\mh Z} \mh C \\
\downarrow & &\downarrow \\
CH^G_{[*]} (\Sigma) & \xrightarrow{\mu_{HN}} & HP_* (\mc H (G))
\end{array}
\end{equation}
commutes for $\Sigma = \beta G$.
By subdividing all polysimplices we may assume that $\beta G$ is a 
simplicial complex, and that $G$ preserves this structure. Let $\beta^{(n)} G$ 
denote the $n$-skeleton of $\beta G$, and $(\beta G)^n$ the collection of $n$-simplices. 
Both inherit a $G$-action from $\beta G$.
We will prove the commutativity of \eqref{eq:3.14} for $\Sigma = \beta^{(n)}$, with 
induction to $n$. 

The set $\beta^{(0)} G$ is a finite union of $G$-spaces of the form $G/U$ with 
$U \in$ CO$ (G)$, so the case $n = 0$ follows from Lemma \ref{lem:3.5}.
Similarly \eqref{eq:3.14} commutes for $\Sigma = (\beta G)^n$.

We consider $(\beta G)^n \times S^n$ as a $G$-space with a trivial action on $S^n$. The
long exact sequence \eqref{eq:3.22} for the pair 
$\big( (\beta G)^n \times S^n , (\beta G)^n \times \mr{point} \big)$ reads
\[
\begin{split}
\cdots & \to CH_p^G ((\beta G)^n ) \to CH_p^G ((\beta G)^n \times S^n ) \xrightarrow{\phi}
CH_{p-n}^G ((\beta G)^n ) \otimes_{\mh C} H_n (S^n ,\mr{point}) \\
& \to CH_{p-1}^G ((\beta G)^n ) \to CH_{p-1}^G ((\beta G)^n \times S^n ) \to \cdots
\end{split}
\]
We note that \pagebreak[3]
\begin{equation*}
CH_p^G ((\beta G)^n \times S^n ) \cong CH_p^G ((\beta G)^n ) \otimes_{\mh C} H_0 (S^n ) 
\oplus CH_{p-n}^G ((\beta G)^n ) \otimes_{\mh C} H_n (S^n ,\mr{point}) ,
\end{equation*}
so $\phi$ is surjective and the sequence splits. Since $\mu_{HN} - \mu' = 0$ on 
$CH_*^G ((\beta G)^n )$, there exists a unique map $f$ making the following diagram commutative:
\begin{equation}\label{eq:3.34}
\begin{array}{ccccc}
\!\! CH_p^G ((\beta G)^n ) \!\! & \to & \!\! CH_p^G ((\beta G)^n \times S^n ) \!\! & \to &
\!\! CH_{p-n}^G ((\beta G)^n ) \! \otimes_{\mh C} \! H_n (S^n ,\mr{point}) \!\! \\
\downarrow 0 & & \downarrow \mu_{HN} - \mu' & & \downarrow f \\
HP_* (\mc H (G)) & = & HP_* (\mc H (G)) & = & HP_* (\mc H (G)) \,. \\
\end{array}
\end{equation}
By the universal property of $\beta G$, there exists a $G$-map 
$h : (\beta G )^n \times S^n \to \beta G$, and it is unique up to homotopy. Hence we may 
assume that $h$ maps $\{ \sigma \} \times S^n \subset (\beta G )^n \times S^n$ to the 
barycenter of $\sigma$ in $\beta G$. By \eqref{eq:3.38} and \eqref{eq:3.36} the middle map 
in \eqref{eq:3.34} factors as
\begin{equation}\label{eq:3.35} 
CH_p^G ((\beta G)^n \times S^n ) \xrightarrow{CH_p^G (h)} CH_p^G (\beta G )  
\xrightarrow{\mu_{HN} - \mu'} HP_* (\mc H (G)) \,.
\end{equation}
But $CH_p^G (h)$ kills the $n$th homology of $S^n$, which in combination with \eqref{eq:3.34}
shows that \eqref{eq:3.35} is zero. Therefore the above map $f$ must also be zero.
As $G$-spaces we have 
\[
\beta^{(n)} G \setminus \beta^{(n-1)} G \cong 
(\beta G)^n \times S^n \setminus \mr{point} \,,
\]
with $G$ acting trivially on the last factor. The corresponding long exact sequence in
equivariant cosheaf homology is
\[
\begin{split}
\cdots & \to CH_p^G ( \beta^{(n-1)} G ) \to CH_p^G ( \beta^{(n)} G ) \to
CH_{p-n}^G ((\beta G)^n ) \otimes_{\mh C} H_n (S^n ,\mr{point}) \\
& \to CH_{p-1}^G (\beta^{(n-1)} G ) \to CH_{p-1}^G ( \beta^{(n)} G ) \to \cdots
\end{split}
\]
By the induction hypothesis $\mu_{HN} - \mu' = 0$ on $CH_*^G ( \beta^{(n-1)} G )$, so
we can write down a commutative diagram
\[
\begin{array}{ccccc}
CH_p^G ( \beta^{(n-1)} G) & \to & CH_p^G (\beta^{(n)} G) & \to &
CH_{p-n}^G ((\beta G)^n ) \otimes_{\mh C} H_n (S^n ,\mr{point}) \\
\downarrow 0 & & \downarrow \mu_{HN} - \mu' & & \downarrow f \\
HP_* (\mc H (G)) & = & HP_* (\mc H (G)) & = & HP_* (\mc H (G)) \,. \\
\end{array}
\]
We already showed that $f = 0$, so $\mu_{HN} - \mu' = 0$ on $CH_p^G (\beta^{(n)} G)$.
Thus \eqref{eq:3.14} commutes for $\Sigma = \beta^{(n)} G$, which completes our induction 
step. $\qquad \Box$
\\[2mm]

The above proof can be compared with \cite[Section 5.1]{Mey1}.

\begin{cor}\label{cor:3.7}
It can be proved with periodic cyclic homology that the Baum--Connes assembly map
\[
\mu \otimes \mr{id} : K_*^G (\beta G) \otimes_{\mh Z} \mh Q \to 
K_* (C_r^* (G)) \otimes_{\mh Z} \mh Q
\]
is an isomorphism for every reductive $p$-adic group $G$.
\end{cor}
\emph{Proof.}
Lemma \ref{lem:3.6} and \eqref{eq:3.12} show that $\mu' = \mu_{HN}$ is an 
isomorphism. Hence all the isomorphisms in the diagram of Theorem 
\ref{thm:3.4} admit mutually independent proofs. With the commutativity of
the diagram we can use any five of them to prove the sixth. 
In particular we can show without using Lafforgue's work that 
\[
K^G_* (\beta G) \otimes_{\mh Z} \mh C \cong 
K_* (C_r^* (G)) \otimes_{\mh Z} \mh C \,,
\]
which is equivalent to $\mu$ being a rational isomorphism. $\qquad \Box$








