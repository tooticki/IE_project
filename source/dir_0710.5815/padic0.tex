\ \vspace{1cm}
\begin{center}
\huge \textbf{Periodic cyclic homology\\ of reductive $p$-adic groups}\\[1cm]
\Large Maarten Solleveld\\[3mm]
\normalsize Mathematisches Institut,
Georg-August-Universit\"at G\"ottingen\\
Bunsenstra\ss e 3-5, 37073 G\"ottingen, Germany\\
email: Maarten.Solleveld@mathematik.uni-goettingen.de \\[5mm]
August 2008 \\[3cm]
\end{center}


\begin{minipage}{13cm}
\textbf{Mathematics Subject Classification (2000).} \\
20G25, 16E40 \\[1cm]
\textbf{Abstract.}\\
Let $G$ be a reductive $p$-adic group, $\mc H (G)$ its Hecke algebra
and $\mc S (G)$ its Schwartz algebra. We will show that these algebras
have the same periodic cyclic homology. This provides an
alternative proof of the Baum--Connes conjecture for $G$, modulo torsion.

As preparation for our main theorem we prove two results that have
independent interest. Firstly, a general comparison theorem for the 
periodic cyclic homology of finite type algebras and
certain Fr\'echet completions thereof. Secondly, a refined form of the 
Langlands classification for $G$, which clarifies the
relation between the smooth spectrum and the tempered spectrum.
\\[1cm]
\textbf{Acknowledgements.}\\
This paper is partly based on the author's PhD-thesis, which was written
at the Universiteit van Amsterdam under the supervision of Eric Opdam. 
The author is grateful for the support and advice that professor Opdam
has given him during his PhD-research.
He would also like to thank Ralf Meyer, Christian Voigt and the referee
for their comments, which lead to substantial clarifications of some proofs.
\end{minipage}


\tableofcontents


\chapter*{Introduction}
\addcontentsline{toc}{chapter}{Introduction}

In this paper we compare different homological invariants of algebras 
associated to reductive $p$-adic groups. Group algebras, or more precisely 
convolution algebras of functions on groups, have always been important 
objects of study in noncommutative geometry. Generally speaking the idea
(or hope) is that the interaction between representation theory, harmonic
analysis, operator algebras and geometry leads to results that can not (yet)
be proven inside only one of these areas. 

By definition a group algebra encodes information about a group, so its 
homological invariants should reflect properties of the group. 
Therefore, whenever one considers two convolution algebras 
associated to the same group, their invariants should be closely related. 
Yet in practice this has to be taken with quite a few grains of salt.
For example the \pch of $\mh C [\mh Z \rtimes C_2 ]$ is isomorphic
to the De Rham-cohomology (with complex coefficients) of the disjoint union of 
$\mh C^\times / (z \sim z^{-1})$ and a point. On the other hand the \pch of
the group-$C^*$-algebra $C^* (\mh Z \rtimes C_2 )$ does not give any new
information: it is the algebra itself in even degrees and it vanishes in odd 
degrees. So finding a meaningful invariant of the group is a matter of both
choosing the right group algebra and the right functor.

For Fr\'echet algebras topological $K$-theory is a good choice, since it is a 
very stable functor. It has the excision property and is invariant under homotopy 
equivalences and under passing to holomorphically closed dense subalgebras. 
Comparing with the above example, the $K$-theory of $C^* (\mh Z \rtimes C_2)$
is again isomorphic to the cohomology of a manifold. But the manifold
has been adjusted to its compact form 
\[
S^1 / (z \sim z^{-1}) \cup \mr{point} \quad \cong \quad [-1,1] \cup \mr{point}
\]
and we must take its singular cohomology with integral coefficients. We remark 
that subalgebras consisting of all functions on $\mh Z \rtimes C_2$ with rapid 
(resp. subexponential) decay have the same $K$-theory.

Nevertheless it can be hard to compute a $K$-group of a lesser-known algebra.
Indeed in the classical picture of $K_0$ one has to find all homotopy classes of 
projectors, a task for which no general procedure exists.
\\[1mm]

Of course there is a wider choice of interesting functors. Arguably the most subtle
one is Hochschild homology ($HH_*$), the oldest homology theory for algebras.
Depending on the circumstances it can be regarded as group cohomology,
(noncommutative) differential forms or as a torsion functor. Moreover Hochschild
homology can be computed with the very explicit bar complex. On the other hand
$HH_*$ does neither have the excision property, nor is it homotopy invariant. 

We mainly discuss \pch $(HP_* )$ in this paper. Although it carries less information 
than Hochschild homology, it is much more stable. The relation between $HH_*$ 
and $HP_*$ is analogous to that between differential forms and De Rham 
cohomology, as the Hochschild--Kostant--Rosenberg theorem makes explicit in the 
case of smooth commutative algebras. It is known that \pch has the excision 
property and is invariant under Morita equivalences, diffeotopy equivalences and
nilpotent extensions.
Together with the link to Hochschild homology these make $HP_*$ computable
in many cases. This functor works especially well on the category of finite type
algebras \cite{KNS}, that is, algebras that are finitely generated modules over the
coordinate ring of some complex affine variety. In this category an important 
principle holds for periodic cyclic homology, namely that it depends only on 
the primitive ideal spectrum of the algebra in question.

A similar principle fails miserably for topological algebras, even for commutative 
ones. For example let $M$ be a compact smooth manifold. Then 
$HP_* (C^\infty (M))$ is the De Rham cohomology of $M$, while $HP_* (C(M))$
just returns the $C^*$-algebra $C(M)$. The underlying reason is that $HP_*$
does not only see the (irreducible) modules of an algebra, it also takes the 
derived category into account. In geometric terms this means that $HP_* (A)$
does not only depend on the primitive ideal spectrum of $A$ as a topological 
space, but also on the structure of the "infinitesimal neighborhoods" of points
in this space. These infinitesimal neighborhoods are automatically right for 
finite type algebras, because they can be derived from the underlying affine variety.
But the spectrum of $C(M)$ does not admit infinitesimal neighborhoods. Indeed,
these have to be related to the powers of a maximal ideal $I$, but they collapse
because $\overline{I^n} = I$ for all $n \in \mh N$.

We remark that this problem can partially be overcome with a clever variation
on $HP_*$, local cyclic homology \cite{Mey3}. This functor gives nice results for
$C^*$-algebras because it is stable under isoradial homomorphisms of complete 
bornological algebras. On the other hand this theory does require an array of new 
techniques.

We will add a new move under which \pch is invariant. Let $\Gamma$ be a finite 
group acting (by $\alpha$) on a nonsingular 
complex affine variety $X$, and suppose that we have a cocycle 
$u: \Gamma \to GL_N (\mc O (X))$. Then $\alpha$ and $u$ combine to 
an action of $\Gamma$ on $M_N (\mc O (X))$:
\begin{equation}
\gamma \cdot f = u_\gamma f^{\alpha (\gamma)} u^{-1}_\gamma \,.
\end{equation}
The algebra of $\Gamma$-invariants $M_N (\mc O (X) )^\Gamma$ has a natural 
Fr\'echet completion, namely $M_N (C^\infty (X) )^\Gamma$. We will show in
Chapter 1 that the inclusion map induces an isomorphism
\begin{equation}\label{eq:0.1}
HP_* \big( M_N (\mc O (X) )^\Gamma \big) \to 
HP_* \big( M_N (C^\infty (X) )^\Gamma \big) \,.
\end{equation}
The proof is based on abelian filtrations of both algebras, that is, on sequences of ideals
such that the successive quotients are Morita equivalent to commutative algebras.
In terms of primitive ideal spectra this means that we have stratifications of finite length
such that all the strata are Hausdorff spaces.
\\[1mm]

Let us discuss these general issues in connection with reductive $p$-adic groups. 
We use this term as an abbreviation of ``the $\mh F$-rational points of a 
connected reductive algebraic group, where $\mh F$ is a non-Archimedean local 
field". Such groups are
important in number theory, especially in relation with the Langlands program.
There are many open problems for reductive $p$-adic groups, for example there is
no definite classification of irreducible smooth representations. There are two general 
strategies to divide the classification problem into pieces, thereby reducing it to 
either supercuspidal or square-integrable representations. 

For the first we start with a 
supercuspidal representation of a Levi-component of a parabolic subgroup of 
our given group $G$. Then we apply parabolic induction to obtain a (not necessarily 
irreducible) smooth $G$-representation. The collection of representations obtained
in this way contains every irreducible object at least once.

The second method involves the Langlands classification, which reduces the problem
to the classification of irreducible tempered $G$-representations. These can be found
as in the first method, replacing supercuspidal by square-integrable representations. 
This kind of induction was studied in \cite{ScZi}. The procedure yields a collection of 
(possibly decomposable) tempered $G$-representations,
in which every irreducible tempered representation appears at least once. 

Our efforts in Chapter 2 result in a refinement of the Langlands classification. To every 
irreducible smooth $G$-representation we associate a quadruple $(P,A,\omega, \chi )$
consisting of a parabolic pair $(P,A)$, a square-integrable representation $\omega$
of the Levi component $Z_G (A)$ and an unramified character $\chi$ of $Z_G (A)$.
Moreover we prove that this quadruple is unique up to $G$-conjugacy. This result is 
useful for comparing the smooth spectrum of $G$ with its
tempered spectrum, and for constructing stratifications of these spectra.
\\[1mm]

Let us consider three convolution algebras associated to a reductive $p$-adic group $G$.
Firstly the reduced $C^*$-algebra $C_r^* (G)$, secondly the Hecke algebra $\mc H (G)$
and thirdly Harish-Chandra's Schwartz algebra $\mc S (G)$. For each of these algebras 
we will study the most appropriate homology theory. For the reduced $C^*$-algebra this 
is topological $K$-theory, and for the Hecke algebra we take periodic cyclic homology. 
For the Schwartz algebra the choice is more difficult. Since it is not a Fr\'echet algebra 
the usual versions of $K$-theory are not even defined for $\mc S (G)$. It is not difficult 
to give an ad-hoc definition, and the natural ways to do so quickly lead to
$K_* (\mc S (G)) \cong K_* (C_r^* (G))$. Nevertheless, we would also like to compute
the \pch of $\mc S (G)$. It is definitely not a good idea to do this with respect to the 
algebraic tensor product, because that would ignore the topology on $\mc S (G)$. As
explained in \cite{Mey}, $\mc S (G)$ is best regarded as a bornological algebra, and
therefore we will study its \pch with respect to the completed bornological tensor product
$\hot_{\mh C}$. 

That this is the right choice is vindicated by two comparison theorems. 
On the one hand the author already proved in \cite{Sol1} that the Chern character 
for $\mc S (G)$ induces an isomorphism
\begin{equation}\label{eq:0.2}
ch \otimes \mr{id} : K_* (C_r ^* (G)) \otimes_{\mh Z} \mh C \to HP_* (\mc S (G) ,\hot_{\mh C}) \,.
\end{equation}
On the other hand we will show in Section \ref{sec:3.1} that the inclusion of $\mc H (G)$
in $\mc S (G)$ induces an isomorphism
\begin{equation}\label{eq:0.3}
HP_* (\mc H (G)) \to HP_* (\mc S (G) ,\hot_{\mh C}) \,.
\end{equation}
Of course both comparison theorems can be decomposed as direct sums over the
Bernstein components of $G$. The proof of \eqref{eq:0.3} is an extension of the ideas 
leading to \eqref{eq:0.1} and is related to the following quote \cite[p. 3]{SSZ}:\\
\emph{``The remarkable picture which emerges is that Bernstein's decomposition of
$\mc M (G)$ into its connected components refines into a stratification of $\mc G (G)$ 
where the strata, at least up to nilpotent elements, are module categories over 
commutative rings. We strongly believe that such a picture holds true for any group $G$."}\\
If this is indeed the case then our methods can be applied to many other groups.
\\[1mm]

The most important application of \eqref{eq:0.2} and \eqref{eq:0.3} lies in their relation
with yet other invariants of $G$. Let $\beta G$ be the affine Bruhat--Tits building of $G$.
The classical paper \cite{BCH} introduced among others the equivariant $K$-homology
$K_*^G (\beta G)$ and the cosheaf homology $CH_*^G (\beta G)$. Let us recall the
known relations between these invariants. The Baum--Connes conjecture for $G$,
proven by Lafforgue \cite{Laf}, asserts that the assembly map
\begin{equation}\label{eq:0.4}
\mu : K_*^G (\beta G) \to K_* (C_r^* (G))
\end{equation}
is an isomorphism. Voigt \cite{Voi4} constructed a Chern character 
\begin{equation}\label{eq:0.5}
ch : K_*^G (\beta G) \to CH_*^G (\beta G)
\end{equation}
which becomes an isomorphism after tensoring the left hand side with $\mh C$.
Furthermore it is already known from \cite{HiNi} that $CH_*^G (\beta G)$ is 
isomorphic to $HP_* (\mc H (G))$. Altogether we get a diagram
\begin{equation}\label{eq:0.6}
\begin{array}{ccc}
K_*^G (\beta G) \otimes_{\mh Z} \mh C & \cong & 
K_* (C_r^* (G)) \otimes_{\mh Z} \mh C \\
\cong & & \cong \\
CH_*^G (\beta G) & \cong & HP_* (\mc H (G))
\end{array}
\end{equation}
whose existence was already conjectured in \cite{BHP2}. We will prove in Section
\ref{sec:3.3} that it commutes. The four isomorphisms 
all have mutually independent proofs, so any three of them can be used to proof
the fourth. None of the proofs is easy, but it seems to the author that 
\eqref{eq:0.4} is the most difficult one. Therefore
it is not unreasonable to say that this diagram provides an alternative way to prove
the Baum--Connes conjecture for reductive $p$-adic groups, modulo torsion.

Returning to our initial broad point of view, we conclude that we used 
representation theory and harmonic analysis to prove results in noncommutative
geometry. It is outlined in \cite{BHP2} how cosheaf homology could be used to
prove representation theoretic results. The author hopes that the present paper
might contribute to the understanding of the issues raised in \cite{BHP2}.







