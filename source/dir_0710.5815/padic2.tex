\chapter{Some representation theory of reductive $p$-adic groups}


\section{Convolution algebras}
\label{sec:2.1}


In this chapter we collect some important results concerning
smooth representations of reductive $p$-adic groups. Good sources
for the theory discussed here are
\cite{BeDe,Car,Sil2,SSZ,Tits,Wal}.

Let $\mh F$ be a non-Archimedean local field with discrete valuation $v$ 
and norm $\norm{.}_{\mh F}$. We assume that the cardinality of the residue 
field is a power $q$ of a prime $p$. Let $\mc G$ be a connected reductive algebraic 
group defined over $\mh F$, and let $G = \mc G (\mh F )$ be the group of 
$\mh F$-rational points. We briefly call $G$ a reductive $p$-adic group.

We denote the collection of compact open subgroups of $G$ by
CO$(G)$. A representation $V$ of $G$ is called smooth if every $v
\in V$ is fixed by a compact open subgroup, or equivalently if $V
= \cup_{K \in \mr{CO}(G)} V^K$. We say that such a smooth representation $V$ 
is admissible if every $V^K$ has finite dimension. For example every smooth 
$G$-representation of finite length is admissible \cite[3.12]{BeDe}.
Let Rep$ (G)$ be the category of
smooth $G$-representations on complex vector spaces, and let Irr$ (G)$ be 
the set of equivalence classes of irreducible objects in Rep$ (G)$.
The Jordan--H\"older content JH$ (V)$ is the collection of all elements of Irr$ (G)$
which are equivalent to a subquotient of the $G$-representation $V$.

Fix a Haar measure $d\mu$ on $G$. Recall that the convolution
product of two functions $f,f' : G \to \mh C$ is defined as
\[
(f * f')(g') = \int_G f(g) f'(g^{-1} g') \, \textup{d}\mu (g) \,.
\]
For $K \in \mr{CO}(G)$ we let $\mc H (G,K)$ be the convolution
algebra of $K$-biinvariant complex-valued compactly supported
functions on $G$. This is called the Hecke algebra of $(G,K)$. Our
main subject of study will be the Hecke algebra of $G$, which
consists of all compactly supported locally constant functions on $G$:
\begin{equation}
\mc H (G) := {\ts \bigcup_{K \in \mr{CO}(G)} } \mc H (G,K) \,.
\end{equation}
For every $K \in \mr{CO}(G)$ there is an idempotent $e_K \in \mc H
(G)$, which is $\mu (K)^{-1}$ times the characteristic function of
$K$. Notice that
\begin{equation}
\mc H (G,K) = e_K \mc H (G) e_K = \mc H (G)^{K \times K} ,
\end{equation}
where $G \times G$ acts on $\mc H (G)$ by left and right
translations. In particular, the nonunital algebra $\mc H (G)$ is
idempotented, which assures that many properties of unital
algebras also hold for $\mc H (G)$.

An $\mc H (G)$-module $V$ is called nondegenerate or essential if
\begin{equation}\label{eq:2.1}
\mc H \cdot V = V ,
\end{equation}
or equivalently if for all $v \in V$ there exists a $K \in \mr{CO}(G)$
such that $e_K \cdot v = v$. A smooth $G$-representation $(\pi ,V)$ is
made into an essential $\mc H (G)$-module by
\begin{equation}\label{eq:2.12}
\pi (f) v = \int_G f (g) \pi (g) v \, \textup{d}\mu (g) \qquad v \in V , f \in \mc H (G) \,.
\end{equation}
This leads to an equivalence between Rep$(G)$ and the category of
essential $\mc H (G)$-modules. Hence we may identify the 
primitive ideal spectrum of $\mc H (G)$ with Irr$ (G)$.

Let $\mc S (G,K)$ be the space of rapidly decreasing
$K$-biinvariant functions on $G$. According to \cite[Theorem
29]{Vig} this is a unital nuclear Fr\'echet *-algebra.
Harish-Chandra's Schwartz algebra consists of all uniformly
locally constant rapidly decreasing functions on $G$:
\begin{equation}
\mc S (G) := {\ts \bigcup_{K \in \mr{CO}(G)} } \mc S (G,K) \,.
\end{equation}
Endowed with the inductive limit topology this is a complete locally convex 
topological algebra with separately continuous multiplication. Clearly
\begin{equation}
\mc S (G,K) = e_K \mc S (G) e_K = \mc S (G)^{K \times K} .
\end{equation}
If $(K_i )_{i=1}^\infty $ is a decreasing sequence of compact open
subgroups of $G$ which forms a neighborhood basis of the unit
element $e \in G$, then
\begin{equation}\label{eq:2.2}
\mc S (G) = {\ts \bigcup_{i=1}^\infty } \mc S (G,K_i )
\end{equation}
is a strict inductive limit of nuclear Fr\'echet spaces. Nevertheless
$\mc S (G)$ is not metrizable. We say that a smooth
$G$-representation $(\pi, V)$ is tempered if the $\mc H (G)$-module structure
extends to $\mc S (G)$. If $(\pi,V)$ is admissible, then there is at most one
such extension, see \cite[p. 51]{SSZ}. Thus we have
\begin{itemize}
\item the category $\mr{Rep}^t (G)$ of tempered smooth
$G$-representations,
\item the space $\mr{Irr}^t (G)$ of equivalence classes of
irreducible objects in $\mr{Rep}^t (G) $,
\item the primitive ideal spectrum of $\mc S (G)$, which by
\cite[p. 52]{SSZ} can be identified with $\mr{Irr}^t (G)$.
\end{itemize}
Furthermore we consider the reduced $C^*$-algebra of $G$. By
definition $C_r^* (G)$ is the completion of $\mc H (G)$ with
respect to the operator norm coming from the left regular
representation of $G$ on $L^2 (G)$. For $K \in \mr{CO}(G)$ let
$C_r^* (G,K)$ be the norm closure of $\mc H (G,K)$ in $B(L^2 (G))$. 
This is a unital type I $C^*$-algebra which contains $\mc S
(G,K)$ as a holomorphically closed dense subalgebra 
\cite[Theorem 29]{Vig}. Moreover by \cite[p. 53]{SSZ}
\begin{equation}
C_r^* (G,K) = e_K C_r^* (G) e_K = C_r^* (G)^{K \times K} .
\end{equation}
Therefore we can construct $C_r^* (G)$ also as an inductive limit
of $C^*$-algebras:
\begin{equation}\label{eq:2.14}
C_r^* (G) = \varinjlim_{K \in \mr{CO}(G)} C_r^* (G,K) \,.
\end{equation}
Having introduced these algebras we will describe the Bernstein
decomposition of Rep$ (G)$.
Suppose that $P$ is a parabolic subgroup of $G$ and that $P = M 
\ltimes N$ where $N$ is the unipotent radical of $P$ and $M$ is a Levi
subgroup. Although $G$ and $M$ are unimodular the modular function
$\delta_P$ of $P$ is general not constant. To be precise
\begin{equation}
\delta_P (m n) = \norm{\det \big( \mr{ad}(m) \big|_{\mf n} \big)
}_{\mh F} \qquad m \in M , n \in N
\end{equation}
where $\mf n$ is the Lie algebra of $N$. For $\sigma \in \mr{Rep}(M)$ one defines
\[
I_P^G (\sigma ) := \mr{Ind}_P^G (\delta_P^{1/2} \otimes \sigma ) \,.
\]
This means that we first inflate $\sigma$ to $P$, then we twist it with $\delta_P^{1/2}$ 
and finally we take the smooth induction to $G$. The twist is useful to preserve unitarity. 
The functor $I_P^G$ is known as parabolic induction. It is transitive in the sense that 
for any parabolic subgroup $Q \subset P$  we have 
\[
I_Q^G = I_P^G \circ I^M_{Q \cap M} \,.
\]
Let $\sigma$ be an irreducible supercuspidal representation of $M$. Thus
the restriction of $\sigma$ to the derived group of $M$ is unitary, but $Z(M)$
may act on $\sigma$ by an arbitrary character. We call $(M,\sigma )$ a cuspidal 
pair, and from it we construct the parabolically induced $G$-representation 
$I_P^G (\sigma )$. For every $(\pi ,V) \in \mr{Irr} (G)$ there is a cuspidal pair 
$(M, \sigma )$, uniquely determined up to $G$-conjugacy, such that 
$V \in \mr{JH} (I_P^G (\sigma ) )$.

We denote the complex torus of nonramified characters of $M$ by
$\xnr (M)$, and the compact subtorus of unitary nonramified
characters by $\xunr (M)$. We say that two cuspidal pairs
$(M,\sigma )$ and $M' ,\sigma' )$ are inertially equivalent if
there exist $\chi \in \xnr (M')$ and $g \in G$ such that $M' = g
M g^{-1}$ and $\sigma' \otimes \chi \cong \sigma^g$. With an
inertial equivalence class $\mf s = [M,\sigma ]_G$ we associate a
subcategory Rep$ (G)^{\mf s}$ of Rep$ (G)$. By definition its
objects are smooth $G$-representations $\pi$ with the following
property: for every $\rho \in \mr{JH} (\pi)$ there
is a $(M,\sigma ) \in \mf s$ such that $\rho$ is a
subrepresentation of $I_P^G (\sigma )$. These blocks Rep$ (G)^{\mf
s}$ give rise to the Bernstein decomposition \cite[Proposition 2.10]{BeDe}
\[
\mr{Rep} (G) = {\ts \prod_{\mf s \in \Omega (G)} } \, \mr{Rep} (G)^{\mf s} \,.
\]
The set $\Omega (G)$ of Bernstein components is countably infinite. 
There are corresponding decompositions of the Hecke and Schwartz 
algebras of $G$ into two-sided ideals:
\begin{align}
\label{eq:2.3} & \mc H (G) = {\ts \bigoplus_{\mf s \in \Omega (G)} }
\, \mc H (G)^{\mf s} \,, \\
\label{eq:2.4} & \mc S (G) \, = {\ts \bigoplus_{\mf s \in \Omega (G)} }
\, \mc S (G)^{\mf s} \,.
\end{align}
For $C_r^* (G)$ this is a less straightforward, since its elements can be
supported on infinitely many Bernstein components. Let $C_r^* (G)^{\mf s}$ 
be the two-sided ideal generated by $\mc H (G)^{\mf s}$. The reduced 
$C^*$-algebra of $G$ decomposes as a direct sum in the $C^*$-algebra sense:
\begin{equation}\label{eq:2.13}
C_r^* (G) = \varinjlim_{\mf S} \bigoplus_{\mf s \in \mf S} C_r^* (G)^{\mf s} \,,
\end{equation}
where the direct limit runs over all finite subsets $\mf S$ of $\Omega (G)$.
For $K \in \mr{CO}(G)$ and $\mf s \in \Omega (G)$ we write
\[
\begin{array}{ccccc}
\mc H (G,K)^{\mf s} & = & \mc H (G)^{\mf s} & \cap & \mc H (G,K) \,, \\
\mc S (G,K)^{\mf s} & = & \mc S (G)^{\mf s} & \cap & \mc S (G,K) \,, \\
C_r^* (G,K)^{\mf s} & = & C_r^* (G)^{\mf s} & \cap & C_r^* (G,K) \,.
\end{array}
\]
Every element of $\mc H (G)$ has a unique decomposition as a sum
of a part in $\mc H (G)^{\mf s}$ and a part in the annihilator of
this ideal. In particular we can write
\begin{equation}
e_K = e_K^{\mf s} + e'_K \in \mc H (G)^{\mf s} \oplus
\bigoplus_{\mf s' \in \Omega (G) \setminus \{\mf s\} } \mc H (G)^{\mf s'} .
\end{equation}
\begin{prop}\label{prop:2.1}
\begin{description}
\item[a)] For fixed $K \in \mr{CO}(G)$ there exist only finitely
many $\mf s \in \Omega (G)$ such that $\mc H (G,K)^{\mf s} \neq 0$.
\item[b)] For every $\mf s \in \Omega (G)$ there exists a $K_{\mf s}
\in \mr{CO}(G)$ such that for all compact open subgroups $K
\subset K_{\mf s}$ the bimodules $e_K^{\mf s} \mc H (G)$ and $\mc
H (G) e_K^{\mf s}$ provide a Morita equivalence between 
\[
\mc H (G)^{\mf s} = \mc H (G) e_K^{\mf s} \mc H (G) \quad \text{and}  
\quad \mc H (G,K)^{\mf s} = e_K^{\mf s} \mc H (G) e_K^{\mf s} \,.
\]
\item[c)] As \textup{b)}, but with $\mc S (G)$ instead of $\mc H (G)$.
\item[d)] As \textup{b)}, but with $C_r^* (G)$ instead of $\mc H (G)$.
\end{description}
\end{prop}
\emph{Proof.} a) See \cite[\S 3.7]{BeDe}.\\
b) By \cite[Corollaire 3.9]{BeDe} there exists a $K_{\mf s} \in \mathrm{CO}(G)$ such that
\[
\mc H (G)^{\mf s} = \mc H (G) e_{K_{\mf s}}^{\mf s} \mc H (G) \,.
\]
For any $K \in \mr{CO}(G)$ with $K \subset K_{\mf s}$ we have 
$e_{K_{\mf s}} \in \mc H (G) e_K \mc H (G)$. Hence
\[
\mc H (G) e_K^{\mf s} \mc H (G) = \big( \mc H (G) e_K \mc H (G) \big) \cap \mc H (G)^{\mf s}
\]
contains $e_{K_{\mf s}}^{\mf s}$ and must equal $\mc H (G)^{\mf s}$.
We note that these $e^{\mf s}_K$ are precisely the special idempotents
constructed in \cite[Proposition 3.13]{BuKu}. \\
c) and d) follow directly from b) and the characterisation of 
$\mc S (G)^{\mf s}$ and $C_r^* (G)^{\mf s}$ as the ideals 
generated by $\mc H (G)^{\mf s}. \qquad \Box$
\\[2mm]

Let $\mf s = [M,\sigma ]_G \in \Omega (G)$ and put
\[
\begin{array}{lll}
N (M,\sigma ) & = & \{ g \in N_G (M) : \sigma^g \cong \sigma \otimes \chi 
\text{ for some } \chi \in \xnr (M) \} \,, \\
W_{\mf s} & = & N (M,\sigma ) / M \,.
\end{array}
\]
\begin{thm}\label{thm:2.15}
For $K \in \mr{CO}(G)$ as in Proposition \textup{\ref{prop:2.1}.b} $\mc H (G,K)^{\mf s}$ 
is a unital finite type algebra with center isomorphic to 
\[
\mc O (\xnr (M) / W_{\mf s} ) = \mc O (\xnr (M) )^{W_{\mf s}} .
\]
There are natural isomorphisms 
\[
\begin{array}{lll}
Z \big( \mc H(G,K_{\mf s})^{\mf s} \big) & \cong & Z \big( \mr{Rep}(G )^{\mf s} \big) \,, \\
\mr{Prim} \big( \mc H(G,K_{\mf s})^{\mf s} \big) & \cong & 
 \mr{Irr}(G) \cap \mr{Rep}(G )^{\mf s} \,, \\
\mr{Prim} \big( \mc S(G,K_{\mf s})^{\mf s} \big) & \cong &
 \mr{Irr}^t(G) \cap \mr{Rep}(G )^{\mf s} \,.
\end{array}
\]
\end{thm}
\emph{Proof.}
This follows from Th\'eor\`eme 2.13 and Corollaire 3.4 of \cite{BeDe},
in combination with Proposition \ref{prop:2.1}. $\qquad \Box$
\\[2mm]

The cohomological dimension of the abelian category Rep$ (G)$ equals the 
rank of $G$, which is by definition the dimension of a maximal split
subtorus of $G$ \cite[Section II.3]{ScSt}. The author does not
know whether the abelian category $\mr{Rep}^t (G)$ has finite
cohomological dimension. Yet one can determine something like
the global dimension of $\mc S (G)$, at the cost of using more
advanced techniques. Namely, according to Meyer \cite[Theorem 29]{Mey} 
the cohomological dimension of the exact category
$\mr{Mod}_b (\mc S (G))$ of complete essential bornological
$\mc S (G)$-modules is also equal to the rank of $G$. The natural
tensor product to work with in this category is the completed
bornological $A$-balanced tensor product, which we denote by
$\hot_A$. For Fr\'echet spaces $\hot_{\mh C}$ agrees with the
completed projective tensor product, so we abbreviate it to $\hot$.
For later use we translate these cohomological dimensions to 
statements about Hochschild homology.

\begin{lem}\label{lem:2.2}
\begin{description}
\item[a)] $HH_n (\mc H (G)) = 0$ for all $n > \mr{rk}(G) $,
\item[b)] $HH_n (\mc S (G),\hot_{\mh C} ) = 0$ for all $n > \mr{rk}(G) $.
\end{description}
\end{lem}
\emph{Proof.} a) can be found in \cite{Nis} but we prefer to
derive it from the above. Let
\begin{equation}\label{eq:2.15}
\mh C \leftarrow P_0 \leftarrow P_1 \leftarrow \cdots \leftarrow
P_{\mr{rk}(G)} \leftarrow 0
\end{equation}
be a projective resolution of the trivial $G$-module $\mh C$.
Endowing $P_m \otimes \mc H (G)$ with the diagonal $G$-action,
$\mc H (G) \leftarrow P_* \otimes \mc H (G)$ becomes a resolution
of $\mc H (G)$ by projective $\mc H (G)$-bimodules. By definition
\begin{equation}
\begin{split}
HH_n (\mc H (G)) & = \mr{Tor}_n^{\mc H (G) \otimes \mc H (G)^\mr{op}}
(\mc H (G), \mc H (G)) \\
& = H_n \big( \mr{Hom}_{\mc H (G) \otimes \mc H (G)^\mr{op}} \big(
P_* \otimes \mc H (G) , \mc H (G) \big) \big) \,,
\end{split}
\end{equation}
which clearly vanishes for $n > \mr{rk}(G)$.\\
b) We will use that the inclusion $\mc H (G) \to \mc S (G)$ is
isocohomological \cite[Theorem 22]{Mey}. According to
\cite[(22)]{Mey} the differential complex
\[
\mc S (G) \leftarrow \mc S (G) \hot_{\mc H (G)} P_* \hot_{\mh C} \mc S (G)
\]
is a projective resolution of $\mc S (G)$ in $\mr{Mod}_b (\mc S (G))$. Hence
\begin{equation}\label{eq:2.5}
\begin{split}
HH_n (\mc S (G), \hot_{\mh C} ) & = \mr{Tor}_n^{\mc S (G)
\hot \mc S (G)^\mr{op}} (\mc S (G),\mc S (G)) \\
& = H_n \big( \mr{Hom}_{\mc S (G) \hot \mc S (G)^\mr{op}} \big( \mc S (G) 
\hot_{\mc H (G)} P_* \hot_{\mh C} \mc S (G) , \mc S (G) \big) \big) \,.
\end{split}
\end{equation}
From \eqref{eq:2.15} we see immediately that this vanishes 
$\forall n > \mr{rk}(G). \qquad \Box$
\\[3mm]
\begin{cor}\label{cor:2.3}
Let $\mf s \in \Omega (G)\,, n > \mr{rk}(G)$ and $K \in \mr{CO}(G)$
be such that $K \subset K_{\mf s}$.
\begin{description}
\item[a)] $HH_n (\mc H (G)^{\mf s}) = 0 = HH_n (\mc H (G,K)^{\mf s})$ ,
\item[b)] $HH_n (\mc S (G)^{\mf s},\hot_{\mh C} ) = 0 =
HH_n (\mc S (G,K)^{\mf s},\hot)$ .
\end{description}
\end{cor}
\emph{Proof.}
b) From \eqref{eq:2.4} and \eqref{eq:2.5} we see that
\begin{equation}
HH_n (\mc S (G),\hot_{\mh C} ) \cong {\ts \bigoplus_{\mf s \in \Omega (G)} }
HH_n (\mc S (G)^{\mf s},\hot_{\mh C} ) \,.
\end{equation}
By Proposition \ref{prop:2.1}.c
\begin{equation}
\begin{split}
& HH_n (\mc S (G,K)^{\mf s},\hot) = \mr{Tor}_n^{\mc S (G,K)^{\mf s} \hot 
\mc S (G,K)^{\mf s,\mr{op}}} (\mc S (G,K)^{\mf s}, \mc S (G,K)^{\mf s}) \cong \\
& \mr{Tor}_n^{\mc S (G)^{\mf s} \hot \mc S (G)^{\mf s,\mr{op}}}
(\mc S (G)^{\mf s}, \mc S (G)^{\mf s}) = HH_n (\mc S (G)^{\mf s},\hot_{\mh C} ) \,,
\end{split}
\end{equation}
where we take the torsion functors in the category of complete
bornological modules. By Lemma \ref{lem:2.2}.b these homology
groups all vanish for $n > \mr{rk}(G)$.\\
a) can be proved in exactly the same way as b), using \eqref{eq:2.3}, 
Lemma \ref{lem:2.2}.a and Proposition \ref{prop:2.1}.b. $ \qquad \Box$ 
\vspace{4mm}





\section{The Plancherel theorem}
\label{sec:2.2}

The Plancherel formula for $G$ is an explicit decomposition of the trace
\[
\mc H (G) \to \mh C \quad,\quad f \mapsto f (e)
\]
in terms of the traces of irreducible $G$-representations. Closely
related is the Plan\-che\-rel theorem, which describes $\mc S (G)$ in
terms of its irreducible representations. This description is due
to Harish-Chandra \cite{HC1,HC2}, although he published only a
sketch of the proof. Harish-Chandra's notes were worked out in
detail by Waldspurger \cite{Wal}. In the present section we recall the
most important ingredients of the Plancherel theorem, relying
almost entirely on the above papers.

A parabolic pair $(P,A)$ consists of a parabolic subgroup $P$ of
$G$ and a maximal split torus $A$ in
the center of some Levi subgroup $M$ of $P$. If $N$ is the
unipotent radical of $P$ then $P = M \ltimes N$ and $M = Z_G (A)$.
Moreover restriction from $M$ to $A$ defines a surjection
$\xnr (M) \to \xnr (A)$ with finite kernel.

The maximal parabolic pair is $(G,A_G )$, where $A_G$ is the unique
maximal split torus of $Z(G)$.
We fix a maximal split torus $A_0$ of $G$, and a minimal parabolic
subgroup $P_0$ containing $A_0$. We call $(P,A)$ semi-standard if
$A \subset A_0$, and standard if moreover $P \supset P_0$. Every
parabolic pair is conjugate to a standard one.

Let $(\omega ,E)$ be an irreducible square-integrable representation of $M$. 
By definition this entails that $E$ is smooth, pre-unitary and admissible. 
Let $(\breve \omega, \breve E)$ be the smooth contragredient representation. 
The admissible $G\times G$-representation
\[
L(\omega ,P) = I_{P \times P}^{G \times G} (E \otimes \breve E) =
I_P^G (E) \otimes I_P^G (\breve E )
\]
is naturally a nonunital Hilbert algebra. Notice that for every
$\chi \in \xunr (M)$ the representation $\omega \otimes \chi$ is
also square-integrable, and that $L(\omega \otimes \chi ,P)$ can
be identified with $L(\omega ,P)$. Let $\mb k_\omega$ be the set of $k
\in \xnr (M)$ such that $\omega \otimes k$ is equivalent to
$\omega$. This is a finite subgroup of $\xunr (M)$. For every $k
\in \mb k_\omega$ there exists a canonical unitary intertwiner
\[
I(k,\omega ) \in \mr{Hom}_{G \times G} (L (\omega ,P),L(\omega \otimes k,P)) \,.
\]

Next we consider the intertwiners associated to elements of
various Weyl groups. Let  $(Q,B)$ be another parabolic pair. Write
$W(A|G|B)$ for the set of all homomorphisms $B \to A$ induced by
inner automorphisms of $G$. This is a group in case $A = B $:
\[
W(G,A) := W(A|G|A) = N_G (A) / Z_G (A) = N_G (A) / M \,.
\]
Let $(Q,A^g )$, with $g \in G$, be yet another parabolic pair, and
put $n = [g] \in W (A^g |G|A)$. The equivalence class of the
$M^g$-representation $(\omega^{g^{-1}},E)$ depends only on $n$ and
may therefore be denoted by $n \omega$. Waldspurger constructs
certain normalized intertwiners $\prefix{^\circ}{c_{Q|P}}(n,\omega
)$. Preferring the simpler notation $I(n,\omega)$ we recall their
properties.

\begin{thm}\label{thm:2.4} \textup{\cite[Paragraphe V]{Wal}} \\
Let $(P,A) ,\, (P',A')$ and $(Q,B)$ be semi-standard
p-pairs, and $n \in W (B|G|A)$. There exists an intertwiner
\[
I (n,\omega \otimes \chi) \in \mr{Hom}_{G \times G} (L (\omega ,P)
, L (n \omega ,Q) )
\]
with the following properties:
\begin{itemize}
\item $\chi \to I (n, \omega \otimes \chi)$ is a rational
function on $\xnr (M) $,
\item $I (n ,\omega \otimes \chi)$ is unitary and regular
for $\chi \in \xunr (M) $,
\item If $n' \in W (A' |G|B)$ then
\[
I (n',n (\omega \otimes \chi) ) I (n,\omega \otimes \chi) = 
I (n' \, n, \omega \otimes \chi) \,.
\]
\end{itemize}
\end{thm}

To define the Fourier transform implementing the Plancherel
isomorphism we introduce a space of induction data, such that
every irreducible tempered representation is a direct summand of
(at least) one of these parabolically induced representations. For
every semi-standard parabolic pair $(P,A)$ choose a set 
$\Delta_M$ of irreducible square-integrable representations of
$M = Z_G (A)$, with the following property. For every
square-integrable $\pi \in \mr{Irr}(M)$ there exists precisely one
$\omega \in \Delta_M$ such that $\pi$ is equivalent to $\omega \otimes \chi$, 
for some $\chi \in \xnr (M)$. We call a triple $(P,A,\omega )$ standard if 
$(P,A)$ is a standard parabolic pair and $\omega \in \Delta_M$.

An induction datum is a quadruple $(P,A,\omega,\chi )$ where
$(P,A)$ is a semi-standard parabolic pair, $\omega \in \Delta_M$
and $\chi \in \xnr (M)$. Let $\Xi$ be the scheme of all induction
data and $\Xi_u$ the smooth submanifold of unitary induction data,
that is, those with $\chi \in \xunr (M)$. Then $\Xi$ and $\Xi_u$ are
countable disjoint unions of complex algebraic tori and compact tori,
respectively. For $\xi = (P,A,\omega,\chi ) \in \Xi$ we put
\[
I(\xi) = I_P^G (\omega \otimes \chi) \,.
\]
By \cite[Lemme III.2.3]{Wal} the representation $I(\xi)$ is
tempered if and only if $\omega \otimes \chi$ is tempered, if and
only if $\xi \in \Xi_u$. Like for cuspidal pairs one can define
inertial equivalence on $\Xi_u$. The set $\Omega^t (G)$ of all
equivalence classes $[P,A,\omega]_G$ is called the Harish-Chandra
spectrum of $G$. It comes with a natural surjection
$\Omega^t (G) \to \Omega (G)$, see \cite[Section 1]{SSZ}. It follows
from Proposition \ref{prop:2.1} and \cite[Th\'eor\`eme VIII.1.2]{Wal} 
that this map is finite-to-one.

Let $\mc L_{\Xi}$ be the vector bundle over $\Xi$ which is trivial
on every component and whose fiber at $\xi$ is $L(\omega,P)$. We
say that a section of this bundle is algebraic (polynomial) or
rational if it is supported on only finitely many components, and
has the required property on every component. Now we can define
the Fourier transform:
\begin{equation}\label{eq:2.6}
\begin{aligned}
& \mc F : \mc H (G) \to \mc O (\Xi ; \mc L_{\Xi}) \,, \\
& \mc F (f) (P,A,\omega,\chi ) = I(P,A,\omega,\chi)(f) \in
L(\omega ,P) \,,
\end{aligned}
\end{equation}
where we used the notation from \eqref{eq:2.12}.
Notice that this differs slightly from $\check f (\omega \otimes
\chi,P)$ as in \cite[\S VII.1]{Wal}. To make it fit better with
its natural adjoint Waldspurger adjusts the Fourier transform. We
will use \eqref{eq:2.6} though, because it is multiplicative.

To formalize the action of the intertwiners on sections of $\mc
L_\Xi$ we construct a locally finite groupoid $\mc W$. The set of
objects of $\mc W$ is $\Xi$ and the morphisms from $\xi$ to $\xi'$
are the pairs $(k,n)$ with the following properties:
\begin{itemize}
\item $k \in \mb k_\omega$ ,
\item $n \in W(A|G|A')$ and $n (A') = A$,
\item $n \omega'$ is equivalent to $\omega \otimes \tilde \chi$
for some $\tilde \chi \in \xnr (M) $.
\end{itemize}
The multiplication in $\mc W$, if possible, is
\[
(k,n) (k',n') = (k (k' \circ n),n n') \,.
\]
Let $\Gamma (\Xi ;\mc L_\Xi)$ be a suitable algebra of sections of
$\mc L_\Xi$. For $f \in \Gamma (\Xi ;\mc L_\Xi)$ we define
\begin{align*}
& k \cdot f (\chi) = I(k, \omega) f(k^{-1} \omega) \,,\\
& n \cdot f (\chi' ) = I(n,\omega ) f (\chi' \circ n) \,.
\end{align*}
Notice that the intertwiners do not stabilize $\mc O (\Xi ;\mc L_\Xi )$ in general.
Nevertheless, we write $\mc O (\Xi ;\mc L_\Xi )^{\mc W}$ for the subalgebra of
$\mc W$-invariant sections, which by construction contains $\mc F (\mc H (G))$.
Because $(\omega,E)$ is admissible,
\[
C^\infty (\xunr (M)) \otimes L(\omega,P)^{K \times K} = C^\infty
(\xunr (M)) \otimes I_P^G (E)^K \otimes I_P^G (\breve E )^K
\]
has a natural Fr\'echet topology, for every $K \in \mr{CO}(G)$. We endow
\[
C_c^\infty (\Xi_u ;\mc L_\Xi) = \varinjlim_{K \in
\mr{CO}(G)} C_c^\infty \big( \Xi_u ; \mc L_\Xi^{K \times K} \big)
\]
with the inductive limit topology. The Plancherel theorem for
reductive $p$-adic groups reads:

\begin{thm}\label{thm:2.5} \textup{\cite{HC2,Wal}} \\
The Fourier transform
\[
\mc F : \mc S (G) \to C_c^\infty (\Xi_u ;\mc L_\Xi )^{\mc W}
\]
is an isomorphism of topological algebras.
\end{thm}

A simple representation theoretic consequence of this important theorem is

\begin{cor}\label{cor:2.15}
For any $w \in \mc W$ and $\xi \in \Xi$ such that $w \xi$ is
defined, the $G$-representations $I(\xi )$ and $I(w \xi )$ have the
same irreducible subquotients, counted with multiplicity.
\end{cor}
\emph{Proof.}
By \cite[Corollary 2.3.3]{Cas} we have to show that the traces of $I(\xi )$ and 
$I(w \xi )$ are the same, in other words, that the function
\begin{equation*}
\mc H (G) \times \xnr (M) \to \mh C : (f ,\chi) \mapsto
\mr{tr}\, I(P,A,\omega ,\chi )(f) - 
\mr{tr}\, I (wP,wA,w \omega ,\chi \circ w^{-1})(f)
\end{equation*}
is identically 0. Because this is a polynomial function of $\chi$,
it suffices to show that it is 0 on $\mc H (G) \times \xunr (M)$.
That follows from Theorem \ref{thm:2.5}. $\qquad \Box$
\\[2mm]

The Plancherel theorem can be used to describe the Fourier transform of
$C_r^* (G)$. For $(\omega,E) \in \Delta_M$ let $\mc K (\omega ,P)$
be the algebra of compact operators on the Hilbert space
completion of $I_P^G (E)$. Notice that
\begin{equation}
\mc K (\omega ,P) = \varinjlim_{K \in \mr{CO}(G)} L (\omega ,P)^{K
\times K}
\end{equation}
in the $C^*$-algebra sense, and that the intertwiner $I (n,
\omega)$ extends to $\mc K (\omega ,P)$ because it is unitary. Let
$\mc K_\Xi$ be the vector bundle over $\Xi$ whose fiber at
$(P,A,\omega ,\chi )$ is $\mc K (\omega ,P)$, and let $C_0 (\Xi_u ;
\mc K_\Xi )$ be the $C^*$-completion of
\[
{\ts \bigoplus_{(P,A,\omega)} } C (\xunr (M) ; \mc K (\omega ,P)) \,.
\]
\begin{thm}\label{thm:2.6} \textup{\cite[Theorem 2.5]{Ply}} \\
The Fourier transform extends to an isomorphism of $C^*$-algebras
\[
C_r^* (G) \to C_0 (\Xi_u ; \mc K_\Xi )^\mc W \,.
\]
\end{thm}

We can also describe the images of the subalgebras $\mc S (G,K)$
and $C_r^* (G,K)$ under the Fourier transform.

\begin{thm}\label{thm:2.7}
Fix $K \in \mr{CO}(G)$. There exists a finite set of standard triples 
$(P_i ,A_i ,\omega_i )$ with the following properties.
\begin{description}
\item[a)] The Fourier transform yields algebra homomorphisms
\[
\begin{array}{rrr}
\mc H (G,K) & \to & \bigoplus_{i=1}^{n_K}
  \big( \mc O (\xnr (M_i )) \otimes L(\omega_i ,P_i
  )^{K \times K} \big)^{\mc W_i} \\
\mc S (G,K) & \to & \bigoplus_{i=1}^{n_K}
  \big( C^\infty (\xunr (M_i )) \otimes L(\omega_i ,P_i
  )^{K \times K} \big)^{\mc W_i} \\
C_r^* (G,K) & \to & \bigoplus_{i=1}^{n_K}
  \big( C (\xunr (M_i )) \otimes L(\omega_i ,P_i )^{K \times K} \big)^{\mc W_i}
\end{array}
\]
where $\mc W_i$ is the isotropy group of $(P_i ,A_i ,\omega_i )$
in $\mc W$.
\item[b)] The first map is injective, the second is an isomorphism of Fr\'echet 
algebras and the third is an isomorphism of $C^*$-algebras.
\item[c)] For every $w \in \mc W_i$ there exists a unitary section
\[
u_w \in C^\infty (\xunr (M_i )) \otimes L(\omega_i ,P_i )^{K
\times K}
\]
which extends to a rational section on $\xnr (M_i )$, such that
\end{description}
\begin{equation}\label{eq:2.7}
w f (\chi ) = u_w (\chi ) f (w^{-1} \chi ) u_w^{-1}(\chi ) \qquad
\forall f \in C (\xunr (M_i )) \otimes L(\omega_i ,P_i )^{K
\times K} .
\end{equation}
\end{thm}

\emph{Proof.} The author already proved this result in
\cite[Theorem 10]{Sol1} but we include the proof anyway. Notice
that, in constrast with Proposition \ref{prop:2.1}, it is not necessary
to require that $K$ is ``small", expect for being compact. That is
because the most tricky (namely, not completely reducible)
representations in a Bernstein component $[M,\sigma ]_G$ appear
only if we twist $\sigma$ by a nonunitary character $\chi \in
\xnr (M) \setminus \xunr (M)$.

According to \cite[Th\'eor\`eme VIII.1.2]{Wal} there are only
finitely many components in the Harish-Chandra spectrum $\Omega^t (G)$
on which the idempotent $e_K$ does not act as 0. Pick one triple
$(P_i ,A_i ,\omega_i )$ for each such component. Now a) and b)
follow immediately from Theorems \ref{thm:2.5} and \ref{thm:2.6}.

Concerning c), every automorphism of
\[
L(\omega_i ,P_i )^{K \times K} \cong \mr{End}_{\mh C} \big( I_P^G
(E)^K \big)
\]
is inner, so \eqref{eq:2.7} holds for some section $u_w$. Using
Theorem \ref{thm:2.4} we can arrange that $u_w$ is rational on
$\xnr (M_i )$ and unitary on $\xunr (M_i ). \qquad \Box$
\vspace{4mm}





\section{The Langlands classification}
\label{sec:2.3}

The Langlands classification describes the relation between the
smooth spectrum of $G$ and the tempered spectra of its Levi subgroups.
Let $(P,A)$ be a semi-standard parabolic pair, let $X^* (A)$ be the lattice of 
algebraic characters of $A$ and put 
\[
\mf a^* = X^* (A) \otimes_{\mh Z} \mh R \cong X^* (M) \otimes_{\mh Z} \mh R \,.
\]
For $A = A_0$ and $A = A_G$ we write $\mf a^* = \mf a_0^*$ and 
$\mf a^* = \mf a_G^*$, respectively. There is a natural homomorphism
\begin{equation}
\begin{split}
& H_M : M \to \mathrm{Hom}_{\mh Z} (X^* (M), \mh R) \cong
  \mathrm{Hom}_{\mh R} (\mf a^* ,\mh R ) \,, \\
& q^{\inp{\chi}{H_M (m)}} = \norm{\chi (m)}_{\mh F} 
\qquad \qquad m \in M , \chi \in X^* (M) \,,
\end{split}
\end{equation}
where $q$ is the cardinality of the residue field of $\mh F$. Conversely, for 
$\nu \in \mf a^*$ we define a nonramified character $\chi_\nu$ of $M$ by
\begin{equation}
\chi_\nu (m) = q^{\inp{\nu}{H_M (m)}} .
\end{equation}
This yields an isomorphism 
\[
\mf a^* \cong \mr{Hom}(M, \mh R_{>0}) \,,
\]
where Hom is taken in the category of topological groups.
Let $Q$ be a parabolic subgroup such that $P \subset Q \subset G$,
and let $\mf q$ be its Lie algebra. It decomposes into $A$-eigenspaces
\begin{equation}
\mf q_\alpha := \{ x \in \mf q : \mr{Ad}(a) x = \alpha (a) x \;
\forall a \in A \}
\end{equation}
with $\alpha \in X^* (A)$. The roots of $Q$ with respect to $A$ are
\begin{equation}
\Sigma (Q,A) := \{ \alpha \in X^* (A) \setminus \{ 1 \} : \mf
q_\alpha \neq 0 \} \,.
\end{equation}
The inclusions $A \to A_0$ and $M_0 \to M$ identify $\mf a^*$ as a direct 
summand of $\mf a_0^*$. We have a root system
\[
\Sigma_0 = \Sigma (G,A_0 ) \subset \mf a_0^*
\]
with positive roots $\Sigma (P_0 ,A_0 )$, simple roots $\Delta_0 = \Delta(P_0, A_0)$ 
and Weyl group $W_0 = W (P_0 ,A_0 )$. 
We fix a $W_0$-invariant inner product $\langle \,, \rangle_0$ 
on $\mf a_0^*$, so that we may identify this vector space with its dual. 
Denote the Lie algebras of $A$ and $A_0$ by $\mf a$ and $\mf a_0$. 
(Notice that $\mf a_0$ is not the dual of $\mf a_0^* $, these are vector 
spaces over different fields.) For $F \subset \Delta_0$ we put
\begin{equation}\label{eq:2.8}
\begin{array}{lll@{\qquad}lll}
\Sigma_F & = & \Sigma_0 \cap \mh R F \; \subset \; \mf a_0^* &
W_F & = & \langle s_\alpha : \alpha \in F \rangle \; \subset \; W_0 \,, \\
\mf a_F & = & \{ x \in \mf a_0 : \alpha (x) = 0 \, \forall
\alpha \in F \} & A_F & = & \exp (\mf a_F ) \,, \\
\mf m_F & = & \big( \bigoplus_{\alpha \in \Sigma_F} \mf g_\alpha
\big) \oplus \mf a_0 & M_F & = & Z_G (A_F ) \,, \\
\mf n_F & = & \bigoplus_{\alpha \in \Sigma (P_0 ,A_0 ) \setminus
\Sigma_F} \mf g_\alpha & N_F & = & \exp (\mf n_F ) \,, \\
\mf p_F & = & \big( \bigoplus_{\alpha \in \Sigma (P_0 ,A_0 ) \cup
\Sigma_F} \mf g_\alpha \big) \oplus \mf a_0 & P_F & = & M_F
\ltimes N_F \,.
\end{array}
\end{equation}
Every standard parabolic pair is of the form $(P_F ,A_F )$ for
some $F \subset \Delta_0$. In this situation $F = \Delta (P_F ,A_F )$ 
is the set of nonzero projections of $\Delta_0 \subset \mf a_0^*$ on 
$\mf a_F^*$, and $W(M_F ,A_0 ) = W_F$. In particular a standard parabolic 
pair is completely determined by either of its two ingredients.

Furthermore we introduce the open and closed positive cones in $\mf a^*$:
\begin{align*}
& \mf a^{*,+} = \{ \nu \in \mf a^* : \inp{\nu}{\alpha}_0 > 0
\,\forall \alpha \in \Delta (P,A) \} \,, \\
& \bar{\mf a}^{*,+} = \{ \nu \in \mf a^* : \inp{\nu}{\alpha}_0 \geq
0 \,\forall \alpha \in \Delta (P,A) \} \,.
\end{align*}
Their antidual is the obtuse negative cone in $\mf a_0^*$ : 
%& \mf a_0^{*,-} = \{ \mu \in \mf a_0^* : \inp{\nu}{\mu} < 0 \; 
%\forall \nu \in \mf a_0^{*,+} \} \,, \\
\[
\bar{\mf a}_0^{*,-} := \{ \mu \in \mf a_0^* : \inp{\nu}{\mu} \leq 0 \; 
\forall \nu \in  \bar{\mf a}_0^{*,+} \} \,.
\]
The set of Langlands data $\Lambda^+$ consists of all quadruples
$\lambda = (P,A,\sigma,\nu )$ such that
\begin{itemize}
\item $(P,A)$ is a standard parabolic pair with Levi component $M
= Z_G (A) $,
\item $\sigma \in \mr{Irr}^t (M) $,
\item $\nu \in \mf a^{*,+} $.
\end{itemize}
Given a Langlands datum $\lambda \in \Lambda^+$ we pick a concrete realization of
$\sigma$ and we construct the admissible $G$-representation 
$I (\lambda ) = I_P^G (\sigma \otimes \chi_\nu )$. Notice that $\lambda$ determines
$I(\lambda )$ only modulo equivalence of $G$-representations. 
The classical Langlands classification for reductive $p$-adic groups reads:

\begin{thm}\label{thm:2.8}
\textup{\textbf{a)}} For every $\lambda \in \Lambda^+$ the $G$-representation
$I (\lambda)$ is indecomposable and has a unique irreducible quotient.\\
We call this the Langlands quotient $J (\lambda )$.\\
\textup{\textbf{b)}} For every $\pi \in \mr{Irr}(G)$ there is a unique
$\lambda \in \Lambda^+$ such that $\pi$ is equivalent to $J (\lambda )$.
\end{thm}
\emph{Proof.} See \cite{Kon} or \cite[\S XI.2]{BoWa}. We note that Konno proves
the uniqueness part only modulo $W_0$-conjugacy. However, two $W_0$-conjugate
Langlands data are necessarily equal, which we will show in a more general setting
in Lemma \ref{lem:2.10}. $\qquad \Box$ 
\\[3mm]

The Langlands datum of $\pi \in \mr{Irr}^t (G)$ is simply $(G,A_G ,\pi ,0)$, and conversely
$J(\lambda )$ cannot be tempered if $\nu \neq 0$. For every $\lambda \in \Lambda^+$
the Langlands quotient is maximal among the irreducible constituents of $I(\lambda )$,
in a suitable sense:

\begin{lem}\label{lem:2.9}
Let $\lambda = (P,A,\sigma,\nu)$ and $\lambda' = (P',A',\sigma',\nu')$ be Langlands data,
and suppose that $J(\lambda' ) \in \mr{JH}(I(\lambda ))$.
\begin{description}
\item[a)] $\nu' - \nu \in \bar{\mf a}_0^{*,-} , A' \subset A$ and $P' \supset P$.
\item[b)] If $\nu' = \nu$ then $\lambda' = \lambda$.
\item[c)] $\mr{End}_G (I(\lambda )) = \mh C$.
\end{description}
\end{lem}
\emph{Proof.} a) and b) The statements about $\nu$ and $\nu'$ are 
\cite[Lemma XI.2.13]{BoWa}. From the definition of $\Lambda^+$ we see that $A' \subset A$ 
and $P' \supset P$ whenever $\nu' - \nu \in \bar{\mf a}_0^{*,-}$.\\
c) Let $\phi \in \mr{End}_G (I(\lambda ))$ and write $M(\lambda ) = \ker (I (\lambda) \to 
J (\lambda ))$. By Theorem \ref{thm:2.8} and the above $M(\lambda)$ and $J(\lambda )$ 
do not have any common irreducible constituents. Hence the composition 
\[
M (\lambda) \xrightarrow{\phi} I(\lambda ) \to J(\lambda )
\]
of $\phi$ with the quotient map is zero, and $\phi (M(\lambda )) \subset M(\lambda)$. 
Therefore $\phi$ induces a map $\phi_J \in \mr{End}_G (J (\lambda))$ and, because 
$J(\lambda )$ is irreducible, $\phi_J = \mu \, \mr{Id}_{J(\lambda )}$ for some 
$\mu \in \mh C$. We want to show that
\[
\psi := \phi - \mu \, \mr{Id}_{I(\lambda )} \in \mr{End}_G (I(\lambda ))
\]
equals zero. By construction $\psi (I(\lambda )) \subset M (\lambda )$. Let $V$ be an 
irreducible quotient representation of $M(\lambda )$ and consider the induced map 
$\psi_V \in \mr{Hom}_G (I(\lambda ),V)$. It follows from Theorem \ref{thm:2.8}.a that 
the smooth contragredient representation $\check{I (\lambda )}$ of $I(\lambda )$ has 
exactly one irreducible submodule, which moreover is equivalent to $\check{J(\lambda )}$. 
Since $\check V$ is not equivalent to the contragredient of $J(\lambda )$, we find
\[
0 = \mr{Hom}_G (\check V , \check{I (\lambda )}) \cong \mr{Hom}_G (I (\lambda ),V) \,.
\]
In particular $\psi_V = 0$ and $\psi (I(\lambda)) \subset \ker (M(\lambda) \to V)$. 
Since $M(\lambda )$ has finite length \cite[p. 30]{BeDe}, we conclude with induction that 
\begin{equation}\label{eq:2.19}
\mr{Hom}_G (I(\lambda ),M(\lambda )) = 0 \,.
\end{equation}
Thus $\psi = 0$ and $\phi = \mu \, \mr{Id}_{I(\lambda )} \,. \qquad \Box$
\\[2mm]

We would like to reformulate Lemma \ref{lem:2.9}.b with a condition on $\sigma$ instead
of on $\nu$. To achieve this we will define a variation on the central character of a 
representation. Suppose that $\pi \in \mr{Irr}(G)$ belongs to the Bernstein component 
$\mf s = [M ,\rho ]_G$. We may assume that $M$ is a standard Levi subgroup and that $\rho$ 
is a unitary supercuspidal $M$-representation. Pick $\chi_\pi \in \xnr (M)$ such that 
$\pi \in \mr{JH}(I_P^G (\rho \otimes \chi_\pi ))$, and consider $\log |\chi_\pi | \in \mf a^*$. 
This does not depend on the choice of $\rho$, and by Theorem \ref{thm:2.15} it is unique modulo 
$W_{\mf s}$. However, we could have chosen another standard Levi subgroup $M'$ conjugate to 
$M$. Since $W(A' |G| A) \subset W_0$, this would lead to 
\[
\log |\chi'_\pi | = w \log | \chi_\pi | \in \mf a_0^* \quad \text{for some } w \in W_0 \,.
\]
Thus we get an invariant
\[
\cc_G (\pi ) := W_0 \log |\chi_\pi | \in \mf a_0^* / W_0 \,,
\]
which can be considered as a substitute for the absolute value of the $Z(\mc H (G))$-character 
of $\pi$. We note that, because the inner product on $\mf a_0^*$ is $W_0$-invariant, 
$\norm{\cc_G (\pi )}$ is well-defined. Furthermore the orthogonal projection of $\cc_G (\pi )$ 
on $\mf a_G^*$ consists of a single element, known as the central exponent of $\pi$. If 
$\pi \in \mr{JH}(I_{P_F}^G (\tau ))$ for some $\tau \in \mr{Irr}(M_F)$ then, by the transitivity 
of parabolic induction
\begin{equation}\label{eq:2.21}
\cc_G (\pi ) = W_0 \, \cc_{M_F} (\tau ) \,.
\end{equation}

\begin{lem}\label{lem:2.13}
Let $\lambda = (P,A,\sigma,\nu)$ and $\lambda' = (P',A',\sigma',\nu')$ be different 
Langlands data, and suppose that $J(\lambda' ) \in \mr{JH}(I(\lambda ))$. Then 
\[
\norm{\cc_{M'}(\sigma' )} > \norm{\cc_M (\sigma )} \,.
\]
\end{lem}
\emph{Proof.}
By \eqref{eq:2.21} all irreducible constituents of $I(\lambda )$ have the same 
$\cc_G$-invariant, so $\cc_G (J (\lambda' )) = \cc_G (J(\lambda ))$. We have
\[
\cc_M (\sigma \otimes \chi_\nu ) = \cc_M (\sigma ) + \nu = 
W_M \log |\chi_\sigma | + \nu \in \mf a_0^* / W_M \,.
\]
Since $\sigma$ is irreducible and tempered, it is a unitary $M$-representation 
\cite[Proposition III.4.1]{Wal}, in particular $\log |\chi_\sigma | = 0$ on $A$. 
But $\nu \in \mf a^*$ is zero on the derived group of $M$, 
so $\inp{\log |\chi_\sigma |}{\nu}_0 = 0$. Hence
\[
\norm{\cc_G (J(\lambda ))}^2 = \norm{\cc_M (\sigma )}^2 + \norm{\nu}^2 ,
\]
and similarly
\[
\norm{\cc_G (J(\lambda' ))}^2 = \norm{\cc_{M'} (\sigma' )}^2 + \norm{\nu'}^2 .
\]
By Lemma 2.9 $\nu' - \nu \in \bar{\mf a}_0^{*,-} \setminus \{ 0 \}$, which by 
\cite[Claim 3.5.1]{Kon} implies $\norm{\nu} > \norm{\nu'}$. 
Since $\norm{\cc_G (J(\lambda ))}^2 = \norm{\cc_G (J(\lambda' ))}^2$, we conclude that 
$\norm{\cc_M (\sigma )}^2 < \norm{\cc_{M'} (\sigma' )}^2 . \; \Box$
\vspace{4mm}




\section{Parametrizing irreducible representations}
\label{sec:2.4}

We will combine the Langlands classification with the Plancherel theorem to 
parametrize of the irreducible smooth $G$-representations. This parametrization is not 
complete, in the sense that certain packets contain more than one irreducible representation, 
but we do have some information about their number. Important for our purposes is that this 
parametrization clearly distinguishes tempered and nontempered representations. 
These results were inspired by unpublished work of Delorme and Opdam \cite{DeOp2} 
on affine Hecke algebras.

For $\xi = (P,A,\omega ,\chi ) \in \Xi$ we define
\begin{equation}\label{eq:2.9}
\begin{array}{lll@{\qquad}lll}
\nu (\xi ) & = & \log |\chi| & \Sigma (\xi ) & = & \{ \alpha \in
\Sigma (P,A) : \inp{\nu (\xi )}{\alpha}_0 = 0 \} \,, \\
M(\xi ) & = & Z_G (A(\xi ))& A(\xi ) & = & \{ a \in A : \alpha (a)
= 1 \;\forall \alpha \in \Sigma (\xi) \} \,, \\
P(\xi ) & = & P M(\xi ) & \omega (\xi ) & = & I_{P \cap M(\xi
)}^{M (\xi )} \big(\omega \otimes \chi |\chi |^{-1} \big) \,.
\end{array}
\end{equation}
By \cite[Lemme III.2.3]{Wal} $\omega (\xi )$ is a tempered
pre-unitary $M (\xi )$-representation. For $\xi \in \Xi_u$ we simply have
\[
\nu (\xi ) = 0 \,,\, \Sigma (\xi ) = \Sigma (P,A) \,,\, A(\xi ) = A_G \,,\,
P(\xi ) = M(\xi ) = G \text{ and } \omega (\xi ) = I(\xi ) \,.
\]
For general $\xi$ these objects are designed to divide parabolic induction 
into stages, like in \cite[\S XI.9]{KnVo}.
The first stage corresponds to the unitary part of $\chi$ and the
second stage to its absolute value. This is possible since
\begin{align}
I_{P(\xi )}^G \big( |\chi | \otimes \omega (\xi ) \big) &\cong
\mr{Ind}_{P(\xi )}^G \big( \delta_{P(\xi )}^{1/2} \otimes |\chi |
\otimes \omega (\xi ) \big) \nonumber \\
&\cong \mr{Ind}_{P(\xi )}^G \big( \delta_{P(\xi )}^{1/2} \otimes
|\chi | \otimes \mr{Ind}_{P \cap M(\xi )}^{M(\xi )} \big(
\delta_{P \cap M(\xi)}^{1/2} \otimes \chi |\chi|^{-1} \otimes
\omega \big) \big) \nonumber \\
&\cong \mr{Ind}_{P M(\xi )}^G \big( \delta_{P M(\xi )}^{1/2}
\otimes \mr{Ind}_{P \cap M(\xi )}^{M(\xi )} \big( \delta_{P \cap
M(\xi)}^{1/2} \otimes \chi \otimes \omega \big) \big) \label{eq:2.10} \\
&\cong \mr{Ind}_{P M(\xi )}^G \big(  \mr{Ind}_{P \cap M(\xi
)}^{M(\xi )} \big( \delta_{P M(\xi )}^{1/2} \otimes \delta_{P
\cap M(\xi)}^{1/2} \otimes \chi \otimes \omega \big) \big) \nonumber \\
&\cong \mr{Ind}_P^G \big( \delta_P^{1/2} \otimes \chi \otimes
\omega \big) \quad = \quad I(\xi ) \,. \nonumber
\end{align}
Clearly we can transfer the positivity condition from Langlands data to induction data. 
We say that $\xi = (P,A,\omega ,\chi ) \in \Xi^+$ if $(P,A)$ is standard and 
$\log |\chi| \in \bar{\mf a}^{*,+}$. This choice of a ``positive cone" is justified 
by the following result.

\begin{lem}\label{lem:2.10}
Every $\xi \in \Xi$ is $\mc W$-associate to an element of $\Xi^+$.
If $\xi_1 , \xi_2 \in \Xi^+$ are $\mc W$-associate, then the
objects $\Sigma (\xi_i ) ,\, A(\xi_i ) ,\, M(\xi_i ) ,\, P(\xi_i )$ and 
$\nu (\xi_i )$ are the same for $i=1$ and $i=2$, while $\omega (\xi_1 )$ 
and $\omega (\xi_2 )$ are equivalent $M(\xi_i )$-representations.
\end{lem}
\emph{Proof.} As we noted before, every parabolic pair is
conjugate to a standard one. By \cite[Section 1.15]{Hum} every
$W_0$-orbit in $\mf a_0^*$ contains a unique point in a positive
chamber $\mf a^{*,+}$ (for a unique $A \subset A_0$).  This proves
the first claim, and it also shows that
\begin{equation}
\log |\chi_1 | = \log |\chi_2 | \in \mf a_0^* \,.
\end{equation}
Hence the $\nu$'s, $\Sigma$'s, $A$'s, $P$'s and $M$'s are the same
for $i=1$ and $i=2$. If now $w \in \mc W$ is such that $w \omega_1
\cong \omega_2$, then by Theorem \ref{thm:2.4}, applied to
$M(\xi_i )$, there is a unitary intertwiner between $\omega (\xi_1
)$ and $\omega (\xi_2 ). \qquad \Box$
\\[2mm]

The link between these positive induction data and the Langlands 
classification is easily provided:

\begin{prop}\label{prop:2.11}
Let $\xi = (P,A,\omega ,\chi ) \in \Xi^+$.
\begin{description}
\item[a)] Let $\tau$ be an irreducible direct summand of $\omega (\xi )$.
Then $(P(\xi ), A (\xi), \tau ,\nu (\xi )) \in \Lambda^+$.
\item[b)] The irreducible quotients of $I(\xi )$ are precisely the modules
$J (P(\xi ), A (\xi), \tau ,\nu (\xi ))$ with $\tau$ as above, and these
are tempered if and only if $\xi \in \Xi_u$.
\item[c)] The functor $I_{P(\xi )}^G$ induces an isomorphism
$\mr{End}_{M(\xi )} (\omega (\xi )) \cong \mr{End}_G (I(\xi ))$.
\end{description}
\end{prop}
\emph{Proof.}
a) follows directly from the definitions \eqref{eq:2.9}.\\
b) follows from a) and Theorem \ref{thm:2.8}.\\
c) Let $\sigma$ be an irreducible direct summand of $\omega (\xi )$, which is not 
equivalent to $\tau$ as a $M(\xi )$-representation. By Lemma \ref{lem:2.9}.b 
\[
J(P(\xi ), A (\xi), \tau ,\nu (\xi )) \notin 
\mr{JH} \big( I(P(\xi ), A (\xi), \sigma ,\nu (\xi )) \big) \,.
\] 
The proof of \eqref{eq:2.19}, with $I(P(\xi ), A (\xi), \sigma ,\nu (\xi ))$ 
in the role of $M(\lambda )$, shows that 
\[
\mr{Hom}_G \big( I \big( P(\xi ), A (\xi), \tau ,\nu (\xi ) \big) 
,I \big( P(\xi ), A (\xi), \sigma ,\nu (\xi ) \big) \big) = 0 \,.
\]
Now the result follows easily from Lemma \ref{lem:2.9}.c. $\qquad \Box$
\\[2mm]

As announced, we can parametrize Irr$(G)$ with our induction data.

\begin{thm}\label{thm:2.12}
For every $\pi \in \mr{Irr}(G)$ there exists a unique association class\\ 
$\mc W (P,A,\omega ,\chi ) \in \Xi / \mc W$ such that the following equivalent statements hold:
\begin{description}
\item[a)] $\pi$ is equivalent to an irreducible quotient of $I (\xi^+ )$, for some\\ 
$\xi^+ \in \mc W (P,A,\omega ,\chi ) \cap \Xi^+$.
\item[b)] $\pi \in \mr{JH}(I (P,A,\omega ,\chi ))$ and $\norm{\cc_M (\omega )}$ 
is maximal with respect to this property.
\end{description}
\end{thm}
\emph{Proof.} 
a) Let $(P_\pi ,A_\pi ,\sigma ,\nu )$ be the Langlands datum associated to $\pi$. 
Write $\mc W^H, \Xi^H$ etcetera for $\mc W, \Xi$, but now corresponding to 
$H = M_\pi = Z_G (A_\pi )$ instead of $G$. By Theorem \ref{thm:2.5}, applied to $H$, 
there exists a unique association class
\[
\mc W^H \xi^H = \mc W^H (P_F \cap H ,A_F ,\omega^H ,\chi^H ) \in \Xi^H_u / \mc W^H
\]
such that $\sigma$ is a direct summand of 
$I^H (\xi ) = I_{P_F \cap H}^H (\omega^H \otimes \chi^H )$. Put 
\[
\xi^+ = (P_F ,A_F ,\omega^H, \chi^H \cdot \chi_\nu ) \in \Xi^+ \,.
\] 
By Proposition \ref{prop:2.11}.b $\pi \in \mr{JH}(I (\xi^+ )$), and by Lemma \ref{lem:2.10} and 
Theorem \ref{thm:2.5} the class $\mc W \xi^+ \in \Xi / \mc W$ is unique for this property.\\
b) In view of Corollary \ref{cor:2.15} and Lemma \ref{lem:2.10} we may assume that 
$\xi = (P,A,\omega ,\chi ) \in \Xi^+$. Suppose that $\pi$ is not equivalent to a quotient of 
$I(\xi )$, and let $\tau$ be an irreducible summand of $\omega (\xi )$ such that 
$\pi \in \mr{JH}(I (\lambda ))$, with $\lambda = (P(\xi ) ,A(\xi ), \tau ,\nu (\xi )) \in \Lambda^+$. 
By Lemma \ref{lem:2.13} and \eqref{eq:2.21}
\[
\norm{c_{M_F}(\omega^H)} = \norm{\cc_{M_\pi}(\sigma)} <  
\norm{\cc_{M(\xi )}(\tau)} = \norm{\cc_M (\omega)} \,.
\]
This contradiction proves that $\pi$ must be equivalent to a quotient of $I(\xi )$.
Hence $\xi$ satisfies the condition from a), which shows that $\mc W \xi$ is unique.
In particular conditions a) and b) are equivalent. $\qquad \Box$
\\[2mm]

Every standard triple $(P,A,\omega )$ gives a series of finite length $G$-representations 
$I(P,A,\omega ,\chi )$, parametrized by $\chi \in \xnr (M)$. Every such series lies in a 
single Bernstein component. Let Irr$(G)_{(P,A,\omega )}$ denote the set of all 
$\pi \in \mr{Irr}(G)$ which are equivalent to an irreducible quotient of $I(\xi )$, for some 
$\xi \in \Xi^+ \cap \mc W (P,A,\omega, \xnr (M) )$. Theorem \ref{thm:2.12} tells us that 
these subsets form a partition of Irr$(G)$. We note that
\[
\mr{Irr}^t (G)_{(P,A,\omega )} := \mr{Irr}(G)_{(P,A,\omega )} \cap \mr{Irr}^t (G)
\]
consists of the direct summands of the $I(P,A,\omega ,\chi )$ with $\chi \in \xunr (M)$.
As a topological space Irr$(G)_{(P,A,\omega )}$ is usually not Hausdorff, since certain
$\chi \in \xnr {M}$ carry more than one point. The following result shows that the 
singularities depend only on the $\mc W$-action, and can therefore already be detected on 
$\mr{Irr}^t (G)_{(P,A,\omega )}$.

\begin{lem}\label{lem:2.16}
Suppose that $t \to \xi_t = (P,A,\omega ,\chi_t )$ is a path in $\Xi$, and that 
$\mc W_\xi := \{ w \in \mc W : w (\xi_t ) = \xi_t \}$ does not depend on $t$. Then
$| JH (I(\xi_t )) \cap \mr{Irr}(G )_{(P,A,\omega )} |$ does not depend on $t$.
\end{lem}
\emph{Proof.}
For $\chi_t$ moving within $\xunr (M)$ this property can be read 
off from Theorem \ref{thm:2.7}. Take $n \in \mc W_{\xi}$. According  to 
Theorem \ref{thm:2.4} the intertwiner $I(n, \omega \otimes \chi )$ extends
holomorphically to a tubular neighborhood $U$ of $\xunr (M)$ in $\xnr (M)$.
Hence the desired result also holds for paths with $\chi_t \in U \: \forall t$.

Now consider any path as in the statement. For every $t_0$ there is an 
$r \in (-1,0]$ and a neighborhood $T$ of $t_0$ such that 
\[
\chi'_t := \chi_t |\chi_t |^r \in U \: \forall t \in T \,.
\]
By Proposition \ref{prop:2.11}.b
\[
| \mr{JH} (I(\xi_t )) \cap \mr{Irr}(G )_{(P,A,\omega )} | = 
| \mr{JH} (I(\xi'_t )) \cap \mr{Irr}(G )_{(P,A,\omega )} | \,.
\]
We just saw that the right hand side is independent of $t \in T$, hence so is the 
left hand side. $\qquad \Box$ 
\\[2mm]

With these partitions of Irr$ (G)$ and $\mr{Irr}^t (G)$ we can construct filtrations of
corresponding algebras, which will be essential in the next chapter.

\begin{lem}\label{lem:2.14}
Let $\mf s \in \Omega (G)$ be a Bernstein component and pick $K_{\mf s} \in \mr{CO}(G)$ 
as in Proposition \textup{\ref{prop:2.1}.b.} There exist standard triples 
$(P_i ,A_i ,\omega_i )$ and filtrations by two-sided ideals
\begin{align*}
&\mc H (G, K_{\mf s})^{\mf s} = \mc H^{\mf s}_0 \supset \mc H^{\mf s}_1 
\supset \cdots \supset \mc H^{\mf s}_{n_{\mf s}} = 0 \,, \\ 
&\mc S (G, K_{\mf s})^{\mf s} \, = \mc S^{\mf s}_0 \, \supset \, \mc S^{\mf s}_1 \,
\supset \cdots \supset \mc S^{\mf s}_{n_{\mf s}} \, = 0 \,, 
\end{align*}
such that
\begin{itemize}
\item[a.] $\mc H_i^{\mf s} \subset \mc S_i^{\mf s}$ ,
\item[b.] Prim$ \big( \mc H_{i-1}^{\mf s} / \mc H_i^{\mf s} \big) 
\cong \mr{Irr}(G)_{(P_i ,A_i ,\omega_i )}$ ,
\item[c.] Prim$ \big( \mc S_{i-1}^{\mf s} / \mc S_i^{\mf s} \big) \, 
\cong \mr{Irr}^t (G)_{(P_i ,A_i ,\omega_i )}$ .
\end{itemize}
\end{lem}
\emph{Proof.}
By \eqref{eq:2.3} $\mc S (G, K_{\mf s})^{\mf s}$ is a direct summand of 
$\mc S (G, K_{\mf s})$, so Theorem \ref{thm:2.7} assures the existence of finitely many 
(say $n_{\mf s}$) standard triples $(P_i ,A_i ,\omega_i )$ such that
\begin{equation}\label{eq:2.11}
\mc S (G,K_{\mf s})^{\mf s} \cong {\ts \bigoplus_i} \big( C^\infty (\xunr (M_i )) \otimes 
L(\omega_i ,P_i )^{K_{\mf s} \times K_{\mf s}} \big)^{\mc W_i} .
\end{equation}
In view of Theorem \ref{thm:2.15} Prim$ \big( \mc H (G, K_{\mf s})^{\mf s} \big)$ is a union 
of series corresponding to standard triples. Looking at the unitary parts of these series 
and at \eqref{eq:2.11}, we find that
\begin{equation}\label{eq:2.18}
\mr{Prim} \big( \mc H (G, K_{\mf s})^{\mf s} \big) \cong 
{\ts \bigcup_i} \mr{Irr}(G )_{(P_i ,A_i ,\omega_i )} \,.
\end{equation}
Number the triples $(P_i ,A_i ,\omega_i )$ from 1 to $n_{\mf s}$, such that 
\[
\norm{\cc_{M_j} (\omega_j )} \geq \norm{\cc_{M_i} (\omega_i )} \quad \mr{if} \quad j \leq i \,. 
\]
We define
\begin{equation}\label{eq:2.22}
\begin{array}{lll}
\mc H^{\mf s}_i & = & \{ h \in \mc H (G, K_{\mf s})^{\mf s} : \pi (h) = 0 \; \forall \pi \in 
\mr{Irr}(G)_{(P_j ,A_j ,\omega_j )} \,, \forall j \leq i \} \,, \\
\mc S^{\mf s}_i & = & \{ h \in \mc S (G, K_{\mf s})^{\mf s} : \pi (h) = 0 \; \forall \pi \in 
\mr{Irr}^t (G)_{(P_j ,A_j ,\omega_j )} \,, \forall j \leq i \} \,.
\end{array}
\end{equation}
Since the Jacobson closure of $\mr{Irr}^t (G)_{(P_j ,A_j ,\omega_j )}$ in Prim$ (\mc H (G))$ 
contains $\mr{Irr} (G)_{(P_j ,A_j ,\omega_j )} $, we have
$\mc H^{\mf s}_i \subset \mc S^{\mf s}_i$. From \eqref{eq:2.11} we get
\begin{equation}\label{eq:2.23}
 \mc S_{i-1}^{\mf s} / \mc S_i^{\mf s} \cong \big( C^\infty (\xunr (M_i )) \otimes 
L(\omega_i ,P_i )^{K_{\mf s} \times K_{\mf s}} \big)^{\mc W_i} ,
\end{equation}
which shows that c) holds. We claim that
\begin{equation*}
{\ts \bigcup_{j \leq i} } \mr{Irr}(G)_{(P_j ,A_j ,\omega_j )}
\end{equation*}
is closed in the Jacobson topology of $\mc H (G, K_{\mf s})^{\mf s}$.
Its Jacobson closure consists of all irreducible subquotients $\pi$ of 
$I(\xi_j ) = I (P_j , A_j ,\omega_j ,\chi )$, for any $j \leq i$ and $\chi \in \xnr (M_j )$.
Suppose that $\pi \not\in \mr{Irr}(G)_{(P_j ,A_j ,\omega_j )}$. By Theorem \ref{thm:2.12}
$\pi \in \mr{Irr}(G)_{(P ,A ,\omega )}$ for some standard triple with
$\norm{\cc_M (\omega )} > \norm{\cc_{M_j} (\omega_j )}$. In view of \eqref{eq:2.18} we must 
have $w(P,A,\omega) = (P_n ,A_n ,\omega_n )$ for some $w \in \mc W$ and $n \in \mh N$. Then
\[
\norm{\cc_{M_n}(\omega_n )} = \norm{\cc_M (\omega )} > \norm{\cc_{M_j} (\omega_j )} \,,
\]
so $n < j \leq i$, proving our claim. Consequently
\[
\mr{Prim} \big( \mc H_i^{\mf s} \big) \cong 
{\ts \bigcup_{j > i} } \mr{Irr}(G)_{(P_j ,A_j ,\omega_j )} \,,
\]
which easily implies b). $\qquad \Box$ 