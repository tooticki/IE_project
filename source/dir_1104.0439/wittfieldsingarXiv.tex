
\documentclass{amsart}
%\documentclass{amsart}
%\documentclass{tglat2e}%%   tglat calls the Class of our Journal.
\parskip 5pt
\textwidth 5in

\usepackage{latexsym, amsthm, amssymb}
%in amsart, add amssymb, amsmath}%, makeidx}
%\makeindex

\usepackage[all]{xy}
%See website http://en.wikibooks.org/wiki/LaTeX/Creating_Graphics#XY-pic
%http://www.tug.org/pracjourn/2006-4/blaga/blaga.pdf

\usepackage[usenames, dvipsnames]{color}
%See website http://people.oregonstate.edu/~peterseb/tex/samples/color-package.html for
%a list of the many color options.  Or, http://en.wikibooks.org/wiki/LaTeX/Colors
%To invoke, use {\color{red}<text>}
%To make a new color, use \definecolor{orange}{rgb}{1,0.5,0}  Here rgb is red green blue.

%%%%% from Asher
\usepackage{mathrsfs} %nice script font
%%% UV - changes in header: %'d \usepackage{mathabx}, \def\divides, \def\notdivides.
%\usepackage{mathabx} % get \notdivides

\theoremstyle{plain}
\newtheorem{Theorem}[equation]{Theorem}
\newtheorem{Lemma}[equation]{Lemma}
\newtheorem{Corollary}[equation]{Corollary}
\newtheorem{Conjecture}[equation]{Conjecture}
\newtheorem*{Conjecture*}{Conjecture}
\newtheorem{Proposition}[equation]{Proposition}

\theoremstyle{definition}
\newtheorem{Definition}[equation]{Definition}
\newtheorem{Example}[equation]{Example}
\newtheorem{Examples}[equation]{Examples}
\newtheorem{Problem}[equation]{Problem}
\newtheorem{Paragraph}[equation]{}

%%   NEW! This is an example of a new environment. It resembles
%% theorem-like environments in usage and definition.
%% You type \newdefinition{<def_name>}{<Definition>} in a preamble
%% or in your own style file and then use automatically numbering
%% environment <def_name> for `<Definition> NN' in text.
%% \newexample is equal to \newdefinition since they
%% produce identical environments. Use whatever you like.
%%   We included some common definitions for convenient typesetting.
%% You can use the following environments:
%%  definition for Definition 57.
%%  example for Example 57.
%%  theorem for Theorem 57.
%%  lemma for Lemma 57.

\theoremstyle{remark}
\newtheorem{Remark}[equation]{Remark}

\numberwithin{equation}{section}%If you remove this, the numbering will be sequential throughout.

%%   NEW! No-numbered environments `remark' and `demo' invented.
%% Used with optional argument they produce it in a suitable way.
%% If no they produce standard `Remark.' and `Proof.' text.
%% Be careful not to use [ right after \begin{remark/demo} ---
%% or protect it stating an empty group in the following way: {}[.

%% Do NOT redefine one- and two-character LaTeX commands
%% (like "\r", "\O", "\L", "\AA", etc.)!


\newcommand{\h}{{\text{\rm h}}}

\newcommand{\A}{{\mathbb A}}
\newcommand{\C}{{\mathbb C}}
\newcommand{\F}{{\mathbb F}}
\newcommand{\Ga}{{\text{\rm G}_{\text{\rm a}}}}
\newcommand{\Gm}{{\text{\rm G}_{\text{\rm m}}}}
\renewcommand{\H}{{\text{\rm H}}}
\newcommand{\K}{{\text{\rm K}}}
\newcommand{\M}{{\text{\rm M}}}
\newcommand{\N}{{\mathbb N}}
\renewcommand{\O}{{\text{\rm O}}}
\newcommand{\Nm}{{\text{\rm N}}}
\renewcommand{\P}{{\mathbb P}}
\newcommand{\Q}{{\mathbb Q}}
\newcommand{\R}{{\mathbb R}}
\renewcommand{\S}{{\mathcal S}}
\newcommand{\T}{{\Cal S_Y}}
\newcommand{\U}{{\Cal S_Z}}
\newcommand{\W}{{\text{\rm W}}}
\newcommand{\X}{{\text{\rm X}}}
\newcommand{\Z}{{\mathbb Z}}

\newcommand{\ab}{{\text{\rm ab}}}
\newcommand{\alg}{{\text{\rm alg}}}
\newcommand{\alt}{{\text{\rm alt}}}
\newcommand{\br}[1]{{\left<{#1}\right>}}
\newcommand{\cd}{{\text{\rm cd}}}
\newcommand{\car}{{\text{\rm char}}}
\newcommand{\codim}{{\text{\rm codim}}}
\newcommand{\cold}{{\text{\rm c}}}
\newcommand{\coker}{{\text{\rm coker}}}
\newcommand{\cok}{{\text{\rm cok}}}
\newcommand{\cor}{{\text{\rm cor}}}
\newcommand{\cont}{{\text{\rm cont}}}
\newcommand{\cs}{{\text{\rm cs}}}
\renewcommand{\det}{{\text{\rm det}}}
\newcommand{\df}{{\,\overset{\text{\rm df}}{=}\,}}
\newcommand{\diag}{{\text{\rm diag}}}
\renewcommand{\div}{{\text{\rm div}}}
\renewcommand{\dim}{{\text{\rm dim}}}
\newcommand{\edge}[1]{{\overline{#1}}}
%\renewcommand{\endpf}{{\hfill $\blacksquare$}}
\renewcommand{\exp}{{\text{\rm exp}}}
\newcommand{\et}{{\text{\rm \'et}}}
\renewcommand{\gcd}{{\text{\rm gcd}}}
\newcommand{\gldim}{{\text{\rm glob\,dim}\,}}
\newcommand{\good}{{\text{\rm good}}}
\newcommand{\gr}{{\text{\rm gr}}}
\newcommand{\height}{{\text{\rm ht}\,}}
\newcommand{\id}{{\text{\rm id}}}
\newcommand{\im}{{\text{\rm im}}}
\newcommand{\ind}{{\text{\rm ind}}}
\renewcommand{\inf}{{\text{\rm inf}\,}}
\newcommand{\init}{{\text{\rm init}}}
\newcommand{\inv}{{\text{\rm inv}}}
\newcommand{\isim}{{\;\overset{\sim}{\longrightarrow}\;}}
\newcommand{\isom}{{\;\simeq\;}}
\renewcommand{\ker}{{\text{\rm ker}}}
\newcommand{\lcm}{{\text{\rm lcm}}}
\newcommand{\length}{{\text{\rm length}}}
\newcommand{\mmu}{{\mbox{\boldmath $\mu$}}}
\newcommand{\mult}{{\text{\rm mult}}}
\newcommand{\ndiv}{{\,\not\big |\,}}
\newcommand{\normal}{{\triangleleft\;}}
\newcommand{\nr}{{\text{\rm nr}}}
\newcommand{\ns}{{\text{\rm ns}}}
\newcommand{\onto}{{\,\twoheadrightarrow\,}}
\newcommand{\op}{\circ}
\newcommand{\ord}{{\text{\rm ord}}}
\newcommand{\per}{{\text{\rm per}}}
\newcommand{\pf}{{\noindent{\it Proof.}\;\;}}
\newcommand{\prob}[1]{{\noindent{\bf{#1}}}}
\newcommand{\rad}{{\text{\rm rad}}}
\newcommand{\ram}{{\text{\rm ram}}}
\newcommand{\ramloc}{{\text{\rm ram.loc.}}}
\newcommand{\red}{{\text{\rm red}}}
\newcommand{\reg}{{\text{\rm reg}}}
\newcommand{\res}{{\text{\rm res}}}
\newcommand{\rk}{{\text{\rm rk}}}
\newcommand{\sep}{{\text{\rm sep}}}
\newcommand{\sgn}{{\text{\rm sgn}}}
\newcommand{\sh}{{\text{\rm sh}}}
\newcommand{\supp}{{\text{\rm supp}}}
\newcommand{\tor}{{\text{\rm tor}}}
\newcommand{\tr}{{\text{\rm t}}}
\newcommand{\trace}{{\text{\rm tr}}}
\newcommand{\vnr}{{\text{\rm $v$-nr}}}
\newcommand{\wt}{{\text{\it wt}}}
\newcommand{\zar}{{\text{\rm Zar}}}

\newcommand{\Alt}{{\text{\rm Alt}}}
\newcommand{\Ann}{{\text{\rm Ann}}}
\newcommand{\Ass}{{\text{\rm Ass}}}
\newcommand{\Aut}{{\text{\rm Aut}}}
\newcommand{\Bil}{{\text{\rm Bil}}}
\newcommand{\Br}{{\text{\rm Br}}}
\newcommand{\Brdim}{{\text{\rm Br.dim}}}
\newcommand{\Bl}{{\text{\rm Bl}}}
\newcommand{\CaCl}{{\text{\rm CaCl}\,}}
\newcommand{\Cl}{{\text{\rm Cl}\,}}
\newcommand{\Cong}{{\text{\rm Cong}}}
\newcommand{\Div}{{\text{\rm Div}\,}}
\newcommand{\End}{{\text{\rm End}}}
\newcommand{\Exp}{{\text{\rm Exp}}}
\newcommand{\Ext}{{\text{\rm Ext}}}
\newcommand{\Frac}{{\text{\rm Frac}\,}}
\newcommand{\Gal}{{\text{\rm Gal}}}
\newcommand{\GL}{{\text{\rm GL}}}
\newcommand{\Hom}{{\text{\rm Hom}}}
\newcommand{\II}{{\Cal I\!\Cal I}}
\newcommand{\Jac}{{\text{\rm Jac}}}
\newcommand{\Log}{{\text{\rm Log}}}
\newcommand{\Norm}{{\text{\rm N}}}
\newcommand{\Ob}{{\text{\rm Ob}}}
\newcommand{\Pf}{{\noindent{\it Proof.}\;\;}}
\newcommand{\PGL}{{\text{\rm PGL}}}
\newcommand{\Pic}{{\text{\rm Pic}\,}}
\newcommand{\Princ}{{\text{\rm Princ}\,}}
\newcommand{\Proj}{{\text{\rm Proj}\,}}
\newcommand{\Rev}{{\text{\rm Rev}}}
\newcommand{\SB}{{\text{\rm SB}}}
\newcommand{\Sch}{{\Cal S\!\text{\it ch}}}
\newcommand{\SK}{{\text{\rm SK}}}
\newcommand{\SL}{{\text{\rm SL}}}
\newcommand{\SO}{{\text{\rm SO}}}
\newcommand{\Supp}{{\text{\rm Supp}}}
\newcommand{\Spec}{{\text{\rm Spec}\,}}
\newcommand{\Sym}{{\text{\rm Sym}\,}}
\newcommand{\Tor}{{\text{\rm Tor}}}
\newcommand{\Zar}{{\text{\rm Zar}}}

\newcounter{listcount}


%\setcounter{secnumdepth}{2}
\begin{document}
%%   Here starts the topmatter.
%% Text must not happen before \maketitle command!


\title[Tame Covers and Cohomology of Relative Curves]
{Tame Covers and Cohomology of Relative Curves
over Complete Discrete Valuation Rings
with Applications to the Brauer Group
}
%%   NEW! Title now has optional argument like other sectioning commands.
%% It will appear in running heads on even pages exept for the first one.
%%   The second mandatory argument produces the full title of your paper.
%% It will appear in running head in case you omit the optional one.
%% This title consists of two lines, add extra \\ to get more lines.
%% We think it is a good idea to avoid math formulas in the title.
%% If any they will be boldface.


\author{Eric Brussel and Eduardo Tengan}
\address 
{Department of Mathematics \& Computer Science\\
Emory University\\
Atlanta, GA 30322\\ USA}
\email{brussel@mathcs.emory.edu}
\address
{Instituto de Ci\^encias Matem\'aticas e de Computa\c c\~ao\\
Universidade de S\~ao Paulo\\
S\~ao Carlos, S\~ao Paulo\\ Brazil}
\email
{etengan@icmc.usp.br}

\subjclass{11G20, 11R58, 12G05, 16K20, 16K50}

%%   This is the example of two-authors article. Separate them by \and.
%%   NEW! Function of \address and \email is similar to that in amsppt.
%% But the usage is different: here you place them INSIDE the argument
%% of \author. \address and \email produce new line automatically, so
%% don't put \\ before them.
%%   Note that if you have thanks, they should go exactly after the
%% authors name preceding the \address. If your paper has many authors,
%% please try to make the same number of lines of the address for all
%% authors.

%\dedicatory{Include a dedication?}
%%   NEW! Dedication is a best way to impress your gratitudes and sympathies.


%\received{???} \accepted{September 22, 2010}
%%%   It's NEW! but you'll never deal with it, it's a headache of
%%% your publisher.


\begin{abstract}
We prove the existence of noncrossed product division algebras and indecomposable
division algebras of unequal period and index over the function field of any $p$-adic
curve, extending the results and methods of \cite{BMT}.
\end{abstract}
%%   We insist on abstract in your paper!

\maketitle

\section{Introduction}

We study the cohomology and the Brauer group of a field $F$ that is finitely generated
and of transcendence degree one over the $p$-adic field $\Q_p$.
Such a field is always the function field of a regular (projective, flat) relative curve $X/\Z_p$.
In \cite{BMT} it was shown that if $F$ admits a {\it smooth} model
$X/\Z_p$ then there exist noncrossed product $F$-division algebras,
and indecomposable $F$-division algebras of unequal prime-power period and index.
These were constructed from objects defined over the generic point $(p)$ of the closed
fiber $X_0=X\otimes_{\Z_p}\F_p$ using a
homomorphism $\lambda:\Br(F_p)'\to\Br(F)$ that splits the restriction map
$\res:\Br(F)\to\Br(F_p)$, where the subscript $p$ denotes completion with respect to 
the valuation defined by $(p)$, and ``$\;'\;$'' denotes ``prime-to-$p$ part''. 
The fields $F$ that have a smooth $X/\Z_p$ include fields such as $\Q_p(t)$ 
but do not include the function
fields of {\it all} $p$-adic curves.

In this paper we generalize the machinery and results of \cite{BMT} to arbitrary $p$-adic curves.
We prove that if $F$ is the function field of a $p$-adic curve then there exist
noncrossed product $F$-division algebras, and indecomposable $F$-division algebras
of unequal prime-power period and index.  The machinery we develop here is used in \cite{BT11b}
to prove that every $F$-division algebra of prime period $\ell$ and index $\ell^2$ decomposes 
into two cyclic $F$-tensor factors, 
hence is a crossed product, generalizing Suresh's result \cite{Sur10},
which assumes roots of unity.  
In the terminology of \cite[Sections 3,4]{ABGV},
this shows the $\Z/\ell$-length in $\H^2(F,\mu_\ell)$ equals the $\ell$-Brauer dimension,
which is two by a theorem of Saltman (\cite[Theorem 3.4]{Sa97}).
In general our work is motivated by the work of Saltman over these fields in \cite{Sa97} and \cite{Sa07}
(see also \cite{Br10}).

We summarize the technical results.
Let $R$ be a complete discrete valuation ring with fraction field $K$,
and let $F$ be a finitely generated field extension of $K$ of transcendence degree one.
Let $X/R$ be a regular (projective, flat) model for $F$
whose closed fiber $X_0$ has normal crossings on $X$.
Let $C=X_{0,\red}$, let $\{C_i\}$ denote the set of irreducible components of $C$,
let $\S$ be the set of singular points of $C$, 
and set $F_C=\prod_i F_{C_i}$, the product of
the completions with respect to the valuations on $F$ defined by the $C_i$.
%Let $\Lambda$ denote the twisted \'etale sheaf $\Z/n(r)$ for a number $n$ invertible on $X$
%and an integer $r$.
We construct for any prime-to-$\car(\kappa(C))$ number $n$, 
any integer $r$, and any $q\geq 0$, a homomorphism
$$
\lambda:\H^q(\O_{C,\S},\Z/n(r))\to\H^q(\O_{X,\S},\Z/n(r))\to\H^q(F,\Z/n(r))
$$
that splits the restriction map $\H^q(\O_{X,\S},\Z/n(r))\to\H^q(\O_{C,\S},\Z/n(r))$.
We use $\lambda$ to construct another map $\lambda$ that splits the subgroup
of the image of $\H^q(F,\Z/n(r))\to\H^q(F_C,\Z/n(r))$
consisting of tuples of classes that are unramified at $\S$ and glue along $\S$.
When $q=2$, $\Z/n(r)=\mu_n$, and $R=\Z_p$, we show that $\lambda$ preserves index.
This allows us to construct indecomposable
$F$-division algebras and noncrossed product $F$-division algebras as mentioned above,
in the same manner as \cite{BMT}.
When the dual graph of the closed fiber $X_0$ has nontrivial topology, i.e., 
nonzero (first) Betti number, we construct cyclic covers of $X$
that are (defined and) trivial at every point of $X$ except the generic point of $X$.
These arise as cyclic covers of the closed fiber $X_0$ that are trivial at every point, and transported
to $X$ via $\lambda$.
When $R=\Z_p$ they are the completely split cyclic covers of 
Saito (\cite{Sai85}).
We thank Colliot-Th\'el\`ene for drawing our attention to these interesting specimens.

\section{Background and Conventions}

We use \cite[Chapter 8,9]{Liu},
\cite[Section 2]{GM}, and \cite[Chapter XIII]{SGA1} for many of the following definitions.

\Paragraph{General Conventions.}
Let $S$ be an excellent scheme, $n$ a number that is invertible on $S$,
and $\Lambda=(\Z/n)(i)$ the twisted \'etale sheaf.
We write $\H^q(S,\Lambda)$ for the \'etale cohomology group,
and if $\Lambda$ is understood (or doesn't matter)
we write $\H^q(S,r)$ instead of $\H^q(S,\Lambda(r))$,
and $\H^q(S)$ in place of $\H^q(S,0)$.
If $S=\Spec A$ for a ring $A$ then we 
write $\H^q(A,r)$.
If $T$ is an integral scheme contained in $S$ we write $\kappa(T)$ for its function field. 
If $T\to S$ is a morphism of schemes then
the restriction $\res_{T|S}:\H^q(T)\to\H^q(S)$ is defined,
and we write $\beta_S=\res_{T|S}(\beta)$.  
If $Z\subset S$ is a subscheme, we write $Z_T$ for the preimage $Z\times_S T$.


\Paragraph{Valuation Theory.}
If $v$ is a valuation on a field $F$ we write $\kappa(v)$ for
the residue field of the valuation ring $\O_v$, and $F_v$ for the completion of $F$
at $v$.  
If $S$ is a connected normal scheme with function field $F$ 
and $v$ arises from a prime divisor $D$ on $S$,
we write $v_D$ for $v$, $\kappa(D)$ for $\kappa(v)$, and $F_D$ for $F_v$.
If $D$ is a sum of prime divisors $D_i$ we write
$F_D=\prod_i F_{D_i}$.
Each $f\in F^*$ defines a divisor $\div(f)=\sum v_D(f)D$,
where the (finite) sum is over prime divisors on $S$.

Recall that if $F=(F,v)$ is a discretely valued field and $\alpha\in\H^q(F,\Lambda)$
then $\alpha$ has a {\it residue} $\partial_v(\alpha)$ in $\H^{q-1}(\kappa(v),\Lambda(-1))$.
%We say $\alpha$ is {\it unramified} with respect to $v$ if $\partial_v(\alpha)=0$,
%and in that case the {\it value} of $\alpha$ at $v$ is
%the element $\alpha(v)=\res_{F|F_{v}}(\alpha)\in\H^q(\kappa(v),\Lambda)\leq\H^q(F_v,\Lambda)$
%(\cite{GMS, 7.13, p.19}).
%If $v$ arises from a prime divisor $D$ on a scheme, we will substitute the notations
%$\partial_D$ and $\alpha(D)$.
%
More generally,
suppose $T$ is a noetherian scheme,
$\xi$ is a generic point of $T$, and $\alpha\in\H^q(T,\Lambda)$.
Then for each discrete valuation $v$ on the field $F=\kappa(\xi_\red)$
$\alpha$ has a residue 
$$
\partial_v(\alpha)\df\partial_v(\alpha_F)\in\H^{q-1}(\kappa(v),\Lambda(-1))
$$
We say $\alpha$ is {\it unramified at $v$} if $\partial_v(\alpha)=0$, {\it ramified at $v$}
if $\partial_v(\alpha)\neq 0$, and {\it tamely ramified at $v$} if $\partial_v(\alpha)$
is contained in the prime-to-$\text{\rm char}(\kappa(v))$ part of $\H^{q-1}(\kappa(v),\Lambda(-1))$.
If $\alpha$ is unramified at $v$ the {\it value} of $\alpha$ at $v$ is the
element 
$$
\alpha(v)=\res_{F|F_v}(\alpha_F)\in\H^q(\kappa(v),\Lambda)
\leq\H^q(F_v,\Lambda)
$$
(see \cite[7.13, p.19]{GMS}).
Suppose $T\to S$ is a birational morphism of noetherian schemes (see \cite[I.2.2.9]{EGAI}).
%Then $T$ and $S$ have the same generic points, and the same local rings.
The {\it ramification locus of $\alpha$ on $S_\red$}
is the sum of the prime divisors on $S_\red$ that determine valuations at
which $\alpha$ is ramified, over all generic points of $S_\red$.

Let $D$ be a divisor on a noetherian normal
scheme $S$, set $U=S-D$, and for each generic point $\xi$ of $\Supp\,D$, let $K_\xi$
denote the fraction field of the discrete valuation ring $\O_{S,\xi}$.
We say a morphism $\rho:T\to S$
is {\it tamely ramified along $D$} if $T$ is normal, $\rho_U:V=T\times_S U\to U$ is \'etale,
and for each generic point $\xi$ of $\Supp\,D$, the \'etale $K_\xi$-algebra $L$ defined by
$\Spec L=V\times_U\Spec K_\xi$ is tamely ramified with respect to $\O_{S,\xi}$.
Since $L/K_\xi$ is \'etale it is a finite product of separable field extensions
of $K_\xi$, and $L$ is tamely ramified if each field extension is tamely ramified 
(with respect to $\O_{S,\xi}$) in the usual sense.
If $S$ is a noetherian scheme whose irreducible components are normal, we'll
say a morphism $\rho:T\to S$ is tamely ramified along $D$ if again $V\to U=S-D$ is \'etale,
and the restriction $\rho_{S_i}$
to each irreducible component $S_i$ of $S$ is tamely ramified along $D_{S_i}$.
If $S=\Spec A$ and $T=\Spec B$, we will also say $B$ is a {\it tamely ramified $A$-algebra}.
We say a map $\rho:T\to S$ of noetherian schemes 
is a {\it cover} if it is finite, generically \'etale, 
and each connected component of $T$ dominates a connected component of $S$.

%We say a cover is (generically) {\it $G$-Galois} for a finite group $G$ if $Y$
%admits a $G$-action that is simple and transitive on the geometric generic fibers.


%If $X/R$ is a normal relative curve and $\rho:Y\to X$ is a cover, 
%then $\dim\,Y=\dim\,X=2$, and $\rho$ preserves dimension
%on irreducible closed subschemes of $Y$ (\cite[2.5.10(b)]{Liu}).
%%note $X$ and $Y$ are integral, so $\rho$ is dominant if and only if all $A\to B$ are 1-1,
%%and then we apply Liu.

\Paragraph{Relative Curves.}\label{setup}
In this paper, $R$ will be a complete discrete valuation ring with residue field $k$
and fraction field $K$, 
$F$ will be a field finitely generated of transcendence degree one over $K$,
and $X/R$ will be
a regular 2-dimensional scheme $X$ that is flat and projective over $\Spec R$
and has function field $K(X)=F$.  We call $X/R$ a {\it regular relative curve}, 
%Note it is then finitely presented, since a projective morphism is of finite type,
%and $X$ is noetherian since it is proper over a noetherian scheme.
write $X_0=X\otimes_R k$ for the closed fiber,
$C=X_{0,\red}$ for the reduced subscheme underlying the closed fiber, 
and $C_1,\dots,C_m$ for the irreducible components of $C$.
We assume each $C_i$ is regular, and at most two of them meet at any closed point of $X$,
a situation that can always  be obtained by blowing up using Lipman's embedded resolution theorem
(see \cite[9.2.4]{Liu}).
For all closed points $z\in X$, we have $\dim\O_{X,z}=2$ by \cite[8.3.4(c)]{Liu},
and since $X$ is regular, $\O_{X,z}$ is factorial by Auslander-Buchsbaum's theorem.

We say an effective divisor $D$ on a relative curve $X/R$ is {\it horizontal}
if each of its irreducible components maps surjectively (hence finitely) to $\Spec R$,
%we use projective for the implication to finite.
and {\it vertical} if its support is contained in the support of the closed fiber.
If $D$ is a reduced and irreducible horizontal divisor then it is flat over 
$\Spec R$, since $R$ is a discrete valuation ring.
Every effective divisor on a 
relative curve $X/R$ is a sum of horizontal and vertical divisors,
and the horizontal prime divisors are exactly the closures of the closed points of the generic fiber
(\cite[8.3.4(b)]{Liu}).
Since $R$ is henselian, each irreducible horizontal divisor has a single closed point.

\Paragraph{Distinguished Divisors.}\label{distinguished}
In general there will be many horizontal divisors on a relative curve $X$
that restrict to a given divisor on $C$.  
In order to construct our lifts of covers and cohomology classes
from $C$ to $X$ we select a single regular horizontal divisor for each closed point, as follows.

\Proposition\label{distingprop}
Assume the setup of \eqref{setup}.
Then for each closed $z\in X$ there exists a regular irreducible horizontal divisor $D\subset X$
that intersects each irreducible component of $C$ passing through $z$ transversally at $z$.
\rm

\begin{proof}
Transversal intersection with a single component is by
\cite[8.3.35(g)]{Liu} and its proof (see also \cite[21.9.12]{EGAIV:d}).
Thus if $z\in C_i\cap C_j$ ($i\neq j$) and $t_i$ and $t_j$ are local equations for $C_i$ and $C_j$,
then we have local equations $f_i$ and $f_j$ for effective regular horizontal divisors
such that $(f_i,t_i)=(f_j,t_j)=\frak m_z\subset\O_{X,z}$.
If $(f_j,t_i)=\frak m_z$ or $(f_i,t_j)=\frak m_z$ then a suitable $D$ is defined locally
by $f_j$  or $f_i$.  Otherwise $(f_i+f_j,t_i)=(f_i+f_j,t_j)=\frak m_z$,
and we define $D$ locally by $f_i+f_j$.
The rest of the proof proceeds as in \cite[3.3.35]{Liu}. 
%Note $D\cap C$ is not a closed point if it is a singular point of $C$.
\end{proof}\rm

We fix a set of these (prime) divisors, 
and let $\mathscr D$ denote the set of supports of the semigroup they generate in $\Div X$.
We will say a divisor $D$ is {\it distinguished}
and write $D\in\mathscr D$ whenever $D$ is reduced and supported in $\mathscr D$.
Though $\mathscr D$ is fixed in principle, we reserve the right to declare any
divisor satisfying the definition to be a member of this set retroactively.
Let $\mathscr D_\S$ denote the subset that {\it avoids $\S$}.
Note that each $D\in\mathscr D$ is a disjoint union of its irreducible components,
each of which meets each irreducible component of $C$
transversally.
%Let $\mathscr D_0$ denote the set of restrictions $D\times_X X_0$ for $D\in\mathscr D$,
%and let $\mathscr D_{0,\red}$ denote the divisors on $X_{0,\red}$.
%If $X_0$ is not reduced, of course, then its irreducible (Cartier) divisors $D_0$ are not 
%(reduced) closed points of $X_0$,
%and the map $\Div(X_0)\to\Div(X_{0,\red})$ is not injective.


%If $D$ is a horizontal prime divisor on $X$ that avoids $\S$,
%and $D_0$ is an irreducible component of $D\cap X_0$, then
%since $X$ is regular with normal crossings,
%the ideal of $D_{0,\red}$ in $\O_{X_0,D_{0,\red}}$ is generated by a regular local parameter for $X_{0,\red}$ 
%together with a regular local parameter for $D$, whereas the ideal of $D_0=D\cap X_0$ and
%$D_{0,\red}=D\cap X_{0,\red}$ is generated by (the images of) the local parameter for $D$.

%For example, $k[x,y]/(x^2)$ is the double $y$-axis.  The origin is generated by $(x,y)$
%on $X$, and by $(\bar x,y)$ on $X_0$.  The point $D\cap X_0$ on $X$ has ideal $(x^2,y)$,
%a ``fat point'', not a closed point.  The point $D\cap X_{0,\red}$ is $(x,y)$.
%To check if a scheme is regular, we check its closed
%points, so, that would be the point $D|_{X_{0,\red}}$, whose max ideal is $(\bar x)$ on $k[x,y]/(y)$.

\section{Structure of Tame Covers}


\Lemma[Structure]\label{structure}
Assume the setup of (\ref{setup}).
Suppose $\rho:Y\to(X,D)$ is a tamely ramified cover, where $D\in\mathscr D$.
Then 
\begin{enumerate}
\item[a)]
The structure map $\rho:Y\to X$ is flat.
\item[b)]
$Y/R$ is a regular relative curve, $Y_{0,\red}=C_Y$,
each irreducible component of $C_Y$ is regular,
$\S_Y$ is the set of singular closed points of $C_Y$,
and exactly two irreducible components of $C_Y$ meet at each point of $\S_Y$.
%$\S_Y$ consists of the reduced preimages of $\S\cap D$ together with the preimages of $\S\backslash D$.
\item[c)]
The support of the irreducible components of $D'_Y$ for $D'\in\mathscr D$
generate a set $\mathscr D_Y$ of distinguished divisors on $Y$.
%\item[d)]  
%If $X$ has normal crossings and $D\cap\S=\varnothing$ then $Y$ has normal crossings.
\end{enumerate}
\rm

\pf
Since $Y\to X$ is finite, $\dim(X)=\dim(Y)=2$ by \cite[5.4.2]{EGAIV:b}, and $Y\to\Spec R$
is projective as the composition of projective morphisms (\cite[3.3.32]{Liu}).
Let $y\in Y$ be a closed point and set $x=f(y)$, $A=\O_{X,x}$, $B'=\O_{Y,x}$, and $B=\O_{Y,y}$.
Choose a geometric point over $x$ that lifts to each point of $Y$ lying over $x$,
and use this in the following to define the strict henselizations with respect to the maximal ideals
of these points.

Since the statements involving $D$ are local and $D$ is a disjoint union of its irreducible
components we may assume $D$ is irreducible.
Let $C_i\subset C$ be a (regular) irreducible component going through $x$,
and let $\{f,t\}\subset A$ be the regular system of parameters formed by local equations for 
the distinguished prime divisor passing through $x$, and for $C_i$, respectively.
Then the strict henselization $A^\sh$ of $A$ with respect to the maximal ideal of $A$
is a two-dimensional regular local ring, faithfully flat over $A$, with regular system
of parameters $\{f,t\}$ (see \cite[18.8]{EGAIV:d}).

If $x\not\in D$ then $B'\otimes_A A^\sh$ is a finite \'etale $A^\sh$-algebra
by base change, since $\rho|_{X-D}$ is finite-\'etale.
If $x\in D$ then $B'\otimes_A A^\sh$ is a finite tamely ramified $A^\sh$-algebra
by \cite[Lemma 2.2.8]{GM}.
By \cite[18.8.10]{EGAIV:d}, $B^\sh$ is a factor of the direct product decomposition of
$B'\otimes_A A^\sh$, hence 
$B^\sh$ is a finite tamely ramified local $A^\sh$-algebra, 
in particular it is a normal local ring, hence it is a normal domain.
%since $B$ is normal by \cite[18.8.12]{EGAIV:d}.
It follows that
$B^\sh$ is the integral closure of $A^\sh$ in the field $\widetilde L\df\Frac B^\sh$.
%Since the inertia groups with respect to height-one primes over $A^\sh$ are subgroups of the
%corresponding inertia groups over $A$, 
%$\widetilde L$ is tamely ramified with respect to the one regular prime $f$ of $A^\sh$ if $D$ runs
%through $x$, otherwise it is unramified over $A^\sh$.
Since the tame fundamental group of the strictly henselian regular local ring $A^\sh$ 
is abelian (\cite[XIII.5.3]{SGA1})
the field extension $\widetilde L/\Frac A^\sh$ is Galois,
and by Abhyankar's Lemma (\cite[A.I.11]{FK}, see also \cite[Corollary 2.3.4]{GM})
$$
B^\sh=A^\sh[T]/(T^e-f)\quad(\text{some }e\geq 1)
$$ 
%[A.I.11] requires $L/K$ Galois and $B$ the normalization of $A$ in $L$. 
By \cite[Lemma 1.8.6]{GM}
$B^\sh$ is a regular (2-dimensional) local ring with system of parameters $\{\root e\of f,t\}$.
Since $B\to B^\sh$ is faithfully flat
and $B^\sh$ is regular, $B$ is regular by flat descent (\cite[6.5.1]{EGAIV:b} or \cite[23.7(i)]{Mat}), 
and since $B$ is the local ring of an arbitrary closed point, we conclude $Y$ is regular.
It follows that $\rho:Y\to X$ is flat by \cite[23.1]{Mat}, proving (a), and since $Y$ is regular
and $Y\to\Spec R$ is flat and projective, $Y/R$ is a regular relative curve.

%Since $A^\sh\to B^\sh$ is flat, every height-one prime of $B^\sh$ lies over a height-one prime of $A^\sh$.
We derive a system of parameters for $B$.
The prime ideal $(\root e\of f)\subset B^\sh$ is the only one lying over $(f)A^\sh$ since,
for $\kappa(f)=\Frac A^\sh/(f)A^\sh$,
the ring $B^\sh\otimes_{A^\sh}\kappa(f)=\kappa(f)[T]/(T^e)$ of the fiber over $\Spec\kappa(f)$ 
consists of a single prime ideal.
The image $(\root e\of f)$ in $\Spec B$ is therefore a unique prime $(g)\subset B$ lying
over $(f)\subset A$, and $(\root e\of f)$ is the unique prime lying over $(g)$.
%since $(\root e\of f)$ is the unique prime lying over $(f)\subset A$
Therefore, since $B\to B^\sh$ is unramified, $(g)B^\sh=(\root e\of f)$.
Since $B\to B^\sh$ is faithfully flat, $IB^\sh\cap B=I$ for all ideals $I$ of $B$ (by e.g.
\cite[Exercise 3.16]{AM}), so since $(g,t)B^\sh=(\root e\of f,t)$ is maximal, 
$(g,t)B^\sh\cap B=(g,t)$ is the maximal ideal of $B$.
Thus $\{g,t\}$ is a regular system for $B$.

Since $t$ is a local equation for $\rho^{-1}C_i$, $\rho^{-1}C_i$ is regular
and irreducible at $y$ for each $C_i$ passing through $x$.
In particular $C_Y=\bigcup_i\rho^{-1} C_i$ is reduced, and so equals $Y_{0,\red}$.  
Since at most two irreducible components of $C$ meet at $x$, the same holds for $C_Y$ at $y$,
and $y$ is a singular point on $C_Y$ if and only if $x=f(y)\in\S$.
This completes the proof of (b).
%If $x\in\S\backslash D$ then $\rho^{-1}(x)$ is reduced
%by base change since $\rho|_{X-D}$
%is \'etale and $x$ is reduced (see \cite[I.3.18]{M}).  
%The preimage of $x\in \S\cap D$ may not be reduced.
%%We don't care!  $x\to X$ is closed, so $\rho^{-1}(x)\to Y$ is closed by base change.

If $D'\in\mathscr D$ is the distinguished (horizontal) prime divisor running through $x$
then there is a single irreducible component of $D_Y'$ passing through $y$, whose 
support $D_{Y,\red}'$ has local equation $g$ at $y$.
Thus each irreducible component of $D_Y'$ covers $D'$, hence $\Spec R$,
hence $D_Y'$ is horizontal.
Since $g$ is part of the regular system $\{g,t\}$ at the arbitrary closed point $y$
we see that $D_{Y,\red}'$ is regular, and since $t$
is a local equation for an arbitrary irreducible component of $C_Y$ passing through $y$,
$D_{Y,\red}'$ intersects each component of $C_Y$ transversally.
%If $D'\neq D$ then $D_Y'$ is already reduced since $\rho|_{X-D}$ is \'etale and $D'$ is reduced.
Thus the support of the irreducible components of $D_Y'$ generate a set
of distinguished divisors $\mathscr D_Y$ for $Y$.  This proves (c).

%Suppose $X$ has normal crossings and $D\cap\S=\varnothing$.
%In this case if $x\in\S$ we take $f$ to be a local equation for the other
%irreducible component of $C$ running through $x$.
%Then the regular system $\{f,t\}$ contains all local equations
%for $C$ at $x$, hence $\{g,t\}$ contains all local equations for $C_Y$ at $y$,
%hence $C_Y$ has normal crossings at $y$.
%Since the local equations for $X_0$ are also local equations for $Y_0$, it follows immediately
%that $Y_0$ has normal crossings at $y$, and since $y$ is arbitrary this proves (d).
%%For example, if $A/(t^m)$ is the local ring for $X_0$ at $x\not\in\S$, then $B/(t^m)B$ is the local
%%ring at $y$, so locally $Y_0=\div t^m$.  At $x\in S$, we have 
%%$A/(t_1^{n_1},t_2^{n_2})$ and $B/(t_1^{n_1},t_2^{n_2})B$ and a similar analysis.

%If $D\cap\S\neq\varnothing$, then this seems to fail, because we can form a regular
%system $t_1,t_2$ that will produce $g$.


\hfill $\blacksquare$



\Lemma\label{covers}
Suppose $X$ is a regular noetherian scheme and
$L$ is an \'etale $K(X)$-algebra that is tamely ramified along a divisor $D$.
Then the normalization $Y$ of $X$ in $L$ defines a tamely ramified cover
$\rho:Y\to(X,D)$.
\rm

\Pf
Since $X$ is regular, its connected components are integral regular schemes, 
hence we may assume $X$ is integral.
Since $L/K(X)$ is \'etale, $L$ is a product of finite separable field extensions of $K(X)$,
hence we may assume $L/K(X)$ is itself a finite separable field extension.
Then the normalization $Y$ exists, $Y$ is normal by definition,
%existence is the comment following \cite{Liu, 4.1.24}.
and $\rho:Y\to X$ is finite by \cite[4.1.25]{Liu}.
Since $Y$ is normal and connected it is irreducible, so $Y$ dominates $X$.
%$\rho$ is flat, so generic point lies over generic point by going down.
%Need this for definition of cover.
Let $U=X-D$, and set $V=Y\times_X U$.
Since $X$ is normal, $Y$ is connected, and $\rho|_V$ is unramified, 
$\rho|_V$ is \'etale by \cite[I.9.11]{SGA1} (see also \cite[I.3.20]{M}).
Therefore $Y\to (X,D)$ is a tamely ramified cover.

\hfill $\blacksquare$



The next lemma shows how distinguished divisors split in tamely ramified covers.

\Lemma\label{lemma2}
Assume the setup of (\ref{setup}).
Suppose $\rho:Y\to (X,D)$ is a tamely ramified cover, where $D\in\mathscr D$,
and $D'\in\mathscr D_\S$ is irreducible.
Suppose $E\subset D_{Y,\red}'$ is a distinguished prime divisor lying over $D'$ as in
Lemma~\ref{structure}(c),
$y= E\times_Y C_Y$, and $x=D'\times_X C$.
Then $y$ and $x$ are regular closed points,
and the ramification (resp. inertia) degree of $v_E$ over $v_{D'}$ 
equals the ramification (resp. inertia)
degree of $v_y$ over $v_x$.
\rm

\Pf
Since we assume \eqref{setup} and $D\in\mathscr D$ we have Lemma~\ref{structure},
which shows $C_Y$ is reduced and $E\subset D_{Y,\red}'$ is distinguished.
Note that either $D'\cap D=\varnothing$ or $D'\subset D$.
Since $D'$ and $E$ are distinguished and avoid the singular points of $X$ and $Y$,
they intersect the reduced closed fibers $C$ and $C_Y$ transversally, hence
$x=D'\times_X C$ and $y=E\times_Y C_Y$ are regular closed points.
We must show that $[\kappa(E):\kappa(D')]=[\kappa(y):\kappa(x)]$
%$[\Frac\O_{E}:\Frac\O_{D'}]=[\Frac\O_y:\Frac\O_x]$,
and that $v_E(f)=v_y(f_0)$, where $f\in\O_{X,D'}$ is a local equation for $D'$ on $X$
and $f_0\in\O_{C,x}$ is a local equation for $x$ on $C$. 
 
Since $D'$ is horizontal and irreducible, 
$D'=\Spec S$ for $S$ a finite local $R$-algebra by \cite[I.4.2]{M},
and $S$ is a discrete valuation ring since $D'$ is regular.
The map $E\to\rho^{-1}D'\to D'$ is finite as a composition of finite morphisms, hence
$E=\Spec T$ for $T$ a finite local $S$-algebra, again a discrete valuation ring since
$E$ is regular.
Since $S$ is a discrete valuation ring and $S\to T$ is finite, $T$ is a free $S$-module of finite
rank, and so $[T:S]$ is well defined.
Since the generic point of $E$ lies over that of $D'$, we have $\Frac T=T\otimes_S\Frac S$,
hence $[\kappa(E):\kappa(D')]=[T:S]$.

Let $A=\O_{X,x}$, $B=\O_{Y,y}$,
let $t$ be a local equation for $C$ at $x$, and
set $A_0=A/(t)$ and $B_0=B/(t)B$, the (reduced) local rings of the fibers 
through $x$ and $y$, as in the proof of Lemma~\ref{structure}.
Already $\kappa(x)=S\otimes_A A_0$ and $\kappa(y)=T\otimes_B B_0$
by the transversality of the intersections.
Since $B_0=B\otimes_A A_0$ we have
%by inspection
$\kappa(y)=T\otimes_A A_0$,
hence $[\kappa(y):\kappa(x)]=[T:S]=[\kappa(E):\kappa(D')]$ by base change.

Let $g\in A$ be defined as above.
%Note that here we use the structure theorem, so we need $D'\cap D=\varnothing$ or $D'\subset D$.
To compute the ramification degree, note that since $B\to B^\sh$ is faithfully flat, 
$(g^e)B=(g^e)B^\sh\cap B=(f)B^\sh\cap B=(f)B$,
hence $g^e=fu$ for some $u\in B^*$.
Since $f$ and $g$ are uniformizers for $v_{D'}$ and $v_{E}$, respectively, it follows that
$e(v_{E}/v_{D'})=v_{E}(f)=e$.
On the other hand, let $f_0$ be the image of $f$ in $A_0$, and let
$g_0$ be the image of $g$ in $B_0$. 
Then $f_0$ cuts out the closed point $x$ on $C$
and $g_0$ cuts out $y$ on $C_Y$ by transversality.
Thus $f_0$ and $g_0$ are uniformizers for $v_x$ and $v_y$, and since $g_0^e=f_0 u_0$,
where $u_0$ is the image of $u$ in $B_0^*$,
we have $e(v_y/v_x)=v_y(f_0)=e$, as desired.
This completes the proof.

\hfill $\blacksquare$


\section{Lifting Cohomology Classes}

\Paragraph\label{setup2}
Let $k$ be a field, and
let $C/k$ be a reduced connected projective curve with regular irreducible components $C_1,\dots,C_m$,
at most two of which meet at any closed point.
Denote the singular points of $C$ by $\S$
and write $\O_{C,\S}$ for the semilocal ring $\varinjlim_U\O_C(U)$,
where $U$ varies over (dense) open subsets of $C$ containing $\S$.
Then $\O_{C,\S}$ is a subring of the rational function ring $\kappa(C)=\prod_i \kappa(C_i)$.
For each $z\in \S\cap C_i$, let $K_{i,z}=\Frac\O_{C_i,z}^\h$, a field since $z$ is a normal point of
$C_i$, and if $\alpha_i\in\H^q(\kappa(C_i))$,
let $\alpha_{i,z}$ denote the image in $\H^q(K_{i,z})$.


\Lemma[Gluing]\label{gluing}
Assume the setup of (\ref{setup2}). 
There exists an element $\alpha\in\H^q(\O_{C,\S},\Lambda)$
that restricts to
$\alpha_C=(\alpha_1,\dots,\alpha_m)\in\bigoplus_i\H^q(\kappa(C_i),\Lambda)$
if and only if $\alpha_i$ is unramified at each $z\in\S\cap C_i$,
and $\alpha_{i,z}=\alpha_{j,z}$ whenever $z\in C_i\cap C_j$.
\rm

\Pf
There is an exact sequence (\cite[III.1.25]{M})
\begin{align*}
0
\longrightarrow \H^0_\S(\O_{C,\S})&\longrightarrow\H^0(\O_{C,\S})\longrightarrow
\H^0(\kappa(C))\longrightarrow\H^1_\S(\O_{C,\S})\longrightarrow\\
&\quad\longrightarrow\H^1(\O_{C,\S})\longrightarrow\H^1(\kappa(C))
\longrightarrow\H^2_\S(\O_{C,\S})\longrightarrow\\
&\qquad\longrightarrow\H^2(\O_{C,\S})\longrightarrow\H^2(\kappa(C))
\longrightarrow\H^3_\S(\O_{C,\S})
\tag{$*$}
\end{align*}
where the maps into the direct sum are restrictions.
Since $\S$ is a disjoint union of closed points,
$\H^q_\S(\O_{C,\S})=\bigoplus_{z\in\S}\H^q_z(\O_{C,\S})=\bigoplus_{z\in \S}\H^q_z(\O_{C,z}^\h)$ 
by excision (\cite[III.1.28, p.93]{M}).
Since $\Lambda$ is a smooth group scheme, $\H^q(\O_{C,z}^\h)=\H^q(\kappa(z))$,
by the cohomological Hensel's lemma \cite[III.3.11(a), p.116]{M}.
Since the $C_k$ are regular and at most two of them meet at any $z\in\S$, 
we have
$\Spec\O_{C,z}^\h-\{z\}=\Spec(K_{i,z}\times K_{j,z})$ for some $i$ and $j$,
%The prime $t_i$ defining $C_{i,z}$ in $\O_{C,z}$ stays prime in $\O_{C,z}^\h$
%by the usual regular system argument with $\{f,t_i\}$.
%We have $\Spec\O_{C,z}^\h=\Spec\O_{X,z}^\h/(t_i t_j)$, and the only primes in 
%the factorial local domain $\O_{X,z}^\h$
%that contain $t_i t_j$ are $\frak m_z,(t_i),(t_j)$.  Thus $\Spec\O_{C,z}^\h=\{\eta_i,eta_j,z\}$,
%where $\eta_i$ is the generic point of $C_i$.
and an ``excised'' exact sequence
\begin{align*}
0
\longrightarrow \H^0_z(\O_{C,\S})&\longrightarrow\H^0(\kappa(z))
\longrightarrow\H^0(K_{i,z}\times K_{j,z})\longrightarrow\H^1_z(\O_{C,\S})\longrightarrow\\
&\quad\longrightarrow\H^1(\kappa(z))\longrightarrow 
\H^1(K_{i,z}\times K_{j,z})\longrightarrow\H^2_z(\O_{C,\S})\longrightarrow\\
&\qquad \longrightarrow\H^2(\kappa(z))\longrightarrow \H^2(K_{i,z}\times K_{j,z})\longrightarrow\H^3_z(\O_{C,\S})\longrightarrow\cdots\\
\end{align*}
where the map $\H^q(\kappa(z))\to\H^q(K_{i,z}\times K_{j,z})=
\H^q(K_{i,z})\oplus\H^q(K_{j,z})$ 
is the diagonal map given by inflation from $\kappa(z)$ to the ``local fields'' $K_{i,z}$ and $K_{j,z}$.
Since $n$ is prime-to-$p$, the map $\H^0(\kappa(z))\to\H^0(K_{i,z})$ is an isomorphism,
so $\H^0_z(\O_{C,\S})=0$, and
%%this can be seen by converting to Galois cohomology, and comparing $\mu_n$ over $\kappa(z)$ and $K_{i,z}$.
%by the Hochschild-Serre spectral sequence, 
for $q\geq 1$ we have short exact Witt-type sequences
\[
0\to\H^q(\kappa(z))\to\H^q(K_{i,z})\xrightarrow{\;\partial_z\,}\H^{q-1}(\kappa(z),-1)\to 0
\]
Thus the long exact sequence breaks up into short exact sequences 
\begin{equation}\label{ses}
0\to\H^q(\kappa(z))\to\H^q(K_{i,z}\times K_{j,z})\to\H^{q+1}_z(\O_{C,\S})\to 0
\qquad (q\geq 0)
\end{equation}
By the compatibility of the localization sequence with the excised sequence, the map
$\H^q(\kappa(C_i))\to\H^{q+1}_z(\O_{C,\S})\leq\H^{q+1}_\S(\O_{C,\S})$ of $(*)$ factors through
$\res_{\kappa(C_i)|K_{i,z}}$.
Therefore an element $\alpha_C=(\alpha_1,\dots,\alpha_m)\in\H^q(\kappa(C))$
maps to zero in $\H_\S^{q+1}(\O_{C,\S})$
if and only if each couple $(\alpha_{i,z},\alpha_{j,z})$ is in the image of some $\bar\alpha\in\H^q(\kappa(z))$;
i.e., $\alpha_{i,z}=\alpha_{j,z}$, and both are unramified.
Thus by the exactness of $(*)$, 
$\alpha_C$ comes from $\H^q(\O_{C,\S})$ if and only if each $\alpha_i$ is unramified at each $z\in\S\cap C_i$,
and $\alpha_{i,z}=\alpha_{j,z}$ whenever $z\in C_i\cap C_j$.


\hfill $\blacksquare$

Suppose $C$ is as in \eqref{setup2}.
Since exactly two irreducible components meet at any $z\in\S$
the {\it dual graph} $G_C$ is defined, and consists of a vertex for each irreducible
component of $C$ and an edge for each singular point, 
such that an edge and a vertex are incident when the corresponding singular point
lies on the corresponding irreducible component (\cite[2.23]{Sai85}, see also \cite[10.1.48]{Liu}).
The (first) Betti number for $G_C$ is $\beta_C\,\df\,\rk(\H_1(G_C,\Z))=N+E-V$,  
where $V,E$ and $N$ are the numbers of vertices, edges, and connected components of $G_C$, 
respectively.


\Lemma\label{injects}
Assume the setup of (\ref{setup2}).
Then:
\begin{enumerate}
\item[a)]
For any integer $r$,
$\H^1(C,\Z/n(r))\to\H^1(\O_{C,\S},\Z/n(r))$ is injective.
\item[b)]
The map $\H^q(\O_{C,\S},\Z/n(q-1))\to\H^q(\kappa(C),\Z/n(q-1))$ is injective
for $q=0,2$, and for $q=1$ we have
$$\H^1(\O_{C,\S},\Z/n)\;\isom\;(\Z/n)^{\beta_C}\oplus \Gamma$$
where 
$(\Z/n)^{\beta_C}$ is the kernel of $\H^1(\O_{C,\S},\Z/n)\to\H^1(\kappa(C),\Z/n)$,
and $\Gamma\leq \H^1(\kappa(C),\Z/n)$ 
is the group of tuples that glue as in Lemma~\ref{gluing}.
\end{enumerate}
\rm

\Pf
We suppress the notation for $\Lambda=\Z/n(r)$.
Let $z\in C-\S$ be a (regular) closed point, and set $U=C-\{z\}$, 
a dense open subset containing $\S$.  
The localization exact sequence is
\begin{align*}
0\longrightarrow\H^0_z(C)\longrightarrow&\H^0(C)\longrightarrow\H^0(U)
\longrightarrow\cdots\\
&\cdots\longrightarrow\H^q_z(C)\longrightarrow\H^q(C)\longrightarrow\H^q(U)\longrightarrow\H^{q+1}_z(C)\longrightarrow\cdots
\end{align*}
%We have $\H^q_z(C)=\H^q_z(\O_{C,z}^\h)$ by \cite{M, III.1.28}.
%Applying the localization sequence to $\Spec\O_{C,z}^\h$ yields
By excision we have an exact sequence
\begin{align*}
0\longrightarrow\H^0_z(C)\longrightarrow&\H^0(\O_{C,z}^\h)
\longrightarrow\H^0(K_z)\longrightarrow\cdots\\
&\cdots\longrightarrow\H^q_z(C)
\longrightarrow\H^q(\O_{C,z}^\h)\longrightarrow\H^q(K_z)
\longrightarrow\H^{q+1}_z(C)\longrightarrow\cdots
\end{align*}
where $K_z=\Frac\O_{C,z}^\h$.
Since $z$ is a regular point $\O_{C,z}^\h$ is a discrete valuation ring,
and by \cite[Section 3.6]{CT95}
we may replace $\H^{q+1}_z(C)$ with $\H^{q-1}(\kappa(z),-1)$,
and the map from $\H^q(U)$, which factors
through $\H^q(K_z)$, is then the residue map $\partial_z$.
We conclude 
$\H^0(C)=\H^0(U)$, and we have a long exact sequence
\begin{align*}
0\longrightarrow&\H^1(C)\longrightarrow\H^1(U)\xrightarrow{\;\partial_z\;}
\H^0(\kappa(z),-1)\longrightarrow \cdots\\
&\cdots\longrightarrow\H^q(C)\longrightarrow\H^q(U)
\xrightarrow{\;\partial_z\;}\H^{q-1}(\kappa(z),-1)\longrightarrow\cdots
\tag{$**$}
\end{align*}
As $\H^1(\O_{C,\S})=\varinjlim_U\H^1(U)$,
where the limit is over dense open subsets of $C$ containing $\S$,
$\H^1(C)\to\H^1(\O_{C,\S})$ is injective by
the exactness of the injective limit functor, proving (a).

For (b) we go back to $\Lambda=\Z/n$.    
By ($*$) we have an exact sequence
\begin{align*}
0\longrightarrow\H^0(\O_{C,\S},\Z/n)\xrightarrow{\,\phi_1\,}\,
&\H^0(\kappa(C),\,\Z/n)\xrightarrow{\,\phi_2\,}\,\H^1_\S(\O_{C,\S}, \Z/n)\xrightarrow{\,\phi_3\,}\,\\
&\xrightarrow{\,\phi_3\,}\,\H^1(\O_{C,\S},\Z/n)\xrightarrow{\,\phi_4\,}\,\H^1(\kappa(C),\Z/n)
\end{align*}
The groups $\H^0(\O_{C,\S},\Z/n)$ and $\H^0(\kappa(C),\Z/n)$
are finite free $\Z/n$-modules whose ranks are the number of $C$'s connected components $N$
and irreducible components $m$, respectively.
We claim $\H^1_\S(\O_{C,\S},\Z/n)$ is a finite free $\Z/n$-module.
For by (\ref{ses}), for each $z\in\S$ we have an exact sequence
$$
0\longrightarrow\H^0(\kappa(z),\Z/n)\longrightarrow\H^0(K_{i,z},\Z/n)\oplus\H^0(K_{j,z},\Z/n)\longrightarrow\H^1_z(\O_{C,z}^\h,\Z/n)\to 0
$$
This shows $\H^1_z(\O_{C,z}^\h,\Z/n)\isom\Z/n$,
and since $\H^1_\S(\O_{C,\S},\Z/n)$ is a finite direct sum of these groups,
it is a finite free $\Z/n$-module, of rank $|\S|$.

The result \cite[27.1]{Fch} implies that a free $\Z/n$-submodule
of a $\Z/n$-module is a direct summand.
Therefore we have a decomposition 
$$
\H^0(\kappa(C),\Z/n)\isom\im(\phi_1)\oplus\im(\phi_2)
%\H^0(\O_{C,\S},\Z/n)\oplus M
$$
and since $\H^0(\kappa(C),\Z/n)$ is a finite free $\Z/n$-module,
$\im(\phi_2)$ is a finite free $\Z/n$-module by the structure theorem for finitely generated 
abelian groups.
Similarly, since $\H^1_\S(\O_{C,\S},\Z/n)$ is a finite free $\Z/n$-module,
$$
\H^1_\S(\O_{C,\S},\Z/n)\isom\im(\phi_2)\oplus \cok(\phi_2)
$$
and 
since $\im(\phi_2)$ and $\H^1_\S(\O_{C,\S},\Z/n)$ are finite free $\Z/n$-modules, so is $\cok(\phi_2)$.
Since $\H^1(\O_{C,\S},\Z/n)$ is a $\Z/n$-module, $\cok(\phi_2)$
is a direct summand of $\H^1(\O_{C,\S},\Z/n)$, again by \cite[27.1]{Fch}.
Thus we have a decomposition
$$
\H^1(\O_{C,\S},\Z/n)\isom\cok(\phi_2)\oplus\im(\phi_4)
$$
Now
we set $\Gamma=\im(\phi_4)$, and compute $\rk(\cok(\phi_2))=N+|\S|-m=\beta_C$. 
This proves the $q=1$ part of (b).

The $q=0$ case of (b) is in the proof of Lemma  \ref{gluing}.
Suppose $q=2$.
To show $\H^2(\O_{C,\S},\mu_n)\to\H^2(\kappa(C),\mu_n)$ is injective,
we will show $\H^1(\kappa(C),\mu_n)\to\H^2_\S(\O_{C,\S},\mu_n)$ is onto
and apply the exactness of ($*$).

For each closed point $z\in C_1\cap C_2\subset \S$, we have a diagram
\[\xymatrix{
&\H^1(\kappa(C_1),\mu_n)\oplus\H^1(\kappa(C_2),\mu_n)\ar[r]\ar[d]&\H^2_z(\O_{C,z}^\h,\mu_n)\ar@{=}[d]\\
0\longrightarrow\H^1(\kappa(z),\mu_n)\ar[r]&\H^1(\kappa(C_1)_z,\mu_n)\oplus\H^1(\kappa(C_2)_z,\mu_n)
\ar[r]&\H^2_z(\O_{C,z}^\h,\mu_n)\longrightarrow 0
}\]
We will show that $\H^1(\kappa(C_1),\mu_n)\oplus\H^1(\kappa(C_2),\mu_n)\to\H^2_z(\O_{C,z}^\h,\mu_n)$ is onto,
by showing the downarrow is onto.
Since $z$ is a regular point of each $C_i$, each $\O_{C_i,z}$ is a discrete valuation 
ring with residue field $\kappa(z)$ and fraction field $\kappa(C_i)$,
and we have a diagram of split short exact sequences
\[\xymatrix{
0\ar[r]&\H^1(\O_{C_i,z},\mu_n)\ar[r]\ar[d]&\H^1(\kappa(C_i),\mu_n)
\ar[r]\ar[d]&\H^0(\kappa(z),\Z/n)\ar@{=}[d]\ar[r]&0\\
0\ar[r]&\H^1(\widehat\O_{C_i,z},\mu_n)\ar[r]&\H^1(\kappa(C_i)_z,\mu_n)
\ar[r]&\H^0(\kappa(z),\Z/n)\ar[r]&0\\
}\]
To show the middle downarrow is onto it suffices (by a standard diagram chase) 
to prove that the left downarrow is onto.
Since $\widehat\O_{C_i,z}$ is henselian $\H^1(\widehat\O_{C_i,z},\mu_n)=\H^1(\kappa(z),\mu_n)$,
and by Kummer theory and Hilbert 90 we have $\H^1(\O_{C_i,z},\mu_n)=\O_{C_i,z}^*/n$
and $\H^1(\kappa(z),\mu_n)=\kappa(z)^*/n$.  
Since $\O_{C_i,z}\to \kappa(z)$ is onto and
$\O_{C_i,z}$ is local, the induced map $\O_{C_i,z}^*\to \kappa(z)^*$ is onto, hence $\H^1(\O_{C_i,z},\mu_n)$
maps onto $\H^1(\kappa(z),\mu_n)$.
We conclude $\H^1(\kappa(C_i),\mu_n)\to\H^1(\kappa(C_i)_z,\mu_n)$ is onto. 
Now each map $\H^1(\kappa(C_1),\mu_n)\oplus\H^1(\kappa(C_2),\mu_n)\to\H^2_z(\O_{C,z}^\h,\mu_n)$ is onto. 

Suppose $(b_z)\in\H^2_\S(\O_{C,\S},\mu_n)=\bigoplus_\S\H^2_z(\O_{C,z}^\h,\mu_n)$.
We have just seen that
for each closed point $z\in C_i\cap C_j$ there exists a pair 
$([a_{i,z}t_{i,z}^{e_i}],[a_{j,z}t_{j,z}^{e_j}])\in \kappa(C_i)^*/n\oplus \kappa(C_j)^*/n$
mapping to $b_z$, for $z$-units $a_{k,z}\in\O_{C_k,z}^*$, $z$-uniformizers
$t_{k,z}\in \kappa(C_k)$, and integers $e_k$, for $k=i,j$.
Let $v_{k,z}$ be the discrete valuation on $\kappa(C_k)$ determined by $z$.
By standard approximation 
(e.g. \cite[XII.1.2]{Lang}) there exist elements $a_k,t_k\in \kappa(C_k)$ such that 
$$
v_{k,z}(a_k-a_{k,z})>0\quad\text{and}\quad v_{k,z}(t_k-t_{k,z})>1\quad\text{for all $z$.}
$$
The image of $a_kt_k^{e_k}$ in $\kappa(C_k)_z^*/n$ is $[a_{k,z}t_{k,z}^{e_k}]$.
Therefore the $m$-tuple 
$$
([a_k t_k^{e_k}])\in\H^1(\kappa(C),\mu_n)
$$
maps to $(b_z)$.
This proves 
%
%Let $v_{i,z}$ be the valuation on $\kappa(C_i)$ defined by $z$.
%Since the $v_{i,z}$ are independent,
%by approximation there exists an element $t_i\in \kappa(C_i)$ such that $v_{i,z}(t_i)=1$ for each $z$,
%and if $\{a_{i,z}\}\subset \kappa(C_i)$ is any subset indexed by $z\in\S\cap C_i$, there 
%exists an element $a_i\in \kappa(C_i)$ such that $a\equiv a_{i,z}(\mod(t_{i,z})$.
%
the induced map $\H^1(\kappa(C),\mu_n)\to\H^2_\S(\O_{C,\S},\mu_n)$ is onto, and completes the proof.

\hfill $\blacksquare$

We will soon need the following technical lemma in order to replace $X_0$ with $C$.


\Lemma\label{reduce}
Suppose $A$ is a noetherian ring.
%I have noetherian in my definition total ring of fractions...
Then $(\Frac A)_\red = \Frac(A_\red)$ if and only if A has no embedded primes.
\rm

\begin{proof}
By definition $\Frac A=S^{-1}A$, where $S=A-\bigcup_{\Ass A}\frak p$, 
and $S^{-1}N_A=N_{S^{-1}A}$ by \cite[3.12]{AM}, hence $S^{-1}(A_\red)=(S^{-1}A)_\red$.  
It remains to show that
$S^{-1}(A_\red) = \Frac(A_\red)$ if and only if $A$ has no embedded primes.  
By $S^{-1}(A_\red)$ of course we mean $f(S)^{-1}(A_\red)$, where $f:A\to A_\red$.  
This localization equals the localization with respect to the multiplicative set T, 
where T is the saturation of $f(S)$ in $A_\red$, and this is the complement of the union of 
prime ideals of $A_\red$ that don't meet $f(S)$ by \cite[Exercise 3.7]{AM}.  
Thus $S^{-1}(A_\red)=\Frac(A_\red)$ if and only if the union of the primes of $A_\red$ that don't meet 
$f(S)$ equals the union of the associated primes of $A_\red$, which are just the minimal primes of 
$A_\red$.  But $A$ and $A_\red$ have identical underlying topological spaces, and the primes of 
$A_\red$ that don't meet $f(S)$ correspond to the primes of $A$ that don't meet $S$, i.e., 
the associated primes. These correspond to the minimal primes of $A_\red$ if and only if 
the associated primes of $A$ are the minimal primes of $A$, i.e., 
$A$ has no embedded primes.
\end{proof}

\Theorem\label{map}
Assume the setup of (\ref{setup}), with $X$ connected.
Then for $q\geq 0$
there is a map $$\lambda:\H^q(\O_{C,\S},\Lambda)\to\H^q(K(X),\Lambda)$$ 
and a commutative diagram
\begin{equation}\label{lambdadiagram}
\xymatrix{
\H^q(\O_{C,\S},\Lambda)\ar[r]^-\lambda\ar[d]_{\oplus\res_i}&\H^q(K(X),\Lambda)\ar[d]^{\oplus_i\res_i}\\
\H^q(\kappa(C),\Lambda)\ar[r]^-\inf&\bigoplus_i\H^q(K(X)_{C_i},\Lambda)
}\end{equation}
such that if $\alpha_0\in\H^q(\O_{C,\S},\Lambda)$ and $\alpha=\lambda(\alpha_0)$
then:
\begin{enumerate}
\item[a)]
$\alpha$ is defined at the generic points of $C_i$, and $\alpha(C_i)=\res_i(\alpha_0)$.
\item[b)]
The ramification locus of $\alpha$ (on $X$) is contained in $\mathscr D_\S$.
\item[c)]
If $D\in\mathscr D_\S$ is prime and $z=D\cap C$,
then $\partial_D\cdot\lambda=\inf_{\kappa(z)|\kappa(D)}\cdot\partial_z$.
\item[d)]
If $\alpha_0$ is unramified at a closed point $z$,
and $D$ is any (horizontal) prime lying over $z$,
then $\alpha$ is unramified at $D$, and has value
$\alpha(D)=\inf_{\kappa(z)|\kappa(D)}(\alpha_0(z))$.
\end{enumerate}
\rm

\Pf
%Since all of the statements are cohomological, we may assume $X_0=C$ is reduced.
Let $D_0$ be an effective divisor on $C$ that avoids $\S$, 
let $D\in\mathscr D_\S$ be the distinguished lift of $D_0$, set $U=X-D$,
and set $U_0=C-D_0$.
Since $X$ and $D$ are regular and $D$ has pure codimension $1$, we have 
$\H^0(X)\isom\H^0(U)$, and an exact Gysin sequence
$$
0\longrightarrow\H^1(X)\longrightarrow\H^1(U)
\xrightarrow{\;\partial_D\;}\H^0(D,-1)\longrightarrow\H^2(X)\longrightarrow\cdots
$$
by Gabber's absolute purity theorem (\cite[Theorem 2.1.1]{Fuj02}) and the standard
construction of the Gysin sequence (\cite[Section 3.2]{CT95}).
(Note that the result in \cite{Fuj02} is stated for the $\Lambda=\Z/n$ case only,
but the result holds in general since the sheaves 
$\mathscr H_D^q(X)$ and $\mathscr H_D^q(X,\Z/n)$ are locally isomorphic,
and the morphism $i^*\Lambda(-1)\to\mathscr H_D^2(X)$ is canonical.)
We use the notation $\partial_D$ since this map
is compatible with the one defined above on $\H^q(K(X))$ when $D$ is prime.


We may replace $X_0$ by $C=X_{0,\red}$ in the cohomological computations below
since $\Lambda$ is finite and $n$ is prime-to-$p$,
by \cite[V.2.4(c)]{M} (see also \cite[II.3.11]{M}).  To substitute $\O_{C,\S}$
and $\kappa(C)$ for $\O_{X_0,\S}$ and $\kappa(X_0)$ we must check that the former
are the canonical reduced quotients of the latter.
But the ring $\O_{X_0,\S}$ can by obtained by localizing some affine open subset $\Spec A_0$
containing $\S$ (which exists since $X_0/k$ is projective)
with respect to the multiplicative set $T=A_0-\bigcup_\S\frak m_z$.
Since $\O_{C,\S}$ is obtained by localizing $A_{0,\red}$ with respect to the image of $T$ in $A_{0,\red}$,
we have $\O_{C,\S}=(\O_{X_0,\S})_\red$ since the formation of the nilradical commutes
with localization (see e.g. \cite[3.12]{AM}).

To show $\kappa(C)=\kappa(X_0)_\red$ it suffices to show $X_0$ has no embedded points
by Lemma~\ref{reduce}.
But if $z$ is any closed point of $X$ then $\O_{X,z}$ is a regular local ring,
and a local equation for the closed fiber $\O_{X,z}\otimes_R k$ passing through $z$ is given by
the uniformizer $p$ in $R$.  Since $\O_{X,z}$
is factorial and at most two components of $X_0$ pass through $z$ we have
$p=\pi_1^{e_1}\pi_2^{e_2}$ for primes $\pi_i$ and numbers $e_i\geq 0$.
The associated primes of $\O_{X,z}/(\pi_1^{e_1}\pi_2^{e_2})$ are evidently just
the $(\pi_i)$, which shows $X_0$ has no embedded point at $z$.


Since $D_0$ is a disjoint union of regular closed points,
by ($**$) and the work that immediately precedes it we have
$\H^0(C)\isom\H^0(U_0)$ and an exact sequence
$$
0\longrightarrow\H^1(C)\longrightarrow\H^1(U_0)\xrightarrow{\;\partial_{D_0}\;}\H^0(\kappa(D_0),-1)
\longrightarrow \H^2(C)\longrightarrow\cdots
$$
Thus we have a commutative ladder
\[\xymatrix{
0\ar[r]&\H^1(X)\ar[r]\ar[d]&\H^1(U)\ar[r]^-{\partial_D}\ar[d]
&\H^0(D,-1)\ar[r]\ar[d]&\H^2(X)\ar[r]\ar[d]&\cdots\\
0\ar[r]&\H^1(C)\ar[r]&\H^1(U_0)\ar[r]^-{\partial_{D_0}}&\H^0(D_0,-1)
\ar[r]&\H^2(C)\ar[r]&\cdots
}\]
Since $R$ is complete,
$\H^q(X)\to\H^q(C)$ and $\H^q(D,-1)\to\H^q(D_0,-1)$ are
isomorphisms for $q\geq 0$ by proper base change (\cite[VI.Corollary 2.7]{M}).
Therefore, in light of the isomorphisms in degree zero and the 5-lemma in degree $q\geq 1$, 
we obtain isomorphisms
$$\H^q(U)\isim\H^q(U_0)$$
for $q\geq 0$.
Let $\widetilde U$ denote the inverse limit over all these open sets $U$ (this is a scheme by \cite[8.2.3]{EGAIV:c}).
Then $\H^q(\widetilde U)$ is the direct limit of the $\H^q(U)$ by \cite[III.1.16]{M}, and since
the direct limit functor is exact we have an isomorphism
$\H^q(\widetilde U)\isim\H^q(\O_{C,\S})$.
Composing the inverse with $\H^q(\widetilde U)\to\H^q(K(X))$ yields our lift
$$\lambda:\H^q(\O_{C,\S})\longrightarrow\H^q(K(X))$$
The commutative diagram (\ref{lambdadiagram}) follows by applying cohomology to the diagram
\[\xymatrix{
\Spec\O_{C,\S}\ar[r]&\widetilde U&\Spec K(X)\ar[l]\\
\Spec \kappa(C_i)\ar[r]\ar[u]&\Spec\O_{K(X)_{C_i}}\ar[u]&\Spec K(X)_{C_i}\ar[u]\ar[l]
}\]
incorporating the isomorphisms induced by the upper and lower left horizontal arrows.
If $\alpha_0\in\H^q(\O_{C,\S})$ and $\alpha=\lambda(\alpha)$,
then since $\widetilde U$ contains $\S$ and the generic points
of the $C_i$, $\alpha$ is defined at these points,
and the formula $\alpha(C_i)=\res_{\O_{C,\S}|\kappa(C_i)}(\alpha_0)$
follows immediately from (\ref{lambdadiagram}), proving (a).
 
If $D$ is a horizontal prime divisor not in $\mathscr D_\S$, then the generic point
$\Spec\kappa(D)$ is contained in $\widetilde U$, hence the map $\H^q(\widetilde U)\to\H^q(K(X)_D)$
factors through $\H^q(\O_{K(X)_D})$, which shows $\partial_D\cdot\lambda=0$.
Thus the ramification locus of any element in the image of $\lambda$ must be contained
in $\mathscr D_\S$, proving (b).
Now if $D\in\mathscr D_\S$ is prime and $z=D\cap C$ then 
$D$ is the prime spectrum of a complete local ring with residue field $\kappa(z)$, and the isomorphism
$$\H^{q-1}(D,-1)\isim\H^{q-1}(z,-1)=\H^{q-1}(\kappa(z),-1)$$
is the standard identification.
Thus the formula $\partial_D\cdot\lambda=\inf_{\kappa(z)|\kappa(D)}\cdot\partial_{z}$ is immediate
by the commutative ladder of Gysin sequences above, proving (c).

Suppose $\alpha=\lambda(\alpha_0)$ has ramification locus $D_\alpha$,
then $D_\alpha\in\mathscr D_\S$.  Set $U=X-D_\alpha$.
%Then as before $\kappa(z)$ is the residue field of the local field $\kappa(D)$.
%, and $\inf_{\kappa(z)|\kappa(D)}(\alpha_0(z))\in \H^q(\kappa(D))$.
If $\alpha_0$ is unramified at a point $z$, then $\alpha$ is unramified at 
every prime divisor $D$ lying over $z$.
For if $D\in\mathscr D_\S$ then $\partial_D(\alpha)=\inf(\partial_{z}(\alpha_0))$
by the formula just proved,
and if $D\not\in\mathscr D_\S$ then $\partial_D(\alpha)=0$ since $D_\alpha\in\mathscr D_\S$.
Thus if $\alpha_0$ is unramified at $z$, and $D$ is a prime divisor lying over $z$,
then $U$ contains $D$.
%Since $U_0$ contains $z$ we have a map $\H^q(U_0)\to\H^q(\widehat\O_{X_0,z})$,
%and the value $\alpha_0(z)$ is the image of $\alpha_0$ under this map.  Similarly,
%the value $\alpha(D)$ is the image of $\alpha$ under the map
%$\H^q(U)\to\H^q(\widehat\O_{X,D})$.
%Since
%$\H^q(\widehat\O_{X_0,z})=\H^q(\kappa(z))$ and $\H^q(\widehat\O_{X,D})=\H^q(\kappa(D))$,
%the last statement now follows from the commutative diagram
%$$
%\alignat 3
%&\H^q(U)\xrightarrow{\qquad}&\H^q(\widehat\O_{X,D})=&\H^q(\kappa(D))\\
%&\quad{\lambda}\big\uparrow &&\qquad\big\uparrow{\inf}\\
%&\H^q(U_0)\xrightarrow{\qquad}&\H^q(\widehat\O_{X_0,z})=&\H^q(\kappa(z))
%\endalignat
%$$
%
The maps $z=\Spec \kappa(z)\to U_0$ and $D\to U$ then induce a commutative diagram
\[\xymatrix{
\H^q(U)\ar[r]^-{\res}\ar[d]^{\res}&\H^q(D)\ar[r]^-{\res}\ar[d]^{\res}&\H^q(\kappa(D))\\
{\H^q(U_0)}{\ar[r]^-{\res}}&{\H^q(z)}\ar[ur]_{\inf}}
\]
Both vertical down-arrows are isomorphisms by proper base change. 
The inverse of the left one is $\lambda$ by definition,
and the composition of the inverse of the right one and the restriction
$\H^q(D)\to\H^q(\kappa(D))$ is inflation, as shown. 
Since $\kappa(D)$ is complete, the top composition of horizontal restrictions factors through the restriction
$\H^q(U)\to\H^q(\widehat\O_{X,D})$ and 
the bottom factors through the restriction
$\H^q(U_0)\to\H^q(\widehat\O_{C,z})$. 
Since these are restriction maps, the images of $\alpha$ and $\alpha_0$ are 
the values $\alpha(D)$ and $\alpha_0(z)$.
We conclude $\inf_{\kappa(z)|\kappa(D)}(\alpha_0(z))=\alpha(D)$, as in (d).

\hfill $\blacksquare$


\Paragraph\label{pi}
By weak approximation \cite[Lemma]{Sa98}
there exists a $\pi\in K(X)$ such that 
$$
\div\,\pi=C+E
$$ 
where $E$ contains no components of $C$, and avoids any finite set of points $\mathcal N$.
We fix such a $\pi$ for $\mathcal N$ containing $\S$. 
For each $i$, the choice of $\pi$ determines a noncanonical ``Witt'' isomorphism 
\[
\H^q(\kappa(C_i))\oplus\H^{q-1}(\kappa(C_i),-1)\isim\;\H^q(K(X)_{C_i})
\]
Taking $(\alpha,\theta)$
to $\alpha+(\pi)\cdot\theta$, where $\alpha$ and $\theta$ are inflated from 
$\kappa(C_i)$ to $K(X)_{C_i}$,
%Note that if $C_i$ is not reduced, then the inflation factors through $\kappa(X^{(i)}_{0,\red})$ on its
%way to $K(X)_{v_i}$.
$(\pi)$ is the image of $\pi$ in $\H^1(K(X)_{C_i},\mu_n)$, and $(\pi)\cdot\theta$
is the cup product.
Although we cannot in general lift all of
$\bigoplus_i\H^q(K(X)_{C_i})$ to $\H^q(K(X)$, we can now prove the following.

\Corollary\label{splits}
Let $(\pi)$ denote the image of $\pi$ in $\H^1(K(X),\mu_n)$.
The choice of $\mathscr D_\S$ and $\pi$ determines a homomorphism for $q\geq 1$,
\begin{align*}
\lambda:\H^q(\O_{C,\S},\Lambda)\oplus\H^{q-1}(\O_{C,\S},\Lambda(-1))&\longrightarrow\;\H^q(K(X),\Lambda)\\
(\alpha_0,\theta_0)&\longmapsto \lambda(\alpha_0)+(\pi)\cdot\lambda(\theta_0)
\end{align*}
such that $(\bigoplus_i\res_{K(X)|K(X)_{C_i}})\cdot\lambda
=\bigoplus_i(\inf_{\kappa(C_i)|K(X)_{C_i}}\cdot\res_{\O_{C,\S}|\kappa(C_i)})$.
\rm

\Pf
This is an immediate consequence of Theorem \ref{map}.

\hfill $\blacksquare$

\Remark\label{connected}
a) Theorem \ref{map} and Corollary \ref{splits} 
apply with obvious amendments to the case where $X$ is not connected.  
For if $X=\coprod_k X_k$ is a decomposition into connected components,
then $K(X)=\prod_k K(X_k)$, $X_0=\coprod_k(X_k)_0$, $\O_{X_0,\S}=\prod_k\O_{(X_k)_0,\S_k}$
(where $\S_k=\S\cap X_k$),
all of the cohomology groups break up into direct sums, and we define the map $\lambda$ 
to be the direct sum of the maps on the summands.
This will come up in the next section.


b) If $X$ is smooth, then $\S$ is empty, and $\O_{C,\S}=\kappa(C)$.
By Witt's theorem we have 
$\H^q(K(X)_C,\Lambda)\isom\H^q(\kappa(C),\Lambda)\oplus\H^{q-1}(\kappa(C),\Lambda(-1))$,
and we obtain a map
$$\lambda:\H^q(K(X)_C,\Lambda)\longrightarrow\H^q(K(X),\Lambda)$$
that splits the restriction map.
This is the map of \cite{BMT}.

\Paragraph{Completely split characters.}
In \cite[2.1]{Sai85} Saito defines a completely split covering
of a noetherian scheme $X$ to be a finite \'etale cover $Y\to X$ such that 
$Y\times_X \Spec \kappa(x)=\coprod\Spec\kappa(x)$, for all closed points $x\in X$. 
%Let $\H^1_\cs(X,\Z/n)$ denote the set of completely split $\Z/n$-Galois \'etale covers of $X$.
We abuse Saito's terminology (see Remark\eqref{remark} below) and in the setup of \eqref{setup} denote by
$\H^1_\cs(C,\Z/n)$ the kernel of the map $\H^1(\O_{C,\S},\Z/n)\to\H^1(\kappa(C),\Z/n)$
in Lemma~\ref{injects}.
If $\beta\in\H^1_\cs(C,\Z/n)$ then $\partial_z(\beta)=0$ for all closed points $z\in C-\S$ since
$\partial_z$ factors through $\kappa(C_i)_z$.
Therefore $\beta$ is defined on $C$, hence $\H^1_\cs(C,\Z/n)\leq\H^1(C,\Z/n)$.
Let $\H^1_\cs(X,\Z/n)$ denote the preimage of $\H^1_\cs(C,\Z/n)$ under the proper
base change isomorphism.

\Proposition\label{cs}
Assume the setup of (\ref{setup}).
Then elements of $\H^1_\cs(C,\Z/n)$ are trivial at all points of $C$,
and the nontrivial elements of $\H^1_\cs(X,\Z/n)$ are trivial at all points of $X$ except for the
generic point $\Spec K(X)$, where they are nontrivial.
\rm

\begin{proof}
Suppose $\beta_0\in\H^1_\cs(C)$.
Then $\beta_0$ is trivial at each generic point of $C$ by definition of $\H^1_\cs(C)$.
If $z\in C$ is a closed point lying on the irreducible component $C_i$ then
the map $\H^1(C)\to\H^1(\kappa(z))$ factors through $\H^1(C_i)$.
Since $C_i$ is regular the map $\H^1(C_i)\to\H^1(\kappa(C_i))$ is injective by purity, 
and consequently $\beta_0(z)=0$ by definition.
Thus the elements of $\H^1_\cs(C)$ are trivial at all points of $C$.

Suppose $\beta=\lambda(\beta_0)\in\H^1_\cs(X)$.
If $x\in X$ is a generic point of some irreducible component
$C_i$ of $C$ then the image of $\beta$ in $\H^1(\kappa(C_i))$
is zero since the map $\H^1_\cs(X,\Z/n)\to\H^1(\kappa(C_i))$ factors through $\H^1_\cs(C)$.
If $x$ is the generic point of a horizontal divisor $D$ with closed point $z$ then 
$\beta(D)=\inf_{\kappa(z)|\kappa(D)}(\beta_0(z))$ by Theorem~\ref{map}(d),
and this is zero since $\beta_0(z)=0$.
If $z$ is a closed point of $X$ then $z$ is on $C$,
and the map $\H^1_\cs(X)\to\H^1(\kappa(z))$ factors through $\H^1_\cs(C)$, hence $\beta$
is trivial at $z$.
Finally, since $X$ is regular the map $\H^1(X)\to\H^1(K(X))$ is injective by purity,
hence $\beta$ is nontrivial at the generic point of $X$.
\end{proof}


\Remark\label{remark}
Proposition~\ref{cs} shows the elements of $\H^1_\cs(C,\Z/n)$ are completely split in the sense of \cite{Sai85}.
However, in the general case our $\H^1_\cs(C,\Z/n)$ does not account for elements 
that are split at every closed point but nontrivial at generic points
of $C$.  This is not an issue if $k$ is finite as shown by Saito in \cite[Theorem 2.4]{Sai85},
since then the $C_i$ have no nontrivial completely split covers, essentially
by Cebotarev's density theorem (see \cite[Lemma 1.7]{Ras}).

%\Remark
%\label{saito}
%If $k$ is finite in Theorem \ref{map},
%then $\lambda$ induces an isomorphism
%$$
%\lambda:\H^1_\cs(C,\Z/n)\isim\H^1_\cs(X_K,\Z/n)
%$$
%For the natural map $\H^1_\cs(X_K)\isim\H^1_\cs(C)$ is 
%an isomorphism by \cite[Proposition 2.2]{Sai85}.
%%Using the finiteness of $k$, Saito shows that 
%%if $\rho_K:Y_K\to X_K$ is a completely split abelian cover,
%%and $Y$ is the integral closure of $X$
%%in the function field of $Y_K$, then $\rho:Y\to X$ is a completely split
%%cover, whose restriction to $C$ is a completely split cover $\rho_0:Y_0\to C$.
%%The canonical map $\H^1_\cs(X_K)\to\H^1_\cs(C)$ is then an isomorphism
%%(\cite{Sai, Proposition 2.2}). 
%By Lemma~\ref{injects}(a), $\H^1(C)\to\H^1(\O_{C,\S})$
%is injective, and $\lambda$ maps $\H^1(C)$ isomorphically onto 
%$\H^1(X)\hookrightarrow\H^1(K(X))$.
%Thus the restriction of $\lambda$ to $\H^1_\cs(C)\leq\H^1(C)$
%factors through $\H^1_\cs(X_K)$, hence the result.





\section{Index Calculation in the Brauer Group}

\Paragraph{Cyclic Covers.}
If $U$ is any scheme, and $\bar u$ is a geometric point,
the fiber functor defines a category equivalence between (finite) \'etale covers of $U$ 
and finite continuous $\pi_1(U,\bar u)$-sets, yielding a canonical isomorphism
$$
\H^1(U,\Z/n)\isom\H^1(\pi_1(U,\bar u),\Z/n)=\Hom_\cont(\pi_1(U,\bar u),\Z/n)
$$
(see \cite[I.2.11]{FK}).
If $\theta\in\H^1(U,\Z/n)$, we will write $U[\theta]$ for the finite cyclic \'etale 
cover determined by $\theta$.  If $U=\Spec A$ is affine, we will write $A[\theta]$
for the corresponding ring, or $A(\theta)$ if $A$ is a field.
If $U$ is a connected normal scheme, and $\theta\in\H^1(U,\Z/n)$ has order $m$, 
then $U[\theta]$ is a disjoint sum of $n/m$ connected $\Z/m$-Galois covers of $U$.

\Lemma\label{cycliccovers}
Assume the setup of (\ref{setup}).
Let $\theta_0\in\H^1(\O_{C,\S},\Z/n)$ be a (tamely ramified) character
with ramification divisor $D_0$ on $C$.
Then the (tame) ramification divisor of $\theta=\lambda(\theta_0)$ is
the distinguished lift $D\in\mathscr D_\S$ of $D_0$ on $X$, and 
$\theta$ defines a tamely ramified cover $\rho:Y\to(X,D)$
as in Lemma~\ref{covers}.
Restriction to $C$ yields a tamely ramified cover
$\rho_0:C_Y\to(C,D_0)$ such that $\O_{C_Y,\S_Y}=\O_{C,\S}[\theta_0]$, 
and the reduced closed fiber
$C_Y$ of $Y$ is the normalization of $C$ in $\kappa(C)(\theta_0)$.
%i.e., the (regular) irreducible components of $C_Y$ are the 
%normalizations of the (regular) irreducible components of $C_i$ in 
%$\kappa(C_i)(\theta_{0,C_i})$.
%Strangely, $Y_0$ is not the integral closure of $X_0$.
\rm

\Pf
The lift $\theta$ is tamely ramified with respect to $D$ by Theorem \ref{map}.
Let $Y$ be the normalization of $X$ in $L=K(X)(\theta)$.
Since $D$ is in $\mathscr D_\S$ and $X/R$ satisfies the setup of \eqref{setup}, by 
Lemma~\ref{covers} $\rho:Y\to(X,D)$ is a tamely ramified cover,
$Y/R$ is a regular relative curve with reduced closed fiber $C_Y$,
the irreducible components of $C_Y$ are regular with singular points $\S_Y$,
and $D_Y$ is in $\mathscr D_{Y,\S_Y}$.

Let $U=X-D$, $V=U\times_X Y$, $U_0=U\times_X X_0$ and $V_0=V\times_Y Y_0$.
Then $\theta_0\in\H^1(U_0)$.
%Since
%$\O_{X_0,\S}=\varinjlim_{\S\subset W_0}\O_{X_0}(W_0)$, with each $W_0$ dense in $X_0$,
%we have $\H^1(\O_{X_0,\S})=\varinjlim_{\S\subset W_0}\H^1(W_0)$, and so we may write
%$\theta_0\in\H^1(U_0)$.
We have $\H^1(U)\isom\Hom(\pi_1(U),\Z/n)$ and $\H^1(U_0)\isom\Hom(\pi_1(U_0),\Z/n)$,
and the restriction map $\res:\H^1(U)\to\H^1(U_0)$, which sends $\theta$ to $\theta_0$, 
is induced by the natural map $\pi_1(U_0)\to\pi_1(U)$,
which is induced on covers by $W\mapsto W\times_U U_0$.
Therefore $V_0=U_0[\theta_0]$.

We show $\O_{C_Y,\mathcal S_Y}=\O_{C,\S}[\theta_0]$.
If $U_0=\Spec A_0\subset C-D_0$ is a dense affine open subset of $C$ containing $\S$,
then its preimage in $C_Y$ is a dense affine open subset $V_0=\Spec B_0$ containing $\mathcal S_Y$.
By base change we have $B_0=A_0[\theta_0]$, and $S^{-1}B_0=\O_{C,\S}[\theta_0]$,
where $S=A_0-\left(\bigcup_{x\in\S}\frak m_x\right)$ is the multiplicative set defining $\O_{C,\S}$.
Since $\mathcal S_Y=\rho^{-1}\S$,
the saturation $T$ of $S$ in $B_0$ is $T=B_0-\left(\bigcup_{y\in\mathcal S_Y}\frak m_y\right)$,
which shows $S^{-1}B_0=\O_{C_Y,\mathcal S_Y}$.
Therefore $\O_{C_Y,\mathcal S_Y}=\O_{C,\S}[\theta_0]$, and it follows immediately
that $\O_{C_Y,\S_Y}=\O_{C,\S}[\theta_0]$.

The map $C_Y\to C$ is finite by base change,
and since each irreducible component of $C_Y$ is regular by Lemma~\ref{structure} 
each irreducible component of $C_Y$ is the normalization of a component of $C$ in 
a field extension which is a direct factor of $\kappa(C_Y)$.
Equivalently, $C_Y$ is the normalization of $C$ in $\kappa(C)(\theta_0)$, by \cite[6.3.7]{EGAII}.
This completes the proof.

\hfill $\blacksquare$

\Paragraph{Index.}\label{index}
Assume the setup of \eqref{setup} with $R=\Z_p$ and $\Lambda=\mu_n$.
Fix $\alpha_C\in\H^2(\O_{C,\S})$ and $\theta_C\in \Gamma\leq\H^1(\O_{C,\S},-1)$, 
and write
\begin{align*}
\alpha_C&=(\alpha_{C_1},\dots,\alpha_{C_m})\in\H^2(\kappa(C))\\
\theta_C&=(\theta_{C_1},\dots,\theta_{C_m})\in\H^1(\kappa(C),-1)
\end{align*}
as per Lemma~\ref{injects}(b).
Suppressing the inflation maps, we form the element
$$
\gamma_C=
\alpha_C+(\pi)\cdot\theta_C=(\alpha_{C_1}+(\pi)\cdot\theta_{C_1},
\dots,\alpha_{C_m}+(\pi)\cdot\theta_{C_m})
\in\bigoplus_{i=1}^m\H^2(K(X)_{C_i})
$$
where $\pi$ is as in \eqref{pi}.
Let $\theta=\lambda(\theta_C)$, and let $Y$ be the normalization of $X$ in $K(X)(\theta)$,
as in Lemma~\ref{cycliccovers}.
Then $C_Y$ is the reduced closed fiber of $Y$, and we write $C_{i,Y}$ for the preimage of $C_i$,
so that $\kappa(C_{i,Y})=\kappa(C_i)(\theta_{C_i})$.
Thus
$$
\alpha_{C_Y}=(\alpha_{C_{1,Y}},\dots,\alpha_{C_{m,Y}})
\;\in\;\bigoplus_{i=1}^m\H^2(\kappa(C_{i,Y}))
$$
Note $\kappa(C_{i,Y})$ is a product of the function fields of the irreducible components of $C_{i,Y}$.
By the (well-known) Nakayama-Witt index formula,
$$
\ind(\alpha_{C_i}+(\pi)\cdot\theta_{C_i})=|\theta_{C_i}|\cdot\ind(\alpha_{C_{i,Y}})
$$
We have $|\theta_C|=\lcm_i\{|\theta_{C_i}|\}$ by Lemma~\ref{injects}(b). 
We now {\it define}
\begin{align*}
\ind(\alpha_{C_Y})\;&\df\;\lcm_i\{\ind(\alpha_{C_{i,Y}})\}\\
\ind(\gamma_C)\;&\df\;|\theta_C|\cdot\ind(\alpha_{C_Y})
\end{align*}

\Theorem\label{preservesindex}
Assume the setup of (\ref{setup}) with $R=\Z_p$.
Let $\Gamma\leq\H^1(\O_{C,\S})$ be as in Lemma~\ref{injects}(b).
Then the map $\lambda:\H^2(\O_{C,\S})\oplus \Gamma\to\H^2(K(X))$
preserves index.
\rm

\Pf
We may assume $X$ is connected.
We identify $\Gamma$ with the image of $\H^1(\O_{C,\S},-1)$ in $\H^1(\kappa(C),-1)$, 
as in Lemma~\ref{injects}(b),
and adopt the notation of (\ref{index}).
Set $\gamma=\lambda(\gamma_C)$, $\alpha=\gamma(\alpha_C)$, and $\theta=\lambda(\theta_C)$,
so that $\gamma=\alpha+(\pi)\cdot\theta$ as in Corollary \ref{splits}.
%We use base change notation,
%so that $\alpha_Y$ is the restriction of $\alpha$ to $K(Y)$,
%$\alpha_{C_Y}$ is the restriction of $\alpha_C$ to $\kappa(C_Y)$
%(which is also the restriction of $\alpha_Y$ to $\kappa(C_Y)$.
By restricting to connected components if necessary
we may assume that $Y$ is connected, hence that $K(Y)$ is a field.
Even so, the cyclic-Galois \'etale $\kappa(C_i)$-algebra $\kappa(C_{i,Y})$ may not be a field.
Since $\theta_C$ is in $\H^1(\O_{C,\S},-1)$ the ramification divisor $D_0$ of $\theta_C$ avoids $\S$,
and the distinguished lift
$D\in\mathscr D_\S$ of $D_0$ is the ramification divisor of $\theta$ by Theorem \ref{map}.
By Lemma~\ref{cycliccovers} $\rho:Y\to (X,D)$ is a cyclic tamely ramified cover,
and by Lemma~\ref{structure} $Y/R$ satisfies the properies of \eqref{setup},
with reduced closed fiber $C_Y$, $\S_Y=\rho^{-1}\S$ the singular points of $C_Y$,
and $\mathscr D_Y$ generated by $D_{Y,\red}$ and the preimages of the other distinguished
divisors of $X$.

The index of $\gamma$ cannot exceed $|\theta|\ind(\alpha_Y)$.
For if $M/K(Y)$ is a separable maximal subfield of the division algebra
associated with $\alpha_Y$, then $M/K(X)$ splits $\gamma$,
and has degree $|\theta|\ind(\alpha_Y)$.
Since in our case $|\theta|=[K(Y):K(X)]=[\kappa(C_Y):\kappa(C)]=|\theta_C|$, 
to prove the theorem it is enough
to prove $\ind(\alpha_Y)=\ind(\alpha_{C_Y})$.

%By Lemma~\ref{cycliccovers}, $\theta_C$ defines a tamely ramified cover $\rho:Y\to (X,D)$ via normalization
%in the cyclic extension $L\df K(X)(\theta)/K(X)$,
%and $Y/R$ is a regular relative curve with normal crossings, 
%such that $\rho^{-1}D$ has normal crossings and intersects $Y_C$ transversally.
%Moreover, the restriction $\rho_0:Y_0\to X_0$ is tamely ramified along $D_0$, $Y_0$ induces $\theta_0$,
%and since they are normal and finite over $X^{(i)}_{0,\red}$,
%the irreducible components of $Y_{0,\red}$ are normalizations in
%the cyclic \'etale extensions $L_0^{(i)}\df F_0^{(i)}(\theta_{C_i})/F_0^{(i)}$.  
%Set $L_0=\prod_i L_0^{(i)}$.

By Lemma~\ref{cycliccovers} $\O_{C,\S}[\theta_C]=\O_{C_Y,\mathcal S_Y}$.
Each $\kappa(C_{i,Y})=\kappa(C_i)(\theta_{C_i})$ is a product of global fields, 
and by class field theory
the division algebra associated with the restriction of
$\alpha_{C_{i,Y}}$ to each field component is cyclic.
Since $\alpha_{C_Y}$ is in $\H^2(\O_{C_Y,\S_Y})$ (by Lemma~\ref{cycliccovers} and Lemma~\ref{gluing}),
$\alpha_{C_{i,Y}}$ is unramified at $\mathcal S_Y$.
By Grunwald-Wang's theorem there exists a tuple
$\psi_{C_Y}=(\psi_{C_{1,Y}},\dots,\psi_{C_m,Y})\in\bigoplus_i\H^1(\kappa(C_{i,Y}),-1)$
such that $|\psi_{C_{i,Y}}|=\ind(\alpha_{C_{i,Y}})$,
$\kappa(\psi_{C_{i,Y}})$ splits $\alpha_{C_{i,Y}}$, and such that
the $\psi_{C_{i,Y}}$ are unramified and equal at the local fields defined by the singular points $\mathcal S_Y$.
Then $\psi_{C_Y}$ comes from $\H^1(\O_{C_Y,\mathcal S_Y},-1)$ by Lemma~\ref{gluing},
and $|\psi_{C_Y}|=\ind(\alpha_{C_Y})$.

By Theorem \ref{map} (and Remark \ref{connected}(a) if $Y$ is not connected)
we have a map $\lambda_Y:\H^1(\O_{C_Y,\mathcal S_Y},-1)\to\H^1(K(Y),-1)$.
Since the distinguished divisors on $Y$ are the (reduced) preimages of those on $X$,
$\lambda_Y$ is compatible with $\lambda$ and the residue maps.
Set $\psi=\lambda_Y(\psi_{C_Y})$, and let $D_\psi$ 
denote the distinguished lift of the ramification divisor $D_{\psi_{C_Y}}$ of $\psi_{C_Y}$ on $C_Y$.
% and $M=L(\psi)$.
By Lemma~\ref{cycliccovers}, $\psi$ determines a cyclic tamely ramified cover $\sigma:Z\to(Y,D_\psi)$
with reduced closed fiber $C_Z$, 
such that $C_Z$ is cyclic and tamely ramified over $(C_Y,D_{\psi_{C_Y}})$,
inducing $\psi_{C_Y}$.
Since $\kappa(C_{i,Z})$ splits $\alpha_{C_{i,Y}}$, $\alpha_{C_Z}=0$.
%let $M_0=\kappa(Z_{0,\red})$, so that $M_0=\prod_i M_0^{(i)}$.
%Since $M_0^{(i)}$ splits $(\alpha_0^{(i)})_{L_0^{(i)}}$,
%$(\alpha_0)_{M_0}=0$. 

Again we may assume $Z$ is connected.
%By construction, $[K(Y):K(X)]=|\theta|=|\theta_0|=[\kappa(Y_0):\kappa(X_0)]$.
%we use connected here, otherwise multiply $|\theta|$ and $|\theta_0|$ by the
%number of components.
By construction, $[K(Z):K(Y)]=|\psi|=|\psi_{C_Y}|=[\kappa(C_Z):\kappa(C_Y)]=\ind(\alpha_{C_Y})$.
By (\ref{lambdadiagram}) we have 
$\ind(\alpha_{C_Y})\leq \ind(\alpha_Y)$, and it remains to show $\alpha_Z=0$.
It is then enough to show $\alpha_Z$ is unramified with respect to all Weil divisors on $Z$,
by \cite[Lemma 3.5]{BMT}.

Let $D'=D\cup\rho(D_\psi)$.  
Since $D_\psi\in\mathscr D_{Y,\S_Y}$, $\rho(D_\psi)\in\mathscr D_\S$, and the composition
$\rho':Z\to (X,D')$ is a tamely ramified cover.
Since $X$ is regular, $\rho'$ is (finite and) flat by Lemma~\ref{structure}, 
and so the image of any prime divisor $J$ of $Z$
is a prime divisor $\rho'(J)=I$ of $X$.
By the functoriality of the residue maps, $\alpha_Z$ can only be ramified at prime divisors lying
over irreducible components of $D_\alpha$.
By Theorem~\ref{map}(a) $D_\alpha$ is in $\mathscr D_\S$, and the divisors of $Z$ lying over $D_\alpha$
are distinguished by Lemma~\ref{structure}(c).
Thus it is enough to show that $\alpha_Z$ is unramified at these distinguished divisors.
Clearly we may assume $D_\alpha$ is irreducible.

Let $E\subset Z$ be a (distinguished) prime divisor lying over $D_\alpha\in\mathscr D_\S$.
Since $D_\alpha\cap D'=\varnothing$ or $D_\alpha\subset D'$,
we have $e(v_E/v_{D_\alpha})=e(v_{E_0}/v_{D_{\alpha_C}})=e$ for some $e\geq 1$,
by Lemma~\ref{lemma2}.
By Lemma~\ref{map} and the functorial behavior of the residue and restriction
maps, we have a commutative diagram
\[\xymatrix{
\H^2(\O_{C,\S})\ar[d]_{\partial_{D_{\alpha_C}}}\ar[r]^-\lambda\ar@{}[dr]|{1}&\H^2(K(X))\ar[d]_{\partial_{D_\alpha}}
\ar[r]^-\res\ar@{}[dr]|{2}&\H^2(K(Z))\ar[d]^{\partial_E}\\
\H^1(\kappa(D_{\alpha_C}),-1)\ar[r]&\H^1(\kappa({D_\alpha}),-1)\ar[r]^-{e\cdot\res}\ar@{}[dr]|{3}&\H^1(\kappa(E),-1)\\
&\H^1(\kappa(D_{\alpha_C}),-1)\ar[r]^-{e\cdot\res}\ar[u]^\inf\ar@{}[dr]|{4}&\H^1(\kappa(E_0),-1)\ar[u]_\inf\\
&\H^2(\O_{C,\S})\ar[u]^{\partial_{D_{\alpha_C}}}\ar[r]^\res&\H^2(\O_{C_Z,\mathcal S_Z})\ar[u]_{\partial_{E_0}}\\
}\]
Since $\alpha=\lambda(\alpha_C)$, 
$\partial_E(\alpha_Z)=e\cdot(\partial_{D_{\alpha_C}}(\alpha_C))_{\kappa(E)}$
by squares (1) and (2), and by square (4),
$\partial_{E_0}(\alpha_{C_Z})=e\cdot(\partial_{D_{\alpha_C}}(\alpha_C))_{\kappa(E_0)}$.
Therefore $\partial_E(\alpha_Z)=\inf_{\kappa(E_0)|\kappa(E)}(\partial_{E_0}(\alpha_{C_Z})$ 
by square (3).
Since 
$\alpha_{C_Z}=0$, we conclude $\partial_E(\alpha_Z)=0$,
as desired.
This proves the theorem.

\hfill $\blacksquare$

\section{Noncrossed Products and Indecomposable Division Algebras}

A (finite-dimensional) 
division algebra $D$ central over a field $F$ is called a {\it noncrossed product}
if it has no Galois maximal subfield.  Its algebra structure then cannot be given
by a Galois 2-cocycle, counter to almost all known division algebra constructions
(see \cite{Ha85} for a construction of a noncrossed product).  
Noncrossed product division algebras were long conjectured to be fictional, until they
were shown to exist by Amitsur in \cite{Am72}. 

We say $D$ is {\it indecomposable} if it does not contain a subalgebra that is also
central over $F$, or equivalently if it is not an $F$-tensor product of two nontrivial
$F$-division algebras.  It is not hard to show that all division algebras of composite period
are decomposable, and that all division algebras of equal prime-power period 
and index are indecomposable, but it is nontrivial to construct indecomposable division algebras of unequal
prime-power period and index.  
The first examples appeared in \cite{ART} and in \cite{Sa79}.
For additional discussion of either of these
topics, see almost any survey treating open problems on division algebras, such as
\cite{ABGV}, \cite{Am82}, or \cite{Sa92}.

We can use Theorem \ref{preservesindex} to prove the existence of 
noncrossed product and indecomposable division algebras over the function
field $F$ of any $p$-adic curve $X_{\Q_p}$.
Noncrossed products over $K(t)$ for $K$ a local field were first
constructed in \cite{Br01a}, and then constructed more sytematically over
the function field of a smooth relative $\Z_p$-curve in \cite{BMT}.
Indecomposable division algebras of unequal period and index were also constructed in 
\cite{BMT}, over the same types of fields.
Modulo gluing,
the method we use below is the same as the one
used in \cite[Theorem 4.3, Corollary 4.8]{BMT}.

\Theorem\label{noncrossed}
Let $F/\Q_p$ be a finitely generated field extension of transcendence degree one.
Let $X/\Z_p$ be a regular relative curve with function field
$F$, let $C_i$ be a reduced irreducible
component of the closed fiber, let $\ell\neq p$ be a prime, 
and let $r$ and $s$ be numbers that are maximal such that $\mu_{\ell^r}\subset \kappa(C_1)$
and $\mu_{\ell^s}\subset \kappa(C_1)(\mu_{\ell^{r+1}})$.
Then there exist noncrossed product $F$-division algebras of period and index
as low as
$\ell^{s+1}$ if $r=0$, and $\ell^{2r+1}$ if $r\neq 0$.
\rm

\Pf
We may assume (without changing $r$ and $s$) that $C$ has regular irreducible components,
at most two of which meet at any closed point of $X$.
The idea is to use the (known) existence of such algebras over the fields
$F_{C_i}$, modify the construction to
produce a class in $\H^2(\O_{C,\S})\oplus\Gamma$, and then apply
Theorem \ref{map} and Theorem \ref{preservesindex} to prove existence over $F$.

By \cite[Theorem 4.7]{BMT}, if $F$ admits a smooth $X$
then there exist noncrossed product division algebras
over $F_C$ of period and index as low as
$\ell^{s+1}$ if $r=0$, and $\ell^{2r+1}$ if $r>0$.
The resulting Brauer class has the form 
$\alpha_{C}+(\pi)\cdot\theta_{C}\in\H^2(F_{C})$,
where $\alpha_{C}\in\H^2(\kappa(C))$ and $\theta_{C}\in\H^1(\kappa(C),-1)$.
A look at the construction, which proceeds exactly as in \cite[Theorem 1]{Br95},
shows we may pre-assign values at any finite set of points of $C$. 
Thus we may thus produce a noncrossed product $F_{C_1}$-division
algebra with class $\gamma_{C_1}=\alpha_{C_1}+(\pi)\cdot\theta_{C_1}$ 
of the desired period and index, and elements $\gamma_{C_i}=\alpha_{C_i}+(\pi)\cdot\theta_{C_i}$ for $i>1$ 
that glue as in Lemma~\ref{gluing}.  Let
$$
\gamma_C=
(\alpha_{C_1}+(\pi)\cdot\theta_{C_1},\dots,
\alpha_{C_m}+(\pi)\cdot\theta_{C_m})\;\in\H^2(\O_{C,\S})\oplus(\pi)\cdot\Gamma\leq\bigoplus_i\H^2(F_{C_i})
$$
Then $\gamma_C$ lifts to $\gamma=\lambda(\gamma_C)\in\H^2(F)$ by Theorem \ref{map}. 
By Theorem \ref{preservesindex}, $\ind(\gamma)=\ind(\gamma_C)=\ell^{s+1}$ if $r=0$, and $\ell^{2r+1}$ if $r\neq 0$.
It is clear that the $F$-division algebra $D$ associated to $\gamma$ is a noncrossed
product, since any Galois maximal subfield of $D$ over $F$ could be constructed over $F_{C_1}$,
contradicting the fact that $D\otimes_F F_{C_1}$ is a noncrossed product $F_{C_1}$-division algebra.

\hfill $\blacksquare$

\Theorem
Let $F/\Q_p$ be a finitely generated field extension of transcendence degree one,
and let $\ell\neq p$ be a prime.  Then there exist indecomposable $F$-division algebras
of (period,index)$=(\ell^a,\ell^b)$, for any numbers $a$ and $b$ satisfying
$1\leq a\leq b\leq 2a-1$.
\rm

\Pf
Let $X$, $C$, $C_i$, and $\S$ be as in Theorem \ref{noncrossed}.
The construction over $F_{C_i}$ is exactly as in \cite[Proposition 4.2]{BMT}
and \cite{Br96b},
and we merely have to observe that we may assume all of the data
in the constructed class $\gamma_{C_i}=\alpha_{C_i}+(\pi)\cdot\theta_{C_i}\in\H^2(F_{C_i})$
is trivial at
the singular points $\S\cap C_i$, so that by Lemma~\ref{gluing}, we may construct
a class $\gamma_C=\alpha_C+(\pi)\cdot\theta_C$ in 
$\H^2(\O_{C,\S})\oplus(\pi)\cdot\Gamma
\leq\bigoplus_i\H^2(F_{C_i})$ whose $i$-th component is $\alpha_{C_i}+(\pi)\cdot\theta_{C_i}$.
This class lifts to a class $\gamma=\lambda(\gamma_C)$ by Theorem \ref{map}, 
and $\ind(\gamma)=\ind(\gamma_C)$ by Theorem \ref{preservesindex}.
Since the indexes are the same, the division algebra $D$ associated to $\gamma$
is indecomposable, since any decomposition would extend to
$D_{C_i}=D\otimes_F F_{C_i}$, contradicting the construction of $\gamma_{C_i}$.

\hfill $\blacksquare$

\bibliographystyle{abbrv} %other choices are plain or abbrv or alpha
\bibliography{hnx.bib}

\end{document}

\comment
The following is Liu, Proposition 4.3.8!
\Lemma{\!\!\label{reduced}}
Let $X$ be a normal scheme, $D$ a normal crossings divisor on $X$,
$\rho:Y\to(X,D)$ a flat tamely ramified cover, 
and $Z$ a closed integral subscheme of $X$ whose generic point $\xi$ belongs to $X-D$.
Then $\rho^{-1}Z=Z\times_X Y$ is reduced.
In particular, if $R$ is a discrete valuation ring, $X/R$ is a relative curve, and $D$ is horizontal,
then $Y_{0,\red}=X_{0,\red}\times_X Y$.
\endlemma\advance\no by 1

\Pf
Let $U=\Spec A$ be an affine neighborhood of $\xi$ on $X$
(which possibly contains points of $D$),
and let $B$ be the ring of $V=\rho^{-1}U$, a finite flat $A$-algebra.
%If $U$ doesn't contain points of $D$, then we may assume $B$ is an \'etale $A$-algebra,
%and then $B/\frak p B$ is reduced since it is \'etale over the domain $A/\frak p$.
Let $\frak p$ be the prime ideal of $A$ corresponding to $\xi$.
Since $Z$ is reduced, 
$A/\frak p$ is a domain with fraction field $\kappa(\xi)$.
The ring of the fiber $\rho^{-1}(U\cap Z)$ is $B/\frak p B$,
and since $A\to B$ is flat, the map $B/\frak pB\to B\otimes_A \kappa(\xi)$ is injective.
%which injects into
%its rational function ring $\Frac(B/\frak p B)=B\otimes_A\kappa(\xi)$, 
%which is the localization of $B/\frak p B$ 
%at the saturation of $(A/\frak p)-\{0\}$ in $B/\frak p B$,
%which is the complement of the union of minimal primes of $B/\frak p B$.
%%Problem:  Why can't $\Spec B/\frak p B$ have embedded points?
The hypotheses imply $\rho$ is \'etale over $\xi$, hence $B\otimes_A\kappa(\xi)$
is a finite separable $\kappa(\xi)$-algebra, which
is reduced since $\kappa(\xi)$ is reduced.
Therefore $B/\frak p B$ is reduced.
Since any localization of a reduced ring is reduced, and $U\cap Z$ is an arbitrary
affine open subset of $Z$,
we conclude the local rings of $\rho^{-1}Z$ are all reduced, hence $\rho^{-1}Z$ is reduced.
The last statement of the lemma follows immediately, since when $D$ is horizontal, the generic
points of $X_{0,\red}$ avoid $D$, and $\rho:Y\to(X,D)$ is flat by Lemma~\ref{structure}.

\hfill $\blacksquare$

%For example, $A=k[x,y]_{(x,y)}$, $B=A[y^{1/2}]$, then $\kappa(x)=k(y)$, 
%and $B\otimes_A k(y)=\Frac(k[y]_{(y)}[y^{1/2}])=k(y^{1/2})$ clearly contains
%$B\otimes_A A/(x)=B/(x)B$, but $B\otimes A/(y)$ has nilpotents.
\endcomment

\comment
We do not need this!
\proclaim{\the\no.\;\; Proposition/Definition}\label{degree}\advance\no by 1
Let $R$ be a discrete valuation ring,
$X/R$ a regular relative curve with normal crossings,
$D$ a regular horizontal divisor, and
$\rho:Y\to (X,D)$ a tamely ramified cover.  
There is a well defined {\it degree} $[\kappa(Y_0):\kappa(X_0)]$, equal to the least common multiple of the degrees
$[\kappa(Y_0^{(i)}):\kappa(X_0^{(i)})]$, where $X_0^{(i)}$ ranges over the irreducible components of 
$X_0$, $Y_0^{(i)}\df X_0^{(i)}\times_X Y$, and $\kappa(X_0^{(i)})$ and $\kappa(Y_0^{(i)})$ denote the
rational function rings, which are the localizations with respect to the minimal primes.
Moreover, we have
$$
[\kappa(Y_0):\kappa(X_0)]=[\kappa(Y_{0,\red}):\kappa(X_{0,\red})]=[\kappa(Y):\kappa(X)]
$$
\endproclaim

Why do we even need this, it's only used to prove equality of ramification behavior in the
lemma after next, and then we recreate the proof!  We can delete it.

But here is the correct statement:

If $\rho:Y\to X$ is a finite flat morphism with $X$ noetherian and connected,
then $\rho$ has a well defined degree, equal to the rank 
$[\kappa(Z\times_X Y)):\kappa(Z)]$, where $Z\subset X$ is any irreducible subscheme, and $\kappa(Z)$ and
$\kappa(Z\times_X Y)$ are the rational function rings.
\endcomment

\comment

This is no longer needed: 2/28/2011
\Lemma{\!\!\label{lemma1}}
Let $R$ be a complete discrete valuation ring with residue field $k$,
and let $X/R$ be a regular relative curve with normal crossings.
Then
\begin{enumerate}
\item[i)]
$X$ is connected if and only if $X_0$ is connected.
\item[ii)]
Any horizontal divisor $D$ on $X$ is a disjoint union of
local schemes, each finite over $\Spec R$.
In particular, any irreducible horizontal divisor has a single closed point.
%Note, each local scheme may be reducible or nonreduced.
\end{enumerate}
\endlemma\advance\no by 1


\Pf
Since $R$ is henselian, (i) follows from \cite[18.5.19]{EGAIV:d}.
Suppose $D$ is horizontal.
Since $D\to \Spec R$ is finite, $D$ is a local scheme $\Spec S$, where $S$ is a finite $R$-algebra.
Since $R$ is henselian, $S$ is a product of finite local $R$-algebras by \cite[I.4.2]{M}, proving (ii).

\hfill $\blacksquare$

So here we see the issue.  A horizontal divisor that is not a mark has
$[\kappa(D):K]\neq [\kappa(D_{0,\red}):k]$, so the (absolute) inertia and ramification degrees of $v_D$ and
$v_{D_0}$ are not equal, over $R$.  But the relative degrees agree, if $Y\to X$
is a cover, and this is all we need.
\endcomment

\comment

Here is the lemma adapted from Liu's proof of distinguished divisor lifting (\cite[8.3.35]{Liu}).

\Pf
Since $X$ has normal crossings, each irreducible component of $X_{0,\red}$
determines a regular parameter $t$ that is part of a regular system
of parameters $\{f,t\}\subset\O_{X,x}$ at $x$.
Thus $\{f,t\}$ determines a basis for the $\kappa(x)$-vector space $\frak m_x/\frak m_x^2$.
Since $f\not\in(t)$, and there is at most one other component of $X_{0,\red}$ running through
$x$, either $f$ or $f+t$ determines a regular divisor not part of $X_{0,\red}$,
and still forms with $t$ a regular system of parameters.  We may assume, then,
that $\div\,f$ is not part of the closed fiber at $x$.

Moreover if we just assume that the irreducible components of $X_{0,\red}$ are regular and
that two of them meet at $x$, then we can choose $f$ so that $\div\,f$ is transverse
to both components.  For suppose $t_1$ and $t_2$ are local equations for the two components
of $X_{0,\red}$ meeting at $x$, and we have $f_1,f_2\in\O_{X,x}$ such that 
$(f_1,t_1)=(f_2,t_2)=\frak m_x$.  If either $(f_2,t_1)=\frak m_x$ or $(f_1,t_2)=\frak m_x$,
then we are done.  If neither, then $(f_1+f_2,t_1)=(f_1+f_2,t_2)=\frak m_x$:
For since $(f_j,t_i)\neq\frak m_x$ for each $i\neq j$, we have $(\bar f_j,\bar t_i)=(\bar t_i)$
in the $k$-vector space $\frak m_x/\frak m_x^2$, hence 
$(\bar f_i+\bar f_j,\bar t_i)=(\bar f_i,\bar t_i)=\frak m_x/\frak m_x^2$.

Clearly the closed point $x$ is an irreducible component of $\div\, f\cap \div\, t$, and 
since $\div\,f$ is not part of the closed fiber at $x$, there
exists an affine open subset $W\subset X$ containing $x$ such that $\div\, f\cap W_{0,\red}=\{x\}$.
Let $E=\div\, f\cap W$, an effective divisor on $W$, and let $E_0=E\cap W_0$.
The structure map $E\to \Spec R$ is quasi-projective, since $W\subset X$ is open and $X\to \Spec R$ is projective,  
and it is quasi-finite since $\dim E_0<\dim W_0=1$ (\cite[2.5.15]{Liu}).
By Zariski's Main Theorem, there exists a finite $\Spec R$-scheme $Z$, an 
affine open neighborhood $U$ of $x$ contained in $W$, and an open immersion $i:E\cap U\to Z$.
Let $D\subset Z$ be the connected component containing $i(x)$.
Since $Z\to \Spec R$ is finite and $R$ is henselian, $Z$ is a disjoint union of its connected components,
all of which are local schemes, finite over $\Spec R$ (\cite[I.4.2]{M}).
Therefore $D$ is a local scheme, finite over $\Spec R$, and since $i$ is an open immersion whose image
already includes the closed point $i(x)$ of $D$,
the image $i(E\cap U)$ contains all of $D$, hence $i^{-1}(D)\isom D$.

We identify $D$ with $i^{-1}(D)$.
Since $D$ is cut out locally at its closed point $x$ by the parameter $f$, it is irreducible and regular.
Since $D\to \Spec R$ is finite, $D$ is closed in $X$:  For a finite morphism of affine schemes is projective
(\cite[Exercise 3.3.22]{Liu}),
hence proper (\cite[3.3.30]{Liu}),
and since the composition $D\to X\to \Spec R$ is proper and $X\to \Spec R$ is separable, 
it follows that $D\to X$ is proper (\cite[3.3.16(e)]{Liu}), hence closed.  
We conclude that $D$ is an irreducible closed subscheme of $X$, finite over $\Spec R$,
hence that it is a regular irreducible horizontal divisor, and $D\cap X_{0,\red}=\{x\}$.

\hfill $\blacksquare$
\endcomment

\comment
\Remark\advance\no by 1
It can be shown that 
if $\kappa(\mu_n)/k$ is cyclic, then $\H^q(\O_{C,\S},\Z/n(q-2))\to\bigoplus_i\H^q(\kappa(C_i),\Z/n(q-2))$ 
is injective for $q=0,1,2$, using \cite[Theorem 5.10]{Sa08, Theorem 5.10} (for $q=1$) 
and \cite{Mkv, Theorem 7} (for $q=2$).

Now suppose that $\kappa(\mu_n)/k$ is cyclic, and let $\Lambda=\Z/n(q-2)$.
Since $\kappa(\mu_n)/k$ is cyclic, the surjectivity of $\H^1(\kappa(C_i),Z/n)\to\H^1(\kappa(C_i)_z,\Z/n)$ 
follows by Theorem 5.10, \cite{Sa1}, which proves the $q=2$ case.

Suppose $q=3$.
Since the $C_i$ are regular at $z$, $\O_{C_i,z}$ is a discrete valuation 
ring with residue field $\kappa(z)$ and function field $\kappa(C_i)$.
By e.g. Section 3.6 and Proposition 3.3.1 in \cite{C-T}, we have a commutative diagram
$$
\alignat 3
0\to&\H^2(\O_{C_i,z},\mu_n)\to &\H^2(\kappa(C_i),\mu_n)\to&\H^1(\kappa(z),\Z/n)\to 0\\
&\qquad\big\downarrow &\qquad\big\downarrow &\qquad\big\|\\
0\to&\H^2(\kappa(z),\mu_n)\to &\H^2(\kappa(C_i)_z,\mu_n)\to&\H^1(\kappa(z),\Z/n)\to 0\\
\endalignat
Since again $\kappa(\mu_n)/k$ is cyclic, the left downarrow is surjective by
Theorem 7 in \cite{Mkv}, hence the middle downarrow is surjective, and the result follows.
\endcomment

\comment
We return to $X_0$ the closed fiber of a regular relative curve $X/R$ over a 
complete discrete valuation ring $R$.
Let $\{X_0^{(i)}\}$ denote the set of irreducible components of $X_0$.
If $X_0$ is reduced, $\alpha_C\in\H^q(\O_{X_0,\S},\Lambda)$,
$\alpha_C^{(i)}$ is the image of $\alpha_C$ in $\H^q(\kappa(X_0^{(i)}),\Lambda)$,
and $D_0\in X_0^{(i)}$ is a prime divisor, then we {\it define}
$$
\align
\partial_{D_0}(\alpha_0)\;&\df\;\partial_{D_0}(\alpha_0^{(i)})\\
\alpha_0(D_0)\;&\df\;\alpha_0^{(i)}(D_0)\quad\text{(if $\partial_{D_0}(\alpha_0)=0$)}
\endalign
$$
If $D_0\in X_0^{(i)}\cap X_{0,j}\subset \S$, then $\alpha_0$ is unramified at $D_0$,
with well defined value $\alpha_0(D_0)=\bar\alpha_0\in\H^q(\kappa(D_0),\Lambda)$, as in Lemma~\ref{gluing}.
If $X_0$ is not reduced, then we define the corresponding objects for $\alpha_0\in\H^q(\O_{X_0,\S},\Lambda)$
with respect to a divisor $D_0$ on $X_0$ such that $D_0\times_{X_0} X_{0,\red}$ is prime
via the natural isomorphisms in cohomology induced by the category equivalence between
$(X_0)_\et$ and $(X_{0,\red})_\et$ in \cite[II.3.11]{M}, which is compatible
with the definition of $\Lambda$ by \cite[V.2.4(c)]{M}.
Technically we might say that, if $X_0$ is not reduced, an element
$\alpha_0\in\H^q(\O_{X_0,\S},\Lambda)$ is ramified with respect to the {\it class}
of divisors $\{D_0':D_0'\times_{X_0}X_{0,\red}=D_0\times_{X_0}X_{0,\red}\}$,
but we will not bother with this subtlety.
\endcomment

\comment
%LOOK ET:  This is the old argument, replaced by the one just above.
Since we do not assume $X_0$ is regular,
the existence of the Gysin sequence for $X_0$ is not automatic.
We produce it as follows.
%First, we may assume that $X_0$ is reduced, by the category equivalence \cite{M, II.3.11} mentioned above.
%The cohomology groups are the right derived functors of the global sections functor
%so the groups will be isomorphic.  The maps between cohomology groups are
%induced by the spectral sequence, which are all maps of these graded objects, all will
%be isomorphic.  
Let $j:U_0\to X_0$ and $i:D_0\to X_0$ be the open and closed immersions, respectively.
%The map $\Lambda\to j_*j^*\Lambda$ is an isomorphism by direct computation \cite{M, II.3.2} 
%(or relative purity \cite{M, VI.5.1}), so $\H^0(X_0,\Lambda)\to\H^0(U_0,\Lambda)$ is an isomorphism.
We claim
$$
\R^q j_*j^*\Lambda=\cases
\Lambda &\text{ if $q=0$}\\
i_*i^*\Lambda(-1) &\text{ if $q=1$}\\
0 &\text{ if $q\geq 2$}
\endcases
$$
Then the Leray spectral sequence
$$\H^p(X_0,\R^q j_*j^*\Lambda)\Rightarrow\H^{p+q}(U_0,\Lambda)$$
degenerates, $\H^0(X_0,\Lambda)\isom\H^0(U_0,\Lambda)$, 
and we obtain the desired sequence
$$
0\to\H^1(X_0,\Lambda)\to
\H^1(U_0,\Lambda)\xrightarrow{\;\partial_{D_0}\;}\H^0(D_0,\Lambda(-1))\to\H^2(X_0,\Lambda)\to\dots
$$
Again we use the notation $\partial_{D_0}$ since this map is compatible with
the residue map.

We prove the claim.
There is a canonical map $\Lambda\to j_*j^*\Lambda$, 
an injection $\R^q j_*j^*\Lambda\to i_*\mathscr H_{D_0}^{q+1}(X_0,\Lambda)$ for $q\geq 1$,
and a natural map $i_*i^*\Lambda(-1)\to i_*\mathscr H_{D_0}^2(X_0,\Lambda)$, where the degree
is $2$ since $D_0$ has pure codimension one in $X_0$ (see \cite[VI.5]{M}).
We now compare values on stalks.
If $\bar x$ is a geometric point centered at a point $x\neq D_0$ (which could be a singular point), 
then $\Lambda_{\bar x}=(j_*j^*\Lambda)_{\bar x}$ and 
$(\R^q j_*j^*\Lambda)_{\bar x}=0$ for $q>0$,
since then $j$ is an isomorphism in a neighborhood of $x$,
and $(i_*\mathscr H_{D_0}^q(X_0,\Lambda))_{\bar x}=(i_*i^*\Lambda(-1))_{\bar x}=0$,
since $x$ is not in the image of $i$.
If $x=D_0$, then $x$ is a regular point of $X_0$ on a regular (irreducible) component,
and we may use Gabber's purity theorem to 
compute $(\Lambda)_{\bar x}\isom(j_*j^*\Lambda)_{\bar x}$, 
$(\R^1 j_*\Lambda)_{\bar x}=(i_*\mathscr H_{D_0}^2(X_0,\Lambda))_{\bar x}=(i_*i^*\Lambda(-1))_{\bar x}$,
and $(\R^q j_*j^*\Lambda)_{\bar x}=(i_*\mathscr H_{D_0}^{q+1}(X_0,\Lambda))_{\bar x}=0$ for $q\geq 2$.
We conclude $\Lambda\isom j_*j^*\Lambda$, 
$$\R^1 j_*j^*\Lambda\isom i_*\mathscr H_{D_0}^2(X_0,\Lambda)\isom i_*i^*\Lambda(-1)\,,$$ and
$\R^q j_*j^*\Lambda=0$ for $q\geq 2$.  This proves the claim.
%(  We could have used relative purity, but for that we need the curves to be smooth over $k$,
%so $k$ would have to be perfect.  This is fine for our purposes, but why do it.
%Again note the result in Milne's book refers only to a local isomorphism, but that is enough
%for our purposes.
%but this applies to the canonical map $i_*i^*\Lambda(-1)\to i_*\mathscr H_{D_0}^2(\Lambda)=\R^1 j_*\Lambda$, 
%see e.g. \cite{C-T, 3.2}.)
\endcomment

\comment
Here is an issue:  Suppose $D\neq D'$ are prime, have the same restriction to $X_{0,\red}$, and are distinct
on $X_0$.  That is, they have different directions, but the same closed point.  Suppose
further that there are 
two characters $\theta_0$ and $\theta_0'$ in $\H^1(\O_{X_0,\S})$, with ramification
$D_0$ and $D_0'$.  Then we have two cyclic extensions of $\kappa(X_0)$, both \'etale, but corresponding
to the same cyclic \'etale extension of $\kappa(X_{0,\red})$.
By one definition we would lift to $D$, by another we would lift to $D'$.
If the two cyclic extensions of $\kappa(X_0)$
are different, this would interfere with the category equivalence between $(\Spec \kappa(X_{0,\red}))_\et$
and $(\Spec \kappa(X_0))_\et$.
We must conclude that the extensions of $\kappa(X_0)$ are identical.
\endcomment




\flushpar
{\bf References.}



\begin{enumerate}
\item[[Am72]]
Amitsur, S. A.:
{\it On central division algebras}, Israel J.\ Math.\
{\bf 12} (1972), 408--422.
\item[[Am82]]
Amitsur, S. A.:
Division algebras, a survey, in {\it Algebraists' homage:
papers in ring theory and related topics} ({N}ew {H}aven, {C}onn.,
1981), Contemp.\  Math., vol.\ 13, Amer.\ Math.\ Soc., Providence, R.I.,
1982, pp.\ 3--26.
\item[[AM]]
Atiyah, M. F., MacDonald, I. G.:
{\it Introduction to Commutative Algebra},
Addison-Wesley, Reading, 1969.
\item[[ART]]
Amitsur, S. A., Rowen, L. H., and Tignol,  J.-P.:
{\it Division algebras
of degree $4$ and $8$ with involution}, Israel J.\ Math.\ {\bf 33}
(1979), 133--148.
\item[[ABGV]]
Auel, A., Brussel, E., Garibaldi, S., and Vishne, U.:
Open problems on central simple algebras,
{\it Transform. Groups} {\bf 16} no.1, (2011) pp. 219--264.
\item[[Br95]]
Brussel, E. S.:
Noncrossed products and nonabelian crossed products over $\Q(t)$ and $\Q((t))$,
{\it Amer. J. Math.},
{\bf 117}, no.2, (1995) pp. 377--393.
\item[[Br96]]
Brussel, E. S.:
Decomposability and embeddability of discretely Henselian division algebras,
{\it Israel J. Math.} {\bf 96} (1996), pp. 141--183.
\item[[BMT]]
Brussel, E., McKinnie, K., Tengan, E.:
Indecomposable and noncrossed product division algebras over function fields of smooth
$p$-adic curves, {\it Adv. in Math.} {\bf 226} (2011) pp. 4316--4337.
\item[[C-T]]
Colliot-Th\'el\`ene, J.-L.:
Birational invariants, purity and the Gersten conjecture,
in {\it $K$-Theory and Algebraic Geometry: Connections with Quadratic Forms
and Division Algebras, Proceedings of Symposia in Pure Mathematics},
B. Jacob and A. Rosenberg eds.,
Vol. 58.1, pp. 1-64,  Amer. Math. Soc., Providence, RI 1995.
\item[[EGA IV]]
Grothendieck, A., Dieudonn\'e, J.:
{\it Etude locale des sch\'emas at des morphismes de sch\'emas,
quatri\`eme partie},
Inst. Hautes \'Etudes Sci. Publ. Math. 32 (1967), 5-361.
\item[[FK]]
Freitag, E., Kiehl, R.:
{\it Etale Cohomology and the Weil Conjecture},
Springer-Verlag, Berlin, 1988.
\item[[Fch]]
Fuchs, L.:
{\it Infinite Abelian Groups, Volume 1}, Academic Press, New York, 1970.
\item[[Fuj]]
Fujiwara, K.:  A proof of the absolute purity conjecture (after Gabber), Algebraic Geometry 2000,
Azumino (Hotaka), {\it Adv. Stud. Pure Math.}, Vol. 36, Math Soc. Japan, Tokyo, 2002, pp. 153-183.
\item[[GM]]
Grothendieck, A, Murre, J.:
The Tame Fundamental Group of a Formal Neighborhood of a Divisor with Normal
Crossings on a Scheme,
{\it Lecture Notes in Mathematics 208}, Springer-Verlag, New York, 1971.
\item[[GMS]]
Garibaldi, S., Serre, J.-P., Merkurjev, A.:
{\it Cohomological Invariants in Galois Cohomology, (University Lecture Series Volume 28)},
American Mathematical Society, 2003.
\item[[L]]
Lang, S.:  {\it Algebra (Revised 3rd Edition)}, 
Springer, 2002.
\item[[Ltb]]
Lichtenbaum, S.:
Curves over discrete valuation rings, {\it Am. J. Math.} {\bf 107} (1978), pp. 151-207.
\item[[Liu]]
Liu, Q.:
{\it Algebraic Geometry and Arithmetic Curves,}
Oxford, New York,  2002.
\item[[Mtm]]
Matsumura, H.:
{\it Commutative Ring Theory, (Cambridge Studies in Advanced Mathematics 8)}, 
Cambridge University Press, Cambridge, 1989.
%\item[[Mkv]]
%Merkurjev, A. S.:
%On the Structure of the Brauer group of fields,
%{\it Math. USSR-Izv} {\bf 27} (1986), no.1, 141--157.
\item[[M]]
Milne, J. S.:
{\it Etale Cohomology.}
Princeton University Press, New Jersey, 1980.
\item[[OPS]]
Ojanguren, M., Parimala, R., Sridharan, R.:
Ketu and the second invariant of a quadratic space,
{\it $K$-Theory} {\bf 7} (1993), pp. 501-515.
\item[[Ras]]
Raskind, W.: 
Abelian class field theory of arithmetic schemes,
in {\it $K$-Theory and Algebraic Geometry: Connections with Quadratic Forms
and Division Algebras, Proceedings of Symposia in Pure Mathematics},
B. Jacob and A. Rosenberg eds.,
Vol. 58.1, Amer. Math. Soc., Providence, RI 1-64.
\item[[Sai]]
Saito, S.:
Class field theory for curves over local fields,
{\it J. Num. Theory} {\bf 21} (1985), 44--80.
\item[[Sa79]]
Saltman, D. J.:
{\it Indecomposable division algebras}, Comm.\ Algebra
{\bf 7} (1979), no.\ 8, 791--817.
%\item[[Sa82]]
%Saltman, D. J.:
%Generic Galois extensions and problems in field theory,
%{\it Adv. in Math.} {\bf 43} (1982), no.3, 250--283.
\item[[Sa92]]
Saltman, D. J.:
Finite dimensional division algebras,
appearing in {\it Azumaya algebras, actions, and modules}, Contemp. Math., vol. 124,
{\it Amer. Math. Soc.}, 1992, pp. 203--214.
%\item[[Sa98]]
%Saltman, D. J.:
%Correction to division algebras over $p$-adic curves, 
%{\it J. Raman. Math. Soc.}, {\bf 13} (1998) 125-129.
%\item[[Sa01]],
%Saltman, D.J.:
%Amitsur and division algebras, in {\it Selected papers of
%{S}.{A}.  {A}mitsur with commentary}, part 2, Amer. Math. Soc., 2001,
%pp.\ 109--115.
\item[[SGA 1]]
Grothendieck, A.: {\it Rev\^etements \'etales et groupe fondamental}, Fasc. I: Expos\'es 1 \`a 5,
Vol. 1960/61 of S\'eminaire de G\'eom\'etrie Alg\'ebrique, IHES, Paris.
\end{enumerate}



\bye.


