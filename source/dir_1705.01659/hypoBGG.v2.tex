\documentclass[reqno,12pt]{amsart}


\usepackage[all]{xypic}
\usepackage{enumerate}
\usepackage{hyperref}
%\usepackage{a4}
%\usepackage[bbgreekl]{mathbbol}
\usepackage{amssymb}
\usepackage{fullpage}
%\usepackage{color}
%\usepackage{amsfonts}
%\usepackage{bbm}
%\DeclareSymbolFontAlphabet{\mathbb}{AMSb}
%\DeclareSymbolFontAlphabet{\mathbbl}{bbold}
\usepackage{scalefnt}

\DeclareMathOperator{\pv}{pv}
\DeclareMathOperator{\Op}{Op}
\DeclareMathOperator{\pr}{pr}
\DeclareMathOperator{\img}{img}
\DeclareMathOperator{\coker}{coker}
\DeclareMathOperator{\ind}{ind}
\DeclareMathOperator{\tind}{t-ind}
\DeclareMathOperator{\supp}{supp}
\DeclareMathOperator{\gr}{gr}
\DeclareMathOperator{\ev}{ev}
\DeclareMathOperator{\rank}{rank}
\DeclareMathOperator\singsupp{sing-supp}
\DeclareMathOperator{\tr}{tr}
\newcommand\goe{\mathfrak g}
\newcommand\poe{\mathfrak p}
\newcommand\noe{\mathfrak n}
\newcommand\zoe{\mathfrak z}
\newcommand\llangle{\langle\!\langle}
\newcommand\rrangle{\rangle\!\rangle}
\newcommand\bbb{|\mkern-2mu|\mkern-2mu|}
\DeclareMathOperator{\Ad}{Ad}
\DeclareMathOperator{\Ext}{Ext}
\DeclareMathOperator{\ad}{ad}
\DeclareMathOperator{\id}{id}
\DeclareMathOperator{\Aut}{Aut}
\DeclareMathOperator{\eend}{end}

\newcommand{\DO}{\mathcal D\mathcal O}
\newcommand{\tDO}{\widetilde{\DO}}

\newcommand\Z{\mathbb Z}
\newcommand\N{\mathbb N}
\newcommand\R{\mathbb R}
\newcommand\C{\mathbb C}
\newcommand\K{\mathbb K}

\newcommand\ii{\mathbf i}

\newcommand\op{\textrm{op}}
\newcommand\cl{\textrm{classical}}
\newcommand\prop{\textrm{prop}}
\newcommand\loc{\textrm{loc}}

\newcommand\itemref[1]{(\ref{#1})}
%\newcommand\red[1]{\textcolor{red}{#1}}

\newcounter{ABC}
\renewcommand{\theABC}{\Alph{ABC}}

\theoremstyle{plain}
  \newtheorem{theorem}{Theorem}[section]
  \newtheorem{corollary}[theorem]{Corollary}
  \newtheorem{lemma}[theorem]{Lemma}
  \newtheorem{proposition}[theorem]{Proposition}
  \newtheorem{thm}[ABC]{Theorem}
\theoremstyle{definition}
  \newtheorem{definition}[theorem]{Definition}
  \newtheorem{example}[theorem]{Example}
  \newtheorem{remark}[theorem]{Remark}






\begin{document}




\title[]{Graded hypoellipticity of BGG sequences}

\author{Shantanu Dave}

\thanks{S.~D.\ was supported by the Austrian Science Fund (FWF) grants P24420 and P28770.}

\address{Shantanu Dave,
         Wolfgang Pauli Institute
         c/o Faculty of Mathematics,
         University of Vienna,
         Oskar-Morgenstern-Platz 1,
         1090 Vienna,
         Austria.}

\email{shantanu.dave@univie.ac.at}

\author{Stefan Haller}

\address{Stefan Haller, 
         Department of Mathematics,
         University of Vienna,
         Oskar-Morgenstern-Platz 1,
         1090 Vienna,
         Austria.}

\email{stefan.haller@univie.ac.at}

\thanks{The second author would like to express his gratitude to the Max Planck Institute for Mathematics in Bonn for the hospitality and financial support during an extended visit which enabled him to pursue this project.
Moreover, he gratefully acknowledges the support of the Austrian Science Fund (FWF) through the START-Project Y963-N35 of Michael Eichmair.}

\begin{abstract}
In this paper we introduce the notion of Rockland sequences of differential operators on filtered manifolds.
This concept generalizes the notion of elliptic sequences and is formulated in terms of a Rockland condition.
We show that these sequences are hypoelliptic and have analytic properties similar to elliptic sequences, providing regularity, maximal hypoelliptic estimates, and Hodge decomposition.
These analytic properties follow from a generalization of the Rockland theorem using the Heisenberg tangent groupoid construction.
The main motivation lies in the fact that Bernstein--Gelfand--Gelfand sequences over regular parabolic geometries are Rockland sequences in a graded sense.
We use the BGG machinery to construct similar sequences for a large class of filtered manifolds, and illustrate these results in some explicit examples.
\end{abstract}

\keywords{Filtered manifold; pseudodifferential operator; hypoelliptic operator; hypoelliptic sequence; BGG sequence; Rockland sequence; Rumin--Seshadri operator; Engel structure; generic rank two distribution in dimension five}

\subjclass[2010]{58J40 (primary) and 58A30, 58A14, 58J10 (secondary)}

\maketitle
\tableofcontents




\section{Introduction}\label{S:intro}





A geometry in the sense of Cartan is a manifold that looks like a homogeneous space.
A homogeneous space is a space $X$ with a large group of symmetries, a Lie group $G$ that acts transitively on $X$ and thus $X=G/H$.
A homogeneous space $X$ has a principal $H$-bundle $G\rightarrow X$ which comes equipped with the Maurer--Cartan form.
Cartan's notion of ``looks like a homogeneous space'' is based upon similar data, the Cartan connection, on a principal $H$-bundle over a smooth manifold.
In the special case when $G$ is a semisimple Lie group and $H$ is a parabolic subgroup, these geometries are known as parabolic geometries, see \cite{CS09} for an introduction.


Traditionally the Lie groups $G$ and $H$ and their representation theory are employed to study Cartan geometries.
For parabolic geometries, \v Cap, Slov\'ak and Sou\v cek \cite{CSS01} have constructed a large class of natural differential operators called (curved) Bernstein--Gelfand--Gelfand sequences or BGG sequences in short, see also \cite{CD01,CS15}.
One aim of this article is to show that BGG sequences over regular parabolic geometries enjoy hypoellipticity.


The manifold underlying a regular parabolic geometry inherits the structure of a filtered manifold.
Using the BGG machinery, we will construct similar sequences for a large class of filtered manifolds, and describe in what sense they are hypoelliptic.
These BGG type sequences depend on a linear connection on a filtered vector bundle and a Kostant type codifferential.
The classical BGG sequences over regular parabolic geometries can be recovered from our construction by a suitable and natural choice of these ingredients.





\subsection{Filtered manifolds and their osculating groups}





A natural structure available on every smooth manifold is its Lie algebra of vector fields.
Various geometric structures on smooth manifolds can be described in terms of a filtration on the tangent bundle which is compatible with the Lie bracket of vector fields.
This compatibility is subsumed in the concept of a filtered manifold.
A filtered manifold is a smooth manifold $M$ together with a filtration of its tangent bundle by smooth subbundles,
$$
TM=T^{-r}M\supseteq\cdots\supseteq T^{-2}M\supseteq T^{-1}M\supseteq T^0M=0,
$$
which is compatible with the Lie bracket of vector fields in the sense that $[X,Y]\in\Gamma^\infty(T^{p+q}M)$ for all $X\in\Gamma^\infty(T^pM)$ and $Y\in\Gamma^\infty(T^qM)$.


To each point $x$ in a filtered manifold $M$ one can assign a simply connected nilpotent Lie group $\mathcal T_xM$, called the osculating group at $x$, which can be regarded as a non-commutative analogue of the tangent space at $x$.
Its Lie algebra is given by
$$
\mathfrak t_xM:=\gr(T_xM):=\sum_pT_x^pM/T_x^{p+1}M
$$
with the (Levi) bracket induced from the Lie bracket of vector fields.
The osculating algebras combine to form a smooth bundle of graded nilpotent Lie algebras $\mathfrak tM$ over $M$, called the bundle of osculating algebras.
Correspondingly, the osculating groups combine to form a smooth bundle of simply connected nilpotent Lie groups $\mathcal TM$ over $M$, called the bundle of osculating groups. 
These play a crucial role in the analysis on filtered manifolds.


Let us mention a few geometric structures that can be described as filtered manifolds with special osculating algebras.
By Frobenius' theorem, foliated manifolds can equivalently be described as filtered manifolds with abelian, but non-trivially graded osculating algebras.
A contact manifold is just a filtered manifold with osculating algebras isomorphic to the Heisenberg algebra.
An Engel manifold is a filtered manifold with osculating algebras isomorphic to a particular (generic) $4$-dimensional graded nilpotent Lie algebra, see Example~\ref{Ex:Engel}. 
A generic rank two distribution in dimension five \cite{C10} is a filtered manifold with osculating algebras isomorphic to a particular (generic) $5$-dimensional graded nilpotent Lie algebra, see Example~\ref{Ex:BGG235}.
Most regular normal parabolic geometries can equivalently be described as filtered manifolds with prescribed osculating algebras, see \cite[Proposition~4.3.1]{CS09}.
Heisenberg manifolds constitute a well studied \cite{P08} class of filtered manifolds with varying osculating algebras which is related to the geometry at the boundary of complex manifolds.




\subsection{Analytic results} 



In this paper we will study operators on filtered manifolds and provide a criterion for their hypoellipticity.
Our main analytic results are based on the calculus for filtered manifolds recently developed by van~Erp and Yuncken.
This calculus is based on the construction of the Heisenberg tangent groupoid \cite{EY15,EY16,CP15} and the idea of essential homogeneity introduced in \cite{DS14}.
In this paper we extend their calculus to operators acting between sections of vector bundles, and complement it with a general Rockland type theorem.
The proof of the latter builds upon harmonic analysis by Christ, Geller, G{\l}owacki, and Polin \cite{CGGP92} and arguments due to Ponge \cite{P08}.
If the underlying manifold is trivially filtered, we recover the classical pseudodifferential calculus, see \cite{DS14}.


For two vector bundles $E$ and $F$ over a filtered manifold $M$, and any complex number $s$, we obtain a class of operators, denoted by $\Psi^s(E,F)$, we will refer to as pseudodifferential operators of Heisenberg order $s$.
These are continuous operators $\Gamma^\infty_c(E)\to\Gamma^\infty(F)$ which extend continuously to pseudolocal operators on distributional sections, $\Gamma^{-\infty}_c(E)\to\Gamma^{-\infty}(F)$.
They can be characterized as operators with a Schwartz kernel that admits an extension to the Heisenberg tangent groupoid which is essentially homogeneous of order $s$.
This extends the Heisenberg filtration on differential operators, that is, a differential operator has Heisenberg order at most $k\in\N_0$ if and only if it is contained in $\Psi^k(E,F)$.


An operator $A\in\Psi^s(E,F)$ has a Heisenberg principal cosymbol $\sigma_x^s(A)\in\Sigma^s_x(E,F)$ at every point $x\in M$.
Here $\Sigma^s_x(E,F)$ denotes the space of regular distributional volume densities on the osculating group $\mathcal T_xM$ with values in $\hom(E_x,F_x)$ which are essentially homogeneous of order $s$, modulo smooth volume densities.
This cosymbol extends the Heisenberg principal (co)symbol of differential operators on filtered manifolds.
The basic properties of this operator class and the Heisenberg principal cosymbol are summarized in Proposition~\ref{P:Psi} below.


Let $\pi\colon\mathcal T_xM\to U(\mathcal H)$ be a non-trivial irreducible unitary representation of the osculating group on a Hilbert space $\mathcal H$, and let $\mathcal H_\infty$ denote the subspace of smooth vectors.
If $A\in\Psi^s(E,F)$, then $\bar\pi(\sigma_x^s(A))$ is a well defined closed unbounded operator on $\mathcal H$, restricting to a map $\bar\pi(\sigma^s_x(A))\colon\mathcal H_\infty\otimes E_x\to\mathcal H_\infty\otimes F_x$.
The operator $A$ is said to satisfy the Rockland condition if $\bar\pi(\sigma^s_x(A))$ is injective on $\mathcal H_\infty\otimes E_x$ for all non-trivial irreducible unitary representations $\pi$ of $\mathcal T_xM$ and every $x\in M$.


The core analytic result is the following Rockland type theorem for general filtered manifolds.
For left invariant differential operators on graded nilpotent Lie groups this was conjectured by Rockland \cite{R78} and proved by Helffer and Nourrigat \cite{HN79}.
Christ, Geller, G{\l}owacki, and Polin constructed a pseudodifferential operator calculus on graded nilpotent Lie groups and established a Rockland type theorem, see \cite{CGGP92}.
For Heisenberg manifolds with varying osculating algebras such a result has been obtained by Ponge \cite{P08}.


\begin{thm}\label{thmA}
Let $E$ and $F$ be two vector bundles over a filtered manifold $M$ and suppose $A\in\Psi^s(E,F)$ satisfies the Rockland condition.
Then there exists a properly supported left parametrix $B\in\Psi^{-s}_\prop(F,E)$, that is, $BA-\id$ is a smoothing operator.
\end{thm}


As a consequence, every operator $A\in\Psi^s(E,F)$ satisfying the Rockland condition is hypoelliptic, that is, if $\psi$ is a compactly supported distributional section of $E$ such that $A\psi$ is smooth, then $\psi$ was smooth.
Over closed manifolds this implies that $\ker(A)$ is a finite dimensional subspace of $\Gamma^\infty(E)$, see Theorem~\ref{T:Rockland} below.


Combining this calculus with a result of Christ, Geller, G{\l}owacki, and Polin, see \cite[Theorem~6.1]{CGGP92}, we construct, for each complex number $s$, an operator $\Lambda_s\in\Psi^s(E)$ which is invertible mod smoothing operators, see Lemma~\ref{L:Lambda}.
This permits to introduce a Heisenberg Sobolev scale, see Proposition~\ref{P:Hs}, and allows to formulate more refined regularity statements, including maximal hypoelliptic estimates, for operators satisfying the Rockland condition, see Corollary~\ref{C:reg}.


We will use Theorem~\ref{thmA} to analyze Rockland sequences.
A sequence of operators,
$$
\cdots\to\Gamma^\infty(E_{i-1})\xrightarrow{A_{i-1}}\Gamma^\infty(E_i)\xrightarrow{A_i}\Gamma^\infty(E_{i+1})\to\cdots,
$$
where $A_i\in\Psi^{s_i}(E_i,E_{i+1})$ will be called Rockland sequence if the corresponding principal symbol sequence is exact in every non-trivial irreducible unitary representation $\pi\colon\mathcal T_xM\to U(\mathcal H)$ at each $x\in M$, that is, the sequence
$$
\cdots\to\mathcal H_\infty\otimes E_{x,i-1}\xrightarrow{\bar\pi(\sigma^{s_{i-1}}_x(A_{i-1}))}\mathcal H_\infty\otimes E_{x,i}\xrightarrow{\bar\pi(\sigma^{s_i}_x(A_i))}\mathcal H_\infty\otimes E_{x,i+1}\to\cdots
$$
is exact.
On trivially filtered manifolds this definition reduces to the well known concept of elliptic sequences.
To study these sequences we consider formal adjoints $A_i^*\in\Psi^{\bar s}(E_{i+1},E_i)$ with respect to standard $L^2$ inner products on $\Gamma^\infty(E_i)$.
Theorem~\ref{thmA} implies that $(A_{i-1}^*,A_i)$ is hypoelliptic, and more refined regularity statements, including maximal hypoelliptic estimates, can be formulated using the Heisenberg Sobolev scale.


On closed manifolds, in case the Rockland sequence forms a complex, i.e.\ $A_iA_{i-1}=0$, it is convenient to use suitable Sobolev adjoints, $A_i^\sharp$ of $A_i$.
These adjoints are constructed such that $A_{i-1}A_{i-1}^\sharp$ and $A_i^\sharp A_i$ have the same Heisenberg order and thus $B_i:=A_{i-1}A_{i-1}^\sharp+A_i^\sharp A_i$ is a Rockland operator.
We obtain a Hodge decomposition
$$
\Gamma^\infty(E_i)=\img(A_{i-1})\oplus\ker(B_i)\oplus\img(A_i^\sharp)
$$
where $\ker(B_i)=\ker(A_{i-1}^\sharp)\cap\ker(A_i)$.
In particular, each cohomology class has a unique harmonic representative, that is, $\ker(A_i)/\img(A_i)=\ker(B_i)$.


For Rockland complexes of differential operators of positive order, one can alternatively follow the Rumin--Seshadri approach \cite{RS12} and consider 
\begin{equation}\label{E:IRS}
\Delta_i:=(A_{i-1}A_{i-1}^*)^{a_{i-1}}+(A_i^*A_i)^{a_i}
\end{equation}
where the numbers $a_i\in\N$ are chosen such that $\kappa:=s_{i-1}a_{i-1}=s_ia_i$.
Then $\Delta_i$ is a differential operator of order at most $2\kappa$ which satisfies the Rockland condition.
We obtain a similar Hodge decomposition, namely,
$$
\Gamma^\infty(E_i)=\img(A_{i-1})\oplus\ker(\Delta_i)\oplus\img(A_i^*),
$$
where $\ker(A_i)/\img(A_i)=\ker(\Delta_i)=\ker(A_{i-1}^*)\cap\ker(A_i)$.


In order to study the BGG type sequences we will construct below, the analysis needs to be modified to fit the filtered setup.
For any two filtered vector bundles $E$ and $F$ we consider a class of operators, $\tilde\Psi^s(E,F)$, which will be called pseudodifferential operators of graded Heisenberg order $s$.
If we identify $E$ and $F$ with the associated graded using splittings of the filtrations, then an operator $A\in\tilde\Psi^s(E,F)$ can be considered as a matrix with entries $A_{qp}\in\Psi^{s+q-p}(\gr_p(E),\gr_q(F))$.
The graded Heisenberg principal cosymbol, $\tilde\sigma^s_x(A)$, can be defined as the matrix obtained by taking the (ordinary) Heisenberg principal symbol of each entry, that is, 
$$
\tilde\sigma^s_x(A)=\sum_{p,q}\sigma^{s+q-p}_x(A_{qp}).
$$
Neither the operator class $\tilde\Psi^s(E,F)$, nor the graded Heisenberg principal cosymbol $\tilde\sigma^s_x(A)$ depend on the choice of splittings.


An operator $A\in\tilde\Psi^s(E,F)$ is called graded Rockland operator if $\bar\pi(\tilde\sigma^s_x(A))\colon\mathcal H_\infty\otimes\gr(E_x)\to\mathcal H_\infty\otimes\gr(F_x)$ is injective for all non-trivial irreducible unitary representations $\pi\colon\mathcal T_xM\to U(\mathcal H)$ and all $x\in M$.
Similarly, a graded Rockland sequence is defined to be a sequence of operators such that its graded Heisenberg principal symbol sequence is exact in each non-trivial irreducible unitary representation of $\mathcal T_xM$ and every $x\in M$.
Theorem~\ref{thmA} has a graded analogue, and all the analysis mentioned above generalizes to this graded setup, see Section~\ref{S:grRockland}.





\subsection{Construction of Rockland sequences}





In this paper will shall construct several examples of Rockland sequences.
The most basic sequence we will consider is the de~Rham sequence associated with a linear connection $\nabla$ on a filtered vector bundle $E$ over a filtered manifold $M$.
This can be characterized as the unique extension of $\nabla$,
\begin{equation}\label{E:nablaE}
\cdots\to\Omega^{k-1}(M;E)\xrightarrow{d_{k-1}^\nabla}\Omega^k(M;E)\xrightarrow{d_k^\nabla}\Omega^{k+1}(M;E)\to\cdots,
\end{equation}
such that the Leibniz rule $d^\nabla(\alpha\wedge\psi)=d\alpha\wedge\psi+(-1)^k\alpha\wedge d^\nabla\psi$ holds for all $\alpha\in\Omega^k(M)$ and $\psi\in\Omega^*(M;E)$.
Here we use the notation $\Omega^k(M;E)=\Gamma^\infty(\Lambda^kT^*M\otimes E)$ for the space of $E$-valued differential forms.


We assume that $\nabla$ is filtration preserving, that is to say, we assume $\nabla_X\psi\in\Gamma^\infty(T^{p+q}M)$ for all $X\in\Gamma^\infty(T^pM)$ and $\psi\in\Gamma^\infty(E^q)$.
Then all operators in the sequence \eqref{E:nablaE} are of graded Heisenberg order at most zero with respect to the induced filtration on the vector bundles $\Lambda^kT^*M\otimes E$.
We will, furthermore, assume that the curvature of $\nabla$ is contained in filtration degree one, that is, we assume $F^\nabla_x(X_1,X_2)\psi\in E_x^{p_1+p_2+p+1}$ for all $X_i\in T_x^{p_i}M$ and $\psi\in E^p_x$.
Linear connections of this kind exist on every filtered vector bundle.
If $E$ is trivially filtered, then all linear connections on $E$ satisfy the two assumptions.
In general, using a splitting of the filtration to identify $E$ with its associated graded, $\gr(E)=\bigoplus_pE^p/E^{p+1}$, each linear connection preserving the grading on $\gr(E)$ will satisfy the two assumptions.
Moreover, all tractor bundles associated with regular parabolic geometries come equipped with a natural linear connection satisfying these assumptions, see Section~\ref{SS:BGG} below.
 

\begin{thm}\label{thmB}
Let $E$ be a filtered vector bundle over a filtered manifold $M$ and suppose $\nabla$ is a filtration preserving linear connection on $E$ such that its curvature is contained in filtration degree one.
Then the de~Rham sequence in \eqref{E:nablaE} is a graded Rockland sequence.
\end{thm}


Essentially, Theorem~\ref{thmB} follows from the fact that the Lie algebra cohomology $H^*(\goe;\mathcal H_\infty)$ vanishes for every finite dimensional nilpotent Lie algebra $\goe$ and its representation on the space of smooth vectors $\mathcal H_\infty$ associated with any non-trivial irreducible unitary representation of the corresponding simply connected nilpotent Lie group on a Hilbert space $\mathcal H$.
This will be established in Section~\ref{SS:linearconnections}, see Proposition~\ref{P:hypo} below.


To construct new sequences, we follow {\v{C}}ap, Slov{\'a}k, and Sou{\v{c}}ek, see \cite{CSS01}, and consider a Kostant type codifferential.
By this we mean a sequence of filtration preserving vector bundle homomorphisms,
$$
\cdots\leftarrow\Omega^{k-1}(M;E)\xleftarrow{\delta_k}\Omega^k(M;E)\xleftarrow{\delta_{k+1}}\Omega^{k+1}(M;E)\leftarrow\cdots,
$$
satisfying  $\delta_k\delta_{k+1}=0$ and two more conditions formulated in Definition~\ref{D:Kdelta} below.
Assuming $\delta_k$ to have locally constant rank, we obtain smooth vector bundles $\ker(\delta_k)$, $\img(\delta_{k+1})$, and $\mathcal H_k:=\ker(\delta_k)/\img(\delta_{k+1})$, which are filtered in a natural way.
We let $\bar\pi_k\colon\ker(\delta_k)\to\mathcal H_k$ denote the natural vector bundle projection.


Using the BGG machinery \cite{CSS01,CD01,CS12,CS15} we will see that there exist operators analogous to the splitting operators in parabolic geometry, see \cite[Theorem~2.4]{CS12}.
More precisely, there exists a unique differential operator $\bar L_k\colon\Gamma^\infty(\mathcal H_k)\to\Omega^k(M;E)$ such that $\delta_k\bar L_k=0$, $\bar\pi_k\bar L_k=\id$, and $\delta_{k+1}d^\nabla_k\bar L_k=0$.
These operators $\bar L_k$ are of graded Heisenberg order zero and permit to define a sequence of differential operators of graded Heisenberg order zero,
\begin{equation}\label{E:Dbar}
\cdots\to\Gamma^\infty(\mathcal H_{k-1})\xrightarrow{\bar D_{k-1}}\Gamma^\infty(\mathcal H_k)\xrightarrow{\bar D_k}\Gamma^\infty(\mathcal H_{k+1})\to\cdots,
\end{equation}
by setting $\bar D_k:=\bar\pi_{k+1}d^\nabla_k\bar L_k$. 


In Section~\ref{SS:subdeRham} we will establish the following result, see Corollary~\ref{C:D}.


\begin{thm}\label{thmC}
The operators in \eqref{E:Dbar} form a graded Rockland sequence.
\end{thm}


A codifferential $\delta$ of maximal rank exists, provided the dimension of the Lie algebra cohomology $H^*(\mathfrak t_xM;\gr(E_x))$ is locally constant in $x$.
Note that the curvature assumption on $\nabla$ implies that $\gr(E_x)$ becomes a graded representation of the graded nilpotent Lie algebra $\mathfrak t_xM$, see Lemma~\ref{L:curv}.
In this case the codifferential $\delta_k$ can be constructed using splittings of the filtrations on the bundles $\Lambda^kT^*M\otimes E\cong\gr(\Lambda^kT^*M\otimes E)=\Lambda^k\mathfrak t^*M\otimes\gr(E)$ and the adjoint of the fiber wise Chevalley--Eilenberg differential, $\partial_{k-1}\colon\Lambda^{k-1}\mathfrak t^*M\otimes\gr(E)\to\Lambda^k\mathfrak t^*M\otimes\gr(E)$, see Remark~\ref{R:Kdeltaexi} for details.
For this codifferential there exists a (non-canonical) isomorphism of smooth vector bundles $\mathcal H_k\cong H^k(\mathfrak tM;\gr(E))=\ker(\partial_k)/\img(\partial_{k-1})$ where the latter denotes the vector bundle with fibers $H^k(\mathfrak t_xM;\gr(E_x))$.


For tractor bundles associated with regular parabolic geometries, however, there exists a natural choice for $\delta$ which is called \emph{Kostant codifferential} and often denoted by $\partial^*$.
In this case the construction above reduces to the construction of the curved BGG sequences, and the operators $\bar L_k$ coincide with the well known splitting operators, see \cite[Theorem~2.4]{CS12} for instance.
As an immediate corollary of Theorem~\ref{thmC} we thus obtain, cf.\ Corollary~\ref{C:BGG}:


\begin{thm}\label{thmD}
All (curved, torsion free) BGG sequences associated with a regular parabolic geometry are graded Rockland sequences.
\end{thm}


To prove Theorem~\ref{thmC}, we shall construct another sequence that, at the principal symbol level, can be combined with the sequence \eqref{E:Dbar} to obtain the de~Rham sequence of Theorem~\ref{thmB}, up to conjugation. 
The Rockland condition for both components then follows from Theorem~\ref{thmB}.
This construction is closely related to the standard BGG machinery.


More precisely, we consider $\Box_k:=d^\nabla_{k-1}\delta_k+\delta_{k+1}d^\nabla_k$, a differential operator of graded Heisenberg order at most zero on $\Omega^k(M;E)$.
The associated graded vector bundle endomorphism $\tilde\Box_k:=\gr(\Box_k)$ on $\gr(\Lambda^kT^*M\otimes E)$ is analogous to Kostant's box operator.
Using the fiber wise projection onto the generalized zero eigenspace of $\tilde\Box_k$, we obtain a vector bundle projector $\tilde P_k$ on $\gr(\Lambda^kT^*M\otimes E)$, providing a decomposition of smooth vector bundles
$$
\gr(\Lambda^kT^*M\otimes E)=\img(\tilde P_k)\oplus\ker(\tilde P_k)
$$
such that $\tilde\Box_k$ is nilpotent on $\img(\tilde P_k)$ and invertible on $\ker(\tilde P_k)$.
We will construct two sequences of differential operators of graded Heisenberg order at most zero,
\begin{equation}\label{E:DDD}
\cdots\to\Gamma^\infty(\img(\tilde P_{k-1}))\xrightarrow{D_{k-1}}\Gamma^\infty(\img(\tilde P_k))\xrightarrow{D_k}\Gamma^\infty(\img(\tilde P_{k+1}))\to\cdots
\end{equation}
and
\begin{equation}\label{E:BBB}
\cdots\to\Gamma^\infty(\ker(\tilde P_{k-1}))\xrightarrow{B_{k-1}}\Gamma^\infty(\ker(\tilde P_k))\xrightarrow{B_k}\Gamma^\infty(\ker(\tilde P_{k+1}))\to\cdots,
\end{equation}
as well as invertible differential operators 
$$
L_k\colon\Gamma^\infty(\gr(\Lambda^kT^*M\otimes E))\to\Omega^k(M;E),
$$
such that the graded Heisenberg principal symbols are related by 
$$
\tilde\sigma^0_x(L^{-1}_{k+1}d_k^\nabla L_k)=\tilde\sigma^0_x(D_k)\oplus\tilde\sigma^0_x(B_k)
$$
at each point $x\in M$.
Theorem~\ref{thmB} readily implies that \eqref{E:DDD} and \eqref{E:BBB} are both graded Rockland sequences.
Moreover, we will construct an invertible differential operator of graded Heisenberg order zero, $V_k\colon\Gamma^\infty(\img(\tilde P_k))\to\Gamma^\infty(\mathcal H_k)$, such that $V_{k+1}^{-1}\bar D_kV_k=D_k$, whence Theorem~\ref{thmC}.


The construction of the operators announced in the preceding paragraph is based on the observation that there exists a unique filtration preserving differential operator
$$
P_k\colon\Omega^k(M;E)\to\Omega^k(M;E)
$$ 
characterized by $P_k\Box_k=\Box_kP_k$, $P_k^2=P_k$ and $\gr(P_k)=\tilde P_k$.
This operator has graded Heisenberg order zero.
Using splittings of the filtrations, $S_k\colon\gr(\Lambda^kT^*M\otimes E)\to\Lambda^kT^*M\otimes E$, we define differential operators of graded Heisenberg order zero, 
$$
L_k:=P_kS_k\tilde P_k+(\id-P_k)S_k(\id-\tilde P_k).
$$
Since $\gr(L_k)=\id$, this differential operator is invertible and its inverse $L^{-1}_k$ is a differential operator of graded Heisenberg order zero too.
Moreover, it conjugates the differential projectors into vector bundle projectors, $L^{-1}_kP_kL_k=\tilde P_k$.
%Combining this with $\tilde\sigma^0_x(d_k^\nabla)\tilde\sigma^0_x(P_k)=\tilde\sigma^0_x(P_{k+1})\tilde\sigma^0_x(d_k^\nabla)$ we obtain $\tilde\sigma^0_x(\tilde P_{k+1}L^{-1}_{k+1}d_k^\nabla L_k)=\tilde\sigma^0_x(L^{-1}_{k+1}d_k^\nabla L_k\tilde P_k)$.
We will verify that the operators $D_k:=\tilde P_{k+1}L_{k+1}^{-1}d^\nabla_kL_k|_{\Gamma^\infty(\img(\tilde P_k))}$, $B_k:=(\id-\tilde P_{k+1})L_{k+1}^{-1}d^\nabla_kL_k|_{\Gamma^\infty(\ker(\tilde P_k))}$, and $V_k:=\bar\pi_kL_k|_{\Gamma^\infty(\img(\tilde P_k))}$ with inverse $V_k^{-1}=L_k^{-1}\bar L_k$ have all the desired properties.
The operator $P_k$ is related to the splitting operator $\bar L_k$ considered above by $\bar L_k\bar\pi_k=P_k|_{\ker(\delta_k)}$.
On regular parabolic geometries $P_k$ coincides with the composition of (5.1) and (5.2) in \cite{CD01}.


As another application, let us now suppose that the linear connection $\nabla$ is flat.
In this case the sequence \eqref{E:nablaE} is known as de~Rham complex, $d^\nabla_kd^\nabla_{k-1}=0$, and computes the cohomology of $M$ with coefficients in the system of local coefficients provided by the flat connection on $E$.
We will see that the sequence of operators $L_{k+1}^{-1}d^\nabla_kL_k$ decouples into a Rumin complex and an acyclic subcomplex.
More precisely, 
$$
L_{k+1}^{-1}d^\nabla_kL_k=D_k\oplus B_k,
$$
and, in particular, $D_kD_{k-1}=0=B_kB_{k-1}$.
In this situation the sequences in \eqref{E:Dbar} and \eqref{E:DDD} will be called Rumin complexes, for they generalize complexes on contact manifolds which have been introduced by Rumin \cite{R00}.
We will show that the sequence $B_k$ is conjugate to an acyclic tensorial complex.
More precisely, we will see that there exist invertible differential operators of graded Heisenberg order at most zero, $G_k$ acting on $\Gamma^\infty(\ker(\tilde P_k))$, such that 
$$
G_{k+1}^{-1}B_kG_k=\partial_k|_{\Gamma^\infty(\ker(\tilde P_k))}
$$ 
where the right hand side denotes the restriction of the Chevalley--Eilenberg differential on $\gr(\Lambda^kT^*M\otimes E)=\Lambda^k\mathfrak t^kM\otimes\gr(E)$ to the invariant acyclic subbundle $\ker(\tilde P_k)$, see Theorem~\ref{T:D}.
Summarizing, we obtain:


\begin{thm}\label{thmE}
If the linear connection $\nabla$ is a flat, then there exist invertible differential operators of graded Heisenberg order zero, $W_k\colon\Gamma^\infty(\mathcal H_k\oplus\ker(\tilde P_k))\to\Omega^k(M;E)$, such that
$$
W^{-1}_{k+1}d^\nabla_kW_k=\bar D_k\oplus(\partial_k|_{\Gamma^\infty(\ker(\tilde P_k))}).
$$
\end{thm}


On a contact manifold, the Rumin complex \eqref{E:Dbar} is Rockland in the ungraded sense and coincides with the complex introduced by Rumin, see Example~\ref{Ex:Rumin} below.
Hypoellipticity of this complex has been established by Rumin in \cite{R00} using the classical Heisenberg calculus.
For generic rank two distributions in dimension five, the Rumin complex is Rockland in the ungraded sense too, see Example~\ref{Ex:BGG235}.
In general, the Rumin complex will only be Rockland in the graded sense, and the graded analysis in Section~\ref{S:grRockland} may be used to study them.
For instance, the Rumin complex associated with an Engel structure will only be Rockland in the graded sense, see Example~\ref{Ex:Engel}.





\subsection{Motivation and outlook}





The work in this paper provides a framework to study filtered manifolds by exploring the analogies to the elliptic case.
Classically, the relation between geometry and topology has been successfully studied by analyzing elliptic operators that arise naturally.
We hope that the hypoellipticity of the operators considered in this paper will allow to relate the geometry of filtered manifolds to global topological properties.
We will now mention some directions which have been motivating our investigations.


By hypoellipticity, Rockland operators on closed filtered manifolds are Fredholm and there is a clear candidate for the index formula.
To be more specific, suppose $E$ and $F$ are two vector bundles over a closed filtered manifold, and consider $A\in\Psi^s(E,F)$ such that $A$ and $A^t$ both satisfy the Rockland condition.
In this situation, the analysis mentioned above implies that $A$ induces a Fredholm operator between appropriate Heisenberg Sobolev spaces, see Corollary~\ref{C:Fredholm}.
We expect that the index of this operator can be computed by an index formula similar to van~Erp's in the contact case, see \cite{E10a} and \cite{C94}.
More precisely, the Rockland condition should guarantee that the Heisenberg principal symbol of $A$ represents a $K$-theory class on the non-commutative cotangent bundle, $[\sigma^s(A)]\in K_0(C^*(\mathcal TM))$, and we expect the index formula $\ind(A)=\tind(\psi([\sigma^s(A)]))$ where $\psi\colon K_0(C^*(\mathcal TM))\to K^0(T^*M)$ denotes the abstract Connes--Thom isomorphism \cite{C81} and $\tind\colon K^0(T^*M)\to\Z$ is the topological index map of Atiyah and Singer \cite{AS68}.
More generally, the Heisenberg principal symbol sequence of every Rockland complex should, in a natural way, represent an element in $K_0(C^*(\mathcal TM))$ which is mapped to the Euler characteristics of the Rockland complex via $\tind\circ\psi\colon K_0(C^*(\mathcal TM))\to\Z$.
We expect explicit index formulas for various parabolic geometries, similar to van~Erp's formula on contact manifolds, see \cite{E10b}.


It seems promising to search for Weitzenb\"ock formulas for the Rumin complex associated with a trivial flat line bundle over a filtered manifold $M$ and combine them with the Hodge decomposition established in this paper, see Corollaries~\ref{C:Hodge} and \ref{C:grHodge}, to obtain analogues of Bochner's vanishing result.
Assuming non-negative curvature, a Weitzenb\"ock formula should imply that every harmonic section of $\mathcal H_k$ is parallel.
Over closed connected manifolds, the Hodge decomposition would thus yield a bound on the $k$-th Betti number, $b_k(M)\leq\rank(\mathcal H_k)$.
If, moreover, the curvature is strictly positive at one point, one would expect $b_k(M)=0$.


Let us specialize the above remarks to a particular $5$-dimensional Cartan geometry and formulate a precise conjecture.
To this end, consider a $5$-manifold $M$ equipped with a generic rank two distribution \cite{C10,BH93,S08}.
More precisely, suppose $T^{-1}M\subseteq TM$ is a distribution of rank two with growth vector $(2,3,5)$, that is, Lie brackets of sections of $T^{-1}M$ span a rank three subbundle $T^{-2}M$ of $TM$ and triple brackets of sections of $T^{-1}M$ span all of $TM$.
Such a filtered manifold can equivalently be described as a regular normal parabolic geometry of type $(G,P)$ where $G$ is the split real form of the exceptional Lie group $G_2$ and $P$ is a particular parabolic subgroup.
Cartan \cite{C10} constructed a curvature tensor $\kappa\in\Gamma^\infty(S^4(T^{-1}M)^*)$ which is a complete obstruction to local flatness.
More precisely, $\kappa$ vanishes if and only if the filtration is locally diffeomorphic to the flat model $G/P$.
Regarding the curvature $\kappa_x$ as a fourth order polynomial on $T^{-1}_xM$, we call $\kappa_x$ non-negative and write $\kappa_x\geq0$, if $\kappa_x(X,X,X,X)\geq0$ for all $X\in T_x^{-1}M$.
Since the corresponding Rumin complex for the trivial flat line bundle has $\rank(\mathcal H_1)=2$, see Example~\ref{Ex:BGG235} below, we conjecture the following to hold true:
If $M$ is closed, connected, and $\kappa\geq0$, then the first Betti number is bounded by $b_1(M)\leq2$. 
If, moreover, $\kappa_x>0$ in at least one point $x$, then $b_1(M)=0$.


Another application we have in mind concerns the extension of Ponge's \cite{P08} spectral analysis on Heisenberg manifolds to more general filtered manifolds.
It appears that asymptotic expansion of the heat kernel, complex powers, and Weyl's law are all within reach for the Rumin--Seshadri operators discussed in this paper.
In particular, there is an obvious generalization of the Rumin--Seshadri analytic torsion \cite{RS12} to closed filtered manifolds which give rise to ungraded Rumin complexes.
Due to the rich structure available on regular parabolic geometries it appears feasible to work out explicit anomaly formulas for the Rumin--Seshadri analytic torsion, expressing to what extent this analytic torsion depends on the $L^2$ inner product used to define the formal adjoints, see~\eqref{E:IRS}.
We hope that the decomposition of the de~Rham complex in Theorem~\ref{thmE} will prove helpful in establishing a comparison result relating the Rumin--Seshadri analytic torsion of the Rumin complex with the Ray--Singer torsion \cite{RS71} of the full de~Rham complex.


The existence of a regular parabolic geometry of a particular type on a given manifold can often be described equivalently in terms of a differential relation.
Formally, these can be solved in terms of homotopy theory.
The subtle question is to what extent Gromov's h-principle \cite{G86} holds true for regular parabolic geometries.
We anticipate that the proposed generalization of the Rumin--Seshadri analytic torsion has the potential to detect a possible failure of the h-principle.
In particular, this might lead to topological obstructions to the existence of regular parabolic geometries on closed manifolds \cite{DH16}, and it might provide a sufficiently strong tool to show that formally homotopic regular parabolic geometries need not be homotopic in general, see \cite{P16}.










\section{Hypoelliptic sequences of differential operators}\label{S:hesequences}










Generalizing elliptic sequences of differential operators, we will, in this section, introduce the notion of \emph{Rockland sequences of differential operators} on filtered manifolds, see Definition~\ref{def.Hypo-seq} below.
We will formulate a hypoellipticity result for these sequences, see Corollary~\ref{C:RShypo}, and a corresponding Hodge decomposition, see Corollary~\ref{C:Hodge}.
These are all immediate corollaries of a Rockland type theorem for differential operators on general filtered manifolds, see Theorem~\ref{T:para}, that generalizes well known results for Heisenberg manifolds \cite{BG88,P08,T84} and filtered manifolds which are locally diffeomorphic to nilpotent Lie groups \cite{CGGP92}.
The proof of Theorem~\ref{T:para} will be presented in Section~\ref{S:PDO}.



\subsection{Preliminaries on operators with distributional kernels}\label{SS:prelim}



Let us start by recalling standard facts about vector valued distributions and setting up our notation.


Let $M$ be a smooth manifold and let $|\Lambda|_M$ denote the bundle of $1$-densities on $M$.
For every smooth complex vector bundle $E$ over $M$, we put $E':=E^*\otimes|\Lambda|_M$ where $E^*$ denotes the dual bundle. 
There is a canonical pairing $\Gamma^\infty_c(E')\times\Gamma^\infty(E)\to\C$, $\langle\phi\otimes dx,\psi\rangle:=\int_M\langle\phi,\psi\rangle dx$  and this provides a continuous inclusion 
$\Gamma^\infty(E)\subseteq\Gamma^{-\infty}(E):=\mathcal D(E)^*$ where $\mathcal D(E):=\Gamma^\infty_c(E')$ denotes the space of test sections and 
$\Gamma^{-\infty}(E)$ is the space of \emph{distributional sections.} 
Similarly, the pairing $\Gamma^\infty(E')\times\Gamma^\infty_c(E)\to\C$ gives rise to a continuous inclusion $\Gamma^\infty_c(E)\subseteq\Gamma^{-\infty}_c(E):=\mathcal E(E)^*$ where $\mathcal E(E):=\Gamma^\infty(E')$ and $\Gamma^{-\infty}_c(E)$ denotes the space of compactly supported distributional sections.
Note that one may identify $E''=E$ since the line bundle $|\Lambda|_M^*\otimes|\Lambda|_M$ admits a canonical trivialization.


These inclusions  extend  continuously to $L^2_\loc(E)\subseteq\Gamma^{-\infty}(E)$ and $L^2_c(E)\subseteq\Gamma^{-\infty}_c(E)$.
Moreover, the canonical pairing extends to a continuous pairing $L_c^2(E')\times L_\loc^2(E)\to\C$ which gives rise to isomorphisms $L_\loc^2(E)^*=L_c^2(E')$ and $L_c^2(E')^*=L_\loc^2(E)$.
If $M$ is compact, then $L^2_c(E)=L^2_\loc(E)$ is a Hilbert space and will be denoted by $L^2(E)$.


Suppose $F$ is another vector bundle over $M$.
Recall that a continuous linear operator $A\colon\Gamma^\infty_c(E)\to\Gamma^{-\infty}(F)$ can be described equivalently by its \emph{distributional (Schwartz) kernel} $k\in\Gamma^{-\infty}(F\boxtimes E')$ via $(A\psi)(x)=\int_Mk(x,y)\psi(y)dy$, where $\psi\in\Gamma^\infty_c(E)$ and $x\in M$.
Here we are using the vector bundle $F\boxtimes E':=\tau^*F\otimes\sigma^*E'$ over $M\times M$ where $\sigma\colon M\times M\to M$, $\sigma(x,y):=y$, and $\tau\colon M\times M\to M$, $\tau(x,y):=x$, denote the natural projections.
More precisely, $\langle\phi,A\psi\rangle=\langle\phi\boxtimes\psi,k\rangle$, for all $\psi\in\Gamma_c^\infty(E)$ and $\phi\in\mathcal D(F)=\Gamma^\infty_c(F')$, 
where $\phi\boxtimes\psi:=(\tau^*\phi)\otimes(\sigma^*\psi)\in\Gamma^\infty_c(F'\boxtimes E)=\mathcal D(F\boxtimes E')$.
\footnote{For the latter identification note that the canonical isomorphism $|\Lambda|_{M\times M}=|\Lambda|_M\boxtimes|\Lambda|_M$ provides a canonical identification 
$(F\boxtimes E')'=F'\boxtimes E''=F'\boxtimes E$ and thus a canonical isomorphism $\mathcal D(F\boxtimes E')=\Gamma_c^\infty((F\boxtimes E')')=\Gamma^\infty_c(F'\boxtimes E)$.}


If $A\colon\Gamma^\infty_c(E)\to\Gamma^{-\infty}(F)$ is an operator with Schwartz kernel $k\in\Gamma^{-\infty}(F\boxtimes E')$, then the transposed kernel, $k^t\in\Gamma^{-\infty}(E'\boxtimes F)$, $k^t(x,y):=k(y,x)^t$, corresponds to the \emph{transpose operator,} $A^t\colon\Gamma^\infty_c(F')\to\Gamma^{-\infty}(E')$, characterized by $\langle\phi,A\psi\rangle=\langle A^t\phi,\psi\rangle$, for all $\psi\in\Gamma^\infty_c(E)=\mathcal D(E')$ and $\phi\in\mathcal D(F)=\Gamma^\infty_c(F')$.
Clearly, $(A^t)^t=A$ up to the canonical identifications $E''=E$ and $F''=F$.


Smooth kernels $k\in\Gamma^\infty(F\boxtimes E')$ correspond precisely to \emph{smoothing operators,} i.e.\ continuous operators $\Gamma^{-\infty}_c(E)\to\Gamma^\infty(F)$.
We will denote the space of smoothing operators by $\mathcal O^{-\infty}(E,F)$.
If $A\in\mathcal O^{-\infty}(E,F)$, then $A^t\in\mathcal O^{-\infty}(F',E')$.


We let $\DO(E,F)$ denote the space of \emph{differential operators} with smooth coefficients, $D\colon\Gamma^\infty(E)\to\Gamma^\infty(F)$.
The Schwartz kernels of differential operators are supported along the diagonal in $M\times M$.
If $D\in\DO(E,F)$, then $D^t\in\DO(F',E')$.


Let $\mathcal O(E,F)$ denote the space of operators $\Gamma^\infty_c(E)\to\Gamma^{-\infty}(F)$ corresponding to Schwartz kernels with wave front set contained in the conormal bundle of the diagonal.
These are precisely the operators whose kernel is smooth away from the diagonal and which map $\Gamma^\infty_c(E)$ continuously into $\Gamma^\infty(F)$.
If $A\in\mathcal O(E,F)$, then $A^t\in\mathcal O(F',E')$.
The transpose permits to extend $A$ continuously to distributional sections, $A\colon\Gamma^{-\infty}_c(E)\to\Gamma^{-\infty}(F)$, such that $\langle A^t\phi,\psi\rangle=\langle\phi,A\psi\rangle$ for all $\psi\in\Gamma^{-\infty}_c(E)$ and $\phi\in\mathcal D(F)=\Gamma^\infty_c(F')$, and this extension is pseudolocal, i.e.\ $\singsupp(A\psi)\subseteq\singsupp(\psi)$
for all $\psi\in\Gamma^{-\infty}_c(E)$.
If $A\in\mathcal O(E,F)$ is properly supported, then it defines continuous linear maps:
$\Gamma^\infty(E)\to\Gamma^\infty(F)$,
$\Gamma^\infty_c(E)\to\Gamma^\infty_c(F)$,
$\Gamma^{-\infty}(E)\to\Gamma^{-\infty}(F)$, and
$\Gamma^{-\infty}_c(E)\to\Gamma^{-\infty}_c(F)$.
Recall that an operator with Schwartz kernel $k$ is called \emph{properly supported} if the projections $\sigma$ and $\tau$ both restrict to proper maps on the support of $k$.
If $B\in\mathcal O(F,G)$ and at least one of $A$ or $B$ is properly supported, then $BA\in\mathcal O(E,G)$ and $(BA)^t=A^tB^t$.
If both are properly supported, then so is their product.
If at least one of $A$ or $B$ is a smoothing operator, the same is true for their product.
If both are differential operators, then so is their product.
Clearly,
$$
\DO(E,F)\subseteq\mathcal O_\prop(E,F)\supseteq\mathcal O^{-\infty}_\prop(E,F),
$$
where the subscript indicates properly supported operators.
In particular, $\mathcal O_\prop(E):=\mathcal O_\prop(E,E)$ is a unital algebra containing the algebra of differential operators $\DO(E):=\DO(E,E)$
and the ideal of properly supported smoothing operators, $\mathcal O^{-\infty}_\prop(E):=\mathcal O^{-\infty}_\prop(E,E)$.


A smooth volume density $dx$ on $M$ and a smooth fiberwise Hermitian inner product $h$ on $E$ give rise to a positive definite sesquilinear form
\begin{equation}\label{E:llrr}
\llangle\psi_1,\psi_2\rrangle_{L^2(E)}=\int_Mh(\psi_1(x),\psi_2(x))dx
=\langle(h\otimes dx)\psi_1,\psi_2\rangle
\end{equation}
where $\psi_1,\psi_2\in\Gamma_c^\infty(E)$.
Here we consider $h\otimes dx\colon \bar E\to E'$ as an vector bundle isomorphism.
When restricted to sections supported in a fixed compact set, this becomes a Hermitian inner product generating the $L^2$-topology.
This sesquilinear form extends continuously to $L^2_c(\bar E)\times L^2_\loc(E)\to\C$, inducing isomorphisms $L^2_\loc(E)^*=L^2_c(\bar E)$ and $L^2_c(\bar E)^*=L^2_\loc(E)$.
With respect to inner products of the form \eqref{E:llrr}, an operator $A\colon\Gamma^\infty_c(E)\to\Gamma^{-\infty}(F)$ with Schwartz kernel $k\in\Gamma^{-\infty}(F\boxtimes E')$ has a \emph{formal adjoint,} $A^*\colon\Gamma^\infty_c(F)\to\Gamma^{-\infty}(E)$, characterized by
\begin{equation}\label{E:A*}
\llangle A^*\phi,\psi\rrangle_{L^2(E)}=\llangle\phi,A\psi\rrangle_{L^2(F)}
\end{equation}
for all $\phi\in\Gamma^\infty_c(F)$ and $\psi\in\Gamma^\infty_c(E)$.
Indeed, in terms of the transpose,
\begin{equation}\label{E:A*At}
A^*=(h_E\otimes dx)^{-1}\circ A^t\circ(h_F\otimes dx),
\end{equation}
where we consider $h_F\otimes dx\colon\bar F\to F'$ and $(h_E\otimes dx)^{-1}\colon E'\to\bar E$ as vector bundle isomorphisms.
Its Schwartz kernel, $k^*\in\Gamma^{-\infty}(E\boxtimes F')$, is given by $k^*(x,y)=k(y,x)^*$.
For $A\in\mathcal O(E,F)$ and $B\in\mathcal O(F,G)$ we obtain $A^*\in\mathcal O(F,E)$, $(A^*)^*=A$, and $(BA)^*=A^*B^*$ provided at least one of the two operators is properly supported.
Clearly, the operator classes $\mathcal O^{-\infty}$ and $\DO$ are invariant under taking formal adjoints.


We let $\Gamma(E)$ denote the space of continuous sections of $E$ equipped with the topology of uniform convergence on compact subsets.
Moreover, we let $\Gamma_c(E)$ denote the space of compactly supported continuous sections equipped with the usual (inductive limit) topology.
Recall that the inclusions $\Gamma^\infty(E)\subseteq\Gamma^{-\infty}(E)$ and $\Gamma_c^\infty(E)\subseteq\Gamma^{-\infty}_c(E)$ extend to continuous inclusions $\Gamma(E)\subseteq\Gamma^{-\infty}(E)$ and $\Gamma_c(E)\subseteq\Gamma^{-\infty}_c(E)$, respectively.







\subsection{Differential operators on filtered manifolds}\label{SS:DO}








A \emph{filtered manifold} \cite{M93,M02,N10} is a smooth manifold $M$ whose tangent bundle comes equipped with a filtration by smooth subbundles,
$$
TM=T^{-r}M\supseteq\cdots\supseteq T^{-2}M\supseteq T^{-1}M\supseteq T^0M=0,
$$
which is compatible with the Lie bracket in the following sense:
If $X\in\Gamma^\infty(T^pM)$ and $Y\in\Gamma^\infty(T^qM)$, then $[X,Y]\in\Gamma^\infty(T^{p+q}M)$.


Let $M$ be a filtered manifold and consider the quotient bundle $\mathfrak t^pM:=T^pM/T^{p+1}M$ with fibers $\mathfrak t_x^pM=T^p_xM/T^{p+1}_xM$.
Recall that the Lie bracket of vector fields induces a tensorial (Levi) bracket $\mathfrak t^pM\otimes\mathfrak t^qM\to\mathfrak t^{p+q}M$ which turns the associated graded $\mathfrak tM:=\bigoplus_p\mathfrak t^pM$ into a bundle of graded nilpotent Lie algebras called the \emph{bundle of osculating algebras}.
Each fiber $\mathfrak t_xM=\bigoplus_p\mathfrak t^p_xM$ is a graded nilpotent Lie algebra which will be referred to as the \emph{osculating algebra at $x$}.
The Lie algebra structure depends smoothly on $x$ but is not assumed to be locally trivial, that is, different fibers may be non-isomorphic as Lie algebras.
In the literature $\mathfrak t_xM$ is also known as the symbol algebra of $M$ at $x$, see \cite{M93,M02,N10}.


\begin{remark}[Examples]
Suppose $\noe=\noe_{-r}\oplus\cdots\oplus\noe_{-1}$ is a graded nilpotent Lie algebra, and let $N$ be a Lie group with Lie algebra $\noe$.
Then the filtration $\noe=\noe^{-r}\supseteq\cdots\supseteq\noe^{-1}\supseteq\noe^0=0$, with $\noe^p:=\bigoplus_{p\leq q}\noe_q$, determines a left invariant filtration of $TN$ which turns $N$ into a filtered manifold with (locally) trivial bundle of osculating algebras and typical fiber $\noe$.

Numerous regular normal parabolic geometries can be described equivalently as filtered manifolds with locally trivial bundle of osculating algebras, see \cite[Proposition~4.3.1]{CS09}.
These include generic rank two distributions in dimension five \cite{C10,BH93,S08,CS09,DH16}, generic rank three distributions in dimension six \cite{B06}, and quaternionic contact structures \cite{B00}.

Heisenberg manifolds \cite{P08} constitute a class of filtered manifolds for which the bundle of osculating algebras need not be locally trivial.
The latter occur naturally as boundaries of complex manifolds.

Contact manifolds are arguably the best know filtered manifolds.
Their bundle of osculating algebras is locally trivial with the Heisenberg algebra as typical fiber.
According to Darboux's theorem, the filtration on a contact manifold is even locally diffeomorphic to the left invariant filtration on the Heisenberg group.
Engel structures on 4-manifolds provide further examples of filtered manifolds that admit local normal forms, see \cite{V09}.

Let us also point out, that foliated manifolds can be considered as filtered manifolds.
In this case the osculating algebras are abelian but non-trivially graded.
\end{remark}


\begin{remark}
There are various conventions on choosing orders/degrees on filtrations and gradings.
Van~Erp and Yuncken \cite{EY15}, for instance, assign positive values to gradings.
We follow the convention prevalent in parabolic geometry, see for example \cite{CS09,M93}.
In this context the choice of negative degree/order is a well established convention.
\end{remark}


A filtration on $M$ induces a (Heisenberg) filtration on differential operators.
If $E$ and $F$ are two smooth vector bundles over $M$, then a differential operator in $\DO(E,F)$ is said to be of \emph{Heisenberg order at most $k$} if, locally, it can be written as a finite linear combination of operators of the form $\Phi\nabla_{X_m}\cdots\nabla_{X_1}$ where $\Phi\in\Gamma^\infty(\hom(E,F))$, $\nabla$ is a linear connection on $E$, and $X_i\in\Gamma^\infty(T^{p_i}M)$ are vector fields such that $-k\leq p_m+\cdots+p_1$.
Denoting the space of these differential operators by $\DO^k(E,F)$, we obtain a filtration on $\DO(E,F)$,
$$
\Gamma^\infty(\hom(E,F))=\DO^0(E,F)\subseteq\DO^1(E,F)\subseteq\DO^2(E,F)\subseteq\cdots,
$$
which is compatible with composition and transposition.
More explicitly, if $A\in\DO^k(E,F)$ and $B\in\DO^l(F,G)$, then $BA\in\DO^{l+k}(E,G)$ and $A^t\in\DO^k(F',E')$.


\begin{remark}[The spaces $\Gamma^r(E)$]\label{R:GammarE}
For $r\in\mathbb N_0$ we let $\Gamma^r(E)$ denotes the Heisenberg analogue of the space of $r$ times continuously differentiable sections of $E$.
More precisely, $\Gamma^r(E)$ denotes the space of all $\psi\in\Gamma^{-\infty}(E)$ such that $A\psi\in\Gamma(F)$ for all differential operators $A\in\DO^r(E,F)$ of Heisenberg order at most $r$ and all vector bundles $F$.
Recall that $\Gamma(F)\subseteq\Gamma^{-\infty}(F)$ denotes the space of continuous sections equipped with the topology of uniform convergence on compact subsets.
We equip $\Gamma^r(E)$ with the coarsest topology such that the maps $A\colon\Gamma^r(E)\to\Gamma(F)$ are continuous for all $A\in\DO^r(E,F)$.
If $r-k\geq0$, then each $A\in\DO^k(E,F)$ induces a continuous operator, $A\colon\Gamma^r(E)\to\Gamma^{r-k}(F)$.
Note that we have continuous inclusions $\cdots\subseteq\Gamma^2(E)\subseteq\Gamma^1(E)\subseteq\Gamma^0(E)$ and topological isomorphisms $\Gamma^0(E)=\Gamma(E)$ as well as $\bigcap_r\Gamma^r(E)=\Gamma^\infty(E)$.
We will denote the compactly supported analogue by $\Gamma^r_c(E)$.
\end{remark}


\begin{remark}[Universal differential operators]\label{R:jk}
Consider a vector bundle $E$ over $M$ and let $J^kE\to E$ denote the bundle of Heisenberg $k$-jets of sections of $E$.
This is a smooth vector bundle whose fiber over $x\in M$ coincides with the vector space of Heisenberg $k$-jets at $x$ of sections of $E$.
Recall that two sections $\psi_1,\psi_2\in\Gamma^\infty(E)$ are said to represent the same Heisenberg $k$-jet at $x$ if $A(\psi_2-\psi_1)(x)=0$ for all differential operators $A\in\DO^k(E,F)$.
Assigning to a section of $E$ its Heisenberg $k$-jet, we obtain a differential operator $j^k\colon\Gamma^\infty(E)\to\Gamma^\infty(J^kE)$.
In fact, $j^k\in\DO^k(E,J^kE)$, and this differential operator is universal in the following sense: For every $A\in\DO^k(E,F)$ there exists a unique smooth vector bundle homomorphism $\alpha\colon J^kE\to F$ such that $A=\alpha\circ j^k$.
We refer to \cite[Section~3.1]{M02}, \cite[Section~1.2.6]{N10} or \cite{N09} for details.
\end{remark}


A differential operator $A\in\DO^k(E,F)$ has a \emph{Heisenberg principal cosymbol} at each $x\in M$,
$$
\sigma_x^k(A)\in\mathcal U_{-k}(\mathfrak t_xM)\otimes\hom(E_x,F_x),
$$
where $\mathcal U_{-k}(\mathfrak t_xM)$ denotes the degree $-k$ part of the universal enveloping algebra of the graded Lie algebra $\mathfrak t_xM=\bigoplus_p\mathfrak t_x^pM$.
More explicitly, $\mathcal U_{-k}(\mathfrak t_xM)$ can be described as the linear subspace of $\mathcal U(\mathfrak t_xM)$ spanned by all elements of the form $X_m\cdots X_1$ where $X_i\in\mathfrak t_x^{p_i}M$ and $-k=p_m+\cdots+p_1$.
The Heisenberg principal cosymbol provides a short exact sequence
$$
0\to\DO^{k-1}(E,F)\to\DO^k(E,F)\xrightarrow{\sigma^k}\Gamma^\infty\bigl(\mathcal U_{-k}(\mathfrak tM)\otimes\hom(E,F)\bigr)\to0
$$
where $\mathcal U_{-k}(\mathfrak tM):=\bigsqcup_{x\in M}\mathcal U_{-k}(\mathfrak t_xM)$ is a smooth vector bundle of finite rank according to the Poincar\'e--Birkhoff--Witt theorem.
Details may be found in \cite[Section~1.2.5]{N10}.


If $A\in\DO^k(E,F)$ and $B\in\DO^l(F,G)$ where $G$ is another vector bundle over $M$, then
\begin{equation}\label{E:sABAt}
\sigma_x^{l+k}(BA)=\sigma_x^l(B)\sigma^k_x(A)
\qquad\text{and}\qquad
\sigma_x^k(A^t)=\sigma_x^k(A)^t.
\end{equation}
To explain the second equation in \eqref{E:sABAt}, we extend $-\id\colon\mathfrak t_xM\to\mathfrak t_xM$ to an anti-automorphism of $\mathcal U(\mathfrak t_xM)$, $\mathbf X\mapsto\mathbf X^t$.
Hence, $(\mathbf X^t)^t=\mathbf X$ and $(\mathbf X\mathbf Y)^t=\mathbf Y^t\mathbf X^t$ for all $\mathbf X,\mathbf Y\in\mathcal U(\mathfrak t_xM)$.
This \emph{antipode} preserves the grading components $\mathcal U_{-k}(\mathfrak t_xM)$.
We extend this further to a transposition $\mathcal U(\mathfrak t_xM)\otimes\hom(E_x,F_x)\to\mathcal U(\mathfrak t_xM)\otimes\hom(F_x',E_x')$ characterized by $(\mathbf X\otimes\Phi)^t:=\mathbf X^t\otimes\Phi^t\otimes\id_{|\Lambda|_{M,x}}$ for all $\Phi\in\hom(E_x,F_x)$ and $\mathbf X\in\mathcal U(\mathfrak t_xM)$ where $\Phi^t\in\hom(F_x^*,E_x^*)$ denotes the linear map dual to $\Phi$.
This is the transposition used in $\sigma_x^k(A)^t$, see~\eqref{E:sABAt}.


If $\nabla$ is a linear connection on $E$ and $X\in\Gamma^\infty(T^{-k}M)$ then $\nabla_X\in\DO^k(E)$ and
\begin{equation}\label{E:snablaX}
\sigma^k(\nabla_X)=[X]\otimes\id_E\in\Gamma^\infty\bigl(\mathcal U_{-k}(\mathfrak tM)\otimes\eend(E)\bigr)
\end{equation}
where $[X]$ denotes the section of $\mathfrak t^{-k}M=T^{-k}M/T^{-k+1}M$ represented by $X$.
This property, together with the multiplicativity in \eqref{E:sABAt} and the requirement $\sigma^0(A)=A$ for all $A\in\DO^0(E,F)=\Gamma^\infty(\hom(E,F))=\Gamma^\infty(\mathcal U_0(\mathfrak tM)\otimes\hom(E,F))$, characterizes the Heisenberg principal symbol uniquely.


\begin{remark}[Formal adjoints]\label{R:sA*}
Suppose $A\in\DO^k(E,F)$ and let $A^*$ denote the formal adjoint with respect to $L^2$ inner products associated with a smooth volume density on $M$ and smooth fiber wise Hermitian inner products $h_E$ and $h_F$ on the vector bundles $E$ and $F$, respectively, see \eqref{E:llrr} and \eqref{E:A*}.
Then $A^*\in\DO^k(F,E)$ and
\begin{equation}\label{E:sA*}
\sigma_x^k(A^*)=\sigma^k_x(A)^*.
\end{equation}
The involution $\mathcal U(\mathfrak t_xM)\otimes\hom(E_x,F_x)\to\mathcal U(\mathfrak t_xM)\otimes\hom(F_x,E_x)$ used on the right hand side can be characterized by $(\mathbf X\otimes\Phi)^*=\mathbf X^t\otimes\Phi^*$ for all $\Phi\in\hom(E_x,F_x)$ and $\mathbf X\in\mathcal U(\mathfrak t_xM)$ where $\Phi^*\in\hom(F_x,E_x)$ denotes the adjoint of $\Phi$ with respect to the inner products $h_{E,x}$ and $h_{F,x}$.
Equation \eqref{E:sA*} follows from \eqref{E:A*At} and \eqref{E:sABAt}.
\end{remark}


A graded Lie algebra has a natural group of \emph{dilation automorphisms.}
Thus, for $\lambda>0$ we let $\dot\delta_\lambda\in\Aut(\mathfrak tM)$ denote the bundle automorphism given by multiplication with $\lambda^{-p}$ on the grading component $\mathfrak t^pM$.
For each $x\in M$, this restricts to an automorphism $\dot\delta_{\lambda,x}\in\Aut(\mathfrak t_xM)$ of the osculating algebra such that $\lim_{\lambda\to0}\dot\delta_{\lambda,x}=0$.
Clearly, $\dot\delta_{\lambda_1\lambda_2}=\dot\delta_{\lambda_1}\dot\delta_{\lambda_2}$ for all $\lambda_1,\lambda_2>0$.
Extending $\dot\delta_{\lambda,x}$ to an automorphism of $\mathcal U(\mathfrak t_xM)$, we can characterize the grading by
\begin{equation}\label{E:Uk}
\mathcal U_{-k}(\mathfrak t_xM)=\bigl\{\mathbf X\in\mathcal U(\mathfrak t_xM):\textrm{$\dot\delta_{\lambda,x}(\mathbf X)=\lambda^k\mathbf X$ for all $\lambda>0$}\bigr\}.
\end{equation}


We let $\mathcal TM\to M$ denote the \emph{bundle of osculating groups.}
For each $x\in M$, the fiber $\mathcal T_xM$ is a simply connected nilpotent Lie group, called the \emph{osculating group at $x$}, with Lie algebra $\mathfrak t_xM$.
The fiber wise exponential map, $\exp\colon\mathfrak tM\to\mathcal TM$, provides an isomorphism of smooth fiber bundles.
The Lie algebra automorphisms $\dot\delta_{\lambda,x}$ integrate to group automorphisms $\delta_{\lambda,x}\in\Aut(\mathcal T_xM)$ which assemble to a smooth bundle automorphism $\delta_\lambda\in\Aut(\mathcal TM)$ such that $\exp\circ\dot\delta_\lambda=\delta_\lambda\circ\exp$.
Clearly, $\delta_{\lambda_1\lambda_2}=\delta_{\lambda_1}\delta_{\lambda_2}$, for all $\lambda_1,\lambda_2>0$.


Since the universal enveloping algebra of $\mathfrak t_xM$ can be identified with the algebra of left invariant differential operators on $\mathcal T_xM$, and in view of \eqref{E:Uk}, the Heisenberg principal symbol of $A\in\DO^k(E,F)$ can equivalently be regarded as a \emph{left invariant differential operator,}
\begin{equation}\label{E:skA}
\sigma_x^k(A)\colon C^\infty(\mathcal T_xM,E_x)\to C^\infty(\mathcal T_xM,F_x),
\end{equation}
which is \emph{homogeneous} of degree $k$, that is,
\begin{equation}\label{E:skAlg}
\sigma^k_x(A)\circ l_g^*=l_g^*\circ\sigma^k_x(A)
\qquad\text{and}\qquad
\sigma^k_x(A)\circ\delta_{\lambda,x}^*=\lambda^k\cdot\delta_{\lambda,x}^*\circ\sigma^k_x(A)
\end{equation}
for all $g\in\mathcal T_xM$ and $\lambda>0$.
Here $l_g^*$ denotes pull back along the left translation, $l_g\colon\mathcal T_xM\to\mathcal T_xM$, $l_g(h):=gh$, and $\delta_{\lambda,x}^*$ denotes pull back along the dilation discussed above.


\begin{remark}
If the filtration on $M$ is trivial, that is to say, if $T^{-1}M=TM$, then the filtration on differential operators is the usual one.
In this case $\mathcal T_xM=T_xM$ is an Abelian Lie group and the principal symbol $\sigma^k_x(A)$ of a differential operator $A$ is a translation invariant (constant coefficient) differential operator on $T_xM$.
\end{remark}







\subsection{Parametrices}\label{SS:paraDO}







As we have seen above, the Heisenberg principal symbols of a differential operator can be described by homogeneous left invariant operators on the osculating Lie groups.
This is the primary reason why the osculating groups and their representation theory, and particularly the Rockland condition, become relevant to the analysis of these operators.
We shall now briefly recall some facts from representation theory necessary to formulate the Rockland condition for differential operators, see Definition~\ref{D:rockland} below, and state the corresponding Rockland type theorem, see Theorem~\ref{T:para}.


Let $G$ be a Lie group with Lie algebra $\goe$. 
Suppose $\pi\colon G\to U(\mathcal H)$ is a \emph{unitary representation} of $G$ on a Hilbert space $\mathcal H$.
These representations will always be assumed to be \emph{strongly continuous}, that is, the map $G\to\mathcal H$, $g\mapsto\pi(g)v$, is assumed to be continuous for every vector $v\in\mathcal H$.
For unitary representations, this is actually equivalent to \emph{weak continuity} which only asserts that the function $G\to\C$, $g\mapsto\llangle\pi(g)v,w\rrangle_{\mathcal H}$, is continuous for any two vectors $v,w\in\mathcal H$, see \cite[Theorem~1 in Appendix~V]{K04}.
Rarely will the representations we shall encounter be continuous with respect to the norm topology on $U(\mathcal H)$.


Recall that $v\in\mathcal H$ is called \emph{smooth vector} if the map $G\to\mathcal H$, $g\mapsto\pi(g)v$, is (strongly) smooth.
According to \cite[Theorem~3 in Appendix~V]{K04} this is equivalent to the weak assumption: the function $G\to\C$, $g\mapsto\llangle\pi(g)v,w\rrangle_{\mathcal H}$, is smooth for all vectors $w\in\mathcal H$.
We will denote the subspace of smooth vectors by $\mathcal H_\infty$.
This is a dense subspace in $\mathcal H$ which is invariant under the operators $\pi(g)$ for all $g\in G$, see \cite[Theorem~4(1) in Appendix~V]{K04}.
For each $X\in\goe$ we may define, see \cite[Theorem~4(2) in Appendix~V]{K04}, 
$$
\pi(X)\colon\mathcal H_\infty\to\mathcal H_\infty,\qquad\pi(X)v:=\tfrac\partial{\partial t}\big|_{t=0}\pi(\exp(tX))v,
$$
where $v\in\mathcal H_\infty$.
By unitarity, $\llangle\pi(X)v,w\rrangle_{\mathcal H}=\llangle v,\pi(-X)w\rrangle_{\mathcal H}$ for all $v,w\in\mathcal H_\infty$.
Hence, $\pi(X)$ has a densely defined adjoint, $\pi(X)^*=\pi(-X)$, and, in particular, $\pi(X)$ is closeable, see \cite[Theorem~4(2) in Appendix~V]{K04}.
Clearly, $\pi([X,Y])=\pi(X)\pi(Y)-\pi(Y)\pi(X)$ for any two $X,Y\in\goe$.
Extending the definition of $\pi$ to the universal enveloping algebra of $\goe$, we obtain $\pi(\mathbf X)\colon\mathcal H_\infty\to\mathcal H_\infty$ for $\mathbf X\in\mathcal U(\goe)$ such that 
\begin{equation}\label{E:piXY}
\pi(\mathbf X)\pi(\mathbf Y)=\pi(\mathbf X\mathbf Y)
\end{equation}
for all $\mathbf X,\mathbf Y\in\mathcal U(\goe)$.
We let $\mathbf X\mapsto\mathbf X^t$ denote the antipode of $\mathcal U(\goe)$ obtained by extending $-\id\colon\goe\to\goe$ to the universal enveloping algebra.
Hence, $(\mathbf X^t)^t=\mathbf X$ and $(\mathbf X\mathbf Y)^t=\mathbf Y^t\mathbf X^t$ for all $\mathbf X,\mathbf Y\in\mathcal U(\goe)$.
For each $\mathbf X\in\mathcal U(\goe)$ we thus have 
\begin{equation}\label{E:piX*}
\pi(\mathbf X)^*=\pi(\mathbf X^t)
\end{equation} 
as operators on $\mathcal H_\infty$.


If $E_0$ and $F_0$ are two finite dimensional vector spaces and $a\in\mathcal U(\goe)\otimes\hom(E_0,F_0)$ we let 
$$
\pi(a)\colon\mathcal H_\infty\otimes E_0\to\mathcal H_\infty\otimes F_0
$$ 
denote the linear operator obtained by linearly extending the definition $\pi(\mathbf X\otimes\Phi):=\pi(\mathbf X)\otimes\Phi$ for all $\Phi\in\hom(E_0,F_0)$ and $\mathbf X\in\mathcal U(\goe)$.
Equivalently, using bases of $E_0$ and $F_0$ to identify $a$ with a matrix with entries in $\mathcal U(\goe)$, the operator $\pi(a)$ corresponds to a matrix of the same size whose entries are operators on $\mathcal H_\infty$ obtained by applying $\pi$ to the corresponding entry of $a$.
The multiplicativity in \eqref{E:piXY} immediately implies
\begin{equation}\label{E:pBA}
\pi(ba)=\pi(b)\pi(a)
\end{equation}
for all $a\in\mathcal U(\goe)\otimes\hom(E_0,F_0)$ and $b\in\mathcal U(\goe)\otimes\hom(F_0,G_0)$ where $G_0$ is another finite dimensional vector space.
If, moreover, $E_0$ and $F_0$ are equipped with Hermitian inner products, then \eqref{E:piX*} leads to 
\begin{equation}\label{E:pA*}
\pi(a)^*=\pi(a^*)
\end{equation}
as operators $\mathcal H_\infty\otimes F_0\to\mathcal H_\infty\otimes E_0$.
Here the adjoint on the left hand side of \eqref{E:pA*} is with respect to the inner products on $\mathcal H_\infty\otimes E_0$ and $\mathcal H_\infty\otimes F_0$ induced by inner products on $E_0$ and $F_0$ and the restriction of the inner product on $\mathcal H$.
On the right hand side of \eqref{E:pA*}, $a^*\in\mathcal U(\goe)\otimes\hom(F_0,E_0)$ is defined by linear extension of $(\mathbf X\otimes\Phi)^*:=\mathbf X^t\otimes\Phi^*$ for all $\mathbf X\in\mathcal U(\goe)$ and $\Phi\in\hom(E_0,F_0)$ where $\Phi^*\in\hom(F_0,E_0)$ denotes the adjoint of $\Phi$.
Equivalently, using orthogonal bases of $E_0$ and $F_0$ to identify $a$ with a matrix with entries in $\mathcal U(\goe)$, $a^*$ corresponds to the matrix obtained by taking the transpose conjugate of $a$ and applying the antipode $\mathbf X\mapsto\mathbf X^t$ to each entry.


\begin{definition}[Rockland condition]\label{D:rockland}
Let $E$ and $F$ be vector bundles over a filtered manifold $M$.
A differential operator $A\in\DO^k(E,F)$ of Heisenberg order at most $k$ is said to satisfy the \emph{Rockland condition} if $\pi(\sigma_x^k(A))\colon\mathcal H_\infty\otimes E_x\to\mathcal H_\infty\otimes F_x$ is injective for every point $x\in M$ and every non-trivial irreducible unitary representation $\pi\colon\mathcal T_xM\to U(\mathcal H)$ of the osculating group $\mathcal T_xM$ on a Hilbert space $\mathcal H$.
Here $\mathcal H_\infty$ denotes the subspace of smooth vectors in $\mathcal H$.
\end{definition}


In Section~\ref{S:PDO} below we will prove the following Rockland \cite{R78} type result:


\begin{theorem}[Left parametrix]\label{T:para}
Let $E$ and $F$ be vector bundles over a filtered manifold $M$ and suppose $A\in\DO^k(E,F)$ is a differential operator of Heisenberg order at most $k$ which satisfies the Rockland condition, see Definition~\ref{D:rockland}.
Then there exists a properly supported left parametrix $B\in\mathcal O_\prop(F,E)$ such that $BA-\id$ is a smoothing operator.
\end{theorem}


Actually, we will show that the parametrix in the preceding theorem has order $-k$ in van~Erp and Yuncken's \cite{EY15} pseudodifferential calculus, see Theorem~\ref{T:Rockland} below.
This will allow us to refine the subsequent hypoellipticity statements, see Section~\ref{SS:sobolev}.


\begin{remark}
Let us point out that several special cases of this result are well know.
To begin with, for trivially filtered manifolds, i.e., $TM=T^{-1}M$, this reduces to the classical, elliptic case.
In this situation all irreducible unitary representations of the (abelian) osculating group are one dimensional, and the Rockland condition at $x\in M$ becomes the familiar condition that the principal symbol of the operator is invertible at every $0\neq\xi\in T_x^*M$.

Another well studied class are the contact and (more generally) Heisenberg manifolds.
For Heisenberg manifolds, a pseudodifferential calculus has been developed independently by Beals--Greiner \cite{BG88} and Taylor \cite{T84}, see also \cite{P08}.
Special cases of Theorem~\ref{T:para} for Heisenberg manifolds can be found in \cite[Theorem~8.4]{BG88} or \cite[Theorem~5.4.1]{P08}.
These investigations can be traced back to the work of Kohn \cite{K65}, Boutet de Monvel \cite{B74}, and Folland--Stein \cite{FS74} on CR manifolds.
For more historical comments we refer to the Introduction of \cite{BG88}.

If the filtration on $M$ is locally diffeomorphic to that on a graded nilpotent Lie group, then the scalar version of Theorem~\ref{T:para} can be found in \cite[Theorem~2.5(d)]{CGGP92}. 
This suffices to study the flat models in parabolic geometry given by the homogeneous spaces $G/P$, as well as topologically stable \cite{P16} structures like contact and Engel manifolds.
\end{remark}


\begin{corollary}[Hypoellipticity]\label{C:hypo}
Let $E$ and $F$ be vector bundles over a filtered manifold $M$ and suppose $A\in\DO^k(E,F)$ is a differential operator of Heisenberg order at most $k$ which satisfies the Rockland condition.
Then $A$ is hypoelliptic, that is, if $\psi$ is a compactly supported distributional section of $E$ and $A\psi$ is smooth, then $\psi$ was smooth.
If, moreover, $M$ is closed, then $\ker(A)$ is a finite dimensional subspace of\/ $\Gamma^\infty(E)$.
\end{corollary}


\begin{proof}
Let $B\in\mathcal O_\prop(F,E)$ be a left parametrix as in Theorem~\ref{T:para}.
Hence, $BA-\id$ is a smoothing operator and $BA\psi-\psi$ is a smooth section of $E$.
Moreover, $BA\psi$ is a smooth, for $B$ maps smooth sections to smooth sections.
Consequently, $\psi$ is a smooth section of $E$.


Assume $M$ to be closed.
By hypoellipticity, $\ker(A)\subseteq\Gamma^\infty(E)\subseteq L^2(E)$.
Since $BA-\id$ is a smoothing operator, the identical map on $\ker(A)$ coincides with the restriction of a smoothing operator.
The latter induces a compact operator on $L^2(E)$ according to the theorem of Arzel\`a--Ascoli.
Hence, every bounded subset of $\ker(A)$ is precompact in $L^2(E)$.
Consequently, $\ker(A)$ has to be finite dimensional.
\end{proof}


\begin{corollary}[Hodge decomposition]\label{C:smooth-Hodge-decomposition}
Let $E$ be a vector bundle over a closed filtered manifold $M$.
Suppose $A\in\DO^k(E)$ satisfies the Rockland condition and is formally selfadjoint, $A^*=A$, with respect to an $L^2$ inner product of the form \eqref{E:llrr}.
Moreover, let $Q$ denote the orthogonal projection onto the (finite dimensional) subspace $\ker(A)\subseteq\Gamma^\infty(E)$.
Then $A+Q$ is invertible with inverse $(A+Q)^{-1}\in\mathcal O(E)$.
Consequently, we have topological isomorphisms and Hodge type decompositions:
\begin{align*}
A+Q\colon\Gamma^\infty(E)&\xrightarrow\cong\Gamma^\infty(E),&\Gamma^\infty(E)&=\ker(A)\oplus A(\Gamma^\infty(E)),\\
A+Q\colon\Gamma^{-\infty}(E)&\xrightarrow\cong\Gamma^{-\infty}(E),&\Gamma^{-\infty}(E)&=\ker(A)\oplus A(\Gamma^{-\infty}(E)).
\end{align*}
\end{corollary}


\begin{proof}
According to Theorem~\ref{T:para} there exists $B\in\mathcal O(E)$ such that $BA-\id$ is a smoothing operator.
Since $A$ is formally selfadjoint, $AB^*-\id$ is a smoothing operator too.
We conclude that $B$ and $B^*$ differ by a smoothing operator.
Hence $P^*=P:=\tfrac12(B+B^*)\in\mathcal O(E)$ is a formally selfadjoint parametrix such that $PA-\id$ and $AP-\id$ are both smoothing operators.


Note that $Q=Q^*$, the orthogonal projection onto $\ker(A)$, is a smoothing operator.
In particular, $A+Q$ is hypoelliptic.
Moreover, $\ker(A+Q)=0$ in view of $A^*=A$.
Since $AP-\id$ is a smoothing operator, arguing as in the proof of Corollary~\ref{C:hypo} shows that $P$ is hypoelliptic and $\ker(P)$ is a finite dimensional subspace of $\Gamma^\infty(E)$.
Adding the orthogonal projection onto $\ker(P)$ to $P=P^*$, we may furthermore assume $\ker(P)=0$.


Consider $G:=(A+Q)P\in\mathcal O(E)$.
Since $G-\id$ is a smoothing operator, it induces a compact operator on every classical Sobolev space $H^s_\cl(E)$.
Hence, $G$ induces a Fredholm operator with vanishing index on $H^s_\cl(E)$ for all real numbers $s$.
By construction, $G$ is injective, whence invertible with bounded inverse on $H^s_\cl(E)$.
Using the classical Sobolev embedding theorem, we conclude that $G$ is invertible on $\Gamma^\infty(E)$ with continuous inverse, $G^{-1}\colon\Gamma^\infty(E)\to\Gamma^\infty(E)$.
Using $G^*=P(A+Q)$, the same argument shows that $G^*$ is invertible on $\Gamma^\infty(E)$ with continuous inverse, $(G^*)^{-1}\colon\Gamma^\infty(E)\to\Gamma^\infty(E)$.
Since $(G^*)^{-1}$ is the formal adjoint of $G^{-1}$, we conclude that $G^{-1}$ extends continuously to distributional sections, $G^{-1}\colon\Gamma^{-\infty}(E)\to\Gamma^{-\infty}(E)$.
Thus, according to the Schwartz kernel theorem, $G^{-1}$ is given by a (distributional) kernel we will denote by $G^{-1}$ too.
The obvious relation $G^{-1}-\id=-(G-\id)G^{-1}$ implies that $G^{-1}-\id$ is a smoothing operator and, consequently, $G^{-1}\in\mathcal O(E)$.
We conclude $(A+Q)^{-1}=PG^{-1}\in\mathcal O(E)$.
The remaining assertions follow at once.
\end{proof}


\begin{corollary}\label{C:smhe}
Let $E$ and $F$ be a vector bundle over a closed filtered manifold $M$.
Suppose $A\in\DO^k(E,F)$ satisfies the Rockland condition, and let $A^*\in\DO^k(F,E)$ denote the formal adjoint with respect to $L^2$ inner products of the form \eqref{E:llrr}.
Then
$$
\Gamma^\infty(E)=\ker(A)\oplus A^*(\Gamma^\infty(F))\qquad\text{and}\qquad
\Gamma^{-\infty}(E)=\ker(A)\oplus A^*(\Gamma^{-\infty}(F)).
$$
\end{corollary}


\begin{proof}
In view of \eqref{E:sABAt}, \eqref{E:sA*}, \eqref{E:pBA}, and \eqref{E:pA*}, the operator $A^*A\in\DO^{2k}(E)$ satisfies the Rockland condition.
Now apply Corollary~\ref{C:smooth-Hodge-decomposition} to this operator, and use $\ker(A^*A)=\ker(A)$.
\end{proof}


\begin{corollary}[Pseudo inverse]\label{C:smooth-pseudoinverse}
Let $E$ and $F$ be smooth vector bundles over a closed filtered manifold $M$, consider $A\in\DO^k(E,F)$, and let $A^*\in\DO^k(F,E)$ denote the formal adjoint with respect to $L^2$ inner products of the form \eqref{E:llrr}.
Assume $A$ and $A^*$ both satisfy the Rockland condition.
Moreover, let $Q$ denote the orthogonal projection onto the (finite dimensional) subspace $\ker(A)\subseteq\Gamma^\infty(E)$, and let $P$ denote the orthogonal projection onto the (finite dimensional) subspace $\ker(A^*)\subseteq\Gamma^\infty(F)$.
Then the pseudo inverse, 
$$
A^+:=(A^*A+Q)^{-1}A^*=A^*(AA^*+P)^{-1},
$$ 
uniquely characterized by the relations 
$$
AA^+=\id-P,\qquad A^+A=\id-Q,\qquad\text{and}\qquad A^+P=0=QA^+,
$$ 
satisfies $A^+\in\mathcal O(F,E)$.
\end{corollary}





\subsection{Rockland sequences}\label{SS:hesDO}





Let us now generalize the concept of elliptic sequences of differential operators to filtered manifolds.
In subsequent sections we will generalize further to the graded setup, see Definition~\ref{D:graded_hypoelliptic_seq}, and pseudodifferential operators, see Definition~\ref{D:gRs}.


\begin{definition}[Rockland sequences of differential operators]\label{def.Hypo-seq}
Let $E_i$ be smooth vector bundles over a filtered manifold $M$.
A sequence of differential operators,
\begin{equation}\label{E:hes}
\cdots\to\Gamma^\infty(E_{i-1})\xrightarrow{A_{i-1}}\Gamma^\infty(E_i)\xrightarrow{A_i}\Gamma^\infty(E_{i+1})\to\cdots,
\end{equation}
with $A_i\in\DO^{k_i}(E_i,E_{i+1})$, is called \emph{Rockland sequence} if
for every $x\in M$ and every non-trivial irreducible unitary representation of the osculating group, $\pi\colon\mathcal T_xM\to U(\mathcal H)$, the sequence
\begin{equation}\label{E:Hisigma}
\cdots\to
\mathcal H_\infty\otimes E_{i-1,x}\xrightarrow{\pi(\sigma^{k_{i-1}}_x(A_{i-1}))}
\mathcal H_\infty\otimes E_{i,x}\xrightarrow{\pi(\sigma^{k_i}_x(A_i))}
\mathcal H_\infty\otimes E_{i+1,x}\to\cdots
\end{equation}
is weakly exact, i.e., the image of the left arrow is contained and dense in the kernel of the right arrow.
Here $\mathcal H_\infty$ denotes the subspace of smooth vectors in the Hilbert space $\mathcal H$.
\end{definition}


\begin{remark}\label{R:weakexactness}
For every Rockland sequence of differential operators, the sequence~\eqref{E:Hisigma} is actually exact in the strict (algebraic) sense.
Indeed, this follows from Lemma~\ref{L:rockseq} and Remark~\ref{R:AtRock} below.
\end{remark}


\begin{remark}\label{R:transposedseq}
If a sequence of differential operators as in \eqref{E:hes} is a Rockland sequence, then so are the transposed sequence,
$$
\cdots\leftarrow\Gamma^\infty(E_{i-1}')\xleftarrow{A_{i-1}^t}\Gamma^\infty(E_i')\xleftarrow{A_i^t}\Gamma^\infty(E_{i+1}')\leftarrow\cdots,
$$
and the sequence of formal adjoints with respect to $L^2$ inner products of the form \eqref{E:llrr}.
\end{remark}


\begin{remark}
Consider the case $E_i=0$ for all $i\neq1,2$.
In other words, consider a sequence with a single differential operator of Heisenberg order at most $k_1$,
$$
0\to\Gamma^\infty(E_1)\xrightarrow{A_1}\Gamma^\infty(E_2)\to0.
$$
Such a sequence is hypoelliptic in the sense of Definition~\ref{def.Hypo-seq} iff, for all points $x\in M$,
$$
\pi(\sigma^{k_1}_x(A_1))\colon\mathcal H_\infty\otimes E_{1,x}\to\mathcal H_\infty\otimes E_{2,x}
$$ 
is injective with dense image for all non-trivial irreducible unitary representations $\pi\colon\mathcal T_xM\to U(\mathcal H)$.
Equivalently, $A_1$ and $A_1^t$ both satisfy the Rockland condition, see Definition~\ref{D:rockland}.
\end{remark}


Consider a Rockland sequence of differential operators as in \eqref{E:hes}.
To study this sequence we shall introduce certain additional structures and operators in analogy with the standard elliptic sequences.
Fix a smooth volume density $dx$ on $M$, let $h_i$ be smooth fiber wise Hermitian inner products on $E_i$, and consider the associated $L^2$ inner products on $\Gamma^\infty_c(E_i)$,
\begin{equation}\label{E:llrrEi}
\llangle\psi_1,\psi_2\rrangle_{L^2(E_i)}
=\int_Mh_i(\psi_1(x),\psi_2(x))dx
=\langle(h_i\otimes dx)\psi_1,\psi_2\rangle
\end{equation}
where $\psi_1,\psi_2\in\Gamma_c^\infty(E_i)$.
Moreover, let $A_i^*\in\DO^{k_i}(E_{i+1},E_i)$ denote the corresponding formal adjoint, that is, $\llangle A_i^*\phi,\psi\rrangle_{L^2(E_i)}=\llangle\phi,A_i\psi\rrangle_{L^2(E_{i+1})}$ for $\psi\in\Gamma^\infty_c(E_i)$ and $\phi\in\Gamma_c^\infty(E_{i+1})$.


Assume $k_i\geq1$, choose positive integers $a_i$ such that 
\begin{equation}\label{E:RSai}
k_{i-1}a_{i-1}=k_ia_i=:\kappa,
\end{equation}
and consider the differential operator $\Delta_i\in\DO^{2\kappa}(E_i)$,
\begin{equation}\label{E:RSDelta}
\Delta_i:=(A_{i-1}A_{i-1}^*)^{a_{i-1}}+(A_i^*A_i)^{a_i}.
\end{equation}
We will refer to these operators as \emph{Rumin--Seshadri operators} since they generalize the fourth order Laplacians associated with the Rumin complex in \cite[Section~2.3]{RS12}.


\begin{lemma}\label{L:rockseq}
The Rumin--Seshadri operators satisfy the Rockland condition.
\end{lemma}


\begin{proof}
Consider $x\in M$ and let $\pi\colon\mathcal T_xM\to U(\mathcal H)$ be a non-trivial irreducible unitary representation.
We equip $\mathcal H_\infty\otimes E_{i,x}$ with the Hermitian inner product provided by the restriction of the scalar product of the Hilbert space $\mathcal H$ and the inner product $h_{i,x}$ on $E_{i,x}$.
Using \eqref{E:sABAt}, \eqref{E:sA*}, \eqref{E:pBA} and \eqref{E:pA*}, we obtain:
$$
\pi(\sigma^\kappa_x(\Delta_i)))
=(B_{i-1}B_{i-1}^*)^{a_{i-1}}+(B_i^*B_i)^{a_i}
$$
where we abbreviate $B_i:=\pi(\sigma_x^{k_i}(A_i))\colon\mathcal H_\infty\otimes E_{i,x}\to\mathcal H_\infty\otimes E_{i+1,x}$ and we are consider $B_i^*=\pi(\sigma_x^{k_i}(A_i^*))\colon\mathcal H_\infty\otimes E_{i+1,x}\to\mathcal H_\infty\otimes E_{i,x}$.
Due to positivity,
\begin{align*}
\ker(\pi(\sigma_x^{2\kappa}(\Delta_i)))
&=\ker\bigl((B_{i-1}B_{i-1}^*)^{a_{i-1}}\bigr)
\cap\ker\bigl((B_i^*B_i)^{a_i}\bigr),
\\
\ker\bigl((B_{i-1}B_{i-1}^*)^{a_{i-1}}\bigr)
&=\ker(B_{i-1}^*)\textrm{, and}
\\
\ker\bigl((B_i^*B_i)^{a_i}\bigr)
&=\ker(B_i).
\end{align*}
Since $A_i$ is a Rockland sequence, we also have $\ker(B_i)\subseteq\overline{\img(B_{i-1})}\perp\ker(B_{i-1}^*)$ and thus $\ker(B_i)\cap\ker(B_{i-1}^*)=0$.
Combining this with the preceding equalities, we obtain $\ker(\pi(\sigma_x^{2\kappa}(\Delta_i)))=0$, i.e., $\Delta_i$ satisfies the Rockland condition.
\end{proof}


Combining Corollary~\ref{C:hypo} and Lemma~\ref{L:rockseq} we see that each Rumin--Seshadri operator is hypoelliptic, that is, if $\psi$ is a distributional section of $E_i$ and $\Delta_i\psi$ is smooth, then $\psi$ was smooth.
For Rockland sequences this immediately implies:


\begin{corollary}\label{C:RShypo}
The differential operator $(A_{i-1}^*,A_i)\colon\Gamma^\infty(E_i)\to\Gamma^\infty(E_{i-1}\oplus E_{i+1})$ is hypoelliptic, that is, if $\psi$ is a distributional section of $E_i$ such that $A_{i-1}^*\psi$ and $A_i\psi$ are both smooth, then $\psi$ was smooth.
Moreover,
\begin{multline}\label{E:kerDAA}
\ker(\Delta_i|_{\Gamma^{-\infty}_c(E_i)})
=\ker(A_{i-1}^*|_{\Gamma^{-\infty}_c(E_i)})\cap\ker(A_i|_{\Gamma^{-\infty}_c(E_i)})
\\=\ker(A_{i-1}^*|_{\Gamma^\infty_c(E_i)})\cap\ker(A_i|_{\Gamma^\infty_c(E_i)}).
\end{multline}
\end{corollary}


Over closed manifolds Corollary~\ref{C:smooth-Hodge-decomposition} applies to $\Delta_i=\Delta_i^*$, hence $\ker(\Delta_i)$ is a finite dimensional subspace of $\Gamma^\infty(E_i)$, and we get Hodge type decompositions as in Corollary~\ref{C:smooth-Hodge-decomposition} for the Rumin--Seshadri operators.
For Rockland complexes over closed manifolds this implies:


\begin{corollary}\label{C:Hodge}
If $M$ is closed and $A_iA_{i-1}=0$, then we have Hodge type decompositions
\begin{align*}
\Gamma^\infty(E_i)&=A_{i-1}(\Gamma^\infty(E_{i-1}))\oplus\ker(\Delta_i)\oplus A_i^*(\Gamma^\infty(E_{i+1})),
\\
\Gamma^{-\infty}(E_i)&=A_{i-1}(\Gamma^{-\infty}(E_{i-1}))\oplus\ker(\Delta_i)\oplus A_i^*(\Gamma^{-\infty}(E_{i+1})),
\end{align*}
and
\begin{align*}
\ker(A_i|_{\Gamma^\infty(E_i)})&=A_{i-1}(\Gamma^\infty(E_{i-1}))\oplus\ker(\Delta_i),
\\
\ker(A_i|_{\Gamma^{-\infty}(E_i)})&=A_{i-1}(\Gamma^{-\infty}(E_{i-1}))\oplus\ker(\Delta_i).
\end{align*}
In particular, every cohomology class admits a unique harmonic representative:
$$
\frac{\ker(A_i|_{\Gamma^{-\infty}(E_i)})}{\img(A_{i-1}|_{\Gamma^{-\infty}(E_{i-1})})}
=\frac{\ker(A_i|_{\Gamma^\infty(E_i)})}{\img(A_{i-1}|_{\Gamma^\infty(E_{i-1})})}
=\ker(\Delta_i)
=\ker(A_{i-1}^*)\cap\ker(A_i).
$$
\end{corollary}


In the subsequent section these regularity statements will be refined by maximal hypoelliptic estimates.
We postpone these more elaborate results because their formulation requires a pseudodifferential calculus for filtered manifolds.








\section{Pseudodifferential operators on filtered manifolds}\label{S:PDO}








The aim of this section is to prove Theorem~\ref{T:para} above which lies at the core of the hypoellipticity results discussed in Sections~\ref{SS:paraDO} and \ref{SS:hesDO}.
The proof we will give relies on the pseudodifferential calculus for filtered manifolds that has recently been developed by van~Erp and Yuncken, see \cite{EY15,EY16}.
Their calculus has only been formulated for scalar valued operators, but can readily be extended to pseudodifferential operators acting between sections of vector bundles, see Proposition~\ref{P:Psi}.
Combining this with harmonic analysis due to Christ, Geller, G{\l}owacki and Polin \cite{CGGP92}, and using arguments of Ponge \cite{P08}, we obtain a Rockland type theorem for general filtered manifolds, see Theorem~\ref{T:Rockland} below.
In Section~\ref{SS:sobolev} we use this calculus to introduce a Heisenberg Sobolev scale, see Proposition~\ref{P:Hs}, and improve upon Corollaries~\ref{C:RShypo} and \ref{C:Hodge} by establishing maximal hypoelliptic estimates, see Corollaries~\ref{C:regrockseq} and \ref{C:HsHodge-seq} below.





\subsection{Smooth groupoids}\label{SS:Groupoids}





Van Erp and Yuncken's pseudodifferential calculus \cite{EY15} has been formulated using the notion of smooth groupoids.
In this section we set up our notations for groupoids and discuss vector bundle valued distributional kernels whose wave front set are conormal to the space of units.
These kernels can be transposed and convolved.
They give rise to a convenient small class of right invariant vertical operators containing smoothing operators and differential operators.


Suppose $\mathcal G$ is a smooth groupoid \cite{R80}.
Let $\mathcal G^{(0)}$ denote the set of objects or \emph{units}, let $\mathcal G^{(1)}$ the set of \emph{arrows}, and let $\iota\colon\mathcal G^{(0)}\to\mathcal G^{(1)}$ denote the inclusion of the objects as identical morphisms.
Moreover, $\sigma\colon\mathcal G^{(1)}\to\mathcal G^{(0)}$ and $\tau\colon\mathcal G^{(1)}\to\mathcal G^{(0)}$ denote \emph{source} and \emph{target} map, respectively, $\mathcal G^{(2)}:=\{(g,h)\in\mathcal G^{(1)}\times \mathcal G^{(1)}:\sigma(g)=\tau(h)\}$ denotes the set of composeable pairs, $\mu\colon\mathcal G^{(2)}\to\mathcal G^{(1)}$ denotes the \emph{multiplication}, and $\nu\colon\mathcal G^{(1)}\to\mathcal G^{(1)}$ denotes the \emph{inversion.}
These maps are subject to the relations:
$\sigma\circ\iota=\id=\tau\circ\iota$,
$\sigma\circ\mu=\sigma\circ\pr_2$,
$\tau\circ\mu=\tau\circ\pr_1$,
$\mu\circ(\id,\iota\circ\sigma)=\id=\mu\circ(\iota\circ\tau,\id)$
$\mu\circ(\mu,\id)=\mu\circ(\id,\mu)$, 
$\sigma\circ\nu=\tau$, $\tau\circ\nu=\sigma$, 
$\mu\circ(\id,\nu)=\iota\circ\tau$, and
$\mu\circ(\nu,\id)=\iota\circ\sigma$.


The smoothness of the groupoid implies that $\mathcal G^{(0)}$ and $\mathcal G^{(1)}$ are smooth manifolds;
$\iota$ is a closed embedding;
$\sigma$ and $\tau$ are surjective submersions;
$\mathcal G^{(2)}$ is a closed submanifold of $\mathcal G^{(1)}\times\mathcal G^{(1)}$;
$\nu$ is an involutive diffeomorphisms, $\nu^2=\id$, which fixes the units, $\nu\circ\iota=\iota$.
Moreover, $\mathcal G_x:=\sigma^{-1}(x)$ and $\mathcal G^x:=\tau^{-1}(x)$ are closed submanifolds of $\mathcal G^{(1)}$ for all $x\in\mathcal G^{(0)}$;
left multiplication $\mu_g:=\mu(g,-)$ with a $g\in\mathcal G^{(1)}$ provides diffeomorphisms $\mu_g\colon\mathcal G^{\sigma(g)}\to\mathcal G^{\tau(g)}$ with inverse $\mu_g^{-1}=\mu_{\nu(g)}$;
and right multiplication $\mu^g:=\mu(-,g)\colon\mathcal G_{\tau(g)}\to\mathcal G_{\sigma(g)}$ is a diffeomorphism with inverse $(\mu^g)^{-1}=\mu^{\nu(g)}$.
We will denote the restriction of $\tau$ to $\mathcal G_x$ by $\tau_x\colon\mathcal G_x\to\mathcal G^{(0)}$.


We let $\Omega_\sigma$ denote the (trivializable) line bundle over $\mathcal G^{(1)}$ obtained by applying the
representation $|\det^{-1}|$ of the general linear group to the vertical bundle $\ker(T\sigma)$ of the submersion $\sigma$.
Analogously, we let $\Omega_\tau$ denote the density bundle of the vertical bundle $\ker(T\tau)$.
Note that the restriction of $\Omega_\sigma$ to the submanifold $\mathcal G_x$ identifies canonically 
with the bundle of volume densities on $\mathcal G_x$, and similarly for $\Omega_\tau$, that is:
\begin{equation}\label{E:Ost}
\Omega_\sigma|_{\mathcal G_x}=|\Lambda|_{\mathcal G_x}
\qquad\text{and}\qquad
\Omega_\tau|_{\mathcal G^x}=|\Lambda|_{\mathcal G^x}.
\end{equation}
Right and left multiplication provide canonical isomorphisms of vector bundles $\ker(T\sigma)=\tau^*\iota^*\ker(T\sigma)$ and $\ker(T\tau)=\sigma^*\iota^*\ker(T\tau)$
which give rise to canonical isomorphisms of line bundles
\begin{equation}\label{E:canalg}
\tau^*\iota^*\Omega_\sigma=\Omega_\sigma
\qquad\text{and}\qquad
\sigma^*\iota^*\Omega_\tau=\Omega_\tau.
\end{equation}
Moreover, the inversion induces canonical isomorphisms $\nu^*\ker(T\sigma)=\ker(T\tau)$, $\nu^*\ker(T\tau)=\ker(T\sigma)$, and $\iota^*\ker(T\sigma)=\iota^*\ker(T\tau)$ 
which in turn induce canonical isomorphisms of line bundles:
\begin{equation}\label{E:nuO}
\nu^*\Omega_\sigma=\Omega_\tau,
\qquad\nu^*\Omega_\tau=\Omega_\sigma,
\qquad\text{and}\qquad
\iota^*\Omega_\sigma=\iota^*\Omega_\tau.
\end{equation}


\begin{definition}\label{D:KGEF}
For every smooth groupoid $\mathcal G$ and any two vector bundles $E$ and $F$ over $\mathcal G^{(0)}$ we let $\mathcal K(\mathcal G;E,F)$ denote the space of all 
$k\in\Gamma^{-\infty}\bigl(\hom(\sigma^*E,\tau^*F)\otimes\Omega_\tau\bigr)$ whose wave front set is contained in the conormal of the embedding 
$\iota\colon\mathcal G^{(0)}\to\mathcal G^{(1)}$.
In particular, these $k$ are assumed to be smooth on $\mathcal G^{(1)}\setminus\iota(\mathcal G^{(0)})$.
We also introduce the notation $\mathcal K^\infty(\mathcal G;E,F):=\Gamma^\infty\bigl(\hom(\sigma^*E,\tau^*F)\otimes\Omega_\tau\bigr)$ for the subspace of smooth kernels.
A kernel $k\in\mathcal K(\mathcal G;E,F)$ is called \emph{properly supported} if $\sigma$ and $\tau$ both restrict to proper maps on the support of $k$.
\end{definition}


The (left) \emph{regular representation} of $k\in\mathcal K(\mathcal G;E,F)$ at $x\in \mathcal G^{(0)}$ is the continuous operator
\begin{equation}\label{E:regrep0}
\Op_x^k\colon\Gamma^\infty_c(\tau_x^*E)\to\Gamma^\infty(\tau_x^*F),
\qquad\Op_x^k(\psi):=k*\psi,
\end{equation}
defined by
\begin{equation}\label{E:regrep}
(k*\psi)(g):=\int_{h\in\mathcal G_x}k(gh^{-1})\psi(h):=\langle(\mu_g\circ\nu)|_{\mathcal G_x}^*k^{\tau(g)},\psi\rangle
\end{equation}
where  $k\in\mathcal K(\mathcal G;E,F)$, $\psi\in\Gamma_c^\infty(\tau_x^*E)$ and $g\in\mathcal G_x$.
To explain the second expression above, recall that conormality permits to regard $k$ as a family of distributions $k^y$ on $\mathcal G^y$ which depend smoothly on $y\in\mathcal G^{(0)}$, see \cite{LMV15}.
Restricting $k$ to $\mathcal G^{\tau(g)}$, pulling back along the diffeomorphisms $(\mu_g\circ\nu)|_{\mathcal G_x}\colon\mathcal G_x\to\mathcal G^x\to\mathcal G^{\tau(g)}$, $(\mu_g\circ\nu)(h):=\mu(g,\nu(h))$, and using the canonical isomorphism of lines $(\mu_g\circ\nu)|_{\mathcal G_x}^*\Omega_\tau=|\Lambda|_{\mathcal G_x}$, see \eqref{E:Ost} and \eqref{E:nuO}, we obtain a family of distributions $(\mu_g\circ\nu)|_{\mathcal G_x}^*k^{\tau(g)}\in\Gamma^{-\infty}\bigl(\hom(\tau_x^*E,F_{\tau(g)})\otimes|\Lambda|_{\mathcal G_x}\bigr)$ which depend smoothly on $g\in\mathcal G_x$.
The brackets on the right hand side in \eqref{E:regrep} denote the canonical pairing 
$$
\Gamma^{-\infty}\bigl(\hom(\tau_x^*E,F_{\tau(g)})\otimes|\Lambda|_{\mathcal G_x}\bigr)\times\Gamma^\infty_c(\tau_x^*E)\to F_{\tau(g)}.
$$
Note that if $k$ is $\sigma$-proper then we obtain a continuous operator $\Op_x^k\colon\Gamma^\infty_c(\tau_x^*E)\to\Gamma^\infty_c(\tau_x^*F)$;
and if $k$ is $\tau$-proper then the same expression defines a continuous operator $\Op_x^k\colon\Gamma^\infty(\tau_x^*E)\to\Gamma^\infty(\tau_x^*F)$.


Note that the left regular representation is right invariant, that is, 
the following diagram commutes, for all $k\in\mathcal K(\mathcal G;E,F)$ and $g\in\mathcal G^{(1)}$:
\begin{equation}\label{E:regright}
\vcenter{
\xymatrix{
\Gamma^\infty_c(\tau_{\sigma(g)}^*E)\ar[rr]^-{\Op_{\sigma(g)}^k}\ar[d]^-\cong_-{(\mu^g)^*}
&&\Gamma^\infty(\tau_{\sigma(g)}^*F)\ar[d]_-\cong^-{(\mu^g)^*}
\\
\Gamma_c^\infty(\tau_{\tau(g)}^*E)\ar[rr]^-{\Op_{\tau(g)}^k}
&&\Gamma^\infty(\tau_{\tau(g)}^*F)
}}
\end{equation}
Here $\mu^g\colon\mathcal G_{\tau(g)}\to\mathcal G_{\sigma(g)}$ denotes the diffeomorphism given by right multiplication, $\mu^g(h):=\mu(h,g)$, which, in view of the relation $\tau\circ\mu^g=\tau$, provides isomorphisms $(\mu^g)^*$ as indicated in the diagram above.


Rather than considering $k$ as a right invariant family of operators $\Op^k_x$ depending smoothly on $x\in\mathcal G^{(0)}$, one can alternatively think of $k$ as representing a continuous operator 
$$
\Op^k\colon\Gamma^\infty_\textrm{$\sigma$-prop}(\tau^*E)\to\Gamma^\infty(\tau^*F).
$$
This operator is $\sigma$-vertical in the sense that it commutes with functions in the image of the homomorphism $\sigma^*\colon C^\infty(\mathcal G^{(0)})\to C^\infty(\mathcal G^{(1)})$.


Let us next describe the \emph{transpose} of a kernel in $\mathcal K(\mathcal G;E,F)$.
For every vector bundle $E$ over $\mathcal G^{(0)}$ we let $E':=E^*\otimes\iota^*\Omega_\sigma$ where $E^*$ denotes the dual bundle.
Note that we have a canonical isomorphism $E''=E$ since the line bundle $\Omega_\sigma^*\otimes\Omega_\sigma$ admits a canonical trivialization.
If $k\in\mathcal K(\mathcal G;E,F)$ the transposed kernel, $k^t\in\mathcal K(\mathcal G;F',E')$, is defined by $k^t(g):=k(g^{-1})^t$.
More precisely, pulling with the inversion and using the canonical isomorphisms $\nu^*\sigma^*E=\tau^*E$, $\nu^*\tau^*F=\sigma^*F$, $\nu^*\Omega_\tau=\Omega_\sigma$ we obtain $\nu^*k\in\Gamma^{-\infty}\bigl(\hom(\tau^*E,\sigma^*F)\otimes\Omega_\sigma\bigr)$.
Applying the vector bundle isomorphism 
$$
\hom(\tau^*E,\sigma^*F)\otimes\Omega_\sigma
\xrightarrow{t}\hom(\sigma^*F^*,\tau^*E^*)\otimes\Omega_\sigma
=\hom(\sigma^*F',\tau^*E')\otimes\Omega_\tau
$$
obtained from fiber wise transposition and the canonical isomorphism of lines $\Omega_\sigma=\sigma^*\iota^*\Omega_\sigma^*\otimes\tau^*\iota^*\Omega_\sigma\otimes\Omega_\tau$, see \eqref{E:canalg} and \eqref{E:nuO}, we obtain $k^t\in\Gamma^{-\infty}\bigl(\hom(\sigma^*F',\tau^*E')\otimes\Omega_\tau\bigr)$.
Since $\nu\circ\iota=\iota$, the wave front set of $k^t$ is contained in the conormal of the embedding $\iota$ and thus $k^t\in\mathcal K(\mathcal G;F',E')$.
Note that $k^t$ is properly supported iff $k$ is.
Clearly, $(k^t)^t=k$ up to the canonical isomorphisms $E''=E$ and $F''=F$.
Moreover, this transposition maps smooth kernels to smooth kernels.


The transpose kernel corresponds to the transposed operator and allows the extension of the regular representation to distributions. 
We shall provide a short description here.
Using the canonical identification $\tau_x^*\iota^*\Omega_\sigma=|\Lambda|_{\mathcal G_x}$, see \eqref{E:Ost} and \eqref{E:canalg}, we obtain isomorphisms $\mathcal D(\tau_x^*F)=\Gamma^\infty_c(\tau_x^*F')$ and $\mathcal E(\tau_x^*E)=\Gamma^\infty(\tau_x^*E')$.
Hence, $\Op^{k^t}_x$ can be regarded as operator
\begin{equation}\label{E:regrepdual}
\mathcal D(\tau_x^*F)=\Gamma^\infty_c(\tau_x^*F')\xrightarrow{\Op_x^{k^t}}\Gamma^\infty(\tau_x^*E')=\mathcal E(\tau_x^*E)
\end{equation}
satisfying $\langle\phi,\Op_x^k(\psi)\rangle=\langle\Op^{k^t}_x(\phi),\psi\rangle$ for all $\psi\in\Gamma^\infty_c(\tau_x^*E)$ and $\phi\in\mathcal D(\tau_x^*F)$.
This permits to extend the regular representation, see \eqref{E:regrep}, to distributional sections,
\begin{equation}\label{E:Opkdistr}
\Op_x^k\colon\Gamma^{-\infty}_c(\tau_x^*E)\to\Gamma^{-\infty}(\tau_x^*F),
\qquad
\langle\phi,\Op_x^k(\psi)\rangle:=\langle\Op^{k^t}_x(\phi),\psi\rangle,
\end{equation}
where $\psi\in\Gamma^{-\infty}_c(\tau_x^*E)$ and $\phi\in\mathcal D(\tau_x^*F)$.
This extension is \emph{pseudolocal}, that is,
\begin{equation}\label{E:singsupp}
\singsupp(\Op_x^k(\psi))\subseteq\singsupp(\psi)
\end{equation}
for all $\psi\in\Gamma^{-\infty}_c(\tau_x^*E)$.
For smooth kernels, $k\in\mathcal K^\infty(\mathcal G;E,F)$, we obtain continuous (smoothing) operators $\Op_x^k\colon\Gamma^{-\infty}_c(\tau_x^*E)\to\Gamma^\infty(\tau_x^*F)$.
If $k$ is properly supported then the operator in \eqref{E:regrepdual} extends to a continuous operator $\mathcal E(\tau_x^*F)\to\mathcal E(\tau_x^*E)$ and corestricts to a continuous operator $\mathcal D(\tau_x^*F)\to\mathcal D(\tau_x^*E)$.
Consequently, a properly supported kernel, $k\in\mathcal K_\prop(\mathcal G;E,F)$, provides continuous operators $\Op_x^k\colon\Gamma^{-\infty}(\tau_x^*E)\to\Gamma^{-\infty}(\tau_x^*F)$ and $\Op^k_x\colon\Gamma^{-\infty}_c(\tau_x^*E)\to\Gamma^{-\infty}_c(\tau_x^*F)$.


Let us now move to the \emph{convolution product} of the kernels.
Suppose $k\in\mathcal K(\mathcal G;E,F)$ and $l\in\mathcal K(\mathcal G;F,G)$ where $G$ is another vector bundle over $\mathcal G^{(0)}$.
If at least one of $k$ or $l$ is properly supported, then one can define a convolution product $l*k\in\mathcal K(\mathcal G;E,G)$ such that 
\begin{equation}\label{E:Opmult}
\Op_x^l\circ\Op_x^k=\Op_x^{l*k}
\end{equation}
for all $x\in\mathcal G^{(0)}$.
To see this note that there is a canonical isomorphism of vector bundles $(\hom(\sigma^*E,\tau^*F)\otimes\Omega_\tau)|_{\mathcal G_x}=\tau_x^*F\otimes E_x'$, see \eqref{E:canalg}.
Hence, the conormality property permits to regard $k$ as a family of distributions $k_x\in\Gamma^{-\infty}(\tau_x^*F)\otimes E'_x$ on $\mathcal G_x$ depending smoothly on $x\in\mathcal G^{(0)}$.
Using the extension of $\Op^l_x$ to distributions, we define $(l*k)_x:=\Op^l_x(k_x)\in\Gamma^{-\infty}(\tau_x^*G)\otimes E'_x$.
Note that this is well defined if $k$ or $l$ is properly supported: If $k$ has proper support then $k_x$ is compactly supported and we can apply $\Op^l_x$; if $l$ has proper support then $\Op^l_x$ can be applied to distributions with arbitrary support.
Note that $(l*k)_x$ is smooth away from $\iota(x)$ since $k_x$ is smooth away from $\iota(x)$ and $\Op^l_x$ is pseudolocal, see \eqref{E:singsupp}.
Moreover, $(l*k)_x$ depends smoothly on $x$ since $k_x$ depends smoothly on $x$ and the family of operators \eqref{E:Opkdistr} depends smoothly on $x$ too.
We conclude that the family $(l*k)_x$, $x\in\mathcal G^{(0)}$, defines a kernel $l*k\in\mathcal K(\mathcal G;E,G)$.
If $k$ and $l$ are both properly supported, then so is $l*k$.
If $k$ or $l$ is smooth, then so is $l*k$.
It is straight forward to check \eqref{E:Opmult}.
Associativity of the convolution follows immediately from \eqref{E:Opmult} since $k$ is completely determined by the collections of operators $\Op^k_x$, $x\in\mathcal G^{(0)}$.
Analogously, we obtain
$$
(l*k)^t=k^t*l^t
$$
provide at least one of $k$ or $l$ has proper support.


Let us denote the space of \emph{complete cosymbols} by
$$
\Sigma(\mathcal G;E,F):=\frac{\mathcal K_\prop(\mathcal G;E,F)}{\mathcal K_\prop^\infty(\mathcal G;E,F)}
=\frac{\mathcal K(\mathcal G;E,F)}{\mathcal K^\infty(\mathcal G;E,F)}.
$$
To see the canonical identification on the right hand side, choose $\chi\in C^\infty_\prop(\mathcal G^{(1)})$ such that $\chi\equiv1$ in a neighborhood of $\iota(\mathcal G^{(0)})$.
For each $k\in\mathcal K(\mathcal G;E,F)$ we thus obtain $\chi k\in\mathcal K_\prop(\mathcal G;E,F)$ and $\chi k-k\in\mathcal K^\infty(\mathcal G;E,F)$.
The convolution product induces an associative multiplication and a compatible transposition,
$$
\Sigma(\mathcal G;E,F)\times\Sigma(\mathcal G;F,G)\xrightarrow*\Sigma(\mathcal G;E,G),\qquad
\Sigma(\mathcal G;E,F)\xrightarrow{t}\Sigma(\mathcal G;F',E').
$$


To discuss \emph{differential operators}, let $\mathcal K_{\mathcal G^{(0)}}(\mathcal G;E,F)$ denote the space of all kernels in $k\in\mathcal K(\mathcal G;E,F)$ which are supported on the space of units, $\supp(k)\subseteq\iota(\mathcal G^{(0)})$.
This class of kernels is invariant under convolution and transposition.
The regular representation $\Op^k_x$ of such a kernel is a differential operator with smooth coefficients.
Every family of right invariant differential operators depending smoothly on $x\in\mathcal G^{(0)}$ can be obtained in this way.
Equivalently, we have a canonical isomorphism, see \cite[Proposition~84]{EY15} for the scalar valued case,
\begin{equation}\label{E:DOG}
\DO_\sigma(\mathcal G;\tau^*E,\tau^*F)^G
=\mathcal K_{\mathcal G^{(0)}}(\mathcal G;E,F)
\end{equation}
where the left hand side denotes the space of all differential operators $\Gamma^\infty(\tau^*E)\to\Gamma^\infty(\tau^*F)$ on $\mathcal G^{(1)}$ which are $\sigma$-vertical, that is, commute with multiplication by functions in the image of the homomorphism $\sigma^*\colon C^\infty(\mathcal G^{(0)})\to C^\infty(\mathcal G^{(1)})$, and which are right invariant, cf.~\eqref{E:regright}.


Let us describe these differential operators more explicitly. 
There is the obvious inclusion 
\begin{equation}\label{E:DOhomEF}
\Gamma^\infty(\hom(E,F))\subseteq\DO_\sigma(\mathcal G;\tau^*E,\tau^*F)^G.
\end{equation}
Next, by restriction to the space of units, one obtains an isomorphism of Lie algebras
\begin{equation}\label{E:XAG}
\mathfrak X_\sigma(\mathcal G)^G=\Gamma^\infty(\mathcal A\mathcal G).
\end{equation}
Here $\mathcal A\mathcal G:=\iota^*\ker(T\sigma)$ denotes the vector bundle underlying the Lie algebroid of $\mathcal G$, and in the left hand side  $\mathfrak X_\sigma(\mathcal G)^G$ denotes the Lie algebra of $\sigma$-vertical vector fields on $\mathcal G$ which are right invariant.
A vector field is called $\sigma$-vertical if it is a section of $\ker(T\sigma)\subseteq T\mathcal G$, equivalently, if it is tangential to the submanifolds $\mathcal G_x$.


Every linear connection on $E$ pulls back to a right invariant connection $\tau^*\nabla$ on $\tau^*E$ and provides an injective linear map
\begin{equation}\label{E:DOnabla}
\Gamma^\infty(\mathcal A\mathcal G)\xrightarrow\nabla\DO_\sigma(\mathcal G;\tau^*E,\tau^*E)^G,\qquad X\mapsto(\tau^*\nabla)_{\tilde X},
\end{equation}
where $\tilde X\in\mathfrak X_\sigma(\mathcal G)^G$ denotes the right invariant extension of $X\in\Gamma^\infty(\mathcal A\mathcal G)$, see \eqref{E:XAG}.
Locally, every right invariant $\sigma$-vertical differential operator can be written as a composition of operators of the form \eqref{E:DOhomEF} and \eqref{E:DOnabla}.
More precise algebraic descriptions in terms of universal enveloping algebras can be found in \cite[Theorem~3]{NWX99} or \cite[Section~7.1]{EY15}.






\subsection{Groupoids on filtered manifolds}






Let us now discuss four groupoids that play an important role in van~Erp and Yuncken's pseudodifferential calculus:
the pair groupoid, the osculating groupoid, the individual osculating groups, and the Heisenberg tangent groupoid.


For every smooth manifold $M$ the \emph{pair groupoid} has objects $\mathcal G^{(0)}=M$, arrows $\mathcal G^{(1)}=M\times M$, units $\iota(x)=(x,x)$, source map $\sigma(x,y)=y$, target map $\tau(x,y)=x$, multiplication $\mu((x,y),(y,z))=(x,z)$ and inversion $\nu(x,y)=(y,x)$, where $x,y,z\in\mathcal G^{(0)}=M$.
For any two $x,y\in\mathcal G^{(0)}$ there exists a unique $g\in\mathcal G^{(1)}$ such that $\tau(g)=x$ and $\sigma(g)=y$.
For any two vector bundles $E$ and $F$ over $\mathcal G^{(0)}=M$ this provides a canonical isomorphism identifying the regular representation at $x$ with the regular representation at $y$, see \eqref{E:regright}.
More explicitly, $\tau_x\colon\mathcal G_x\to\mathcal G^{(0)}=M$ is a diffeomorphism which identifies all the operators $\Op^k_x$ with a single operator $\Op^k\colon\Gamma^\infty_c(E)\to\Gamma^\infty(F)$ such that the following diagram commutes for all $x\in\mathcal G^{(0)}$:
\begin{equation}
\vcenter{
\xymatrix{
\Gamma^\infty_c(E)\ar[rr]^-{\Op^k}\ar[d]^-\cong_-{\tau_x^*}
&&\Gamma^\infty(F)\ar[d]_-\cong^-{\tau_x^*}
\\
\Gamma_c^\infty(\tau_x^*E)\ar[rr]^-{\Op^k_x}
&&\Gamma^\infty(\tau_x^*F)
}}
\end{equation}
For the pair groupoid the kernels $\mathcal K(M\times M;E,F)$ introduced in the preceding section, see Definition~\ref{D:KGEF}, correspond precisely to operators in the class $\mathcal O(E,F)$ with composition and transposition of operators as discussed in Section~\ref{SS:prelim}.
The kernels which are supported on the space of units correspond precisely to $\DO(E,F)$, the space of differential operators with smooth coefficients. 


For every filtered manifold $M$ the bundle of osculating groups $\pi\colon\mathcal TM\to M$ is a bundle of (nilpotent) Lie groups and gives rise to \emph{osculating groupoid} with objects $\mathcal G^{(0)}=M$ and arrows $\mathcal G^{(1)}=\mathcal TM$.
In this case $\sigma=\pi=\tau$ and $\iota\colon M\to\mathcal TM$ is the zero section, assigning the neutral element in the group $\mathcal T_xM$ to $x\in M$.
Two arrows $g$ and $h$ are composeable iff they lie in the same fiber, $g,h\in\mathcal T_xM$ say, and $\mu(g,h)$ is given by multiplication in the group $\mathcal T_xM$.
If $E$ and $F$ are two vector bundles over $M$, then the regular representation of $k\in\mathcal K(\mathcal TM;E,F)$ at $x\in M$ identifies to a right invariant operator on the group $\mathcal T_xM$ with matrix valued convolution kernel $k^x\in\Gamma^{-\infty}(|\Lambda|_{\mathcal T_xM})\otimes\hom(E_x,F_x)$:
\begin{equation}
\vcenter{
\xymatrix{
C^\infty_c(\mathcal T_xM,E_x)\ar[rr]^-{k^x*-}\ar@{=}[d]
&&C^\infty(\mathcal T_xM,F_x)\ar@{=}[d]
\\
\Gamma_c^\infty(\tau_x^*E)\ar[rr]^-{\Op^k_x}
&&\Gamma^\infty(\tau_x^*F)
}}
\end{equation}
The kernels which are supported on the space of units correspond precisely to the space of differential operators $\Gamma^\infty(\pi^*E)\to\Gamma^\infty(\pi^*F)$ which are vertical, i.e.\ commute with functions in the image of the homomorphism $\pi^*\colon C^\infty(M)\to C^\infty(\mathcal TM)$, and restrict to right invariant operators on each fiber $\mathcal T_xM$.


Each \emph{osculating group} $\mathcal T_xM$ can be considered as a groupoid, and the inclusion $\mathcal T_xM\subseteq\mathcal TM$ is a map of groupoids.
Correspondingly, restriction provides a map
\begin{equation}\label{E:evx}
\ev_x^*\colon\mathcal K(\mathcal TM;E,F)\to\mathcal K(\mathcal T_xM;E_x,F_x)
\end{equation}
which is compatible with composition and transposition.


For each $\lambda>0$, the scaling automorphism $\delta_\lambda\colon\mathcal TM\to\mathcal TM$ is an automorphisms of groupoids.
It induces an action on $\mathcal K(\mathcal TM;E,F)$ such that 
$$
(\delta_{x,\lambda})_*\circ\Op^k_x
=\Op^{(\delta_\lambda)_*k}_x\circ(\delta_{x,\lambda})_*
$$
for all $x\in\mathcal G^{(0)}$ where $(\delta_{x,\lambda})_*$ denotes the action on $C^\infty_c(\mathcal T_xM,E_x)$ and $C^\infty(\mathcal T_xM,F_x)$ induced by the scaling automorphism $\delta_{x,\lambda}\colon\mathcal T_xM\to\mathcal T_xM$.
Clearly, the map \eqref{E:evx} intertwines the actions induced by $\delta_\lambda$ and $\delta_{x,\lambda}$.


Let us finally consider the \emph{Heisenberg tangent groupoid} $\mathbb TM$ which has recently been constructed by van~Erp and Yuncken \cite{EY15,EY16}, see also \cite{CP15}.
This is a smooth groupoid associated to a filtered manifold $M$ with objects $\mathcal G^{(0)}=M\times\R$ and arrows
$$
\mathcal G^{(1)}=\mathbb TM=(\mathcal T^\op M\times\{0\})\sqcup(M\times M\times\R^\times).
$$
Here $\mathcal T^\op M$ denotes the groupoid $\mathcal TM$ discussed in the preceding paragraph, equipped with the opposite multiplication, $\mu^\op(g,h)=\mu(h,g)$, so that restriction along $M\subseteq\mathcal TM$ provides an isomorphism of Lie algebroids $\mathfrak tM=\mathcal A(\mathcal T^\op M)$.
\footnote{The opposite groupoid $\mathcal T^\op M$ mediates between two conflicting, yet common, conventions we are following: The Lie algebra of a Lie group is usually defined by restricting the Lie bracket to \emph{left invariant} vector fields, while the Lie algebroid of a smooth groupoid is defined using \emph{right invariant} vector fields, see~\eqref{E:XAG}.}
The decomposition above only describes $\mathbb TM$ ``algebraically'', the subtle ingredient is the smooth structure along $\mathcal T^\op M\times\{0\}$.
We refer to \cite{EY15,EY16} for details.  
The evaluations $\ev_0\colon\mathbb TM\to\mathcal T^\op M$ and $\ev_t\colon\mathbb TM\to M\times M$ are smooth maps of groupoids for all $t\in\R^\times$.
These give rise to homomorphisms of Lie algebras
$$
\Gamma^\infty(\mathfrak tM)
=\Gamma^\infty(\mathcal A(\mathcal T^\op M))\xleftarrow{\ev_0}
\Gamma^\infty(\mathcal A(\mathbb TM))\xrightarrow{\ev_t}
\Gamma^\infty(\mathcal A(M\times M))=\mathfrak X(M).
$$
Using the evaluations $\ev_t$ for $t\neq0$, we obtain an isomorphism of Lie algebras
\begin{equation}\label{E:ATTM}
\Gamma^\infty(\mathcal A(\mathbb TM))
=\bigl\{\mathbb X\in\Gamma^\infty(TM\times\R):\textrm{$\partial_t^k|_{t=0}\mathbb X\in\Gamma^\infty(T^{-k}M)$ for all $k\in\N_0$}\bigr\}
\end{equation}
in such a way that the bracket and anchor correspond to the (constant) bracket and anchor of the Lie algebroid $TM\times\R$, see \cite[Section~3.5]{EY15}.
Moreover, for every $\mathbb X\in\Gamma^\infty(\mathcal A(\mathbb TM))$,
\begin{equation}\label{E:ev0}
\ev_0(\mathbb X)=\lim_{t\to0}\delta_tS^{-1}\ev_t(\mathbb X),
\end{equation}
where $S\colon\mathfrak tM\to TM$ is any splitting of the filtration,\footnote{These splittings are called degradings by van~Erp and Yuncken.}
see \cite[Section~3.5]{EY15}.
This is indeed independent of the splitting $S$, see \cite[Lemma~32]{EY15}.
Every splitting $S$ provides a non-canonical isomorphism of $C^\infty(M\times\R)$ modules, 
$$
\Gamma^\infty(\mathcal A(\mathbb TM))\cong\Gamma^\infty(\mathfrak tM\times\R),
\qquad\mathbb X\mapsto\bigl((x,t)\mapsto\delta_tS^{-1}\mathbb X(x,t)\bigr)
$$
see~\cite[Lemma~30]{EY15}.
The construction of the tangent groupoid in \cite{EY15,EY16} is accomplished by integrating the Lie algebroid $\mathcal A(\mathbb TM)$, using a theorem of Nistor, see \cite{N00}.


The scaling automorphism $\delta_\lambda$ on $\mathcal TM$ can be extended to an automorphism of the groupoid $\mathbb TM$, called the \emph{zoom action}, by putting $\delta_\lambda(x,y,t):=(x,y,\lambda^{-1}t)$ for all $x,y\in M$, $t\in\R^\times$ and $\lambda>0$, see \cite[Definition~34]{EY15}.
If $E$ and $F$ are two vector bundles over $M$, we will consider the vector bundles $\mathbb E:=E\times\R$ and $\mathbb F:=F\times\R$ over the space of units $M\times\R$.
Although the dilations $\delta_\lambda$ do not restrict to the identical map on the space of units, they still act canonically on $\mathcal K(\mathbb TM;\mathbb E,\mathbb F)$ such that the diagram
\begin{equation}\label{E:TTMev}
\vcenter{
\xymatrix{
\mathcal K(\mathcal T^\op M;E,F)\ar[d]^-{(\delta_\lambda)_*}
&\mathcal K(\mathbb TM;\mathbb E,\mathbb F)\ar[r]^-{\ev_t^*}\ar[l]_-{\ev_0^*}\ar[d]^-{(\delta_\lambda)_*}
&\mathcal K(M\times M;E,F)\ar@{=}[d]
\\
\mathcal K(\mathcal T^\op M;E,F)
&\mathcal K(\mathbb TM;\mathbb E,\mathbb F)\ar[r]^-{\ev^*_{t/\lambda}}\ar[l]_-{\ev_0^*}
&\mathcal K(M\times M;E,F)
}}
\end{equation}
commutes for all $t\in\mathbb R^\times$ and $\lambda>0$.
Clearly, the evaluation maps $\ev_0^*$ and $\ev_t^*$ above are compatible with composition and transposition.


The zoom action was introduced in \cite{DS14} for Connes' tangent groupoid.
It provides a way to describe the classical pseudodifferential operators.
The main intention here is to recover a similar calculus.






\subsection{Van~Erp and Yuncken's pseudodifferential calculus}\label{SS:calculus}






Following van~Erp and Yuncken \cite[Section~4.1]{EY15}, we will now introduce vector valued Heisenberg classical pseudodifferential operators.
The basic properties of this calculus are summarized in Propositions~\ref{P:Psi} and \ref{P:Psimap} below, the Rockland theorem can be found in Theorem~\ref{T:Rockland}.


In the flat case, that is, if the filtration on $M$ locally is diffeomorphic to the left invariant filtration on a graded nilpotent Lie group, the calculus described below (in the scalar case) coincides with the one constructed by Christ, Geller, G{\l}owacki, and Polin in \cite{CGGP92}, see Remark~\ref{R:CGGP} below.
On Heisenberg manifolds it reduces to the calculus in \cite{P08}.


\begin{definition}[van~Erp and Yuncken]
For each complex number $s$ we let $\Psi^s(E,F)\subseteq\mathcal O(E,F)$ denote the space of operators whose Schwartz kernel $k\in\mathcal K(M\times M;E,F)$ can be extended across the tangent groupoid to a kernel $\K\in\mathcal K(\mathbb TM;\mathbb E,\mathbb F)$ which is \emph{essentially homogeneous} of order $s$, that is, $\ev_1^*\K=k$ and, for all $\lambda>0$,
$$
(\delta_\lambda)_*\K=\lambda^s\K\mod\mathcal K^\infty(\mathbb TM;\mathbb E,\mathbb F).
$$
These are called \emph{kernels} or \emph{pseudodifferential operators} of Heisenberg order $s$.
\end{definition}


Defining the space of \emph{principal cosymbols of order $s$} by
$$
\Sigma^s(E,F)
:=\left\{k\in\frac{\mathcal K(\mathcal TM;E,F)}{\mathcal K^\infty(\mathcal TM;E,F)}:\textrm{$(\delta_\lambda)_*k=\lambda^sk$ for all $\lambda>0$}\right\},
$$
cf.\ \cite[Definition~46]{EY15}, we obtain a principal cosymbol map,
$$
\sigma^s\colon\Psi^s(E,F)\to\Sigma^s(E,F),\qquad\sigma^s(A):=\nu^*\ev_0^*\K,
$$
where $\K\in\mathcal K(\mathbb TM;E,F)$ is any essentially homogeneous kernel of order $s$ extending the Schwartz kernel of $A$, cf.\ \cite[Definition~48]{EY15}.
Here $\nu\colon\mathcal TM\to\mathcal T^\op M$ denotes the isomorphism of groupoids given by inversion.
For this symbol map to be well defined one needs to show that if $\K$ is essentially homogeneous of order $s$ and $\ev_1^*\K$ is smooth, then $\ev_0^*\K$ is smooth too.
The argument for this is the same as in the scalar case given in \cite[Proposition~45]{EY15}.
For $k\in\N_0$ there is a canonical inclusion,
\begin{equation}\label{E:incUS}
\Gamma^\infty(\mathcal U_{-k}(\mathfrak tM)\otimes\hom(E,F))\subseteq\Sigma^k(E,F)
\end{equation}
provided by regarding both sides as right invariant $\sigma$-vertical operators on $\mathcal T^\op M$.


\begin{remark}\label{R:conormal}
Note that in the original definition in \cite{EY15} the kernel $\K$ is a priori not assumed to be smooth away from the units but the essential homogeneity forces this conormality, see \cite[Proposition~41]{EY15}.
Hence, the calculus obtained here is equivalent to the original one.
\end{remark}


The following basic properties of the Heisenberg calculus have been established in \cite{EY15} for scalar valued operators. 


\begin{proposition}[Van~Erp--Yuncken]\label{P:Psi}
Let $E$, $F$ and $G$ be vector bundles over a filtered manifold $M$ and let $s$ be any complex number.
Then the following hold true:
\begin{enumerate}[(a)]
\item\label{P:Psi:O}
We have $\Psi^s(E,F)\subseteq\mathcal O(E,F)$.
Hence, every $A\in\Psi^s(E,F)$ induces continuous operators
$A\colon\Gamma_c^\infty(E)\to\Gamma^\infty(F)$ and $A\colon\Gamma^{-\infty}_c(E)\to\Gamma^{-\infty}(F)$.
If, moreover, $A$ is properly supported, then we obtain continuous operators $\Gamma^\infty(E)\to\Gamma^\infty(F)$, $\Gamma^\infty_c(E)\to\Gamma^\infty_c(F)$, $\Gamma^{-\infty}(E)\to\Gamma^{-\infty}(F)$, and $\Gamma^{-\infty}_c(E)\to\Gamma^{-\infty}_c(F)$.
\item\label{P:Psi:symbsequ}
We have $\Psi^{s-1}(E,F)\subseteq\Psi^s(E,F)$, and the following sequence is exact:
$$
0\to\Psi^{s-1}(E,F)\to\Psi^s(E,F)\xrightarrow{\sigma^s}\Sigma^s(E,F)\to0
$$
\item\label{P:Psi:smooth}
$\bigcap_{k\in\N}\Psi^{s-k}(E,F)=\mathcal O^{-\infty}(E,F)$.
\item\label{P:Psi:mult}
If $A\in\Psi^{s_1}(E,F)$, $B\in\Psi^{s_2}(F,G)$, and at least one of the two is properly supported, then $BA\in\Psi^{s_2+s_1}(E,G)$ and $\sigma^{s_2+s_1}(BA)=\sigma^{s_2}(B)\sigma^{s_1}(A)$.
\item\label{P:Psi:trans}
If $A\in\Psi^s(E,F)$, then $A^t\in\Psi^s(F',E')$ and $\sigma^s(A^t)=\sigma^s(A)^t$.
\item\label{P:Psi:DO}
$\DO^k(E,F)=\DO(E,F)\cap\Psi^k(E,F)$ for all $k\in\N_0$, and the principal symbol considered here extends the one for differential operators via the canonical inclusion \eqref{E:incUS}.
\item\label{P:Psi:parametrix}
Let $A\in\Psi^s(E,F)$ and assume that there exists $b\in\Sigma^{-s}(F,E)$ such that $b\,\sigma^s(A)=1$.
Then there exists a left parametrix $B\in\Psi^{-s}_\prop(F,E)$ such that $\sigma^{-s}(B)=b$ and $BA-\id$ is a smoothing operator. 
Moreover, an analogous statement involving right parametrices holds true.
\end{enumerate}
\end{proposition}


\begin{proof}
The proof is a straight forward generalization of the arguments in \cite{EY15}.
The statement $\Psi^s(E,F)\subseteq\mathcal O(E,F)$ in \itemref{P:Psi:O} is trivial, given our definition, cf.\ Remark~\ref{R:conormal}.
Every operator in $\mathcal O(E,F)$ induces continuous maps as indicated, see Section~\ref{SS:prelim}.


Part \itemref{P:Psi:symbsequ} can be proved exactly as in \cite[Lemma~50 and Corollary~53]{EY15}.


Part \itemref{P:Psi:smooth} can be proved exactly as in \cite[Corollary~73]{EY15}.


Part \itemref{P:Psi:mult} can be proved exactly as in \cite[Proposition~49]{EY15}:
Let $\K\in\mathcal K(\mathbb TM;\mathbb E,\mathbb F)$ and $\mathbb L\in\mathcal K(\mathbb TM;\mathbb F,\mathbb G)$ be essentially homogeneous of order $s_1$ and $s_2$, respectively, such that $\ev_1^*\mathbb K$ and $\ev_1^*\mathbb L$ are the Schwartz kernels of $A$ and $B$, respectively. 
If one of $A$ or $B$ is properly supported, then the corresponding $\mathbb K$ or $\mathbb L$, can be chosen properly supported too.
Then, $\mathbb L*\mathbb K\in\mathcal K(\mathbb TM;\mathbb E,\mathbb G)$ and $(\delta_\lambda)_*(\mathbb L*\mathbb K)=(\delta_\lambda)_*\mathbb L*(\delta_\lambda)_*\mathbb K=\lambda^{s_2+s_1}\mathbb L*\mathbb K\mod\mathcal K^\infty(\mathbb TM;\mathbb E,\mathbb G)$.
Hence, $\mathbb L*\mathbb K$ is essentially homogeneous of order $s_1+s_2$, and since $\ev_1^*(\mathbb L*\mathbb K)=\ev_1^*\mathbb L*\ev_1^*\mathbb K$ is the Schwartz kernel of $BA$, we obtain $BA\in\Psi^{s_1+s_2}(E,G)$.
Moreover, $\sigma^{s_1+s_2}(BA)=\nu^*\ev_0^*(\mathbb L*\mathbb K)=\nu^*\ev_0^*\mathbb L*\nu^*\ev_0^*\mathbb L=\sigma^{s_2}(B)\sigma^{s_1}(A)$.


To see \itemref{P:Psi:trans} let $\K\in\mathcal K(\mathbb TM;\mathbb E,\mathbb F)$ be essentially homogeneous of order $s$ such that $\ev_1^*\mathbb K$ is the Schwartz kernel of $A$. 
Then $\mathbb K^t\in\mathcal K(\mathbb TM;\mathbb F',\mathbb E')$ and $(\delta_\lambda)_*\mathbb K^t=((\delta_\lambda)_*\mathbb K)^t=\lambda^s\mathbb K^t\mod\mathcal K^\infty(\mathbb TM;\mathbb F',\mathbb E')$.
Hence, $\mathbb K^t$ is essentially homogeneous of order $s$, and since $\ev_1^*\mathbb K^t=(\ev_1^*\mathbb K)^t\in\mathcal K(M\times M;F',E')$ is the Schwartz kernel of $A^t$, we obtain $A^t\in\Psi^s(F',E')$.
Moreover, $\sigma^s(A^t)=\nu^*\ev_0^*\mathbb K^t=(\nu^*\ev_0^*\mathbb K)^t=\sigma^s(A)^t$.


Part \itemref{P:Psi:DO} can be proof exactly as in \cite[Section~7.3]{EY15}.
Let us first consider $A\in\Gamma^\infty(\hom(E,F))=\DO^0(E,F)$.
In view of \eqref{E:DOG} and \eqref{E:DOhomEF}, the pull back $\pr_1^*A\in\Gamma^\infty(\hom(\mathbb E,\mathbb F))$ provides a kernel in $\mathcal K_{M\times\R}(\mathbb TM;\mathbb E,\mathbb F)$ which is strictly homogeneous of order zero, and such that $\ev_1^*\mathbb K$ is the Schwartz kernel of $A$.
Hence, $A\in\Psi^0(E,F)$ and $\nu^*\sigma^0(A)=\ev_0^*\mathbb K=A$.
To discuss actual differential operators note that if $X\in\Gamma^\infty(T^{-k}M)$ then $\mathbb X(x,t):=t^kX(x)$ provides an extension $\mathbb X\in\Gamma^\infty(\mathcal A(\mathbb TM))$ which is strictly homogeneous of order $k$ and such that $\ev_1^*\mathbb X=X$, see \eqref{E:ATTM} or \cite[Section~7.3]{EY15}.
Moreover, in view of \eqref{E:ev0}, $\ev_0^*\mathbb X=\pi_{-k}(X)$ where $\pi_{-k}\colon T^{-k}M\to\gr_{-k}(TM)$ denotes the canonical vector bundle projection.
Now suppose $\nabla$ is a linear connection on $E$ and consider the induced connection $\nabla^{\mathbb E}:=\pr_1^*\nabla$ on $\mathbb E$.
Using \eqref{E:DOG} and \eqref{E:DOnabla} we see that $\nabla^{\mathbb E}_{\mathbb X}$ provides a kernel $\mathbb K\in\mathcal K_{M\times\R}(\mathbb TM;\mathbb E,\mathbb E)$ which is strictly homogeneous of order $k$ and such that $\ev_1^*\mathbb K$ is the Schwartz kernel of the operator $\nabla_X\in\DO^k(E)$.
Hence, $\nabla_X\in\Psi^k(E)$ and $\nu^*\sigma^k(\nabla_X)=\ev_0^*(\nabla^{\mathbb E}_{\mathbb X})=\pi_{-k}(X)\otimes\id$ which coincides with the principal symbol for differential operators, see~\eqref{E:snablaX}.
This completes the proof of part \itemref{P:Psi:DO} since, locally, every differential operator can be written as a finite composition of vector bundle homomorphisms and operators of the form $\nabla_X$.


Part \itemref{P:Psi:parametrix} can be proved in the standard way, assuming that the calculus is asymptotically complete. 
In the scalar case this is described in \cite[Theorem~76]{EY15} similar to the standard argument and generalizes to the vector valued case.
W.l.o.g.\ we may assume $A\in\Psi^s_\prop(E,F)$.
In view of \itemref{P:Psi:symbsequ}, there exists $\tilde B\in\Psi^{-s}_\prop(F,E)$ such that  $\sigma^{-s}(\tilde B)=b$.
Using \itemref{P:Psi:mult}, we obtain $\id-\tilde BA\in\Psi^0_\prop(E)$ and $\sigma^0(\id-\tilde BA)=\sigma^0(\id)-\sigma^{-s}(\tilde B)\sigma^s(A)=1-b\,\sigma^s(A)=0$.
In view of \itemref{P:Psi:symbsequ}, we conclude $R:=\id-\tilde BA\in\Psi^{-1}_\prop(E)$.
Hence, by asymptotic completeness, there exists $S\in\Psi^0_\prop(E)$ such that 
$$
S-\sum_{j=0}^{k-1}R^j\in\Psi^{-k}_\prop(E),
$$
for all $k\in\N$.
In particular, $S-\id\in\Psi^{-1}_\prop(E)$ and thus $\sigma^0(S)=\sigma^0(\id)=1$, see \itemref{P:Psi:symbsequ}.
Moreover, $S(\id-R)-\id\in\Psi^{-k}_\prop(E)$, for all $k\in\N$.
Using \itemref{P:Psi:smooth}, we conclude $S(\id-R)=\id\mod\mathcal O^{-\infty}(E)$.
Putting $B:=S\tilde B\in\Psi^{-s}_\prop(F,E)$, we obtain $\sigma^{-s}(B)=\sigma^0(S)\sigma^{-s}(\tilde B)=1\cdot b=b$ and $BA=S\tilde BA=S(\id-R)=\id\mod\mathcal O^{-\infty}(E)$.
The construction of right parametrices can be reduced to the construction of left parametrices using transposition.
\end{proof}


\begin{remark}
For Heisenberg manifolds, a statement similar to Proposition~\ref{P:Psi}\itemref{P:Psi:parametrix} can be found in \cite[Proposition~3.3.1]{P08}.
\end{remark}


\begin{remark}[Formal adjoints]\label{R:Psi:adjoint}
If $A\in\Psi^s(E,F)$ and $A^*$ denotes the formal adjoint w.r.\ to inner products of the form \eqref{E:llrr}, then $A^*\in\Psi^{\bar s}(F,E)$ and $\sigma^s(A^*)=\sigma^{\bar s}(A)^*$.
Indeed, in view of \eqref{E:A*At} this follows immediately from the assertions \itemref{P:Psi:mult}, \itemref{P:Psi:trans}, and \itemref{P:Psi:DO} of Proposition~\ref{P:Psi}.
Here the adjoint of the cosymbol, $\sigma^s(A)^*$, is understood as follows:
If $k\in\mathcal K(\mathcal TM;E,F)$, then $k^*\in\mathcal K(\mathcal TM;F,E)$ is defined by $k^*(g)=k(g^{-1})^*$ where $g
\in\mathcal T_xM$, and the right hand side denotes the adjoint of $k(g^{-1})$ with respect to the inner products $h_{E,x}$ and $h_{F,x}$ on $E_x$ and $F_x$, respectively.
Since this star preserves the subspace $\mathcal K^\infty(\mathcal TM;E,F)$ and commutes with $\delta_\lambda$, it induces an involution $\Sigma^{\bar s}(F,E)\xrightarrow{*}\Sigma^s(E,F)$, for each complex $s$.
For $s\in\N_0$ this extends the involution in Remark~\ref{R:sA*} via the inclusion \eqref{E:incUS}.
\end{remark}


\begin{remark}[Asymptotic expansion in exponential coordinates]\label{R:expcoor}
Let us use exponential coordinates as in \cite{EY15} to identify an open neighborhood $U$ of the zero section in $\mathcal TM$ with an open neighborhood $V$ of the diagonal in $M\times M$,
$$
\mathcal TM\supseteq U\xrightarrow\varphi V\subseteq M\times M.
$$
This diffeomorphism is obtained by restricting the composition
$$
\mathcal TM\xleftarrow{\exp}\mathfrak tM\xrightarrow{-S}TM\supseteq U'\xrightarrow{(p,\exp^\nabla)}M\times M.
$$
Here $\exp$ denotes the fiber wise exponential map;
$S$ is a splitting of the filtration;
$\exp^\nabla$ denotes the exponential map associated with a linear connection on the tangent bundle which preserves the grading $TM=\bigoplus_pS(\mathfrak t^pM)$; 
$U'$ is an open neighborhood of the zero section in $TM$ on which $\exp^\nabla$ is defined and a diffeomorphism onto its image; and
$p\colon TM\to M$ denotes the canonical projection, cf.\ \cite[Equation~(22) in Section~3.7]{EY15}.
Let $\pi\colon\mathcal TM\to M$ denote the canonical projection and, after possibly shrinking $U$, fix an isomorphism of vector bundles $\tilde\varphi$ over $\varphi$, 
$$
\xymatrix{
\bigl(\pi^*\hom(E,F)\otimes\Omega_\pi\bigr)|_U\ar[d]\ar[r]^-{\tilde\varphi}_-\cong&(F\boxtimes E')|_V\ar[d]
\\
U\ar[r]^-\varphi_\cong&V,
}
$$ 
which restricts to the tautological identification over the zero section/diagonal. 
For every Schwartz kernel $k\in\Gamma^{-\infty}(F\boxtimes E')$ we obtain $\tilde\varphi^*(k|_V)\in\Gamma^{-\infty}((\pi^*\hom(E,F)\otimes\Omega_\pi)|_U)$.
If $k$ is the kernel of an operator $A\in\Psi^s(E,F)$, then there exist an asymptotic expansion of the form
\begin{equation}\label{E:asexp}
\tilde\varphi^*(k|_V)\sim k_0+k_1+k_2+\cdots,
\end{equation}
where $k_j\in\Sigma^{s-j}(E,F)$ and $\sigma^s(A)=k_0$.
More precisely, for every $r\in\N$ there exists $N\in\N$ such that $\tilde\varphi^*(k|_V)-\sum_{j=0}^Nk_j$ is of class $C^r$ on $U$.
Indeed, if $l\in\mathcal K(\mathcal TM;E,F)$ is essentially homogeneous of order $s$ and $\chi\in C^\infty(M\times M)$ is supported in $V$ and $s\equiv1$ in a neighborhood of the diagonal, then $\chi\cdot(\tilde\varphi_*l)\in\Gamma^{-\infty}(F\boxtimes E')$ is the Schwartz kernel of an operator $B\in\Psi^s(E,F)$ with $\sigma^s(B)=l$, cf.\ the proof of \cite[Lemma~50]{EY15}.
Combining this with Proposition~\ref{P:Psi}\itemref{P:Psi:symbsequ}, we inductively obtain $k_j$ as above such that $k-\chi\cdot\tilde\varphi_*(\sum_{j=0}^Nk_j)\in\Psi^{s-N-1}(E,F)$.
Moreover, the proof of Proposition~\ref{P:Psi}\itemref{P:Psi:smooth} shows that for each $r\in\N$ there exists $N_0$ such that for all $N_1\geq N_0$ the kernels of operators in $\Psi^{s-N_1}(E,F)$ are in $\Gamma^r(F\boxtimes E')$, whence the asymptotic expansion \eqref{E:asexp}. 
Conversely, if a Schwartz kernel $k$ is smooth away from the diagonal and admits an asymptotic expansion of the form \eqref{E:asexp}, then the corresponding operator is in $\Psi^s(E,F)$, see \cite[Theorem~76]{EY15}. 
Actually, any sequence $k_j$ can be realized.
\end{remark}


Let us now link the principal cosymbols of van~Erp and Yuncken with the principal cosymbols used by Christ, Geller, G{\l}owacki, and Polin in \cite{CGGP92}.
For every complex $s$, put 
$$
\mathcal P^s(\mathcal TM;E,F):=\bigl\{k\in\mathcal K^\infty(\mathcal TM;E,F):\textrm{$(\delta_\lambda)_*k=\lambda^sk$ for all $\lambda>0$}\bigr\}.
$$ 
Clearly, $\mathcal P^s(\mathcal TM;E,F)=0$ if $-s-n\notin\mathbb N_0$ where 
\begin{equation}\label{E:hdim}
n:=-\sum_pp\cdot\rank(\mathfrak t^pM)=-\sum_pp\cdot\rank(T^pM/T^{p+1}M)
\end{equation} 
denotes the \emph{homogeneous dimension} of $M$.
For $-s-n\in\mathbb N_0$, by Taylor's theorem, this is the space of smooth kernels which restrict to (matrices of) polynomial volume densities of homogeneous degree $s$ on each fiber $\mathcal T_xM$, using the exponential map to canonically identify $\mathcal T_xM$ with the graded vector space $\mathfrak t_xM$.
Equivalently, these can be characterized as smooth families of polynomial volume densities of homogeneous degree $s$ on the fibers $\mathcal T_xM=\mathfrak t_xM$, smoothly parametrized by $x\in M$.


\begin{lemma}\label{L:EYvsCGGP}
The identical map on $\mathcal K(\mathcal TM;E,F)$ induces a canonical identification
$$
\Sigma^s(E,F)=\left\{k\in\frac{\mathcal K(\mathcal TM;E,F)}{\mathcal P^s(\mathcal TM;E,F)}:\text{$(\delta_\lambda)_*k=\lambda^sk$ for all $\lambda>0$}\right\}.
$$
Moreover, with respect to a homogeneous norm $|-|$ on $\mathfrak tM$, and using the fiber wise exponential map, $\exp\colon\mathfrak tM\to\mathcal TM$,
every kernel $k\in\mathcal K(\mathcal TM;E,F)$ which is essentially homogeneous of order $s$ can be written in the form
\begin{equation}\label{E:EYvsCGGP}
k=k_\infty+k_s+p_s\log|\exp^{-1}(-)|
\end{equation}
where $k_\infty\in\mathcal K^\infty(\mathcal TM;E,F)$, $k_s\in\mathcal K(\mathcal TM;E,F)$ homogeneous of order $s$, that is, $(\delta_\lambda)_*k_s=\lambda^sk_s$ for all $\lambda>0$, and $p_s\in\mathcal P^s(\mathcal TM;E,F)$.
If $-s-n\notin\mathbb N_0$, then $p_s=0$ and the decomposition in \eqref{E:EYvsCGGP} is unique.
If $-s-n\in\mathbb N_0$, then the decomposition in \eqref{E:EYvsCGGP} is unique up to adding a kernel in $\mathcal P^s(\mathcal TM;E,F)$ to $k_s$ and subtracting it from $k_\infty$ in turn.
\end{lemma}


\begin{proof}
We begin by showing that each element of $\Sigma^s(E,F)$ admits a representative in $\mathcal K(\mathcal TM;E,F)$ which is homogeneous of order $s$ mod $\mathcal P^s(\mathcal TM;E,F)$.
To this end let $k\in\mathcal K_\prop(\mathcal TM;E,F)$ be properly supported and essentially homogeneous of order $s$.
Use the diffeomorphism $\mathcal TM=\mathfrak tM$ provided by the fiber wise exponential map, to identify $k$ with an element in $\Gamma^{-\infty}_\prop(\mathfrak tM;p^*\hom(E,F)\otimes\Omega_p)$, and let $\hat k\in\Gamma^\infty(\mathfrak t^*M;\pi^*\hom(E,F))$ denote its fiber wise Fourier transform, see for instance \cite[Section~5]{EY15}.
Here $p\colon\mathfrak tM\to M$ and $\pi\colon\mathfrak t^*M\to M$ denote the canonical vector bundle projections.
More explicitly, for $\xi\in\mathfrak t_x^*M$ we have 
$$
\hat k(\xi)=\langle k_x,e^{-\mathbf i(\xi,-)}\rangle=\int_{\mathfrak t_xM}k_x(X)e^{-\mathbf i(\xi,X)}dX\in\hom(E_x,F_x).
$$
Recall that the restriction of $k$ to the fiber $\mathfrak t_xM$ is a volume density with values in $\hom(E_x,F_x)$, whence this pairing (integral) is unambiguously defined.
Since $k$ is essentially homogeneous of order $s$ and properly supported, $\hat k$ is homogeneous of order $s$ mod $\mathcal S(\mathfrak t^*M;\pi^*\hom(E,F))$, the fiber wise Schwartz space.
More precisely, 
$$
(\dot\delta'_\lambda)^*\hat k-\lambda^s\hat k\in\mathcal S(\mathfrak t^*M;\pi^*\hom(E,F)),
$$ 
for all $\lambda>0$.
Here $\dot\delta'_\lambda:=\dot\delta_\lambda^*\colon\mathfrak t^*M\to\mathfrak t^*M$ denotes the vector bundle automorphism dual to $\dot\delta_\lambda\colon\mathfrak tM\to\mathfrak tM$, that is, $\langle\dot\delta_\lambda'\xi,X\rangle=\langle\xi,\dot\delta_\lambda X\rangle$ for all $\xi\in\mathfrak t^*_xM$ and $X\in\mathfrak t_xM$.


According to \cite[Lemma~58]{EY15}, there exists $f\in\Gamma^\infty(\mathfrak t^*M\setminus M;\pi^*\hom(E,F))$ such that $(\dot\delta'_\lambda)^*f=\lambda^sf$ for all $\lambda>0$, and
\begin{equation}\label{E:QWERTY}
(1-\eta)\cdot(\hat k-f)\in\mathcal S(\mathfrak t^*M,\pi^*\hom(E,F)),
\end{equation}
where $\eta\in C^\infty_\prop(\mathfrak t^*M)$ is a properly supported smooth bump function on $\mathfrak t^*M$ such that $\eta\equiv1$ in a neighborhood of $M\subseteq\mathfrak t^*M$.


We extend $f$ to a globally defined distribution $\tilde f\in\Gamma^{-\infty}(\mathfrak t^*M;\pi^*\hom(E,F))$ by putting
$$
\langle\psi,\tilde f\rangle
:=\int_{|\xi|\geq1}\psi f
+\int_{|\xi|\leq1}\left(\psi-\sum_{i=n}^N\psi_i\right)f
+\sum_{\substack{n\leq i\leq N\\i+s\neq0}}(i+s)^{-1}\mu(\psi_if)
$$
for every test section $\psi\in\mathcal D(\pi^*\hom(E,F))=\Gamma^\infty_c((\pi^*\hom(E,F))^*\otimes|\Lambda|_{\mathfrak t^*M})$, see \cite[Section~3]{CGGP92} or \cite[Proof of Proposition~6.15]{FS74}.
Here $|-|$ is a fiber wise homogeneous norm on $\mathfrak t^*M$, i.e., $|\dot\delta'_\lambda\xi|=\lambda|\xi|$ for all $\xi\in\mathfrak t^*M$ and $\lambda>0$;
$N$ is any integer such that $N+1+\Re(s)>0$;
and $\psi_i\in\Gamma^\infty((\pi^*\hom(E,F))^*\otimes|\Lambda|_{\mathfrak t^*M})$ are homogeneous (fiber wise Taylor polynomials) such that $(\dot\delta'_\lambda)^*\psi_i=\lambda^i\psi_i$ for all $\lambda>0$ and $(\dot\delta'_\lambda)^*\bigl(\psi-\sum_{i=n}^N\psi_i\bigr)=O(\lambda^{N+1})$ as $\lambda\to0$.
Moreover, for every $g\in\Gamma^\infty(|\Lambda|_{\mathfrak t^*M\setminus M})$ which is homogeneous of degree $z$, i.e., $(\dot\delta'_\lambda)^*g=\lambda^zg$ for all $\lambda>0$, and such that $\pi(\supp(g))$ has compact closure, $\mu(g)$ denotes the unique number such that, for all $0<r<R$, we have
$$
\int_{r\leq|\xi|\leq R}g
=\begin{cases}\mu(g)z^{-1}(R^z-r^z)&\textrm{if $z\neq0$, and}\\\mu(g)\log(R/r)&\textrm{if $z=0$.}\end{cases}
$$ 
If $-s-n\notin\mathbb N_0$, then $\tilde f$ is the unique distributional extension of $f$, and $(\dot\delta'_\lambda)^*\tilde f=\lambda^s\tilde f$, for all $\lambda>0$.
If $-s-n\in\mathbb N_0$, then $\langle\psi,\Delta\rangle:=\mu(\psi_{-s}f)$ defines a distribution $\Delta\in\Gamma^{-\infty}(\mathfrak t^*M;\pi^*\hom(E,F))$ which is supported on $M\subseteq\mathfrak t^*M$, and homogeneous of order $s$, that is $(\dot\delta_\lambda')^*\Delta=\lambda^s\Delta$ for all $\lambda>0$.
Moreover, $(\dot\delta_\lambda')^*\tilde f=\lambda^s\tilde f+\lambda^s\log(\lambda)\Delta$, for all $\lambda>0$.


Let $l\in\mathcal K(\mathcal TM;E,F)$ be the unique fiber wise tempered distribution with Fourier transform $\hat l=\tilde f$.
In view of \eqref{E:QWERTY}, we have $k-l\in\mathcal K^\infty(\mathcal TM;E,F)$.
If $-s-n\notin\mathbb N_0$, then $(\delta_\lambda)_*l=\lambda^sl$, for all $\lambda>0$.
If $-s-n\in\mathbb N_0$, we let $p\in\mathcal P^s(\mathcal TM;E,F)$ denote the unique fiber wise tempered distribution with Fourier transform $\hat p=\Delta$, and obtain $(\delta_\lambda)_*l=\lambda^sl+\lambda^s\log(\lambda)p$, for all $\lambda>0$.
In either case $l$ is homogeneous mod $\mathcal P^s(\mathcal TM;E,F)$.
To establish a decomposition as in \eqref{E:EYvsCGGP}, it remains to observe that $l-p\log|\exp^{-1}(-)|\in\mathcal K(\mathcal TM;E,F)$ is homogeneous of order $s$, for we have $(\delta_\lambda)_*(l-p\log|\exp^{-1}(-)|)=\lambda^s(l-p\log|\exp^{-1}(-)|)$ for all $\lambda>0$.


To complete the proof, it remains to observe that every smooth kernel $k\in\mathcal K^\infty(\mathcal TM;E,F)$ which is homogeneous mod $\mathcal P^s(\mathcal TM;E,F)$ is actually contained in $\mathcal P^s(\mathcal TM;E,F)$.
Indeed, if $\lambda^{-s}(\delta_\lambda)_*k=k+p_\lambda$ with $p_\lambda\in\mathcal P^s(\mathcal TM;E,F)$, then the fiber wise Taylor series of $f$ along $M\subseteq\mathcal TM$ contains only terms which are homogeneous of order $s$.
Hence, $\lambda^{-s}(\delta_\lambda)_*k$ converges to an element in $\mathcal P^s(\mathcal TM;E,F)$ as $\lambda\to0$, and, thus, $k\in\mathcal P^s(\mathcal TM;E,F)$.
\end{proof}


\begin{proposition}\label{P:Psimap}
Let $E$ and $F$ be vector bundles over a filtered manifold $M$ of homogeneous dimension $n$, see \eqref{E:hdim}.
Consider $A\in\Psi^s(E,F)$ where $s$ is some complex number, and let $k\in\Gamma^{-\infty}(F\boxtimes E')$ denote the corresponding Schwartz kernel.
Then the following hold true:
\begin{enumerate}[(a)]
\item\label{P:Psimap:L2cont}
If $\Re s\leq0$, then $A$ induces a continuous operator $A\colon L^2_c(E)\to L^2_\loc(F)$.
If, moreover, $A$ is properly supported, then we obtain continuous operators $A\colon L^2_c(E)\to L^2_c(F)$ and $A\colon L^2_\loc(E)\to L^2_\loc(F)$.
For closed $M$ we have a bounded operator $A\colon L^2(E)\to L^2(F)$.
\item\label{P:Psimap:compact}
If $\Re s<0$, then $A$ induces a compact operator $A\colon L^2_c(E)\to L^2_\loc(F)$.
If, moreover, $A$ is properly supported, then it induces compact operators $A\colon L^2_\loc(E)\to L^2_\loc(F)$ and $A\colon L^2_c(E)\to L_c^2(F)$.
For closed $M$ we obtain a compact operator $A\colon L^2(E)\to L^2(F)$.
\item\label{P:Psimap:kL2}
If $\Re s<-n/2$, then $A$ induces a continuous operator $A\colon L^2_c(E)\to\Gamma(F)$.
If $A$ is properly supported, then it induces continuous operators $A\colon L^2_\loc(E)\to\Gamma(F)$ and $A\colon L^2_c(E)\to\Gamma_c(F)$.
For closed $M$ we obtain a continuous operator $A\colon L^2(E)\to\Gamma(F)$.
\footnote{Recall that $\Gamma(F)$ and $\Gamma_c(F)$ denote the standard spaces of (compactly supported) continuous sections.}
\item\label{P:Psimap:cont}
If $\Re s<-n$, then the kernel $k$ is continuous.
If, moreover, $M$ is closed and $E=F$, then $A\colon L^2(E)\to L^2(E)$ is trace class and
$$
\tr_{L^2(E)}(A)=\int_M\tr_E(\iota^*k)=\int_{x\in M}\tr_{E_x}(k(x,x))
$$
where $\iota^*k\in\Gamma^\infty(\eend(E)\otimes|\Lambda|_M)$ denotes the restriction of the kernel to the diagonal.
\end{enumerate}
\end{proposition}


\begin{proof}
To show \itemref{P:Psimap:L2cont}, suppose $\Re s\leq0$.
The symbol estimate in \cite[Corollary~67]{EY15} implies that the full symbol of $A$, i.e.\ the fiber wise Fourier transform of the full cosymbol, with respect to exponential coordinates as in Remark~\ref{R:expcoor}, is in the standard class $\mathcal S^{s/m}_{1/m,0}$ where $m$ is such that $\mathfrak tM=\bigoplus_{j=1}^m\mathfrak t^{-j}M$, cf.\ \cite[Proposition~10.22]{BG88}.
This symbol class is not invariant under general coordinate change.
Nevertheless, boundedness on $L^2$ is a well known consequence, see for instance \cite[Theorem~6.1]{S01}.


To show \itemref{P:Psimap:compact} suppose $\Re s<0$.
Using Remark~\ref{R:expcoor} and Lemma~\ref{L:EYvsCGGP} we see that the kernel provides a family $k(x,-)\in L^1_\loc(F_x\otimes E')$, smoothly parametrized by $x\in M$; and the same holds true for the transposed kernel, see Proposition~\ref{P:Psi}\itemref{P:Psi:trans}.
In particular, given two compact subsets $K$ and $L$ of $M$, there exists a constant $C\geq0$ such that $\sup_{x\in L}\int_K|k(x,y)|dy\leq C$ and $\sup_{y\in K}\int_L|k(x,y)|dx\leq C$.
Hence, according to Schur's lemma, see \cite[Lemma~9.1]{S01} or \cite[Lemma~15.2]{FS74} for instance, the operator norm of the composition
\begin{equation}\label{E:AKL}
L^2_K(E)\subseteq L^2_c(E)\xrightarrow{A}L^2_\loc(F)\to L^2_L(F)
\end{equation}
is bounded by $C$, i.e., $\|A\psi\|_{L^2_L(F)}\leq C\|\psi\|_{L^2_K(E)}$ for all $\psi\in L^2_K(E)$.
Writing $k=\chi k+(1-\chi)k$, where $\chi\in C^\infty(M\times M,[0,1])$ and $\chi\equiv1$ in a neighborhood of the diagonal, we obtain a decomposition $A=A'+R$ where $R$ is a smoothing operator with kernel $(1-\chi)k$ and $A'\in\Psi^s(E,F)$ has kernel $\chi k$.
Given $\varepsilon>0$, we may choose $\chi$ such that $\sup_{x\in L}\int_K|\chi k(x,y)|dy\leq\varepsilon$ and $\sup_{y\in K}\int_L|\chi k(x,y)|dx\leq\varepsilon$ and, consequently, $\|A'\psi\|_{L^2_L(F)}\leq\varepsilon\|\psi\|_{L^2_K(E)}$.
We conclude that the composition in \eqref{E:AKL} can be approximated by smoothing operators.
Since the latter are compact, we conclude that the composition in \eqref{E:AKL} is  compact too.


To show \itemref{P:Psimap:kL2} we suppose $\Re s<-n/2$.
Using Remark~\ref{R:expcoor} and Lemma~\ref{L:EYvsCGGP} we see that the kernel provides a smooth family $k(x,-)\in L^2_\loc(F_x\otimes E')$, parametrized by $x\in M$.
In particular, given two compact subsets $K$ and $L$ of $M$, there exists a constant $C\geq0$ such that $\sup_{x\in L}\int_K|k(x,y)|^2dy\leq C^2$.
Using the Cauchy--Schwarz inequality, we obtain $\sup_{x\in L}|(A\psi)(x)|\leq C\|\psi\|_{L^2_K(E)}$ for all $\psi\in L^2_K(E)$.
Hence, $A$ maps $L^2_c(E)$ continuously into $\Gamma(F)$.
The remaining statements are now obvious.


To show \itemref{P:Psimap:cont}, we suppose $\Re s<-n$.
Using Remark~\ref{R:expcoor} and Lemma~\ref{L:EYvsCGGP} we see that the kernel provides a smooth family $k(x,-)\in\Gamma(F_x\otimes E')$, parametrized by $x\in M$.
Clearly, this implies that $k$ is continuous.
The remaining assertions are now obvious.
\end{proof}





\subsection{Parametrices and Rockland condition}\label{SS:para}




In this section we will establish a Rockland \cite{R78} type result characterizing (left) invertible cosymbols and the existence of (left) parametrices in terms of irreducible unitary representations of the osculating groups, see Theorem~\ref{T:Rockland} below.
This generalizes \cite[Theorem~2.5(d)]{CGGP92} in two ways: (1) The geometry need not be flat, that is, the filtered manifold need not by locally diffeomorphic to the left invariant filtration on a graded nilpotent Lie group; and (2) we treat operators acting on sections of vector bundles.
For Heisenberg manifolds this can be found in \cite[Theorems~3.3.10 and 5.4.1]{P08}.


For every $x\in M$ we let
$$
\Sigma_x^s(E,F):=
\left\{k\in\frac{\mathcal K(\mathcal T_xM;E_x,F_x)}{\mathcal K^\infty(\mathcal T_xM;E_x,F_x)}:\textrm{$(\delta_\lambda)_*k=\lambda^sk$ for all $\lambda>0$}\right\}
$$
denote the space principal cosymbols of order $s$ at $x$.
Restriction provides a linear map $\ev_x\colon\Sigma^s(E,F)\to\Sigma^s_x(E,F)$ which is compatible with convolution and transposition.
Composing this with the principal symbol map, we obtain 
$$
\sigma^s_x\colon\Psi^s(E,F)\to\Sigma^s_x(E,F).
$$
We will refer to $\sigma^s_x(A)$ as the principal cosymbol of $A\in\Psi^s(E,F)$ at $x$.


Following \cite{CGGP92} and \cite[Section~3.3.2]{P08}, we will now formulate a (matrix) Rockland type condition for cosymbols in $\Sigma_x^s(E,F)$.
Let $\mathcal P(\mathcal T_xM)=\bigoplus_{j=0}^\infty\mathcal P^{-n-j}(\mathcal T_xM)$ denote the space of polynomial volume densities on $\mathcal T_xM$, where $n$ denotes the homogeneous dimension of $M$, see~\eqref{E:hdim}.
Recall that $\mathcal P(\mathcal T_xM)$ is invariant under translation and inversion.
For a finite dimensional vector space $E_0$, we let $\mathcal S_0(\mathcal T_xM;E_0)$ denote the subspace of all $f$ in the $E_0$-valued Schwartz space $\mathcal S(\mathcal T_xM;E_0)$ such that $\int_{\mathcal T_xM}pf=0$ for all $p\in\mathcal P(\mathcal T_xM)$.


In view of Lemma~\ref{L:EYvsCGGP}, every cosymbol $a\in\Sigma_x^s(E,F)$ can be represented in the form
\begin{equation}\label{E:akplog}
a=k+p\log|\exp^{-1}(-)|
\end{equation}
where $k\in\mathcal K(\mathcal T_xM;E_x,F_x)$ is homogeneous of order $s$, that is $(\delta_\lambda)_*k=\lambda^sk$ for all $\lambda>0$, and $p\in\mathcal P^s(\mathcal T_xM;E_x,F_x)$.
If $-s-n\notin\mathbb N_0$, then $p=0$ and $k$ is unique.
If $-s-n\in\mathbb N_0$, then $p$ is unique but $k$ comes with an ambiguity in $\mathcal P^s(\mathcal T_xM;E_x,F_x)$.
Hence, the restriction of the left regular representation,
\begin{equation}\label{E:aS0}
\mathcal S_0(\mathcal T_xM;E_x)\to\mathcal S_0(\mathcal T_xM;F_x),\qquad f\mapsto a*f,
\end{equation}
does not depend on the representative for $a\in\Sigma^s_x(E,F)$, provided it is of the form \eqref{E:akplog}.


Suppose $\pi\colon\mathcal T_xM\to U(\mathcal H)$ is a non-trivial irreducible unitary representation of the osculating group $\mathcal T_xM$ on a Hilbert space $\mathcal H$.
Let $\mathcal H_0$ denote the subspace of $\mathcal H$ spanned by elements of the form $\pi(fdg)v$ where $v\in\mathcal H$, $f\in\mathcal S_0(\mathcal T_xM)$, $dg$ denotes a (left) invariant volume density on $\mathcal T_xM$, and $\pi(fdg):=\int_{\mathcal T_xM}\pi(g)f(g)dg\in\mathcal K(\mathcal H)$.
Since $\mathcal H_0$ non-trivial and invariant under $\pi(g)$ for all $g\in\mathcal T_xM$, the subspace $\mathcal H_0$ is dense in $\mathcal H$.
Note that $\mathcal H_0\otimes E_x$ is spanned by vectors of the form $\pi(fdg)v$ where $v\in\mathcal H$, $f\in\mathcal S_0(\mathcal T_xM;E_x)$, and $\pi(fdg):=\int_{\mathcal T_xM}\pi(g)f(g)dg\in\mathcal K(\mathcal H,\mathcal H\otimes E_x)$.


Still assuming a representative of the form \eqref{E:akplog}, we define an unbounded operator $\pi(a)$ from $\mathcal H\otimes E_x$ to $\mathcal H\otimes F_x$ by
$$
\pi(a)\colon\mathcal H_0\otimes E_x\to\mathcal H_0\otimes F_x,\qquad\pi(a)\pi(fdg)v:=\pi(a*fdg)v,
$$ 
for all $f\in\mathcal S_0(\mathcal T_xM;E_x)$ and $v\in\mathcal H$, cf.~\eqref{E:aS0}.
As explained in \cite[Section~2]{CGGP92}, this definition of $\pi(a)$ is unambiguous.
Moreover, $\pi(a)$ is closeable, for $\pi(a^*)$ is a densely defined adjoint.
We denote its closure by $\bar\pi(a)$.
It is well know, see \cite{CGGP92}, \cite{G91}, or \cite[Proposition~3.3.6]{P08}, that the domain of definition of $\bar\pi(a)$ contains the space of smooth vectors, $\mathcal H_\infty\otimes E_x$, and this subspace is mapped into $\mathcal H_\infty\otimes F_x$ by $\bar\pi(a)$.
Furthermore, on $\mathcal H_\infty\otimes E_x$, we have
\begin{equation}\label{E:piab}
\bar\pi(ba)=\bar\pi(b)\bar\pi(a)
\end{equation}
whenever $a\in\Sigma^s_x(E,F)$ and $b\in\Sigma^{\tilde s}_x(F,G)$.
Moreover, with respect to inner products on $E_x$ and $F_x$ and the associated inner products on $\mathcal H\otimes E_x$ and $\mathcal H\otimes F_x$, we have 
\begin{equation}\label{E:pia*}
\bar\pi(a)^*=\bar\pi(a^*)
\end{equation}
on $\mathcal H_\infty\otimes F_x$.
If $a$ is the cosymbol of a differential operator then, on $\mathcal H_\infty\otimes E_x$, the operator $\bar\pi(a)$ coincides with $\pi(a)$ considered in Section~\ref{SS:paraDO}, see \cite[Remark~3.3.7]{P08}, and thus the following definition is consistent with Definition~\ref{D:rockland}.


\begin{definition}[Matrix Rockland condition]\label{D:Rockland}
A principal cosymbol $a\in\Sigma_x^s(E,F)$ at $x\in M$ is said to satisfy the \emph{Rockland condition} if, for every non-trivial irreducible unitary representation $\pi\colon\mathcal T_xM\to U(\mathcal H)$, the unbounded operator $\bar\pi(a)$ is injective on $\mathcal H_\infty\otimes E_x$. 
An operator $A\in\Psi^s(E,F)$ is said to satisfy the Rockland condition if its principal cosymbol, $\sigma^s_x(A)\in\Sigma^s_x(E,F)$, satisfies the Rockland condition at each point $x\in M$.
\end{definition}
  

We will need the following matrix version of a Rockland \cite{R78} type theorem due to Christ, Geller, G{\l}owacki, and Polin, see \cite[Theorem~6.2]{CGGP92} or \cite{G91} for the order zero case.
For differential operators such a statement can be found in van~Erp's thesis \cite{E05}.


\begin{lemma}[Christ--Geller--G{\l}owacki--Polin]\label{L:Rockland}
A principal cosymbol $a\in\Sigma_x^s(E,F)$ at $x\in M$ satisfies the Rockland condition iff it admits a left inverse $b\in\Sigma_x^{-s}(F,E)$, that is, $ba=1$.
\end{lemma}


\begin{proof}
The necessity of the Rockland condition is obvious, see \eqref{E:piab}.
To prove the non-trivial implication, we will present an elementary argument, reducing the statement to the well known scalar case due to Christ--Geller--G{\l}owacki--Polin, see \cite[Theorem~6.2]{CGGP92}.
For $s=0$ this has been proved in \cite{G91}.


With respect to bases of $E_x$ and $F_x$, the cosymbol $a\in\Sigma^s_x(E,F)$ corresponds to a matrix $A$ with entries in $\Sigma^s_x$, the space of principal cosymbols of order $s$ at $x$ for scalar operators.
Moreover, $\bar\pi(a)$ corresponds to the matrix $\bar\pi(A)$ obtained by applying the representation $\pi$ to each entry of $A$, that is, $(\bar\pi(A))_{ij}=\bar\pi(A_{ij})$.
Hence, $a$ satisfies the Rockland condition in Definition~\ref{D:Rockland} iff the matrix $\bar\pi(A)$ acts injectively on $(\mathcal H_\infty)^m$.
Here $m:=\dim(E_x)$ is the number of columns of $A$.
Using induction on $m$, we will show that there exists a matrix $B$ with entries in $\Sigma^{-s}_x$ such that $BA=I_m$ where $I_m$ denotes the $m\times m$ unit matrix.


Clearly, $A^*A$ is an $(m\times m)$-matrix with entries in $\Sigma^{2s}_x$ which satisfies the Rockland condition, see \eqref{E:piab} and \eqref{E:pia*}.
Let $y:=\sum_j(A_{j1})^*A_{j1}\in\Sigma^{2s}_x$ denote the entry in the upper left corner of $A^*A$.
Clearly, $y$ satisfies the (scalar) Rockland condition.
Hence, according to \cite[Theorem~6.2]{CGGP92}, there exists $z\in\Sigma^{-2s}_x$ such that $zy=1$.
Since $y^*=y$, we also have $yz=1$, whence $z$ is invertible.
Hence the diagonal matrix $D:=zI_m$ is invertible and, thus, $DA^*A$ satisfies the Rockland condition. 
The matrix $DA^*A$ has entries in $\Sigma^0_x$ and, by construction, its entry in the upper left corner is 1. 
Performing elementary row operations, we find an invertible matrix $L$ with entries in $\Sigma^0_x$ such that
$$
LDA^*A=\begin{pmatrix}1&*\\0&Y\end{pmatrix}
$$
where $Y$ is an $(m-1)\times(m-1)$-matrix with entries in $\Sigma^0_x$. 
Since $L$ is invertible, $LDA^*A$ satisfies the Rockland condition and, thus, $Y$ satisfies the Rockland condition too. 
By induction, there exists an $(m-1)\times(m-1)$-matrix $Z$ with entries in $\Sigma^0_x$ such that $ZY=I_{m-1}$. 
The matrix $C:=\begin{pmatrix}1&0\\0&Z\end{pmatrix}$ has entries in $\Sigma^0_x$ and, by construction, 
$$
CLDA^*A=\begin{pmatrix}1&*\\0&I_{m-1}\end{pmatrix}.
$$
Performing further elementary row operations, we find an invertible matrix $U$ with entries in $\Sigma^0_x$ such that $UCLDA^*A=I_m$. 
Hence, the matrix $B:=UCLDA^*$ has entries in $\Sigma^{-s}_x$ and satisfies $BA=I_m$.
\end{proof}


\begin{lemma}\label{L:pointwiseinv}
Let $E$ and $F$ be two vector bundles over a filtered manifold $M$.
Consider $a\in\Sigma^s(E,F)$ and suppose $\ev_{x_0}(a)$ is left invertible at some point $x_0\in M$, that is, there exists $b_{x_0}\in\Sigma_{x_0}^{-s}(F,E)$ such that $b_{x_0}\ev_{x_0}(a)=1$.
Then there exists an open neighborhood $U$ of $x_0$ and $b\in\Sigma^{-s}(F|_U,E|_U)$ such that $ba|_U=1$.
Moreover, a similar statement involving right inverses holds true.
\end{lemma}


\begin{proof}
If the bundle of osculating algebras is locally trivial, then this follows from \cite{CGGP92}, at least in the scalar case.
To handle general bundles of osculating algebras, we will proceed as in \cite[Section~3.3.3]{P08} where Ponge considers (in general non-contact) Heisenberg manifolds with varying osculating algebras.


Choose $\tilde b\in\Sigma^{-s}(F,E)$ such that $\ev_{x_0}(\tilde b)=b_{x_0}$.
Putting $c:=\tilde ba$, we have $c\in\Sigma^0(E)$ and $\ev_{x_0}(c)=1$.
It suffices to find an open neighborhood $U$ of $x_0$ in $M$ such that $c|_U$ is invertible in $\Sigma^0(E|_U)$, for then $b:=c|_U^{-1}\tilde b$ is the desired local left inverse of $a$.

We will identify $\Sigma^0(E)=\hat\Sigma^0(E)$ where $\hat\Sigma^0(E)$ denotes the space of all $k\in\mathcal K(\mathcal TM;E,E)$ which are strictly homogeneous of order zero, that is, $(\delta_\lambda)_*k=k$ for all $\lambda>0$, see Lemma~\ref{L:EYvsCGGP}.
Convolution with $k\in\hat\Sigma^0_x(E)$ gives rise to a bounded operator on $L^2(\mathcal T_xM)\otimes E_x$, see \cite[Theorem~6.19]{FS82}.
We fix an auxiliary fiber wise Hermitian inner product on $E$, as well as a smooth family of invariant volume densities on the osculating groups $\mathcal T_xM$, and let $\bbb k\bbb_x$ denote the operator norm with respect to the associated Hermitian inner product on $L^2(\mathcal T_xM)\otimes E_x$.


We fix a fiber wise homogeneous norm $|-|$ on $\mathcal TM$ which is smooth on $\mathcal TM\setminus M$.
This permits to decompose each $k\in\hat\Sigma^0_x(E)$ uniquely in the form $k=c_x(k)\delta_x+\pv_x(k)$ where $c_x(k)\in\eend(E_x)$, and $\pv_x(k)\in\hat\Sigma^0_x(E)$ is the principal value distribution
$$
\langle\pv_x(k),\psi\rangle:=\lim_{\varepsilon\to0}\int_{\{g\in\mathcal T_xM:|g|\geq\varepsilon\}}k\psi,
$$
see \cite[Proposition~6.13]{FS82}.
For each $r\in\N_0$ we consider the norm
$$
\|k\|_{x,r}:=|c_x(k)|+\left(\int_{\mathcal S_xM}|j^r(k|_{\mathcal S_xM})|^2\mu_x\right)^{1/2},
$$
where $k\in\hat\Sigma^0_x(E)$.
Here $j^r(k|_{\mathcal S_xM})$ denotes the $r$-jet of the restriction of $k$ to the sphere $\mathcal S_xM:=\{g\in\mathcal T_xM:|g|=1\}$, and we use smooth fiber wise Hermitian metric on the $r$-jet bundle $J^r(\mathcal S_xM,E_x)$ which depends smoothly on $x$, and we are using a smooth volume density $\mu_x$ on the sphere $\mathcal S_xM$ which depends smoothly on $x$.


There exists $r_0\in\N_0$ and constants $C_x\geq0$ such that $\bbb k\bbb_x\leq C_x\|k\|_{x,r_0}$ for all $k\in\hat\Sigma^0_x(E)$.
This follows from a result due to Folland and Stein, see \cite[Theorem~6.19]{FS82}, and the Sobolev embedding theorem.
Ponge observed, see \cite[Lemma~3.3.13]{P08}, that the proof of Folland and Stein allows to choose the constants $C_x$ uniformly over compact subsets of $M$.
To make this more precise, we put, for every compact subset $L$ of $M$ and each $k\in\hat\Sigma^0(E)$,
$$
\bbb k\bbb_L:=\sup_{x\in L}\bbb\ev_x(k)\bbb_x\qquad\text{and}\qquad \|k\|_{L,r}:=\sup_{x\in L}\|\ev_x(k)\|_{x,r}.
$$
Then there exists a constant $C_L\geq0$ such that
\begin{equation}\label{E:bbb}
\bbb k\bbb_L\leq C_L\|k\|_{L,r_0}
\end{equation}
holds for all $k\in\hat\Sigma^0(E)$.\footnote{Actually, $r_0=0$ appears to be sufficient, see \cite{C88} and \cite[Remark~3.3.14]{P08}, but we won't need that.}
Moreover, see \cite[Lemma~3.3.15]{P08} and \cite[Lemma~5.7]{CGGP92}, for every $r\geq r_0$ there exists a constant $C_{L,r}\geq0$ such that 
$$
\|k_2*k_1\|_{L,r}\leq C_{L,r}\Bigl(\|k_2\|_{L,r}\cdot\bbb k_1\bbb_L+\bbb k_2\bbb_L\cdot\|k_1\|_{L,r}\Bigr)
$$
holds for all $k_1,k_2\in\hat\Sigma^0(E)$.
As in \cite[Lemma~4]{C88}, this gives 
$$
\|k^{2i}\|^{1/2i}_{L,r}\leq(2C_{L,r})^{1/2i}\sqrt{\bbb k^i\bbb^{1/i}_L}\sqrt{\|k^i\|_{L,r}^{1/i}}\leq(2C_{L,r})^{1/2i}\sqrt{\bbb k\bbb}_L\sqrt{\|k^i\|_{L,r}^{1/i}}
$$ 
and passing to the limit, we obtain
\begin{equation}\label{E:geom}
\lim_{i\to\infty}\|k^i\|^{1/i}_{L,r}\leq\bbb k\bbb_L
\end{equation}
for all $r\geq r_0$ and all $k\in\hat\Sigma^0(E)$.


To invert $c$ we write $c=1-k$ where $k\in\hat\Sigma^0(E)$.
Then $\ev_{x_0}(k)=0$, and there exists a compact neighborhood $L$ of $x_0$ such that $\bbb k\bbb_L<1$, see \eqref{E:bbb}.
In view of \eqref{E:geom}, the Neumann series $\sum_{i=0}^\infty k^i$ converges with respect to the norm $\|-\|_{L,r}$ for all $r\geq r_0$.
For each $x\in L$ we conclude that $\ev_x(c)$ is invertible in $\hat\Sigma^0_x(E)$ with inverse $\ev_x(c)^{-1}=\sum_{i=0}^\infty\ev_x(k)^i$.
This also shows that $\ev_x(c)^{-1}$ depends continuously on $x\in L$.
Proceeding as in the proof of \cite[Proposition~3.3.11]{P08}, we see that $\ev_x(c)^{-1}$ actually depends smoothly on $x$ in the interior $U$ of $L$, see also \cite[Proposition~5.10]{CGGP92}.
Hence, $c|_U$ is invertible in $\hat\Sigma^0(E|_U)$ with inverse $c|_U^{-1}=\sum_{i=0}^\infty k|_U^i$.
\end{proof}


Combining these results, we obtain the following Rockland \cite{R78} type theorem generalizing \cite[Theorem~2.5(d)]{CGGP92}, see also \cite[Theorem~0.1]{HN79} and \cite[Theorem~3.3.10 and 5.4.1]{P08}.


\begin{theorem}\label{T:Rockland}
Let $E$ and $F$ be vector bundles over a filtered manifold $M$. 
Let $s$ be a complex number, and suppose $A\in\Psi^s(E,F)$ satisfies the Rockland condition.
Then there exists a left parametrix $B\in\Psi^{-s}_\prop(F,E)$ such that $BA-\id$ is a smoothing operator.
In particular, $A$ is hypoelliptic, that is, if $\psi$ is a compactly supported distributional section of $E$ and $A\psi$ is smooth, then $\psi$ was smooth.
If, moreover, $M$ is closed, then $\ker(A)$ is a finite dimensional subspace of\/ $\Gamma^\infty(E)$.
\end{theorem}


\begin{proof}
According to Lemma~\ref{L:Rockland} the principal cosymbol $\sigma^s_x(A)$ admits a left inverse at each point $x\in M$.
Hence, in view of Lemma~\ref{L:pointwiseinv}, we see that the principal cosymbol $\sigma^s(A)$ locally admits left inverses.
Using a smooth partition of unity on $M$, we obtain a global left inverse $b\in\Sigma^{-s}(F,E)$ such that $b\sigma^s(A)=1$.
Applying Proposition~\ref{P:Psi}\itemref{P:Psi:parametrix}, we obtain $B\in\Psi^{-s}_\prop(F,E)$ such that $BA-\id$ is a smoothing operator.
The remaining assertions are now obvious, cf.\ the proof of Corollary~\ref{C:hypo} above.
\end{proof}


Note that Theorem~\ref{T:para} follows immediately from Theorem~\ref{T:Rockland}.


\begin{corollary}\label{C:PsiinvA}
Let $E$ be a vector bundle over a closed filtered manifold $M$.
Suppose $A\in\Psi^s(E)$ satisfies the Rockland condition and is formally selfadjoint, $A^*=A$, with respect to an $L^2$ inner product of the form \eqref{E:llrr}, and let $Q$ denote the orthogonal projection onto the (finite dimensional) subspace $\ker(A)\subseteq\Gamma^\infty(E)$.
Then $A+Q$ is invertible with inverse $(A+Q)^{-1}\in\Psi^{-s}(E)$.
\end{corollary}


\begin{proof}
Proceeding exactly as in the proof of Corollary~\ref{C:smooth-Hodge-decomposition}, we start with a left parametrix $B\in\Psi^{-s}(E)$, see Theorem~\ref{T:Rockland}, and observe that the injective and formally selfadjoint parametrix $P$ constructed there is contained in $\Psi^{-s}(E)$.
As explained in the proof of Corollary~\ref{C:smooth-Hodge-decomposition}, $G:=(A+Q)P\in\Psi^0(E)$ is invertible, and $G^{-1}-\id$ is a smoothing operator.
Clearly, this implies $G^{-1}\in\Psi^0(E)$, whence $(A+Q)^{-1}=PG^{-1}\in\Psi^{-s}(E)$.
\end{proof}


\begin{corollary}[Pseudo inverse]\label{C:pseudoinverse}
Let $E$ and $F$ be smooth vector bundles over a closed filtered manifold $M$, consider $A\in\Psi^s(E,F)$, and let $A^*\in\DO^{\bar s}(F,E)$ denote the formal adjoint with respect to $L^2$ inner products of the form \eqref{E:llrr}.
Assume $A$ and $A^*$ both satisfy the Rockland condition.
Moreover, let $Q$ denote the orthogonal projection onto the (finite dimensional) subspace $\ker(A)\subseteq\Gamma^\infty(E)$, and let $P$ denote the orthogonal projection onto the (finite dimensional) subspace $\ker(A^*)\subseteq\Gamma^\infty(F)$.
Then the pseudo inverse, 
$$
A^+:=(A^*A+Q)^{-1}A^*=A^*(AA^*+P)^{-1},
$$ 
uniquely characterized by the relations 
$$
AA^+=\id-P,\qquad A^+A=\id-Q,\qquad\text{and}\qquad A^+P=0=QA^+,
$$ 
satisfies $A^+\in\Psi^{-s}(F,E)$.
\end{corollary}


Combining these observations with a result from \cite{CGGP92} permits to construct operators of arbitrary order which are invertible up to smoothing operators.
More precisely, we have the following:


\begin{lemma}\label{L:Lambda}
Let $E$ be a vector bundle over a filtered manifold $M$.
Then, for every complex number $s$, there exist $\Lambda\in\Psi^s_\prop(E)$ and $\Lambda'\in\Psi^{-s}_\prop(E)$ 
such that $\Lambda\Lambda'-\id$ and $\Lambda'\Lambda-\id$ are both smoothing operators.
Moreover, $\Lambda$ and $\Lambda'$ may be chosen so that they act injectively on $\Gamma^{-\infty}_c(E)$.
On a closed manifold there even exist $\Lambda\in\Psi^s(E)$ and $\Lambda'\in\Psi^{-s}(E)$ such that $\Lambda\Lambda'=\id=\Lambda'\Lambda$.
\end{lemma}


\begin{proof}
For each $x\in M$ we fix an invertible cosymbol $a_x\in\Sigma^{s/4}_x(E)$ at $x$, see \cite[Theorem~6.1]{CGGP92}.
We extend these to cosymbols $\tilde a_x\in\Sigma^{s/4}(E)$ such that $\ev_x(\tilde a_x)=a_x$.
By Lemma~\ref{L:pointwiseinv}, for each $x\in M$, there exists an open neighborhood $U_x$ of $x$ such that $\tilde a_x|_{U_x}$ is invertible in $\Sigma^{s/4}(E|_{U_x})$.
Fix a smooth partition of unity $\lambda_x$, $x\in M$, such that $\supp(\lambda_x)\subseteq U_x$ for each $x\in M$.
With respect to a fiber wise Hermitian metric on $E$, we consider 
$$
a:=\sum_{x\in M}\lambda_x\tilde a_x^*\tilde a_x\in\Sigma^{s/2}(E).
$$
We claim that $a$ is an invertible cosymbol.
To see this note first that, in view of Lemma~\ref{L:pointwiseinv}, it suffices to show that $\ev_y(a)$ admits an inverse in $\Sigma^{-s/2}_y(E)$ for each $y\in M$.
For fixed $y\in M$, there exists $x\in M$ such that $\lambda_x(y)>0$ and, by construction, $\ev_y(\tilde a_x)$ is invertible.
Using \eqref{E:piab} and \eqref{E:pia*} we conclude that $\ev_y(a)$ satisfies the Rockland condition and, thus, admits a left inverse in $\Sigma^{-s/2}_y(E)$, see Lemma~\ref{L:Rockland}.
Actually, $\ev_y(a)$ is invertible since $\ev_y(a)=\ev_y(a)^*$.
This shows that $a$ is invertible, hence there exists $a'\in\Sigma^{-s/2}(E)$ such that $aa'=1=a'a$.


According to Proposition~\ref{P:Psi}\itemref{P:Psi:symbsequ} there exist $A\in\Psi^{s/2}_\prop(E)$ such that $\sigma^{s/2}(A)=a$.
Using Proposition~\ref{P:Psi}\itemref{P:Psi:parametrix} we obtain $A'\in\Psi^{-s/2}_\prop(E)$ such that $R:=A'A-\id$ is a smoothing operator.
Proposition~\ref{P:Psi}\itemref{P:Psi:parametrix} also gives $A''\in\Psi_\prop^{-s/2}(E)$ such that $AA''-\id$ is a smoothing operator.
Since $A''$ differs from $A'$ by a smoothing operator, $AA'-\id$ is a smoothing operator too.


Consider $\Lambda\in\Psi^s_\prop(E)$ and $\Lambda'\in\Psi^{-s}_\prop(E)$ defined by 
$$
\Lambda:=A^*A+R^*R\qquad\text{and}\qquad\Lambda':=A'(A')^*+RR^*,
$$ 
where the adjoints are with respect to an $L^2$ inner product of the form \eqref{E:llrr}.
One readily verifies that $\Lambda'\Lambda-\id$ and $\Lambda\Lambda'-\id$ are both smoothing operators.
In particular, $\Lambda$ is hypoelliptic and, thus, every distributional section in the kernel of $\Lambda$ has to be smooth.
Using~\eqref{E:A*}, we conclude, $\ker(\Lambda|_{\Gamma^{-\infty}_c(E)})\subseteq\ker(A)\cap\ker(R)\subseteq\ker(\id)=0$.
Analogously, $\ker(\Lambda'|_{\Gamma^{-\infty}_c(E)})\subseteq\ker((A')^*)\cap\ker(R^*)\subseteq\ker(\id)=0$ in view of $R^*=A^*(A')^*-\id$.
If the underlying manifold is closed, then $\Lambda^{-1}\in\Psi^{-s}(E)$ according to Corollary~\ref{C:PsiinvA}.
\end{proof}


\begin{remark}[Right parametrix]\label{R:rightpara}
If $A\in\Psi^s(E,F)$ and $A^t$ satisfies the Rockland condition, then there exists a right parametrix $B\in\Psi_\prop^{-s}(F,E)$ such that $AB-\id$ is a smoothing operator.
Indeed, by Theorem~\ref{T:Rockland} there exists $B'\in\Psi_\prop^{-s}(E',F')$ such that $B'A^t-\id$ is a smoothing operator.
Hence, $B:=(B')^t$ is the desired right parametrix for $A$.
\end{remark}


\begin{remark}\label{R:AAtRock}
If $A\in\Psi^s(E,F)$ is such that $A$ and $A^t$ both satisfies the Rockland condition, then there exists a parametrix $B\in\Psi_\prop^{-s}(F,E)$ such that $AB-\id$ and $BA-\id$ are both smoothing operators.
This follows immediately from Theorem~\ref{T:Rockland} and Remark~\ref{R:rightpara} since every left parametrix differs from any right parametrix by a smoothing operator.
\end{remark}


\begin{remark}\label{R:AtRock}
For $A\in\Psi^s(E,F)$ and $x\in M$ the following are equivalent:
\begin{enumerate}[(a)]
\item $A^t$ satisfies the Rockland condition at $x$.
\item $A^*$ satisfies the Rockland condition at $x$.
\item $\bar\pi(\sigma_x^s(A))\colon\mathcal H_\infty\otimes E_x\to\mathcal H_\infty\otimes F_x$ has dense image, for all non-trivial irreducible unitary representations $\pi\colon\mathcal T_xM\to U(\mathcal H)$.
\item $\bar\pi(\sigma_x^s(A))\colon\mathcal H_\infty\otimes E_x\to\mathcal H_\infty\otimes F_x$ is onto, for all non-trivial irreducible unitary representations $\pi\colon\mathcal
 T_xM\to U(\mathcal H)$.
\end{enumerate}
Indeed, the equivalence (a)$\Leftrightarrow$(b) is clear in view of \eqref{E:A*At}.
The equivalence (b)$\Leftrightarrow$(c) follows from $\bar\pi(\sigma^s_x(A))^*=\bar\pi(\sigma^{\bar s}_x(A^*))$, see \eqref{E:pia*} and Remark~\ref{R:Psi:adjoint}.
To see the implication (a)$\Rightarrow$(d) suppose $A^t$ satisfies the Rockland condition at $x$.
According to Lemma~\ref{L:Rockland}, there exists $b\in\Sigma^{-s}_x(E',F')$ such that $b\sigma^s_x(A^t)=1$.
Transposing this equation, we obtain $\sigma^s_x(A)b^t=1$, see Proposition~\ref{P:Psi}\itemref{P:Psi:trans}.
In view of \eqref{E:piab}, this implies (d), for $\bar\pi(b^t)$ maps $\mathcal H_\infty\otimes F_x$ into $\mathcal H_\infty\otimes E_x$.
\end{remark}


\begin{remark}\label{R:CGGP}
In the flat case, that is to say, if the filtration on $M$ is locally diffeomorphic to the left invariant filtration on a graded nilpotent Lie group, the calculus described above coincides with the calculus of Christ, Geller, G{\l}owacki, and Polin \cite{CGGP92}, cf.\ Remark~\ref{R:expcoor}.
On Heisenberg manifolds it specializes to the Heisenberg calculus, see \cite{BG88,P08,T84}, which builds upon work of Boutet de Monvel \cite{B74}, Folland--Stein \cite{FS74} and Dynin \cite{D75,D76}, see also \cite{BGH75}, \cite{EM00}, \cite{G76}, \cite{H75}, and \cite{RS76}.
The equivalence with the Heisenberg calculus follows from \cite[Theorems~15.39 and 15.49]{BG88}, see also \cite[Proposition~3.1.15]{P08}, for the exponential coordinates used in Remark~\ref{R:expcoor} are clearly privileged coordinates in the sense of \cite[Definition~2.1.10]{P08}.
For trivially filtered manifolds, that is $TM=T^{-1}M$, we recover classical pseudodifferential operators, see \cite[Section~8]{EY15} and \cite{DS14}.
\end{remark}





\subsection{The Heisenberg Sobolev scale}\label{SS:sobolev}





The properties of the operator class $\Psi^s(E,F)$ discussed above permit to introduce a Heisenberg Sobolev scale on filtered manifolds which can be used to refine the hypoellipticity results in Section~\ref{SS:hesDO}, see Corollaries~\ref{C:regrockseq} and \ref{C:HsHodge-seq} at the end of this section.
The main properties of this Sobolev scale are summarized in Proposition~\ref{P:Hs} below, a refined regularity statement including maximal hypoelliptic estimates can be found in Corollary~\ref{C:reg}.


For non-degenerate CR manifolds the origins of this Sobolev scale can be traced back to a paper of Folland and Stein \cite{FS74} where $L^p$ Sobolev spaces for integral $s$ are constructed using differential operators, see \cite[Section~15]{FS74}, and maximal hypoelliptic estimates for Kohn's Laplacian are established, see \cite[Theorem~16.6]{FS74} and also \cite[Theorem~16.7]{FS74}.
For Heisenberg manifolds satisfying only the bracket generating condition $H+[H,H]=TM$, a full Sobolev scale has been constructed by Ponge using complex powers of subLaplacians, see \cite[Section~5.5]{P08} and \cite[Propositions~5.5.9 ad 5.5.14]{P08}.
Maximal hypoelliptic estimates can also be found in \cite{HN85} and the work of Beals and Greiner, see \cite[Theorem~18.31]{BG88}.



For every real $s$ we let $H^s_\loc(E)$ denote the space of all distributional sections $\psi\in\Gamma^{-\infty}(E)$ such that $A\psi\in L^2_\loc(F)$ for all $A\in\Psi^s_\prop(E,F)$ and all vector bundles $F$.
We equip $H^s_\loc(E)$ with the coarsest topology such that the maps $A\colon H^s_\loc(E)\to L^2_\loc(F)$ are continuous, for all $A\in\Psi_\prop^s(E,F)$.
Analogously, let $H^s_c(E)$ denote the space of compactly supported distributional sections $\psi\in\Gamma_c^{-\infty}(E)$ such that $A\psi\in L^2_\loc(F)$ for all $A\in\Psi^s(E,F)$, and equip $H^s_c(E)$ with the coarsest topology such that the corresponding maps $A\colon H^s_c(E)\to L^2_\loc(F)$ are continuous, for all $A\in\Psi^s(E,F)$ and all vector bundles $F$.


\begin{proposition}\label{P:Hs}
Let $E$ and $F$ be vector bundles over a filtered manifold $M$.
\begin{enumerate}[(a)]
\item\label{P:Hs:locfilt}
$H^s_\loc(E)$ is a complete locally convex vector space, and we have continuous inclusions
$$
\Gamma^\infty(E)\subseteq H^{s_2}_\loc(E)\subseteq H^{s_1}_\loc(E)\subseteq\Gamma^{-\infty}(E)
$$
whenever $s_1\leq s_2$.
Moreover, $H^0_\loc(E)=L^2_\loc(E)$ as locally convex spaces.

\item\label{P:Hs:cfilt}
$H^s_c(E)$ is a complete locally convex vector space, and we have continuous inclusions
$$
\Gamma^\infty_c(E)\subseteq H^{s_2}_c(E)\subseteq H^{s_1}_c(E)\subseteq\Gamma^{-\infty}_c(E)
$$
whenever $s_1\leq s_2$.
Moreover, $H^0_c(E)=L^2_c(E)$ as locally convex spaces.

\item\label{P:Hs:pairing}
The canonical pairing $\Gamma^\infty_c(E')\times\Gamma^\infty(E)\to\C$ extends to a pairing 
$$
H^{-s}_c(E')\times H^s_\loc(E)\to\C
$$ 
inducing linear bijections $H^s_\loc(E)^*=H^{-s}_c(E')$ and $H^{-s}_c(E')^*=H^s_\loc(E)$.
If, moreover, $M$ is closed, then $H^s_c(E)=H^s_\loc(E)$ is a Hilbert space we denote by $H^s(E)$, and the pairing induces an isomorphism of Hilbert spaces, $H^s(E)^*=H^{-s}(E')$.

\item\label{P:Hs:operators}
Each $A\in\Psi^r(E,F)$ restricts to continuous operator $A\colon H^s_c(E)\to H_\loc^{s-\Re(r)}(F)$.
Moreover, if $A$ is properly supported, then we obtain continuous operators $A\colon H^s_c(E)\to H_c^{s-\Re(r)}(F)$ and $A\colon H^s_\loc(E)\to H_\loc^{s-\Re(r)}(F)$.
On a closed manifold we obtain a bounded operator $A\colon H^s(E)\to H^{s-\Re(r)}(F)$.


\item\label{P:Hs:innerproduct}
Assume $M$ closed and let $\Lambda_s\in\Psi^s(E)$ be invertible with $\Lambda_s^{-1}\in\Psi^{-s}(E)$, see Lemma~\ref{L:Lambda}.
If $\llangle-,-\rrangle_{L^2(E)}$ is any Hermitian inner product generating the topology on $L^2(E)$, then
\begin{equation}\label{E:llrrSobolev}
\llangle\psi_1,\psi_2\rrangle_{H^s(E)}:=\llangle\Lambda_s\psi_1,\Lambda_s\psi_2\rrangle_{L^2(E)},
\end{equation}
$\psi_1,\psi_2\in H^s(E)$, is a Hermitian inner product generating the topology on $H^s(E)$.


\item\label{P:Hs:compact}
If $s_1<s_2$, then the inclusions $H^{s_2}_\loc(E)\subseteq H^{s_1}_\loc(E)$ and $H^{s_2}_c(E)\subseteq H^{s_1}_c(E)$ are compact.
For closed $M$ we obtain a compact inclusion $H^{s_2}(E)\subseteq H^{s_1}(E)$.

\item\label{P:Hs:Sobolevemb}
If $M$ has homogeneous dimension $n$, see~\eqref{E:hdim}, then we have continuous Sobolev embeddings $H^s_\loc(E)\subseteq\Gamma^r(E)$ and $H^s_c(E)\subseteq\Gamma^r_c(E)$ for all $r\in\mathbb N_0$ with $r<s-n/2$, see Remark~\ref{R:GammarE}.
In particular, we obtain isomorphisms of locally convex spaces,
$$
\Gamma^\infty(E)=\bigcap_s H^s_\loc(E)
\qquad\text{and}\qquad
\Gamma_c^\infty(E)=\bigcap_s H^s_c(E),
$$ 
as well as $\Gamma^{-\infty}(E)=\bigcup_sH^s_\loc(E)$ and\/ $\Gamma^{-\infty}_c(E)=\bigcup_sH^s_c(E)$.
\end{enumerate}
\end{proposition}


\begin{proof}
The proof is a routine extension, cf.\ for instance \cite[Section~\S7]{S01} for the classical case, of the results established in the preceding sections.
For each complex number $s$ we choose operators $\Lambda_s\in\Psi^s_\prop(E)$, $\Lambda'_s\in\Psi^{-s}_\prop(E)$, and $R_s\in\mathcal O^{-\infty}_\prop(E)$ such that, see Lemma~\ref{L:Lambda},
$$
\Lambda_s'\Lambda_s=\id+R_s.
$$ 

We have a continuous inclusion $\Gamma^\infty(E)\subseteq H^s_\loc(E)$ since every operator $A\in\Psi^s_\prop(E,F)$ induces a continuous map $A\colon\Gamma^\infty(E)\to\Gamma^\infty(F)$, see Proposition~\ref{P:Psi}\itemref{P:Psi:O}, and the inclusion $\Gamma^\infty(F)\subseteq L^2_\loc(F)$ is continuous.
Using Proposition~\ref{P:Psi}\itemref{P:Psi:O} we see that the composition
$H^s_\loc(E)\xrightarrow{\Lambda_s}L^2_\loc(E)\subseteq\Gamma^{-\infty}(E)\xrightarrow{\Lambda'_s}\Gamma^{-\infty}(E)$
is continuous.
Moreover, $\mathcal O^{-\infty}_\prop(E)\subseteq\Psi^s_\prop(E)$, see Proposition~\ref{P:Psi}\itemref{P:Psi:smooth}, and thus the composition 
$H^s_\loc(E)\xrightarrow{R_s}L^2_\loc(E)\subseteq\Gamma^{-\infty}(E)$ is continuous.
We conclude that $\id=\Lambda_s'\Lambda_s-R_s$ induces a continuous map $H^s_\loc(E)\subseteq\Gamma^{-\infty}(E)$.
To see that we have continuous inclusions $H^{s_2}_\loc(E)\subseteq H^{s_1}_\loc(E)$ for all $s_1\leq s_2$, we have to show that each $A\in\Psi^{s_1}_\prop(E,F)$ induces a continuous operator $A\colon H^{s_2}_\loc(E)\to L^2_\loc(E)$.
To achieve that, 
note that $A=\Lambda'_{s_2-s_1}\Lambda_{s_2-s_1}A-R_{s_2-s_1}A$.
Moreover, $\Lambda_{s_2-s_1}A\in\Psi^{s_2}_\prop(E,F)$, see Proposition~\ref{P:Psi}\itemref{P:Psi:mult}, and thus $\Lambda_{s_2-s_1}A\colon H^{s_2}_\loc(E)\to L^2_\loc(F)$ is continuous.
Moreover, $\Lambda'_{s_2-s_1}$ induces a continuous operator $L^2_\loc(F)\to L^2_\loc(F)$ since $s_1-s_2\leq0$, see Proposition~\ref{P:Psimap}\itemref{P:Psimap:L2cont}.
Furthermore, $R_{s_2-s_1}A\in\mathcal O^{-\infty}_\prop(E,F)\subseteq\Psi^{s_2}_\prop(E,F)$ in view of Proposition~\ref{P:Psi}\itemref{P:Psi:smooth} and thus $R_{s_2-s_1}A\colon H^{s_2}_\loc(E)\to L^2_\loc(F)$ is continuous.
We conclude that $A=\Lambda'_{s_2-s_1}\Lambda_{s_2-s_1}A-R_{s_2-s_1}A$ induces a continuous map $H^{s_2}_\loc(E)\to L^2_\loc(F)$, whence the continuous inclusion $H^{s_2}_\loc(E)\subseteq H^{s_1}_\loc(E)$.
The completeness of $H^s_\loc(E)$ follows from the continuity of the inclusion $H^s_\loc(E)\subseteq\Gamma^{-\infty}(E)$, the completeness of the spaces $\Gamma^{-\infty}(E)$, the fact that each $A\in\Psi^s_\prop(E,F)$ induces a continuous operator $A\colon\Gamma^{-\infty}(E)\to\Gamma^{-\infty}(F)$, see Proposition~\ref{P:Psi}\itemref{P:Psi:O}, and the completeness of the spaces $L^2_\loc(F)$.
We have a continuous inclusion $H^0_\loc(E)\subseteq L^2_\loc(E)$ since $\id\in\Psi^0_\prop(E)$, see Proposition~\ref{P:Psi}\itemref{P:Psi:DO}.
By Proposition~\ref{P:Psimap}\itemref{P:Psimap:L2cont}, we also have the converse continuous inclusion $L^2_\loc(E)\subseteq H^0_\loc(E)$.
This shows $H^0_\loc(E)=L^2_\loc(E)$ and completes the proof of \itemref{P:Hs:locfilt}.
Part \itemref{P:Hs:cfilt} can be proved analogously.


To see \itemref{P:Hs:pairing},
note that the canonical pairing can be written as
\begin{equation}\label{E:pairHs}
\langle\phi,\psi\rangle
=\langle(\Lambda'_s)^t\phi,\Lambda_s\psi\rangle
-\langle\phi,R_s\psi\rangle,
\end{equation}
where $\phi\in\Gamma^\infty_c(E')$, $\psi\in\Gamma^\infty_c(E)$, and
$(\Lambda'_s)^t\in\Psi^{-s}_\prop(E')$ according to Proposition~\ref{P:Psi}\itemref{P:Psi:trans}.
Recall that the canonical pairing extends to a pairing $L^2_c(E')\times L^2_\loc(E)\to\C$.
Since $\Lambda_s\colon H^s_\loc(E)\to L^2_\loc(E)$ and $(\Lambda_s')^t\colon H^{-s}_c(E')\to L^2_c(E')$ are continuous, we see that the term $\langle(\Lambda'_s)^t\phi,\Lambda_s\psi\rangle$ extends to a separately continuous bilinear map $H^{-s}_c(E')\times H^s_\loc(E)\to\C$.
Since $R_s$ induces a continuous operator $R_s\colon\Gamma^{-\infty}_\loc(E)\to\Gamma^\infty_\loc(E)$, the term $\langle\phi,R_s\psi\rangle$ actually extends to a separately continuous bilinear map $\Gamma^{-\infty}_c(E')\times\Gamma^{-\infty}_\loc(E)\to\C$.
Using \itemref{P:Hs:locfilt}, \itemref{P:Hs:cfilt} and \eqref{E:pairHs} we conclude that the canonical pairing extends to a separately continuous bilinear map $H^{-s}_c(E')\times H^s_\loc(E)\to\C$.
Let us now verify that the induced linear map
\begin{equation}\label{E:HsHslin}
H^s_\loc(E)\to H^{-s}_c(E')^*
\end{equation}
is bijective.
Since the inclusion $\mathcal D(E)=\Gamma^\infty_c(E')\subseteq H^{-s}_c(E')$ is continuous, we obtain a continuous map $H^{-s}_c(E')^*\to\mathcal D(E)^*=\Gamma^{-\infty}(E)$ which, when composed with \eqref{E:HsHslin}, yields the canonical inclusion $H^s_\loc(E)\subseteq\Gamma^{-\infty}(E)$. This immediately implies that \eqref{E:HsHslin} is injective.
To see that it is onto, suppose $\alpha\in H^{-s}_c(E')^*$.
The preceding considerations show that there exists $\psi\in\Gamma^{-\infty}(E)$ such that $\langle\phi,\psi\rangle=\alpha(\phi)$ for all $\phi\in\mathcal D(E)$.
It remains to show that $\psi\in H^s_\loc(E)$.
To check this, let $A\in\Psi^s_\prop(E,F)$.
Then $\langle\tilde\phi,A\psi\rangle=\langle A^t\tilde\phi,\psi\rangle=(\alpha\circ A^t)(\tilde\phi)$ extends continuously to all $\tilde\phi\in L^2_c(F')$.
Since the pairing $L^2_c(F')\times L^2_\loc(F)\to\C$ induces a linear bijection $L^2_\loc(F)=L^2_c(F')^*$, we conclude $A\psi\in L^2_\loc(F)$, whence $\psi\in H^s_\loc(E)$.
Analogously, one can verify that the induced linear map $H^{-s}_c(E')\to H^s_\loc(E)^*$ is a bijection.
If $M$ is closed, then $\Lambda_s$ may be assumed to be invertible with inverse $\Lambda_s^{-1}\in\Psi^{-s}(E)$, see Lemma~\ref{L:Lambda}, hence $\Lambda_s$ induces a topological isomorphism $H^s(E)\cong L^2(E)$, whence $H^s(E)$ is a Hilbert space.
This also implies that the canonical pairing induces an isomorphism of Hilbert spaces, $H^s(E)^*=H^{-s}(E')$, for we have $\langle\phi,\psi\rangle=\langle(\Lambda_s^{-1})^t\phi,\Lambda_s\psi\rangle$ and the canonical pairing induces an isomorphism of Hilbert spaces $L^2(E)^*=L^2(E')$.


The mapping properties in \itemref{P:Hs:operators} are immediate consequences of Proposition~\ref{P:Psi}\itemref{P:Psi:mult}, provided $r$ is real.
For complex $r$, we use Lemma~\ref{L:Lambda} and Proposition~\ref{P:Psimap}\itemref{P:Psimap:L2cont}.


The statement in \itemref{P:Hs:innerproduct} is now obvious, for $\Lambda_s\colon H^s(E)\to L^2(E)$ is a topological isomorphism with inverse induced by $\Lambda_s^{-1}\colon L^2(E)=H^0(E)\to H^s(E)$, see \itemref{P:Hs:operators}.


To prove \itemref{P:Hs:compact}, note that $\Lambda_{s_1}\Lambda_{s_2}'\in\Psi^{s_1-s_2}_\prop(E)$ and $s_1-s_2<0$, hence $\Lambda_{s_1}\Lambda_{s_2}'$ induces a compact operator on $L^2_\loc(E)$, see Proposition~\ref{P:Psimap}\itemref{P:Psimap:compact}.
Hence, the operator $\Lambda_{s_1}'\Lambda_{s_1}\Lambda_{s_2}'\Lambda_{s_2}\colon H^{s_2}_\loc(E)\to H^{s_1}_\loc(E)$ is compact.
The latter operator differs from the canonical inclusion by a properly smoothing operator for we have $\Lambda_{s_1}'\Lambda_{s_1}\Lambda_{s_2}'\Lambda_{s_2}=\id+R_{s_1}+R_{s_2}+R_{s_1}R_{s_2}$.
Moreover, properly supported smoothing operators induce compact operators on each Sobolev space $H^s_\loc(E)$ for the continuous inclusion $\Gamma^\infty(E)\subseteq H^s_\loc(E)$ is compact in view of the Heine--Borel property of $\Gamma^\infty(E)$ which asserts that the identical map on $\Gamma^\infty(E)$ is compact (Arzel\`a--Ascoli).
We conclude that the inclusion $H^{s_2}_\loc(E)\subseteq H^{s_1}_\loc(E)$ is compact.
Similarly, one shows that the inclusion $H^{s_2}_c(E)\subseteq H^{s_1}_c(E)$ is compact.


To prove \itemref{P:Hs:Sobolevemb} we have to show that every differential operator $D\in\DO^r(E,F)$ of Heisenberg order at most $r<s-n/2$ induces a continuous operator $H^s_\loc(E)\to\Gamma(F)$.
By Proposition~\ref{P:Psimap}\itemref{P:Psimap:kL2}, $D\Lambda_s'\in\Psi_\prop^{r-s}(E,F)$ induces a continuous operator $D\Lambda_s'\colon L^2_\loc(E)\to\Gamma(F)$ and, thus, $D\Lambda_s'\Lambda_s\colon H^s_\loc(E)\to\Gamma(F)$ is continuous.
Since the inclusion $\Gamma^\infty(E)\subseteq\Gamma(E)$ is continuous, $DR_s\in\mathcal O^{-\infty}_\prop(E,F)$ induces a continuous operator $DR_s\colon L^2(E)\to\Gamma(F)$.
Using $D=D\Lambda_s'\Lambda_s+DR_s$, we conclude that $D\colon H^s_\loc(E)\to\Gamma(F)$ is indeed continuous, whence the continuous inclusion $H^s_\loc(E)\subseteq\Gamma^r(E)$.
Analogously, one verifies the continuous inclusion $H^s_c(E)\subseteq\Gamma_c^r(E)$, provided $r<s-n/2$.
%The remaining statements follow from the well known topological isomorphisms: $\Gamma^\infty(E)=\bigcap_r\Gamma^r(E)$, $\Gamma^\infty_c(E)=\bigcap_r\Gamma^r_c(E)$, $\Gamma^{-\infty}(E)=\bigcup_r\Gamma^r_c(E')^*$, and $\Gamma^{-\infty}_c(E)=\bigcup_r\Gamma^r(E')^*$.
\end{proof}


\begin{remark}
For non-degenerate CR manifolds a boundedness statement as in Proposition~\ref{P:Hs}\itemref{P:Hs:operators} can be traced back to \cite[Theorem~15.19]{FS74}.
For Heisenberg manifolds satisfying the bracket generating condition $H+[H,H]=TM$ a similar statement has been established in \cite[Proposition~5.5.8]{P08}.
\end{remark}


\begin{remark}\label{R:CrHr}
Given $r\in\N_0$ one might wonder if the space $H^r_\loc(E)$ can be characterized as the space of all $\psi\in\Gamma^{-\infty}(E)$ such that $D\psi\in L^2_\loc(F)$ for all differential operators $D\in\DO^r(E,F)$ of Heisenberg order at most $r$; and if these differential operators suffice to generate the topology on $H^r_\loc(E)$.
Moreover, one might ask if $\Gamma^r(E)$ includes (continuously) in $H^r_\loc(E)$, see Remark~\ref{R:GammarE}.
For general $M$ and $r$ neither of these properties will hold true.
However, if $r$ is a common multiple of $\{1,2,\dotsc,m\}$ where $m$ is such that $\mathfrak tM=\bigoplus_{j=1}^m\mathfrak t^{-j}M$, then the universal differential operator $j^r\in\DO^r(E,J^rE)$, see Remark~\ref{R:jk}, satisfies the Rockland condition and we obtain affirmative answers to all the questions above.
This will remain true for the more values of $r$, the more non-commutative the osculating algebras are.
If $T^{-1}M$ is bracket generating, this holds true for all $r$, cf.\ \cite{FS74,BG88,P08}.
Similar remarks apply to the compactly supported case. 
\end{remark}


This Sobolev scale permits to refine the hypoellipticity statement in Theorem~\ref{T:Rockland}, providing a common generalization of well known results, see for instance \cite[Propositions~5.5.9 and 5.5.14]{P08},
\cite[Theorem~16.6]{FS74}, \cite{HN85}, and \cite[Theorem~18.31]{BG88}.


\begin{corollary}[Regularity and maximal hypoelliptic estimate]\label{C:reg}
Let $E$ and $F$ be vector bundles over a filtered manifold $M$, and suppose $A\in\Psi^k(E,F)$ satisfies the Rockland condition where $k$ is some complex number.
If $\psi\in\Gamma^{-\infty}_c(E)$ and $A\psi\in H_\loc^{s-\Re(k)}(F)$ for some real number $s$, then $\psi\in H^s_c(E)$.
If, moreover, $M$ is closed and $\tilde s\leq s$, then there exists a constant $C=C_{A,s,\tilde s}\geq0$ such that the maximal hypoelliptic estimate
$$
\|\psi\|_{H^s(E)}\leq C\left(\|\psi\|_{H^{\tilde s}(E)}+\|A\psi\|_{H^{s-\Re(k)}(F)}\right)
$$
holds for all $\psi\in H^s(E)$.
Here we are using any norms generating the Hilbert space topologies on the corresponding Heisenberg Sobolev spaces.
If, moreover, $Q$ denotes the orthogonal projection, with respect to an inner product of the form \eqref{E:llrr}, onto the (finite dimensional) subspace $\ker(A)\subseteq\Gamma^\infty(E)$, then there exists a constant $C=C_{A,s}\geq0$ such that the maximal hypoelliptic estimate
$$
\|\psi\|_{H^s(E)}\leq C\left(\|Q\psi\|_{\ker(A)}+\|A\psi\|_{H^{s-\Re(k)}(F)}\right)
$$
holds for all $\psi\in H^s(E)$. Here $\|-\|_{\ker(A)}$ denotes any norm on $\ker(A)$.
\end{corollary}


\begin{proof}
For the first part use a left parametrix as in Theorem~\ref{T:Rockland} and Proposition~\ref{P:Hs}\itemref{P:Hs:locfilt}\&\itemref{P:Hs:operators}.
To see the second hypoelliptic estimate, observe that $A^*A\in\Psi^{2\Re(k)}(E)$ satisfies the Rockland condition and $\ker(A^*A)=\ker(A)$.
Hence, $B:=(A^*A+Q)^{-1}A^*\in\Psi^{-k}(F,E)$, see Corollary~\ref{C:PsiinvA}, and $\id=Q+BA$.
\end{proof}


\begin{remark}
Let us complement the regularity statement above with the following description of the topologies on the Heisenberg Sobolev spaces.
Suppose $A\in\Psi^k_\prop(E,F)$ satisfies the Rockland condition and put $s=\Re(k)$.
Then the topologies on $H^s_\loc(E)$ and $H^s_c(E)$ coincide with the topologies induced from the embeddings
$$
H^s_\loc(E)\xrightarrow{(A,\id)}L^2_\loc(F)\times\Gamma^{-\infty}(E)
\qquad\text{and}\qquad
H^s_c(E)\xrightarrow{(A,\id)}L^2_c(F)\times\Gamma^{-\infty}_c(E),
$$
respectively.
Moreover, if $B\in\Psi^{-r}_\prop(F,E)$ is a left parametrix such that $R:=BA-\id$ is a smoothing operator, see Theorem~\ref{T:Rockland}, then the topologies on $H^s_\loc(E)$ and $H^s_c(E)$ coincide with the topologies induced from the embeddings
$$
H^s_\loc(E)\xrightarrow{(A,R)}L^2_\loc(F)\times\Gamma^\infty(E)
\qquad\text{and}\qquad
H^s_c(E)\xrightarrow{(A,R)}L^2_c(F)\times\Gamma^\infty_c(E),
$$
respectively.
Indeed, $B-R$ provides a continuous left inverse for the first two inclusions; and $B-\id$ provides a continuous left inverse for the other two inclusions.
\end{remark}


Accordingly, the Hodge type decomposition for formally selfadjoint Rockland operators on closed filtered manifolds in Corollary~\ref{C:smooth-Hodge-decomposition} admits the following refinement:


\begin{corollary}[Hodge decomposition]\label{C:HsHodge}
Let $E$ be a vector bundle over a closed filtered manifold $M$.
Suppose $A\in\Psi^k(E)$ satisfies the Rockland condition and is formally selfadjoint, $A^*=A$, with respect to an $L^2$ inner product of the form~\eqref{E:llrr}.
Moreover, let $Q$ denote the orthogonal projection onto the (finite dimensional) subspace $\ker(A)\subseteq\Gamma^\infty(E)$.
Then, for every real number $s$, we have a topological isomorphism and a Hodge type decomposition
$$
A+Q\colon H^{s+\Re(k)}(E)\xrightarrow\cong H^s(E),
\qquad H^s(E)=\ker(A)\oplus A\bigl(H^{s+\Re(k)}(E)\bigr).
$$
\end{corollary}


\begin{proof}
This follows from Corollary~\ref{C:PsiinvA} and Proposition~\ref{P:Hs}\itemref{P:Hs:operators}.
\end{proof}


Corollary~\ref{C:smhe} admits the following refinement:


\begin{corollary}\label{C:he}
Let $E$ and $F$ be a vector bundles over a closed filtered manifold $M$.
Suppose $A\in\Psi^k(E,F)$ satisfies the Rockland condition, and let $A^*\in\Psi^{\bar k}(F,E)$ denote the formal adjoint with respect to $L^2$ inner products of the form \eqref{E:llrr}.
Then, for every real number $s$, we have a Hodge type decomposition
$$
H^s(E)=\ker(A)\oplus A^*\bigl(H^{s+\Re(k)}(F)\bigr).
$$
\end{corollary}


\begin{proof}
Apply Corollary~\ref{C:HsHodge} to the operator $A^*A$, observe that $A^*A$ satisfies the Rockland condition, and use $\ker(A^*A)=\ker(A)$.
\end{proof}


\begin{corollary}[Fredholm operators and index]\label{C:Fredholm}
Let $E$ and $F$ be a vector bundles over a closed filtered manifold $M$.
Suppose $A\in\Psi^k(E,F)$ is such that $A$ and $A^t$ both satisfy the Rockland condition, cf.\ Remark~\ref{R:AtRock}.
Then, for every real number $s$, we have an induced Fredholm operator $A\colon H^s(E)\to H^{s-\Re(k)}(F)$ whose index is independent of $s$ and can be expressed as
$$
\ind(A)=\dim\ker(A)-\dim\ker(A^t).
$$
Moreover, this index only depends on the Heisenberg principal cosymbol $\sigma^k(A)\in\Sigma^k(E,F)$.
\end{corollary}


\begin{proof}
According to Remark~\ref{R:AAtRock} there exists a parametrix $B\in\Psi^{-k}(F,E)$ such that $BA-\id$ and $AB-\id$ are both smoothing operators.
Since the inclusion $\Gamma^\infty(E)\subseteq H^s(E)$ is compact, see Proposition~\ref{P:Hs}\itemref{P:Hs:compact}, smoothing operators are compact on the Sobolev spaces $H^s(E)$ and $H^{s-\Re(k)}(F)$, respectively.
Hence, $B\colon H^{s-\Re(k)}(F)\to H^s(E)$ provides an inverse of $A$ mod compact operators.
Consequently, $A\colon  H^s(E)\to H^{s-\Re(k)}(F)$ is Fredholm.
Moreover, the canonical pairing induces a canonical isomorphism $\coker(A)=\ker(A^t)$, see Proposition~\ref{P:Hs}\itemref{P:Hs:pairing}, whence the index formula above.
By regularity, $\ker(A)$ and $\ker(A^t)$ consist of smooth sections and, thus, do not depend on $s$.
If $A'\in\Psi^k(E,F)$ is another operator with the same Heisenberg principal cosymbol, $\sigma^k(A')=\sigma^k(A)$, then $A'-A\in\Psi^{k-1}(E,F)$ in view of Proposition~\ref{P:Psi}\itemref{P:Psi:symbsequ}, hence $A'-A\colon H^s(E)\to H^{s-\Re(k)}(F)$ is a compact operator according to Proposition~\ref{P:Hs}\itemref{P:Hs:operators}\&\itemref{P:Hs:compact}, and thus $\ind(A')=\ind(A)$.
\end{proof}


Let us now apply these results to the Rockland sequences considered in Section~\ref{SS:hesDO}.
Recall that the Rumin--Seshadri operators $\Delta_i$ satisfy the Rockland condition, see Lemma~\ref{L:rockseq}.
Hence, the refined regularity statement in Corollary~\ref{C:reg} applies to $\Delta_i$.
Moreover, since $\Delta_i$ is formally selfadjoint, we have Hodge type decompositions and maximal hypoelliptic estimates for $\Delta_i$ as in Corollary~\ref{C:HsHodge}, provided the underlying manifold is closed.
For the Rockland sequences, we immediately obtain the following refinement of Corollary~\ref{C:RShypo}.


\begin{corollary}[Regularity and maximal hypoelliptic estimate]\label{C:regrockseq}
In the situation of Corollary~\ref{C:RShypo}, if $\psi\in\Gamma^{-\infty}(E_i)$ is such that $A_i\psi\in H_\loc^{s-k_i}(E_{i+1})$ and $A_{i-1}^*\in H_\loc^{s-k_{i-1}}(E_{i-1})$ for some real number $s$, then $\psi\in H^s_\loc(E_i)$.
Moreover, if $M$ is closed, then there exists a constant $C=C_{A_i,s}\geq0$ such that the maximal hypoelliptic estimate
$$
\|\psi\|_{H^s(E_i)}\leq C\Bigl(\|A_{i-1}^*\psi\|_{H^{s-k_{i-1}}(E_{i-1})}+\|Q_i\psi\|_{\ker(\Delta_i)}+\|A_i\psi\|_{H^{s-k_i}(E_{i+1})}\Bigr)
$$
holds for all $\psi\in H^s(E_i)$.
Here $\|-\|_{H^s(E_i)}$ are any norms generating the Hilbert space topology on the Heisenberg Sobolev spaces $H^s(E_i)$, $Q_i$ denotes the orthogonal projection onto the (finite dimensional) subspace $\ker(\Delta_i)=\ker(A_{i-1}^*)\cap\ker(A_i)\subseteq\Gamma^\infty(E_i)$, and $\|-\|_{\ker(\Delta_i)}$ denotes any norm on $\ker(\Delta_i)$.
\end{corollary}


For Rockland complexes over closed manifolds we immediately obtain the following Hodge decomposition, refining the statement in Corollary \ref{C:Hodge}:


\begin{corollary}[Hodge decomposition]\label{C:HsHodge-seq}
In the situation of Corollary~\ref{C:Hodge} we have
$$
H^s(E_i)=A_{i-1}(H^{s+k_{i-1}}(E_{i-1}))\oplus\ker(\Delta_i)\oplus A_i^*(H^{s+k_i}(E_{i+1}))
$$
and
$$
\ker(A_i|_{H^s(E_i)})=A_{i-1}(H^{s+k_{i-1}}(E_{i-1}))\oplus\ker(\Delta_i),
$$
for every real number $s$.
\end{corollary}


\begin{remark}
For the $\bar\partial$ complex on a non-degenerate CR manifold the preceding statement can be found in \cite[Theorem~17.1]{FS74} for $s=0$.
\end{remark}


The preceding two corollaries remain true for Rockland sequences of pseudodifferential operators.
We will formulate this precisely and more generally in Section~\ref{S:grRockland} below.











\section{Graded hypoelliptic sequences}\label{S:grhesequences}












The de~Rham differentials on a filtered manifold will in general have Heisenberg order strictly larger than one, and the de~Rham complex will in general not be Rockland in the sense of Definition~\ref{def.Hypo-seq}.
In this section we will see that the de~Rham complex is, however, Rockland in an appropriate graded sense.
More generally, this remains true for the sequence obtained by extending a linear connection to vector bundle valued differential forms, provided the curvature satisfies an algebraic condition, see Proposition~\ref{P:hypo} below.
Regarding these de~Rham sequences as graded Rockland sequences, permits to extract (split off) graded Rockland sequences of (higher order) differential operators acting between vector bundles of smaller rank, see Theorem~\ref{T:D} and Corollary~\ref{C:D} below.
Since all curved BGG sequences \cite{CSS01,CD01} of regular parabolic geometries appear in this way, the latter are all graded Rockland, see Corollary~\ref{C:BGG} below.
In some cases, the sequences thus obtain are even Rockland in the ungraded sense of Section~\ref{SS:hesDO}.
In particular, this is happens for the BGG sequences associated with a generic rank two distribution in dimension five, see Example~\ref{Ex:BGG235} below.


The construction we will present in Sections~\ref{SS:DP} and \ref{SS:splitting-operators} below extends parts of the BGG machinery \cite{CSS01,CD01,CS15} to more general filtered manifolds, and reproduces the ``torsion free'' \cite{CD01} BGG sequences on regular parabolic geometries, see Section~\ref{SS:BGG}.
We will keep track of the graded Heisenberg principal symbol throughout the construction to conclude that the resulting sequence is graded Rockland.
The fundamental object we construct is a natural differential projector, see Lemma~\ref{L:EP}(b), which restricts and descends to the splitting operator in the classical BGG setup.
Combining this differential projector with splittings of the filtrations, we obtain an (unnatural) invertible differential operator which we use to conjugate the original sequence.
On the level of graded principal Heisenberg symbol, the conjugated sequence decouples as a direct sum of two graded Rockland sequences, see Proposition~\ref{P:DB} below.
One of these two sequences generalizes the curved BGG sequence.
If the original sequence was a complex, then the conjugated complex decomposes as a direct sum of this generalized BGG complex and a complementary complex which is conjugate to an acyclic tensorial complex.
We hope that this decomposition will prove useful when comparing analytic torsion of BGG complexes \cite{RS12} with the Ray--Singer torsion \cite{RS71} of the full de~Rham complex.
As an example we will work out explicitly, how the Rumin complex appears as a direct summand in the de~Rham complex, see Example~\ref{Ex:Rumin} below.



\subsection{Filtered vector bundles and differential operators}\label{SS:fVBDO}



In this section we introduce the filtration by graded Heisenberg order on differential operators acting between sections of filtered vector bundles over filtered manifolds. 
We discuss the corresponding graded Heisenberg principal symbol and establish some of its basic properties.
For trivially filtered vector bundles these concepts reduce to the filtration by Heisenberg order and the principal Heisenberg symbol discussed in Section~\ref{SS:DO} above.


Let $M$ be a filtered manifold.
Suppose $E$ is a filtered vector bundle over $M$, i.e., a smooth vector bundle which comes equipped with a filtration by smooth subbundles,
$$
\cdots\supseteq E^{p-1}\supseteq E^p\supseteq E^{p+1}\supseteq\cdots.
$$ 
We will always assume that the filtration is full, that is, $E=\bigcup_pE^p$ and $\bigcap_pE^p=0$.
Put $\gr_p(E):=E^p/E^{p+1}$ and let $\gr(E):=\bigoplus_p\gr_p(E)$ denote the associated graded vector bundle equipped with the filtration by the subbundles $\gr^q(E):=\bigoplus_{q\leq p}\gr_p(E)$.
By a \emph{splitting of the filtration} on $E$ we mean a filtration preserving vector bundle isomorphism $S_E\colon\gr(E)\to E$ which induces the identity on the level of associated graded, $\gr(S_E)=\id_{\gr(E)}$.
More explicitly, $S_E$ maps $\gr_p(E)$ into $E^p$ such that the composition with the projection $E^p\to E^p/E^{p+1}=\gr_p(E)$ is the identity.
Such splittings always exist, in fact the space of all splittings is convex, hence contractible.


Suppose $F$ is another filtered vector bundle over $M$, and let $S_F\colon\gr(F)\to F$ be a splitting for its filtration.
A differential operator $A\in\DO(E,F)$ is said to have \emph{graded Heisenberg order} at most $k$ if the operator $S_F^{-1}AS_E\in\DO(\gr(E),\gr(F))$ has the following property:
The component $(S_F^{-1}AS_E)_{q,p}\in\DO(\gr_p(E),\gr_q(F))$ in the decomposition according to the gradings, $S_F^{-1}AS_E=\sum_{p,q}(S_F^{-1}AS_E)_{q,p}$, has Heisenberg order at most $k+q-p$.
Consider the Heisenberg principal symbols of these components, 
$$
\sigma_x^{k+q-p}((S_F^{-1}AS_E)_{q,p})\colon C^\infty(\mathcal T_xM,\gr_p(E_x))\to C^\infty(\mathcal T_xM,\gr_q(F_x)),
$$ 
see Section~\ref{SS:DO}, and define the \emph{graded Heisenberg principal symbol}
$$
\tilde\sigma_x^k(A)\colon C^\infty(\mathcal T_xM,\gr(E_x))\to C^\infty(\mathcal T_xM,\gr(F_x))
$$
by
$$
\tilde\sigma^k_x(A):=\sum_{p,q}\sigma^{k+q-p}_x\bigl((S_F^{-1}AS_E)_{q,p}\bigr).
$$
This is a left invariant differential operator which is \emph{homogeneous of degree $k$ in the graded sense}, that is, $\tilde\sigma^k_x(A)\circ l_g^*=l_g^*\circ\tilde\sigma_x^k(A)$ and
$$
\tilde\sigma^k_x(A)\circ\delta^{E_x}_\lambda\circ\delta_{\lambda,x}^*
=\lambda^k\cdot\delta^{F_x}_\lambda\circ\delta_{\lambda,x}^*\circ\tilde\sigma^k_x(A)
$$
for all $g\in\mathcal T_xM$ and $\lambda>0$, see \eqref{E:skAlg}.
Here $\delta^{E_x}_\lambda\in\Aut(\gr(E_x))$ denotes the isomorphism given by multiplication with $\lambda^p$ on the component $\gr_p(E_x)$.
Equivalently, the graded Heisenberg principal symbol at $x$ can be considered as an element of the degree $-k$ component of $\mathcal U(\mathfrak t_xM)\otimes\hom(\gr(E_x),\gr(F_x))$, that is,
$$
\tilde\sigma^k_x(A)\in
\bigl(\mathcal U(\mathfrak t_xM)\otimes\hom(\gr(E_x),\gr(F_x))\bigr)_{-k}
:=
\bigoplus_{p,q}\mathcal U_{-k+q-p}(\mathfrak t_xM)\otimes\hom(\gr_p(E_x),\gr_q(F_x)).
$$
Note that any two splittings of a filtered vector bundle differ (multiplicatively) by a filtration preserving vector bundle isomorphism inducing the identity on the associated graded.
Thus, the filtration of differential operators by graded Heisenberg order does not depend on the choice of splittings $S_E$ and $S_F$, and the graded principal Heisenberg symbol is independent of this choice too.
We obtain a short exact sequence
$$
0\to\tDO^{k-1}(E,F)\to\tDO^k(E,F)\xrightarrow{\tilde\sigma^k}\Gamma^\infty\Bigl(\bigl(\mathcal U(\mathfrak tM)\otimes\hom(\gr(E),\gr(F))\bigr)_{-k}\Bigr)\to0
$$
where $\tDO^k(E,F)$ denotes the space of differential operators in $\DO(E,F)$ which are of graded Heisenberg order at most $k$.


If $G$ is another filtered vector bundle, and $B\in\DO(F,G)$ is a differential operator of graded Heisenberg order at most $l$, then the composition $BA\in\DO(E,G)$ has graded Heisenberg order at most $l+k$ and 
\begin{equation}\label{E:tsAB}
\tilde\sigma^{l+k}_x(BA)=\tilde\sigma^l_x(B)\tilde\sigma_x^k(A).
\end{equation}
This follows readily from \eqref{E:sABAt}.
Moreover, the transposed operator $A^t\in\DO(F',E')$ is of graded Heisenberg order at most $k$ and
\begin{equation}\label{E:grsAt}
\tilde\sigma^k_x(A^t)=\tilde\sigma_x^k(A)^t.
\end{equation}
Here the filtration on the bundle $E'=E^*\otimes|\Lambda|_M=\hom(E,|\Lambda|_M)$ is defined such that a section of $E'$ is in filtration degree $p$ iff it pairs $E^q$ into the $(p+q)$-th filtration subspace of the trivially filtered line bundle $|\Lambda|_M=|\Lambda|_M^0\supseteq|\Lambda|_M^1=0$, i.e.\ iff it vanishes on $E^{-p+1}$.
Note that the canonical isomorphism $\gr_p(E')=\hom(E^{-p}/E^{-p+1},|\Lambda|_M)=\gr_{-p}(E)'$ provides a canonical isomorphism of filtered vector bundles $\gr(E')=\gr(E)'$.
\footnote{The degree $p$ filtration subbundle of $\gr(E)$ is $\gr(E)^p=\bigoplus_{q\geq p}\gr_q(E)$ and thus the filtration on $\gr(E)'$ canonically identifies to $(\gr(E)')^p=\bigoplus_{q\leq-p}\gr_q(E)'=\bigoplus_{q\leq-p}\gr_{-q}(E')=\gr(E')^p$.}
If $S_E\colon\gr(E)\to E$ is a splitting of the filtration on $E$, then $S_E^t\colon E'\to\gr(E)'=\gr(E')$ is filtration preserving and $S_{E'}:=(S_E^t)^{-1}\colon\gr(E')\to E'$ is a splitting of the filtration on $E'$.
Moreover, $(S_{E'}^{-1}A^tS_{F'})_{p,q}=((S_F^{-1}AS_E)_{-q,-p})^t=((S_F^{-1}AS_E)^t)_{p,q}$ and thus \eqref{E:grsAt} follows at once from \eqref{E:sABAt}.


\begin{remark}\label{R:grsigma}
Every differential operator $A\in\tDO^k(E,F)$ of graded Heisenberg order at most $k$ maps $\Gamma^\infty(E^p)$ into $\Gamma^\infty(F^{p-k})$. 
Indeed, $(S_F^{-1}AS_E)_{q,p}=0$ for $k+q-p<0$ since there are no non-trivial differential operators of negative order.
Moreover, the induced operator $\gr_k(A)\colon\Gamma^\infty(\gr_p(E))\to\Gamma^\infty(\gr_{p-k}(F))$ is tensorial and the corresponding vector bundle homomorphism $\gr_k(A)\colon\gr_p(E)\to\gr_{p-k}(F)$ coincides with the corresponding component of the graded principal Heisenberg symbol, $(\tilde\sigma^k(A))_{p-k,p}$.
\end{remark}


Generalizing Definition~\ref{def.Hypo-seq} we have:


\begin{definition}[Graded Rockland sequences of differential operators]\label{D:graded_hypoelliptic_seq}
Let $E_i$ be filtered vector bundles over a filtered manifold $M$.
A sequence of differential operators
$$
\cdots\to\Gamma^\infty(E_{i-1})\xrightarrow{A_{i-1}}\Gamma^\infty(E_i)\xrightarrow{A_i}\Gamma^\infty(E_{i+1})\to\cdots
$$ 
which are of graded Heisenberg order at most $k_i$, respectively, is said to be \emph{graded Rockland sequence} if, for each $x\in M$ and every non-trivial irreducible unitary representation $\pi\colon\mathcal T_xM\to U(\mathcal H)$ of the osculating group $\mathcal T_xM$ the sequence
$$
\cdots\to
\mathcal H_\infty\otimes\gr(E_{i-1,x})\xrightarrow{\pi(\tilde\sigma^{k_{i-1}}_x(A_{i-1}))}
\mathcal H_\infty\otimes\gr(E_{i,x})\xrightarrow{\pi(\tilde\sigma^{k_i}_x(A_i))}
\mathcal H_\infty\otimes\gr(E_{i+1,x})\to\cdots
$$
is weakly exact, i.e., the image of each arrow is contained and dense in the kernel of the subsequent arrow.
Here $\mathcal H_\infty$ denotes the subspace of smooth vectors in the Hilbert space $\mathcal H$.
\end{definition}






\subsection{Differential projectors}\label{SS:DP}






In this section we consider a sequence of differential operators acting between sections of filtered vector bundles over a filtered manifold.
Following the BGG machinery \cite{CSS01,CD01,CS15} we will present a construction which permits to extract (split off) sequences of differential operators acting between sections of certain subbundles.
If the original sequence was graded Rockland, then so is the new one, see Proposition~\ref{P:DB}(b) below.
 

The construction is based on the following simple fact which will be used repeatedly below.
A similar argument can be found in the construction of curved BGG sequences, see \cite{CSS01} and \cite[Theorem~5.2]{CD01}.


\begin{lemma}\label{L:inv}
Let $E$ and $F$ be filtered vector bundles over a filtered manifold $M$.
Suppose $A\colon\Gamma^\infty(E)\to\Gamma^\infty(F)$ is a differential operator of graded Heisenberg order at most zero and suppose the induced vector bundle homomorphism, $\tilde A\colon\gr(E)\to\gr(F)$, is invertible, cf.\ Remark~\ref{R:grsigma}.
Then $A$ is invertible and its inverse, $A^{-1}\colon\Gamma^\infty(F)\to\Gamma^\infty(E)$, is a differential operator of graded Heisenberg order at most zero.
\end{lemma}


\begin{proof}
Choose splittings of the filtrations $S_E\colon\gr(E)\to E$ and $S_F\colon\gr(F)\to F$.
Then the differential operator 
$$
\id-AS_E\tilde A^{-1}S_F^{-1}\colon\Gamma^\infty(F)\to\Gamma^\infty(F)
$$ 
is nilpotent for it induces zero on the associated graded.
Hence, the inverse of $A$ can be expressed using the finite Neumann series,
\begin{equation}\label{E:invneumann}
A^{-1}=S_E\tilde A^{-1}S_F^{-1}\sum_{n=0}^\infty\Bigl(\id-AS_E\tilde A^{-1}S_F^{-1}\Bigr)^n.
\end{equation}
In view of this formula, $A^{-1}$ has graded Heisenberg order at most zero.
\end{proof}


\begin{lemma}\label{L:EP}
Consider a filtered vector bundle $E$ over a filtered manifold $M$.
Suppose $\Box\colon\Gamma^\infty(E)\to\Gamma^\infty(E)$ is a differential operator of graded Heisenberg order at most zero, and let $\tilde\Box\colon\gr(E)\to\gr(E)$ denote the associated graded vector bundle endomorphism, see Remark~\ref{R:grsigma}.
For each $x\in M$ let $\tilde P_x\in\eend(\gr(E_x))$ denote the spectral projection onto the generalized zero eigenspace of\/ $\tilde\Box_x\in\eend(\gr(E_x))$.
Assume that the rank of $\tilde P_x$ is locally constant in $x\in M$.

\begin{enumerate}[(a)]
\item 
Then $\tilde P\colon\gr(E)\to\gr(E)$ is a smooth vector bundle homomorphism,
$\tilde P^2=\tilde P$, $\tilde P\tilde\Box=\tilde\Box\tilde P$, and we obtain a decomposition of graded vector bundles,
\begin{equation}\label{E:tpdeco}
\gr(E)=\img(\tilde P)\oplus\ker(\tilde P),
\end{equation}
invariant under $\tilde\Box$, and such that $\tilde\Box$ is nilpotent on $\img(\tilde P)$ and invertible on $\ker(\tilde P)$.


\item
There exists a unique filtration preserving differential operator $P\colon\Gamma^\infty(E)\to\Gamma^\infty(E)$ such that $P^2=P$, $P\Box=\Box P$, and $\gr(\Box)=\tilde\Box$.
This operator $P$ has graded Heisenberg order at most zero and provides a decomposition of filtered vector spaces,
\begin{equation}\label{E:Pdeco}
\Gamma^\infty(E)=\img(P)\oplus\ker(P),
\end{equation}
which is invariant under $\Box$ and such that $\Box$ is nilpotent on $\img(P):=P(\Gamma^\infty(E))$ and invertible on $\ker(P):=\{\psi\in\Gamma^\infty(E):\Box\psi=0\}$.


\item
Let $S\colon\gr(E)\to E$ be a splitting of the filtration.
Then
\begin{equation}\label{E:tL}
L\colon\Gamma^\infty(\gr(E))\to\Gamma^\infty(E),\qquad
L:=PS\tilde P+(\id-P)S(\id-\tilde P),
\end{equation}
is an invertible differential operator of graded Heisenberg order at most zero such that $\gr(L)=\id$ and $L^{-1}PL=\tilde P$.
Hence, $L$ induces filtration preserving isomorphisms
$$
L\colon\Gamma^\infty(\img(\tilde P))\xrightarrow\cong\img(P)
\qquad\text{and}\qquad
L\colon\Gamma^\infty(\ker(\tilde P))\xrightarrow\cong\ker(P).
$$
Moreover, $L^{-1}\Box L$ is a differential operator of graded Heisenberg order at most zero satisfying $\gr(L^{-1}\Box L)=\tilde\Box$.
Furthermore, $L^{-1}\Box L$ preserves the decomposition \eqref{E:tpdeco}, its restriction to $\Gamma^\infty(\img(\tilde P))$ is nilpotent, and its restriction to $\Gamma^\infty(\ker(\tilde P))$ is invertible.
\end{enumerate}
\end{lemma}


\begin{proof}
Part (a) is well known.
To prove the existence of an operator $P$ as in (b) we assume, for a moment, that there exists $\varepsilon>0$ such that all non-trivial eigenvalues of $\tilde\Box_x$ lie outside the disk of radius $2\varepsilon$ centered at the origin in the complex plane for all $x\in M$.
Hence, the vector bundle homomorphism $z-\tilde\Box$ is invertible, for all $0<|z|<2\varepsilon$.
According to Lemma~\ref{L:inv}, $(z-\Box)^{-1}$ is a family of differential operators of graded Heisenberg order at most zero depending rationally on $z$ for $|z|<2\varepsilon$.
Hence,
\begin{equation}\label{E:P}
P:=\frac1{2\pi\ii}\oint_{|z|=\varepsilon}(z-\Box)^{-1}dz,
\end{equation}
defines a differential operator of graded Heisenberg order at most zero.
Clearly, $P^2=P$ and $\Box P=P\Box$.
Moreover, $P$ is filtration preserving and $\gr(P)=\tilde P$ in view of $\tilde\Box_x=\frac1{2\pi\mathbf i}\oint_{|z|=\varepsilon}(z-\tilde \Box_x)^{-1}dz$.
In particular, $P$ gives rise to a decomposition of filtered vector spaces which is invariant under $\Box$ as indicated in \eqref{E:Pdeco}.
In general, the $\varepsilon$ used above, will only exist locally.
However, since the operator in \eqref{E:P} does not depend on the choice of $\varepsilon$, these locally defined differential operators match up and give rise to a globally defined differential operator $P$ with said properties.
Using $P^2=P$ and $\tilde P^2=\tilde P$, we immediately obtain $PL=L\tilde P$.
In view of $\gr(P)=\tilde P$ and $\gr(S)=\id$, we have $\gr(L)=\id$, hence $L$ is invertible according to Lemma~\ref{L:inv} and $L^{-1}$ is a differential operator of graded Heisenberg order at most zero.
Since $\Box$ preserves the decomposition \eqref{E:Pdeco}, $L^{-1}\Box L$ preserves the decomposition \eqref{E:tpdeco}.
From $\gr(L)=\id$, we get $\gr(L^{-1}\Box L)=\tilde\Box$.
Using the latter it is easy to see that $\Box$ is nilpotent on $\img(P)$ and invertible on $\ker(P)$.
Indeed, since $\tilde\Box$ is nilpotent on $\img(\tilde P)$, we conclude that $L^{-1}\Box L$ is nilpotent on $\Gamma^\infty(\img(\tilde P))$, and thus $\Box$ is nilpotent on $\img(P)$.
Furthermore, since $\tilde\Box$ is invertible on $\ker(\tilde P)$, Lemma~\ref{L:inv} implies that $L^{-1}\Box L$ is invertible on $\Gamma^\infty(\ker(\tilde P))$, and thus $\Box$ is invertible on $\ker(P)$.
Using the latter property one readily checks the uniqueness assertion in (b).
\end{proof}


After these preparations let us now turn to the construction of the sequences mentioned at the beginning of this section.
Let $E_i$ be filtered vector bundles over a filtered manifold $M$, and consider a sequence of differential operators,
\begin{equation}\label{E:EAE}
\cdots\to\Gamma^\infty(E_{i-1})\xrightarrow{A_{i-1}}\Gamma^\infty(E_i)\xrightarrow{A_i}\Gamma^\infty(E_{i+1})\to\cdots,
\end{equation}
such that $A_i$ is of graded Heisenberg order at most $k_i$.
Suppose
\begin{equation}\label{E:EdelE}
\cdots\leftarrow\Gamma^\infty(E_{i-1})\xleftarrow{\delta_i}\Gamma^\infty(E_i)\xleftarrow{\delta_{i+1}}\Gamma^\infty(E_{i+1})\leftarrow\cdots
\end{equation}
is a sequence of differential operators such that $\delta_i$ is of graded Heisenberg order at most $-k_{i-1}$.
Then the differential operators
\begin{equation}\label{E:EboxE}
\Box_i\colon\Gamma^\infty(E_i)\to\Gamma^\infty(E_i),\qquad
\Box_i:=A_{i-1}\delta_i+\delta_{i+1}A_i,
\end{equation}
are of graded Heisenberg order at most zero, see Section~\ref{SS:fVBDO}.
\footnote{In subsequent sections we will restrict our attention to the case when $\delta_i$ are (tensorial) vector bundle homomorphisms satisfying $\delta_i\delta_{i+1}=0$, but these restrictions would not be helpful here.}


We let $\tilde\Box_i\colon\gr(E_i)\to\gr(E_i)$, $\tilde\Box_i:=\gr(\Box_i)$, denote the associated graded vector bundle homomorphism, see Remark~\ref{R:grsigma}.
Moreover, for each $x\in M$, we let $\tilde P_{x,i}\colon\gr(E_{x,i})\to \gr(E_{x,i})$ denote the spectral projection onto the generalized zero eigenspace of $\tilde\Box_{x,i}\in\eend(\gr(E_{x,i}))$.
Note that the projectors $\tilde P_{x,i}$ preserve the grading on $\gr(E_{x,i})$.
We assume that the rank of $\tilde P_{x,i}$ is locally constant in $x$. 
Consequently, these fiber wise projectors provide a smooth vector bundle projector $\tilde P_i\colon\gr(E_i)\to\gr(E_i)$ and we obtain a decomposition of graded vector bundles, 
\begin{equation}\label{E:decogrE}
\gr(E_i)=\img(\tilde P_i)\oplus\ker(\tilde P_i),
\end{equation}
which is invariant under $\tilde\Box_i$, see Lemma~\ref{L:EP}(a).
By construction, $\tilde\Box_i$ is nilpotent on $\img(\tilde P_i)$ and invertible on $\ker(\tilde P_i)$.
We let $\tilde A_i\colon\gr_*(E_i)\to\gr_{*-k_i}(E_{i+1})$ and $\tilde\delta_i\colon\gr_*(E_i)\to\gr_{*+k_{i-1}}(E_{i-1})$ denote the associated graded vector bundle homomorphisms of $A_i$ and $\delta_i$, respectively, see Remark~\ref{R:grsigma}.
Clearly, $\tilde\Box_i=\tilde A_{i-1}\tilde\delta_i+\tilde\delta_{i+1}\tilde A_i$, see \eqref{E:EboxE}.
If $\tilde A_i\tilde A_{i-1}=0$ for all $i$, then $\tilde\Box_{i+1}\tilde A_i=\tilde A_i\tilde\Box_i$, $\tilde P_{i+1}\tilde A_i=\tilde A_i\tilde P_i$, and, thus, $\tilde A_i$ preserves the decompositions \eqref{E:decogrE}.
Similarly, if $\tilde\delta_{i-1}\tilde\delta_i=0$ for all $i$, then $\tilde\Box_{i-1}\tilde\delta_i=\tilde\delta_i\tilde\Box_i$, $\tilde P_{i-1}\tilde\delta_i=\tilde\delta_i\tilde P_i$, and $\tilde\delta_i$ preserves the decompositions in \eqref{E:decogrE}.


According to Lemma~\ref{L:EP}(b), there exists a unique filtration preserving differential operator 
$$
P_i\colon\Gamma^\infty(E_i)\to\Gamma^\infty(E_i)
$$ 
such that $P_i^2=P_i$, $P_i\Box_i=\Box_iP_i$, and $\gr(P_i)=\tilde P_i$.
These projectors are of graded Heisenberg order at most zero and provide a decomposition $\Gamma^\infty(E_i)=\img(P_i)\oplus\ker(P_i)$ such that $\Box_i$ is nilpotent on $\img(P_i)$ and invertible on $\ker(P_i)$.


We fix splittings for the filtrations, $S_i\colon\gr(E_i)\to E_i$, and consider the differential operators 
$$
L_i\colon\Gamma^\infty(\gr(E_i))\to\Gamma^\infty(E_i),\qquad
L_i:=P_iS_i\tilde P_i+(\id-P_i)S_i(\id-\tilde P_i).
$$
In view of Lemma~\ref{L:EP}(c), $L_i$ is an invertible differential operator of graded Heisenberg order at most zero such that $\gr(L_i)=\id$ and $L_i^{-1}P_iL_i=\tilde P_i$.
Moreover, $L^{-1}_i\Box_iL_i$ is a differential operator of graded Heisenberg order at most zero such that $\gr(L_i^{-1}\Box_iL_i)=\tilde\Box_i$.
Furthermore, $L^{-1}_i\Box_iL_i$ preserves the decomposition \eqref{E:decogrE}, its restriction to $\Gamma^\infty(\img(\tilde P_i))$ is nilpotent and its restriction to $\Gamma^\infty(\ker(\tilde P_i))$ is invertible.


Conjugating the original sequence \eqref{E:EAE} by $L_i$, we obtain two sequences,
\begin{equation}\label{E:EDE}
\cdots\to\Gamma^\infty(\img(\tilde P_{i-1}))\xrightarrow{D_{i-1}}\Gamma^\infty(\img(\tilde P_i))\xrightarrow{D_i}\Gamma^\infty(\img(\tilde P_{i+1}))\to\cdots
\end{equation}
and
\begin{equation}\label{E:EBE}
\cdots\to\Gamma^\infty(\ker(\tilde P_{i-1}))\xrightarrow{B_{i-1}}\Gamma^\infty(\ker(\tilde P_i))\xrightarrow{B_i}\Gamma^\infty(\ker(\tilde P_{i+1}))\to\cdots,
\end{equation}
where 
\begin{equation}\label{E:defDB}
D_i:=\tilde P_{i+1}L_{i+1}^{-1}A_iL_i|_{\Gamma^\infty(\img(\tilde P_i))}
\qquad\text{and}\qquad
B_i:=(\id-\tilde P_{i+1})L_{i+1}^{-1}A_iL_i|_{\Gamma^\infty(\ker(\tilde P_i))}
\end{equation}
are differential operators of graded Heisenberg order at most $k_i$.


\begin{proposition}\label{P:DB}
In this situation the following hold true:
\begin{enumerate}[(a)]
\item
If $\tilde\sigma^{k_i}_x(A_i)\tilde\sigma^{k_{i-1}}_x(A_{i-1})=0$ for all $i$ and $x$, then 
$$
\tilde\sigma^{k_i}_x(L^{-1}_{i+1}A_iL_i)=\tilde\sigma^{k_i}_x(D_i)\oplus\tilde\sigma^{k_i}_x(B_i).
$$

\item 
If the sequence \eqref{E:EAE} is graded Rockland, then so are the sequences \eqref{E:EDE} and \eqref{E:EBE}.

\item
If $A_iA_{i-1}=0$ for all $i$, then the operator $L_{i+1}^{-1}A_iL_i$ decouples,
$$
L^{-1}_{i+1}A_iL_i=D_i\oplus B_i,
$$ 
and we have $D_iD_{i-1}=0=B_iB_{i-1}$.
In this situation, $G_i\colon\Gamma^\infty(\ker(\tilde P_i))\to\Gamma^\infty(\ker(\tilde P_i))$,
$$
G_i:=B_{i-1}(\id-\tilde P_{i-1})\tilde\delta_i\tilde\Box_i^{-1}
+(\id-\tilde P_i)\tilde\delta_{i+1}\tilde\Box_{i+1}^{-1}\tilde A_i,
$$ 
is an invertible differential operator of graded Heisenberg order at most zero with $\gr(G_i)=\id$ that conjugates the complex \eqref{E:EBE} into an acyclic tensorial complex, that is, 
$$
G_{i+1}^{-1}B_iG_i=\tilde A_i|_{\Gamma^\infty(\ker(\tilde P_i))}.
$$
Moreover, the restriction $L_i\colon\Gamma^\infty(\img(\tilde P_i))\to\Gamma^\infty(E_i)$ provides a chain map, that is, $A_iL_i|_{\Gamma^\infty(\img(\tilde P_i))}=L_{i+1}D_i$, which induces an isomorphism between the cohomologies of \eqref{E:EDE} and \eqref{E:EAE}.
More precisely, $\pi_i:=\tilde P_iL_i^{-1}\colon\Gamma^\infty(E_i)\to\Gamma^\infty(\img(\tilde P_i))$, is a chain map, $D_i\pi_i=\pi_{i+1}A_i$, which is inverse up to homotopy, i.e., $\pi_iL_i|_{\Gamma^\infty(\img(\tilde P_i))}=\id$ and\/ $\id-L_i\pi_i=A_{i-1}h_i+h_{i+1}A_i$ where $h_i\colon\Gamma^\infty(E_i)\to\Gamma^\infty(E_{i-1})$ is a differential operator of graded Heisenberg order at most $-k_{i-1}$ given by $h_i:=L_{i-1}G_{i-1}(\id-\tilde P_i)\tilde\delta_i\tilde\Box_i^{-1}G_i^{-1}(\id-\tilde P_i)L_i^{-1}$.
\end{enumerate}
\end{proposition}


\begin{proof}
To show part (c) suppose $A_iA_{i-1}=0$ for all $i$.
Then $\Box_{i+1}A_i=A_i\Box_i$, see \eqref{E:EboxE}, and thus $P_{i+1}A_i=A_iP_i$, see \eqref{E:P}.
Using the relation $L_i^{-1}P_iL_i=\tilde P_i$ from Lemma~\ref{L:EP}(c), we obtain $\tilde P_{i+1}(L_{i+1}^{-1}A_iL_i)=(L_{i+1}^{-1}A_iL_i)\tilde P_i$, hence $\tilde P_{i+1}(L_{i+1}^{-1}A_iL_i)(\id-\tilde P_i)=0=(\id-\tilde P_{i+1})(L_{i+1}^{-1}A_iL_i)\tilde P_i$,
$$
L^{-1}_{i+1}A_iL_i=\tilde P_{i+1}(L^{-1}_{i+1}A_iL_i)\tilde P_i
+(\id-\tilde P_{i+1})(L^{-1}_{i+1}A_iL_i)(\id-\tilde P_i),
$$ 
and thus $L^{-1}_{i+1}A_iL=D_i\oplus B_i$, cf.\ \eqref{E:defDB}.
Clearly, $A_iA_{i-1}=0$ implies $D_iD_{i-1}=0=B_iB_{i-1}$ and $B_iG_i=G_{i+1}\tilde A_i$.
Clearly, $\gr(G_i)=\id$ and, thus, $G_i$ is invertible according to Lemma~\ref{L:inv}.
The remaining assertions in (c) are now straight forward.


To show part (a) suppose $\tilde\sigma^{k_i}_x(A_i)\tilde\sigma^{k_{i-1}}_x(A_{i-1})=0$ for all $i$ and all $x\in M$.
Using the multiplicativity of the graded Heisenberg principal symbol, see \eqref{E:tsAB}, we conclude $\tilde\sigma^0_x(\Box_{i+1})\tilde\sigma^{k_i}_x(A_i)=\tilde\sigma_x^{k_i}(A_i)\tilde\sigma^0_x(\Box_i)$, see \eqref{E:EboxE}, and thus $\tilde\sigma^0_x(P_{i+1})\tilde\sigma^{k_i}_x(A_i)=\tilde\sigma^{k_i}_x(A_i)\tilde\sigma^0_x(P_i)$, see \eqref{E:P}.
Combining this with $L_i^{-1}P_iL_i=\tilde P_i$ from Lemma~\ref{L:EP}(c), we obtain $\tilde\sigma_x^{k_i}\bigl(\tilde P_{i+1}(L_{i+1}^{-1}A_iL_i)\bigr)=\tilde\sigma^{k_i}_x\bigl((L_{i+1}^{-1}A_iL_i)\tilde P_i\bigr)$, hence 
$\tilde\sigma^{k_i}_x\bigl(\tilde P_{i+1}(L_{i+1}^{-1}A_iL_i)(\id-\tilde P_i)\bigr)=0=\tilde\sigma^{k_i}_x\bigl((\id-\tilde P_{i+1})(L_{i+1}^{-1}A_iL_i)\tilde P_i\bigr)$,
$$
\tilde\sigma^{k_i}_x(L^{-1}_{i+1}A_iL_i)
=\tilde\sigma^{k_i}_x\bigl(\tilde P_{i+1}(L^{-1}_{i+1}A_iL_i)\tilde P_i\bigr)
+\tilde\sigma^{k_i}_x\bigl((\id-\tilde P_{i+1})(L^{-1}_{i+1}A_iL_i)(\id-\tilde P_i)\bigr),
$$ 
and thus $\tilde\sigma^{k_i}_x(L^{-1}_{i+1}A_iL_i)=\tilde\sigma^{k_i}_x(D_i)\oplus\tilde\sigma^{k_i}_x(B_i)$.


To see (b) suppose the sequence \eqref{E:EAE} is graded Rockland, that is, the graded Heisenberg principal symbol sequences $\sigma^{k_i}_x(A_i)$ are exact in every non-trivial irreducible unitary representation of $\mathcal T_xM$.
Clearly, the conjugated sequence $\tilde\sigma^{k_i}_x(L_{i+1}^{-1}A_iL_i)=\tilde\sigma^0_x(L_{i+1})^{-1}\tilde\sigma^{k_i}_x(A_i)\tilde\sigma^0_x(L_i)$ has the same property.
Note that $\tilde\sigma^{k_i}_x(A_i)\tilde\sigma^{k_{i-1}}_x(A_{i-1})=0$ since this relation holds true in every irreducible unitary representation of $\mathcal T_xM$.
Hence part (a), permits to conclude that the graded Heisenberg principal symbol sequences $\tilde\sigma^{k_i}_x(D_i)$ and $\tilde\sigma^{k_i}_x(B_i)$ are  exact in every non-trivial irreducible unitary representation of $\mathcal T_xM$ too.
Hence, the sequences $D_i$ and $B_i$ are graded Rockland.
\end{proof}


\begin{remark}
If the $\delta_i$ are such that the spectra of $\tilde\Box_{i,x}$ are locally constant in $x\in M$, then the construction discussed above can be refined.
In this case, for each eigenvalue $\lambda$ of $\tilde\Box_{i,x}$, we let $\tilde P^{(\lambda)}_{i,x}$ denote the spectral projection onto the corresponding generalized eigenspace.
These fiber wise projectors give rise to a vector bundle projector $\tilde P^{(\lambda)}_i$ on $\gr(E_i)$ which preserves the grading and commutes with $\tilde\Box_i$.
Clearly, $\sum_\lambda\tilde P^{(\lambda)}_i=\id$ and $\tilde P^{(\lambda)}_i\tilde P^{(\mu)}_i=0=\tilde P^{(\mu)}_i\tilde P^{(\lambda)}_i$ for any two different eigenvalues $\lambda\neq\mu$, and we obtain a decomposition of graded vector bundles,
\begin{equation}\label{E:finerdeco}
\gr(E_i)=\bigoplus_\lambda\img\bigl(\tilde P^{(\lambda)}_i\bigr)
\end{equation}
where the finite direct sum is over all eigenvalues $\lambda$ of $\tilde\Box_i$.
Proceeding as in Lemma~\ref{L:EP} above one readily shows that there exists a unique filtration preserving differential operator 
$$
P^{(\lambda)}_i\colon\Gamma^\infty(E_i)\to\Gamma^\infty(E_i)
$$ 
such that $(P^{(\lambda)}_i)^2=P^{(\lambda)}_i$, $P^{(\lambda)}_i\Box_i=\Box_i P^{(\lambda)}_i$, and $\gr(P^{(\lambda)}_i)=\tilde P^{(\lambda)}_i$.
These differential projectors can be constructed using a contour integral as in \eqref{E:P}, they are of Heisenberg order at most zero, they satisfy the completeness relation $\sum_\lambda P^{(\lambda)}_i=\id$ and $P^{(\lambda)}_iP^{(\mu)}_i=0=P^{(\mu)}_iP^{(\lambda)}_i$ for any two different eigenvalues $\lambda\neq\mu$, and they provide a finite decomposition of filtered vector spaces,
$$
\Gamma^\infty(E_i)=\bigoplus_\lambda\img\bigl(P^{(\lambda)}_i\bigr)
$$
such that $\Box_i-\lambda$ is nilpotent on $\img(P^{(\lambda)}_i)$.
Using splittings of the filtrations, $S_i\colon\gr(E_i)\to E_i$, we obtain an invertible differential operator of graded Heisenberg order at most zero,
$$
\hat L_i\colon\Gamma^\infty(\gr(E_i))\to\Gamma^\infty(E_i),\qquad
\hat L_i:=\sum_\lambda P^{(\lambda)}_iS_i\tilde P^{(\lambda)}_i,
$$
such that $\gr(\hat L_i)=\id$ and $\hat L^{-1}_iP^{(\lambda)}_i\hat L_i=\tilde P^{(\lambda)}_i$.
Hence, $\hat L_i$ restricts to isomorphisms of filtered vector spaces,
$$
\Gamma^\infty\bigl(\img\bigl(\tilde P^{(\lambda)}_i\bigr)\bigr)\xrightarrow\cong\img\bigl(P^{(\lambda)}_i\bigr)
$$
and $\hat L^{-1}_i\Box_i\hat L_i$ is a differential operator of graded Heisenberg order at most zero which preserves the decomposition \eqref{E:finerdeco} and satisfies $\gr(\hat L^{-1}_i\Box_i\hat L_i)=\tilde\Box_i$.
Conjugating the original sequence \eqref{E:EAE} with $\hat L_i$ we obtain sequences of differential operators, one for each $\lambda$,
\begin{equation}\label{E:hatDl}
\cdots\to\Gamma^\infty(\img(\tilde P^{(\lambda)}_{i-1}))\xrightarrow{D^{(\lambda)}_{i-1}}\Gamma^\infty(\img(\tilde P^{(\lambda)}_{i}))\xrightarrow{D^{(\lambda)}_{i}}\Gamma^\infty(\img(\tilde P^{(\lambda)}_{i+1}))\to\cdots
\end{equation}
where $D^{(\lambda)}_i:=\tilde P^{(\lambda)}_{i+1}\hat L^{-1}_{i+1}A_i\hat L_i|_{\img(\tilde P^{(\lambda)}_i)}$ is of graded Heisenberg order at most zero.
For $\lambda=0$ we recover the sequence discussed before, i.e., $D^{(0)}_i=D_i$.
Similar to Proposition~\ref{P:DB}, we have:
\begin{enumerate}[(a)]
\item 
If $\tilde\sigma^{k_i}_x(A_i)\tilde\sigma^{k_{i-1}}_x(A_{i-1})=0$ for all $i$ and $x$, then $\tilde\sigma^{k_i}_x(\hat L^{-1}_{i+1}A_i\hat L_i)=\bigoplus_\lambda\tilde\sigma^{k_i}_x(D_i^{(\lambda)})$.

\item 
If the sequence \eqref{E:EAE} is graded Rockland, then so is the sequence \eqref{E:hatDl} for each $\lambda$.

\item 
If $A_iA_{i-1}=0$ for all $i$, then the operator $\hat L_{i+1}^{-1}A_i\hat L_i$ decouples as $\hat L_{i+1}^{-1}A_i\hat L_i=\bigoplus_\lambda D^{(\lambda)}_i$ and we have $D_i^{(\lambda)}D^{(\lambda)}_{i-1}=0$ for all $\lambda$ and all $i$.
For all $\lambda\neq0$, the complex \eqref{E:hatDl} is acyclic and conjugate to an algebraic complex, that is, there exist invertible differential operators of graded Heisenberg order at most zero, $G_i^{(\lambda)}\colon\Gamma^\infty(\img(\tilde P^{(\lambda)}_i))\to\Gamma^\infty(\img(\tilde P_i^{(\lambda)}))$, such that $\gr(G_i^{(\lambda)})=\id$ and $(G_{i+1}^{(\lambda)})^{-1}D_i^{(\lambda)}G_i^{(\lambda)}=\tilde A_i|_{\Gamma^\infty(\tilde P_i^{(\lambda)})}$.
\end{enumerate}
\end{remark}


We conclude this section with an example for which we describe the operators $P$, $L$, $D$ and $B$ more explicitly.
The choice of filtration is motivated by the filtration on differential forms on contact manifolds which will be discussed subsequently in Example~\ref{Ex:Rumin}.


\begin{example}\label{Ex:preRumin}
Consider vector bundles $E_k=E_k'\oplus E_k''$ over a filtered manifold, and a sequence of differential operators, $A_k\colon\Gamma^\infty(E_k)\to\Gamma^\infty(E_{k+1})$.
Introduce a filtration on $E_k$ such that 
$$
E_k=E_k^k\supseteq E_k^{k+1}\supseteq E_k^{k+2}=0
\qquad\text{where}\qquad
E_k^{k+1}=E_k''.
$$
We assume $A_k$ is of graded Heisenberg order at most zero.
Equivalently, $A_k$ is of the form
$$
A_k=\begin{pmatrix}a'_k&e_k\\b_k&-a''_k\end{pmatrix}
$$
where $b_k\colon\Gamma^\infty(E_k')\to\Gamma^\infty(E_{k+1}'')$ is a differential operator of Heisenberg order at most two, $a'_k\colon\Gamma^\infty(E_k')\to\Gamma^\infty(E_{k+1}')$ and $a''_k\colon\Gamma^\infty(E_k'')\to\Gamma^\infty(E_{k+1}'')$ are two differential operators of Heisenberg order at most one, and $e_k\colon E_k''\to E_{k+1}'$ is a vector bundle homomorphism.
Moreover, we consider an operator $\delta_k\colon\Gamma^\infty(E_k)\to\Gamma^\infty(E_{k-1})$ of Heisenberg order at most zero, that is, $\delta_k$ has the form
$$
\delta_k=\begin{pmatrix}0&0\\f_k&0\end{pmatrix}
$$
where $f_k\colon E_k'\to E_{k-1}''$ is a vector bundle homomorphism.
For sake of simplicity in notation, we will drop the subscript of the operators indicating the vector bundle.
Clearly,
$$
\Box=[A,\delta]=\begin{pmatrix}ef&0\\fa'-a''f&fe\end{pmatrix},
\quad
\tilde A=\begin{pmatrix}0&e\\0&0\end{pmatrix},
\quad
\tilde\delta=\begin{pmatrix}0&0\\f&0\end{pmatrix},
\quad
\tilde\Box=\begin{pmatrix}ef&0\\0&fe\end{pmatrix}.
$$
We assume that the rank of $\tilde\Box$ is locally constant, i.e., the ranks of $ef$ and $fe$ are both locally constant.
Let $\tilde P=1-\tilde Q$ denote the fiber wise spectral projection onto the generalized zero eigenspace of $\tilde\Box$.
Introduce vector bundle projectors $\tilde P'=1-\tilde Q'$ and $\tilde P''=1-\tilde Q''$ such that
$$
\tilde P=\begin{pmatrix}\tilde P'&0\\0&\tilde P''\end{pmatrix}
\qquad\text{and}\qquad
\tilde Q=\begin{pmatrix}\tilde Q'&0\\0&\tilde Q''\end{pmatrix}.
$$
Since $\tilde P$ commutes with $\tilde A$ and $\tilde\delta$, we have
$\tilde P'e=e\tilde P''$ and $\tilde P''f=f\tilde P'$.
As $\tilde\Box$ is invertible on $\ker\tilde P$, we may define a vector bundle homomorphism $e_k^\dag\colon E'_{k+1}\to E''_k$ such that $ee^\dag=\tilde Q'$, $e^\dag e=\tilde Q''$, and $e^\dag\tilde P'=0=\tilde P''e^\dag$.

Assume, moreover, $A^2=0$, that is, $(a')^2+eb=0$, $(a'')^2+be=0$, $a'e=ea''$, and $ba'=a''b$.
In this situation, the differential projector $P$ characterized in Lemma~\ref{L:EP}(b) is given by
$$
P=\begin{pmatrix}\tilde P'&0\\R&\tilde P''\end{pmatrix}
\qquad\text{where}\qquad
R:=a''e^\dag-e^\dag a'.
$$
Indeed, using $\tilde Q''a''e^\dag=e^\dag a'\tilde Q'$, we find $R=R\tilde P'+\tilde P''R$ and thus $P^2=P$.
Moreover, $[\tilde P',a']=eR$, $[\tilde P'',a'']=Re$, $b\tilde P'-\tilde P''b=a''R+Ra'$ and thus $[A,P]=0$. 
Obviously, $[\delta,P]=0$ and, consequently, $[\Box,P]=0$.
For the operator $L$ in Lemma~\ref{L:EP}(c) we find
$$
L=P\tilde P+Q\tilde Q=\begin{pmatrix}\id&0\\-\tilde P''a''e^\dag-e^\dag a'\tilde P'&\id\end{pmatrix},
$$
where $Q=\id-P$.
Moreover, $L^{-1}\delta L=\tilde\delta$,
$$
L^{-1}\Box L=\begin{pmatrix}ef&0\\\tilde P''(fa'-a''f)\tilde P'+\tilde Q''(fa'-a''f)\tilde Q'&fe\end{pmatrix},
$$
and
$$
L^{-1}AL=\begin{pmatrix}\tilde P'a'\tilde P'+\tilde Q'a'\tilde Q'&e
\\\tilde P''(a''e^\dag a'+b)\tilde P'-\tilde Q''a''e^\dag a'\tilde Q'&-\tilde P''a''\tilde P''-\tilde Q''a''\tilde Q''\end{pmatrix}.
$$
The operators $D_k\colon\Gamma^\infty(\img(\tilde P_k))\to\Gamma^\infty(\img(\tilde P_{k+1}))$ and $B_k\colon\Gamma^\infty(\ker(\tilde P_k))\to\Gamma^\infty(\ker(\tilde P_{k+1}))$ thus take the form
$$
D=\begin{pmatrix}\tilde P'a'\tilde P'&\tilde P'e\tilde P''
\\\tilde P''(a''e^\dag a'+b)\tilde P'&-\tilde P''a''\tilde P''\end{pmatrix}
\qquad\text{and}\qquad
B=\begin{pmatrix}\tilde Q'a'\tilde Q'&\tilde Q'e\tilde Q''
\\-\tilde Q''a''e^\dag a'\tilde Q'&-\tilde Q''a''\tilde Q''\end{pmatrix}
$$
w.r.\ to the decompositions $\img(\tilde P_k)=\img(\tilde P'_k)\oplus\img(\tilde P_k'')$ and $\ker(\tilde P_k)=\ker(\tilde P'_k)\oplus\ker(\tilde P_k'')$.
\end{example}






\subsection{Splitting operators}\label{SS:splitting-operators}






The sequences \eqref{E:EDE} and \eqref{E:EBE} constructed above depend on the operators $A_i$ and $\delta_i$, see \eqref{E:EAE} and \eqref{E:EdelE} but also on the splittings $S_i\colon\gr(E_i)\to E_i$. 
We will now specialize to a situation in which one can construct a variant of the sequence \eqref{E:EDE} which does not depend on the splittings $S_i$, but only on $A_i$ and $\delta_i$.
The sequences obtain in this way, see Proposition~\ref{P:D} below, generalize the curved BGG sequences \cite{CSS01,CD01} discussed in Section~\ref{SS:BGG}.


We continue to use the notation from the preceding sections.
In particular, $E_i$ are filtered vector bundles over a filtered manifold $M$, and we consider a sequence,
\begin{equation}\label{E:EAE2}
\cdots\to\Gamma^\infty(E_{i-1})\xrightarrow{A_{i-1}}\Gamma^\infty(E_i)\xrightarrow{A_i}\Gamma^\infty(E_{i+1})\to\cdots,
\end{equation}
where $A_i$ is a differential operator of graded Heisenberg order at most $k_i$.


\begin{definition}[Codifferential of Kostant type]\label{D:Kdelta}
A sequence of vector bundle homomorphisms,
\begin{equation}\label{E:admEdelE}
\cdots\leftarrow E_{i-1}\xleftarrow{\delta_i}E_i\xleftarrow{\delta_{i+1}}E_{i+1}\leftarrow\cdots,
\end{equation}
will be called a \emph{codifferential of Kostant type} for the sequence \eqref{E:EAE2} if it has the following properties:
\begin{enumerate}[(i)]
\item
$\delta_i\delta_{i+1}=0$ for all $i$.
\item
$\delta_i$ maps the filtration space $E_i^p$ into $E_{i-1}^{p+k_{i-1}}$.\footnote{In particular, $\delta_i$ is of graded Heisenberg order at most $-{k_{i-1}}$.}
\item
There exist splittings of the filtrations, $S_i\colon\gr(E_i)\to E_i$, such that $\tilde\delta_i=S_{i-1}^{-1}\delta_iS_i$.
\footnote{Only the existence of such splittings is required, $S_i$ is not part of the data.}
\item
$\tilde\delta_{i,x}\tilde P_{i,x}=0$ for each $x\in M$.
\end{enumerate}
Here $\tilde A_i\colon\gr_*(E_i)\to\gr_{*-k_i}(E_{i+1})$, 
$\tilde\delta_i\colon\gr_*(E_i)\to\gr_{*+k_{i-1}}(E_{i-1})$, and
$\tilde\Box_i\colon\gr_*(E_i)\to\gr_*(E_i)$
denote the vector bundle homomorphisms induced by $A_i$, $\delta_i$ and $\Box_i=A_{i-1}\delta_i+\delta_{i+1}A_i$ on the associated graded vector bundles, respectively, see Remark~\ref{R:grsigma}.
Moreover, $\tilde P_{i,x}\colon E_{i,x}\to E_{i,x}$ denotes the spectral projection onto the generalized zero eigenspace of $\tilde\Box_{i,x}$.
\end{definition}



\begin{lemma}\label{L:L}
Consider a sequence of vector bundle homomorphisms $\delta_i$ as in \eqref{E:admEdelE} which is a codifferential of Kostant type for the sequence \eqref{E:EAE2}, see Definition~\ref{D:Kdelta}, and assume that the rank of $\delta_{x,i}$ is locally constant in $x$, for each $i$.
Then the following hold true:

\begin{enumerate}[(a)]
\item
The rank of $\tilde P_{i,x}$ is locally constant in $x$, these families provide smooth vector bundle projectors, $\tilde P_i\colon\gr(E_i)\to\gr(E_i)$, such that $\tilde\delta_i\tilde\delta_{i+1}=0$, $\tilde\Box_{i-1}\tilde\delta_i=\tilde\delta_i\tilde\Box_i$, $\tilde P_i\tilde\Box_i=\tilde\Box_i\tilde P_i$, $\tilde P_{i-1}\tilde\delta_i=\tilde\delta_i\tilde P_i=0$, and we have a decomposition of graded  vector bundles,
\begin{equation}\label{E:kertddeco}
\ker(\tilde\delta_i)=\img(\tilde P_i)\oplus\img(\tilde\delta_{i+1}).
\end{equation}

\item
Let $P_i\colon\Gamma^\infty(E_i)\to\Gamma^\infty(E_i)$ denote the unique filtration preserving differential operator such that $P_i^2=P_i$, $P_i\Box_i=\Box_iP_i$ and $\gr(P_i)=\tilde P_i$, see Lemma~\ref{L:EP}(b).
Then $\Box_{i-1}\delta_i=\delta_i\Box_i$, $P_{i-1}\delta_i=\delta_iP_i=0$, and we obtain a decomposition of filtered spaces,
\begin{equation}\label{E:kertdeco}
\Gamma^\infty(\ker(\delta_i))=\img(P_i)\oplus\Gamma^\infty(\img(\delta_{i+1})).
\end{equation}
Moreover, the quotient bundle $\mathcal H_i:=\ker(\delta_i)/\img(\delta_{i+1})$ is a filtered vector bundle and $P_i$ factors to a differential operator of graded Heisenberg order at most zero,
$$
\bar L_i\colon\Gamma^\infty(\mathcal H_i)\xrightarrow\cong\img(P_i)\subseteq\Gamma^\infty(E_i),
$$
which is inverse to the restriction of the canonical projection $\bar\pi_i\colon\ker(\delta_i)\to\mathcal H_i$, that is, $\bar\pi_i\bar L_i=\id$ and $\bar L_i\bar\pi_i=P_i$.


\item
Assume, moreover, $\tilde A_{i+1}\tilde A_i=0$ for all $i$.
Then $\tilde\Box_{i+1}\tilde A_i=\tilde A_i\tilde\Box_i$, $\tilde P_{i+1}\tilde A_i=\tilde A_i\tilde P_i$, $\tilde\delta_{i+1}\tilde A_i\tilde P_i=0$, and 
\begin{equation}\label{E:kertbox}
\img(\tilde P_i)=\ker(\tilde\Box_i)
=\ker(\tilde\delta_i)\cap\ker(\tilde\delta_{i+1}\tilde A_i).
\end{equation}
Moreover, $\delta_{i+1}A_iP_i=0$, and 
\begin{equation}\label{E:kerbox}
\img(P_i)=\ker(\Box_i)=\ker(\delta_i)\cap\ker(\delta_{i+1}A_i).
\end{equation}
%In particular, the differential operator $\bar L_i$ above can be characterized as the unique operator $\bar L_i\colon\Gamma^\infty(\mathcal H_i)\to\Gamma^\infty(E_i)$ such that $\delta_i\bar L_i=0$, $\pi_i\bar L_i=\id$, and $\delta_iA_iL_i=0$.
\end{enumerate}
\end{lemma}


\begin{proof}
In view of $\tilde\delta_{i-1,x}\tilde\delta_{i,x}=0$ and $\tilde\delta_{i,x}\tilde P_{i,x}=0$ we also have $\tilde\Box_{i-1,x}\tilde\delta_{i,x}=\tilde\delta_{i,x}\tilde\Box_{i,x}$ and $\tilde P_{i-1,x}\tilde\delta_{i,x}=\tilde\delta_{i,x}\tilde P_{i,x}=0$.
Moreover, since $\tilde A_{i-1,x}\tilde\delta_{i,x}+\tilde\delta_{i+1,x}\tilde A_{i,x}=\tilde\Box_{i,x}$ is invertible on $\ker(\tilde P_{i,x})$, it is straight forward to establish the finite dimensional decomposition $\ker(\tilde\delta_{i,x})=\img(\tilde P_{i,x})\oplus\img(\tilde\delta_{i+1,x})$ for each $x\in M$.
Since $\tilde\delta_{i,x}$ is assumed to have locally constant rank, the same must be true for $\tilde P_{i,x}$.
Hence, $\tilde P_i$ is a smooth vector bundle homomorphism, and \eqref{E:kertddeco} is a decomposition of smooth vector bundles.
The remaining assertions in (a) are obvious.


Clearly, $\delta_i\Box_i=\Box_{i-1}\delta_i$ and $\delta_iP_i=P_{i-1}\delta_i$, see \eqref{E:EboxE} and \eqref{E:P}.
To show $\delta_iP_i=0$, we consider the operator 
\begin{equation}\label{E:Li4}
L_i\colon\Gamma^\infty(\gr(E_i))\to\Gamma^\infty(E_i),\qquad
L_i=P_iS_i\tilde P_i+(\id-P_i)S_i(\id-\tilde P_i),
\end{equation}
cf.~\eqref{E:tL} in Lemma~\ref{L:EP}, where the splittings $S_i$ are as in Definition~\ref{D:Kdelta}(iii).
Then $L_{i-1}^{-1}\delta_iL_i=\tilde\delta_i$ and $L^{-1}_iP_iL_i=\tilde P_i$.
The assertion $\delta_iP_i=0$ thus follows from $\tilde\delta_i\tilde P_i=0$, see Definition~\ref{D:Kdelta}(iv).
Since $A_{i-1}\delta_i+\delta_{i+1}A_i=\Box_i$ is invertible on $\ker(P_i)$, it is now straight forward to establish the decomposition \eqref{E:kertdeco}.
Clearly, $\img(\delta_{i+1})$, $\ker(\delta_i)$ and $\mathcal H_i=\ker(\delta_i)/\img(\delta_{i+1})$ are smooth vector bundles.
Moreover, $P_i$ factorizes to a linear map $\bar L_i\colon\Gamma^\infty(\mathcal H_i)\to\img(P_i)$ such that $\bar L_i\bar\pi_i=P_i$ and $\bar\pi_i\bar L_i=\id$.
To see that $\bar L_i$ is a differential operator of graded Heisenberg order at most zero, it suffices to observe that $\bar L_i$ can be written as the composition of $P_i$ with a filtration preserving vector bundle homomorphism $\mathcal H_i\to\ker(\delta_i)$.


If $\tilde A_i\tilde A_{i-1}=0$, then the assertions $\tilde\Box_{i+1}\tilde A_i=\tilde A_i\tilde\Box_i$, $\tilde P_{i+1}\tilde A_i=\tilde A_i\tilde P_i$, and $\tilde\delta_{i+1}\tilde A_i\tilde P_i=0$, as well as the equalities in \eqref{E:kertbox} are now obvious.
Using $\Box_i=\delta_{i+1}A_i+A_{i-1}\delta_i$, $\delta_iP_i=0=P_{i-1}\delta_i$, $\Box_iP_i=P_i\Box_i$ and $P^2_i=P_i$, we obtain $\delta_{i+1}A_iP_i=\Box_iP_i=P_i\Box_iP_i=0$, whence the equalities in \eqref{E:kerbox}.
The remaining assertions follow at once.
\end{proof}


\begin{proposition}\label{P:D}
Let $E_i$ be filtered vector bundles over a filtered manifold $M$, and consider a sequence of differential operators,
\begin{equation}\label{E:EAE3}
\cdots\to\Gamma^\infty(E_{i-1})\xrightarrow{A_{i-1}}\Gamma^\infty(E_i)\xrightarrow{A_i}\Gamma^\infty(E_{i+1})\to\cdots,
\end{equation}
such that $A_i$ is of graded Heisenberg order at most $k_i$ and $\tilde A_i\tilde A_{i-1}=0$ for all $i$.
Moreover, let 
$$
\cdots\leftarrow E_{i-1}\xleftarrow{\delta_i}E_i\xleftarrow{\delta_{i+1}}E_{i+1}\leftarrow\cdots
$$
be a codifferential of Kostant type for the sequence \eqref{E:EAE3}, see Definition~\ref{D:Kdelta}.
Assume that each $\delta_i$ has locally constant rank, and let $\bar\pi_i\colon\ker(\delta_i)\to\mathcal H_i:=\ker(\delta_i)/\img(\delta_{i+1})$ denote the canonical vector bundle projection.
Then the following hold true:

\begin{enumerate}[(a)]
\item
There exists a unique differential operator $\bar L_i\colon\Gamma^\infty(\mathcal H_i)\to\Gamma^\infty(E_i)$ such that $\delta_i\bar L_i=0$, $\bar\pi_i\bar L_i=\id$, and $\delta_{i+1}A_i\bar L_i=0$.
Moreover, $\bar L_i$ is of graded Heisenberg order at most zero, and the differential operator 
\begin{equation}\label{E:defbarD}
\bar D_i\colon\Gamma^\infty(\mathcal H_i)\to\Gamma^\infty(\mathcal H_{i+1}),\qquad\bar D_i:=\bar\pi_{i+1}A_i\bar L_i,
\end{equation}
is of graded Heisenberg order at most $k_i$.


\item
If the sequence \eqref{E:EAE3} is graded Rockland, then so is the sequence:
\begin{equation}\label{E:D}
\cdots\to\Gamma^\infty(\mathcal H_{i-1})\xrightarrow{\bar D_{i-1}}\Gamma^\infty(\mathcal H_i)\xrightarrow{\bar D_i}\Gamma^\infty(\mathcal H_{i+1})\to\cdots.
\end{equation}

\item
If $A_iA_{i-1}=0$, then $\bar D_i\bar D_{i-1}=0$, and $\bar L_i\colon\Gamma^\infty(\mathcal H_i)\to\Gamma^\infty(E_i)$ is a chain map, $A_i\bar L_i=\bar L_{i+1}\bar D_i$, inducing an isomorphism between the cohomologies of \eqref{E:D} and \eqref{E:EAE3}.

\item
There exist invertible differential operators, $V_i\colon\Gamma^\infty(\img(\tilde P_i))\to\Gamma^\infty(\mathcal H_i)$ with inverse $V_i^{-1}\colon\Gamma^\infty(\mathcal H_i)\to\Gamma^\infty(\img(\tilde P_i))$, both of graded Heisenberg order at most zero, such that 
$$
V_{i+1}^{-1}\bar D_iV_i=D_i,
$$
where $D_i\colon\Gamma^\infty(\img(\tilde P_i))\to\Gamma^\infty(\img(\tilde P_{i+1}))$ denotes the operator considered in Section~\ref{SS:DP}, see \eqref{E:defDB}, associated with splittings $S_i$ as in Definition~\ref{D:Kdelta}(iii).
\end{enumerate}
\end{proposition}


\begin{proof}
Part (a) follows immediately from Lemma~\ref{L:L}(b)\&(c).
Recall the differential operator $L_i$ associated with the splittings $S_i$, see \eqref{E:Li4}.
The differential operator $V_i\colon\Gamma^\infty(\img(\tilde P_i))\to\Gamma^\infty(\mathcal H_i)$, $V_i:=\bar\pi_iL_i|_{\img(\tilde P_i)}$ has graded Heisenberg order at most zero and so does its inverse, $V_i^{-1}\colon\Gamma^\infty(\mathcal H_i)\to\Gamma^\infty(\img(\tilde P_i))$, $V_i^{-1}=L_i^{-1}\bar L_i$.
Clearly, $V_{i+1}^{-1}\bar D_iV_i=D_i$, whence (d).
Combining this with Proposition~\ref{P:DB}(b)\&(c), we obtain the statements in (b) in (c).
\end{proof}


The operators $\bar L_i$ in Proposition~\ref{P:D}(a) above are direct generalizations of the well known splitting operators in parabolic geometry, see \cite[Theorem~2.4]{CS12}.
The operators $\bar D_i$ generalize the BGG operators.

\begin{remark}
If the filtration on each $\mathcal H_i$ is trivial, then \eqref{E:D} is a Rockland sequence in the ungraded sense of Definition~\ref{def.Hypo-seq}.
\end{remark}


\begin{remark}
For the Kostant type codifferential $\delta_i=0$ all statements in Proposition~\ref{P:D} are trivially true, $\bar\pi_i=\id=\bar L_i$. 
\end{remark}


\begin{remark}\label{R:Kdeltabound}
If $\tilde A_i\tilde A_{i-1}=0$, then the rank of a Kostant type codifferential is bounded by
\begin{equation}\label{E:Kdeltaesti}
\sum_i\rank(\delta_{i,x})\leq\sum_i\rank(\tilde A_{i,x}).
\end{equation}
Indeed, from $\tilde\delta_i\tilde\delta_{i+1}=0$ and $\tilde\delta_i\tilde P_i=0$ we get $\ker(\tilde\delta_{i,x})=\img(\tilde P_{i,x})\oplus\img(\tilde\delta_{i+1,x})$ and, consequently, $\ker(\delta_{i,x})/\img(\delta_{i+1,x})\cong\ker(\tilde\delta_{i,x})/\img(\tilde\delta_{i+1,x})\cong\img(\tilde P_{i,x})$.
Moreover, $\tilde A_{i,x}$ preserves the decomposition $\gr(E_i)=\img(\tilde P_{i,x})\oplus\ker(\tilde P_{i,x})$ and its restriction to $\ker(\tilde P_{i,x})$ is acyclic since $\tilde\Box_{i,x}$ is invertible on $\ker(\tilde P_{i,x})$.
We conclude 
\begin{equation}\label{E:dimHH}
\dim\bigl(\ker(\tilde A_{i,x})/\img(\tilde A_{i-1,x})\bigr)
\leq\dim\bigl(\ker(\delta_{i,x})/\img(\delta_{i+1,x})\bigr)
\end{equation}
for all $i$.
To show \eqref{E:Kdeltaesti} it thus remains to recall that the equation $\dim(C)=2\rank(d)+\dim(\ker(d)/\img(d))$ holds for every finite dimensional vector space $C$ and every linear map $d\colon C\to C$ satisfying $d^2=0$.
Also note that we have equality in \eqref{E:Kdeltaesti} if and only if we have equality in \eqref{E:dimHH} for all $i$.
In this case the codifferential is of maximal rank, and we have $\mathcal H_{x,i}\cong\ker(\tilde A_{i,x})/\img(\tilde A_{i-1,x})$.
\end{remark}


\begin{remark}\label{R:Kdeltaexi}
If $\tilde A_i\tilde A_{i-1}=0$, then there exists a Kostant type codifferential of maximal rank.
To construct such a codifferential, fix fiber wise graded Hermitian inner products on the graded vector bundles $\gr(E_i)$, let $\tilde A_i^*\colon\gr_*(E_{i+1})\to\gr_{*-k_i}(E_i)$ denote the fiber wise adjoint of $\tilde A_i\colon\gr_*(E_i)\to\gr_{*+k_i}(E_{i+1})$, and consider $\delta_i:=S_{i-1}\tilde A_{i-1}^*S_i^{-1}$ where $S_i\colon\gr(E_i)\to E_i$ are some splittings for the filtrations.
In this case $\tilde\Box_i=\tilde A_i^*\tilde A_i+\tilde A_{i-1}\tilde A_{i-1}^*$, hence $\img(\tilde P_{i,x})=\ker(\tilde\Box_{i,x})=\ker(\tilde A_{i,x})\cap\ker(\tilde\delta_{i,x})$ 
and 
$$
\gr(E_{i,x})=\img(\tilde A_{i-1,x})\oplus\img(\tilde P_{i,x})\oplus\img(\tilde\delta_{i+1,x})
$$
for each $x\in M$.
We conclude that $\delta$ is a Kostant type codifferential of maximal rank, see Definition~\ref{D:Kdelta} and Remark~\ref{R:Kdeltabound}.
If the dimension of $\ker(\tilde A_{i,x})/\img(\tilde A_{i-1,x})$ is locally constant for all $i$, then $\delta_{i,x}$ has locally constant rank for all $i$, hence $\delta_i$ meets the assumptions in Proposition~\ref{P:D}, we obtain a sequence of operators $\bar D_i$ as in \eqref{E:D}, acting between sections of the vector bundles 
$$
\mathcal H_i\cong\ker(\tilde A_i)/\img(\tilde A_{i-1}),
$$ 
and this is a graded Rockland sequence provided the original sequence $A_i$ was.
\end{remark}






\subsection{Linear connections on filtered manifolds}\label{SS:linearconnections}





In this section we consider a linear connection on a filtered vector bundle over a filtered manifold and its extension to bundle valued differential forms.
We will show that this is a graded Rockland sequence in the sense of Definition~\ref{D:graded_hypoelliptic_seq} provided the linear connection is filtration preserving and its curvature is contained in filtration degree one, see Proposition~\ref{P:hypo} below.


Suppose $E$ is a filtered vector bundle over a filtered manifold $M$.
We consider the induced filtration on the vector bundles $\Lambda^kT^*M\otimes E$.
To be explicit, $\psi\in\Lambda^kT^*_xM\otimes E_x$ is in filtration degree $p$ iff, for all tangent vectors $X_i\in T^{p_i}_xM$, we have $\psi(X_1,\dotsc,X_k)\in E_x^{p+p_1+\cdots+p_k}$.
Let us introduce the following notation for the associated graded vector bundle:
\begin{equation}\label{E:CkME}
C^k(M;E):=
\gr(\Lambda^kT^*M\otimes E)=\Lambda^k\mathfrak t^*M\otimes\gr(E).
\end{equation}
We will denote the grading by $C^k(M;E)=\bigoplus_pC^k(M;E)_p$.


Suppose $\nabla$ is a linear connection on $E$ such that
$\nabla_X\psi\in\Gamma^\infty(E^{p+q})$ for all $X\in\Gamma^\infty(T^pM)$ and $\psi\in\Gamma^\infty(E^q)$.
In other words, $\nabla\colon\Gamma^\infty(E)\to\Omega^1(M;E)$, is assumed to be filtration preserving.
It is easy to see that filtration preserving connections always exist, the space of all such connections is affine over the space of filtration preserving vector bundle homomorphisms $E\to T^*M\otimes E$.
The Leibniz rule implies that the induced operator on the associated graded bundles, $\omega:=\gr(\nabla)\colon\Gamma^\infty(\gr(E))\to\Gamma^\infty(\gr(T^*M\otimes E))$, is tensorial, i.e.\
$$
\omega\in\Gamma^\infty\bigl(C^1(M;\eend(E))_0\bigr)
$$
where $C^1(M;\eend(E))=\gr(T^*M\otimes\eend(E))=\mathfrak t^*M\otimes\eend(\gr(E))$ with $0$-th grading component $C^1(M;\eend(E))_0=\bigoplus_{p,q}(\mathfrak t^pM)^*\otimes\hom(\gr_q(E),\gr_{p+q}(E))$.
Let 
\begin{equation}\label{E:dnabla}
d^\nabla\colon\Omega^*(M;E)\to\Omega^{*+1}(M;E)
\end{equation}
denote the usual extension of $\nabla$ characterized by the Leibniz rule,
\begin{equation}\label{E:leibn}
d^\nabla(\alpha\wedge\psi)=d\alpha\wedge\psi+(-1)^k\alpha\wedge d^\nabla\psi,
\end{equation}
for all $\alpha\in\Omega^k(M)$ and all $\psi\in\Omega^*(M;E)$.
Recall the explicit formula
\begin{multline}\label{E:dnablaf}
(d^\nabla\psi)(X_0,\dotsc,X_k)
=\sum_{i=0}^k(-1)^i\nabla_{X_i}\psi(X_0,\dotsc,\hat i,\dotsc,X_k)
\\+\sum_{0\leq i<j\leq k}(-1)^{i+j}\psi([X_i,X_j],X_0,\dotsc,\hat i,\dotsc,\hat j,\dotsc,X_k)
\end{multline}
for $\psi\in\Omega^k(M;E)$ and vector fields $X_0,\dotsc,X_k$.


\begin{lemma}\label{L:symb}
Let $\nabla\colon\Gamma^\infty(E)\to\Omega^1(M;E)$ be a filtration preserving linear connection on a filtered vector bundle $E$ over a filtered manifold $M$.
Then the extension $d^\nabla\colon\Omega^k(M;E)\to\Omega^{k+1}(M;E)$ is a differential operator of graded Heisenberg order at most zero.
Moreover, the graded principal Heisenberg symbol at $x\in M$ fits into the following commutative diagram:
$$
\xymatrix{
C^\infty\bigl(\mathcal T_xM,C_x^k(M;E)\bigr)\ar@{=}[r]\ar[d]^-{\tilde\sigma^0_x(d^\nabla)}
&\Omega^k\bigl(\mathcal T_xM;\gr(E_x)\bigr)\ar[d]^-{d+\omega_x\wedge-}
%&\Omega^k(\mathcal T_xM)\otimes\gr(E_x)\ar[d]^-{d+\omega_x\wedge-}
\\
C^\infty\bigl(\mathcal T_xM,C_x^{k+1}(M;E)\bigr)\ar@{=}[r]
&\Omega^{k+1}\bigl(\mathcal T_xM;\gr(E_x)\bigr)
%&\Omega^{k+1}(\mathcal T_xM)\otimes\gr(E_x)
}
$$
Here the horizontal identifications are obtained by tensorizing the identity on $\gr(E_x)$ with the identification $C^\infty(\mathcal T_xM,\Lambda^k\mathfrak t_x^*M)=\Omega^k(\mathcal T_xM)$ induced by left trivialization of the tangent bundle of the group $\mathcal T_xM$.
Moreover, $\omega_x\in C^1_x(M;\eend(E))_0$ is considered as a left invariant $\eend(\gr(E_x))$-valued $1$-form on $\mathcal T_xM$.
\end{lemma}


\begin{proof}
This follows readily from  \eqref{E:dnablaf} and \eqref{E:snablaX}.
Alternatively, this can be understood as follows.
Let $A\colon E\to T^*M\otimes E$ be a filtration preserving vector bundle homomorphism. 
Then $\nabla+A$ is another filtration preserving linear connection on $E$, and $\omega^{\nabla+A}=\omega^\nabla+\tilde A$ where $\tilde A=\gr(A)\colon\gr(E)\to\gr(T^*M\otimes E)$ denotes the associate graded vector bundle homomorphism induced by $A$.
Moreover, $d^{\nabla+A}=d^\nabla+A\wedge-$ and thus $\tilde\sigma^0_x(d^{\nabla+A})=\tilde\sigma^0_x(d^\nabla)+\tilde A_x\wedge-$.
We conclude that the statement of the lemma holds for $\nabla$ iff it holds for $\nabla+A$.
Since this statement is local, and since the space of filtration preserving linear connections on $E$ is affine over the space of filtration preserving vector bundle homomorphisms $E\to T^*M\otimes E$, we may w.l.o.g.\ assume that $\nabla$ is the trivial connection on a trivial bundle $E=M\times E_0$.
By compatibility with direct sums, it suffices to consider the trivial line bundle $E=M\times\C$, that is, we may assume $d^\nabla=d$, the de~Rham differential on $\Omega^*(M)$.
In view of the Leibniz rule, it suffices to show that $d\colon\Omega^k(M)\to\Omega^{k+1}(M)$ has graded Heisenberg order at most zero and $\tilde\sigma^0_x(d)=d$, for $k=0,1$.
This, however, can readily be checked.
\end{proof}


Using the Leibniz rule (or Lemma~\ref{L:symb} above), one readily checks that the differential operator \eqref{E:dnabla} is filtration preserving.
Moreover, the operator induced on the associated graded is tensorial, given by a vector bundle homomorphism of bidegree $(1,0)$,
\begin{equation}\label{E:delomega}
\partial^\omega\colon C^k(M;E)_p\to C^{k+1}(M;E)_p,\qquad
\partial^\omega:=\gr(d^\nabla),
\end{equation}
which can be characterized as the unique extension of $C^0(M;E)\xrightarrow\omega C^1(M;E)$ satisfying the Leibniz rule
\begin{equation}\label{E:leibdel}
\partial^\omega(\alpha\wedge\psi)
=\partial\alpha\wedge\psi+(-1)^k\alpha\wedge \partial^\omega\psi,
\end{equation}
for all $\alpha\in\Lambda^k\mathfrak t^*_xM$ and $\psi\in C_x^*(M;E)$.
Here $\partial\colon\Lambda^k\mathfrak t^*M\to\Lambda^{k+1}\mathfrak t^*M$ denotes the fiber wise Chevalley--Eilenberg differential.



\begin{lemma}\label{L:curv}
For each $x\in M$ the following are equivalent:
\begin{enumerate}[(a)]
\item\label{L:curc:a} The curvature $F^\nabla_x\in\Omega_x^2(M;\eend(E))$ is contained in filtration degree one, that is, for all $X_i\in T_x^{p_i}M$ and $\psi\in E^p_x$ we have $F_x^\nabla(X_1,X_2)\psi\in E^{p+p_1+p_2+1}_x$.
\item\label{L:curc:b} $\omega_x\colon\mathfrak t_xM\to\eend(\gr(E_x))$ provides a graded representation of the graded nilpotent Lie algebra $\mathfrak t_xM$ on the graded vector space $\gr(E_x)$.
\item\label{L:curc:c} $(\partial_x^\omega)^2=0$.
\item\label{L:curc:d} $\tilde\sigma^0_x(d^\nabla)^2=0$.
\end{enumerate}
In this case, the Chevalley--Eilenberg differential of the Lie algebra $\mathfrak t_xM$ with coefficients in the representation $\gr(E_x)$ coincides with \eqref{E:delomega} at the point $x$.
\end{lemma}


\begin{proof}
Recall that $(d^\nabla)^2=F^\nabla\wedge-$ is tensorial.
Using using \eqref{E:tsAB} and Remark~\ref{R:grsigma}, we obtain
$$
\tilde\sigma^0_x(d^\nabla)^2
=\tilde\sigma_x^0((d^\nabla)^2)
%=\gr(F_x^\nabla\wedge-)
=\gr_x((d^\nabla)^2)
=\gr_x(d^\nabla)^2
=(\partial_x^\omega)^2,
$$
whence the equivalences \itemref{L:curc:a}$\Leftrightarrow$\itemref{L:curc:c}$\Leftrightarrow$\itemref{L:curc:d}.
The equivalence \itemref{L:curc:b}$\Leftrightarrow$\itemref{L:curc:c} is well known, and so are the remaining assertions.
\end{proof}


\begin{proposition}\label{P:hypo}
Let $\nabla\colon\Gamma^\infty(E)\to\Omega^1(M;E)$ be a filtration preserving linear connection on a filtered vector bundle $E$ over a filtered manifold $M$.
If the curvature of $\nabla$ is contained in filtration degree one, see Lemma~\ref{L:curv}, then 
\begin{equation}\label{E:dnablak}
\cdots\to\Omega^{k-1}(M;E)\xrightarrow{d^\nabla}\Omega^k(M;E)
\xrightarrow{d^\nabla}\Omega^{k+1}(M;E)\to\cdots
\end{equation}
is a graded Rockland sequence.
\end{proposition}


\begin{proof}
Fix $x\in M$, consider the nilpotent Lie group $G:=\mathcal T_xM$ with Lie algebra $\goe:=\mathfrak t_xM$ and the $\goe$-module $V:=\gr(E_x)$, see Lemma~\ref{L:curv}.
Hence, $C_x^k(M;E)=\Lambda^k\goe^*\otimes V$.
According to Lemma~\ref{L:symb}, the Heisenberg principal symbol sequence
$$
\cdots\to C^\infty(G,\Lambda^k\goe^*\otimes V)\xrightarrow{\tilde\sigma_x^0(d^\nabla)}
C^\infty(G,\Lambda^{k+1}\goe^*\otimes V)\to\cdots
$$
is isomorphic to the Chevalley--Eilenberg complex of the Lie algebra $\goe$ with values in the $\goe$-representation $C^\infty(G)\otimes V$.
More explicitley, if $X_j$ is a basis of $\goe$ and $\alpha^j$ denotes the dual basis of $\goe^*$ then, in $\mathcal U(\goe)\otimes\hom\bigl(\Lambda^k\goe^*\otimes V,\Lambda^{k+1}\goe^*\otimes V\bigr)$, we have
\begin{equation}\label{E:ChE}
\tilde\sigma_x^0(d^\nabla)=\sum_jX_j\otimes e_{\alpha^j}+\sum_j1\otimes e_{d\alpha^j}i_{X_j}+1\otimes\omega_x
\end{equation}
Here $e_{\alpha^j}\in\hom\bigl(\Lambda^k\goe^*\otimes V,\Lambda^{k+1}\goe^*\otimes V\bigr)$ denotes the exterior product with $\alpha^j\in\goe^*$; $i_{X_j}\in\hom\bigl(\Lambda^k\goe^*\otimes V,\Lambda^{k-1}\goe^*\otimes V\bigr)$ denotes the contraction with $X_j\in\goe$; $e_{d\alpha^j}\in\hom\bigl(\Lambda^{k-1}\goe^*\otimes V,\Lambda^{k+1}\goe^*\otimes V\bigr)$ denotes the exterior product with $d\alpha^j\in\Lambda^2\goe^*$; and $\omega_x\in\hom\bigl(\Lambda^k\goe^*\otimes V,\Lambda^{k+1}\goe^*\otimes V\bigr)$ denotes the exterior product with the representation $\omega_x\colon\goe\to\eend(V)$.

Suppose $\pi\colon G\to U(\mathcal H)$ is a non-trivial irreducible unitary representation on a Hilbert space $\mathcal H$, and let $\mathcal H_\infty$ denote the space of smooth vectors.
Using \eqref{E:ChE} one readily checks, see Section~\ref{SS:paraDO}, that the sequence
\begin{equation*}\label{E:tbse}
\cdots\to\mathcal H_\infty\otimes\Lambda^k\goe^*\otimes V\xrightarrow{\pi(\tilde\sigma^0_x(d^\nabla))}\mathcal H_\infty\otimes\Lambda^{k+1}\goe^*\otimes V\to\cdots
\end{equation*}
is isomorphic to the Chevalley--Eilenberg complex of the Lie algebra $\goe$ with values in the $\goe$-representation $\mathcal H_\infty\otimes V$.
Consequently, it remains to show that the Lie algebra cohomology of $\goe$ with coefficients in the $\goe$-module $\mathcal H_\infty\otimes V$ vanishes, that is, $H^*(\goe;\mathcal H_\infty\otimes V)=0$.


Since $\goe$ is nilpotent, there exists a $1$-dimensional subspace $W\subseteq V$ on which $\goe$ acts trivially.
The corresponding short exact sequence of $\goe$-modules, $0\to W\to V\to V/W\to0$, yields a short exact sequence of $\goe$-modules $0\to\mathcal H_\infty\to\mathcal H_\infty\otimes  V\to\mathcal H_\infty\otimes  V/W\to0$ which, in turn, gives rise to a long exact sequence:
$$
\cdots\to H^q(\goe;\mathcal H_\infty)\to H^q(\goe;\mathcal H_\infty\otimes V)\to H^q(\goe;\mathcal H_\infty\otimes V/W)\xrightarrow\partial H^{q+1}(\goe;\mathcal H_\infty)\to\cdots
$$
Hence, by induction on the dimension of $V$, it suffices to show $H^q(\goe;\mathcal H_\infty)=0$, for all $q$.
The statement thus follows from Lemma~\ref{L:Hnoe} below.
\end{proof}


\begin{lemma}\label{L:Hnoe}
Consider a non-trivial irreducible unitary representation of a finite dimensional simply connected nilpotent Lie group $G$ on a Hilbert space $\mathcal H$, and the associated representation of the corresponding Lie algebra $\goe$ on the space of smooth vectors $\mathcal H_\infty$.
Then the Lie algebra cohomology of $\goe$ with coefficients in $\mathcal H_\infty$ is trivial, that is, $H^*(\goe;\mathcal H_\infty)=0$.
\end{lemma}


\begin{proof}
The proof proceeds by induction on the dimension of $\goe$.
Since $\goe$ is nilpotent, there exists a 1-dimensional central subalgebra $\zoe\subseteq\goe$.
Recall that there is a Hochschild--Serre spectral sequence \cite{HS53} converging to $H^*(\goe;\mathcal H_\infty)$ with $E_2$-term 
$$
E_2^{p,q}\cong H^p\bigl(\goe/\zoe;H^q(\zoe;\mathcal H_\infty)\bigr).
$$
Since the representation of $G$ on the Hilbert space $\mathcal H$ is irreducible, $\zoe$ acts by scalars on $\mathcal H_\infty$, see \cite[Theorem~5 in Appendix~V]{K04}.
If this action is non-trivial, then $H^*(\zoe;\mathcal H_\infty)=0$ and, consequently, the $E_2$-term vanishes.
We may thus assume that the action of $\zoe$ on $\mathcal H_\infty$ is trivial. 
Hence, $H^*(\zoe;\mathcal H_\infty)=H^0(\zoe;\mathcal H_\infty)\oplus H^1(\zoe;\mathcal H_\infty)\cong\mathcal H_\infty\oplus\mathcal H_\infty$ as $\goe/\zoe$-modules.


Consider the closed connected central ideal $Z:=\exp(\zoe)$ in $G$, and note that $G/Z$ is a simply connected nilpotent Lie group with Lie algebra $\goe/\zoe$.
Since $\mathcal H_\infty$ is dense in the Hilbert space $\mathcal H$, the ideal $Z$ acts trivially on $\mathcal H$, see \cite[Theorem~4 in Appendix~V]{K04}.
Hence the representation of $G$ factors through a representation of $G/Z$ on $\mathcal H$.
Clearly, this is a non-trivial irreducible unitary representation of $G/Z$ whose space of smooth vectors coincides with $\mathcal H_\infty$.
Hence, $H^*(\goe/\zoe;\mathcal H_\infty)=0$, by induction.
We conclude $H^p(\goe/\zoe;H^*(\zoe;\mathcal H_\infty))\cong H^p(\goe/\zoe;\mathcal H_\infty\oplus\mathcal H_\infty)=H^p(\goe/\zoe;\mathcal H_\infty)\oplus H^p(\goe/\zoe;\mathcal H_\infty)=0$.
Thus the $E_2$-term vanishes, and the proof is complete.
\end{proof}






\subsection{Subcomplexes of the de~Rham complex}\label{SS:subdeRham}






In this section we apply the observations and constructions from Sections~\ref{SS:DP} and \ref{SS:splitting-operators} to the de~Rham sequence associated with a filtration preserving linear connection.
This directly leads to the main result of this section, see Theorem~\ref{T:D} and Corollary~\ref{C:D} below.
In the subsequent Section~\ref{SS:BGG} we will apply this to the curved BGG sequences in parabolic geometry which appear as a special case of the sequences considered here.
In this section we will discuss how the Rumin complex fits into our picture and present a graded Rockland sequence for Engel structures.


Let $\nabla$ be a filtration preserving linear connection on a filtered vector bundle $E$ over a filtered manifold $M$, and consider its extension to $E$-valued differential forms characterized by the Leibniz rule in \eqref{E:leibn},
\begin{equation}\label{E:Tdnabla}
\cdots\to\Omega^{k-1}(M;E)\xrightarrow{d^\nabla_{k-1}}\Omega^k(M;E)\xrightarrow{d^\nabla_k}\Omega^{k+1}(M;E)\to\cdots.
\end{equation}
Consider a sequence of differential operators
\begin{equation}\label{E:Tdelta}
\cdots\leftarrow\Omega^{k-1}(M;E)\xleftarrow{\delta_k}\Omega^k(M;E)\xleftarrow{\delta_{k+1}}\Omega^{k+1}(M;E)\leftarrow\cdots
\end{equation}
which are of graded Heisenberg order at most zero.
Then the differential operator
$$
\Box_k\colon\Omega^k(M;E)\to\Omega^k(M;E),\qquad
\Box_k:=d^\nabla_{k-1}\delta_k+\delta_{k+1}d^\nabla_k,
$$
is of graded Heisenberg order at most zero.
Let 
\begin{align*}
\partial_k^\omega&\colon C^k(M;E)\to C^{k+1}(M;E),&
\partial_k^\omega&:=\gr(d^\nabla_k),
\\
\tilde\delta_k&\colon C^k(M;E)\to C^{k-1}(M;E),&
\tilde\delta_k&:=\gr(\delta_k),
\\
\tilde\Box_k&\colon C^k(M;E)\to C^k(M;E),&
\tilde\Box_k&:=\gr(\Box_k)=\partial^\omega_{k-1}\tilde\delta_k+\tilde\delta_{k+1}\partial^\omega_k,
\end{align*}
denote the associated graded vector bundle homomorphisms, see Remark~\ref{R:grsigma}, \eqref{E:CkME}, and \eqref{E:delomega}.


For each $x\in M$, let $\tilde P_{k,x}\colon C_x^k(M;E)\to C_x^k(M;E)$ denote the spectral projection onto the generalized zero eigenspace of $\tilde\Box_{x,k}\colon C_x^k(M;E)\to C^k_x(M;E)$, the restriction of $\tilde\Box_k$ to the fiber over $x$.
Assume that the rank of $\tilde P_{k,x}$ is locally constant in $x$ for each $k$.
Then, see Lemma~\ref{L:EP}(a), $\tilde P_k\colon C^k(M;E)\to C^k(M;E)$ is a smooth vector bundle homomorphism,
$\tilde P_k^2=\tilde P_k$, $\tilde P_k\tilde\Box_k=\tilde\Box_k\tilde P_k$,  and we obtain a decomposition of graded vector bundles,
\begin{equation}\label{E:tpdeco2}
C^k(M;E)=\img(\tilde P_k)\oplus\ker(\tilde P_k),
\end{equation}
which is invariant under $\tilde\Box_k$.
Moreover, $\tilde\Box_k$ is nilpotent on $\img(\tilde P_k)$ and invertible on $\ker(\tilde P_k)$.


According to Lemma~\ref{L:EP}(b), there exists a unique filtration preserving differential operator
$$
P_k\colon\Omega^k(M;E)\to\Omega^k(M;E)
$$ 
such that $P_k^2=P_k$, $P_k\Box_k=\Box_kP_k$, and $\gr(P_k)=\tilde P_k$.
This operator $P_k$ has graded Heisenberg order at most zero and provides a decomposition of filtered vector spaces, 
\begin{equation}\label{E:Pdeco2}
\Omega^k(M;E)=\img(P_k)\oplus\ker(P_k),
\end{equation}
invariant under $\Box_k$ and such that $\Box_k$ is nilpotent on $\img(P_k)$ and invertible on $\ker(P_k)$.
If
$$
C^k(M,E)=\gr(\Lambda^kT^*M\otimes E)\xrightarrow{S_k}\Lambda^kT^*M\otimes E,
$$ 
are splittings of the filtrations, then according to Lemma~\ref{L:EP}(c)
\begin{equation}\label{E:tL2}
L_k\colon\Gamma^\infty(C^k(M;E))\to\Omega^k(M;E),\qquad
L_k:=P_kS_k\tilde P_k+(\id-P_k)S_k(\id-\tilde P_k),
\end{equation}
is an invertible differential operator of graded Heisenberg order at most zero such that $\gr(L_k)=\id$ and $L^{-1}_kP_kL_k=\tilde P_k$.
Hence, $L_k$ induces filtration preserving isomorphisms
$$
L_k\colon\Gamma^\infty(\img(\tilde P_k))\xrightarrow\cong\img(P_k)
\qquad\text{and}\qquad
L_k\colon\Gamma^\infty(\ker(\tilde P_k))\xrightarrow\cong\ker(P_k).
$$
Moreover, $L^{-1}_k\Box_kL_k$ is a differential operator of graded Heisenberg order at most zero preserving the decomposition \eqref{E:tpdeco2} and satisfying $\gr(L^{-1}_k\Box_kL_k)=\tilde\Box_k$.
Putting 
\begin{equation}\label{E:defDB2}
D_k:=\tilde P_{k+1}L_{k+1}^{-1}d^\nabla_kL_k|_{\Gamma^\infty(\img(\tilde P_k))}
\qquad\text{and}\qquad
B_k:=(\id-\tilde P_{k+1})L_{k+1}^{-1}d^\nabla_kL_k|_{\Gamma^\infty(\ker(\tilde P_k))}
\end{equation}
we obtain two sequences of differential operators,
\begin{equation}\label{E:TEDE}
\cdots\to\Gamma^\infty(\img(\tilde P_{k-1}))\xrightarrow{D_{k-1}}\Gamma^\infty(\img(\tilde P_k))\xrightarrow{D_k}\Gamma^\infty(\img(\tilde P_{k+1}))\to\cdots
\end{equation}
and
\begin{equation}\label{E:TEBE}
\cdots\to\Gamma^\infty(\ker(\tilde P_{k-1}))\xrightarrow{B_{k-1}}\Gamma^\infty(\ker(\tilde P_k))\xrightarrow{B_k}\Gamma^\infty(\ker(\tilde P_{k+1}))\to\cdots,
\end{equation}
all of which have graded Heisenberg order at most zero.


Combining Proposition~\ref{P:DB} with Proposition~\ref{P:hypo}, we immediately obtain:


\begin{theorem}\label{T:D}
In this situation the following hold true:
\begin{enumerate}[(a)]
\item
If the curvature of $\nabla$ is contained in filtration degree one, cf.\ Lemma~\ref{L:curv}, then \eqref{E:TEDE} and \eqref{E:TEBE} are both graded Rockland sequences.
\item
If the curvature of\/ $\nabla$ vanishes, then the operator $L_{k+1}^{-1}d^\nabla_kL_k$ decouples,
$$
L^{-1}_{k+1}d^\nabla_kL_k=D_k\oplus B_k,
$$ 
and $D_kD_{k-1}=0=B_kB_{k-1}$ for all $k$.
In this situation, $G_k\colon\Gamma^\infty(\ker(\tilde P_k))\to\Gamma^\infty(\ker(\tilde P_k))$,
$$
G_k:=B_{k-1}(\id-\tilde P_{k-1})\tilde\delta_k\tilde\Box_k^{-1}
+(\id-\tilde P_k)\tilde\delta_{k+1}\tilde\Box_{k+1}^{-1}\partial^\omega_k,
$$ 
is an invertible differential operator of graded Heisenberg order at most zero with $\gr(G_k)=\id$ that conjugates the complex \eqref{E:TEBE} into an acyclic tensorial complex, that is, 
$$
G_{k+1}^{-1}B_kG_k=\partial^\omega_k|_{\Gamma^\infty(\ker(\tilde P_k))}.
$$
Moreover, the restriction $L_k\colon\Gamma^\infty(\img(\tilde P_k))\to\Omega^k(M;E)$ provides a chain map, $d^\nabla_kL_k|_{\Gamma^\infty(\img(\tilde P_k))}=L_{k+1}D_k$, inducing an isomorphism between the cohomologies of \eqref{E:TEDE} and \eqref{E:Tdnabla}.
More precisely, $\pi_k:=\tilde P_kL_k^{-1}\colon\Omega^k(M;E)\to\Gamma^\infty(\img(\tilde P_k))$, is a chain map, $D_k\pi_k=\pi_{k+1}d^\nabla_k$, which is inverse up to homotopy, i.e., $\pi_kL_k|_{\Gamma^\infty(\img(\tilde P_k))}=\id$ and $\id-L_k\pi_k=d^\nabla_{k-1}h_k+h_{k+1}d^\nabla_k$ where the homotopy $h_k\colon\Omega^k(M;E)\to\Omega^{k-1}(M;E)$ is a differential operator of graded Heisenberg order at most zero given by $h_k:=L_{k-1}G_{k-1}(\id-\tilde P_k)\tilde\delta_k\tilde\Box_k^{-1}G_k^{-1}(\id-\tilde P_k)L_k^{-1}$.
\end{enumerate}
\end{theorem}


Combining Proposition~\ref{P:D} with Proposition~\ref{P:hypo} we immediately obtain:


\begin{corollary}\label{C:D}
Let $\nabla$ be a filtration preserving linear connection on a filtered vector bundle $E$ over a filtered manifold $M$ whose curvature is contained in filtration degree one, and consider its extension to $E$-valued differential forms characterized by the Leibniz rule, see~\eqref{E:leibn},
\begin{equation}\label{E:Cdnabla}
\cdots\to\Omega^{k-1}(M;E)\xrightarrow{d^\nabla}\Omega^k(M;E)\xrightarrow{d^\nabla}\Omega^{k+1}(M;E)\to\cdots.
\end{equation}
Moreover, consider a codifferential of Kostant type, see Definition~\ref{D:Kdelta},
$$
\cdots\leftarrow\Lambda^{k-1}T^*M\otimes E\xleftarrow{\delta_i}\Lambda^kT^*M\otimes E\xleftarrow{\delta_{i+1}}\Lambda^{k+1}T^*M\otimes E\leftarrow\cdots,
$$
which has locally constant rank, and let $\bar\pi_k\colon\ker(\delta_k)\to\mathcal H_k:=\ker(\delta_k)/\img(\delta_{k+1})$ denote the canonical vector bundle projection.
Then the following hold true:
\begin{enumerate}[(a)]
\item
There exists a unique differential operator $\bar L_k\colon\Gamma^\infty(\mathcal H_k)\to\Omega^k(M;E)$ such that $\delta_k\bar L_k=0$, $\bar\pi_k\bar L_k=\id$, and $\delta_{k+1}d^\nabla_k\bar L_k=0$.
Moreover, $\bar L_k$ is of graded Heisenberg order at most zero.
\item
The differential operator
$$
\bar D_k\colon\Gamma^\infty(\mathcal H_k)\to\Gamma^\infty(\mathcal H_{k+1}),\qquad\bar D_k:=\bar\pi_{k+1}d^\nabla_kL_k,
$$ 
is of graded Heisenberg order at most zero, and 
\begin{equation}\label{E:bD}
\cdots\to\Gamma^\infty(\mathcal H_{k-1})\xrightarrow{\bar D_{k-1}}\Gamma^\infty(\mathcal H_k)\xrightarrow{\bar D_k}\Gamma^\infty(\mathcal H_{k+1})\to\cdots
\end{equation}
is graded Rockland sequence.
\item
If $\nabla$ has vanishing curvature, then $\bar D_k\bar D_{k-1}=0$, and $\bar L_k\colon\Gamma^\infty(\mathcal H_k)\to\Omega^k(M;E)$ provides a chain map, $d^\nabla_k\bar L_k=\bar L_{k+1}\bar D_k$, inducing an isomorphism between the cohomologies of \eqref{E:bD} and \eqref{E:Cdnabla}.
\item
There exist invertible differential operators, $V_i\colon\Gamma^\infty(\img(\tilde P_i))\to\Gamma^\infty(\mathcal H_i)$ with inverse $V_i^{-1}\colon\Gamma^\infty(\mathcal H_i)\to\Gamma^\infty(\img(\tilde P_i))$, both of graded Heisenberg order at most zero, such that 
$$
V_{i+1}^{-1}\bar D_iV_i=D_i,
$$ 
where $D_i\colon\Gamma^\infty(\img(\tilde P_i))\to\Gamma^\infty(\img(\tilde P_{i+1}))$ denotes the operator considered in Theorem~\ref{T:D}, see \eqref{E:defDB2}, corresponding to splittings $S_k$ as in Definition~\ref{D:Kdelta}(iii).
\end{enumerate}
\end{corollary}


The operators $\bar L_k$ and $\bar D_k$ in Corollary~\ref{C:D} are direct generalizations of the splitting operators and the curved BGG operators in parabolic geometry, respectively.
This aspect will be addressed in Section~\ref{SS:BGG} below.
The differential projector $P_k$ generalizes the operator obtained by composing (5.1) with (5.2) in \cite{CD01}.


As a first example we will now explain how Rumin's complex \cite{R90,R94} appears among the sequences \eqref{E:bD} considered in Corollary~\ref{C:D}.
Hypoellipticity of this sequence has been established by Rumin, see \cite[Section~3]{R94}.
Let us mention that Rumin and Seshadri \cite{RS12} have introduced an analytic torsion based on hypoelliptic Laplacians associated with Rumin's complex.
We expect that their construction can be be extended to all (ungraded) sequences appearing in Corollary~\ref{C:D}(b).


We will present an approach to Rumin's complex which uses a contact form and the associated splitting of the contact distribution.
This leads to differential operators $L_k$ providing a decomposition of the de~Rham complex as in Theorem~\ref{T:D}(b).
The Rumin complex appears as a direct summand in the de~Rham complex, the complementary complex is conjugate to an acyclic tensorial complex.


\begin{example}[Rumin complex]\label{Ex:Rumin}
Let $\eta$ be a contact form on a manifold $M$ of dimension $2n+1$.
The contact distribution turns $M$ into a filtered manifold, 
$$
TM=T^{-2}M\supseteq T^{-1}M\supseteq T^0M=0,\qquad\text{where}\qquad T^{-1}M:=H:=\ker(\eta).
$$
The induced filtration on the de~Rham complex takes has form
$$
\Omega^k(M)=\Omega^k(M)^k\supseteq\Omega^k(M)^{k+1}\supseteq\Omega^k(M)^{k+2}=0
$$
where $\Omega^k(M)^{k+1}=\{\alpha\in\Omega^k(M):\iota^*\alpha=0\}$ denotes the space of differential forms which vanish on the contact distribution.
Here $\iota\colon H\to TM$ denotes the inclusion and $\iota^*\colon\Omega^k(M)\to\Gamma^\infty(\Lambda^kH^*)$ the corresponding restriction.
The contact form provides a splitting for this filtration and gives rise to an identification
\begin{equation}\label{E:contactforms}
\Omega^k(M)\cong\Gamma^\infty(\Lambda^kH^*)\oplus\Gamma^\infty(\Lambda^{k-1}H^*),\qquad\alpha\mapsto(\iota^*\alpha,\iota^*i_X\alpha).
\end{equation}
Here $X\in\mathfrak X(M)$ denotes the Reeb vector field associated with $\eta$, that is, the unique vector field such that $\eta(X)=1$ and $i_Xd\eta=0$.
Via the identification \eqref{E:contactforms}, the non-trivial filtration subspace $\Omega^k(M)^{k+1}$ corresponds to the second factor, $\Gamma^\infty(\Lambda^{k-1}H^*)$.


To describe the de~Rham differential, put $\omega:=\iota^*d\eta\in\Gamma^\infty(\Lambda^2H^*)$ and let 
$$
e_k\colon\Lambda^kH^*\to\Lambda^{k+2}H^*,\qquad
e_k\psi:=\omega\wedge\psi,
$$ 
denote the vector bundle homomorphism given by wedge product with $\omega$.
Furthermore, let 
$$
a_k\colon\Gamma^\infty(\Lambda^kH^*)\to\Gamma^\infty(\Lambda^{k+1}H^*),\qquad a_k\psi:=\iota^*d\tilde\psi,
$$ 
where $\tilde\psi\in\Omega^k(M)$ denotes the unique form such that $\iota^*\tilde\psi=\psi$ and $i_X\tilde\psi=0$. 
One readily verifies that $a_k$ is a differential operator of Heisenberg order at most one.
Moreover, let 
$$
b_k\colon\Gamma^\infty(\Lambda^kH^*)\to\Gamma^\infty(\Lambda^kH^*),\qquad b_k\psi:=L_X\psi,
$$ 
denote the operator given by Lie derivative with respect to the Reeb vector field.
Note that the flow of the Reeb vector field preserves the contact bundle and, thus, gives rise to a Lie derivative on sections of $\Lambda^kH^*$ as indicated.
Also note that $b_k$ is a differential operator of Heisenberg order at most two.
Via the identification \eqref{E:contactforms}, the de~Rham differential $d_k\colon\Omega^k(M)\to\Omega^{k+1}(M)$ takes the form, see \cite[Section~2]{R00},
$$
d_k=\begin{pmatrix}a_k&e_{k-1}\\b_k&-a_{k-1}\end{pmatrix}.
$$


The contact form also provides an codifferential $\delta_k\colon\Omega^k(M)\to\Omega^{k-2}(M)$.
Indeed, since $\omega\in\Lambda^2H^*$ is non-degenerate in each fiber, it provides an isomorphism of vector bundles, $\sharp\colon H^*\cong H$, which leads to the symplectic dual, $\sharp\omega\in\Gamma^\infty(\Lambda^2H)$. 
Contraction with $\sharp\omega$ provides a vector bundle homomorphism 
$$
f_k\colon\Lambda^kH^*\to\Lambda^{k-2}H^*,\qquad f_k\psi:=i_{\sharp\omega}\psi.
$$
Via the identification \eqref{E:contactforms}, the codifferential is given by the matrix
$$
\delta_k:=\begin{pmatrix}0&0\\f_k&0\end{pmatrix}.
$$
Via the identification \eqref{E:contactforms} we have, cf.\ Example~\ref{Ex:preRumin},
$$
\tilde\Box_k=\begin{pmatrix}e_{k-2}f_k&0\\0&f_{k+1}e_{k-1}\end{pmatrix}.
$$
To describe the generalized zero eigenspaces of $\tilde\Box$ recall that the action of $e$ and $f$ turn the fibers of $\bigoplus_k\Lambda^kH^*$ into $\mathfrak s\mathfrak l_2$-modules, that is, $e_{k-2}f_k-f_{k+2}e_k=(k-n)\id$.
Hence, the vector bundle $\bigoplus_k\Lambda^kH^*$ decomposed canonically into isotypical $\mathfrak s\mathfrak l_2$-components. This decomposition gives rise to canonical vector bundle projections 
$$
\tilde P'_k\colon\Lambda^kH^*\to\ker(f_k)
\qquad\text{and}\qquad
\tilde P''_k\colon\Lambda^kH^*\to\ker(e_k)
$$ 
onto the primitive elements.
One readily checks that 
$$
\tilde P_k=\begin{pmatrix}\tilde P'_k&0\\0&\tilde P''_{k-1}\end{pmatrix}
$$
coincides with the fiber wise spectral projection onto the generalized zero eigenspace of $\tilde\Box_k$.
We conclude that $\delta_k$ is a Kostant type codifferential of constant rank, see Definition~\ref{D:Kdelta}.
Note that $\tilde P'_k=0$ for all $k>n$, $\tilde P'_n=\tilde P''_n$, and $\tilde P''_k=0$ for all $k<n$.
Thus, the sequence \eqref{E:TEDE} is ungraded, taking the form:
\begin{multline}\label{E:RuminD}
0\to\Gamma^\infty(\ker(f_0))\xrightarrow{D_0}
\Gamma^\infty(\ker(f_1))\xrightarrow{D_1}
\Gamma^\infty(\ker(f_2))\to\cdots
\\
\cdots\to\Gamma^\infty(\ker(f_{n-1}))\xrightarrow{D_{n-1}}
\Gamma^\infty(\ker(f_n))\xrightarrow{D_n}
\Gamma^\infty(\ker(e_n))\xrightarrow{D_{n+1}}
\Gamma^\infty(\ker(e_{n+1}))\to\cdots
\\
\cdots\to\Gamma^\infty(\ker(e_{2n-2}))\xrightarrow{D_{2n-1}}
\Gamma^\infty(\ker(e_{2n-1}))\xrightarrow{D_{2n}}
\Gamma^\infty(\ker(e_{2n}))\to0.
\end{multline}
From Example~\ref{Ex:preRumin} we obtain
$$
P_k=\begin{pmatrix}\tilde P'_k&0\\a_{k-2}e^\dag_k-e^\dag_{k+1}a_k&\tilde P''_{k-1}\end{pmatrix}\qquad\text{and}\qquad
L_k=\begin{pmatrix}\id&0\\-\tilde P''_{k-1}a_{k-2}e^\dag_k-e^\dag_{k+1}a_k\tilde P'_k&\id\end{pmatrix}
$$
where the vector bundle homomorphism $e_k^\dag\colon\Lambda^kH^*\to\Lambda^{k-2}H^*$ is uniquely characterized by $e_{k-2}e^\dag_k=\tilde Q'_k:=\id-\tilde P'_k$, $e^\dag_{k+2}e_k=\tilde Q''_k:=\id-\tilde P''_k$ and $e^\dag_k\tilde P'_k=0=\tilde P''_{k-2}e^\dag_k$.
Moreover,
\begin{equation}\label{E:RuminDD}
D_k=\begin{cases}
\tilde P'_{k+1}a_k|_{\Gamma^\infty(\ker(f_k))}&\text{for $0\leq k<n$,}
\\
\tilde P''_n(a_{n-1}e^\dag_{n+1}a_n+b_n)|_{\Gamma^\infty(\ker(f_n))}&\text{for $k=n$, and}
\\
-\tilde P''_ka_{k-1}|_{\Gamma^\infty(\ker(e_{k-1}))}&\text{for $n<k\leq2n$.}
\end{cases}
\end{equation}

Using the identification \eqref{E:contactforms} to define invertible differential operators, 
$$
U_k\colon\Gamma^\infty(\Lambda^kH^*)\oplus\Gamma^\infty(\Lambda^{k-1}H^*)\to\Omega^k(M),\qquad
U_k:=\begin{pmatrix}\id&0\\-a_{k-2}e^\dag_k-e^\dag_{k+1}a_k\tilde P'_k&\id\end{pmatrix},
$$
we find
$$
U_{k+1}^{-1}d_kU_k=\underbrace{\begin{pmatrix}\tilde P_{k+1}'a_k\tilde P_k'&0\\\tilde P_k''(a_{k-1}e^\dag_{k+1}a_k+b_k)\tilde P'_k&-\tilde P''_ka_{k-1}\tilde P''_{k-1}\end{pmatrix}}_{D_k}
+\begin{pmatrix}0&\tilde Q'_{k+1}e_{k-1}\tilde Q''_{k-1}\\0&0\end{pmatrix}
$$
In other words, conjugating the de~Rham complex with the differential operators $U_k$ above, it decomposes as a direct sum of the complex \eqref{E:RuminD} and the acyclic tensorial complex:
$$
\cdots\to\Gamma^\infty\bigl(\ker(\tilde P_{k-1})\bigr)
\xrightarrow{\begin{pmatrix}0&e_{k-1}\\0&0\end{pmatrix}}
\Gamma^\infty\bigl(\ker(\tilde P_k)\bigr)
\xrightarrow{\begin{pmatrix}0&e_k\\0&0\end{pmatrix}}
\Gamma^\infty\bigl(\ker(\tilde P_{k+1})\bigr)
\to\cdots.
$$


Up to a conformal factor, the codifferential $\delta_k\colon\Omega^k(M)\to\Omega^{k-1}(M)$ considered above does not depend on the contact form $\eta$, but only on the contact distribution $H$. 
Consequently, the vector bundle projections $\bar\pi_k\colon\ker(\delta_k)\to\mathcal H_k:=\ker(\delta_k)/\img(\delta_{k+1})$ and the splitting operator $\bar L_k\colon\Gamma^\infty(\mathcal H_k)\to\Omega^k(M)$ in Corollary~\ref{C:D}(a) only depend on the contact distribution too.
Hence, the corresponding Rockland complex, see \eqref{E:bD} in Corollary~\ref{C:D}(b),
$$
\cdots\to\Gamma^\infty(\mathcal H_{k-1})\xrightarrow{\bar D_{k-1}}\Gamma^\infty(\mathcal H_k)\xrightarrow{\bar D_k}\Gamma^\infty(\mathcal H_{k+1})\to\cdots,
$$
$\bar D_k=\bar\pi_{k+1}d_k\bar L_k$, only depends on the contact distribution.
One readily verifies that this complex is canonically isomorphic to Rumin's complex, see \cite{R00}.
By Corollary~\ref{C:D}(d) the latter is conjugate to the complex \eqref{E:RuminD}.
Actually, a description of Rumin's complex in the form \eqref{E:RuminD} with differential \eqref{E:RuminDD} is implicit in \cite[Section~3]{R00}.
\end{example}


We conclude this section with a 4-dimensional geometry for which the sequence in Theorem~\ref{T:D}(b) turns out to be graded Rockland but not Rockland in the ungraded sense.


\begin{example}[Engel structures]\label{Ex:Engel}
Recall that an Engel structure \cite{P16} on a smooth 4-manifold $M$ is a smooth rank two distribution $T^{-1}M\subseteq TM$ with growth vector $(2,3,4)$.
More explicitly, Lie brackets of sections of $T^{-1}M$ generate a rank three bundle $T^{-2}M$, and triple brackets of sections of $T^{-1}M$ generate all of $TM$.
Hence, $M$ is a filtered manifold,
$$
TM=T^{-3}M\supseteq T^{-2}M\supseteq T^{-1}M\supseteq T^0M=0.
$$
One readily checks that the bundle of osculating algebras is locally trivial with typical fiber isomorphic to the graded nilpotent Lie algebra $\goe=\goe_{-3}\oplus\goe_{-2}\oplus\goe_{-1}$ with non-trivial brackets 
$$
[X_1,X_2]=X_3,\qquad [X_1,X_3]=X_4,\qquad [X_2,X_3]=0,
$$ 
where $X_1,X_2$ is a basis of $\goe_{-1}$, $X_3$ is a basis of $\goe_{-2}$ and $X_4$ is a basis of $\goe_{-3}$.
If $G$ denotes the simply connected nilpotent Lie group with Lie algebra $\goe$, then the left invariant distribution corresponding to $\goe_{-1}$ provides an Engel structure on $G$.
Locally, every Engel structure is diffeomorphic to this left invariant structure.
According to a result of Vogel \cite{V09}, every closed parallelizable smooth 4-manifold admits an (orientable) Engel structure.


We will exhibit explicit formulas for the graded Heisenberg principal symbol of the operators $D_k$ in Theorem~\ref{T:D} corresponding to the de~Rham complex, that is, the trivial flat line bundle $E$. 
We work on $M=G$ equipped with the left invariant Engel structure mentioned before.
We use the left invariant splitting for $S\colon\gr(\Lambda^kT^*G)\to\Lambda^kT^*G$ provided by the decomposition $\goe=\goe_{-3}\oplus\goe_{-2}\oplus\goe_{-1}$.
Moreover, we consider the left invariant codifferential $\delta_k\colon\Lambda^kT^*G\to\Lambda^{k-1}T^*G$ which, at the identity, is dual to the Chevalley--Eilenberg differential $\partial_{k-1}\colon\Lambda^{k-1}\goe^*\to\Lambda^k\goe^*$ with respect to the basis $X_1,\dotsc,X_4$. 
Clearly, this is a Kostant type codifferential of maximal rank.
In this situation, Theorem~\ref{T:D}(b) provides a graded Rockland complex of left invariant operators, see also \cite{BENG12}:
\begin{equation}\label{E:DEngel}
C^\infty(G)\xrightarrow{D_0}
C^\infty(G)^2\xrightarrow{D_1}
\begin{array}{c}C^\infty(G)\\\oplus\\ C^\infty(G)\end{array}\xrightarrow{D_2}
C^\infty(G)^2\xrightarrow{D_3}
C^\infty(G).
\end{equation}
Using matrices with entries in the universal enveloping algebra of $\goe$, and using the notation $X_{i_1\dotsc i_k}=X_{i_1}\cdots X_{i_k}$, these operators can be expressed as:
$$
D_0=\left(\begin{array}{c}X_1\\X_2\end{array}\right)
$$

$$
D_1=\left(\begin{array}{cc}-X_{22}&X_{12}-2X_3\\\hline-X_4-X_{112}-X_{13}&X_{111}\end{array}\right)
$$

$$
D_2=\left(\begin{array}{c|c}X_{111}&-X_3-X_{12}\\ 3X_4-3X_{13}+X_{112}&-X_{22}\end{array}\right)
$$

$$
D_3=\left(\begin{array}{cc}-X_2&X_1\end{array}\right)
$$
A detailed verification of these claims can be found in Appendix~\ref{A:Engel}.
The direct sum symbol at the middle space in the sequence \eqref{E:DEngel} indicates that its filtration is non-trivial.
Correspondingly, the operators $D_1$ and $D_2$ are not homogeneous, whence the lines in their matrices separating different degrees.
Evidently, the sequence \eqref{E:DEngel} fails to be Rockland in the ungraded sense.
\end{example}






\subsection{BGG sequences}\label{SS:BGG}





Every regular parabolic geometry has an underlying filtered manifold $M$ whose bundle of osculating Lie algebras is locally trivial.
The Cartan connection induces a linear connection $\nabla$ on every associated tractor bundle $E$ over $M$.
This linear connection is filtration preserving and its curvature is contained in filtration degree one.
Moreover, Kostant's codifferential provides a vector bundle homomorphism $\delta\colon\Lambda^kT^*M\otimes E\to\Lambda^{k-1}T^*M\otimes E$, satisfying the assumptions in Corollary~\ref{C:D}.
In this situation, the operator $\bar L$ in Corollary~\ref{C:D}(a) coincides with the splitting operator in \cite[Theorem~2.4]{CS12}, see also \cite{CSS01,CD01}.
Furthermore, the sequence in Corollary~\ref{C:D}(b) reduces to the  ``torsion free BGG sequence'' in \cite[Section~5]{CD01} and coincides with the sequence considered in \cite[Section~2.4]{CS12}.
For torsion free parabolic geometries this coincides with the original curved BGG sequence constructed by \v Cap, Slov\'ak and Sou\v cek in \cite{CSS01}.
From Corollary~\ref{C:D}(b) we conclude that these BGG operators have graded Heisenberg order at most zero and form a graded Rockland sequence, see Corollary~\ref{C:BGG} below.


Let us point out that there is a normalization condition for the curvature of the Cartan connection such that normal regular parabolic geometries can equivalently be described by underlying geometric structures, see Theorem~3.1.14, Section~3.1.16, and the historical remarks at the end of Section~3 in \cite{CS09}.
For a large class of parabolic geometries this underlying geometric structure consists merely of the underlying filtered manifold, \cite[Proposition~4.3.1]{CS09}.
These provide intriguing classes of filtered manifolds to which one can associate graded Rockland sequences of differential operators in a natural way.


In the remaining part of this section we will, for the reader's convenience, briefly recall basic facts on parabolic geometries and provided detailed references supporting the claims made above.
We closely follow the presentation in \cite{CS09}, \cite{CSS01} and \cite[Section~2]{CS12}.


Consider a $|k|$-graded semisimple Lie algebra
\begin{equation}\label{E:grading}
\goe=\goe_{-k}\oplus\cdots\oplus\goe_{-1}\oplus\goe_0\oplus\goe_1\oplus\cdots\oplus\goe_k.
\end{equation}
More precisely, $[\goe_i,\goe_j]\subseteq\goe_{i+j}$ for all $i,j$, and the subalgebra $\goe_-:=\goe_{-k}\oplus\cdots\oplus\goe_{-1}$ is generated by $\goe_{-1}$, see \cite[Definition~3.1.2]{CS09}.
Consider the filtration 
\begin{equation}\label{E:filt}
\goe=\goe^{-k}\supseteq\goe^{-k+1}\supseteq\cdots\supseteq\goe^{-1}\supseteq\goe^0\supseteq\goe^1\supseteq\cdots\supseteq\goe^{k-1}\supseteq\goe^k
\end{equation}
where $\goe^i:=\goe_i\oplus\cdots\oplus\goe_k$.
The subalgebras $\goe_0$ and $\poe:=\goe^0=\goe_0\oplus\cdots\oplus\goe_k$ can be characterized as grading and filtration preserving subalgebras, respectively.
More precisely, $\goe_0=\{X\in\goe\mid\forall i:\ad_X(\goe_i)\subseteq\goe_i\}$ and $\poe=\{X\in\goe\mid\forall i:\ad_X(\goe^i)\subseteq\goe^i\}$, see\cite[Lemma~3.1.3(1)]{CS09}.
Also note that $\poe_+:=\goe^1=\goe_1\oplus\cdots\oplus\goe_k$ is a nilpotent ideal in $\poe$.


Let $G$ be a not necessarily connected Lie group with Lie algebra $\goe$.
Then
\begin{equation}\label{E:Pt}
\{g\in G\mid\forall i:\Ad_g(\goe^i)\subseteq\goe^i\}
\end{equation}
is a closed subgroup of $G$ with Lie algebra $\poe$, see \cite[Lemma~3.1.3(2)]{CS09}.
Let $P$ be a parabolic subgroup of $G$ corresponding to the $|k|$-grading \eqref{E:grading}, i.e.\ a subgroup between \eqref{E:Pt}
and its connected component, see \cite[Definition~3.1.3]{CS09}. Hence, $P$ has Lie algebra $\poe$, and
the corresponding Levi subgroup,
$$
G_0:=\{g\in P\mid\forall i:\Ad_g(\goe_i)\subseteq\goe_i\},
$$
has Lie algebra $\goe_0$. According to \cite[Theorem~3.1.3]{CS09} we have a diffeomorphism
\begin{equation}\label{E:PG0iso}
G_0\times\poe_+\cong P,\qquad (g,X)\mapsto g\exp(X).
\end{equation}
In particular, $P_+:=\exp(\poe_+)$ is a closed normal nilpotent subgroup of $P$, and the inclusion $G_0\subseteq P$ induces a canonical isomorphism $G_0=P/P_+$.
Note that $P_+$ acts trivially on the associated graded, $\gr(\goe)$, of the filtration \eqref{E:filt}. 
In particular, $\gr(\goe/\poe)$ can be considered as a representation of $P/P_+=G_0$. 
The inclusion $\goe_-\subseteq\goe$ induces a canonical isomorphism of $G_0$-modules,
\begin{equation}\label{E:grgp}
\goe_-=\gr(\goe/\poe).
\end{equation}


A parabolic geometry of type $(G,P)$ consists of a principal $P$-bundle $p\colon\mathcal G\to M$ and a Cartan connection $\omega\in\Omega^1(\mathcal G;\goe)$, see \cite[Definition~3.1.4 and Section~1.5]{CS09}.
Hence, the $\goe$-valued $1$-form $\omega$ provides a $P$-equivariant trivialization of the tangent bundle $T\mathcal G$, that is, $\omega_u\colon T_u\mathcal G\to\goe$ is a linear isomorphism for each $u\in\mathcal G$, and $(r^g)^*\omega=\Ad_{g^{-1}}\omega$ for all $g\in P$,
where $r^g$ denotes the principal right action of $g\in P$ on $\mathcal G$.
Moreover, $\omega$ reproduces the generators of the right $P$-action, i.e.\ for all $X\in\poe$, we have $\omega(\zeta_X)=X$ where $\zeta_X:=\frac d{dt}|_0r^{\exp(tX)}$ denotes the fundamental vector field.
The prototypical example of a parabolic geometry of type $(G,P)$ is its flat model, that is, the generalized flag variety $G/P$ with the canonical projection $G\to G/P$ and the Maurer--Cartan form on $G$.


Using the Cartan connection, the filtration \eqref{E:filt} provides a filtration of the tangent bundle $T\mathcal G$ by $P$-invariant subbundles, 
$$
T\mathcal G=T^{-k}\mathcal G\supseteq T^{-k+1}\mathcal G\supseteq\cdots\supseteq T^k\mathcal G,
$$
where $T^i_u\mathcal G:=\omega_u^{-1}(\goe^i)$ for $u\in\mathcal G$. 
Note that $T^0\mathcal G$ coincides with the vertical bundle of the projection $p\colon\mathcal G\to M$.
Hence there exists a unique filtration of $TM$ by subbundles,
\begin{equation}\label{E:TMfilt}
TM=T^{-k}M\supseteq T^{-k+1}M\supseteq\cdots\supseteq T^{-1}M,
\end{equation}
such that $(Tp)^{-1}(T^iM)=T^i\mathcal G$.
The Cartan connection induces an isomorphism
\begin{equation}\label{E:TMgp}
TM\cong\mathcal G\times_P(\goe/\poe)
\end{equation}
intertwining the filtration \eqref{E:TMfilt} with the filtration induced from \eqref{E:filt}.
Using \eqref{E:grgp} for the associated graded we obtain an isomorphism of vector bundles,
\begin{equation}\label{E:grTM}
\gr(TM)\cong\mathcal G_0\times_{G_0}\goe_-.
\end{equation}
Here $\mathcal G_0:=\mathcal G/P_+$ is considered as a principal $G_0$-bundle over $M$.


We assume that the Cartan connection $\omega$ is regular, see \cite[Definition~3.1.7]{CS09}.
Hence, the filtration on $TM$, see~\eqref{E:TMfilt}, turns $M$ into a filtered manifold, and the corresponding Levi bracket $\gr(TM)\otimes\gr(TM)\to\gr(TM)$ induced by the Lie bracket of vector fields coincides with the algebraic bracket induced by the Lie bracket $\goe_-\otimes\goe_-\to\goe_-$ via \eqref{E:grTM}.
In other words, using the notation from Section~\ref{SS:DO}, the Cartan connection of a regular parabolic geometry provides an isomorphism of bundles of graded nilpotent Lie algebras
\begin{equation}\label{E:tMG0}
\mathfrak tM\cong\mathcal G_0\times_{G_0}\goe_-.
\end{equation}


Recall that the Cartan connection induces a principal connection on $\mathcal P\times_PG$.
More precisely, there exists a unique principal connection on the principal $G$-bundle $\mathcal G\times_PG\to M$ which restricts to the Cartan connection $\omega$ along the inclusion 
$\mathcal G\subseteq\mathcal G\times_PG$, see \cite[Theorem~1.5.6]{CS09}.
Consequently, for every finite dimensional $G$-representation $\mathbb E$, the Cartan connection induces a linear connection $\nabla$ on the associated tractor bundle 
$$
E:=\mathcal G\times_P\mathbb E=(\mathcal G\times_PG)\times_G\mathbb E.
$$
Recall that $\mathbb E$ admits a grading, $\mathbb E=\mathbb E_{-l}\oplus\cdots\oplus\mathbb E_l$, which is compatible with the grading of $\goe$, i.e.\ $X\cdot v\in\mathbb E_{i+j}$ for all $X\in\goe_i$ and $X\in\mathbb E_j$.
Indeed, there exists a unique grading element in $\goe$ which acts by multiplication with $j$ on the component $\goe_j$, see \cite[Proposition~3.1.2(1)]{CS09}, and the eigenspaces of its action on $\mathbb E$ provide the desired decomposition.
The grading of $\mathbb E$ is $G_0$-invariant since the uniqueness of the the grading element implies that it is stabilized by $G_0$.
Hence, the associated filtration $\mathbb E^i:=\bigoplus_{j\geq i}\mathbb E_j$ is $P$-invariant, see \eqref{E:PG0iso}.
Moreover, $P_+$ acts trivially on the associated graded, and $\gr(\mathbb E)=\mathbb E$ as representations of $P/P_+=G_0$.
The $P$-invariant filtration of $\mathbb E$ induces a filtration of $E$ by subbundles $E^i:=\mathcal G\times_P\mathbb E^i$.
Clearly, the linear connection $\nabla$ is filtration preserving, that is, for all $X\in\Gamma^\infty(T^pM)$ and $\psi\in\Gamma^\infty(E^q)$ we have $\nabla_X\psi\in\Gamma^\infty(E^{p+q})$.
Since the Cartan connection is assumed to be regular, its curvature $F^\nabla\in\Omega^2(M;\eend(E))$ is contained in filtration degree one, that is, for all $X_i\in\Gamma^\infty(T^{p_i}M)$ and $\psi\in\Gamma^\infty(E^q)$ we have $F^\nabla(X_1,X_2)\psi\in\Gamma^\infty(E^{p_1+p_2+q+1})$, see \cite[Corollary~3.1.8(2) and Theorem~3.1.22(3)]{CS09}.
The isomorphism \eqref{E:TMgp} induces an isomorphism
\begin{equation}\label{E:forms}
\Lambda^kT^*M\otimes E\cong\mathcal G\times_P(\Lambda^k(\goe/\poe)^*\otimes\mathbb E)
\end{equation}
which intertwines the filtration on $\Lambda^kT^*M\otimes E$ with the one induced from the filtration on $\Lambda^k(\goe/\poe)^*\otimes\mathbb E$.
Moreover, \eqref{E:grgp} provides an isomorphism of $G_0$-modules,
\begin{equation}\label{E:123}
\gr\bigl(\Lambda^k(\goe/\poe)^*\otimes\mathbb E\bigr)=C^k(\goe_-;\mathbb E),
\end{equation}
where $C^k(\goe_-;\mathbb E):=\Lambda^k\goe_-^*\otimes\mathbb E$. 
Hence, \eqref{E:forms} induces an isomorphism
\begin{equation}\label{E:grforms}
\gr\bigl(\Lambda^kT^*M\otimes E\bigr)
\cong\mathcal G_0\times_{G_0}C^k(\goe_-;\mathbb E).
\end{equation}
The extension $d^\nabla\colon\Omega^k(M;E)\to\Omega^{k+1}(M;E)$ characterized by the Leibniz rule, see \eqref{E:leibn}, is filtration preserving, and via \eqref{E:grforms} we have
\begin{equation}\label{E:grd}
\gr(d^\nabla)=\mathcal G_0\times_{G_0}\partial_{\goe_-}
\end{equation}
where $\partial_{\goe_-}\colon C^k(\goe_-;\mathbb E)\to C^{k+1}(\goe_-;\mathbb E)$ denotes the differential in the standard complex computing Lie algebra cohomology $H^*(\goe_-;\mathbb E)$, see Lemma~\ref{L:curv}.


Let $\delta_{\poe_+}\colon\Lambda^k\poe_+\otimes\mathbb E\to\Lambda^{k-1}\poe_+\otimes\mathbb E$ denote the differential in the standard complex computing the Lie algebra homology $H_*(\poe_+;\mathbb E)$ with coefficients in $\mathbb E$. 
Since $\delta_{\poe_+}$ is $P$-equivariant, and since the Killing form provides an isomorphism of $P$-modules $(\goe/\poe)^*\cong\poe_+$, the differential $\delta_{\poe_+}$ dualizes to a $P$-equivariant map
\begin{equation}\label{E:del*gp}
\delta_{\goe/\poe}\colon\Lambda^k(\goe/\poe)^*\otimes\mathbb E\to\Lambda^{k-1}(\goe/\poe)^*\otimes\mathbb E.
\end{equation}
Via the identification \eqref{E:forms}, it gives rise to a vector bundle homomorphism,
\begin{equation}\label{E:deltaVV}
\delta\colon\Lambda^kT^*M\otimes E\to\Lambda^{k-1}T^*M\otimes E,\qquad\delta:=\mathcal G\times_P\delta_{\goe/\poe}.
\end{equation}
In the literature \cite{CSS01,CS12,CS09} this homomorphism is often denoted $\partial^*$.
Clearly, $\delta^2=0$. 
Moreover, $\delta$ is filtration preserving and via \eqref{E:grforms}
\begin{equation}\label{E:grdel}
\gr(\delta)=\mathcal G_0\times_{G_0}\delta_{\goe_-},
\end{equation}
where $\delta_{\goe_-}\colon C^k(\goe_-;\mathbb E)\to C^{k-1}(\goe_-;\mathbb E)$ is obtained from \eqref{E:del*gp} by passing to the associated graded and using the identification \eqref{E:123}, that is, $\delta_{\goe_-}=\gr(\delta_{\goe/\poe})$.
Actually, $\delta_{\goe/\poe}$ is grading preserving, hence $\delta_{\goe_-}=\delta_{\goe/\poe}$ via the isomorphism of $G_0$-modules 
\begin{equation}\label{E:g-gp}
\Lambda^k\goe_-^*\otimes\mathbb E=\Lambda^k(\goe/\poe)^*\otimes\mathbb E
\end{equation}
induced by the identification $\goe_-=\goe/\poe$.


Recall that a Weyl structure is a $G_0$-equivariant section of the principal $P_+$-bundle $\mathcal G\to\mathcal G_0=\mathcal G/P_+$, see \cite[Definition~5.1.1]{CS09}.
Global Weyl structures always exist, see \cite[Proposition~5.1.1]{CS09}. Moreover, the (contractible) space of sections of the bundle of groups $\mathcal G_0\times_{G_0}P_+$
acts free and transitively on the space of Weyl structures.
Using the isomorphism of $G_0$-modules \eqref{E:g-gp}, every Weyl structure $\mathcal G_0\to\mathcal G$ induces a filtration preserving isomorphism of vector bundles,
\begin{equation*}\label{E:sigma}
\mathcal G_0\times_{G_0}C^k(\goe_-;\mathbb E)\xrightarrow\cong\mathcal G\times_P\bigl((\Lambda^k(\goe/\poe)^*\otimes\mathbb E\bigr),
\end{equation*}
inducing the identity on the associated graded, see \eqref{E:123}.
Via \eqref{E:forms} and \eqref{E:grforms} this corresponds to a splitting of the filtration
$$
S\colon\gr(\Lambda^kT^*M\otimes E)\to\Lambda^kT^*M\otimes E
$$
satisfying $\delta\circ S=S\circ\gr(\delta)$.


Kostant \cite{K61} observed that $\delta_{\goe_-}$ and $\partial_{\goe_-}$ are adjoint with respect to positive definite inner products on the spaces $C^k(\goe_-;\mathbb E)$, see \cite[Proposition~3.1.1]{CS09}.
Hence the Laplacian
$$
\Box_{\goe_-}\colon C^*(\goe_-;\mathbb E)\to C^*(\goe_-;\mathbb E),\qquad
\Box_{\goe_-}:=\delta_{\goe_-}\circ\partial_{\goe_-}+\partial_{\goe_-}\circ\delta_{\goe_-},
$$
gives rise to a finite dimensional Hodge decomposition
\begin{equation}\label{E:fdhodge}
C^*(\goe_-;\mathbb E)=\img(\delta_{\goe_-})\oplus\ker(\Box_{\goe_-})\oplus\img(\partial_{\goe_-}).
\end{equation}
Using \eqref{E:grd} and \eqref{E:grdel} we see that $\Box\colon\Omega^*(M;E)\to\Omega^*(M;E)$, $\Box=\delta\circ d^\nabla+d^\nabla\circ\delta$, is filtration preserving, and via the identification \eqref{E:grforms} we have
\begin{equation}\label{E:grBox}
\gr(\Box)=\mathcal G_0\times_{G_0}\Box_{\goe_-}.
\end{equation}
For the fiber wise projection $\tilde P\colon\gr(\Lambda^kT^*M\otimes E)\to\gr(\Lambda^kT^*M\otimes E)$ onto the (generalized) zero eigenspace of $\gr(\Box)$, we obtain, via \eqref{E:grforms},
$$
\tilde P=\mathcal G_0\times_{G_0}P_{\goe_-}
$$
where $P_{\goe_-}\colon C^*(\goe_-;\mathbb E)\to C^*(\goe_-;\mathbb E)$ denotes the projection onto $\ker(\Box_{\goe_-})$ along the decomposition \eqref{E:fdhodge}.
In particular, $\gr(\delta)\circ\tilde P=0$.


From the discussion above we conclude that the homomorphism $\delta$, see \eqref{E:deltaVV}, is a Kostant type codifferential of maximal rank for the linear connection $\nabla$ on the tractor bundle $E$, see Definition~\ref{D:Kdelta} and Remark~\ref{R:Kdeltabound}.
In this situation Corollary~\ref{C:D}(a) reduces to the statement in \cite[Theorem~2.4]{CS12}, see also \cite{CSS01,CD01}.
Since $P_+$ acts trivially on $H_*(\poe_+;\mathbb E)$ the $P$-action on $H_*(\poe_+;\mathbb E)$ factors to an action by $P/P_+=G_0$ and we have canonical identifications:
$$
\mathcal H_*
=\mathcal G_0\times_{G_0}H_*(\poe_+;\mathbb E)
=\mathcal G_0\times_{G_0}\ker(\Box_{\goe_-})
=\mathcal G_0\times_{G_0}H^*(\goe_-;\mathbb E).
$$
The corresponding sequence of differential operators in Corollary~\ref{C:D}(b)
coincides with one version of (curved) BGG sequences that can be found in the literature.
This is called ``torsion free BGG sequence'' in \cite[Section~5]{CD01} and coincides with the sequence constructed in \cite[Section~2.4]{CS12}.
For torsion free parabolic geometries, see \cite[Section~1.5.7]{CS09}, this coincides with the original curved BGG sequence constructed by \v Cap, Slov\'ak and Sou\v cek in \cite{CSS01}.
From Corollary~\ref{C:D}(b) we thus obtain


\begin{corollary}\label{C:BGG}
The (torsion free) BGG sequence associated to a regular parabolic geometry of type $(G,P)$ and a finite dimensional $G$-representation is a graded Rockland sequence of differential operators which have graded Heisenberg order at most zero.
\end{corollary}


Kostant's version of the Bott--Borel--Weil theorem permits to effectively compute the homologies $H_k(\poe_+;\mathbb E)$ as modules over $\goe_0$.
More precisely, $H_k(\poe_+;\mathbb E)$ decomposes as a direct sum of irreducible $\goe_0$-modules whose dominant weights can be read off the Hasse diagram of $\poe$, see \cite{K61} or \cite[Theorem~3.3.5 and Proposition~3.3.6]{CS09}.
Moreover, the grading element acts as a scalar on each of these irreducible components which can easily be computed too, see \cite[Section~3.2.12]{CS09}.
Consequently, representation theory permits to determine the decomposition according to the grading $H_k(\poe_+;\mathbb E)=\bigoplus_pH_k(\poe_+;\mathbb E)_p$.
Decomposing the BGG operators accordingly, $\bar D_k=\sum_{p,q}(\bar D_k)_{qp}$, with
\begin{equation}\label{E:Dkqp}
\Gamma^\infty\bigl(\mathcal G_0\times_{G_0}H_k(\poe_+;\mathbb E)_p\bigr)
\xrightarrow{(\bar D_k)_{qp}}
\Gamma^\infty\bigl(\mathcal G_0\times_{G_0}H_{k+1}(\poe_+;\mathbb E)_q\bigr),
\end{equation}
one part of Corollary~\ref{C:BGG} asserts that the differential operator in \eqref{E:Dkqp} is of Heisenberg order at most $q-p$, a statement which appears to be well known.
Using a frame $(\mathcal G_0)_x\cong G_0$ at $x\in M$, the Heisenberg principal symbol of the operator \eqref{E:Dkqp} at $x$ can be regarded as a left invariant differential operator
$$
C^\infty\bigl(\mathcal T_xM,H_k(\poe_+;\mathbb E)_p\bigr)\xrightarrow{\sigma^{q-p}_x((\bar D_k)_{qp})}C^\infty\bigl(\mathcal T_xM,H_{k+1}(\poe_+;\mathbb E)_q\bigr)
$$
which is homogeneous of order $q-p$.
The second part of Corollary~\ref{C:BGG} asserts that these Heisenberg principal symbols combine to form a sequence of left invariant differential operators
\begin{equation}\label{E:tsBGG}
\cdots\to C^\infty\bigl(\mathcal T_xM,H_k(\poe_+;\mathbb E)\bigr)
\xrightarrow{\tilde\sigma^0_x(\bar D_k)}
C^\infty\bigl(\mathcal T_xM,H_{k+1}(\poe_+;\mathbb E)\bigr)\to\cdots
\end{equation}
where $\tilde\sigma_x^0(\bar D_k)=\sum_{p,q}\sigma^{q-p}_x((\bar D_k)_{qp})$, which is Rockland in the sense that it becomes exact in every non-trivial irreducible unitary representation of $\mathcal T_xM$.
Up to the isomorphism $\mathcal T_xM\cong G_-:=\exp(\goe_-)$ provided by the frame, the graded Heisenberg principal symbol of a BGG operator in \eqref{E:tsBGG} coincides with the corresponding BGG operator on the flat model $G/P$ restricted along the local diffeomorphism $G_-\to G/P$ obtained form the inclusion $G_-\subseteq G$.


If the homology $H_k(\poe_+;\mathbb E)$ is concentrated in a single degree for each $k$, that is, if there exist numbers $p_k$ such that $H_k(\poe_+;\mathbb E)=H_k(\poe_+;\mathbb E)_{p_k}$, then the corresponding BGG operator $\bar D_k\colon\Gamma^\infty(\mathcal H_k)\to\Gamma^\infty(\mathcal H_{k+1})$ is of Heisenberg order at most $p_{k+1}-p_k$ and the BGG sequence is Rockland in the ungraded sense, see Definition~\ref{def.Hypo-seq}.
If, moreover, $p_{k+1}-p_k\geq1$, then the analytic results established in the preceding sections are applicable, see, in particular, Corollaries~\ref{C:RShypo}, \ref{C:Hodge}, \ref{C:regrockseq}, and \ref{C:HsHodge-seq}.
Below we will discuss a classical example of this type.
The analysis for genuinely filtered sequences will be provided in Section~\ref{S:grRockland}.


\begin{example}[Generic rank two distributions in dimension five]\label{Ex:BGG235}
Let $M$ be a 5-manifold equipped with a rank two distribution of Cartan type, $T^{-1}M\subseteq TM$, see \cite{C10,BH93,S08}.
In other words, $T^{-1}M$ is a rank two subbundle of $TM$ with growth vector $(2,3,5)$, that is, Lie brackets of sections of $T^{-1}M$ span a rank three subbundle $T^{-2}M$ of $TM$ and triple brackets of sections of $T^{-1}M$ span all of $TM$.
These geometric structures are also known as generic rank two distributions in dimension five, see \cite{S08,CS09a}.
The topological obstructions to global existence of such a distribution are well understood in the orientable case, see \cite[Theorem~1]{DH16}.
On open 5-manifolds, Gromov's h-principle is applicable and permits to establish existence once the topological requirements are met, see \cite[Theorem~2]{DH16}.
It is unclear, however, if there are further geometric obstructions on closed 5-manifolds.
Whether rank two distributions of Cartan type also abide by an h-principle on closed manifolds, appears to be an intriguing open question and is a major motivation for our investigation of hypoelliptic sequences.


In his celebrated paper \cite{C10} Cartan has shown that, up to isomorphism, there exists a unique regular normal parabolic geometry of type $(G,P)$ on $M$ with underlying filtration:
$$
TM=T^{-3}M\supseteq T^{-2}M\supseteq T^{-1}M\supseteq T^0M=0.
$$
Here $G$ denotes the split real form of the exceptional Lie group $G_2$ and $P$ denotes the maximal parabolic subgroup corresponding to the shorter simple root.
Hence, every finite dimensional representation $\mathbb E$ of $G$ gives rise to a curved BGG sequence on $M$,
\begin{equation}\label{E:BGG235}
\Gamma^\infty(\mathcal H_0)\xrightarrow{\bar D_0}
\Gamma^\infty(\mathcal H_1)\xrightarrow{\bar D_1}
\Gamma^\infty(\mathcal H_2)\xrightarrow{\bar D_2}
\Gamma^\infty(\mathcal H_3)\xrightarrow{\bar D_3}
\Gamma^\infty(\mathcal H_4)\xrightarrow{\bar D_4}
\Gamma^\infty(\mathcal H_5),
\end{equation}
where $\mathcal H_k:=\mathcal G_0\times_{G_0}H_k(\poe_+;\mathbb E)$.
We label the longer simple root of the $G_2$ root system by $\alpha_1$ and let $\alpha_2$ denote the shorter simple root.
Hence, $2\alpha_1+3\alpha_2$ is the highest root, and the corresponding fundamental weights are $\lambda_1=2\alpha_1+3\alpha_2$ and $\lambda_2=\alpha_1+2\alpha_2$.
Suppose $\mathbb E$ is the irreducible complex representation with highest weight $a\lambda_1+b\lambda_2$ where $a,b\in\N_0$.
Then $H_k(\poe_+;\mathbb E)$ is an irreducible complex module of $\goe_0\subseteq\goe_0^\C\cong\mathfrak g\mathfrak l_2(\C)$ which can readily be determined by working out the Hasse diagram of $\poe^\C$, see \cite[Section 3.2.16]{CS09}, and using Kostant's version of the Bott--Borel--Weil theorem, see \cite[Theorem~3.3.5 and Proposition~3.3.6]{CS09}.
Denoting the highest weight of $H_k(\poe_+;\mathbb E)$ by $a_k\lambda_1+b_k\lambda_2$, we obtain the first three columns in the following table:
\footnote{As formulated in \cite[Theorem~3.3.5]{CS09}, Kostant's version of the Bott--Borel--Weil theorem computes the cohomology $H^k(\poe_+;\mathbb E)$.
Using the following facts, this permits to work out the homology $H_k(\poe_+;\mathbb E)$ as well:
$H^k(\poe_+;\mathbb E^*)\cong H_k(\poe_+;\mathbb E)^*$ as $\goe_0$-modules; 
$\mathbb E\cong\mathbb E^*$ as $\goe$-modules; 
If $W$ is an irreducible $\goe_0$ module with highest weight $a\lambda_1+b\lambda_2$, then $W^*$ is an irreducible $\goe_0$ module with highest weight $a'\lambda_1+b'\lambda_2$, where $a'=a$ and $b'=-3a-b$.}
\begin{equation}\label{E:G2table}
\begin{array}{r||r|r||r|r}
k & a_k & b_k & \dim H_k(\poe_+;\mathbb E) & p_k
\\\hline\hline
0 & a      & -3a-b    & a+1    & -3a-2b  
\\
1 & a+b+1  & -3a-2b-1 & a+b+2  & -3a-b+1 
\\
2 & 2a+b+2 & -3a-2b-1 & 2a+b+3 & -b+4
\\
3 & 2a+b+2 & -3a-b    & 2a+b+3 & b+6
\\
4 & a+b+1  & -b+3     & a+b+2  & 3a+b+9
\\
5 & a      & b+5      &  a+1   & 3a+2b+10
\end{array}
\end{equation}
The grading element acts by multiplication with the scalar $p_k=3a_k+2b_k$ on $H_k(\poe_+;\mathbb E)$, see \cite[Section~3.2.12]{CS09}, whence the last column.
Moreover, the highest weight of $H_k(\poe_+;\mathbb E)$ considered as $\mathfrak s\mathfrak l_2(\C)$-module is $a_k$ times the fundamental weight of $\mathfrak s\mathfrak l_2(\C)$, hence $\dim H_k(\poe_+;\mathbb E)=a_k+1$, whence the remaining column.
Since $\bar D_k$ is of Heisenberg order $p_{k+1}-p_k$, we conclude that $\bar D_0$ and $\bar D_4$ are of Heisenberg order $b+1$; $\bar D_1$ and $\bar D_3$ are of Heisenberg order $3(a+1)$; and $\bar D_2$ is of Heisenberg order $2(b+1)$.
According to Corollary~\ref{C:BGG} the sequence \eqref{E:BGG235} is Rockland in the ungraded sense, see Definition~\ref{def.Hypo-seq}.
In particular, the differential operators
$\bar D_0^*\bar D_0$, 
$(\bar D_0\bar D_0^*)^{3(a+1)}+(\bar D_1^*\bar D_1)^{b+1}$, 
$(\bar D_1\bar D_1^*)^{2(b+1)}+(\bar D_2^*\bar D_2)^{3(a+1)}$,
$(\bar D_2\bar D_2^*)^{3(a+1)}+(\bar D_3^*\bar D_3)^{2(b+1)}$,
$(\bar D_3\bar D_3^*)^{b+1}+(\bar D_4^*\bar D_4)^{3(a+1)}$, and
$\bar D_4\bar D_4^*$ 
are all hypoelliptic and maximal hypoelliptic estimates are available, see Lemma~\ref{L:rockseq}, Theorem~\ref{T:Rockland}, as well as, Corollaries~\ref{C:PsiinvA}, \ref{C:reg}, and \ref{C:HsHodge}.
Here $\bar D_k^*$ denotes the formal adjoint of $\bar D_k$ with respect to any fiber wise Hermitian metrics on the vector bundles $\mathcal H_k$ and any volume density on $M$.


Let us finally put down explicit formulas for the Heisenberg principal symbol of the BGG operators corresponding to the trivial representation $\mathbb E$.
We consider these operators as left invariant differential operators on the simply connected nilpotent Lie group $G_-$.
According to the discussion above, this  BGG sequence has the form
\begin{equation}\label{E:BGG235dR}
C^\infty(G_-)\xrightarrow{D_0}
C^\infty(G_-)^2\xrightarrow{D_1}
C^\infty(G_-)^3\xrightarrow{D_2}
C^\infty(G_-)^3\xrightarrow{D_3}
C^\infty(G_-)^2\xrightarrow{D_4}
C^\infty(G_-)
\end{equation}
where $D_0$ and $D_4$ are homogeneous of degree $1$; $D_1$ and $D_3$ are homogeneous of degree $3$; and $D_2$ is homogeneous of degree $2$, see also \cite{BENG12}.
Using matrices with entries in the universal enveloping algebra of $\goe_-$ these operators can be expressed as:

$$
D_0=\left(\begin{array}{c}
X_1\\
X_2
\end{array}\right)
$$

$$
D_1=\left(\begin{array}{cc}
-X_4-X_{112}-X_{13}&X_{111}\\
-X_5-X_{122}&X_{112}-2X_{13}\\
-X_{222}&X_{122}-3X_{23}
\end{array}\right)
$$

$$
D_2=\left(\begin{array}{ccc}
-X_{12}-X_3&X_{11}& 0\\         
-X_{22}&-3X_3&  X_{11}\\ 
0&        -X_{22}&X_{12}-2X_3\\
\end{array}\right)
$$

$$
D_3=\left(\begin{array}{ccc}
X_{122}+X_{23}-2X_5&-X_{112}+X_4&X_{111}\\
X_{222}&-X_{122}+2X_{32}&X_{112}-3X_{13}+3X_4
\end{array}\right)
$$

$$
D_4=\left(\begin{array}{cc}
-X_2&X_1
\end{array}\right)
$$
Here $X_5,X_4|X_3|X_2,X_1$ is a graded basis of $\goe_-=\goe_{-3}\oplus\goe_{-2}\oplus\goe_{-1}$ such that
$$
[X_1,X_2]=X_3,\quad[X_1,X_3]=X_4,\quad[X_2,X_3]=X_5.
$$
The vertical bars above indicate that $X_1,X_2$ is a basis of $\goe_{-1}$, $X_3$ is a basis of $\goe_{-2}$, and $X_4,X_5$ is a basis of $\goe_{-3}$.
Moreover, we use the notation $X_{i_1\dotsc i_k}=X_{i_1}\cdots X_{i_k}$.
These formulas are derived in Appendix~\ref{A:G2} below.
\end{example}












\section{Graded hypoelliptic sequences}\label{S:grRockland}










In this section we adapt the analysis discussed in Section~\ref{S:PDO} to the filtered setup required to deal with the sequences constructed in Section~\ref{S:grhesequences}.
Everything generalizes effortlessly, but one bit: Formal adjoints of graded (pseudo)differential operators are in general only available if the underlying manifold is closed.
This is related to the fact that we can construct invertible $\Lambda_s\in\Psi^s(E)$ with $\Lambda_s^{-1}\in\Psi^{-s}(E)$ only on closed manifolds, see Lemma~\ref{L:Lambda} and \eqref{E:Asharp} below.





\subsection{Graded pseudodifferential operators}\label{SS:gradedhypo} 





The concept of graded Heisenberg order for differential operators introduced in Section~\ref{SS:fVBDO} can be generalized to pseudodifferential operators in a straight forward manner as follows.
Let $E$ and $F$ be two filtered vector bundles over a filtered manifold $M$, suppose $A\in\mathcal O(E,F)$, and let $s$ be a complex number.
Choose splittings of the filtrations, $S_E\colon\gr(E)\to E$ and $S_F\colon\gr(F)\to F$ and decompose the operator accordingly, $S_F^{-1}AS_E=\sum_{q,p}(S_F^{-1}AS_E)_{q,p}$, where $(S_F^{-1}AS_E)_{q,p}\in\mathcal O(\gr_p(E),\gr_q(F))$.
We say $A$ has \emph{graded Heisenberg order} $s$ if $(S_F^{-1}AS_E)_{q,p}\in\Psi^{s+q-p}(\gr_p(E),\gr_q(F))$ for all $p$ and $q$.
We let $\tilde\Psi^s(E,F)$ denote the space of pseudodifferential operators of graded Heisenberg order $s$.
One readily checks that this space does not depend on the choice of splittings $S_E$ and $S_F$.


Let us define the \emph{space of principal cosymbols of graded order $s$} by
$$
\tilde\Sigma^s(E,F)
:=\left\{k\in\frac{\mathcal K(\mathcal TM;\gr(E),\gr(F))}{\mathcal K^\infty(\mathcal TM;\gr(E),\gr(F))}:\textrm{$(\delta_\lambda)_*k=\lambda^s\delta_\lambda^Fk\delta^E_{1/\lambda}$ for all $\lambda>0$}\right\},
$$
where $\delta^E_\lambda\in\Aut(\gr(E))$ denotes the automorphism given by multiplication with $\lambda^p$ on the grading component $\gr_p(E)$.
These are essentially homogeneous kernels in a graded sense, taking the grading on $\gr(E)$ and $\gr(F)$ into account.
They can be canonically identified with matrices of ordinary principal cosymbols,
\begin{equation}\label{E:tSigma}
\tilde\Sigma^s(E,F)=\bigoplus_{q,p}\Sigma^{s+q-p}(\gr_p(E),\gr_q(F)).
\end{equation}


For $A\in\tilde\Psi^s(E,F)$ we define the \emph{graded Heisenberg principal cosymbol} $\tilde\sigma^s(A)\in\tilde\Sigma^s(E,F)$ by $\tilde\sigma^s(A):=\sum_{p,q}\sigma^{s+q-p}\bigl((S_F^{-1}AS_E)_{q,p}\bigr)$ where $\sigma^{s+q-p}((S_F^{-1}AS_E)_{q,p})\in\Sigma^s(\gr_p(E),\gr_q(F))$ are the Heisenberg principal symbols of the components.
One readily checks that the graded principal Heisenberg cosymbol is independent of the choice of splittings $S_E$ and $S_F$.
From Proposition~\ref{P:Psi}\itemref{P:Psi:symbsequ} we immediately obtain a short exact sequence:
$$
0\to\tilde\Psi^{s-1}(E,F)\to\tilde\Psi^s(E,F)\xrightarrow{\tilde\sigma^s}\tilde\Sigma^s(E,F)\to0.
$$
If $A\in\tilde\Psi^s(E,F)$ and $B\in\tilde\Psi^r(F,G)$, then $BA\in\tilde\Psi^{r+s}(E,G)$ and
\begin{equation}\label{E:grsigmaAB}
\tilde\sigma^{r+s}(BA)=\tilde\sigma^r(B)\tilde\sigma^s(A),
\end{equation}
provided at least one operator is properly supported.
Moreover, $A^t\in\tilde\Psi^s(F',E')$ and
\begin{equation}\label{E:grsigmaAt}
\tilde\sigma^s(A^t)=\tilde\sigma^s(A)^t.
\end{equation}
These two properties follow immediately from the corresponding statements in the ungraded case, see Proposition~\ref{P:Psi}\itemref{P:Psi:mult}\&\itemref{P:Psi:trans}.
Recall that the bundle $E'=E^*\otimes|\Lambda|_{M}$ is equipped with the dual filtration as explained in Section~\ref{SS:fVBDO}.


For trivially filtered vector bundles these concepts clearly reduce the ungraded case discussed in Section~\ref{SS:calculus}.
Moreover, for differential operators we recover the graded Heisenberg order and graded Heisenberg symbol from Section~\ref{SS:fVBDO}.
More precisely, for every non-negative integer $k$, we have $\DO(E,F)\cap\tilde\Psi^k(E,F)=\widetilde{\DO}^k(E,F)$, and the graded Heisenberg principal symbol from Section~\ref{SS:fVBDO} coincides with principal Heisenberg cosymbol introduced in this section via the canonical inclusion
\begin{equation}\label{E:incUSgr}
\bigl(\mathcal U(\mathfrak tM)\otimes\hom(\gr(E),\gr(F))\bigr)_{-k}\subseteq\tilde\Sigma^k(E,F),
\end{equation}
see Proposition~\ref{P:Psi}\itemref{P:Psi:DO} and \eqref{E:incUS}.


\begin{lemma}\label{L:grLambda}
Let $E$ be a filtered vector bundle over a filtered manifold $M$, and let $\mathbf E$ denote the same vector bundle equipped with the trivial filtration, $\mathbf E=\mathbf E^0\supseteq\mathbf E^1=0$.
For every complex number $s$ there exist $\tilde\Lambda_s\in\tilde\Psi_\prop^s(E,\mathbf E)$ and $\tilde\Lambda_s'\in\tilde\Psi^{-s}_\prop(\mathbf E,E)$ such that $\tilde\Lambda_s\tilde\Lambda_s'-\id$ and $\tilde\Lambda_s'\tilde\Lambda_s-\id$ are both smoothing operators.
Moreover, these operators may be chosen such that $\tilde\Lambda_s\colon\Gamma^{-\infty}_c(E)\to\Gamma^{-\infty}_c(\mathbf E)$ and $\tilde\Lambda_s'\colon\Gamma^{-\infty}_c(\mathbf E)\to\Gamma^{-\infty}_c(E)$ are injective.
On a closed manifold these operators may even be chosen such that $\tilde\Lambda_s\tilde\Lambda_s'=\id$ and $\tilde\Lambda_s'\tilde\Lambda_s=\id$.
\end{lemma}


\begin{proof}
Choose a splitting of the filtration, $S\colon\gr(E)\to E$.
Let $\Lambda_{s-p}\in\Psi^{s-p}_\prop(\gr_p(E))$ and $\Lambda_{s-p}'\in\Psi^{-(s-p)}_\prop(\gr_p(E))$ be as in Lemma~\ref{L:Lambda}.
Then the operators
$$
\tilde\Lambda_s:=S\bigl(\textstyle\bigoplus_p\Lambda_{s-p}\bigr)S^{-1}
\qquad\text{and}\qquad
\tilde\Lambda_s':=S\bigl(\textstyle\bigoplus_p\Lambda'_{s-p}\bigr)S^{-1}
$$
have the desired properties.
\end{proof}





\subsection{Graded Heisenberg Sobolev scale}\label{SS:grsobolev}





Let $E$ be a filtered vector bundle over a filtered manifold $M$.
For each real number $s$ we let $\tilde H^s_\loc(E)$ denote the space of all distributional sections $\psi\in\Gamma^{-\infty}(E)$ such that $A\psi\in L^2_\loc(\mathbf F)$ for all $A\in\tilde\Psi^s_\prop(E,\mathbf F)$ and all trivially filtered vector bundles $\mathbf F$ over $M$, that is, $\mathbf F=\mathbf F^0\supseteq\mathbf F^1=0$.
We equip $\tilde H^s_\loc(E)$ with the coarsest topology such that $A\colon\tilde H^s_\loc(E)\to L^2_\loc(\mathbf F)$ is continuous for all such $A\in\tilde\Psi^s_\prop(E,\mathbf F)$.
Similarly, we let $\tilde H^s_c(E)$ denote the space of all compactly supported distributional sections $\psi\in\Gamma_c^{-\infty}(E)$ such that $A\psi\in L^2_\loc(\mathbf F)$ for all $A\in\tilde\Psi^s(E,\mathbf F)$ and all trivially filtered vector bundles $\mathbf F$ over $M$.
We equip $\tilde H^s_c(E)$ with the coarsest topology such that $A\colon\tilde H^s_c(E)\to L^2_\loc(\mathbf F)$ is continuous for all such $A\in\tilde\Psi^s(E,\mathbf F)$.
We will refer to these spaces as \emph{graded Heisenberg Sobolev spaces}.
Any splitting of the filtration on $E$ gives rise to non-canonical topological isomorphisms
$$
\tilde H^s_\loc(E)\cong\bigoplus_p H^{s-p}_\loc\bigl(\gr_p(E)\bigr)
\qquad\text{and}\qquad
\tilde H^s_c(E)\cong\bigoplus_pH^{s-p}_c\bigl(\gr_p(E)\bigr).
$$


Generalizing Proposition~\ref{P:Hs}\itemref{P:Hs:locfilt}\&\itemref{P:Hs:cfilt}, we have continuous inclusions
$$
\Gamma^\infty(E)\subseteq\tilde H^{s_2}_\loc(E)\subseteq\tilde H^{s_1}_\loc(E)\subseteq\Gamma^{-\infty}(E)
$$
and
$$
\Gamma^\infty_c(E)\subseteq\tilde H^{s_2}_c(E)\subseteq\tilde H^{s_1}_c(E)\subseteq\Gamma_c^{-\infty}(E)
$$
for all real numbers $s_1\leq s_2$.
Moreover, if $F$ is another filtered vector bundle over $M$, then each $A\in\tilde\Psi^k(E,F)$ induces continuous operators $A\colon\tilde H^s_c(E)\to\tilde H_\loc^{s-\Re(k)}(F)$ for all real $s$, cf.\ Proposition~\ref{P:Hs}\itemref{P:Hs:operators}.
For properly supported $A\in\tilde\Psi^k_\prop(E,F)$ we obtain continuous operators $A\colon\tilde H^s_c(E)\to\tilde H_c^{s-\Re(k)}(F)$ and $A\colon\tilde H^s_\loc(E)\to\tilde H_\loc^{s-\Re(k)}(F)$ for all real $s$.
As in Proposition~\ref{P:Hs}\itemref{P:Hs:pairing}, the canonical pairing $\Gamma^\infty_c(E')\times\Gamma^\infty(E)\to\C$ extends to a pairing
$$
\tilde H^{-s}_c(E')\times\tilde H^s_\loc(E)\to\C
$$
inducing linear bijections $\tilde H^s_\loc(E)^*=\tilde H^{-s}_c(E')$ and $\tilde H^{-s}_c(E')^*=\tilde H^s_\loc(E)$.
If, moreover, $M$ is closed, then $\tilde H^s_c(E)=\tilde H^s_\loc(E)$ is a Hilbert space we denote by $\tilde H^s(E)$, and the pairing induces an isomorphism of Hilbert spaces, $\tilde H^s(E)^*=\tilde H^{-s}(E')$.
This can all be proved as in Proposition~\ref{P:Hs} using Lemma~\ref{L:grLambda}.


Suppose $M$ is closed.
Fix a smooth volume density on $M$ and a smooth fiber wise Hermitian metric $h$ on $\mathbf E$.
Moreover, let $s$ be a real number, choose invertible $\tilde\Lambda_s\in\tilde\Psi^s(E,\mathbf E)$ with inverse $\tilde\Lambda_s^{-1}\in\tilde\Psi^{-s}(\mathbf E,E)$, see Lemma~\ref{L:grLambda}, and consider the associated Hermitian inner product, cf.~\eqref{E:llrr},
\begin{equation}\label{E:grHSllrr}
\llangle\psi_1,\psi_2\rrangle_{\tilde H^s(E)}
:=\llangle\tilde\Lambda_s\psi_1,\tilde\Lambda_s\psi_2\rrangle_{L^2(\mathbf E)}
=\langle\tilde\Lambda_s^t(h\otimes dx)\tilde\Lambda_s\psi_1,\psi_2\rangle
\end{equation}
where $\psi_1,\psi_2\in\Gamma^\infty(E)$.
In the expression on the right hand side $h\otimes dx\colon\bar{\mathbf E}\to\mathbf E'$ is considered as a vector bundle isomorphism, $\tilde\Lambda_s^t\in\tilde\Psi^s(\mathbf E',E')$, and $\langle-,-\rangle$ denotes the canonical pairing for sections of $E$.
The sesquilinear form in~\eqref{E:grHSllrr} extends to an inner product generating the Hilbert space topology on the graded Heisenberg Sobolev space $\tilde H^s(E)$.


With respect to inner products on $\tilde H^{s_1}(E)$ and $\tilde H^{s_2}(F)$ as above, every $A\in\tilde\Psi^k(E,F)$ admits a formal adjoint, $A^\sharp\in\tilde\Psi^{\bar k+2(s_2-s_1)}(F,E)$ such that
\begin{equation}\label{E:Asharpllrr}
\llangle A^\sharp\phi,\psi\rrangle_{\tilde H^{s_1}(E)}=\llangle\phi,A\psi\rrangle_{\tilde H^{s_2}(F)}
\end{equation}
for all $\psi\in\Gamma^\infty(E)$ and $\phi\in\Gamma^\infty(F)$.
Indeed,
\begin{equation}\label{E:Asharp}
A^\sharp
=\tilde\Lambda_{E,s_1}^{-1}(\tilde\Lambda_{F,s_2}A\tilde\Lambda_{E,s_1}^{-1})^*\tilde\Lambda_{F,s_2}
=(\tilde\Lambda_{E,s_1}^t(h_E\otimes dx)\tilde\Lambda_{E,s_1})^{-1}\,A^t\,\tilde\Lambda_{F,s_2}^t(h_F\otimes dx)\tilde\Lambda_{F,s_2}.
\end{equation}
In the first expression the star denotes the adjoint of $\tilde\Lambda_{F,s_2}A\tilde\Lambda_{E,s_1}^{-1}\in\Psi^{k+s_2-s_1}(\mathbf E,\mathbf F)$ with respect to the $L^2$ inner products associated with the fiber wise Hermitian metrics $h_E$ and $h_F$ and the volume density $dx$.
One readily verifies:
\begin{equation}\label{E:BAsharp}
(BA)^\sharp=A^\sharp B^\sharp
\qquad\text{and}\qquad
(A^\sharp)^\sharp=A.
\end{equation}







\subsection{Graded Rockland operators}






Let $E$ and $F$ be filtered vector bundles over a filtered manifold $M$.
To formulate the graded Rockland condition for operators in $\tilde\Psi^s(E,F)$, we begin by extending the definition of $\bar\pi(a)$ to graded cosymbols $a\in\tilde\Sigma^s_x(E,F)$ at $x\in M$, where $\pi\colon\mathcal T_xM\to U(\mathcal H)$ is a non-trivial irreducible unitary representation of the osculating group:
Write $a=\sum_{p,q}a_{p,q}$ according to the decomposition \eqref{E:tSigma} with $a_{q,p}\in\Sigma^{s+q-p}_x(\gr_p(E),\gr_q(F))$, and put $\bar\pi(a):=\sum_{p,q}\bar\pi(a_{q,p})$ where $\bar\pi(a_{q,p})$ denotes the unbounded operator from $\mathcal H\otimes\gr_p(E_x)$ to $\mathcal H\otimes\gr_q(F_x)$ described in Section~\ref{SS:para}.
Hence, $\bar\pi(a)$ is an unbounded operator form $\mathcal H\otimes\gr(E_x)$ to $\mathcal H\otimes\gr(F_x)$.
Moreover, the subspace $\mathcal H_\infty\otimes\gr(E_x)$ is contained in the domain of definition and mapped into $\mathcal H_\infty\otimes\gr(F_x)$.
From \eqref{E:piab} we immediately obtain 
\begin{equation}\label{E:grpiab}
\bar\pi(ba)=\bar\pi(b)\bar\pi(a)
\end{equation}
for all $a\in\tilde\Sigma^s_x(E,F)$ and $b\in\tilde\Sigma^{s'}_x(F,G)$.
For trivially filtered vector bundles, this clearly specializes to the definition in Section~\ref{SS:para}.
If $k$ is a non-negative integer and, see~\eqref{E:incUSgr}, $a\in\bigl(\mathcal U(\mathfrak t_xM)\otimes\hom(\gr(E_x),\gr(F_x))\bigr)_{-k}\subseteq\tilde\Sigma^k_x(E,F)$ then, on $\mathcal H_\infty\otimes\gr(E_x)$, the operator $\bar\pi(a)$ coincides with $\pi(a)$ considered in Section~\ref{SS:fVBDO}, cf.\ Definition~\ref{D:graded_hypoelliptic_seq}.


Generalizing Definition~\ref{D:Rockland} to the graded situation we have:


\begin{definition}[Graded Rockland condition]\label{D:graded-Rockland}
Let $E$ and $F$ be filtered vector bundles over a filtered manifold $M$.
A graded principal cosymbol $a\in\tilde\Sigma_x^s(E,F)$ at $x\in M$ is said to satisfy the \emph{graded Rockland condition} if, for every non-trivial irreducible unitary representation $\pi\colon\mathcal T_xM\to U(\mathcal H)$, the unbounded operator $\bar\pi(a)$ is injective on $\mathcal H_\infty\otimes\gr(E_x)$.
An operator $A\in\tilde\Psi^s(E,F)$ is said to satisfy the \emph{graded Rockland condition} if its graded principal cosymbol, $\tilde\sigma^s_x(A)\in\tilde\Sigma^s_x(E,F)$, satisfies the graded Rockland condition at each point $x\in M$.
\end{definition}


We obtain the following generalization of Theorem~\ref{T:Rockland} and Corollary~\ref{C:reg}.


\begin{corollary}[Left parametrix and graded regularity]\label{C:graded-regularity}
Let $E$ and $F$ be two filtered vector bundles over a filtered manifold $M$, let $k$ be a complex number, and suppose $A\in\tilde\Psi^k(E,F)$ satisfies the graded Rockland condition.
Then there exists a properly supported left parametrix $B\in\tilde\Psi^{-k}_\prop(F,E)$ such that $BA-\id$ is a smoothing operator.
In particular, $A$ is hypoelliptic, that is, if $\psi$ is a compactly supported distributional section of $E$ and $A\psi$ is smooth, then $\psi$ was smooth.
More precisely, $\psi\in\Gamma^{-\infty}_c(E)$ and $A\psi\in\tilde H^{r-\Re(k)}_\loc(F)$, then $\psi\in\tilde H^r_c(E)$.
If, moreover, $M$ is closed, then $\ker(A)$ is a finite dimensional subspace of\/ $\Gamma^\infty(E)$, and for every $r'\leq r$ there exists a constant $C=C_{A,r,r'}\geq0$ such that the maximal graded hypoelliptic estimate
\begin{equation}\label{E:grmhesti}
\|\psi\|_{\tilde H^r(E)}\leq C\left(\|\psi\|_{\tilde H^{r'}(E)}+\|A\psi\|_{\tilde H^{r-\Re(k)}(F)}\right)
\end{equation}
holds for all $\psi\in\tilde H^r(E)$.
Here we are using any norms generating the Hilbert space topologies on the corresponding graded Heisenberg Sobolev spaces.
Moreover, if $Q$ denotes the orthogonal projection, with respect to an inner product of the form \eqref{E:grHSllrr}, onto the (finite dimensional) subspace $\ker(A)\subseteq\Gamma^\infty(E)$, then there exists a constant $C=C_{A,r,s}\geq0$ such that the maximal graded hypoelliptic estimate
\begin{equation}\label{E:grmheestiQ}
\|\psi\|_{\tilde H^r(E)}\leq C\left(\|Q\psi\|+\|A\psi\|_{\tilde H^{r-\Re(k)}(F)}\right)
\end{equation}
holds for all $\psi\in\tilde H^r(E)$. Here $\|-\|$ denotes any norm on $\ker(A)$.
\end{corollary}


\begin{proof}
Let $\mathbf E$ and $\mathbf F$ denote the vector bundles $E$ and $F$ equipped with the trivial filtrations, respectively, that is to say, $\mathbf E=\mathbf E^0\supseteq\mathbf E^1=0$ and $\mathbf F=\mathbf F^0\supseteq\mathbf F^1=0$.
According to Lemma~\ref{L:grLambda} there exist $\tilde\Lambda_E\in\tilde\Psi^0(E,\mathbf E)$, $\tilde\Lambda_E'\in\tilde\Psi^0(\mathbf E,E)$, $\tilde\Lambda_F\in\tilde\Psi^0(F,\mathbf F)$, $\tilde\Lambda_F'\in\tilde\Psi^0(\mathbf F,F)$ such that $\tilde\Lambda_E\tilde\Lambda_E'-\id$, $\tilde\Lambda_E'\tilde\Lambda_E-\id$, $\tilde\Lambda_F\tilde\Lambda_F'-\id$, and $\tilde\Lambda_F'\tilde\Lambda_F-\id$ are all smoothing operators.
Then $\mathbf A:=\tilde\Lambda_FA\tilde\Lambda_E'\in\Psi^k(\mathbf E,\mathbf F)$ has Heisenberg order $k$ in the ungraded sense, and
$$
\sigma_x^k(\mathbf A)=\tilde\sigma^k_x(\mathbf A)=\tilde\sigma^0_x(\tilde\Lambda_F)\tilde\sigma^k_x(A)\tilde\sigma^0_x(\tilde\Lambda_E'),
$$
see \eqref{E:grsigmaAB}.
Since $A$ satisfies the graded Rockland condition, and since $\tilde\sigma^0_x(\tilde\Lambda_F)$ and $\tilde\sigma^0_x(\tilde\Lambda_E')$ are invertible with inverses $\tilde\sigma^0_x(\tilde\Lambda_F')$ and $\tilde\sigma^0_x(\tilde\Lambda_E)$, respectively, we conclude that $\mathbf A$ satisfies the (ungraded) Rockland condition, see \eqref{E:grpiab}.
Hence, by Theorem~\ref{T:Rockland}, there exists a left parametrix $\mathbf B\in\Psi^{-k}_\prop(\mathbf F,\mathbf E)$ such that $\mathbf B\mathbf A-\id$ is a smoothing operator.
Putting $B:=\tilde\Lambda_E'\mathbf B\tilde\Lambda_F\in\tilde\Psi^{-k}(F,E)$ and using the fact that $\tilde\Lambda_E'\tilde\Lambda_E-\id$ is a smoothing operator, we see that $BA-\id$ is a smoothing operator.
Hence, $B$ is the desired left parametrix.
The hypoellipticity statements follow immediately from the mapping properties of $B$, that is, $B\colon\Gamma^\infty(F)\to\Gamma^\infty(E)$ and $B\colon\tilde H_\loc^{r-\Re(k)}(F)\to\tilde H^r_\loc(E)$.
Assume $M$ closed.
As in Corollary~\ref{C:hypo} we see that $\ker(A)$ is a finite dimensional subspace of $\Gamma^\infty(E)$.
For the maximal graded hypoelliptic estimate \eqref{E:grmhesti} we use boundedness of $B\colon\tilde H^{r-\Re(k)}(F)\to\tilde H^r(E)$ and the fact that smoothing operators induce bounded operators $\tilde H^{r'}(E)\to\tilde H^r(E)$.
To see the other hypoelliptic estimate, we consider the formal adjoint $A^\sharp\in\tilde\Psi^{\bar k}(F,E)$ with respect to inner products of the form \eqref{E:grHSllrr}, see \eqref{E:Asharp} with $s_1=s_2=s$.
Clearly, $A^\sharp A\in\tilde\Psi^{2\Re(k)}(E)$ satisfies the graded Rockland condition and $\ker(A^\sharp A)=\ker(A)$.
Hence, Corollary~\ref{C:graded-Hodge} below implies that $A^\sharp A+Q$ is invertible with inverse $(A^\sharp A+Q)^{-1}\in\tilde\Psi^{-2\Re(k)}(E)$.
Thus, $B':=(A^\sharp A+Q)^{-1}A^\sharp\in\tilde\Psi^{-k}(F,E)$ is a parametrix such that $B'A=\id-Q$, whence \eqref{E:grmheestiQ}.
\end{proof}


We obtain the following generalization of Corollaries~\ref{C:smooth-Hodge-decomposition}, \ref{C:PsiinvA}, and \ref{C:HsHodge}.


\begin{corollary}[Graded Hodge decomposition]\label{C:graded-Hodge}
Let $E$ be a filtered vector bundle over a closed filtered manifold $M$.
Suppose $A\in\tilde\Psi^k(E)$ satisfies the graded Rockland condition and is formally selfadjoint, $A^\sharp=A$, with respect to a graded Sobolev inner product of the form~\eqref{E:grHSllrr}.
Moreover, let $Q$ denotes the orthogonal projection onto the (finite dimensional) subspace $\ker(A)\subseteq\Gamma^\infty(E)$ with respect to the inner product \eqref{E:grHSllrr}.
Then $A+Q$ is invertible with inverse $(A+Q)^{-1}\in\tilde\Psi^{-k}(E)$.
Consequently, we have topological isomorphisms and Hodge type decompositions:
\begin{align*}
A+Q\colon\Gamma^\infty(E)&\xrightarrow\cong\Gamma^\infty(E)&\Gamma^\infty(E)&=\ker(A)\oplus A(\Gamma^\infty(E))\\
A+Q\colon\tilde H^r(E)&\xrightarrow\cong\tilde H^{r-\Re(k)}(E)&\tilde H^{r-\Re(k)}(E)&=\ker(A)\oplus A(\tilde H^r(E))\\
A+Q\colon\Gamma^{-\infty}(E)&\xrightarrow\cong\Gamma^{-\infty}(E)&\Gamma^{-\infty}(E)&=\ker(A)\oplus A(\Gamma^{-\infty}(E))
\end{align*}
\end{corollary}


\begin{proof}
Let $\mathbf E$ denote the vector bundle $E$ equipped with the trivial filtration, $\mathbf E=\mathbf E^0\supseteq\mathbf E^1=0$.
Recall that $A^\sharp=\tilde\Lambda_s^{-1}(\tilde\Lambda_sA\tilde\Lambda_s^{-1})^*\tilde\Lambda_s$, see~\eqref{E:Asharp}.
Hence, the assumption $A^\sharp=A$ implies that $\mathbf A:=\tilde\Lambda_sA\tilde\Lambda_s^{-1}\in\Psi^k(\mathbf E)$ is formally selfadjoint with respect to the $L^2$ inner product~\eqref{E:llrr}, that is, $\mathbf A^*=\mathbf A$.
Moreover, $\mathbf A$ satisfies the (ungraded) Rockland condition for $\tilde\sigma^s_x(\tilde\Lambda_s)$ is invertible with inverse $\tilde\sigma^{-s}_x(\tilde\Lambda_s^{-1})$, see~\eqref{E:grsigmaAB}.
Hence, according to Corollary~\ref{C:PsiinvA}, $\mathbf A+\mathbf Q\in\Psi^k(\mathbf E)$ is invertible with inverse $(\mathbf A+\mathbf Q)^{-1}\in\Psi^{-k}(\mathbf E)$, where $\mathbf Q\in\mathcal O^{-\infty}(\mathbf E)$ denotes the orthogonal projection onto $\ker(\mathbf A)$, a finite dimensional subspace of $\Gamma^\infty(\mathbf E)$.
Note that $Q:=\tilde\Lambda_s^{-1}\mathbf Q\tilde\Lambda_s\in\mathcal O^{-\infty}(E)$ is the orthogonal projection onto $\ker(A)$ with respect to the inner product \eqref{E:grHSllrr}.
Conjugating with $\tilde\Lambda_s$, we conclude that $A+Q\in\tilde\Psi^k(E)$ is invertible with inverse $(A+Q)^{-1}=\tilde\Lambda_s^{-1}(\mathbf A+\mathbf Q)^{-1}\tilde\Lambda_s\in\tilde\Psi^{-k}(E)$.
The remaining assertions follow at once.
\end{proof}


We obtain the following generalization of Corollaries~\ref{C:smhe} and \ref{C:he}:


\begin{corollary}\label{C:grmaxhesti}
Let $E$ and $F$ be two filtered vector bundles over a closed filtered manifold $M$, suppose $A\in\tilde\Psi^k(E,F)$ satisfies the graded Rockland condition, and let $A^\sharp\in\tilde\Psi^{\bar k+2(s_2-s_1)}(F,E)$ denote the formal adjoint with respect to inner products of the form \eqref{E:grHSllrr}, see \eqref{E:Asharp}.
Then:
\begin{align*}
\Gamma^\infty(E)&=\ker(A)\oplus A^\sharp(\Gamma^\infty(F))\\
\tilde H^r(E)&=\ker(A)\oplus A^\sharp(\tilde H^{r+\Re(k)+2(s_2-s_1)}(F))\\
\Gamma^{-\infty}(E)&=\ker(A)\oplus A^\sharp(\Gamma^{-\infty}(F))
\end{align*}
\end{corollary}


\begin{proof}
Observe that $A^\sharp A\in\tilde\Psi^{2\Re(k)+2(s_2-s_1)}(E)$ satisfies the graded Rockland condition.
Apply Corollary~\ref{C:graded-Hodge} to this operator and use $\ker(A^\sharp A)=\ker(A)$.
\end{proof}


We have the following generalization of Corollary~\ref{C:Fredholm}:


\begin{corollary}[Fredholm operators and index]\label{C:graded-Fredholm}
Let $E$ and $F$ be a filtered vector bundles over a closed filtered manifold $M$.
Suppose $A\in\tilde\Psi^k(E,F)$ is such that $A$ and $A^t$ both satisfy the graded Rockland condition.
Then, for every real number $r$, we have an induced Fredholm operator $A\colon\tilde H^r(E)\to\tilde H^{r-\Re(k)}(F)$ whose index is independent of $r$ and can be expressed as
$$
\ind(A)=\dim\ker A-\dim\ker A^t.
$$
Moreover, this index depends only on the graded Heisenberg principal cosymbol $\tilde\sigma^k(A)\in\tilde\Sigma^k(E,F)$.
\end{corollary}


\begin{proof}
Using Corollary~\ref{C:graded-regularity} we obtain a parametrix $B\in\tilde\Psi^{-k}(F,E)$ such that $BA-\id$ and $AB-\id$ are both smoothing operators, cf.\ Remark~\ref{R:AAtRock}.
We may now proceed exactly as in the proof of Corollary~\ref{C:Fredholm}.
\end{proof}




\subsection{Graded Rockland sequences}\label{SS:gRs}





Throughout this section we assume $M$ to be a closed filtered manifold.
Suppose $E_i$ are filtered vector bundles over $M$, and consider a sequence of operators
\begin{equation}\label{E:grRseq}
\cdots\to\Gamma^\infty(E_{i-1})\xrightarrow{A_{i-1}}\Gamma^\infty(E_i)\xrightarrow{A_i}\Gamma^\infty(E_{i+1})\to\cdots
\end{equation}
where $A_i\in\tilde\Psi^{k_i}(E_i,E_{i+1})$ for some complex numbers $k_i$.
Generalizing Definition~\ref{D:graded_hypoelliptic_seq} for sequences of differential operators, we make the following



\begin{definition}[Graded Rockland sequence]\label{D:gRs}
A sequence of operators as above is said to be a \emph{graded Rockland sequence} if, for every $x\in M$ and every non-trivial irreducible unitary representation $\pi\colon\mathcal T_xM\to U(\mathcal H)$, the sequence
$$
\cdots\to\mathcal H_\infty\otimes\gr(E_{{i-1,x}})\xrightarrow{\bar\pi(\tilde\sigma^{k_{i-1}}_x(A_{i-1}))}
\mathcal H_\infty\otimes\gr(E_{i,x})\xrightarrow{\bar\pi(\tilde\sigma^{k_i}_x(A_i))}
\mathcal H_\infty\otimes\gr(E_{i+1,x})\to\cdots
$$
is weakly exact, that is, the image of each arrow is contained and dense in the kernel of the subsequent arrow.
Here $\mathcal H_\infty$ denotes the subspace of smooth vectors in $\mathcal H$.
\end{definition}


Suppose the sequence in \eqref{E:grRseq} is a Rockland sequence.
Fix real numbers $s_i$ such that $\Re(k_i)+s_{i+1}-s_i$ is independent of $i$, and put
\begin{equation}\label{E:kappa}
\kappa:=\Re(k_i)+s_{i+1}-s_i.
\end{equation}
Let $\mathbf E_i$ denote the vector bundle $E_i$ considered as a trivially filtered bundle, that is, equipped with the filtration $\mathbf E_i=\mathbf E_i^0\supseteq\mathbf E^1_i=0$.
Fix a smooth volume density on $M$ as well as smooth fiber wise Hermitian inner products $h_i$ on $\mathbf E_i$ and let $\llangle-,-\rrangle_{L^2(\mathbf E_i)}$ denote the associated $L^2$ inner product on sections of $\mathbf E_i$, see \eqref{E:llrr}.
Moreover, choose invertible $\tilde\Lambda_i\in\tilde\Psi^{s_i}(E_i,\mathbf E_i)$ with $\tilde\Lambda_i^{-1}\in\tilde\Psi^{-s_i}(\mathbf E_i,E_i)$, see Lemma~\ref{L:grLambda}, and consider the associated graded Sobolev inner product on sections of $E_i$,
\begin{equation}\label{E:grHSllrri}
\llangle\psi_1,\psi_2\rrangle_{\tilde H^{s_i}(E_i)}
:=\llangle\tilde\Lambda_i\psi_1,\tilde\Lambda_i\psi\rrangle_{L^2(\mathbf E_i)}
=\langle\tilde\Lambda_i^t(h_i\otimes dx)\tilde\Lambda_i\psi_1,\psi_2\rangle
\end{equation}
where $\psi_1,\psi_2\in\Gamma^\infty(E_i)$.
Let $A_i^\sharp\in\tilde\Psi^{2\kappa-k_i}(E_{i+1},E_i)$, 
$$
A_i^\sharp
=\tilde\Lambda_i^{-1}(\tilde\Lambda_{i+1}A_i\tilde\Lambda_i^{-1})^*\tilde\Lambda_{i+1}
=(\tilde\Lambda_i^t(h_i\otimes dx)\tilde\Lambda_i)^{-1}\,A^t_i\,\tilde\Lambda_{i+1}^t(h_{i+1}\otimes dx)\tilde\Lambda_{i+1}
$$
denote the formal adjoint of $A_i$ with respect to these inner products, that is, 
$$
\llangle A_i^\sharp\phi,\psi\rrangle_{\tilde H^{s_i}(E_i)}=\llangle\phi,A_i\psi\rrangle_{\tilde H^{s_{i+1}}(E_{i+1})}
$$ 
for all $\psi\in\Gamma^\infty(E_i)$ and $\phi\in\Gamma^\infty(E_{i+1})$.
Let us finally consider the Laplace type operators 
$$
B_i:=A_{i-1}A_{i-1}^\sharp+A_i^\sharp A_i.
$$
Note that $B_i=B_i^\sharp\in\tilde\Psi^{2\kappa}(E_i)$, see \eqref{E:kappa} and \eqref{E:BAsharp}.


\begin{lemma}\label{L:Delta}
The operator $B_i$ satisfies the graded Rockland condition.
\end{lemma}


\begin{proof}
Note that $\mathbf B_i:=\tilde\Lambda_iB_i\tilde\Lambda_i^{-1}\in\Psi^{2\kappa}(\mathbf E_i)$ is of the form $\mathbf B_i=\mathbf A_i^*\mathbf A_i+\mathbf A_{i-1}\mathbf A_{i-1}^*$ where $\mathbf A_i:=\tilde\Lambda_{i+1}A_i\tilde\Lambda_i^{-1}\in\Psi^{k_i+s_{i+1}-s_i}(\mathbf E_i,\mathbf E_{i+1})$.
Using \eqref{E:piab}, \eqref{E:pia*}, and Remark~\ref{R:Psi:adjoint}, we obtain
\begin{multline*}
\bar\pi(\sigma^{2\kappa}_x(\mathbf B_i))
=\bar\pi(\sigma^{k_{i-1}+s_i-s_{i-1}}_x(\mathbf A_{i-1}))\bar\pi(\sigma^{k_{i-1}+s_i-s_{i-1}}_x(\mathbf A_{i-1}))^*
\\+\bar\pi(\sigma^{k_i+s_{i+1}-s_i}_x(\mathbf A_i))^*\bar\pi(\sigma^{k_i+s_{i+1}-s_i}_x(\mathbf A_i)).
\end{multline*}
Since the operators $\mathbf A_i$ form an (ungraded) Rockland sequence, one readily concludes that $\mathbf B_i$ satisfies the (ungraded) Rockland condition, cf.\ the proof of Lemma~\ref{L:rockseq}.
Clearly, this implies that $B_i$ satisfies the graded Rockland condition, see \eqref{E:grpiab}.
\end{proof}


In view of Lemma~\ref{L:Delta}, Corollary~\ref{C:graded-Hodge} applies to each of the operators $B_i$ and we obtain the following generalization of Corollaries~\ref{C:RShypo}, \ref{C:Hodge}, \ref{C:regrockseq}, and \ref{C:HsHodge-seq}:


\begin{corollary}\label{C:grHodge}
The operator $(A_{i-1}^\sharp,A_i)$ is hypoelliptic, that is, if $\psi$ is a distributional section of $E_i$ such that $A_{i-1}^\sharp\psi$ and $A_i\psi$ are both smooth, then $\psi$ was smooth.
More precisely, if $\psi\in\Gamma^{-\infty}(E_i)$ is such that $A_{i-1}^\sharp\psi\in\tilde H^{r-2\kappa+\Re(k_{i-1})}(E_{i-1})$ and $A_i\psi\in\tilde H^{r-\Re(k_i)}(E_{i+1})$, then $\psi\in\tilde H^r(E_i)$.
Moreover, there exists a constant $C=C_{A_i,r}\geq0$ such that the maximal graded hypoelliptic estimate
\begin{equation*}
\|\psi\|_{\tilde H^r(E_i)}\leq C\left(\|A^\sharp_{i-1}\psi\|_{\tilde H^{r-2\kappa+\Re(k_{i-1})}(E_{i-1})}+\|Q_i\psi\|_{\ker(B_i)}+\|A_i\psi\|_{\tilde H^{r-\Re(k_i)}(E_{i+1})}\right)
\end{equation*}
holds for all $\psi\in\tilde H^r(E_i)$.
Here $r$ is any real number, $Q_i$ denotes the orthogonal projection onto the (finite dimensional) subspace $\ker(B_i)\subseteq\Gamma^\infty(E_i)$ with respect to the inner product~\eqref{E:grHSllrri}, $\|-\|_{\ker(B_i)}$ denotes any norm on $\ker(B_i)$, and $\|-\|_{\tilde H^r(E_i)}$ is any norm generating the Hilbert space topology on the graded Sobolev space $\tilde H^r(E_i)$. 
Furthermore,
$$
\ker(B_i)=\ker(A_{i-1}^\sharp|_{\Gamma^{-\infty}(E_i)})\cap\ker(A_i|_{\Gamma^{-\infty}(E_i)})
=\ker(A_{i-1}^\sharp|_{\Gamma^\infty(E_i)})\cap\ker(A_i|_{\Gamma^\infty(E_i)}).
$$
If, moreover, $A_iA_{i-1}=0$, then we have Hodge type decompositions
\begin{align*}
\Gamma^\infty(E_i)&=A_{i-1}(\Gamma^\infty(E_{i-1}))\oplus\ker(B_i)\oplus A_i^\sharp(\Gamma^\infty(E_{i+1}))\\
\tilde H^r(E_i)&=A_{i-1}\bigl(\tilde H^{r+\Re(k_{i-1})}(E_{i-1})\bigr)\oplus\ker(B_i)\oplus A_i^\sharp\bigl(\tilde H^{2\kappa-\Re(k_i)}(E_{i+1})\bigr)\\
\Gamma^{-\infty}(E_i)&=A_{i-1}(\Gamma^{-\infty}(E_{i-1}))\oplus\ker(B_i)\oplus A_i^\sharp(\Gamma^{-\infty}(E_{i+1}))
\end{align*}
as well as:
\begin{align*}
\ker(A_i|_{\Gamma^\infty(E_i)})&=A_{i-1}(\Gamma^\infty(E_{i-1}))\oplus\ker(B_i)\\
\ker(A_i|_{\tilde H^r(E_i)})&=A_{i-1}(\tilde H^{r+\Re(k_{i-1})}(E_{i-1}))\oplus\ker(B_i)\\
\ker(A_i|_{\Gamma^{-\infty}(E_i)})&=A_{i-1}(\Gamma^{-\infty}(E_{i-1}))\oplus\ker(B_i)
\end{align*}
In particular, every cohomology class admits a unique harmonic representative:
$$
\frac{\ker(A_i|_{\Gamma^{-\infty}(E_i)})}{\img(A_{i-1}|_{\Gamma^{-\infty}(E_{i-1})})}
=\frac{\ker(A_i|_{\Gamma^\infty(E_i)})}{\img(A_{i-1}|_{\Gamma^\infty(E_{i-1})})}
=\ker(B_i)
=\ker(A_{i-1}^\sharp)\cap\ker(A_i).
$$
\end{corollary}





\appendix






\section{Engel structures}\label{A:Engel}




The purpose of this appendix is to verify the formulas for the Rockland complex \eqref{E:DEngel} in Example~\ref{Ex:Engel}.
Some of the subsequent computations have been carried out using the computer algebra system Singular \cite{DGPS16}.


Let $\goe=\goe_{-3}\oplus\goe_{-2}\oplus\goe_{-1}$ denote the graded nilpotent Lie algebra with graded basis $X_4|X_3|X_2,X_1$ and non-trivial brackets
$$
[X_1,X_2]=X_3,\quad[X_1,X_3]=X_4,\quad[X_2,X_3]=0.
$$
The vertical bars above indicate that $X_1,X_2$ is a basis of $\goe_{-1}$, $X_3$ is a basis of $\goe_{-2}$, and $X_4$ is a basis of $\goe_{-3}$.
Note that the osculating algebras of an Engel structure are isomorphic to this graded nilpotent Lie algebra $\goe$.


Let $\alpha^1,\alpha^2|\alpha^3|\alpha^4$ denote the dual basis of $\goe^*=\goe^*_1\oplus\goe^*_2\oplus\goe^*_3$, that is $\alpha^i(X_j)=\delta^i_j$.
We use the following ordered basis of $\Lambda^q\goe^*$:
\begin{align*}
\Lambda^0\goe^*&:1\\
\Lambda^1\goe^*&:\alpha^1,\alpha^2|\alpha^3|\alpha^4\\
\Lambda^2\goe^*&:\alpha^{12}|\alpha^{13},\alpha^{23}|\alpha^{14},\alpha^{24}|\alpha^{34}\\
\Lambda^3\goe^*&:\alpha^{123}|\alpha^{124}|\alpha^{134},\alpha^{234}\\
\Lambda^4\goe^*&:\alpha^{1234}
\end{align*}
Here we use the notation $\alpha^{i_1\cdots i_k}=\alpha^{i_1}\wedge\cdots\wedge\alpha^{i_k}$.


With respect to the ordered bases above, the Chevalley--Eilenberg differentials $\partial_q\colon\Lambda^q\goe^*\to\Lambda^{q+1}\goe^*$
%, $(\partial_q\alpha)(Y_0,\dotsc,Y_q)=\sum_{i<j}(-1)^{i+j}\alpha([Y_i,Y_j],Y_0,\dotsc,\hat Y_i,\dotsc,\hat Y_j,\dotsc,Y_q)$,
correspond to the following matrices where the lines indicate the different degrees:

$$
\partial_0=\left(\begin{array}{c}
0\\0\\\hline0\\\hline0
\end{array}\right)
$$

$$
\partial_1=\left(\begin{array}{cc|c|c}
0&0&-1&0\\
\hline
0&0&0&-1\\
0&0&0&0\\
\hline
0&0&0&0\\
0&0&0&0\\
\hline
0&0&0&0
\end{array}\right)
$$

$$
\partial_2=\left(\begin{array}{c|cc|cc|c}
0&0&0&0&-1&0\\\hline
0&0&0&0&0&-1\\\hline
0&0&0&0&0&0\\
0&0&0&0&0&0
\end{array}\right)
$$

$$
\partial_3=\left(\begin{array}{c|c|cc}
0&0&0&0
\end{array}\right)
$$


Using the identification $\Omega^q(G)=C^\infty(G)\otimes\Lambda^q\goe^*=C^\infty(G)^{\left(\substack{4\\q}\right)}$ provided by left trivialization of $TG$ and the graded basis above, the exterior derivative $d_q\colon\Omega^q(G)\to\Omega^{q+1}(G)$ on the nilpotent Lie group $G$ can be expressed by the following matrices.
Here $X_i$ can be understood as a left invariant differential operator, more precisely, differentiation with respect to the left invariant vector field $X_i$. Equivalently, we may regard these matrices to have entries in the universal enveloping algebra of $\goe$. 

$$
d_0=\left(\begin{array}{c}
X_1\\X_2\\\hline X_3\\\hline X_4
\end{array}\right)
$$

$$
d_1=\left(\begin{array}{cc|c|c}
-X_2&X_1&-1&0\\
\hline
-X_3&0&X_1&-1\\
0&-X_3&X_2&0\\
\hline
-X_4&0&0&X_1\\
0&-X_4&0&X_2\\
\hline
0&0&-X_4&X_3
\end{array}\right)
$$

$$
d_2=\left(\begin{array}{c|cc|cc|c}
X_3&-X_2&X_1&0&-1&0\\\hline
X_4&0&0&-X_2&X_1&-1\\\hline
0&X_4&0&-X_3&0&X_1\\
0&0&X_4&0&-X_3&X_2
\end{array}\right)
$$

$$
d_3=\left(\begin{array}{c|c|cc}
-X_4&X_3&-X_2&X_1
\end{array}\right)
$$

For the codifferential $\delta_q\colon\Lambda^q\goe^*\to\Lambda^{q-1}\goe^*$ we use the adjoint of $\partial$ with respect to the inner product which makes our basis $\alpha^{i_1\cdots i_k}$ orthonormal. 
In matrix form this reads:

$$
\tilde\delta_1=\delta_1=\left(\begin{array}{cc|c|c}
0&0&0&0
\end{array}\right)
$$

$$
\tilde\delta_2=\delta_2=\left(\begin{array}{c|cc|cc|c}
0&0&0&0&0&0\\
0&0&0&0&0&0\\\hline
-1&0&0&0&0&0\\\hline
0&-1&0&0&0&0
\end{array}\right)
$$

$$
\tilde\delta_3=\delta_3=\left(\begin{array}{c|c|cc}
0&0&0&0\\\hline
0&0&0&0\\
0&0&0&0\\\hline
0&0&0&0\\
-1&0&0&0\\\hline
0&-1&0&0
\end{array}\right)
$$

$$
\tilde\delta_4=\delta_4=\left(\begin{array}{c}
0\\\hline
0\\\hline
0\\
0
\end{array}\right)
$$

For the algebraic $\tilde\Box_q\colon\Lambda^q\goe^*\to\Lambda^q\goe^*$, $\tilde\Box_q=\tilde\delta_{q+1}\partial_q+\partial_{q-1}\tilde\delta_q$, we obtain:

$$
\tilde\Box_0=\left(\begin{array}{c}
0
\end{array}\right)
$$

$$
\tilde\Box_1=\left(\begin{array}{cc|c|c}
0&0&0&0\\
0&0&0&0\\\hline
0&0&1&0\\\hline
0&0&0&1
\end{array}\right)
$$

$$
\tilde\Box_2=\left(\begin{array}{c|cc|cc|c}
1&0&0&0&0&0\\\hline
0&1&0&0&0&0\\
0&0&0&0&0&0\\\hline
0&0&0&0&0&0\\
0&0&0&0&1&0\\\hline
0&0&0&0&0&1
\end{array}\right)
$$

$$
\tilde\Box_3=\left(\begin{array}{c|c|cc}
1&0&0&0\\\hline
0&1&0&0\\\hline
0&0&0&0\\
0&0&0&0
\end{array}\right)
$$

$$
\tilde\Box_4=\left(\begin{array}{c}
0
\end{array}\right)
$$

For the spectral projectors onto the zero eigenspaces, $\tilde P\colon\Lambda^q\goe^*\to\Lambda^q\goe^*$, we find:

$$
\tilde P_0=\left(\begin{array}{c}
1
\end{array}\right)
$$

$$
\tilde P_1=\left(\begin{array}{cc|c|c}
1&0&0&0\\
0&1&0&0\\\hline
0&0&0&0\\\hline
0&0&0&0
\end{array}\right)
$$

$$
\tilde P_2=\left(\begin{array}{c|cc|cc|c}
0&0&0&0&0&0\\\hline
0&0&0&0&0&0\\
0&0&1&0&0&0\\\hline
0&0&0&1&0&0\\
0&0&0&0&0&0\\\hline
0&0&0&0&0&0
\end{array}\right)
$$

$$
\tilde P_3=\left(\begin{array}{c|c|cc}
0&0&0&0\\\hline
0&0&0&0\\\hline
0&0&1&0\\
0&0&0&1
\end{array}\right)
$$

$$
\tilde P_4=\left(\begin{array}{c}
1
\end{array}\right)
$$

Moreover, $\Box_q\colon\Omega^q(G)\to\Omega^q(G)$, $\Box_q=\delta_{q+1}d_q+d_{q-1}\delta_q$, becomes:

$$
\Box_0=\left(\begin{array}{c}
0
\end{array}\right)
$$

$$
\Box_1=\left(\begin{array}{cc|c|c}
0&0&0&0\\
0&0&0&0\\\hline
X_2&-X_1&1&0\\\hline
X_3&0&-X_1&1
\end{array}\right)
$$

$$
\Box_2=\left(\begin{array}{c|cc|cc|c}
1&0&0&0&0&0\\\hline
-X_1&1&0&0&0&0\\
-X_2&0&0&0&0&0\\\hline
0&-X_1&0&0&0&0\\
-X_3&0&-X_1&0&1&0\\\hline
0&-X_3&0&X_2&-X_1&1
\end{array}\right)
$$

$$
\Box_3=\left(\begin{array}{c|c|cc}
1&0&0&0\\\hline
-X_1&1&0&0\\\hline
0&-X_1&0&0\\
X_3&-X_2&0&0
\end{array}\right)
$$

$$
\Box_4=\left(\begin{array}{c}
0
\end{array}\right)
$$

For the differential projectors $P_q\colon\Omega^q(G)\to\Omega^q(G)$ we find:

$$
P_0=\left(\begin{array}{c}
1
\end{array}\right)
$$

$$
P_1=\left(\begin{array}{cc|c|c}
1&0&0&0\\
0&1&0&0\\\hline
-X_2&X_1&0&0\\\hline
-X_{12}-X_3&X_{11}&0&0
\end{array}\right)
$$

$$
P_2=\left(\begin{array}{c|cc|cc|c}
0&0&0&0&0&0\\\hline
0&0&0&0&0&0\\
X_2&0&1&0&0&0\\\hline
X_{11}&X_1&0&1&0&0\\
X_{12}&0&X_1&0&0&0\\\hline
2X_{13}-X_4&X_3-X_{12}&X_{11}&-X_2&0&0
\end{array}\right)
$$

$$
P_3=\left(\begin{array}{c|c|cc}
0&0&0&0\\
0&0&0&0\\\hline
X_{11}&X_1&1&0\\\hline
X_{12}-2X_3&X_2&0&1
\end{array}\right)
$$

$$
P_4=\left(\begin{array}{c}
1
\end{array}\right)
$$

For $L_q\colon\Omega^q(G)\to\Omega^q(G)$, $L_q=P_q\tilde P_q+(1-P_q)(1-\tilde P_q)$, we obtain:

$$
L_0=\left(\begin{array}{c}
1
\end{array}\right)
$$

$$
L_1=\left(\begin{array}{cc|c|c}
1&0&0&0\\
0&1&0&0\\\hline
-X_2&X_1&1&0\\\hline
-X_{12}-X_3&X_{11}&0&1
\end{array}\right)
$$

$$
L_2=\left(\begin{array}{c|cc|cc|c}
1&0&0&0&0&0\\\hline
0&1&0&0&0&0\\
-X_2&0&1&0&0&0\\\hline
-X_{11}&-X_1&0&1&0&0\\
-X_{12}&0&X_1&0&1&0\\\hline
X_4-2X_{13}&X_{12}-X_3&X_{11}&-X_2&0&1
\end{array}\right)
$$

$$
L_3=\left(\begin{array}{c|c|cc}
1&0&0&0\\
0&1&0&0\\\hline
-X_{11}&-X_1&1&0\\\hline
2X_3-X_{12}&-X_2&0&1
\end{array}\right)
$$

$$
L_4=\left(\begin{array}{c}
1
\end{array}\right)
$$


And for their inverses, $L_q^{-1}\colon\Omega^q(G)\to\Omega^q(G)$, we have:


$$
L_0^{-1}=\left(\begin{array}{c}
1
\end{array}\right)
$$

$$
L_1^{-1}=\left(\begin{array}{cc|c|c}
1&0&0&0\\
0&1&0&0\\\hline
X_2&-X_1&1&0\\\hline
X_{12}+X_3&-X_{11}&0&1
\end{array}\right)
$$

$$
L_2^{-1}=\left(\begin{array}{c|cc|cc|c}
1&0&0&0&0&0\\\hline
0&1&0&0&0&0\\
X_2&0&1&0&0&0\\\hline
X_{11}&X_1&0&1&0&0\\
0&0&-X_1&0&1&0\\\hline
0&0&-X_{11}&X_2&0&1
\end{array}\right)
$$

$$
L_3^{-1}=\left(\begin{array}{c|c|cc}
1&0&0&0\\\hline
0&1&0&0\\\hline
X_{11}&X_1&1&0\\
X_{12}-2X_3&X_2&0&1
\end{array}\right)
$$

$$
L_4^{-1}=\left(\begin{array}{c}
1
\end{array}\right)
$$

The conjugates $L_{q+1}^{-1}d_qL_q\colon\Omega^q(G)\to\Omega^{q+1}(G)$ are:

$$
L_1^{-1}d_0L_0=\left(\begin{array}{c}
X_1\\X_2\\\hline0\\\hline0
\end{array}\right)
$$

$$
L_2^{-1}d_1L_1=\left(\begin{array}{cc|c|c}
0&0&-1&0\\
\hline
0&0&X_1&-1\\
-X_{22}&X_{12}-2X_3&0&0\\
\hline
-X_4-X_{112}-X_{13}&X_{111}&0&0\\
0&0&-X_{12}&X_2\\
\hline
0&0&-X_{112}-X_4&X_{12}
\end{array}\right)
$$

$$
L_3^{-1}d_2L_2=\left(\begin{array}{c|cc|cc|c}
X_3&-X_2&0&0&-1&0\\\hline
X_4&0&0&0&X_1&-1\\\hline
0&0&X_{111}&-X_3-X_{12}&0&0\\
0&0&3X_4-3X_{13}+X_{112}&-X_{22}&0&0
\end{array}\right)
$$

$$
L_4^{-1}d_3L_3=\left(\begin{array}{c|c|cc}
0&0&-X_2&X_1
\end{array}\right)
$$

We will use the following ordered bases of $\img(\tilde P_q)$ and $\ker(\tilde P_q)$:
\begin{align*}
\img(\tilde P_0)&:1
&\ker(\tilde P_0)&:\emptyset
\\
\img(\tilde P_1)&:\alpha^1,\alpha^2
&\ker(\tilde P_1)&:\alpha^3|\alpha^4
\\
\img(\tilde P_2)&:\alpha^{23}|\alpha^{14}
&\ker(\tilde P_2)&:\alpha^{12}|\alpha^{13}|\alpha^{24}|\alpha^{34}
\\
\img(\tilde P_3)&:\alpha^{134},\alpha^{234}
&\ker(\tilde P_3)&:\alpha^{123}|\alpha^{124}
\\
\img(\tilde P_4)&:\alpha^{1234}
&\ker(\tilde P_4)&:\emptyset
\end{align*}


Then $D_q=\tilde P_{q+1}L_{q+1}^{-1}d_qL_q|_{\img(\tilde P_q)}\colon\Gamma^\infty(\img(\tilde P_q))\to\Gamma^\infty(\img(\tilde P_{q+1}))$ becomes:

$$
D_0=\left(\begin{array}{c}X_1\\X_2\end{array}\right)
$$

$$
D_1=\left(\begin{array}{cc}-X_{22}&X_{12}-2X_3\\\hline-X_4-X_{112}-X_{13}&X_{111}\end{array}\right)
$$

$$
D_2=\left(\begin{array}{c|c}X_{111}&-X_3-X_{12}\\ 3X_4-3X_{13}+X_{112}&-X_{22}\end{array}\right)
$$

$$
D_3=\left(\begin{array}{cc}-X_2&X_1\end{array}\right)
$$


This proves the formulas for the Rockland complex \eqref{E:DEngel} in Example~\ref{Ex:Engel}.
Let us also point out that $B_q=(\id-\tilde P_{q+1})L_{q+1}^{-1}d_qL_q|_{\ker(\tilde P_q)}\colon\Gamma^\infty(\ker(\tilde P_q))\to\Gamma^\infty(\ker(\tilde P_{q+1}))$ become:

$$
B_1=\left(\begin{array}{c|c}-1&0\\\hline X_1&-1\\\hline-X_{12}&X_2\\\hline -X_{112}-X_4&X_{12}\end{array}\right)
$$

$$
B_2=\left(\begin{array}{c|c|c|c}X_3&-X_2&-1&0\\\hline X_4&0&X_1&-1\end{array}\right)
$$

Furthermore, the operators $G_q\colon\Gamma^\infty(\ker(\tilde P_q))\to\Gamma^\infty(\ker(\tilde P_q))$ satisfying

$$
G_2^{-1}B_1G_1=\partial_1|_{\ker(\tilde P_1)}=\left(\begin{array}{c|c}
-1&0\\\hline
0& -1\\\hline
0& 0\\\hline
0& 0  
\end{array}\right)
$$

$$
G_3^{-1}B_2G_2=\partial_2|_{\ker(\tilde P_2)}=\left(\begin{array}{c|c|c|c}
0&0&-1&0\\\hline
0&0&0& -1
\end{array}\right),
$$

see Theorem~\ref{T:D}(b), are given by:

$$
G_1=\left(\begin{array}{c|c}
1&0\\\hline
0&1 
\end{array}\right)
$$

$$
G_2=\left(\begin{array}{c|c|c|c}
1&         0&     0&0\\\hline
-X_1&       1&     0&0\\\hline
X_{12}&     -X_2&   1&0\\\hline
X_{112}+X_4&-X_{12}&0&1 
\end{array}\right)
$$

$$
G_3=\left(\begin{array}{c|c}
1&  0\\\hline
-X_1&1 
\end{array}\right)
$$

And their inverses, $G_q^{-1}\colon\Gamma^\infty(\ker(\tilde P_q))\to\Gamma^\infty(\ker(\tilde P_q))$, are:

$$
G^{-1}_1=\left(\begin{array}{c|c}
1&0\\\hline
0&1 
\end{array}\right)
$$

$$
G^{-1}_2=\left(\begin{array}{c|c|c|c}
1&        0&    0&0\\\hline
X_1&       1&    0&0\\\hline
-X_3&      X_2&   1&0\\\hline
-X_{13}-X_4&X_{12}&0&1 
\end{array}\right)
$$

$$
G^{-1}_3=\left(\begin{array}{c|c}
1& 0\\\hline
X_1&1 
\end{array}\right)
$$










\section{Generic rank two distributions in dimension five}\label{A:G2}








The purpose of this appendix is to verify the formulas for the Rockland complex \eqref{E:BGG235dR} in Example~\ref{Ex:BGG235}.
Most of the subsequent computations have been carried out using the computer algebra system Singular \cite{DGPS16}.


Let $\goe_-=\goe_{-3}\oplus\goe_{-2}\oplus\goe_{-1}$ denote the graded nilpotent Lie algebra with graded basis $X_5,X_4|X_3|X_2,X_1$ and non-trivial brackets
$$
[X_1,X_2]=X_3,\quad[X_1,X_3]=X_4,\quad[X_2,X_3]=X_5.
$$
The vertical bars above indicate that $X_1,X_2$ is a basis of $\goe_{-1}$, $X_3$ is a basis of $\goe_{-2}$, and $X_4,X_5$ is a basis of $\goe_{-3}$.
Note that the osculating algebras of a generic rank two distribution in dimension five are isomorphic to this graded nilpotent Lie algebra $\goe_-$.


Let $\alpha^1,\alpha^2|\alpha^3|\alpha^4,\alpha^5$ denote the dual basis of $\goe^*_-=\goe^*_1\oplus\goe^*_2\oplus\goe^*_3$, that is $\alpha^i(X_j)=\delta^i_j$.
We use the following ordered basis of $\Lambda^q\goe^*_-$:
\begin{align*}
\Lambda^0\goe^*_-&:1\\
\Lambda^1\goe^*_-&:\alpha^1,\alpha^2|\alpha^3|\alpha^4,\alpha^5\\
\Lambda^2\goe^*_-&:\alpha^{12}|\alpha^{13},\alpha^{23}|\alpha^{14},\alpha^{15},\alpha^{24},\alpha^{25}|\alpha^{34},\alpha^{35}|\alpha^{45}\\
\Lambda^3\goe^*_-&:\alpha^{123}|\alpha^{124},\alpha^{125}|\alpha^{134},\alpha^{135},\alpha^{234},\alpha^{235}|\alpha^{145},\alpha^{245}|\alpha^{345}\\
\Lambda^4\goe^*_-&:\alpha^{1234},\alpha^{1235}|\alpha^{1245}|\alpha^{1345},\alpha^{2345}\\
\Lambda^5\goe^*_-&:\alpha^{12345}
\end{align*}
Here we use the notation $\alpha^{i_1\cdots i_k}=\alpha^{i_1}\wedge\cdots\wedge\alpha^{i_k}$.


With respect to the ordered bases above, the Chevalley--Eilenberg differentials $\partial_q\colon\Lambda^q\goe^*_-\to\Lambda^{q+1}\goe^*_-$
%, $(\partial_q\alpha)(Y_0,\dotsc,Y_q)=\sum_{i<j}(-1)^{i+j}\alpha([Y_i,Y_j],Y_0,\dotsc,\hat Y_i,\dotsc,\hat Y_j,\dotsc,Y_q)$,
correspond to the following matrices where the lines indicate the different degrees:

$$
\partial_0=\left(\begin{array}{c}
0\\0\\\hline0\\\hline0\\0
\end{array}\right)
$$

$$
\partial_1=\left(\begin{array}{cc|c|cc}
0&0&-1&0&0\\
\hline
0&0&0&-1&0\\
0&0&0&0&-1\\
\hline
0&0&0&0&0\\
0&0&0&0&0\\
0&0&0&0&0\\
0&0&0&0&0\\
\hline
0&0&0&0&0\\
0&0&0&0&0\\
\hline
0&0&0&0&0
\end{array}\right)
$$

$$
\partial_2=\left(\begin{array}{c|cc|cccc|cc|c}
0&0&0&0&1&-1&0&0&0&0\\
\hline
0&0&0&0&0&0&0&-1&0&0\\
0&0&0&0&0&0&0&0&-1&0\\
\hline
0&0&0&0&0&0&0&0&0&0\\
0&0&0&0&0&0&0&0&0&-1\\
0&0&0&0&0&0&0&0&0&1\\
0&0&0&0&0&0&0&0&0&0\\
\hline
0&0&0&0&0&0&0&0&0&0\\
0&0&0&0&0&0&0&0&0&0\\
\hline
0&0&0&0&0&0&0&0&0&0
\end{array}\right)
$$

$$
\partial_3=\left(\begin{array}{c|cc|cccc|cc|c}
0&0&0&0&0&0&0&-1&0&0\\
0&0&0&0&0&0&0&0&-1&0\\
\hline
0&0&0&0&0&0&0&0&0&-1\\
\hline
0&0&0&0&0&0&0&0&0&0\\
0&0&0&0&0&0&0&0&0&0
\end{array}\right)
$$

$$
\partial_4=\left(\begin{array}{cc|c|cc}
0&0&0&0&0
\end{array}\right)
$$

Using the identification $\Omega^q(G_-)=C^\infty(G_-)\otimes\Lambda^q\goe^*_-=C^\infty(G_-)^{\left(\substack{5\\q}\right)}$ provided by left trivialization of $TG_-$ and the graded basis above, the exterior derivative $d_q\colon\Omega^q(G_-)\to\Omega^{q+1}(G_-)$ on the nilpotent Lie group $G_-$ can be expressed by the following matrices.
Here $X_i$ can be understood as a left invariant differential operator, more precisely, differentiation withe respect to the left invariant vector field $X_i$.
Equivalently, we may regard these matrices to have entries in the universal enveloping algebra of $\goe_-$.

$$
d_0=\left(\begin{array}{c}
X_1\\
X_2\\
\hline
X_3\\
\hline
X_4\\
X_5
\end{array}\right)
$$

$$
d_1=\left(\begin{array}{cc|c|cc}
-X_2&X_1&-1&0&0\\
\hline
-X_3&0&X_1&-1&0\\
0&-X_3&X_2&0&-1\\
\hline
-X_4&0&0&X_1&0\\
-X_5&0&0&0&X_1\\
0&-X_4&0&X_2&0\\
0&-X_5&0&0&X_2\\
\hline
0&0&-X_4&X_3&0\\
\hline
0&0&-X_5&0&X_3\\
0&0&0&-X_5&X_4
\end{array}\right)
$$

$$
d_2=\left(\begin{array}{c|cc|cccc|cc|c}
X_3&-X_2&X_1&0&1&-1&0&0&0&0\\
\hline
X_4&0&0&-X_2&0&X_1&0&-1&0&0\\
X_5&0&  0& 0&  -X_2&0&  X_1& 0& -1& 0\\
\hline
0& X_4& 0& -X_3&0&  0&  0&  X_1&0&  0\\
0& X_5& 0& 0&  -X_3&0&  0&  0& X_1& -1\\
0& 0&  X_4&0&  0&  -X_3&0&  X_2&0&  1\\
0& 0&  X_5&0&  0&  0&  -X_3&0& X_2& 0\\
\hline
0& 0&  0& X_5& -X_4&0&  0&  0& 0&  X_1\\
0& 0&  0& 0&  0&  X_5& -X_4&0& 0&  X_2\\
\hline
0& 0&  0& 0&  0&  0&  0&  X_5&-X_4&X_3
\end{array}\right)
$$

$$
d_3=\left(\begin{array}{c|cc|cccc|cc|c}
-X_4&X_3&0&-X_2&0&X_1&0&-1&0&0\\
-X_5&0&X_3&0&-X_2&0&X_1&0&-1&0\\
\hline
0&-X_5&X_4&0&0&0&0&-X_2&X_1&-1\\
\hline
0&0&0&-X_5&X_4&0&0&-X_3&0&X_1\\
0&0&0&0&0&-X_5&X_4&0&-X_3&X_2
\end{array}\right)
$$

$$
d_4=\left(\begin{array}{cc|c|cc}
X_5&-X_4&X_3&-X_2&X_1
\end{array}\right)
$$

For the codifferential $\delta_q\colon\Lambda^q\goe^*_-\to\Lambda^{q-1}\goe^*_-$ we use the adjoint of $\partial$ with respect to the inner product which makes our basis $\alpha^{i_1\cdots i_k}$ orthonormal.
In matrix form this reads:

$$
\tilde\delta_1=\delta_1=\left(\begin{array}{cc|c|cc}
0&0&0&0&0
\end{array}\right)
$$

$$
\tilde\delta_2=\delta_2=\left(\begin{array}{c|cc|cccc|cc|c}
0& 0& 0& 0&0&0&0&0&0&0\\
0& 0& 0& 0&0&0&0&0&0&0\\
\hline
-1&0& 0& 0&0&0&0&0&0&0\\
\hline
0& -1&0& 0&0&0&0&0&0&0\\
0& 0& -1&0&0&0&0&0&0&0
\end{array}\right)
$$

$$
\tilde\delta_3=\delta_3=\left(\begin{array}{c|cc|cccc|cc|c}
0& 0& 0& 0&0& 0&0&0&0&0\\
\hline
0& 0& 0& 0&0& 0&0&0&0&0\\
0& 0& 0& 0&0& 0&0&0&0&0\\
\hline
0& 0& 0& 0&0& 0&0&0&0&0\\
1& 0& 0& 0&0& 0&0&0&0&0\\
-1&0& 0& 0&0& 0&0&0&0&0\\
0& 0& 0& 0&0& 0&0&0&0&0\\
\hline
0& -1&0& 0&0& 0&0&0&0&0\\
0& 0& -1&0&0& 0&0&0&0&0\\
\hline
0& 0& 0& 0&-1&1&0&0&0&0
\end{array}\right)
$$

$$
\tilde\delta_4=\delta_4=\left(\begin{array}{cc|c|cc}
0& 0& 0& 0&0\\
\hline
0& 0& 0& 0&0\\
0& 0& 0& 0&0\\
\hline
0& 0& 0& 0&0\\
0& 0& 0& 0&0\\
0& 0& 0& 0&0\\
0& 0& 0& 0&0\\
\hline
-1&0& 0& 0&0\\
0& -1&0& 0&0\\
\hline
0& 0& -1&0&0
\end{array}\right)
$$

$$
\tilde\delta_5=\delta_5=\left(\begin{array}{c}
0\\
0\\
\hline
0\\
\hline
0\\
0
\end{array}\right)
$$


For the algebraic $\tilde\Box_q\colon\Lambda^q\goe^*_-\to\Lambda^q\goe^*_-$, $\tilde\Box_q=\tilde\delta_{q+1}\partial_q+\partial_{q-1}\tilde\delta_q$, we obtain:

$$
\tilde\Box_0=\left(\begin{array}{c}
0
\end{array}\right)
$$

$$
\tilde\Box_1=\left(\begin{array}{cc|c|cc}
0&0&0&0&0\\
0&0&0&0&0\\
\hline
0&0&1&0&0\\
\hline
0&0&0&1&0\\
0&0&0&0&1
\end{array}\right)
$$

$$
\tilde\Box_2=\left(\begin{array}{c|cc|cccc|cc|c}
1&0&0&0&0& 0& 0&0&0&0\\
\hline
0&1&0&0&0& 0& 0&0&0&0\\
0&0&1&0&0& 0& 0&0&0&0\\
\hline
0&0&0&0&0& 0& 0&0&0&0\\
0&0&0&0&1& -1&0&0&0&0\\
0&0&0&0&-1&1& 0&0&0&0\\
0&0&0&0&0& 0& 0&0&0&0\\
\hline
0&0&0&0&0& 0& 0&1&0&0\\
0&0&0&0&0& 0& 0&0&1&0\\
\hline
0&0&0&0&0& 0& 0&0&0&2
\end{array}\right)
$$

$$
\tilde\Box_3=\left(\begin{array}{c|cc|cccc|cc|c}
2&0&0&0&0& 0& 0&0&0&0\\
\hline
0&1&0&0&0& 0& 0&0&0&0\\
0&0&1&0&0& 0& 0&0&0&0\\
\hline
0&0&0&0&0& 0& 0&0&0&0\\
0&0&0&0&1& -1&0&0&0&0\\
0&0&0&0&-1&1& 0&0&0&0\\
0&0&0&0&0& 0& 0&0&0&0\\
\hline
0&0&0&0&0& 0& 0&1&0&0\\
0&0&0&0&0& 0& 0&0&1&0\\
\hline
0&0&0&0&0& 0& 0&0&0&1
\end{array}\right)
$$

$$
\tilde\Box_4=\left(\begin{array}{cc|c|cc}
1&0&0&0&0\\
0&1&0&0&0\\
\hline
0&0&1&0&0\\
\hline
0&0&0&0&0\\
0&0&0&0&0
\end{array}\right)
$$

$$
\tilde\Box_5=\left(\begin{array}{c}
0
\end{array}\right)
$$


For the spectral projectors onto the zero eigenspaces, $\tilde P_q\colon\Lambda^q\goe^*_-\to\Lambda^q\goe^*_-$, we find:

$$
\tilde P_0=\left(\begin{array}{c}
1
\end{array}\right)
$$

$$
\tilde P_1=\left(\begin{array}{cc|c|cc}
1&0&0&0&0\\
0&1&0&0&0\\\hline
0&0&0&0&0\\\hline
0&0&0&0&0\\
0&0&0&0&0
\end{array}\right)
$$

$$
\tilde P_2=\left(\begin{array}{c|cc|cccc|cc|c}
0&0&0&0&0&  0&  0&0&0&0\\\hline
0&0&0&0&0&  0&  0&0&0&0\\
0&0&0&0&0&  0&  0&0&0&0\\\hline
0&0&0&1&0&  0&  0&0&0&0\\
0&0&0&0&1/2&1/2&0&0&0&0\\
0&0&0&0&1/2&1/2&0&0&0&0\\
0&0&0&0&0&  0&  1&0&0&0\\\hline
0&0&0&0&0&  0&  0&0&0&0\\
0&0&0&0&0&  0&  0&0&0&0\\\hline
0&0&0&0&0&  0&  0&0&0&0
\end{array}\right)
$$

$$
\tilde P_3=\left(\begin{array}{c|cc|cccc|cc|c}
0&0&0&0&0&  0&  0&0&0&0\\\hline
0&0&0&0&0&  0&  0&0&0&0\\
0&0&0&0&0&  0&  0&0&0&0\\\hline
0&0&0&1&0&  0&  0&0&0&0\\
0&0&0&0&1/2&1/2&0&0&0&0\\
0&0&0&0&1/2&1/2&0&0&0&0\\
0&0&0&0&0&  0&  1&0&0&0\\\hline
0&0&0&0&0&  0&  0&0&0&0\\
0&0&0&0&0&  0&  0&0&0&0\\\hline
0&0&0&0&0&  0&  0&0&0&0
\end{array}\right)
$$

$$
\tilde P_4=\left(\begin{array}{cc|c|cc}
0&0&0&0&0\\
0&0&0&0&0\\\hline
0&0&0&0&0\\\hline
0&0&0&1&0\\
0&0&0&0&1
\end{array}\right)
$$

$$
\tilde P_5=\left(\begin{array}{c}
1
\end{array}\right)
$$


Moreover, $\Box_q\colon\Omega^q(G_-)\to\Omega^q(G_-)$, $\Box_q=\delta_{q+1}d_q+d_{q-1}\delta_q$, becomes:

$$
\Box_0=\left(\begin{array}{c}
0
\end{array}\right)
$$

$$
\Box_1=\left(\begin{array}{cc|c|cc}
0& 0&  0&  0&0\\
0& 0&  0&  0&0\\\hline
X_2&-X_1&1&  0&0\\\hline
X_3&0&  -X_1&1&0\\
0& X_3& -X_2&0&1
\end{array}\right)
$$

$$
\Box_2=\left(\begin{array}{c|cc|cccc|cc|c}
1&  0&  0&  0& 0& 0&  0&  0& 0&  0\\\hline
-X_1&1&  0&  0& 0& 0&  0&  0& 0&  0\\
-X_2&0&  1&  0& 0& 0&  0&  0& 0&  0\\\hline
0&  -X_1&0&  0& 0& 0&  0&  0& 0&  0\\
X_3& -X_2&0&  0& 1& -1& 0&  0& 0&  0\\
-X_3&0&  -X_1&0& -1&1&  0&  0& 0&  0\\
0&  0&  -X_2&0& 0& 0&  0&  0& 0&  0\\\hline
0&  -X_3&0&  X_2&0& -X_1&0&  1& 0&  0\\
0&  0&  -X_3&0& X_2&0&  -X_1&0& 1&  0\\\hline
0&  0&  0&  0& X_3&-X_3&0&  X_2&-X_1&2
\end{array}\right)
$$

$$
\Box_3=\left(\begin{array}{c|cc|cccc|cc|c}
2&  0&  0&  0& 0&  0& 0&  0& 0&  0\\\hline
-X_1&1&  0&  0& 0&  0& 0&  0& 0&  0\\
-X_2&0&  1&  0& 0&  0& 0&  0& 0&  0\\\hline
0&  -X_1&0&  0& 0&  0& 0&  0& 0&  0\\
-X_3&0&  -X_1&0& 1&  -1&0&  0& 0&  0\\
X_3& -X_2&0&  0& -1& 1& 0&  0& 0&  0\\
0&  0&  -X_2&0& 0&  0& 0&  0& 0&  0\\\hline
0&  -X_3&0&  X_2&-X_1&0& 0&  1& 0&  0\\
0&  0&  -X_3&0& 0&  X_2&-X_1&0& 1&  0\\\hline
0&  0&  0&  0& -X_3&X_3&0&  X_2&-X_1&1
\end{array}\right)
$$

$$
\Box_4=\left(\begin{array}{cc|c|cc}
1& 0&  0&  0&0\\
0& 1&  0&  0&0\\\hline
X_2&-X_1&1&  0&0\\\hline
X_3&0&  -X_1&0&0\\
0& X_3& -X_2&0&0
\end{array}\right)
$$

$$
\Box_5=\left(\begin{array}{c}
0
\end{array}\right)
$$


For the differential projectors $P_q\colon\Omega^q(G_-)\to\Omega^q(G_-)$ we find:

$$
P_0=\left(\begin{array}{c}
1
\end{array}\right)
$$

$$
P_1=\left(\begin{array}{cc|c|cc}
1&        0&         0&0&0\\
0&        1&         0&0&0\\\hline
-X_2&      X_1&        0&0&0\\\hline
-X_{12}-X_3&X_{11}&      0&0&0\\
-X_{22}&    X_{12}-2X_3&0&0&0
\end{array}\right)
$$

$$
P_2=\scalefont{.5}{\left(\begin{array}{c|cc|cccc|cc|c}
0&      0&      0&      0&      0&      0&      0&      0&0&0\\\hline
0&      0&      0&      0&      0&      0&      0&      0&0&0\\
0&      0&      0&      0&      0&      0&      0&      0&0&0\\\hline
X_{11}&   X_1&     0&      1&      0&      0&      0&      0&0&0\\
X_{12}-\tfrac12X_3& \tfrac12X_2& \tfrac12X_1& 0&      1/2&    1/2&    0&      0&0&0\\
X_{12}-\tfrac12X_3& \tfrac12X_2& \tfrac12X_1& 0&      1/2&    1/2&    0&      0&0&0\\
X_{22}&   0&      X_2&     0&      0&      0&      1&      0&0&0\\\hline
\tfrac32X_{13}-X_4&X_3-\tfrac12X_{12}&\tfrac12X_{11}& -X_2&    \tfrac12X_1& \tfrac12X_1& 0&      0&0&0\\
\tfrac32X_{23}-X_5&-\tfrac12X_{22}&\tfrac12X_{12}+\tfrac12X_3& 0&      -\tfrac12X_2&-\tfrac12X_2&X_1&     0&0&0\\\hline
\tfrac34X_{33}+\tfrac12X_{24}-\tfrac12X_{15}&\tfrac14X_5-\tfrac34X_{23}&\tfrac34X_{13}-\tfrac14X_4&\tfrac12X_{22}&\tfrac14X_3-\tfrac12X_{12}&\tfrac14X_3-\tfrac12X_{12}&\tfrac12X_{11}&0&0&0
\end{array}\right)}
$$

$$
P_3=\scalefont{.5}{\left(\begin{array}{c|cc|cccc|cc|c}
0&      0&      0&      0&   0&      0&      0&   0&0&0\\\hline
0&      0&      0&      0&   0&      0&      0&   0&0&0\\
0&      0&      0&      0&   0&      0&      0&   0&0&0\\\hline
\tfrac12X_{11}& X_1&     0&      1&   0&      0&      0&   0&0&0\\
\tfrac12X_{12}-\tfrac14X_3& \tfrac12X_2& \tfrac12X_1& 0&   1/2&    1/2&    0&   0&0&0\\
\tfrac12X_{12}-\tfrac14X_3& \tfrac12X_2& \tfrac12X_1& 0&   1/2&    1/2&    0&   0&0&0\\
\tfrac12X_{22}& 0&      X_2&     0&   0&      0&      1&   0&0&0\\\hline
\tfrac34X_{13}-\frac12X_4&X_3-\tfrac12X_{12}&\tfrac12X_{11}& -X_2& \tfrac12X_1& \tfrac12X_1& 0&   0&0&0\\
\tfrac34X_{23}-\frac12X_5&-\tfrac12X_{22}&\tfrac12X_{12}+\tfrac12X_3& 0&   -\tfrac12X_2&-\tfrac12X_2&X_1&  0&0&0\\\hline
\tfrac34X_{33}+\tfrac12X_{24}-\tfrac12X_{15}&\tfrac12X_5-\tfrac32X_{23}&\tfrac32X_{13}-\tfrac12X_4&X_{22}&\tfrac12X_3-X_{12}&\tfrac12X_3-X_{12}&X_{11}&0&0&0
\end{array}\right)}
$$

$$
P_4=\left(\begin{array}{cc|c|cc}
0&          0&          0&  0&0\\
0&          0&          0&  0&0\\\hline
0&          0&          0&  0&0\\\hline
-X_{12}-X_3&X_{11}&     X_1&1&0\\
-X_{22}&    X_{12}-2X_3&X_2&0&1
\end{array}\right)
$$

$$
P_5=\left(\begin{array}{c}
1
\end{array}\right)
$$

For $L_q\colon\Omega^q(G_-)\to\Omega^q(G_-)$, $L_q=P_q\tilde P_q+(1-P_q)(1-\tilde P_q)$, we obtain:

$$
L_0=\left(\begin{array}{c}
1
\end{array}\right)
$$

$$
L_1=\left(\begin{array}{cc|c|cc}
1&        0&         0&0&0\\
0&        1&         0&0&0\\\hline
-X_2&      X_1&        1&0&0\\\hline
-X_{12}-X_3&X_{11}&      0&1&0\\
-X_{22}&    X_{12}-2X_3&0&0&1
\end{array}\right)
$$

$$
L_2=\scalefont{.5}{\left(\begin{array}{c|cc|cccc|cc|c}
1&0&0&0&0&0&0&0&0&0\\\hline
0&1&0&0&0&0&0&0&0&0\\
0&0&1&0&0&0&0&0&0&0\\\hline
-X_{11}&-X_1&0&1&0&0&0&0&0&0\\
-X_{12}+\tfrac12X_3&-\tfrac12X_2&-\tfrac12X_1&0&1&0&0&0&0&0\\
-X_{12}+\tfrac12X_3&-\tfrac12X_2&-\tfrac12X_1&0&0&1&0&0&0&0\\
-X_{22}&0&-X_2&0&0&0&1&0&0&0\\\hline
-\tfrac32X_{13}+X_4&\tfrac12X_{12}-X_3&-\tfrac12X_{11}&-X_2&\tfrac12X_1&\tfrac12X_1&0&1&0&0\\
-\tfrac32X_{23}+X_5&\tfrac12X_{22}&-\tfrac12X_{12}-\tfrac12X_3&0&-\tfrac12X_2&-\tfrac12X_2&X_1&0&1&0\\\hline
-\tfrac34X_{33}-\tfrac12X_{24}+\tfrac12X_{15}&\tfrac34X_{23}-\tfrac14X_5&-\tfrac34X_{13}+\tfrac14X_4&\tfrac12X_{22}&\tfrac14X_3-\tfrac12X_{12}&\tfrac14X_3-\tfrac12X_{12}&\tfrac12X_{11}&0&0&1
\end{array}\right)}
$$

$$
L_3=\scalefont{0.5}{\left(\begin{array}{c|cc|cccc|cc|c}
1&0&0&0&0&0&0&0&0&0\\\hline
0&1&0&0&0&0&0&0&0&0\\
0&0&1&0&0&0&0&0&0&0\\\hline
-\tfrac12X_{11}&-X_1&0&1&0&0&0&0&0&0\\
-\frac12X_{12}+\tfrac14X_3&-\tfrac12X_2&-\tfrac12X_1&0&1&0&0&0&0&0\\
-\tfrac12X_{12}+\tfrac14X_3&-\tfrac12X_2&-\tfrac12X_1&0&0&1&0&0&0&0\\
-\tfrac12X_{22}&0&-X_2&0&0&0&1&0&0&0\\\hline
-\frac34X_{13}+\tfrac12X_4&\frac12X_{12}-X_3&-\frac12X_{11}&-X_2&\tfrac12X_1&\tfrac12X_1&0&1&0&0\\
-\frac34X_{23}+\tfrac12X_5&\tfrac12X_{22}&-\tfrac12X_{12}-\tfrac12X_3&0&-\tfrac12X_2&-\tfrac12X_2&X_1&0&1&0\\\hline
-\tfrac34X_{33}-\tfrac12X_{24}+\tfrac12X_{15}&\tfrac32X_{23}-\tfrac12X_5&-\tfrac32X_{13}+\tfrac12X_4&X_{22}&\tfrac12X_3-X_{12}&\tfrac12X_3-X_{12}&X_{11}&0&0&1
\end{array}\right)}
$$

$$
L_4=\left(\begin{array}{cc|c|cc}
1&       0&          0&  0&0\\
0&       1&          0&  0&0\\\hline
0&       0&          1&  0&0\\\hline
X_{12}+X_3&-X_{11}&      -X_1&1&0\\
X_{22}&    -X_{12}+2X_3&-X_2&0&1
\end{array}\right)
$$

$$
L_5=\left(\begin{array}{c}
1
\end{array}\right)
$$


And for their inverses, $L_q^{-1}\colon\Omega^q(G_-)\to\Omega^q(G_-)$, we have:

$$
L^{-1}_0=\left(\begin{array}{c}
1
\end{array}\right)
$$

$$
L^{-1}_1=\left(\begin{array}{cc|c|cc}
1&       0&          0&0&0\\
0&       1&          0&0&0\\\hline
X_2&      -X_1&        1&0&0\\\hline
X_{12}+X_3&-X_{11}&      0&1&0\\
X_{22}&    -X_{12}+2X_3&0&0&1
\end{array}\right)
$$

$$
L^{-1}_2=\left(\begin{array}{c|cc|cccc|cc|c}
1&     0&     0&     0&      0&      0&      0&      0&0&0\\\hline
0&     1&     0&     0&      0&      0&      0&      0&0&0\\
0&     0&     1&     0&      0&      0&      0&      0&0&0\\\hline
X_{11}&  X_1&    0&     1&      0&      0&      0&      0&0&0\\
X_{12}-\tfrac12X_3&\tfrac12X_2&\tfrac12X_1&0&      1&      0&      0&      0&0&0\\
X_{12}-\tfrac12X_3&\tfrac12X_2&\tfrac12X_1&0&      0&      1&      0&      0&0&0\\
X_{22}&  0&     X_2&    0&      0&      0&      1&      0&0&0\\\hline
0&     0&     0&     X_2&     -\tfrac12X_1&-\tfrac12X_1&0&      1&0&0\\
0&     0&     0&     0&      \tfrac12X_2& \tfrac12X_2& -X_1&    0&1&0\\\hline
0&     0&     0&-\tfrac12X_{22}&\tfrac12X_{12}-\tfrac14X_3&\tfrac12X_{12}-\tfrac14X_3&-\tfrac12X_{11}&0&0&1
\end{array}\right)
$$

$$
L^{-1}_3=\left(\begin{array}{c|cc|cccc|cc|c}
1&               0&     0&     0&    0&           0&           0&    0&0&0\\\hline
0&               1&     0&     0&    0&           0&           0&    0&0&0\\
0&               0&     1&     0&    0&           0&           0&    0&0&0\\\hline
\tfrac12X_{11}&        X_1&    0&     1&    0&           0&           0&    0&0&0\\
\tfrac12X_{12}-\tfrac14X_3&\tfrac12X_2&\tfrac12X_1&0&    1&           0&           0&    0&0&0\\
\tfrac12X_{12}-\tfrac14X_3&\tfrac12X_2&\tfrac12X_1&0&    0&           1&           0&    0&0&0\\
\tfrac12X_{22}&        0&     X_2&    0&    0&           0&           1&    0&0&0\\\hline
0&               0&     0&     X_2&   -\tfrac12X_1&     -\tfrac12X_1&     0&    1&0&0\\
0&               0&     0&     0&    \tfrac12X_2&      \tfrac12X_2&      -X_1&  0&1&0\\\hline
0&               0&     0&     -X_{22}&X_{12}-\tfrac12X_3&X_{12}-\tfrac12X_3&-X_{11}&0&0&1
\end{array}\right)
$$

$$
L^{-1}_4=\left(\begin{array}{cc|c|cc}
1&        0&         0& 0&0\\
0&        1&         0& 0&0\\\hline
0&        0&         1& 0&0\\\hline
-X_{12}-X_3&X_{11}&      X_1&1&0\\
-X_{22}&    X_{12}-2X_3&X_2&0&1
\end{array}\right)
$$

$$
L^{-1}_5=\left(\begin{array}{c}
1
\end{array}\right)
$$

The conjugates $L_{q+1}^{-1}d_qL_q\colon\Omega^q(G_-)\to\Omega^{q+1}(G_-)$ are:

$$
L^{-1}_1d_0L_0=\left(\begin{array}{c}
X_1\\
X_2\\\hline
0\\\hline
0\\
0
\end{array}\right)
$$

$$
L^{-1}_2d_1L_1=\left(\begin{array}{cc|c|cc}
0&                0&              -1& 0&               0\\\hline
0&                0&              X_1& -1&              0\\
0&                0&              X_2& 0&               -1\\\hline
-X_{112}-X_{13}-X_4&X_{111}&           0&  0&               0\\
-X_{122}-X_5&      X_{112}-2X_{13}&0&  -\tfrac12X_2&         \tfrac12X_1\\
-X_{122}-X_5&      X_{112}-2X_{13}&0&  \tfrac12X_2&          -\tfrac12X_1\\
-X_{222}&            X_{122}-3X_{23}&0&  0&               0\\\hline
0&                0&              -X_4&\tfrac12X_{12}&       -\tfrac12X_{11}\\
0&                0&              -X_5&\tfrac12X_{22}&        -\tfrac12X_{12}+\tfrac12X_3\\\hline
0&                0&              0&  \tfrac34X_{23}-\tfrac54X_5&-\tfrac34X_{13}+\tfrac54X_4
\end{array}\right)
$$

$$
L^{-1}_3d_2L_2=\scalefont{.5}{\left(\begin{array}{c|cc|cccc|cc|c}
X_3&-X_2&    X_1&     0&     1&      -1&     0&     0&      0&      0\\\hline
X_4&0&      0&      0&     -\tfrac12X_1&\tfrac12X_1& 0&     -1&     0&      0\\
X_5&0&      0&      0&     -\tfrac12X_2&\tfrac12X_2& 0&     0&      -1&     0\\\hline
0& 0&      0&  -X_{12}-X_3&\tfrac12X_{11}&\tfrac12X_{11}& 0&     0&      0&      0\\
0& \tfrac12X_5& -\tfrac12X_4&-\tfrac12X_{22}&-\tfrac54X_3&-\tfrac14X_3&\tfrac12X_{11}&-\tfrac12X_2&\tfrac12X_1& -1\\
0& -\tfrac12X_5&\tfrac12X_4&-\tfrac12X_{22}&-\tfrac14X_3&-\tfrac54X_3&\tfrac12X_{11}&\tfrac12X_2& -\tfrac12X_1&1\\
0& 0&      0&      0&-\tfrac12X_{22}&-\tfrac12X_{22}&X_{12}-2X_3&0&      0&      0\\\hline
0&X_{24}-\tfrac12X_{15}&-\tfrac12X_{14}&0&-\tfrac12X_4&\tfrac12X_4&0&\tfrac12X_{12}-X_3&-\tfrac12X_{11}&X_1\\
0&\tfrac12X_{25}&\tfrac12X_{24}-X_{15}&0&-\tfrac12X_5&\tfrac12X_5&0&\tfrac12X_{22}&-\tfrac12X_{12}-\tfrac12X_3&X_2\\\hline
0&X_{125}-X_{224}-\tfrac12X_{35}&X_{124}-X_{115}-\tfrac12X_{34}&0&0&0&0&\tfrac32X_{23}+\tfrac12X_5&-\tfrac32X_{13}-\frac12X_4&X_3
\end{array}\right)}
$$

$$
L^{-1}_4d_3L_3=\scalefont{.5}{\left(\begin{array}{c|cc|cccc|cc|c}
-X_4&X_3&0&0&-\tfrac12X_1&\tfrac12X_1&0&-1&0&0\\
-X_5&0&X_3&0&-\tfrac12X_2&\tfrac12X_2&0&0&-1&0\\\hline
0&-X_5&X_4&0&0&0&0&-X_2&X_1&-1\\\hline
0&0&0&X_{122}+X_{23}-2X_5&-X_{112}+X_4&-X_{112}+X_4&X_{111}&0&0&0\\
0&0&0&X_{222}&2X_{23}-X_{122}-2X_5&2X_{23}-X_{122}-2X_5&X_{112}-3X_{13}+3X_4&0&0&0
\end{array}\right)}
$$

$$
L^{-1}_5d_4L_4=\left(\begin{array}{cc|c|cc}
0&0&0&-X_2&X_1
\end{array}\right)
$$


We will use the following ordered bases of $\img(\tilde P_q)$ and $\ker(\tilde P_q)$:


\begin{align*}
\img(\tilde P_0)&:1
&\ker(\tilde P_0)&:\emptyset
\\
\img(\tilde P_1)&:\alpha^1,\alpha^2
&\ker(\tilde P_1)&:\alpha^3|\alpha^4,\alpha^5
\\
\img(\tilde P_2)&:\alpha^{14},\alpha^{15}+\alpha^{24},\alpha^{25}
&\ker(\tilde P_2)&:\alpha^{12}|\alpha^{13},\alpha^{23}|\tfrac12(\alpha^{15}-\alpha^{24})|\alpha^{34},\alpha^{35}|\alpha^{45}
\\
\img(\tilde P_3)&:\alpha^{134},\tfrac12(\alpha^{135}+\alpha^{234}),\alpha^{235}
&\ker(\tilde P_3)&:\alpha^{123}|\alpha^{124},\alpha^{125}|\alpha^{135}-\alpha^{234}|\alpha^{145},\alpha^{245}|\alpha^{345}
\\
\img(\tilde P_4)&:\alpha^{1345},\alpha^{2345}
&\ker(\tilde P_4)&:\alpha^{1234},\alpha^{1235}|\alpha^{1245}
\\
\img(\tilde P_5)&:\alpha^{12345}
&\ker(\tilde P_5)&:\emptyset
\end{align*}


Then $D_q=\tilde P_{q+1}L_{q+1}^{-1}d_qL_q|_{\img(\tilde P_q)}\colon\Gamma^\infty(\img(\tilde P_q))\to\Gamma^\infty(\img(\tilde P_{q+1}))$ becomes:

$$
D_0=\left(\begin{array}{c}
X_1\\
X_2
\end{array}\right)
$$

$$
D_1=\left(\begin{array}{cc}
-X_{112}-X_{13}-X_4&X_{111}\\           
-X_{122}-X_5&      X_{112}-2X_{13}\\
-X_{222}&            X_{122}-3X_{23}
\end{array}\right)
$$

$$
D_2=\left(\begin{array}{ccc}
-X_{12}-X_3&X_{11}& 0\\         
-X_{22}&-3X_3&  X_{11}\\ 
0&        -X_{22}&X_{12}-2X_3\\
\end{array}\right)
$$

$$
D_3=\left(\begin{array}{ccc}
X_{122}+X_{23}-2X_5&-X_{112}+X_4&X_{111}\\
X_{222}&-X_{122}+2X_{23}-2X_5&X_{112}-3X_{13}+3X_4
\end{array}\right)
$$

$$
D_4=\left(\begin{array}{cc}
-X_2&X_1
\end{array}\right)
$$

This proves the formulas for the Rockland complex \eqref{E:BGG235dR} in Example~\ref{Ex:BGG235}.
Let us also point out that $B_q=(\id-\tilde P_{q+1})L_{q+1}^{-1}d_qL_q|_{\ker(\tilde P_q)}\colon\Gamma^\infty(\ker(\tilde P_q))\to\Gamma^\infty(\ker(\tilde P_{q+1}))$ become:

$$
B_1=\left(\begin{array}{c|cc}
-1& 0&               0\\\hline
X_1& -1&              0\\
X_2& 0&               -1\\\hline
0&  -X_2&         X_1\\\hline
-X_4&\tfrac12X_{12}&       -\tfrac12X_{11}\\
-X_5&\tfrac12X_{22}&        -\tfrac12X_{12}+\tfrac12X_3\\\hline
0&  \tfrac34X_{23}-\tfrac54X_5&-\tfrac34X_{13}+\tfrac54X_4
\end{array}\right)
$$

$$
B_2=\left(\begin{array}{c|cc|c|cc|c}
X_3&-X_2&    X_1&     1&           0&      0&      0\\\hline
X_4&0&      0&      -\tfrac12X_1& -1&     0&      0\\
X_5&0&      0&      -\tfrac12X_2& 0&      -1&     0\\\hline
0& \tfrac12X_5& -\tfrac12X_4&-\tfrac12X_3& -\tfrac12X_2&\tfrac12X_1& -1\\\hline
0&X_{24}-\tfrac12X_{15}&-\tfrac12X_{14}&-\tfrac12X_4&\tfrac12X_{12}-X_3&-\tfrac12X_{11}&X_1\\
0&\tfrac12X_{25}&\tfrac12X_{24}-X_{15}&-\tfrac12X_5&\tfrac12X_{22}&-\tfrac12X_{12}-\tfrac12X_3&X_2\\\hline
0&-X_{224}+X_{125}-\tfrac12X_{35}&X_{124}-X_{115}-\tfrac12X_{34}&0&\tfrac32X_{23}+\tfrac12X_5&-\tfrac32X_{13}-\tfrac12X_4&X_3
\end{array}\right)
$$

$$
B_3=\left(\begin{array}{c|cc|c|cc|c}
-X_4&X_3& 0& -X_1&-1& 0& 0\\
-X_5&0&  X_3&-X_2&0&  -1&0\\\hline
0&  -X_5&X_4&    0&     -X_2&X_1&-1
\end{array}\right)
$$


Furthermore, the operators $G_q\colon\Gamma^\infty(\ker(\tilde P_q))\to\Gamma^\infty(\ker(\tilde P_q))$ satisfying

$$
G_2^{-1}B_1G_1=\partial_1|_{\ker(\tilde P_1)}=\left(\begin{array}{c|cc}
-1&0& 0\\\hline
0& -1&0\\ 
0& 0& -1\\\hline
0& 0& 0\\\hline
0& 0& 0\\
0& 0& 0\\\hline
0& 0& 0  
\end{array}\right)
$$

$$
G_3^{-1}B_2G_2=\partial_2|_{\ker(\tilde P_2)}=\left(\begin{array}{c|cc|c|cc|c}
0&0&0&1&0& 0& 0\\\hline
0&0&0&0&-1&0& 0\\
0&0&0&0&0& -1&0\\\hline
0&0&0&0&0& 0& -1\\\hline
0&0&0&0&0& 0& 0\\
0&0&0&0&0& 0& 0\\\hline
0&0&0&0&0& 0& 0  
\end{array}\right)
$$

$$
G_4^{-1}B_3G_3=\partial_3|_{\ker(\tilde P_3)}=\left(\begin{array}{c|cc|c|cc|c}
0&0&0&0&-1&0& 0\\
0&0&0&0&0& -1&0\\\hline
0&0&0&0&0& 0& -1
\end{array}\right)
$$

see Theorem~\ref{T:D}(b), are given by:

$$
G_1=\left(\begin{array}{c|cc}
1&0&0\\\hline
0&1&0\\
0&0&1 
\end{array}\right)
$$

$$
G_2=\left(\begin{array}{c|cc|c|cc|c}
1&  0&                0&               0&0&0&0\\\hline
-X_1&1&                0&               0&0&0&0\\
-X_2&0&                1&               0&0&0&0\\\hline
0&  X_2&               -X_1&             1&0&0&0\\\hline
X_4& -\tfrac12X_{12}&       \tfrac12X_{11}&        0&1&0&0\\
X_5& -\tfrac12X_{22}&        \tfrac12X_{12}-\tfrac12X_3&0&0&1&0\\\hline
0&  -\tfrac34X_{23}+\tfrac54X_5&\tfrac34X_{13}-\tfrac54X_4&0&0&0&1 
\end{array}\right)
$$

$$
G_3=\left(\begin{array}{c|cc|c|cc|c}
1&      0&                0&               0&  0&0&0\\\hline
-\tfrac12X_1&1&                0&               0&  0&0&0\\
-\tfrac12X_2&0&                1&               0&  0&0&0\\\hline
-\tfrac12X_3&\tfrac12X_2&           -\tfrac12X_1&         1&  0&0&0\\\hline
-\tfrac12X_4&-\tfrac12X_{12}+X_3&    \tfrac12X_{11}&        -X_1&1&0&0\\
-\tfrac12X_5&-\tfrac12X_{22}&        \tfrac12X_{12}+\tfrac12X_3&-X_2&0&1&0\\\hline
0&      -\tfrac32X_{23}-\tfrac12X_5&\tfrac32X_{13}+\tfrac12X_4&-X_3&0&0&1 
\end{array}\right)
$$

$$
G_4=\left(\begin{array}{cc|c}
1& 0&  0\\
0& 1&  0\\\hline
X_2&-X_1&1 
\end{array}\right)
$$

Their inverses, $G_q^{-1}\colon\Gamma^\infty(\ker(\tilde P_q))\to\Gamma^\infty(\ker(\tilde P_q))$, are:

$$
G^{-1}_1=\left(\begin{array}{c|cc}
1&0&0\\\hline
0&1&0\\
0&0&1 
\end{array}\right)
$$

$$
G^{-1}_2=\left(\begin{array}{c|cc|c|cc|c}
1&                            0&               0&                0&0&0&0\\\hline
X_1&                           1&               0&                0&0&0&0\\
X_2&                           0&               1&                0&0&0&0\\\hline
X_3&                           -X_2&             X_1&               1&0&0&0\\\hline
-\tfrac12X_{13}-X_4&                \tfrac12X_{12}&       -\tfrac12X_{11}&        0&1&0&0\\
-\tfrac12X_{23}-X_5&                \tfrac12X_{22}&        -\tfrac12X_{12}+\tfrac12X_3&0&0&1&0\\\hline
-\tfrac34X_{33}+\tfrac12X_{24}-\tfrac12X_{15}&\tfrac34X_{23}-\tfrac54X_5&-\tfrac34X_{13}+\tfrac54X_4&0&0&0&1 
\end{array}\right)
$$

$$
G^{-1}_3=\left(\begin{array}{c|cc|c|cc|c}
1&           0&       0&        0& 0&0&0\\\hline
\tfrac12X_1&      1&       0&        0& 0&0&0\\
\tfrac12X_2&      0&       1&        0& 0&0&0\\\hline
\tfrac34X_3&      -\tfrac12X_2& \tfrac12X_1&   1& 0&0&0\\\hline
X_4&          -X_3&     0&        X_1&1&0&0\\
X_5&          0&       -X_3&      X_2&0&1&0\\\hline
-X_{24}+X_{15}&X_{23}+X_5&-X_{13}-X_4&X_3&0&0&1 
\end{array}\right)
$$

$$
G^{-1}_4=\left(\begin{array}{cc|c}
1&  0& 0\\
0&  1& 0\\\hline
-X_2&X_1&1 
\end{array}\right)
$$






%\bibliographystyle{alpha}
%\bibliographystyle{plain}
\bibliographystyle{abbrv}
  \bibliography{hypo}




\end{document}
