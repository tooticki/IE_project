%%% 2015_01_05, valojaban Jan 11 (egesz nap) es kis jav: jan 12.
% Marc 3,2007
% From z-furedi@mail1.math.uiuc.edu Wed Jul 12 10:47:03 2006

\documentclass[11pt]{article}
\usepackage{amssymb,amsmath,amsfonts,mathtools}
\usepackage{epsfig,graphicx}
\usepackage{latexsym}
\newtheorem{theorem}{Theorem}
\newtheorem{lemma}[theorem]{Lemma}
\newtheorem{coro}[theorem]{Corollary}
\newtheorem{cl}[theorem]{Claim}
\newtheorem{pr}[theorem]{Proposition}
\newtheorem{defi}[theorem]{Definition}
\def\qed
 {\ifhmode\unskip\nobreak\hfill$\Box$\medskip\fi
 \ifmmode\eqno{\Box}\fi}


% %%%% ------------------------------------------------------------------------
\parskip=2mm plus 2pt%%% 3mm
  \headsep 20pt            %    Space between running head and text.
  \tolerance=1000
\renewcommand{\baselinestretch}{1.2}  %%% 1.5
\setlength{\textwidth}{162mm}   % 160
\setlength{\textheight}{230mm}  % 232
\setlength{\topmargin}{-12mm}   % -16
\setlength{\evensidemargin}{0mm}
\setlength{\oddsidemargin}{0mm}
% \textwidth=6in
% \hoffset=-0.3in
% %%%% ------------------------------------------------------------------------


\def\ex{{\rm ex}}
\def\EE{{\mathcal E}}
\newcommand{\eps}{{\varepsilon}}
\def\ed{{\rm ed}}
\def\FF{{\mathcal F}}
\def\half{{1\over 2}}
\def\GG{{\mathcal G}}
\def\HH{{\mathcal H}}
\def\LL{{\mathcal L}}
\def\NN{{\mathcal N}}
\def\PP{{\mathcal P}}
\def\RR{{\mathcal R}}



\pagestyle{myheadings}
\markright{{\small\sc  Z. F\"uredi:}
   {\it\small Stability of extremal graphs}}

\begin{document}

\title{{% \huge
A proof of the stability of extremal graphs,\\
 Simonovits' stability from Szemer\'edi's regularity}}
% A proof of  \\  the stability of extremal graphs}}
% \author{{\bf Zolt\'an F\"uredi}}

\author{Zolt\'an F\"uredi
\thanks{%Alfr\'ed R\'enyi Institute of Mathematics, 13--15 Re\'altanoda Street, 1053 Budapest, Hungary. \newline
        %E-mail: {\tt z-furedi@illinois.edu}.\newline\indent
Research supported in part by the Hungarian National Science Foundation OTKA 104343,
 by the Simons Foundation Collaboration Grant \#317487,
and by the European Research Council Advanced Investigators Grant 267195.
%  \newline
% This work was done while the authors visited the Department of Mathematics and Computer Science, Emory University, Atlanta, GA, USA.
\newline\indent
{\it 2010 Mathematics Subject Classifications:}
05C35. \hfill \jobname
\newline\indent
{\it Key Words}:  Tur\'an number,  extremal graphs, stability.  \hfill\today
}
}
\date{Alfr\'ed R\'enyi Institute of Mathematics,\\ 13--15 Re\'altanoda Street, 1053 Budapest, Hungary. \\
E-mail: {\tt z-furedi@illinois.edu}}

%This work was done while the authors visited the Department of Mathematics and Computer Science, Emory University, Atlanta, GA, USA.}




\maketitle





\maketitle

\begin{abstract}
The following sharpening of Tur\'an's theorem is proved.
Let $T_{n,p}$ denote the complete $p$--partite graph of order $n$ having the maximum number of edges.
If $G$ is an $n$-vertex $K_{p+1}$-free graph with
  $e(T_{n,p})-t$ edges then there exists an (at most) $p$-chromatic subgraph $H_0$ such that
  $e(H_0)\geq e(G)-t$.

Using this result we present a concise, contemporary proof (i.e., one using
Szemer\'edi's regularity lemma) for the classical stability result
of Simonovits~\cite{Sim_Tihany}.



\end{abstract}
%%%%%%%%%%%%%%%%%%%%%%%%%%%%%%%%%%%%%%%%%%%%%%%%%%



\section{The Tur\'an problem} % and the stability of the extremal graphs}


Given a graph $G$ with vertex set $V(G)$ and vertex set $\EE(G)$ its number of edges is denoted by $e(G)$.
The neighborhood of a vertex $x\in V$ is denoted by $N(x)$, note that $x\notin N(x)$.
For any $A\subset V$ the restricted neighborhood $N_G(x|A)$ stands for $N(x)\cap A$.
Similarly, $\deg_G(x|A):=|N(x)\cap A|$.
If the graph is well understood from the text we leave out subscripts.
The {\it Tur\'an graph} $T_{n,p}$ is the largest $p$-chromatic graph having $n$ vertices, $n,p\geq 1$.
Given a partition $(V_1, \dots, V_p)$ of $V$ the {\it complete multipartite} graph
 $K(V_1, \dots, V_p)$ has vertex set $V$ and all the edges joining distinct partite sets.
$A\bigtriangleup B$ stands for the symmetric difference of the sets $A$ and $B$.
For further notations and notions undefined here see, e.g., the monograph of Bollob\'as~\cite{BolExt}.

Tur\'an~\cite{TuranA} proved that if an $n$ vertex graph $G$ has at least
 $e(T_{n,p})$ edges then it contains a complete subgraph $K_{p+1}$, except if $G=T_{n,p}$.
Given a class of graphs $\LL$, a graph $G$ is called $\LL$-{\it free} if it does not contain any subgraph isomorphic to any member of $\LL$.
The {\it Tur\'an number} $\ex(n,\LL)$ is defined as the largest size of an $n$-vertex, $\LL$-free graph.
Erd\H os and Simonovits~\cite{ErdSimLim} gave a general asymptotic for the Tur\'an number as follows.
Let $p+1:=\min \{\chi(L): L\in\LL\}$. Then
\begin{equation}\label{eq1}
\ex(n,\LL)=\left(1-\frac{1}{p}\right)\binom{n}{2}+o(n^2) \quad{\rm as} \quad n\to \infty.
  \end{equation}
They also showed that if $G$ is an extremal graph, i.e., $e(G)=\ex(n, \LL)$,
 then it can be obtained from $T_{n,p}$ by adding and deleting at most $o(n^2)$ edges.
This result is usually called Erd\H os--Stone--Simonovits theorem, although it was proved first in~\cite{ErdSimLim},
 but indeed (\ref{eq1}) easily follows from a result of Erd\H os and Stone~\cite{ErdStone}.

The aim of this paper is to present a new proof for the following stronger version of (\ref{eq1}), a structural stability theorem,
 originally proved by Erd\H os and  Simonovits~\cite{ErdSimLim}, Erd\H os~\cite{Erd_Rome,Erd_Tihany}, and Simonovits~\cite{Sim_Tihany}.
For every $\varepsilon>0$ and forbidden subgraph class $\LL$ there is a $\delta>0$, and $n_0$ such that
 if  $n>n_0$ and $G$ is an $n$-vertex, $\LL$-free graph with
\begin{equation*}%\label{}
 e(G)\ge\left(1-\frac{1}{p}\right)\binom{n}{2}-\delta n^2,
   \end{equation*}
then
\begin{equation}\label{eq2}
|\EE(G_n) \bigtriangleup \EE(T_{n,p})|\le \varepsilon n^2.
 \end{equation}
I.e., one can change (add and delete) at most $\varepsilon n^2$ edges of $G$
 and obtain a complete $p$-partite graph.
In other words, if an $n$-vertex $\LL$-free graph $G$ is almost extremal,
$\min \{ \chi(L): L\in \LL\}=p+1$, then the structure of $G$ is close to a $p$-partite Tur\'an graph.
This result is usually called Simonovits' stability of the extremum.

Our main tool is a very simple proof for the case $\LL=\{K_{p+1}\}$.

Stability results are usually  more important than their extremal counterparts.
That is why there are so many investigations concerning the {\it edit distance} of graphs.
Let $G_1=(V,\EE_1)$ and $G_2=(V,\EE_2)$ be two (finite, undirected) graphs on the same vertex set.
The {\it edit distance} from $G_1$ to $G_2$ is $\ed(G_1,G_2):=|\EE_1 \bigtriangleup  \EE_2|$.
Let $\PP$ denote a class of graphs and $G$ be a fixed graph.
The edit distance from $G$ to $\PP$ is $\ed(G,\PP)=\min\{\ed(G,F):F\in \PP,V(G)=V(F)\}$.
This notion was explicitly introduced in~\cite{AxeMartin2006}, Alon and Stav~\cite{AloSta} proved connections with Tur\'an theory.
For more recent results see Martin~\cite{Martin2015}.





\section{How to make a $K_{p+1}$-free graph $p$-chromatic}

Ever since Erd\H os~\cite{ErdBipartite} observed that one can always delete at most $e/2$ edges from any graph $G$
 to make it bipartite there are many generalizations and applications of this
(see, e.g., Alon~\cite{Alo} for a more precise form).
Here we prove a version dealing with a narrower class of graphs.
Recall that $e(T_{n,p}):= \max\{ e(K(V_1, \dots, V_p)): \sum |V_i|=n\}$, the maximum size of a $p$-chromatic graph.

\begin{theorem}\label{th1}
Suppose that $K_{p+1}\not\subset G$, $|V(G)|=n$, $t\geq 0$, and
 $$ e(G)= e(T_{n,p})-t.$$
Then there exists an (at most) $p$-chromatic subgraph $H_0$,
 $\EE(H_0)\subset \EE(G)$ such that
$$  e(H_0)\geq e(G)-t. $$
\end{theorem}

\begin{coro}[Stability of $\ex(n, K_{p+1})$]\label{co1}\enskip
Suppose that $G$ is $K_{p+1}$-free with $e(G)\geq e(T_{n,p})- t$.
Then there is a complete $p$-chromatic graph $K:=K(V_1, \dots, V_p)$ with $V(K)=V(G)$,
  such that
$$   |\EE(G)\bigtriangleup \EE(K)|\leq 3t. $$
\end{coro}
Indeed, delete $t$ edges of $G$ to obtain the $p$-chromatic $H_0$.
Since $e(H_0)\geq e(T_{n,p})-2t$ one can add at most $2t$ edges to make it a complete $p$-partite graph.
(Here $V_i=\emptyset$ is allowed). \qed

There are other more exact stability results, e.g.,
Hanson and Toft~\cite{HanTof} showed that for  $t< n/(2p)-O(1)$ the graph $G$ itself is $p$-chromatic,
 there is no need to delete any edge.
Some results of E. Gy\H ori~\cite{Gyo} implies a stronger form, namely that $e(H_0)\geq e(G)-O(t^2/n^2)$.
Erd\H os, Gy\H ori, and  Simonovits~\cite{ErdGyoSim} considers only dense triangle-free graphs.
The advantage of our Theorem~\ref{th1} is that it contains no $\varepsilon, \delta, n_0$,
 it is true for every $n$, $p$ and $t$.

The inequality in Corollary~\ref{co1} is simple because we estimate the edit distance of $G$ from
 a not necessarily balanced $p$ partite graph $K$.
If we are interested in $\ed(G,T_{n,p})$ then we can use the following inequality.
If $e(K((V_1, \dots, V_p))\geq e(T_{n,p})-2t$, then a simple calculation shows that
the sizes of $V_i$'s should be 'close' to $n/p$ (more exactly we get
$4t\geq \sum_i (|V_i|-(n/p))^2$) and hence
\begin{equation}\label{eq:co2}
  \ed(K,T_{n,p})\leq n \sqrt{t/p}
\end{equation}


\noindent
{\bf Proof of Theorem~\ref{th1}.} \quad
We find the large $p$-partite subgraph $H_0\subset G$ by analyzing
Erd\H os' degree majorization algorithm~\cite{ErdTur} what he used to prove Tur\'an's theorem.
Our input is the $K_{p+1}$-free graph $G$ and the output is a partition $V_1, V_2, \dots, V_p$
 of $V(G)$ such that  $\sum_i e(G|V_i) \leq t$.

Let $x_1\in V(G)$ be a vertex of maximum degree and  let $V_1:= V\setminus N(x_1)$, $V_1^+:= V\setminus V_1$.
Note that $x_1\in V_1$ and $\deg(x)\leq |V_1^+|$ for all $x\in V_1$. Hence
\begin{equation*}%\label{}
   2e(G|V_1)+ e(V_1, V_1^+ )=\sum_{x\in V_1} \deg(x)\leq |V_1||V_1^+|.
   \end{equation*}
In general, define $V_0^+:=V(G)$ and let $x_i$ be a vertex of maximum degree of the graph  $G|V_{i-1}^+$,
% on the rest of the vertices, of $G\setminus (V_1\cup \dots V_{i-1})$ and
 let $V_i:= V_{i-1}^+\setminus N(x_i)$, $V_i^+:=V(G)\setminus (V_1\cup \dots \cup V_i)$.
We have $x_i\in V_i$, $\deg(x_i, V_{i-1}^+)=|V_i^+|$ and
\begin{equation}\label{eq4}
   2e(G|V_i)+ e(V_i, V_i^+ )=\sum_{x\in V_i} \deg(x| V_{i-1}^+)\leq |V_i||V_i^+|.
   \end{equation}
The procedure stops in $s$ steps when no more vertices left, i.e., if $V_1\cup \dots \cup V_s=V(G)$.
Note that $s\leq p$ because $\{ x_1, x_2, \dots , x_s\}$
 span a complete graph.

Add up the left hand sides of (\ref{eq4}) for $1\leq i\leq s$, we get  $e(G)+ \left(\sum_i e(G|V_i)\right)$.
The sum of the right hand sides is exactly  $e(K(V_1, V_2, \dots , V_s))$.
We obtain
  $$
  e(T_{n,p})-t+ \left(\sum_i e(G|V_i)\right)= e(G)+ \left(\sum_i e(G|V_i)\right)\leq e(K(V_1, V_2, \dots , V_p))
    \leq e(T_{n,p})
    $$
implying  $\sum_i e(G|V_i)\leq t $.  \qed




\section{Az application of the Removal Lemma}

We only need a simple consequence of Szemer\'edi's Regularity Lemma.
Recall that the graph $H$ contains a homomorphic image of $F$ if there is a mapping
 $\varphi:V(F)\to V(H)$ such that the image of each $F$-edge is an $H$-edge.
There is a $\varphi:V(F)\to V(K_s)$ homomorphism if and only if $s\geq \chi(H)$.
If there is no any $\varphi:V(F)\to V(H)$ homomorphism then $H$ is called  ${\rm hom}(F)$-free.

\begin{lemma}[A simple form of the Removal Lemma]\label{le1}\enskip
For every  $\alpha>0$ and graph $F$ there is an $n_1$ such that
 if  $n>n_1$ and $G$ is an $n$-vertex, $F$-free graph then it contains a ${\rm hom}(F)$-free subgraph $H$
 with $e(H)> e(G)-\alpha n^2$.
   \end{lemma}

This means that $H$ does not contain any homomorphic image of $F$ as a subgraph, especially
 if $\chi(F)=p+1$ then $H$ is $K_{p+1}$-free.
The Removal Lemma can be attributed to Ruzsa and Szemer\'edi~\cite{RuzSze}.
It appears in a more explicit form in \cite{ErdFraRod} and \cite{Fur}.
For a survey of applications of Szemer\'edi's regularity lemma in graph theory see Koml\'os-Simonovits~\cite{KomlosSim}
 or Koml\'os-Shokoufandeh-Simonovits-Szemer\'edi~\cite{KomShoSimSze}.

\noindent
{\bf Proof of (\ref{eq2})} using Lemma~\ref{le1} and Corollary~\ref{co1}.\enskip
Suppose that $F\in \LL$, $\chi(F)=p+1$ and $\alpha>0 $ an arbitrary real.
Suppose that $G$ is $F$-free with $n> n_1(F, \alpha)$ and $e(G)> e(T_{n,p})-
 \alpha n^2$.
We have to show that the edit distance of $G$ to $T_{n,p}$ is small.
First we claim that the edit distance of $G$ to a complete $p$--partite graph $K(V_1, \dots, V_p)$ is at most $7\alpha n^2$.
Indeed, using the Removal Lemma we obtain a $K_{p+1}$-free subgraph $H$ of $G$ such that
  $e(H)>e(G)-\alpha n^2 > e(T_{n,p})- 2\alpha n^2$.
Apply Theorem~\ref{th1} to $H$ we get a $p$--partite $H_0$ with $e(H_0)> e(T_{n,p})- 4\alpha n^2$.
Then Corollary~\ref{co1} yields a $K:=K(V_1, \dots, V_p)$ with $\ed(K,H) < 6\alpha n^2$, giving
 $\ed(K, G)\leq 7\alpha n^2$.

Since  $e(K)\geq e(H_0)> e(T_{n,p})-4\alpha n^2$, we can use (\ref{eq:co2}) with $t=2\alpha n^2$
 to get $\ed(K, T_{n,p})\leq n^2 \sqrt{2\alpha/p}$.
This completes the proof that $\ed(G,T_{n,p})\leq(7\alpha + \sqrt{2\alpha/p})n^2$.
\qed


%{\small
\noindent
{\bf Acknowledgments.}\quad
The author is greatly thankful to % I.~B\'ar\'any and
 M.~Simonovits for helpful conversations.
\newline
This result was first presented in a public lecture at Charles University, Prague, July 2006.
%Carnegie Mellon University, May 2007.
Since then there were several references to it, e.g., in~\cite{Mub}.
%}



% \newpage
% \normalsize

\begin{thebibliography}{10}
\parskip=0pt
\small

\bibitem{Alo} N. Alon: Bipartite subgraphs.
{\it Combinatorica} {\bf 16} (1996), 301--311.

\bibitem{AloSta} N. Alon and U. Stav: % Uri
The maximum edit distance from hereditary graph properties.
{\it J. Combin. Theory, Ser. B} {\bf 98} (2008), % no. 4,
 672--697.

\bibitem{AxeMartin2006} M. Axenovich and R. Martin:
Avoiding patterns in matrices via a small number of changes.
{\it SIAM J. Discrete Math.} {\bf 20} (2006), %no. 1,
49--54.

\bibitem{BolExt} B. Bollob\'as: {\it Extremal Graph Theory}. London
Math. Soc. Monographs,  Academic Press, 1978.

\bibitem{ErdBipartite} P. Erd\H{o}s:
On even subgraphs of graphs. (Hungarian), {\it Mat. Lapok} {\bf 18} (1967), 283--288.

\bibitem{ErdTur} P. Erd\H{o}s:
On the graph theorem of Tur\'an. (Hungarian)
{\it Mat. Lapok} {\bf 21} (1970), 249--251 (1971).

\bibitem{Erd_Rome} P. Erd\H{o}s:
Some recent results on extremal problems in graph theory.
{\it Theory of Graphs}, International Symp. Rome, 1966, 118--123.

\bibitem{Erd_Tihany} P. Erd\H{o}s:
On some new inequalities concerning extremal properties
of graphs. {\it Theory of Graphs},
Proc. Coll. Tihany, (Hungary) 1966, 77--81.

\bibitem{ErdFraRod} P. Erd\H os, P. Frankl, and V. R\"odl:
The asymptotic number of graphs not containing a fixed subgraph
and a problem for hypergraphs having no exponent.
{\it Graphs Combin.} {\bf 2} (1986), 113--121.

\bibitem{ErdGyoSim} P. Erd\H os, E. Gy\H ori, and M. Simonovits:
How many edges should be deleted to make a triangle-free graph bipartite?
{\it  Sets, graphs and numbers} (Budapest, 1991), 239--263,
{\it Colloq. Math. Soc. J\'anos Bolyai}, {\bf 60}, North-Holland, Amsterdam, 1992.

\bibitem{ErdSimLim} P. Erd\H{o}s and M. Simonovits:
A limit theorem in graph theory.
{\it Studia Sci. Math. Hungar.} {\bf 1} (1966), 51--57.

\bibitem{ErdStone} P. Erd\H{o}s and A. H. Stone:
On the structure of linear graphs.
{\it Bull. Amer.  Math. Soc.} {\bf 52} (1946), 1089--1091.

\bibitem{Fur} Z. F\"uredi:
Extremal hypergraphs and combinatorial geometry.
{\it Proc. Internat. Congress of Mathematicians},
Vol. 1, 2 (Z\"urich, 1994), 1343--1352, Birkh\"auser, Basel, 1995.

\bibitem{Gyo} E. Gy\H ori:
On the number of edge disjoint cliques in graphs of given size.
{\it Combinatorica} {\bf 11} (1991), 231--243.

\bibitem{HanTof} D. Hanson and B. Toft:
$k$-saturated graphs of chromatic number at least $k$.
{\it Ars Combin.} {\bf 31} (1991), 159--164.

\bibitem{KomlosSim} J. Koml\'os and M. Simonovits:
Szemer\'edi's regularity lemma and its applications in graph theory.
{\it Combinatorics, Paul Erd\H os is eighty}, Vol. 2 (Keszthely, 1993), 295--352,
Bolyai Soc. Math. Stud., {\bf 2}, Budapest, 1996.

\bibitem{KomShoSimSze} J. Koml\'os, Ali Shokoufandeh, M. Simonovits, and E. Szemer\'edi:
The regularity lemma and its applications in graph theory.
{\it Theoretical aspects of computer science}
(Tehran, 2000), 84--112,
{\it Lecture Notes in Comput. Sci.}, {\bf 2292}, Springer, Berlin, 2002.

\bibitem{Martin2015} R. R. Martin:
On the computation of edit distance functions.
{\it Discrete Math.} {\bf 338} (2015), %no. 2,
291--305.

\bibitem{Mub} D. Mubayi:
Books versus triangles, {\it Journal of Graph Theory} {\bf 70} (2012), % no. 2,
171--179.

\bibitem{RuzSze} I. Z. Ruzsa and E. Szemer\'edi:
Triple systems with no six points carrying three triangles.
{\it Combinatorics} Proc. Fifth Hungarian Colloq., Keszthely, 1976, Vol. II, pp. 939--945,
{\it Colloq. Math. Soc. J\'anos Bolyai}, {\bf 18}, North-Holland, Amsterdam-New York, 1978.

\bibitem{Sim_Tihany} M. Simonovits:
A method for solving extremal problems in graph theory.
{\it Theory of Graphs}, (P. Erd\H{o}s and G. Katona, eds.)
Proc. Coll. Tihany (1966), 279--319.

\bibitem{SzemRegu} E. Szemer\'edi: Regular partitions of graphs.
{\it Problemes Combinatoires et Theorie des Graphes} (ed. I.-C. Bermond et al.),
CNRS, {\bf 260} Paris, 1978, pp. 399--401.

\bibitem{TuranA} P. Tur\'an:
On an extremal problem in graph theory. {\it Matematikai Lapok}, {\bf 48} (1941),
 436--452 (in Hungarian).
Reprinted in English {\it in:}
 {\it Collected papers of Paul Tur\'an}. Akad\'emiai Kiad\'o,
Budapest, 1989. Vol 1--3.



\end{thebibliography}

\end{document}
-------------------------------------------


\def\Fig#1/#2\par{\vskip3mm\begin{center}
 \boxit{
 \epsfig{figure=#1.eps}
 \vskip1mm
 {\small \sc \Rd #2}
 }\end{center}\vskip4mm}
\def\text#1{~\mbox{#1}~}
\def\separule{\bigskip\hrule\vskip1mm\hrule\bigskip}
% \def\binom#1#2{{#1\choose#2}}
% \def\Imply{\quad\Longrightarrow\quad }






\begin{theorem} {\rm (Erd\H os, 1970)}\newline
Suppose  $K_{p+1}\not\subset G$,
 then $\exists$ a $p$-chromatic $H$ on the same vertex set,
$$V(H)=V(G)$$
 majorizing the degrees:
$$  \deg_H(x)\geq \deg_G(x)$$
for every $x\in V$.
\end{theorem}


\bibitem{KomSarkSzemBlowUp}
J. Koml\'os, G. N. S\'ark\"ozy, and E. Szemer\'edi:
  Blow-up lemma.  {\it Combinatorica} {\bf 17} (1997), 109--123.

\bibitem{SimNes}
M. Simonovits:
How to solve a Tur\'an type extremal graph problem?
Contemporary trends in discrete mathematics (Stirin Castle, 1997), 283--305,
{\it DIMACS Ser. Discrete Math. Theoret. Comput.  Sci.}, {\bf 49},
Amer. Math. Soc., Providence, RI, 1999.

\bibitem{TuranB} P. Tur\'an:
On the theory of graphs.  {\it Colloq. Math.} {\bf 3} (1954), 19--30.
