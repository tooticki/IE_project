\documentclass[10pt]{amsart}

\usepackage{amsfonts,amssymb,amsthm,amsmath,euscript}

%\usepackage[headings]{fullpage}

\setlength{\textheight}{21cm}

\hoffset=-0.2in
\voffset=-2mm
\setlength{\textwidth}{144mm}


\usepackage{amsfonts,amssymb,amsthm,amsmath,euscript}
\usepackage{units}
\usepackage{subfigure}
\usepackage{graphicx}
\usepackage{tikz}
\usepackage{pinlabel}
\usepackage[colorlinks,citecolor=blue,linkcolor=red]{hyperref}
\input{xy}
\xyoption{all}



\usepackage[english]{babel}
\usepackage[ansinew]{inputenc}
\usepackage{amsmath, amsthm, amssymb, graphicx, mathrsfs}
\usepackage{color}

%\usepackage{pinlabel}


%\hoffset=-0.2in
%\setlength{\textwidth}{137mm}



%\newcommand{\mathieu}[1]{{\color{blue}#1}}
\newcommand{\mathieu}[1]{{\color{blue}#1}\marginpar{change}}

\newcommand{\Ab}{\mathbb A}
\newcommand{\Ac}{\mathcal A}
\newcommand{\Af}{\mathbf A}
\newcommand{\Bb}{\mathbb B}
\newcommand{\Bc}{\mathcal B}
\newcommand{\Cb}{\mathbb C}
\newcommand{\Cc}{\mathcal C}
\newcommand{\Ec}{\mathcal E}
\newcommand{\Fb}{\mathbb F}
\newcommand{\Fc}{\mathcal F}
\newcommand{\Hc}{\mathcal H}
\newcommand{\Kc}{\mathcal K}
\newcommand{\Lc}{\mathcal G}
\newcommand{\Ms}{\mathscr M}
\newcommand{\Nb}{\mathbb N}
\newcommand{\Nc}{\mathcal N}
\newcommand{\Pb}{\mathbb P}
\newcommand{\Pc}{\mathcal P}
\newcommand{\Ps}{\mathscr P}
\newcommand{\Rb}{\mathbb R}
\newcommand{\Sc}{\mathcal S}
\newcommand{\Tc}{\mathcal T}
\newcommand{\Zb}{\mathbb Z}

\newcommand{\inv}{^{-1}}
\newcommand{\epm}{^{\pm 1}}
\newcommand{\la}{\langle}
\newcommand{\ra}{\rangle}
\newcommand{\len}[1]{\left\| #1 \right\|}

\makeatletter %Definition of \overrightharpoonup
   \def\rightharpoonupfill@{\arrowfill@\relbar\relbar\rightharpoonup}
   \newcommand{\overrightharpoonup}{%
   \mathpalette{\overarrow@\rightharpoonupfill@}}
\makeatother 
\makeatletter %Definition of \overleftharpoonup
   \def\leftharpoonupfill@{\arrowfill@\leftharpoonup\relbar\relbar}
   \newcommand{\overleftharpoonup}{%
   \mathpalette{\overarrow@\leftharpoonupfill@}}
\makeatother 
\newcommand{\ora}{\overrightharpoonup}%{\overrightarrow }
\newcommand{\ola}{\overleftharpoonup}%{\overleftarrow }

%\DeclareMathOperator{\Aut}{Aut}
%\DeclareMathOperator{\Out}{Out}

\DeclareMathOperator{\Inn}{Inn}

%\DeclareMathOperator{\PCurr}{{\Pb}Curr}



\DeclareMathOperator{\vol}{vol}
\DeclareMathOperator{\ill}{ill}
\DeclareMathOperator{\im}{im}
\DeclareMathOperator{\rk}{rk}
\DeclareMathOperator{\id}{Id}


%%%%% from CHL3,  6. april 06 %%%%%%%

\newcommand{\cal}{\mathcal}
\newcommand\<{\langle}
\renewcommand\>{\rangle}
\let\Box\square
\newcommand{\fol}{\text{\rm F\o l}\,}
\newcommand{\folx}{\text{\rm F\o l}_X}
\newcommand{\folxk}{\text{\rm F\o l}_{X_k}}
\newcommand{\foly}{\text{\rm F\o l}_Y}
\newcommand{\bdx}{\partial_X\!}
\newcommand{\bdxk}{\partial_{X_k}\!}
\newcommand{\bdy}{\partial_Y\!}
\newcommand{\bdst}{\partial^*\!}
\newcommand{\LL}{{\cal L}}
\newcommand{\CA}{{\cal A}}
\newcommand{\CB}{{\cal B}}
\newcommand{\CC}{{\cal C}}
\newcommand{\CG}{{\cal G}}
\newcommand{\CH}{{\cal H}}
\newcommand{\CM}{{\cal M}}
\newcommand{\CP}{{\cal P}}
\newcommand{\CR}{{\cal R}}
\newcommand{\CT}{{\cal T}}
\newcommand{\LG}{\mathfrak L}
\newcommand{\Fm}{\mathfrak m}
\newcommand{\hr}{\mathfrak H}
\newcommand{\DB}{\|}
\newcommand{\G}{\Gamma}
\newcommand{\wbar}{\overline}
\newcommand{\boxx}{\unskip \nopagebreak \hfill $\square$}
%%%%%%%%%
\newcommand{\reals}{\mathbb R}
\newcommand{\integers}{\mathbb Z}
\newcommand{\vbar}{\, | \,}
\newcommand{\chop}{\dagger}
\newcommand{\doublevbar}{||}
\newcommand{\dvb}{\|}
\def\epsilon{\varepsilon}
\def\phi{\varphi}
%\def\psi{\varpsi}
\def\Pr{\mathbb P}
\def\hat{\widehat}


\newcommand{\Supp}{\mbox{Supp}}
\newcommand{\supp}{\mbox{Supp}}
\newcommand{\Curr}{\mbox{Curr}}
\newcommand{\Out}{\mbox{Out}}
\newcommand{\Aut}{\mbox{Aut}}

%\newcommand{\inv}{^{-1}}

\newcommand{\pmo}{^{\pm 1}}
\newcommand{\BBT}{\mbox{BBT}}
\newcommand{\Ax}{\mbox{Ax}}
\newcommand{\obs}{^{\mbox{\scriptsize obs}}}
\newcommand{\rat}{_{\mbox{\scriptsize rat}}}
\newcommand{\integ}{_{\mbox{\scriptsize int}}}
\newcommand{\smsm}{\smallsetminus}

%\newcommand{\vol}{\mbox{vol}}

\newcommand{\Stab}{\mbox{Stab}}
\newcommand{\Id}{\mbox{Id}}



%%%%%

\newcommand{\PCN}{\Pr\Curr(\FN)}
%\newcommand{\cvnbar}{{\overline{cv}}_N}
%\newcommand{\CVnbar}{{\overline{CV}}_N}


\newcommand{\FN}{F_N}   % F ou F_n ou F_N ?
\newcommand{\cvn}{\mbox{cv}_N}
%\newcommand{\cvnbar}{\mbox{\overline{cv}}_N}
\newcommand{\cvnbar}{\overline{\mbox{cv}}_N}
\newcommand{\CVN}{\mbox{CV}_N}
\newcommand{\CVNbar}{\overline{\mbox{CV}}_N}
%\newcommand{\cvn}{\cvn}
%\newcommand{\CVN}{\cvn}
%\newcommand{\CQ}{Q}     % l'application Q: \cal Q ou Q ?
\newcommand{\CQ}{{\cal Q}}
\newcommand{\Tobs}{\widehat T^{\mbox{\scriptsize obs}}}
\newcommand{\Tobszero}{\widehat T_{0}^{\mbox{\scriptsize obs}}}
\newcommand{\Tobsone}{\widehat T_{1}^{\mbox{\scriptsize obs}}}
\newcommand{\Tprobs}{\widehat{T'}^{\mbox{\scriptsize obs}}}


\newcommand{\FM}{F_M}   % F ou F_n ou F_N ?
\newcommand{\FA}{F({\CA})}
\newcommand{\cvm}{\mbox{cv}_M}
%\newcommand{\cvnbar}{\mbox{\overline{cv}}_N}
\newcommand{\cvmbar}{\overline{\mbox{cv}}_M}
\newcommand{\CVM}{\mbox{CV}_M}


\newcommand{\PCurr}{\Pr\Curr(\FN)}


\newcommand{\R}{\mathbb R}
\newcommand{\Z}{\mathbb Z}
\newcommand{\Q}{\mathbb Q}
\newcommand{\N}{\mathbb N}
\newcommand{\Hy}{\mathbb H}
\newcommand{\A}{{\cal A}}
\renewcommand{\L}{{\cal L}}



\usepackage{amsmath,amsthm,amssymb}
\newtheorem{definition}{Definition}
\newtheorem{question}{Question}
\newtheorem{proposition}{Proposition}[section]
\newtheorem{theorem}[proposition]{Theorem}
\newtheorem{example}{Example}

\newtheorem{theor}{Theorem}
\renewcommand{\thetheor}{\Alph{theor}}
\newtheorem{corol}[theor]{Corollary}
\renewcommand{\thecorol}{\Alph{theor}}



\newtheorem{thm}{Theorem}[section]
\newtheorem{lem}[thm]{Lemma}
\newtheorem{cor}[thm]{Corollary}
\newtheorem{conj}[thm]{Conjecture}
\newtheorem{prop}[thm]{Proposition}
\theoremstyle{definition}
\newtheorem{defn}[thm]{Definition}
\newtheorem{notation}[thm]{Notation}
\newtheorem{conv}[thm]{Convention}
\newtheorem{rem}[thm]{Remark}
\newtheorem{exmp}[thm]{Example}
\newtheorem{prop-defn}[thm]{Proposition-Definition}
\newtheorem{lede}[thm]{Lemma-Definition}
\newtheorem{warning}[thm]{Warning}
\newtheorem{history}[thm]{Historical Note}
\newtheorem{openp}[thm]{Open Problem}
\newtheorem{prob}[thm]{Problem}
\newtheorem{case}{Case}
\newtheorem{step}{Step}
\newtheorem{claim}{Claim}
\newtheorem*{claim*}{Claim}

%%%%%%%%%%%%%%%%%%%%%%%%%%%%%%%%%%%%%%%%%%%%%%%%%%%%%%%%%%%%%%%%%%



%\newtheorem{theorem}{Theorem}
%\newtheorem{question}{Question}
%\newtheorem{proposition}[theorem]{Proposition}
%\newtheorem{lemma}[theorem]{Lemma}
%\newtheorem{corollary}[theorem]{Corollary}
%\newtheorem{conv}[theorem]{Convention}

%\theoremstyle{remark}
%\newtheorem*{remark}{Remark}
%\newtheorem{example}{Example}
%\newtheorem{defn}{Definition}
%\newtheorem{case}{Case}
%\newtheorem{claim}{Claim}
%\newtheorem*{claim*}{Claim}
%\newtheorem{step}{Step}


\begin{document}


\title{Corrigendum to: ``Spectral rigidity of automorphic orbits in free groups''}


\author{Mathieu Carette}
\address{\tt SST/IRMP, 
Chemin du Cyclotron 2, bte L7.01.01, 
1348 Louvain-la-Neuve,
Belgium}
\email{\tt mathieu.carette@uclouvain.be}

\author{Stefano Francaviglia}

\address{\tt Dipartimento di Matematica of the University of Bologna, 
Pizza Porta S. Donato 5, 40126 Bologna, Italy}
\email{\tt francavi@dm.unibo.it }


\author{Ilya Kapovich}
\address{\tt Department of Mathematics, University of Illinois at
 Urbana-Champaign, 1409 West Green Street, Urbana, IL 61801, USA
 \newline http://www.math.uiuc.edu/\~{}kapovich/} \email{\tt
  kapovich@math.uiuc.edu}

\author{Armando Martino}
\address{\tt School of Mathematics,
University of Southampton;
Highfield, Southampton,
SO17 1BJ, United Kingdom }
\email{\tt A.Martino@soton.ac.uk}



%\thanks{\today}


	
\begin{abstract}
Lemma 5.1 in our paper~\cite{CFKM} says that every infinite normal subgroup of $\Out(F_N)$ contains a fully irreducible element; this lemma was substantively used
in the proof of the main result, Theorem~A in \cite{CFKM}.
Our proof of Lemma 5.1 in \cite{CFKM} relied on a subgroup classification result of
Handel-Mosher~\cite{HM}, originally stated in \cite{HM} for arbitrary
subgroups $H\le \Out(F_N)$. It subsequently turned out (see p. 1 in \cite{HM1}) that the proof
of the Handel-Mosher theorem needs the assumption that $H$ be finitely
generated. Here we provide an alternative proof of Lemma~5.1 from
\cite{CFKM}, which uses the corrected version of the Handel-Mosher
theorem and relies on the 0-acylindricity of the action of $\Out(F_N)$
on the free factor complex (due to Bestvina-Mann-Reynolds).
\end{abstract}
	

\thanks{The first author is a Postdoctoral Researcher of the F.R.S.-FNRS (Belgium). The third author was supported by Collaboration Grant no. 279836 from the Simons Foundation}


\subjclass[2000]{Primary 20F, Secondary 57M, 37D}
	
	
	
\maketitle

%\tableofcontents

	
\section{Introduction}

The purpose of this note is to correct a gap in our
paper~\cite{CFKM}. The proof of the main result, Theorem~A, of~\cite{CFKM},
substantively relies on Theorem~1.1 of Handel and
Mosher~\cite{HM} about classification of subgroups of $\Out(F_N)$.
Originally Theorem~1.1 was stated in \cite{HM} for arbitrary subgroups
$H\le \Out(F_N)$, and we applied that statement in~\cite{CFKM}.  After
our paper~\cite{CFKM} was published, we were informed that the proof of Theorem~1.1 in \cite{HM} only goes through under
the additional assumption that $H\le\Out(F_N)$ be finitely generated;
see the footnote at p.1 of \cite{HM1}.

The specific use of Theorem~1.1 of \cite{HM} in \cite{CFKM} occurs in
the proof of Lemma~5.1 in \cite{CFKM}. This proof no longer works,
when the Handel-Mosher result is replaced by its finitely generated
version.  This situation has created a gap in the proof of Lemma~5.1 given
in \cite{CFKM}.

In this corrigendum we fix this gap and provide an alternative proof of Lemma~5.1. Thus Theorem~A in \cite{CFKM} and all the other results proved
there remain valid in their original form.  Lemma~5.1 in \cite{CFKM} stated the following:

\begin{prop}\label{prop:fix}
Let $N\ge 2$ and let $H\le \Out(F_N)$ be an infinite normal
subgroup. Then $H$ contains some fully irreducible element $\phi$.
\end{prop} 
Note that for $N\ge 3$ every nontrivial normal subgroup of
$\Out(F_N)$ is infinite, but $\Out(F_2)$ does possess a finite
nontrivial normal subgroup (namely the center of $\Out(F_2)$, which is
cyclic of order $2$). Recall also that an element $\phi\in\Out(F_N)$
is called \emph{fully irreducible} or \emph{iwip} (for ``irreducible
with irreducible powers'') if no positive power of $\phi$ preserves
the conjugacy class of a proper free factor of $F_N$.

The original formulation of Theorem~1.1 in \cite{HM} said that for an
arbitrary subgroup $H\le\Out(F_N)$, either $H$ contains a fully
irreducible element or $H$ has a subgroup of finite index $H_0$ such
that $H_0$ preserves the conjugacy class of some proper free factor of
$F_N$. As noted above, it turns out that the proof of
Theorem~1.1 in \cite{HM} only goes through under the additional
assumption that $H$ be finitely generated.



The new proof of Lemma~5.1 of \cite{CFKM}, presented here,
is quite different from our original argument in \cite{CFKM}, although the proof
still relies on the corrected finitely generated version of the
Handel-Mosher subgroup classification theorem. Another key
ingredient in this new argument is the proof, due to Bestvina, Mann
and Reynolds, of 0-acylindricity of the $\Out(F_N)$ action on the
free factor complex $\mathcal{FF}_N$.

The proof of 0-acylindricity was communicated to us
by Bestvina and Reynolds. Since this proof does not appear
anywhere in the literature, we include it here for completeness; see
Proposition~\ref{prop:acyl} below.

We are grateful to Ric Wade for bringing to our attention the issue
with the original formulation of Theorem~1.1 of \cite{HM}. We are also
grateful to Mladen Bestvina and Patrick Reynolds for explaining to us
the argument for establishing 0-acylindricity of
$\mathcal{FF}_N$. Finally, we thank Martin Lustig for very helpful
discussions regarding the train track theory.


\section{$0$-Acylindricity}


We will use the terminology and notations from~\cite{CFKM}. In
particular, $\cvn$ denotes the (unprojectivized) Outer space, $\CVN$
denotes the projectivized Outer space, $\cvnbar$ denotes the closure
of $\cvn$ in the hyperbolic length function topology, and $\CVNbar$ denotes the projectvization of $\cvnbar$, so that $\CVNbar$ is the standard compactification of $\CVN$.


Following \cite{Rey,BR}, if $T\in\cvnbar$ and $A\le F_N$ is a proper free factor of $F_N$, we say that $A$ \emph{reduces $T$} if there exists an $A$-invariant subtree $T'$ of $T$ such that $A$ acts on $T'$ with dense orbits (the subtree $T'$ is allowed to consist of a single point). For $T\in \cvnbar$ denote by $\mathcal R(T)$ the set of all proper free factors of $F_N$ which reduce $T$.  Note that in many cases $\mathcal R(T)$ is the empty set.


\subsection{Two useful lemmas}

Lemma~\ref{lem:reduce} and Lemma~\ref{lem:reduce1} below are
related statements, and either one  can be used to complete the
proof of 0-acylindricity. The proof of Lemma~\ref{lem:reduce} relies
on the use of relative train tracks and is based on an argument of
Brian Mann as communicated to us by Bestvina. The proof of
Lemma~\ref{lem:reduce1} relies on the use of geodesic currents and
algebraic laminations, and is due to Patrick Reynolds. We include both
lemmas here since each of them is interesting in its own right.


\begin{lem}\label{lem:reduce}
Let $h\in \Out(F_N)$ be an element of infinite order and let $A$ be a
proper free factor of $F_N$ such that $h[A]=[A]$. Then there exist a positive power $\psi=h^m$ of $h$ and a nontrivial
free factor $A'$ of $A$ with the following property:

 Let $T_0$ in $\cvn$ be arbitrary and let $n_i\to\infty$ be a
 subsequence such that for some $T\in\cvnbar$ we have $\lim_{i\to\infty} [T_0]\psi^{n_i}=[T]$ in $\CVNbar$.
 Then $A'$ reduces $T$.
\end{lem}


\begin{proof}
Among all the $h$-periodic, up to conjugacy, proper free factors of
$A$, choose a free factor $A'$ of smallest rank. Replace $h$ by a
positive power $\psi=h^m$ so that $\psi[A']=[A']$.

Let $T_0, n_i$ and $T$ be as in the statement of the lemma.

Now, using the construction of Bestvina-Handel~\cite{BH92},  there exists a relative train-track $f:\Gamma\to\Gamma$ for $\psi$ with
marking $\alpha:F_N\to\pi_1(\Gamma)$ such that the lowest
stratum $\Gamma_0$ of $\Gamma$ corresponds to $A'$.

Let $T_1\in \cvn$ be the point in Outer space determined by $\Gamma$
(with, say, the simplicial metric on $\Gamma$).

Then, after possibly passing to a further subsequence of $n_i$, we may assume that $[T_1]\psi^{n_i}$ converges to some tree $[S]\in \CVNbar$ as
$i\to\infty$.  We will first show that $A'$ reduces $S$.

For some $c_i\ge 0$ we have $c_iT_1\psi^{n_i}\to_{n\to\infty} S$ in $\cvnbar$.
Since $\psi\in\Out(F_N)$ has infinite order, we have $\lim_{n\to\infty}
c_i=0$.


If $A'$ has rank $1$, then $A'=\langle a \rangle$ and $\psi([a])=[a^{\pm
  1}]$. Since $c_i$ converges to $0$, we then have
$||a||_S=\lim_{i\to\infty} ||a||_{c_iT_1\psi^{n_i}}=\lim_{i\to\infty} c_i
||\psi^{n_i}(a)||_{T_1}=\lim_{i\to\infty} c_i ||a||_{T_1}=0$.
Thus $A'$ fixes a point of $S$ and hence $A'$ reduces $S$.


Suppose now that $A'$ has rank at least $2$. Then $\psi$ is
exponentially growing and the restriction of $\psi$ to $A'$ (after a
conjugation) is a fully irreducible automorphism $\psi'$ of $A'$. Let $\lambda$ be the largest P.F. eigenvalue among all the strata of $\Gamma$.
Theorem~6.2 of \cite{L} then implies that there exists $k\ge 0$ such that we can choose $c_i=\lambda^{-n_i}n_i^{-k}$ for the rescaling constants, 
so that $\lambda^{-n_i}n_i^{-k}T_1 \psi^{n_i}$ converges to (a scalar multiple of) $S$ in $\cvnbar$.  Let
$\lambda'$ be the P.F. eigenvalue for the bottom stratum of $\Gamma$,
that is $\lambda'>1$ is the stretch factor of the fully irreducible
$\psi'\in \Out(A')$. Thus for every nontrivial $a'\in A'$ (except possibly for $a'$ conjugate to a power of the peripheral curve, if $\psi'$ is a  non-hyperbolic fully irreducible and is induced by a pseudo-Anosov of a compact surface with a single boundary component)  we see that $||\psi^n(a')||_{T_1}$ grows as $(\lambda')^n$.


If $\lambda'<\lambda$ then for each $a'\in A'$  $||\psi^n(a')||_{T_1}$ grows as at a smaller exponential rate than $\lambda^n$ and hence, by the above choice of the rescaling constants $c_i$, we get
 $||a'||_S=0$. Thus in this case $A'$ fixes a point of $S$, so that $A'$ reduces $S$.

Suppose now that $\lambda=\lambda'$.  

If $k>0$, then, since $c_i=\lambda^{-n_i}n_i^{-k}$, we see that for every $a'\in A'$ the sequence  $\lambda^{-n_i}n_i^{-k}||a'||_{T_1 \psi^{n_i}}=\lambda^{-n_i}n_i^{-k}||\psi^{n_i}(a)||_{T_1}$ is bounded above by $c' n_i^{-k}\to_{i\to\infty} 0$ (where $c'>0$ is some constant depending on $a'$) and hence $||a'||_S=0$.  Again, in this case $A'$ fixes a point of $S$ and thus reduces $S$.


Finally, suppose that $k=0$, so that $c_i=\lambda^{-n_i}$.
Then for every $a'\in A'$ we have $||a'||_S=\lim_{i\to\infty} \frac{||\psi^{n_i}(a') ||_{T_1}}{\lambda^{n_i}}=||a'||_{Y_+}$ where $[Y_+]$ in
the compactified Outer space  of $A'$ is the attracting tree of
$\psi'$. Therefore the minimal $A'$-invariant subtree $S_{A'}$ of $S$ is
$A'$-equivariantly isometric to $Y_+$. Since $\psi'$ is a fully
irreducible automorphism of $A'$, the action of $A'$ on $Y_+=S_{A'}$ has
dense orbits (see \cite{LL}, for example). Thus $A'$ reduces $S$.

Now we return to the original tree $T_0\in\cvn$ such that $[T_0]\psi^{n_i}\to_{i\to\infty} [T]\in\CVNbar$.
Since we also have $[T_1]\psi^{n_i}\to_{i\to\infty} [S]$, a general result of Sela~\cite{Sela} implies that  there exists a
bi-Lipschitz $F_N$-equivariant homeomorphism $f: S\to T$. 
The fact that $A'$ reduces $S$ now implies that $A'$ also reduces $T$.  Indeed,
let $Y$ be an $A$-invariant subtree of $S$ such that $A'$ acts on $Y$ with dense orbits. Then $f(Y)$ is a connected subset of $T$ and hence $f(Y)$ is a subtree of $T$.
Since $A'$ acts on $Y$ with dense orbits and $f$ is bi-Lipschitz and $F_N$-equivariant, it follows that $A'$ acts on $f(Y)$ with dense orbits as well. Thus $A'$ reduces $T$, as claimed.
\end{proof}







If a tree $T\in\cvnbar$ does not have dense $F_N$-orbits, then it is known (see \cite{CHL2,Rey,KL3}) that $T$ canonically decomposes as a "graph of actions".
In this case there exists an $F_N$-equivariant distance non-increasing map $f:T\to Y$ for some very small simplicial metric tree $Y\in\cvnbar$ such for every vertex $v$
  of $Y$ the stabilizer $Stab_{F_N}(v)$ acts with dense orbits on some $Stab_{F_N}(v)$-invariant subtree $T_v$ of $T$ (where $T_v$ may be a single point). Moreover, the tree $Y$ is obtained from $T$ be collapsing each $T_v$ to a point, where $v$ varies over all vertices of $Y$.  In this situation we will say that  $Y\in\cvn$ is the \emph{simplicial tree associated} to $T$. 
Note that if $T\in\cvn$ then $Y=T$.

\begin{lem}\label{lem:reduce1}
Let $A$ be a proper free factor of $F_N$ and let $h_n\in \Out(F_N)$ be an infinite sequence of distinct elements of $\Out(F_N)$ such that $h_n([A])=[A]$ for all $n\ge 1$.
Let $T_0\in\cvn$ and $T\in\cvnbar$ be such that $[T_0]h_n\to [T]$ in
$\CVNbar$ as $n\to\infty$. Then:
\begin{enumerate}
\item If $T$ has dense $F_N$-orbits then there exists a nontrivial free factor $A'$ of $A$ such that $A'$
 reduces $T$.

\item If $T$ does not have dense $F_N$-orbits and $Y\in\cvnbar$ is the associated simplicial tree, then some
  nontrivial free factor $A'$ of $A$ reduces $Y$.

\end{enumerate}

\end{lem}
\begin{proof}
 

There exist $c_n\ge 0$ such that $\lim_{n\to\infty} c_n T_0h_n=T$ in $\cvnbar$. Since the elements $h_n$ are distinct and the action of $\Out(F_N)$ on $\CVN$ is properly discontinuous, it follows that $T\in \cvnbar\setminus\cvn$ and that $\lim_{n\to\infty}c_n=0$.
For every $n\ge 1$ choose a representative $\beta_n\in \Aut(F_N)$ of the outer automorphism $h_n$ such that $\beta_n(A)=A$.


Choose a nontrivial element $a\in A$ and put $a_n=\beta_n^{-1}(a)\in A$.
Then 
\[
c_n||a_n||_{T_0h_n} =||a_n||_{c_nT_0h_n}=||a_n||_{c_nT_0\beta_n}=c_n||\beta_n(a_n)||_{T_0}=c_n||a||_{T_0}\to_{n\to\infty} 0.
\]

In $\PCN$ we have $\lim_{n\to\infty}
\frac{1}{||a_n||_{T_0}}\eta_{a_n}=\mu\ne 0$.  
Here, for a nontrivial element $g\in F_N$, $\eta_g\in \Curr(F_N)$ is the
``counting current'' associated to $g$~\cite{Ka}.

Since $T_0\in\cvn$, there exists $c>0$ such that $||g||_{T_0}\ge c>0$ for all $g\in F_N$.
Hence $||a_n||_{T_0}\ge c$ and therefore
\begin{gather*}
\langle T,\mu\rangle=\lim_{n\to\infty} \langle c_nT_0h_n,
\frac{1}{||a_n||_{T_0}}\eta_{a_n}\rangle=
\lim_{n\to\infty} \frac{c_n}{||a_n||_{T_0}} \langle T_0h_n, \eta_{a_n}\rangle=\\
\lim_{n\to\infty} \frac{c_n}{||a_n||_{T_0}} ||a_n||_{T_0h_n}=\lim_{n\to\infty} \frac{c_n}{||a_n||_{T_0}}||a||_{T_0} = 0.
\end{gather*}
Here $\langle\ ,\ \rangle:\cvnbar\times Curr(F_N)\to \mathbb R$ is the
continuous ``geometric intersection form'' constructed in \cite{KL2}.


Since $\langle T,\mu\rangle=0$, by the main result of \cite{KL3} we have $supp(\mu)\subseteq
L(T)$, where $supp(\mu)$ is the support of $\mu$ and $L(T)$ is the
dual algebraic lamination of $T$ (see \cite{CHL2} for background about dual algebraic laminations associated to elements of $\cvnbar$). Since $a_n\in A$ for all $n\ge 1$, the construction of $\mu$ implies that there is a leaf of $L(T)$ that is carried by $A$.

If $T$ has dense $F_N$-orbits, then by Corollary~6.7 of~\cite{Rey},
there exists a nontrivial free factor $A'$ of $A$ such that $A'$
reduces $T$, and part (1) of the lemma holds.

Suppose now that $T$ does not have dense $F_N$-orbits, and let $Y\in\cvnbar$  be the associated simplicial tree for $T$.

Then by Lemma~10.2 of \cite{KL3}, we have $L(T)\subseteq
L(Y)$.  Since the factor $A$ carries a leaf of $L(T)$, it follows
that $A$ also carries a leaf $\ell$ of $L(Y)$. The  description of the dual lamination of a very small simplicial tree,
given in Lemma~8.2 of \cite{KL3}, then implies that $A$
contains some nontrivial element acting elliptically on $Y$. Namely, consider a free basis $X$ of $F_N$ such that $X$ contains as a subset a free basis $X'$ of $A$.  Lemma~8.2 of \cite{KL3} now implies that for some vertex group $U$ of $T'$ and for the Stallings core subgroup graph $\Delta_U$~\cite{KM} (with oriented edges labelled by elements of $X^{\pm 1}$) representing the conjugacy class of $U$, there exists a bi-infinite reduced edge-path $\gamma$ in $\Delta_U$ corresponding to the leaf $\ell$ of $L(Y)$. The fact that $\ell$ is carried by $A$ means that all the edges of $\gamma$ are labelled by elements of $(X')^{\pm 1}$. Since $\gamma$ is an infinite reduced path, we can find an immersed circuit as a subpath of $\gamma$. Then the label $a'$ of this circuit is a nontrivial element of $A$ whose conjugate belongs to $U$ and thus $a'$ acts elliptically on $Y$.

If $A$ fixes a point of $Y$, then $A$ reduces $Y$ and we are
done. Otherwise consider the minimal $A$-invariant subtree $Y_A$ of
$Y$. Then the quotient graph of groups $\mathbb A=Y_A//A$ gives a
nontrivial very small splitting of $A$ with at least one nontrivial
vertex group. The general structural result (Lemma 4.1 of~\cite{BF93}) about very small
simplicial splittings of free groups then implies that there exists a nontrivial free factor $A'$ of $A$ such that
a conjugate of $A'$ is contained in some vertex group of $\mathbb
A$. Then $A'$ fixes a vertex of the tree $Y_A$ and therefore $A'$
reduces $Y$, as required.

\end{proof}

\subsection{0-acylindricity}

For $N\ge 2$ let $\mathcal{FF_N}$ be the free factor graph for
$F_N$. The vertices of $\mathcal{FF_N}$ are the conjugacy classes
$[A]$ of proper free factors $A$ of $F_N$. For $N\ge 3$ the adjacency
of vertices in $\mathcal F_N$ corresponds to containment: two distinct
vertices $[A]$ and $[B]$ are adjacent if there exist representatives
$A$ of $[A]$ and $B$ of $[B]$ such that $A\le B$ or $B\le A$. For
$N=2$ the definition of adjacency is somewhat different, see
\cite{BF11} for details. It is known that for $N\ge 2$ the graph
$\mathcal{FF}_N$ is connected and Gromov-hyperbolic~\cite{BF11,KRa,HH}. There is a
natural action of $\Out(F_N)$ on $\mathcal{FF}_N$ by simplicial isometries.

We denote the simplicial metric on $\mathcal{FF}_N$ by $d$.

\begin{prop}[0-acylindricity of the free factor complex]\label{prop:acyl}
There exists a constant $M\ge 1$ with the following property:


If $N\ge 2$ and if $A,B$ are proper free factors of $F_N$ such that $d([A],[B])>M$ then
\[
Stab_{\Out(F_N)}([A]) \cap Stab_{\Out(F_N)}([B]) 
\]
is finite.
\end{prop}

\begin{proof}
Corollary~5.3 of \cite{BR} implies that there exists a constant $C>0$ (independent of the rank $N$ of $F_N$)  such that if $T\in\cvnbar$ admits a reducing factor then the set $\mathcal R(T)$ of all reducing factors for $T$ has diameter $\le C$ in $\mathcal{FF}_N$.  Take $M=C+2$. Let $A,B$ be proper free factors of $F_N$ such that $d([A],[B])>M$.
Put $H:=Stab_{\Out(F_N)}([A]) \cap Stab_{\Out(F_N)}([B])$. We claim that $H$ is finite.

Indeed, suppose not, and $H$ is infinite. Since $\Out(F_N)$ is virtually torsion-free, it follows that there exists an element $h\in H$ of infinite order.
Let $T_0\in \cvn$ be arbitrary and let $T\in\cvnbar$ be such that
$\lim_{i\to\infty} [T_0h^{n_i}]=[T]$ for some subsequence $n_i\to\infty$.


Since $h[A]=[A]$ and $h[B]=[B]$, by Lemma~\ref{lem:reduce} or,
alternatively, by Lemma~\ref{lem:reduce1}, there exist a tree
$S\in\cvnbar$ and
nontrivial free factors $A'$ of $A$ and $B'$ of $B$ such that $A'$ and
$B'$ both reduce $S$. When appealing to Lemma~\ref{lem:reduce}, we can
take $S=T$. If appealing to Lemma~\ref{lem:reduce1}, we can take $S=T$
if $T$ has dense $F_N$ orbits and otherwise we can take $S=Y$ where
$Y\in\cvnbar$ is the simplicial tree associated to $T$.

Note that if $N=2$ then $A'=A$, $B'=B$, and the
factors $A,B$ are infinite cyclic.

Then $d([A'],[A])\le 1$, $d([B'],[B])\le 1$ and therefore $d([A'],[B'])> M-2=C$.
Thus the set of reducing factors for $S$ has diameter $>C$ in $\mathcal{FF}_N$, which contradicts the choice of $C$.
\end{proof}


\section{The proof of Proposition~\ref{prop:fix}}
We can now recover Lemma~5.1 of \cite{CFKM}, which is stated as
Proposition~\ref{prop:fix} above.


\begin{proof}[Proof of  Proposition~\ref{prop:fix}]
Suppose that $H\le Out(F_N)$ is an infinite normal subgroup but that $H$ does not contain a fully irreducible element.
Since $\Out(F_N)$ is virtually torsion-free and $H$ is infinite, it  follows that H contains an element $\phi$ of infinite order.

By assumption on $H$, $\phi$ is not fully irreducible and hence, after replacing $\phi$ by a positive power, $\phi$ fixes the conjugacy class $[A]$ of a proper free factor $A$ of $F_N$.


Now let $M\ge 1$ be the 0-acylindricity constant provided by Proposition~\ref{prop:acyl}. We choose a fully irreducible $\theta\in \Out(F_N)$ and look at the conjugates $\alpha_n= \theta^n \phi\theta^{-n}$.
Note that $\alpha_n$ fixes the conjugacy class $\theta^n[A]$.  Since $H$ is normal, we have $\alpha_n\in H$, and thus the subgroup $L_n=\langle \phi, \alpha_n\rangle$ is contained in $H$.
The subgroup $L_n$ of $\Out(F_N)$ is finitely generated so the (corrected)
Handel-Mosher subgroup classification theorem~\cite{HM,HM1} does apply to $L_n$.

Since we assumed that $H$ contains no fully irreducible elements, $L_n$ must contain a subgroup $K_n$ of finite index which preserves a vertex $[B_n]$ of the free factor complex. Hence some positive powers of $\phi$ and of $\alpha_n$ preserve $[B_n]$.

Since $\theta$ is fully irreducible, we have $d([A], \theta^n[A])\to\infty$  as $n\to\infty$ (see \cite{BF08,KL2}). We choose $n\ge 1$ big enough so that $d([A], \theta^n[A]) > 2M+1$.

Then either $d([A], [B_n])>M$ or $d([B_n], \theta^n[A])>M$.


In the first case we get that some positive power $\phi^i$ of $\phi$ fixes both the vertices $[A]$ and $[B_n]$ of $\mathcal{FF}_N$, so that $\phi^i$ belongs to the intersection of their stabilizers. This contradicts 0-acylindricity since $\phi$ has infinite order.


In the second case for some $i>0$ the element $\alpha_n^i=\theta^n\phi^i\theta^{-n}\in H$ fixes both $[B_n]$ and $\theta^n[A]$. Thus $\alpha_n^i$ belongs to the intersection of the stabilizers of $[B_n]$ and $\theta^n[A]$, which again contradicts 0-acylindricity, since $\alpha_n$ has infinite order.
\end{proof}

\begin{rem}
Let $\phi\in H$ be a fully irreducible element provided by
Proposition~\ref{prop:fix}. Theorem~8.5 of Dahmani, Guirardel and
Osin~\cite{DGO} then implies that for some $m\ge 1$ the normal closure
$U=ncl(\phi^m)$ of $\phi^m$ in $\Out(F_N)$ is free of infinite rank
and every nontrivial element of $U$ is fully irreducible. Since $H\le
\Out(F_N)$ is normal by assumption, we have $U\le H$.
\end{rem}



\begin{thebibliography}{ABC}




\bibitem{BH92} M.~Bestvina, and M.~Handel, \emph{Train tracks and
automorphisms of free groups.} Ann.  of Math.  (2) \textbf{135}
(1992), no.  1, 1--51


\bibitem{BF93} M. Bestvina and M. Feighn, \emph{Outer Limits},
preprint, 1993; \\ http://andromeda.rutgers.edu/\~{}feighn/papers/outer.pdf


\bibitem{BF11}
M. Bestvina and M. Feighn, \emph{Hyperbolicity of the complex of free factors}, Jour. Amer. Math. Soc., to appear;  arXiv:1107.3308




\bibitem{BF08}
M.~Bestvina, M.~Feighn, \emph{A hyperbolic $Out(F_n)$ complex},  Groups Geom. Dyn.  \textbf{4}  (2010),  no. 1, 31-~58


\bibitem{BR}
M. Bestvina and P. Reynolds,
\emph{The boundary of the complex of free factors}, 
preprint, 2012; arXiv:1211.3608



\bibitem{CFKM}
M. Carette, S. Francaviglia, I. Kapovich and A. Martino,
\emph{Spectral rigidity of automorphic orbits in free groups}, \textbf{12} (2012), 1457--1486



\bibitem{CHL2}
 T.~Coulbois, A.~Hilion, and M.~Lustig, \emph{ $\mathbb R$-trees and
 laminations for free groups II: The dual lamination of an $\mathbb
 R$-tree,} J. Lond. Math. Soc. (2)  \textbf{78}  (2008),  no. 3,
737--754


\bibitem{DGO}
F. Dahmani, V. Guirardel, D. Osin,
\emph{Hyperbolically embedded subgroups and rotating families in groups acting on hyperbolic spaces}, preprint, 2011; arXiv:1111.7048

%\bibitem{Gui}
%V.~Guirardel, \emph{Approximations of stable actions on $R$-trees.}
%Comment.  Math.  Helv.  \textbf{73} (1998), no.  1, 89--121


%\bibitem{Gui1}
%V.~Guirardel, \emph{Dynamics of ${\rm Out}(F\sb n)$ on the boundary of
%outer space.} Ann.  Sci.  \'Ecole Norm.  Sup.  (4) \textbf{33} (2000),
%no.  4, 433--465


\bibitem{HM}
M. Handel and L.~Mosher,
\emph{Subgroup classification in $Out (F_n)$}, preprint,
arXiv:0908.1255


\bibitem{HM1}
M. Handel and L.~Mosher,
\emph{Subgroup decomposition in $Out(F_n)$: Introduction and Research
Announcement}; preprint, 2013; arXiv:1302.2681



\bibitem{HH}
A. Hilion and C. Horbez,
\emph{The hyperbolicity of the sphere complex via surgery paths},
preprint, 2012;  arXiv:1210.6183 

\bibitem{Ka}
I. Kapovich, \emph{Currents on free groups}, in: ``Topological and asymptotic aspects of group theory'', 149--176, Contemp. Math., 394, Amer. Math. Soc., Providence, RI, 2006

\bibitem{KL2}
I.~Kapovich and M.~Lustig, \emph{Geometric Intersection Number and analogues of the Curve Complex for free
groups}, Geometry \& Topology \textbf{13} (2009),  1805--1833


\bibitem{KL3}
I.~Kapovich and M.~Lustig, \emph{Intersection form, laminations and
currents on free groups}, Geom. Funct. Analysis (GAFA), \textbf{19} (2010), no. 5, 1426--1467

\bibitem{KM}
I.~Kapovich and A.~Myasnikov, \emph{Stallings foldings and the
subgroup structure of free groups}, J. Algebra \textbf{248} (2002), no. 2, 608--668

\bibitem{KRa}
I. Kapovich and K. Rafi,
\emph{On hyperbolicity of free splitting and free factor complexes}, Groups, Geom. Dynam.,  to appear;  arXiv:1206.3626 

\bibitem{L}
G. Levitt, \emph{Counting growth types of automorphisms of free groups.} Geom. Funct. Anal. \textbf{19} (2009), no. 4, 1119--1146

\bibitem{LL} 
G. Levitt and M. Lustig
\emph{Irreducible automorphisms of $F_n$ have north-south dynamics on compactified outer space.} J. Inst. Math. Jussieu \textbf{2} (2003), no. 1, 59--72

%\bibitem{Pau89} F. Paulin, \emph{The Gromov topology on $R$-trees.}
%  Topology Appl.  \textbf{32} (1989), no.  3, 197--221

\bibitem{Rey}
P. Reynolds, \emph{Reducing systems for very small trees}, preprint, 2012; arXiv:1211.3378 


\bibitem{Sela}
Z. Sela, \emph{The Nielsen-Thurston classification and automorphisms of a free group. I.} Duke Math. J. \textbf{84} (1996), no. 2, 379-397




\end{thebibliography}	

	
\end{document}
