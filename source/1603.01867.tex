\documentclass[11pt, reqno,amsmath,amsthm,amssymb,amscd]{amsart}
%\documentclass[11pt,twoside]{article}
\usepackage{amsmath,amssymb}
\usepackage{graphics}
\usepackage[usenames]{color}
\openup 5pt \pagestyle{plain} \oddsidemargin -10pt \evensidemargin
-10pt \topmargin -50pt \textwidth 6.45truein \textheight 9.32truein
\parskip .030 truein
\baselineskip 3.1pt \lineskip 3.1pt \numberwithin{equation}{section}
\def\AAu{{{\cal A}_u}}
\def\BBu{{{\cal B}_{uv}}}
\def\BByx{\C((y^{-1}))((x^{\mbox{\tiny $\Q_-$}}))}
\def\CCu{{\cal C}_{uv}}
\def\FF{{\,\hat{\textbf{\textit{$\!$F$\ssc\,$}}}}}\def\pPP{{\,\hat{\textbf{\textit{$\!$P$\ssc\,$}}}}}
\def\lead{_{\rm lead}}\def\CA{{\textbf{CA }}}\def\CA{converges absolutely }
\def \NN{{\textbf{\textit{N}}}}
\def\aa{{m_0}}\def\bb{{m}}\def\rDS{\unrhd}
\def\tr{{\rm tr}}\def\DSS{\DS}\def\rDSS{\rDS}
\def\QJ{{quasi-Jacobi}}
\def\TT{{\mathbb T}}
\def\igo{{\rm igo}}\def\inv{{\rm neg}}
\def\QJ{{Jacobi }}
\def\Aut{{\rm Aut}}
\def\th{\theta}
\def\DS{\unlhd}
\def\ign{_{\rm igno}}
\def\de{e}
\def\Dhat{\ddot}
\def\Plar{\raisebox{0pt}{\mbox{${\sc\prod\limits_{\leftharpoondown}{\!\ssc\,}}$}}}
\def\Prar{\raisebox{0pt}{\mbox{${\sc\prod\limits_{\rightharpoondown}{\!\ssc\,}}$}}}
\def\pri#1{{\textbf{\textit{#1}}}}
\def\priF{{F}}\def\hht#1{\check{#1}}
\def\UPo#1{_{\la#1\ra}}
\def\UP#1{_{\la-#1\ra}}
\def\Comp#1{_{\la#1\ra}}
\def\aaa{\sigma}
\def\gcd{{\rm gcd}}
\def\Coeff{{\rm C_{oeff}}}
\def\comp#1{{}_{[#1]}{}}
\def\notdiv{\mbox{$\not|$}}
\def\one{{1\hskip-4pt 1}}
\def\adddot{$\!\!\!${\bf.}\ \ }
\def\ptl{\partial}
\def\Supp{{\rm Supp}}
\def\res{{\rm Res}}
\def\deg{{\rm deg}}
\def\max{{\rm max}}
\def\SUM#1#2{\mbox{$\sum\limits_{#1}^{#2}$}}
\def\ad{{\ssc\,}{\rm ad}}
\def\DD{\hbox{$I\hskip -4pt D$}}\def\sDD{\hbox{$\sc I\hskip -2.5pt D$}}
\def \N{\hbox{$I\hskip -3pt N$}}
\def \Z{\hbox{$Z\hskip -5.2pt Z$}}
\def \LL{\hbox{$I\hskip -5.2pt L$}}
\def\sZ{\hbox{$\sc Z\hskip -4.2pt Z$}}
\def \Q{\hbox{$Q\hskip -5pt \vrule height 6pt depth 0pt\hskip 6pt$}}
\def \R{\mathbb{R}}
\def \C{\hbox{$C\hskip -5pt \vrule height 6pt depth 0pt \hskip 6pt$}}
\def \sC{{\hbox{$\sc C\hskip -5pt \vrule height 5pt depth 0pt \hskip 6pt$}}}
\def\qed{\ \ \ifhmode\unskip\nobreak\fi\ifmmode\ifinner
         \else\hskip5pt\fi\fi
 \hbox{\hskip5pt\vrule width4pt height6pt depth1.5pt\hskip 1 pt}}
\def\a{\alpha}
\def\b{\beta}
\def\d{\delta}
\def\D{\Delta}
\def\g{\mbox{${\ssc\!}$\Large$\gamma$}}
%\def\g{\gamma}
\def\G{\Gamma}
\def\l{\lambda}
\def\o{\omiga}
\def\p{\psi}
\def\Vir{\mbox{Vir}}
\def\Si{\Sigma}
\def\si{\sigma}
\def\sc{\scriptstyle}
\def\ssc{\scriptscriptstyle}
\def\dis{\displaystyle}
\def\cl{\centerline}
\def\DD{{\cal D}}
\def\rl{\rightline}
\def\nl{\newline}
\def\ol{\overline}
\def\ul{\underline}
\def\wt{\widetilde}
\def\wh{\widehat}
\def\rar{\rightarrow}
\def\Rar{\Rightarrow}
\def\lar{\leftarrow}
\def\D{\Delta}
\def\Res{{\rm Res}}
\def\Lar{\Leftarrow}
\def\Rla{\Longleftrightarrow}
\def\Lra{\Longleftrightarrow}
\def\bs{\backslash}
\def\hs{\hspace*}
\def\rb{\raisebox}
\def\vs{\vspace*}
\def\VS#1{}
\def\PAR{}
\def\NOindent{}
\def\ra{\rangle}
\def\la{\langle}
\def\PP{{\cal P}}\def\PP{\C(x)((y))}
\def\ni{\noindent}
\def\hi{\hangindent}
\def\barx{x}
\def\bary{y}
\def\ha{\hangafter}
\def\WW{{\cal W}}
\def\AA{{\cal A}}
\def\BB{{\cal B}}
\def\CC{{\cal C}}
\def\II{{\cal I}}
\def\C{\mathbb{C}}
\def\F{\mathbb{C}}
\def\Z{\mathbb{Z}}\def\bB{{\textbf{\textit{b}}}}
\def\N{\mathbb{N}}
\def\Q{\mathbb{Q}}\def\SS{{\textbf{\textit{s}}}}\def\kk{\uu}
\def\uu{{\textbf{\textit{k}}}}
\def\DD{{\textbf{\textit{D}}}}
\def\dd{{\textbf{\textit{d}}}}
\def\lel{{\textbf{\textit{l}}}}\def\aa{{\textbf{\textit{a}}}}\def\bb{{\textbf{\textit{b}}}}
\def\bfit#1{{\textbf{\textit{#1}}}}\def\ii{{\textbf{\textit{i}}}}
\newtheorem{clai}{Claim}
\newtheorem{subc}{Subclaim}
\newtheorem{theo}{Theorem}[section]
\newtheorem{nota}[theo]{Notation}
\newtheorem{assu}[theo]{Assumption}
\newtheorem{mtheo}{Main Theorem}
\newtheorem{lemm}[theo]{Lemma}
\newtheorem{fact}[theo]{Fact}
\newtheorem{stat}[theo]{Statement}
\newtheorem{prop}[theo]{Proposition}
\newtheorem{rema}[theo]{Remark}
\newtheorem{exam}[theo]{Example}
\newtheorem{conv}[theo]{Convention}
\newtheorem{coro}[theo]{Corollary}
\newtheorem{conj}[theo]{Conjecture}
\newtheorem{defi}[theo]{Definition}
\def\bq{{\mathit{\mathbf{r}{\ssc\,}}}}\def\bF{\CiL F}\def\bG{\CiL G}\def\baf{\CiL f}\def\bag{\CiL g}\def\bau{\CiL U}\def\bav{\CiL V}
\def\bx{\CiL x}\def\by{\CiL y}
\def\Su{{{\frak s}{\ssc\,}}}\def\M{{\textbf{\textit{M}}}}\def\K{{\textbf{\textit{${\kappa}$}}}}
\def\th{{\theta}}\def\KA{}\def\KA{{\mathbf{\frak u}{\ssc\,\!}}}\def\pq{{\frak c}}
\def\Su{{{\frak s}{\ssc\,}}}\def\M{{\textbf{\textit{M}}}}\def\K{{\textbf{\textit{${\kappa}$}}}}
\def\ep{{\textbf{\textit{\footnotesize$\mathbf{\mathcal{E}}$}}}}
\def\scep{{{\small\textbf{\textit{\footnotesize$\sc\mathbf{\mathcal{E}}$}}}}}
%\def\sep{{\textbf{\textit{\footnotesize$\mathbf{\mathcal{\sc E}}$}}}}
\def\ffp{{\textbf{\textit{p}}}}
\def\th{{\theta}}\def\KA{}\def\KA{{\mathbf{\frak u}{\ssc\,\!}}}\def\pq{{\frak c}}
\def\equa#1#2{\begin{equation}\label{#1}\mbox{$#2$}\end{equation}}
\def\equan#1#2{\begin{equation*}\mbox{$#2$}\end{equation*}}
\def\re{_{\rm re}}\def\im{_{\rm im}}
\def\TH{}\def\tau{}
\def\NOUSE#1{}

\baselineskip15pt
\lineskip5.1pt
\begin{document}
\title{Generalizations of local bijectivity of Keller maps\\
and a proof of $2$-dimensional Jacobian conjecture
%proof of $n$-dimensional Jacobian conjecture
}
\author{Yucai Su\\
\email{
\lowercase{ycsu@tongji.edu.cn}}}}


\date{\noindent\today\\ \indent Supported by NSF grant  11431010 of China\\ \indent {\it Mathematics Subject
Classification (2000):} 14R15, 14E20, 13B10, 13B25, 17B63}

\begin{abstract}
Let $(F,G)\in{\mathbb C}[x,y]^2$ be a Jacobian pair and $\sigma:(a,b)\mapsto(F(a,b),G(a,b))$ for $(a,b)\in{\mathbb C}^2$  the corresponding
Keller map. The local bijectivity of Keller maps tells that for  $p\in{\mathbb C}^2$,
there exist neighborhoods ${\mathcal O}_p$ of $p$ and ${\mathcal O}_{\sigma(p)}$ of $\sigma(p)$
such that $\sigma_p=\sigma|_{{\mathcal O}_p}: {\mathcal O}_p\to{\mathcal O}_{\sigma(p)}$ is a bijection. Thus if there  exist $p_0,p_1\in{\mathbb C}^2$ with $p_0\ne p_1$, $\sigma(p_0)=\sigma(p_1)$,
then the local bijectivity implies that $\sigma_{p_1}^{-1}\sigma_{p_0}:{\mathcal O}_{p_0}\to {\mathcal O}_{p_1}$
is a bijection between some neighborhoods of $p_0$ and $p_1$. We generalize this result in various aspects, which lead us to give
a proof of injectivity of Keller maps and thus
 the $2$-dimensional Jacobian conjecture. Among those generalizations, one is the following (cf.~Theorem 1.5): For  any $(p_0,p_1)=\big((x_0,y_0),(x_1,y_1)\big)\in{\mathbb C}^2\times{\mathbb C}^2$
satisfying   $p_0\ne p_1$, $\sigma(p_0)=\sigma(p_1)$,
$\kappa_0\le|x_1|\le\kappa_1|x_0|^{\kappa_2}+\kappa_3\le\kappa_4|x_1|+\kappa_5$,
$\ell_{p_0,p_1}:=\frac{|y_1|}{|x_1|^{\kappa_6}}\ge\kappa_7$ for some preassigned $\kappa_i\in{\mathbb R}_{>0}$,
%$\theta_i\in{\mathbb R}_{\ne0}$ and $\lambda_i\in{\mathbb C}$,
there exists $(q_0,q_1)\in{\mathbb C}^2\times{\mathbb C}^2$ satisfying the same conditions, and furthermore
$\ell_{q_0,q_1}>\ell_{p_0,p_1}$.
%
%Finally we generalize the results to $n$-dimensional case.
%
%\keywords{Quantized walled Brauer algebras, decomposition numbers}
\end{abstract}
\sloppy \maketitle
\tableofcontents
%\newpage

\def\CiL#1{{\stackrel{\ssc\circ}{#1}}}\def\DoT#1{{\stackrel{\ssc\bullet}{#1}}}
\def\OO#1{{\mathcal O}_{#1}}\def\P{{\cal W}}\def\NP{{\rm NP}}


\section{Main theorem}\label{sect1}
Let us start with an arbitrary
{\it Jacobian pair} $(F,G)\in{\mathbb C}[x,y]^2$, i.e., a pair of polynomials on two variables $x,y$ with a nonzero constant {\it Jacobian determinant} \equa{JcDtT}{\mbox{$\dis
J(F,G):=\left|\begin{array}{cc}\dis\frac{\ptl F}{\ptl x}&\dis\frac{\ptl F}{\ptl y}\\[10pt]
\dis\frac{\ptl G}{\ptl x}&\dis\frac{\ptl G}{\ptl y}\end{array}\right|\in\C_{\ne0}$.}}
Assume that the corresponding {\it Keller map} $\sigma:\C^2\to\C^2$ sending, for $p=(a,b)\in{\mathbb C}^2$,\equa{KeMaPP}{\mbox{$p\mapsto (F(p),G(p)):=(F(a,b),G(a,b))$,}}
is not injective, namely,
for some $p_0=(x_0,y_0),\ p_1=(x_1,y_1)\in{\mathbb C}^2$,
\equa{=simag}{\mbox{$\sigma(p_0)=\sigma(p_1)$, \ \  $p_0\ne p_1$.}}
The local bijectivity
of  Keller maps says that for $p\in{\mathbb C}^2$, there exist neighborhoods $\OO p$ of $p$ and $\OO {\si(p)}$ of $\sigma(p)$ such that $\sigma_p=\sigma|_{\OO p}$ is a bijection between these two
neighborhoods. This implies that  $\sigma_{p_1}^{-1}\sigma_{p_0}:\OO {p_0}\to \OO {p_1}$ is a bijection between some neighborhoods $\OO{p_0}$ of $p_0$ and $\OO{p_1}$ of $p_1$ (we may assume $\OO{p_0}$ and $\OO{p_1}$ are disjoint), i.e., any $q_0\in\OO{p_0}$ is in $1$--$1$ correspondence with $q_1\in\OO{p_1}$ such that
{\mbox{$\si(q_0)=\si(q_1)$ and $q_0\ne q_1$.}}
In this paper we   generalize this result in various aspects, which lead us to present
a proof of injectivity of Keller maps, which implies the well-known Jacobian conjecture (see, e.g., the References%
%
%; we will generalize the following result to $n$-dimensional case in the last section
).
\begin{theo}\label{MAINT} {\bf(Main Theorem)} \
Let $(F,G)\in\C[x,y]^2$ be a {Jacobian pair}.
Then
the {Keller map} $\si$
 is injective. In particular, the $2$-dimensional Jacobian conjecture holds, i.e.,
$F,G$ are generators of $\C[x,y]$.
\end{theo}

First we give some formulations.
\NOUSE{%
Choose any $\a,\b\in\C$ such that $(^{\ \,\ \ \  \a}_{\frac14(x_1-x_0)}\ {}^{\,\  \ \ \ \b}_{\frac14(y_1-y_0)})$ is an invertible matrix. Applying the variable change \equa{Var11111}{(x,y)\mapsto(x,y,1)
\left(\begin{array}{cc}\a&\b\\ \dis \frac14(x_1-x_0)&\dis\frac14( y_1-y_0)\\ x_0 &y_0\end{array}\right),}
where the right-hand side is regarded as a vector-matrix multiplication, by \eqref{1=simag}, we can assume
\equa{=simag}{\mbox{$\sigma(\ffp_0)=\sigma(\ffp_1)$ with $\ffp_0=(0,0),\ \ffp_1=(0,4)$.}}
}%
Fix (once and for all)
a sufficiently large $\ell\in\R_{>0}$. %and let $\TH\in\C_{\ne0}$ (later on we may need choose different $\TH$'s, cf.~Theorem \ref{real00-inj}).
Applying the following
variable change%  [which does not affect \eqref{=simag}]
, %, where
%$\ell\in\Z_{>0}$ is some sufficiently large integer)
\equa{newvara}{(x,y)\mapsto\big(x+%(\TH x+y-4)
(\TH x+y)^\ell,\TH x+y\big),}
and rescaling $F,G$, we can assume
\equa{wheraraaa}{\Supp\, F\subset\D_{0,\xi,\eta},\ \ \ \  F_L=(\TH x+y)^m,\ \ \ \ J(F,G)=1,}
where \begin{itemize}\lineskip7pt
\item
$\Supp\,F:=\big\{(i,j)\in\Z_{\ge0}^2\,|\,\Coeff( F,x^iy^j)\ne0\big\}$ is the {\it  support} of $ F$ [cf.~Convention \ref{conv1}\,(2)\,(iv) for notation $\Coeff(F,x^iy^j){\sc\,}$],
\item
 $\D_{0,\xi,\eta}$ is the triangular with vertices $0=(0,0),\,\xi=(m,0),\,
\eta=(0,m)$ for some $m\in\Z_{>0}$,
\item $L$ is the edge of $\Supp\, F$ linking vertices $\xi,\eta$,
\item  $ F_L$, which we refer to as the {\it leading part} of $F$,  is the part of $ F$ corresponding to the edge $L$
 (which means that $\Supp\, F_L=L\cap\Supp\, F$).
\end{itemize}
The reason we take the variable change \eqref{newvara} is to use the leading part $ F_L$ of $ F$ to control $ F$ in some sense [cf.~\eqref{sim1aqa}${\ssc\,}$], which guides
us to obtain Theorem \ref{lemm-a1}.

Throughout the paper, we use the following notations,
\begin{eqnarray}
\label{V=1-}&\!\!\!\!\!\!\!\!\!\!\!\!\!\!\!\!\!\!&
(p_0,p_1)=\big((x_0,y_0),(x_1,y_1)\big)\in\C^2\times\C^2\cong\C^4,\\[4pt]
\label{V=0}&\!\!\!\!\!\!\!\!\!\!\!\!\!\!\!\!\!\!&
V=\big\{(p_0,p_1)=\big((x_0,y_0),(x_1,y_1)\big)\in\C^4\,\big|\,\si(p_0)=\si(p_1),
\,p_0\ne p_1\big\},\\[4pt]
\label{V=1}&\!\!\!\!\!\!\!\!\!\!\!\!\!\!\!\!\!\!&
V_{\xi_0,\xi_1}=\big\{(p_0,p_1)=\big((x_0,y_0),(x_1,y_1)\big)\in V\ \big|\ x_0=\xi_0,\,x_1=\xi_1\big\},
\end{eqnarray}
for any $\xi_0,\xi_1\in\C.$
Then $V\ne\emptyset$ by assumption \eqref{=simag}.
The main result used in the proof of Theorem \ref{MAINT} is the following.
\begin{theo}\label{real-inj-1}\begin{itemize}\item[\rm(i)]
There exist $\xi_0,\xi_1\in\C$ such that $V_{\xi_0,\xi_1}=\emptyset$.
\item[\rm(ii)]
Fix any $\xi_0,\xi_1\in\C$ satisfying {\rm(i)}. Denote, for $(p_0,p_1)=\big((x_0,y_0),(x_1,y_1)\big)\in V$,
\equa{d-1-2-3-4}{\dd_{p_0,p_1}=|x_0-\xi_0|^2+|x_1-\xi_1|^2.}
Then for any $(p_0,p_1)\in V$, there exists $(q_0,q_1)=\big((\dot x_0,\dot y_0),(\dot x_1,\dot y_1)\big)\in V$ such that\equa{GSoo}{\dis  \dd_{q_0,q_1}< \dd_{p_0,p_1}.}
\end{itemize}\end{theo}

After a proof of this result, it is then not surprising that it can be used  to give a proof of Theorem \ref{MAINT} by taking some kind of ``limit'' [cf.~\eqref{Acon111}${\ssc\,}$], which can  guide us to derive a contradiction.
We would like to mention that at a first sight, Theorem \ref{real-inj-1}\,(i) seems to be obvious, however its proof is highly nontrivial to us, it needs several results, which we state below. Here is the first one.
 \begin{theo}\label{lemm-a1} Denote,
for $(p_0,p_1)=\big((x_0,y_0),(x_1,y_1)\big)\in V$,
\equa{mqp1234}{h_{p_0,p_1}=\max\big\{|\TH x_1|,|y_1|,|\TH x_0|,|y_0|\big\},}
and call $h_{p_0,p_1}$ the {\bf height} of $(p_0,p_1)$.
%Fix any sufficiently small $\tau\in\R_{>0}$.
There exists %some sufficiently large
$\SS_0\in\R_{>0}$
$($depending on %$\TH$, $\tau$,
$m=\deg\,F,$ $\deg\,G$ and coefficients of $F$ and $G{\sc\,})$
satisfying the following: For any $(p_0,p_1)=\big((x_0,y_0),(x_1,y_1)\big)\in V$ with
\equa{mqp1234-1}{h_{p_0,p_1}\ge\SS_0,}
we must have
\equa{mqp1234-2}{|\TH x_0+y_0|<\tau h_{_{\sc p_0,p_1}}^{^{\sc\frac{m}{m+1}}},\ \ \ \ \ |\TH x_1+y_1|<\tau  h_{_{\sc p_0,p_1}}^{^{\sc\frac{m}{m+1}}}.}
In particular if $h_{p_0,p_1}=\max\big\{|\TH x_t|,|y_t|\big\}$ $($for some $t\in\{0,1\}{\ssc\,})$, then
\equa{mqp1234-2+}{|a-b|<\tau n_{_{\sc t}}^{^{\sc\frac{m+1}{m+2}}}\mbox{ for any $a,b\in\big\{|\TH x_t|,|y_t|,h_{p_0,p_1}\big\}$, }}
where $n_t=\min\big\{|\TH x_t|,|y_t|\big\}$.
\end{theo}

To prove  Theorem $\ref{real-inj-1}$\,(i), we assume conversely that
\equa{CSweaua}{V_{\xi_0,\xi_1}\ne\emptyset\mbox{ \ for all \ }\xi_0,\xi_1\in\C.}
Then we are able to obtain the following.
\begin{theo}\label{AddLeeme--0}
 Under the assumption \eqref{CSweaua}, we have the following.\begin{itemize}\item[\rm(i)]The following subset of $V$ is a nonempty closed bounded subset of $\C^4$ for any
 $k_0,k_1\in\R_{\ge0}$,
\begin{eqnarray}
\label{Ak=}&\!\!\!\!\!\!\!\!\!\!\!\!\!\!&
A_{k_0,k_1}=\big\{(p_0,p_1)=\big((x_0,y_0),(x_1,y_1)\big)\in V\ \big|\ |x_0|=k_0,\,|x_1|=k_1\big\}.
\end{eqnarray}
\item[\rm(ii)]
The following is a well-defined %continuous
function on $k_0,k_1\in\R_{\ge0}$,
\begin{eqnarray}
\label{Ak=1}&\!\!\!\!\!\!\!\!\!\!\!\!\!\!\!\!\!\!\!\!&
\g_{k_0,k_1}=\max \,\big\{|y_1|\ \big|\ (p_0,p_1)=\big((x_0,y_0),(x_1,y_1)\big)\in A_{k_0,k_1}\big\}.
\end{eqnarray}
\item[\rm(iii)]The
$\g_{k_0,k_1}$ is an ``almost strictly'' increasing function on both variables $k_0,k_1\in\R_{\ge0}$ in the following sense,
\begin{eqnarray}
\label{wePPPP1}
&&\!\!\!\!\!\!\!\!\!\!\!\!\!\!\!\!\!\!\!\!\!\!\!\!
\mbox{{\rm(a)\ }
$\g_{k'_0,k_1}>\g_{k_0,k_1}$ if  $k'_0>k_0\ge0,\,k_1\ge0$},\nonumber\\[4pt]
&&\!\!\!\!\!\!\!\!\!\!\!\!\!\!\!\!\!\!\!\!\!\!\!\!
\mbox{{\rm(b)\ }
$\g_{k_0,k'_1}>\g_{k_0,k_1}$ if  $k_0>0,\,k'_1>k_1\ge0$}
%,\nonumber\\[4pt]
%%\label{wePPPP2}
%&&\!\!\!\!\!\!\!\!\!\!\!\!\!\!\!\!\!\!\!\!\!\!\!\!
%\mbox{{\rm(c)}\
%$\g_{0,k'_1}\ge\g_{0,k_1}$ \ \ \,  if $k'_1> k_1$}
.\end{eqnarray}
\end{itemize}\end{theo}






This result is then used to prove the following.
\begin{theo}\label{real00-inj}%There exists suitable $\th\in\C_{\ne0}$ in \eqref{newvara} such that we have the following.
\begin{itemize}\item[\rm(1)]
There exist $\kappa_i\in\R_{>0}%,\,\l_i\in\C%,\th\in\Z_{\ge1}
$ %with $\kk$ being sufficiently large %\,\th_i\in\R_{\ne0}%,\,\l\!\in\!\C
 %with $\kappa_2\!<\!\kappa_4,\,\kappa_3<1$ and $|\l|\!<\!1$
%$\kappa_1,\kappa_3<1,\,\kappa_5>4$, $\kappa_2<\kappa_4$ and $\kappa_2$ being sufficiently large
such that the following hold.
\begin{itemize}
\item[\rm(i)]
Denote by $V_0$ the subset of $V$ such that %either
all its elements $(p_0,p_1)\!=\!\big((x_0,y_0),(x_1,y_1)\big)$ simultaneously satisfy
%The subset $V_0$ of $V$ containing the elements $(p_0,p_1)=\big((x_0,y_0),(x_1,y_1)\big)$ satisfying
one of %either
\eqref{ToSayas} or %else all its elements satisfy
\eqref{ToSayas0} $[$cf.~\eqref{LetNSoOP}${\ssc\,}]$%
%is a nonempty set:
. Then $V_0\ne\emptyset$.
%the following is a nonempty set $($the condition implies that $x_0,x_1,y_1\ne0$ and $x_0,x_1$ are bounded$)$:
\begin{eqnarray}
\label{ToSayas}
\!\!\!\!\!\!\!\!\!\!\!\!&\!\!\!\!\!\!\!\!\!\!\!\! &
{\rm(a)\ }\dis\kappa_0\le|x_1|\le\kappa_1|x_0|^{\kappa_2}+\kappa_3\le
\kappa_4|x_1|+\kappa_5,
 \ \ %\nonumber\\[6pt]
%\!\!\!\!\!\!\!\!\!\!\!\!&\!\!\!\!\!\!\!\!\!\!\!\!\!\!\!\!\!\!\!\!\!\!\!\!\!\!\!\!\!\!\!\!\!\!\!\!\!&
{\rm(b)\ }\ell_{p_0,p_1}:=\frac{|y_1|}{|x_1|^{\kappa_6}}\ge\kappa_7, % \mbox{ and}
%%\mbox{ in case }\eqref{ToSayas}, \mbox{ \ and \ $\kappa_2\ne1$ in case \eqref{ToSayas0}},
\\[7pt] \label{ToSayas0}
\!\!\!\!\!\!\!\!\!\!\!\!&\!\!\!\!\!\!\!\!\!\!\!\!\!\!\!\!\!\!\!\!\!\!\!\!\!\!\!\!\!\!\!\!\!\!\!\!\!&
{\rm(a)\,}\dis1\!\le\!\kappa_1|x_0^3x_1y_1|^{\kappa_2}
\!\le\!\kappa_3 |x_0x_1^3y_1^2|\!\le\!\kappa_4|x_0^3x_1y_1|^{\kappa_5}\!\le\!\kappa_6,
\nonumber\\[4pt]
%%\label{11ToSayas}
\!\!\!\!\!\!\!\!\!\!\!\!&\!\!\!\!\!\!\!\!\!\!\!\!\!\!\!\!\!\!\!\!\!\!\!\!\!\!\!\!&
{\rm(b)\,}\kappa_7\le|x_1|\le\kappa_8,\ \ \ \
{\rm(c)\, }\ell_{p_0,p_1}\!:=\!
|x_1|^2\Big(\frac{|x_1^3y_1|^{\kappa_9}}{|x_0|^{\kappa_{10}}}+\kappa_{11}\Big)
\!\ge\!\kappa_{12},
\end{eqnarray}
where
we require
\begin{eqnarray}
\label{-EiathA}
&&%\!\!\!\!\!\!\!\!\!\!\!\!\!\!\!\!\!\!\!\!\!\!\!\!\!\!\!\!\!\!\!
\kappa_{4}\!<\!1, \   \kappa_{3} \!<\!\min\{\kappa_0,\kappa_{5}\}
\mbox{ in case \eqref{ToSayas}}%, and }\kappa_2,\kappa_5\notin\Big\{\frac13,2\Big\}\mbox{ in case \eqref{ToSayas0}}
%\mbox{\ in case \eqref{ToSayas},  and }\kappa_5\!<\!\kappa_2\mbox{ in case \eqref{ToSayas0}}
,
%%\nonumber
%% \\[4pt]
%%&&\!\!\!\!\!\!\!\!\!\!\!\!\!\!\!\!\!\!\!\!\!\!\!\!\!
%% \ \ \dis\kappa_1>\kappa_3,\ \kappa_5<\kappa_7\kappa_0^{\kappa_6}
%%\mbox{  in case \eqref{ToSayas0}}.
\end{eqnarray}
and in any case,
%%in case \eqref{ToSayas0},
$\kappa_i%,\l_i%,\th
$ will be chosen such that there exists $\th_i\in\R_{>0}$ satisfying: when conditions hold, we have %$[$cf.~\eqref{LetNSoOP}$]$
\equa{-EiathA0}{\mbox{$\th_0\le|x_0|,
|x_1|%,%\big|\l_1x_1+\l_2y_1(x_0x_1)^{-1}\big|
\le\th_1$, \ and \ $%|x_1+\l_2|,
%|\l x_0-x_1^{\th}|,%|\l_1 x_1+\kappa_1|,|\l_2x_0x_1^{-1}+\kappa_3|,
|y_1|\ge\th_0.$ %\ \ and \ \ $|y_1+\l_2|\ge\th_0$ in case  \eqref{ToSayas0}.
}}
% %and in case \eqref{ToSayas0}, %$\kappa_5>1$ and
%%$\kappa_i,\l$ will be chosen such that when conditions \eqref{ToSayas0} hold, we have $[$cf.~\eqref{LetNSoOP}$]$
%%\equa{-EiathA1}{\mbox{$x_0^{\pm1},x_1^{\pm1},(x_0^{-1}+\l)^{\pm1},y_1^{\pm1}$ are well defined and bounded.}}
\item[\rm(ii)]For any $(p_0,p_1)\in V_0$, at most only one equality can hold in
{\rm\eqref{ToSayas}}\,{\rm(a)} or {\rm\eqref{ToSayas0}}\,{\rm(a)};
furthermore, %the last equality of {\rm\eqref{ToSayas0}}\,{\rm(a)} cannot occur.
none of the first or last equality %, the last equality
of
{\rm\eqref{ToSayas0}}\,{\rm(a)} or any equalities of {\rm\eqref{ToSayas0}}\,{\rm(b)}
%further none of the last three equalities of {\rm\eqref{ToSayas0}}\,{\rm(a)} %, the last equality cannot occur, the second inequality
%automatically holds, the first and third  equalities cannot simultaneously
can occur.
\end{itemize}
\item[\rm(2)]
For any $(p_0,p_1)\in V_0$, there exists $(q_0,q_1)=\big((\dot x_0,\dot y_0),(\dot x_1,\dot y_1)\big)\in V_0$
such that \equa{GSoo+}{\dis \ell_{q_0,q_1}> \ell_{p_0,p_1}.}
\end{itemize}\end{theo}

\begin{rema}\label{MARK1}\rm\begin{itemize}\item[(1)]We wish to mention that Theorem \ref{real00-inj} is the most important and difficult part of the paper. Throughout the paper we will frequently use the local bijectivity of Keller maps. Theorem \ref{real00-inj}\,(2) says that
[assume for example, we have case \eqref{ToSayas}${\ssc\,}$]
\begin{eqnarray}
\label{MUST115}
&&\!\!\!\!\!\!\!\!\!\!\!\!\!\!\!\!\!\!\!\!\!\!\!\!\!\!\!\!\!\!\!\!
{\rm(a)\  }
\kappa_0\le|\dot x_1|\le\kappa_1|\dot x_0|^{\kappa_2}+\kappa_3
\le\kappa_4|\dot x_1|+\kappa_5,
\ \ \  \ {\rm(b)\  }
\frac{|\dot y_1|}{|\dot x_1|^{\kappa_6}}>\frac{|y_1|}{|x_1|^{\kappa_6}},
\\[4pt]
%\label{MUST115-}
&&\!\!\!\!\!\!\!\!\!\!\!\!\!\!\!\!\!\!\!\!\!\!\!\!\!\!\!\!\!\!\!\!
%\ \ \ \ \ \ \ \
{\rm(c)\   }\si(q_0)=\si(q_1),
\ \ \ \  \, \ \ \ \ \ \ \ \ \ \ \ \ \ \ \ \, \ \ \ \ \ \ \ \ \ \ \ \ \ \ \ \ \ \,{\rm(d)\   }q_0\ne q_1.\end{eqnarray}
If we regard $\dot x_0,\dot y_0,\dot x_1,\dot y_1$ as 4 free variables, then  the local bijectivity always allows us to obtain
\eqref{MUST115}\,(c), which imposes two restrictions on 4 variables. We can impose at most two more ``nontrivial'' restrictions on them [we regard
\eqref{MUST115}\,(d) as a trivial restriction, see below].
The main difficulty for us is how to impose two more ``solvable'' restrictions on variables [see (3) below]
to control $\dot x_0,\dot y_0,\dot x_1,\dot y_1$ in order to achieve our goal  of ``deriving a contradiction''. However, it seems to us that two more restrictions are always
insufficient  to achieve the goal. Here,
condition \eqref{MUST115}\,(b)
imposes one more restriction, and we have one free variable left.
However there are 3 restrictions in
\eqref{MUST115}\,(a), thus in general there will be no solutions. Thanks to Theorem \ref{real00-inj}\,(1)\,(ii) [see (2) below], we only need to take care of one restriction in
\eqref{MUST115}\,(a)
each time [cf.~\eqref{@such111that=2}%--\eqref{saysome+}
${\ssc\,}$]
since we are always under a ``local'' situation (i.e., we are only concerned with a small neighborhood of some points each time), and thus the inequation in
\eqref{MUST115}\,(a) is solvable [we do not need to consider condition
\eqref{MUST115}\,(d) under the ``local'' situation, we only need to take care of it when we take some kind of ``limit'', cf.~\eqref{Acon111}${\ssc\,}$].
\item[(2)]Condition  \eqref{MUST115}\,(b)
is not only used to control $|\dot x_1|$ and $|\dot y_1|$,
but also used to take the ``limit''; while  \eqref{MUST115}\,(a)
is used to control $|\dot x_0|$ and $|\dot x_1|$.
%,
We remark that the requirement ``${\ssc\,}\kappa_{3}<\kappa_{5}{\ssc\,}$''
in
\eqref{ToSayas}\,(a) or respectively
 $\kappa_{11}>0$ in \eqref{ToSayas0}\,(c)
will guarantee  %[cf.~\eqref{kakak}${\ssc\,}$] that $\kk^{\kk}|x_0|-|y_1|>0$, which
%is very crucial in order  to ensure that %\eqref{MUST115}\,(b)
the correspondent inequation
\begin{eqnarray}\label{SoLLL}
&\!\!\!\!\!\!\!\!\!\!\!\!\!\!\!\!\!\!\!\!&
%{\rm (i)\ }
\kappa_{1} |\dot x_0|^{\kappa_2}+\kappa_3\le\kappa_{4}|\dot x_1|+\kappa_{5},\mbox{ \ or  respectively,}
\nonumber\\[4pt]
&\!\!\!\!\!\!\!\!\!\!\!\!\!\!\!\!\!\!\!\!&
%\ \ %{\rm (ii)\ }
|\dot x_1|^2\Big(\frac{|\dot x_1^3\dot y_1|^{\kappa_9}}{|\dot x_0|^{\kappa_{10}}}+\kappa_{11}\Big)
>
|x_1|^2\Big(\frac{|x_1^3y_1|^{\kappa_9}}{|x_0|^{\kappa_{10}}}+\kappa_{11}\Big)
,
\end{eqnarray}
%
%$\big[$resp., $|\dot x_1|\big(\frac{|\dot y_1|}{|\dot x_0|^{\kappa_5}}+\kappa_6\big)>
%|x_1|\big(\frac{| y_1|}{| x_0|^{\kappa_5}}+\kappa_6\big){\sc\,}\big]$
is solvable [see (3) below, Remark \ref{FinalRem} and \eqref{C12BsM}$\ssc\,$].
% and the last paragraph after \eqref{2MEMEME}${\ssc\,}$].
%The rest  requirements in
%\eqref{-EiathA} are to ensure that $\dot x_0,\,\dot x_1$ are bounded and
%nonzero% [thus all terms in \eqref{MUST115} are well defined]
%\item[(3)]
%Further
%we use $|\l x_0-\kappa_3|$ instead of  $|\l|\cdot|x_0|-\kappa_3$ in %the term after the second inequality of
%\eqref{ToSayas0}\,(b) is also to guarantee that the corresponding inequation
%\eqref{SoLLL}\,(ii) is solvable.
%there is a negative term in the numerator of \eqref{ToSayas0}\,(b), thus the
%corresponding inequation \eqref{GSoo+}
%to the second inequality of \eqref{LetNSoOP}\,(i) will have two negative terms
%[see \eqref{1MEMEME}\,(i){$\sc\,$}] and
%is in general %such an inequation is
%unsolvable, however we have designed \eqref{ToSayas0} [cf.~\eqref{LetNSoOP}] so that the inequation is solvable [cf.~Remark \ref{remaFFFF} and \eqref{1MEMEME}].
%
%\eqref{ToSayas0}\,(a) is in general unsolvable since there will be two negative terms in the corresponding inequation
%[see~\eqref{1MEMEME}${\ssc\,}$], however we have designed \eqref{ToSayas0} so that the inequation is solvable.
Finally we would like to mention that to find  conditions like the ones in \eqref{ToSayas} or \eqref{ToSayas0}
satisfying Theorem \ref{real00-inj}\,(1)\,(ii)  has been extremely difficult for us, we achieve this by using Theorem \ref{AddLeeme--0} to prove
%two
several %three
technical lemmas (cf.~Assumption \ref{assu1111} and
Lemmas \ref{G--lemm-assum1--}--\ref{TheoHold}).
\item[(3)]
One may   expect to have some statements such as one of the following:
\begin{itemize}\item[(i)]
For any $(p_0,p_1)\in V$ with $|x_0|^{\th_0}+|x_1|^{\th_1}\le\SS$ (for some $\th_i,\SS\in\R_{>0}$), there exists $(q_0,q_1)\in V$ such that
\equa{Assumq1}{{\rm(a)\ }|\dot x_0|^{\th_0}+|\dot x_1|^{\th_1}\le\SS,\ \ \ \ \ \ \ \ \ \
{\rm(b)\ }|\dot y_1|^{\th_2}>|y_1|^{\th_2}.}
\item[(ii)] For any $(p_0,p_1)\in V$ with $|x_1|^{\th_1}\le|y_1|^{\th_2}+\SS$,
%[where $\th,b\in\Q_{>0}$ are fixed numbers with $\th<1$; we need to add a
%positive power $\th<1$ and a positive $b$ as will be mentioned in \eqref{saysome+}
%and Remark \ref{FinalRem} so that the first inequation below is solvable],
there exists
 $(q_0,q_1)\in V$ such that
\equa{Assumq1+}{{\rm(a)\ }|\dot x_1|^{\th_1}\le|\dot y_1|^{\th_2}+\SS,\ \ \ \ \ \ {\rm(b)\ }|\dot y_0|^{\th_3}-|\dot x_0|^{\th_4}>|y_0|^{\th_3}-|x_0|^{\th_4}.}
\end{itemize}
If a statement such as one of the above  could be obtained, then  a proof of Theorem \ref{MAINT} would be easier. At a first sight, the condition \eqref{Assumq1}\,(a) [or  \eqref{GSoo}${\ssc\,}$] only imposes one restriction on variables,
however it in fact contains 2 hidden restrictions [see arguments after \eqref{2DDDbar1}${\ssc\,}$] simply because the left-hand side of ``\,$<$\,'' has 2 positive terms with absolute values containing
variables% [if we replace ``\,$<$\,'' by ``\,$>$\,'', then it is indeed only one restriction; we remark that more {\bf positive} terms with absolute values containing variables at the left-hand side of ``\,$<$\,'' (or the right-hand side of ``\,$>$\,''), there are more restrictions on variables]
.
The second condition in \eqref{Assumq1+} is unsolvable as will be explained in Remark \ref{FinalRem}.
We would also like to point out that to obtain Theorem \ref{MAINT}, we always need to take some kind of ``limit'' [cf.~\eqref{Acon111}${\ssc\,}$] to derive a contradiction. Thus some condition such as  \eqref{GSoo},
\eqref{MUST115}\,(b),  \eqref{Assumq1}\,(b) or \eqref{Assumq1+}\,(b)  is necessary in order to take the ``limit''.
\end{itemize}\end{rema}















\NOUSE{%
This result is then used to prove the following.
\begin{theo}\label{real00-inj}%There exists suitable $\th\in\C_{\ne0}$ in \eqref{newvara} such that we have the following.
\begin{itemize}\item[\rm(1)]
There exist $\kappa_i\in\R_{>0},\,\l_i\in\C%,\th\in\Z_{\ge1}
$ %with $\kk$ being sufficiently large %\,\th_i\in\R_{\ne0}%,\,\l\!\in\!\C
 %with $\kappa_2\!<\!\kappa_4,\,\kappa_3<1$ and $|\l|\!<\!1$
%$\kappa_1,\kappa_3<1,\,\kappa_5>4$, $\kappa_2<\kappa_4$ and $\kappa_2$ being sufficiently large
such that the following hold.
\begin{itemize}
\item[\rm(i)]
Denote by $V_0$ the subset of $V$ such that %either
all its elements $(p_0,p_1)\!=\!\big((x_0,y_0),(x_1,y_1)\big)$ simultaneously satisfy
%The subset $V_0$ of $V$ containing the elements $(p_0,p_1)=\big((x_0,y_0),(x_1,y_1)\big)$ satisfying
one of %either
\eqref{ToSayas} or %else all its elements satisfy
\eqref{ToSayas0} $[$cf.~\eqref{LetNSoOP}${\ssc\,}]$%
%is a nonempty set:
. Then $V_0\ne\emptyset$.
%the following is a nonempty set $($the condition implies that $x_0,x_1,y_1\ne0$ and $x_0,x_1$ are bounded$)$:
\begin{eqnarray}
\label{ToSayas}
\!\!\!\!\!\!\!\!\!\!\!\!&\!\!\!\!\!\!\!\!\!\!\!\! &
{\rm(a)\ }\dis\kappa_0\le|x_1|\le\kappa_1|x_0|^{\kappa_2}+\kappa_3\le
\kappa_4|x_1|+\kappa_5,
 \ \ %\nonumber\\[6pt]
%\!\!\!\!\!\!\!\!\!\!\!\!&\!\!\!\!\!\!\!\!\!\!\!\!\!\!\!\!\!\!\!\!\!\!\!\!\!\!\!\!\!\!\!\!\!\!\!\!\!&
{\rm(b)\ }\ell_{p_0,p_1}:=\frac{|y_1|}{|x_1|^{\kappa_6}}\ge\kappa_7, % \mbox{ and}
%%\mbox{ in case }\eqref{ToSayas}, \mbox{ \ and \ $\kappa_2\ne1$ in case \eqref{ToSayas0}},
\\[7pt] \label{ToSayas0}
\!\!\!\!\!\!\!\!\!\!\!\!&\!\!\!\!\!\!\!\!\!\!\!\!\!\!\!\!\!\!\!\!\!\!\!\!\!\!\!\!\!\!\!\!\!\!\!\!\!&
{\rm(a)\,}\dis1\!\le\!|\l_1x_0\!-\!\l_2x_1|^{\kappa_1}
\!\le\! |\l_3x_0\!+\!\l_4x_1|\!\le\!|\l_1x_0\!-\!\l_2x_1|^{\kappa_2}\!\le\!\kappa_3,
 \ %\nonumber\\[4pt]
%%\label{11ToSayas}
%\!\!\!\!\!\!\!\!\!\!\!\!&\!\!\!\!\!\!\!\!\!\!\!\!\!\!\!\!\!\!\!\!\!\!\!\!\!\!\!\!&
%{\rm(b)\,}|x_0|\!\le\!\kappa_4,\ \
{\rm(b)\, }\ell_{p_0,p_1}\!:=\!
\frac{|y_1|\!+\!\kappa_4}{|x_1|^2}
\!\ge\!\kappa_{5},
\end{eqnarray}
where in case \eqref{ToSayas}
we require
\begin{eqnarray}
\label{-EiathA}
&&%\!\!\!\!\!\!\!\!\!\!\!\!\!\!\!\!\!\!\!\!\!\!\!\!\!\!\!\!\!\!\!
\kappa_{4}\!<\!1, \   \kappa_{3} \!<\!\min\{\kappa_0,\kappa_{5}\}
%\mbox{\ in case \eqref{ToSayas},  and }\kappa_5\!<\!\kappa_2\mbox{ in case \eqref{ToSayas0}}
,
%%\nonumber
%% \\[4pt]
%%&&\!\!\!\!\!\!\!\!\!\!\!\!\!\!\!\!\!\!\!\!\!\!\!\!\!
%% \ \ \dis\kappa_1>\kappa_3,\ \kappa_5<\kappa_7\kappa_0^{\kappa_6}
%%\mbox{  in case \eqref{ToSayas0}}.
\end{eqnarray}
and in any case,
%%in case \eqref{ToSayas0},
$\kappa_i,\l_i%,\th
$ will be chosen such that there exists $\th_i\in\R_{>0}$ satisfying: when conditions hold, we have %$[$cf.~\eqref{LetNSoOP}$]$
\equa{-EiathA0}{\mbox{$\th_0\le|x_0|,
|x_1|%,%\big|\l_1x_1+\l_2y_1(x_0x_1)^{-1}\big|
\le\th_1$, \ and \ $%|x_1+\l_2|,
%|\l x_0-x_1^{\th}|,%|\l_1 x_1+\kappa_1|,|\l_2x_0x_1^{-1}+\kappa_3|,
|y_1|\ge\th_0.$ %\ \ and \ \ $|y_1+\l_2|\ge\th_0$ in case  \eqref{ToSayas0}.
}}
% %and in case \eqref{ToSayas0}, %$\kappa_5>1$ and
%%$\kappa_i,\l$ will be chosen such that when conditions \eqref{ToSayas0} hold, we have $[$cf.~\eqref{LetNSoOP}$]$
%%\equa{-EiathA1}{\mbox{$x_0^{\pm1},x_1^{\pm1},(x_0^{-1}+\l)^{\pm1},y_1^{\pm1}$ are well defined and bounded.}}
\item[\rm(ii)]For any $(p_0,p_1)\in V_0$, at most only one equality can hold in
{\rm\eqref{ToSayas}}\,{\rm(a)} or {\rm\eqref{ToSayas0}}\,{\rm(a)};
furthermore, %the last equality of {\rm\eqref{ToSayas0}}\,{\rm(a)} cannot occur.
none of the first or last equality %, the last equality
of
{\rm\eqref{ToSayas0}}\,{\rm(a)} %or equality in {\rm\eqref{ToSayas0}}\,{\rm(b)}
%further none of the last three equalities of {\rm\eqref{ToSayas0}}\,{\rm(a)} %, the last equality cannot occur, the second inequality
%automatically holds, the first and third  equalities cannot simultaneously
can occur.
\end{itemize}
\item[\rm(2)]
For any $(p_0,p_1)\in V_0$, there exists $(q_0,q_1)=\big((\dot x_0,\dot y_0),(\dot x_1,\dot y_1)\big)\in V_0$
such that \equa{GSoo+}{\dis \ell_{q_0,q_1}> \ell_{p_0,p_1}.}
\end{itemize}\end{theo}
%
\begin{rema}\label{MARK1}\rm\begin{itemize}\item[(1)]We wish to mention that Theorem \ref{real00-inj} is the most important and difficult part of the paper. Throughout the paper we will frequently use the local bijectivity of Keller maps. Theorem \ref{real00-inj}\,(2) says that
[assume for example, we have case \eqref{ToSayas}${\ssc\,}$]
\begin{eqnarray}
\label{MUST115}
&&\!\!\!\!\!\!\!\!\!\!\!\!\!\!\!\!\!\!\!\!\!\!\!\!\!\!\!\!\!\!\!\!
{\rm(a)\  }
\kappa_0\le|\dot x_1|\le\kappa_1|\dot x_0|^{\kappa_2}+\kappa_3
\le\kappa_4|\dot x_1|+\kappa_5,
\ \ \  \ {\rm(b)\  }
\frac{|\dot y_1|}{|\dot x_1|^{\kappa_6}}>\frac{|y_1|}{|x_1|^{\kappa_6}},
\\[4pt]
%\label{MUST115-}
&&\!\!\!\!\!\!\!\!\!\!\!\!\!\!\!\!\!\!\!\!\!\!\!\!\!\!\!\!\!\!\!\!
%\ \ \ \ \ \ \ \
{\rm(c)\   }\si(q_0)=\si(q_1),
\ \ \ \  \, \ \ \ \ \ \ \ \ \ \ \ \ \ \ \ \, \ \ \ \ \ \ \ \ \ \ \ \ \ \ \ \ \ \,{\rm(d)\   }q_0\ne q_1.\end{eqnarray}
If we regard $\dot x_0,\dot y_0,\dot x_1,\dot y_1$ as 4 free variables, then  the local bijectivity always allows us to obtain
\eqref{MUST115}\,(c), which imposes two restrictions on 4 variables. We can impose at most two more ``nontrivial'' restrictions on them [we regard
\eqref{MUST115}\,(d) as a trivial restriction, see below].
The main difficulty for us is how to impose two more ``solvable'' restrictions on variables [see (3) below]
to control $\dot x_0,\dot y_0,\dot x_1,\dot y_1$ in order to achieve our goal  of ``deriving a contradiction''. However, it seems to us that two more restrictions are always
insufficient  to achieve the goal. Here,
condition \eqref{MUST115}\,(b)
imposes one more restriction, and we have one free variable left.
However there are 3 restrictions in
\eqref{MUST115}\,(a), thus in general there will be no solutions. Thanks to Theorem \ref{real00-inj}\,(1)\,(ii) [see (2) below], we only need to take care of one restriction in
\eqref{MUST115}\,(a)
each time [cf.~\eqref{@such111that=2}%--\eqref{saysome+}
${\ssc\,}$]
since we are always under a ``local'' situation (i.e., we are only concerned with a small neighborhood of some points each time), and thus the inequation in
\eqref{MUST115}\,(a) is solvable [we do not need to consider condition
\eqref{MUST115}\,(d) under the ``local'' situation, we only need to take care of it when we take some kind of ``limit'', cf.~\eqref{Acon111}${\ssc\,}$].
\item[(2)]Condition  \eqref{MUST115}\,(b)
is not only used to control $|\dot x_1|$ and $|\dot y_1|$,
but also used to take the ``limit''; while  \eqref{MUST115}\,(a)
is used to control $|\dot x_0|$ and $|\dot x_1|$.
%,
We remark that the requirement ``${\ssc\,}\kappa_{3}<\kappa_{5}{\ssc\,}$''
in
\eqref{ToSayas}\,(a) or respectively
 $\kappa_{4}>0$ in \eqref{ToSayas0}\,(c)
will guarantee  %[cf.~\eqref{kakak}${\ssc\,}$] that $\kk^{\kk}|x_0|-|y_1|>0$, which
%is very crucial in order  to ensure that %\eqref{MUST115}\,(b)
the correspondent inequation
\begin{eqnarray}\label{SoLLL}
&\!\!\!\!\!\!\!\!\!\!\!\!\!\!\!\!\!\!\!\!&
%{\rm (i)\ }
\kappa_{1} |\dot x_0|^{\kappa_2}+\kappa_3\le\kappa_{4}|\dot x_1|+\kappa_{5},\mbox{ \ or  respectively,}
%\nonumber\\[4pt]
%&\!\!\!\!\!\!\!\!\!\!\!\!\!\!\!\!\!\!\!\!&
\ \ %{\rm (ii)\ }
\frac{|\dot y_1|+\kappa_4}{|\dot x_1|^2}
>
\frac{|y_1|+\kappa_4}{|x_1|^2}
,
\end{eqnarray}
%
%$\big[$resp., $|\dot x_1|\big(\frac{|\dot y_1|}{|\dot x_0|^{\kappa_5}}+\kappa_6\big)>
%|x_1|\big(\frac{| y_1|}{| x_0|^{\kappa_5}}+\kappa_6\big){\sc\,}\big]$
is solvable [see (3) below, Remark \ref{FinalRem} and \eqref{C12BsM}$\ssc\,$].
% and the last paragraph after \eqref{2MEMEME}${\ssc\,}$].
%The rest  requirements in
%\eqref{-EiathA} are to ensure that $\dot x_0,\,\dot x_1$ are bounded and
%nonzero% [thus all terms in \eqref{MUST115} are well defined]
%\item[(3)]
%Further
%we use $|\l x_0-\kappa_3|$ instead of  $|\l|\cdot|x_0|-\kappa_3$ in %the term after the second inequality of
%\eqref{ToSayas0}\,(b) is also to guarantee that the corresponding inequation
%\eqref{SoLLL}\,(ii) is solvable.
%there is a negative term in the numerator of \eqref{ToSayas0}\,(b), thus the
%corresponding inequation \eqref{GSoo+}
%to the second inequality of \eqref{LetNSoOP}\,(i) will have two negative terms
%[see \eqref{1MEMEME}\,(i){$\sc\,$}] and
%is in general %such an inequation is
%unsolvable, however we have designed \eqref{ToSayas0} [cf.~\eqref{LetNSoOP}] so that the inequation is solvable [cf.~Remark \ref{remaFFFF} and \eqref{1MEMEME}].
%
%\eqref{ToSayas0}\,(a) is in general unsolvable since there will be two negative terms in the corresponding inequation
%[see~\eqref{1MEMEME}${\ssc\,}$], however we have designed \eqref{ToSayas0} so that the inequation is solvable.
Finally we would like to mention that to find  conditions like the ones in \eqref{ToSayas} or \eqref{ToSayas0}
satisfying Theorem \ref{real00-inj}\,(1)\,(ii)  has been extremely difficult for us, we achieve this by using Theorem \ref{AddLeeme--0} to prove
%two
several %three
technical lemmas (cf.~Assumption \ref{assu1111} and
Lemmas \ref{G--lemm-assum1--}--\ref{TheoHold}).
\item[(3)]
One may   expect to have some statements such as one of the following:
\begin{itemize}\item[(i)]
For any $(p_0,p_1)\in V$ with $|x_0|^{\th_0}+|x_1|^{\th_1}\le\SS$ (for some $\th_i,\SS\in\R_{>0}$), there exists $(q_0,q_1)\in V$ such that
\equa{Assumq1}{{\rm(a)\ }|\dot x_0|^{\th_0}+|\dot x_1|^{\th_1}\le\SS,\ \ \ \ \ \ \ \ \ \
{\rm(b)\ }|\dot y_1|^{\th_2}>|y_1|^{\th_2}.}
\item[(ii)] For any $(p_0,p_1)\in V$ with $|x_1|^{\th_1}\le|y_1|^{\th_2}+\SS$,
%[where $\th,b\in\Q_{>0}$ are fixed numbers with $\th<1$; we need to add a
%positive power $\th<1$ and a positive $b$ as will be mentioned in \eqref{saysome+}
%and Remark \ref{FinalRem} so that the first inequation below is solvable],
there exists
 $(q_0,q_1)\in V$ such that
\equa{Assumq1+}{{\rm(a)\ }|\dot x_1|^{\th_1}\le|\dot y_1|^{\th_2}+\SS,\ \ \ \ \ \ {\rm(b)\ }|\dot y_0|^{\th_3}-|\dot x_0|^{\th_4}>|y_0|^{\th_3}-|x_0|^{\th_4}.}
\end{itemize}
If a statement such as one of the above  could be obtained, then  a proof of Theorem \ref{MAINT} would be easier. At a first sight, the condition \eqref{Assumq1}\,(a) [or  \eqref{GSoo}${\ssc\,}$] only imposes one restriction on variables,
however it in fact contains 2 hidden restrictions [see arguments after \eqref{2DDDbar1}${\ssc\,}$] simply because the left-hand side of ``\,$<$\,'' has 2 positive terms with absolute values containing
variables% [if we replace ``\,$<$\,'' by ``\,$>$\,'', then it is indeed only one restriction; we remark that more {\bf positive} terms with absolute values containing variables at the left-hand side of ``\,$<$\,'' (or the right-hand side of ``\,$>$\,''), there are more restrictions on variables]
.
The second condition in \eqref{Assumq1+} is unsolvable as will be explained in Remark \ref{FinalRem}.
We would also like to point out that to obtain Theorem \ref{MAINT}, we always need to take some kind of ``limit'' [cf.~\eqref{Acon111}${\ssc\,}$] to derive a contradiction. Thus some condition such as  \eqref{GSoo},
\eqref{MUST115}\,(b),  \eqref{Assumq1}\,(b) or \eqref{Assumq1+}\,(b)  is necessary in order to take the ``limit''.
\end{itemize}\end{rema}
}


\section{Some preparations}

We need some conventions and notations, which, for easy reference, are listed as follows.
\begin{conv}\label{conv1}\rm\begin{enumerate}\item[(1)]A complex number is written as $a=a\re+a\im\ii$ for some
$a\re,a\im\in\R$, where $\ii=\sqrt{-1}$.
If $a^b$ appears in an expression, then we assume $b\in\R$, and in case $a\ne0$, we
interpret $a^b$ as the unique complex number $r^be^{b\th\ii}$ by writing
$a=re^{\th\ii}$ for some $r\in\R_{>0}$, $-\pi<\th\le\pi$ and $e$ is the natural number.
\item[\rm(2)]
Let $P=\sum_{i\in\Z_{\ge0}} p_iy^{\a+i}\in\PP
$ with $\a\in\Z,\,
p_i\in\C(x)$.\begin{itemize}\item[(i)]Assume $p_0=1$. For any $\b\in\Q$  with $\a\b\in\Z$,
we always interpret $P^\b$ as \equa{P-as}{P^{\b}=
y^{\a\b}\Big(1+\sum\limits_{j=1}^\infty{\dis\binom{\b}{j}}\Big(\sum\limits_{i\in
\Z_{>0}}
p_iy^{i}\Big)^j \Big) \in\PP,}
where in general, $\binom{k}{\l_1,\l_2,...,\l_i}$ is the  {\it multi-nomial coefficient}
$\frac{k(k-1)\cdots(k-(\l_1+\l_2+\cdots+\l_i)+1)}{\l_1!\l_2!\cdots\l_i!}$.
Then \equa{Holsss}{\mbox{$(P^\b)^{\b'}=P^{\b\b'}$ for any $\b,\b'\in\Q$ with $\a\b,\a\b\b'\in\Z$.}} If $p_0\ne1$, then $p_0^\b$ is in general a multi-valued function, and if we fix a choice of $p_0^\b$, then \eqref{Holsss} only holds when $\b'\in\Z$ [fortunately we will only encounter this situation, cf.~\eqref{1-928384} and statements after it].

\item[(ii)]
For $Q_1,Q_2\in\PP$, we use the following notation [as long as it is algebraically a well-defined element in $\PP {\sc\,}$]
\equa{P(qq)}{P(Q_1,Q_2)=P|_{(x,y)=(Q_1,Q_2)}=\sum\limits_{i}p_{i}(Q_1)Q_2^{\a+i}.}
\item[(iii)]If $Q_1,Q_2\in\C$ with $Q_2\ne0$, we also use \eqref{P(qq)} to denote a well-defined complex number as long as
$p_{i}(Q_1)$ exists for all posible $i$ and
the series \eqref{P(qq)} \CA$\!{\sc\!}.$
\item[(iv)]
For $Q=\sum_{i\in
\Z_{\ge0}}q_iy^{\b+i}\in\PP$, by comparing coefficients of $y^{\b+i}$ for $i\ge0$, there exists uniquely $b_i\in\C(x)$ such that
\equa{usia}{Q=\mbox{$\sum\limits_{i=0}^\infty$}b_iP^{\frac{\b+i}{\a}}.}
We call $b_i$ the {\it coefficient of $P^{\frac{\b+i}{\a}}$ in $Q$,} and denote by $\Coeff(Q,P^{\frac{\b+i}{\a}})$. If $Q=\sum_{i,j}q_{ij}x^iy^j$ with $q_{ij}\in\C$, we also denote $\Coeff(Q,x^iy^j)=q_{ij}$.
\end{itemize}
\item[(3)] Throughout the paper, we need  two independent parameters $\kk\gg1$ (i.e., $\kk\to\infty$) and $\ep\to0$% (for instance, we take
%$k\in\R_{>0}$ such that $k\gg1$, and set $\ep=k^{-1}\to0$)
. We use the following convention:
 Symbols $\SS,\SS_j$ for $j\ge0$ always  denote some
(sufficiently large) numbers independent of
$\ep,\kk$. We use $O(\ep^i)$ for  $i\in\Q_{\ge0}$ to denote any element
$P$ in $\PP$ (or especially in $\C$)
such that
$P(\dot x,\dot y)$ \CA and $|\ep^{-i}P(\dot x,\dot y)|<\SS$ for some fixed
$\SS$, where  $(\dot x,\dot y)$ is in some required region which will be specified in the context.

\end{enumerate}
\end{conv}

Let $P=\sum_{j}p_{j}y^{j}\in\PP,\,p_j\in\C(x) $, and $(x_0,y_0)\in\C^2$ with $y_0\ne0$.
If  $p_j(x_0)$ exists for all possible $j$,
and $%\equa{cong}{
z_0=\sum%\limits
_{j} |p_{j}(x_0)y_0^{j}|$ %,}
converges,
then $z_0$ is called the {\it absolute converging value}
of $P$ at  $(x_0,y_0)$, denoted by $A_{(x_0,y_0)}(P)$ (or  by $A_{(y_0)}(P)$ if $P$ does not depend on $x$).
\begin{defi}\label{contro}\rm
\begin{itemize}\item[(1)]
Let $P$ be as above and $Q=\sum_iq_iy^i\in\C((y)),\,q_i\in\R_{\ge0}$, $x_0\in\C$. If $p_i(x_0)$ exists and \equa{COmmm}{\mbox{$|p_i(x_0)|\le q_i$
for all possible
$i$,}}
then we say $Q$ is a {\it controlling function} for $P$ on
%(or with respect to)
$y$ at point $x_0$, and denote
 \equa{MDEooooo}{\mbox{$P\,\DS^{x_0}_y\, Q$  \ or \  $Q\,\rDS^{x_0}_y\, P$,}}
or  $P\,\DS_y\, Q$   or   $Q\,\rDS_y\, P$ when there is no confusion.
In particular if $P,Q$ do not depend on $y$ then we write $P\DS^{x_0} Q$ or $Q\rDS^{x_0} P$ (thus $a\DS b$ for $a,b\in\C$ simply means $|a|\le b$).
\item[(2)]An element in $\C((y))$ with non-negative coefficients (such as $Q$ above) is called a {\it controlling function} on $y$.
\item[(3)]
If $Q=q_0y^\a+\sum_{j>0} q_{j}y^{\a+j}\in\C((y))$ is a controlling function on
 $y$ with $q_i\in\R_{\ge0}$ and $q_0>0$,  then we always use the same symbol with
subscripts ``\,$_\igo$\,'' and
``\,$_\inv$\,'' to denote the  elements
\equa{P---}{Q_\igo=q_0^{-1}\sum\limits_{j>0} q_{j}y^{j}, \ \ \
Q_\inv=q_0y^\a\Big(1-q_0^{-1}\sum\limits_{j>0} q_{j}y^{j}\Big)=q_0y^\a(1-Q_\igo).}
We call $Q_\igo$ the {\it ignored  part} of $Q$, and $Q_\inv$ the {\it negative correspondence of} $Q$
[in sense of \eqref{MAMS1} and \eqref{MAMS1+1}, where $a,-k$ are nonpositive].
\end{itemize}\end{defi}

\begin{lemm}\label{ds-lemm}\begin{itemize}\item[\rm(1)]
If \equa{P-Q===}{\mbox{$P=p_0y^\a+\sum\limits_{j>0} p_{j}y^{\a+j}\in\PP,\ \ \
Q=q_0y^\a+\sum\limits_{j>0} q_{j}y^{\a+j}\in\C((y%^{\frac1\NN}
)),$}} with $P\DS^{x_0}_y Q$, $x_0\in\C$ and $|p_0(x_0)|=q_0\in\R_{>0}$,
then
\begin{eqnarray}\label{MAMS1}\!\!\!\!\!\!\!\!&\!\!\!\!\!\!\!\!\!&{\rm(a)\ }\dis\frac{\ptl P}{\ptl y}\ \DS^{x_0}_y\ \,  \pm{\sc\,}
\frac{d Q}{d y},\ \ \ \ \ \ \ {\rm(b)\ }P^a\ \DS^{x_0}_y\ \, Q_\inv^a\ \DS_y\ \, (q_0y^{\a})^{-b}Q_\inv^{a+b}\mbox{ for $a,b\in\Q_-$,}
\\[7pt]
\label{MAMS1+1}\!\!\!\!\!\!\!\!&\!\!\!\!\!\!\!\!\!&
Q^k\ \DS_y\ \, (q_0y^{\a})^{2k}Q_\inv^{-k}\ \DS_y\ \,\left\{\begin{array}{ll}\dis
\frac{(q_0y^{\a})^k}{1-k
Q_\igo}&\mbox{if }k\in\Z_{\ge1},\\[12pt]\dis
(q_0y^{\a})^k\Big(1+\dis \frac{kQ_\igo}{1-Q_\igo}\Big)&\mbox{if }k\in\Q_{\ge0}\mbox{ with }k<1.
\end{array}\right.
\end{eqnarray}
where  $\eqref{MAMS1}\,(a)$ holds under the condition: either both $P$ and $Q$ are power series of
$y$ $($in this case the sign is ``\,$+$\,''$)$, or else both are polynomials on $y^{-1}$ $($in this case the sign is ``\,$-$\,''$)$.
\item[\rm(2)]If $x_0,y_0\in\C$ %, $y_1\in\R_{>0}$
with $y_0\ne0$,  and $P_1\DS^{x_0}_y\, Q_1,\ P_2\DS^{x_0}_y\, Q_2$, then
\equa{ABV1}{A_{(x_0,y_0)}(P_1P_2)\le A_{(y_0)}(Q_1)A_{(y_0)}(Q_2)=Q_1(|y_0|)Q_2(|y_0|).}
\end{itemize}
\end{lemm}{\it Proof.} (2)
and \eqref{MAMS1}\,(a) are obvious, \eqref{MAMS1}\,(b) and \eqref{MAMS1+1}
are obtained by
 noting that for $a,b\in\Q_-$ and $i\in\Z_{>0}$, one has $$\mbox{$\!\!\!\!\dis(-1)^i\binom{a}{i}=
 \Big|\binom{a}{i}\Big|{\ssc\!}\le{\ssc\!}\Big|\binom{a+b}{i}\Big|{\ssc\!}={\ssc\!}(-1)^i\binom{a+b}{i}$, \ \ $\dis \binom{k}{i}{\ssc\!}\le{\ssc\!}\Big|\binom{-k}{i}\Big|{\ssc\!}\le{\ssc\!} \left\{\!\!\begin{array}{ll}k^i\!\!&\mbox{if  }k\in\Z_{\ge1},\\[4pt]
k\!\!&\mbox{if  }0<k\in\Q_{<1}.\!\!\!\!\end{array}\right.$}\eqno{\begin{array}{r}\\[4pt]\Box\end{array}}$$
%\hfill$\Box$\vskip7pt

\def\bB{{\textbf{\textit{b}}}}\def\zz{y}%Let $z$ be a parameter.
Take
\equa{Faa}{\tilde F=\tilde f_1\zz +\sum\limits_{i=2}^\infty\tilde  f_i \zz ^i \in\C(x)[[\zz ]],
}
with $\tilde f_i\in\C(x)$ and $\tilde f_1\ne0.$
\def\tf{f}Regarding $\tilde F$ as a function on $\zz $ (with fixed $x$), we have the {\it formal inverse function} denoted by
$\zz _{{\tilde F}}\in\C(x)[[\tilde F]]$ such that [cf.~\eqref{usia}${\ssc\,}$]
\begin{eqnarray}\!\!\!\!\!\!\!\!
\label{i-Faa}\zz &\!\!=\!\!&\zz _{{\tilde F}}(\tilde F)=\bB_1\tilde F+\mbox{$\sum\limits_{i=2}^\infty$} \bB_i {\tilde F}^i,
\end{eqnarray}
with $\bB_i=\Coeff(\zz ,\tilde F^i)\in\C(x)$ being determined by $\bB_1=\tilde f_1^{-1}\in\C(x)$
 and
\begin{eqnarray}
\!\!\!\!\!\!\!\!\!\!\!\!\!\!
\label{i-Faa+}\bB_i&\!\!=\!\!&
-\mbox{$\sum\limits_{j=1}^{i-1}$}\bB_j\tilde f_1^{j-i}
\mbox{$\sum\limits_{\ell=0}^{j}$}
{\dis\binom{j}{\ell}}\mbox{$\sum\limits_{\stackrel{\sc n\in\Z_{\ge0},\,\l_1,\l_2,...,\l_n\ge0}{{}^{\ }_{\sc\l_1+2\l_2+\cdots+n\l_n={\ssc\,}i-j}}}$}{\dis\binom{\ell}{\l_1,\l_2,...,\l_n}}
\tilde f_1^{-\l_1-\l_2-\cdots-\l_n}\tilde \tf_2^{\l_1}\tilde \tf_3^{\l_2}\cdots \tilde \tf_{n+1}^{\l_n},
\end{eqnarray}
for $i\ge2$, which %where $\bB_0=0$ and \eqref{i-Faa+}
is obtained by comparing the coefficients of $\zz ^{i}$ in \eqref{i-Faa}.
\begin{lemm}\label{lemm2222}Let $($with $\hat a_i\in\R_{\ge0},\,\hat a_1>0{\ssc\,})$,
\equa{AJHAHA}{\FF =\hat a_1\zz +\sum\limits_{i=2}^\infty \hat a_i\zz ^i\in\C[[\zz ]]
\mbox{ and $\FF _\inv =\hat a_1\zz -\sum\limits_{i=2}^\infty \hat a_i\zz ^i$},} be a controlling function on $\zz $ and its negative correspendence $[$cf.~$\eqref{P---}{\ssc\,}]$, and let
\equa{ha--f-}{\zz =\hat \zz _\inv (\FF _\inv )=\hat \bB_1\FF _\inv +\sum\limits_{i=2}^\infty\hat \bB_i\FF _\inv ^i,}
 be  the formal inverse function of $\FF _\inv$, where
$\hat\bB_1=\hat a_1^{-1}$ and $\hat\bB_i=\Coeff(\zz ,\FF _\inv ^i)\in
\C$. Then
\begin{itemize}
\item[\rm(1)] $\hat \zz _\inv (\FF _\inv )$ is a controlling function on $\FF _\inv $, i.e.,
\equa{bBi>0}{\mbox{$\dis \hat \bB_i=\Coeff(\zz ,\FF _\inv ^i)\ge0$ for $i\ge1$.}}
\item[\rm(2)]If $\tilde F\DS^{x_0}_\zz \FF $ with $\tilde F$ as in \eqref{Faa} and
$\tilde f_i(x_0)$ exists for all possible  $i$ and $|\tilde f_1(x_0)|=\hat a_1$,
then $\zz =\zz _{{\tilde F}}(\tilde F)\DS_{{\tilde F}}^{x_0}\,\hat \zz _\inv (\tilde F)$,
i.e.,
\equa{HAHAHAHHJ}{\mbox{$\bB_i\DS^{x_0}\,\hat \bB_i$,}}
where $\bB_i=\Coeff(\zz ,\tilde F^i)$ is as in \eqref{i-Faa}, and $\bB_i\DS^{x_0}\,\hat \bB_i$ means that
$|\bB_i(x_0)|\le\hat \bB_i$.
In particular
\equa{y-ds-}{\zz \DS_\zz \,\hat \zz _\inv (\FF ),}
where the right side of ``\,$\DS_\zz $\,'' is regarded as a function of $\zz $ by substituting $\FF $ by \eqref{AJHAHA}.
\end{itemize}
\end{lemm}{\it{Proof.~}}~Note that (1) follows from (2) by simply taking $\tilde F=\hat a_1\zz $. Thus we prove (2). We want to prove
\equa{to0000}{\dis\frac{\partial^i \zz }{{^{\ssc\,}}\partial {\tilde F}^i }\DS^{x_0}_\zz \, \frac{d^i \zz }{d \FF{} _\inv ^i } \mbox{ for } i\ge1,}
where the left-hand side is understood as that we first use \eqref{i-Faa} to regard $\zz $ as a function on $\tilde F$ (with parameter $x$) and apply $\frac{\partial^i }{{^{\ssc\,}}\partial {\tilde F}^i }$ to it, then regard the result as a function on $\zz $ (and the like for the right-hand side, which does not contain the parameter $x$).
By \eqref{MAMS1}, we have $\frac{\ptl\tilde  F}{\ptl \zz }\DS^{x_0}_\zz \,\frac{d \FF }{d \zz }$, and thus $$\left(\frac{\ptl \tilde F}{\ptl \zz }\right)^{-1}\DS^{x_0}_\zz \,\left(\Big(\frac{d \FF }{d \zz }\Big)_\inv\right) ^{-1}=\left(\frac{d\FF_\inv }{d \zz }\right)^{-1},$$ i.e., $\frac{\partial \zz }{\partial\tilde  F}\DS^{x_0}_\zz \,
\frac{d \zz }{d \FF_\inv }$ and  \eqref{to0000} holds for $i=1$. Inductively, by Lemma \ref{ds-lemm},
\begin{eqnarray}
\frac{\partial^i \zz }{\partial \tilde F^i }&\!\!=\!\!&\frac{\ptl}{\ptl \tilde F}\Big(\frac{\partial^{i-1} \zz }{\partial\tilde  F^{i-1} }\Big)=\frac{\ptl}{\ptl \zz }\Big(\frac{\partial^{i-1} \zz }{\partial \tilde F^{i-1} }\Big)
\Big({\frac{\ptl\tilde  F}{\ptl \zz }}\Big)^{-1}\nonumber\\[6pt]
&\DS^{x_0}_\zz \!\!&
\frac{d}{d \zz }\Big(\frac{d^{i-1} \zz }{d \FF{} _\inv ^{i-1} }\Big)
\Big({\frac{d \FF_\inv }{d \zz }}\Big)^{-1}=\frac{d^i \zz }{d \FF{}_\inv ^i }\,.
\end{eqnarray}
This proves \eqref{to0000}. Using \eqref{to0000} and noting from \eqref{i-Faa} and \eqref{ha--f-}, we have $$\mbox{
$\begin{array}{llll}\dis\bB_{i%+1
}\!\!\!&\dis=\frac{1}{i!}\frac{\partial^i \zz }{\partial \tilde F^i }\Big|_{\tilde F=0}=\frac{1}{i!}\frac{\partial^i \zz }{\partial\tilde  F^i }\Big|_{\zz =0}\\[11pt]
\!\!\!&\dis\DS^{x_0}\,
\frac{1}{i!}\frac{d^i \zz }{d \FF{} _\inv ^i }\Big|_{\zz =0}=
\frac{1}{i!}\frac{d^i \zz }{d \FF{} _\inv ^i }\Big|_{\FF _\inv =0}=\hat \bB_{i%+1
}\end{array}$.
}$$
This proves \eqref{HAHAHAHHJ}. Since $\tilde F\DS^{x_0}_\zz \,\FF $ and $\hat \zz _\inv $ is a controlling function, we have
$\hat \zz _\inv (\tilde F)\DS^{x_0}_\zz \, \hat \zz _\inv (\FF )$. This together with \eqref{HAHAHAHHJ}
proves \eqref{y-ds-}.\hfill$\Box$


\section{Proof of Theorem \ref{lemm-a1}}\label{Th3Sec}

%First note that for the purpose of proving Theorem \ref{lemm-a1}, we can exchange variables $x$ and $y$ with $\th$
%being changed to $\th^{-1}$.
First we use \eqref{mqp1234-2} to prove \eqref{mqp1234-2+}: By exchanging $x$ and $y$ if necessary, we may assume $h_{p_0,p_1}=|y_t|,$ $n_t=|\TH x_t|$. Then the only nontrivial case in \eqref{mqp1234-2+} is the case when $a=|y_t|,$ $b=|\TH x_t|$. In this case, we have
\begin{eqnarray}
\label{InTSC}&\!\!\!\!\!\!\!\!\!\!&
|a-b|=\big||y_t|-|\TH x_t|\big|\le|y_t+\TH x_t|<\tau h_{_{\sc p_0,p_1}}^{^{\sc \frac{m}{m+1}}}
\nonumber\\[4pt]
&\!\!\!\!\!\!\!\!\!\!&
\phantom{|a-b|}=\tau|y_{t}|^{\frac{m}{m+1}}<\tau|\TH x_t|^{\frac{m+1}{m+2}}=\tau n_{_{\sc t}}^{^{\sc
\frac{m+1}{m+2}}},\end{eqnarray}
 where the last inequality follows from the fact that by \eqref{mqp1234-2}, we have (when $|y_t|=h_{p_0,p_1}\ge\SS_0$ is sufficiently large)
\equa{WHAHA}{\mbox{$\dis |\TH x_t|>|y_t|-\tau h_{_{\sc p_0,p_1}}^{^{\sc \frac{m}{m+1}}}
=|y_t|-\tau |y_t|^{\frac{m}{m+1}}>|y_t|^{\frac{m(m+2)}{(m+1)^2}}$.}}
To prove \eqref{mqp1234-2}, assume conversely that there exists $(p_{0i},p_{1i})$ $=\big((x_{0i},y_{0i}),(x_{1i},y_{1i})\big)\in V$
for any $i\in\Z_{>0}$ satisfying
\equa{mqp1234-3}{h_{p_{0i},p_{1i}}\ge i,}
such that at least one of the following does not hold:
\equa{mqp1234-4}{{\rm (i)\ \ }|\TH x_{0i}+y_{0i}|<\tau h^{^{\sc \frac{m}{m+1}}}_{_{\sc p_{0i},p_{1i}}},\ \ \ {\rm (ii)\ \ }|\TH x_{1i}+y_{1i}|<\tau h^{^{\sc \frac{m}{m+1}}}_{_{\sc p_{0i},p_{1i}}}.}
Thus we obtain a sequence $(p_{0i},p_{1i})$, $i=1,2,...$
Since at least one of the conditions in \eqref{mqp1234-4} cannot hold for infinite many $i$'s. If necessary by replacing the sequence by a subsequence [if the sequence $(p_{0i},p_{1i})$ is replaced by the subsequence  $(p_{0,{i_j}},p_{1,{i_j}})$, then we always have $i_j\ge j$; thus \eqref{mqp1234-3} still holds after the replacement], we may assume one of the conditions in \eqref{mqp1234-4} does not hold for all $i$.
If necessary by switching $p_{0i}$ and $p_{1i}$, we can assume \eqref{mqp1234-4}\,(i) cannot hold for all $i$, i.e.,
\equa{mqp1234-5}{|\TH x_{0i}+y_{0i}|\ge\tau h^{^{\sc \frac{m}{m+1}}}_{_{\sc p_{0i},p_{1i}}}\to\infty\mbox{ for all }i\gg1.}
We need to use the following notations:
\equa{MSM1111}{\mbox{$a_i\sim  b_i,\ \ a_i\prec b_i,\ \ a_i\preceq b_i$,}} means respectively \equan{SMSMDMD}{\mbox{$\dis\SS_1<
 \Big|\frac{a_i}{b_i} \Big|<\SS_2,\ \ \lim\limits_{i\to\infty} \frac{a_i}{b_i}=0,\ \ \Big|
\frac{a_i}{b_i}\Big|\le\SS_1$,}}
for some fixed $\SS_1,\SS_2\in\R_{>0}$.
By \eqref{wheraraaa}, we can write, for some $f_{jk}\in\C$,
\equa{Fasa}{ F= F_L+F_1\mbox{ with }F_1=\sum\limits_{j=0}^{m-1}y^{m-1-j}\sum\limits_{k=0}^jf_{jk}x^k.}
%Now \eqref{mqp1234-5} tells that at least a subsequence of $|x_{0i}|$ or $|y_{0i}|$ tends to the infinity.
%If necessary by replacing the sequence by a subsequence and/or by switching $x$ and $y$, we may assume $|x_{0i}|\ge|y_{0i}|$ and  $|x_{0i}|\to\infty$. Then using the fact that
Since $|\TH x_{0i}|,|y_{0i}|\le h_{p_{0i},p_{1i}}$,
by \eqref{mqp1234-5} and \eqref{Fasa}, we have \equa{WEHSSSS}{\mbox{$\dis F_1(p_{0i})\preceq h_{_{\sc p_{0i},p_{1i}}}^{^{\sc m-1}}\prec(\TH x_{0i}+y_{0i})^m
=F_L(p_{0i})$,}} and thus
[we remark that although it is possible that
$(\TH x_{0i}+y_{0i})^m\prec h_{_{\sc p_{0i},p_{1i}}}^{^{\sc m}}$, it is very crucial that we have \eqref{Wo(999)}${\ssc\,}$]
\equa{Wo(999)}{\mbox{$\dis F(p_{0i})\sim F_L(p_{0i})=(\TH x_{0i}+y_{0i})^m$.}}
Similarly,
%since $x_{1i},y_{1i}\preceq h_{p_{0i},p_{1i}}$, we have
$F_1(p_{1i})\preceq h_{_{\sc p_{0i},p_{1i}}}^{^{\sc m-1}}\prec(\TH x_{0i}+y_{0i})^m=F_L(p_{0i})$.
%Thus in order for $ F(p_{0i})$ and $ F(p_{1i})$ to be equal,  we must have
We obtain the following important fact:
\begin{eqnarray}\label{sim1a?qa}
&\!\!\!\!\!\!\!\!\!\!\!\!\!\!&
\dis1=\frac{F(p_{1i})}{F(p_{0i})}
=\lim_{i\to\infty}\frac{F(p_{1i})}{F(p_{0i})}=\lim_{i\to\infty}
\frac{\frac{F_1(p_{1i})}{F_L(p_{0i})}
+\frac{F_L(p_{1i})}{F_L(p_{0i})}}{\frac{F_1(p_{0i})}{F_L(p_{0i})}+1}
\nonumber\\[4pt]
&\!\!\!\!\!\!\!\!\!\!\!\!\!\!&
\phantom{1}=\lim_{i\to\infty}\frac{F_L(p_{1i})}{F_L(p_{0i})}=\lim_{i\to\infty}
\frac{(\TH x_{1i}+y_{1i})^m}{(\TH x_{0i}+y_{0i})^m}.\end{eqnarray}
Therefore, by replacing the sequence by a subsequence, we have\equa{sim1aqa}{\dis
\lim_{i\to\infty}\frac{\TH x_{1i}+y_{1i}}{\TH x_{0i}+y_{0i}}=\omega,}where $\omega$ is  some $m$-th root of unity.
Furthermore, when $i\gg1$, by \eqref{mqp1234-5} we have [cf.~Convention \ref{conv1}\,(3)]
\equa{AlllCASE}{\dis\ep:=\frac
{\,h_{_{\sc p_{0i},p_{1i}}}^{^{\sc m-1}}\,}{{\dis \b_{_{\sc 0}}^{{\sc m}}}}\to0,\mbox{ \ where \ }\b_0:=\TH x_{0i}+y_{0i}.}
Set
\equa{bob1=}{\dis\b_1:=\frac{\TH x_{1i}+y_{1i}}{\TH x_{0i}+y_{0i}}-1\ \ \to\ \ \omega-1.}
\begin{rema}\label{emakaaa}\rm Before continuing, we would like to remark that our idea is to take some variable change [cf.~\eqref{0-SSS1}${\ssc\,}$]
to send the leading part $ F_L$ of $ F$ to a ``leading term'' [cf.~\eqref{ansnwnw}${\ssc\,}$], which has the highest absolute value when $(x,y)$ is set to $p_{0i}$ or $p_{1i}$ [cf.~\eqref{sss1==}${\ssc\,}$] so that when we expand it
as a power series of $y$, it \CA [cf.~\eqref{+1-928384}${\ssc\,}$], and further, the inverse function \CA (cf.~Lemma \ref{newlemm}). Then we can derive a contradiction [cf.~\eqref{Liebd----}${\ssc\,}$].
\end{rema}
\noindent{\it Proof of Theorem \ref{lemm-a1}.~}
Now we begin our proof of \eqref{mqp1234-2} in Theorem \ref{lemm-a1} as follows.
Since either $x_{0i}\ne x_{1i}$ or  $y_{0i}\ne y_{1i}$ for all $i$,
replacing the sequence by a subsequence and/or exchanging $x$ and $y$ if necessary, we may assume $y_{0i}\ne y_{1i}$ for all $i$.
Set [where $\ii=\sqrt{-1}$, cf.~the statement after \eqref{928384} to see why we need to choose such a $\b_2$]
 \equa{u1==}{\dis u_1=1+\b_1x+\b_2x(1-x)\in\C[x],
\mbox{ \ where \ }\b_2=\left\{\begin{array}{ll}0&\mbox{if \ }\omega\ne-1,\\[4pt]\ii&\mbox{else.}\end{array}\right.} We have
\begin{lemm}\label{Theste}
There exists some $\delta>0$ independent of $\ep$ such that
\equa{delta0}{|u_1(a)|>\delta\mbox{ for }0\le a\le1\mbox{ \ $($when $i\gg1)$}.}
\end{lemm}{\it Proof.~}Fix $\d_1\in\R_{>0}$ to be sufficiently small.\begin{itemize}\item
First assume $\omega=1$ (then $\b_2=0$). By \eqref{bob1=}, we can then assume $|\b_1|<\d_1$. Then for $0\le a\le1$, we have $|u_1(a)|\ge1-|\b_1|a\ge1-\d_1$.\item
Next assume $\omega=-1$ (then $\b_2=\ii$). We can then assume $|\b_{1\rm\,im}|<\d_1^2$ [cf.~Convention \ref{conv1}\,(1)] and $2-\d_1^2\le |\b_{1\rm\,re}|\le2+\d_1^2$ by \eqref{bob1=}.
For $\d_1\le a\le 1-\d_1$, we have \equa{HAV@1}{\mbox{$\ \ \ |u_1(a)|\ge|u_1(a)\im|=|\b_{1\rm\,im}a+\b_{2\rm\,im}a(1-a)|\ge a(1-a)-|\b_{1\rm\,im}|a\ge\d_1(1-\d_1)-\d_1^2$.}\!\!\!\!\!\!\!\!\!}
If $0\le a\le \d_1$, we have $|u_1(a)|\ge|u_1(a)\re|=|1+\b_{1\rm\,re}x|\ge1-(2+\d_1^2)\d_1$.
If $1-\d_1\le a\le1$, then $|u_1(a)|\ge|u_1(a)\re|=|1+\b_{1\rm\,re}x|\ge(2-\d_1)(1-\d_1)-1$.
\item
Now assume $\omega\ne\pm1$ (then $\b_2=0$). We can then assume $|\b_{1\rm\,im}|\ge\d_1$ and $|\b_1|\le2+\d_1$. If $0\le a\le\d_1$, we have
$|u_1(a)|\ge1-|\b_1|\d_1\ge1-(2+\d_1)\d_1$. If $\d_1\le a\le1$, then
$|u_1(a)|\ge |u_1(a)\im|=|\b_{1\rm\,im}|\d_1\ge\d_1^2$.\end{itemize}
 In any case we can choose $\d>0$ such that
\eqref{delta0} holds.\hfill$\Box$\vskip5pt
%\equa{b2==}{\mbox{ $u_0\!:=\!1\!+\!(\omega\!-\!1)x\!+\!\b_2x(1\!-\!x)\!\in\!\C[x]$ does not have a root in $\{a\in\R\,|\,0\le a\le1\}$.}}
%Since $u_1\!\to\!u_0$ when $i\!\to\!\infty$, by \eqref{bob1=} and \eqref{b2==},
We set [cf.~Remark \ref{emakaaa}, our purpose is to use the variable change \eqref{SSS1} to send the leading part $ F_L$ of $ F$ to the element
\eqref{ansnwnw}  which is a term (the ``leading term'') with the lowest degree of  $y$ in $\hat F$, cf.~\eqref{928384}${\ssc\,}$]
\begin{eqnarray}
\label{0-SSS1}&\!\!\!\!\!\!\!\!\!\!\!\!\!\!\!\!&
u=\b_0u_1y^{-1}-\TH v,\ \ \  \ \ \ v=y_{i0}+\b_3x,\ \ \ \ \b_3=y_{1i}-y_{0i},\\[4pt]
\label{SSS1}&\!\!\!\!\!\!\!\!\!\!\!\!\!\!\!\!&
 \hat F=%(\TH
 \b_0%)
^{-m}  F(u,v),\ \ \ \ \ \ \hat G=%\TH^m
 \b_0^{m-1}\b_3^{-1}  G(u,v)\in\C[x,u_1^{-1},y^{-1}]\subset\C(x)[y^{\pm1}].
\end{eqnarray}
Then one can verify that $J(u,v)=-\frac{\ptl u}{\ptl y}{\sc\,}\frac{d v}{dx}=\b_0\b_3u_1y^{-2}$ and \begin{eqnarray}
\label{++-SSS1---}&\!\!\!\!\!\!\!\!\!\!\!\!\!\!\!\!&\!\!\!\!\!\!\!
J(\hat F,\hat G)=u_1y^{-2},\  \ \big(\hat F(q_0),\hat G(q_0)\big)=\big(\hat F(q_1),\hat G(q_1)\big),
\mbox{ where $q_0=(0,1),\, q_1=(1,1)$.}\end{eqnarray}
Note that the leading part $ F_L$ of $ F$  contributes to  $\hat F$ the following
element (which is the only term  in $\hat F$ with the lowest $y$-degree $-m$, referred to as the
{\it leading term} of $\hat F$):
\equa{ansnwnw}{%(\TH
\b_0%)
^{-m}(\TH u+v)^m=u_1^{m}y^{-m}.}
Since all coefficients of $x$ and $y^{-1}$ in $u$ or $v$ have absolute values being $\preceq$ the height $ h_{p_{0i},p_{1i}}$
[cf.~\eqref{0-SSS1}${\ssc\,}$],
due to
 the factor $\b_{0}^{-m}$ in $\hat F$ [cf.~\eqref{SSS1}${\ssc\,}$],  we see from \eqref{Fasa} and \eqref{AlllCASE} that other terms of $ F$ (i.e., terms in $F_1$) can only contribute $O(\ep^1)$ elements to $\hat F$ [cf.~Convention \ref{conv1}\,(3) for notation $O(\ep^j)$]. Thus we can write, for some $ f_{j}=O(\ep^1)\in\C[x,u_1^{-1}]\subset\C(x)$,
\begin{eqnarray}
\label{928384}&\!\!\!\!\!\!\!\!\!\!\!\!\!\!\!\!\!\!\!\!&
\hat F=u_1^{m}y^{-m}\Big(1+\mbox{$\sum\limits_{j=1}^{m}$} f_{j}y^j\Big).
\end{eqnarray}
%and some $\bar m\in\Z_{>0}$ (in our case here $\bar m=m$; in Case 2 below $\bar m=2m$).
By \eqref{delta0}, we see that
 $ f_j(a)$ for $0\le a\le 1$ is well-defined for any $j$ and $f_j(a)=O(\ep^1)$ [this is why we need to choose some $\b_2$ to satisfy \eqref{delta0}${\ssc\,}$].
Set \equa{sss1==}{\mbox{$\SS_1=\max\big\{| f_j(a)|\,\,\big|\,\,1\le j\le%\bar
m,\,0\le a\le 1\big\}=O(\ep^1)$.}} Let $0\le a\le 1$. Take [here we choose an $m$-th root of $u_1^{m}$ to be $u_1$, this choice will not cause any problem since we will only encounter integral powers of $u_1$ below, cf.~\eqref{Holsss}${\ssc\,}$]
\begin{eqnarray}
\label{1-928384}&\!\!\!\!\!\!\!\!\!\!\!\!\!\!\!\!\!\!\!\!&
P:={\hat F^{-\frac1m}}\in u_1^{-1} y+y^2\C(x)[[y]],\\
\label{AA+1-928384}&\!\!\!\!\!\!\!\!\!\!\!\!\!\!\!\!\!\!\!\!&
 \FF=|u_1(a)|^{m}y^{-m}(1+\hat F_-),\mbox{ where }\hat F_-=\SS_1
\mbox{$\sum\limits_{j=1}^{%\bar
m}$}y^j=O(\ep^1).
\end{eqnarray}
We have (cf.~Definition \ref{contro})
\begin{eqnarray}
\label{+1-928384}&\!\!\!\!\!\!\!\!\!\!\!\!\!\!\!\!\!\!\!\!&
\hat F\ \DS^{a}_y\ \FF,\ \ \ \ \ \ P\ \DS^a_y\ \pPP:=|u_1(a)|^{-1}y(1-\hat F_-)^{-\frac1m}.
\end{eqnarray}
Thus $\hat F,P$ \CA [by Lemma \ref{ds-lemm}\,(3)% and \eqref{1-928384}
] when setting $x=a$ and $y=1$. Let
\equa{P000===}{P_0:=P|_{(x,y)=(0,1)}=1+O(\ep^1),}where the last equality can be easily seen from \eqref{928384} and \eqref{1-928384} by noting that $u_1(0)=1$.
Write [cf.~\eqref{usia} and \eqref{i-Faa}; assume that $\hat G$ has the lowest $y$-degree $-m^{\ssc G}{\ssc\,}$]
\begin{eqnarray}\label{y==inv}
\!\!\!\!\!\!\!\!\!\!\!\!&\!\!\!\!\!\!\!\!\!\!\!\!&
y=u_1 P+\mbox{$\sum\limits_{i=2}^\infty$} \bB_i P^i\mbox{ for some }\bB_i\in%\C[x,u_1^{-1}]\subset
\C(x),
\\[6pt]
\label{GasFG}\!\!\!\!\!\!\!\!\!\!\!\!&\!\!\!\!\!\!\!\!\!\!\!\!&
\hat G=\mbox{$\sum\limits_{i=-m^{\ssc G}}^\infty$} c_iP^i\mbox{ for some }c_i\in\C(x)
.
\end{eqnarray}
To continue the proof of Theorem \ref{lemm-a1}, we need the following lemma.
First, let $0\le a\le1$.
\begin{lemm}\label{newlemm}\begin{itemize}\item[\rm(1)]
The series in \eqref{y==inv}
\CA when setting $(x,P)$ to $(a,P_0)$, and
\equa{ACVa0}
{Y_0(a):=y|_{(x,P)=(a,P_0)}=u_1(a)+O(\ep^1).}
\item[\rm(2)]Regarding $\big(\frac{\ptl \hat F}{\ptl y}\big)^{-1}$ as a series of
$P$, it   \CA
when setting $(x,P)$ to $(a,P_0)$. Furthermore,
\equa{F'}{\dis\Big(\frac{\ptl \hat F}{\ptl y}\Big)^{-1}\Big|_{(x,P)=(a,P_0)}=-m^{-1}u_1(a)+O(\ep^1).}
\item[\rm(3)]The series in \eqref{GasFG}   \CA when setting $(x,P)$ to $(a,P_0)$.
\end{itemize}\end{lemm}
\noindent{\it Proof.~} (1) (cf.~Remark \ref{simRe}) Note that the negative  correspondence of $\pPP$ is [cf.~\eqref{P---}${\ssc\,}$]
\equa{IndF}{\pPP_\inv:=2|u_1(a)|^{-1}y-\pPP=2|u_1(a)|^{-1}y-|u_1(a)|^{-1}y(1-F_-)^{-\frac1{%\bar
m}}.}
Let $y_{\inv}{\ssc}= y_{\inv}(\pPP_\inv)$ be the inverse function of $\pPP_\inv$ [cf.~\eqref{ha--f-}${\ssc\,}$]. Then Lemma \ref{lemm2222} shows that\equa{le-firt}{\mbox{$y(P){\ }\DSS^{a}_{P} {\ssc\ }y_{\inv}
(P)$.}}
Thus to see whether the series  in \eqref{y==inv} [which is the left-hand side of \eqref{le-firt}${\ssc\,}$]  \CA when setting $(x,P)$ to $(a,P_0)$, it suffices to see if the series
$y_{\inv}(P)$ [which is the right-hand side of \eqref{le-firt}${\ssc\,}$] converges when setting $P$ to  $|P_0|$. The latter is equivalent to whether \eqref{IndF} has the
solution for $y$ when  $\pPP_\inv$ is set to $|P_0|$
(note that the solution, if exists, must be unique by noting that a controlling function which is a nontrivial power series of $y$ must be a strictly increasing function).
Note from \eqref{P000===}
that\equa{AbsF0=}{\mbox{ $|P_0|=1+w$
for some $w\in\R$ such that $-\SS_2\ep\le w\le\SS_3\ep$,}}
for some $\SS_2,\SS_3$.
Consider the right-hand side of \eqref{IndF}:\begin{itemize}\item if we set  $y$ to $|u_1(a)|-\SS_4\ep$ for some sufficiently large $\SS_4$, then it
 obviously has some value $1+w_1$ with $w_1<-\SS_2\ep\le w$;\item
 if we set
 $y$ to $|u_1(a)|+\SS_5\ep$ for some sufficiently large $\SS_5$, then it has some value $1+w_2$
 with $w_2>\SS_3\ep\ge w$.\end{itemize}
  Since the right-hand side of \eqref{IndF} is a continuous function on $y$, this shows that there exists (unique) $y_0\in\R_{>0}$ such that
 \equa{y0===}{\pPP_\inv|_{y=y_0}=|P_0|,\mbox{ \ \ and obviously,\ \ }y_0=|u_1(a)|+O(\ep^1),} i.e.,
 \eqref{IndF} has the solution  $y=y_0$ when $\pPP_\inv$ is set to $|P_0|$,
and thus the first part of (1) follows.
 As for \eqref{ACVa0}, note that $Y_0(a)$ is the solution of $y$ in the equation $P_0=P|_{x=a}$. Using
\eqref{ACVa0} in this equation, we see it holds up to $O(\ep^1)$.

(2) By Lemmas \ref{ds-lemm} and \ref{lemm2222}, using \eqref{1-928384}--\eqref{+1-928384}, we have
\begin{eqnarray}\label{P'==}
&\!\!\!\!\!\!\!\!\!\!\!\!&\Big(\frac{\ptl \hat F}{\ptl y}\Big)^{-1}
\ \DS^a_y\ \frac{y^{m+1}}{m|u_1(a)|^{m}(1-Q_-)}
%\cdot\frac{1}{}
\ \DS%^a
_{P}\ \frac{y^{m+1}}{m|u_1(a)|^{m}(1-Q_-)}
%\cdot\frac{1}{1-Q_-}
\Big|_{y=y_\inv(P)}, \end{eqnarray} where $Q_-
%=\frac{d\hat F_-}{dy}
=\SS_1\mbox{$\sum_{j=1}^{%\bar
m}%\big|
\frac{m-j}{m}%\big|
y^j$}=O(\ep^1)$ [cf.~\eqref{AA+1-928384}${\ssc\,}$].
The right-hand side of \eqref{P'==} (a controlling function)
converges obviously when setting $P$ to $|P_0|$ since by \eqref{y0===}, we have $y_\inv(|P_0|)=y_0=|u_1(a)|+O(\ep^1)$ and so
\equa{Q====0}{\mbox{$\dis0\le Q_-\big|_{y=y_\inv(|P_0|)}=O(\ep^1)<1$.}}
This proves the first statement of (2) [cf.~\eqref{ABV1}${\ssc\,}$].
As for the second statement, note that setting $(x,P)$ to $(a,P_0)$ is equivalent to setting $(x,y)$ to $\big(a,Y_0(a)\big)$. Then \eqref{F'} follows from \eqref{928384} and \eqref{ACVa0}.

(3)  follows from (1) since $\hat G|_{x=a}$ is a polynomial on $y^{\pm1}$.
This proves  Lemma \ref{newlemm}.\hfill$\Box$
\begin{rema}\rm\label{simRe}By \eqref{928384}, \equa{lead}{\hat F=\mbox{the leading term}+O(\ep^1),}
where $O(\ep^1)$ is a finite combination of powers of $y^{\pm1}$. In this case, we can in fact easily choose a simpler controlling function $\pPP$ for $P$ [cf.~\eqref{+1-928384}${\ssc\,}$]: \equa{P==aaa}{\mbox{$\pPP=|u_1(a)|^{-1}y\sum\limits_{i=0}^\infty\ep^{\d_1 i}y^i=\dis\frac{|u_1(a)|^{-1}y}{1-\ep^{\d_1}y}$,}} where $\d_1\in\R_{>0}$ is some fixed sufficiently small number. Then \equa{lead2}{\dis\pPP_\inv=|u(a)|^{-1}y\Big(1-\frac{\ep^{\d_1}y}{1-\ep^{\d_1}y}\Big)=
\frac{|u(a)|^{-1}y(1-2\ep^{\d_1}y)}{1-\ep^{\d_1}y},} and we can explicitly write down the inverse function of $\pPP_\inv$ by solving $y$ from \eqref{lead2} to obtain $y_{\inv}(\pPP_\inv)$ [which, by Lemma \ref{lemm2222}\,(1), must be a controlling function on $\pPP_\inv$ (although it is not obvious to see)]
\begin{eqnarray}
\label{lead3}&\!\!\!\!\!\!\!\!\!\!\!\!\!\!\!\!&
\dis y_{\inv}(\pPP_\inv)=\frac{1+\ep^{\d_1}|u(a)|\pPP_\inv-A}{4\ep^{\d_1}}, \mbox{ \ where }\\[6pt]
 &\!\!\!\!\!\!\!\!\!\!\!\!\!\!\!\!&
\nonumber A=\Big(1-6\ep^{\d_1}|u(a)|
\pPP_\inv+\ep^{2\d_1}|u(a)|^2\pPP_\inv^2\Big)^{\frac12}.\end{eqnarray}
From this, one easily sees that the right-hand side of \eqref{le-firt} converges when  $P$ is set to $|P_0|$, i.e., \eqref{lead3} converges when $\pPP_\inv$ is set to $|P_0|$ [if we expand $A$  as a power series of $\pPP_\inv$, it converges absolutely when $\pPP_\inv$ is set to $|P_0|$ simply because there appear the factors $\ep^{\d_1}$ and $\ep^{2\d_1}$]. Thus the proof of Lemma \ref{newlemm}\,(1) is easier (we have used the above proof as it can be adapted in some more general situation).
\end{rema}

Now we return to our proof of Theorem \ref{lemm-a1}.
By Lemma \ref{newlemm}\,(3), we are now safe to set $(x,y)$ to $(0,1)$ and $(1,1)$ [which is equivalent to setting
$(x,P)$ to $(0,P_0)$ and $(1,P_0)$ respectively] in \eqref{GasFG} to obtain
\begin{eqnarray}
\label{GasFG-set}\!\!\!\!\!\!\!\!\!\!\!\!&\!\!\!\!\!\!\!\!\!\!\!\!&
0=\hat G(1,1)-\hat G(0,1)=\mbox{$\sum\limits_{i=-m^{\ssc G}}^\infty$} \big(c_i(1)-c_i(0)\big)P_0^i.
\end{eqnarray}
Denote
\begin{eqnarray}
\label{LieANAbdbd1}&\!\!\!\!\!\!\!\!\!\!\!\!\!\!\!\!\!\!\!\!\!\!\!\!&
Q:=-
{J(\hat F,\hat G)}\Big(\frac{\ptl \hat F}{\ptl y}\Big)^{-1}=-{u_1y^{-2}}\Big(\frac{\ptl \hat F}{\ptl y}\Big)^{-1},
\end{eqnarray}where the last equality follows from \eqref{++-SSS1---}.
Take the Jacobian of $\hat F$ with \eqref{GasFG}, by \eqref{++-SSS1---} and \eqref{LieANAbdbd1}, we
obtain [by regarding $Q$ as in $\PP$]
\begin{eqnarray}\label{Liebdbd1}&\!\!\!\!\!\!\!\!\!\!\!\!\!\!\!\!\!\!\!\!\!\!\!\!&Q=
\mbox{$\sum\limits_{i=-m^{\ssc G}}^\infty$}\frac{d c_{i}}{dx}P^i.
\end{eqnarray}
By \eqref{LieANAbdbd1} and the proof of
Lemma  \ref{newlemm},
we see that when $P$ is set to $P_0$, the series  in \eqref{Liebdbd1}
  \CA and in fact uniformly for $x\in[0,1]:=\{x\in\R\,|\,0\le x\le1\}$.
This together with \eqref{GasFG-set} and \eqref{Liebdbd1}
implies
\begin{eqnarray}\label{Liebd----}&\!\!\!\!\!\!\!\!\!\!\!\!\!\!\!\!\!\!\!\!\!\!\!\!&0
=\mbox{$\sum\limits_{i=-m^{\ssc G}}^\infty$} (c_i(1)-c_i({0}))P_0^i=
\mbox{$\sum\limits_{i=-m^{\ssc G}}^\infty$}\int_{0}^{1} \frac{dc_{i}}{dx}P_0^idx
\nonumber\\[6pt]
&\!\!\!\!\!\!\!\!\!\!\!\!\!\!\!\!\!\!\!\!\!\!\!\!&\phantom{0}
=\int_{0}^{1}\mbox{$\sum\limits_{i=-m^{\ssc G}}^\infty$} \frac{dc_{i}}{dx}P_0^idx
=\int_{0}^{1}Q\,\big|_{P=P_0}dx=\int_{0}^{1} Q\,\big|_{y=Y_0(x)}dx
\nonumber\\[6pt]
&\!\!\!\!\!\!\!\!\!\!\!\!\!\!\!\!\!\!\!\!\!\!\!\!&\phantom{0}
=m^{-1}
\int_0^{1} dx+O(\ep^1)=
m^{-1}+O(\ep^1),\end{eqnarray}which is a contradiction,
where the sixth equality of \eqref{Liebd----} follows from \eqref{LieANAbdbd1}, \eqref{ACVa0} and \eqref{F'},  and the fifth follows by noting that $Q\big|_{P=P_0}$ means that  we need  to express $Q$ as an element in
$\C(x)[[P]]$ [i.e., use \eqref{y==inv}  to substitute $y$, that is exactly the equation \eqref{Liebdbd1}${\ssc\,}$] then set $P$ to $P_0$, which is equivalent to directly
  setting $y$ to $Y_0(x)$ in $Q$ [cf.~\eqref{ACVa0}${\ssc\,}$].
%Thus this case cannot occur.\vskip4pt
%\noindent{\bf Case 2}: {\it Assume $y_{1i}=y_{0i}$ for $i\gg1$.} Then $\b_1\ne0$ by \eqref{bob1=} (otherwise we would obtain $p_{0i}=p_{1i}$).
%Similar to \eqref{u1==}, we set
% \equa{+u1==}{u_1\!=\!1+\!\b_1u_2,\ \ \ u_2=x\!+\!\bar\b_2x(1\!-\!x),}
%where $\bar\b_2\in\C$ is chosen such that \eqref{delta0} holds [such $\bar\b_2$ can be easily found, for example if $\omega=1$, i.e., $\lim_{i\to\infty}\b_1=0$ by \eqref{bob1=}, then we can simply choose $\bar\b_2=0$; if $\omega\ne1$, then we can take $\bar\b_2=\b_1^{-1}\b_2$ such that $\b_2$ satisfies \eqref{b2==}${\ssc\,}$].
%Then similar to \eqref{0-SSS1} and \eqref{SSS1}, we take (where $\tilde\b_3=y_{1i}-\b_3$ for
% any $\b_3\in\C$ with $\tilde\b_3\ne0$)
%\begin{eqnarray}
%\label{2+++0-SSS1}&\!\!\!\!\!\!\!\!\!\!\!\!\!\!\!\!\!\!\!\!\!&
%u\!=\!\b_0u_1y^{-1}\!-\!v,\,  v\!=\!\tilde\b_3y\!+\!\b_3, \, \ \
% \hat F\!=\!\b_0^{-m}  F(u,v),\, \ \
% \hat G\!=\!\tilde\b_3^{-1}\b_1^{-1}\b_0^{m-1}  G(u,v)\!\in\!\C(x)[y^{\pm1}].
%\end{eqnarray}
%Then $J(\hat F,\hat G)=\frac{d u_2}{dx}y^{-1}$, and we have the last equation of \eqref{++-SSS1---}. Now following exactly the same arguments in Case 1 [the only difference is now $\bar m=2m$, cf.~\eqref{928384}], we can obtain [cf.~\eqref{Liebd----}]
%\equa{Liebd----1}{\dis 0=m^{-1}\int_0^1\frac{du_2}{dx}dx+O(\ep^1)=m^{-1}+O(\ep^1),}
%which again is a contradiction. Thus Case 2 cannot occur either, which
The contradiction  means that if \eqref{mqp1234-3} holds, then we must have \eqref{mqp1234-4}. The proof of Theorem \ref{lemm-a1}
is now completed.
\hfill$\Box$


\section{Proof of Theorem \ref{AddLeeme--0}}
\noindent{\it Proof of Theorem \ref{AddLeeme--0}.~}%To prove Theorem \ref{AddLeeme--0}\,(i), take $\SS_1$ to be sufficiently large.
To prove the boundness of $A_{k_0,k_1}$ defined in \eqref{Ak=}, assume
\equa{SeSSS}{\mbox{$\dis (p_{0i},p_{1i})=\big((x_{0i},y_{0i}),(x_{1i},y_{1i})\big)\in A_{k_0,k_1},\,i=1,2,...,$}} is a sequence such that the height $h_{p_{0i},p_{1i}}\to\infty$. By definition, we have
$|x_{0i}|=k_0,\,|x_{1i}|=k_1$. Thus $|y_{0i}|=h_{p_{0i},p_{1i}}$ or $|y_{1i}|=h_{p_{0i},p_{1i}}$, in any case,
at least one inequation of \eqref{mqp1234-2} is violated. Hence $A_{k_0,k_1}$ is bounded.
To prove the closeness of $A_{k_0,k_1}$, let \eqref{SeSSS} %$(p_{0i},p_{1i})=\big((x_{0i},y_{0i}),(x_{1i},y_{1i})\big)\in A_{k_0,k_1}$, $i=1,2,...,$
be a sequence converging to some $(p_0,p_1)=\big((x_0,y_0),(x_1,y_1)\big)\in\C^4$. Then $\si(p_0)=\si(p_1)$ and $|x_0|=k_0,\,|x_1|=k_1$. We must have $p_0\ne p_1$ [otherwise, the local bijectivity of $\si$ does not hold at the point $p_0$, cf.~arguments after \eqref{(qqq)}${\ssc\,}$], i.e., $(p_0,p_1)\in A_{k_0,k_1}$, and so
 $A_{k_0,k_1}$  is a closed set in $\C^4$, namely, we have Theorem \ref{AddLeeme--0}\,(i). From this, we see
that $\g_{k_0,k_1}$ in \eqref{Ak=1} is well-defined.
%
\NOUSE{To prove the continuousness of $\g_{k_0,k_1}$,
let $(k_{0i},k_{1i})\in\R_{\ge0}^2$, $i=1,2,...$ be a sequence such that
\equa{1-Take1111}{\mbox{$\dis\lim_{i\to\infty}(k_{0i},k_{1i})=(k_0,k_1)$,}}
and $\lim_{i\to\infty}\g_{k_{0i},k_{1i}}=\g$ for some $k_0,k_1,\g\in\R_{\ge0}$. Take $(p_0,p_1)=\big((x_0,y_0),(x_1,y_1)\big)\in V$ with \equa{2-Take1111}{\mbox{$|x_0|=k_0,$ \ \ \ $|x_1|=k_1$,}} and $|y_1|=\g_{k_0,k_1}$. The local bijectivity of Keller maps implies that when $i\gg1$ there exist $p_{0i}=(x_{0i},y_{0i})$  and $p_{1i}=(x_{1i},y_{1i})$ in some neighborhoods of $p_0$ and $p_1$ respectively with $(p_{0i},p_{1i})\in V$ such that \equa{Take1111}{\mbox{$|x_{0i}|=k_{0i}$, \ \ \ $|x_{1i}|=k_{1i}$,}}
and $\lim_{i\to\infty}(p_{0i},p_{1i})=(p_0,p_1)$ by \eqref{1-Take1111}--\eqref{Take1111}. By \eqref{Take1111} and
by definition, $\g_{k_{0i},k_{1i}}\ge |y_{1i}|$. Therefore, \equa{Take31111}{\mbox{$\dis\g=\lim_{i\to\infty}\g_{k_{0i},k_{1i}}\ge\lim_{i\to\infty}|y_{1i}|=|y_1|=\g_{k_0,k_1}$.}} Now for each $i$, take $(p'_{0i},p'_{1i})=\big((x'_{0i},y'_{0i}),(x'_{1i},y'_{1i})\big)\in V$ such that
 \equa{Take21111}{\mbox{$|x'_{0i}|=k_{0i},$ \ \ \ $|x'_{1i}|=k_{1i}$ \ \ \ and \ \ \ $|y'_{1i}|=\g_{k_{0i},k_{1i}}$.}}
The sequence $(p'_{0i},p'_{1i})$, $i=1,2,...$ is bounded by \eqref{mqp1234-2} and the fact that $\{x'_{0i},x'_{1i}\,|\,i=1,2,...\}$ is bounded. Replacing the sequence by a subsequence if necessary, we may assume
$(p'_{0i},p'_{1i})$ converges to some $(p'_0,p'_1)=\big((x'_0,y'_0),(x'_1,y'_1)\big)\in\C^4$. As the proof above, we must have $p'_0\ne p'_1$. By \eqref{1-Take1111} and \eqref{Take21111}, $(p'_0,p'_1)\in A_{k_0,k_1}$ and $\g=\lim_{i\to\infty}\g_{k_{0i},k_{1i}}=\lim_{i\to\infty}|y'_{1i}|=|y_1'|\le\g_{k_0,k_1}$, where the inequality follows from the definition of $\g_{k_0,k_1}$. This together with  \eqref{Take31111} shows that $\g=\g_{k_0,k_1}$, therefore the function $\g_{k_0,k_1}$ is  continuous, i.e., we have Theorem \ref{AddLeeme--0}\,(ii).
}
%

Now we prove Theorem \ref{AddLeeme--0}\,(iii). We will prove \eqref{wePPPP1}\,(b) [the proof for \eqref{wePPPP1}\,(a) is similar, but simpler,
cf.~Remark \ref{Mamamam}]. %the remark in the paragraph before \eqref{@suchRethat=2-for}].
%By taking $\lim_{k_0\to0}$, we obtain \eqref{wePPPP1}\,(c).
%We prove that $\g_{k_0,k_1}$ is a strictly increasing function on $k_1$ if $k_0>0$ is fixed
First %from definition
%
%and \eqref{=simag},
we %have
claim that
%
\equa{ClM0}{\g_{0,0}>0%\ge4
.}
To see this, by definition, there exists $\big((0,\tilde y_0),(0,\tilde y_1)\big)\in A_{0,0}$ for some $\tilde y_0,\tilde y_1\in\C$ with $\tilde y_0\ne\tilde y_1$, thus also $\big((0,\tilde y_1),(0,\tilde y_0)\big)\in A_{0,0}$. By definition,
$\g_{0,0}\ge\max\{|\tilde y_0|,|\tilde y_1|\}>0$, i.e., we have \eqref{ClM0}.

Fix $k_0>0$.
For any given $k'_1>0$, let
\begin{eqnarray}\label{LetA=}
\b\!\!&=&\!\!\max\{\g_{k_0,k_1}\,|\,k_1\le k'_1\}\nonumber\\[4pt]
&=&\!\!\max\big\{\,|y_1|\ \big|\ (p_0,p_1)=
\big((x_0,y_0),(x_1,y_1)\big)\in V,\, |x_0|=k_0,\,|x_1|\le k'_1\,\big\}.\end{eqnarray}
 Assume
conversely that
there exists $k_1<k'_1$ with $\g_{k_0,k_1}=\b$.
We want
to use the local bijectivity of Keller maps  to obtain a contradiction.
Let $\ep>0$ be a parameter such that $\ep\to0$ [cf.~Convention \ref{conv1}\,(3)]. Let \equa{MSAMSMS}
{\mbox{$(\tilde p_0,\tilde p_1)=\big((\tilde x_0,\tilde y_0),(\tilde x_1,\tilde y_1)\big)\in V$ with $|\tilde x_0|=k_0,\,|\tilde x_1|=k_1$, $|\tilde y_1|=\b$.}}
Set
(and define $ G_0, G_1$ similarly)
\equa{-ANewf0F1++}{F_0= F(\tilde x_0+x,\tilde y_0+y),\
\ \ F_1= F(\tilde x_1+x,\tilde y_1+y).
}
Denote
\equa{-Aa0b0}{\tilde a_0=\Coeff(F_0,x^1y^0)
,\ \ \ \tilde b_0= \Coeff(F_0,x^0y^1)
,\ \ \ \tilde a=\Coeff(F_1,x^1y^0)
,\ \ \ \tilde b= \Coeff(F_1,x^0y^1)
.}
We use $\tilde c_0,\tilde d_0,\tilde c,\tilde d$  to denote the corresponding elements for $G_0,G_1$.
Then $A_0=(^{\tilde a_0}_{\tilde b_0}\ ^{\tilde c_0}_{\tilde d_0}), A=(^{\tilde a}_{\tilde b}\ ^{\tilde c}_{\tilde d})$ are invertible $2\times2$ matrices such that
${\rm det\,}A_0={\rm det\,}A=J(F,G)=1$. For the purpose of proving Theorem \ref{AddLeeme--0}\,(iii), we can replace $( F, G)$ by
$( F, G)A_0^{-1}$% for $i=0,1$
, then $A_0$ becomes
 $A_0=I_2$ (the $2\times2$ identity matrix), and
$AA^{-1}_0$ becomes the new $A$, which is in fact the following matrix,
\equa{-AA===}{A=\left(\begin{array}{cc}\tilde a&\tilde c\\[4pt]\tilde b&\tilde d\end{array}\right)=\left(\begin{array}{ll}F_x(\tilde p_1)G_y(\tilde p_0)-G_x(\tilde p_1)
F_y(\tilde p_0)&G_x(\tilde p_1)F_x(\tilde p_0)-F_x(\tilde p_1)G_x(\tilde p_0)\\[6pt]
F_y(\tilde p_1)G_y(\tilde p_0)-G_y(\tilde p_1)F_y(\tilde p_0)&G_y(\tilde p_1)F_x(\tilde p_0)-F_y(\tilde p_1)G_x(\tilde p_0)\end{array}\right),}
where the subscript ``$\,_{x}\,$'' (resp., ``$\,_{y}\,$'') stands for the partial derivative $\frac{\ptl}{\ptl x}$ (resp., $\frac{\ptl}{\ptl y}$), and $F,G$ are original $F,G$ [i.e., before replaced by $( F, G)A_0^{-1}\,$].
From this, for later use, we obtain the following: There exist $\SS_1\in\R_{>0}$ (depending on degrees and  coefficients of $F,G$)
such that for any $(\tilde p_1,\tilde p_2)\in V$, we have
\equa{bBbbb}{{\rm(i)\ }
|\tilde a|,|\tilde b|,|\tilde c|,|\tilde d|\le h_{_{\sc\tilde  p_0,\tilde  p_1}}^{^{\sc \SS_1}}
,\mbox{ \ \ (ii)\ either $|\tilde a|\ge
h_{_{\sc\tilde  p_0,\tilde  p_1}}^{^{\sc-\SS_1}}$ or $|\tilde b|\ge
h_{_{\sc \tilde p_0,\tilde  p_1}}^{^{\sc-\SS_1}}$,}
}
where (ii) follows from (i) and the fact that $\tilde a\tilde c-\tilde b\tilde d=1$.
We
can  write (here ``\,$\equiv$\,'' means equal modulo terms with degrees $\ge3$ by defining
$\deg\,x=\deg\,y=1$%; we remark that for the purpose of proving Theorem \ref{AddLeeme--0}\,(iii), we not need terms with degree two, however since the following arguments will be used more than once, we also write terms with degree two here]
)
\begin{eqnarray}
\label{4-1}&\!\!\!\!\!\!\!\!\!\!\!\!\!\!\!\!\!\!\!\!\!\!\!\!\!\!\!\!\!\!&
F_0\equiv \a_F+x+a_1x^2+a_2xy+a_3y^2
, \ \ \ \ \ F_1\equiv \a_F+\tilde ax+ \tilde b y+a_4x^2+a_5xy+a_6y^2
,\\[4pt]
\label{4-2}&\!\!\!\!\!\!\!\!\!\!\!\!\!\!\!\!\!\!\!\!\!\!\!\!\!\!\!\!\!\!&
G_0\equiv \a_G+y+a_7x^2+a_8xy+a_9y^2
, \  \ \ \, \, G_1\equiv  \a_G+\tilde cx+ \tilde d y+a_{10}x^2+a_{11}xy+a_{12}y^2,
\end{eqnarray}
for some $a_i\in\C$, where $\a_F= F(\tilde x_0,\tilde y_0),\,\a_G=G(\tilde x_0,\tilde y_0)$. For any $s,t,u,v\in\C$, denote \equa{q0q1}{
q_0:=(\dot x_0,\dot y_0)=(\tilde x_0+ s\ep,\tilde y_0+t\ep),\ \ q_1:=(\dot x_1, \dot y_1)=
(\tilde x_1
+u
\ep,\tilde y_1+v\ep).}
The local bijectivity of Keller maps says that for any $u,v\in\C$ (cf.~Remark \ref{u-vRema}),
there exist $s,t\in\C$
such that $(q_0,q_1)\in V$.
\begin{rema}\rm\label{u-vRema} When we consider the local bijectivity of Keller maps,
we always assume $u,v\in\C$ are bounded by some fixed $\SS\in\R_{>0}$ (which is independent of $\ep$, and we can assume $\ep$ as small as we wish, for instance $\ep<\SS^{-\SS^\SS}$).\end{rema}

In fact we can use \eqref{4-1} and \eqref{4-2} to solve $s,t
$ up to $\ep^2$; for instance,
\equa{st=}{s=s_0+O(\ep^2
), \ \ \ s_0= \tilde au+ \tilde bv+(\a_1u^2+\a_2uv+\a_3v^2)\ep,
}
for some $\a_i\in\C$.
We want to choose suitable $u,v$ such that
\begin{eqnarray}
\label{@suchthat=2-for}&\!\!\!\!\!\!\!\!\!\!\!\!\!\!\!\!\!\!\!\!\!\!\!\!&
{\rm (I)\ \ }|\dot x_0|=|\tilde x_0+s\ep|=|\tilde x_0|\ \,(\,=k_0>0\,){\ssc\,},\ \ \ {\rm (II)\ \ }|\dot y_1|=|\tilde y_1+v\ep|>|\tilde y_1|\ \,(\,=\b\,){\ssc\,}.\end{eqnarray}
If $\tilde a\ne0$ in \eqref{st=}, then we can easily first choose $v$ to satisfy (II), then choose $u$ to satisfy (I)
[cf.~\eqref{st=}; we can also regard $s$ as a free variable and solve $u=\tilde a^{-1}(s-\tilde bv)+O(\ep^1)$ from \eqref{st=}, and then using the fact that $x_0\ne0$ we can solve $s$ from \eqref{@suchthat=2-for}\,(I) as in  \eqref{SMSMDndndnd} below].
Since $|x_1|=k_1<k'_1$, we also have $|\dot x_1|<k'_1$ [since $\ep\to0$, cf.~\eqref{q0q1}${\ssc\,}$].
This means that we can choose $(q_0,q_1)\in V$ with $|\dot x_0|=k_0,\,|\dot x_1|<k'_1$, but $|\dot y_1|>\b$, which is a contradiction with the definition of $\b$ in \eqref{LetA=}.
Now assume $\tilde a=0$ (and so $\tilde b\ne0,\,\tilde c\ne0$). In this case the situation is more complicated.
\begin{rema}\rm\label{Mamamam}Before continuing, we remark that the proof of \eqref{wePPPP1}\,(a) %for the case $k_0>k_0,\,k'_1=k_1\ge0$
is easier: in that case condition \eqref{@suchthat=2-for}\,(I) should be replaced by the condition $|\tilde x_1+u\ep|=|\tilde x_1|$, which can be easily satisfied even in case $k_1=0$ (i.e., $\tilde x_1=0$). Thus \eqref{wePPPP1}\,(a) holds.\end{rema}

Now we continue our proof. Since $|\tilde x_0|=k_0>0$, and $|\tilde y_1|=\b\ge\g_{k_0,0}>\g_{0,0}>0%\ge4
$
[where the first inequality follows from
the definition of $\b$ in
\eqref{LetA=}, the second from
\eqref{wePPPP1}\,(a),
 and the last from \eqref{ClM0}${\ssc\,}$], we can rewrite
\eqref{@suchthat=2-for} as [cf.~\eqref{st=}${\ssc\,}$]
\begin{eqnarray}
\label{@suchRethat=2-for}&\!\!\!\!\!\!\!\!\!\!\!\!\!\!\!\!\!\!\!\!\!\!\!\!&
{\rm (I)'\  }|1+\hat s\ep|=1, \ \ {\rm (II)'\  }|1+\hat v\ep|>1,\mbox{ where }\hat v=\tilde y_1^{-1}v,\ \ \hat s=\tilde x_0^{-1}s,
\end{eqnarray}
and regard $\hat v$ as a new variable. Set
[see also arguments after \eqref{TaKa}${\ssc\,}$],
\equa{SetFtS}{\mbox{$\dis\hat F_0=\tilde x_0F_0(\tilde x_0^{-1}x,y),\ \ \hat G_0=G_0(\tilde x_0^{-1}x,y), \ \ \hat F_1=\tilde x_0F_1(x,\tilde y_1^{-1}y), \ \ \hat G_1=G_1(x,\tilde y_1^{-1}y),$}}
 and
 rewrite
[cf.~\eqref{4-1} and \eqref{4-2}; by subtracting $ F_i, G_i$ by some constants we may assume $\a_F=\a_G=0$; %by abusing notations,
we now %still
use $b,c,d$, which are %now
different from $\tilde b,\tilde c,\tilde d$ in \eqref{4-1} and \eqref{4-2}, to denote the coefficients of linear parts of $\hat F_1,\hat G_1$]
\begin{eqnarray}
\label{4-3}&\!\!\!\!\!\!\!\!\!\!\!\!\!\!\!\!\!\!\!\!\!\!\!\!\!\!\!\!\!\!&
\hat F_0=
\mbox{$\sum\limits_{i\ge2}a_iy^i+x\Big(1+\sum\limits_{i\ge2}\bar a_iy^i\Big)+\cdots, \ \ \hat F_1= \sum\limits_{i\ge2}b_iz^i+b y\Big(1+\sum\limits_{i\ge2}\bar b_iz^i\Big)+\cdots$},
\\[6pt]
\label{4-4}&\!\!\!\!\!\!\!\!\!\!\!\!\!\!\!\!\!\!\!\!\!\!\!\!\!\!\!\!\!\!&
\hat G_0=y+\mbox{$\sum\limits_{i\ge2}c_iy^i+\cdots,\ \ \ \ \ \,   \ \ \ \  \ \ \ \ \ \ \ \ \ \ \,\hat G_1=z+\sum\limits_{i\ge2}d_iz^i+\cdots
,\ \ \ \mbox{ \ where \ } z=cx+d y,$}
\end{eqnarray}
for some $a_i,\bar a_i,b_i,\bar b_i,c_i,d_i\in\C$,  and where we regard $\hat F_1,\hat G_1$ as polynomials on $y,z$ and we omit  terms with $x$-degree $\ge2$ in
$\hat F_0$ (or $\ge1$ in $\hat G_0$, and the likes for $\hat F_1,\hat G_1$), which will be irrelevant to our computations below.
In this case, by \eqref{4-3}, we have
\equa{NuUUU}{\hat s=b\hat v+O(\ep^1).}
If $b\im\ne0$ [cf.~Convention \ref{conv1}\,(1)], we can always choose suitable $\hat v\in\C$ with $\hat v\re>0$ such that both (I)$'$ and (II)$'$ in \eqref{@suchRethat=2-for} hold. Alteratively,  we can also regard $\hat s$ as a free variable [and solve $\hat v=b^{-1}\hat s+O(\ep^1)$ from \eqref{NuUUU}${\ssc\,}$] and determine $\hat s$ by solving $\hat s\re$  from
\eqref{@suchRethat=2-for}\,(I)$'$ to obtain \equa{SMSMDndndnd}{\mbox{$\dis\hat s\re=\frac{-1+(1-\hat s\im^2\ep^2)^{\frac12}}{2}=-\frac{\hat s\im^2\ep}{2}+O(\ep^3)$,}}
then choose $\hat s\im$ [with $(b^{-1}\hat s)\re>0$] to satisfy \eqref{@suchRethat=2-for}\,(II)$'$.

Now assume
$b\in\R_{\ne0}$.
We claim  \equa{Ciaiaia}{\mbox{$(a_i,c_i)\ne(b_i,d_i)$ for at least one $i\ge2$, }}
otherwise we would in particular obtain  (and the like for $ G$)
\begin{eqnarray}
\label{ObataDDD}&&\!\!\!\!\!\!\!\!\!\!\!\!\!\!\!\!\!\!\!\!\!\!\!\!
F(\tilde x_0,\tilde y_0+\kk)=\tilde x_0^{-1}\hat F_0\big|_{(x,y)=(0,\kk)}=\tilde x_0^{-1}\hat F_1\big|_{(x,y)=({\kk}{c^{-1}},0)}= F(\tilde x_1+{\kk}{c^{-1}},\tilde y_1), \mbox{ i.e.,}\\[4pt]
\label{1ObataDDD}&&\!\!\!\!\!\!\!\!\!\!\!\!\!\!\!\!\!\!\!\!\!\!\!\!
\si(\bar p_0)=\si(\bar p_1),\end{eqnarray}
with $\bar p_0=(\bar x_0,\bar y_0)=(\tilde x_0,\tilde y_0+\kk),\ \bar p_1=(\bar x_1,\bar y_1)=(\tilde x_1+{\kk}{c^{-1}},\tilde y_1)$
for
all $\kk\gg1$ [cf.~Convention \ref{conv1}\,(3)]. Then $h_{\bar p_0,\bar p_1}\sim\kk$ [when $\kk\gg1$, cf.~\eqref{MSM1111}${\ssc\,}$], and  $|\bar x_0+\bar y_0|\sim\kk\succ h_{_{\sc \bar p_0,\bar p_1}}^{^{\sc \frac{m}{m+1}}}$, a contradiction with  \eqref{mqp1234-2}.
Thus \eqref{Ciaiaia} holds. Then, if necessary by replacing $\hat  G_i$ by $\hat  G_i+\hat  F_i^2$ for $i=0,1$ (which does not change the linear parts of $\hat F_0,\hat F_1,\hat G_0,\hat G_1$), we may assume \equa{SASSUME}{\mbox{$c_{i_0}\ne d_{i_0}$ for some minimal $i_0\ge2$.}}
By replacing $\hat  F_j$ by $\hat  F_j+\sum_{i=2}^{2i_0}\b_i \hat G^{{\ssc\,}i}_j$ for some $\b_i\in\C$ and $j=0,1$, thanks to the term $y$ in $\hat G_0$, we can then suppose\equa{ai-=-0}{\mbox{ $a_i=0$ for $2\le i\le 2i_0$.}}
Now we need to consider two cases.\vskip4pt
\noindent{\bf Case 1}: {\it Assume $b_k\ne0$ for some $k\le 2i_0$.} Take minimal such $k\ge2$.
Setting [the second equation amounts to setting $z=w\ep$ in \eqref{4-4}${\ssc\,}$],\equa{[MAMMS]}{\hat v=\check v\ep^{k-1},\ \
\dis u=  c^{-1}{w}- c^{-1}{  d}{\ssc\,}\check v\ep^{k-1},}  and regarding $\check v,w$ as new variables,
we can then solve from \eqref{4-3} and \eqref{4-4}
[cf.~\eqref{st=}, \eqref{@suchRethat=2-for} and \eqref{NuUUU}; observe that all omitted terms and all coefficients $\bar a_i$'s, $\bar b_i$'s do not contribute to our solution of $\hat s$ up to $\ep^k$] to obtain,
for some nonzero $b'\in\C$,
\equa{s=a}{\hat s=( b\check v+b'w^k)\ep^{k-1}+O(\ep^k).}
Using this and the first equation of \eqref{[MAMMS]} in
\eqref{@suchRethat=2-for}, one can then easily see that \eqref{@suchthat=2-for} have solutions [by taking, for example, $\check v>0$ so that (II)$'$ holds and then
 choosing $w$ to satisfy (I)$'$].
\vskip4pt
\noindent{\bf Case 2}: {\it Assume $b_i=0$ for $0\le i\le 2i_0$.}
By computing the coefficients \equa{CsssII}{\mbox{$\dis\Coeff\big(J(F_0,G_0),x^0y^i\big)=0=\Coeff\big(J(F_1,G_1),x^0y^i\big)$, \ $i\ge1$,}} and induction on
$i$ for $1\le i<i_0$, one can easily obtain that $\bar a_i=\bar b_i$ for $i<i_0$ and $\bar a_{i_0}\ne\bar b_{i_0}$   by \eqref{SASSUME} and \eqref{ai-=-0}.
In this case, by setting [the first equation below means that $\check v\ep$ contributes a {positive} $O(\ep^{2i_0})$ element to
the left-hand side of \eqref{@suchRethat=2-for}\,(II)$'$ since it does not have a real part, in particular (II)$'$ holds], \equa{AMSMSM23333}{\dis \hat v=v_1\ii\ep^{i_0-1}\mbox{ \ for \ }v_1\in\R_{\ne0},
\ \ \ \ \ u=c^{-1}{w}-c^{-1}{d}v_1\ii\ep^{i_0-1},}
 we can then solve from \eqref{4-3} and \eqref{4-4} to obtain, for some nonzero $b''\in\C$
(all omitted terms do not contribute to our solution of $\hat s$ up to $\ep^{2i_0}$),
\equa{s=a+1}{\hat s\ep=bv_1\ii\ep^{i_0}+b''v_1\ii w^{i_0}\ep^{2i_0}+O(\ep^{2i_0+1}).}
Since $b\in\R_{\ne0},$ we see that
\eqref{s=a+1} can only contribute an $O(\ep^{2i_0})$ element to  \eqref{@suchRethat=2-for}\,(I)$'$.
Using \eqref{s=a+1} and the first equation of \eqref{AMSMSM23333} in
\eqref{@suchRethat=2-for},  one can again  see that \eqref{@suchthat=2-for} have solutions by choosing suitable $w$.
This proves
Theorem \ref{AddLeeme--0}.\hfill$\Box$\vskip7pt


\section{Proof of Theorem \ref{real00-inj}\,(1)}


%
To prove Theorem \ref{real00-inj}\,(1),
let us make the following assumption [cf.~Remark \ref{MARK1}\,(3)].
\begin{assu}\label{assu1111}
Assume Theorem {\rm\ref{real00-inj}\,(1)} is not true.
\end{assu}


Under this assumption, we have
\begin{lemm}\label{G--lemm-assum1--}For any $\d,k,k_0,k_1\in\R_{>0}$ with $k>1,\,\d<\frac1m$, we have $\g_{k^{1+\d}k_0,kk_1}<k\g_{k_0,k_1}$.
 \end{lemm}\noindent{\it Proof.~}%The proof is essentially the same as that of Lemma \ref{G--lemm-assum1--}.
Assume the result is not true, then by choosing $\d'$ with $\d<\d'<\frac1m$ and by Theorem \ref{AddLeeme--0}\,(iii), we may assume
$\g_{\bar k^{1+\d'}k_0,\bar kk_1}> \bar k\g_{k_0,k_1}$ for some $\bar k,k_0,k_1\in\R_{>0}$ with $\bar k>1.$
Thus we can choose a sufficiently small $\d_1\in\R_{>0}$ satisfying (the following holds when $\d_1=0$ thus also hold when $\d_1>0$ is sufficiently small)
\equa{GMSMSM0999}{\dis
\frac{k_1^{1+\d_1}\g_{\bar k^{1+\d'}k_0,\bar kk_1}}{(\bar kk_1)^{1+\d_1}}>
\g_{k_0,k_1}.}
Take $\kk\gg1$.
We define $V_0$ to be the subset of $V$ consisting of elements $(p_0,p_1)=\big((x_0,y_0),(x_1,y_1)\big)$ satisfying
[our aim is to design the following to satisfy
Theorem {\rm\ref{real00-inj}\,(1)}${\ssc\,}$]
\begin{eqnarray}
\label{GMSMSMSMS}\dis\!\!\!\!\!\!\!\!\!\!\!\!\!\!\!\!\!\!&&
{\rm(a)\ }k_1\le|x_1|\le \frac{k_1}{\,k_0^{\frac1{1+\d'}}\,}(1-\kk^{-3})|x_0|^{\frac1{1+\d'}}+ k_1(\kk^{-1}-\kk^{-2}+\kk^{-3})\le
(1-\kk^{-2})|x_1|+ k_1\kk^{-1},
\nonumber\\[4pt]
%\label{1MSMSMSMS}
\dis\!\!\!\!\!\!\!\!\!\!\!\!\!\!\!\!\!\!&&
{\rm(b)\ }\frac{|y_1|}{\,|x_1|^{1+\d_1}\,}\ge
\frac{\g_{\bar k^{1+\d'}k_0,\bar kk_1}}{(\bar kk_1)^{1+\d_1}}.
\end{eqnarray}
Then  we can rewrite the above as the form in \eqref{ToSayas},
obviously we have \eqref{-EiathA0} [note from the second and last inequalities of  \eqref{GMSMSMSMS}\,(a) that $|x_1|\le k_1\kk$].
Further,
by definition, there exists\equa{GEXiSTS}{\mbox{ $(\check p_0,\check p_1)=\big((\check x_0,\check y_0),(\check x_1,\check y_1)\big)\in V$ with
$|\check x_0|=\bar k^{1+\d'}k_0,\,|\check x_1|=\bar kk_1$, $|\check y_1|=\g_{\bar k^{1+\d'}k_0,\bar kk_1}$.}} Then one can easily see that $(\check p_0,\check p_1)\in V_0$, i.e., $V_0\ne\emptyset.$

Let $(p_0,p_1)\in V_0$.
If the first two equalities, or the first and last equalities, hold in \eqref{GMSMSMSMS}\,(a), then we obtain that $|x_1|=k_1,\,|x_0|\le k_0$,
but
by \eqref{GMSMSM0999} and \eqref{GMSMSMSMS}\,(b), $|y_1|>\g_{k_0,k_1}$, a contradiction with the definition of $\g_{k_0,k_1}$ and/or Theorem \ref{AddLeeme--0}\,(iii).

If  %the first and last equalities, or
the last two equalities hold in \eqref{GMSMSMSMS}\,(a), then one
obtains that $|x_1|\sim\kk,\,|x_0|\sim\kk^{1+\d'}$ [when $\kk\gg1$, cf.~\eqref{MSM1111}; see Remark \ref{rema-kk}${\ssc\,}$].
Note that
\eqref{mqp1234-2+}
in particular implies that either $h_{p_0,p_1}\sim|x_0|\sim|y_0|$ or $h_{p_0,p_1}\sim|x_1|\sim|y_1|$, in any case we obtain
%
%
%Thus \eqref{mqp1234-2+} implies
that $h_{p_0,p_1}\preceq\kk^{1+\d'}$.
By \eqref{GMSMSMSMS}\,(b), we have
$|y_1|\succeq|x_1|^{1+\d_1}\sim \kk^{1+\d_1}$, and so
(where the part ``\,$\succ$\,'' follows by noting from $\d'<\frac1m$ that $|y_1|\succeq\kk^{1+\d_1}\succ\kk\succeq \kk^{\frac{(1+\d')m}{m+1}}\ssc\,$)
\equa{AnDSo}{|x_1+y_1|\ge|y_1|-|x_1|\sim|y_1|\succ \kk^{\frac{(1+\d')m}{m+1}}\succeq h_{_{\sc p_0,p_1}}^{^{\sc\frac{m}{m+1}}},}
 a contradiction with
\eqref{mqp1234-2}. This shows that Theorem
{\rm\ref{real00-inj}\,(1)} holds, a contradiction with Assumption \ref{assu1111}.
The lemma is proven.\hfill$\Box$\vskip5pt

\NOUSE{%
%
Under this assumption, we have
\begin{lemm}\label{G--lemm-assum1--}For any $k,k_0,k_1\in\R_{>0}$ with $k>1$, we have $\g_{kk_0,kk_1}\le k\g_{k_0,k_1}$.
 \end{lemm}\noindent{\it Proof.~}Assume conversely that
$\g_{\bar kk_0,\bar kk_1}> \bar k\g_{k_0,k_1}$ for some $\bar k,k_0,k_1\in\R_{>0}$ with $\bar k>1.$
Then we can choose a sufficiently small $\d\in\R_{>0}$ satisfying (the following holds when $\d=0$ thus also hold when $\d>0$ is sufficiently small)
\equa{MSMSM0999}{\dis
\frac{k_1^{1+\d}\g_{\bar kk_0,\bar kk_1}}{(\bar kk_1)^{1+\d}}>
\g_{k_0,k_1}.}
Take $\kk\gg1$.
We define $V_0$ to be the subset of $V$ consisting of elements $(p_0,p_1)=\big((x_0,y_0),(x_1,y_1)\big)$ satisfying
[our aim is to design the following to satisfy
Theorem {\rm\ref{real00-inj}\,(1)}${\ssc\,}$]
\begin{eqnarray}
\label{MSMSMSMS}\dis\!\!\!\!\!\!\!\!\!\!\!\!\!\!\!\!\!\!&&
{\rm(a)\ }k_1\le|x_1|\le \frac{k_1}{k_0}(1-\kk^{-3})|x_0|+ k_1(\kk^{-1}-\kk^{-2}+\kk^{-3})\le
(1-\kk^{-2})|x_1|+ k_1\kk^{-1},
\nonumber\\[4pt]
%\label{1MSMSMSMS}
\dis\!\!\!\!\!\!\!\!\!\!\!\!\!\!\!\!\!\!&&
{\rm(b)\ }\frac{|y_1|}{|x_1|^{1+\d}}\ge
\frac{\g_{\bar kk_0,\bar kk_1}}{(\bar kk_1)^{1+\d}}.
\end{eqnarray}
Then  we can rewrite the above as the form in \eqref{ToSayas},
obviously we have \eqref{-EiathA0} [note from the second and last inequalities of  \eqref{MSMSMSMS}\,(a) that $|x_1|\le k_1\kk$].
Further,
by definition, there exists\equa{EXiSTS}{\mbox{ $(\check p_0,\check p_1)=\big((\check x_0,\check y_0),(\check x_1,\check y_1)\big)\in V$ with
$|\check x_0|=\bar kk_0,\,|\check x_1|=\bar kk_1$ and $|\check y_1|=\g_{\bar kk_0,\bar kk_1}$.}} Then one can easily see that $(\check p_0,\check p_1)\in V_0$, i.e., $V_0\ne\emptyset.$
%
Let $(p_0,p_1)\in V_0$.
If the first two equalities, or the first and last equalities, hold in \eqref{MSMSMSMS}\,(a), then we obtain that $|x_1|=k_1,\,|x_0|\le k_0$,
but
by \eqref{MSMSM0999} and \eqref{MSMSMSMS}\,(b), $|y_1|>\g_{k_0,k_1}$, a contradiction with the definition of $\g_{k_0,k_1}$ and/or Theorem \ref{AddLeeme--0}\,(iii).
If  %the first and last equalities, or
the last two equalities hold in \eqref{MSMSMSMS}\,(a), then one
obtains that $|x_0|\sim|x_1|\sim\kk$ [when $\kk\gg0$, cf.~\eqref{MSM1111}${\ssc\,}$],
and by \eqref{MSMSMSMS}\,(b),
\equa{ANdBBBB}{\mbox{$\dis|y_1|\succeq|x_1|^{1+\d}\sim k_1\kk^{1+\d}\succ\max\{| x_0|,|x_1|\}$.}}
Note that
\eqref{mqp1234-2+}
in particular implies that either $h_{p_0,p_1}\sim|x_0|\sim|y_0|$ or $h_{p_0,p_1}\sim|x_1|\sim|y_1|$, in any case we have
$|x_1+y_1|\ge|y_1|-|x_1|\sim|y_1|\succ h_{_{\sc p_0,p_1}}^{^{\sc\frac{m}{m+1}}}$, a contradiction with \eqref{mqp1234-2}.  This shows that Theorem
{\rm\ref{real00-inj}\,(1)} holds, a contradiction with Assumption \ref{assu1111}.
The lemma is proven.\hfill$\Box$
\begin{lemm}\label{Anlemmmmmm}For any $k_0,k_1\in\R_{>0}$, we have $\g_{k_0,k_1}\ge\TH  k_1.$
\end{lemm}\noindent{\it Proof.~}Assume \equa{Ass1}{\mbox{$\g_{k_0,k_1}<\TH k_1$ for some $k_0,k_1>0$.}} Denote $\a=\frac{\gamma_{k_0,k_1}}{k_1}<1\TH $. By
\eqref{Ass1} and Lemma \ref{G--lemm-assum1--}, we have
$\g_{\kk k_0,\kk k_1}\le \kk\g_{k_0,k_1}=\kk k_1\a$ for all $\kk\gg1$. Let
$(p_0,p_1)\in V$ with $|x_0|=\kk k_0$, $|x_1|=\kk k_1$, $|y_1|=\g_{\kk k_0,\kk k_1}\le\kk k_1\a$.
Then as in the proof of the previous lemma, we have $h_{p_0,p_1}\sim\kk$, but then
 $|\TH x_1+y_1|\ge|\TH x_1|-|y_1|\ge(
 1%\TH
 -\a)\kk k_1>\tau h_{p_0,p_1}^{\frac{m}{m+1}}$, which is a contradiction with
\eqref{mqp1234-2}. This proves Lemma \ref{Anlemmmmmm}.\hfill$\Box$
\vskip7pt
%
}%
%
\begin{rema}\label{rema-kk}\rm Note that when we design the system \eqref{GMSMSMSMS}, $\kk$ is simply some fixed positive real number. When we say $\kk\gg1$, it  means that we may need to choose sufficiently large $\kk$ such that the system \eqref{GMSMSMSMS} can satisfy our requirement. This will also apply to some similar situations later.
\end{rema}
\begin{lemm}\label{Anlemmmmmm}For any $k_0,k_1\in\R_{>0}$, we have $\g_{k_0,k_1}>\TH  k_1.$
\end{lemm}\noindent{\it Proof.~}Assume the result is not true, then by choosing $k'_0\in\R_{>0}$ with $k'_0<k_0$, we may assume \equa{Ass1}{\mbox{$\g_{k'_0,k_1}<\TH k_1$ for some $k'_0,k_1>0$.}} Denote $\a=\frac{\gamma_{k'_0,k_1}}{k_1}<1\TH $. By
\eqref{Ass1} and Lemma \ref{G--lemm-assum1--}, we have
$\g_{\kk k'_0,\kk k_1}< \kk\g_{k'_0,k_1}=\kk k_1\a$ for all $\kk\gg1$. Let
$(p_0,p_1)\in V$ with $|x_0|=\kk k'_0$, $|x_1|=\kk k_1$, $|y_1|=\g_{\kk k'_0,\kk k_1}\le\kk k_1\a$.
Then as in the proof of the previous lemma, we have $h_{p_0,p_1}\sim\kk$ when $\kk\gg1$, but then
\equa{ButTherne}{|\TH x_1+y_1|\ge|\TH x_1|-|y_1|\ge(
 1%\TH
 -\a)\kk k_1>\tau h_{_{\sc p_0,p_1}}^{^{\sc\frac{m}{m+1}}},} which is a contradiction with
\eqref{mqp1234-2}. This proves Lemma \ref{Anlemmmmmm}.\hfill$\Box$
\vskip7pt

%
%\begin{lemm}\label{ThwHold}Theorem {\rm\ref{real00-inj}\,(1)} holds.
%\end{lemm}\noindent{\it Proof.~}
 % and fix a sufficiently small $\d>0$.
%We define $V_0$ to be the subset of $V$ consisting of elements $(p_0,p_1)=\big((x_0,y_0),(x_1,y_1)\big)$ satisfying
%\begin{eqnarray}
%\label{11111NSoOP}&&\!\!\!\!\!\!\!\!\!\!\!\!\!\!\!\!\!\!\!\!\!\!\!\!\!\!\!\!
%\dis{\rm(i)\ }
%(1-\d)\kk\le|x_1|\le\kk\g_{\kk,\kk}^{-2}|y_1|^2\le
%(1+\d)\kk,
%\\[6pt]
%\label{11111NewAdded-NSoOP}&&\!\!\!\!\!\!\!\!\!\!\!\!\!\!\!\!\!\!\!\!\!\!\!\!\!\!\!\!
% \ \ {\rm(ii)\ }
%\frac{|x_1|^2+\kk}{|x_0|^2+\kk^{-\kk}}\ge\frac{\kk^2+\kk}{\kk^2+\kk^{-\kk}}.\end{eqnarray}
%Then we have \eqref{ToSayas0}. Further, by definition there exists


Let $\kk\gg1$.
We take \begin{eqnarray}\label{TaKa}
&\!\!\!\!\!\!\!\!\!\!\!\!\!\!&
(\bar p_0,\bar p_1)=\big((\bar x_0,\bar y_0),(\bar x_1,\bar y_1)\big)\in A_{\kk,\kk}\mbox{ with }%\nonumber\\[4pt]
%&\!\!\!\!\!\!\!\!\!\!\!\!\!\!&
|\bar x_0|=%1,\ \
|\bar x_1|=\kk
,\
|\bar y_1|=\g_{\kk,\kk}>%\g_{1,\kk}\!\ge
\kk,\end{eqnarray}
 where the inequalities follow from %Theorem \ref{AddLeeme--0}\,(iii) and %\eqref{ClM0}.
Lemma \ref{Anlemmmmmm}.
%Then one can verify that $(p_0,p_1)\in V_0$, i.e., $V_0\ne\emptyset$.
%First note that $\g_{\kk,\kk}<(1+\d^2)\kk$, otherwise by \eqref{mqp1234-2+} we have either $h_{\bar p_0,\bar p_1}\sim|x_0|\sim\kk$ or $h_{\bar p_0,\bar p_1}\sim|x_1|\sim\kk$ and we would have $|\bar x_1+\bar y_1|\ge|\bar y_1|-|\bar x_1|\sim|\bar y_1|\succeq\kk\succ\kk^{\frac{m}{m+1}}\sim
%h_{_{\sc\bar p_0,\bar p_1}}^{^{\sc\frac{m}{m+1}}}$, a contradiction with \eqref{mqp1234-2}.
%For any $(p_0,p_1)=\big((x_0,y_0),(x_1,y_1)\big)\in V_0$, we always have $|x_0|\preceq\kk$. If the first two equalities, or the first and last equalities, hold in
%\eqref{11111NSoOP}\,(i), then $|x_1|=(1-\d)\kk,\,|y_1|\ge(1-\d)^{\frac12}\kk$, and as above we have $|x_1+y_1|\ge|y_1|-|x_1|\sim\kk\succ h_{_{\sc p_0, p_1}}^{^{\sc\frac{m}{m+1}}}$, a contradiction with \eqref{mqp1234-2}.
%If  the last two equalities hold in \eqref{11111NSoOP}\,(i), then $|x_1|=(1+\d)\kk,\,|y_1|=(1+\d)^{\frac12}\g_{\kk,\kk}<(1+\d)^{\frac12}(1+\d^2)\kk$, and
%%by \eqref{11111NSoOP}\,(ii), $|x_0|\preceq\kk$. Thus we must have $h_{p_0,p_1}\sim\kk$ by \eqref{mqp1234-2+}, but then
%we have $|x_1+y_1|\ge|x_1|-|y_1|\sim\kk\succ h_{_{\sc p_0,p_1}}^{^{\sc\frac{m}{m+1}}}$, a contradiction with \eqref{mqp1234-2+}. Thus Theorem {\rm\ref{real00-inj}\,(1)} holds, which is a contradiction with Assumption \ref{assu1111}. This shows that Assumption \ref{assu1111} must be wrong, i.e., we have the lemma.\hfill$\Box$\vskip7pt
%
%
%
{%
Similar to \eqref{-ANewf0F1++} (but not exactly), we define
\equa{Ne2wf0F1++}{F_0= F\big(\bar x_0(1+x),\bar y_0+y\big),\
\ \ F_1= F\big(\bar x_1(1+x),\bar y_1(1+y)\big),
}
and define $ G_0, G_1$ similarly [thus the matrices $A_0,A$ defined after
\eqref{-Aa0b0} now have determinant ${\rm det\,}A_0=\bar x_0 J(F,G)\ne0$, ${\rm det\,}A=\bar x_1\bar y_1 J(F,G)\ne0$, and again by replacing $( F_i, G_i)$ by $( F_i, G_i)A_0^{-1}$ for $i=0,1$, we can assume $A_0=I_2$].
Define $q_0,q_1$ accordingly [similar to, but a slightly different from, \eqref{q0q1}, simply due to the different definitions in \eqref{Ne2wf0F1++} and \eqref{-ANewf0F1++}; note that $\ep$ may depend on $\kk$: in general the larger $\kk$ is, the smaller $\ep$; but any in case once $\kk$ is chosen we can always choose sufficiently small $\ep$, cf.~also Remark \ref{u-vRema}],
\equa{1+++q0q1}{
q_0:=(\dot x_0,\dot y_0)=\big(\bar x_0(1+ s\ep),\bar y_0+t\ep\big),\ \ q_1:=(\dot x_1, \dot y_1)=
\big(\bar x_1(1
+u
\ep),\bar y_1(1+v\ep)\big).}
In particular, we have \eqref{4-1}, \eqref{4-2},
and as in \eqref{st=},
\begin{eqnarray}
\label{@suc2hthat=4}&\!\!\!\!\!\!\!\!\!\!\!\!\!\!\!\!\!\!\!\!\!\!\!\!&
s=-a_\kk u+b_\kk v+O(\ep^1),
\end{eqnarray}
where we have written the coefficients of $u,v$
as $-a_\kk,b_\kk$ to emphasis that they may depend on $\kk$ and that they
are in fact positive %indicate that it is in fact negative
as shown in the next lemma. First to avoid confusion we remark that
if we still use $\tilde a,\tilde b$ to denote the elements
defined in \eqref{-AA===} [with $(\tilde p_0,\tilde p_1)$ replaced by $(\bar p_0,\bar p_1)$], then
$-a_\kk,b_\kk$ in \eqref{@suc2hthat=4} are not $\tilde a,\tilde b$  simply because
of the different definition of \eqref{Ne2wf0F1++} from \eqref{-ANewf0F1++}; in fact, we have the following relations:
\equa{akk-bkk}{a_\kk=-\bar x_0^{-1}\tilde a\bar x_1,\ \ \ \ b_\kk=\bar x_0^{-1}\tilde b\bar y_1.}
Therefore, by \eqref{bBbbb} there exists some fixed $\SS_0\in\R_{>0}$ such that
\equa{akk-bkk+}{|a_\kk|,|b_\kk|\le\kk^{\SS_0}.}


 % (note that $a',b$ may depend on $\kk$).
\begin{lemm}\label{anLemmA}We have $a_\kk>0,\,b_\kk>0$.\end{lemm}
\noindent{\it Proof.~}First assume
$a_{\kk\,\rm im}\ne0$ or $a_{\kk\,\rm re}<0$ or  $b_{\kk\,\rm im}\ne0$ or $b_{\kk\,\rm re}<0$ [cf.~Convention \ref{conv1}\,(1)]. Then from \eqref{@suc2hthat=4}
one can easily choose $u,v$ [with $u\im\ne0$, $u\re<0$, $v\im\ne0$, $v\re>0$ such that either $(a_\kk u)\re>0$ or
$(b_\kk v)\re<0$, and so $s\re<0$] satisfying [cf.~\eqref{1+++q0q1} and \eqref{@suc2hthat=4}${\ssc\,}$]
\begin{eqnarray}
\label{suchThaT}&\!\!\!\!\!\!\!\!\!\!\!\!\!\!\!\!\!\!\!\!\!\!\!\!\!\!\!\!\!\!\!\!&
0<k_0:=|\dot x_0|=\kk|1+s\ep|<\kk,\ 0<k_1:=|\dot x_1|=\kk|1+u\ep|<\kk,
\nonumber\\[4pt]
&\!\!\!\!\!\!\!\!\!\!\!\!\!\!\!\!&
|\dot y_1|=\g_{\kk,\kk}|1+v\ep|>\g_{\kk,\kk},\end{eqnarray}
i.e., $0<k_0<\kk,\,0<k_1<\kk $ with $\g_{k_0,k_1}\ge |\dot y_1|>\g_{\kk,\kk }$,
a contradiction with Theorem \ref{AddLeeme--0}\,(iii). Thus $a_\kk \ge0$, $b_\kk \ge0$.
%
If $a_\kk =0$,
similar to arguments after \eqref{@suchRethat=2-for} [see also arguments after \eqref{2DDDbar1}${\ssc\,}$],
we have two possible cases [cf.~\eqref{[MAMMS]}, \eqref{s=a} and \eqref{AMSMSM23333}, \eqref{s=a+1}${\ssc\,}$]:
\begin{eqnarray}
&\!\!\!\!\!\!\!\!\!\!\!\!\!\!\!\!\!\!\!&\label{1-MAMMS}
v=\hat v\ep^{k-1},\ \ \ \ \
\dis u=  c^{-1}{w}- c^{-1}{  d}{\ssc\,}\hat v\ep^{k-1},\ \ \ \ \
s=( b_\kk\hat v+b'w^k)\ep^{k-1}+O(\ep^k),\\[4pt]
&\!\!\!\!\!\!\!\!\!\!\!\!\!\!\!\!\!\!\!\!&\label{1-AMSMSM23333}
\dis v=v_1\ii\ep^{i_0-1},
\ \ u=c^{-1}{w}-c^{-1}{d}v_1\ii\ep^{i_0-1},\ \
s\ep=b_\kk v_1\ii\ep^{i_0}+b''v_1\ii w^{i_0}\ep^{2i_0}+O(\ep^{2i_0+1}),
\end{eqnarray}
where $\hat v,w\in\C,\,v_1\in\R_{\ne0}$, $b',b''\in\C_{\ne0}$, $k,i_0\in\Z_{\ge2}$. Assume we have the case \eqref{1-MAMMS} [the case
\eqref{1-AMSMSM23333} is similar], we can first choose $\hat v$ with $\hat v\re>0$ so that the last inequation of \eqref{suchThaT} holds, then choose $w$ with $(c^{-1}w)\re<-1$ (sufficiently smaller than $-1$) and $(b'w^k)\re<0$ (sufficiently smaller than $-1$, such $w$ can be always chosen since $k\ge2$) such that the first two inequations of \eqref{suchThaT} hold. Thus \eqref{suchThaT} holds, and as before we obtain a contradiction. Therefore $a_\kk >0$. Similarly $b_\kk >0$.
The lemma is proven.\hfill$\Box$

\begin{lemm}\label{YYYy1==}For any fixed $\d\in\R_{>0}$, we have $\kk<\g_{\kk,\kk}<(1+\d^4)\kk.$
\end{lemm}\noindent{\it Proof.~}Otherwise we would then obtain that $|\bar x_1+\bar y_1|\ge|\bar y_1|-|\bar x_1|\ge\d^4\kk\succ\kk^{\frac{m}{m+1}}\sim h_{_{\sc \bar p_0,\bar p_1}}^{^{\sc\frac{m}{m+1}}}$ (when $\kk\gg1$), a contradiction with \eqref{mqp1234-2}.\hfill$\Box$

\begin{lemm}\label{NeMore-}
%Theorem {\rm\ref{real00-inj}\,(1)} holds.
For any fixed $\d\in\R_{>0}$ with $\d<\frac1m$, we have $b_\kk\ge1+\d+a_\kk$ for all $\kk>0$.
%For any $\l>1$ and $n_1\in\Z$, whenever $\kk\gg1$, we have
%\equa{WehHAH}{\g_{\kk^{\l_0\l},\kk^{-\kk+n_1}}\ge\TH \kk^{\l}.}
\end{lemm}\noindent{\it Proof.~}%Assume conversely that \eqref{WehHAH} is not true.
%Let $\kk\gg1$,
Assume the lemma does not hold, then we can choose sufficiently small $\d_1>0$ (which can depend on $\kk$)
such that
\equa{NSNENNENE}{(1+\d_1)b_\kk<1+\d-\d_1+a_\kk.}
Let  $\ell\gg\kk$ (we can assume $\ep<\ell^{-\ell}$, cf.~Remark \ref{u-vRema}). %and $\d=\kk^{-1}$
 %Fix $\TH,\d,\d_0,\ell\in\R_{>0}$ such that $\TH\l_0\l<1$, and $\d,\d_0$ are sufficiently small, and
%$\ell$ is sufficiently large.
We define $V_0$ to be the subset of $V$ consisting of elements $(p_0,p_1)=\big((x_0,y_0),(x_1,y_1)\big)$ satisfying
[again our purpose is to design the following to satisfy
Theorem {\rm\ref{real00-inj}\,(1)}${\ssc\,}$]
\begin{eqnarray}
\label{NSoOP}&&\!\!\!\!\!\!\!\!\!\!\!\!\!\!\!\!\!\!\!\!\!\!\!\!\!\!\!\!
\dis{\rm(i)\ }
\kk\le|x_1|\le(1-\ell^{-3})\kk^{\frac{\d}{1+\d}}|x_0|^{\frac1{1+\d}}+\kk(\ell^{-1}-\ell^{-2}+\ell^{-3})\le
(1-\ell^{-2})|x_1|+\kk\ell^{-1},
%1\le|x_1|\le|x_0|\le2|x_1|^{1-\kk^{-1}},
\nonumber\\[6pt]
%\label{NewAdded-NSoOP}
&&\!\!\!\!\!\!\!\!\!\!\!\!\!\!\!\!\!\!\!\!\!\!\!\!\!\!\!\!
{\rm(ii)\ }
\frac{|y_1|}{|x_1|^{1+\d_1}}\ge\frac{\g_{\kk,\kk}}{\kk^{1+\d_1}}
(1+\ep^2).
%|y_1|^{\frac12}-\kk^{\kk}|x_0|\ge\g_{1,1}^{\frac12}+\ep^2-\kk^\kk.
\end{eqnarray}
%where $\a_1=\a_2-k^{-\ell}$, $\a_2=k^{\TH\l_0\l}-k^{\TH\l_0\l(1-\d)}$.
%If we rewrite the above as the form in
Then we have \eqref{ToSayas} and \eqref{-EiathA0} %with $\l=0$
[using the second and last inequalities of \eqref{NSoOP}\,(i)]%, then %$\mu=0$ and %$(t_1,t_2,t_3)=(0,1,0)$ and
%we have \eqref{-EiathA}% and \eqref{EiathA}\,(c)
.

\begin{rema}\label{Fimsmsm}\rm
%We regard $\kk$ and $\ep$ as fixed [
Recall from statements inside the bracket before \eqref{1+++q0q1} and Remark \ref{u-vRema}
that when $\kk$ is fixed, $\ep$ can be fixed, and we can assume $\ep<\kk^{-\kk}$.
We here emphasis that the $\ep$ used in the above design of  the system of inequations in \eqref{LetNSoOP} is exactly the same as that used in the local bijectivity of Keller maps in
\eqref{1+++q0q1}. There is no any problem in doing this since our design does not need to use  the local bijectivity of Keller maps, we only use the local bijectivity of Keller maps to show
that the set $V_0$ is nonempty [in the sense of defining the system \eqref{NSoOP}, $\ell,\kk,\ep$ are simply some chosen positive real numbers, cf.~Remark \ref{rema-kk}$\ssc\,$].
\end{rema}

For any $(p_0,p_1)\in V_0$, if the first two equalities, or the first and last equalities, hold in \eqref{NSoOP}\,(i), then $|x_1|=\kk,\,|x_0|\le\kk$, but by
\eqref{NSoOP}\,(ii), $|y_1|>\g_{\kk,\kk}$, a contradiction with the definition of $\g_{1,1}$ and/or Theorem \ref{AddLeeme--0}\,(iii).

If the %first and last equalities, or the
last two equalities
hold in \eqref{NSoOP}\,(i),
then we can  obtain $|x_1|\sim\ell$ (when $\ell\gg\kk$ and $\kk$ is regarded as fixed; cf.~Remarks
 \ref{rema-kk} and \ref{Fimsmsm}$\ssc\,$), $|x_0|\preceq\ell^{1+\d}$, but by
%\equa{F1rst}{\mbox{$|x_1|=\kk^{-\kk+n_1-1}<\kk^{-\kk+n_1},$ \ \  $|x_0|\le\kk^{\l_0\l}$,}}
%where the second inequation follows by noting from \eqref{NSoOP} that since $|x_0|$ has the negative power $-\frac1{(\l_0-1)\l}$, when $|x_1|$ is reduced by the factor $\kk^{-1}$, the $|x_0|$ can be at most multiplied by a factor $\kk^{(\l_0-1)\l}(1+\a)$ for some $\a=O(\kk^{-1})$.
%By Theorem \ref{AddLeeme--0}\,(iii) and definition \eqref{Ak=1}, we obtain
%\equa{F2rst}{\mbox{$\g_{\kk^{\l_0\l},\kk^{-\kk+n_1}}>\g_{|x_0|,|x_1|}\ge|y_1|>|x_1|^{-\l}\kk^{\l(-\kk+n_1)}\g_{1,\kk^{-\kk+n_1}}>\kk^\l$,}}
%where the last two inequalities follow from \eqref{TaKa} and
%\eqref{NewAdded-NSoOP}, i.e., we have \eqref{WehHAH} (and the proof is completed in this case).
%a contradiction with the assumption at the beginning of the proof.
\eqref{NSoOP}\,(ii), %\eqref{NewAdded-NSoOP}, $|y_1|>\g_{\kk,\kk}$,
%a contradiction with the definition of $\g_{\kk,\kk}$ and/or Theorem \ref{AddLeeme--0}\,(iii).
%If the last two  equalities hold in \eqref{NSoOP}, then $|x_1|=\kk\ell$, $|x_0|\sim\ell^{1+\d}$ (when $\ell\gg\kk$ and $\kk$ is regarded as fixed), but by
%\eqref{NewAdded-NSoOP}, $|y_1|>\a\ell$, where $\a=\frac{\gamma_{\kk,\kk}}{\kk}>1$ by \eqref{TaKa}.
$|y_1|\succeq\ell^{1+\d_1}\succ|x_1|$. Again by \eqref{mqp1234-2+}, we must have either $h_{p_0,p_1}\sim|x_0|\sim|y_0|$ or $h_{p_0,p_1}\sim|x_1|\sim|y_1|$, in any case we have $h_{p_0,p_1}\preceq\ell^{1+\d}<\ell^{1+\frac1m}.$
But then $|x_1+y_1|\ge|y_1|-|x_1|\sim|y_1|\succ h_{_{\sc p_0,p_1}}^{^{\sc\frac{m}{m+1}}}$, a contradiction with
\eqref{mqp1234-2}.
%$h_{p_0,p_1}\sim|x_0|\prec|y_1|\preceq h_{p_0,p_1}$ [where the part $\preceq$ is obtained by definition \eqref{mqp1234}], a contradiction.
%
%As in the proof of Lemma \ref{G--lemm-assum1--}, we have $h_{p_0,p_1}\preceq\ell^{1+\d}\prec\ell^{\frac{m+1}{m}}$, and then
%$|x_1+y_1|\ge|y_1|-|x_1|\ge(\a-1)\ell\sim\ell\succ h_{_{\sc p_0,p_1}}^{^{\sc\frac{m}{m+1}}}$,  a contradiction with \eqref{mqp1234-2}.
%, a contradiction with definition \eqref{Ak=1}.
%$|x_1|=\d^{-1}=\kk$, and $|x_0|=(\frac{\kk-\d+\d^2-\d^3}{1-\d^3})^{\l_0\l}<\kk^{\l_0\l}$, and by
%\eqref{NSoOP}\,(ii), $|y_1|>\g_{1,1}\kk^\l>\TH\kk^\l.$
%In particular, by Theorem \ref{AddLeeme--0}\,(iii), we have $\g_{\kk^{\l_0\l},\kk^{-\kk}}\ge\g_{|x_0|,|x_1|}\ge|y_1|>\kk^\l$,
%
Hence %Therefore %the last two inequalities cannot simultaneously hold in \eqref{NSoOP}f\,(i), i.e.,
 Theorem \ref{real00-inj}\,(1)\,(ii) holds.
%

Next, we want to choose suitable $u,v$ such that
\eqref{NSoOP} holds for $(q_0,q_1)$ [defined in \eqref{1+++q0q1}${\ssc\,}$], i.e.,
\begin{eqnarray}
\label{1NSoOP}&&\!\!\!\!\!\!\!\!\!\!\!\!\!\!\!\!\!\!
\dis{\rm(i)\ }
1\le|1+u\ep|\le(1-\ell^{-3})|1+s\ep|^{\frac1{1+\d}}+\ell^{-1}-\ell^{-2}+\ell^{-3}\le
(1-\ell^{-2})|1+u\ep|+\ell^{-1},
%1\le|x_1|\le|x_0|\le2|x_1|^{1-\kk^{-1}},
\nonumber\\[5pt]
%\label{MAMAM1NSoOP}
&&\!\!\!\!\!\!\!\!\!\!\!\!\!\!\!\!\!\!
   {\rm(ii)\ }
\frac{|1+v\ep|}{|1+u\ep|^{1+\d_1}}\ge1+\ep^2.
\end{eqnarray}
The second strict inequality automatically holds in \eqref{1NSoOP}\,(i) (we can assume $\ep<\ell^{-\ell}$, cf.~Remark \ref{u-vRema}).
If we take
\equa{LetF3rst}{\mbox{$\dis u=1,\ \ \ v=\frac{a_\kk+1+\d-\d_1}{b_\kk},$
\ \ and so $\dis s=1+\d-\d_1+O(\ep^1)$ \ by \eqref{@suc2hthat=4},}}  by comparing the coefficients of $\ep^1$, we see that
all strict inequalities hold in \eqref{1NSoOP}\,(i). Further, the coefficient of $\ep^1$ in the left hand-side of \eqref{1NSoOP}\,(ii) is
$\frac{a_\kk+1+\d-\d_1}{b_\kk}-(1+\d_1)>0$ by \eqref{NSNENNENE}.
%which is obviously positive (since $\g_{1,1}>0$ and $\kk\gg1$).
We see that $(q_0,q_1)\in V_0$, i.e., $V_0\ne\emptyset,$
a contradiction with Assumption \ref{assu1111},
%namely, %. Thus the assumption  at the beginning of the proof is wrong, i.e.,
We have Lemma \ref{NeMore-}.
\hfill$\Box$
%\vskip7pt
%
%
%

\begin{lemm}\label{NeMMMMore-}
For any fixed $\d\in\R_{>0}$, we have $(1-\d^5)b_\kk\le1+a_\kk$ for all $\kk\gg1$.
%For any $\l>1$ and $n_1\in\Z$, whenever $\kk\gg1$, we have
%\equa{WehHAH}{\g_{\kk^{\l_0\l},\kk^{-\kk+n_1}}\ge\TH \kk^{\l}.}
\end{lemm}\noindent{\it Proof.~}%Assume conversely that \eqref{WehHAH} is not true.
Let $\kk\gg1$ and we  assume $\ep<\kk^{-\kk}$ (cf.~Remark \ref{u-vRema}). Define $V_0$ to be the subset of $V$ consisting of elements $(p_0,p_1)=\big((x_0,y_0),(x_1,y_1)\big)$ satisfying [again our purpose is to design the following to satisfy
Theorem {\rm\ref{real00-inj}\,(1)}; cf.~Remarks \ref{rema-kk} and \ref{Fimsmsm}$\ssc\,$]
\begin{eqnarray}
\label{NSoMMMOP}&&\!\!\!\!\!\!\!\!\!\!\!\!\!\!\!\!\!\!\!\!\!\!\!\!\!\!\!\!
\dis{\rm(i)\ }
(1-\d^5)\kk\le|x_1|\le(1-\kk^{-3})|x_0|+\kk^{-2}\le
(1-\kk^{-2})|x_1|+\kk^{-1},
\nonumber\\[6pt]
%\label{NewAdded-NSoMMMOP}
&&\!\!\!\!\!\!\!\!\!\!\!\!\!\!\!\!\!\!\!\!\!\!\!\!\!\!\!\!
{\rm(ii)\ }
\frac{|y_1|}{|x_1|^{1-\d^5}}\ge\frac{\g_{\kk,\kk}}{\kk}
(1+\ep^2).\end{eqnarray}
%where $\a_1=\a_2-k^{-\ell}$, $\a_2=k^{\TH\l_0\l}-k^{\TH\l_0\l(1-\d^4)}$.
%If we rewrite the above as the form in
We have \eqref{ToSayas} and \eqref{-EiathA0}% (with $\l=0$)%, then %$\mu=0$ and %$(t_1,t_2,t_3)=(0,1,0)$ and
%we have \eqref{-EiathA}% and \eqref{EiathA}\,(c)
.
For any $(p_0,p_1)\in V_0$, if the first two equalities, or the first and last equalities, hold in \eqref{NSoMMMOP}\,(i), then we can  obtain $|x_1|=(1-\d^5)\kk$,
$|x_0|\le\kk$,
%\equa{F1rst}{\mbox{$|x_1|=\kk^{-\kk+n_1-1}<\kk^{-\kk+n_1},$ \ \  $|x_0|\le\kk^{\l_0\l}$,}}
%where the second inequation follows by noting from \eqref{NSoOP} that since $|x_0|$ has the negative power $-\frac1{(\l_0-1)\l}$, when $|x_1|$ is reduced by the factor $\kk^{-1}$, the $|x_0|$ can be at most multiplied by a factor $\kk^{(\l_0-1)\l}(1+\a)$ for some $\a=O(\kk^{-1})$.
%By Theorem \ref{AddLeeme--0}\,(iii) and definition \eqref{Ak=1}, we obtain
%\equa{F2rst}{\mbox{$\g_{\kk^{\l_0\l},\kk^{-\kk+n_1}}>\g_{|x_0|,|x_1|}\ge|y_1|>|x_1|^{-\l}\kk^{\l(-\kk+n_1)}\g_{1,\kk^{-\kk+n_1}}>\kk^\l$,}}
%where the last two inequalities follow from \eqref{TaKa} and
%\eqref{NewAdded-NSoOP}, i.e., we have \eqref{WehHAH} (and the proof is completed in this case).
%a contradiction with the assumption at the beginning of the proof.
and by \eqref{NSoMMMOP}\,(ii), we have \equa{yyy6666}{\mbox{$\dis|y_1|>(1-\d^5)^{1-\d^5}\g_{\kk,\kk}\ge\Big(1-\d^5+\d^{10}+O(\d^{15})\Big)\kk>|x_1|$.}}
As before, we would obtain that $|x_1+y_1|\ge|y_1|-|x_1|\sim|y_1|\sim\kk\succ\kk^{\frac{m}{m+1}}\sim h_{_{\sc p_0,p_1}}^{^{\sc\frac{m}{m+1}}}$ (when $\kk\gg1$, cf.~Remarks \ref{rema-kk} and \ref{Fimsmsm}$\ssc\,$), a contradiction with
with  \eqref{mqp1234-2}.

If the last two equalities hold in \eqref{NSoMMMOP}\,(i), then $|x_1|=|x_0|=\kk$, but by
\eqref{NSoMMMOP}\,(ii), $|y_1|>\g_{\kk,\kk}$, a contradiction with definition \eqref{Ak=1}.
%$|x_1|=\d^4^{-1}=\kk$, and $|x_0|=(\frac{\kk-\d^4+\d^4^2-\d^4^3}{1-\d^4^3})^{\l_0\l}<\kk^{\l_0\l}$, and by
%\eqref{NSoOP}\,(ii), $|y_1|>\g_{1,1}\kk^\l>\TH\kk^\l.$
%In particular, by Theorem \ref{AddLeeme--0}\,(iii), we have $\g_{\kk^{\l_0\l},\kk^{-\kk}}\ge\g_{|x_0|,|x_1|}\ge|y_1|>\kk^\l$,
%
Hence %Therefore %the last two inequalities cannot simultaneously hold in \eqref{NSoOP}\,(i), i.e.,
 Theorem \ref{real00-inj}\,(1)\,(ii) holds.
%

Next, we want to choose suitable $u,v$ such that
\eqref{NSoOP} holds for $(q_0,q_1)$ [defined in \eqref{1+++q0q1}${\ssc\,}$], i.e.,
\begin{eqnarray}
\label{1NSoMMMOP}&&\!\!\!\!\!\!\!\!\!\!\!\!\!\!\!\!\!\!
\dis{\rm(i)\ }
1-\d^5 \le |1+u\ep| \le
(1-\kk^{-3})|1+s\ep|
+\kk^{-3}\le
(1-\kk^{-2})|1+u\ep|+\kk^{-2},
\nonumber\\[5pt]
%\label{MAMAM1NSoMMMOP}
&&\!\!\!\!\!\!\!\!\!\!\!\!\!\!\!\!\!\!
{\rm(ii)\ }
\frac{|1+v\ep|}{|1+u\ep|^{1-\d^5}}\ge1+\ep^2.\end{eqnarray}
The first strict inequality automatically holds in \eqref{1NSoMMMOP}\,(i).
If we take
\equa{LetF3rMMMst}{\mbox{$\dis u=-1,\ \ \ v=-\frac{1+a_\kk}{b_\kk},$
\ \ and so $\dis s=-1+O(\ep^1)$ \ by \eqref{@suc2hthat=4},}}  by comparing the coefficients of $\ep^1$, we see that
all strict inequalities hold in \eqref{1NSoMMMOP}\,(i). Further, the coefficient of $\ep^1$ in the left hand-side of \eqref{1NSoMMMOP}\,(ii) is
$1-\d^5-\frac{1+a_\kk}{b_\kk}$, which is positive if the assertion of the lemma is not true; in this case, we see that $(q_0,q_1)\in V_0$, i.e., $V_0\ne\emptyset,$
and we obtain a contradiction with Assumption \ref{assu1111},
namely, %. Thus the assumption  at the beginning of the proof is wrong, i.e.,
we have Lemma \ref{NeMMMMore-}.\hfill$\Box$\vskip7pt

%
%
%If we choose $\d$ in Lemma \ref{NeMMMMore-} to be $\d^4$, then
The above two lemmas show that $a_\kk\ge\frac{1-\d^4(1+\d)}{\d^4}$. Since $\d$ is arbitrarily sufficiently small number, we see that $a_\kk$ (thus also $b_\kk$) is unbounded, i.e.,
\equa{Akk-bkk}{\dis\lim_{\kk\to\infty}a_\kk=\lim_{\kk\to\infty}b_\kk=\infty,\mbox{ and
in fact, }\lim_{\kk\to\infty}\frac{a_\kk}{b_\kk}=1.
}
%





















\begin{rema}\label{remaFFFF}\rm
\begin{itemize}\item[(i)]
We wish to emphasis  that the following design \eqref{LetNSoOP} has been the hardest part for us to obtain [from the proof below we will see  why we have to design such a complicated
system of inequations in \eqref{LetNSoOP}$\ssc\,$].
\item[(ii)]From our proof above, one can see that in order to achieve our task, we must choose the power (denoted as $\a_0$) of $|x_1|$ in
\eqref{GMSMSMSMS}\,(b), \eqref{NSoOP}\,(ii) and \eqref{NSoMMMOP}\,(ii) to be different from one,
and in case $\a_0<1$ we must choose $\a_0$ to be independent of $\kk$ as in \eqref{NSoMMMOP}\,(ii),
and therefore we have to choose $v$ to be different from $u$ [as in \eqref{LetF3rst} and \eqref{LetF3rMMMst}$\ssc\,$] so that
\eqref{1NSoOP}\,(ii) and \eqref{1NSoMMMOP}\,(ii) can hold.
However because of \eqref{Akk-bkk}, our task becomes extremely difficult, simply because of the fact that any choice of $v\ne u$ (and in case $\a_0<1$, $v,u$ must be negative),
with $v$ to be differ from $u$ by a number which is independent of $\kk$, will possibly force $s=-a_\kk u+b_\kk v+O(\ep^1)$ to be too large
[by \eqref{Akk-bkk}%; cf~also \eqref{11F3rst}
$\ssc\,$]; thus we have to try in some other way as shown below.
%
\item[(iii)]We wish to mention that %, in $C_2$ of \eqref{LetNSoOP}, we
%use
%$\big|2\frac{X_1^2}{Y_1}-X_1\big|$
%instead of
%$2\frac{|X_1|^2}{|Y_1|}-|X_1|$,
%and
in $C_5$ of \eqref{LetNSoOP}\,(iii), we
add the term $\ep^4$ %,
 in order for the technical reason that
 the inequation %s
 corresponding to %$C_1\le C_2$ and
\eqref{GSoo+} is %are
solvable %, and
%in $C_5$ of \eqref{LetNSoOP}\,(ii), we add  %??????????????in the numerator
%
%\equa{MEMEMEM3-303}{\mbox{
%$\dis\left|\frac{(1-\ep^2)\big(\bar x_0^{-1}x_0\big)^{\frac2{\scep}}}{\big(\bar y_1^{-1}y_1\big)^{\frac{1+\omega }{\scep}}}+\ep^2
%\frac{\big(\bar x_0^{-1}x_0\big)^{\frac{10}{\omega \scep}}}{\big(\bar y_1^{-1}y_1\big)^{-\frac{5(2+\omega )}{\omega \scep}}}\right|$,}}
%\equan{IndOf}{\mbox{$\dis
%\left|(1\!-\!\a_0\ep)\mbox{\Large$\Big($}\frac{X_0}{X_1^{1-\d_0}}\mbox{\Large$\Big)$}^{\ell_0^3}
%\!-\!\a_0\ep\mbox{\Large$\Big($}\frac{X_0Y_1}{X_1^{2-\d_0+\d_0^2}}\mbox{\Large$\Big)$}^{\ell_0^4}\right|$,
%%in $C_2$
%}}
%$|(1\!-\!\ep)\bar x_1^{-1}x_1-\ep\bar x_0^{-1}x_0|$
%instead of
%$(1-\ep^2)\big(\kk^{-1}|x_0|\big)^{\frac2{\scep}}\big(\g_{\kk,\kk}^{-1}|y_1|\big)^{-\frac{1+\omega }{\scep}}+\ep^2
%\big(\kk^{-1}|x_0|\big)^{\frac{10}{\omega \scep}}\big(\g_{\kk,\kk}^{-1}|y_1|\big)^{-\frac{5(2+\omega )}{\omega \scep}}$
%
%$|\d\bar x_0^{-1}x_0+1|$ in the denominator instead of
%$\d\kk^{-1}|x_0|+1$ to guarantee that
%
%I, we use the numerator
%$|\frac{x_1}{\bar x_1}+\frac{\d\bar x_0\bar x_1}{\bar y_1}\frac{y_1}{x_0x_1}|$ in $C_2$
%instead of $\frac{|x_1|}{\kk}+\frac{\d\kk^2}{\gamma_{\kk,\kk}}\big|\frac{y_1}{x_0x_1}\big|$
%\equan{IndOf2}{\mbox{$
%\dis(1\!-\!\a_0\ep)\left|\frac{X_0}{X_1^{1-\d_0}}\right|^{\ell_0^3}
%\!-\!\a_0\ep\left|\frac{X_0Y_1}{X_1^{2-\d_0+\d_0^2}}\right|^{\ell_0^4}$,}}
%$(1\!-\!\ep)\kk^{-1}|x_1|-\ep\kk^{-1}|x_0|$
%is also
%for the technical reason that
%is in order to guarantee that
%the corresponding inequation \eqref{GSoo+} %corresponding  to $C_1\le C_2$ %\eqref{GSoo+}
%
%
%We will see from our proof below that the factor $
%\big(\frac{(\kk^{-1}|x_0|)^{-\scep}+2\scep\kk^{-1}|x_0|}{1+2\scep}\big)^{\frac1{\scep}}$ of $C_2$ in
%\eqref{LetNSoOP}\,(i) is very crucial, which plays the following three important roles:\begin{itemize}\item[(a)]
% we can have case (3) in the proof of Lemma \ref{TheoHold};
% \item[(b)]we can have $C_1<C_2$ in \eqref{MSMSMSMSMS}\,(i);\item[(c)]the inequation corresponding to
%$C_2\le C_3$
%is solvable
[cf.~\eqref{1MEMEME}\,(i)$\ssc\,$].
%From the following proof we will see  why we have to design such a complicated
%system of inequations in \eqref{LetNSoOP}.
%
%
% \end{itemize}
%We remark that we use $|2\bar x_0^{-1}x_0-1|$ instead of $2\kk^{-1}|x_0|-1$ in \eqref{LetNSoOP}\,(ii)
%is to guarantee that
%the inequation corresponding to it  is solvable (cf.~Remark \ref{FinalRem}).
% there is a negative term in the
%denominator
%numerator of \eqref{LetNSoOP}\,(ii), thus the corresponding inequation \eqref{GSoo+}
%to the second inequality of \eqref{LetNSoOP}\,(i) will have two negative terms
%[see \eqref{1MEMEME}\,(i){$\sc\,$}] and
%is in general %such an inequation is
%unsolvable, however we have designed \eqref{LetNSoOP} so that the inequation is solvable [cf.~\eqref{1MEMEME}].
%The reason %From %\eqref{WEhHaaa} and
%the proof of Lemma \ref{TheoHold} below,  we will see why
%we have to design such a
%complicated system of inequations in \eqref{LetNSoOP} is to guarantee that we have Lemma \ref{NeLmeme} and
%\eqref{isisidjd} below.
%We would also like to emphasis that the reason we use
%$\big|\bar x_0x_0^{-1}-(1+\ep)\bar x_1^{-2}x_1^2\big|$
% in $C_2$ instead of
%$\kk|x_0|^{-1}-(1+\ep)\kk^{-2}|x_1|^2$
%is to guarantee that
%the inequation corresponding to the third inequality of \eqref{LetNSoOP}\,(i) is solvable (cf.~Remark \ref{FinalRem}).
%\item[(2)]We remark that the proof of Lemma \ref{TheoHold} below in fact does not need to use any result obtained so far except Theorem $\ref{lemm-a1}$.
\end{itemize}
\end{rema}
%
\NOUSE{%
\begin{lemm}\label{fiX-klemm}For any fixed $\kk\gg1$, there exists $k'>\kk$ such that $\g_{k',k'}=\frac{k'}{\kk}\g_{\kk,\kk}.$
\end{lemm}
%
We shall prove this lemma by assuming conversely we have the following.
\begin{assu}\label{assu1111+1}
Assume Lemma {\rm\ref{fiX-klemm}} is not true.
\end{assu}
%
%
%
%
First we reformulate the Jacobian pair $(F,G)$. Denote
\equa{sssA00}{\dis A'_{1,1}=\big\{\,(p_0,p_1)=\big((x_0,y_0),(x_1,y_1)\big)\in V\ \big|\ |x_0|,|x_1|\le1\,\big\}=\raisebox{-4pt}{\mbox{${}^{\dis\ \ \ \ \, \mbox{\Large$\cup$}}_{0\le k_1,k_2\le1}$}}A_{k_1,k_2}.}
Theorem \ref{Ak=}\,(i) shows that there exists a fixed sufficiently large number $\eta_0\in\R_{>0}$ such that \equa{SuThata}{\mbox{$h_{p_0,p_1}<\eta_0$ for $(p_0,p_1)\in A'_{1,1}$.}}
 Take $\eta=\eta_0^2+\eta_0$, and define the Jacobian pair $(F^\eta,G^\eta)$ as follows,
\equa{MSMSMNewFG}{F^\eta=F(x,y-\eta),\ \ G^\eta=G(x,y-\eta).}
Then all results obtained so far are valid for the pair $(F^\eta,G^\eta)$. We use same symbols with the superscript ``\,${}^{\eta}$\,'' to denote elements associated with $(F^\eta,G^\eta)$. In particular, we have
\equa{sssA00-eta}{\dis A'^\eta_{1,1}=\big\{\,(p_0^\eta,p_1^\eta)=\big((x_0^\eta,y_0^\eta),(x_1^\eta,y_1^\eta)\big)\in V^\eta\ \big|\ \big((x_0^\eta,y_0^\eta-\eta),(x_1^\eta,y_1^\eta-\eta)\big)\in A'_{1,1}\,\}.}
Therefore, for any $(p_0^\eta,p_1^\eta)\in A'^\eta_{1,1}$, we have $|y_1^\eta|\ge\eta-|y_1^\eta-\eta|>\eta_0^2$
by \eqref{SuThata}. Thus by replacing $(F,G)$ by $(F^\eta,G^\eta)$, we may assume
\equa{sssA00-eta+}{|y_1|>\eta_0^2\mbox{ for any }(p_0,p_1)=\big((x_0,y_0),(x_1,y_1)\big)\in A'_{1,1}.
%\mbox{ where $\eta_0\in\R_{>0}$ is some sufficiently large number}.
}
%
%
}%
\NOUSE{%
First we generalize Lemma \ref{G--lemm-assum1--} as follows.
\begin{lemm}\label{G--lemm-assum1--}For any $\d,k,k_0,k_1\in\R_{>0}$ with $k>1,\,\d<\frac1m$, we have $\g_{k^{1+\d}k_0,kk_1}<k\g_{k_0,k_1}$.
 \end{lemm}\noindent{\it Proof.~}The proof is essentially the same as that of Lemma \ref{G--lemm-assum1--}. Assume the result is not true, then by choosing $\d'$ with $\d<\d'<\frac1m$ and by Theorem \ref{AddLeeme--0}\,(iii), we may assume
$\g_{\bar k^{1+\d'}k_0,\bar kk_1}> \bar k\g_{k_0,k_1}$ for some $\bar k,k_0,k_1\in\R_{>0}$ with $\bar k>1.$
Thus we can choose a sufficiently small $\d_1\in\R_{>0}$ satisfying (the following holds when $\d_1=0$ thus also hold when $\d_1>0$ is sufficiently small)
\equa{GMSMSM0999}{\dis
\frac{k_1^{1+\d_1}\g_{\bar k^{1+\d'}k_0,\bar kk_1}}{(\bar kk_1)^{1+\d_1}}>
\g_{k_0,k_1}.}
Take $\kk\gg1$.
We define $V_0$ to be the subset of $V$ consisting of elements $(p_0,p_1)=\big((x_0,y_0),(x_1,y_1)\big)$ satisfying
[our aim is to design the following to satisfy
Theorem {\rm\ref{real00-inj}\,(1)}${\ssc\,}$]
\begin{eqnarray}
\label{GMSMSMSMS}\dis\!\!\!\!\!\!\!\!\!\!\!\!\!\!\!\!\!\!&&
{\rm(a)\ }k_1\le|x_1|\le \frac{k_1}{\,k_0^{\frac1{1+\d'}}\,}(1-\kk^{-3})|x_0|^{\frac1{1+\d'}}+ k_1(\kk^{-1}-\kk^{-2}+\kk^{-3})\le
(1-\kk^{-2})|x_1|+ k_1\kk^{-1},
\nonumber\\[4pt]
%\label{1MSMSMSMS}
\dis\!\!\!\!\!\!\!\!\!\!\!\!\!\!\!\!\!\!&&
{\rm(b)\ }\frac{|y_1|}{\,|x_1|^{1+\d_1}\,}\ge
\frac{\g_{\bar k^{1+\d'}k_0,\bar kk_1}}{(\bar kk_1)^{1+\d_1}}.
\end{eqnarray}
Then  we can rewrite the above as the form in \eqref{ToSayas},
obviously we have \eqref{-EiathA0} [note from the second and last inequalities of  \eqref{GMSMSMSMS}\,(a) that $|x_1|\le k_1\kk$].
Further,
by definition, there exists\equa{GEXiSTS}{\mbox{ $(\check p_0,\check p_1)=\big((\check x_0,\check y_0),(\check x_1,\check y_1)\big)\in V$ with
$|\check x_0|=\bar k^{1+\d'}k_0,\,|\check x_1|=\bar kk_1$ and $|\check y_1|=\g_{\bar k^{1+\d'}k_0,\bar kk_1}$.}} Then one can easily see that $(\check p_0,\check p_1)\in V_0$, i.e., $V_0\ne\emptyset.$
%
Let $(p_0,p_1)\in V_0$.
If the first two equalities, or the first and last equalities, hold in \eqref{GMSMSMSMS}\,(a), then we obtain that $|x_1|=k_1,\,|x_0|\le k_0$,
but
by \eqref{GMSMSM0999} and \eqref{GMSMSMSMS}\,(b), $|y_1|>\g_{k_0,k_1}$, a contradiction with the definition of $\g_{k_0,k_1}$ and/or Theorem \ref{AddLeeme--0}\,(iii).
If  %the first and last equalities, or
the last two equalities hold in \eqref{GMSMSMSMS}\,(a), then one
obtains that $|x_1|\sim\kk,\,|x_0|\sim\kk^{1+\d'}$ [when $\kk\gg0$, cf.~\eqref{MSM1111}${\ssc\,}$].
Thus \eqref{mqp1234-2+} implies that $h_{p_0,p_1}\preceq\kk^{1+\d'}$.
By \eqref{GMSMSMSMS}\,(b), we have
$|y_1|\succeq|x_1|^{1+\d_1}\sim \kk^{1+\d_1}$, and so
(where the part ``\,$\succ$\,'' follows by noting that $\d'<\frac1m$)
\equa{AnDSo}{|x_1+y_1|\ge|y_1|-|x_1|\sim|y_1|\succ \kk^{\frac{(1+\d')m}{m+1}}\succeq h_{_{\sc p_0,p_1}}^{^{\sc\frac{m}{m+1}}},}
 a contradiction with
\eqref{mqp1234-2}. This shows that Theorem
{\rm\ref{real00-inj}\,(1)} holds, a contradiction with Assumption \ref{assu1111}.
The lemma is proven.\hfill$\Box$\vskip5pt
}%














Let as before
$%\ell\gg
\kk\gg1$ and $\d$ be sufficiently small
(%we may assume $\kk\gg(d^{-1})^{d^{-1}},\,\d^{-1}\gg\ell^\ell$,
and  assume $\ep\ll\kk^{-\kk}%\ell^{-\ell}
$, cf.~Remark \ref{u-vRema}${\ssc\,}$)%
.
%We take $\ell_1,\ell\in\R_{>0}$ such that \equa{MAmsndj}{\kk\gg\ell\gg\ell_1\gg1,\ \
%\ \ 0<\ep\ll\kk^{-1}\ll\d:=\ell^{-1}\ll\d_1:=\ell_1^{-1}\ll1.}
%We take $\ell_1,\ell\in\R_{>0}$ (and assume $\ep\ll\ell^{-\ell}$, cf.~Remark \ref{u-vRema}${\ssc\,}$) such that
%\equa{ell-0ell1}{\ell\gg\ell_1\gg\kk%\d ^{-1}
%,\ \ \ \ \
%\ \ 0<\ep\ll\d:=\ell^{-1}\ll\d_1:=\ell_1^{-1}\ll\kk^{-1}%\d
%.}
%
Denote %(cf.~Lemma \ref{NeMore-}$\ssc\,$)
[%where $\ln(\cdot)$ is the natural logarithmic function;
note from Lemmas \ref{NeMore-}, \ref{NeMMMMore-} and \eqref{Akk-bkk} that
$v_0>0$ and  we can assume $0<v_0<\d  ^N$ for any fixed $N\in\R_{>0}$;
we define the following in order to have \eqref{11F3rst}%
% and case (d) in the proof of Lemma \ref{TheoHold}
%; where $\ln(\cdot)$ is the natural logarithmic function
$\ssc\,$],
\begin{eqnarray}\label{v0=1=1=1=1}
v_0:=\frac{b_\kk-a_\kk-1}{b_\kk}=O(\d ^N)
%,\ \ \ \
%\a_1=\frac{\ep}{\ln\Big(\frac{\d}{\scep^4}\Big)},\ \ \ \a_2=\frac1{1+v_0}
%\a_0:=\frac{\ln\Big(\frac{2}{\scep^3}\Big)}{1 - v_0 + v_0^6}\gg\kk
%,\
%0\!<\!v_0\!=\!O(\d ^N)\mbox{ for any fixed }N\!\in\!\R_{>0}
.
\end{eqnarray}
%
%
%Of course we
%can assume  $\frac{\ln(\scep)}{\scep}$ is an integer [where $\ln(\cdot)$ is the natural logarithmic function] so that the element
%inside the absolute sign $|\cdot|$ of $C_1$ in \eqref{LetNSoOP} is well-defined.
%
%
%
%
%
%For some technical reason, we need some notations.
%Take $\ell_0\in\Z_{>0}$, $\ep,\ell_1,\ell ,\kk\in\R_{>0}$ such that
%\equa{tech-1}{1\ll\ell_0\ll\ell_1\ll\ell \ll\kk,\ \ \ep\ll\kk^{-\kk}.}
%Now let as before, $\kk\gg1$ and $\d\in\R_{>0}$ is sufficiently small (and assume $\ep\ll\kk^{-\kk}$, cf.~Remark \ref{u-vRema}${\ssc\,}$).
%Denote [note from Lemmas \ref{NeMore-}, \ref{NeMMMMore-} and \eqref{Akk-bkk} that
%(we may assume that the $\d_1$ appeared here is much smaller than the $\d$ appeared in Lemma \ref{NeMore-} so that $v_0>0$); in fact we can assume
%$0<v_0<\d ^N$ for any fixed $N\in\R_{>0}\ssc\,$]
%\begin{eqnarray}
%\label{tech-2}
%\!\!\!\!\!\!&\!\!\!\!\!\!\!\!\!\!\!\!\!\!\!\!\!\!\!\!\!\!&
%\dis 1{\sc}\gg{\sc}\d {\sc}:={\sc}\ell_0{\sc}\gg{\sc}\d_1{\sc}:={\sc}\ell_1{\sc}\gg{\sc}\d {\sc}:={\sc}\ell {\sc}\gg{\sc} v_0\gg\ep{\sc}>{\sc}0,\ \mbox{ where }
%v_0{\sc}={\sc}\frac{b_\kk{\sc}-{\sc}a_\kk{\sc}-{\sc}1}{b_\kk}{\sc}>{\sc}0,
%\nonumber\\[4pt]
%\label{tech-3}
%\!\!\!\!\!\!&\!\!\!\!\!\!\!\!\!\!\!\!\!\!\!\!\!\!\!\!\!\!&
%\a_0:=\frac{\d -(1+\d )v_0-v_0^6}{\d +v_0}=1+O(v_1^1).
%\frac{\ell_0}{2}-1+\frac{\d }{2}+\frac{v_0}{\d ^4}-\frac{v_0^5}{2}=\frac{\ell_0}{2}-1+O(\d ^1)
%\end{eqnarray}
%
%\begin{eqnarray}
%\label{tech-2++}
%v_0:=\frac{b_\kk-a_\kk-1}{3b_\kk},\ \ \ \ 0<v_0<\d^N\mbox{ for any fixed $N\in\R_{>0}$}.
%\end{eqnarray}
%
%
%
%
\NOUSE{%
\NOUSE{%
 with $\kk\gg1$ and $\ep<\kk^{-\kk}$ such that $\frac1{\scep}$ is an integer.
Denote (where $\ln(\cdot)$ is the natural logarithmic function)
\equa{Daa1+++1}{\dis v_0=\frac{b_\kk-a_\kk-1}{b_\kk},\ \a_0=-\frac{\ep^2\ln(1-\ep(1+\ep)^{-\frac1{\scep^2}})}{2\ln(1+\ep)},\ \
\a_1=\ep(1+\ep)^{-\frac1{\scep^2}}.}
Then we have [where $\d>0$ is any fixed sufficiently small number]
\equa{Daa1+++2}{\dis 0<v_0<\d^N,\ \ 0<\a_0,\a_1<\ep^N\mbox{ for any fixed \ } N\in\Z_{>0}.}
Take $\ell\in\R_{>0}$ such that \equa{Daa1+++3}{\ell\gg\Big(\frac1{\a_0\a_1}\Big)^{\frac1{\a_0\a_1}}\mbox{ \
(thus $\ell^{-1}\ll\ep\ll v_0\ssc\,$)}.}
%
%
 $\ep, j  ,\ell,\kk\in\Q_{>0}$ such that \equa{SuThahsh}{\mbox{
$ j  \gg1,\ \ \ell\gg j  ^{ j  },\ \ \kk\gg\ell^{\ell},\ \ \ \ep<\kk^\kk$%
%let $\kk%,\d
%$ be as above %$\gg1$ and $\d\in\R_{>0}$ being sufficiently small %
%Take $\ell\gg\kk$
%(and assume $\ep<\kk^{-\kk}$, cf.~Remark \ref{u-vRema})
.
}}Denote\equa{Td-d1}{\mbox{$\dis \d=\ell^{-1},\ \ \ \omega = j  ^{-1}$ \ \ (thus \ $\ep\ll\kk^{-1}\ll\d\ll\omega\ssc\, $).}}
%We choose $\ell\gg\frac1{\scep}$ (for instance, $\ell=(\frac1\scep)^{\frac1{\scep}}$).
\NOUSE{%
Denote
\equa{smsmsdddd}{\dis s_0=3b_\kk-a_\kk.}
 %Take $\ell=\d^{-1}$.
%
We need some notations. Denote (where $e$ is the natural number and $\ln(\cdot)$ is the natural logarithmic function)
\begin{eqnarray}
\label{NoTTTaa}
&\!\!\!\!\!\!\!\!\!\!\!\!\!\!\!\!\!\!\!\!&
\a_0\!=\!\Big(\frac{\big((1\!+\!\ep)^{\frac1{\scep}}\!+\!1\!-\!e^{-1}\big)(1\!+\!\ep)^{-\frac2{\scep}}}{2\!-\!
e^{-1}}\Big)^{\scep^2}\!=\!s_0\ep^2\mbox{ with }s_0\!=\!\ln\Big(\frac{e^2+e-1}{e^2(2e-1)}\Big)+O(\ep^1)<0,
\nonumber\\[4pt]
&\!\!\!\!\!\!\!\!\!\!\!\!\!\!\!\!\!\!\!\!&
v_0:=\frac{b_\kk-a_\kk-1}{2b_\kk+u_1a_\kk+2}>0,\ \ \a_1=\a_{11}v_0-3\a_{10}v_0^2\mbox{ with }\a_{10}\!=\!\ln\Big(\frac{e^2(2e\!-\!1)}{3(e^2\!+\!e\!-\!1)}\Big)^{-1}\approx5.48.
\end{eqnarray}
Note from Lemmas \ref{NeMore-}, \ref{NeMMMMore-} and \eqref{Akk-bkk} that we may assume $v_0<\d^{N}$ for any fixed $N\in\R_{>0}{\ssc\,}$.
%We denote
%[]
%\equa{v1====}{\dis v_0=\frac{b_\kk-a_\kk-1}{2b_\kk-1}>0\mbox{ (and so $v_0<\d^N$ for any fixed $N\in\R_{>0}$)}, \ \
%\a_0=\frac72+13v_0+53v_0^2.}
}%
%Take $u_1\in\R_{\ge0}$ with $u_1\le1%\a_\kk^{-1}
%$ such that $v_0^{-1}$ is an integer, where $v_0$ is defined below
%
\NOUSE{%
We also denote [%where we fix some $\a_1\in\R_{>2}$ such that $\a_2$ defined below is an integer;
note that we may assume $0<v_0<\d^{N}$ for any fixed $N\in\R_{>0}$ by
Lemmas \ref{NeMore-}, \ref{NeMMMMore-} and \eqref{Akk-bkk}; cf.~\eqref{c1==010101} and \eqref{c1==010101+} to see why we define such numbers]
%thus $\a_2>\d^{-N}
%${\ssc\,}$]
\begin{eqnarray}
\label{v2=Aq}
&\!\!\!\!\!\!\!\!\!\!\!\!\!\!\!\!\!\!\!\!\!\!\!\!\!\!\!\!&
 v_0=\frac{b_\kk-a_\kk-1- \omega^5
 }{b_\kk+1}
,%\ \ \a_2=\frac{1+\a_1+(2\a_1+3)v_2}{\a_1v_2}\gg1,\ \ \a_1\in\R_{>2}
\mbox{ \ \ and \ \ $0<v_0=O(\d^{N})$ for any fixed $N\in\R_{>0}$%,\ \ and \ $v_0^{-1}\in\Z_{>0}$
}
%,\nonumber\\[4pt]
%&\!\!\!\!\!\!\!\!\!\!\!\!\!\!\!\!\!\!\!\!\!\!\!\!\!\!\!\!&
%\a_1\!=\!\frac{(1\!+\!\d)\big(\d^3\!+\!(1\!+\!\d^2)v_0\big)}{\d(1\!+\!\d^2)}\!=\!\d^2\!+\!\d^3\!-\!\d^4\!+\!O(\d^5),\ \ \
%\a_2\!=\!\frac{2(1\!+\!\d)(\d\!-\!v_0)}{\d}\!=\!2(1\!+\!\d)\!+\!O(\d^5)
.
\end{eqnarray}
%
%
}%
%
%Take $\ell\in\R_{>0}$, and of course we can take $\kk\gg1$ to be sufficiently large such that we can assume the following,
%\equa{d-dddd}{\kk\gg\ell^\ell,\ \ \ \ell\gg\d^{-\d^{-1}}.}
% Of course we may assume $\d^{-1}$ is integral [so that $(\bar y_1^{-1}y_1)^{\d^{-1}}$ is well-defined].
%Take
%fixed $\ell_0\in\R_{>0}$ and
%[see \eqref{C4===334} for the reason why we define $\ell_0,\ell_1$ as below]
%\equa{ell2==1}{\dis\ell_1=\ell_0+2+\Big(-\frac12+\frac{3\ell_0}{2}+\ell_0^2\Big)\ep,\ \ \ \ell_0=100.}
%Now we regard $\kk$ as fixed, and take $\ell\gg\kk$ (for instance $\ell>\kk^{\kk}$), and we take
%$\ep=\ell^{-\frac13}$ (cf.~Remarks \ref{u-vRema} and \ref{Fimsmsm}).
%Now%
%
%Take
%We choose $\a_{20},\a_{30}\in\R$ with $0\le\a_{20},\a_{30}<1$ such that the numbers $\a_2,\a_3$ defined below are integers (one can simply regard $\a_{20},\a_{30}$ as zeros since we will only conduct our compustions below up to $O(v_0^3)\ssc\,$),
%\equa{a-1-a2-a3}{\dis\a_1=\frac{(1+2v_0-v_0^4)v_0^3}{5-2v_0+v_0^3+2v_0^4}.}
%
We also assume $\ep\in\Q_{>0}$ is chosen
%to be sufficiently small (for instance, $\ep<\kk^{-\kk}$, cf.~Remark \ref{u-vRema})
such that
$\frac1{\scep},\frac{1+\omega}{\scep}$ are integral [so that the elements inside the absolute sign $|\cdot|$ of $C_2$ in \eqref{LetNSoOP}\,(i) are well-defined].
%
}%
%
}%
%
%
%
%
%
%
%
%
%
%
%
%
%
%
%
%
%
\NOUSE{%
We can assume $\d \in\Q_{>0}$ such that $\frac1{\d }$ is an anteger [so that the elements inside the absulate sign $|\cdot|$ in $C_2$ of \eqref{LetNSoOP}\,(i) are well-defined]. Denote
\equa{alphsooo}{\dis\a_0=\frac{\d (1-3\d )
+(1+3\d ^2)v_0}{\d (3-2\d )+3(1+\d ^2)v_0}=
=\frac13-\frac{7\d }{9}-\frac{14\d ^2}{27}+O(\d ^3).}
%
}%
For convenience, we simply denote [cf.~\eqref{TaKa} for notations $\bar x_i,\,\bar y_1\ssc\,$],
\equa{SimMMSMS}{\dis Y_1=\bar y_1^{-1}y_1,\ \ X_i=\bar x_i^{-1}x_i,\ i=0,1.}
%
%
%
%
%
%
%
%
%
%
%
%
%
%
%
%
%
%
%
%
%
%
%
%
%
%
%
%
%
%We take $\ell\in\R_{>0}$ such that $\ell\gg\ep^{-1}$.
%Adding  $v_0$ by an $O(v_0^5)$ element (which will not affect our computation below), we may assume $v_0$ is a rational number. Then we choose any $\a_1\in\Z_{>0}$ such that $\frac{\a_1}{v_0},\,\frac{\a_1}{v_0(1-v_0)}$ are integers [thus the complex numbers inside the absolute sign $|\cdot|$ of $C_2$ below are well-defined].
%Now
Define $V_0$ to be the subset of $V$ consisting of elements $(p_0,p_1)=\big((x_0,y_0),(x_1,y_1)\big)$ satisfying
%[cf.~\eqref{TaKa}${\ssc\,}$] %(where $e$ is the natural number)
[again our purpose is to design the following to satisfy
Theorem {\rm\ref{real00-inj}\,(1)}%$\ssc\,$; cf.~\eqref{TaKa} for notations $\bar x_0,\,\bar x_1%,\,\bar y_1,\,
%$
; %also
cf.~Remarks \ref{rema-kk} and \ref{Fimsmsm}$\ssc\,$% and $
%\g_{\kk,\kk}{\ssc\,}
%$
%; obviously we can choose $\ep$ such that $\ell:=\frac1{\scep}$ and $\frac{\d}{\scep}%,\,\frac1{\omega ^2\scep}
%$ are integers so that
%the complex numbers inside of absolute sign $|\cdot|$ of $C_1$
%are well-defined%
%; where $e$ is the natural number%
%note also that $\d^{-1}\big((\kk^{-1}|x_0|)^{\ell^{-1}}-(1-\d)\big)$ is a positive number thus there is no
%problem for taking any power of this number
%cf.~Remark \ref{Fimsmsm}
%;]
%(where $e$ is the natural number),
]\begin{eqnarray}
\!\!\!\!\!\!\!\!\!\!\!&\!\!\!\!\!\!\!\!\!\!\!\!\!\!\!\!\!\!\!\!\!\!  &
\dis{\rm(i)\  }
1\le C_1:=|X_0^3X_1Y_1|^{\frac1{5-v_0}-v_0^5}
\le C_2:=|X_0X_1^3Y_1^2|^{\frac1{6-2v_0}}
\nonumber\\[2pt]
\!\!\!\!\!\!\!\!\!\!\!\!\!\!\!\!\!\!\!\!\!\!\!\!\!\!\!\!\!\!\!\!\!\!\!\!\!\!\!\!\!\!\!\!  \!\!\!\!\!\!\!\!\!\!\!\!\!\!\!\!\!\!\!\!\!\!&\!\!\!\!\!\!\!\!\!\!\!\!\!\!\!\!\!\!\!\!\!\!  &
\phantom{\dis{\rm(i)\  }
1}\le C_3:=|X_0^3X_1Y_1|^{\frac1{5-v_0}+v_0^5}\le C_4:=\ell%^{\frac1{5-v_0}+v_0^5}
,\ \ \mbox{ where \ } \ell=\Big(\frac1{v_0}\Big)^{\frac1{v_0^{10}}},
\nonumber
\\ %[2pt]
\!\!\!\!\!\!\!\!\!\!\!\!\!\!\!\!\!\!\!\!\!\!\!\!\!\!\!\!\!\!\!\!\!\!\!\!\!\!\!\!\!\!\!\!  \!\!\!\!\!\!\!\!\!\!\!\!\!\!\!\!\!\!\!\!\!\!&\!\!\!\!\!\!\!\!\!\!\!\!\!\!\!\!\!\!\!\!\!\!  &
%\end{eqnarray}
%\begin{eqnarray}
\label{LetNSoOP}
{\rm(ii)\ }\ep\le|X_1|\le\ep^{-1},\ \ \ \ \ {\rm(iii)\ }
C_5:=|X_1|^2\Big(\frac{|X_1^3X_0Y_1^2|^{\frac{3}{6-2v_0}+v_0^5}}{|X_0|^5}+\ep^4\Big)\ge1+v_0^5\ep.
\end{eqnarray}%Note from Lemmas \ref{NeMore-}, \ref{NeMMMMore-} and \eqref{Akk-bkk} that $0<v_0<\d^5$.
%where $\ln(\cdot)$ is the natural logarithmic function.
%
%where $\a_1=\a_2-k^{-\kk}$, $\a_2=k^{\TH\l_0\l}-k^{\TH\l_0\l(1-\d)}$.
%where $\a_1=1+(1+2\d)b_\kk-a_\kk>0$ by Lemma \ref{NeMore-}.
%
%
%where $\a_1\in\R_{>0}$ is some sufficiently large number (which may depend on $\kk$), and $\ell\gg\ep^{-1}$.
Obviously we can rewrite \eqref{LetNSoOP} as the form in
\eqref{ToSayas0}% with $\kappa_2=(6-2v_0)\big(\frac{1}{5-v_0}-v_0^5\big),\kappa_5=(6-2v_0)\big(\frac{1}{5-v_0}+v_0^5\big)\notin\big\{\frac13,2\big\}$, thus,
%\eqref{-EiathA} holds%
% [note that we can always choose $\ell\gg\kk$ such that $\frac{2c_\kk+1}{c_\kk-1}\pm\ell^{-5}\ne2\ssc\,$, thus
%condition \eqref{-EiathA} holds]% and obviously
%we have \eqref{-EiathA0} [using  \eqref{LetNSoOP}\,(ii) and the fact that $\ep\le C_2\le\ep^{-1}$, we see that
%$\ep\ell^{-1}\le|Y_1|\le\ep^{-1}\ell$, then using the fact that $\ep\le C_1\le\ep^{-1}$ we see that \eqref{ToSayas0}
%holds for $|X_0|$% is nonzero and bounded
%]%
%\le\ell_1^{v_0+\d^3}$, we have $1\le|X_0|\le\ell_1$; then
%using \eqref{LetNSoOP}\,(ii),\,(iii), we have $1\le|X_1|\le\ell\ssc\,$]%
%[usig the fact that $1\le C_1\le C_3$, we obtain that $\d\le|X_0|\le1$, then using \eqref{LetNSoOP}\,(iii), we see $|Y_1|\ge\th_0$ for some $\th_0\in\R_{>0}\ssc$].
%
% [cf.~Remark \ref{remaFFFF}\,(iii)$\ssc\,$]%, and \eqref{-EiathA0} holds% with $\eta=\ep$
%. Note that conditions \eqref{LetNSoOP} imply the following [cf.~\eqref{TaKa}%; recall that $\ep<\kk^{-\kk}$, and $a_\kk,b_\kk\le\kk^{\SS_0}$
%for some fixed $\SS_0$, cf.~\eqref{bBbbb} and \eqref{akk-bkk}
.
%
%
%
\begin{rema}\label{NotaRemak}\rm
%\begin{itemize}\item[(i)]
We emphasis once again (cf.~Remarks \ref{rema-kk} and \ref{Fimsmsm}$\ssc\,$) that there is no problem to use the $\kk,\ep$ in our design of the system \eqref{LetNSoOP}. In the sense of defining the system \eqref{LetNSoOP}, $\kk,\ep$ are
only some chosen numbers (however in order for the system to satisfy our requirement, we may need to choose these numbers
to satisfy the %following
condition
%\equa{MMSMSMnnsnsn}{\mbox{
$
\dis\ep^{-1}%\gg\ell
\gg\kk\gg1{\ssc\,})
.
$% to be sufficiently large and $\ep>0$ to be sufficiently small
%.}}
%We remark that in order to avoid  possible non-integral powers of complex numbers, $|C_{10}|,\,C_2$   should be read as
%$|C_{10}|=\frac{|X_0|^{1+\d }}{|Y_1|}$, $C_2=\Big|\a_0\frac{X_1X_0^{3(1+\d )\d ^{-2}}}{Y_1^{3\d ^{-2}}}+(1-\a_0)Y_1^3\Big|$,
% we have written as in \eqref{LetNSoOP}
%just for simplicity. % (note that $\ell_0\in\Z_{>0}{\ssc\,})$.
%This will also apply to some similar situations below.
%\item[(ii)]Note that %since $\ell_2\gg\ell\gg\ep^{-1}$,
%for convenience, we can omit the terms $\ep^4,\ep^3$ in \eqref{LetNSoOP}\,(ii) in the following computation (since $\ep^{-1}\gg|X_0|\ssc\,$).
%\item[(iii)]Note that in \eqref{ImMpP}\,(c), if $|x_1|\succeq\ep^{-1}$ then $\d^5$ can be replaced by $\kk^{-5}$, and
%if $|x_1|\succeq\ell$ then $\d^5$ can be replaced by $\ep^{5}$.
%\end{itemize}
\end{rema}%
%
%
%
%
%
%
\begin{lemm}\label{NeLmeme}%There exists some $\th_0\in\R_{>0}$ such that
%\begin{itemize}
%\item[\rm(1)]The second equality of \eqref{LetNSoOP}\,{\rm(i)} implies that $x_1=\bar x_1$.
%\item[\rm(2)]
When conditions \eqref{LetNSoOP} hold we have the following %$($for any sufficiently small $\d>0{\ssc\,})$% $[$cf.~$\eqref{MAmsndj}{\ssc\,}]$% $[$%where $\ell\gg\kk$, for instance $\ell>\kk^\kk$; %for some $\th_0\in\R_{>0}$,
[in particular we have \eqref{-EiathA0}%; where
%$\ln(\cdot)$ is the natural logarithmic function%$e$ is the natural number
%; we also remark that $(d)$ does not necessarily hold if $|y_1|$ is not big enough $($cf.~proof of Lemma $\ref{YYYy1==}{\ssc\,})
${\ssc\,}]$.
\begin{eqnarray}
\label{ImMpP}
&\!\!\!\!\!\!\!\!\!\!\!\!\! &\!\!\!\!\!\!\!\!\!\!\!\!\!
{\rm(a)\  }
\ep\le|X_1|\le\ep^{-1},
 \ \ %\ \ \ \ \ \ \ \ \ \ \ \ \ \ \ \ \ \ \ \ \ \,
{\rm(b)\  }
\ep\le|X_0|\le\ep^{-1}, %|X_1|^{-\frac12}\le |Y_1|\le\big(C_4|X_1|^{-1}\big)^{\frac12},
%\nonumber\\[6pt]
 \ \
%&\!\!\!\!\!\!\!\!\!\!\!\!\! &\!\!\!\!\!\!\!\!\!\!\!\!\!
{\rm(c) \ }%|Y_1|^{-1}\le|X_0|\le2^{\frac1{v_0^3}}|Y_1|^{-1},
%\ \ {\rm(d) \ }
(1\!-\!\d ^5)|X_1|\le|Y_1|\le(1\!+\!\d ^5)|X_1|.
%\g_{\kk,\kk}^{-1}|y_1|\ge\Big(1+(v_0-v_0^2)\ep\Big)\a_2^{1-2v_0}+O(\ep^2)=1+\big(v_0+O(v_0^2)\big)\ep+O(\ep^2)>1
%,\mbox{ where }\nonumber\\[6pt]
%&\!\!\!\!\!\!\!\!\!\!\!\!\!\!\!\!\!\!\!\!\!\!\!\!&
%%\mbox{ where }
%\a_1=\Big({\ssc}2\Big({\ssc}1-\frac{\ep}{v_0}{\ssc}\Big)
%{\ssc}\Big)^{{\ssc}-\frac{v_0^3\scep}{2-\scep}}{\ssc}=1-
%\frac{\ln(2)v_0^3\ep}{2}+O(\ep^2),
%\ \ \a_2=\Big(2\a_1^{\frac{2-\scep}{v_0^3\scep}}\a_3\Big)^{v_0^3\scep}=1+O(v_0^3)\ep+O(\ep^2),
%\nonumber\\[4pt]
%&\!\!\!\!\!\!\!\!\!\!\!\!\! &\!\!\!\!\!\!\!\!\!\!\!\!\!
%\
%\ \ \ \
%{\rm(d)\  }
%(1-\d^4)|X_1|\le|Y_1|\le(1+\d^4)|X_1|,
% \
% \,
%\a_3=\a_1^{\frac1{v_0^3}}{\ssc}
% -\Big({\ssc}1-\frac{\ep}{v_0}{\ssc}
% \Big)=\Big({\ssc}\frac1{v_0}-\frac{\ln(2)}{2}{\ssc}\Big)\ep+O(\ep^2)>0.
%
%(1-\d^4)\g_{\kk,\kk}^{-1}|y_1|\le\kk^{-1}|x_1|\le(1+\d^4)\g_{\kk,\kk}^{-1}|y_1|.
%
%\ \ \nonumber\\
%&&
%\!\!\!\!\!\!\!\!\!\!\!\!\!\!\!\!\!\!\!\!\!\!\!\!\!\!\!\!
% \ \ \ \ \
%{\rm(e)\ }1-\d^4\le|X_1|\le\ell^2.
\end{eqnarray}
%where $\ell\in\R_{>0}$ such that $\ell\gg\ep^{-1}$.
%In particular we have \eqref{-EiathA0}.
% \item[\rm(2)]The second equality of \eqref{LetNSoOP}\,(i) implies that $|y_1|=\g_{\kk,\kk}$.
% \end{itemize}
\end{lemm}\noindent{\it Proof.~}Equ.~\eqref{ImMpP}\,(a) is simply \eqref{LetNSoOP}\,(ii).
If $|X_0|<\ep$, using that fact that $1\le C_1\le C_4$ and Equ.~\eqref{ImMpP}\,(a), we see that $|Y_1|\succeq\ep^{-2}\succ|X_0|$
 when $\ep^{-1}\gg\kk$
[cf.~Remark \ref{NotaRemak}$\ssc\,$].
Thus
by   \eqref{mqp1234-2}, \eqref{mqp1234-2+} and notation \eqref{SimMMSMS}, we must have $h_{p_0,p_1}\sim|x_1|\sim|y_1|\gg\ep^{-1}$, a contradiction with \eqref{ImMpP}\,(a).
Similarly, if $|X_0|>\ep^{-1}$, then $|Y_1|\preceq\ep^2$. By \eqref{LetNSoOP}\,(iii), we have \equa{SDmdeme}{\dis|X_0|^{\frac92}\le|X_1|^{\frac72+O(v_0^1)}|Y_1|^{1+O(v_0^1)},} thus
$|X_1|\succeq\ep^{-\frac{13}{7}+O(v_0^1)}\gg\ep^{-1}$, again a contradiction with \eqref{ImMpP}\,(a). Thus we have  \eqref{ImMpP}\,(b).
We claim  \equa{cLtHHH}{\mbox{$\dis \max\{|X_1|,|Y_1|\}\ge\d$.}}
 Otherwise, by the fact that $1\le C_2$, we obtain that $|X_0|>\d^{-5}$, a contradiction with \eqref{SDmdeme}. Thus
we have \eqref{cLtHHH}, which together with
\eqref{SDmdeme} [and notation \eqref{SimMMSMS}$\ssc\,$] shows that $\kk\preceq\max\{|x_1|,|y_1|\}$ when $\kk\gg1$ [cf.~Remark \ref{NotaRemak}$\ssc\,$],
 and $|x_0|\preceq \max\{|x_1|^{1+\d},|y_1|^{1+\d}\}$ [cf.~\eqref{v0=1=1=1=1}$\ssc\,$].
Thus   \eqref{mqp1234-2}, \eqref{mqp1234-2+} and notation \eqref{SimMMSMS} imply that we must have 
$h_{p_0,p_1}\preceq|x_1|^{1+\d}$ and $|x_1|\sim|y_1|$, %a contradiction with \eqref{ImMpP}\,(a).
%
%
%If $|X_1|<\ep$, using the fact that $1\le C_2\le C_4$, we obtain that $|Y_1|\succeq\ep^{-\frac12}$ when $\ep^{-1}\gg\kk$
%T[cf.~Remark \ref{NotaRemak}$\ssc\,$], and $|X_0|\preceq\ep^{\frac12}\prec|Y_1|$ by the fact that $1\le C_1\le C_4$. Thus
%by   \eqref{mqp1234-2}, \eqref{mqp1234-2+} and notation \eqref{SimMMSMS}, we must have $h_{p_0,p_1}\sim|x_1|\sim|y_1|$, a contradiction.
%
%If $|X_1|>\ep^{-1}$, similarly, we can obtain that $|Y_1|\preceq\ep^{\frac12}$, and by \eqref{LetNSoOP}\,(ii), \equa{M03048}{\mbox{$\dis|X_0|^{\frac{3}{2}+O(v_0^1)}\le|X_1|\cdot|Y_1|^{\frac12+O(v_0^1)}\prec|X_1|$, \ and thus $|X_0|\prec|X_1|.$}}
%Hence again we have
% $h_{p_0,p_1}\sim|x_1|\sim|y_1|$, a contradiction. Thus we have
%\eqref{ImMpP}\,(a).
%Then \eqref{ImMpP}\,(b) and (c) simply follow from \eqref{LetNSoOP}\,(i).
%
%Using the fact that $1\le C_2$ and the first inequality of \eqref{M03048} [and notation  \eqref{SimMMSMS}$\ssc\,$], we see that
%$\kk\preceq\max\{|x_0|,|y_1|\}$ when $\kk\gg1$ and $|x_0|\preceq\max\{|x_0|^{1+\d},|y_1|\}$ [cf.~\eqref{v0=1=1=1=1}$\ssc\,$], thus
%\eqref{mqp1234-2} and \eqref{mqp1234-2+} imply that $h_{p_0,p_1}\preceq|x_1|^{1+\d}\sim|y_1|^{1+\d}$,
%
%Equ.~\eqref{ImMpP}\,(a) and (b) follow from the facts that $1\le C_1,C_3\le C_4$.
%from this, % and notation \eqref{SimMMSMS} and the fact that $C_1\le C_3$, we obtain that $\kk\preceq|x_0|\preceq|x_1|^{1+\d}$ when $\kk\gg1$ [cf.~Remark \ref{NotaRemak} and \eqref{v0=1=1=1=1}$\ssc\,$].
%By \eqref{mqp1234-2+}, we must have $h_{p_0,p_1}\preceq|y_1|^{1+\d}\sim|x_1|^{1+\d}$, which implies
from this we obtain \eqref{ImMpP}\,(c)  as in the proof of
Lemma \ref{YYYy1==}.\hfill$\Box$
%, we see
%that $\kk\preceq|x_1|$ and $|x_0|\preceq|x_1|$ when $\ell\gg\kk\gg1$ [cf.~Remark \ref{NotaRemak}\,(i)$\ssc\,$].
%
%If $|X_1|<\frac14$, then $|X_0|<\frac12$ (by the fact that $1\le C_2$), but then $C_4<1$, a contradiction.
%Thus we have the first inequality of \eqref{ImMpP}\,(a). From this and the fact that $1\le C_2$ [and notation \eqref{SimMMSMS}$\ssc\,$], we see
%that $\kk\preceq|x_1|$ and $|x_0|\preceq|x_1|$ when $\ell\gg\kk\gg1$ [cf.~Remark \ref{NotaRemak}\,(i)$\ssc\,$]. Thus
%by \eqref{mqp1234-2+}, we must have $h_{p_0,p_1}\sim|y_1|\sim|x_1|$, which implies \eqref{ImMpP}\,(c)  as in the proof of
%Lemma \ref{YYYy1==}.
%
%Assume $|X_1|>\ell^4$. By \eqref{ImMpP}\,(c),
%By \eqref{mqp1234-2+} and the fact that $1\le C_2$ [and notation \eqref{SimMMSMS}$\ssc\,$], we must have
%we have $h_{p_0,p_1}\sim|y_1|\sim|x_1|$ when $\ell\gg\kk$
%[note that when we say $\ell\gg\kk$,
%we regard $\kk$ as fixed; cf.~Remark \ref{NotaRemak}\,(i)$\ssc\,$].
%However, we have
%\equa{USUSUS}{
%|Y_1|\le\ell^{\frac{c_k-1}{2c_k+1}}|X_0|<\ell^{\frac12}|X_0|\le\ell^{\frac12}|X_1|^{\frac12}<\ell^{-1}|X_1|,}
%a contradiction. Thus we have \eqref{ImMpP}\,(a), which together with the fact that $1\le C_2\le\ell$ implies \eqref{ImMpP}\,(b).



%We will need the following very facts,
%\equa{VySkM}{\dis
%{\rm(i)}^{\pm} \ %\frac1
%{\Big|3\frac{X_1^3}{X_0^2}-2\Big|}%\le
%\ge\pm\Big(%\frac{1}
%{3\frac{|X_1|^3}{|X_0|^2}-2}\Big),\ \ \ \ \
%{\rm(ii)}\ \ %\frac1
%{\Big|3\frac{X_1^3}{X_0^2}-2\Big|}%\ge
%\le%\frac1
%{3\frac{|X_1|^3}{|X_0|^2}+2}.
%}
\NOUSE{%
Equ.~\eqref{ImMpP}\,(a) is simply \eqref{LetNSoOP}\,(ii). From this and \eqref{LetNSoOP}\,(iii) [and notation \eqref{SimMMSMS}$\ssc\,$], we see that $|x_0|\preceq|x_1|\sim\kk$ when $\kk\gg\ell$
[note that when we say $\kk\gg\ell$,
we regard $\ell,\ell_1$ as fixed; cf.~Remark %\ref{rema-kk}$\ssc\,$] [cf.~Remark
\ref{NotaRemak}\,(i)$\ssc\,$].
Thus we have \eqref{ImMpP}\,(c)  as in the proof of
Lemma \ref{YYYy1==}.
%
If $|X_0|<\d^{\frac13}$, then using \eqref{ImMpP}\,(c), we have that $C_2\succeq\ell\gg\ell_1$ when $\ell=\d^{-1}\gg\ell_1$
[note that when we say $\ell\gg\ell_1$,
we regard $\kk,\ell_1$ as fixed
(but we still require that $\kk\gg\ell$); see also Remark
\ref{NotaRemak}\,(i)$\ssc\,$], a contradiction. Similarly, if $|X_0|>\ell^{\frac13}$ [then $\frac{1}{|X_0|^3}=O(\d)\ssc\,$], then
$C_2=\frac12+O(\d)<1$, again a contradiction. Thus we have \eqref{ImMpP}\,(b).
}%
%cf.~Remark %\ref{rema-kk}$\ssc\,$] [cf.~Remark
%\ref{NotaRemak}\,(i)$\ssc\,$].
%
%If $|X_1|<\d$, then by the fact that $1\le C_1\le\ell_1$ and $\ell_1\gg\ell=\d^{-1}$ when $\ell\gg\ell_1$
%[note that when we say $\ell\gg\ell_1$,
%we regard $\ell_1,\kk$ as fixed (but we still require that $\kk\gg\ell$; see also Remark %\ref{rema-kk}$\ssc\,$] [cf.~Remark
%\ref{NotaRemak}\,(i)$\ssc\,$], we must have $|X_0|\preceq\d^2$. Then using \eqref{VySkM}\,(i)${}^+$, we can obtain
%that $C_2\preceq\d\ll1$, a contradiction. If $|X_1|>\ell$, then similarly, we obtain that $|X_0|\succeq\ell^2$ and
%using \eqref{VySkM}\,(i)${}^-$, we obtain that $C_2\le\frac1{2+O(\d)}<1$, again a contradiction. Thus we have \eqref{ImMpP}\,(a).
%Now using %\eqref{LetNSoOP}\,(ii) and
%the fact that $1\le C_3\le \ell_1$, we have \eqref{ImMpP}\,(b).
%Using \eqref{ImMpP}\,(a),\,(b) and notation \eqref{SimMMSMS}, we see that $|x_0|\sim|x_1|\sim\kk$ when $\kk\gg\ell$
%[note that when we say $\kk\gg\ell$,
%we regard $\ell,\ell_1$ as fixed], thus
% as in the proof of
%Lemma \ref{YYYy1==}, we  have \eqref{ImMpP}\,(c).
%
%when we say
%
%we see that $|X_0|,|Y_1|\le\ep$, thus by notation \eqref{SimMMSMS}, $|x_0|,|x_1|,|y_1|<1$, a contradiction with
%\eqref{sssA00-eta+}. Assume $|X_1|>\ell^{\frac1{\scep^2}}$.
%By the fact that $1\le C_2$, we see that $|x_0|\preceq |x_1|^{1+\d+v_0}$  when $\ell\gg\ep$ (cf.~Remark \ref{NotaRemak}$\ssc\,$), thus
%\eqref{mqp1234-2} and \eqref{mqp1234-2+} imply that we must have (as we can assume $\d+v_0<\frac1m\ssc\,$), \equa{Mehhhhhh}{\mbox{$\dis h_{p_0,p_1}\preceq|x_1|^{1+\d+v_0}\sim|y_1|^{1+\d+v_0}$,}}
%and so (noting that $\ell\gg\ep^{-1}\ssc\,$), we have $C_1\sim |X_1|^{\frac{1+\d+v_0}{v_0}(1-\ell_1^{-1})\scep}\gg\ell$, a contradiction.
%Hence we have \eqref{ImMpP}. Now using the facts that $1\le C_1,C_2\le\ell$, we have \eqref{-EiathA0}.
%
%$|Y_1|\preceq\ell^{-1}\ll1$ when $\ell\gg\ep^{-1}$. Thus $|X_0|\preceq\ell^{-1}\ll1$ by the fact $C_3\le\ep^{-1}$. We obtain a contradiction with \eqref{sssA00-eta+}.
%%
%Assume $|X_1|\ge\ell$. Then $|Y_1|\sim\ell$ by the fact that $\ep\le C_2\le\ep^{-1}$.
%Then $|X_0|\sim$
%
%Denote $C'_1=\ep^2 X_0+(1-\ep^2)X_1 $, $C'_2=
%(1+\ep^2)X_0-\ep^2 X_1$.
%Then one can easily verify
%\equa{Thsmememe}{\dis
%{\rm(i)\ }C_1'=\frac{\ep^2}{1+\ep^2}C'_2+\frac{X_1}{1+\ep^2},\ \ \ \
%{\rm(ii)\ }C_2'=\frac{X_0}{1-\ep^2}-\frac{\ep^2}{1-\ep^2}C'_1.
%}
\NOUSE{%
We will need the following very simple facts,
\begin{eqnarray}
\label{Vemsms}
&\!\!\!\!\!\!\!\!\!\!\!\!\!\!\!\!&
{\rm(i)}^{\pm}\
\Big|(1+\ep^{-5})X_0^{^{\sc\frac{\a_1}{v_0}}}-\ep^{-5} Y_1^{^{\sc\frac{\a_1}{v_0(1-v_0)}}}\Big|\ge
\pm\Big((1+\ep^{-5})|X_0|{^{\sc\frac{\a_1}{v_0}}}-\ep^{-5} |Y_1|{^{\sc\frac{\a_1}{v_0(1-v_0)}}}\Big),\nonumber
\\[4pt]
&\!\!\!\!\!\!\!\!\!\!\!\!\!\!\!\!&
{\rm(ii)}\ \
\Big|(1+\ep^{-5})X_0^{^{\sc\frac{\a_1}{v_0}}}-\ep^{-5} Y_1^{^{\sc\frac{\a_1}{v_0(1-v_0)}}}\Big|\le
(1+\ep^{-5})|X_0|{^{\sc\frac{\a_1}{v_0}}}+\ep^{-5} |Y_1|{^{\sc\frac{\a_1}{v_0(1-v_0)}}}.
\end{eqnarray}
}%
%Equ.~\eqref{ImMpP}\,(a) simply follows from the facts that $1\le C_1,\,C_3\le C_4$. From this we see that
%$\kk\preceq|x_1|$ when $%\ell\gg\ep^{-1}\gg
%\kk\gg1$
%(cf.~notation \eqref{SimMMSMS} and Remark \ref{NotaRemak}$\ssc\,$).
%Then by \eqref{LetNSoOP}\,(ii), we see that $|x_0|\le|y_1|^{1+2\d}$. Thus
%\eqref{mqp1234-2} and \eqref{mqp1234-2+} imply that we must have $h_{p_0,p_1}\sim|x_1|^{1+2\d}\sim|y_1|^{1+2\d}$.
%Then we have \eqref{ImMpP}\,(c) as in the proof of Lemma \ref{YYYy1==} [cf.~Remark \ref{NotaRemak}\,(iii)$\ssc\,$].
%
%If $|X_0|<1-\d$, using \eqref{Vemsms}\,(i)${}^-$ and the fact that $C_2\le C_3$, we obtain
%\begin{eqnarray}
%\label{od0ei3jnnn}
%&\!\!\!\!\!\!\!\!\!\!\!\!\!\!\!\!\!\!&
%\ep^{-5}|Y_1|^{\frac{\a_1}{v_0(1-v_0)}}
%\le(1+\ep^{-5})(1-\d)^{\frac{\a_1}{v_0}}+|X_1|^{\frac{\a_1(1+v_0^5)}{v_0}}
%\nonumber\\[4pt]
%&\!\!\!\!\!\!\!\!\!\!\!\!\!\!\!\!\!\!&
%\phantom{\ep^{-5}|Y_1|^{\frac{\a_1}{v_0(1-v_0)}}}
%\le\ep^{-5}\Big(1-\frac{\a_1}{v_0}\d+O(\d^2)\Big)+1-\frac{\a_1\d}{v_0}+O(\d^2)+|X_1|^{\frac{\a_1(1+v_0^5)}{v_0}}
%\nonumber\\[4pt]
%&\!\!\!\!\!\!\!\!\!\!\!\!\!\!\!\!\!\!&
%\phantom{\ep^{-5}|Y_1|^{\frac{\a_1}{v_0(1-v_0)}}}
%\le\Big(\ep^{-5}\Big(1-\frac{\a_1}{v_0}\d+O(\d^2)\Big)+1-\frac{\a_1\d}{v_0}+O(\d^2)+1\Big)
%|X_1|^{\frac{\a_1}{v_0(1-v_0)}},
%\end{eqnarray}
%where the last inequality follows by noting that $|X_1|\ge1$ and $\frac{\a_1(1+v_0^5)}{v_0}<\frac{\a_1}{v_0(1-v_0)}$.
%By noting that $0<\ep\ll v_0\ll\d$, we obtain from \eqref{od0ei3jnnn} that
%$|Y_1|\le\big(1-\d+O(\d^2)\big)|X_1|$, a contradiction with \eqref{ImMpP}\,(c). If $|X_0|>\ell^2$, then by
%\eqref{LetNSoOP}\,(ii)  [cf.~Remark \ref{NotaRemak}\,(iii)$\ssc\,$] and the fact that $|X_1|\le\ell$, we have $|Y_1|\ge\ell^{5-4v_0}\gg|X_1|$, again  a contradiction with \eqref{ImMpP}\,(c).
%We have \eqref{ImMpP}\,(b).
%Using \eqref{Thsmememe} and the facts that $C_2,C_3\le C_4=1+O(\ep^4)$, we immediate obtain the second inequalities of
%\eqref{ImMpP}\,(a),\,(b). If $|X_0|<1-\d$, then $1\le|C_2'|\le (1+\ep^2)(1-\d)+\ep^2|X_1|$ and so $|X_1|\ge\frac{\d-(1-\d)\scep^2}{\scep^2}\gg1+O(\ep^2)$, a contradiction. Thus we have \eqref{ImMpP}\,(a).
%If $|X_1|<1-\d$, then \eqref{LetNSoOP}\,(ii) implies that $|Y_1|>(1-\d)^{-\frac1{4\scep^2}}\gg\ep^{-1}\gg\max\{|X_0|,|X_1|\}\succeq\kk$
% when $\ep^{-1}\gg\kk\gg1$
%contradiction.
%Thus we have \eqref{ImMpP}\,(b).
%
%By \eqref{ImMpP}\,(a),\,(b) and notation \eqref{SimMMSMS}, we see that $|x_0|\sim|x_1|\sim\kk$ when $\kk\gg1$. Thus
%\eqref{mqp1234-2} and \eqref{mqp1234-2+} imply that $h_{p_0,p_1}\sim\kk$.
%
%
%\eqref{LetNSoOP}\,(i) can be rewritten as
%\equa{Re-LetNSoOP}{\!\!\!\!
%\dis1\!\le\! C_1\!:=\!\Big|\frac{1\!+\!\d}{1\!-\!\d}C'_2\!-\!\frac{2\d}{1\!-\!\d}X_1\Big|^{1-\kk^{-5}}
%\!\le\! C_2\!:=\!|C_2'|^{1+\frac{2\d}{1-\d}}\!\le\!\Big|\frac{1\!+\!\d}{1\!-\!\d}C'_2\!-\!
%\frac{2\d}{1\!-\!\d}X_1\Big|^{1+\kk^{-5}}\!\le\!2^{1+\kk^{-5}}.\!\!\!\!}
%If $|X_1|<\d^2$, then from the fact that $C_1\le C_2$, we obtain
%[where $e$ is the natural number; recall that $1\le |C'_2|\le2^{\frac{1-\d}{1+\d}}$,
%here and the following we evaluate elements up to $O(\kk^{-5}){\ssc\,}$] \equa{m44444}{\dis\!\!\!\!\!\!
%\frac{1\!+\!\d}{1\!-\!\d}|C'_2|\!+\!O(\d^3)\!\le\! |C'_2|^{1+\frac{2\d}{1+\d}}, \mbox{  thus, } |C'_2|\!\ge\!\Big(\frac{1\!+\!\d}{1\!-\!\d}\!+\!O(\d^3)\Big)^{\frac{1+\d}{2\d}}\!\ge \! e\!+\!O(\d^1)\!>\!2^{\frac{1-\d}{1+\d}},\!\!\!\!}
%a contradiction.
%Thus we have the first inequality of
%\eqref{ImMpP}\,(a), from which and using the fact that $(1+\d)|X_0|\le2+\d|X_1|$
%(obtained from the fact that $C_3\le C_4\ssc\,$), we see that
%$\kk\preceq|x_0|\preceq|x_1|$ when $\kk\gg1$ (cf.~notation \eqref{SimMMSMS} and Remarks \ref{rema-kk} and \ref{Fimsmsm}$\ssc\,$). Thus by
%\eqref{mqp1234-2} and \eqref{mqp1234-2+},
%
%we must have $h_{p_0,p_1}\sim|x_1|\sim|y_1|$.
%Then we have \eqref{ImMpP}\,(c)
%
%Now assume $|X_1|>1+\d$. Then \eqref{LetNSoOP}\,(ii) implies that $|Y_1|\ge(1+\d)|X_1|$, a contradiction with \eqref{ImMpP}\,(c). Thus we have \eqref{ImMpP}\,(a).
%Finally \eqref{ImMpP}\,(b) follows from \eqref{ImMpP}\,(a) and the facts that $1\le C_1,C_3\le C_4$.
%
%Equ.~\eqref{ImMpP}\,(a),\,(b) simply follow from the facts that $1\le C_1,\,C_2\le C_4$.
%By \eqref{ImMpP}\,(a),\,(b) and the fact that $C_1\le C_2$, we see that $\kk\preceq|x_0|\preceq|x_1|$ when $\ell=\ep^{-1}\gg\kk\gg1$ [cf.~Remark \ref{NotaRemak} and notation \eqref{SimMMSMS}$\ssc\,$], which together with
%
%Assume $|X_1|>\ell^{\ell^2}$.
%Then $\frac{|Y_1|}{|X_1|^5}\preceq\ep^{4\ell^2}$ when $\ell=\ep^{-1}\gg\kk$. By \eqref{11simMSMpp}\,(i)${}^-$, we see that $C_2>C_4$, a contradiction.
%By \eqref{ImMpP}\,(c), we have $|Y_1|\sim|X_1|\succeq\ell^4$
% %[thus $\frac1{|X_1|}\preceq\ep^4{\ssc\,}$]
% when $\ell=\ep^{-1}\gg\kk$.
%Thus $\frac{|Y_1|}{|X_1|^5}\preceq\ep^{16}$ and $\frac1{|X_1|}\preceq\ep^4,$ and so
%by \eqref{11simMSMpp}\,(ii), we have $C_2\preceq \ep^{\frac{\scep}{\a_0}}<1$, a contradiction with \eqref{LetNSoOP}\,(i).
%
%Using the fact that $1\le C_1\le C_3$, we have \eqref{ImMpP}\,(a), and we have
%\eqref{ImMpP}\,(b) by \eqref{LetNSoOP}\,(ii). Using \eqref{ImMpP}\,(a),\,(b), %we see that
%
%The first inequality of
%\eqref{ImMpP}\,(b) follows from \eqref{LetNSoOP}\,(iii) and the fact that $C_1\ge1$. The second inequality of
%\eqref{ImMpP}\,(b) follows from \eqref{ImMpP}\,(a) and the fact that $1\le C_1$.
%
%
%Observe the following.\begin{itemize}\item[(1)]
%Equ.~\eqref{ImMpP}\,(a) is simply \eqref{LetNSoOP}\,(ii), and \eqref{ImMpP}\,(b) follows from the fact that $1\le C_2\le C_4$, in addition,
%\eqref{ImMpP}\,(c) follows from \eqref{LetNSoOP}\,(iii).\item[(2)]
%By \eqref{ImMpP}\,(a),\,(b),
%\eqref{mqp1234-2}
%and notation \eqref{SimMMSMS},
%we see that $h_{p_0,p_1}\sim\kk\sim|x_0|\sim|y_1|$ (when $\kk\gg\ell$; note that when we say $\kk\gg\ell$, we regard $\ell_1,\ell$ as fixed; see also Remark \ref{rema-kk}$\ssc\,$]. From this, as in the proof of
%Lemma \ref{YYYy1==}, we  have \eqref{ImMpP}\,(c).
%\item[(3)]The first inequality of \eqref{ImMpP}\,(e) follows from
%\eqref{ImMpP}\,(c),\,(d). If $|X_1|>\ell^2$, by \eqref{ImMpP}\,(d), $|Y_1|\sim|X_1|$
%[when $\ell=\d^{-1}\gg\ell_1$; note that when we say $\ell\gg\ell_1$, we regard $\ell_0,\ell_1,\kk$ as fixed
%(but we still required $\kk\gg\ell{\ssc\,}){\ssc\,}$; see also Remark  \ref{rema-kk}$\ssc\,$],
%but then by
%\eqref{ImMpP}\,(a), see that $C_2\ll1$, a contradiction.
%
%
\NOUSE{%
We will need the following simple fact,
\begin{eqnarray}
\label{simMSMpp}
&\!\!\!\!\!\!\!\!\!\!\!\!\!\!\!\!\!\!\!\!\!\!\!\!&
{\rm(i)}^{\pm}\ \big|(1\!-\!\ep)\bar x_1^{-1}x_1-\ep\bar x_0^{-1}x_0\big|
\ge\pm\big((1\!-\!\ep)\kk^{-1}|x_1|-\ep\kk^{-1}|x_0|\big),
\nonumber\\[4pt]
&\!\!\!\!\!\!\!\!\!\!\!\!\!\!\!\!\!\!\!\!\!\!\!\!&
{\rm(ii)}\, \,\ \big|(1\!-\!\ep)\bar x_1^{-1}x_1-\ep\bar x_0^{-1}x_0\big|
\le(1\!-\!\ep)\kk^{-1}|x_1|+\ep\kk^{-1}|x_0|.
\end{eqnarray}
%
%
%
%
%We prove as follows.\begin{itemize}\item[(1)]
%If $|X_0|<\ep$, i.e., $|X_0|=O(\ep^1)$, then by the fact that $1\le C_1\le C_4$, we must have $|Y_1|=O(\ep^1)$.
%If $|X_1|\ge\kk^{-1}$ then we would obtain a contradiction that $C_2\gg C_4$.
%Thus $|X_1|<\kk^{-1}$, i.e., $|x_1|<1$, but then we obtain a contradiction with \eqref{sssA00-eta+}.
%\item[(2)]If $|X_0|>\big(\frac1{v_0}\big)^{\frac1{v_0}}$, then as in case (1), we must have $|Y_1|\sim|X_0|\succeq \big(\frac1{v_0}\big)^{\frac1{v_0}}$ (when $\kk\gg1$, i.e., $v_0\ll1$). Thus $|y_1|\sim|x_0|\succ\kk \big(\frac1{v_0}\big)^{\frac1{v_0}}$. By
%\eqref{mqp1234-2}, we must have $h_{p_0,p_1}\sim|y_1|$. Then as in the proof of Lemma \ref{YYYy1==}, we must have $|x_1|\sim|y_1|$, but then $C_2\sim\frac1{|Y_1|^{2v_0}}\ll1$, a contradiction. We have \eqref{ImMpP}\,(a).
%\item[(3)]As in case (1), we have the first inequality of \eqref{ImMpP}\,(b). If $|X_1|>\big(\frac1{v_0}\big)^{\frac2{v_0}}$, using \eqref{ImMpP}\,(a) and as in case (2), we have
%$h_{p_0,p_1}\sim|X_1|\sim|Y_1|$, but then $C_1\ll1$ by \eqref{ImMpP}\,(a), a contradiction. We have \eqref{ImMpP}\,(b).
%\item[(4)]Equ.~\eqref{ImMpP}\,(c) is obtained from \eqref{LetNSoOP}\,(ii).
\hfill$\Box$
%\end{itemize}
%
%
\NOUSE{%
%
Now
\eqref{ImMpP}\,(a) follows from the facts that $1\le C_1$ and $C_3\le C_4$. %is simply \eqref{LetNSoOP}\,(ii).
By \eqref{LetNSoOP}\,(ii), we have \eqref{ImMpP}\,(b).
We claim the following
\equa{mdmdh}{h_{p_0,p_1}\sim |x_1|\sim |y_1|\mbox{ \ (when $\kk\gg1$)}.}
To see this, first assume $|x_0|\prec\kk$, then \eqref{ImMpP}\,(a) implies that $|x_1|\sim\kk\succ|x_0|$, thus by
\eqref{mqp1234-2}, we must have \eqref{mdmdh}. Now assume $|x_0|\succeq\kk$, then
\eqref{ImMpP}\,(b) shows that $\kk\preceq|x_0|\preceq|y_1|$, again by
\eqref{mqp1234-2}, we must have \eqref{mdmdh}.
Now by \eqref{mdmdh}, as in the proof of Lemma \ref{YYYy1==}, we obtain \eqref{ImMpP}\,(d).
The last inequality of \eqref{ImMpP}\,(c) follows from \eqref{simMSMpp}\,(i)${}^-$ and the fact that $C_2\le C_4$.
%
We claim that there exists some $\th_0\in\R_{>0}$ such that the first inequality of  \eqref{ImMpP}\,(c) can follow from \eqref{LetNSoOP}\,(i).
Assume it is not so.
Then it means that [the above has in particular proved that $(x_0,x_1)$ must be bounded when \eqref{LetNSoOP}\,(i) holds]
 there exists a converging sequence
$(x_{0i},x_{1i}),\,i=1,2,...$, each $(x_{0i},x_{1i})$ satisfies
\eqref{LetNSoOP}\,(i), such that $x_{0i}$ converges to zero. Denote $\lim_{i\to\infty}\kk^{-1}|x_{i1}|=\hat x_1$.
Using $(x_{0i},x_{1i})$ in \eqref{LetNSoOP}\,(i) and taking the limit, we obtain that $1\le\hat x_1^{\frac13-\scep}\le(1-\ep)\hat x_1\le\hat x_1^{(\frac13-\scep)(1+\scep^2)}$
%
%
%
%
% and the fact that $C_2\ge1$ (thus $C_{20}\ge1$), we immediately obtain
%that $|Y_1|^{2\d ^4}\ge1+\d ^4\ep$, i.e., we have
%\eqref{ImMpP}\,(b). By \eqref{ImMpP}\,(a),\,(b), we see that $|x_0|\sim\kk\preceq|y_1|$ [when $\kk\gg\ell $; cf.~\eqref{TaKa}, \eqref{SimMMSMS} and Lemma \ref{YYYy1==}; note that when we say $\kk\gg\ell$, we regard $\ell_0,\ell_1,\ell$ as fixed]. From this, as in the proof of
%Lemma \ref{YYYy1==}, we obtain \eqref{ImMpP}\,(d). By \eqref{ImMpP}\,(b),\,(d), we have the first inequality
%of \eqref{ImMpP}\,(c).
%
%If $|X_1|>\ell$, then by %\eqref{ImMpP}\,(d) and
%the fact that $1\le C_1%\le C_4
%$, we obtain %,  when $\ell\gg\ell_1$, that
%$|X_0|>\ell%\sim|X_1|^{1-\d +\d ^2}\gg\ell
%$, a contradiction with \eqref{ImMpP}\,(a).
%From this and the fact that
%$1\le C_1\le C_4$, we immediately have
%\eqref{ImMpP}\,(b) [cf.~\eqref{Td-d1}$\ssc\,$].
%
\NOUSE{%
We will need the following simple facts.
\begin{eqnarray}\label{SimPPPP}
\!\!\!\!\!\!\!\!\!\!\!\!\!\!\!\!&&
{\rm(i)^{\pm}\ }
\Big|2\big(\bar x_1^{-1}x_1\big)^{\frac{2}{v_0^3\scep}}-\big(\bar x_0^{-1}x_0\big)^{\frac{1}{v_0^3\scep}}\Big|
\ge
\pm\Big(2\big(\kk^{-1}|x_1|\big)^{\frac{2}{v_0^3\scep}}-\big(\kk^{-1}|x_0|\big)^{\frac{1}{v_0^3\scep}}\Big)
,\nonumber\\[4pt]
\!\!\!\!\!\!\!\!\!\!\!\!\!\!\!\!&&
{\rm(ii)\ \ \, }
\Big|2\big(\bar x_1^{-1}x_1\big)^{\frac{2}{v_0^3\scep}}-\big(\bar x_0^{-1}x_0\big)^{\frac{1}{v_0^3\scep}}\Big|
\le
2\big(\kk^{-1}|x_1|\big)^{\frac{2}{v_0^3\scep}}+\big(\kk^{-1}|x_0|\big)^{\frac{1}{v_0^3\scep}}
.\end{eqnarray}Now
\eqref{ImMpP}\,(a) follows from
the fact that $1\le C_2\le C_3$. The first inequality of \eqref{ImMpP}\,(b) follows from the fact $C_1\le C_2$ and \eqref{SimPPPP}\,(i)${}^+$, and the second from the fact that $C_1\le C_3$ and \eqref{SimPPPP}\,(i)${}^+$.
Obviously \eqref{ImMpP}\,(c) follows from \eqref{ImMpP}\,(b) and \eqref{LetNSoOP}\,(ii).
\NOUSE{%
\eqref{SimPPPP}\,(i),\,(ii)${}^{+}$ and the fact that $1\le C_2\le1+\d$.
Using \eqref{SimPPPP}\,(iii),\,(iv)${}^{+}$ and the facts that $1\le C_1\le1+\d$ and $1\le C_2\le1+\d$, we obtain
\eqref{ImMpP}\,(a), where
the first equality is obtained by noting the following [since we only compute the following up to $O(\d^3)$, we may assume $v_0=0$ by \eqref{v2=Aq}$\ssc\,$]
\equa{F0000}{\dis
\frac{(1+\d)^{-\a_1}(1+\d^2)\!-\!1}{\d^2}=
\frac{(1+\d)^{-(\d^2+\d^3-\d^4)}(1+\d^2)-1+O(\d^5)}{\d^2}=1-\d-\frac{\d^2}{2}+O(\d^3).
}
%
}%
}%
%By \eqref{ImMpP}\,(a),\,(b) and \eqref{mqp1234-2+}, we see that $h_{p_0,p_1}\preceq\kk$ (when $\kk\gg\ell$), from this
%we obtain  \eqref{ImMpP}\,(c) as in the proof of Lemma \ref{YYYy1==}.
%
%
% and the second inequality follows from \eqref{SimPPPP}\,(ii)${}^{\pm}$ and the fact that $C_2\le C_4$. Using $1\le C_1\le C_4$, we obtain \eqref{ImMpP}\,(b). Now \eqref{ImMpP}\,(c) follows from
%\eqref{LetNSoOP}\,(ii) and the fact that $C_2\ge1$.
\NOUSE{%
We prove as follows.\begin{itemize}\item[(1)]
If $|x_0|<\d$ (then $|x_0|<1$), using \eqref{v1====}, Lemma \ref{YYYy1==} and the fact that $1\le C_2\le1+\d$, we have $|y_1|=\d+O(\d^2)<1$, and
using $1\le C_1\le 1+\d$, we have $|x_1|=\d+O(\d^2)<1$, a contradiction with \eqref{sssA00-eta+}. We have the first inequality of \eqref{ImMpP}\,(a).
\item[(2)]If $\ell_0:=|x_0|>\ell,$ then as in case (1), we have $|y_1|\sim\ell_0^{\frac1{1+2v_0}}$ (when $\ell\gg\kk$), and
        \equa{x--1}{\dis |x_1|\sim
    \big(\ell_0^{\frac1{1+2v_0}-(\frac12+v_0)}\big)^{\frac1{\frac12+v_0}}
    =\ell^{\a_1}\prec|y_1|,
    \mbox{ where }\a_1=1-8v_0+O(v_0^2)<\frac1{1+2v_0},}
which implies a contradiction as in \eqref{AnDSo} by noting that $h_{p_0,p_0}\!\sim\!|x_0|\!\sim\!\ell_0\prec|y_1|^{\frac{m+1}{m}}$.
We have the last inequality of \eqref{ImMpP}\,(a).
\item[(3)]The first inequality of \eqref{ImMpP}\,(b) follows from \eqref{LetNSoOP}\,(ii) and the fact that $C_1^{-1}\ge(1+\d)^{-1}$. The last inequality of \eqref{ImMpP}\,(b) can be proven as in case (2).
\item[(4)]One can easily obtain \eqref{ImMpP}\,(c) by the fact that $1\le C_1\le1+\d$.
\item[(5)]We have \eqref{ImMpP}\,(d) as in the proof of Lemma \ref{YYYy1==}. \hfill$\Box$   \end{itemize}
}%\hfill$\Box$%
%Obviously we have (a) by \eqref{LetNSoOP}\,(i).
%Then we have (b) by the fact that $\d\le C_1\le C_3$. The first inequality of (c) follows from the second inequality of (b),
%\eqref{LetNSoOP}\,(iii) and the fact that $\g_{\kk,\kk}>\kk$.
%Part (a) follows from the fact that $\d\le C_1\le\C_3$. We will need the following simple fact
% and (b) from \eqref{LetNSoOP}\,(i).
%Observe that (a) follows from the facts that $1\le C_1$ and $C_3\le C_4$. Then (b) follows from
%\eqref{LetNSoOP}\,(ii) and (iii). %, and
%(c) from the fact that $1\le C_3\le C_4$. %Using \eqref{LetNSoOP}\,(i), One can easily verify (a) and (b).
%Using $C_0\le C_1\le C_3$, we obtain (a) (and also $C_3\le1$).
%One can easily obtain (a), (b) from \eqref{LetNSoOP}\,(i).
%We will need the following simple fact
%[cf.~\eqref{TaKa}],
%Using the fact that $1\le C_1,C_2$, we obtain the first inequality of (a) or (b). Using \eqref{msmsm}\,(ii) and the fact that $C_3\le C_4$, we have the second inequality of (b). Then
%the second inequality of (a) follows from the fact that $C_1\le C_2$.
%
%
%Using $\d\le C_2$ and \eqref{msmsm}\,(i), we obtain the first inequality of (b) by noting that $\d^\scep=1+O(\ep^1)$, using $C_2\le C_3$, \eqref{msmsm}\,(ii)  and (a), we have the second inequality of (b).
%Using \eqref{msmsm} and $C_2\le C_3$ and the facts that $\kk^{-1}|x_1|\le1$ and $\frac{\ln(2)}{3\scep}\gg1$, we obtain \equa{Wemdmd}{\mbox{$\!\!\!\!\!\!\dis-\kk^{-1}|x_1|\le-(\kk^{-1}|x_1|)^{\frac{\ln(2)}{3\scep}(1-\scep)}\le
%\kk^{-1}\big(|x_0|-(1-\ep)|x_1|\big)\le(\kk^{-1}|x_1|)^{\frac{\ln(2)}{3\scep}(1-\scep)}\le \kk^{-1}|x_1|\!\!\!$,}}i.e., we have (b).
%
%By the first and last inequalities of \eqref{LetNSoOP}\,(i), we have (a).
%From this and the second and third inequalities of  \eqref{LetNSoOP}\,(i),
%we immediately obtain (b) [noting that $(1-\d)^{\scep}=1+O(\ep^1){\sc\,}$].
%Using
% \eqref{LetNSoOP}\,(ii) and (b), we have (c).
%By (a) %, (b)
%We will need the following simple facts,
%\begin{eqnarray}
%\label{SimPls}
%&\!\!\!\!\!\!\!\!\!\!\!\!\!\!\!\!\!\!\!\!&
%{\rm(i)}^{\pm}\ \
%\pm\big(\kk|x_0|^{-1}\!-\!(1\!+\!\ep)\kk^{-2}|x_1|^2\big)\le
%\big|\bar x_0x_0^{-1}-(1+\ep)\bar x_1^{-2}x_1^2\big|,
%\ \ {\rm(ii)\ }
%(1\!+\!\ep)\kk^{-2}|x_1|^2\!-\!\kk|x_0|^{-1}\le
%|\bar x_0x_0^{-1}-(1+\ep)\bar x_1^{-2}x_1^2|,
%\nonumber\\[4pt]
%&\!\!\!\!\!\!\!\!\!\!\!\!\!\!\!\!\!\!\!\!&
%{\rm(ii)\ \ \ \,}\big|\bar x_0x_0^{-1}-(1+\ep)\bar x_1^{-2}x_1^2\big|
%\le
%\kk|x_0|^{-1}\!+\!(1\!+\!\ep)\kk^{-2}|x_1|^2.\end{eqnarray}
%Denote by $C'_2$ the element inside the absolute sign $|\cdot|$ of $C_2$.
%If $\kk^{-1}|x_0|>\d^{-2}$, using \eqref{TaKa}, \eqref{SimPls}\,(a) and \eqref{ImMpP}\,(i)$^-$, we see
%\equa{CPrime}{\mbox{ $\dis\frac{1+\ep-\d^2}{\ep}<\frac{(1+\ep)\kk^{-2}|x_1|^2-\kk|x_0|^{-1}}{\ep}\le
%|C'_2|=C_2\le\d^{-1}$,}}
%which is a contradiction. Similarly, we can obtain a contradiction if $\kk^{-1}|x_0|<\d^2$
%by \eqref{TaKa}, \eqref{SimPls}\,(a) and \eqref{ImMpP}\,(i)$^+$.
%Thus we have (b).
%Then (a), (b)
%, using
% thus $C_2=|C'_2|^{\frac1{\scep}}<1-\d$ (when $\kk\gg1$ by noting that $\frac1{\scep}>\kk^{\kk}$),
% a contradiction with the fact that $C_2\ge1-\d$. Thus $\kk^{-1}|x_0|\le1+\d$. This, together with
% \eqref{ImMpP}\,(a)
%and \eqref{mqp1234-2+}, we see that $h_{p_0,p_1}\preceq\kk$, from this we obtain the second inequality of (c) as in %follows from
%the proof of Lemma \ref{YYYy1==}. %Using (a), (c) and \eqref{LetNSoOP}, we can obtain that $\kk^{-1}|x_0|\ge1-\d$, i.e., we have (b).
%
%
% and the fact that $(1-\d)^{-\ell}>\ell^{10}$ (when $\ell\gg\kk$),
%we obtain that $C_2\ge\ell^7>C_4=1+O(\d^1)+O(\ell^{-1})$, a contradiction.
%If $\kk^{-1}|x_0|>1+\d$, using \eqref{SimPls}\,(a), \eqref{TaKa} and the following facts: $(1+\d)^{-\ell}<\ell^{-10}$,
%$\tilde y_1:=\g_{\kk,\kk}^{-1}|y_1|\ge1$ and $\a_1:=\frac{(\ell+1)(1-\scep)}{\ell^{-2}+\ell}>1$, we obtain
%\equa{EMwsdwjwejhn}{\dis\frac{\ell^{-2}\tilde y_1^{\a_1}}{\ell^{-2}+\ell}
%=C_1-\frac{\ell\tilde y_1^{\a_1}}{\ell^{-2}+\ell}
%\le
%C_1-\frac{\ell\tilde y_1}{\ell^{-2}+\ell}\le C_2-\frac{\ell\tilde y_1}{\ell^{-2}+\ell}\le\frac{\ell^{-12}}{\ell^{-2}+\ell},}
%a contradiction with the fact that $\tilde y_1\ge1$. We obtain (b).
%
% follows from %the first and last inequalities of
%\eqref{LetNSoOP}\,(i).
%By (a), (b) and
% \eqref{mqp1234-2+}, we see that $h_{p_0,p_1}\preceq\kk$, and then
%(c) follows from the proof of Lemma \ref{YYYy1==}. %Finally (d) follows from (a) and (c).
%, and (b) from (a) and \eqref{LetNSoOP}\,(ii). Using \eqref{TaKa} and
%the following simple facts,
%\begin{eqnarray}\label{UsUEUEUE}
%&\!\!\!\!\!\!\!\!\!\!\!\!\!\!\!\!\!\!\!\!&
%{\rm(i)\ }|\bar x_0x_0^{-1}+\kk^{-1}(\bar y_1^{-1}y_1)^{\kk}|\le\kk|x_0|^{-1}+\kk^{-1}(\g_{\kk,\kk}^{-1}|y_1|)^{\kk},
%\nonumber\\[4pt]
%&\!\!\!\!\!\!\!\!\!\!\!\!\!\!\!\!\!\!\!\!&
%{\rm(ii)\ }
%|\bar x_0x_0^{-1}+\kk^{-1}(\bar y_1^{-1}y_1)^{\kk}|\ge\kk^{-1}(\g_{\kk,\kk}^{-1}|y_1|)^{\kk}|-\kk|x_0|^{-1},
%\end{eqnarray}
%we obtain (d) by
%\eqref{LetNSoOP}\,(i) and (a) [we remind that $\ep<\kk^{-\kk}$ and $(1-\d)^\scep=1+O(\ep^1){\ssc\,}$].
%If $\kk|x_0|<1-\d$, then using $1\le C_1$, we obtain that $\kk^{-1}|x_1|\le(1-\d)^2<1$
%[thus $h_{p_0,p_1}\preceq\kk$].
%Now by (a), (d) and \eqref{mqp1234-2+}, we see $h_{p_0,p_1}\preceq\kk$, and then
%(c) follows from
%and as in the proof of Lemma \ref{YYYy1==}, we obtain a contradiction with (d). Then using $C_2\le C_4$, $C_1\ge1$
%and \eqref{TaKa}, we obtain $(\d^{-1}+1)\kk^{-1}|x_0|\le C_4+\d^{-1}$, i.e., we have (a).
%If $\kk^{-1}|x_1|<1-\d^4$, we can again  obtain a contradiction with (d) from (a), (d) and the proof of Lemma \ref{YYYy1==}. Now (b) follows from (a) and $C_1\ge1$.
%By (a) and (b), we have (c)  as in
%the proof of Lemma \ref{YYYy1==}.
%
%
%we obtain $\d\le\kk^{-1}|x_0x_1|\le1$ (and also $C_4\le1$), then using $C_3\le C_4$ and \eqref{TaKa}, we have
%$(\d^{-1}+1)\kk^{-1}|x_0|\le(\d\kk^2)^{-1}|x_0x_1|+1\le\d^{-2}+1$, thus $\kk|x_0|\le\d^{-1}$.
%If $\kk^{-1}|x_0|<\d^2$, then using $C_3\le C_4\le1$, we obtain $\d^{-1}\kk^2|x_0x_1|^{-1}\le C_3-(\d^{-1}+1)\d^2\le1-(\d^{-1}+1)\d^2$, thus $\kk^{-2}|x_0x_1|\ge\d^{-1}$,
% but then $C_2\le C_3$ cannot hold, i.e., we have \eqref{ImMpP}\,(a).
%Using $C_1\le C_2\le C_4\le1$ and (a), we immediately obtain (b).
%
%Assume the second equality of \eqref{LetNSoOP}\,{\rm(i)} holds.
%Denote $\tilde x_1=\bar x_1^{-1}x_1$, then
%\equa{WeWhaVe}{(1+\d)|\tilde x_1|^{\frac1{1+\d}}=|\tilde x_1-\d|+2\d\ge|\tilde x_1|+\d.}
%Denote the real function $f(x)=(1+\d)x^{\frac1{1+\d}}-x-\d$ on the nonnegative variable $x$. We have
%$f(1)=\frac{df(1)}{dx}=0,\,\frac{d^2f(1)}{dx^2}<0$ and $f(0)<0,\lim_{x\to+\infty}f(x)<0$ and $\frac{df(x)}{dx}=0\Leftrightarrow x=1$. Therefore $0=f(1)=\max\{f(x)\,|\,x\in\R_{\ge0}\}$, and so \eqref{WeWhaVe}
%implies $|\tilde x_1|=1$ and $|\tilde x_1-\d|=|\tilde x_1|-\d$, from which we obtain that $\tilde x_1=1$.
%
%(2) Inequations \eqref{ImMpP}\,(a), (b) follow from the first and last inequalities of
%\eqref{LetNSoOP}\,(i);
%Now (c) follows from
%(a) and the fact that
%$\lim_{\scep\to0}\big(1\!-\!(1\!+\!\d)\ep\big)^{-\frac3{\scep}}=e^{3(1+\d)}<e^{3(1+2\d)}$; (d) follows the last two inequalities of  \eqref{LetNSoOP}\,(i); and finally (e) follows from
%One can easily see that the conditions imply the following,
%where the assertion about $|x_1|$ follows from %the last inequality of \eqref{LetNSoOP}\,(i), and the assertion about $|x_1|$ follows from %\eqref{LetNSoOP}\,(ii) and
%the proof of Lemma \ref{YYYy1==}.
%Take for instance $a_0=20$, if (d) does not holds, then \eqref{LetNSoOP}\,(ii) cannot holds by (a)--(c).
%To prove (d), denote $\bar y_1^{-1}y_1+\ell^2=\a$. %, and assume $|\a|<\ep$.
%Then [using \eqref{TaKa} and the second inequality of \eqref{LetNSoOP}\,(i)]
%\begin{eqnarray}
%\label{SSSS}
%\!\!\!\!\!\!\!\!\!\!\!\!\!\!\!\!\!\!\!\!&\!\!\!\!\!\!\!\!\!\!&
%\ell^2-|\a|+1+\ep^7\le\g_{\kk,\kk}^{-1}|y_1|+1+2\ep^7
%\le
%(1+\ep^7)|\bar y_1^{-1}y_1+1|\nonumber\\[4pt]
%\!\!\!\!\!\!\!\!\!\!\!\!\!\!\!\!\!\!\!\!&\!\!\!\!\!\!\!\!\!\!&
%\phantom{\ell^2-|\a|+1+\ep^7}=(1+\ep^7)|\a-\ell^2+1|\le(1+\ep^7)(|\a|+\ell^2-1).
%\end{eqnarray}
%Thus  $|\a|\ge\frac{2+2\scep^7-\ell^2\scep^7}{2+\scep^7}=1-\frac{\scep}{2}+O(\ep^2)>1-\ep$
% (by the fact that $\ell=\ep^{-3}$).
%
%(2) If the second equality holds in \eqref{LetNSoOP}\,(i), then we obtain
%$\g_{\kk,\kk}^{-1}|y_1|+1+2\ep^7=(1+\ep^7)|\bar y_1^{-1}y_1+1|$
%\hfill$\Box$
%Clearly we have the assertion about $|x_1|$ in \eqref{-EiathA0}. Note that
%if $\kk^{-1}|x_0|\le\frac12$ or $\kk^{-1}|x_0|\ge5$ then $C_1\le C_2$ cannot hold, thus we also have assertion about $|x_0|$ in \eqref{-EiathA0}. Then the assertion about $|y_1|$  in \eqref{-EiathA0} follows from \eqref{LetNSoOP}\,(ii).
%
%When \eqref{LetNSoOP}\,(i) holds, one can easily obtain the following,
%\equa{MwWSWMW}{\mbox{$\!\!\!\!\!\!\!\!\!\!\!\!\!\!\!\!\!\!\dis1\!-\!\d\!\le\!\kk^{-1}|x_1|\!\le\!1,\
%1\!-\!\ep\!-\!\frac{(1\!-\!\kk^{-3})\ep}{1\!-\!\d\!-\!\kk^{-3}}\!\le\!\kk^{-1}|x_0|\!\le\!1\!-\!\ep\!+\!\frac{(1\!-\!\kk^{-3})\ep}{1\!-\!\d\!-\!\kk^{-3}}
%,\ 1\!-\!2\d\!\le\!\g_{\kk,\kk}^{-1}|y_1|\!\le\!1\!+\!\d,\!\!\!\!\!\!\!\!\!\!\!\!\!\!$
%}}
%where the assertion about $|y_1|$  follows from the proof of Lemma \ref{YYYy1==}.
%Then %us by \eqref{TaKa},
%\eqref{LetNSoOP}\,(i) implies %and Lemma \ref{YYYy1==}
%that $1-\ep\le\kk^{-1}|x_0|\le\g_{\kk,\kk}|y_1|\le\ep^4$, t
% (the assertion about
%$|x_1|$ follows from the proof of Lemma \ref{YYYy1==}).
%
%
%
%One can easily observe the following: there exist $\eta_0\,\eta_1\in\R_{>0}$ such that when \eqref{LetNSoOP} holds, we have [from the last two inequalities of
%\eqref{LetNSoOP}\,(i) we obtain the assertion for $|x_0|$; thus we also have the assertion for $|x_1|$; the assertion for $|y_1|$ follows from \eqref{LetNSoOP} and the proof of Lemma \ref{YYYy1==}]
%\equa{WEhHaaa}{\eta_0\le|x_0|,|x_1|,|\bar x_1^{-1}x_1+1|,|y_1|\le\eta_1.}
%Using \eqref{TaKa} and the fact that $\kk^{-1}|x_0|-\d\le|\bar x_0^{-1}x_0-\d|$,
%we obtain  the following simple fact.
%\begin{lemm}\label{FnMlemm}The second inequality of {\rm\eqref{LetNSoOP}\,(i)} implies that $\kk^{-1}|x_0|\le1.$
%$\kk^{-1}|x_0|=(\kk\ep)^{-1}|x_0-(1-\ep)\bar x_0|$, then $\kk^{-1}|x_0|\ge\frac{1-\scep}{1+\scep}$.
%\end{lemm}
%
%
%
%
%\begin{lemm}\label{AothLemm}The second inequality of \eqref{LetNSoOP}\,${\rm(i)}$ automatically holds.
%\end{lemm}\noindent{\it Proof.~}Assume conversely $(\kk|x_0|)^{\frac12}>\frac{|2\bar x_0^{-1}x_0-1|+3}{4}$. Then using
%$|2\bar x_0^{-1}x_0-1|\ge2\kk^{-1}|x_0|-1$, we obtain that $0>\big((\kk^{-1}|x_0|)^{\frac12}-1\big)^2$, a contradiction.\hfill$\Box$
%
%
%
%
%
%
%
%
}}%
%
%
%
%
\begin{lemm}\label{TheoHold}
Theorem $\ref{real00-inj}\,(1)$ holds.\end{lemm}
 %let $\kk\gg1$ and
\noindent{\it Proof.~}\NOUSE{We will need the following very simple fact.
\begin{eqnarray}
\label{11simMSMpp}
&\!\!\!\!\!\!\!\!\!\!\!\!\!\!\!\!\!\!\!\!\!\!\!\!&
{\rm(i)}^{\pm}\
\Big|X_1^{(1+3\d)\ell}-(e^{3\d}-e^{-1+\d^2})X_0^{\ell}\Big|
\ge\pm
\Big(|X_1|^{(1+3\d)\ell}-(e^{3\d}-e^{-1+\d^2})|X_0|^{\ell}\Big),
%\ \ \
\nonumber\\[4pt]
&\!\!\!\!\!\!\!\!\!\!\!\!\!\!\!\!\!\!\!\!\!\!\!\!&
{\rm(ii)}\, \,\
\Big|X_1^{(1+3\d)\ell}-(e^{3\d}-e^{-1+\d^2})X_0^{\ell}\Big|
\le|X_1|^{(1+3\d)\ell}+(e^{3\d}-e^{-1+\d^2})|X_0|^{\ell}.
\end{eqnarray}}%
%
\NOUSE{%
First we need the following simple fact,
\begin{eqnarray}
\label{Si000}
&\!\!\!\!\!\!\!\!\!\!\!\!\!\!\!\!\!\!\!&
{\rm(i)^{\pm}\ }
\Big|(1-\a_0\ep)\frac{Y_1}{X_0^2}-\a_0\ep\Big|
\ge\pm\Big((1-\a_0\ep)\frac{|Y_1|}{|X_0|^2}-\a_0\ep\Big),
\nonumber\\[4pt]
&\!\!\!\!\!\!\!\!\!\!\!\!\!\!\!\!\!\!\!\!\!\!\!\!\!\!\!\!&
{\rm(ii)\ \, \ }
\Big|(1-\a_0\ep)\frac{Y_1}{X_0^2}-\a_0\ep\Big|
\le(1-\a_0\ep)\frac{|Y_1|}{|X_0|^2}+\a_0\ep
.
\end{eqnarray}
}%Assume conversely that \eqref{WehHAH} is not true.
%Let $\kk\gg1$ and
%% $\d =\d^{\frac15}\in\R_{>0}$ with
%$\d$ be %being
%as above %be sufficiently small %and $\d=\kk^{-1}$
%and take $\ell\gg\kk$
%(and we can assume $\ep<\kk^{-\kk}$, cf.~Remark \ref{u-vRema}).
%$\d\in\R_{>0}$ be as above. %sufficiently small.
%Denote \equa{NDSMS}{\dis s_0=-a_\kk+(1+\d+\kk^{-\ell})b_\kk,\ \ \a_0=\Big(\frac{s_0}{s_0+2}\Big)^{\frac12}+\kk^{-\ell}<1.}
 % and \eqref{-EiathA}. % and %, then %$\mu=0$ and %$(t_1,t_2,t_3)=(0,1,0)$ and
%First note that when conditions \eqref{LetNSoOP} hold, we have \equa{x01===}{\mbox{$\dis\kk\le|x_1|\preceq\ell$
% (when $\ell\gg\kk$ and $\kk$ is regarded as fixed),}}
%where the first inequality follows from the last two inequalities of \eqref{LetNSoOP}\,(i) and
%the $\preceq$ follows from the first inequality.
%Then we can obtain \equa{x01===0}{\mbox{$\dis\ell^{-1}\kk\le\frac{\ell^{-1}(|x_1|-\a_0\kk)}{1-\a_0}\le|x_0|\le\frac{|x_1|-\a_0\kk}{1-\a_0}\preceq\ell$.}}
%we have
%\eqref{-EiathA1} by noting that when \eqref{LetNSoOP}\,(i) holds, we can first deduce that \equa{SMSMSMSMS}{\mbox{$\dis\kk\le|x_1|\le\Big(\frac{1+\d}{1-\d}-\d\Big)\kk=\frac{1+\d^2}{1-\d}\kk$,}}
%then deduce that $x_0^{\pm1},(x_0^{-1}+(\d^{-1}-1)\bar x_0^{-1})^{\pm1}$ are nonzero and bounded, from which we can also obtain that $y_1^{\pm1}$ is
%nonzero and bounded by \eqref{mqp1234-2}% and \eqref{EiathA}\,(c)
%.
%We need the following simple fact, %note that when \eqref{LetNSoOP}\,(i) holds, we have
%\equa{HAHSHSHSD}{\mbox{$\!\!\!\!\!\!\dis|x_1^{-1}{\sc}+\d\bar x_1^{-1}|\le|x_1|^{-1}{\sc}+\d\kk^{-1}$,   i.e.,  $\dis|x_1|\le\frac{1}{|x_1^{-1}{\sc}+\d\bar x_1^{-1}|-\d\kk^{-1}}$
%%, thus $|x_1|\!\le\!\kk$ if \eqref{LetNSoOP} holds
%.}\!\!\!\!}
%We will need the following simple fact
%\equa{SImMM}{\dis\Big|x_0^{-1}\!+\!(\d^{-1}\!-\!1)\bar x_0^{-1}\Big|\!\le\!|x_0|^{-1}\!+\!(\d^{-1}\!-\!1)\kk^{-1},\mbox{ i.e., }
%|x_0|\!\le\!\frac1{\Big|x_0^{-1}\!+\!(\d^{-1}\!-\!1)\bar x_0^{-1}\Big|\!-\!(\d^{-1}\!-\!1)\kk^{-1}}.}
%First we remark that when conditions \eqref{LetNSoOP} hold, we have $|x_0|\sim\kk$ and $|y_1|\sim\g_{\kk,\kk}\ge\kk$ [cf.~\eqref{suchThaT}], thus by
%\eqref{mqp1234-2}, we must have $|x_1|\preceq\kk$. We have \eqref{-EiathA0}.
%
%We will need the following simple fact,
%First notice that if the
%third equality of \eqref{LetNSoOP}\,(i) holds, then we can solve
%\equa{M8938383}{\mbox{$\dis
%{\rm(i)\ }
%\Big|\frac{x_1}{\bar x_1}\!+\!\frac{\d\bar x_0\bar x_1}{\bar y_1}\frac{y_1}{x_0x_1}\Big|
%\le\frac{|x_1|}{\kk}\!+\!\frac{\d\kk^2}{\g_{\kk,\kk}}\Big|\frac{y_1}{x_0x_1}\Big|,\ \
%{\rm(ii)\ }
%\Big|\frac{x_1}{\bar x_1}\!+\!\frac{\d\bar x_0\bar x_1}{\bar y_1}\frac{y_1}{x_0x_1}\Big|
%\ge-\frac{|x_1|}{\kk}\!+\!\frac{\d\kk^2}{\g_{\kk,\kk}}\Big|\frac{y_1}{x_0x_1}\Big|
%$.}}
%Now
\NOUSE{%
We will need the following simple fact
[cf.~\eqref{TaKa}],
\begin{eqnarray}
\label{msmsm}
&\!\!\!\!\!\!\!\!\!\!\!\!\!\!\!\!\!\!&
{\rm(i)}^{\pm}\ \
|\bar x_0^{-1}x_0+\d|\ge\pm
\big(\kk^{-1}|x_0|-\d\big),
\ \ \ \ %\nonumber\\[4pt]
%&\!\!\!\!\!\!\!\!\!\!\!\!\!\!\!\!\!\!&
{\rm(ii)\ \ \ }
|\bar x_0^{-1}x_0+\d|\le
\kk^{-1}|x_0|+\d,
\end{eqnarray}
%
For simplicity we denote $Y_1=\g_{\kk,\kk}|y_1|,\,X_i=\kk^{-1}|x_i|,\,i=0,1$.
%
We will need the following simple fact,
\begin{eqnarray}
\label{msmsm}
&\!\!\!\!\!\!\!\!\!\!\!\!\!\!\!\!\!\!&
{\rm(i)^{\pm}\, \ }
\Big|(1-\ep^2)\big(\frac{X_0}{Y_1}\big)^{\frac1{\scep}}
+\ep^2\big(\frac{X_0}{Y_1^{1+\omega}}\big)^{\frac{10}{\scep}}\Big|
\ge
\pm
\Big((1-\ep^2)\big|\frac{X_0}{Y_1}\big|^{\frac1{\scep}}
-\ep^2\big|\frac{X_0}{Y_1^{1+\omega}}\big|^{\frac{10}{\scep}}\Big),
\nonumber\\[4pt]
&\!\!\!\!\!\!\!\!\!\!\!\!\!\!\!\!\!\!&
{\rm(ii)\ \, \ }
\Big|(1-\ep^2)\big(\frac{X_0}{Y_1}\big)^{\frac1{\scep}}
+\ep^2\big(\frac{X_0}{Y_1^{1+\omega}}\big)^{\frac{10}{\scep}}\Big|
\le
(1-\ep^2)\big|\frac{X_0}{Y_1}\big|^{\frac1{\scep}}
+\ep^2\big|\frac{X_0}{Y_1^{1+\omega}}\big|^{\frac{10}{\scep}}
.\end{eqnarray}
%
}%
%
%
%
%
%
%
%
%
%
%
%
%
%
%
%
%
%
%
%
%
%
%
%
%
%
%
%We will need the following very simple fact.
%\begin{eqnarray}
%\label{VerySM}
%\!\!\!\!\!\!\!\!\!\!\!\!\!\!\!\!\!\!\!\!\!\!\!\!&&
%{\rm(i)^{\pm}\ \ }
%\mbox{\Large$\Big|$}(1-\a_0)\frac{Y_1}{X_1^2}+\a_0\frac{X_0^4}{X_1^3}\mbox{\Large$\Big|$}
%\ge\pm\mbox{\Large$\Big($}(1-\a_0)\frac{|Y_1|}{|X_1|^2}-\a_0\frac{|X_0|^4}{|X_1|^3}\mbox{\Large$\Big)$},
%\nonumber\\[2pt]
%\!\!\!\!\!\!\!\!\!\!\!\!\!\!\!\!\!\!\!\!\!\!\!\!&&
%{\rm(ii)\ \ \ \,}
%\mbox{\Large$\Big|$}(1-\a_0)\frac{Y_1}{X_1^2}+\a_0\frac{X_0^4}{X_1^3}\mbox{\Large$\Big|$}
%\le(1-\a_0)\frac{|Y_1|}{|X_1|^2}+\a_0\frac{|X_0|^4}{|X_1|^3}.
%\end{eqnarray}
%We some very simple facts, which for easy reference are stated below,
%\equa{VERYEasy}{\dis{\rm(i)}^{\pm}\
%\Big|2\frac{X_1^2}{Y_1}-X_1\Big|\ge\pm\Big(2\frac{|X_1|^2}{|Y_1|}-|X_1|\Big),
%\ \ \ \
%{\rm(ii)\ }\Big|2\frac{X_1^2}{Y_1}-X_1\Big|\ge 2\frac{|X_1|^2}{|Y_1|}+|X_1|.
%}
For any $(p_0,p_1)\in V_0$, %we have the following.
%\begin{itemize}\item[(1)]
if the first equality holds,
or the second and third equalities hold, in \eqref{LetNSoOP}\,(i), i.e., $C_1=C_2=C_3=1$. Then
$|X_0^3X_1Y_1|=|X_1^3X_0Y_1^2|=1$ and so $|X_1|^2=|X_0|^5$, but by \eqref{LetNSoOP}\,(iii), we have $|X_0|^5<|X_1|^2$, a contradiction.

Assume the last equality holds in
%%then $|X_0|=1$ and by
\eqref{LetNSoOP}\,(i). Then %, $|X_1|<1$, a contradiction with the fact that $1\le C_3$.
\equa{ThEhHoo}{|X_0^3X_1Y_1|=\ell^{5-v_0+O(v_0^5)}, \ \ |X_1^3X_0Y_1^2|=\ell^{6-2v_0+O(v_0^5)}.}
By \eqref{ImMpP}\,(c), we can write $|X_1|=|Y_1|\big(1+O(\d^5)\big)$. Then $|X_0^3Y_1^2|\big(1+O(\d^5)\big)=\ell^{5-v_0+O(v_0^5)}$, $|X_0Y_1^5|\big(1+O(\d^5)\big)=\ell^{6-2v_0+O(v_0^5)}$, i.e.,
\equa{1ThEhHoo}{|X_0|=\ell^{1-\frac{v_0}{13}+O(v_0^5)}\big(1+O(\d^5)\big), \ \ |Y_1|=\ell^{1-\frac{5v_0}{13}+O(v_0^5)}\big(1+O(\d^5)\big).}
Thus, $|X_1|=\ell^{1-\frac{5v_0}{13}+O(v_0^5)}\big(1+O(\d^5)\big)$.
Then up to $\ep^4$, we have (by noting that $C_2=\ell^{1+O(v_0^5)}\ssc\,$), \equa{Cs515151}{\mbox{$\dis
C_5=|X_1|^2\frac{\ell^{3+O(v_0^5)}}{|X_0|^5}=\ell^{-\frac{5v_0}{13}+O(v_0^5)}\big(1+O(\d^5)\big)\ll1$ [since $\ell=\big(\frac1{v_0}\big)^{\frac1{v_0^{10}}}\ssc\,$],}} a contradiction.

If the first equality holds in \eqref{LetNSoOP}\,(ii), i.e., $|X_1|=\ep$, then $|Y_1|\sim\ep$ when $\ep^{-1}\gg\kk$ by
\eqref{ImMpP}\,(c). Then $|X_0|\sim\ep^{-5}$ by the fact that $1\le C_2\le C_4$, but then $C_1\sim\ep^{-13}\gg C_4$, a contradiction.
Similarly, if  the last equality holds in \eqref{LetNSoOP}\,(ii), i.e., $|X_1|=\ep^{-1}$, then $|Y_1|\sim\ep^{-1},\,|X_0|\sim\ep^5$, and
$C_1\sim \ep^{13}\ll1$, again a contradiction.
\NOUSE{%
Then
\equa{Fooele}{\dis{\rm(i)\ }|X_0|=|X_1|^{1+\frac{2v_0^5}{1+2v_0-v_0^5}}>1, \ \ \ {\rm(ii)\ }|Y_1|\ge\frac{2|X_1|^2}{|X_1|+1}>|X_1|,}
where the inequality in \eqref{Fooele}\,(i) follows from the (first) equality and \eqref{LetNSoOP}\,(ii), and the first inequality of \eqref{Fooele}\,(ii) follows from
\eqref{VERYEasy}\,(i)${}^+$ and the fact that $C_2=1$.
%
, i.e., $C_1=C_2=C_3=1$.
Then $|X_1|=|Y_1|=|X_0|>1$, where the inequality follows from \eqref{LetNSoOP}\,(ii).
%Then $\frac32\frac{|Y_1|}{|X_1^3X_0|}-\frac12\le1\le\frac12\frac{|Y_1|}{|X_0^3X_1|}+\frac12$, thus $|X_3|\le|X_1$, a contradiction with
%\eqref{LetNSoOP}\,(iii).
Denote $\bar k_0:=|Y_1|>1$, then [cf.~notation  \eqref{SimMMSMS}$\ssc\,$]
we have $\bar k_0\kk=|Y_1|\kk=|X_1|\kk=|x_1|=|x_0|$.
By definition \eqref{Ak=1} and Theorem \ref{AddLeeme--0}\,(iii), we obtain
\equa{ByDsllo}{\g_{\bar k_0\kk,\bar k_0\kk}=\g_{|x_0|,|x_1|}\ge|y_1|=\g_{\kk,\kk}|Y_1|=\bar k\g_{\kk,\kk},}
which is a contradiction with Lemma \ref{G--lemm-assum1--}.
%
%
Assume the last equality of \eqref{LetNSoOP}\,(i) holds. Then $C_1=\ell^{1+O(\ell^{-5})}$, $C_2=\ell^{1+O(\ell^{-5})}$ and
\begin{eqnarray}
\label{Mosoweirn}
&\!\!\!\!\!\!\!\!\!\!\!\!\!\!\!\!\!\!\!\!\!&
|Y_1|=\ell^{\frac{c_\kk-1}{2c_\kk+1}+O(\ell^{-5})}|X_0|
=\ell^{\frac{c_\kk-1}{2c_\kk+1}+O(\ell^{-5})}\Big(\ell^{-\frac12+O(\ell^{-5})}|X_1|^{\frac12}\Big)
=\ell^{-\frac{3}{2(2c_\kk+1)}+O(\ell^{-5})}|X_1|^{\frac12}
\nonumber\\[4pt]
&\!\!\!\!\!\!\!\!\!\!\!\!\!\!\!\!\!\!\!\!\!&
\phantom{|Y_1|}
\sim
\ell^{-\frac{3}{2(2c_\kk+1)}+O(\ell^{-5})}|Y_1|^{\frac12}\mbox{ when }\ell\gg\kk.
\end{eqnarray}
Thus $|Y_1|\sim\ell^{-\frac{3}{2c_\kk+1}+O(\ell^{-5})}\sim\ell^{-\frac{3}{2c_\kk+1}}\ll1$ when $\ell\gg\kk$
[cf.~Remark \ref{NotaRemak}\,(i)$\ssc\,$], a contradiction with \eqref{ImMpP}\,(a),\,(c).
%
%
Then up to $O(v_0^5)$ [note that $0<v_0\ll\d\ll\d_1$, cf.~\eqref{MAmsndj} and \eqref{v0=1=1=1=1}$\ssc\,$], we have
$\frac{|Y_1|}{|X_0^3X_1|}\sim\ell_1$ when $\ell_1\gg1$ [note that when we say $\ell_1\gg1$, we regard $\kk,\ell$ as fixed (but we still require $\kk\gg\ell$), cf.~Remark
{NotaRemak}\,(i)$\ssc\,$].
%
[using \eqref{VySkM}\,(i)${}^+$, %where the last equation is obtained by noting
Remark \ref{NotaRemak}\,(ii) and %using the first two in
\eqref{LetNSoOP}\,(ii)$\ssc\,$]
\equa{MSMSMDpapa}{|X_0|=|X_1|^2,\ \
|X_1|^3\le|X_0|^2,\ \ |Y_1|>\frac{|X_1|^4}{|X_0|^3}.
%|X_0|=|X_1|^{1+\d+v_0}=|Y_1|^{(1+\d+v_0)(1-\scep)},\ \ \ C_4=|Y_1|^{v_0+O(\ep^1)}>1.
}
In particular, $|Y_1|>1$.
%
%assume the first equality holds, or the second and third equalities hold, %two equalities hold
%in \eqref{LetNSoOP}\,(i), i.e.,
%$C_1=C_2=C_3=C'_1=C'_2=1$% [cf.~\eqref{Thsmememe}$\ssc\,$]
%. Then [using \eqref{Vemsms}\,(i)${}^-$ and \eqref{LetNSoOP}\,(ii)$\ssc\,$]
%\begin{eqnarray}
%\label{0595999dnnsd}
%&\!\!\!\!\!\!\!\!\!\!\!\!\!\!\!\!&
%{\rm(i)\ }|X_1|=1,\ \ \ep^{-5}|Y_1|^{\frac{\a_1}{v_0(1-v_0)}}\ge(1+\ep^{-5})|X_0|^{\a_1}{v_0}-1.
%\end{eqnarray}
%
%
%
%\eqref{Thsmememe}\,(i) shows that $1\le\frac{\scep^2+|X_1|}{1+\ep^2}$, i.e., $|X_1|\ge1$. Similarly,
%\eqref{Thsmememe}\,(ii) shows that $|X_0|\le1$.
%Further, $$1=|C_1'|\ge(1-\ep^2)|X_1|-\ep^2|X_0|\ge(1-\ep^2)|X_1|-\ep^2,$$
% i.e., $|X_1|\le1+2\ep^2+O(\ep^3)$.
%Then \eqref{LetNSoOP}\,(ii) implies that $$|Y_1|\!+\!O(\ep^4)
%\!\ge\!\big(1\!+\!(1\!-\!\kk^{-5})\ep\big)|x_1|^{-\frac{1}{4\scep^2}}
%\!\ge\!\big(1\!+\!(1\!-\!\kk^{-5})\ep\big)
%\Big(1\!-\!\frac{\ep}{2}\!+\!O(\ep^2)\Big)\!>\!1\!+\!\Big(\frac12\!-\!\kk^{-5}\Big)\ep\!+\!O(\ep^2).$$
%Thus we obtain $|Y_1|>|X_1|\ge1$.
%
%\equa{Then1234}{{\rm(i)\, }(1+\d)|X_0|-\d|X_1|\le1\mbox{ $\Big($i.e., $\dis|X_0|\le\frac{1+\d|X_1|}{1+\d}$\,$\Big)$},\ \ \ {\rm(ii)\ }(1-\d)|X_0|+\d|X_1|\ge1.}
%Using the first inequation  in the second inequation, we obtain
%$(1-\d)\frac{1+\d|X_1|}{1+\d}+\d|X_1|\ge1$, i.e., $|X_1|\ge1$.
%Using this in the first inequation, we obtain $|X_0|\le|X_1|$.
%% from which we obtain that $|X_0|\le1,\,|X_1|\ge1$.
%If $|X_1|=1$, i.e., $|x_0|\le\kk,\,|x_1|=\kk$, and $|y_1|>\g_{\kk,\kk}$ [by \eqref{LetNSoOP}\,(ii) and notation \eqref{SimMMSMS}$\ssc\,$], we obtain a contradiction with
%definition of $\g_{\kk,\kk}$ and/or Theorem \ref{AddLeeme--0}\,(iii).
%Thus assume $|X_1|>1$.
%
%
%$|x_0|=|x_1|=\kk$ [cf.~notation \eqref{SimMMSMS}$\ssc\,$], and $|y_1|>\g_{\kk,\kk}$ [using \eqref{LetNSoOP}\,(ii) and the
%fact that $C_3\ge1\ssc\,$], a contradiction with the definition of $\g_{\kk,\kk}$.
%
%
%Assume the third equality holds in \eqref{LetNSoOP}\,(i). Then
%we obtain
%\equa{meme9494845}{(1-\ep^2)|X_1|-\ep^2\le C_3^{\frac{1}{2}}=|X_1|^{\frac12}.}
%First assume $|X_1|\ge1+5\ep^2$. Then $(1-\ep^2)|X_1|^{\frac12}-1>0$, and we have
%\equa{1meme9494845}{(1-\ep^2)|X_1|^{\frac12}-1\le|X_1|^{\frac12}\Big((1-\ep^2)|X_1|^{\frac12}-1\Big)
%=(1-\ep)^2|X_1|-|X_1|^{\frac12}\le\ep^2.}
%This implies that $|X_1|\le1+4\ep^2+O(\ep^3)<1+5\ep^2$, a contradiction with the assumption. Thus $|X_1|<1+5\ep^2$.
%
%
%[using %\eqref{ImMpP}\,(b),
%\eqref{11simMSMpp}\,(i)${}^+$ and
%\eqref{LetNSoOP}\,(ii)%;
%note that when we say $\ell_1\gg\kk$, we regard $\ell,\kk$ as fixed
%(but we still required $\ell\gg\ell_1{\ssc\,})$% and that $v_0\ll\d\ll\d_1$
%, see also Remark \ref{rema-kk}
%; we compute the following up to $O(\ep^2)$ and compute the coefficient of $\ep^1$ up to $O(\d^5)$, thus for this purpose $v_0$ can be omitted by \eqref{v0=1=1=1=1}; where $\ln(\cdot)$ is the natural
%logarithmic function],
%\begin{eqnarray}
%\label{msssssss}
%&  \!\!\!\!\! \!\!\!\!\! \!\!\!\!\! \!\!\!\!\!&
%|X_0|=e^{\frac{\scep}{1-\d^2}}=1+\frac{\ep}{1-\d^2}=1+(1+\d^2+\d^4)\ep,
%\nonumber\\[4pt]
%&\!\!\!\!\!\!\!\!\!\!\!\!\!\!\!\!\!\!&
%|X_1|\le\Big(1+(e^{3\d}-e^{1-\d^2})e^{\frac1{1-\d^2}}\Big)^{\frac{\scep}{1+3\d}}
%=1+\frac{\ln\Big(1+(e^{3\d}-e^{1-\d^2})e^{\frac1{1-\d^2}}\Big)\ep}{1+3\d}
%\nonumber\\[2pt]
%&\!\!\!\!\!\!\!\!\!\!\!\!\!\!\!\!\!\!&
%\phantom{|X_1|}
%=
%1+\Big(1+\d^2-3\d^3+(10-e^{-1})\d^4\Big)\ep,
%\nonumber\\[4pt]
%&\!\!\!\!\!\!\!\!\!\!\!\!\!\!\!\!\!\!&
%|Y_1|\!\ge\!\Big({\sc\!}(1\!+\!3\d \ep)e^{\frac{\scep}{1-\d^2}}{\sc\!}\Big)^{\frac1{1+3\d}}\!=\!1
%\!+\!\Big({\sc\!}\frac{3\d}{1\!+\!3\d}\!+\!\frac{1}{(1\!+\!3\d)(1\!-\!\d^2)}{\sc\!}\Big)\ep
%\!=\!1\!+\!(1\!+\!\d^2\!-\!3\d^3\!+\!10\d^4)\ep.
%\end{eqnarray}
%Thus $|X_1|<|Y_1|$ and $|X_0|<|Y_1|^{1+3\d}$.
%If $|X_1|\le1$, then using notation \eqref{SimMMSMS}, we have $|x_1|\le\kk$, $|x_0|=\kk$, but $|y_1|>\g_{\kk,\kk}$, we obtain
%a contradiction with the definition of $\g_{\kk,\kk}$. Thus assume $|X_1|>1$.
%
%We claim
%\equa{1-msssssss}{\dis
%|Y_1|>|X_1|.}
%To prove the claim, first assume $|X_1|>\ell$. Then [using \eqref{ImMpP}\,(c)$\ssc\,$], we have
%$|Y_1|\sim|X_1|$, and so $\frac{|Y_1|}{|X_1|^5}\sim\frac{1}{|X_1|^4}\preceq\ep^4\ll|X_1|-\ep^2$
%when $\ep=\ell^{-1}\ll1$ (cf.~Remark \ref{NotaRemak}$\ssc\,$), a contradiction with
%the last inequation of \eqref{msssssss}. Thus we must have $|X_1|\le\ell$.
%Assume conversely $|Y_1|\le|X_1|$% (then $|X_1|^{-1}\ge\ep\ssc\,$)
%. We have [where $v_1=v_0(1-v_0^4)>0$;
%using the second equation of \eqref{msssssss} and the facts that $|X_1|^{-1}\ge\ep$, $|X_1|^{-1}<1$ (thus $\frac{O(\scep^3)}{|X_1|}=O(\ep^3)\ssc\,$],
%\begin{eqnarray}
%\label{SmemS}
%&\!\!\!\!\!\!\!\!\!\!\!\!\!\!\!\!\!\!\!\!\!\!\!\!\!\! &
%\frac{|Y_1|}{|X_1|^5}\le\frac{1}{|X_1|^4}\le\frac1{|X_1|}\Big(1-3v_1(1-2v_1\ep)\ep+O(\ep^3)\Big)
%=\frac1{|X_1|}-\frac{3v_1(1-2v_1\ep)\ep}{|X_1|}+O(\ep^3)
%\nonumber\\[4pt]
%&\!\!\!\!\!\!\!\!\!\!\!\!\!\!\!\!\!\!\!\!\!\!\!\!\!\!&\phantom{\frac{|Y_1|}{|X_1|^5}}
%\le\frac1{|X_1|}-3v_1(1-2v_1\ep)\ep^2+O(\ep^3)=\frac1{|X_1|}-3v_1\ep^2+O(\ep^3)\le|X_1|-3v_1\ep^2+O(\ep^3),
%\end{eqnarray}
%which is a contradiction with the last inequation of \eqref{msssssss}.
%Thus we have \eqref{1-msssssss}.
%Denote $\bar k:=|Y_1|%$. Then $\bar k
%>1$. By
%notation \eqref{SimMMSMS} and \eqref{msssssss}, %\eqref{1-msssssss},
%we have
%\equa{msssssss3}{\dis \bar k^{1+3\d}\kk>|X_0| \kk=|x_0|,\ \  \bar k\kk=|Y_1|\kk>|X_1|\kk=|x_1|.}
By definition \eqref{Ak=1} and Theorem \ref{AddLeeme--0}\,(iii) [cf.~notation \eqref{SimMMSMS}$\ssc\,$]% and the second equation of \eqref{x000wowow}
, we obtain
\equa{ByDsllo}{\g_{\bar k^{1+\d+v_0}\kk,\bar k\kk}>\g_{|x_0|,|x_1|}\ge|y_1|=\g_{\kk,\kk}|Y_1|=\bar k\g_{\kk,\kk},}
which is a contradiction with Lemma \ref{G--lemm-assum1--} (we can assume $\d+v_0<\frac1{m}\ssc\,$).
%
%Thus $|x_0|\prec|y_1|$ [cf.~notation \eqref{SimMMSMS}$\ssc\,$], and then
%  \eqref{mqp1234-2+} implies that we must have that $h_{p_0,p_1}\sim|x_1|\sim|y_1|$ when $\ell_1\gg\kk$.
%We obtain\equa{aDadsbmnm}{\dis
%\d_1^5\sim C_{20}=\frac{|Y_1|}{|X_1|^2}\sim|Y_1|^{-1}\preceq \d_{_{\sc1}}^{^{\sc\frac{6+2v_0}{v_0}}}\ll\d_1^5{\ssc\,},}
% which is a contradiction.
%
%  , and then
%(c) follows from the proof of Lemma \ref{YYYy1==}. %Finally (d) follows from (a) and (c).
%, and (b) from (a) and \eqref{LetNSoOP}\,(ii). Using \eqref{TaKa} and
%
%
%Then by \eqref{LetNSoOP}\,(iii), we have $|Y_1|\succeq\ell_1^3$. However by \eqref{ImMpP}\,(c), we obtain that
%\equa{msssssss1}{\dis
%C_{20}=\frac{Y_1}{X_1^2}\sim\frac1{Y_1}\sim\d_1^3,} a contradiction with \eqref{msssssss}.
%
%\item[(2)]
Now assume the last equality holds in \eqref{LetNSoOP}\,(i). Then we obtain
[similar to \eqref{MSMSMDpapa}$\ssc\,$] %[up to $O(\kk^{-5})\ssc\,$]
\begin{eqnarray}
\label{ememem0000}
\!\!\!\!\!\!\!\!&\!\!\!\!\!\!\!\!\!\!\!\!\!\!\!\!\!\!&
{\rm(i)\ }|X_1|=\ell^{\frac{v_0}{\d+v_0}+O(\ell_1^{-1})}|Y_1|^{1-\scep},\ \
\nonumber\\[4pt]
\!\!\!\!\!\!\!\!&\!\!\!\!\!\!\!\!\!\!\!\!\!\!\!\!\!\!&
{\rm(ii)\ }|X_0|=\ell^{-1+O(\ell_1^{-1})}|X_1|^{1+\d+v_0}=\ell^{\frac{v_0(1+\d+v_0)-\d-v_0}
{\d+v_0}+O(\ell_1^{-1})}|y_1|^{1+\d+v_0+O(\scep^1)},\nonumber\\[4pt]
\!\!\!\!\!\!\!\!&\!\!\!\!\!\!\!\!\!\!\!\!\!\!\!\!\!\!&
{\rm(iii)\ }C_4\!=\!\ell^{\frac{\d-(1+\d)v_0}{v_0}\cdot\frac{v_0}{\d+v_0}+\frac{v_0(1+\d+v_0)-\d-v_0}
{\d+v_0}+O(\ell_1^{-1})}|Y_1|^{v_0+O(\scep^1)}\!=\!\ell^{\frac{-v_0(1-v_0)}{\d+v_0}+O(\ell_1^{-1})}|Y_1|^{v_0+O(\scep^1)}
\!>\!1.
\end{eqnarray}
Thus $|Y_1|\ge\ell^{\frac1{\d+v_0}+O(\scep^1)}\gg1$ when $\ell\gg\ep^{-1}$ [cf.~Remark \ref{NotaRemak}$\ssc\,$), and \eqref{MMSMSMnnsnsn}$\ssc\,$]. Thus we have
\eqref{Mehhhhhh}. Then $C_1\sim|X_1|^{\frac{(\d+v_0)\scep}{v_0}}\sim\ell^{1+O(\ell_1^{-1})}$,
i.e., $|X_1|\sim\ell^{\frac{v_0(1+O(\ell_1^{-1}))}{(\d+v_0)\scep}}$, and so by the first equality of  \eqref{ememem0000}\,(ii), we have
$|X_0|\sim\ell^{\frac{(1+\d+v_0)v_0(1+O(\ell_1^{-1}))}{(\d+v_0)\scep}}.$ By \eqref{LetNSoOP}\,(ii)
[we remark
that the only purpose of \eqref{ememem0000}\,(iii) is to estimate that $|Y_1|\gg1$ so that we can obtain \eqref{Mehhhhhh}, \eqref{ememem0000}\,(iii) cannot be used in the following computation since
we only compute the power of $|Y_1|$ up to $O(\ep^1)$ in \eqref{ememem0000}\,(iii)$\ssc\,$]
\equa{C4==2l2l}{C_4\sim\frac{|X_0|}{|Y_1|^{1+\d}}\sim
\frac{|X_0|}{|Y_1|^{1+\d}}.}
%
%
Using the same arguments after \eqref{Then1234}, one  can  easily solve
 [where $\ln(\cdot)$ is the
 natural logarithmic function],
\equa{N9djfrfjr}{\dis|X_1|\ge\frac{-2(1-\d)+(1+\d)2^{\frac{1+\d}{1-\d}}}{2\d}=2+2\ln(2)+O(\d^1)>1+\d,} a contradiction with
\eqref{ImMpP}\,(a).
%
%[using  \eqref{ImMpP} and \eqref{LetNSoOP}\,(i)$\ssc\,$]
%\equa{9NeWsss}{1\le|X_0|=|X_1|^{1+3\d}, \ \ |Y_1|^{1+3\d-v_0^5}>|X_0|, \ \ \mbox{ thus, }\ |Y_1|>|X_1|.}
%We can obtain a contradiction with Lemma \ref{G--lemm-assum1--} as in the previous case.
%
%\item[(3)]
%Assume the last equality holds in \eqref{LetNSoOP}\,(i), i.e., $|X_1|=\ell$.
%Using \eqref{11simMSMpp}\,(i)${}^+$ and the fact that $C_1\le C_4$, we obtain
%\equa{|X000|}{(e^{3\d}-e^{-1-\d^2})|X_0|^{\ell}\ge|X_1|^{(1+3\d)\ell}-C_4,\mbox{ \ i.e., \ }|X_0|\succeq|X_1|^{1+3\d}\mbox{ when }\ell\gg\kk.}
%However, by \eqref{LetNSoOP}\,(ii) and \eqref{ImMpP}\,(c), we have $|X_0|\preceq|Y_1|^{1+3\d-v_0^5}\sim|X_1|^{1+3\d-v_0^5}\prec|X_1|^{1+3\d}$
%when $\ell \gg\kk$ [cf.~Remark \ref{NotaRemak}% and notation \eqref{SimMMSMS}
%; note that $\ell^{-1}=\ep\ll v_0\ssc\,$], a contradiction with \eqref{|X000|}.
%
%by \eqref{LetNSoOP}\,(ii), we have $|X_1|\ge\ell^5$,
%a contradiction with \eqref{ImMpP}\,(b).
%, i.e., $C_1=C_2=C_3=1$.
%Then when $\ell_1\gg\kk$
%[note that when we say $\ell_1\gg\kk$, we regard $\ell,\kk$ as fixed
%(but we still required $\ell\gg\ell_1{\ssc\,})$ and that $\d=\ell^{-1}\ll\d_1$; see also Remark \ref{rema-kk}$\ssc\,$], we have
%[using %\eqref{VerySM}\,(ii) and
%\eqref{LetNSoOP}\,(iii)$\ssc\,$],
%\equa{msssssss2}{\dis |X_0|=\ell_1,\ \ \ \frac{|X_0|^{3v_0}\cdot|Y_1|}{|X_1|^{1+v_0}}\sim\ell_1,\ \ \ |X_1|>|X_0|.}
%The last inequation and \eqref{mqp1234-2+} imply that we must have $h_{p_0,p_1}\sim|x_1|\sim|y_1|$ when $\ell_1\gg\kk$.
%
%
% then we obtain [cf.~\eqref{ImMpP}\,(b) and \eqref{VerySM}\,(i) and note that $\frac{|X_0|^6}{|X_1|^4}=\frac{|C_1|^3}{|X_1|}<1\ssc\,$]
%\equa{Wejsjsj}{\dis1<|X_1|=|X_0|^2,\ \ \ |X_0|^3<|Y_1|.}
%If $|X_1|\le1$ [this means that $|x_0|=\kk,\,|x_1|\le\kk$ but $|y_1|>\g_{\kk,\kk}$, cf notation \eqref{SimMMSMS}$\ssc\,$], we obtain a contradiction with definition of $\g_{\kk,\kk}$. Thus assume $|X_1|>1$.
%\item[(3)] Assume the first equality of \eqref{LetNSoOP}\,(ii) holds, i.e., $|X_1|=\d$.
%By \eqref{LetNSoOP}\,(i), we have $C_1\sim C_{20}\sim 1$ when $\ell\gg\ell_1$
%[note that when we say $\ell\gg\ell_1$, we regard $\ell_1,\kk$ as fixed, see also Remark \ref{rema-kk}$\ssc\,$].
%Thus $|Y_1|=|C_{20}X_1^2|\sim|X_1|^2\sim\d^2$
%[one should be very careful that when $|X_1|=\d$, we cannot expect that $|Y_1|\sim|X_1|$ since in this case
%$|X_1|\ll1\ll|X_0|\ssc\,$], and then $C_4\sim|Y_1|^{v_0}\sim\d^{2v_0}\ll1$ (note that $\d=\ell^{-1}\ll v_0$ when $\ell\gg\ll_1\gg\kk\ssc\,$), a contradiction with \eqref{LetNSoOP}\,(iii).
%
%Then by \eqref{ImMpP}\,(c), we have $|Y_1|\sim\d$ when $\ell=\d^{-1}\gg\ell_1$
% and
%$C_1\sim C_{20}\sim1$ when $\ell\gg\ell_1$. Then $C_4\sim|Y_1|\sim\d\ll1$, a contradiction with \eqref{LetNSoOP}\,(iii).
%\item[(4)] Assume the last equality of \eqref{LetNSoOP}\,(ii) holds, i.e., $|X_1|=\ell$.
%Then $|x_0|\ll|x_1|$ when $\ell\gg\ell_1\gg\kk$  [cf.~notation \eqref{SimMMSMS}$\ssc\,$],
%and
%  \eqref{mqp1234-2+} implies that we must have that $h_{p_0,p_1}\sim|x_1|\sim|y_1|$ when $\ell\gg\ell_1\gg\kk$.
%Then $C_{20}=\frac{Y_1}{X_1^2}\sim\d$ and $C_2\sim\d\ll1$, a contradiction with \eqref{LetNSoOP}\,(i).
% Then $|Y_1|\sim\ell$ by \eqref{ImMpP}\,(c), but then $C_{20}\sim\d$ and $C_2\sim\d\ll\d ^5$, a contradiction with \eqref{LetNSoOP}\,(i).
%, then
%$C_1\sim C_2\sim C_3\sim\ell_1$ when $\ell_1=\d_1^{-1}\gg1%\d ^{-1}
%$
%a contradiction with \eqref{ImMpP}\,(d).
%By \eqref{ImMpP}\,(c) and \eqref{LetNSoOP}\,(iii), we have $|Y_1|\sim|X_1|\succeq\ell_1$, and from $|X_0|=(C_1X_1)^{\frac12}$, we have \equa{metetetete}{\mbox{$\dis
%\frac{Y_1}{X_0^3}\!=\!\frac{Y_1}{(C_1X_1)^{\frac32}}\!\sim\!\frac1{C_1^{\frac32}X_1^{\frac12}}\!\ll\!\frac1{|X_0|}$, and thus $\dis \frac{Y_1}{X_0^3}+\frac{X_0^6}{X_1^4}\sim C_1^2\sim\ell_1$,}\!\!\!} and we see that $C_2\ll1$,
%a contradiction.
%
%\item[(3)] Assume the first equality holds in  \eqref{LetNSoOP}\,(ii), i.e., $|X_0|=\d$.
%Using the facts that $1\le C_1\le C_4$ and that $\ell=\d^{-1}\gg\ell_1$
%[note that when we say $\ell\gg\ell_1$, we regard $\ell_1,\kk$ as fixed
%(but we still required $\kk\gg\ell{\ssc\,})$ and that $v_0\ll\d$, see also Remark \ref{rema-kk}$\ssc\,$], we have
%$|X_1|\sim\d^2$, but then $C_5\sim\d\ll1$, a contradiction.
%\item[(4)]Assume the last equality  holds in  \eqref{LetNSoOP}\,(ii), i.e., $|X_0|=\ell$.
%
%
%.
%[the following computations are conducted up to $O(v_0)=O(\d^N)$, cf.~\eqref{v2=Aq}$\ssc\,$] we have
%\end{itemize}
%
%
}%
%
%
%
%
%
%
%
%
%
%
%
%
%
%
%
%
%
%
%
%
%
%
%
%
%
%
%
%
%
%
%
%
%
%
%
%we have the following.
%\begin{itemize}
%\item[(1)]
%If the first equality holds, or the second and third equalities hold, in
%\eqref{LetNSoOP}\,(i),
%
% hods, then $|X_0|=\d$. Using
%\eqref{ImMpP}\,(b) and the fact that  $1\le C_1\le C_4$, we obtain
%that $|X_1|\sim|X_0|^{\frac1{1+\d }}=\d^{\frac1{1+\d }}\ll|Y_1|$
%[when $\ell=\d^{-1}\gg\ell_1$; note that when we say $\ell\gg\ell_1$, we regard $\ell_0,\ell_1,\kk$ as fixed (but we still required $\kk\gg\ell{\ssc\,}){\ssc\,}$],
%a contradiction with \eqref{ImMpP}\,(d).
%
% (when $\ell=\d^{-1}\gg j$). Thus $|X_1|\sim \d$ by \eqref{ImMpP}\,(c), and so $C_5\sim\d^{1+\omega^5-1}=\d^{\omega^5}<1$, a contradiction with \eqref{LetNSoOP}\,(iii).
%\item[(2)]
%If the last equality of \eqref{LetNSoOP}\,(ii) hods, then $|X_0|=\ell$. As in case (1),
%we obtain
%[using \eqref{ImMpP}\,(d)$\ssc\,$] that $|Y_1|\sim|X_1|\sim\ell^{\frac1{1+\d }}$ (when $\ell\gg\ell_1$).
%Then
%\equa{C22=}{\mbox{$\dis
%C_{22}=\frac{|Y_1|^{1+\d ^2}}{|X_0|}\sim\ell^{\frac{1+\d ^2}{1+\d }-1}=\ell^{-\frac{\d (1-\d )}{1+\d }}=\d^{\frac{\d (1-\d )}{1+\d }}\ll1$
%when $\ell\gg\ell_1$.}}
%Thus $C_2<1$ (when $\ell\gg\ell_1$), a contradiction.
%
%
%$\big|\frac{X_0}{X_1^{1-\d }}\big|\sim\ell^{\frac{\d ^2}{1-\d +\d ^2}}\gg\ell_1^{3\ell_0^4}$.
%This together with \eqref{simMSMpp}\,(i)${}^+$ contradicts the fact that $C_2\le C_4$.
%
%
%From this and \eqref{msmsm}\,(i)${}^+$, together with the fact that $\ell\gg j$, we see that $C_2>C_4$, a contradiction.
%
%Thus the first term inside the absolute sign $|\cdot|$ of $C_2$ is $\sim$
%
%
%Thus $\kk^{-1}|x_1|\sim \d^{\frac2{2+\omega}}$ by \eqref{ImMpP}\,(c), and so $C_4\sim\d^{\frac{4(1+\omega)}{2+\omega}-\frac{2}{2+\omega}-1}=\d^{\frac{3\omega}{2+\omega}}<1$, a contradiction with \eqref{LetNSoOP}\,(iii).
%\item[(3)]
%If the first equality holds, or the second and third equalities holds,
%in \eqref{LetNSoOP}\,(i), i.e., $C_1=C_2=C_3=1$, then we  obtain [cf.~\eqref{ImMpP}\,(c)$\ssc\,$]
%\equa{x000wowow}{|Y_1|=|X_1X_0^{\d }|,\ \ \ |X_0|=|Y_1|^{1+\d },\ \ \ |Y_1|>1.}
%In particular, $|X_0|>1$ and  $|X_1|<|Y_1|$.
%
%From this and \eqref{ImMpP}\,(b),  we obtain that $|X_1|=|Y_1|^{\frac{1+\d ^2}{1+\d }}>1$ and $|X_1|<|Y_1|$.
%Denote $\bar k=|Y_1|>1$. Then \equa{thDenans}{\mbox{$\dis \bar k^{1+\d }\kk
%=|X_0|\kk=|x_0|$, \ \
%$\dis \bar k\kk=|Y_1|\kk>|X_1|\kk=|x_1|.$}}
%
\NOUSE{%
%
%For convenience, we denote $Y_1=\g_{\kk,\kk}^{-1}|y_1|$, $X_i=\kk^{-1}|x_i|,\, i=0,1$.
then by \eqref{msmsm}\,(ii),
we obtain \equa{Me9393}{\mbox{$\dis
\frac{X_0}{Y_1}=1$ and $\dis\frac{X_0^2}{Y_1}\ge1$,  thus $\dis Y_1\ge1$.}} Further by  \eqref{msmsm}\,(i)${}^+$ and the fact that $C_1=1$, we obtain
\equa{DMememme}{\dis\frac{X_0^2}{Y_1^{1}}\le\Big(\frac{1+\ep^2}{1-\ep^2}\Big)^{\scep}=1+O(\ep^3).}
By \eqref{Me9393} and \eqref{DMememme}, we see that $|X_0|\le1+O(\ep^3)$. From this and \eqref{LetNSoOP}\,(iii), we have that $|X_1|<1$. If $\bar k:=|X_0|=\kk^{-1}|x_0|\le1$, we obtain a contradiction with
definition of $\g_{\kk,\kk}$ and/or Theorem \ref{AddLeeme--0}\,(iii).
Thus assume $\bar k>1$. %Denote $\tilde k=\bar k^{\frac1{1+\frac{\omega}2}}$, t
Then
$\bar k\kk=|x_0|$, $\bar k\kk>|x_1|$.
\NOUSE{%
Then $\kk^{-1}|x_1|=$, and by \eqref{Si000}\,(ii), we have
$(\kk^{-1}|x_1|\big)^{\frac{1+2\scep}{\scep^2}}\ge\frac{1-5\scep}{1-6\scep}$, thus $\kk^{-1}|x_1|>1$, and
$\g_{\kk,\kk}^{-1}|y_1|>\kk^{-1}|x_1|$ by \eqref{LetNSoOP}\,(ii). Denote $k_0:=\kk^{-1}|x_1|>1$. Then
$|x_1|=k_0\kk,|x_0|<k_0\kk,$
%
%
%First by \eqref{ImMpP}\,(b), we have $\kk^{-1}|x_1|\ge1$, and
%by \eqref{SimPPPP}\,(iii) and the facts that $C_1=C_2=1$, we have $|x_0|\ge|x_1|\ge\kk$. Then by
%\eqref{LetNSoOP}\,(ii), we obtain \equa{Kskeke}{\dis
%\g_{\kk,\kk}^{-1}|y_1|>\frac{\kk^{-1}|x_0|^2}{|x_1|}\ge\kk^{-1}|x_0|\ge1.}
%If $|x_0|=\kk$, then \eqref{Kskeke} gives a contradiction with the definition of $\g_{\kk,\kk}$. Thus assume $k_0:=\kk^{-1}|x_0|>1$. Then $k_0\kk=|x_0|$,\, $k_0\kk\ge|x_1|$,
and}%
%
%
}%
%
%Thus by definition \eqref{Ak=1} and Theorem \ref{AddLeeme--0}\,(iii)% and the second equation of \eqref{x000wowow}
%, we obtain
%\equa{ByDsllo}{\g_{\bar k^{1+\d }\kk,\bar k\kk}>\g_{|x_0|,|x_1|}\ge|y_1|=\g_{\kk,\kk}|Y_1|=\bar k\g_{\kk,\kk},}
%which is a contradiction with Lemma \ref{G--lemm-assum1--}.
%
%[where \eqref{WeoOOOO}\,(c) follows from \eqref{LetNSoOP}\,(ii)$\ssc\,$]
%\equa{WeoOOOO}{{\rm(a)\ }Y_1=(X_0X_1)^{\frac12+v_0},\ \ \ {\rm(b)\ }X_0=Y_1^{1+2v_0},\ \ \
%{\rm(c)\ }Y_1>X_0^{1-\frac52v_0}
%.}
%By \eqref{WeoOOOO}\,(b) and (c), we deduce that $Y_1>Y_1^{1-\frac12v_0-5v_0^2}$, thus $Y_1>1$ and so $X_0=Y_1^{1+2v_0}>1$. By \eqref{WeoOOOO}\,(a) and (b), we have $Y_1=Y_1^{(\frac12+v_0)(1+2v_0)}X_1^{\frac12+v_0}$, i.e., $Y_1^{\frac12-2v_0-2v_0^2}=X_1^{\frac12+v_0}$, thus we obtain that $1<X_1<Y_1=X_0^{\frac1{1+2v_0}}$.
%Denote $k_0=X_0^{\frac1{1+2v_0}}$. Then
%\equa{ThenWhHaba}{\!\!\!\!\!\!|x_0|\!=\!X_0\kk\!=\!k_0^{1+2v_0}\kk,\,
%|x_1|\!=\!X_1\kk\!<\!X_0^{\frac1{1+2v_0}}\kk\!=\!k_0\kk,\,
%|y_1|\!=\!Y_1\g_{\kk,\kk}\!=\!X_0^{\frac1{1+2v_0}}\g_{\kk,\kk}\!=\!k_0\g_{\kk,\kk}.\!\!\!\!}
%By definition \eqref{Ak=1} and Theorem \ref{AddLeeme--0}\,(iii),
%we obtain
%\equa{MDMDMD333}{\g_{k_0^{1+2v_0}\kk,k_0\kk}\ge\g_{|x_0|,|x_1|}\ge|y_1|\ge k_0\g_{\kk,\kk},}
%which is a contradiction with Lemma \ref{G--lemm-assum1--} since $2v_0<\frac1m$ by \eqref{v1====}.
%
%i.e., $C_1=\d^2$, using \eqref{msmsm}\,(i)${}^-$, we obtain
%that $\kk^{-1}|x_0|\ge\d\big(1+O(\d^1)\big)$, and then by \eqref{LetNSoOP}\,(ii), we can obtain
%that $\g_{\kk,\kk}^{-1}|y_1|\ge\d^{-1}\big(1+O(\d^1)\big)\kk^{-1}|x_1|$, a contradiction with \eqref{ImMpP}\,(b).
%
%if the last two equalities hold, %or the second and third equalities hold,
%then
%
%[using \eqref{ImMpP}\,(a),
%\eqref{LetNSoOP}\,(ii) and the simple fact that $|a|\ge|a+b|-|b|$ for $a,b\in\C$; we compute the following up to $O(\ep^1)\ssc\,$]
%\equa{MSmdjejer}{\mbox{$\dis\g_{\kk,\kk}^{-1}|y_1|>\frac{(\kk^{-1}|x_0|)^2}{\kk^{-1}|x_1|}\ge\frac{\Big(3(1+\d)^4-2(\kk^{-1}|x_1|)^{-4}\Big)^2}{\kk^{-1}|x_1|}>1+8\d$,}} a contradiction with \eqref{ImMpP}\,(c).
%
%\item[(2)]
%If the last equality holds in
% \eqref{LetNSoOP}\,(i), then when $\ell_1\gg\ell_0$ [note that when we say $\ell_1\gg\ell_0$, we regard $\ell_0,\ell,\kk$ as fixed (but we still require $\kk\gg\ell\gg\ell_1$; see also  Remark  \ref{rema-kk}${\ssc\,}$);
%note also from \eqref{tech-2} that $v_0<\d^N$ for any fixed $N\in\R_{>0}$ and that $\d\ll\d $, we can regard $\a_0$ as $1$, and $v_0^5$ appeared in $C_3,\,C_4$ can be omitted in our computations below], we have the following
%\begin{eqnarray}
%\label{M373611}
%&\!\!\!\!\!\!\!\!\!\!\!\!\!\!\!\!\!\!\!\!\!\!&
%{\rm(i)\  }\Big|\frac{X_1X_0^{\d }}{Y_1}\Big|=C_{10}=\ell_1,
%\nonumber\\[6pt]
%&\!\!\!\!\!\!\!\!\!\!\!\!\!\!\!\!\!\!\!\!\!\!&
% \ \ {\rm(ii)\ }\Big|\frac{Y_1^{1+\d }}{X_0}\Big|=C_2\sim\ell_1,\ \  {\rm(iii)\ }
%|Y_1|\ge(C_{10}C_2)^{\frac1{3\d }}\sim\ell_1^{{\frac{2\ell_0}{3}}}.
%\end{eqnarray}
%We have $|Y_1|\sim|X_1|$ by \eqref{ImMpP}\,(d).
%This together with \eqref{M373611}\,(i) implies that $|X_0|\sim\ell_1^{\frac1{\d }}=\ell_1^{\ell_0}.$
%Thus
%\begin{eqnarray}
%\label{M373611+}
%&\!\!\!\!\!\!\!\!\!\!\!\!\!\!\!\!\!\!\!\!\!\!&
%\frac{1}{|Y_0|^{\d -\d ^2}}\sim
%\frac{|X_0|}{|X_1|^{1+\d }}\cdot\frac{|Y_1|^{1+\d ^2}}{|X_0|}\sim\ell_1^{\a_0\ell_0^2+\ell_0(1-\a_0\ell_0^2)},\mbox{ i.e., }\nonumber\\[6pt]
%&\!\!\!\!\!\!\!\!\!\!\!\!\!\!\!\!\!\!\!\!\!\!&
%|Y_0|\sim\ell_1^{\a_1},\ \ \a_1:=-\frac{\a_0\ell_0^2+\ell_0(1-\a_0\ell_0^2)}{\d -\d ^2}=
%\end{eqnarray}
%
%$\big|\frac{X_0}{Y_1}\big|=j$ (when $j\gg1$).
%If $\big|\frac{X_0^2}{Y_1}\big|\not\sim j^{j^{20}}$, then using
%\eqref{msmsm} [cf.\eqref{SuThahsh}--\eqref{v2=Aq}$\ssc\,$], we see either $C_1\le C_2$ or $C_2\le C_3$ cannot hold. Thus $\big|\frac{X_0^2}{Y_1}\big|\sim j^{\a_0}\sim j^{j^{20}}$
%
%
%
%
%
% \equa{mememj}{\dis
%\kk^{-1}|x_0|=\big(\kk^{-1}|x_1|\big)^{\frac{1+2\scep}{1+5\scep}},} using this,
%\eqref{Si000}\,(ii), and the fact that $C_1=1$, we obtain
%\equa{9ei3j3}{\dis(1-\ep)\big(\kk^{-1}|x_1|\big)^{\frac{1+2\scep}{\scep^2}}
%\ge1.} %i.e., $(\kk^{-1}|x_1|\big)^{\frac{10(1+2\scep)}{\scep(1+5\scep)}}>\frac1{1-\scep}$.
%Thus $\kk^{-1}|x_1|>1$. From this, \eqref{mememj} and \eqref{Si000}\,(ii), we obtain $1<\kk^{-1}|x_0|<\kk^{-1}|x_1|<\g_{\kk,\kk}|y_1|$. As in case (1), we obtain a contradiction.
%\item[(3)]
%If the second and third equality holds in
% \eqref{LetNSoOP}\,(i), then again we have \eqref{mememj}, from this, \eqref{Si000}\,(ii), and the fact that $C_1=C_2$, we obtain
%\equa{+9ei3j3}{\dis(1-\ep)\big(\kk^{-1}|x_1|\big)^{\frac{1+2\scep}{\scep^2}}
%\ge\big(\kk^{-1}|x_1|\big)^{\frac{(1-5\scep)(1+2\scep)}
%{\scep^2(1+5\scep)}},} i.e., $(\kk^{-1}|x_1|\big)^{\frac{10(1+2\scep)}{\scep(1+5\scep)}}>\frac1{1-\scep}$. Thus $\kk^{-1}|x_1|>1$. As in case (2),
%From this, \eqref{mememj} and \eqref{Si000}\,(ii), we obtain $1<\kk^{-1}|x_0|<\kk^{-1}|x_1|<\g_{\kk,\kk}|y_1|$. As in case (1),
%we obtain a contradiction.
%
%\item[\rm(4)]If the last equality holds in
% \eqref{LetNSoOP}\,(i),
%then $\kk^{-1}|x_1|=\ell$. If $\kk^{-1}|x_0|\ge\kk^{-1}|x_1|$, then by
%\eqref{Si000}\,(i)${}^-$ (and the fact that $\ell\gg\frac1{\scep}\ssc\,$), we obtain $C_1>C_2$
%(since $5\ep\big(\kk^{-1}|x_0|\big)^{\frac{1+5\scep}{\scep^2}}\gg(\kk^{-1}|x_0|\big)^{\frac{1-5\scep}{\scep^2}}{\ssc\,}$), a contradiction. Thus $\kk^{-1}|x_0|<\kk^{-1}|x_1|$, then $\g_{\kk,\kk}^{-1}|y_1|>\kk^{-1}|x_1|$ by
%\eqref{LetNSoOP}\,(i). Again we obtain a contradiction as in case (3).
\NOUSE{%
%$C_2=1+\d$, and by \eqref{SimPPPP}\,(i),\,(ii)${}^+$,\,(iii), and \eqref{LetNSoOP}\,(ii), we have
%[since we only compute the following up to $O(\d^3)$, we may assume $v_0=0$ by \eqref{v2=Aq}$\ssc\,$]
\begin{eqnarray}
\label{x1ToBe}
&\!\!\!\!\!\!\!\!\!\!\!\!\!\!\!\!\!\!\!\!\!\!\!\!&
{\rm(i)\ }\kk^{-1}|x_1|\ge\frac{(1+\d)^2-1}{\d}=2+\d,
\ \ \
{\rm(ii)\ }\kk^{-1}|x_1|\le\frac{(1+\d)^2+1}{\d}=2\d^{-1}+2+\d,
\nonumber\\[4pt]
&\!\!\!\!\!\!\!\!\!\!\!\!\!\!\!\!\!\!\!\!\!\!\!\!&
{\rm(iii)\ }\kk^{-1}|x_0|\ge\frac{(1+\d)^{-\a_1}(\d^2+1)-1}{\d^2}\kk^{-1}|x_1|=
2\d^{-1}-2\d-\frac{5\d^2}{3}+O(\d^3),
\nonumber\\[6pt]
&\!\!\!\!\!\!\!\!\!\!\!\!\!\!\!\!\!\!\!\!\!\!\!\!&
{\rm(iv)\ }\g_{\kk,\kk}^{-1}|y_1|>(1+\d)^{\a_2}\Big|\frac{x_0}{x_1}\Big|\kk^{-1}|x_0|\ge(1+\d)^{\a_2}
\frac{(1+\d)^{-\a_1}(\d^2+1)-1}{\d^2}\Big(2\d^{-1} - 2 \d - \frac{5\d^2}{3}\Big)
\nonumber\\[4pt]
&\!\!\!\!\!\!\!\!\!\!\!\!\!\!\!\!\!\!\!\!\!\!\!\!&
\phantom{{\rm(iv)\ }\g_{\kk,\kk}^{-1}|y_1|}
\ge(1+\d)^{2(1+\d)}\Big(1-\d-\frac{\d^2}{2}+O(\d^3)\Big)\Big(2\d^{-1} - 2 \d - \frac{5\d^2}{3}\Big)
\end{eqnarray}
%
$\big|\frac{x_0}{x_1}\big|
\ge1-\d-\frac{\d^2}{2}+O(\d^3)$ by \eqref{ImMpP}\,(a). Thus
by \eqref{LetNSoOP}\,(ii), we have
\equa{gGgG}{\dis
\frac{\g_{\kk,\kk}^{-1}|y_1|}{\kk^{-1}|x_1|}
>\Big|\frac{x_0}{x_1}\Big|^2C_2^{a_2}
\ge\Big(1-\d-\frac{\d^2}{2}\Big)^2(1+\d)^{2(1+\d)}+O(\d^3)
\ge1+\d+O(\d^2),}which is a contradiction with \eqref{ImMpP}\,(c).
}\NOUSE{%
%{v2=Aq}
%
%{SimPPPP}
% $\kk^{-1}|x_1|\ge$
% then up to $O(v_0^2)$, we have
%\equa{+WeoOOOO}{\!\!\!\!\!\!{\rm(a)\ }Y_1=(1+\d)(X_0X_1)^{\frac12+v_0},\ \ \ {\rm(b)\ }X_0=(1+\d)Y_1^{1+2v_0},\ \ \
%{\rm(c)\ }Y_1>(1+\d)^{-1}X_0^{1-\frac52v_0}
%.\!\!\!\!\!\!}
By \eqref{+WeoOOOO}\,(b) and (c), we obtain that \equa{Y1====}{Y_1>(1+\d)^{-1}\big((1+\d)Y_1^{1+2v_0}\big)^{1-\frac52v_0}=(1+\d)^{-\frac52v_0}Y_1^{1-\frac12v_0-5v_0^2},} i.e., $Y_1>(1+\d)^{\frac{-\frac52v_0}{\frac12v_0+5v_0^2}}=(1+\d)^{-5}+O(v_0^1).$
This proves that $|y_1|\succeq\kk$ and $|x_0|\preceq|y_1|^{1+\d}$ by
\eqref{+WeoOOOO}\,(b). Thus we have \eqref{ImMpP}\,(d), but
%
 i.e., $\kk^{-1}|x_1|=1$. Using $C_1\le C_2$ and \eqref{LetNSoOP}\,(ii), we obtain $\g_{\kk,\kk}^{-1}|y_1|>1$. Since $|x_0|\le\kk$ by \eqref{ImMpP}\,(a), we obtain a contradiction with
definition of $\g_{\kk,\kk}$ and/or Theorem \ref{AddLeeme--0}\,(iii).
\item[(3)]If the second and third equalities hold in \eqref{LetNSoOP}\,(i), i.e., $C_1=C_2=C_3$, then
 \eqref{LetNSoOP}\,(ii) gives that $|y_1|>\g_{\kk,\kk}$, this together with \eqref{ImMpP}\,(a) again implies a
  contradiction as in case (2).
}%
%
%If the first two equalities hold in \eqref{LetNSoOP}\,(i), then $-\ep\le1-\ep-\kk^{-1}|x_1|\le\ep$ (thus $\kk^{-1}|x_1|\le1$), and
%$k:=\kk^{-1}|x_0|\ge3-2(\kk^{-1}|x_1|)^{-4}\ge1$, further,
%$\g_{\kk,\kk}^{-1}|y_1|>\kk^{-1}|x_0|=k\ge1$. If $k=1$, then we obtain a contradiction with
%definition of $\g_{\kk,\kk}$ and/or Theorem \ref{AddLeeme--0}\,(iii). Thus assume $k>1$.
%
%
%
% then $C_1=C_2=C_3=\kk^{-1}|x_1|=1$, thus $k:=\kk^{-1}|x_0|>1$ by \eqref{LetNSoOP}\,(iii).
%Using \eqref{M8938383}\,(i) and the fact that $C_2=1$, we obtain
%If we denote $k_0=k_1=\kk$, then
%$kk_0=|x_0|$ and $kk_1>|x_1|$. Thus by the definition of $\g_{k_0,k_1}$ and Theorem \ref{AddLeeme--0}\,(iii), we obtain the following very crucial fact,
%\equa{Amsmw334}{\mbox{$\dis\g_{kk_0,kk_1}>\g_{|x_0|,|x_1|}\ge|y_1|\ge k\g_{\kk,\kk}=k\g_{k_0,k_1}$,}}
% which is a contradiction with Lemma \ref{G--lemm-assum1--}.
%
%then $\g_{\kk,\kk}^{-1}|y_1|=\d^{-1}$,
%but by \eqref{LetNSoOP}\,(ii) and \eqref{ImMpP}\,(a), we obtain that $|y_1|>2^{1+\d}|x_1|=2\big(1+\ln(2)\d+O(\d^2)\big)|x_1|$ (where $\ln(\cdot)$ is the natural
%logarithmic function),
%
%<\frac{2(2\d^{-2}+1)+1}{(2\d^{-2}+1)^{1-\d}}=O(\d^0)$,
%a contradiction with \eqref{ImMpP}\,(c).
%
%but by \eqref{TaKa}, \eqref{ImMpP}\,(a) and
%\eqref{LetNSoOP}\,(ii), we have
% $|y_1|>\frac{2^{1-\d^4}\kk}{2+(1 - 2 \ln(2)\d+O(\d^2)}>(1-\d^4)|x_1|$ (note that $1-2\ln(2)<0$),
%
% $|x_1|<|x_2|$ (note that
%$<0$), and
%but by , $|y_1|\gg\ell$ (when $\ell\gg1$),
%a contradiction with
%\eqref{ImMpP}\,(c).
%and
%so $|x_1|=|x_0|=\kk$, but by \eqref{LetNSoOP}\,(ii), $|y_1|>\g_{\kk,\kk}$, a contradiction with definition
% \eqref{Ak=1}.
%Using \eqref{SimPls}\,(i)$^{\pm}$ and the fact that $C_2=1$, we obtain
%$\frac{1}{1+2\ep}\le\kk^{-1}|x_0|\le1$.
%Thus by definition \eqref{Ak=1} and Theorem \ref{AddLeeme--0}\,(iii), we must have $|y_1|\le\g_{|x_0|,|x_1|}<\g_{\kk,\kk}$.
%Expanding $C_i$ as power series of $\ep$ up to $O(\ep^2)$, we have,
%%can then compute [we compute up to $O(\ep^2)$, where $e$ is the natural number]
%\equa{SMDMD222}{C_{41}=\frac{1+5\ep}{e^4},\,C_{43}\le
%\Big(
%=e^4(1-4\ep),\,C_{44}.}
%
%(where $e$ is the natural number)
%\equa{msmsmx0==}{\dis\d\kk|x_0|^{-1}\ge(1-\d)^{\scep}(1+\d)-(1-\d)^{\d^{-1}}=1+\d+O(\ep^1)-\big(e^{-1}+O(\d^1)\big),}
%by noting that $\frac{b_\kk}{a_\kk+1}>1+\frac{\d}{a_\kk+1}$ $\big[$by Lemma \ref{NeMore-}; thus
%$\big(\frac{b_\kk}{a_\kk+1}\big)^\ell>\ell^5{\ssc\,}\big]$ and $(1-\d)^{\scep}=1+O(\ep^1)$, $(1-\d)^\ell<\ell^{-1}$. Then \eqref{msmsmx0==}
%which contradicts \eqref{ImMpP}\,(b).
%
%$C_1=C_2=C_3=1$, and $|y_1|=\g_{\kk,\kk}$ [thus $|x_1|<\kk$ by \eqref{LetNSoOP}\,(ii)],
%and $(\kk^{-1}|x_0|)^{-\ell}\ge1$ by \eqref{SimPls}\,(a), i.e., $|x_0|\le\kk$.
%, and by \eqref{LetNSoOP}\,(ii), up to $O(\ep^4)$,  we have
%[note that we have defined $\ell_1$ in \eqref{ell2==1} in order to have the following]
%\equa{C4===334}{(\kk^{-1}|x_1|)^\scep \!\le\!(\kk^{-1}|x_1|)^\scep  C_3\!=\!1\!+\!(\ell_1\!-\!\eta \!-\!2)\ep^2\!+\!\frac{\ep^3}{2}(-3\!+\!\ell_1\!-\!4\ell_0\!-\!2\ell_0^2)\!=\!1\!-\!
%\ep^3\!<\!1,}
%i.e., $|x_1|<\kk$.
%Thus we obtain that $\g_{|x_0|,|x_1|}\ge|y_1|=\g_{\kk,\kk}$, a contradiction with
%definition of $\g_{\kk,\kk}$ and/or Theorem \ref{AddLeeme--0}\,(iii).
%
%[using \eqref{LetNSoOP}\,(ii)] and $|y_1|\ge\big(1+O(\kk^{-1})+O(\ep^1)\big)
%\g_{\kk,\kk}\ge\big(1+O(\kk^{-1})+O(\ep^1)\big)\kk$ [by \eqref{ImMpP}\,(d) and \eqref{TaKa}], a contradiction with \eqref{ImMpP}\,(c).
%
% and $C_2=1$, we have
%$(\d^{-1}+1)\kk^{-1}|x_0|\le C_2+\d^{-1}\kk|x_0^2x_1|=\d^{-1}+1$, i.e., $\kk^{-1}|x_0|\le1$, and so $\kk^{-1}|x_1|\le1$ by $C_1=1$. This and \eqref{ImMpP}\,(d) mean that $\g_{|x_0|,|x_1|}\ge|y_1|>\g_{\kk,\kk}$, a
%a contradiction with definition of $\g_{\kk,\kk}$ and/or
%Theorem \ref{AddLeeme--0}\,(iii).
%
%
%and using $C_3\le C_4$, we obtain $\kk^{-1}|x_0|\ge\d^{-1}+O(\d^0)$, and thus
%$\kk^{-1}|x_1|\le\d+O(\d^2)$, but by \eqref{LetNSoOP}\,(ii), $\g_{\kk,\kk}^{-1}|y_1|\ge\d^{\frac12}+O(\d^{\frac32})$, a contradiction with  \eqref{TaKa} and \eqref{ImMpP}\,(c).
%
%but by \eqref{LetNSoOP}\,(ii) and \eqref{ImMpP}\,(c),\,(d),
%up to $O(\ep^2)$, we have $\kk^{-1}|x_1|<30(\g_{\kk,\kk}|y_1|)^2\d<40\g_{\kk,\kk}^{-1}|y_1|\d$, a contradiction with the first inequality of \eqref{ImMpP}\,(e);
%
\NOUSE{%
\item[(2)]If the last equality of \eqref{LetNSoOP}\,(i) holds, then $\kk^{-1}|x_0|=\ell$ [and thus $\kk^{-1}|x_0|>\ell$ by \eqref{LetNSoOP}\,(iii)${\ssc\,}$], and
thus $C_2,C_3,C_4$ are elements which are $\preceq\ell^{1+\d^{4}}$ [when $\ell\gg\d^{-\d^{-1}}$; for convinience, we use notations  as in \eqref{MSM1111}; here $\a\sim\b,\,\a\preceq\b$ and $\a\prec\b$ mean respectively that $\frac{\a}{\b}=O(\ell^0)=\frac{\b}{\a}$, $\frac{\a}{\b}=O(\ell^0)$ and $\frac{\a}{\b}=O(\ell^{-c})$ for some $c>0\ssc\,$].
Thus the left-hand sides of both inequations of \eqref{M8938383}
%a contradiction with \eqref{LetNSoOP}\,(i)
are elements which are $\preceq\ell^{\frac{1}{\a}(1+\d^4)}\prec\ell$ [cf.~\eqref{LetNSoOP}\,(iv)$\ssc\,$], therefore the second term in the right-hand side of \eqref{M8938383}\,(i) must be a number which is $\sim\ell$.
This implies that $\ell^3\preceq\g_{\kk,\kk}^{-1}|y_1|$, a contradiction with
\eqref{ImMpP}\,(c).
%then $\kk^{-1}|x_0|=1$, $\g_{\kk,\kk}^{-1}|y_1|$
%
%
%but by \eqref{TaKa} and \eqref{LetNSoOP}\,(ii), $|y_1|\ge1-\ell^{-4}$, again a contradiction with
%\eqref{ImMpP}\,(c).
%
\item[(3)] If the equality of \eqref{LetNSoOP}\,(ii) holds, then $\kk^{-1}|x_0|=\ell^4$.
If $\g_{\kk}^{-1}|y_1|<\ell^3$, then the second term side the absolute sign $|\cdot|$ in $C_2$ is an $O(\ell^{-1})$ element
% as in case (2) above, we can obtain that $\ell^4\preceq\g_{\kk,\kk}|y_1|$
(since $1\le\kk^{-1}|x_1|$), which
%
again a contradiction with \eqref{ImMpP}\,(c).
%
}%
%
%By \eqref{M8938383}, we see that the first and
%
%If the third equality of  \eqref{LetNSoOP}\,(i) holds, then we can solve again a contradiction with
%\eqref{ImMpP}\,(a).
%\item[(4)]If the first two equalities of  \eqref{LetNSoOP}\,(i) holds, then $|x_0|=|x_1|=\kk$, but by
%\eqref{LetNSoOP}\,(ii), $|y_1|>\g_{\kk,\kk}$,
%a contradiction with \eqref{ImMpP}\,(c) and (d).
%then $C_1=C_2=C_3=1$, thus $|y_1|=\g_{\kk,\kk}$, and $|x_0|\le\kk$ [using
%\eqref{SimPls}\,(a)], and so $|x_1|<\kk$ by \eqref{LetNSoOP}\,(ii). By definition \eqref{Ak=1},
%$\g_{|x_0|,|x_1|}\ge|y_1|=\g_{\kk,\kk}$, a contradiction with Theorem \ref{AddLeeme--0}\,(iii).
%then $C_1=C_2=C_3=1$, and so $|x_0|=\kk$ [thus $|x_1|<\kk$ by  \eqref{ImMpP}\,${\rm(b)}{\ssc\,}]$, and using \eqref{UsUEUEUE}\,(i) and the fact that $C_2=1$, we obtain $|y_1|\ge\g_{\kk,\kk}$,
%which means that $\g_{|x_0|,|x_1|}\ge|y_1|\ge\g_{\kk,\kk}$,
%a contradiction with
%definition of $\g_{\kk,\kk}$. % and/or Theorem \ref{AddLeeme--0}\,(iii).
%we have $(\d^{-1}+1)\kk^{-1}|x_0|\ge
%\d^{-2(2\d^{-1}+1)(1-\d^4)}-\d^{-1}\kk|x_0^{-2}x_1|=\d^{-2(2\d^{-1}+1)(1-\d^4)}-\d$, and using $C_3=C_4$, we have
%$\kk^{-1}|x_1|$
%
%$C_3=C_4=1$, and so $(\d^{-1}+1)\kk^{-1}|x_0|-\d^{-1}\le C_3=1$, i.e.,
%
%
%but by \eqref{LetNSoOP}\,(ii) and \eqref{ImMpP}\,(b)--(d),
%up to $O(\ep^2)$, we have $\kk^{-1}|x_1|<30(\g_{\kk,\kk}|y_1|)^2=30\g_{\kk,\kk}|y_1|\d$, again a contradiction with the first inequality of \eqref{ImMpP}\,(e);
%\item[3]if the second and third equalities hold  in \eqref{LetNSoOP}\,(i).
%thus  $1-(1+\d)\ep\le\kk^{-1}|x_0|\le1-(1-\d)\ep$ by  \eqref{TaKa}, and
%$(\kk^{-1}|x_0|)^{-\frac3{\scep}}\le\big(1-(1+\d)\ep\big)^{-\frac3{\sc\ep}}<e^{4(1-\d)}$ (where $e$ is the natural number),
%$\g_{\kk,\kk}|y_1|$
%by \eqref{LetNSoOP}\,(ii)], and
%$-\d^3\le\kk^{-1}|x_0|-\d\le\d^3$ [thus $\kk^{-1}x_0|\ge\d+O(\d^2){\ssc\,}$], and
%by \eqref{TaKa} and the third inequality of \eqref{LetNSoOP}\,(i), we have $|y_1|\ge|x_0|=\d+O(\d^3)>|x_1|$
%[thus $|x_1+y_1|\ge|y_1|-|x_1|\ge\d^2\kk\succ\kk^{\frac{m}{m+1}}\sim h_{_{\sc p_0,p_1}}^{^{\sc\frac{m}{m+1}}}$],
%and we can obtain a contradiction as before;
%\item[(2)]if the second equality holds, by Lemma \ref{FnMlemm}, we have $|\bar x_0^{-1}x_0-\d|=(1-\d)\kk^{-1}|x_0|\le1-\d$,
%but by
%definition \eqref{Ak=1} and
%\eqref{TaKa} and
%\eqref{LetNSoOP}\,(ii), $|y_1|>\g_{\kk,\kk}$, %and by definition \ref{Ak=1}, we obtain that $\g_{|x_0|,|x_1|}\ge|y_1|>\g_{\kk,\kk}$,
%a contradiction with definition of $\g_{\kk,\kk}$% and/or
%Theorem \ref{AddLeeme--0}\,(iii)
%.
%If the first and last equalities, or the last two equalities, holds in \eqref{LetNSoOP}\,(i), then $|x_1|=(1+\d)\kk$
%[and $|x_0|<\kk$ by \eqref{ImMpP}],
%and by \eqref{TaKa}, \eqref{LetNSoOP}\,(ii), $|y_1|>(1+\d)^{1+\d}\kk>(1+\d+\d^2)\kk>|x_1|$
%[thus $|x_1+y_1|\ge|y_1|-|x_1|\ge\d^2\kk\succ\kk^{\frac{m}{m+1}}\sim h_{_{\sc p_0,p_1}}^{^{\sc\frac{m}{m+1}}}$],
% and \eqref{ImMpP}, up to $O(\ep^1)$ [then $|y_1|$ can be regarded as $\g_{\kk,\kk}{\ssc\,}$], we have
%$|x_1|\le(1+\d)^{-2+\d^4}\kk<|y_1|$,
%and we can obtain a contradiction as before.
%
%a contradiction with %definition of $\g_{\kk,\kk}$ and/or
%Theorem \ref{AddLeeme--0}\,(iii).
%If the last two equalities hold
%in \eqref{LetNSoOP}\,(i), then $|x_0|=2\kk$, $|x_1|<\kk$, but by
%\eqref{TaKa} and \eqref{LetNSoOP}\,(ii),
%$|y_1|>2\kk>|x_1|$, and we can obtain a contradiction as before.
%If the first two equalities holds in \eqref{LetNSoOP}\,(i),
%then $|x_1|=(1-\d)\kk$, $|x_0|=\frac{1-\d-\ep^3}{1-\ep^3}\kk$,
% but by
%\eqref{TaKa} and \eqref{LetNSoOP}\,(ii), up to $O(\ep^6)$, we have $|y_1|>(1-\d)\big(\frac{1-\d-\ep^3}{(1-\d)(1-\ep^3)}\big)^{\frac1{\ep^3}}$, and we obtain a contradiction as before.
%
%by \eqref{MwWSWMW}. This means that $|x_1|<\kk$ but $\g_{|x_0|,|x_1|}\ge|y_1|\ge\g_{\kk,\kk}$ [the first inequality follows from definition \eqref{Ak=1}], a contradiction with Theorem \ref{AddLeeme--0}\,(iii).
%If the first and last equalities hold in \eqref{LetNSoOP}\,(i), then $|x_0|\le\frac{\kk}{1-\d}$.
%The first two equalities cannot simultaneously hold in \eqref{LetNSoOP}(i).
%\end{itemize}
Hence  Theorem \ref{real00-inj}\,(1)\,(ii) holds.
%
%
\NOUSE{%
last two equalities, or the first and last equalities,
hold in \eqref{LetNSoOP}\,(i), then
%\equa{THUSaSLO}{\mbox{$\dis|x_1|=\frac{|x_1+\d+\kk^{-2}|}{1+\d+\kk^{-2}}\le\frac{|x_1|+\d+\kk^{-2}}{1+\d+\kk^{-2}}$, i.e., $|x_1|\le\kk$,}}
we can obtain that
$|x_0|\le\kk,$ $|y_1|=\g_{\kk,\kk}$% and $|x_0^{-1}\!+\!(\d^{-1}\!-\!1)\bar x_0^{-1}|=\d^{-1}\kk^{-1}$ and so $|x_0|\le\kk$ by \eqref{SImMM}
, and by \eqref{LetNSoOP}\,(ii), $|x_1|<\kk$, which means that
$\g_{|x_0|,|x_1|}\ge|y_1|=\g_{\kk,\kk}$, a
%thus also $|x_0|\le\kk$, but by \eqref{LetNSoOP}\,(ii) (using the fact that $|x_0|=|x_1|$), \equa{y1000poasoas}{\mbox{$\dis|y_1|^{1-\d}>\g^{1-\d}_{\kk,\kk}\Big((1+\d^{-1})-\d^{-1}\kk^{\d-\d^2+\d^3}|x_0|^{\d-\d^2+\d^3}\Big)\ge
%\g^{1-\d}_{\kk,\kk}$,}}
contradiction with %definition of $\g_{\kk,\kk}$% and/or
Theorem \ref{AddLeeme--0}\,(iii).
If the first two %and last
equalities %, or the first and last equalities,
hold
\eqref{LetNSoOP}\,(i), then
$|y_1|=(1+\d)\g_{\kk,\kk}\ge(1+\d)\kk$ [cf.~\eqref{TaKa}${\ssc\,}$],
$|x_0|=(1-\d)(1+\d)^{-\frac{2-\d}{1-\d}}\kk=(1-3\d)\kk$
[up to $O(\d^2)$], and by \eqref{LetNSoOP}\,(ii), up to $O(\d^2)$, we have
$|x_1|\le(1-3\d)\kk<|y_1|$,
%
% [cf.~\eqref{TaKa}${\ssc\,}$]\equa{WHASSS}{\mbox{$\dis|y_1|=\frac{2-\d}{2-2\d}\g_{\kk,\kk}=\Big(1+\frac{\d}{2}\Big)\kk$, and \
%$\dis|x_0|=(1-\d)\Big(1+\frac{\d}{2}\Big)^{\frac53}\kk=\Big(1-\frac{\d}{6}\Big)\kk<|y_1|$
%
%$|y_1|=\frac12(\kk+1)\g_{\kk,\kk}\succeq\kk^2$
%,}}
%$|x_0|\preceq1$,
 %[and of course $|x_0|\preceq\ell$ by \eqref{x01===0}${\ssc\,}$],
%but by
%\eqref{LetNSoOP}\,(ii), we obtain $|x_1|<|x_0|<|y_1|$,
and we have
$|x_1+y_1|\ge|y_1|-|x_1|\sim|y_1|\succ\kk^{\frac{m}{m+1}}\sim h_{_{\sc p_0,p_1}}^{^{\sc\frac{m}{m+1}}}$, a contradiction
%
%up to $O(\d^1)+O(\ep^1)+O(\kk^{-1})$, we have
%[note that when \eqref{SMSMSMSMS} holds, $|x_1|$ and $\kk^{\kk^{-5}}|x_1|^{1-\kk^{-5}}$ only differ by an $O(\kk^{-3})$ element]
%\begin{eqnarray}
%\label{x01sx--}&\!\!\!\!\!\!\!\!\!\!\!\!\!\!\!\!\!\!&
%|x_1|=\frac{1+\d^2}{1-\d}\kk=(1+\d)\kk, \
%|x_0^{-1}\!+\!(\d^{-1}\!-\!1)\bar x_0^{-1}|\!=\!\frac{1\!+\!\d^{-1}}{1\!-\!\d}\!-\!1\!=\!(\d^{-1}\!+\!1\!+\!2\d)\kk
%,
%\nonumber\\[6pt]&\!\!\!\!\!\!\!\!\!\!\!\!\!\!\!\!\!\!\!\!\!\!\!\!&
%|y_1|\!\ge\!\Big(\frac{1\!+\!\d^2}{1\!-\!\d}\Big)^{\frac{1-\d}2-\d^4}
%\d\Big(\frac{1\!+\!\d^{-1}}{1\!-\!\d}\!-\!1\Big)
%\kk
%\nonumber\\[4pt]&\!\!\!\!\!\!\!\!\!\!\!\!\!\!\!&\phantom{|y_1|}
%\!=\!\Big(1\!+\frac{\d}{2}\Big)\Big(1\!+\!\d\Big)\kk
%\!=\!\Big(1+\frac{3\d}{2}\Big)\kk>|x_1|,
%\end{eqnarray}
%where the last inequation follows from \eqref{LetNSoOP}\,(ii) and
%\eqref{TaKa}. As before %(note that $|x_0|\le\kk$)
%we can obtain that $|x_1+y_1|\ge|y_1|-|x_1|\sim|y_1|\succ\kk^{\frac{m}{m+1}}\sim h_{_{\sc p_0,p_1}}^{^{\sc\frac{m}{m+1}}}$,
a contradiction with \eqref{mqp1234-2}.
%If the first and last  equalities hold
%in \eqref{LetNSoOP}\,(i) then  up to $O(\d^3)$ and up to $O(\ep^1)$, we have % [the following computations are all]
%\begin{eqnarray}
%\label{x01sx}&\!\!\!\!\!\!\!\!\!\!\!\!\!\!\!&
%|x_1|=\kk,\ \ \
%|x_0|\ge\frac{1-\d}{1+\d}\kk
%=(1-2\d+2\d^2-2\d^3)\kk,
%\nonumber\\[6pt]&\!\!\!\!\!\!\!\!\!\!\!\!\!\!\!&
%|y_1|\ge\Big(\frac{1-\d}{1+\d+\d^2}\Big)^{\frac{4+\d}{2+\d}-\d^4}
%\Big(\frac{1+\d}{1-\d}\Big)
%\kk
%\nonumber\\[4pt]&\!\!\!\!\!\!\!\!\!\!\!\!\!\!\!&
%\phantom{|y_1|}=(1-4\d+7\d^2-6\d^3)(1+2\d+2\d^2+2\d^3)\kk=(1-2\d+2\d^2+2\d^3)\kk>|x_1|,
%\end{eqnarray}
%where the last inequation follows from \eqref{LetNSoOP}\,(ii) and
%\eqref{TaKa}. As before (note that $|x_0|\le\kk$) we can obtain that $|x_1+y_1|\ge|y_1|-|x_1|\sim|y_1|\succ\kk^{\frac{m}{m+1}}\sim h_{_{\sc p_0,p_1}}^{^{\sc\frac{m}{m+1}}}$, a contradiction with \eqref{mqp1234-2}.
%If the first and last two equalities hold in \eqref{LetNSoOP}\,(i),
%then as above, we have
%\begin{eqnarray}
%\label{aax01sx}&\!\!\!\!\!\!\!\!\!\!\!\!\!\!\!&
%|x_1|=\frac{4(1+\d)}{4+\d}\kk=\Big(1+
%\frac34\d\Big)\kk,\ \ \
%|x_0|=\frac{5(1+\d)}{5+2\d}\kk
%=\Big(1+\frac35\d\Big)\kk,
%\nonumber\\[6pt]&\!\!\!\!\!\!\!\!\!\!\!\!\!\!\!&
%|y_1|\ge\Big(\frac{4(1+\d)}{4+\d}\Big)^{\frac{47}{15}-\d^2}
%\Big(\frac{5+2\d}{5(1+\d)}\Big)
%\kk=\Big(1+\frac{47}{20}d\Big)\Big(1-\frac35\d\Big)=\Big(1+\frac{7}{4}\d\Big)\kk>|x_1|,
%\end{eqnarray}
%then up to $O(\d^3)$, we have $|x_1|=\frac{1}{1-\d}\kk=(1+\d+\d^2)\kk$, $|x_0|=(1+\d)\kk$, and by
%\eqref{LetNSoOP}\,(i) and \eqref{TaKa}, $|y_1|>\frac{1+\d}{(1-\d)^{\frac{\d}{1+\d}}}\kk=
%(1+\d+2\d^2)\kk>|x_1|$,
%again we have a contradiction.
%[we compute up to $O(\d^4)\,$]
%\begin{eqnarray}
%\label{x1=101010}&\!\!\!\!\!\!\!\!\!\!\!\!\!\!\!\!\!\!\!\!\!&
%|x_1|=(1-\d)^{-1}\kk=\big(1+\d+\d^2+\d^3%+O(\d^4)
%\big)\kk,
%\ \ %\\
%\label{x0=101010}&\!\!\!\!\!\!\!\!\!\!\!\!\!\!\!\!\!\!\!\!\!&
%|x_0|\ge\frac{\kk^{1-\frac2{1+\d}}|x_1|^{\frac2{1+d}}}{1+\d}=\big(1+\d+\d^3%+O(\d^4)
%\big)\kk,
%\end{eqnarray}
%and $|x_0|\le\kk^{1-\frac2{1+\d}}|x_1|^{\frac2{1+d}}\le(1-\d)^{\frac2{1+\d}}\kk.$
%But by \eqref{LetNSoOP}\,(ii), we have [up to $O(\d^4)\,]$
%\equa{y1=01020202}{\dis|y_1|\ge
%\frac{|x_0|^{1+\d-\d ^4}}{\kk^{\d-\d ^4}}=\big(1+\d+\d^2+\frac32\d^3%+O(\d^4)
%\big)\kk>|x_1|.}
%$|x_0|=(1-\d )\kk,$ $|x_1|=\big(\frac{1+\d}{1-\d}-\d\big)^{-1}\kk=\frac{1-\d }{1+\d ^2}\kk=\big(1-\d -\d ^2+\d ^3+O(\d ^4)\big)\kk,$ but by
%\eqref{LetNSoOP}\,(ii), we have [using \eqref{TaKa}, we compute up to $O(\ep^2)$ (recall that $\ep<\kk^{-\kk}$)]
%\equa{YYSYSYSSSSS}{\!\!\!\!\!\!\!\!\dis|y_1|\!\ge\!
%\frac{\dis\kk^{2(1+\d )}\g_{\kk,\kk}|x_1|^{3+\d ^4}}{\kk^{3+\d ^4}|x_0|^{2(1+\d )}}\!\ge\!\frac{\big(1\!-\!\d\!-\!\d^2\!+\!\d^3\!+\!O(\ep^3)\big)^{3+\d^4}}{(1-\d)^{2(1+\d)}}\kk\!=\!
%\Big(1\!-\!\d\! -\!\d ^2\!+\!2\d ^3\!+\!O(\d ^4)\Big)\kk\!>\!|x_1|,\!\!\!\!}
%We have $|x_1+y_1|\ge|y_1|-|x_1|\sim|y_1|\succ\kk^{\frac{m}{m+1}}\sim h_{_{\sc p_0,p_1}}^{^{\sc\frac{m}{m+1}}}$, a contradiction with \eqref{mqp1234-2}.
%If the first and last equalities hold in \eqref{LetNSoOP}\,(i), then
%$|x_1|=\kk$, $|x_0|=(1-\d )\kk$, and by \eqref{LetNSoOP}\,(ii) [as in \eqref{YYSYSYSSSSS}],
%$|y_1|\ge\big(1+2\d +O(\d ^2)\big)\kk>|x_1|$, and again we can obtain a contradiction with
%\eqref{mqp1234-2}.
% and using the inequality of \eqref{THUSaSLO}, we have $|x_0|\le(1+\d)^{-1}\kk+O(\kk^{-1})$ (thus $\frac{|x_1|}{|x_0|}\ge1-\d^2$), but by
%\eqref{LetNSoOP}\,(ii),
%\equa{NOTTTTTaaa}{\dis|y_1|^{1-\d}\!{\sc\!}>\!\g^{1-\d}_{\kk,\kk}\Big((1\!+\!\d^{-1})(1\!-\!\d^2)\!-\!\d^{-1}(1\!+\!\d)^{-\d+\d^2-\d^3}\Big)=
%\g_{\kk,\kk}^{1-\d}\Big(1-\frac{5\d^2}{2}+O(\d^3)\Big)
%,}
%thus using \eqref{TaKa}, we have $|y_1|\ge(1-\frac{5\d^2}{2}+O(\d^3))\kk>|x_1|$.
%As before, we have $|x_1+y_1|\ge|y_1|-|x_1|\sim\kk\succ\kk^{\frac{m}{m+1}}\sim h_{_{\sc p_0,p_1}}^{^{\sc\frac{m}{m+1}}}$, a contradiction with  \eqref{mqp1234-2}.
%If the first and last equalities hold in \eqref{LetNSoOP}\,(i), then $|x_1|=|x_0|=\kk$, but by \eqref{LetNSoOP}\,(ii), $|y_1|>\g_{\kk,\kk}$,
%  a contradiction with definition of $\g_{\kk,\kk}$. % and/or Theorem \ref{AddLeeme--0}\,(iii).
%%If the first and last equalities hold in \eqref{LetNSoOP}\,(i), then $|x_0|=(1+\d)\kk$, and
%%$|x_1|\le\frac{\kk}{1+\d+\d^2}$ by \eqref{HAHSHSHSD}, but by \eqref{LetNSoOP}\,(ii), $|y_1|>\g_{\kk,\kk}$,
%If  the last two equalities hold in \eqref{LetNSoOP}\,(i), then
%$|x_1^{-1}\!+\!\d\bar x_1^{-1}|=(1+\d)^2\kk^{-1}$ [thus $|x_1|\le(1+\d+\d^2)^{-1}\kk=(1-\d+O(\d^3))\kk$ by \eqref{HAHSHSHSD}], and
%$|x_0|=(1+\d)\kk+O(\kk^{-1})$, but by
%\eqref{LetNSoOP}\,(ii), we have [up to $O(\kk^{-1})$] \equa{y1-upto}{\mbox{$|y_1|\ge(1+\d)^{-(1-\d-\d^4)}\kk=(1-\d+\d^2+O(\d^3))\kk$.}} As before, we have $|x_1+y_1|\ge|y_1|-|x_1|\sim\kk\succ\kk^{\frac{m}{m+1}}\sim h_{_{\sc p_0,p_1}}^{^{\sc\frac{m}{m+1}}}$, a contradiction
%with  \eqref{mqp1234-2}.
%
%
%[using \eqref{TaKa}, and we estimate up to $O(\kk^{-1})$] %[note that if \eqref{LetNSoOP}\,(ii) holds it must also hold when $\ep^2,\ep^3$ are removed since $\ep^3\ll\ep^2$],
%\equa{MusTHAHA}{|y_1|\ge\Big(\frac{(1-\d)(\kk\g^{\kk^{-1}}_{\kk,\kk}+\kk^{-\kk-3})-\kk^{-\kk-3}}{|x_0|}\Big)^{\frac1\d}=(1-\d)^{\frac1\d}\kk+O(\kk^{-2}).
%}
%As in the previous lemma, we must have $h_{p_0,p_1}\sim\kk$,
%but then $|x_1+y_1|\ge|y_1|-|x_1|\ge((1-\d)^{\frac1\d}-1)\kk+O(\kk^{-2})\sim\kk\succ h_{_{\sc p_0,p_1}}^{^{\sc\frac{m}{m+1}}}$, a contradiction with
%\eqref{mqp1234-2}.
%the first two equalities, or the first and last equalities, cannot simultaneously hold in \eqref{LetNSoOP}.
%, then we can  obtain $|x_1|=\kk$, $|x_0|\le\kk$,
%\equa{F1rst}{\mbox{$|x_1|=\kk^{-\kk+n_1-1}<\kk^{-\kk+n_1},$ \ \  $|x_0|\le\kk^{\l_0\l}$,}}
%where the second inequation follows by noting from \eqref{NSoOP} that since $|x_0|$ has the negative power $-\frac1{(\l_0-1)\l}$, when $|x_1|$ is reduced by the factor $\kk^{-1}$, the $|x_0|$ can be at most multiplied by a factor $\kk^{(\l_0-1)\l}(1+\a)$ for some $\a=O(\kk^{-1})$.
%By Theorem \ref{AddLeeme--0}\,(iii) and definition \eqref{Ak=1}, we obtain
%\equa{F2rst}{\mbox{$\g_{\kk^{\l_0\l},\kk^{-\kk+n_1}}>\g_{|x_0|,|x_1|}\ge|y_1|>|x_1|^{-\l}\kk^{\l(-\kk+n_1)}\g_{1,\kk^{-\kk+n_1}}>\kk^\l$,}}
%where the last two inequalities follow from \eqref{TaKa} and
%\eqref{NewAdded-NSoOP}, i.e., we have \eqref{WehHAH} (and the proof is completed in this case).
%a contradiction with the assumption at the beginning of the proof.
%and by \eqref{LetNewAdded-NSoOP}, $|y_1|>\g_{\kk,\kk}$,
%a contradiction with the definition of $\g_{\kk,\kk}$ and/or Theorem \ref{AddLeeme--0}\,(iii).
%If the last two  equalities hold in \eqref{LetNSoOP}, then $|x_1|=(1+\d)\kk$, $|x_0|\sim\kk$, but by
%\eqref{LetNewAdded-NSoOP},
%we have
%\equa{KAMSMSy}{|y_1|>(1+\d)^{2-e^{-\d}},}
%
% where $\a=\frac{\gamma_{\kk,\kk}}{\kk}>1$ by \eqref{TaKa}.
%As in the proof of Lemma \ref{G--lemm-assum1--}, we have $h_{p_0,p_1}\preceq\kk^{1+\d}\prec\kk^{\frac{m+1}{m}}$, and then
%$|x_1+y_1|\ge|y_1|-|x_1|\ge(\a-1)\kk\sim\kk\succ h_{_{\sc p_0,p_1}}^{^{\sc\frac{m}{m+1}}}$,  a contradiction with \eqref{mqp1234-2}.
%, a contradiction with definition \eqref{Ak=1}.
%$|x_1|=\d^{-1}=\kk$, and $|x_0|=(\frac{\kk-\d+\d^2-\d^3}{1-\d^3})^{\l_0\l}<\kk^{\l_0\l}$, and by
%\eqref{NSoOP}\,(ii), $|y_1|>\g_{1,1}\kk^\l>\TH\kk^\l.$
%In particular, by Theorem \ref{AddLeeme--0}\,(iii), we have $\g_{\kk^{\l_0\l},\kk^{-\kk}}\ge\g_{|x_0|,|x_1|}\ge|y_1|>\kk^\l$,
%
Hence %Therefore %the last two inequalities cannot simultaneously hold in \eqref{NSoOP}\,(i), i.e.,
 Theorem \ref{real00-inj}\,(1)\,(ii) holds.
}%
%
%One can easily verify from \eqref{TaKa} that $(\bar p_0,\bar p_1)\in V_0$, i.e.,
%






















Next, we want to choose suitable $u,v$ such that
\eqref{LetNSoOP} holds for $(q_0,q_1)$ [defined in \eqref{1+++q0q1}${\ssc\,}$], namely%
%[%estimate the inequations up to $\ep^2$ except \eqref{MSMSMSMSMS}\,(iii); further we
%will take $u,v,s$ so that the values inside the  absolute sign $|\cdot|$ will be positive, thus the absolute sign can be removed
%we emphasis again (cf.~Remark \ref{Fimsmsm}) that our choices of $u,v,s$ only depend on $\bar p_0,\bar p_1)$
%(which is defined in \eqref{TaKa}$\ssc\,$), thus are irrelevant to $\ell_1,\ell$]
,
\begin{eqnarray}
\label{MSMSMSMSMS}
&\!\!\!\!\!\!\!\!\!\!\!\!\!\!\!\!\!\!\!\!\!\!  &
\dis{\rm(i)\  }
1\le C_1:=|(1+s\ep)^3(1+u\ep)(1+v\ep)|^{\frac1{5-v_0}-v_0^5}
\le C_2:=|(1+s\ep)(1+u\ep)^2(1+v\ep)^2|^{\frac1{6-2v_0}}
\nonumber\\[2pt]
\!\!\!\!\!\!\!\!\!\!\!\!\!\!\!\!\!\!\!\!\!\!\!\!\!\!\!\!\!\!\!\!\!\!\!\!\!\!\!\!\!\!\!\!  \!\!\!\!\!\!\!\!\!\!\!\!\!\!\!\!\!\!\!\!\!\!&\!\!\!\!\!\!\!\!\!\!\!\!\!\!\!\!\!\!\!\!\!\!  &
\phantom{\dis{\rm(i)\  }
1}\le C_3:=|(1+s\ep)^3(1+u\ep)(1+v\ep)|^{\frac1{5-v_0}+v_0^5}\le C_4:=\ell%^{\frac1{5-v_0}+v_0^5}
,\ \ \ \ \
{\rm(ii)\ }\ep\le|1+u\ep|\le\ep^{-1},
\nonumber\\[2pt]
\!\!\!\!\!\!\!\!\!\!\!\!\!\!\!\!\!\!\!\!\!\!\!\!\!\!\!\!\!\!\!\!\!\!\!\!\!\!\!\!\!\!\!\!  \!\!\!\!\!\!\!\!\!\!\!\!\!\!\!\!\!\!\!\!\!\!&\!\!\!\!\!\!\!\!\!\!\!\!\!\!\!\!\!\!\!\!\!\!  &
{\rm(iii)\ }
C_5:=|1+u\ep|^2\Big(\frac{|(1+u\ep)^3(1+s\ep)(1+v\ep)^2|^{\frac{3}{6-2v_0}+v_0^5}}{|1+s\ep|^5}+\ep^4\Big)\ge1+v_0^5\ep.
\end{eqnarray}
Note that \eqref{MSMSMSMSMS}\,(ii) and the last strict inequality of \eqref{MSMSMSMSMS}\,(i) automatically hold.
%Since $b_\kk>0$, we can regard $s$ as a free variable and solve $v$ from \eqref{@suc2hthat=4} to obtain
%\equa{nVpooo}{\dis v=\frac{a_\kk u+s}{b_\kk}+O(\ep^1).}
Take %(recall that $c_\kk=a_\kk+b_\kk\ssc\,$)%[cf.~\eqref{v1====}$\ssc\,$]%[for some sufficiently large $\ell$ (which can depend on $\kk$)]
\begin{eqnarray}
\label{11F3rst}
&&\!\!\!\!\!\!\!\!\!\!\!\!\!\!\!\!\!\!
\dis u=1, \  v=1-v_0,\mbox{ and so, \ }
%\nonumber\\[4pt]
%&&\!\!\!\!\!\!\!\!\!\!\!\!\!\!\!\!\!\!
%\dis
s=1+O(\ep^1)\mbox{ by \eqref{@suc2hthat=4} and \eqref{v0=1=1=1=1}.}\end{eqnarray}
%where
%the assertion about $v$ follows from \eqref{@suc2hthat=4} and \eqref{v0=1=1=1=1}
%[we have defined $v_0$ in \eqref{v0=1=1=1=1} in order to have \eqref{11F3rst}$\ssc\,$]. %\eqref{@suc2hthat=4}, %\eqref{nVpooo},
%\eqref{Akk-bkk}
%and Lemma \ref{NeMore-}.
Then the coefficients of $\ep^1$ in $C_1,\,C_2$ and $C_3$ are respectively
\equa{mememememe}{\dis c_1:=1-(5-v_0)v_0^5,\ \ \ \ \ c_2:=
1,\ \ \ \ \ c_3:=1+(5-v_0)v_0^5,} i.e., we have all strict inequalities in
\eqref{MSMSMSMSMS}\,(i).
%
%$c_1=1-v_0$ and $c_3=1-v_0+v_0^5-v_0^6$. To compute $c_2$ (the coefficient of $\ep^1$ in $C_2$), observe the following.
%\begin{itemize}\item[(a)] $u\ep,v\ep$ do not contribute to $c_2$, thus the numerator in $C_2$ can be replaced by $2$
%for the purpose of computing $c_2$;
%\item[(b)] $2^{\frac{\scep}{\a_0}}=1+\frac{\ln(2)\scep}{\a_0}+O(\ep^2)$;
%\item[(c)] $(\ep^{-3})^{\frac{\scep}{\a_0}}=1-\frac{3\ln(\scep)\scep}{\a_0}+O(\ep^2)$;
%\item[(d)] $c_2=\frac{\ln(2)}{\a_0}-\frac{3\ln(\scep)}{\a_0}=\frac{\ln\big(\frac2{\scep^3}\big)}{\a_0}=1-v_0+v_0^6$ by \eqref{v0=1=1=1=1} [we have defined $\a_0$ in \eqref{v0=1=1=1=1} in order to have this].
%\end{itemize}
%Thus $0<c_1<c_2<c_3$, i.e., we have all strict inequalities of \eqref{MSMSMSMSMS}\,(i).
Further, the coefficient of $\ep^1$ in $C_5$ is $(6-2v_0)v_0^5>v_0^5$, i.e., \eqref{MSMSMSMSMS}\,(ii) holds.
%
% is [using \eqref{v0=1=1=1=1}$\ssc\,$]
%\equa{c1==010101}{\dis c_2=-2\a_0+v-2u=-(6+v_0-\d^3)-(1-v_0)+2=-5+\d^3,}
%and the coefficients pf $\ep^1$ in $C_1$ and $C_3$ are respectively $c_1=-5$ and $c_3=-5+\d^2$. Thus
%by \eqref{v2=Aq} [this is why we define $\a_1$ in \eqref{v2=Aq}$\ssc\,$], i.e.,
%all strict inequalities of
% \eqref{MSMSMSMSMS}\,(i)hold. The coefficient of $\ep^1$ in $C_4$ is \equa{c4==010101}{\dis c_4=2u-(1+2v_0)s-(1-v_0)v=-2+1+2v_0+(1-v_0)^2=v_0^2.} Thus
%\eqref{MSMSMSMSMS}\,(iii) holds.
% The coefficient of $\ep^1$ in $C_2$ is $c_2=\frac{\d }{1+\d}>c_1$. We have \eqref{MSMSMSMSMS}\,(i).
%The coefficient of $\ep$ in the left-hand side of \eqref{MSMSMSMSMS}\,(ii) is the following [this is why we define $\a_2$ in \eqref{v2=Aq}$\ssc\,$]
%\equa{c1==010101+}{\dis\Big(1-\frac{\a_2\d}{\d+1}\Big)u+v_0-2s
%=\big(1-2(\d-v_0)\big)+(1-v_0)-2(1-\d)
%=v_0>0.}
\NOUSE{%
Then one can easily observe
\equa{eEuusus}{\mbox{$
\dis\d^2<C_1=
1-\Big(1+\frac{v_0}{2}\Big)\ep+O(\ep^2)
<C_2=1-\ep-\ep^2
<C_3=1-\ep<1$,}} i.e., all strict inequalities of \eqref{MSMSMSMSMS}\,(i) hold. Further one can easily see that the coefficient of $\ep^1$ of the left-hand side of \eqref{MSMSMSMSMS}\,(ii) is $v+\frac{v_0s}{2}-u=\frac{v_0}{2}>0$.
}%
%
% by \eqref{Akk-bkk} and Lemma \ref{NeMore-}.
%Then $C_2=1+\frac{\d^4}{2}\ep$ and thus it
%is straightforward to see that all strict inequalities hold in \eqref{MSMSMSMSMS}
%[the left-hand side of \eqref{MSMSMSMSMS}\,(iii) is $1+\ep^2+O(\ep^3)>1+\ep^3\ssc\,$].
  %by comparing the coefficients of $\ep^1$ [note that the coefficients of $\ep^1$ in
%$\frac{4}{5}|1+s\ep|\big(|1+u\ep|^{-1}\!{\sc\!}+\!\frac14\big)$ and in $\frac{3}{4\big(|1+u\ep|^{-1}
%\!{\sc\!}-\!\frac14\big)}$ are respectively
%$s-\frac{4u}{5}=\frac{4}{3}-\kk^{-\ell}$ and $\frac{4u}{3}=\frac43$], we see that
%all strict inequalities hold in \eqref{1LetNSoOP}\,(i). Further, the coefficient of $\ep^1$ in the left hand-side of \eqref{1LetNSoOP}\,(ii) is
%\equa{siSiS}{\mbox{$\dis s+v-\frac{47}{15}u
%=\frac{a_\kk-(1-\d^2)b_\kk+\frac{32}{15}-\kk^{-\ell}(1-b_\kk)}{b_\kk},$}}
% which, by Lemma \ref{NeMMMMore-} %(by replacing $\d$ there to be $\d^5$)
% and \eqref{Akk-bkk}, is positive when $\ell$ is sufficiently large,
%[by taking $\d$ in Lemma \ref{NeMMMMore-} to be $\d^5$ and using \eqref{Akk-bkk}${\ssc\,}$]. We
%
%Then [we compute $C_1,C_2,C_3$ up to $O(\ep^4)$ and $C_0$ up to $O(\ep^1)$, where $\ln$ is the natural
%logarithm; note that we have chosen $s$ in \eqref{F3rst} so that $C_2$ has the required form below];
%\begin{eqnarray}
%\label{isisidjd}
%&\!\!\!\!\!\!\!\!\!\!\!\!\!\!\!\!\!\!\!\!&
%C_0\!=\!1\!+\!\ln(1\!-\!\d)\ep\!<\!
%C_1\!=\!1\!-\!\ep^2\!-\!\frac{\ep^3}{2}
%\!<\!C_2\!=\!1-\ep^2
%\!<\!C_3\!=\!1\!-\!\ep^2\!+\!\frac{\ep^3}{2}\!<\!1,\end{eqnarray}
% [where the second strict inequality follows from Lemma \ref{NeMore-}; also note that $\ep<\ep^{1-\scep^5}<\ep+\ep^5$],
%i.e.,
%\eqref{MSMSMSMSMS}\,(i) hold.
\NOUSE{To persuade readers that our computations above have no problem, one can simply set
$s=-\d^{-2}+s_1\ep+s_2\ep^2$ for some $s_1,s_2\in\R$, then one can easily see, by observing that the coefficient of $\ep^1$ in $\d(1+s\ep)^{-1}+(1+v\ep)^{\d^{-1}}$ is zero (we remind again that $\ep<\kk^{-\kk}$) and so   $C_2$ can be expanded as a power series of $\ep$,  that there are solutions for $v_0,v_2$ so that $C_2$ has the above form (cf.~Remark \ref{remaFFFF}).
}%
%Further, the coefficient of $\ep^1$ in \eqref{MSMSMSMSMS}\,(ii) is
%$v-s=\frac{b_\kk-a_\kk-1}{b_\kk}\ge\frac{\d}{b_\kk}>0$ by %\eqref{F3rst}. \eqref{nVpooo}, \eqref{Akk-bkk} and
%Lemma \ref{NeMore-}.
%by comparing the coefficients of $\ep^1$ [note that the coefficients of $\ep^1$ in
%
%
%
%, we see that the strictly inequality holds since $v>-1$.
%Further the  coefficient of $\ep^1$ in $C_3$ is $v-4u-\frac52=\frac52>0$% by Lemma \ref{NeMore-} and \eqref{Akk-bkk}
%, i.e., we have  \eqref{MSMSMSMSMS}\,(ii).
$\!$Hence $(q_0,q_1)\in V_0$, i.e.,
%
$V_0\ne\emptyset,$
and we obtain a contradiction with Assumption \ref{assu1111}. This shows that Assumption \ref{assu1111} must be wrong, namely, we  have the lemma.\hfill$\Box$\vskip7pt
%

\NOUSE{
%Theorem \ref{AddLeeme--0}\,(iii) and \eqref{ClM0}.
%\vskip7pt
%
\NOUSE{Note that if we use the definition \eqref{-ANewf0F1++}
[with $(\tilde p_0,\tilde p_1)$ replaced by $(\bar p_0,\bar p_1)$, cf.~\eqref{TaKa}${\ssc\,}$]
and denote the matrix elements defined in
\eqref{-AA===} by $\tilde a,\tilde b,\tilde c,\tilde d$, then by comparing \eqref{Ne2wf0F1++} with
 \eqref{-ANewf0F1++}, we in fact have the following
\equa{akk-bkk}{a'=-\bar x_0^{-1}\tilde a\bar x_1,\ \ b=\bar x_0\tilde b\bar y_1.}
Since $(\bar p_0,\bar p_1)\in A_{1,\kk^{-\kk+n_1}}$, Theorem \ref{AddLeeme--0}\,(i) shows that
the height $h_{\bar p_0,\bar p_1}$ is bounded by some fixed number. In particular by \eqref{bBbbb}, \eqref{TaKa}
and \eqref{akk-bkk}, we obtain
\equa{a'-blesst}{|a'|<\kk^{-\kk+n_1}\SS_1,\ \ |b|<\SS_1\mbox{ for some fixed }\SS_1>0.}
Now we take fixed $\l_0\in\R_{>0}$ (i.e., $\l_0$ is independent of $\kk$) with \equa{aal0=0}{\mbox{$\l_0>a'+b+3$.}}
\begin{lemm}\label{NeMore-}For any $\l>1$ and $n_1\in\R$, whenever $\kk\gg1$, we have
\equa{WehHAH}{\g_{\kk^{\l_0\l},\kk^{-\kk+n_1}}\ge\TH \kk^{\l}.}
\end{lemm}\noindent{\it Proof.~}%Assume conversely that \eqref{WehHAH} is not true.
Let  $\kk\gg1$ %and $\d=\kk^{-1}$
(and we can assume $\ep<\kk^{-\kk^\kk}$, cf.~Remark \ref{u-vRema}). %Fix $\TH,\d,\d ,\kk\in\R_{>0}$ such that $\TH\l_0\l<1$, and $\d,\d $ are sufficiently small, and
%$\kk$ is sufficiently large.
We define $V_0$ to be the subset of $V$ consisting of elements $(p_0,p_1)=\big((x_0,y_0),(x_1,y_1)\big)$ satisfying
\begin{eqnarray}
\label{NSoOP?}&&\!\!\!\!\!\!\!\!\!\!\!\!\!\!\!\!\!\!\!\!\!\!\!\!\!\!\!\!
%\dis{\rm(i)\ }
\kk^{-\kk+n_1-1}\le|x_1|\le(1-\kk^{-3})\kk^{-\kk+n_1}|x_0|^{-\frac1{(\l_0-1)\l}}+\kk^{-\kk+n_1-3}\le
(1-\kk^{-2})|x_1|+\kk^{-\kk+n_1-2},
\\[6pt]
\label{NewAdded-NSoOP}&&\!\!\!\!\!\!\!\!\!\!\!\!\!\!\!\!\!\!\!\!\!\!\!\!\!\!\!\!
% \ \ {\rm(ii)\ }
|x_1|^\l\cdot|y_1|\ge\kk^{\l(-\kk+n_1)}\g_{1,\kk^{-\kk+n_1}}
(1+\ep^2).\end{eqnarray}
%where $\a_1=\a_2-k^{-\kk}$, $\a_2=k^{\TH\l_0\l}-k^{\TH\l_0\l(1-\d)}$.
If we rewrite the above as the form in \eqref{ToSayas}, then %$\mu=0$ and %$(t_1,t_2,t_3)=(0,1,0)$ and
we have \eqref{-EiathA}% and \eqref{EiathA}\,(c)
.
For any $(p_0,p_1)\in V_0$, if the first two equalities, or the first and last equalities, hold in \eqref{NSoOP}, then we can  obtain \equa{F1rst}{\mbox{$|x_1|=\kk^{-\kk+n_1-1}<\kk^{-\kk+n_1},$ \ \  $|x_0|\le\kk^{\l_0\l}$,}}
where the second inequation follows by noting from \eqref{NSoOP} that since $|x_0|$ has the negative power $-\frac1{(\l_0-1)\l}$, when $|x_1|$ is reduced by the factor $\kk^{-1}$, the $|x_0|$ can be at most multiplied by a factor $\kk^{(\l_0-1)\l}(1+\a)$ for some $\a=O(\kk^{-1})$.
By Theorem \ref{AddLeeme--0}\,(iii) and definition \eqref{Ak=1}, we obtain
\equa{F2rst}{\mbox{$\g_{\kk^{\l_0\l},\kk^{-\kk+n_1}}>\g_{|x_0|,|x_1|}\ge|y_1|>|x_1|^{-\l}\kk^{\l(-\kk+n_1)}\g_{1,\kk^{-\kk+n_1}}>\kk^\l$,}}
where the last two inequalities follow from \eqref{TaKa} and
\eqref{NewAdded-NSoOP}, i.e., we have \eqref{WehHAH} (and the proof is completed in this case).
%a contradiction with the assumption at the beginning of the proof.
%a contradiction with the definition of $\g_{1,1}$ and/or Theorem \ref{AddLeeme--0}\,(iii).
If the last two  equalities hold in \eqref{NSoOP}, then $|x_1|=\kk^{-\kk+n_1}$, $|x_0|=1$, but by
\eqref{NewAdded-NSoOP}, $|y_1|>\g_{1,\kk^{-\kk+n_1}}$, a contradiction with definition \eqref{Ak=1}.
%$|x_1|=\d^{-1}=\kk$, and $|x_0|=(\frac{\kk-\d+\d^2-\d^3}{1-\d^3})^{\l_0\l}<\kk^{\l_0\l}$, and by
%\eqref{NSoOP}\,(ii), $|y_1|>\g_{1,1}\kk^\l>\TH\kk^\l.$
%In particular, by Theorem \ref{AddLeeme--0}\,(iii), we have $\g_{\kk^{\l_0\l},\kk^{-\kk}}\ge\g_{|x_0|,|x_1|}\ge|y_1|>\kk^\l$,
%
Hence %Therefore %the last two inequalities cannot simultaneously hold in \eqref{NSoOP}\,(i), i.e.,
 Theorem \ref{real00-inj}\,(1)\,(ii) holds.
%
%
Next, we want to choose suitable $u,v$ such that
\eqref{NSoOP} holds for $(q_0,q_1)$ [defined in \eqref{1+++q0q1}${\ssc\,}$], i.e.,
\begin{eqnarray}
\label{1NSoOP}&&\!\!\!\!\!\!\!\!\!\!\!\!\!\!\!\!\!\!
%\dis{\rm(a)\, }
\kk^{-1} \le |1+u\ep| \le
(1-\kk^{-3})|1+s\ep|^{-\frac1{(\l_0-1)\l}}
+\kk^{-3}\le
(1-\kk^{-2})|1+u\ep|+\kk^{-2},
\\[5pt]
\label{MAMAM1NSoOP}&&\!\!\!\!\!\!\!\!\!\!\!\!\!\!\!\!\!\!
%   {\rm(b)\, }
|1+u\ep|^\l\cdot|1+v\ep|\ge1+\ep^2.\end{eqnarray}
The first strict inequality automatically holds in \eqref{1NSoOP}.
If we take
\equa{F3rst}{\mbox{$\dis u=-1,\ \ \ v=\frac{(\l_0-1)\l-a'}{b},$
\ \ and so $\dis -\frac{s}{(\l_0-1)\l}=-1+O(\ep^1)$ \ by \eqref{@suc2hthat=4},}}  by comparing the coefficients of $\ep^1$, we see that
all strict inequalities hold in \eqref{1NSoOP} and \eqref{MAMAM1NSoOP} [by noting that the coefficient of $\ep^1$ in the left hand-side of \eqref{MAMAM1NSoOP} is $\l u+v=\frac{(\l_0-1-b)\l-a'}{b}>0$ by \eqref{aal0=0} and the fact that $\l>1$]. This shows that $(q_0,q_1)\in V_0$, i.e., $V_0\ne\emptyset,$ and we obtain a contradiction with Assumption \ref{assu1111},
namely, %. Thus the assumption  at the beginning of the proof is wrong, i.e.,
we have Lemma \ref{NeMore-}.\hfill$\Box$\vskip7pt
%
Now for any $\l\in\R_{>2}$, $n_1\in\R$ and $\kk\gg1$, we take
\begin{eqnarray}
\label{Anooooo+1}&\!\!\!\!\!\!\!\!\!\!\!\!\!\!\!\!\!\!\!\!&
(\bar p_0,\bar p_1)=\big((\bar x_0,\bar y_0),(\bar x_1,\bar y_1)\big)\in V
\mbox{ with }\nonumber\\[4pt]
&\!\!\!\!\!\!\!\!\!\!\!\!\!\!\!\!\!\!\!\!&
|\bar x_0|=\kk^{\l_0\l},\ |\bar x_1|=\kk^{-\kk+n_1},\
|\bar y_1|=\g_{\kk^{\l_0\l},\kk^{-\kk+n_1}}\ge\TH \kk^\l.\end{eqnarray}
Define $F_0,F_1,G_0,G_1,\,(q_0,q_1)$ as in \eqref{Ne2wf0F1++} and \eqref{1+++q0q1}. Then we have \eqref{@suc2hthat=4}, which is
rewritten as
\equa{RwTw}{s=-a_\kk u+b_\kk v+O(\ep^1),}
to emphasis that $a_\kk,b_\kk$ may depend on $\kk$. The same proof as that  of Lemma \ref{anLemmA} shows that $a_\kk,b_\kk>0$.
Noting from \eqref{Anooooo+1} and  the proof of Theorem \ref{AddLeeme--0}\,(i), we
see that the height $h_{\bar p_0,\bar p_1}$ is bounded by $\kk^{\SS'_2}$ for some fixed $\SS'_2>0$. Thus similar to \eqref{a'-blesst} [cf.~\eqref{bBbbb}${\ssc\,}$], we have
\equa{a'-blesst1}{|a_\kk|<\kk^{-\kk+n_1+\SS_2}\SS_1,\ \ |b_\kk|<\kk^{\SS_2}\SS_1\mbox{ for some fixed }\SS_1,\SS_2>0.}
%From now on we fix  sufficiently small $\th\in\R_{>0}$. % such that $3\th<\frac1m$.
%We want to prove the following statement.
%\begin{lemm}\label{stat1}
%For any $\l_0,\l\in\R_{>0},\,n_1\in\R$ with $\l_0>1+3\th$ such that \eqref{WehHAH} holds, there
% exist $\l'_0,\l'\in\R_{>0},\,n_2\in\Z$, which are independent of $\kk$, %(except that $\l'$ may depend on $\l$ and $\l'_0$ may depend on $\l_0$)
% such that
% \equa{KAKAInde}{\l'_0<\l_0(1-\th^3),\ \ \l'>\l\mbox{ \ and }\g_{\kk^{\l'_0\l'},\kk^{-\kk+n_1+n_2}}>\kk^{\l'}\mbox{ \ when }\kk\gg1.}
%\end{lemm}
}%
\begin{lemm}\label{An+NeMore}
%Assume  \eqref{KAKAInde} does not hold. Then %for any sufficiently small ${\th^2} \in\R_{>0}$, we have
%\equa{TheTT}{\mbox{$\dis (1+{\th^2})b_\kk\ge\l_0+\frac {a_\kk}{\l}$ for all $\kk\gg1$.}}
Fix any $\d>0$ with $\d<\frac1m.$ We have $b_\kk\ge1+\d+\a_\kk$ $($for all $\kk>0{\ssc\,})$.
\end{lemm}\noindent{\it Proof.~}
%We regard $\kk$ as fixed.
Assume conversely $b_\kk<1+\a_\kk$. Choose $v_0>0$ (which may depend on $\kk$) such that
\equa{SUUUSUS}{b_\kk<1+\d-v_0+a_\kk.}
Take $\ell\gg\kk$ (and we may assume $\ep<\ell^{-\ell}$, cf.~Remark \ref{u-vRema}).
%
%Assume conversely $(1+{\th^2})b_\kk<\l_0+\frac {a_\kk}{\l}$. Then we can %Take $\d=\frac{{\th^2} }{1-{\th^2} }$, and
%choose $v_0>0$ (which can depend on $\kk$) such that
%\equa{d-th0}{\l(1+{\th^2})b_\kk<\l_0\l(1-v_0)+a_\kk.}
%Then we can choose some sufficiently small ${\th^2} >0$ (which may depend on $\kk$) such that
%%\equa{AnSuchThat}{-a_\kk+b_\kk(1+2{\th^2} )<1.}
%%We regard $\kk$ as fixed and take $\ell\gg\kk$, ${\th^2}=\ell^{-1}$ (and we can assume $\ep<\ell^{-\ell}$, cf.~Remark \ref{u-vRema}).
%Take $\ell\in\Z_{>0}$ to be sufficiently large (which can depend on $\kk$) such that
%\equa{ssssssssssssssssssss}{\kk^{-\ell}<v_0.}
Define the subset $V_0$ of $V$ consisting of elements $(p_0,p_1)=\big((x_0,y_0),(x_1,y_1)\big)$ satisfying
\begin{eqnarray}
\label{An0FL1}&&\!\!\!\!\!\!\!\!\!\!\!\!\!\!\!\!\!\!\!\!\!\!\!\!\!\!\!\!\!\!\!
\kk\le|x_1|\le(1-\ell^{-3})\kk^{\frac{\d}{1+\d}}|x_0|^{\frac1{1+\d}}+\kk(\ell^{-1}-\ell^{-2}+\ell^{-3})
%\nonumber\\[4pt]
%&&\!\!\!\!\!\!\!\!\!\!\!\!\!\!\!\!\!\!\!\!\!\!\!\!\!\!\!\!\!\!\!
%\phantom{\kk^{-\kk+n_1}}
\le
(1-\ell^{-2})|x_1|+\kk\ell^{-1},\
\\[8pt] \label{An?0FL1}&&\!\!\!\!\!\!\!\!\!\!\!\!\!\!\!\!\!\!\!\!\!\!\!\!\!\!\!\!\!\!\!
%\ {\rm(ii)\ }
\frac{|y_1|}{|x_1|}\ge\frac{\g_{\kk,\kk}}{\kk}(1+\ep^2).
\end{eqnarray}
Again we have \eqref{ToSayas}.
For any $(p_0,p_1)\in V_0$,
if the first two equalities, or the first and last equalities,  hold in \eqref{An0FL1},
then $|x_1|=\kk$ and $|x_0|\le\kk$, and by  \eqref{An?0FL1}, $|y_1|>\g_{\kk,\kk}$,
a contradiction
 with definition \eqref{Ak=1} and/or Theorem \ref{AddLeeme--0}\,(iii).
%As before, we obtain a contradiction.
If the last two inequalities hold in
\eqref{An0FL1}, then
$|x_1|=\kk\ell$ and $|x_0|\sim\ell^{1+\d}$ (when $\ell\gg\kk$ and $\kk$ is regarded fixed), but by \eqref{An?0FL1},
$|y_1|=\a\kk\ell$ with $\a>\frac{\g_{\kk,\kk}}{\kk}>1$. By \eqref{mqp1234-2+}, we must have
$h_{p_0,p_1}\sim\ell^{1+\d}$, but then $|x_1+y_1|\ge|y_1|-|x_1|=(\a-1)\ell\sim\ell\succ h_{_{\sc p_0,p_1}}^{^{\sc\frac{m}{m+1}}}$ (since $\d<\frac1m$), a contradiction with
\eqref{mqp1234-2}.
 Thus
%a contradiction as before.
Theorem \ref{real00-inj}\,(1)\,(ii) holds.
%
Now we want to choose suitable $u,v$ such that
\eqref{An0FL1} holds for $(q_0,q_1)$ [defined in \eqref{1+++q0q1}${\ssc\,}$], i.e.,
\begin{eqnarray}
\label{An+0FL1}&&\!\!\!\!\!\!\!\!\!\!\!\!\!\!\!
\dis{\rm(i)\ }
1\le|1+u\ep|\le(1-\ell^{-3})|1+s\ep|^{\frac{1}{1+\d}}+\ell^{-1}-\ell^{-2}+\ell^{-3}\nonumber
\le
(1-\ell^{-2})|1+u\ep|+\ell^{-1},
\\[6pt]
\label{AAAAAAAn+0FL1}&&\!\!\!\!\!\!\!\!\!\!\!\!\!\!
  {\rm(ii)\ }
\frac{|1+v\ep|}{|1+u\ep|}\ge1+\ep^{2}.\end{eqnarray}
The second strict inequality automatically holds in \eqref{An+0FL1}\,(i) (when $\ell\gg\kk$). %(we may assume $\ep<\min\{\kk^{-\kk^\kk},v_0^{\kk}\}$, cf.~Remark \ref{u-vRema}).
Take \equa{S2ncdos}{\mbox{$\dis u=1,\ v=\frac{1+\d-v_0+a_\kk}{b_\kk}$% with $v_0>0$
, \  and so $s=1+\d-v_0+O(\ep^1)$ by \eqref{@suc2hthat=4}.}}
%then  by \eqref{RwTw} and \eqref{ssssssssssssssssssss}, \equa{sooooaaa}{\mbox{$\dis(1-\kk^{-3-\ell})\frac{s}{\l_0\l}=(1-\kk^{-3-\ell})(1-v_0)+O(\ep^1)<1-\kk^{-2-\ell}$.}}
Thus \eqref{An+0FL1}\,(i) holds by comparing the coefficients of $\ep^1$.
Further, the coefficient of $\ep^1$ in the left-hand side of \eqref{An+0FL1}\,(ii) is
$v-u=\frac{1+\d-v_0+a_\kk}{b_\kk}-1>0$ by \eqref{SUUUSUS}.
%
%\equa{vaaaaa}{\mbox{$\dis v-\l(1+{\th^2})u=
%\frac{\l_0\l(1-v_0)+a_\kk-\l(1+{\th^2})b_\kk}{b_\kk}>0$,}}
%where the inequality follows from  \eqref{d-th0}.
 %, is positive when $v_0=0$, thus also
%positive when $v_0>0$ is sufficiently small (note that $v_0$ can depend on $\kk$ since when we consider the local bijectivity of Keller maps, we regard $\kk$ as fixed).
This shows that $(q_0,q_1)\in V_0$, i.e., $V_0\ne\emptyset,$ and we obtain a contradiction with Assumption \ref{assu1111}. This proves the lemma.
\hfill$\Box$
%
%
\NOUSE{\begin{lemm}\label{An+NeMore00}
Assume  \eqref{KAKAInde} does not hold. Then %for any sufficiently small ${\th^2} \in\R_{>0}$, we have
\equa{TheTT00}{\mbox{$\dis (1-{\th^2})b_\kk\le\l_0+\frac {a_\kk}{\l}$ for all $\kk\gg1$.}}
\end{lemm}\noindent{\it Proof.~}%Assume conversely $b_\kk<(1-{\th^2}^3 )(\l_0+\frac {a_\kk}{\l})$. Take $\d=\frac{{\th^2}^3 }{1-{\th^2}^3 }$, and choose $v_0>0$ (which can depend on $\kk$) such that
%\equa{d-th0}{\l(1-\d)b_\kk<\l_0\l(1-v_0)+a_\kk.}
%Then we can choose some sufficiently small ${\th^2} >0$ (which may depend on $\kk$) such that
%\equa{AnSuchThat}{-a_\kk+b_\kk(1+2{\th^2} )<1.}
%We regard $\kk$ as fixed and take $\ell\gg\kk$, $\d=\ell^{-1}$ (and we can assume $\ep<\ell^{-\ell}$, cf.~Remark \ref{u-vRema}).
%Take $\ell\in\Z_{>0}$ to be sufficiently large (which can depend on $\kk$) such that
%\equa{ssssssssssssssssssss}{\kk^{-\ell}<v_0.}
aaaaaaaaaaaaaaaWe define the subset $V_0$ of $V$ consisting of elements $(p_0,p_1)=\big((x_0,y_0),(x_1,y_1)\big)$ satisfying
\begin{eqnarray}
\label{An0FL100}&&\!\!\!\!\!\!\!\!\!\!\!\!\!\!\!\!\!\!\!\!\!\!\!\!\!\!\!\!
\kk^{-\kk+n_1-1}\le|x_1|\le(1-\kk^{-3})\kk^{-\kk+n_1-1}|x_0|^{\frac1{\l_0\l}}+
\kk^{-\kk+n_1-3}
%\nonumber\\[4pt]
%&&\!\!\!\!\!\!\!\!\!\!\!\!\!\!\!\!\!\!\!\!\!\!\!\!\!\!\!\!\!\!\!
%\phantom{\kk^{-\kk+n_1-1}}
\le
(1-\kk^{-2})|x_1|+\kk^{-\kk+n_1-2},\
\\[8pt] \label{An?0FL100}&&\!\!\!\!\!\!\!\!\!\!\!\!\!\!\!\!\!\!\!\!\!\!\!\!\!\!\!\!
%\ {\rm(ii)\ }
\frac{|y_1|}{|x_1|^{(1-{\th^2})\l}}\ge\frac{\g_{\kk^{\l_0\l},\kk^{-\kk+n_1}}}{\kk^{(-\kk+n_1)(1-{\th^2})\l}}(1+\ep^2).
\end{eqnarray}
Again we have \eqref{ToSayas}.
For any $(p_0,p_1)\in V_0$,
if the last two equalities hold in \eqref{An0FL100}, then $|x_1|=\kk^{-\kk+n_1},\,|x_0|=\kk^{\l_0\l}$, but by \eqref{An?0FL100}, $|y_1|>\g_{\kk^{\l_0\l},\kk^{-\kk+n_1}}$,
a contradiction with definition \eqref{Ak=1}.
If the first two equalities, or the first and last equalities,  hold in \eqref{An0FL100},
then $|x_1|=\kk^{-\kk+n_1}$ and $|x_0|\sim1$, and by  \eqref{An?0FL100} and \eqref{WehHAH}, $|y_1|>\kk^{{\th^2}\l}\succ|x_0|\succ|x_1|$,
and as above we can get a contradiction
 with \eqref{mqp1234-2}. %definition \eqref{Ak=1} and/or Theorem \ref{AddLeeme--0}\,(iii).
%As before, we obtain a contradiction.
%If the last two inequalities hold in
%\eqref{An0FL1}, then
%$|x_1|=\kk^{-\kk+n_1+1}$ and $|x_0|\sim\kk^{2\l_0\l}\prec\kk^{2\l_0\l(1+\d^2)}$
%[by noting from \eqref{An0FL1} that $|x_0|$ has the power ${\frac1{\l_0\l}}$, cf.~\eqref{MSM1111} for notation $\sim$ and $\prec\,$],
%i.e.,
%\equa{Uaaaa00}{\mbox{$\dis|x_1|=\kk^{-\kk+n_1+1},\,|x_0|<\kk^{\l_0'\l'}$, where
%$\dis\l_0'=\l_0\frac{1+\d^2}{1+\frac{\d}{2}}<\l_0$, $\l'=2\l(1+\frac{\d}{2})>\l$,}}
%and by Theorem \ref{AddLeeme--0}\,(iii), \eqref{WehHAH} and \eqref{An?0FL1}, we have
%\equa{Uaaaa1}{\mbox{$\g_{\kk^{\l_0'\l'},\kk^{-\kk+n_1+1}}\ge\g_{|x_0|,|x_1|}\ge|y_1|>
%\g_{\kk^{\l_0\l},\kk^{-\kk+n_1}}\kk^{(1+\d)\l}\ge\kk^{\l+(1+\d)\l}=\kk^{\l'}$.}}
%This together with \eqref{Uaaaa} contradicts the assumption that \eqref{KAKAInde} does not hold.
Thus
%a contradiction as before.
Theorem \ref{real00-inj}\,(1)\,(ii) holds.
%
Now we want to choose suitable $u,v$ such that
\eqref{An0FL1} holds for $(q_0,q_1)$ [defined in \eqref{1+++q0q1}${\ssc\,}$], i.e.,
\begin{eqnarray}
\label{An+0FL100}&&\!\!\!\!\!\!\!\!\!\!\!\!\!\!\!
%\dis{\rm(i)\, }
\kk^{-1}\le|1+u\ep|\le(1-\kk^{-3})|1+s\ep|^{\frac1{\l_0\l}}+\kk^{-3}
%\nonumber
%\\
%&&\!\!\!\!\!\!\!\!\!\!\!\!\!\!\!
%\phantom{1}
\le
(1-\kk^{-2})|1+u\ep|+\kk^{-2},
\\[6pt]%  {\rm(ii)\, }
\label{AAAAAAAn+0FL100}&&\!\!\!\!\!\!\!\!\!\!\!\!\!\!
\frac{|1+v\ep|}{|1+u\ep|^{\l(1-\th^2)}}\ge1+\ep^{2}.\end{eqnarray}
The first strict inequality automatically holds in \eqref{An+0FL1} (we may assume $\ep<\kk^{-\kk^\kk}$, cf.~Remark \ref{u-vRema}).
Take \equa{S2ncdos00}{\mbox{$\dis u=-1,\ \ \ v=-\frac{\l_0\l+a_\kk}{b_\kk}$ % with $v_0>0$
and so $\dis\frac{s}{\l_0\l}=-1+O(\ep^1)$ by \eqref{RwTw}.}}
Thus \eqref{An+0FL100} holds by comparing the coefficients of $\ep^1$.
Further, the coefficient of $\ep^1$ in the left-hand side of \eqref{AAAAAAAn+0FL100} is
$v-\l(1-\th^2)u=
\frac{\l_0\l+a_\kk-\l(1-\th^2)b_\kk}{b_\kk}$, which is positive if \eqref{TheTT00} does not hold, and in this case
%where the inequality follows from  \eqref{d-th0}.
 %, is positive when $v_0=0$, thus also
%positive when $v_0>0$ is sufficiently small (note that $v_0$ can depend on $\kk$ since when we consider the local bijectivity of Keller maps, we regard $\kk$ as fixed).
we have that $(q_0,q_1)\in V_0$, i.e., $V_0\ne\emptyset,$ and we obtain a contradiction with Assumption \ref{assu1111}. This proves the lemma.
\hfill$\Box$
 \begin{lemm}\label{+NeMore}The inequations \eqref{KAKAInde} hold.
 %
% Fixed any $\d >0$ with $\d <\frac1{m}$. We have $b_\kk\ge1+a_\kk+\d $ for all $\kk>0$.
\end{lemm}\noindent{\it Proof.~}By the previous two lemmas, we can assume
%Lemma \ref{An+NeMore}, we may assume that \eqref{TheTT} holds.
%Take $s_0=1-{\th^2}+\frac{1-{\th^2}^3}{(1-{\th^2})\l_0}$. Then
\equa{AAAs0===}{\dis(1-\th^2)b_\kk\le\l_0+\frac{a_\kk}{\l}\le(1+\th^2)b_\kk,
\mbox{ and $|a_\kk|\to0$ (when $\kk\to\infty$) by \eqref{a'-blesst1}}
%1-{\th^2}+\frac{1+a_\kk-{\th^2}}{b_\kk}<s_0%=1-{\th^2}+\frac{1+a_\kk-{\th^2}}{b_\kk}+{\th^2}^3
%<1-{\th^2}+\frac1{(1-{\th^2}^2)\l_0}+{\th^2}^3<1+\frac{1}{\l_0}
.}
\NOUSE{where the first inequality follows from \eqref{TheTT} and the first inequation of \eqref{a'-blesst1}, while the last strict inequality follows from the fact that $\l_0>1+3{\th^2}$ (by noting that if we set $\l_0$ to be $1+3\th$ then the strict inequality holds, thus also holds when $\l_0>1+3\th$). We take
\begin{eqnarray}
\label{AlESS}&&\!\!\!\!\!\!\!\!\!\!\!\!\!\!\!\!\!\!\!\!\!
\dis \rho :=\frac{\l_0\l\th^2}{1-(1-\th^2)\l_0(s_0-1)}>0,
\end{eqnarray}
where the inequality in \eqref{AlESS} follows from the fact that $\l_0(s_0-1)<1$ by \eqref{s0===}. Let
\begin{eqnarray}
\label{1AlESS}&&\!\!\!\!\!\!\!\!\!\!\!\!\!\!\!\!\!\!\!\!\!
\l':=\l-\rho (s_0-1)=\frac{\l\Big(1-(1-\th^2)\l_0(s_0-1)-\l_0\th^2(s_0-1)\Big)}{1-(1-\th^2)\l_0(s_0-1)}\nonumber\\
&&\!\!\!\!\!\!\!\!\!\!\!\!\!\!\!\!\!\!\!\!\!
\phantom{\l':}=
\frac{\l\Big(1-\l_0(s_0-1)\Big)}{1-(1-\th^2)\l_0(s_0-1)}
>\l,
\end{eqnarray}
where the inequality  follows from the fact that both
the numerator and the denominator are positive and that $0<1-\th^2<1$. Set
\begin{eqnarray}
\label{2AlESS}
&&\!\!\!\!\!\!\!\!\!\!\!\!\!\!\!\l'_0:=(\l_0\l-\rho)\cdot\frac1{\l'}
=\frac{\l_0\l\Big(1-(1-\th^2)\l_0(s_0-1)-\th^2\Big)}{1-(1-\th^2)\l_0(s_0-1)}\cdot
\frac{1-(1-\th^2)\l_0(s_0-1)}{\l\Big(1-\l_0(s_0-1)\Big)}\nonumber
\\
&&\!\!\!\!\!\!\!\!\!\!\!\!\!\!\!\phantom{\l'_0:}
=
(1-\th^2)\l_0.
\end{eqnarray}
Then $\l'_0,\l'$ are independent of $n_1,\kk$.
%Assume conversely $b_\kk<1+a_\kk+\d $. Then we can choose some sufficiently small $\th >0$ (which may depend on $\kk$) such that
%\equa{SuchThat}{-a_\kk+b_\kk(1+2\th )<1+\d \mbox{ \ and \ }\th <\d .}
%We regard $\kk$ as fixed and take $\ell\gg\kk^\kk$ %, $\d=\ell^{-1}$
%(and we can assume $\ep<\ell^{-\ell}$, cf.~Remark \ref{u-vRema}).
%
}%
%
}
\begin{lemm}\label{Nowl0===}%If $b<a'+1$, then
Theorem $\ref{real00-inj}\,(1)$ holds.
\end{lemm}\noindent{\it Proof.~}%Fix a sufficiently $\eta>0$.
Let $\kk\gg1$ and $\d =\frac{\d}{2}$ with $\d$ being as above. Let $\ell\gg\kk$ (and we may assume $\ep<\ell^{-\ell}$, cf.~Remark \ref{u-vRema}).
We define the subset $V_0$ of $V$ consisting of elements $(p_0,p_1)=\big((x_0,y_0),(x_1,y_1)\big)$ satisfying
[cf.~\eqref{TaKa}${\ssc\,}$]\begin{eqnarray}
\label{0FL1}&&\!\!\!\!\!\!\!\!\!\!\!\!\!\!\!\!\!\!\!\!\!\!\!\!\!\!\!\!\!\!\!
{\rm(i)\ }(1-\d )(1+\d )\kk\le|x_1+\bar x_1\d |\le
(1+\d )|x_0|
%\nonumber\\[4pt]
%&&\!\!\!\!\!\!\!\!\!\!\!\!\!\!\!\!\!\!\!\!\!\!\!\!\!\!\!\!\!\!\!
%\phantom{(1-\d )(1+\d )\kk}
\le(1-\kk^{-1})|x_1|+\d \kk-1\le\ell,
\nonumber\\[6pt] %\label{?0FL1}
&&\!\!\!\!\!\!\!\!\!\!\!\!\!\!\!\!\!\!\!\!\!\!\!\!\!\!\!\!\!\!\!
{\rm(ii)\ }
\frac{|x_1|\cdot|y_1|^{1-\kk^{-2}}+\ep^3}{|x_0|^2}\ge\frac{\kk\g_{\kk,\kk}^{1-\kk^{-2}}+\ep^3}{\kk}(1+\ep^2).
\end{eqnarray}
%where  $\eta>1$ will be determined later.
%If we rewrite the above as the form in \eqref{ToSayas}, ????????????then %$\mu=0$ and %$(t_1,t_2,t_3)=(0,1,0)$ and
Then we have \eqref{ToSayas0}. %we have \eqref{-EiathA}.
%First we remark that
%\equa{ReaThat}{\dis|3\bar x_1^{-1}\!-\!x_1^{-1}|\ge3\kk^{\kk-n_1}\!-\!|x_1|^{-1},
%\mbox{ \ i.e., \ }|x_1|\le\frac{1}{3\kk^{\kk-n_1}-|3\bar x_1^{-1}\!-\!x_1^{-1}|}.}
For any $(p_0,p_1)\in V_0$, %as in the previous lemma,
if the last two equalities, or the first and third equalities, hold in \eqref{0FL1}, then
%$|3\bar x_1^{-1}-x_1^{-1}|=2\kk^{\kk-n_1}$ [and so $|x_1|\le\kk^{-\kk+n_1}$ by \eqref{ReaThat}${\ssc\,}$], and
%
$|x_0|=\kk^{\l_0\l}$, but by
\eqref{0FL1}\,(ii), $|y_1|>\g_{\kk^{\l_0\l},\kk^{-\kk+n_1}}$, a contradiction with definition \eqref{Ak=1} and/or Theorem \ref{AddLeeme--0}\,(iii).
If the first two equalities, or the first and last equalities, hold in \eqref{0FL1}, then
$%\begin{eqnarray}
%\label{W01929293}&\!\!\!\!\!\!\!\!\!\!\!\!\!\!\!\!\!\!\!\!\!\!\!\!&
|x_1|\le\frac{\kk^{-\kk+n_1}}{3-2\kk^{-\l_0\l}}<\kk^{-\kk+n_1+1},\, |x_0|\sim1,$ %\end{eqnarray}
but Theorem \ref{AddLeeme--0}\,(iii) and \eqref{0FL1}\,(ii), we have
$|y_1|\ge\kk^{\l-1}\succ|x_0|\succ|x_1|$. As in the proof of the previous lemma, we obtain a contradiction.
 Thus Theorem \ref{real00-inj}\,(1)\,(ii) holds.
%
Now we want to choose suitable $u,v$ such that
\eqref{0FL1} holds for $(q_0,q_1)$ [defined in \eqref{1+++q0q1}${\ssc\,}$], i.e.,
\begin{eqnarray}
\label{+0FL1}&&\!\!\!\!\!\!\!\!\!\!\!\!\!\!\!\!\!\!\!\!\!\!\!\!\!\!\!\!\!\!\!
\kk^{-\frac{\l_0\l}{2}}\le\frac{|3-(1+u\ep)^{-1}|}{2}\le
(1\!-\!\kk^{-\kk+n_1-3})|1+s\ep|+\kk^{-\kk+n_1-3}
\nonumber\\[4pt]
&&\!\!\!\!\!\!\!\!\!\!\!\!\!\!\!\!\!\!\!\!\!\!\!\!\!\!\!\!\!\!\!
\phantom{\kk^{-\frac{\l_0\l}{2}}}\le\frac{1\!-\!\kk^{-\kk+n_1-2}}{2}\Big|3\!-\!(1+u\ep)^{-1}\Big|+\frac{\kk^{-\kk+n_1-2}}{2},
\\[6pt] \label{+?0FL1}&&\!\!\!\!\!\!\!\!\!\!\!\!\!\!\!\!\!\!\!\!\!\!\!\!\!\!\!\!\!\!\!
%\ {\rm(ii)\ }
|1+(u+v)\ep+uv\ep^2|\ge1+\ep^2.
\end{eqnarray}
The first strict inequality automatically holds in \eqref{+0FL1}\,(i).
Take \equa{s3333}{\mbox{$\dis u=1,\ \ \ v=-\frac{a_\kk+\frac{1}{2}}{b_\kk}$, \ \ then $\dis s=-\frac{1}{2}+O(\ep^1)>-1$ by \eqref{RwTw}.}}
By comparing the coefficients of $\ep^1$, we see that  \eqref{+0FL1}\,(i) holds.
Further, the coefficient of $\ep^1$ in the left-hand side of \eqref{+0FL1}\,(ii) is
$1-\frac{a_\kk+\frac{1}{2}}{b_\kk}>0$ by Lemma \ref{An+NeMore}.
This shows that $(q_0,q_1)\in V_0$, i.e., $V_0\ne\emptyset,$ and we obtain a contradiction with Assumption \ref{assu1111}. This shows that
Assumption \ref{assu1111} is wrong, i.e., we have the lemma.
\hfill$\Box$\vskip7pt
\NOUSE{\noindent{\it Proof.~}We start with fixed $\l_0,\l,n_1$ such that \eqref{WehHAH} holds. Apply \eqref{KAKAInde} repeatedly (only a finite times) until we reach some $\l_0,\l,n_1$ with $\l_0\le1+3\th<1+\frac1m$.
Now let $(\bar p_0,\bar p_1)$ be as in \eqref{Anooooo+1}. Then \eqref{mqp1234-2} shows that $h_{\bar p_0,\bar p_1}\sim|x_0|\sim|y_0|$ or
$h_{\bar p_0,\bar p_1}\sim|x_0|\sim|y_1|$ (when $\kk\gg1$), in any case (using the fact that $\l_0<1+\frac1m$) we have $|x_1+y_1|\succeq|y_1|\succ\tau h_{_{\sc\bar p_0,\bar p_1}}^{^{\sc \frac{m}{m+1}}}$, a contradiction
with \eqref{mqp1234-1}. This shows that Assumption \ref{assu1111} is wrong, i.e., we have Theorem $\ref{real00-inj}\,(1)$.\hfill$\Box$\vskip7pt
}}}


\section{Proof of Theorem \ref{real00-inj}\,(2)}

\noindent{\it Proof of Theorem \ref{real00-inj}\,(2).~}Now we prove Theorem {\rm\ref{real00-inj}}\,(2).
Let $(p_0,p_1)=\big((x_0,y_0),(x_1,y_1)\big)\in  V_0$, i.e.,  \eqref{ToSayas} or \eqref{ToSayas0} [cf.~\eqref{LetNSoOP}${\ssc\,}$]
holds.
Note that \eqref{ToSayas}--\eqref{-EiathA0}
imply that  $x_0,x_1,y_1\ne0$. % [cf.~\eqref{WEhHaaa} for case \eqref{ToSayas0}${\ssc\,}$].
% [note that for the case \eqref{ToSayas0}, we in fact have \eqref{LetNSoOP}, from this we can see $y_1$ cannot be zero]
%
%
%We need to choose $(q_0,q_1)$ to satisfy
%\eqref{MUST115}.
Similar to \eqref{-ANewf0F1++} and \eqref{Ne2wf0F1++}, we define
\equa{0000Ne2wf0F1++}{F_0= F\big(x_0(1+x),y_0+y\big),\
\ \ F_1= F\big(x_1(1+x),y_1(1+y)\big),
}
%where $\kappa_0=1$ if $y_1=0$ or $\kappa_0=y_1$ else,
and define $ G_0, G_1$ similarly% [thus the matrices $A_0,A$ defined after
. %\eqref{-Aa0b0} now have determinant ${\rm det\,}A_0= x_0 J(F,G)\ne0$, ${\rm det\,}A= x_1\kappa_0 J(F,G)\ne0$, and again by replacing $( F_i, G_i)$ by $( F_i, G_i)A_0^{-1}$ for $i=0,1$, we can assume $A_0=I_2$].
Define $q_0,q_1$ accordingly [similar to \eqref{q0q1} and \eqref{1+++q0q1}${\ssc\,}$],
% we define
%\equa{Ne2??wf0F1++}{F_0= F\big(x_0(1+x),y_0+y\big),\
%\ \ F_1= F\big(x_1(1+x),y_1+\kappa_0y\big),
%}
%and define $ G_0, G_1$ similarly.
%Define $q_0,q_1$ accordingly
%[note a slightly different from \eqref{q0q1}${\ssc\,}$]
\equa{1++??+q0q1}{
q_0:=(\dot x_0,\dot y_0)=\big(x_0(1+ s\ep), y_0+t\ep\big),\ \ q_1:=(\dot x_1, \dot y_1)=
\big(x_1(1
+u
\ep), y_1(1+v\ep)\big).}
As in \eqref{st=} and \eqref{@suc2hthat=4}, we have, for some $\a_i\in\C$,
\equa{s1t=}{s=s_0+O(\ep^2
), \ \ \ s_0= au+ bv+(\a_1u^2+\a_2uv+\a_3v^2)\ep.
}

\begin{rema}\rm\label{FaMREE}
We remark that the $\ep$ here shall be regarded to be different from that in the previous results, here $\ep$ may be much smaller than the previous $\ep$. If we denote the previous $\ep$ as $\ep_1$, whenever necessary we can assume our new $\ep$ satisfies that $\ep<\kk^{-\kk},\ell^{-\ell},\ep_1^{\scep_1^{-1}}$, where $\kk,\ell$ are as before (cf.~Remark \ref{u-vRema}).
\end{rema}

Now we consider two cases.\vskip5pt
\noindent{\bf Case 1}: {\it
Assume we have \eqref{ToSayas}}.
If no equality (resp.,  the first equality) holds in \eqref{ToSayas}\,(a), we only need to consider
\eqref{MUST115}\,(b) [resp., \eqref{MUST115}\,(b) and the first inequality of \eqref{MUST115}\,(a)], which can be easily done. Thus by Theorem \ref{real00-inj}\,(1)\,(ii), we may assume that the last equality of \eqref{ToSayas}\,(a) holds [the proof for the case with the second equality of \eqref{ToSayas}\,(a) is exactly similar],
 i.e.,
\equa{ISISIS}{\dis\kappa_1|x_0|^{\kappa_2}+\kappa_3=\kappa_4|x_1|+\kappa_5.}
%\equa{LastHH}{|x_0|
%=\kappa_1'|x_1|
%+\kappa_2' \mbox{ with $\dis\kappa_1'=\kappa_3\kappa_1^{-1}>0,\ \kappa_2=(\kappa_4-\kappa_2)\kappa_1^{-1}>0$.}}
%%Note that $x_1+\l\ne0$ since $|x_0|\ge1$ and $|\l|<1$.
%%
%%
%%Then \eqref{ToSayas} implies
%%\equa{ImSTTTT}{\mbox{$\d \le|x_1|\le1$, \ $\dis1\le|x_0|\le\Big(\frac{1-\d ^2}{\d -\d ^2}\Big)^{\frac1{\th }}$, \ $|y_1|\ge\d_1^{\frac1{|x_1|}}\ge4$.}}
then %In our case here,
we need to choose $u,v$ so that $(q_0,q_1)$ satisfies \eqref{MUST115}\,(b) and the last inequality of \eqref{MUST115}\,(a), namely
%[here we assume $y_1\ne0$ (thus $\kappa_0=y_1$); if $y_1=0$ then the term $|1+v\ep|^{\frac12}$ in $C_2$ defined below
%should be replaced by $|v\ep|^{\frac12}$ (which contributes a positive $O(\ep^{\frac12})$ element to
%$C_2$ as long as $v\ne0$), thus \eqref{@such111that=2} holds for any nonzero $v$ and the proof becomes easier],
\begin{eqnarray}
\label{@such111that=2}&\!\!\!\!\!\!\!\!\!\!\!\!\!\!\!\!\!\!\!\!\!\!\!\!&
{\rm(i)\ }C_1:=|1+v\ep|\cdot
|1+u\ep|^{-\kappa_6}-1>0,
\nonumber\\[6pt] %\label{@suchthat=3}
&\!\!\!\!\!\!\!\!\!\!\!\!\!\!\!\!\!\!\!\!\!\!\!\!&
{\rm(ii)\ }C_2:=-|1+s\ep|^{\kappa_2}+\kappa_4'|1+u\ep|+\kappa'_5\ge0
,\end{eqnarray}
where \equa{ka4p}{\mbox{$\dis\kappa_4'=\frac{\kappa_4|x_1|}{\kappa_1|x_0|^{\kappa_2}}>0$, $\dis\kappa_5'=
\frac{\kappa_5-\kappa_3}{|x_0|^{\kappa_2}}>0$ by \eqref{-EiathA}, \
thus $\dis\kappa_4'+\kappa_5'-1=0$ by \eqref{ISISIS}.}} %We claim that
%\equa{kakak}{\kappa_4>0\mbox{ \ (thus $\kappa_3>1$)},}
%otherwise we would have $|y_1|\ge(\kk^{\kk}|x_0|)^2\ge\kk^{2\kk}$
%by condition \eqref{ToSayas} [cf.~\eqref{NSoOP}${\ssc\,}$], and as in the proof of Lemma \ref{NeMore-}, we would obtain a contradiction.
%
%
%If all strict inequalities hold in \eqref{ToSayas}\,(a), then \eqref{@suchthat=2} holds automatically, and we only need to consider \eqref{@suchthat=3}, such a case is easier. Thus we assume that at least one [and by Theorem {\rm\ref{real00-inj}}\,(1)\,(ii), only one] equality holds in  condition \eqref{ToSayas}\,(a). Say (the proofs for  other cases are similar, cf.~Remark \ref{FinalRem}),
%\equa{saysome}{|x_1|=(1-\d ^2)|x_0|^{-\d }+\d ^2.}
%Then we only need to choose $u,v$ to satisfy \eqref{@suchthat=3} and the following (cf.~Remark \ref{MARK1}),
%\equa{saysome+}{C_1:=(1-\d ^2)|x_0|^{-\d }\cdot|1+s\ep|^{-\d }+\d ^2-|x_1|\cdot|1+u\ep|>0.}
%Setting \equa{StTV=}{v=\big(-u+2^{-1}u^2\ep|x_1|\ln\kappa_5\big)|x_1|\ln\kappa_5+w\ep,}
%and regarding $w$ as a new variable,
%First assume $b\ne0$.
%Using \eqref{s1t=}, if we
Set, for  any fixed $w\in\C$ with $w\re>0$,  \begin{eqnarray}
\label{SetV}&\!\!\!\!\!\!\!\!\!\!\!\!\!\!\!\!&
\mbox{$\dis v=\kappa_6 u+\frac12\kappa_6(\kappa_6-1)u^2\ep+w\ep$,% where
}
%\\
%\label{SetV1}&\!\!\!\!\!\!\!\!\!\!\!\!\!\!\!\!&
%\kappa'_1=\frac{\kappa_2(\kappa_2+1)}{2b}-\frac{\a_1}{b}+
%\frac{(a+\kappa_2)\a_2}{b^2}-\frac{(a+\kappa_2)^2\a_3}{b^3},
\end{eqnarray}
then one can easily see that
$C_1=|1+w\ep^2|-1+O(\ep^3)=w\re\ep^2+O(\ep^3)>0$, i.e.,
\eqref{@such111that=2}\,(i) holds.
%we can then rewrite \eqref{@suchthat=3} as (by computing the coefficient of $\ep^1$ in $C_0$)
%\begin{eqnarray}
%\label{y1==2}&\!\!\!\!\!\!\!\!\!\!\!\!\!\!&C_0\!=\!\kappa_5^{|x_1|\cdot|1+u\ep|-|x_1|}|1\!+\!v\ep|\!=\!|1\!+\!w\ep^2|\!+\!O(\ep^3)\!>\!1.
%\end{eqnarray}
%Now choose $w\in\C$ (with $w\re>0$) so that \eqref{y1==2} holds, and fix such a $w$.
Using  \eqref{SetV} in \eqref{s1t=}, we
obtain, for some $\tilde\a_i\in\C$,
\equa{SSSS0000}{s=\tilde\a_0u+(\tilde\a_1u^2+\tilde\a_2w)\ep+O(\ep^2)
\mbox{ %where $u=-\kappa_2u$, which is regarded as a new variable,}}
.}}
Using \eqref{SetV} and \eqref{SSSS0000} in \eqref{@such111that=2}\,(ii), we  can then rewrite $C_2$ as [by writing $(1+s\ep)^{\kappa_2}=1+\kappa_2s\ep+\frac{\kappa_2(\kappa_2-1)}{2}s^2\ep^2+O(\ep^3)\,$]
 \equa{tildeS=}{C_2=-
\big|1+\tilde\a_3u\ep+(\tilde\a_4u^2+\tilde\a_5w)\ep^2\big|+\kappa_4'|1+u\ep|+\kappa_5'+O(\ep^3)\ge0,}
for some $\tilde\a_i\in\C$.
%
%Using this in \eqref{@such111that=2}\,(ii), we immediately obtain
%we can rewrite \eqref{@such111that=2} as (where $u=\frac{\th_1u}{x_1^{\th_1}+\l'}$, which is regarded as a new variable)
%\equa{@suchthat=2}{C'_1:=|1+\a_1u\ep+(\a_2 u^2+\a_3w)\ep^2|-\kappa''_1|1+u\ep+\a_4\bar  u^2\ep^2|-\kappa''_2+O(\ep^3)<0,}
%form some $\a_i\in\C,\,\kappa''_i\in\R_{>0}$ with $\kappa''_1+\kappa''_2=1$.
By comparing the coefficients of $\ep^1$ in \eqref{tildeS=},
we immediately obtain that if $\tilde\a_3\ne
\kappa_4'$, %the following does not hold,
then
 we can   choose $u$ [with $\big((\kappa'_4-\tilde\a_3)u\big)\re>0\ssc\,$] to satisfy  \eqref{tildeS=}.
%\equa{tiLA}{\dis\tilde\a_1=
%\kappa_1''.}

Assume $\tilde\a_3=
\kappa_4'$. %we have \eqref{tiLA}.
Then we see that $C_2$ in \eqref{tildeS=} is an $O(\ep^2)$ element. In this case, since we do not know
what are values of $\tilde\a_4$, our strategy is to compute the following coefficient %$\tilde\b$ defined below which is independent of $\tilde\a_2$
[%we remark (cf.~Remark \ref{FinalRem}) that here is the only place which needs some computations;
cf.~Convention \ref{conv1}\,(1)\,(2) for notations ``\,$\re$\,', ``\,$\im$\,'' and  $\Coeff$; also note that $u\re^2=(u\re)^2\ssc\,$], \equa{Coeffff}{\mbox{$\tilde\b=\tilde\b_1+\tilde\b_2$ \ with \ $\tilde\b_1
=\Coeff(C_2,u\re^2\ep^2)$ and $\tilde\b_2=\Coeff(C_2,u\im^2\ep^2).$}} Observe
that $\tilde\a_4$ %(similarly $\tilde\a_4$)
does not contribute to
$\tilde\b$ by noting the following  \equa{Premmmeiim}{(\tilde\a_4u^2\ep^2)\re=
\big(\tilde\a_{4{\ssc\,} \rm re}(u\re^2-u\im^2)+2\tilde\a_{4{\ssc\,}\rm im}u\re u\im\big)\ep^2,}
and that the imaginary part of $\tilde\a_4u^2\ep^2$ can only contribute an $O(\ep^4)$ element to $C_2$ in \eqref{tildeS=}.
Thus for the purpose of computing $\tilde\b$, we may assume $\tilde\a_4=\tilde\a_5=0$ %for $i\ne3$
(then the computation becomes much easier).
Since $\tilde\a_3=\kappa_4'$ %Noting from \eqref{tiLA} that $\tilde\a_1$
is real,
it is straightforward to compute that
\equa{WeRERE}{\dis\tilde\b=\frac12(\kappa_4'-\tilde\a_3^2)=\frac12\kappa_4'(1-\kappa_4')>0,}
by  \eqref{tildeS=} and \eqref{ka4p}
[remark: the fact that $\tilde\b$ is positive
is very crucial for the inequation \eqref{tildeS=} being solvable for any unknown $\tilde\a_i\in\C$ in \eqref{tildeS=}, cf.~Remark \ref{FinalRem}]. By \eqref{Coeffff} and \eqref{WeRERE}, either $\tilde\b_1>0$ or $\tilde\b_2>0,$ and we can then choose $u$ with sufficiently large $u\re
>0$ or respectively
$u\im>0$ to guarantee that \eqref{@such111that=2}\,(ii) [i.e., \eqref{tildeS=}${\ssc\,}$] holds (when $w$ is fixed). This completes the
proof of Theorem {\rm\ref{real00-inj}}\,(2)  for the case \eqref{ToSayas}.
\vskip5pt
\noindent{\bf Case 2}: {\it Assume we have \eqref{ToSayas0}.}
%
%Now assume $b=0$.
%
%First we assume $x_0\ne0$. In this case,
%%we define $F_0$ in \eqref{Ne2wf0F1++} as $F_0=F(x_0+x,y_0+y)$ (and the like for $G_0$) and define $q_0$ in \eqref{1++??+q0q1} as $q_0=(x_0+s \ep,y_0+t \ep)$.
%%
By Theorem \ref{real00-inj}\,(1)\,(ii), %we %need to consider two cases.\vskip4pt
%\noindent\noindent{\bf Subcase 2.1}:
%First
assume that we have the third equality of \eqref{ToSayas0}\,(a) %
[the arguments for the proof of the case with the second equality of
\eqref{ToSayas0}\,(a) is exactly similar% (to Case 1) but simpler because   in this case
%the corresponding inequation of \eqref{1MEMEME}\,(i)  only contains one
%negative term
]%
% (the proof for the case with any other equality of \eqref{ToSayas0}\,(a) is exactly similar)
, namely% [we use notations in \eqref{SimMMSMS} and \eqref{LetNSoOP}
% and denote $\bar x_0^{-1}\dot x_0,\bar x_1^{-1}\dot x_1,\bar y_1^{-1}\dot y_1$
%%,\kk^{-1}\dot x_0,\kk^{-1}\dot x_1,\g_{\kk,\kk}^{-1}\dot y_1
% by $\dot X_0,\dot X_1,\dot Y_1
%%\dot{\tilde x}_0,\dot{\tilde x}_1,\dot{\tilde y}_1
%$ respectively; further, to avoid confusion we denote the $\ep^3$ in
%\eqref{LetNSoOP}\,(ii) by $\kappa_4$ as in \eqref{ToSayas0}\,(b) (as mentioned in Remark \ref{FaMREE}, the $\ep$ in
%\eqref{LetNSoOP} is not the $\ep$ using here)$\ssc\,$]
,
\equa{i.s.SSS}{\mbox{$\dis\kappa'_4-1=0$, \ \ where $\dis\kappa'_4=
\frac{\kappa_4|x_0^3x_1y_1|^{\kappa_5}}{\kappa_3|x_0x_1^3y_1^2|}$.}}
%
%As above, say
%we have the second equality of \eqref{ToSayas0}\,(a) [as in Case 1, the proof for other cases is similar by noting that every corresponding inequation can be written as the form in \eqref{1MEMEME} with at most one negative term; if there are only two terms in an inequation, for example the inequation corresponding to the first or the last equality of \eqref{ToSayas0}\,(a), then we can rewrite the inequation as the form in
%\eqref{@such111that=2}\,(i); we remark that for the inequation corresponding to the first equality of \eqref{ToSayas0}\,(a), we need to consider the cases whether or not $a=0$, and in case $a=0$ we need to use \eqref{1-MAMMS} or \eqref{1-AMSMSM23333} to obtain solutions],
%i.e.,
%\equa{MSMSMSMS0000}{|y_1|^{-1}-\kappa_4=\kappa_3\kappa_1^{-1}|x_0|^{-\kappa_2}\cdot|y_1|.}
%By writing
%\equa{WyYaaa}{\mbox{$\dis|\dot x_0+\l|^{\kappa_4}=|x_0+\l|^{\kappa_4}\cdot\Big|1+\frac{\kappa_4x_0s\ep}{x_0+\l}+\frac{\kappa_4(\kappa_4-1)x_0s^2\ep^2}{2(x_0+\l)}+O(\ep^3)\Big|$
%,}} % we re-denote $\eta_1^{-1}a$, with $a$ being as in \eqref{s1t=}, as $a$ so that $s$ is shown as in \eqref{1MEMEME}],
%
%$\big|\l\dot x_1-\frac12\big|=\big|\l x_1-\frac12\big|\cdot|1+\eta_2 u\ep|$ with $\eta_2=\frac{\l x_1}{\l x_1-\frac12}\ne0$ [note from \eqref{ToSayas0}\,(b) that $\l x_1-\frac12\ne0$],
%Denote
%\equa{DeC1P}{\dis\!\!\!
%C'_1\!=\!\frac{|x_1|^{\kappa_1}}{\kappa_2|x_0|^{\kappa_3}|\dot x_1|^{\kappa_1}}\big(\kappa_2|\dot x_0|^{\kappa_3}\!-\!\kappa_4|\dot x_1|\!-\!|\dot x_1|^{\kappa_1}\big),\ \ C'_2\!=\!
%\frac{|\dot x_0|}{|y_1|}\Big(\frac{|\dot y_1|+\kappa_8}{|\dot x_0|}-\frac{|y_1|+\kappa_8}{|x_0|}
%\Big).\!\!\!}
Then by writing [using \eqref{s1t=} and \eqref{i.s.SSS}${\ssc\,}$] \begin{eqnarray}
\label{MSMSMSMSM2222}
\!\!\!\!\!&\!\!\!\!\!\!\!\!\!\!\!\!\!\!\!\!\!\!\!\!&
\frac{\kappa_4|x_0^3x_1y_1|^{\kappa_5}}{\kappa_3|x_0x_1^3y_1^2|}
\!=\!\Big|1+\Big(\big((3\kappa_5-1)a+\kappa_5-3\big)u+\big((3\kappa_5-1)b+\kappa_5-2\big)v\Big)\ep+\cdots\Big|,
%\nonumber\\[4pt]
%\!\!\!\!\!&\!\!\!\!\!\!\!\!\!\!\!\!\!\!\!\!\!\!\!\!&
%\phantom{\frac{|\l_1\dot x_0-\l_2 \dot x_1|^{\kappa_2}}{|\l_3\dot x_0+\l_4 \dot x_1|}}
%\!=\!\Big|1\!+\!(\tilde\a_3u\!+\!\tilde\a_4v)\ep\!+\!(\tilde\a_5u^2\!+\!\tilde\a_6uv\!+\!\tilde\a_7v^2)\ep^2\Big|\!+\!O(\ep^3),
\end{eqnarray}
%for some $\tilde\a_i\in\C$%
%(and writing $\frac{|\dot y_1|}{|\dot x_0|^2\cdot|\l_1\dot x_1+\kappa_1|^{\kappa_5}}$ similarly)
%,
%
\NOUSE{%
by writing $\big($where $\eta=\l_1x_1+\l_2y_1(x_0x_1)^{-1}{\ssc\,}\big)$
\begin{eqnarray}\label{WbYsats}
&\!\!\!\!\!\!\!\!\!\!\!\!\!\!\!\!\!\!\!\!&
\dis\frac{
|\dot x_1|^{\kappa_1}}
{\Big|\l_1\dot x_1+\l_2\dot y_1(\dot x_0\dot x_1)^{-1}\Big|}
=\frac{|x_1|^{\kappa_1}}{|\eta|}\cdot\frac{
\Big|1+\kappa_1u\ep+
\frac12\kappa_1(\kappa_1-1)u\ep^2+\cdots\Big|}
{\Big|1+\eta^{-1}_1\big(\l_1u
+\l_2y_1(x_0x_1)^{-1}(v-u-s)\big)\ep+\cdots\Big|},\end{eqnarray}
and using \eqref{s1t=},
}%
we only need to find suitable $u,v$ to satisfy, for some $\tilde\a_i\in\C$
[cf.~\eqref{s1t=}${\ssc\,}$],
%we remark that all $\kappa'_i$'s below are positive numbers)% $\big[$for some $\tilde\a_i\in\C$, where
%$\kappa'_4=(\kappa_2|x_0|^{\kappa_3})^{-1}\kappa_4|x_1|$, $\kappa_0'=1-\kappa_4'$ and
%$\kappa'_8=\kappa_8|y_1|^{-1}{\ssc\,}]$
\begin{eqnarray}
\label{1MEMEME}
&&\!\!\!\!\!\!\!\!\!\!\!\!\!\!\!\!\!\!\!\!\!
{\rm(i)\ }C_1'\!:=\!\big|1+(\tilde\a_3u+\tilde\a_4v)\ep+(\tilde\a_5u^2+\tilde\a_6uv+\tilde\a_7v^2)\ep^2\big|-1+O(\ep^3)\ge0
%(\tilde\a_3u+\tilde\a_4v)\ep+(\tilde\a_5u^2+\tilde\a_6uv+\tilde\a_7v^2)\ep^2|-1+O(\ep^3)\ge0,
%\nonumber\\[4pt]
%&&\!\!\!\!\!\!\!\!\!\!\!\!\!\!\!\!\!\!\!\!\!\!\!\!
\nonumber\\[6pt]
&&\!\!\!\!\!\!\!\!\!\!\!\!\!\!\!\!\!\!\!\!\!
{\rm(ii)\ }
C'_2:=\big|1+(\tilde\a_8u+\tilde\a_9v)\ep+(\tilde\a_{10}u^2+\tilde\a_{11}uv+\tilde\a_{12}v^2)\ep^2|
+\kappa_{11}'\nonumber\\[4pt]
&&\!\!\!\!\!\!\!\!\!\!\!\!\!\!\!\!\!\!\!\!\!
\phantom{{\rm(ii)\ }C'_2:=}
-(1+\kappa'_{11})|1-2u\ep+3u^2\ep^2\big|+O(\ep^3) >0,
%\mbox{ where }\kappa_0\!=\!
%\frac{\tilde y_1\!-\!(1\!-\!\ell^{-4})}{\ell^{-4}\tilde x_1^{\ell^4}}\!\ge\!1\mbox{\ [by \eqref{LetNSoOP}\,(ii)]}.
\end{eqnarray}
for some $\tilde\a_i\in\C$,
where $\kappa'_{11}\!=\!\kappa_{11}|x_1^3y_1^2|^{-\kappa_{9}}|x_0|^{\kappa_{10}}$, and \eqref{1MEMEME}\,(ii) is obtained by rewriting \eqref{GSoo+}
%[i.e., $|\dot x_0|(|\dot x_1|^{\kappa_7}+\kappa_8)>|x_0|(|x_1|^{\kappa_7}+\kappa_8)\ssc\,$]
as $\frac{|\dot x_1^3\dot y_1^2|^{\kappa_9}}{|\dot x_0|^{\kappa_{10}}}+\kappa_{11}-
|x_0|^2\big(\frac{|x_1^3 y_1^2|^{\kappa_9}}{|x_0|^{\kappa_{10}}}+\kappa_{11}\big)|\dot x_1|^{-2}>0$.
One can easily compute
\equa{a-4===}{\tilde\a_4=(3\kappa_5-1)b+\kappa_5-2
%\mbox{ \ \ [recall from \eqref{-EiathA} that $\kappa_5\ne0\ssc\,$]}
,\ \ \ \tilde\a_{9}=2\kappa_9-\kappa_{10}b.}
%, $
%\kappa'_4=|y_1|^{-1}\kappa_4$ and for some $\eta_i\in\C$ with
%\equa{WiAISI}{\mbox{$\dis\eta_1=\eta_0\l_1x_0,\ \ \eta_2=\eta_0\l_2x_1^{-4}$, \ $\eta_0=(\l_1x_0+\l_2x_1^{-4})^{-1}$, and $\eta_4=-4(\kappa'_0-\l_0x_1)^{-1}\l_0x_1$.}}
%
%where $\kappa'_5=\kappa_5|x_0|^{-1}$ and \equa{slksld}{\tilde\a_4=\eta^{-1}\l_2y_1(x_0x_1)^{-1}(b-1).}
%First assume $b\ne0$ [cf.~\eqref{s1t=}]. Then one can observe from
%\eqref{1MEMEME}\,(i) that as in Case 1, we can always set $v=(\eta_1b)^{-1}\big(\eta_5u+(\eta_6u^2+w)\ep\big)$ for some $\eta_i,u,w\in\C$ with $w\re>0$ such that $C'_1$ becomes $C'_1=|1+w\ep^2|-1+O(\ep^3)=w\re\ep^2+O(\ep^3)>0$, i.e,
%\eqref{1MEMEME}\,(i) holds.
%
\NOUSE{%
Observing from \eqref{MSMSMSMSM2222}, we have the following,
\equa{smsm4444}{\dis\tilde\a_4=\kappa_4b-\l_4,\mbox{ where }\l_4=\frac{\l_1y_1x_1^{-5}}{\l_1y_1x_1^{-5}+\l_2x_1^{-1}}\ne0.}
}%
%
First assume %$b\ne0,$ i.e.,
$\tilde\a_4\ne0$ [cf.~\eqref{1MEMEME}\,(i){$\ssc\,$}]. Then we can set, for some $\tilde\a_{13},w\in\C$ with $w\re>0$ [cf.~\eqref{s1t=}$\ssc\,$],
\equa{v====}{v=\tilde\a_4^{-1}\Big(-\tilde\a_3u+(\tilde\a_{13}u^2+w)\ep\Big),
}
so that $C'_1$ can become $C'_1=|1+w\ep^2|-1+O(\ep^3)=w\re\ep^2+O(\ep^3)>0$,
%$C'_2$ can become %such that the terms inside the first absolute sign $|\cdot|$ can become
%$1+(1+\kappa'_5)u\ep+O(\ep^3)$, and so
%\eqref{1MEMEME}\,(ii) becomes
%(thus $\kappa_4>0$ is crucial in our proof)
%\equa{C333s}{\dis C'_2=|1+v\ep|+\kappa_4'-(1+\kappa'_4)|1+(1+\kappa'_4)^{-1}v\ep|+O(\ep^3)=\frac{\kappa'_4v\im^2\ep^2}{2(1+\kappa'_4)}+O(\ep^3)>0,}
% then
% one can easily compute that $C'_2$ becomes \equa{C2Beomm}{\mbox{$\dis C'_2=\frac12(1+\kappa'_8)\kappa_8'\kappa_9^2u\im^2\ep^2+O(\ep^3)>0$,}}
i.e.,  \eqref{1MEMEME}\,(i) holds.
Then using \eqref{v====},  we can rewrite $C'_2$
 as (for some $\tilde\a_i\in\C$)
%becomes
%an element  depending  on $v$ of the following form (for some $\eta_i\in\C$)
\equa{C12BsM}{C'_2=|1+\tilde\a_{14}u\ep+(\tilde\a_{15}u^2+\tilde\a_{16}w)\ep^2|
+\kappa'_{11}-(1+\kappa'_{11})\big|1-2u\ep+3u^2\ep^2\big|
+O(\ep^{3}).}
%such that $(\eta_7,\eta_9)\ne(0,0)$.
Observe the following. \begin{itemize}\item
If $c_0:=\tilde\a_{14}+2(1+\kappa'_{11})\ne0$, one can immediately choose $u\in\C$ with $(c_0u)\re>0$ to guarantee
that $C'_2=(c_0u)\re\ep+O(\ep^2)>0$.
\item
If $c_0=0$ (then $\tilde\a_{14}$ is a real number),
%
%
%thus $\kappa_4>0$ is crucial in our proof
%
\NOUSE{%
%
If $C'_1$ is independent of $v$, then \eqref{1MEMEME}\,(i) holds automatically. Otherwise,
for some $k\in\Z_{\ge1}$ and nonzero $\eta_7\in\C$. We can always choose $v\in\C$ with $(\eta_7v^k)\re>0$ and $v\im\ne0$ to guarantee that both of inequations in
 \eqref{1MEMEME} hold.
%
Finally assume $a=0$ [thus $\tilde\a_4\ne0$ by \eqref{slksld}]. Then any choice of $u$ only contributes an $O(\ep^2)$ element to $C'_2$, and we can always first choose $u\in\C$ as in Case 1 such that \eqref{1MEMEME}\,(ii) holds, then choose $v\in\C$ with $(\tilde\a_4v)\re>0$ to be sufficiently large so that \eqref{1MEMEME}\,(i) holds.
}%
%
%We simply set $v=0$. If $C'_1$ is independent of $u$ [then \eqref{1MEMEME}\,(i) holds automatically], we only need to choose $u$ to satisfy
%\eqref{1MEMEME}\,(ii), which can be done as in Case 1 [cf.~arguments after \eqref{tildeS=} and Remark \ref{FinalRem}].
%
%if $C'_1$ does not depend on $u$% (the fact is that this is impossible but we do not need this fact)
%, i.e., %$s=0$ [cf.~\eqref{4-1}--\eqref{q0q1} and \eqref{s1t=}${\ssc\,}$], then
%$C'_1=0$ [namely, \eqref{1MEMEME}\,(i) trivially holds],
%we can easily choose $u$ as above to satisfy  \eqref{1MEMEME}\,(ii) [notice that
%the last term of \eqref{1MEMEME}\,(ii) is $(1+\kappa'_6)|1+u\ep|$: if $a\ne1+\kappa'_6$, we immediately have a solution, otherwise process as above];
%
%
%
%We do not need the actual values of $\tilde\a_3,\tilde\a_5$, however we need the
%following important fact  [which can be  directly observed from \eqref{1MEMEME}\,(i) simply because of the factor $(1+u\ep)^{-1}$ in the last term of \eqref{1MEMEME}\,(i){$\ssc\,$}],\equa{ImAMSM}{\mbox{$(\tilde\a_3,\tilde\a_5)\ne(0,0)$.}}
%In fact, one can compute \equa{a3-a5}{\mbox{$\tilde\a_3=-\kappa'_2\big(a+b(1+\kappa'_3)\big)\kappa_9$, \ \ $\dis\tilde\a_5=\frac{(1-\kappa'_4)\tilde\a_3}{\kappa'_2}-\kappa'_4$.}}
%\begin{itemize}\item[(a)]
%If $c_1:=\eta_7-(1+\kappa'_2)\eta_9\ne0$, then we can choose $v\in\C$ with $v\im\ne0$ and $(c_1v)\re>0$ to guarantee that both of \eqref{1MEMEME} hold.
%\item[(b)] Assume $c_1=0$.
%Say $\eta_7\ne0$ (the proof for $\eta_9\ne0$ is similar). Set $v=\eta_7^{-1}\bar v$ with  $\bar v\in\C$ being regarded as a new variable.
then as in Case 1 [cf.~\eqref{Coeffff}{$\ssc\,$}], we need to compute
$\tilde\b=\tilde\b_1+\tilde\b_2$ with  $\tilde\b_1=\Coeff(C_2',u\re^2\ep^2)$ and $\tilde\b_2=\Coeff(C'_2,u\im^2\ep^2)$, and it is easy to obtain that
 $\tilde\b=2\kappa'_{11}(1+\kappa'_{11})>0$
(thus the fact that $\kappa_{11}>0$ is very crucial in our proof).
Then we can choose $u\in\C$ with
%
%$(\eta_7^{-1}\bar v)\im=v\im\ne0$ and`
%\equan{SMSMS}{\mbox{(b1) \
%$\bar v\re^2\gg\bar v\im^2$ in case $\tilde\b_1>0$; \ \
%\ \ \ (b2) \
% $\bar v\im^2\gg\bar v\re^2$ in case $\tilde\b_2>0$;}}
%to guarantee that both of \eqref{1MEMEME} hold.
%\end{itemize}
%
$u^2\re>0$  in case $\tilde\b_1>0$ (or $u^2\im>0$ in case $\tilde\b_2>0$) to be sufficiently large to guarantee that $C'_2>0$.
\end{itemize}

Now assume $\tilde\a_4=0$, i.e., $b=-\frac{\kappa_5-2}{3\kappa_5-1}$.
We claim that $\tilde\a_9\ne0$. To see this, using notations in \eqref{LetNSoOP}, we in fact have
[thus $b=\frac4{13}+O(v_0^1)\ssc\,$]
\begin{eqnarray}
\label{kekeke-gen}
&\!\!\!\!\!\!\!\!\!\!\!\!\!\!\!\!&
\kappa_5=\frac{6-2v_0}{5-v_0}+O(v_0^5)=\frac65+O(v_0^1),\ \ \ \ \
\kappa_9=\frac{3}{6-2v_0}+O(v_0^5)=\frac12+O(v_0^1),\nonumber
\\[4pt] &\!\!\!\!\!\!\!\!\!\!\!\!\!\!\!\!& \kappa_{10}=5-\frac{3}{6-2v_0}+O(v_0^5)=\frac{9}{2}+O(v_0^1).
\end{eqnarray}
Thus $\tilde\a_9=2\kappa_9-\kappa_{10}b=
-\frac{5}{13}+O(v_0^1)\ne0$
 [note that in case  the second equality of
\eqref{ToSayas0}\,(a) holds we still have no problem since $\kappa_2=\kappa_5+O(v_0^5)$ by \eqref{LetNSoOP}$\ssc\,$].
%
% [then $\tilde\a_{10}=-b\ne0$ by \eqref{a-4===}$\ssc\,$]%
%.
We simply set $u=0$% [then \eqref{1MEMEME}\,(ii) becomes very simple, namely, $C'_2=|1+v\ep|-1\ssc\,$]
. Then we have the following.\begin{itemize}\item
If $C'_1$ is independent of $v$, then $C'_1=0$, i.e., \eqref{1MEMEME}\,(i) automatically holds, and we
can easily choose $v\in\C$ with $(\tilde\a_{9}v)\re>0$ so that $C_2'=(\tilde\a_{9}v)\re\ep+O(\ep^2)>0$, i.e., \eqref{1MEMEME}\,(ii) holds.
\item Otherwise, $C'_1=|1+b'v^k\ep^k|-1+O(\ep^{k+1})$ for some $b'\in\C_{\ne0}$ and $k\in\Z_{\ge2}$, and we can always choose $v\in\C$ with $(b'v^k)\re>0$ and $(\tilde\a_{9}v)\re>0$ (such  $v$ always exists simply because $k\ge2$) to guarantee that    both of \eqref{1MEMEME} hold.
\end{itemize}
\NOUSE{%
Now assume $a=0$. Then any choose
%
\noindent{\bf Subcase 2.2}: Finally, assume we have the first equality of \eqref{ToSayas0}\,(a). Then \eqref{1MEMEME}\,(i) should be replaced by the inequation \equa{FCDou}{\mbox{$C''_1=|1+s\ep|-1\ge0$.}}
If $a\ne0$ or respectively, $b\ne0$, then as in \eqref{SetV}, by setting \equa{uandVSA}{\mbox{$u=\tilde\a_7v+(\tilde\a_8v^2+\tilde\a_9w)\ep$, \ or respectively, \ $v=\tilde\a_7u+(\tilde\a_8u^2+\tilde\a_9w)\ep$,}} for some $\tilde\a_i\in\C$ and $w\in\C$ with $w\re>0$ so that $C''_1$ can become that $C''_1=w\re\ep^2+O(\ep^3)>0$. Then again as in Case 1 (cf.~Remark  \ref{FinalRem}), we can choose $v$, or respectively, $u$, to guarantee that
\eqref{1MEMEME}\,(ii) holds.
}%
%
\NOUSE{%
The main difference of this system of inequations from the one in Case 1 is that there are two
negative terms in \eqref{1MEMEME}\,(ii) (cf.~Remarks \ref{remaFFFF} and \ref{FinalRem}).
%
%Observe the important fact that the left-hand sides of both inequations have only one negative term, which guarantees that the
%inequations are always solvable as shown below (cf.~Remark \ref{FinalRem}).
%
%where $\kappa'_{11}=\kappa_{11}\big(\big|x_0^5y_1\big(\l x_1-\frac12\big)\big|\big)^{-1}|\l x_1+1|^4.$
%
However, we have designed \eqref{ToSayas0} [cf.~\eqref{LetNSoOP}{$\ssc\,$}] so that this system of inequations is still solvable as shown below.
%
%\vskip4pt
%\noindent{\bf Subcase 2.1.~}First assume
%$b\ne0$ [cf.~\eqref{s1t=}${\ssc\,}$]. Then we can regard $s$ as a free variable and solve $v$ from \eqref{s1t=} to obtain
%\equa{vToBo}{\!\!\!\!v\!=\!\hat a u\!+\!\hat b s\!+\!(\hat \a_1 u^2\!+\!\hat\a_2 u s\!+\!\hat \a_3 s^2)\ep\!+\!O(\ep^2)\mbox{ for some $\hat a,\hat b,\hat\a_i\!\in\!\C$ with $\hat a\!=\!-a b^{-1},\,\hat b\!=\!b^{-1}\ne0$.}\!\!\!\!}
First observe the following.\begin{itemize}
\item If $a\im\ne0$ [cf.~\eqref{s1t=}],
then we can set $v=0$ and choose $u\in\C$ with
$u\re<0$ [then the coefficient of $\ep^1$ in $C'_2$ is
$-\ell^8\tilde x_1^{-\ell^4}\big(
\tilde y_1-(1-\ell^{-4})\big)u\re
>0$,
%recall that $\kappa_i,\kappa'_i>0$ and  $\kappa_1<1$ by \eqref{-EiathA}${\ssc\,}$]
and with $c_0:=\big(\ell^{-1}\tilde x_0^{\ell^{-1}}a-\tilde x_1)u\big)\re>0$ [then the coefficient of $\ep^1$ in $C'_1$ is $c_0>0{\ssc\,}$] so that both of \eqref{1MEMEME} hold.
\item If $b\im\ne0$, then we can set $u=0$ and choose
$v\in\C$ with $v\re>0$ [then the coefficient of
$\ep^1$ in $C'_2$ is $\ell^4\tilde x_1^{-\ell^4}\tilde y_1v\re>0{\ssc\,}$]
and with $c_1:=(\ell^{-1}\tilde x_0^{\ell^{-1}}bv)\re>0$ [then the coefficient of $\ep^1$ in $C'_1$ is  $c_1>0{\ssc\,}$]  so that both of \eqref{1MEMEME} hold.
\end{itemize}
Thus assume $a,b\in\R$. Then the following determinant is a real number:
%, and we may assume it is zero:
\equa{DeterM}{\left|\begin{array}{cc}
\ell^{-1}\tilde x_0^{\ell^{-1}}a-\tilde x_1&-\big(\kappa_1+\kappa'_4(1-\kappa_1)\big)\\[4pt]
\hat b-(1+\kappa'_8)&\hat a\end{array}\right|=0,}
otherwise we can choose $s,u\in\R$ such that the coefficients of $\ep^1$ in both $C'_1$ and $C'_2$ are any given positive numbers to guarantee that both of \eqref{1MEMEME} hold.
%
Now we set [we do not need to know the actual value of  $\hat\a_4$ but for sure such $\hat\a_4$ exists by regarding it as a parameter then solving it to ensure that the first term in $C'_1$ has the required form in \eqref{1MEMEME+}\,(i) below]
\equa{s==www}{s=\eta_1u+\hat\a_4u^2\ep\mbox{ for some $\eta_1\in\R,\,\hat\a_4\in\C$ with $\eta_1=\kappa_3^{-1}\big(\kappa_1+\kappa'_4(1-\kappa_1)\big)$,}}
so that \eqref{1MEMEME} can become [we do not need to know the actual value of $\eta_0$;
we wish to mention that it is very crucial that there is the term $u^2\ep^2$ inside the first absolute sign $|\cdot|$ of $C'_1$, which contributes the element $(u\re^2-u\im^2)\ep^2+O(\ep^4)$ to $C'_1$],
\begin{eqnarray}
\label{1MEMEME+}
&\!\!\!\!\!\!\!\!\!\!\!\!\!\!\!\!\!\!&
{\rm(i)\ }C_1':=\Big|1+\kappa'_4(1-\kappa_1)u\ep+(\kappa'_4\tilde\a_6+1)u^2\ep^2\Big|
-\kappa'_4\Big|1+(1-\kappa_1)u+\tilde\a_6u^2\ep^2\Big|-\kappa'_0+O(\ep^3)
\nonumber\\[4pt]
&\!\!\!\!\!\!\!\!\!\!\!\!\!\!\!\!\!\!&
\phantom{{\rm(i)\ }C_1'}=
(u\re^2+\eta_0u\im^2)\ep^2+O(\ep^3)
\ge0\mbox{ for some $\eta_0\in\R$},
%\nonumber\\[4pt]
%&\!\!\!\!\!\!\!\!\!\!\!\!\!\!\!\!\!\!&
\nonumber\\[6pt]
%\label{MEMEME}
&\!\!\!\!\!\!\!\!\!\!\!\!\!\!\!\!\!\!&
{\rm(ii)\ }
C'_2:=|1+\eta_2u\ep+\hat b\hat\a_4u^2\ep^2|+\kappa'_8
-(1+\kappa'_8)
|1+\eta_1u\ep+\hat \a_4u^2\ep^2|>0,\mbox{ with }
\nonumber\\[4pt]
&\!\!\!\!\!\!\!\!\!\!\!\!\!\!\!\!\!\!&
\eta_2=
\hat a+\hat b\eta_1
=\eta_1(2\hat b-1-
\kappa'_8),
\end{eqnarray}
where the second equality in $\eta_2$ follows from \eqref{DeterM} [namely $\hat a=(\hat b-1-\kappa'_8)\eta_1$].
We have the following.
\begin{itemize}\item
If $b\ne1+\kappa'_8$, i.e., $c_1:=\eta_2-(1+\kappa'_8)\eta_1\in\R_{\ne0}$, we can then choose $u\in\R_{\ne0}$ [then
\eqref{1MEMEME+}\,(i) holds]
and $c_1u\re>0$ [then \eqref{1MEMEME+}\,(ii) holds] so that both of \eqref{1MEMEME+} hold.
\item Assume $b=1+\kappa'_8$ [thus $\hat a=0$ by \eqref{DeterM}${\ssc\,}$].
Then it is very crucial that  $C'_2$ becomes
\begin{eqnarray}
\label{1MEMEME+2}
&&\!\!\!\!\!\!\!\!\!\!\!\!\!\!\!\!\!\!\!\!\!\!\!\!\!\!\!\!\!\!\!\!
C'_2:=|1+(1+\kappa'_8)\eta_1u\ep+(1+\kappa'_8)\hat\a_4u^2\ep^2|+\kappa'_8
-(1+\kappa'_8)
|1+\eta_1u\ep+\hat \a_4u^2\ep^2|,\nonumber\\[4pt]
&&\!\!\!\!\!\!\!\!\!\!\!\!\!\!\!\!\!\!\!\!\!\!\!\!\!\!\!\!\!\!\!\!
\phantom{C'_2}=\frac{\eta_1^2\kappa'_8(\kappa'_8+1)u\im^2\ep^2}{2}+O(\ep^3)>0.
\end{eqnarray}
One can easily choose $u\in\C$ with $u\im\ne0$ [then \eqref{1MEMEME+2} holds, i.e., \eqref{1MEMEME+}\,(ii) holds] and $u\re>0$ being sufficiently large
such that $u\re^2+\eta_0u\im^2>0$
[then \eqref{1MEMEME+}\,(i) holds] to ensure we have solutions for inequations \eqref{1MEMEME+}.
%Alternatively, we can give another proof for this case as follows: Since $\hat a=0$, any choice of $u$ can only contribute an $O(\ep^2)$ element to $C'_2$, we can always first choose $s\in\C$ with
\end{itemize}
\vskip4pt
\noindent{\bf Subcase 2.2.~}Finally assume $b=0$ [cf.~\eqref{s1t=}${\ssc\,}$]. Then any choice of $v$ can only contribute an $O(\ep^2)$ element to $C'_1$ in \eqref{1MEMEME}\,(i). In this case, we can always first choose
$u\in\C$ with \equa{c1===0}{\mbox{$\dis
c_2:=\Big(\big(\kappa_3a-\kappa_1-\kappa'_4(1-\kappa_1)\big)u\Big)\re>0$,}}
so that the coefficient of
$\ep^1$ in $C'_1$ is $c_2>0,$
to satisfy \eqref{1MEMEME}\,(i), then choose $v\in\R_{>0}$ with $c_3:=v-(1+\kappa_8') a\re u>0$
[then the coefficient of
$\ep^1$ in $C'_2$ is $c_3>0{\ssc\,}$]
to satisfy \eqref{1MEMEME}\,(ii).
}%


\NOUSE{%
$\tilde\a_3a\ne0$ [observing the part $\tilde\a_3s\ep$ in $C'_1$, cf.~\eqref{s1t=}${\ssc\,}$].
Then as in Case 1 [cf.~\eqref{SetV}${\ssc\,}$], we can
set%; we remind that $x_0\ne0$, cf.~\eqref{-EiathA0}
,
\equa{NEWSETv}{\mbox{$\dis u\!=\!-(\tilde\a_3 a)^{-1}\Big((\tilde\a_3b+\tilde\a_4)v+
(\tilde\a_7 v^2+w)\ep\Big)$ for some $\tilde\a_7,w\in\C$ with $w\re>0$,
% for some $\eta_i,w\!\in\!\C$ with $w\re\!>\!0$ and
%$\dis\eta_0\!=\!-\frac{a\kappa_4x_0\!+\!\kappa_2(x_0\!+\!\l)}{b\kappa_4x_0}$
}}
so that
$C_1'$ can become $C'_1=|1+w\ep^2|+O(\ep^3)-1=w\re\ep^2+O(\ep^3)>0$, i.e.,
\eqref{1MEMEME}\,(i) holds.
% [up to $O(\ep^3){\ssc\,}$]\equa{C1bBB}{\dis
%C'_1\!=\!\kappa'_1\!+\!|1\!-\!(1\!+\!\kappa'_1)v\ep\!+\!(1\!+\!\kappa'_1)v^2\ep^2|\!-\!(1\!+\!\kappa'_1)|1\!-\!v\ep\!+\!v^2\ep^2|
%\!=\!\frac{\kappa'_1(\kappa'_1+1)v\im^2\ep^2}{2}\!>\!0,}
%which holds as long as $v\im\ne0$.
Then \eqref{1MEMEME}\,(ii) becomes
\begin{eqnarray}
\label{2MEMEME}&&\!\!\!\!\!\!\!\!\!\!\!\!\!\!\!\!
C'_2\!:=\!|1\!+\tilde\a_8v\ep+\tilde\a_9v^2\ep^2|\!+\!\kappa'_6
\!-\!(1\!+\!\kappa'_6)|1\!+\!
\tilde\a_{10}v\ep\!+\!\tilde\a_{11}v^2\ep^2|\!+\!O(\ep^3)\!>\!0,
\end{eqnarray}
for some $\tilde\a_i\!\in\!\C$ % with $\tilde\a_3=-\frac{\kappa_2(x_0+\l)\kappa_6}{\kappa_4x_0}\ne0$ [which can be easily computed using \eqref{s1t=} and \eqref{NEWSETv}${\ssc\,}$]
with $\tilde\a_8=-\frac{\tilde\a_4}{\tilde\a_3}\ne0$.
Now as in Case 1,\begin{itemize}\item if $c_0:=\tilde\a_8-(1+\kappa'_6)\tilde\a_{10}
\ne0$ then we can immediately
 choose suitable $v\in\C$ with $v\im\ne0$ and with  $(c_0v)\re>0$ to satisfy \eqref{2MEMEME}
 [thus both inequations of \eqref{1MEMEME} hold];
\item otherwise we set $\bar v=\tilde\a_8v$ [cf.~the first term of \eqref{2MEMEME}${\ssc\,}$], and  compute
$\tilde\b=\tilde\b_1+\tilde\b_2$ with $\tilde\b_1=\Coeff(C'_2,\bar v\re^2\ep^2)$, $\tilde\b_2=\Coeff(C'_2,\bar v\im^2\ep^2)$. As in \eqref{WeRERE}, $\tilde\b>0$ (cf.~Remark \ref{FinalRem}), and we can always choose $v$
%with $v\im\ne0$ and
with $\bar v\re\ne0$ %(such that $|\bar v\re|$ is sufficiently large)
if $\tilde\b_1>0$ or $\bar v\im\ne0$  %(such that $|\bar v\im|$ is sufficiently large)
if $\tilde\b_2>0$
 to ensure that there are solutions for  \eqref{2MEMEME}.
\end{itemize}
Now assume $\tilde\a_3a=0$.
Set $v=0$, and
%
%If $\tilde\a_3=0$, then similar to \eqref{NEWSETv}, by setting $v=\tilde\a_{12}u+(\tilde\a_{13}u^2+\tilde\a_4^{-1}w)\ep$ for some
%$\tilde\a_{i},w\in\C$ with $w\re>0$, we can guarantee that \eqref{1MEMEME}\,(i) holds.
%Then as above, we can choose suitable $u$ to satisfy \eqref{1MEMEME}\,(ii). If $\tilde\a_3\ne0,a=0$ (then $b\ne0$),
%
%
%Set $v=0$.
\begin{itemize}\item
if $C'_1$ does not depend on $u$% (the fact is that this is impossible but we do not need this fact)
, i.e., %$s=0$ [cf.~\eqref{4-1}--\eqref{q0q1} and \eqref{s1t=}${\ssc\,}$], then
$C'_1=0$ [namely, \eqref{1MEMEME}\,(i) trivially holds],
we can easily choose $u$ as above to satisfy  \eqref{1MEMEME}\,(ii) [notice that
the last term of \eqref{1MEMEME}\,(ii) is $(1+\kappa'_6)|1+u\ep|$: if $a\ne1+\kappa'_6$, we immediately have a solution, otherwise process as above];
\item
otherwise, $C'_1=|1+b'u^k\ep^{k-1}|-1+O(\ep^k)$ for some $b'\in\C_{\ne0}$, $k\in\Z_{\ge2}$,
and we can choose
$u$  to satisfy both  \eqref{1MEMEME}\,(ii) (by considering  as above whether or not $a\ne1+\kappa'_6$)
and \eqref{1MEMEME}\,(i) [by requiring that $(b'u^k)\re>0{\ssc\,}$],
 %and with $u\re<0$ so that   holds
such $u$ always exists simply because $k\ge2$.\end{itemize}
}%
%and we can always choose $v\in\C$ with $(b'v^k)\re>0$ so that $C'_2>0$.
%
%Finally assume we have the last equality of \eqref{ToSayas0}\,(a), i.e., $\kappa_3-\kappa_4|y_1|^{-1}=1$. In this case \eqref{1MEMEME}\,(i) is replaced by $C'_3:=|1+v\ep|-1\le0$, which holds trivially if we simply set $v=0$. Then we can choose $u$ to satisfy \eqref{1MEMEME}\,(ii) as above [note that the coefficient of $u$ in the last term of \eqref{1MEMEME}\,(ii) is nonzero, which guarantees that everything can be done as above].
%
%
%
%If $a\ne\eta_1$, we can always first choose $u$ [with $\big((\eta_1-a)u\big)\re>0$] to satisfy \eqref{1MEMEME}\,(i) then choose $v$ to satisfy
%\eqref{1MEMEME}\,(ii). Thus assume $a=\eta_1$.
%Take $u=0$ in \eqref{1MEMEME}. If $s$ does not depend on $v$ [i.e., $s=0$,
%cf.~\eqref{4-1}--\eqref{q0q1} and \eqref{s1t=}; note that the fact is that this is impossible but we do not need this fact], then \eqref{1MEMEME}\,(i) trivially holds and we can easily choose $v$ (with $v\re>0$) to satisfy \eqref{1MEMEME}\,(ii); otherwise, $s=b'v^k\ep^{k-1}+O(\ep^k)$ for some $b'\in\C_{\ne0}$, $k\in\Z_{\ge2}$, and we can choose
% $v$ [with $(b'v^k)\re<0$ so that  \eqref{1MEMEME}\,(i) holds and with $v\re>0$ so that  \eqref{1MEMEME}\,(ii) holds] to satisfy \eqref{1MEMEME} (such $v$ always exists simply because $k\ge2$).
%
%
%
%We need to use
% arguments after \eqref{@suchthat=2-for} to obtain the similar versions of either \eqref{[MAMMS]} and \eqref{s=a}, or else
%\eqref{AMSMSM23333} and \eqref{s=a+1} to obtain a solution [more precisely, see \eqref{IeI1} and \eqref{1IeI1}${\ssc\,}$].
%
%This proves Theorem \ref{real00-inj}.
%\hfill$\Box$
%
%
%the coefficient of $v$ in $C_2$ of \eqref{1MEMEME}\,(ii) is $1$, as above we can set $v=\eta_1u+\eta_2u^2\ep$ for some $\eta_i\in\C$ so that
%\eqref{1MEMEME} becomes
%\begin{eqnarray}
%\label{2MEMEME}&&\!\!\!\!\!\!\!\!\!\!\!\!\!\!\!\!\!\!\!\!\!\!\!\!
%{\rm(i)\ }C''_1:=-|1-2u\ep+4u^2\ep^2|
%+\kappa_2'+\kappa_3'\Big|1+\tilde\a_{11}u\ep+\tilde\a_{12}u^2\ep^2\Big|+O(\ep^3)
%>0,\nonumber\\[6pt]
%\label{MEMEME}
%&&\!\!\!\!\!\!\!\!\!\!\!\!\!\!\!\!\!\!\!\!\!\!\!\!
%{\rm(ii)\ }C''_2:=\Big|1+\kappa'_6(\kappa_6u\ep+\tilde\a_7u^2\ep^2)\Big|+
%\kappa_5'-\kappa_6'|1+\kappa_6u\ep+\tilde\a_7u^2\ep^2|+O(\ep^3)
%\nonumber\\[4pt]
%\label{MEMEME}
%&&\!\!\!\!\!\!\!\!\!\!\!\!\!\!\!\!\!\!\!\!\!\!\!\!
%\phantom{{\rm(ii)\ }C''_2:}=\frac12\kappa_6^2\kappa_6'(\kappa'_6-1)u\im^2
%+O(\ep^3)>0,
%\mbox{ where }\\[6pt]
%\label{2MEMEME}&&\!\!\!\!\!\!\!\!\!\!\!\!\!\!
%\kappa_1'=\frac{|x_0|^2(|x_1|^2+\kappa_3)}{|x_1|^{2}(|x_0|^2+\kappa_4)}>0, \ \ \kappa_2'=\frac{\kappa_3|x_0|^2-\kappa_4|x_1|^2}{|x_1|^2(|x_0|^2+\kappa_4)}.
%\end{eqnarray}for some $\tilde\a_i\in\C$.
%Again,
%\eqref{2MEMEME}\,(ii) holds automatically as long as $u\im\ne0$, and
%if $\kappa'_3\tilde\a_{11}\ne-2$ we immediately have solutions for \eqref{2MEMEME}, otherwise by computing
%$\b''=\Coeff(C''_1,u\re^2\ep^2)+\Coeff(C''_1,u\im^2\ep^2)$, which is positive
% for any nonzero $v\im$, then choose suitable $v$ to satisfy
%\eqref{1MEMEME}\,(ii).
%to ensure that there are solutions for  \eqref{2MEMEME}.
%
\NOUSE
{
that the terms inside the first absolute value become
Say $\tilde\b_1>0$ (the proof is similar if $\tilde\b_2>0$).
If $\eta:=\tilde\a_{9\,\rm re}-\kappa_6'\kappa_6\ne0$, we can always choose $u\in\R$ with $\eta u>0$ and $|u|$ being sufficiently large to guarantee that both $C_1>0$ in \eqref{1MEMEME00} and $C_2>0$ in \eqref{MEMEME00}.
If $\eta=0$ and $\tilde\a_{9\,\rm im}\ne0$, we can always choose $u\in\C$ with $\tilde\a_{9\,\rm im}u\im>0$
and $u\re$ be sufficiently large to guarantee that both $C_1>0$  and $C_2>0$.
%
%
As in \eqref{@such111that=2}, \eqref{@suchthat=3} and \eqref{SetV}, we first
set $v=2u+ u^2\ep+w\ep$ for some $w\in\R_{>0}$ to guarantee that $C'_1<1$. Then $C'_2$ can be rewritten as
\begin{eqnarray}\label{NeDDDDDD}
C'_2:=|1+u\ep|^2-\kappa_1'|1+\a_1u\ep+\a_2u^2\ep^2|^2+\kappa'_2+O(\ep^3)>0.
\end{eqnarray}
If $\kappa_2'>0$, then exactly as above, we can find some $u$ satisfying \eqref{NeDDDDDD} (cf.~Remark \ref{FinalRem}). Thus assume $\kappa'_2\le0.$
Note from \eqref{11111NSoOP} that when condition \eqref{ToSayas0}\,(a) holds, we have some sufficiently large $\kk$ such that $|x_1|\sim|y_1|\sim\kk$ and $|x_0|\preceq\kk$.
Thus we must also have $|y_0|\preceq\kk$ by \eqref{mqp1234-2}.
%
Note that since $\kappa_3=\kk,\kappa_4=\kk^{-\kk}$ [cf.~\eqref{11111NSoOP}${\ssc\,}$], we obtain that $|x_0|^2<\kk^{-\kk}$ and from \eqref{2MEMEME} that $\kappa'_1\preceq\kk^{-\kk}$.
%
%Now assume $b=0$.
If $a\ne\kappa_2$ then we can always first choose $u$ to satisfy \eqref{@such111that=2}\,(i), then choose $v$ to satisfy \eqref{@such111that=2}\,(ii) (the choice of $v$ does not affect the coefficient of $\ep^1$ in $C'_2$).
%
Thus assume $b=0,\,a=\kappa_2$. Take $u=0$ in \eqref{@such111that=2}. If $s$ does not depend on $v$ [i.e., $s=0$,
cf.~\eqref{4-1}--\eqref{q0q1} and \eqref{s1t=}${\ssc\,}$], then \eqref{@such111that=2}\,(i) holds trivially and we can easily choose $v$ to satisfy \eqref{@such111that=2}\,(ii); otherwise, $s=b'v^k\ep^{k-1}+O(\ep^k)$ for some $b'\in\C_{\ne0}$, $k\in\Z_{\ge2}$, and we can choose
 $v$ to satisfy both \eqref{@such111that=2}\,(i) and \eqref{@such111that=2}\,(ii) (such $v$ always exists simply because $k\ge2$).
%
%We need to use
% arguments after \eqref{@suchthat=2-for} to obtain the similar versions of either \eqref{[MAMMS]} and \eqref{s=a}, or else
%\eqref{AMSMSM23333} and \eqref{s=a+1} to obtain a solution [more precisely, see \eqref{IeI1} and \eqref{1IeI1}${\ssc\,}$].
}%
%\vskip4pt
\noindent This proves Theorem \ref{real00-inj}.
\hfill$\Box$


\begin{rema}\label{FinalRem}\rm
(cf.~Remark \ref{MARK1})
Assume that we have the following inequation on variable $u$, where $\a_1,\a_2,\b_1,\b_2\in\R_{>0}$, and $a_1,a_2,a_3\in\C$ are some unknown complex numbers:
\equa{FinalRem1}{\a_1|1+a_1u\ep+a_2u^2\ep^2+a_3\ep^2|^{\b_1}<\a_2|1+u\ep|^{\b_2}+\a_1-\a_2+O(\ep^3).}
Then from the proof of \eqref{@such111that=2}, one can see that this inequation is solvable for {\bf any unknown} $a_1,a_2,a_3\in\C$ if and only if $\a_1>\a_2$ [the reason that
\eqref{1MEMEME}\,(i) is solvable is that we have designed \eqref{1MEMEME}\,(i) to be compatible with \eqref{1MEMEME}\,(ii)${\ssc\,}$].
%From this we see that if the last equality holds in \eqref{ToSayas}\,(a), then the inequation we need to consider is solvable (since $\kappa_{2}>\kappa_{4}$). We also like to mention that if the first equality holds  in \eqref{ToSayas}\,(a),
%then the inequation we need to consider is simply the one
%$|1\!+\!u\ep|>1$, which can be easily satisfied.
\end{rema}


















































\section{Proofs of Theorems \ref{MAINT} and \ref{real-inj-1}}

\noindent{\it Proof of Theorem \ref{real-inj-1}.~}
To prove Theorem \ref{real-inj-1}\,(i), we use Theorem {\rm\ref{real00-inj}}.
Denote [cf.~\eqref{ToSayas}, \eqref{ToSayas0}${\ssc\,}$]
\equa{DeBAR-D=}{ L=\{ \ell_{p_0,p_1}\,|\,(p_0,p_1)\in V_0\},\ \ \
 \ell={\rm sup\,} L\in\R_{>0}\cup\{+\infty\}\mbox{ (the supremum of $ L$)}.}
By definition, there exists a sequence $(p_{0i},p_{1i}):=\big((x_{0i},y_{0i}),(x_{1i},y_{1i})\big)\in V_0$, $i=1,2,...$, i.e., $p_{0i}{\ssc}\ne{\ssc} p_{1i}$ and
[assume we have case \eqref{ToSayas} as the proof the case \eqref{ToSayas0} is exactly similar],
\begin{eqnarray}\label{Acon111}
&\!\!\!\!\!\!\!\!\!\!\!\!\!\!\!\!\!\!\!\!&
\sigma(p_{0i})=\sigma(p_{1i}),\,\
\kappa_0\le|x_{1i}|\le\kappa_1|x_{0i}|^{\kappa_2}+\kappa_3\le
\kappa_4|x_{1i}|+\kappa_5,
\ \ \ell_i:=\frac{|y_{1i}|}{|x_{1i}|^{\kappa_6}}\ge\kappa_7,
\end{eqnarray}
such that  $\lim_{i\to\infty} \ell_i= \ell$ (cf.~Remark \ref{MARK1}$\ssc\,$).
By \eqref{-EiathA0},
$|x_{0i}|,\,|x_{1i}|$ are bounded.
%[note that for the case , i.e.,
%\eqref{LetNSoOP}, we  have \eqref{WEhHaaa}${\ssc\,}$].
By \eqref{mqp1234-2}, we see that
$|y_{0i}|,|y_{1i}|$ are also bounded [as in the proof of Theorem \ref{AddLeeme--0}\,(i)$\ssc\,$].
Thus by replacing the sequence by a subsequence, we may assume
\equa{(qqq)}{\lim\limits_{i\to\infty}(p_{0i},p_{1i})=(p_0,p_1)=\big((x_0,y_0),(x_1,y_1)\big)\in\C^4.}
First suppose $p_0=p_1$. Then by \eqref{(qqq)}, for any neighborhood $\OO{p_0}$ of $p_0$,
there exists $N_0$ such that  $p_{0i},p_{1i}\in\OO{p_0}$ when $i>N_0$, but $p_{0i}\ne  p_{1i},\,\si(p_{0i})=\si(p_{1i})$, which is a contradiction with the local bijectivity of Keller maps.
Thus $p_0\ne p_1$. By taking the limit $i\to\infty$, we see  that \eqref{-EiathA0} is satisfied by $x_0,x_1,y_1$ (in particular $x_1\ne0$) and all conditions in \eqref{ToSayas}  hold
for $(p_0,p_1)$.
Thus $(p_0,p_1)\in  V_0$.
 Therefore by Theorem {\rm\ref{real00-inj}}\,(2), there exists $(q_0,q_1)=\big((\dot x_0,\dot y_0),(\dot x_1,\dot y_1)\big)\in V_0$ such that $ \ell_{q_0,q_1}> \ell_{p_0,p_1}= \ell$,  a contradiction with \eqref{DeBAR-D=}. This proves that \eqref{CSweaua} is not true, i.e., we have Theorem \ref{real-inj-1}\,(i).

To prove Theorem \ref{real-inj-1}\,(ii), as in \eqref{MSAMSMS} and \eqref{Ne2wf0F1++},
take $(p_0,p_1)=\big((x_0,y_0),(x_1,y_1)\big)\in V$ and set
(and define $G_0,G_1$ similarly)
\begin{eqnarray}
\label{Newf0F1++00}&\!\!\!\!\!\!\!\!\!\!\!\!\!\!&
F_0= F(x_0+\a_0x,y_0+y),\ \ \ \ \ \  \ \ \ \ \ \ \ \ \,
\ \ F_1= F(x_1+\a_1x,y_1+y),\mbox{ \ \  where}\\[6pt]
\label{0+Newf0F1++00}&\!\!\!\!\!\!\!\!\!\!\!\!\!\!&
\a_0=\left\{\begin{array}{cl}1&\mbox{if }x_0=\xi_0,\\[4pt]
x_0-\xi_0&\mbox{else},\end{array}\right.\ \ \ \ \ \ \ \ \ \ \ \ \
\a_1=\left\{\begin{array}{cl}1&\mbox{if }x_1=\xi_1,\\[4pt]
x_1-\xi_1&\mbox{else.}\end{array}\right.
\end{eqnarray}
Define $q_0,q_1$ accordingly [cf.~\eqref{q0q1} and \eqref{1+++q0q1}${\ssc\,}$].
Then we have as in \eqref{st=} and \eqref{s1t=},
\equa{AsIn2.2}{s=au+bv+O(\ep^1).}
 Note from Theorem \ref{real-inj-1}\,(i) that $(x_0,x_1)\ne(\xi_0,\xi_1)$.

 First suppose $x_0\ne\xi_0,\,x_1\ne\xi_1$ (then $\a_0=x_0-\xi_0,\,\a_1=x_1-\xi_1$).
In this case, we need to choose $u,v$ such   that,
\equa{2DDDbar1}{C_0:=\b_0|1+s\ep|^2+\b_1|1+u\ep|^2-(\b_0+\b_1)<0,} where
$\b_0=|x_0-\xi_0|^2,\  \b_1=|x_1-\xi_1|^2$.
Using \eqref{AsIn2.2} in \eqref{2DDDbar1}, we immediately see (by comparing the coefficients of $\ep^1$) that if $b\ne0$ or $a\ne-{\b_0}{\b_1^{-1}}$, then we have a solution for \eqref{2DDDbar1}. Thus assume $b=0,\,a=-{\b_1}{\b_0^{-1}}$
[then $d\ne0$ in \eqref{4-2} and $a$ is real]. In this case, using arguments after \eqref{@suchthat=2-for},
we have the similar versions of either \eqref{[MAMMS]} and \eqref{s=a}, or else
\eqref{AMSMSM23333} and \eqref{s=a+1}, i.e.,
\begin{eqnarray}
\label{IeI1}\!\!\!\!\!\!\!\!\!\!\!\!\!\!\!\!\!\!\!\!&&
 u=\hat u\ep^{k-1},\ \ \ \ \ \ \,  v=d^{-1}w-d^{-1}c\hat u\ep^{k-1},\   \  \ \ \ \ s=(a \hat  u+b'w^k)\ep^{k-1}+O(\ep^k),\mbox{ \ \ or else}\\[4pt]
\label{1IeI1}\!\!\!\!\!\!\!\!\!\!\!\!\!\!\!\!\!\!\!\!&&
u=u_1\ii\ep^{i_0-1},\ \ \ v=d^{-1}w-d^{-1}cu_1\ii\ep^{i_0-1},\ \ \ s\ep=a u_1\ii\ep^{i_0}+b''u_1\ii w^{i_0}\ep^{2i_0}+O(\ep^{2i_0+1}),
\end{eqnarray}for some $b',b'',u,w\in\C_{\ne0}$, $u_1\in\R_{\ne0}$, $k,i_0\in\Z_{>0}$,
one can again find a solution for the inequation \eqref{2DDDbar1}.

Now if $x_0=\xi_0$ (thus $x_1\ne\xi_1$), then the first term of $C_0$ becomes $|s\ep|^2=O(\ep^2)$ and we can easily choose any $u$ with $u\re<0$ to satisfy that $C_0<0$. Similarly, if $x_1=\xi_1$ (thus $x_0\ne\xi_0$), then the second term of $C_0$ becomes $|u\ep|^2=O(\ep^2)$ and we can easily choose $u$ with $(au)\re<0$ (in case $a\ne0$) or $v$ with $(bv)\re<0$ (in case $b\ne0$) to satisfy that $C_0<0$.
This proves Theorem \ref{real-inj-1}.\hfill$\Box$\vskip7pt












\noindent{\it Proof of Theorem \ref{MAINT}.~}Finally we are able to prove
Theorem \ref{MAINT}.
The second assertion of Theorem \ref{MAINT} follows from \cite{K-m1,K-m2}. To prove the first statement, assume conversely that there exists a Jacobian
 pair $(F,G)\in\C[x,y]^2$ satisfying \eqref{wheraraaa}
such that  \eqref{=simag} holds.
Then we have Theorem \ref{real-inj-1}.
Similar to the proof  of  Theorem \ref{real-inj-1},
denote $\DD=\{\dd_{p_0,p_1}\,|\,(p_0,p_1)\in V\}$ [cf.~\eqref{d-1-2-3-4}${\ssc\,}$], and set
$\dd={\rm inf\,} \DD\in\R_{\ge0}$ (the infimum of $\DD$).
By definition, there exists a sequence $(p_{0i},p_{1i}):=\big((x_{0i},y_{0i}),(x_{1i},y_{1i})\big)\in V$, $i=1,2,...$,
 such that $\lim_{i\to\infty} \dd_{p_{0i},p_{1i}}=\dd$. Then $\{x_{0i},x_{1i}\,|\,i=1,2,...\}$ is bounded by \eqref{d-1-2-3-4}.
Thus
$\{y_{0i},y_{1i}\,|\,i=1,2,...\}$ is also bounded by \eqref{mqp1234-2}.
By replacing the sequence by a subsequence, we can then assume \eqref{(qqq)}. Now
arguments after \eqref{(qqq)} show that $(p_0,p_1)\in V$, but $(x_0,x_1)\ne(\xi_0,\xi_1)$ by Theorem \ref{real-inj-1}\,(i), i.e., $\dd>0$. Then by Theorem \ref{real-inj-1}\,(ii), we can then obtain a contradiction with the definition of $\dd$.
This proves Theorem \ref{MAINT}.\hfill$\Box$
%
%
%
\NOUSE{%
%
\section{Generalizations: the $n$-dimensional case}\label{sect1-gen}
%
\def\ff{{\textbf{\textit{f}}}}%
\def\pp{{\textbf{\textit{p}}}}%
\def\qq{{\textbf{\textit{q}}}}%
\def\lL{{\textbf{\textit{l}}}}%
\def\uU{{\textbf{\textit{u}}}}%
\def\sS{{\textbf{\textit{s}}}}%
\subsection{Main results}
Denote $A_n=\C[x_1,...,x_n]$, and
let $\ff=(f_1,...,f_n)\in A_n^n$ be a
{\it Jacobian $n$-tuple}, i.e., the rank $n$ Jacobian determinant
$J(\ff):=\big|\frac{\ptl f_i}{\ptl x_j}\big|_{n\times n}\in\C_{\ne0}$. Let $\sigma:\C^n\to\C^n$ 
be the associated  {\it Keller map},
sending
$\pp\mapsto \ff(\pp)=(f_1(\pp),...,f_n(\pp))$  for $\pp=(p_1,...,p_n)\in{\mathbb C}^n$.
\begin{theo}\label{MAINT-gen}
The Keller map $\si$
 is injective. In particular, the \,$n$-dimensional Jacobian conjecture holds, i.e.,
$A_n=\C[f_1,...,f_n]$.
\end{theo}
%
Let us assume the map $\si$ is not injective, namely,
for some $\pp_i=(x_{i1},...,x_{in})\in{\mathbb C}^n$, $i=0,1$,
\equa{=simag-gen}{\mbox{$\sigma(\pp_0)=\sigma(\pp_1)$, \ \  $\pp_0\ne \pp_1$.}}
Without loss of generality, we may assume, for $i=1,...,n$ and some $m\in\Z_{>1}$,
\equa{fi-sform-gen}{f_i=x_i^m+h_i\mbox{ for some }h_i\in A_n\mbox{ with }\deg\,h_i<m.}
Throughout this section, we denote
\begin{eqnarray}
\label{V=1--gen}&\!\!\!\!\!\!\!\!\!\!\!\!\!\!\!\!\!\!&
(\pp_0,\pp_1)=\big((x_{01},...,x_{0n}),(x_{11},...,x_{1n})\big)\in\C^n\times\C^n\cong\C^{2n},\\[4pt]
\label{V=0-gen}&\!\!\!\!\!\!\!\!\!\!\!\!\!\!\!\!\!\!&
V=\big\{(\pp_0,\pp_1)\in\C^{2n}\,\big|\,\si(\pp_0)=\si(\pp_1),
\,\pp_0\ne \pp_1\big\},\\[4pt]
\label{V=1-gen}&\!\!\!\!\!\!\!\!\!\!\!\!\!\!\!\!\!\!&
V_{\pp}=\big\{(\pp_0,\pp_1)\in V\ \big|\ \pp_1=\pp\big\},
\end{eqnarray}
for $i=0,1$ and any $\pp\in\C^n.$
Then $V\ne\emptyset$ by assumption \eqref{=simag-gen}.
Analogous to Theorem \ref{real-inj-1}, we have the following.
\begin{theo}\label{real-inj-1-gen}\begin{itemize}\item[\rm(i)]
There exist $\pp\in\C^n$ such that $V_{\pp}=\emptyset$.
\item[\rm(ii)]
Fix any $\pp=(p_1,...,p_n)\in\C^n$ satisfying {\rm(i)}.
Denote
$\dd_{\pp_0,\pp_1}=\mbox{$\sum%\limits
_{i=1}^n$}|x_{1i}-p_i|^2$ for $(\pp_0,\pp_1)\in V$.
Then for any $(\pp_0,\pp_1)\in V$, there exists $(\qq_0,\qq_1)\in V$ such that
$\dd_{\qq_0,\qq_1}< \dd_{\pp_0,\pp_1}.$
\end{itemize}\end{theo}
Analogous to Theorem \ref{lemm-a1}, we have the following.
 \begin{theo}\label{lemm-a1-gen} Denote
$h_{\pp_0,\pp_1}=\max\big\{|x_{ij}|\,|\,i=0,1,\,j=1,...,n\big\}$
for $(\pp_0,\pp_1)\in V$.
There exists $\SS_0\in\R_{>0}$
satisfying the following: For any $(\pp_0,\pp_1)\in V$ with
$h_{\pp_0,\pp_1}\ge\SS_0$, we must have, for $i=1,...,n$,
\equa{mqp1234-2-gen}{|x_{0i}^m-x_{1i}^m|< h_{_{\sc \pp_0,\pp_1}}^{{\sc\frac{m^2}{m+1}}}.}
\end{theo}
}%
%
\NOUSE{%
%
%
Assume
\equa{CSweaua-gen}{V_{\pp}\ne\emptyset\mbox{ \ for all \ }\pp\in\C^n.}
Then
analogous to Theorem \ref{real00-inj}, we have
the following.
\begin{theo}\label{real00-inj-gen}
\begin{itemize}\item[\rm(1)]
There exist $\kappa_i,\kappa'_i\in\R_{>0},\,\eta_i\in\R_{\ne0}$
and a permutation $(i_1,...,i_n)$ of $(1,...,n)$
such that the following hold.
\begin{itemize}
\item[\rm(i)]
Denote by $V_0$ the subset of $V$ such that %either
all its elements $(p_0,p_1)\!=\!\big((x_0,y_0),(x_1,y_1)\big)$ simultaneously satisfy
one of \eqref{ToSayas-gen} or \eqref{ToSayas0-gen}. Then $V_0\ne\emptyset$.
\begin{eqnarray}
\label{ToSayas-gen}
 & \!\!\!\!\!\!\! &
{\rm(a)\ }\dis\kappa_0\le
\kappa_{1}|x_{1,i_1}|^{\eta_{1}}
\le\cdots\le\kappa_{n}|x_{1,i_{n}}|^{\eta_{n}}\le\kappa_{n+1},
%\nonumber\\[4pt]
%\!\!\!\!\!\!\!\!\!\!\!\!&\!\!\!\!\!\!\!\!\!\!\!\!\!\!\!\!\!\!\!\!\!\!\!\!\!\!\!\!&
\ \ {\rm(b)\, } \ell_{\pp_0,\pp_1}:=
\frac{|x_{01}|+\kappa_{n+2}}{|x_{1,n}|^{\kappa_{n+3}}}
\!\ge\!\kappa_{n+4},
\\[6pt]
\label{ToSayas0-gen}
 &\!\!\!\!\!\!\!\!\!\!\!\!\!\!\!\! \!\!\!\!\!\!\!\! &
{\rm(a)\ }\dis\kappa_0\!\le\!\kappa'_n|x_{1n}|^{-1}
\!\le\!\cdots\!\le\!\kappa'_3|x_{13}|^{-1}\!\le\!
%\nonumber\\[2pt]
%\!\!\!\!\!\!\!\!\!\!\!\!&\!\!\!\!\!\!\!\!\!\!\!\!\!\!\!\!\!\!\!\!\!\!\!\!\!\!\!\!\!\!\!\!\!\!\!\!&
%\phantom{{\rm(a)\ }\dis1}\le
\kappa_1|x_{12}^3x_{11}x_{01}|^{\kappa_2}
\!\le\! \kappa_3|x_{12}x_{11}^3x_{01}^2|^{\kappa_4}
\!\le\! \kappa_5|x_{12}^3x_{11}x_{01}|^{\kappa_6}
\!\le\!\kappa_{7},
\nonumber\\[4pt]
\!\!\!\!\!\!\!\!\!\!\!\!&\!\!\!\!\!\!\!\!\!\!\!\!\!\!\!\!\!\!\!\!\!\!\!\!\!\!\!\!\!\!\!\!\!\!\!\!&
{\rm(b)\ } \kappa_8\le|x_{11}|\le\kappa_9,\ \ \
{\rm(c)\ }
\ell_{\pp_0,\pp_1}:=
|x_{11}|^2\Big(\frac{|x_{11}^3x_{01}^2|^{\kappa_{10}}}{|x_{12}|^{\kappa_{11}}}+\kappa_{12}\Big)
\!\ge\!\kappa_{13},
\end{eqnarray}
where
$\kappa_i,\kappa'_i,\eta_{i}$
will be chosen such that there exists
$\th_i\in\R_{>0}$ satisfying: when conditions hold, we have, for $i=1,...,n$,
\equa{-EiathA0-gen}{\mbox{$\th_0\le|x_{1i}|\le\th_1$, \ and \
$|y_1|\ge\th_0.$
}}
\item[\rm(ii)]For any $(p_0,p_1)\in V_0$,
the last equality of {\rm\eqref{ToSayas-gen}}\,{\rm(a)} cannot hold, and the first $\,n$
 equalities cannot simultaneously hold;
none of the last equality
of {\rm\eqref{ToSayas0-gen}}\,{\rm(a)} or any equality of {\rm\eqref{ToSayas0-gen}}\,{\rm(b)} can occur, the first $n-1$ equalities
of {\rm\eqref{ToSayas0-gen}}\,{\rm(a)} cannot simultaneously hold, the $n$-th and $(n+1)$-th equalities of
{\rm\eqref{ToSayas0-gen}}\,{\rm(a)} cannot simultaneously hold%, and at most only $n-1$ equalities
%of {\rm\eqref{ToSayas0-gen}}\,{\rm(a)} can simultaneously hold
.\end{itemize}
\item[\rm(2)]
For any $(\pp_0,\pp_1)\in V_0$, there exists $(\qq_0,\qq_1)\in V_0$
such that $\ell_{\qq_0,\qq_1}> \ell_{\pp_0,\pp_1}.$
\end{itemize}\end{theo}
\subsection{Proof of Theorems \ref{lemm-a1-gen} and \ref{real00-inj-gen}\,(1)}
To prove \eqref{mqp1234-2-gen}, assume conversely that there exists $(\pp_0,\pp_0)\in V$ with $h_{\pp_0,\pp_1}=\kk\gg1$ such that
\equa{a0-geegne}{\mbox{$\dis \a_0:=|x_{0i}^m-x_{1i}^m|>h^{{\sc \frac{m^2}{m+1}}}_{_{\sc \pp_{0i},\pp_{1i}}}$ for some $i$.}}
Then by \eqref{fi-sform-gen}, we have $|f_i(\pp_0)-f_i(\pp_1)|\ge\a_0-\a_1$, where \equa{Whwhw-gen}{\mbox{$\dis\a_1=|h_i(\pp_0)-h_i(\pp_1)|\preceq \kk^{m-1}\prec\kk^{\frac{m^2}{m+1}}<\a_0$ (when $\kk\gg1$),}}
a contradiction with the fact that $f_i(\pp_0)=f_i(\pp_1)$. Thus we have Theorem \ref{lemm-a1-gen}.
%
%
Let $\kk\gg1$ and take $\tilde\pp_1=(\eta_1\kk,...,\eta_n\kk)\in\C^n$, where we first simply take  $\eta_n=\cdots=\eta_n=1$. By
\eqref{CSweaua-gen}, there exists $\tilde\pp_0$ written as $\tilde\pp_1=(\eta'_1\kk,...,\eta'_n\kk)\in\C^n$ for some $\eta'_i\in\C$ such that $(\tilde\pp_0,\tilde\pp_1)\in V$. We claim
that for any fixed $\d>0$, we have
\equa{WeHhhave-gen}{|\eta_i|(1-\d^4)\le|\eta'_i|\le|\eta_i|(1+\d^4)\mbox{ for }i=1,...,n.}
Set $\eta=\max\big\{|\eta_i|,|\eta'_i|\,\big|\,i=1,...,n\big\}$.
Assume $\eta=\eta'_j>\eta_j(1+\d^4)$ for some $j$. Then $h_{\tilde \pp_0,\tilde \pp_1}=\eta\kk$, and
\equa{bBut-Gen}{\mbox{$\!\!\!\!\dis|\tilde x_{0j}^m\!-\!\tilde x_{1j}^m|\!=\!|\eta_j'^{{\ssc\,}m}\kk^m\!-\!\eta_j^m\kk^m|
\!\ge\!\big(1\!-\!\frac{1}{(1\!+\!\d^4)^m}\big)\big(|\eta_j'|\kk\big)^m\sim
h_{_{\sc\tilde \pp_0,\tilde \pp_1}}^m\succ h_{_{\sc\tilde \pp_0,\tilde \pp_1}}^{\frac{m^2}{m+1}}$ when $\!\kk\!\gg1$,}} which is a contradiction with
\eqref{mqp1234-2-gen}. This in particular shows that $h_{{\sc\tilde \pp_0,\tilde \pp_1}}\le(1+\d^4)\kk$.
Now assume $|\eta'_\ell|<|\eta_\ell|(1-\d^4)$ for some $\ell$. Then
\equa{1bBut-Gen}{\mbox{$\!\!\!\!\dis|\tilde x_{0\ell}^m\!-\!\tilde x_{1\ell}^m|\!=\!
|\eta_\ell^m\kk^m-\eta_\ell'^{{\ssc\,}m}\kk^m|
\!\ge\!\big(1\!-\!\frac{1}{(1\!+\!\d^4)^m}\big)\big(|\eta_\ell|\kk\big)^m\sim
h_{_{\sc\tilde \pp_0,\tilde \pp_1}}^m\succ h_{_{\sc\tilde \pp_0,\tilde \pp_1}}^{\frac{m^2}{m+1}}$ when $\!\kk\!\gg1$,}}
again a contradiction with \eqref{mqp1234-2-gen}.
Thus \eqref{WeHhhave-gen} hold. In particular, $\eta_i'\ne0$.
If necessary by rearranging variables $x_i$'s [and rearranging $f_i$'s accordingly so that \eqref{fi-sform-gen} holds]
and/or exchanging $\tilde\pp_0$ and $\tilde\pp_1$, we may assume $|\eta_1'|>|\eta_1|$. Denote $\xi_i=|\eta_i|$. This shows that there exists $(\tilde\pp_0,\tilde\pp_1)=\big((\tilde x_{01},...,\tilde x_{0n}),(\tilde x_{11},...,\tilde x_{1n})\big)\in V$ such that, for $i=1,...,n$,
\equa{SuSCHEtheg-gen}{|\tilde x_{1i}|=\xi_i\kk, \ \ \ 1-\d^4\le\xi_i\le1+\d^4,\ \ \ |\tilde x_{01}|>|\tilde x_{11}|.}
%
As before, let us make the following assumption.
\begin{assu}\label{assu-gen}
Assume Theorem {\rm\ref{real00-inj-gen}\,(1)} is not true.\end{assu}
%
Analogous to Theorem \ref{AddLeeme--0}\,(i),\,(ii),
one can define, for $\lL=( l_1,..., l_n)\in\R_{>0}^n$,
\begin{eqnarray}
\label{Ak=-gen}&\!\!\!\!\!\!\!\!\!\!\!\!\!\!&
A_{\lL}=\big\{(\pp_0,\pp_1)\in V\ \big|\ |x_{1i}|= l_i,\,i=1,...,n\big\},\ \ \ \
\g_{\lL}=\max \,\big\{|x_{01}|\ \big|\ (\pp_0,\pp_1)\in A_{\lL}\big\}.
\end{eqnarray}
Then $A_{\lL}$  is a nonempty closed bounded subset of $V_0$
 and
$\g_{\lL}$ is a well-defined
function on $\lL\in\R_{>0}^n$.
We do not need to prove whether or not $\g_\lL$ is an increasing function, however we have the following very crucial fact, which is
analogous to Lemma \ref{G--lemm-assum1--}.
%we have the following.
\begin{lemm}\label{Lemma1-gen}
Let $\d_i,k,k_i,l_i\in\R_{>0}$ except that $\d_1=0$, such that $\d_i<\frac1m$, $k\ge\max\{k_i\,|\,i=1,...,n\}$ and  $k>1$. Then $($we do not have the strict inequality since we do not know whether or not $\g_\lL$ is a strictly increasing function$)$
\equa{equ-Lemma1-gen}{\g_{k_1^{1+\d_1}l_1,...,k_n^{1+\d_n}l_n}\le k\g_{l_1,...,l_n}.
}
\end{lemm}\noindent{\it Proof.~}We arrange $k_i$ in the following order,   for some permutation
$(i_1,...,i_n)$ of $(1,...,n)$,
\equa{FoRMmmm}{\mbox{$\dis k_{i_{n_1}}\le\cdots\le k_{i_1}\le1<k_{i_{n_1+1}}\le\cdots\le k_{i_n}$.}} Assume \eqref{equ-Lemma1-gen} is not true, then we can choose sufficiently small $\widetilde\d_1,\widetilde\d_2>0$ such that (note that $k_{i_n}\le k\ssc\,$)
\equa{2equ-Lemma1-gen}{
\kappa_0l_{i_n}^{1+\widetilde\d_2}-\widetilde\d_1>\g_{l_1,...,l_n},\mbox{ where }
\kappa_0:=\frac{\g_{k_1^{1+\d_1}l_1,...,k_n^{1+\d_n}l_n}\,+\,\widetilde\d_1}{{}^{^{^{}}}(k_{i_n}l_{i_n})^{1+\widetilde\d_2}}.
}
Choose sufficiently small $\th>0$ such that \equa{ksmthe}{\mbox{$\dis
k_{i_{n_1}}^{-\th}<k_{i_{n_1+1}}$ (if $n_1=n$ we regard $k_{i_{n_1+1}}$ as $k$ which is $>1\ssc\,$),}} and let $\kk\gg1$.
We define $V_0$ such that its elements $(\pp_0,\pp_1)$ satisfy the following (cf.~Remark \ref{NotaRemak}).
\begin{eqnarray}\label{V0-1-gen}
&\!\!\!\!\!\!\!\!\!\!\!\!\!\!\!\!\!\!\!\!\!\!\!\!&
{\rm(i)\ }1\le (l_{i_1}|x_{1,i_1}|^{-1})^{\frac{\th}{1+\d_{i_1}}}\le\cdots\le
 (l_{i_{n_1}}|x_{1,i_{n_1}}|^{-1})^{\frac{\th}{1+\d_{i_{n_1}}}}
\nonumber\\[4pt]
&\!\!\!\!\!\!\!\!\!\!\!\!\!\!\!\!\!\!\!\!\!\!\!\!&
\phantom{{\rm(i)\ }1}
 \le
  (l_{i_{n_1+1}}^{-1}|x_{1,i_{n_1+1}}|)^{\frac{1}{1+\d_{i_1}}}\le\cdots\le
   (l_{i_n}^{-1}|x_{1,i_n}|)^{\frac{1}{1+\d_{i_n}}}\le\kk,
\ \ \ \ \ \ \ {\rm(ii)\ }\frac{|x_{01}|+\widetilde\d_1}{|x_{1,i_n}|^{1+\widetilde\d_2}}\ge\kappa_0.
    \end{eqnarray}
    Then we can rewrite the above as the form in \eqref{ToSayas-gen}.
If the first $n$ equalities of \eqref{V0-1-gen}\,(i) hold, then $|x_i|=l_i$, but by \eqref{V0-1-gen}\,(ii) and \eqref{2equ-Lemma1-gen}, we obtain $|x_{01}|>\g_{l_1,...,l_n}$, a contradiction with definition \eqref{Ak=-gen}.
If the last equality of \eqref{V0-1-gen}\,(i) hold, then \equa{ThHHHweww}{\mbox{$\dis|x_{1,i_n}|\sim\kk^{1+\d_{i_n}}$, \ \  $\dis|x_{1,i_j}|\preceq\kk^{1+\d_{i_j}}$ for $j<n$, \ \ \ and \
$|x_{11}|\preceq\kk$ (since $\d_1=0\ssc\,$).}} However by  \eqref{V0-1-gen}\,(ii), we have that $|x_{01}|\succeq\kk^{(1+\d_{i_n})(1+\widetilde\d_2)}\succ\kk\succeq|x_{11}|$ when $\kk\gg1$.
Let $\d=\max\{\d_i\,|\,i=1,...,n\}$. Then $|x_{1i}|\preceq|x_{01}|^{1+\d}$ for $i=2,...,n$.
Thus by \eqref{mqp1234-2-gen} and as in \eqref{mqp1234-2+}, we see that we must have
$h_{\pp_0,\pp_1}\preceq|x_{01}|^{1+\d}$ and $|x_{01}|\sim|x_{11}|$, a contradiction. Thus Theorem {\rm\ref{real00-inj-gen}\,(1)\,(ii)} holds.
%
By definition, there exists $(\tilde\pp_0,\tilde\pp_1)\in V$ with $|\tilde x_{1i}|=k_i^{1+\d_i}l_i$ for $i=1,...,n$ and $|x_{01}|=\g_{k_1^{1+\d_1}l_1,...,k_n^{1+\d_n}l_n}$.
We see that \eqref{V0-1-gen} holds for $(\tilde\pp_0,\tilde\pp_1)$: if we use $(\tilde\pp_0,\tilde\pp_1)$ in \eqref{V0-1-gen}\,(i), then
the inequation becomes \equa{InDeeee}{\mbox{$\dis 1\le k_{i_1}^{-\th}\le\cdots\le k_{i_{n_1}}^{-\th}\le k_{i_{n_1+1}}\le\cdots\le k_{i_n}\le\kk$,}} which indeed holds by \eqref{FoRMmmm} and \eqref{ksmthe}. Thus $V_0\ne\emptyset$, i.e., Theorem
\ref{real00-inj-gen}\,(1) holds, a contradiction with Assumption \ref{assu-gen}.\hfill$\Box$\vskip5pt
%
Exactly as the proof of Lemma \ref{Anlemmmmmm}, we have [in our case here, we cannot obtain the strict inequality as in \eqref{equ-Lemma1-gen}$\ssc\,$]
\equa{MSMDEME}{\g_{l_1,...,l_n}\ge l_1\mbox{ for all }\lL=(l_1,...,l_n)\in\R_{>0}^n.}
%
\NOUSE{
Then analogous to Theorem \ref{AddLeeme--0}\,(iii),
 we have
the following.
\begin{theo}\label{AddLeeme--0-gen}
The
$\g_{\lL}$ is a weakly  increasing function on each variable $ l_i>0$, i.e.,
\begin{eqnarray}
\label{wePPPP1-gen}
&&\!\!\!\!\!\!\!\!\!\!\!\!\!\!\!\!\!\!\!\!\!\!\!\!
\mbox{
$\g_{ l_1,..., l_{i-1}, l'_i, l_{i+1},..., l_n}\ge\g_{\lL}$ if  $ l'_i\ge l_i>0$}.\end{eqnarray}
\end{theo}
%
%
%
%
To prove \eqref{wePPPP1-gen} (we consider the case with $i=1$ as the proof is similar for other cases), assume it is not true. Then there exists $\lL=( l_1,..., l_n)\in\R_{>0}^n$ such that
\equa{G-LLBigGG-gen}{
%{\rm (i)\ }\g_{\lL}=\max\{\g_{ l'_1, l_2,..., l_n}\,|\, l_1-\ep<l'_1\le l_1\},\ \ \ \ {\rm (ii)\ }
\g_{\bar l_1, l_2,..., l_n}<\g_{\lL}
\mbox{ for some $\bar l_1$ with $ l_1<\bar l_1< l_1+\ep$, where $0<\ep\ll1$.}} In particular $\g_{\lL}>0$.
%
}%
%
%
%
%
%
Let $\kk\gg1$. Take, for $i=1,...,n$ [where $\xi_i$'s are as in \eqref{SuSCHEtheg-gen}$\ssc\,$],
\equa{MSAMSMS-gen}
{\mbox{$(\tilde \pp_0,\tilde \pp_1)\in V$ with $|\tilde x_{1i}|=\xi_i\kk$, \ $|\tilde x_{01}|=\g_{\xi_1\kk,...,\xi_n\kk}>\xi_1\kk$,}}
where the strict inequality follows from \eqref{SuSCHEtheg-gen}. By \eqref{WeHhhave-gen}, we have
%Then similar to Lemma \ref{YYYy1==}, we have, for any fixed sufficiently small $\d>0$ [if $\g_{\kk,...,\kk}>(1+\d^4)\kk$, then we must have $h_{\tilde\pp_0,\tilde\pp_1}=\g_{\kk,...,\kk}$, and $|x_{0i}^m-x_{1i}^m|\ge\g_{\kk,...,\kk}^m\big(1-(1+\d^4)^m\big)\succ h_{_{\sc\tilde\pp_0,\tilde\pp_1}}^{{\sc\frac{m^2}{m+1}}}$, a contradiction with
%\eqref{mqp1234-2-gen}$\ssc\,$],
\equa{142-MSAMSMS-gen}{\xi_1\kk<\g_{\xi_1\kk,...,\xi_n\kk}\le\xi_1(1+\d^4)\kk.}
As in \eqref{-ANewf0F1++}, we set $\ff^{(i)}=(f_1^{(i)},...,f_n^{(i)})$ for $i=0,1$,  with
[thus we have two Jacobian $n$-tuple $\ff^{(0)}$ and $\ff^{(1)}\ssc\,$],
\equa{-ANewf0F1++-gen}{
f_j^{(0)}=f_j\big(\tilde x_{01}(1+x_1),\tilde x_{02}+x_2,...,\tilde x_{in}+x_n\big),\ \
f_j^{(1)}=f_j\big(\tilde x_{11}(1+x_1),...,\tilde x_{in}(1+x_n)\big).}
As in \eqref{4-1}, we can assume, for $j=1,...,n$
[we only write the linear part; we just remind  that $\ff^{(i)}$ for $i=0,1$, may not satisfy \eqref{fi-sform-gen} as we may need to replace them by $\ff^{(i)}A$ for some nondegenerate $n\times n$ matrix $A$ as in the proof of Theorem \ref{AddLeeme--0}\,(iii), but anyway we do not need $\ff^{(i)}$ to satisfy \eqref{fi-sform-gen}, we only use $\ff^{(i)}$ for $i=0,1,$ in the local bijectivity of Keller maps to show that there exists $(\qq_0,\qq_1)$ which will satisfy our requirement to be specified in the context],
\begin{eqnarray}
\label{4-1-gen}&\!\!\!\!\!\!\!\!\!\!\!\!\!\!\!\!\!\!\!\!\!\!\!\!\!\!\!\!\!\!&
f^{(0)}_j=x_j
, \ \ \ \ \ f^{(1)}_j=\mbox{$\sum\limits_{k=1}^n$}a_{jk}x_k,
\end{eqnarray}
for some $a_{jk}\in\C$. For any $\sS=(s_1,...,s_n),\uU=(u_1,...,u_n)\in\C^n$, as in \eqref{q0q1}, we denote
$\qq_i=(\dot x_{i1},...,\dot x_{in})$, $i=0,1$, with, for $j=2,...,n,\,k=1,...,n$,
\equa{q0q1-gen}{\dot x_{01}=\tilde x_{01}(1+s_1\ep),\ \ \ \ \ \
\dot x_{0j}=\tilde x_{0j}\!+\! s_j\ep,\ \ \ \ \ \
\dot x_{1k}=\tilde x_{1k}(1+u_k\ep).}
Then as in \eqref{st=}, in order for $(\qq_0,\qq_1)\in V$, we can regard $\uU$ as free and solve $\sS$ to obtain 
(we only write down $s_1$ as we do not need
$s_j$ with $j>1$, where $a_j=a_{1j}$ and some $b_{ij}\in\C$, they can depend on $\kk\ssc\,$),
\equa{st=-gen}{s_1=\mbox{$\sum\limits_{i=1}^n$}a_iu_i+
\mbox{$\sum\limits_{1\le i\le j\le n}$}b_{ij}u_iu_j\ep
+O(\ep^2).
}
\begin{lemm}\label{2-LemmA-gen}For any $\d,
\d_i\ge0$ with $\d_i<\frac1m$ and $\d_1=0,$ we have, for $i=1,...,n$ $[$we do not have the strict inequality in \eqref{Euq2-LemmA-gen}$\,{\rm(i)}{\ssc\,}]$,
\begin{eqnarray}\label{Euq2-LemmA-gen}
{\rm(i)\ }a_i\ge0,\ \ \ \ %\nonumber\\[4pt]
{\rm(ii)\ }%a_j>0\mbox{ for some }j>1,\ \ \  \ {\rm(iii)\ }
\mbox{$\sum\limits_{i=1}^n$}(1+\d_i)a_i<1,\ \ \ \ {\rm(iii)\ }\mbox{$\sum\limits_{i=1}^n$}a_i>1-\d
.
\end{eqnarray}
\end{lemm}\noindent{\it Proof.~}{(i)\ }To simplify notations and without loss of generality, assume $a_1\notin\R_{\ge0}$. We take $u_i=0$ for $i>1$, and take $u_1\in\C$ to satisfy [cf.~Convention \ref{conv1}\,(1) for notations ``\,${}\re$\,'', ``\,${}\im$\,'']: \equa{MSMndeneene}{\!\!\!\mbox{(a) \ $u_1<0$ if $a_{1\rm\,re}<0$, \ \ or else (b) \ $|1+u_1\ep|=1$ and $(a_1u_1)\re>0$ [cf.~\eqref{SMSMDndndnd}$\ssc\,$] if $a_{1\rm\,im}\ne0$.}\!\!} Then by \eqref{q0q1-gen}, we have
$|\dot x_{11}|=k_1|\tilde x_1|=k_1\xi_1\kk$ with $0<k_1:=|1+u_1\ep|<1$ in case (a) or $k_1=1$ in case (b), and
$|\dot x_{1i}|=|\tilde x_{1i}|=k_i\xi_i\kk$ with $k_i=1$ for $i=2,...,n$, and
\equa{Ansnns-gen}{\mbox{$\!\!\!\!\!\dis |\dot x_{01}|=|\tilde x_{01}|\big(1+(a_1u_1)\re\ep+O(\ep^2)\big)=k_0\g_{\xi_1\kk,...,\xi_n\kk},$ where $k_0:=1+(a_1u_1)\re\ep+O(\ep^2)>1$.}\!\!\!\!} Take $k>1$ with $k<k_0$. Then $k\ge\max\{k_i\,|\,i=1,...,n\}$. By definition \eqref{Ak=-gen}, we have $\g_{k_1\xi_1\kk,...,k_n\xi_n\kk}\ge|\dot x_{01}|>k\g_{\xi_1\kk,...,\xi_n\kk}$, a contradiction with Lemma \ref{Lemma1-gen}.
%
{\rm(ii) }%
%
\NOUSE{%
%
Assume $a_i=0$ for $i=2,...,n$.
Note that $s_1$ is a power series of $u_1,...,u_n$.
As in the proof of Theorem \ref{AddLeeme--0}\,(iii) [cf.~\eqref{s=a} and \eqref{AMSMSM23333}$\ssc\,$], we may assume that we have two possible cases:
we can either find \begin{itemize}\item[(a)]a term $u_\a:=b_\a u_1^{\a_1}\cdots u_n^{\a_n}\ep^{|\a|-1}$ in $s_1$ for some $b_\a\in\C_{\ne0}$ and $\a=(\a_1,...,\a_n)\in\Z_{\ge0}^n$ with $|\a|:=\sum_{i=1}^n\a_i\ge2$ being minimal and $\a_1=0$; or else
   \item[(b)]a term $u_\b=b_\b u_1u_2^{\b_2}\cdots u_n^{\b_n}\ep^{|\b|-1}$ in $s_1$ for some
$b_\b\in\C_{\ne0}$ and $\b=(\b_1,\b_2,...,\b_n)\in\Z_{\ge0}^n$ with $|\b|\ge2$ being minimal and $\b_1=1$ such that $s_1$ does not contain any term $u_\gamma=b_\gamma u_1^{\gamma_1}\cdots u_n^{\gamma_n}$ with $\gamma_1=0$, $|\gamma|\le2|\b|$ and $b_\gamma\ne0$.
\end{itemize}
In case (a), we take $u_i=0$ if $\a_i=0$, or else $u_i=u_0$, where $u_0\in\C_{\ne0}$ satisfies that $|1+u_0\ep|=1$ and $(b_\a u_0^{|\a|})\re>0$
 [cf.~\eqref{SMSMDndndnd}$\ssc\,$]. Then $|\dot x_i|=\kk$ for $i=1,...,n$, but \equa{BuTTTTT}{\mbox{$\dis|\dot x_{01}|=\big(1+(b_\a u_0^{|\a|})\re\ep^{|\a|}+O(\ep^{|\a|+1})\big)\g_{\kk,...,\kk}>\g_{\kk,...,\kk}$,}} a contradiction with definition \eqref{Ak=-gen}.
  In case (b), we take, for $i=2,...,n$ (where $\ii=\sqrt{-1}\ssc\,$),
\equa{TTTTTata-gen}{u_1=\ii\ep^{|\b|},\ \ \ u_i=\left\{\begin{array}{cc}0&\mbox{ if }\b_i=0,\\[4pt]u_0&\mbox{ otherwise},\end{array}\right.}
where $u_0\in\C_{\ne0}$ satisfies that $|1+u_0\ep|=1$ and $(b_\a\ii u_0^{|\b|})\re\ge2$.
Then $k_1:=|1+u_1\ep|=1+\ep^{2|\b|}+O(\ep^{2|\b|+1})>1$, and %for $i>1$, we take $u_i=0$ if $\b_i=0$ or $u_i=u_0$ if $\b_i\ne0$,
%Then
\equa{x01=====}{\mbox{$\dis|\dot x_{01}|=k_0\g_{\kk,...,\kk},\mbox{ where }k_0=1+(b_\a\ii u_0^{|\b|})\re\ep^{2|\b|}+O(\ep^{2|\b|+1})>k_1.$}}
Take $k_i=1$ for $i>1$, and $k>1$ with $k_1<k<k_0$. Then by definition \eqref{Ak=-gen}, we have $\g_{k_1\kk,...,k_n\kk}\ge|\dot x_{01}|>k\g_{\kk,...,\kk}$, a contradiction with Lemma \ref{Lemma1-gen}.
%
%
{(iii)\ }%
%
%
}%
Assume $\kappa_0:=\sum_{i=0}^m(1+\d_i)a_i\ge1$. 
We arrange $\d_i$ in the following order, for some permutation $(i_1,...,i_n)$ of $(1,...,n$) with $i_n=1$ (note that $\d_1=0\ssc\,$),
\equa{DE1-gen}{0\le\d_{i_n}\le\cdots\le\d_{i_1}.}
Take \equa{Tsksks-geeen}{\mbox{$\a=2\sum\limits_{1\le i\le j\le n}|b_{ij\,\rm re}|(1+\d_i)(1+\d_j)+\sum\limits_{i=1}^n\d_i>0$.}} Let $\ell\gg\kk$.
We define $V_0$ such that its elements $(\pp_0,\pp_1)$ satisfy the following  (we assume $0<\ep\ll\ell^{-\ell}$, cf.~Remark \ref{NotaRemak}),
\begin{eqnarray}
\label{-1-Let-gen}
&\!\!\!\!\!\!\!\!\!\!\!\!\!\!\!\!&
{\rm(a)\ }
1\le\big((\xi_{i_n}\kk)^{-1}|x_{1,i_n}|\big)^{\frac1{1+\d_{i_n}}}
\le\cdots\le\big((\xi_{i_1}\kk)^{-1}|x_{1,i_1}|\big)^{\frac1{1+\d_{i_1}}}\le\ell,\nonumber\\[4pt]
&\!\!\!\!\!\!\!\!\!\!\!\!\!\!\!\!&
{\rm(b)\ }A_1:=\frac{\g_{\xi_1\kk,...,\xi_n\kk}^{-1}|x_{01}|+\ep^4}{\big((\xi_{i_1}\kk)^{-1}|x_{1,i_1}|\big)^{\frac{1-\a\scep}{1+\d_{i_1}}}}\ge1+\ep^3.
\end{eqnarray}
Then we can rewrite the above as the form in \eqref{ToSayas-gen}.
If the first $n$ equalities hold in \eqref{-1-Let-gen}\,(a), then $|x_{1i}|=\xi_i\kk$ for $i=1,...,n$, but 
$|x_{01}|>\g_{\xi_1\kk,...,\xi_n\kk}$ by
\eqref{-1-Let-gen}\,(b), a contradiction with definition \eqref{Ak=-gen}. If the
last equality holds in \eqref{1-Let-gen}\,(i), then we have [using \eqref{-1-Let-gen}\,(b) and the fact that $i_n=1$ and thus $(\xi_{1}\kk)^{-1}|x_{11}|=\big((\xi_{i_n}\kk)^{-1}|x_{1,i_n}|\big)^{\frac1{1+\d_{i_n}}}\le\ell\ssc\,$],
\equa{ThHHSHS-gennnn}{\mbox{$
\dis|x_{1,i_1}|=\xi_{i_1}\kk\ell^{1+\d_{i_1}}, \ \ \ \ \ |x_{01}|>\g_{\xi_1\kk,...,\xi_n\kk\sc\,}\ell^{1-\a\scep},\ \ \ \ \ |x_{11}|\le\xi_1\kk\ell.$}}
Using \eqref{-1-Let-gen}\,(a) and \eqref{mqp1234-2-gen} [and cf.~\eqref{mqp1234-2+}$\ssc\,$], we see that $h_{\pp_0,\pp_1}\preceq\kk\ell^{1+\bar\d}$
when $\ell\gg\kk\gg1$, where $\bar\d=\max\{\d_i\,|\,i=1,...,n\}<\frac1m$. However, by 
\eqref{ThHHSHS-gennnn}, we have
\equa{1+ThHHSHS-gennnn}{\dis|x_{01}^m-x_{11}^m|\ge(\xi_1\kk\ell)^m
\Big(\frac{\g_{\xi_1\kk,...,\xi_n\kk}^m}{(\xi_1\kk)^m}\ell^{-\a m\scep}-1\Big)>\a_1(\xi_1\kk\ell)^m\sim\ell^m\succ h_{_{\sc\pp_0,\pp_1}}^{\frac{m^2}{m+1}}\mbox{ when }\ell\gg\kk,
}
for some $\a_1\in\R_{>0}$ (which can depend on $\kk$),
where the strict inequality follows by noting the following: We have $\ell^{-\a m\scep}>1-(\a m+1)\ln(\ell)\ep$ [where $\ln(\cdot)$ is the natural logarithmic function], and by \eqref{MSAMSMS-gen}, $\frac{\g_{\xi_1\kk,...,\xi_n\kk}^m}{(\xi_1\kk)^m}=1+\a_0$ for some $\a_0>0$ (which can depend on $\kk$). Take $\a_1=\frac{\a_0}{2}$, then
\equa{ThUSLast=-gen}{\mbox{$\dis\frac{\g_{\xi_1\kk,...,\xi_n\kk}^m}{(xi_1\kk)^m}\ell^{-\a m\scep}-1\ge(1+\a_0)\Big(1-(\a m+1)\ln(\ell)\ep\Big)-1>\frac{\a_0}{2}=\a_1$ when $\ep^{-1}\gg\ell$.}}
Now \eqref{1+ThHHSHS-gennnn} is a contradiction with \eqref{mqp1234-2-gen}. Thus we have Theorem \ref{real00-inj-gen}\,(1)\,(i).
%
%
Take $u_i=1+\d_i$ for $i=1,...,n$. By \eqref{MSAMSMS-gen}, \eqref{q0q1-gen} and \eqref{DE1-gen}, we have 
\begin{eqnarray}
\label{Mmdme-gennn}
\!\!\!\!\!\!\!\!\!\!\!\!\!\!\!\!&&
\big((\xi_{i_j}\kk)^{-1}|\dot x_{1,i_j}|\big)^{\frac1{1+\d_{i_j}}}=C_1:=|1\!+\!(1\!+\!\d_{i_j})\ep|^{\frac1{1+\d_{i_j}}}
\nonumber\\[4pt]
\!\!\!\!\!\!\!\!\!\!\!\!\!\!\!\!&&\phantom{\big((\xi_{i_j}\kk)^{-1}|\dot x_{1,i_j}|\big)^{\frac1{1+\d_{i_j}}}}
\le
C_2:=|1\!+\!(1\!+\!\d_{i_{j+1}})\ep|^{\frac1{1+\d_{i_{j+1}}}}
=\big((\xi_{i_{j+1}}\kk)^{-1}|\dot x_{1,i_{j+1}}|\big)^{\frac1{1+\d_{i_{j+1}}}}
,\end{eqnarray}
i.e., \eqref{-1-Let-gen}\,(a) holds for $(\qq_0,\qq_1)$, where the inequality follows by noting that the equality holds if $\d_{i_j}=d_{i_{j+1}}$, and
if $\d_{i_j}<d_{i_{j+1}}$, then $C_2-C_1=(\d_{i_{j+1}}-\d_{i_j})\ep^2+O(\ep^3)>0$.
Note from \eqref{st=-gen} that 
(the imaginary part $b_{ij\rm\,im}$ of $b_{ij}$ can only contribute to $\ep^4$ elements to $|1+s_1\ep|\ssc\,$),
\equa{NoteFROM=-gen}{\mbox{$\dis |1+s_1\ep|=1+\kappa_0%\mbox{$\sum\limits_{i=1}^n$}(1+\d_i)a_i\ep
+\a'\ep^2+O(\ep^3)$, where $\dis\a'=\mbox{$\sum\limits_{1\le i\le j\le n}$}b_{ij\rm\,re}(1+\d_i)(1+\d_j)$.}}
Denote $A_1$ by $\dot A_1$ when $x_{ij}$ is replaced by $\dot x_{ij}$.
Then $\Coeff(\dot A_1,\ep)=\kappa_0-\frac{u_{i_1}}{1+\d_i}=\kappa_0-1>0$ if $\kappa_0>1$. If $\kappa_0=1$, then $\Coeff(\dot A_1,\ep)=0$ and
by \eqref{Tsksks-geeen} and \eqref{NoteFROM=-gen}, we have \equa{Anfdd-gen}{\mbox{$\Coeff(\dot A_1,\ep^2)=\a'+\a+\frac{\d_{i_1}}{2}>0$.}}
Thus \eqref{-1-Let-gen}\,(b) holds for $(\qq_0,\qq_1)$. We obtain a contradiction with Assumption \ref{assu-gen}.
%
%
\NOUSE{%
Using \eqref{Euq2-LemmA-gen}\,(i),\,(ii), by replacing $\d_i$ with some $\d'_i>\d_i$ and $\d'_i<\frac1m$ for some $i>1$ with $a_i>0$, we may assume $\kappa_0>1$.
Take $u_i=1+\d_i$ for $i=1,...,n$ (note that $\d_1=0\ssc\,$). Then
\equa{msms-gen}{\mbox{$\dis|\dot x_{1i}|=|1+u_i\ep|\kk=k_i^{1+\d_i}\kk$, \ \ \ and  \ $\dis|\dot x_{01}|=k_0\g_{\kk,...,\kk}$,
}}
 with $k_i:=\big(1+(1+\d_i)\ep\big)^{\frac1{1+\d_i}}\le1+\ep$ and $ k_0=|1+s_1\ep|=1+\kappa_0\ep+O(\ep^2)>1+\ep$.
Take $k>1$ with $k<k_0$.
Then we obtain that $\g_{k_1^{1+\d_1}\kk,...,k_n^{1+\d_n}\kk}\ge|x_{01}|>k\g_{\kk,...,\kk}$, again a contradiction with Lemma \ref{Lemma1-gen}.
%
}%
%
%
{\rm(iii)\ }Assume $\sum_{i=1}^n a_i\le1-\d$. We define $V_0$ such that its elements $(\pp_0,\pp_1)$ satisfy the following (cf.~Remark \ref{NotaRemak}),
\begin{eqnarray}
\label{1-Let-gen}
\!\!\!\!\!&\!\!\!\!\!\!\!\!\!\!\!\!\!\!\!\!\!\!\!\!\!&
{\rm(a)\ }
1\le\xi_1\kk|x_{1n}|^{-1}\le\cdots\le\xi_n\kk|x_{11}|^{-1}\le(1-\d)^{-1},\ \ {\rm(b)\ }\frac{\g_{\xi_1\kk,...,\xi_n\kk}^{-1}|x_{01}|+\ep^3}{\big((\xi_1\kk)^{-1}|x_{11}|\big)^{1-\d+\d^2}}\ge1+\ep^2.
\end{eqnarray}
If the first $n$ equalities hold in \eqref{1-Let-gen}\,(a), then $|x_{1i}|=\xi_i\kk$ for $i=1,...,n$, but $|x_{01}|>\g_{\xi_1\kk,...,\xi_n\kk}$ by
\eqref{1-Let-gen}\,(b), a contradiction with definition \eqref{Ak=-gen}. If the
last equality holds in \eqref{1-Let-gen}\,(a), then $|x_{11}|=(1-\d)\xi_1\kk$, $|x_{1i}|\le\xi_i\kk$ for $i>1$, but by
\eqref{1-Let-gen}\,(b), \equa{ButBY-gen}{\mbox{$\g_{\xi_1\kk,...,\xi_n\kk}^{-1}|x_{01}|>(1-\d)^{1-\d+\d^2}=1-\d+\d^2+O(\d^3)$.}}
Thus by \eqref{MSAMSMS-gen}, we have $|x_{01}|>\big(1+\d^2+O(\d^3)\big)|x_{11}|$, and we can obtain a contradiction as in the proof of \eqref{WeHhhave-gen}. Thus we have Theorem \ref{real00-inj-gen}\,(1)\,(i).
%
Take $u_i=-1$, then \eqref{1-Let-gen}\,(a) trivially holds for  $(\qq_0,\qq_1)$, further the coefficient of $\ep^1$ in the left-hand side of \eqref{1-Let-gen}\,(b) is
$1-\d+\d^2-\sum_{i=1}^n a_i\ge\d^2>0$, i.e.,
\eqref{1-Let-gen}\,(b) holds for $(\qq_0,\qq_1)$. We obtain a contradiction with Assumption \ref{assu-gen}.
\hfill$\Box$\vskip5pt
%
By Lemma \ref{2-LemmA-gen}, we in particular obtain that $0\le a_1<1$, and (by choosing $\d$ in the lemma to be $\d^2$, and $\d_i$ to be $\d\ssc\,$), 
%
\equa{MSmsms-gen}{\mbox{$\sum\limits_{i=1}^n a_i>1-\d^2>(1-\d^2)\Big(a_1+\sum\limits_{i=2}^n(1+\d)a_i\Big)$,}}
which, similar to \eqref{Akk-bkk}, implies that $\lim_{\kk\to\infty}a_1=1$ and $\lim_{\kk\to\infty}a_i=0$ for $i>1$.
Now denote, for $i=1,...,n$ [cf.~\eqref{MSAMSMS-gen} for notation $\tilde x_{ij}\ssc\,$],
\equa{XnEW-gen}{\mbox{$\dis X_i=\tilde x_{1i}^{-1}x_{1i},\ \ \ Y_1=\tilde x_{01}^{-1}x_{01},\ \ \ v_0=1-(a_1+a_2)>0$.}}
Thus we can assume $v_0<\d$ for any fixed $\d>0$. Choose $\ell\in\R_{>0}$ with $\ell\gg\kk^{\kk}$ and assume $0<\ep\ll\ell^{-\ell}$.
Similar to \eqref{LetNSoOP}, we define $V_0$ such that its elements $(\pp_0,\pp_1)$ satisfy the following (cf.~Remark \ref{NotaRemak}),
\begin{eqnarray}
\label{Let-gen}
&\!\!\!\!\!\!\!\!\!\!\!\!\!\!\!\!&
{\rm(i)\ }
1\le|X_n|^{-1}\le\cdots\le|X_3|^{-1}\le C_1:=|X_2^3X_1Y_1|^{\frac1{5-2v_0}-\ell^{-5}}\le C_2:=|X_2X_1^3Y_1^2|^{\frac1{6-2v_0}}
\nonumber\\[4pt]
&\!\!\!\!\!\!\!\!\!\!\!\!\!\!\!\!&
\phantom{{\rm(i)\ }1}
\le C_3:=|X_2^3X_1Y_1|^{\frac1{5-2v_0}+\ell^{-5}}\le\ell,\nonumber\\[4pt]
&\!\!\!\!\!\!\!\!\!\!\!\!\!\!\!\!&
{\rm(ii)\ }\ep\le|X_1|\le\ep^{-1},\ \ \ \ {\rm(iii)\ }
|X_1|^2\Big(\frac{|X_1^3X_2Y_1^2|^{\frac3{6-2v_0}+\ell^{-5}}}{|X_2|^5}+\ep^4\Big)\ge1+\ell^{-5}\ep.
\end{eqnarray}Then we can rewrite \eqref{Let-gen} as the form in \eqref{ToSayas0-gen}.
When \eqref{Let-gen}\,(i) holds, we have $|X_i|\le1$ for $i\ge3$, i.e., $|x_{1i}|\le\xi_i\kk$. As in \eqref{cLtHHH}, we have
$\max\{|X_1|,|Y_1|\}\ge\d$, which implies that $|x_{1i}|\preceq\kk\preceq\max\{|x_{11}|,|x_{01}|\}$ when $\kk\gg1$. Thus as 
 in Lemma \ref{NeLmeme}, one can easily obtain the following, for any fixed $\d>0$,
\equa{Msksk-gen}{\ep\le|X_2|\le\ep^{-1},\ \ \ \ \ \ (1-\d^5)|X_1|\le|Y_1|\le(1+\d^5)|X_1|.}
If the first $n-1$ equalities, or the $n$-th and $(n+1)$-th equalities, hold in \eqref{Let-gen}\,(i), then $|X_3|=\cdots=|X_n|=C_1=C_2=C_3=1$, and we can then obtain that $|X_1|^2=|X_2|^5$, but \eqref{Let-gen}\,(iii) shows that $|X_2|^5<|X_1|^2$, a contradiction.
If the last equality of \eqref{Let-gen}\,(i) holds, the same arguments after \eqref{ThEhHoo} can give a contradiction (noting the fact that $|x_{1i}|\le\xi_i\kk$ for $i\ge3\ssc\,$).
As in the proof of Lemma \ref{TheoHold},
no equality of \eqref{Let-gen}\,(ii) can hold.
Thus Theorem
\ref{real00-inj-gen}\,(1)\,(i) holds.
%
If we take $u_3=\cdots=u_n=0$, $u_1=u_2=1$, then $s_1=a_1+a_2+O(\ep^1)=1-v_0+O(\ep^1)$, as in \eqref{MSMSMSMSMS}--\eqref{mememememe}, we see that
\eqref{Let-gen} holds, i.e., $V_0\ne\emptyset$. We obtain a contradiction with Assumption
\ref{assu-gen}. This shows that Assumption \ref{assu-gen} is wrong, i.e., we have Theorem \ref{real00-inj-gen}\,(1).
%
\NOUSE{%
(such $u_0$ always exists since $|\a|\ge2\ssc\,$).
%
We want to choose suitable $\uU$ such that, for $j=2,...,n$,
\begin{eqnarray}
\label{@suchthat=2-for-gen}&\!\!\!\!\!\!\!\!\!\!\!\!\!\!\!\!\!\!\!\!\!\!\!\!&
{\rm (i)\ }|\dot x_{1j}|=|\tilde x_{1j}|\cdot|1+u_j\ep|=|\tilde x_{1j}|,\ \ \ \ {\rm(ii)\ }
|\dot x_{11}|=|\tilde x_{11}|\cdot|1+u_1\ep|\le|\tilde x_{11}|,\nonumber\\[4pt]
&\!\!\!\!\!\!\!\!\!\!\!\!\!\!\!\!\!\!\!\!\!\!\!\!&
{\rm(iii)\ }|\dot x_{01}|=|\tilde x_{01}|\cdot|1+s_1\ep|>|x_{01}|.\end{eqnarray}
Equ.~\eqref{G-LLBigGG-gen}\,(ii) with the fact that $\bar l_1> l_1$ implies that we can choose some $\uU$
[denoted  as  $\uU'=(u_1',...,u'_n)\ssc$]
satisfying
\eqref{@suchthat=2-for-gen}\,(I) and
$|\dot x_{11}|>|\tilde x_{11}|$ (which means that $u'_{1\rm\,re}>0$) such that $|\dot x_{01}|<|x_{01}|$
(which means that $s_{1\rm\,re}<0$).
Now we fix $u_j=u'_j$ for $j=2,...,n$. Then
 $s_1$ becomes a power series function of $u_1$, written as
\equa{s-1-AsASS}{s_1=\a_1 u_1^k\ep^{k-1}+O(\ep^k),}
where $k\in\Z_{\ge1}$ is smallest such that $\a_1\ne0$ (if.
First assume $k=1$. Then using the fact that when $u_1=u'_1$ we have $s_{1\rm\,re}<0$, we obtain that when $u_1=-u'_1$ (thus $u_{1\rm\,\re}<0\ssc\,$) we have $s_{1\rm\,re}>0$, i.e., we have a solution for \eqref{@suchthat=2-for-gen}.
%
%
%
Thus we can choose $\uU$ satisfying \eqref{@suchthat=2-for-gen}\,(I) and
$|\dot x_{11}|<|\tilde x_{11}|$ such that $|\dot x_{01}|>|x_{01}|$
[noting that we can fix $u_2,...,u_n$ satisfying \eqref{@suchthat=2-for-gen}\,(I) so that
and that \eqref{@suchthat=2-for-gen}\,(II) means that $\tilde x_{11}\ne0$ (as otherwise it would imply that it holds for all $u_1$ (with $|u_1|$ being sufficiently small), which is impossible),
we can find the solution of $u_1$ to
satisfy \eqref{@suchthat=2-for-gen}$\ssc\,$]. We obtain a contradiction with \eqref{G-LLBigGG-gen}\,(i).
This proves Theorem \ref{AddLeeme--0-gen}.
}%
%
\subsection{Proof of Theorem \ref{real00-inj-gen}\,(2)}
We consider two cases.
%\vskip2pt
%
\noindent{\bf Case 1}: {\it First assume we have case $\eqref{ToSayas-gen}$.}
Define $\ff^{(i)}$ for $i=0,1$ as in \eqref{-ANewf0F1++-gen} (with $\tilde x_{ij}$ replaced by $x_{ij}\ssc\,$).
Then we have \eqref{4-1-gen}--\eqref{st=-gen} (we remark that the $\ep$ here is much smaller than the previous $\ep$, cf.~Remark \ref{FaMREE}).
We want to choose suitable $\uU$ such that \eqref{ToSayas-gen} holds by $(\qq_0,\qq_1)$.
Assume we have the maximal number of equalities in \eqref{ToSayas-gen}\,(a) (if the number of equalities is less, then the number of inequations we need to consider is less, and therefore the proof is easier), i.e., assume all equalities hold except the last and the $n_1$-th for some $n_1\le n$.
We take $u_{i_j}=0$ for $j<n_1$ [so that the first $n_1-1$ inequalities of \eqref{ToSayas-gen}\,(a) automatically hold for $(\qq_0,\qq_1)\ssc\,$].
Similar to \eqref{SetV}, we regard $u_{i_n}$ as a free variable, and for $j=n-1,...,n_1$,
we take
\equa{u-i-gen}{\mbox{$\dis
u_{i_j}=\frac{\eta_{i_{j+1}}}{\eta_{i_j}}\Big(u_{i_{j+1}}+\frac{\eta_{i_{j+1}}-\eta_{i_j}}{2\eta_{i_j}}u_{i_{j+1}}^2 \ep\Big)+w_j\ep$,}}
for some fixed $w_j\in\C$ with $(\eta_{i_j}w_j)\re<0$, then
\equa{Mth-gennn}{\mbox{$\dis \frac{\kappa_{i_{j+1}}|\dot x_{1,i_{j+1}}|^{\eta_{i_{j+1}}}}{\kappa_{i_j}|\dot x_{1,i_j}|^{\eta_{i_j}}}=1-(\eta_{i_j}w_j)\re\ep^2+O(\ep^3)>1$,}} i.e., \eqref{ToSayas-gen}\,(a) holds. Then
$\dot x_{01}$ becomes $\dot x_{01}=x_{01}\big(1+\a_1u_{i_n}\ep+\a_2u_{i_n}^2\ep^2+w\ep^2+O(\ep^3)\big)$ for some $\a_i,w\in\C$ (where $w$ depends on $w_j$'s).
As in the proof of Theorem \ref{real00-inj}\,(2), we can always choose suitable $u_{i_n}$ to satisfy [simply because of the crucial fact that $\kappa_{n+2}>0$, cf.~arguments after \eqref{C12BsM}$\ssc\,$]
\equa{memermfrn}{\dis\frac{|\dot x_{01}|+\kappa_{n+2}}{|\dot x_{1,i_n}|^{\kappa_{n+3}}}
=
\frac{|x_{01}|\big(1+\a_1u_{i_n}\ep+\a_2u_{i_n}^2\ep^2+w\ep^2\big)+\kappa_{n+2}}{\big(|x_{1,i_n}|(1+u_{i_n}\ep)\big)^{\kappa_{n+3}}}+O(\ep^3)>
\frac{|x_{01}|+\kappa_{n+2}}{|x_{1,i_n}|^{\kappa_{n+3}}}.
}
This proves  Theorem \ref{real00-inj-gen}\,(2) for case \eqref{ToSayas-gen}.
%\vskip2pt
%
\noindent{\bf Case 2}: {\it Now assume we have case  $\eqref{ToSayas0-gen}$}.
Before proceeding our proof, we wish to persuade readers that we have $n$ free variables $u_1,...,u_n$ to solve at most $n$ inequations which have been designed to be  compatible, thus there are always solutions for $u_1,...,u_n$.
%
As in Case 1, assume the maximal number of equalities in \eqref{ToSayas0-gen}\,(a) hold.
By Theorem \ref{real00-inj-gen}\,(1)\,(ii), we can assume that there exists some $n_1\in\Z_{\ge1}$ with $n_1\le n$ such that all equalities hold in \eqref{ToSayas0-gen}\,(a) except the $(n-n_1+1)$-th, the $(n+1)$-th and the last
[the proof is similar if the  $(n+1)$-th is replaced by the  $n$-th], namely (where all weak inequalities become equalities),
\begin{eqnarray}
\label{i.e.-gen1}
&\!\!\!\!\!\!\!\!\!\!\!\!\!\!\!\!\!\!\!\!&
1\le|x_{1n}|^{-1}\le\cdots\le|x_{1,n_1-1}|^{-1}<|x_{1,n_1}|^{-1}\le\cdots\le A_0:=|x_{13}|^{-1}
\nonumber\\[4pt]&\!\!\!\!\!\!\!\!\!\!\!\!\!\!\!\!\!\!\!\!&
\phantom{1}\le
A_1:=\kappa_1|x_{12}^3x_{11}x_{01}|^{\kappa_2}<
A_2:=\kappa_3|x_{12}x_{11}^3x_{01}^2|^{\kappa_4}\le A_3:=\kappa_5|x_{12}^3x_{11}x_{01}|^{\kappa_6}<\kappa_7,
\end{eqnarray}
As in Case 1, we set  [thus the first $n-n_1-1$ inequalities of \eqref{ToSayas0-gen}\,(a) automatically hold when we use $\dot x_{ij}$ defined in \eqref{q0q1-gen} to substitute $x_{ij}\ssc\,$],
\equa{WeSet-gen}{u_n=\cdots=u_{n_1+1}=0.}
We need to choose $u_1,...,u_{n_1}$ so that all inequalities of \eqref{i.e.-gen1} hold for $(\qq_0,\qq_1)$, and further,
\equa{FuTHER-gen}{\dis \dot A_4:=\frac{|x_{12}|^{\kappa_{11}}\cdot|\dot x_{11}\dot x_{01}^2|^{\kappa_{10}}}{|x_{11}x_{01}^2|^{\kappa_{10}}\cdot|\dot x_{12}|^{\kappa_{11}}}+\kappa'_{12}-(1+\kappa'_{12})\frac{|\dot x_{11}|^{-2}}{|x_{11}|^{-2}}>0, \mbox{ \ \ where }\kappa'_{12}=
\frac{\kappa_{12}|x_{12}|^{\kappa_{11}}}{|x_{11}x_{01}^2|^{\kappa_{10}}}.}
If $n_1=2$, then the problem becomes the $n=2$ case, which can be done exactly as in the
proof of Case 2 of Theorem \ref{real00-inj}\,(2). Thus assume $n_1\ge3$.
We regard $u_1,u_2,u_3$ as free variables and set
[thus the first $n-2$ inequalities of \eqref{ToSayas0-gen}\,(a) automatically hold when we use $\dot x_{ij}$ to substitute $x_{ij}\ssc\,$],
\equa{1WeSet-gen}{u_{n_1}=\cdots=u_4=u_3.}
Note that \eqref{1WeSet-gen} together with \eqref{WeSet-gen} implies [cf.~\eqref{st=-gen}$\ssc\,$]
\equa{msmeme-gen}{s_1=a_1u_1+a_2u_2+\a u_3+O(\ep^1),\mbox{ where }\a=\mbox{$\sum\limits_{i=3}^{n_1}$}a_i.}
In addition, note from \eqref{Let-gen} that we in fact have [noting that $0<\ell^{-1}\ll v_0$; cf.~\eqref{kekeke-gen} and noting a slightly different notations in
\eqref{ToSayas0-gen} from that in \eqref{ToSayas0}$\ssc\,$\vspace*{-5pt}],
\equa{kapppa-en}{\dis\!\!
\kappa_2=\frac15\!+\!O(v_0^1), \ \ \kappa_4=\frac16\!+\!O(v_0^1),\  \ \kappa_6=\frac15\!+\!O(v_0^1),\ \
\kappa_{10}=\frac12\!+\!O(v_0^1), \ \ \kappa_{11}=\frac92\!+\!O(v_0^1).\!\!}
Denote $A_i$ in \eqref{i.e.-gen1} by $\dot A_i$ when $x_{ij}$ is replaced by $\dot x_{ij}$ for $i=0,1,2,3$.
Now we consider three subcases.
%\vskip2pt
%
\noindent{\bf Subcase 2.1}: {\it Assume $\Coeff\big(\frac{\dot A_0}{\dot A_1},u_3\ep\big)\ne0$, i.e., $-1\ne\kappa_2\a$} [note that here $\Coeff\big(\frac{\dot A_0}{\dot A_1},u_3\ep\big)$ means the coefficient of $u_3\ep$ inside the absolute sign $|\cdot|$ of $\frac{\dot A_0}{\dot A_1}\ssc\,$].
Then similar to \eqref{v====}, by \vspace*{-5pt}setting
\equa{SettingU3-gen}{\dis u_3=-\frac{\kappa_2}{1+\a\kappa_2}\Big((1+a_1)u_1+(3+a_2)u_2+
(\tilde\a_1u_1^2+\tilde\a_2u_1u_2+\tilde\a_3u_2^2+w)\ep\Big),}
for some $\tilde\a_i,w\in\C$ with $w\re<0$ (note that we only need to know the coefficients of the linear parts of $u_1,$ and $u_2$), we see that $\frac{\dot A_1}{\dot A_0}=1-\kappa_2w\re\ep^2+O(\ep^3)>0$, i.e., the $(n-1)$-th strict inequality of
\eqref{i.e.-gen1} holds for $(\qq_0,\qq_1)$. Using \eqref{msmeme-gen} and \eqref{SettingU3-gen}, one can
 compute [note that in the following, $\Coeff\big(\frac{\dot A_3}{\dot A_2},u_2\ep\big)$ means the coefficient
 of $u_2\ep$ inside the absolute sign $|\cdot|$ of $\frac{\dot A_3}{\dot A_2}$; we remark that \eqref{CoMooo-gen} and \eqref{A4-Coeff-gen} are the only parts which need to be carefully computed and thus we write down them in details\vspace*{-5pt}]
\begin{eqnarray}
\label{CoMooo-gen}
\!\!\!\!\!\!\!\!\!\!\!\!\!\!\!\!\!\!\!\!\!\!\!\!\!\!\!\!\! &&
c_0:=\Coeff\Big(\frac{\dot A_3}{\dot A_2},u_2\ep\Big)=\kappa_6\Big(a_2+\a\Coeff(u_3,u_2)+3\Big)-\kappa_4\Big(2\big(a_2+\a\Coeff(u_3,u_2)\big)+1\Big)
\nonumber\\[4pt]
\!\!\!\!\!\!\!\!\!\!\!\!\!\!\!\!\!\!\!\!\!\!\!\!\!\!\!\!\!\!\!\!&&
\phantom{c_0}
=\!\kappa_6\Big(a_2\!-\frac{\a\kappa_2(3\!+\!a_2)}{1\!+\!\a\kappa_2}
\!+\!3\!\Big)\!
-\!\kappa_4\Big(2a_2\!
-\frac{2\a\kappa_2(3\!+\!a_2)}{1\!+\!\a\kappa_2}
\!+\!1\!\Big)\!
=\!
\frac{(3\!+\!a_2)\kappa_6\!+\!(-1\!-\!2a_2\!+\!5\a\kappa_2)\kappa_4}{1\!+\!\a\kappa_2}.
\end{eqnarray}
If $c_0\ne0$, then as in \eqref{SettingU3-gen}, we can first set $u_2=\tilde\a_4u_1+(\tilde\a_5u_1^2+\tilde\a_6w_1)\ep$ for some $\tilde\a_i,w_1\in\C$ so that $\frac{\dot A_3}{\dot A_2}$ can becomes $\frac{\dot A_3}{\dot A_2}=1+w_{1\rm\,re}\ep^2+O(\ep^3)>0$, then use the free variable $u_1$ to solve the inequation \eqref{FuTHER-gen}, which can be done exactly as in the arguments after \eqref{C12BsM} (by noting that the coefficient of $u_1\ep$ in the last term of $\dot A_4$ is nonzero).
%
Now assume $c_0=0$, i.e.\vspace*{-5pt},
\equa{ap=-gen}{\dis\a=\frac{(1+2a_2)\kappa_4-(3+a_2)\kappa_6}{5\kappa_2\kappa_4}
.}
Then [using \eqref{kapppa-en}; note that in the following, $\Coeff(\dot A_4,u_2\ep)$ means the coefficient of $u_2\ep$ inside the absolute sign $|\cdot|$ of the first term of $\dot A_4$ in \eqref{FuTHER-gen}$\ssc\,$\vspace*{-5pt}]
\begin{eqnarray}
\label{A4-Coeff-gen}
&\!\!\!\!\!\!\!\!\!\!\!\!\!\!\!\!&
\Coeff(\dot A_4,u_2\ep)=2\kappa_{10}\big(a_2+\a\Coeff(u_3,u_2)\big)-\kappa_{11}
=2\kappa_{10}\Big(a_2
-\frac{\kappa_2\a(3\!+\!a_2)}{1\!+\!\a\kappa_2}\Big)
-\kappa_{11}\nonumber\\[4pt]
&\!\!\!\!\!\!\!\!\!\!\!\!\!\!\!\!&
\phantom{\Coeff(\dot A_4,u_2\ep)}
=2\kappa_{10}\Big(a_2
-\frac{\kappa_2\big((1+2a_2)\kappa_4-(3+a_2)\kappa_6\big)(3\!+\!a_2)}{5\kappa_2\kappa_4\!+\!\big((1+2a_2)\kappa_4-(3+a_2)\kappa_6\big)\kappa_2}\Big)
-\kappa_{11}\nonumber\\[4pt]
&\!\!\!\!\!\!\!\!\!\!\!\!\!\!\!\!&
\phantom{\Coeff(\dot A_4,u_2\ep)}
=\frac{-2(\kappa_{10}+\kappa_{11})\kappa_4+(6\kappa_{10}+\kappa_{11})\kappa_6}{2\kappa_4-\kappa_6}
=\frac{-2(\frac12+\frac92)\frac16+(3+\frac92)\frac15}{\frac13-\frac15}+O(v_0^1)
\nonumber\\[4pt]
&\!\!\!\!\!\!\!\!\!\!\!\!\!\!\!\!&
\phantom{\Coeff(\dot A_4,u_2\ep)}
=-\frac{5}{4}+O(v_0^1)\ne0.
\end{eqnarray}
Thus by setting $u_1=0$ and using the same arguments after \eqref{kekeke-gen}, we can use $u_2$ to solve the inequations $\dot A_3\ge\dot A_2$ and $\dot A_4>0$.
%\vskip2pt
%
\noindent{\bf Subcase 2.2}: {\it Assume $\a=-\frac1{\kappa_2}$, $a_2\ne-3$}.
Then $\frac{\dot A_1}{\dot A_0}=
\big|1+\kappa_2((1+a_1)u_1+(3+a_2)u_2)\ep+O(\ep^2)\big|$.
By setting \equa{Setu2==-gen}{\dis u_2=-\frac{1+a_1}{3 + a_2}\Big(u_1+(\tilde\a_6u_1^2+\tilde\a_8u_1u_3+\tilde\a_9u_3^2)\ep)\Big)+\frac{w_2\ep}{3+a_2},}
for some $\tilde\a_i,w_2\in\C$ with $w_{2\rm\,re}>0$, so that $\frac{\dot A_1}{\dot A_0}$ can becomes $\frac{\dot A_1}{\dot A_0}=1+\kappa_2w_{2\rm\,re}\ep^2+O(\ep^3)>1$, i.e.,
 the $(n-1)$-th strict inequality of
\eqref{i.e.-gen1} holds for $(\qq_0,\qq_1)$. One can easily compute that the
coefficient of $u_3\ep$ in $\frac{\dot A_3}{\dot A_2}$ is
[note that $u_2$ in \eqref{Setu2==-gen} does not contain the linear part of $u_3$, which makes the following computation easier]
\equa{Coefffff-geneee}{\mbox{$\dis c_1:=(\kappa_6-2\kappa_4)\a=-5\Big(\frac15-\frac13\Big)+O(v_0^1)=\frac23+O(v_0^1)\ne0$.}}
Thus using arguments after \eqref{CoMooo-gen}, by setting $u_3=\tilde\a_{10}u_1+(\tilde\a_{11}u_1^2+\tilde\a_{12}w_3)\ep$ for some $\tilde\a_i,w_3\in\C$ with $w_{3\rm\,re}>0$ (so that we can have $\frac{\dot A_3}{\dot A_2}=1+w_{3\rm\,\re}\ep^2+O(\ep^3)>1\ssc\,$), and
we can find some  $u_1$ to satisfy the inequation $\dot A_4>0$.
%\vskip2pt
%
\noindent{\bf Subcase 2.3}: {\it Finally assume $\a=-\frac1{\kappa_2}$, $a_2=-3$}. Then
\begin{eqnarray}\label{MSMSMSMenene-gen}
\!\!\!\!\!\!\!\!\!\!\!\!\!\!\!\!\!\!&&
\frac{\dot A_1}{\dot A_0}=\Big|1+(1+a_1)\kappa_2u_1\ep+O(\ep^2)\Big|,
\nonumber\\[4pt]
\!\!\!\!\!\!\!\!\!\!\!\!\!\!\!\!\!\!&&
\frac{\dot A_3}{\dot A_2}
=\Big|1+\Big(-(1+a_1)(2\kappa_4-\kappa_6)u_1+5\kappa_4u_2+\frac{2\kappa_4-\kappa_6}{\kappa_2}u_3\Big)\ep+O(\ep^2)\Big|
,\nonumber\\[4pt]
\!\!\!\!\!\!\!\!\!\!\!\!\!\!\!\!\!\!&&
\mbox{the first term of $\dot A_4$ \ }
=\ \Big|1+\Big((1+2a_1)\kappa_{10}u_1-(6\kappa_{10}+\kappa_{11})u_2-\frac{2\kappa_{10}}{\kappa_2}u_3\Big)\ep\Big|.
\end{eqnarray}
By setting \equa{Finana-gen}{\mbox{$\dis u_1=0,\ \ \ \ \ u_2=\frac{-2\kappa_4 +\kappa_6}{5\kappa_2\kappa_4}u_3+\tilde\a_{13}u_3^2\ep + w_4\ep$,}} for some
$\tilde\a_i,w_4\in\C$ with $w_{4\rm\,re}>0$ so that $\frac{\dot A_3}{\dot A_2}$ can become $\frac{\dot A_3}{\dot A_2}=1+w_{4\rm\,re}\ep^2+O(\ep^3)>1$, i.e., the $(n+1)$-th strict inequality of \eqref{i.e.-gen1}\,(i) holds for $(\qq_0,\qq_1)$.
Now it is not difficult to compute that the coefficient of $u_3\ep$ in $\dot A_4$ is
[cf.~the last equation of \eqref{MSMSMSMenene-gen}$\ssc\,$]
\equa{Msmdmdmd-gennnn}{\mbox{$\dis c_2:=\frac{2(\kappa_{10}+\kappa_{11})\kappa_4-
(6\kappa_{10}+\kappa_{11})\kappa_6}{5\kappa_2\kappa_4}=1+O(v_0^1)>0$.}}
Thus similar to  the arguments after \eqref{C12BsM}, we can simply take $u_3=w_4$, then 
$\dot A_4=c_2w_{4\rm\,re}\ep^2+O(\ep^3)>0$, i.e., \eqref{FuTHER-gen} holds. If 
$\frac{\dot A_1}{\dot A_0}$ does not depend on $w_4$ (i.e.,
$\frac{\dot A_1}{\dot A_0}=1\ssc\,$),
then $\dot A_0\le\dot A_1$ trivially holds. Otherwise $\frac{\dot A_1}{\dot A_0}
 =1+b'w_4^k\ep^{k_1}+O(\ep^{k_1+1})$ for some $b'\in\C_{\ne0}$ and $k,k_1\in\Z_{\ge2}$, and we can always choose
$w_4\in\C$ satisfying both $w_{4\rm\,re}>0$ and  $(b'w_4^k)\re>0$ (since $k\ge2$). This proves  Theorem \ref{real00-inj-gen}\,(2).
%
%
%we can assume the first $n-2$ equalities and the $(n+1)$-th equality in \eqref{ToSayas0-gen}\,(a) hold
%(the proof is similar to the above for other cases).
%Then we can set $u_3=\cdots=u_{n}=0$ [then the first $n-2$ inequalities of \eqref{ToSayas0-gen}\,(a) automatically hold] and regard $u_1,u_2$ as free variables and then solve them to satisfy $(n+1)$-th inequality in \eqref{ToSayas0-gen}\,(a) and
%$\ell_{\qq_0,\qq_1}>\ell_{\pp_0,\pp_1}$, which can be done exactly as in $n=2$ case
%[cf.~Case 2 . This proves  Theorem \ref{real00-inj-gen}\,(2).
%
\NOUSE{%
\eqref{ToSayas-gen}\,(b) becomes
%
%
Let $n_2=\max\{j\,|\,a_{i_j}\ne0\}$. If $n_2<n$, then we can take  $u_{i_n}=-1$ $\big[$then $\frac{|\dot x_{01}|+\kappa_{n+2}}{|\dot x_{1n}|^{\kappa_{n+3}}}=
\frac{|x_{01}|+O(\scep^2)+\kappa_{n+2}}{\big(|x_{1n}|(1-\scep)\big)^{\kappa_{n+3}}}>
\frac{|x_{01}|+\kappa_{n+2}}{|x_{1n}|^{\kappa_{n+3}}}$, i.e., \eqref{ToSayas-gen}\,(b) holds for $(\qq_0,\qq_1){\ssc\,}\big]$, and take $u_{i_j}$ for $j=n_1,...,n-1$
%
}%
%
%
\subsection{Proof of Theorems \ref{MAINT-gen} and \ref{real-inj-1-gen}}
The proof of Theorem \ref{real-inj-1-gen}\,(i) can be done exactly as in the proof of Theorem \ref{real-inj-1}\,(i).
The proof of Theorem \ref{real-inj-1-gen}\,(ii) is very simple: Assume $x_{1k}\ne p_{1k}$ for some $k$, then we can simply take $u_j=0$ for $j\ne k$
(then $|\dot x_{1j}-p_j|^2=|x_{1j}-p_j|^2\ssc\,$),
 and take $u_k$ to satisfy that $|\dot x_{1k}-p_k|^2<|x_{1k}-p_k|^2$, then $\dd_{\qq_0,\qq_1}<\dd_{\pp_0,\pp_1}$.
%
The proof of Theorem \ref{MAINT-gen} is exactly the same as that of  Theorems \ref{MAINT}.
%
}%
%
%{}%
%
%

{ \small\footnotesize \lineskip=2pt
\parskip2pt

\begin{thebibliography}{9999}\label{referencessss}
\small \lineskip=4pt
\parskip2pt
\def\BI#1{\bibitem{#1}\label{#1}}

\BI{A}
A.~Abdesselam,
The Jacobian conjecture as a problem of perturbative quantum field theory,
{\it Ann.~Henri Poincar\'e} {\bf4} (2003), 199--215.

\BI{Abh}S.~Abhyankar, Some thoughts on the Jacobian conjecture.~I., {\it J.~Algebra} {\bf319} (2008), 493--548.

\BI{AO}
H.~Appelgate, H.~Onishi,
The Jacobian conjecture in two variables,
{\it J.~Pure Appl.~Algebra} {\bf37} (1985), 215--227.

\BI{AV} P.K.~Adjamagbo, A.~van den Essen, A proof of the equivalence of the
Dixmier, Jacobian and Poisson conjectures, {\it Acta Math. Vietnam.}
{\bf32} (2007), 205--214.


\BI{B} H.~Bass,
The Jacobian conjecture,
{\it Algebra and its Applications} (New Delhi, 1981), 1--8,
Lecture Notes in Pure and Appl.~Math.~{\bf91}, Dekker, New York,
1984.

\BI{BCW} H.~Bass, E.H.~Connell, D.~Wright,
The Jacobian conjecture: reduction of degree and
formal expansion of the inverse,
{\it Bull.~Amer.~Math.~Soc.}~{\bf7}
(1982), 287--330.

\BI{BK} A.~Belov-Kanel, M.~Kontsevich, The Jacobian Conjecture
is stably equivalent to the Dixmier Conjecture, {\it Mosc. Math. J.}
{\bf7} (2007), 209--218.

\BI{K-m1}
A.~Bialynicki-Birula, M.~Rosenlicht,
Injective morphisms of real algebraic varieties, {\it
Proc. Amer. Math. Soc.} {\bf13} (1962), 200--203.


\BI{CCS}
Z.~Charzy\'nski, J.~Chadzy\'nski, P.~Skibi\'nski, A contribution to
Keller's Jacobian conjecture. IV. {\it
Bull.~Soc.~Sci.~Lett.~\L\'od\'z} {\bf39} (1989), no.~11, 6 pp.

\BI{CM}
M.~Chamberland, G.~Meisters,
A mountain pass to the Jacobian conjecture,
{\it Canad.~Math. Bull.} {\bf 41} (1998), 442--451.

\BI{Di} J.~Dixmier, Sur les algebres de Weyl,
{\it Bull. Soc. Math. France} {\bf96} (1968), 209--242.

\BI{D}
L.M.~Dru\.zkowski,
The Jacobian conjecture: survey of some results,
{\it Topics in Complex Analysis} (Warsaw, 1992), 163--171,
Banach Center Publ., 31, Polish Acad.~Sci., Warsaw, 1995.


\BI{DV} M.~de Bondt, A.~van den Essen,
A reduction of the Jacobian conjecture to the symmetric case,
{\it Proc.~Amer.~Math.~Soc.} {\bf133} (2005), 2201--2205.

\BI{ES} G.P.~Egorychev, V.A.~Stepanenko,
The combinatorial identity on the Jacobian conjecture,
{\it Acta Appl.~Math.} {\bf85} (2005), 111--120.

\BI{H}
E.~Hamann,
Algebraic observations on the Jacobian conjecture,
{\it J.~Algebra} {\bf265} (2003), 539--561.

\BI{J}
Z.~Jelonek,
The Jacobian conjecture and the extensions of polynomial embeddings,
{\it Math. Ann.} {\bf 294} (1992), 289--293.

\BI{K}
S.~Kaliman,
On the Jacobian conjecture, {\it Proc.~Amer.~Math.~Soc.}~{\bf 117} (1993),
45--51.


\BI{KM}
T.~Kambayashi, M.~Miyanishi,
On two recent views of the Jacobian conjecture,
{\it Affine Algebraic Geometry}, 113--138,
Contemp.~Math.~{\bf369}, Amer.~Math.~Soc.,
Providence, RI, 2005.

\BI{Ki}
M.~Kirezci, The Jacobian conjecture. I, II. {\it \.Istanbul
Tek.~\"Univ.~B\"ul.}~{\bf 43} (1990), 421--436, 451--457.


\BI{MTU}L.~Makar-Limanov, U.~Turusbekova, U.~Umirbaev,
Automorphisms and derivations of free Poisson algebras in two variables,
{\it J. Algebra} {\bf322} (2009), 3318--3330.

\BI{M1}
T.T.~Moh, On the Jacobian conjecture and the configurations of
roots, {\it J. Reine Angew. Math.} {\bf340} (1983), 140--212.


\BI{N}
M.~Nagata,
Some remarks on the two-dimensional Jacobian conjecture,
{\it Chinese J.~Math.} {\bf17} (1989), 1--7.


\BI{K-m2}
D.J.~Newman,
One-one polynomial maps,
{\it Proc. Amer. Math. Soc.} {\bf11} (1960), 867--870.



\BI{No}
A.~Nowicki,
On the Jacobian conjecture in two variables.
{\it J.~Pure Appl.~Algebra} {\bf50} (1988), 195--207.

\BI{R}
K.~Rusek, A geometric approach to Keller's Jacobian conjecture, {\it
Math.~Ann.} {\bf264} (1983), 315--320.

\BI{S}
S. Smale, Mathematical problems for the next century,
{\it Math.~Intelligencer} {\bf20}
(1998), 7-15.

\BI{SW}
M.H.~Shih, J.W.~Wu,
On a discrete version of the Jacobian conjecture of dynamical systems,
{\it Nonlinear Anal.}~{\bf 34} (1998), 779--789.

\BI{SY}
V.~Shpilrain, J.T.~Yu,
Polynomial retracts and the Jacobian conjecture,
{\it Trans.~Amer.~Math. Soc.}~{\bf352} (2000), 477--484.


\BI{SU} Y. Su, Poisson algebras, Weyl algebras and Jacobi pairs, arXiv:1107.1115v9.

\BI{SX} Y.~Su, X.~Xu,
Central simple Poisson algebras,
{\it Science in China A}~{\bf47} (2004), 245--263.
 ��

\BI{V1} A.~van den Essen,
{\it Polynomial automorphisms and the Jacobian conjecture,}
Progress in Mathematics {\bf190}, Birkh\"auser Verlag, Basel, 2000.

\BI{V2} A.~van den Essen,
The sixtieth anniversary of the Jacobian conjecture: a new approach,
{\it Polynomial automorphisms and related topics,}
Ann.~Polon.~Math. {\bf76} (2001), 77--87.

\BI{W} D.~Wright,
The Jacobian conjecture: ideal membership questions and recent advances,
{\it Affine Algebraic Geometry}, 261--276,
Contemp.~Math.~{\bf369}, Amer.~Math.~Soc., Providence,
RI, 2005.

\BI{Y} J.T.~Yu,
Remarks on the Jacobian conjecture, {\it J. Algebra} {\bf188}
(1997), 90--96.


\end{thebibliography}
}
\end{document}
