%%%%    Final version submitted to the ArXiv January 14, 2017, and submitted for publication
%%%%   Version containing comments by Bowen, Seward, Weiss
%%%% Preliminary version sent to Bowen Seward Lindenstrauss Hochamn and Weiss, December 8, 2016
%%%%%%    Entropy for hyperfinite relations / Markov groups, started May 2016

\documentclass[a4paper,10pt,reqno]{amsart}
%\documentclass[a4paper,11pt]{amsart}


\usepackage{amssymb,amsmath,amsfonts} %\allowdisplaybreak
\usepackage{a4wide}
\usepackage{graphicx}
\usepackage{hyperref}
\usepackage{color}
\usepackage{amsthm}
\usepackage{enumerate}
\numberwithin{equation}{section}    % Gleichungen mit Sections nummeriert
%
\theoremstyle{plain}
\newtheorem{Theorem}{Theorem}[section]
\newtheorem{Proposition}[Theorem]{Proposition}
\newtheorem{Corollary}[Theorem]{Corollary}
\newtheorem{Lemma}[Theorem]{Lemma}
\theoremstyle{definition}
\newtheorem{Assumption}[Theorem]{Assumption}
\newtheorem{Definition}[Theorem]{Definition}
\newtheorem{Example}[Theorem]{Example}
\newtheorem{Examples}[Theorem]{Examples}
\newtheorem{assumption}[Theorem]{Assumption}
\theoremstyle{remark}
\newtheorem{Remark}[Theorem]{Remark}
%\newtheorem{Proof}[Theorem]{Proof}



%
\renewcommand{\qedsymbol}{$\blacksquare$}
%\renewcommand{\epsilon}{\varepsilon}
%\renewcommand{\phi}{\varphi}
\newcommand{\one}{\mathbf{1}} % char. Funktion
\newcommand{\indlu}[1]{\ ^{#1}\!}
\newcommand{\ov}[2]{\begin{pmatrix}{#1}\\{#2}\end{pmatrix}}
\newcommand{\heis}[3]{\begin{pmatrix}{#1}\\{#2}\\{#3}\end{pmatrix}}
\newcommand{\dotcup}{\ensuremath{\mathaccent\cdot\cup}}
\newcommand{\ato}[2]{\genfrac{}{}{0pt}{2}{#1}{#2}}
\DeclareMathOperator{\Tr}{Tr}
\DeclareMathOperator{\ggT}{ggT}
\DeclareMathOperator{\Hom}{Hom}
\DeclareMathOperator{\diam}{diam}
\DeclareMathOperator{\supp}{supp}
\DeclareMathOperator{\rank}{rank}
\DeclareMathOperator{\id}{id}
\newcommand{\1}[0]{\textbf 1}

\newcommand{\EE}{\mathbb{E}}
\newcommand{\FF}{\mathbb{F}}
\newcommand{\RR}{\mathbb{R}}
\newcommand{\CC}{\mathbb{C}}
\newcommand{\NN}{\mathbb{N}}
\newcommand{\TT}{\mathbb{T}}
\newcommand{\ZZ}{\mathbb{Z}}
\newcommand{\PP}{\mathbb{P}}
\renewcommand{\SS}{\mathbb{S}}


\newcommand{\cC}{\mathcal{C}}
\newcommand{\cE}{\mathcal{E}}
\newcommand{\cF}{\mathcal{F}}
\newcommand{\cP}{\mathcal{P}}
\newcommand{\cD}{\mathcal{D}}
\newcommand{\cB}{\mathcal{B}}
\newcommand{\cH}{\mathcal{H}}
\newcommand{\cS}{\mathcal{S}}
\newcommand{\cA}{\mathcal{A}}
\newcommand{\cZ}{\mathcal{Z}}
\newcommand{\cG}{\mathcal{G}}
\newcommand{\cL}{\mathcal{L}}
\newcommand{\cX}{\mathcal{X}}
\newcommand{\cN}{\mathcal{N}}
\newcommand{\cO}{\mathcal{O}}
\newcommand{\cK}{\mathcal{K}}
\newcommand{\cQ}{\mathcal{Q}}
\newcommand{\cR}{\mathcal{R}}
\newcommand{\cJ}{\mathcal{J}}
\newcommand{\cI}{\mathcal{I}}
\newcommand{\cT}{\mathcal{T}}
\newcommand{\cM}{\mathcal{M}}
\newcommand{\cU}{\mathcal{U}}

%  MACROS
\newcommand\norm[1]{\left\|#1\right\|}
\newcommand\biggnorm[1]{\biggl\|#1\biggr\|}
\newcommand\abs[1]{\left|#1\right|}
\newcommand\bigabs[1]{\bigl|#1\bigr|}
\newcommand\biggabs[1]{\biggl|#1\biggr|}
\newcommand\inn[1]{\left\langle #1 \right\rangle}
\newcommand\set[1]{\left\{{#1}\right\}}
%\newcommand\except[1]{\setminus\set{#1}}
\renewcommand\Bar[1]{\overline{#1}}


%% Semi direct products

\def\lsemi{\hbox{$>\!\!\!\triangleleft$}}
\def\rsemi{\hbox{$\triangleright\!\!\!<$}}

\def\bnu{\bar{\nu}}







\newcommand{\hm}[1]{\textbf{*}\leavevmode{\marginpar{\tiny%
$\hbox to 0mm{\hspace*{-0.5mm}$\leftarrow$\hss}%
\vcenter{\vrule depth 0.1mm height 0.1mm width \the\marginparwidth}%
\hbox to 0mm{\hss$\rightarrow$\hspace*{-0.5mm}}$\\\relax\raggedright #1}}}
%

\begin{document}

%opening
\title[The Shannon-McMillan-Breiman theorem for group actions]{The Shannon-McMillan-Breiman theorem \\ beyond amenable groups}

%\subjclass{999999 ?????????}
\keywords{Ergodic group actions, entropy, measured equivalence relations, 
hyperfiniteness.}

\date{\today}

%\author[Amos Nevo, Felix Pogorzelski]{Amos Nevo, Felix Pogorzelski}
\author{Amos Nevo}
\address{Technion, Israel Institute of Technology}
\email{anevo@tx.technion.ac.il}
%\address{Technion Campus Amado Building, 32000 Haifa, Israel}

\author{Felix Pogorzelski}
\address{Technion, Israel Institute of Technology}
\email{ felixp@technion.ac.il}
%\urladdr{http://www.analysis-lenz.uni-jena.de/Team/Felix+Pogorzelski.html}

\thanks{The first author acknowledges the support of  ISF grant \# 2022962. The second author gratefully acknowledges support through a Technion Fine fellowship }


%\begin{document}
\begin{abstract}
We introduce a new isomorphism-invariant notion of entropy for measure preserving actions of arbitrary countable groups on probability spaces, which we call cocycle entropy. We develop methods to show that cocycle entropy satisfies many of the properties of classical amenable entropy theory, but applies in much greater generality to actions of non-amenable groups.  
%The class of groups in question may in fact include all countable groups. 
One key ingredient in our approach is a proof of a  subadditive convergence principle which is valid for measure-preserving amenable equivalence relations, going beyond the Ornstein-Weiss Lemma for amenable groups. 

For a large class of countable groups, which may in fact include all of them, 
%we use hyperfinite exhaustions of amenable equivalence relations to 
we prove the Shannon-McMillan-Breiman pointwise convergence theorem for cocycle entropy in their measure-preserving actions.  
 
We also compare cocycle entropy to Rokhlin entropy, and using an important recent result of Seward
we show that they coincide for free, ergodic actions of any countable group in the class. Finally, we use the example of the free group to demonstrate the geometric significance of the entropy equipartition property implied by the Shannon-McMillan-Breiman theorem.  

%The class of groups in question consists of groups admitting an injective cocycle from a probability measure preserving amenable equivalence relation to the group in question, a condition that may hold for the class of all countable groups. 

\end{abstract}

\maketitle
%\begin{center}
%Preliminary version
%\end{center}


%\begin{center}
%{\bf  Address : } 
%Technion, Israel Institute of Technology \\
%Technion City, 32000 Haifa, Israel\\
%{\tt felixp@tx.technion.ac.il, anevo@tx.technion.ac.il}\\
%\end{center}

\tableofcontents
%{\bf Keywords:} entropy theory, ergodic actions, measured equivalence relations, 
%hyperfiniteness. \\ %-- Entropy. \\

%{Mathematics subject classification:} %\quad {05C25, 05C80, 46L60, 47A35}.


\section{Introduction}\label{sec:intro}

%In this paper, we propose a new notion of entropy for 
%probabilty measure preserving (p.m.p.) actions of countable 
%groups on probability spaces, and develop some of its properties. 
%
% Our approach is via 
%ergodic theorems for general countable groups.
%In the proposed framework, we do not only obtain the corresponding 
%convergence theorems for amenable groups as a special case, 
%but likewise provide the first results of this kind for non-amenable groups. 
%In fact, the class of non-amenable groups which fall in our
%setting is vast. 
%The proofs rely on the amenability of naturally arising measured 
%equivalence relations (In the sense of :::ZIMMER/CFW???:::). 
%Since the connection to the group action is given via 
%some measurable cocycle, our new notion will be named
%{\em cocylce entropy}.\\ 
%More precisely, the main achievements of the present paper are the 
%following.
 
\noindent{\bf Classical amenable entropy theory.}
The notion of entropy as an isomorphism invariant of dynamical systems originated in the work of
Kolmogorov and Sinai, who defined entropy for probability measure preserving (p.m.p.) actions of $\ZZ$  and studied its basic properties \cite{Ko58, Ko59, Si59}. Entropy theory has been developed extensively and has played a major role in the theory of dynamical systems ever since. One of the main themes of development has been the program of extending entropy theory to the class of dynamical systems defined by p.m.p.\@ actions of amenable groups. In particular, \cite{OW87} settled the problem of classifying Bernoulli
shifts for countable amenable groups by the value of the entropy of the underlying dynamical system. A central goal of classical entropy theory has been  "entropy equipartition" results, established by Shannon \cite{Sh48}, McMillan \cite{Mc53} and Breiman \cite{Br57} and culminating in the Shannon-McMillan-Breiman pointwise convergence theorem, a major achievement of the classical theory. In the more general context of amenable groups, the Shannon-McMillan mean convergence theorem for countable amenable groups has been established by Kieffer \cite{Ki75}, and a subadditive ergodic theorem for entropy has been established by Ollagnier \cite{Ol85}. Further, pointwise convergence for countable amenable groups admitting regular F{\o}lner sequences is due to Ornstein and Weiss \cite{OW83}. The Shannon-McMillan-Breiman pointwise pointwise convergence theorem for tempered F{\o}lner satisfying a mild
growth condition was established by Lindenstrauss \cite{Li01}. This latter result was extended by Weiss \cite{W03} to the case of non-ergodic actions and without a growth assumption on the sequences in question. The proof is based on previous joint work of Ornstein and Weiss,  
and avoids the random tiling and covering arguments used in \cite{Li01}. 
%Although we adapt the overall strategy of Lindenstrauss to prove the Shannon-McMillan-Breiman theorem for many 
%non-amenable groups, we are also able to dispense with random tilings.   

 Anticipating our discussion below, let us note that our approach to entropy theory has the significant advantage that it treats amenable and non-amenable groups on an equal footing, and thus allows us to make extensive use of the methods and ideas developed in the discussion of entropy theory of amenable groups, and apply them to the setting of non-amenable group. % in \cite{Li01}. 

  
\noindent{\bf Entropy for actions of non-amenable groups.} 
The problem of developing entropy theory for p.m.p.\@ actions
of non-amenable countable groups remained open for nearly half a century, but has seen remarkable ground-breaking progress in recent years. 
The first major breakthrough was the introduction by Bowen (see \cite{Bo10b}\cite{Bo10c}\cite{Bo12})  of the concept of sofic 
entropy and the developments of its main properties as an isomorphism-invariant for p.m.p.\@ actions of sofic groups.  While this is certainly a very large class of groups, it is not yet known whether it includes all countable groups or not. 
%Bowen has used sofic entropy to solve the classification problem for Bernoulli shifts for groups in this class.  
Bowen's notion of sofic entropy was developed further by Kerr and Li,  \cite{Ke13, KL11}. In particular, sofic entropy constitutes a major extension of  
the classical entropy theory of p.m.p.\@ group actions, since for amenable groups, sofic entropy
coincides with the classical Kolmogorov-Sinai entropy  \cite{Bo12, KL13}. 
 
 Another important breakthrough in entropy theory, for actions of completely
arbitrary groups, has recently been obtained by Seward, who introduced the 
concept of Rokhlin entropy. As for the case of sofic entropy, for  
amenable groups this notion coincides with the 
classical Kolmogorov-Sinai entropy as well, a result due to Rokhlin for the classical case of $\ZZ$-actions, and to Seward  and Tucker-Drob
in general \cite{ST12}.  
Furthermore, Seward proved several remarkable results about Rokhlin entropy, 
%In fact, all of his considerations even deal with relative Rokhlin
%entropy, i.e.\@ they allow for conditioning on $\Gamma$-invariant
%sub $\sigma$-algebras. We do not pursue this goal in this work. 
including the fact that every ergodic
group action with finite Rokhlin entropy admits a {\em finite}
generating partition \cite[Thm. 1.3]{Se15a}. 
%(The theorem is much stronger since it also gives quantitative 
%information on the numbers of sets in the partions, as well as
%on their probabilities.)
This greatly extends the classical finite generator theorems of
Rokhlin and Krieger \cite{Ro67, Kr70}, to actions of general countable groups.
%The statement is given the stronger form as considered for $\ZZ$ in \cite{De74}.
This result is material for our work since the
Shannon-McMillan-Breiman theorem proposed in Theorem~\ref{thm:MAIN_SMBhyp} 
below is valid in the context of
finite partitions.
%It is further shown in \cite{Se15a} that for
%free actions of amenable groups, Rokhlin entropy coincides with
%the classical Kolmogorov-Sinai entropy.
Moreover, in \cite{Se15b, Se16}, Seward 
implicitly  deals with another notion of entropy which we will call 
{\em finitary entropy} below. % (:::STUDIED BEFORE? WHO? WHAT NAME? :::). 
It follows from the fundamental inequality stated in \cite[Thm. 1.5]{Se16} that finitary entropy and Rokhin entropy coincide 
for free, ergodic p.m.p.\@ actions of arbitrary
countable groups, a result that will be crucial in our considerations below. 


\noindent{\bf Main results of the present paper.}  
Let us note that while the three notions of entropy mentioned above - sofic, Rokhlin and finitary - generalize the classical entropy of amenable groups, none of them gives rise to an entropy equipartition theorem in the form of a Shannon-McMillan mean convergence theorem or a Shannon-McMillan-Breiman pointwise convergence theorem. The present paper is devoted to solving this problem, as follows. 


\begin{itemize}
\item We define a new isomorphism-invariant notion of entropy for probability measure preserving (p.m.p.) actions of a very extensive class of groups, which may in fact coincide with the class of all countable infinite groups. We establish the existence of this invariant, which we call cocycle entropy, via a new  
general subadditive convergence theorem for actions of non-amenable groups 
%going beyond the known results for amenable groups 
(see Theorem~\ref{thm:MAINcocycleentropy}).   
\item %We compare the concepts of cocycle entropy with other notions, and  
We show that cocycle entropy is naturally bounded from below by finitary entropy
and bounded from above by Rokhlin entropy (see Theorem~\ref{thm:ENT}).  
%Using an important recent result of Seward, we show 
It follows that for free, ergodic p.m.p.\@ countable group actions, cocycle entropy coincides
with Rokhlin entropy and finitary entropy. 
%(see \@ Theorem~\ref{thm:ENT}). 
%This relies on 
%and implies that for sofic groups, 
%the sofic, Rokhlin, finitary and cocycle entropies of 
%the cocycle entropy of Bernoulli shifts is 
%equal to the Shannon entropy of the base space. 
\item We prove an analog of the Shannon-McMillan-Breiman pointwise convergence theorem for cocycle entropy,  under a suitable ergodicity assumption (see  
Theorem~\ref{thm:MAIN_SMBhyp}). As a consequence, an analog of the Shannon-McMillan mean convergence theorem follows (see Corollary \ref{cor:MAIN_SML1}). 
%Since this setting allows for p.m.p.\@ actions of a vast class of non-amenable
%groups, this significantly extends the geometric framework of the
%classical results. 
\end{itemize}
Let us add the following comments. 


1) 
%The existence of cocycle entropy is based on a natural generalization of classical subadditive convergence theorems to the case of actions of non-amenable groups.  
In the classical theory one defines the entropy of a partition as a limit 
of the normalized Shannon entropy of refinements of
that partition. The crucial tool in this undertaking are
abstract subadditive convergence theorems.  
The corresponding statement for F{\o}lner
sequences in amenable groups is also called the Ornstein-Weiss Lemma.
In the context of discrete, amenable structures, 
there are multiple versions of this result in the literature, 
cf.\@ e.g.\@ \cite{OW87,Gr99,LW00,CCK14,Po14}.
%The ergodic theorems leading to the definition of 
%entropy
%underscore its dynamical aspect, provide the 
%Using amenable equivalence relations, we show in Theorem~\ref{thm:INT}
%the validity of an integrated version of the Ornstein-WeissLemma. 
We will formulate a new subadditive convergence principle and use it to establish convergence in the mean 
%(with respect to the measure on the relation) 
of the normalized Shannon entropy of 
a sequence of successive refinements of a given partition of the underlying probability space, thus establishing 
the existence of cocycle entropy (see Theorem~\ref{thm:MAINcocycleentropy}). 
%The application for the definition of 
%cocycle entropy for a fixed partition is given by 
%

2) The subadditive convergence theorem we establish is the first subadditive convergence principle 
that we are aware of for actions of non-amenable groups. 
It is valid for general subadditive functions and its usefulness is not restricted to convergence of measure-theoretic information functions (see Theorem \ref{thm:INT}). In particular, it can be used to 
define a notion of topological entropy for certain actions of 
non-amenable groups on compact metric spaces, thus raising also the possibility of a variational principle for cocycle entropy.  
%We plan to address this problem in future work. 

%3) Cocycle entropy 
%is directly connected to the dynamics of the group action, 
%a fact which is underscored also by the entropy equiparatition property it satisfies, expressed by   
%the Shannon-McMillan-Breiman theorem. 
%The Shannon-McMillan-Breiman theorem is concerned with 
%the pointwise almost-everywhere convergence of the 
%information function for a fixed, finite partition. 
%Under an additional ergodicity assumption on the 
%equivalence relation under consideration, we provide in
%Theorem~\ref{thm:MAIN_SMBhyp} the Shannon-McMillan-Breiman for cocycle
%entropy. The corresponding Shannon-McMillan theorem
%($L^1$-convergence) is given in Corollary~\ref{cor:MAIN_SML1}.
%Those results go far beyond the existing results for
%p.m.p.\@ actions of amenable groups. In fact, this 
%gives a pointwise almost everywhere ergodic theorem
%for a very rich class of non-amenable groups.  
%The latter includes lattices in connected semisimple Lie groups without compact factors
%\cite{BN13b} and Gromov hyperbolic groups, cf.\@ \cite{BNhyp}.
%Thus, from
%a geometric point of view, Theorem~\ref{thm:MAIN_SMBhyp} 
%supersedes the classical
%results for amenable groups.
%%By exploiting amenability properties of measured equivalence
%%relations arising naturally with p.m.p. group actions,
%
%

\noindent{\bf The method of the proof : from amenable groups to amenable equivalence relations.}
Given a p.m.p.\@ action of a group $\Gamma$ on $(X,\lambda)$, a main preoccupation in ergodic theory is to establish convergence properties for averages defined by a sequence of finite subsets $\cF_n\subset \Gamma$. Thus ergodic theorems concern the averages
$\frac{1}{\abs{\cF_n}}\sum_{\gamma\in \cF_n}f(\gamma^{-1}x)$ for a measurable function $f$ on $X$, 
and the Shannon-McMillan-Breiman theorem concerns the convergence of the normalized information functions, given in integrated form by 
$$\frac{1}{\abs{\cF_n}}H\left(\bigvee_{\gamma\in \cF_n} \gamma^{-1}\cP\right),$$
where $\cP$ is a (finite) partition of $X$ and $H$ denotes the Shannon entropy. 

An mentioned briefly above, an elaborate and very useful set of techniques was developed when the group $\Gamma$ is amenable, and the sequences $\cF_n$ are asymptotically invariant (often satisfying some additional properties). These techniques allow for rather complete solutions to both convergence problems in the amenable case. 
Remarkably, it is possible to develop a point of view that treats amenable groups and non-amenable groups 
on an equal footing, thus raising the very attractive possibility of leveraging the arguments of classical amenable ergodic theory to deduce analogous results in the non-amenable case. The point view in question is based on the following heuristics. Suppose that $\cR$ is an {\it amenable} probability-measure-preserving Borel equivalence relation on a space $Y$, with countable classes. Suppose that $\alpha : \cR\to \Gamma$ is a measurable cocycle. Let there be given a sequence of finite subset functions $\cF_n$ with 
$\cF_n(y)\subset [y]_\cR$, which are asymptotically invariant under the relation in a suitable sense. We then obtain a collection of finite subsets of $\Gamma$, depending on the parameter $y\in Y$, given by  $y\mapsto \set{\alpha(y,z)\,;\, z \in \cF_n(y)}$. We can then consider the convergence of the averages 
$\frac{1}{\abs{\cF_n(y)}}\sum_{z\in \cF_n(y)}f(\alpha(y,z)x)$, with $f$ being a measurable function on $X$, and the convergence of the information functions, given in integrated form by 
\begin{equation} \label{formula}
\frac{1}{\abs{\cF_n(y)}}H\left(\bigvee_{z\in \cF_n(y)} \alpha(y,z)\cP\right),
\end{equation}
where $\cP$ is a finite partition of $X$. Given this set-up, it is to be expected that the amenability of the equivalence relation $\cR$ and the asymptotic invariance of the subset functions $\cF_n(y)$ can be utilized using the classical arguments of amenable ergodic theory to prove convergence. 

This point of view was initiated and developed in \cite{BN13a}\cite{BN13b}\cite{BN15a}\cite{BN15b}, where it was applied to prove mean and pointwise convergence theorems for averages of functions on $X$. The present paper is concerned with the application of this method to the case of convergence of information functions.
For this purpose, we consider a specific kind of asymptotically invariant subset functions $\cF_n$, namely we choose them to be  
an increasing sequence of subequivalence relations $\cR_n$ of $\cR$ consisting of finite classes, such that $\cR=\bigcup_{n\in \NN} \cR_n$ (a hyperfinite exhaustion). The generality of our approach is based on the fact that every amenable p.m.p. equivalence relation admits such a sequence, a result due to Connes, Feldman and Weiss \cite{CFW81}. A natural question that arises here is to what extent can general asymptotically invariant sequences $\cF_n$ (which can often be defined in geometric terms, e.g.\@ via horospheres in hyperbolic groups)  
be approximated by hyperfinite sequences.  
%This connection is clarified for the free group in Section~\ref{sec:freegroup}. 
We plan to address this issue and the issue of establishing amenability of equivalence
relations in terms of combinatorial asymptotic invariance, in a separate paper. 

It is a vindication of the point of view described above that our proof of the Shannon-McMillan-Breiman theorem %(see Theorem~\ref{thm:MAIN_SMBhyp} below) in fact 
proceeds by adapting the intricate overall
strategy and some of the ingenious arguments developed in the amenable group case  by E. Lindenstrauss in \cite{Li01}, to
the setting of amenable equivalence relations. 
%Those adaptions are non-trivial for the following reasons.
%Firstly, we have to deal with 
%uncountably many averaging sequences at one time.
%In \cite{Li01}, the author deals with tempered 
%F{\o}lner sequences which can be obtained by
%extracting subsequences from arbitrary F{\o}lner 
%sequences. A similar phenomenon occurs in our setting:
%in Proposition~\ref{prop:approxrelationsexist} we show that 
%F{\o}lner sequences generating amenable equivalence
%relations allow for a suitable subcollection 
%providing pointwise approximate hyperfinite exhaustions.\\
%Secondly, instead of employing group multiplication, we need to 
%deal with the action of a dense subgroup of
%the full group $\operatorname{Aut}(\cR)$ on the base space $Y$.  \\
As in the situation of amenable groups, there are two major ingredients 
for the proof. The first is an abstract covering lemma for 
the averaging sequences under consideration. We prove 
a corresponding assertion in Lemma~\ref{lemma:abstrcomb}.
Using the hyperfinite  structure of the equivalence relation, we
are able to avoid some difficult technicalities that appear in \cite{Li01} such as the construction of random
collection of tilings and establishing control of the overlapping of different tiles. 
The second ingredient is a suitable pointwise ergodic 
theorem. For the classical setting, it was observed in
\cite{OW83} that the pointwise ergodic theorem is 
the crucial dynamical tool for the proof of the
Shannon-McMillan-Breiman theorem. 
We will use  the pointwise ergodic theorem proved in \cite{BN13b}, but note that 
in our present setting of hyperfinite equivalence relations, the required pointwise convergence also follows in a straightforward manner from the martingale convergence theorem. 

%XXXXXXXX
%
%
%Amenability of measured equivalence relation can also be characterized
%in terms of F{\o}lner type sequences, see e.g.\@ \cite{CFW81,Ka97}. 
%For the proofs of Birkhoff type pointwise ergodic theorems in 
%\cite{BN13a, BN13b, BN15, BNhyp}, the authors deal with sequences
%with a priori weaker invariance properties. Those sequences will simply be 
%called {\em asymptotically invariant} below. To clarify the 
%relation to the notion used in the present paper, 
%
% We show in Theoren~\ref{thm:MAIN_folneramenable} 
%that measured equivalence relations allowing for asymptotically
%invariant sequences are amenable in the sense of \cite{CFW81}. 
%
%
%
%In \cite{W03}, Weiss gives  Shannon-McMillan-Breiman theorem of the 
Let us turn to comment briefly on precursors to our approach in the literature. The usefulness of coycles for problems in ergodic theory, and in particular for entropy questions, has already been observed in \cite{RW00}, where Rudolph and Weiss show that p.m.p. actions of amenable groups which have the 
completely positive entropy property have strong mixing properties. For the proof of the main theorem, 
the authors employ orbit equivalence theory, where cocycles mapping from the orbit equivalence relation of one (amenable) group into another (amenable) group appear naturally. 
Another interesting precursor to our approach has been developed by Danilenko \cite{Da01} and by Danilenko and Park \cite{DP02}. 
In particular, \cite{Da01} introduced the information function 
(\ref{formula}) for a given amenable relation $\cR$ and a cocycle $\alpha : \cR \to \Gamma$. Its properties were studied in \cite{Da01} and \cite{DP02} when $\Gamma$ is amenable. However, the notion of entropy considered by Danilenko and Park is not suitable for our purposes, since in all the examples we will consider in this paper it assumes the value $\infty$. 
In \cite{Av10}, Avni studies entropy defined via cocycles arising from cross sections in locally compact amenable groups. Among other interesting results, he proves a 
Shannon-McMillan type theorem for the underlying notion of entropy in this case.  

We note, however, that considering cocycles taking values in non-amenble groups and establishing the validity of the corresponding subadditive convergence principle, 
as well as of the Shannon-McMillan-Breiman theorem, have no precedents in the literature that we are aware of. 


\noindent{\bf Entropy equipartition and the geometric significance of cocycle entropy.}
 One of the most important aspects of the classical Shannon-McMillan-Breiman theorem is that it establishes an "entropy equipartition" property of the dynamical system under consideration. This property asserts that the size of a typical atom in the partition $\cF_n=\bigvee_{\gamma\in \cF_n}\gamma^{-1}\cP$ is comparable  
to $\exp\left(-h(\cP)\abs{\cF_n} \right)$, where $h(\cP)$ is the Shannon entropy of the partition $\cP$. In particular, refining $\cP$ by the action of the elements $\gamma\in \cF_n$ in the space $X$ eventually produces refined partitions with most atoms of roughly the same size.  The analog of the Shannon-McMillan-Breiman theorem that we prove for cocycle entropy implies the analogous equipartition property, upon refining the partition $\cP$ of $X$ by the sets $\set{\alpha(y,z)\,;\, z \in \cR_n(y)}\subset \Gamma$, and thus expresses a property which is directly connected to the dynamics of the group action on $X$. 

It is a very natural problem to study what are the sets of group elements that arise 
in the form $\set{\alpha(y,z)\,;\, z \in \cR_n(y)}$, namely as the cocycle images of hyperfinite exhaustions $\cR_n(y)$. We emphasize that in many geometric situations, it is possible to give a very concrete and meaningful description of such sets, and this fact constitutes one of the main advantages of the definition of cocycle entropy. We will exemplify this statement in complete detail in \S \ref{sec:freegroup} below for actions of finitely generated free groups $\FF_r$. In that case, choosing $\cR$ as the horospherical (=synchronous tail) relation on the boundary $\partial \FF_r=Y$ of the free group, and $\alpha$ the canonical cocycle on it, the equivalence classes $[\omega]_\cR$ (where $\omega\in \partial \FF_r$) constitute a combinatorial construction of the unstable leaves of the horospherical foliation. Their images $\alpha(\omega,\xi)\in \FF_r$ for $\xi\in \cR_n(\omega)$ under the canonical cocycle coincide with the intersection of the word metric ball $B_{2n}(e)$ in $\FF_r$ with the horoball  based at $\omega$ and passing through $e$. 
%This set is called the horospherical ball of radius $2n$ determined by $\eta$ and denoted by $B_{2n}^\eta$
Thus the Shannon-McMillan-Breiman theorem asserts in this case that refining a partition of $X$ along the sequence  of (the inverses of) horospherical balls of increasing radii in the group has the entropy equipartition property almost surely. Namely, the information functions of the refined partitions converge to the cocycle entropy of the partition. 
We remark that a similar geometric interpretation holds in much greater generality, and serves as evidence for  the existence of deep connections between the theory of p.m.p.\@ actions of a non-amenable group, and the theory of its {\it amenable} actions, particularly its actions on its Poisson boundaries. %In the interest of brevity, 
We will briefly comment further on these topics in \S\S 7 and 8 below, and plan to give a more detailed exposition elsewhere. 
%Examples : free groups, hyperbolic groups, Markov groups, synchronuous tail relation.



\noindent{\bf Organization of the present paper.} 
The paper is organized as follows.
In  Section~\ref{sec:MAIN} we present and discuss the 
main results of our work in detail. Section~\ref{sec:amenable} is devoted
to the discussion of amenability of measured equivalence relations and to hyperfinite exhaustions. 
Next, we prove a subadditive convergence theorem which amounts to an integrated Ornstein-Weiss type lemma in Section~\ref{sec:subadditive}.
We then establish pointwise covering and tiling lemmas for hyperfinite 
exhaustions in amenable equivalence relations in Section~\ref{sec:tiling}.
Those statements will be the crucial tools in the proof of the
Shannon-McMillan-Breiman theorem. This latter assertion is
proven in Section~\ref{sec:SMB}, along with the Shannon-McMillan
$L^1$-convergence theorem. In Section~\ref{sec:ergodicity}, we 
describe the generality of the framework in which the Shannon-McMillan-Breiman
theorem holds true. 
 We illustrate the convergence theorems by explicating the
case of the free group in Section~\ref{sec:freegroup}. \\
%In the last Section~\ref{sec:folneramenable}, we prove 
%that measured equivalence relations admitting asymptotically invariant
%averaging sequences are amenable in the sense considered in the
%present paper. 

\noindent{\bf Acknowledgements.}
The authors would like to thank 
%Lewis Bowen, Mike Hochman, Elon Lindenstrauss, Roman Sauer, 
%Brandon Seward and Benjamin Weiss for enlightening discussions.  
%Moreover, thanks go to
 Lewis Bowen, Brandon Seward and Benjamin Weiss for 
several enlightening and useful conversations and for their comments on a preliminary version of the present manuscript.  
%{\bf AMOS: STH NEED TO BE ADDED?}

%The first author acknowledges the  support of an ISF grant. The second author gratefully acknowledges support through a Technion Fine fellowship. 
\section{Statement of main results} \label{sec:MAIN}

In the present section, we will briefly mention some key definitions, and then state our main results. A detailed discussion of all relevant concepts will appear in subsequent sections devoted to the proofs of the main results.  

Throughout the paper, we consider standard Borel equivalence relations with countable classes 
$\cR\subset Y\times Y$, where $(Y,\nu)$ is a probability space. $[y]=\cR(y)\subset Y$ denotes the $\cR$-class of $y\in Y$. We assume that the equivalence relation preserves the probability measure $\nu$. It is further assumed that $\cR$ is hyperfinite, or equivalently,  that $\cR$ is amenable in the sense of \cite{CFW81}.  
%and recall that this property was shown\cite{CFW81} that amenable
%equivalence relations are in fact hyperfinite,
Thus  $\cR$ can be 
written as an increasing union of equivalence subrelations $\cR_n\subset \cR$, each 
with finite classes, i.e.
\[
\cR(y) := \bigcup_{n=1}^{\infty} \cR_n(y),\,\text{ for $\nu$-almost every } y\in Y.
\] 
Such a representation of $\cR$ %as a union of a sequence of finite subrelations $(\cR_n)$
 will be called a {\em hyperfinite exhaustion}.  If for every $n$, the relations $\cR_n$ are bounded, namely 
 the size of the equivalence classes $\cR_n(y)$ is essentially bounded (with a bound depending on $n$), we will call the representation a 
 {\it bounded hyperfinite exhaustion}. Let  us note that hyperfinite relations always admit a bounded hyperfinite exhaustion, as will be seen below. In fact it is possible to construct a hyperfinite exhaustion satisfying $\abs{\cR_n(y)}\le n$ almost surely, as noted in \cite{W84} in a more general context. %\cite[p. 420]

%Our discussion will make use of certain approximation arguments, which will concern 
%%the following F{\o}lner type 
%a sequence of bounded measurable subset functions $\cF_n : Y\to Fin(Y)$. Here $Fin(Y)$ is the set of finite subsets of $Y$ with its standard Borel structure, and  $\cF_n(y)$ is a finite set contained in $\cR(y)$, such that $\abs{\cF_n(y)}$ is essentially bounded. 
%%Subset functions satusfy $\cF_n(y)\subset \cR(y) $ a.e., namely 
%%consist of $\cR$-equivalent points. 
%The sequence $(\cF_n)$ will
%be called {\em approximate hyperfinite exhaustion} if there is a 
%{\bf bounded} hyperfinite exhaustion such that 
%\[
%\lim_{n \to \infty} \int_Y \frac{\big| \cF_n(y) \,\triangle \, \cR_n(y)\big|}
%{\big| \cR_n(y) \big|}\,d\nu(y) = 0
%\] 
%The sequence $(\cF_n)$ will be called a {\em pointwise approximate hyperfinite exhaustion} if for a.e. $y$
%% the integrands themselves already converge to zero $\nu$-almost everywhere, namely 
%\[\lim_{n\to \infty} \frac{\big| \cF_n(y) \,\triangle \, \cR_n(y)\big|}
%{\big| \cR_n(y) \big|}=0\,.
%\]
%%{\bf  Standing assumption.} We will assume throughout the paper that $(\cR_n)$ is a bounded hyperfinite exhaustion. 
%
%Pointwise approximate hyperfinite exhaustion can be verified to exist
%in many
%situations, cf.\@ Section~\ref{sec:amenable}. 
%

\subsection{Probability measure preserving actions of groups and cocycle extensions}


Throughout the paper, $\Gamma$ denotes a countable group. 
The collection of all finite subsets of $\Gamma$ is denoted
by $Fin\big( \Gamma \big)$. We consider a probability measure preserving (p.m.p.) ergodic action 
$\Gamma \curvearrowright (X,\lambda)$, and aim to define a notion of entropy for the action.

To that end, assume that there is an amenable p.m.p.\@ equivalence relation $\cR$
over $(Y,\nu)$, admitting a measurable cocycle $\alpha : \cR\to \Gamma$. 
%which is linked to the group action via a cocycle. Precisely,
%Namely we are given a measurable map $\alpha: \cR \to \Gamma$ such that 
Thus for $\nu$-almost every $y \in Y$ and every $w,u,z \in [y]=\cR(y)$, 
the cocycle identity holds :
\[
\alpha\big(z,u\big) = \alpha\big(z,w \big)\cdot \alpha\big(w,u \big).
\]
A crucial construction in our discussion is the equivalence relation, denoted $\cR^X$, which is the extension of the equivalence relation $\cR$ by the cocycle $\alpha$ and the $\Gamma$-action on $X$. 
The {\em extended equivalence relation $\cR^X$}  over
$\big( X \times Y, \lambda \times \nu \big)$ is defined by the condition 
\[
\big( (x,y), (x^\prime,y^\prime) \big) \in \cR^X \Longleftrightarrow y\cR y^\prime \,\,\text{ and } x=\alpha(y,y^\prime)x^\prime
\,.\]
%Note that $\cR^X$ is defined equivalently by the condition
%\[
%\big( (x,y), (x^\prime,y^\prime) \big) \in \cR^X \Longleftrightarrow y\cR y^\prime \,\,\text {and } 
%\, \exists\, g \in\Gamma:\,
%g = \alpha(y,y^\prime), \,\, gx^\prime = x.
%\]


Assuming that the measure $\nu$ is $\cR$-invariant, it follows that $ \lambda \times \nu$ is $\cR^X$-invariant,
since the $\Gamma$-action on $X$ preserves $\lambda$. The projection map $\pi : \cR^X\to \cR$ given by $(x,y)\to y$ is {\it class injective}, namely injective on almost every $\cR^X$-equivalence class.  



Further, it 
is well known that an extension of an amenable action is amenable, and thus in particular if $\cR$ is amenable, so is $\cR^X$. Thus, when $\cR$ is hyperfinite, so is $\cR^X$.  But since the extension is class-injective, in fact every hyperfinite exhaustion $(\cR_n)$ of $\cR$ can be canonically lifted to a hyperfinite exhaustion $(\cR_n^X)$ of $\cR^X$, via $\cR_n^X((x,y))=\set{(\alpha(z,y)x,z)\,;\, z\in \cR_n(y)}$. Note that if $(\cR_n)$ is a bounded hyperfinite exhuastion, then so is $(\cR_n^X)$, with the same bounds on the equivalence classes.  



The cocycle $\alpha$ is called {\it class  
injective} if for $\nu$-almost every $y \in Y$, 
we have that $\alpha(y,z) \neq \alpha(y,w)$ whenever $w\neq z$.
%So $(y,z)$ shall be contained in $\cR$ if and only if there is 
%some $g \in \Gamma$ such that $g = \alpha(z,y)$. 
In order to avoid degenerate cases, we will assume in the
sequel that the cocycle under consideration is class injective. 



\subsection{Definition of cocycle entropy} 


In order to define the measure theoretic entropy of the p.m.p.\@ $\Gamma$-action on $X$, recall first the following. For a 
countable measurable partition 
$\cP = \{A_i\,;\,i\in I\}$ of $X$ the Shannon entropy $H(\cP)$ 
is defined as
\[
H\big( \cP \big):= - \sum_{A \in \cP} \lambda(A)\,\log\big( \lambda(A) \big),
\] 
where we use the convention that $0\cdot \log\,0 = 0$.  
For two countable partitions $\cP$ and $\cQ$, their common refinement
is $\cP \vee \cQ:= \big\{ P \cap Q\,|\, P \in \cP,\, Q\in \cQ \}$. 
%This leads to the definition of refinements induced by a finite subset 
%of $\Gamma$. Precisely, for a countable partition
%$\cP$ and 
For a finite set $F \in Fin(\Gamma)$, we set
\[
\cP^F := \bigvee_{g \in F} g^{-1}\,\cP,
\] 
where $g^{-1}\cP=\set{g^{-1}A_i\,;\, i\in I}$. If $F$ is the empty set, then we define $\cP^F$ to be the trivial partition, which of course has Shannon entropy zero. 
% means the obvious, namely shifting every set $P \in \cP$by $g^{-1}$. 
Given two partitions $\cP$ and $\cQ$,  
$\cQ$ is called {\em finer than} $\cP$ or a {\em refinement of} $\cP$, denoted 
$\cQ \geq \cP$, if for every $Q \in \cQ$, there is some $P \in \cP$ such 
that $Q \subseteq P$ up to $\lambda$-measure zero. 


Now for a subset function $\cF: Y \to Fin(Y)$ (as always, with $\cF(y)\subset \cR(y)$ a.e.) and a countable 
partition $\cP$ of $X$ with $H(\cP) < \infty$, we consider the entropy function 
\[
h^{\cP}(\cF): Y \to [0,\infty): h^{\cP}(\cF)(y) := 
H\Big( \bigvee_{z \in \cF(y)} \alpha(z,y)^{-1}\,\cP \Big).
\]


We can now state our first main theorem which shows that we can attach 
a notion of cocycle entropy to every partition $\cP$ with 
finite Shannon entropy. 

\begin{Theorem}\label{thm:MAINcocycleentropy}
Given a class injective cocycle on $\cR$,   
for every countable partition $\cP$ of $X$ with $H(\cP) < \infty$, there is a number $h_{\cP}^{*}(\alpha)$
such that for every bounded hyperfinite exhaustion $(\cR_n)$,
\begin{eqnarray*}
h_{\cP}^{*}(\alpha) := \lim_{n \to \infty} \int_Y \frac{h^{\cP}\big( \cR_n \big)(y)}{\big| \cR_n(y)\big|}\, d\nu(y).
\end{eqnarray*} 
\end{Theorem}
Let us highlight the crucial fact that it is part of the conclusion of Theorem \ref{thm:MAINcocycleentropy} that the limit is independent of the choice of the bounded hyperfinite exhaustion.
\begin{Remark}
As can be seen from the proof of Theorem \ref{thm:MAINcocycleentropy}, the  
assumption on the cocycle being class injective is not necessary for this convergence result. But we dispense
with this additional generality in order to make sure that the values $h_{\cP}^{*}(\alpha)$ accurately 
reflect entropy theoretic information regarding the action of $\Gamma$ on $X$.
\end{Remark}

%The proof of the theorem is based on a general 
%Ornstein-Weiss type lemma, see 
%Theorem \ref{thm:INT} below for the statement and further elaboration. This result applies in fact to {\it general subadditive functions}, not just to the measure-entropy information function mentioned above.  We note that in the world of p.m.p.\@ actions of non-amenable groups, Theorem \ref{thm:INT} seems to be the first instance of a subadditive convergence principle. 
 
In the following, the number $h_{\cP}^{*}(\alpha)$ shall be called {\em the cocycle entropy} of the
partition $\cP$ for the action $\Gamma \curvearrowright (X,\lambda)$. Let us proceed to use it 
to define a notion of measure-theoretic entropy for the dynamical system $\Gamma \curvearrowright (X,\lambda)$ itself.  For this, we must restrict ourselves to {\em generating partitions},
i.e.\@ countable partitions $\cP$ for $X$ such that (modulo null sets) 
\[
\bigvee_{g \in \Gamma} g \, \cP = \cB\,.
\]
%where $A \circeq B$ if $\lambda(A \triangle B) =0$.  
%We now define :
% the {\it  cocycle entropy} for the dynamical system  $\Gamma \curvearrowright (X,\lambda)$, as follows.

\begin{Definition}[Cocycle entropy]
 Let $(X,\lambda)$ be a p.m.p.\@ ergodic action of $\Gamma$.
Let $\alpha: \cR \to \Gamma$ be a class injective cocycle defined on a hyperfinite relation. The number 
\begin{eqnarray*}
h^{C}(\Gamma \curvearrowright X) := \inf\big\{h^{*}_{\cP}(\alpha)\,| \, \cP \mbox{ countable, generating partition}\big\}
\end{eqnarray*}
is called the {\em cocycle entropy} for the action $\Gamma \curvearrowright (X,\lambda)$ and for the injective cocycle $\alpha: \cR \to \Gamma$.  
%(As the abbreviations indicate, the infimum is taken over all countable, generating partions.)
\end{Definition}

%This is a new concept for attaching an entropy value to p.m.p.\@ group actions.
As we shall see presently, the cocycle entropy of a free ergodic action is an intrinsic invariant, independent of the p.m.p.\@ hyperfinite relation $\cR$, the bounded hyperfinite exhaustion $\cR_n$, and the class injective cocycle $\alpha$ used to define it. 
%equivalence was defined using an class injective cocycle $\alpha$ defined on an equivalence relation $\cR$. But in fact, the same value  is obtained  for {\em any} class  
%injective coycle on a hyperfinite p.m.p. equivalence relation. 
This remarkable fact is ultimately based on the important recent result established by Seward relating two completely different notions of entropy, which we now proceed to describe. 


%{\bf but what about the trivial one point relation and the trivial cocycle, which is injective !!!!!!!!}
%More concretely, we will compare our definition with {\em Rokhlin entropy} 
%and find out that for free actions, both notions coincide. 

\subsection{Cocycle entropy, finitary entropy and Rokhlin entropy}

The notion of Rokhlin entropy has its origin in Rokhlin's studies of entropy for a single transformation. It has recently been greatly generalized and studied intensively
in the context of general group actions by Seward in the series of papers \cite{Se15a, Se15b, Se16}. 
Is particular, Seward defines  
\begin{Definition}[Rokhlin entropy]
Let $\Gamma \curvearrowright (X,\lambda)$ be a p.m.p.\@ ergodic group action. Then, the number
\[
h^{\operatorname{Rok}}(\Gamma \curvearrowright X) := \inf\big\{ H(\cP)\,|\, 
\cP \mbox{ countable, generating partition}\big\}
\]
is called the {\em Rokhlin entropy} of the group action. 
\end{Definition}

Seward and Tucker-Drob showed in \cite{ST12}
that for free ergodic actions of amenable groups, Rokhlin entropy coincides
with the classical Kolmogorov-Sinai entropy. 
In a recent important breakthrough  Seward \cite{Se16} established the following upper bound for the Rokhlin entropy for every countable, generating partition $\cP$, assuming the $\Gamma$ action on $X$ is essentially free. 
\begin{eqnarray}\label{thm:seward}
h^{\operatorname{Rok}}\big( \Gamma \curvearrowright X \big) \leq 
\inf_{T \in Fin(\Gamma)} \frac{1}{|T|}\, H\Big( \bigvee_{g \in T} g^{-1}\cP \Big).
\end{eqnarray}
(In fact, Theorem~1.5 in \cite{Se16} shows a stronger statement
involving entropy conditioned on $\Gamma$-invariant $\sigma$-algebras.) 



Let us now note that by the standard subadditivity property of Shannon entropy (see Proposition~\ref{prop:propertiesofH}~(iii) below), the right hand side of (\ref{thm:seward}) is bounded from above by $H(\cP)$. 
Hence, by passing to the infimum over all generating partitions, we obtain 
in fact equality of the lower and the upper bound. Let us therefore define the following notion of {\it finitary entropy}.  

%The following notion of entropy is inspired by the investigations of
%Seward mentioned above. (:::ORIGIN:::?)

\begin{Definition}[Finitary entropy]
%We define
%{\em finitary entropy} as
\[
h^{\operatorname{fin}}\big( \Gamma \curvearrowright X \big)
:= \inf\Big\{ \inf_{T \in Fin(\Gamma)} H\Big( \bigvee_{g \in T} g^{-1}\cP \Big) /|T| \, \,\Big|\, 
\cP \mbox{ countable, generating partition}\Big\}.
\]
\end{Definition}
By the preceding discussion, it follows from Seward's inequality (\ref{thm:seward}) that these two notions of entropy coincide for ergodic (essentially) free actions: $h^{\operatorname{fin}}\big( \Gamma \curvearrowright X \big)=h^{\operatorname{Rok}}(\Gamma \curvearrowright X)$. 


Using this result, it is not hard to show that our definition of cocycle 
entropy above gives rise to the same value as well. 
%It turns out that for essentially free p.m.p.\@ actions, the concepts
%of cocylce entropy, finitary entropy and Rokhlin entropy conincide. 
%This leads to the following theorem, which is a straightforward 
%consequence of Seward's result, given by inequality~\eqref{thm:seward}.


\begin{Theorem} \label{thm:ENT}
Assume that $\Gamma \curvearrowright (X,\lambda)$ is an ergodic essentially free p.m.p.\@ action. Then,
for every amenable p.m.p.\@ equivalence relation $\cR$ over $(Y, \nu)$ and every class injective cocycle 
$\alpha: \cR \to \Gamma$, we obtain
\begin{eqnarray*}
h^{\operatorname{fin}}\big( \Gamma \curvearrowright X \big)=h^C\big(\Gamma \curvearrowright X, \alpha\big) = h^{\operatorname{Rok}}\big(\Gamma \curvearrowright X\big).
\end{eqnarray*}
\end{Theorem}

\begin{proof}
Let $\cP$ be a countable partition with finite Shannon entropy. Assume that $(\cR_n)$
is any bounded hyperfinite exhaustion for $\cR$. Since the 
cocycle $\alpha$ is class injective, and $\cR_n(y)$ is a finite set almost surely, we obtain 
\begin{eqnarray*}
\inf_{T \in Fin(\Gamma)} \frac{1}{|T|}\,H\Big( \bigvee_{g \in T} g^{-1}\cP \Big)
\leq \frac{h^{\cP}\big( \cR_n \big)(y)}{\big| \cR_n(y) \big|} \leq H\big( \cP \big)\,,
\end{eqnarray*}
for each $n \in \NN$ and for $\nu$-almost every $y \in Y$.
Note that these inequalities remain valid even if $H(\cP) = \infty$.
In the case that there exist generating partitions with finite Shannon 
entropy, we 
integrate over $Y$ and pass to the limit as $n\to \infty$. By Theorem \ref{thm:MAINcocycleentropy} we conclude that $ h^{\operatorname{fin}}(\Gamma \curvearrowright X)\le h^\ast_\cP(\alpha)\le H(\cP)$
independently of the cocycle $\alpha$. Now taking the infimum over all generating partitions,  since by the discussion above % by inequality~\eqref{thm:seward} 
$h^{\operatorname{Rok}}(\Gamma \curvearrowright X) = h^{\operatorname{fin}}(\Gamma \curvearrowright X)$, the above inequality immediately implies equality of all three notions of entropy. 

%{\bf Using Theorem 2.1 requires BOUNDEDNESS of $\cF_n $ ????????}
%Let $\cP$ be an arbitrary partition with $H(\cP)< \infty$. For a finite set $F \subseteq \Gamma$,
%we write $\cP^F := \bigvee_{g \in F} g^{-1}\cP$. It is clear that for each $n \in \NN$, 
%\begin{eqnarray*}
%\inf_{\emptyset \neq F \, \operatorname{fin}} \frac{H\big( \cP^F \big)}{|F|} \leq \int_Y h^{\cP}_n(y) \,d\nu(y)
%\end{eqnarray*}
%On the other hand, the right hand side of the latter inequality is clearly bounded above by $H(\cP)$. 
%Passing to the infimum over all generating partitions $\cP$ with $H(\cP) < \infty$, we obtain
%\begin{eqnarray*}
%\inf_{\cP \, \operatorname{gen}} \inf_{\emptyset \neq\tilde{} F \, \operatorname{fin}} \frac{H\big( \cP^F \big)}{|F|}
%\leq h(X, \alpha) \leq h^{\operatorname{Rok}}(G \curvearrowright X). 
%\end{eqnarray*}
%By a theorem of Seward, cf.\@ \cite{Se16}, Theorem~1.5, for essentially 
%%free actions, the left hand side of the above double 
%inequality
%is equal to $h^{\operatorname{Rok}}(G \curvearrowright X)$ as well. This
%concludes the proof of the theorem. 
\end{proof}

To conclude this subsection, let us comment briefly on the relation of the concept of entropy described above to Bowen's sofic entropy introduced in \cite{Bo10c}. 
%Sofic entropy has been introduced 
%for p.m.p.\@ actions of sofic groups by Bowen in \cite{Bo10}
%and sharpened by Kerr and Li in \cite{KL11, Ke13}. 
First, we note the fact that 
for every sofic approximation $\Sigma$ of a sofic group $\Gamma$,
the corresponding sofic entropy $h_{\Sigma}^{\operatorname{sof}}
(\Gamma \curvearrowright
X)$ is bounded above by Rohklin entropy, cf.\@ \cite{Bo10c, Ke13}. 
It follows from Theorem~\ref{thm:ENT} that for essentially free actions,
the same is true if we 
replace Rokhlin entropy by cocycle entropy. 
Consequently, we can easily conclude that for sofic groups $\Gamma$ and a Bernoulli shift
$\Gamma \curvearrowright (\mathcal{A}^{\Gamma}, \bigotimes_{\Gamma} {\bf p})$
with $\cA$ being a countable alphabet and ${\bf p} = (p_a)$ denoting a probability vector
with $H({\bf p}):= -\sum_{a \in \cA} p_a\,\log p_a < \infty$, the cocycle entropy is
equal to $H({\bf p})$. Indeed, the cocycle entropy is less than or equal to $H({\bf p})$
since the sets $\big\{ [x_a = i], i \in \cA \big\}$ form a generating partition of $\cA^{\Gamma}$
with Shannon entropy $H({\bf p})$. Bowen shown that the sofic entropy
for Bernoulli shifts is equal to $H({\bf p})$, cf.\@ Theorem~8.1 in~\cite{Bo10c} (see
also the work of Kerr \cite{Ke13}, Theorem~4.2). Hence, since
cocycle entropy is bounded from below by sofic entropy, we obtain what we claimed.  
To the best of our knowledge, the question of determining Rokhlin entropy
for Bernoulli shifts over general countable groups is still open. 

\subsection{The Shannon-McMillan-Breiman theorem}
The generalization that we propose of the Shannon-McMillan-Breiman theorem concerns mean or pointwise almost everywhere convergence of a sequence of natural {\em information functions} on $X$. For
a given countable partition $\cP$ with $H(\cP) < \infty$, we set
\[
\cJ(\cP)(x):= - \log\lambda\big( \cP(x) \big),
\] 
where we define $\cP(x)$ to be the unique set $A \in \cP$ containing 
the point $x$ (namely the $\cP$-{\em name of} $x$).
Note that by definition, we have 
\[
H\big( \cP \big) = \int_X \cJ\big( \cP \big)(x)\,d\lambda(x)\,.
\]
%and this relation also shows that the information function is 
%well defined almost surely. \\
Given an ergodic
p.m.p.\@ group action $\Gamma \curvearrowright (X,\lambda)$, and a class-injective cocycle $\alpha : \cR \to \Gamma$ consider the 
extended relation $\cR^X$ over $X \times Y$. 
% is defined as
%\[
%\big( (x,y), (t,z) \big) \in \cR^X \Longleftrightarrow \, \exists\, g \in\Gamma:\,
%g = \alpha(z,y), \,\, gx = t.
%\]
%Note that $\cR^X$ is measure preserving and hyperfinite as well, cf.\@ SOURCE::::.
For the proof of the Shannon-McMillan-Breiman theorem, we will 
assume that the extended relation $\cR^X$ is {\em ergodic}. 
Though being a non-trivial assumption on the 
equivalence relation and the action under consideration, ergodicity of the extension is satisfied in 
many situations. One important example is that of arbitrary ergodic actions of irreducible lattices in connected semisimple Lie groups with finite center. In general, a sufficient condition in order
to guarantee ergodicity of $\cR^X$ is {\em weak mixing} of the relation
$\cR$ over $(Y,\nu)$, as defined in \cite{BN13a}.
  For a more detailed elaboration
of these issues, we refer the reader to Section  7. %~\ref{sec:SMB}. 

We are now
able to state the Shannon-McMillan-Breiman pointwise convergence theorem in our context.



\begin{Theorem}%[SMB-Theorem for hyperfinite equivalence relations] 
\label{thm:MAIN_SMBhyp}
Let $\Gamma \curvearrowright (X,\lambda)$ be an ergodic essentially free p.m.p.\@ action. 
%Let $\Gamma$ be a countable group which acts on a probability space $(X,\lambda)$
%by p.m.p. 
Assume that $\cR$ is an amenable, p.m.p.\@ equivalence
relation, and $\alpha : \cR\to \Gamma$ is a class injective cocycle, such that the extended relation $\cR^X$ is ergodic. 
%over a probability space $(Y,\nu)$ which gives rise to a measurable,
%fiberwise injective
%cocycle $\alpha: \cR \to \Gamma$ and such that the extended relation $\cR^X$ is ergodic. \\
Then, for every bounded hyperfinite exhaustion $(\cR_n)$ satisfying the growth condition
\[
\lim_{n \to \infty} \operatorname{ess}\,\operatorname{inf}_y |\cR_n(y)|/\log\,n = \infty 
\]
the information functions satisfy the following convergence property. 
Given a finite partition $\cP$ of $X$, for $(\lambda \times \nu)$-almost every $(x,y) \in X \times Y$, 
\[
\lim_{n \to \infty} \frac{\cJ\big( \cP^{\cR_n(y)}(x) \big)}{|\cR_n(y)|} = h^{*}_{\cP}(\alpha),
\]
where $h^{*}_{\cP}(\alpha)$ is the cocycle entropy of the partition
$\cP$. 
\end{Theorem}



%As pointed out before, those results are the first of their kind for actions
%of p.m.p.\@ actions of non-amenable groups. Thus, they go far beyond the 
%classical results, cf.\@ \cite{Ki75, OW83, Li01}. \\
Let us point out that the growth condition on the $(\cF_n)$ is very mild, and in fact one can always find bounded hyperfinite exhaustions
which satisfy it. To see this, recall that any two ergodic p.m.p.\@ actions of amenable
groups are orbit equivalent. This fact goes back to Dye \cite{Dy59} for a pair of (ergodic) 
$\ZZ$-actions and was stated in full generality in \cite{OW80}. For a survey on the
topic of orbit-equivalence, see also \cite{Ga00}. Now in \cite{CFW81} it was shown that any amenable equivalence relation is hyperfinite and any hyperfinite relation is generated by the action of a single transformation. This action is orbit equivalent to the standard odometer action, and we can transfer the  hyperfinite exhaustion of the odometer to the underlying equivalence relation (using the orbit equivalence). It follows that bounded hyperfinite
exhaustions satisfying the growth condition required in Theorem~\ref{thm:MAIN_SMBhyp} always exist. 

We note that the analogous growth condition for (tempered) F{\o}lner sequences appears also in E. Lindenstrauss'
proof of the Shannon-McMillan-Breiman theorem for amenable groups, see Theorem~1.3 in~\cite{Li01}. In the survey paper \cite{W03} Weiss gives a deterministic combinatorial proof of the Shannon-McMillan-Breiman theorem based on previous joint work of Ornstein and Weiss. 
This proof is valid for general tempered
F{\o}lner sequences even without an additional growth condition and also for non-ergodic
actions, see Theorem~6.2 in \cite{W03}.

As a corollary of Theorem~\ref{thm:MAIN_SMBhyp}, we obtain the corresponding 
Shannon-McMillan theorem, asserting convergence of the information functions in $L^1$. 

\begin{Corollary} \label{cor:MAIN_SML1}
 Convergence in Theorem \ref{thm:MAIN_SMBhyp} holds
for $\nu$-almost every $y \in Y$ in the $L^1(X,\lambda)$-norm, and also  
in the $L^1(X \times Y, \lambda \times \nu)$-norm. 
\end{Corollary}

%Due to lack of an argument in order to bound the information function, 
We do not know whether
one can also expect convergence in 
$L^1(Y,\nu)$ for $\lambda$-almost every $x \in X$. 
%This would follows from a suitable bound on the information functions, which is currently unavailable. 


\section{Amenable equivalence relations} \label{sec:amenable}

In this section, we discuss measured Borel 
equivalence relations which are amenable in the sense of 
Connes, Feldman and Weiss \cite{CFW81}. This condition was shown in that paper to be equivalent to hyperfiniteness. 


\subsection{Measurable equivalence relations}
Consider a Borel measurable equivalence relation $\cR$ defined
over a standard Borel probability space $(Y, \mathcal{B}(Y), \nu)$, namely  $\cR$ is a Borel measurable subset
of $Y \times Y$ with the properties
\begin{itemize}
\item $(y,y) \in \cR$ for all $y \in Y$,
\item if $(y,z) \in \cR$, then $(z,y) \in  \cR$ for all $y,z \in Y$,
\item if $(y,z) \in \cR$ and $(z,w) \in \cR$, then $(y,w) \in \cR$ for all $y,z,w \in Y$.
\end{itemize}
Two points $y$ and $z$ are called $\cR$-equivalent points if $(y,z) \in \cR$,  and the equivalence class is denoted $\cR(y)=[y]=[y]_\cR$. Each $y\in Y$ determines a left and a right fiber in $\cR$, given by 
 $\cR^y=\set{(y,z)\,;\, z\cR y}\subset \cR$ and $\cR_y=\set{(z,y)\,;\, z\cR y}\subset \cR$. 
We will always assume that for almost every $y$, the fiber $\cR^y$ (and hence also $\cR_y$) 
is countable. $c^y$ will denote the counting measure on $\cR^y$, and $ c_y$ the counting measure on $\cR_y$. Integrating the counting measures
over $Y$, we obtain two $\sigma$-finite (but in general not finite) measures on $\cR$, namely 
$\tilde{\nu}_l=\int_Y c^y d\nu(y)$ and $\tilde{\nu}_r=\int_Y c_yd\nu(y)$. The measure $\nu$ on $Y$ is called $\cR$-non-singular if these two measures are equivalent. Note that $(\cR,\tilde{\nu}_l)$ is a standard Borel space, and $\pi_l:\cR\to Y$ given by $\pi_l(y,z)=y$ is a measurable factor map, and similarly for $\tilde{\nu}_r$ and $\pi_r$.   Note that under the coordinate projection $\pi_l :\cR\to Y$,  
the integral above expresses the measure disintegration of $\tilde{\nu}_l$ with respect to $\nu$. 
If $\tilde{\nu}_l=\tilde{\nu}_r$ then we denote it by $\tilde{\nu}$, and then $\tilde{\nu}$ as well as $\nu$  are called {\it $\cR$-invariant}, and  $\cR$ is  called a probability-measure-preserving (p.m.p.) equivalence relation. This is the only case we will consider below.

%Equivalently $m$ is $\cR$-invariant if and only if $m_r= m_l \circ j$ where $j:\cR \to \cR$ is defined by $j(y,z):= (z,y)$. 
%We will only consider the case of $m$ being $\cR$-invariant.
 %Other important objects for our investigation are 
 
An {\em inner automorphism} of the relation $\cR$ is a measurable mapping
$\phi: Y \to Y$ which is almost surely bijective with measurable inverse
and with its graph $\operatorname{gr}(\phi)$ being contained in $\cR$.
The collection of all inner automorphisms gives rise to a group
$\operatorname{Aut}(\cR)=[\cR]$, called the {\em full group} of $\cR$. 
A countable subset $\Phi_0 \in \operatorname{Aut}(\cR)$ is said to be {\em generating}
(for $\cR$) if for $\tilde{\nu}$-almost all $(y,z) \in \cR$ there is some $\phi \in \Phi_0$
such that $z= \phi(y)$. 
%Fur our purposes, we will assume that there
%is a set $\Phi$ which is generating for $\cR$ and which generates 
$\Phi_0$ of course generates a {\em countable} subgroup of $\operatorname{Aut}(\cR)$, denoted $\Phi$.

For measurable subsets $A, B \subseteq Y$ of positive measure, we say that $\psi$ is a {\em
partial transformation} if $\psi: A \to B$ is measurable, 
essentially bijective with measurable inverse and again,  $\operatorname{gr}(\psi) \subseteq
\cR$.  
%Furthermore, we denote the space of all fiberwise finite subsets of $\cR$ by $\cF(\cR)$. 
The space $Fin(Y)$ of finite subsets of a Borel space $Y$ is a Borel space in a natural way, using the obvious Borel structure on $\bigcup_{n\in \NN}Y^n/Sym(n)$. 
Measurable mappings of the form $\cF: Y \to Fin(Y)$ satisfying that for almost every $y$, the set $\cF(y)$ 
consists of finitely many points equivalent to $y$, are called 
{\em subset functions} of $\cR$. The possibility that $\cF(y)=\phi$ is the empty set is allowed. We consider two subset functions $\cT, \cS$ to be equal if 
the set $\{y \,|\, \cS(y) \neq \cT(y)\}$ has zero measure. We write $\cS\subset \cT$ if $\cS(y)\subset \cT(y)$ for $\nu$-almost every $y$. 
 Subset functions
can be composed with each other, inverted, and subtracted from each other. We refer to \cite{BN13a} and \cite{BN13b} for a full discussion, and recall here the following definitions. 
\begin{eqnarray*}
\cS \circ \cT(y) &:=& \bigcup_{z \in \cT(y)} \cS(z), \\
\cT^{-1}(y)  &:=& \big\{ z \in [y] \,|\, y \in \cT(z)  \big\}, \\
\big( \cS \setminus \cT \big) (y) &:=& \cS(y) \setminus \cT(y). 
\end{eqnarray*} 
%We will equivalently write $\cS\cT$ instead of 
%$\cS \circ \cT$ for the composition of two subset functions. 
A finite non-empty set $D \subset \operatorname{Aut}(\cR)$ gives rise to the subset function 
$\cD(y)=\set{\phi(y)\,;\, \phi\in D}$. Given a 
subset function $\cT$ and a finite set $D \subset \operatorname{Aut}(\cR)$, $\cD\circ \cT$ is defined as above, and is given by 
\[
\cD \circ\cT(y) := \bigcup_{\phi \in D} \phi\big( \cT(y) \big) .
\]
We will also use the notation $D\circ \cT$ for this expression. 
%Note that since inner automorphisms preserve the equivalence relation
%structure, this operation gives rise to subset functions as defined
%above. 
%In some situations, we will need to transform a given 
%subset function $\cT$ by a combination $\cD$ of a finite set 
%$D \in \operatorname{Aut}(\cR)$ and a bounded subset function $S$. 
%For this, we write and define $\cD \circ T:= D\circ \big(S\circ T \big)$,
%which, again, defines another subset function. 
A subset $\cK \subseteq \cR$ is said to be {\em bounded} if  
\[
\|\cK\|:= \operatorname{ess-sup}_y \max \big\{ |\cK_y|, |\cK^y| \big\}< \infty,
\]
where $\cK_y := \big\{ z \in [y] \,|\, (z,y) \in \cK \big\}$
and $\cK^y:= \big\{ z \in [y]\,|\, (y,z) \in \cK \big\}$.
Analogously, we say that a subset function $\cT$ is {\em bounded}, if
$\|\cT\| := \operatorname{ess-sup}_y \max\, \big\{ |\cT(y)|, |\cT^{-1}(y)| \big\}$
is finite. 
%In case of a combination $\cD$ of a finite set $D \subset \operatorname{Aut}(\cR)$
%with a bounded subset function $S$, we use the notation
%$\|D\|:= \|D \circ S\|$.


\subsection{Hyperfiniteness and amenability}

The relation $\cR$ is called hyperfinite if there exists a
sequence $(\cR_n)$ consisting of subrelations $\cR_n\subset \cR$, where each $\cR_n$ has finite
classes, satisfying 
\[
\cR_n \subseteq \cR_{n+1}\,\,\,,\,\,\,\cR = \bigcup_{n=1}^{\infty} \cR_n.
\] 
We refer to such a sequence as a {\em hyperfinite exhaustion} of
$\cR$. Note that each $\cR_n$ is a subset function as defined above. 
If each $\cR_n$ is a bounded subset function we will call the hyperfinite exhaustion a {\it bounded hyperfinite exhaustion}. 
%{\bf We will discuss the notion of amenability of equivalence relations and some of its equivalent forms in detail in section \ref{sec:folneramenable} ????????}. Here we just 
Recall that it was proved by Connes, Feldman and Weiss  \cite[Thm. 10]{CFW81} that $\cR$ being amenable
is equivalent to $\cR$ being {\em hyperfinite}. 





%We now recall (see \cite{BN13a}\cite{BN13b}) the notion of bounded subset functions which have 
%an asymptotic invariance (F\o lner type) property with respect to inner automorphisms 
%of the relation. 

%\begin{Definition}[Asymptotically invariant sequences] \label{defi:asinv}
%A sequence $(\cF_n)$ of bounded subset functions is called {\em asymptotically invariant}
%with respect to $\cR$ if there is a countable set $\Phi \in \operatorname{Aut}(\cR)$
%which generates $\cR$ and such that for all $\phi \in \Phi$, 
%%and every 
%$m \in \NN$, 
%we have
%\[
%\lim_{n \to \infty} \frac{\big| \cF_n(y) \, \triangle\, \phi\big(\cF_n(y) \big) \big|}
%{\big|\cF_n(y) \big|} = 0
%\]
%for $\nu$-almost every $y \in Y$.
%\end{Definition}
%{\bf DO WE STILL NEED THIS DEFINITION?}

%Sequences $(\cR_n)$ as in the above definition 
%are a natural replacement for F{\o}lner sequences
%in amenable groups.
Let us note that 
hyperfinite exhaustions are asymptotically invariant under inner automorphisms of finite rank, in the following sense (see also \cite{BN13a}, \cite{BN13b}).

\begin{Proposition} \label{prop:finitephin}
If $(\cR_n)$ is a bounded hyperfinite exhaustion of the relation $\cR$, then there 
exists an increasing sequence of finite subgroups $\Phi_n \subseteq \operatorname{Aut}(\cR)$, $n \geq 1$,
such that $\Phi := \bigcup_{n \geq 1} \Phi_n$ is generating for $\cR$ 
and for all $n \in \NN$ and $\phi \in \Phi_n$, the graph $\operatorname{gr}(\phi)$
is contained in $\cR_n$.

In particular, for every $\phi \in  \Phi $,
there is an $n_0 \in \NN$ such that $\cR_n(y) \, \triangle \, \phi(\cR_n(y)) = \emptyset$ 
for $\nu$-almost every $y \in Y$ and every $n \geq n_0$. 
\end{Proposition}  
%{\bf Note that this proof uses BOUNDEDNESS of $\cR_n$ !!!!!!!!}
\begin{proof}
Let $\cT$ be any equivalence subrelation of $\cR$ with finite classes of bounded size. We can divide $Y$ to finitely many measurable $\cT$-invariant sets where the size of $\cT(y)$ is fixed, and we can restrict $\cT$ to one of them. Without loss of generality we can thus assume that $\cT$ has classes of fixed size $N$ in $Y$.  As is well known (see e.g. \cite[\S 4]{FM77}, or \cite[Lem. 3b]{CFW81}) the factor space $Y/\cT$ consisting of equivalence classes of $\cT$ is a standard Borel space and admits measurable sections $J_1, \dots,J_N : Y/\cT \to Y$ such that $\set{J_i(\cT(y))\,;\, 1\le i\le N}=[y]_\cT=\cT(y)$ for almost every $y$. Define the cyclic permutation $\sigma_{\cT(y)}$ given by 
$J_1(\cT(y))\mapsto  J_2(\cT(y))\mapsto  \cdots J_N(\cT(y))\mapsto J_1(\cT(y))$ in each class, whose cycle length is $N$. Denote by $\phi_\cT$ the map on $Y$ which coincides with $\sigma_{\cT(y)}$ on each class $\cT(y)$, and denote the cyclic group generated by $\phi_\cT$ by $\Phi_\cT$. Then $\phi_\cT$ is measurable and constitutes an inner automorphism of $\cR$ which leaves invariant almost every class of the relation $\cT$, and the group $\Phi_\cT$ generates the relation $\cT$.  

Applying this procedure to each of the finite bounded relations $\cR_n$ and taking the union of the corresponding groups, the stated result follows. 

\end{proof} 

%\begin{proof}
%Let $n \in \NN$. By \cite[Thm. 10]{CFW81}, the relation $\cR_n$ can be generated by one finite-order 
%p.m.p transformation $T_n$ on $Y$. If $N$ is the order of $T_n$, then 
%%Set $N:= \| \cR_n \|$. Since $T_n(y) = y$ if and only if $[y] = \{y\}$(for $y \in Y$), 
%the collection of inner automorphisms defined by 
%\[
%\Phi_n := \big\{ T_n^{i} \,|\, -N \leq i \leq N \big\}
%\]
%is generating for $\cR_n$. By definition, we have $|\Phi_n| \leq 2N+1$,
%in particular, $\Phi_n$ is finite. Since $(\cR_n)$ generates $\cR$
%as a union of subset functions, the set $\Phi:= \bigcup_{n=1}^{\infty} \Phi_n$
%is generating as a set of inner automorphisms. Now take $\phi$ out of
%the group $\langle \Phi \rangle$ generated by the countable set
%$\Phi$. Then $\phi$ is a finite composition of elements taken from
%the $\Phi_n$. Set $n_0 \in \NN$ to be the largest of those $n$
%such that one component of $\phi$ is contained in $\Phi_n$.
%Then for all $n \geq n_0$, we have that $\operatorname{gr}(\phi) \subseteq 
%\cR_n$. It follows that $\phi\big( \cR_n \big) = \cR_n$ for those $n$,
%establishing the second part of the proposition.  
%\end{proof}
%



 


%\begin{proof}
%Assume that $(\cF_n)$ is a F{\o}lner sequence with  
%$\cF_n \subseteq \cF_{n+1}$ for all $n \in \NN$. Then, there is some countable
%set $\Phi \subseteq \operatorname{Aut}(\cR)$ which is generating for $\cR$ and 
%such that for every $\phi \in \Phi$,
%%%%and 
%%%%such that for $m$-almost every $(y,z) \in \cR$,
%%%we get
%\begin{eqnarray} \label{eqn:reiter}
%\lim_{n \to \infty} \frac{\big| \cF_n(y) \,\triangle\, \phi\big( \cF_n(y)\big) \big|}{\big| \cF_n(y) \big|} = 0
%\end{eqnarray}
%for $\nu$-almost every $y \in Y$. 
%By Proposition~\ref{prop:approxrelation}, for every $n \in \NN$, we find finite equivalence 
%relations $\cT = \cR^{\prime}_n$ with $\cR^{\prime}_n \subseteq \cF_n$ and 
%$\lim_n m\big( \cF_n \setminus \cR^{\prime}_n \big) = 0$. We define
%$\cR_n := \bigcup_{j \leq n} \cR_j^{\prime}$. Since $
%\cR^{\prime}_n \subseteq \cF_n \subseteq \cF_{n+1}$,
%we obtain $\cR_n \subseteq \cF_n$ for all $n \in \NN$. The integrated approximation
%condition is clearly satisfied by definition of the measure $m$. 
%In order complete the proof 
%of assertion~(i), we still need to show that $\coprod_{n \geq 1} \cR_n
%= \cR$. Since this will follow from the considerations concerning pointwise 
%approximation, we immediately turn to the proof of the second part 
%of the proposition.
%Concerning the assertion~(ii) about pointwise approximation, we can pass to a subsequence,
%say $(n_k)$ in order to obtain
%\begin{eqnarray} \label{eqn:reiter2}
%\lim_{k\to \infty} \frac{\big| \cF_{n_k}(y) \setminus \cR_{n_k}^{\prime}(y) \big|}
%{\big| \cR_{n_k}^{\prime}(y) \big|}  =0
%\end{eqnarray}
%for $\nu$-almost all $y \in Y$. 
%It remains to show that $(\cR^{*}_{k})$ with $\cR_k^{*} := \bigcup_{j=1}^k \cR^{\prime}_{n_j}$ 
%is a hyperfinite exhaustion. (Then, since indeed $\cR_k^{\prime} \subseteq \cR_k^{*} \subseteq \cF_{n_k}$ for
%all $k$, it follows that $\cF_{n_k}$ is an approximate
%hyperfinite exhaustion.) By definition, we have $\cR_k^{*} \subseteq \cR_{k+1}^{*}$.
%Moreover, note that by~\eqref{eqn:reiter} 
%and~\eqref{eqn:reiter2}, we have
%\[
%\lim_{k \to \infty} \frac{\big| \cR_{n_k}^{\prime}(y) \, \triangle\, \phi\big(\cR_{n_k}^{\prime}(y) \big) \big|}
%{\big| \cR_{n_k}^{\prime}(y) \big|}= 0
%\] 
%for all $\phi \in \Phi$ and for $\nu$-almost every $y \in Y$. 
%So assume $y$ is taken from the convergence set and assume that $(y,z) \in\cR$. Since the
%set $\Phi$ is generating for $\cR$, there is some $\phi \in \Phi$ such that $z=\phi(y)$.
%Now the latter limit relation implies that for large enough $k$, there must 
%be some element $w_k \in \cR_{n_k}^{\prime}(y) \cap \phi\big(\cR_{n_k}^{\prime}(y)\big)$. 
%This, in turn, gives that $z$ is $\cR^{*}_k$-equivalent to $y$. Hence, we get indeed 
%$\coprod_{k} \cR^{*}_k = \cR$ and the proof is finished. 
%\end{proof}

%\begin{Remark}
%Thus whenever an  asymptotically invariant (F{\o}lner) sequence satisfies the condition 
%$\bigcup_n \cF_n = \cR$, by Theorem~\ref{thm:MAIN_folneramenable} below,
%it gives rise to (pointwise)
%approximate hyperfinite exhaustions, cf.\@ e.g.\@ 
%\cite{BN13a,BNhyp}. \\
%%(:::Question: Is $\cup_n \cF_n = \cR$ automatically true for F{\o}lner sequences? I guess not:::)
%\end{Remark}


%However, for most of our considerations, we have to deal with 
%sequences which are at the same time pointwise hyperfinite approximations 
%and asymptotically invariant.
%Again, the natural candidates for this are hyperfinite exhaustions. 

We proceed to state and prove another lemma which will be useful in our considerations below. 
The statement of part (i) is very close to \cite[Lem. 2.8, 2.9]{BN13b}, but we give full details in order to eliminate the assumption of uniformity of the sets $\cR_n$ that appears there.  
Part (ii) is close to \cite[Cor. 2.3]{Da01} but we will use the form stated below so again give full details. 


\begin{Lemma} \label{lemma:hyperfinitefolner}
Let $\cR$ be hyperfinite and $(\cR_n)$ a bounded hyperfinite exhaustion.   
%a composition of a finite set 
%$D \subset \operatorname{Aut}(\cR)$ and a bounded subrelation $\cT$ of $\cR$. 
%Then  
\begin{enumerate}[(i)]
\item
Assume that the subset function $\cD$ is bounded. Then

\[
\lim_{n \to \infty} \int_Y \frac{\big| \cR_n(y)\,\triangle\, \cD\circ\cR_n(y) \big|}
{\big| \cR_n(y) \big|} \,d\nu(y) = 0.
\]
\item Assume that  $\cT\subset \cR$ is a bounded sub-equivalence relation. For every $\varepsilon > 0$, and for all large enough $n \in \NN$ 
(depending on $\varepsilon$) 
there is a set $Y_n \subseteq Y$ with $\nu(Y_n) \geq 1- {\varepsilon}$ such that for all
$y \in Y_n$
\begin{eqnarray} \label{eqn:hyperfinitefolner}
\frac{\big|\big\{ z \in \cR_n(y)\,|\, \cT(z) \subseteq \cR_n(y)  \big\}\big|}
{\big| \cR_n(y) \big|}
> 1- \varepsilon\cdot \norm{\cT}.
\end{eqnarray}
\end{enumerate}
\end{Lemma}


\begin{proof}
(i). First let us assume that $\cD$ is defined by a finite set of inner automorphisms $D=\set{\psi_i, 1\le i \le N}\subset [\cR]$. It suffices to show that for each $1\le i\le N$
$$\lim_{n \to \infty} \int_Y \frac{\big| \cR_n(y)\, \triangle \, \psi_i(\cR_n(y)) \big|}
{\big| \cR_n(y) \big|} \,d\nu(y) = 0.$$
%Fix $\psi=\psi_i^{\pm 1}$ {\bf ???}. There exist inner automorphisms $\phi_j, j\in \NN$ in $\Phi$ (as defined in the previous Lemma), such that  $\psi(x)=\phi_{j}(x)$ on a set $Y_j$, and %$\coprod_{j\in \NN} Y_j=Y$, since $\Phi$ generates the relation. Fix $\epsilon > 0$, and $N(\epsilon)$ so that $\nu\left(\coprod_{j=1}^{N(\epsilon)} Y_j\right) > 1-\epsilon$. Denoting $Z=\coprod_{j > %N(\epsilon)} Y_j$, and  considering $\cR_n(y)\cap  Z $ and $\cR_n(y)\cap (Y\setminus Z)$, we conclude {\bf TRUE!?!?!?}
%$$ \cR_n(y)\,\setminus \psi(\cR_n(y))\subset \left( \cR_n(y)\,\setminus\bigcup_{j=1}^{N(\epsilon)} \, \phi_j(\cR_n(y)) \right) \bigcup \left(\cR_n(y)\cap Z\right)$$
%and hence 
%$$\int_Y  \frac{\abs{\cR_n(y)\,\setminus \psi(\cR_n(y))}}{\abs{\cR_y}}d\nu(y) \le  \sum_{j=1}^{N(\epsilon)}\int_Y\frac{\abs{ \cR_n(y)\,\setminus\, \phi_j(\cR_n(y))}}{\abs{\cR_n(y)}}d\nu(y)+\int_Y %\frac{\abs{\cR_n(y)\cap Z}}{\abs{\cR_n(y)}}d\nu(y)$$
%Since $\phi_j(\cR_n(y))=\cR_n(y)$ for fixed $j$ and all sufficiently large $n$, the first summand of the r.h.s. converges to zero as $n\to \infty$. The second summand converges to $\nu(Z)$ by the %martingale convergence theorem. Finally, a general bounded subset $\cU$ has the property that there exists a finite set $D$ of inner automorphisms such that $\cU(y)\subset \set{\psi(y)\,;\, \psi \in %D}=:\cD(y)$ (see the proof of \cite[Lem. 2.9]{BN13b}), and the result follows. 
Using the invariance of the measure $\tilde{\nu}$ and $\cR_n(y) = \cR_n(z)$ for $z \in \cR_n(y)$, 
we compute for $\psi = \psi_i$
\begin{eqnarray*}
\int_Y \frac{\big| \psi\big(\cR_n(y)\big) \, \triangle\, \cR_n(y) \big|}{\big| \cR_n(y) \big|}\, d\nu(y) &\leq&  
\int_Y \frac{1}{\big| \cR_n(y) \big|} \sum_{z \in \cR_n(y)} \big| \{\psi(z)\} \setminus \cR_n(y) \big|\, d\nu(y) \\
&& \quad\quad + \int_Y \frac{1}{\big| \cR_n(y) \big|} \sum_{z \in \cR_n(y)} 
\big| \{\psi^{-1}(z)\} \setminus \cR_n(y) \big|\,  d\nu(y) \\
&=& \int_Y \sum_{y \in \cR_n(z)} \big| \cR_n(z) \big|^{-1}  \big| \{\psi(z)\} \setminus \cR_n(z) \big|\, d\nu(z) \\
&& \quad\quad + \int_Y \sum_{y \in \cR_n(z)} \big| \cR_n(z) \big|^{-1} \, \big| 
\{\psi^{-1}(z)\} \setminus \cR_n(z) \big|\, d\nu(z) \\
&=& \int_Y \big| \{\psi(z)\} \setminus \cR_n(z) \big|\, d\nu(z) + \int_Y \, \big| 
\{\psi^{-1}(z)\} \setminus \cR_n(z) \big|\, d\nu(z). 
%&\leq& \int_Y \sum_{y \in \cR_n(z)} \big| \cR_n(y) \big|^{-1}\,\big| \cT(z) \setminus \cR_n(y) \big|\, d\nu(z) \\
%&& \quad \quad  + \int_Y \big| \cR_n(y) \big|^{-1} \sum_{z \in \cR_n(y)} \big| D^{-1}(z) \setminus \cR_n(y) \big|\, d\nu(y)
%&=& \int_Y \sum_{y \in \cR_n(z)} \big| \cR_n(z) \big|^{-1}\,\big| \cT(z) \setminus \cR_n(z) \big|\, d\nu(z) \\
%&=& \int_Y \big| \cT(z) \setminus \cR_n(z) \big|\,d\nu(z).
\end{eqnarray*}
Now since the union of the $\cR_n$ is the full relation $\cR$, we obtain that the integrand tends
to zero pointwise almost surely as $n \to \infty$. Since all integrands are bounded by $1$, 
we can use the dominated convergene theorem to deduce that the above integrals converges to zero as well. 
 In the general case, for every bounded subset function $\cU$, there exists a finite set of inner automorphisms $D$ such that $\cU(y)\subset \cD(y)$ for a.e. $y$, as shown in the proof of Lemma 2.9 in \cite{BN13b}. This establishes
the validity of the assertion (i).\\
%The invariance of the measure $\tilde{\nu}$ on $\cR$ means that for every kernel $K(y,z)$ on $\cR$
%\[\int_Y \sum_{z\cR y} K(y,z) d\nu(y)=\int_Y \sum_{y\cR z} K(y,z)d\nu(z)\,.\]
%Recall that  $\cD\circ \cR_n(y)=\bigcup_{z\in \cR_n(y)} \cD(z)$.
%Thus 
%$$ \cD\circ\cR_n(y) \, \triangle\, \cR_n(y)=%\big(\bigcup_{z\in \cR_n(y)} \cD(z)\big)\triangle \cR_n(y)
%\big(\bigcup_{z\in \cR_n(y)} \cD(z)\setminus \cR_n(y)\big)\cup \big(\cR_n(y)\setminus \bigcup_{z\in \cR_n(y)} \cD(z)\big)
%$$
%Therefore using $\cR_n(y) = \cR_n(z)$ for $z \in \cR_n(y)$, and applying the invariance of the measure to the kernels 
%$$K_1(y,z)= \frac{1}{\big| \cR_n(y) \big|} \sum_{z \in \cR_n(y)} \big| \cD(z) \setminus \cR_n(y) \big|\,\,\text{ and } 
%K_2(y,z)=\frac{1}{\big| \cR_n(y) \big|} \sum_{z \in \cR_n(y)}  
%\big|  \cR_n(y)\setminus \cD(z) \big|\, $$
%we compute 
%$$
%\int_Y \frac{\big| \cD\circ\cR_n(y) \, \triangle\, \cR_n(y) \big|}{\big| \cR_n(y) \big|}\, d\nu(y) \leq $$  
%$$\int_Y \frac{1}{\big| \cR_n(y) \big|} \sum_{z \in \cR_n(y)} \big| \cD(z) \setminus \cR_n(y) \big|\, d\nu(y) + \int_Y \frac{1}{\big| \cR_n(y) \big|} \sum_{z \in \cR_n(y)}  
%\big|  \cR_n(y)\setminus \cD(z) \big|\,  d\nu(y) $$
%$$= \int_Y \sum_{y \in \cR_n(z)} \big| \cR_n(z) \big|^{-1}  \big| \cD(z) \setminus \cR_n(z) \big|\, d\nu(z) 
%+ \int_Y \sum_{y \in \cR_n(z)} \big| \cR_n(z) \big|^{-1}  \big|  \cR_n(z)\setminus \cD(z) \big|\, d\nu(z) $$
%%&&+ \int_Y \sum_{y \in \cR_n(z)} \big| \cR_n(z) \big|^{-1} \,|D|\, \max_{\phi\in D} \big| 
%%\phi^{-1}(z) \setminus \cR_n(z) \big|\, d\nu(z) \\
%$$= \int_Y \big| \cD(z) \setminus \cR_n(z) \big|\, d\nu(z) +\int_Y \big| \cR_n(z) \setminus \cD(z) \big|\, d\nu(z) \,.$$
% %+ \int_Y |D|\, \max_{\phi\in D} \big| 
%%\phi^{-1}(z) \setminus \cR_n(z) \big|\, d\nu(z). $$
%%&\leq& \int_Y \sum_{y \in \cR_n(z)} \big| \cR_n(y) \big|^{-1}\,\big| \cT(z) \setminus \cR_n(y) \big|\, d\nu(z) \\
%%&& \quad \quad  + \int_Y \big| \cR_n(y) \big|^{-1} \sum_{z \in \cR_n(y)} \big| D^{-1}(z) \setminus \cR_n(y) \big|\, d\nu(y)\\
%%&=& \int_Y \sum_{y \in \cR_n(z)} \big| \cR_n(z) \big|^{-1}\,\big| \cT(z) \setminus \cR_n(z) \big|\, d\nu(z) \\
%%&=& \int_Y \big| \cT(z) \setminus \cR_n(z) \big|\,d\nu(z).
%%\end{eqnarray*}
%%
%
%
%
%
%%
%%\begin{eqnarray*}
%%\int_Y \frac{\big| \cD\circ\cR_n(y) \, \triangle\, \cR_n(y) \big|}{\big| \cR_n(y) \big|}\, d\nu(y) &\leq&  
%%\int_Y \frac{1}{\big| \cR_n(y) \big|} \sum_{z \in \cR_n(y)} \big| \cD(z) \setminus \cR_n(y) \big|\, d\nu(y) \\
%%&& \quad\quad + \int_Y \frac{1}{\big| \cR_n(y) \big|} \sum_{z \in \cR_n(y)}  |D|\,\max_{\phi \in D} \,
%%\big| \phi^{-1}(z) \setminus \cR_n(y) \big|\,  d\nu(y) \\
%%&=& \int_Y \sum_{y \in \cR_n(z)} \big| \cR_n(z) \big|^{-1}  \big| \cD(z) \setminus \cR_n(z) \big|\, d\nu(z) \\
%%&& \quad\quad + \int_Y \sum_{y \in \cR_n(z)} \big| \cR_n(z) \big|^{-1} \,|D|\, \max_{\phi\in D} \big| 
%%\phi^{-1}(z) \setminus \cR_n(z) \big|\, d\nu(z) \\
%%&=& \int_Y \big| \cD(z) \setminus \cR_n(z) \big|\, d\nu(z) \\
%%&& \quad\quad + \int_Y |D|\, \max_{\phi\in D} \big| 
%%\phi^{-1}(z) \setminus \cR_n(z) \big|\, d\nu(z). \\
%%&\leq& \int_Y \sum_{y \in \cR_n(z)} \big| \cR_n(y) \big|^{-1}\,\big| \cT(z) \setminus \cR_n(y) \big|\, d\nu(z) \\
%%&& \quad \quad  + \int_Y \big| \cR_n(y) \big|^{-1} \sum_{z \in \cR_n(y)} \big| D^{-1}(z) \setminus \cR_n(y) \big|\, d\nu(y)\\
%%&=& \int_Y \sum_{y \in \cR_n(z)} \big| \cR_n(z) \big|^{-1}\,\big| \cT(z) \setminus \cR_n(z) \big|\, d\nu(z) \\
%%&=& \int_Y \big| \cT(z) \setminus \cR_n(z) \big|\,d\nu(z).
%%\end{eqnarray*}
%Now since the union of the $\cR_n$ generates the full relation $\cR$, and the subset function $\cD$ is bounded, so that $\abs{D(z)}$ is uniformly bounded and $\abs{R_n(z)}\to \infty$, we conclude that the integrand tends
%to zero pointwise almost surely as $n \to \infty$. Since $\|\cD\| < \infty$, we can use the dominated 
%convergene theorem to deduce that the above integral converges to zero as well. This establishes
%the validity of the assertion (i).\\
(ii). Fix $\varepsilon> 0$. 
By (i), we can choose
$n(\varepsilon) \in \NN$ such that for $n\ge n(\varepsilon)$
\begin{eqnarray*}
\int_Y \frac{\big| \cT \circ \cR_n(y) \setminus \cR_n(y) \big|}{\big| \cR_n(y) \big|}\, d\nu(y) < \varepsilon^2.
\end{eqnarray*}
%Let $\tilde{Y} \subseteq Y$ be a conull set such that every $\phi \in D$ is one-to-one on all
%points $w \in  \cT(z)$, where $z \in \tilde{Y}$. 
It is a consequence of Markov's inequality that we can find  a set $Y_n \subset Y$ with 
$\nu(Y_n) \geq 1 - \varepsilon$ and such that for all $y \in Y_n$
\begin{eqnarray} \label{eqn:hyperfincontra}
\frac{ \big|\cT \circ \cR_n(y) \setminus \cR_n(y) \big|}{\big| \cR_n(y) \big|} < \varepsilon\,.
\end{eqnarray}


Now suppose $y\in Y_n$ and \eqref{eqn:hyperfinitefolner} fails to hold for some $\varepsilon > 0$.
Then there are at least $\varepsilon\,\|\cT\|\,|\cR_n(y)|$ many elements $z \in \cR_n(y)$ with
$\cT(z) \setminus \cR_n(y)$ containing at least one element $e_z$.  
$\cT$ is a sub equivalence relation consisting of disjoint cells, and each cell $\cT(z)$ contains at most $\norm{\cT}$ points. Thus there are at least $\varepsilon\,|\cR_n(y)|$ many distinct
such elements $e_z$. However, this contradicts the inequality~\eqref{eqn:hyperfincontra}.
\end{proof}
%We are now in position to prove the existence of strong
%F{\o}lner sequences. 


%\begin{Proposition}[Existence of strong F{\o}lner sequences] \label{prop:strongfolnerex}
%Let $\cR$ be amenable and assume that $\Phi \subseteq \operatorname{Aut}(\cR)$
%is a countable set which is generating for $\cR$.
% Suppose that $(\cF_n)$ is a sequence of bounded 
%subset functions satisfying $\cF_{n} \subseteq \cF_{n+1}$
%and 
%\[
%\lim_{n \to \infty} \frac{\big| \cF_n(y) \,\triangle\, \cF_n(z) \big|}{\big| \cF_n(y) \big|} = 0
%\]
%for $m$-almost every $(y,z) \in \cR$. Then, one can extract a subsequence
%from $(\cF_n)$ which is a strong F{\o}lner sequence.
%\end{Proposition}


%\begin{proof}
%We have shown in Proposition~\ref{prop:approxrelationsexist}~(ii) that we can extract
%a subsequence from $(\cF_n)$, say $(\cF_{n_k})$, 
%which is a pointwise approximate hyperfinite exhaustion. Denote the correspondingwhen
%approximating hyperfinite exhaustion by $(\cR_k)$. Now let $m \in \NN$ and take
%$\phi \in \Phi$. 
%By Lemma~\ref{lemma:hyperfinitefolner} applied to $D = \{\phi\}$ and
%$\cT = \cR_m$, we can pass 
%to another subsequence, say $(\cR_{k_l})$ in order to obtain
%\[
%\lim_{l \to \infty}\frac{\big| \cR_{k_l}(y) \, \triangle \phi\big(
%\cR_m \circ \cR_{k_l}(y) \big) \big|}{|\cR_{k_l}(y)|} = 0
%\]
%for $\nu$-almost every $y \in Y$. By a diagonal sequence argument, we obtain
%a subsequence $(\cR_l^{*}) \subseteq (\cR_{k})$ such that 
%the above limit relation holds along that subsequence simultaneously 
%for every $\phi \in \Phi$ and every $m \in \NN$. We conclude the 
%proof by taking the corresponding subsequence of the sequence $\cF_{n_k}$.
%\end{proof}

\section{Subadditive convergence and entropy} \label{sec:subadditive}

The notion of subadditivity and the convergence of subadditive functions along a suitable family of sets are two key concepts in the development of ergodic theory of amenable groups. In the present section we will extend the scope of this concept and develop a natural notion of subadditivity for p.m.p.\@ amenable equivalence relations, along bounded hyperfinite exhaustions. This will be the crucial tool in the proof of Theorem~\ref{thm:MAINcocycleentropy}, establishing the existence of cocycle entropy as a limit of normalized information functions. By the standard subadditivity properties
of Shannon entropy, we show in Proposition~\ref{prop:hpsubadditive}
that the information functions $h^{\cP}(\cR_n(y))$ are subadditive functions in the sense
proposed below. Applying the subadditive convergence lemma below immediately gives the validity of Theorem~\ref{thm:MAINcocycleentropy}.  


 


Let us turn to state the definition of subadditive functions.
Recall that a measurable subset function is a measurable map $\cF:Y\to Fin(Y)$, satisfying that $\cF(y)$ is a finite subset of $\cR(y)$ for almost every $y$.  $\cF$ is bounded if $\abs{\cF(y)}$ and $\abs{\cF^{-1}(y)}$ are essentially bounded. Let $\cB\cS\cF(\cR)$ denote the space of measurable bounded subset functions on the equivalence 
relation $\cR$.   
\begin{Definition}[Subadditive functions] \label{defi:subadditive}
A mapping $\mathfrak{H}: \cB\cS\cF(\cR) \to \operatorname{Map}\big(Y, [0,\infty) \big)$ is called {\em subadditive} if 
\begin{itemize}
\item for every bounded subset function $\cA$, the function $\mathfrak{H}(\cA): Y \to [0,\infty)$ is measurable; 
\item $\mathfrak{H}$ is {\em bounded} in the sense that there exists $C> 0$ such that for all subset functions
$\cA$, we have $\mathfrak{H}(\cA)(y) \leq C\,|\cA(y)|$ a.e.;
\item $\mathfrak{H}$ is $\cR$-{\em invariant}, i.e.\@ for every bounded sub-equivalence relation $\cT \subseteq \cR$, one has $h(\cT)(y)=h(\cT)(z)$,
whenever $y\cT z$. 
%{\bf ???????? there is a function $\tilde{h}_y : Fin([y])
% \to [0,\infty)$ such 
%that for all $\cA \in \cS\cF(\cR)$ and all $z \in [y]$, 
%we have $h(\cA)(y) = \tilde{h}_z\big(\cA(y)\big)$;  ?????????????}
%\item $\mathfrak{H}$ is {\em monotone} in the sense that if $\cA,\cB$ are two bounded subset functions satisfying  $\cA (y)\subseteq \cB(y)$ a.e.,
%then $\mathfrak{H}(\cA)(y) \leq \mathfrak{H}(\cB)(y)$ for $\nu$-almost every $y \in Y$;
\item $\mathfrak{H}$ has the following {\em subadditivity property}:
if $\cA,\cA_i$ are bounded subset functions that satisfy $\cA(y):= \coprod_{i=1}^m \cA_i(z_i)$ then for a.e. $y\in Y$
\[
\mathfrak{H}\big( \cA \big)(y) \leq \sum_{i=1}^m \mathfrak{H}\big( \cA_i\big)(z_i)\,,
\]
where $\set{z_1(y),\dots,z_{m(y)}(y)}\subset \cA(y)$ is a set of points depending measurably on $y$.   
\end{itemize}
\end{Definition}

%With a slight abuse of notation, we write $h(\cF(y)):= h(\cF)(y)$ for
%subadditive (in particular invariant) functions.
We now prove the subadditive convergence theorem.

\begin{Theorem} \label{thm:INT}
Let $\cR$ be p.m.p.\@ and hyperfinite and let $\mathfrak{H}:\cB\cS\cF(\cR) \to \operatorname{Map}\big(Y,[0,\infty)\big)$ be subadditive. Then, 
%there exists $\mathfrak{H}^{*} \geq 0$ such that 
for all bounded hyperfinite exhaustions $(\cR_n)$, the following holds: 
\begin{eqnarray*}
\mathfrak{H}^{*} := \inf_{\cT \subseteq \cR} \int_Y \frac{\mathfrak{H}\big( \cT\big)(y)}{|\cT(y)|}\,d\nu(y)  
= \lim_{n \to \infty} \int_Y \frac{\mathfrak{H}\big( \cR_n\big)(y)}{\big| \cR_n(y) \big|}\, d\nu(y),
\end{eqnarray*}
where the infimum is taken with respect to all non-trivial, bounded sub-equivalence relations $\cT \subseteq \cR$. 
\end{Theorem}


\begin{proof}
%We first prove the convergence for a hyperfinite exhaustion $(\cR_n)$, and note that this part of the proof is valid  for {\it any} hyperfinite exhaustion without the assumption of boundedness. This %assumption will play a role later on in the proof, when comparing the values that arise from two hyperfinite exhaustions. 

%To this end, fix $n \in \NN$, and then $ \cR_{n+1}(y)$ is a disjoint union of finite number of $\cR_n$-classes, say $m=m(y)$. We can choose $m$ points $z_1,\dots,z_m$, one in each of these disjoint %cells, such that the choice of representatives is measurable in $y$, since an equivalence relation with finite classes admits measurable sections. Thus $\cR_{n+1}(y)=\coprod_{i=1}^m \cR_n(z_i)$, and %using the $\cR_n$-invariance of the function $\mathfrak{H}(\cR_n(\cdot))$ and the subadditivity property of $\mathfrak{H}$ we deduce that 
%\begin{eqnarray*}
%\mathfrak{H}\big( \cR_{n+1}\big)(y)  \leq \sum_{z \in \cR_{n+1}(y)} \frac{\mathfrak{H}\big( \cR_n\big)(z)}{\big| \cR_n(z) \big|}.
%\end{eqnarray*}
%For brevity, set $\mathfrak{H}^{*}_n(y):= \mathfrak{H}\big( \cR_n\big)(y) /\big| \cR_n(y) \big|$.
%Using the $\cR$-invariance of the measure $\tilde{\nu}$, as well as $\cR_{n+1}(y) = \cR_{n+1}(z)$ for
%$z \in \cR_{n+1}(y)$, we compute
%\begin{eqnarray*}
%\int_Y \mathfrak{H}^{*}_{n+1}(y)\, d\nu(y) &\leq& \int_Y \sum_{z \in \cR_{n+1}(y)} \frac{\mathfrak{H}^{*}_n(z)}{|\cR_{n+1}(y)|} \, d\nu(y)\\
%\mbox{ (invariance of } \nu \mbox{) } &=& 	\int_Y   \sum_{y \in \cR_{n+1}(z)} \frac{\mathfrak{H}^{*}_n(z)}{|\cR_{n+1}(y)|} \, d\nu(z)\\ 
%\big( \cR_{n+1}(y) = \cR_{n+1}(z) \big) &=& \int_{Y} |\cR_{n+1}(z)|^{-1}\,\sum_{y \in \cR_{n+1}(z)} \mathfrak{H}^{*}_n(z)\,d\nu(z) \\
%&=& \int_Y \mathfrak{H}_n^{*}(z) \, d\nu(z). 
%\end{eqnarray*}
%This shows monotonicity of the integral expression and the existence of the limit 
%follows immediately. % (Here, we allow $-\infty$ to be a limit point as well.) 

Let $(\cR_n)$ be a non-trivial, bounded hyperfinite exhaustion and denote by $\cT$ be an arbitrary
sub-equivalence relation of $\cR$. It is enough to show that 
\begin{eqnarray} \label{eqn:subaddtoshow}
\limsup_{n \to \infty} \int_Y \frac{\mathfrak{H}\big(\cR_n\big)(y)}{|\cR_n(y)|}\,d\nu(y)
\leq \int_Y \frac{\mathfrak{H}\big(\cT \big)(y)}{|\cT(y)|}\,d\nu(y).
\end{eqnarray}
To this end, 
%us now show that the limit is independent of the choice of the bounded hyperfinite exhaustion. 
%To this end, 
%let $(\cR_n)$ and $(\cR_m^{\prime})$ be two bounded hyperfinite exhaustions so 
%that 
%\[
%%\cR = \bigcup_{n=1}^{\infty} \cR_n = \bigcup_{m=1}^{\infty} \cR_m^{\prime}.
%\]
fix $\varepsilon > 0$, as well as $m \in \NN$. We apply Lemma~\ref{lemma:hyperfinitefolner} (ii)
to $\cT$. Hence, for large enough $n \in \NN$, there is $Y_n \subseteq Y$ with 
$\nu(Y_n) > 1 - \varepsilon$ and for all $y \in Y_n$, we have $\big| \mathring{\cR}_n(y) \big|
\geq (1-\varepsilon\,\|\cT\|)\,\big|\cR_n(y)\big|$, where $\mathring{\cR}_n(y):=
\{ z \in \cR_n(y)\,|\, \cT(z) \subseteq \cR_n(y) \}$. 

$\mathring{\cR}_n(y)$ is of course a bounded subset function (possibly assuming the empty set as a value), and consider now the disjoint decomposition 
$\cR_n(y)= 
\mathring{\cR}_n(y)\coprod \big(\cR_n(y)\setminus  \mathring{\cR}_n(y)\big)$. % for $z\in  \mathring{\cR}_n(y)$.
Applying the subadditivity of $\mathfrak{H}$ to this decomposition, 
   this yields for a.e. $y\in Y$ %$y \in Y_n$ %and $z\in \mathring{\cR}_n(y)$
\begin{eqnarray} \label{eqn:esspart} 
 \mathfrak{H}(\cR_n)(y)\le \mathfrak{H}(\mathring{\cR}_n)(y)+\mathfrak{H}(\cR_n\setminus \mathring{\cR}_n)(y)\,.
\end{eqnarray}
Now note that if $z\in \mathring{\cR}_n(y)$ and $w\cT z$, then $w \in \mathring{\cR}_n(y)$ as well, so that 
$\mathring{\cR}_n(y)$ is a union of disjoint $\cT$-classes.  Hence, we can choose points $z_1, \dots, z_m$ in each of 
the resulting cells such that the choice of representatives is measurable in $y$ (since an equivalence relation with 
finite classes admits measurable sections). Hence, $\mathring{\cR}_n(y) = \coprod_{j=1}^m \cT(z_j)$, and using the 
subadditivity property of $\mathfrak{H}$, along with the invariance property with respect to $\cT$, we get 
\begin{eqnarray*}
\mathfrak{H}(\mathring{\cR}_n)(y) \leq \sum_{z \in \mathring{\cR}_n(y)} \frac{\mathfrak{H}\big(\cT \big)(z)}{|\cT(z)|.}
\end{eqnarray*} 
Dividing inequality~\eqref{eqn:esspart} by $\abs{\cR_n(y)}$, 
and using the  boundedness of $\mathfrak{H}$ applied to the second summand, we conclude that for a.e. $y\in Y_n$ :
\[
\mathfrak{H}^{*}_{n}(y) \leq \sum_{z \in \mathring{\cR}_{n}(y)} \frac{{\mathfrak{H}^{*}_{\cT}}(z)}{|\cR_{n}(y)|}
+ C\,\varepsilon\,\|\cT\|\,,
\]
 where, for brevity, we have set 
$\mathfrak{H}^{*}_{\cT}(y) := \mathfrak{H}\big( \cT\big)(y) /\big| \cT(y) \big|$.
%By the monotonicity assumption on $\mathfrak{H}$, one further gets
%\[
%\mathfrak{H}^{*}_{n}(y) \leq \sum_{z \in {\cR}_{n}(y)} \frac{{\mathfrak{H}^{*\prime}_m}(z)}{|\cR_{n}(y)|}
%+ C\,\varepsilon\,\|\cR_m^{\prime}\|
%\]
By the boundedness of $\mathfrak{H}$, clearly  $\mathfrak{H}^{*}_{n}(y) \leq C$ for all $y \in Y$ and since $\mathfrak{H}^{*}_{\cT} \geq 0$, 
and as $\nu(Y_n)< \varepsilon$, we integrate over
 $Y=Y_n \coprod \big(Y\setminus Y_n\big)$ in order to arrive at
\begin{eqnarray*}
 \int_Y \mathfrak{H}^{*}_{n}(y)\, d\nu(y) \leq 
\int_Y \sum_{z \in \cR_{n}(y)} \frac{{\mathfrak{H}^{*}_{\cT}}(z)}{|\cR_{n}(y)|} \, d\nu(y) 
+  C\, (1 + \|\cT\|)\, \varepsilon. 
\end{eqnarray*}
Now using the $\cR$-invariance of the measure $\tilde{\nu}$, along with the fact that
$\cR_n(y) = \cR_n(z)$ for $z\cR_n y$, we obtain
\begin{eqnarray*}
 \int_Y  \sum_{z \in \cR_{n}(y)} \frac{{\mathfrak{H}^{*}_{\cT}}(z)}{|\cR_{n}(y)|} \, d\nu(y) &=& 
 \int_Y \sum_{y \in \cR_n(z)}   \frac{{\mathfrak{H}^{*}_{\cT}}(z)}{|\cR_{n}(z)|} \, d\nu(z) =
 \int_Y {\mathfrak{H}^{*}_{\cT}}(z)\,d\nu(z).
\end{eqnarray*}
%as well as the previous convergence result along bounded hyperfinite exhaustions,  
Consequently, one gets
\[
\limsup_{n \to \infty} \int_Y \mathfrak{H}^{*}_{n}(y)\, d\nu(y) \leq \int_Y {\mathfrak{H}^{*}_{\cT}}(z)\,d\nu(z) + 
C\,(1+ \|\cT\|)\,\varepsilon. 
\]
Letting $\varepsilon \to 0$, we see that the inequality~\ref{eqn:subaddtoshow} is verified.
%Finally, it remains to show convergence for arbitrary bounded approximate hyperfinite exhaustions 
%$(\cF_n)$. To this end, we find a hyperfinite approximation $(\cR_n)$ with $\cR_n \subseteq \cF_n$ {\bf WHY IS THIS A LEGITIMATE ASSUMPTION ????????}
%and such that 
%\begin{eqnarray} \label{eqn:null!}
%\lim_{n \to \infty} \int_Y \frac{\big| \cF_n(y) \setminus \cR_n(y) \big|}{\big| \cR_n(y) \big|}\,d\nu(y) = 0.
%\end{eqnarray}
%Now a short calculation using boundedness, monotonicity, and subadditivity of $\mathfrak{H}$ shows that
%\begin{eqnarray*}
%\frac{\mathfrak{H}\big( \cF_n(y) \big)}{\big| \cF_n(y) \big|} \leq \frac{\mathfrak{H}\big( \cR_n(y) \big)}{\big| \cF_n(y) \big|} 
%+ \, \frac{\mathfrak{H}( \cF_n(y) \setminus \cR_n(y))}{\big| \cF_n(y) \big|} 
%\leq
%\frac{\mathfrak{H}\big( \cR_n(y) \big)}{\big| \cR_n(y) \big|} 
%+C \, \frac{\big| \cF_n(y) \setminus \cR_n(y) \big|}{\big| \cR_n(y) \big|}
%\end{eqnarray*}
%{\bf WHY is $\abs{\cR_n(Y)}\le \abs{\cF_n(y)}$ ?????????? this should be argued for an integral, not pointwise. Work with $\cF_n(y) \Delta \cR_n(y)$, and so under integral convergence to zero the sizes are within $(1+\epsilon)$ on a set of measure $(1-\epsilon)$ and the complement there is a global bound on the size   }
%
%for every $n \in \NN$ and $\nu$-almost every $y \in Y$. We integrate the above 
%inequalities over $Y$ and let $n \to \infty$. Due to the limit relation~\eqref{eqn:null!},
%we obtain the desired convergence. 
\end{proof}


Before proceeding with the proof of 
Theorem~\ref{thm:MAINcocycleentropy}, let us recall the following standard properties of the Shannon entropy of countable partitions.
% are well known
%and can be verified by standard calculations. 

\begin{Proposition}\label{prop:propertiesofH}
Let $(X,\lambda)$ be a p.m.p.\@ action of $\Gamma$. 
Fix $F\in Fin(\Gamma)$ and $g \in \Gamma$. Further, let 
two countable partitions $\cP$ and $\cQ$ of $X$ be given. Then,
\begin{enumerate}[(i)]
\item $H(\cP \vee \cQ) \leq H(\cP) + H(\cQ)$;
\item $H\big( \bigvee_{g \in F}g^{-1}\cP \big) \leq H(\cP)\,|F|$;
\item $H(\cP) \leq H(\cQ)$ if $\cQ \geq \cP$.
\item $H(g\,\cP)= H(\cP)$;

\end{enumerate}
\end{Proposition}

\begin{proof}
See e.g. \cite[Ch. 10, \S 6]{CFS}. 
\end{proof}

The following proposition shows that for a countable 
partition $\cP$ with $H(\cP)<\infty$, choosing for $\mathfrak{H}$ 
the Shannon entropy map $h^{\cP}: \mathcal{B}\cS\cF(\cR) \to \operatorname{Map}\big(Y, [0,\infty)\big)$
as defined in Section~\ref{sec:MAIN}
gives rise to a subadditive function. 

\begin{Proposition}\label{prop:hpsubadditive}
Fix a countable partition $\cP$ of $(X,\lambda)$ with finite Shannon entropy $H(\cP)$.
Then the mapping $h^{\cP}: \cB\cS\cF(\cR) \to \operatorname{Map}\big(Y, [0,\infty)\big)$ 
is subadditive according to
Definition~\ref{defi:subadditive}.
\end{Proposition}

\begin{proof}
Recall that we have defined 
\[
h^{\cP}(\cF): Y \to [0,\infty) \text{\,\,\,via :\,\,\,} h^{\cP}(\cF)(y) := 
H\Big( \bigvee_{w \in \cF(y)} \alpha(w,y)^{-1}\,\cP \Big).
\]

The boundedness and the subadditivity property for $h^{\cP}$ 
are easily verified by properties (i) and (ii) listed in Proposition~\ref{prop:propertiesofH}.
Concerning the invariance property, fix an arbitrary bounded sub-equivalence relation $\cT \subseteq \cR$.
Let $y,z \in Y$ be such that $y \cT z$. We need to show that 
$h^{\cP}(\cT)(z) = h^{\cP}(\cT)(y)$. To this end, note that by the cocycle identity, 
the set of group elements $\big\{ \alpha(w,z) \,|\, w \in \cT(z) \big\}$ is equal 
to the set $\big\{  \alpha(w,y) \,|\, w \in \cT(y) \big\} \cdot \alpha(y,z)$. Hence, the desired
invariance follows from part~(iv) of Proposition~\ref{prop:propertiesofH}.
It remains to show the measurability statement. To this end, let $\cA$ be any  
measurable bounded subset function. 
Note first that given a 
partition $\cP$, the function $h^{\cP}(\cA)(\cdot)$ takes at most countably many different values, 
since $Fin(\Gamma)$ is countable. More precisely, those values only depend on the finite 
collection
of group elements of the form $\{ \alpha(w,y)\, |\, w \in \cA(y)\} \in Fin(\Gamma)$ 
for $y \in Y$. Since $\cA$ and the cocycle $\alpha$ are measurable by assumption, the function
$\overline{h}(\cA): \cR \to [0, \infty)$, defined by 
$\overline{h}(\cA)\big( (y,z) \big):= H\Big( \bigvee_{w \in \cA(y)} \alpha(w,z)^{-1}\cP \Big)$
is measurable. Using the cocycle identity $\alpha(z,w)\alpha(w,y)=\alpha(z,y)$ we observe that 
$\overline{h}(\cA)\left((y,z)\right) = \overline{h}(\cA)\big( (y,y) \big)$ for all $z \in [y]$, by property (iv) in Proposition~\ref{prop:hpsubadditive}. 
Consequently, we can write $h^{\cP}(\cA) := \overline{h}(\cA) \circ \ell_{\Delta}$, 
where $\ell_{\Delta}: Y\to \cR$ is the lift $y\mapsto(y,y)$ of $Y$ to the diagonal. Since the latter function  is measurable, so is $h(\cA)$.
\end{proof}



%\begin{Proposition}
%The function 
%$$y\mapsto h_n^\cP(y)= \frac{H\left(\bigvee_{y^\prime\in \cR_n(y)}\alpha(y^\prime, y)^{-1}\cP\right)}{\abs{\cR_n(y)}}$$ is an $\cR_n$-invariant function on $Y$. 
%\end{Proposition} 
%\begin{proof}
%Assume that $y''\cR_n y$. Then, for every $y^\prime\in \cR_n(y)$, by the cocycle identity 
%$\alpha(y,y'')\alpha(y'',y^\prime)=\alpha(y, y^\prime)$. Furthermore, clearly $\cR_n(y)=\cR_n(y'')$. 
%Thus the set of group elements $\alpha(y'',y^\prime)$ as $y^\prime$ ranges over $\cR_n(y'')$ is equal to the set of group elements $\alpha(y,y^\prime)$  as $y^\prime$ ranges over $\cR_n(y)$, shifted %by multiplication by the fixed group element 
%$\alpha(y,y'')^{-1}$. Since a fixed shift does not change the value of the Shannon entropy of a partition, we conclude that $h_n^\cP(y)=h_n^\cP(y'')$ so that $h_n^\cP$ is an $\cR_n$-invariant %function. 
%\end{proof} 

%The sequence of functions $h_n^\cP$, $n \in \NN$ has the following subadditivity property. Each equivalence class $\cR_{n+1}(y)$ can be written as a  disjoint union of $k_n(y)$  $\cR_n$-equivalence %classes, a fact that we denote by 
%$$\cR_{n+1}(y)=\coprod_{i=1}^{k_n(y)} \cR_n(w_i(y))$$ 
%so that $w_i$, $1\le i \le k_n(y)$ is a choice of representatives from each of the $\cR_n$-equivalence classes which are contained in the $\cR_{n+1}$-equivalence class $\cR_{n+1}(y)$. The fact that it %is possible to choose the representatives $w_i(y)$ as measurable functions of $y$ follows from the fact that any equivalence relation with finite classes is type $I$, or smooth, namely it admits %measurable sections. 

%\begin{Proposition} \label{prop:subadd}
%For almost every $y \in Y$ and each $n\in \NN$, we have
% $$h_{n+1}^\cP(y)\le \sum_{i=1}^{k_n(y)}\frac{\abs{\cR_n(w_i(y))}}{\abs{\cR_{n+1}(y)}} h_n^\cP(w_i(y)).$$
%\end{Proposition}
%\begin{proof}
%The Shannon entropy of two partitions of $X$ satisfies $H(\cQ\vee \cQ^\prime)\le H(\cQ)+H(\cQ^\prime)$. Since $\cR_{n+1}(y)$ is a disjoint union of $k_n(y)$ sets, The partition $\cP^{\cR_{n+1}(y)}$ is %given by  
%$\bigvee_{i=1}^{k_n(y)}\cP^{\cR_n(w_i(y))}$, and hence the inequality follows from the definition of the functions $h_n^\cP(\cdot)$. 
%\end{proof} 


%For the proof of our integrated version of the Ornstein-Weiss lemma, 
%we need the following technical proposition.

%\begin{Proposition} \label{prop:hyperfinitefolner}
%Let $(\cR_n)$ be hyperfinite and assume that $\cT$ is a bounded, finite $\cR$-equivalence relation.
%Then, for every $\varepsilon > 0$, and for all large enough $n \in \NN$ (depending on $\varepsilon$) 
%there is a set $Y_n \subseteq Y$ with $\nu(Y_n) \geq 1- {\varepsilon}$ such that for all
%$y \in Y_n$, we get
%\begin{eqnarray} \label{eqn:hyperfinitefolner}
%\frac{\big|\big\{ z \in \cR_n(y)\,|\, \cT(z) \subseteq \cR_n(y)  \big\}\big|}{\big| \cR_n(y) \big|}
%> 1- \varepsilon\,\|\cT\|,
%\end{eqnarray}
%where $\|\cT\| := \operatorname{esssup}_y \big| \cT(y) \big|$.
%\end{Proposition} 

%\begin{proof}
%Using the invariance of the measure $\nu \circ c$ and $\cR_n(y) = \cR_n(z)$ for $z \in \cR_n(y)$, 
%we compute
%%\begin{eqnarray*}
%\int_Y \frac{\big| \cT\cR_n(y) \setminus \cR_n(y) \big|}{\big| \cR_n(y) \big|}\, d\nu(y) &\leq& 
%\int_Y \big| \cR_n(y) \big|^{-1} \sum_{z \in \cR_n(y)} \big| \cT(z) \setminus \cR_n(y) \big|\, d\nu(y) \\
%&\leq& \int_Y \sum_{y \in \cR_n(z)} \big| \cR_n(y) \big|^{-1}\,\big| \cT(z) \setminus \cR_n(y) \big|\, d\nu(z) \\
%&=& \int_Y \sum_{y \in \cR_n(z)} \big| \cR_n(z) \big|^{-1}\,\big| \cT(z) \setminus \cR_n(z) \big|\, d\nu(z) \\
%&=& \int_Y \big| \cT(z) \setminus \cR_n(z) \big|\,d\nu(z).
%\end{eqnarray*}
%Now since the union of the $\cR_n$ generates the full relation $\cR$, we obtain that the integrand tends
%to zero pointwise almost surely as $n \to \infty$. Since $\|\cT\| < \infty$, we can use the dominated 
%convergene theorem to deduce that the above integral converges to zero as well. Hence, we can choose
%$n \in \NN$ such that 
%\begin{eqnarray*}
%\int_Y \frac{\big| \cT\cR_n(y) \setminus \cR_n(y) \big|}{\big| \cR_n(y) \big|}\, d\nu(y) < \varepsilon^2.
%\end{eqnarray*}
%It is a consequence of Markov's inequality that we can find  a set $Y_n \subseteq Y$ with 
%$\nu(Y_n) \geq 1 - \varepsilon$ and such that 
%\begin{eqnarray} \label{eqn:hyperfincontra}
%\frac{ \big|\cT\cR_n(y) \setminus \cR_n(y) \big|}{\big| \cR_n(y) \big|} < \varepsilon
%\end{eqnarray}
%%for all $y \in Y_n$. Now assume that \eqref{eqn:hyperfinitefolner} fails to hold.
%Then there are at least $\varepsilon\,\|\cT\|\,|\cR_n(y)|$ many elements $z \in \cR_n(y)$ with
%$\cT(z) \setminus \cR_n(y)$ containing at least one element $e_z$. Since 
%$\cT$ is an equivalence relation, there are at least $\varepsilon\,|\cR_n(y)|$ many distinct
%such elements $e_z$. But this contradicts the inequality~\eqref{eqn:hyperfincontra}.


%\end{proof}


%\begin{Theorem} \label{thm:INT}
%The limit
%\[
%h(X, \cP, \alpha) := \lim_{n \to \infty} \int_Y h^{\cP}_n(y)\,d\nu(y)
%\]
%%exists. \\
%Moreover, $h(X, \cP, \alpha)$ does not depend on the approximation sequence $(\cR_n)$. 
%\end{Theorem}

We can now complete the proof of Theorem~\ref{thm:MAINcocycleentropy}. 

\begin{proof}[Proof of Theorem~\ref{thm:MAINcocycleentropy}]
By Proposition~\ref{prop:hpsubadditive}, the mapping $h^{\cP}: \cB\cS\cF(\cR) \to 
\operatorname{Map}\big(Y, [0,\infty)\big)$ is 
subadditive. 
%Therefore, we only need to verify the invariance assumption appearing in Theorem~\ref{thm:INT}, namely that if $y\cR_n z$ (namely $\cR_n(y)=\cR_n(z)$) then 
%$h^{\cP}(\cR_n)(y)
% = h^{\cP}(\cR_n)(z)$. By definition 
%$h^{\cP}(\cR_n)(y)=H\Big( \bigvee_{w \in \cR_n(y)} \alpha(w,y)^{-1}\,\cP \Big)$, and the set of group elements 
%$\set{\alpha(w,y)\,;\, w\in \cR_n(y)}$ equals the set of group elements $\set{\alpha(w,z)\alpha(z,y)\,;\, w\in \cR_n(y)}$, by the cocycle identity.  The latter set is equal to 
%$\set{\alpha(w,z)\,;\, w\in \cR_n(w)}\cdot \alpha(z,y)$, so that  the computation of the Shannon entropy involves  the two partitions $\bigvee_{w \in \cR_n(y)} \alpha(w,y)^{-1}\,\cP$ 
%and $\alpha(z,y)^{-1} \bigvee_{w \in \cR_n(y)} \alpha(w,z)^{-1}\,\cP$. By property (iv) in 
%Proposition~\ref{prop:hpsubadditive}, the stated invariance property holds. 
Hence, the convergence claimed in the theorem, as well as the fact that the limit does not depend on the hyperfinite exhaustion, both follow from Theorem~\ref{thm:INT}, and this concludes the proof of Theorem~\ref{thm:MAINcocycleentropy}. 
\end{proof}


As noted in the introduction, Theorem~\ref{thm:INT} establishes a general subadditive
convergence principle, valid for all functions in the large class of
subadditive set functions over equivalence relations, as described in Definition~\ref{defi:subadditive}. 
%Convergence can then be established in the mean, i.e.\@ along the approximants
%integrated over all classes. 
%For amenable groups and graphs, subadditive convergence results have been obtained
%before, see e.g.\@ \cite{Gr99, LW00,CCK14,Po14}. However the  
%result for measured equivalence relations that we establish below seems to be the first such subadditive
%convergence theorems along finite subsets in non-amenable groups. 
%Theorem~\ref{thm:INT} is not confined to
%measure entropy but holds for a wider class of subadditive functions,  
In particular, it can be applied to prove an analog of Theorem~\ref{thm:MAINcocycleentropy} for a natural notion 
of topological entropy in the present context. 







\section{Pointwise covering lemmas} \label{sec:tiling}

We now establish pointwise decomposition results for subset functions which will provide
the central tool for proving the Shannon-McMillan-Breiman theorem in the 
next section. Our main lemma builds on the techniques developed by Lindenstrauss
for the proof of the corresponding covering lemma for tempered F{\o}lner 
sequences in amenable groups (see \cite[Lem. 2.4]{Li01}). However, working with hyperfinite 
exhaustions, we are able to avoid two difficult  technicalities: 
\begin{itemize}
\item we do not need to use an auxiliary random parameter in order to be able to choose the desired coverings with high probability;
\item we are able to produce strictly disjoint coverings, and the discussion 
of $\delta$-disjointness (see \cite[Lem. 2.6, 2.7]{Li01}) becomes unnecessary.
\end{itemize}
Instead, we exploit the disjointness properties inherent in a sequence of nested equivalence relations.

As usual, $\cR$ will denote a p.m.p amenable equivalence relation
over $(Y,\nu)$ with $\tilde{\nu}$ denoting the invariant measure on $\cR$, and 
$(\cR_n)$ will denote a bounded hyperfinite exhaustion for $\cR$.

We start with the following elementary covering (and disjointification) lemma. 
 

\begin{Proposition} \label{prop:basiccov}
Let $N,L \in \NN_{\ge 1}$ with $N < L$ and consider an arbitrary finite sequence of subset functions 
$\cB_j \subseteq \cR_L$, $1 \leq j \leq N$. Further, for $y\in Y$, consider a collection of classes (of the relations $\cR_n$ where $1\le n\le N$) given by 
\[
\mathfrak{F}(y) := \big\{ \cR_{n(j)}(w)\,|\, w \in \cB_j(y), 1 \leq j \leq N \big\}.
\]

Then, for a.e. $y \in Y$, 
we can extract from $\mathfrak{F}(y)$ a disjoint subcollection $\mathfrak{S}(y)$ of classes 
such that 
\begin{eqnarray*}
\coprod_{C \in \mathfrak{S}(y)} C \supset \bigcup_{j=1}^N \cB_j(y)\,, \text{ and so }
\sum_{C \in \mathfrak{S}(y)} \abs{C} \geq \Big| \bigcup_{j=1}^N \cB_j(y) \Big|. 
\end{eqnarray*}
\end{Proposition}

\begin{proof}
Fix $y \in Y$.
We note first that for $z_1, z_2 \in \cR_L(y)$ and $1 \leq j_1, j_2 \leq N$,
there are three possibilities for the inclusion relation between $\cR_{j_1}(z_1)$ 
and $\cR_{j_2}(z_2)$:
\begin{itemize}
\item  $\cR_{j_1}(z_1) \cap \cR_{j_2}(z_2) = \emptyset$,
\item  $\cR_{j_1}(z_1) \subseteq \cR_{j_2}(z_2)$, \quad or
\item  $\cR_{j_1}(z_1) \supseteq \cR_{j_2}(z_2)$.
\end{itemize}
Any collection $\mathfrak{F}(y)$ of sets with the latter property has the property that the union of its constituents has a unique representation as a disjoint union of some of the constituents. 
To find this representation explicitly,  namely to choose the  subcollection
$\mathfrak{S}(y)$, enumerate all the elements in $\mathfrak{F}(y)$ and give them
distinct labels collected in an index set $\cI$. Then run the following checking algorithm.
\begin{enumerate}[(1)]
\item Set $\mathfrak{S}^{*}(y) = \emptyset$, $\cI^{*} = \cI$.
\item Check an arbitrary class $C \in \mathfrak{F}(y)$ 
with its corresponding label being contained in $\cI^{*}$. 
\item Given $C$, there are two possibilities:
\begin{enumerate}[(A)]
\item Either for all $1 \leq j \leq N$ and $z \in \cB_j(y)$ such that 
$\cR_{n(j)}(z)\cap C \neq \emptyset$, we have $C \supseteq \cR_{n(j)}(z)$,  
\item or there is some $1 \leq j_1 \leq N$, $z_1 \in \cB_{j_1}(y)$ such that
$C \subseteq \cR_{n(j_1)}(z_1)$ and $C \neq \cR_{n(j_1)}(z_1)$.
\end{enumerate}
\item Only in case of (A),  add $C$ to the subcollection $\mathfrak{S}^{*}(y)$. 
Then remove from $\cI^{*}$ all labels corresponding to classes
$\cR_{n(j)}(z) \in \mathfrak{F}(y)$ being contained in $C$. If the new set $\cI^{*} = \emptyset$,
 jump to step (5), otherwise  return to step (2).  
\item We have $\cI^{*} = \emptyset$ (meaning all classes have been checked) and we set 
$\mathfrak{S}(y) = \mathfrak{S}^{*}(y)$. This is the collection we aim for. 
\end{enumerate}
By construction, for a.e. $Y$, the elements $C \in \mathfrak{S}(y)$ are pairwise disjoint. Also, we made
sure that for every $1 \leq j \leq N$, every single $b \in \cB_j(y)$ is 
contained in some class $C$ taken into $\mathfrak{S}(y)$. This proves
the above inequality. 
\end{proof}

We now prove the main covering lemma, motivated by 
\cite[Lemma 2.1]{Li01}. 

\begin{Lemma}%[Abstract covering lemma]
 \label{lemma:abstrcomb}
Fix $0 < \delta < 1$ and fix an (arbitrary) finite non-empty set $D\subset \Phi$. 
% \subseteq \operatorname{Aut}(\cR)$.
Then, for sufficiently large $M \in \NN$, depending only on $D$ and $\delta$, the  following property holds. 

Let  ${\cT}_{i,j} \subseteq \cR_L$ 
$(1 \leq i \leq M, 1 \leq j \leq N_i)$ be  
an array of subset functions, such that for a.e. $y$, $\cT_{i,j}(y)=\cR_{n(i,j)}(y)$, where $n(i,j)\le L$. Assume that for $2\le i \le M$ and every $1\le j\le N_i$
\begin{eqnarray} \label{eqn:tempinfiber}
\Big| \bigcup_{k < i} D \circ \big({\cT}^{-1}_{k,*} \,{\cT}_{i,j} \big) \Big| \leq (1+\delta)\,\big|{\cT}_{i,j}\big|
\end{eqnarray}
almost surely, where ${\cT}_{k,*} := \bigcup_{j=1}^{N_k} {\cT}_{k,j}$ for $1 \leq k \leq M$. %$1 \leq i \leq M$. 

Then, given another array $\cB_{i,j} \subseteq \cR_L$
for $1 \leq i \leq M$ and $1 \leq j \leq N_i$, for $\nu$-almost every $y\in Y$ 
there are disjoint subcollections
\begin{eqnarray*}
\mathfrak{S}(y) \subseteq \big\{ {\cT}_{i,j}(w)\,|\, w \in \cB_{i,j}(y), 1 \leq i \leq M, 1 \leq j \leq N_i \big\}
\end{eqnarray*}
such that 
\begin{eqnarray*}
\sum_{C \in \mathfrak{S}(y)} \abs{C} \geq (1-\delta)\, \min_{1 \leq i \leq M} \Big|D \circ \bigcup_{j=1}^{N_i} \cB_{i,j}(y) \Big|.
\end{eqnarray*}
\end{Lemma}


\begin{proof}
Fix a conull set $Y_0 \subseteq Y$ such that for all $y \in Y_0$ the inequality~\eqref{eqn:tempinfiber}
is fulfilled.  For $i=M$, apply Proposition~\ref{prop:basiccov} to the subset functions defined as  $\cB_j :=\cB_{M,j}$, where $1 \leq j \leq N_M$.
This way, by passing to another conull subset of $Y_0$, for $\nu$-a.e.\@ $y \in Y_0$, we obtain a disjoint subcollection 
\[
\mathfrak{S}_M(y) \subseteq \big\{ {\cT}_{M,j}(w)\,|\, 1 \leq j \leq N_M, w \in \cB_{M,j}(y) \big\}.
\]
with 
\[
\sum_{C \in \mathfrak{S}_M(y)} \abs{C} \geq \Big| \bigcup_{j=1}^{N_M} \cB_{M,j}(y) \Big|.
\]
Proceeding iteratively, for $i < M$, we set
\[
\tilde{\cB}_{i,j}(y) := \cB_{i,j}(y) \setminus \bigcup_{l > i} {\cT}^{-1}_{i,j} \circ (\bigcup \mathfrak{S}_l )(y),
\]
where $1 \leq j \leq N_i$ and $\bigcup \mathfrak{S}_l$ denotes  the union over all sets in $\mathfrak{S}_l$.
Applying Proposition~\ref{prop:basiccov} again, this time to the subset functions defined as  $\cB_j := \tilde{\cB}_{i,j}$, $1 \leq j \leq N_i$
gives a subcollection
\[
\mathfrak{S}_i (y)\subseteq \big\{ {\cT}_{i,j}(w), 1 \leq j \leq N_i, w \in \tilde{\cB}_{i,j}(y) \big\}
\] 
with the corresponding covering property, namely 
\[
\sum_{C \in \mathfrak{S}_i(y)} \abs{C} \geq \Big| \bigcup_{j=1}^{N_i} \tilde{\cB}_{i,j}(y) \Big|.
\]

Having constructed $\mathfrak{S}_i$ for all $1 \leq i \leq M$,
we finally set $\mathfrak{S} := \bigcup_{i=1}^M \mathfrak{S}_i$. By construction, $\mathfrak{S}$ is a disjoint collection of
sets. We are going to show
\begin{eqnarray} \label{eqn:whattoshow}
\sum_{i=m}^M \sum_{C \in \mathfrak{S}_i(y)} \abs{C} \geq \min\big\{1-\delta, (M-m+1) \,\frac{\delta^2}{|D|} \big\}
\cdot \min_{m \leq i \leq M} \Big| D \circ \bigcup_{j=1}^{N_i} \cB_{i,j}(y) \Big|
\end{eqnarray}
for all $1 \leq m \leq M$ and $\nu$-almost every $y\in Y$. 
The proof is by induction on $m=M,\dots, 1$, starting with $M$. 
By the first case in our argument 
above,
\[
\sum_{C \in \mathfrak{S}_M(y)} \abs{C}\geq \Big| \bigcup_{j=1}^{N_M} \cB_{M,j}(y) \Big| 
\geq \frac{1}{|D|} \, 
\Big| D\circ\bigcup_{j=1}^{N_M} \cB_{M,j}(y) \Big|,
\]
which shows the validity of an inequality which is in fact stronger than \ref{eqn:whattoshow}. 
In order to show the claim for $m < M$, we assume that for the element $y \in Y$ under consideration,
\begin{eqnarray} \label{eqn:nothingtoprove}
\sum_{l > m} \sum_{C \in \mathfrak{S}_l(y)} \abs{C} < (1-\delta)\, \min_{m \leq i \leq M} 
\Big| D \circ \bigcup_{j=1}^{N_i} \cB_{i,j}(y) \Big|,
\end{eqnarray}
since, otherwise, there is nothing to prove. By construction of the collection $\mathfrak{S}_m(y)$,
we have
\begin{eqnarray} \label{eqn:covaux11}
\sum_{C \in \mathfrak{S}_m(y)} \abs{C}\geq \Big| \bigcup_{j=1}^{N_m} \tilde{\cB}_{m,j}(y) \Big|.
\end{eqnarray}
It follows from 
\[
\tilde{\cB}_{m,j}(y) = {\cB}_{m,j}(y) \setminus  \bigcup_{l > m} \bigcup_{C \in \mathfrak{S}_l(y)} {\cT}_{m,j}^{-1}\,C
\]
that 
\begin{eqnarray} \label{eqn:covaux22}
\Big(D \circ \bigcup_{j=1}^{N_m} \tilde{\cB}_{m,j}\Big)(y) \supseteq \Big(D\circ \bigcup_{j=1}^{N_m} \cB_{m,j}\Big)(y) 
\setminus \bigcup_{l > m} \bigcup_{C \in\mathfrak{S}_l(y)}
D \circ \Big( \bigcup_{j=1}^{N_m} {\cT}^{-1}_{m,j}\, C \Big)\,.
\end{eqnarray}
Recall that by assumption (\ref{eqn:tempinfiber}), we have
\[
\Big| D \circ \Big( \bigcup_{j=1}^{N_m} \cT_{m,j}^{-1}\,C  \Big) \Big| \leq (1+\delta)\, \big| C \big|
\]
for a.e.\@ $y \in Y_0$, where $C \in \bigcup_{l > m} \mathfrak{S}_l(y)$. Hence, by \eqref{eqn:covaux22},
\begin{eqnarray*}
\Big| \bigcup_{j=1}^{N_m} \tilde{\cB}_{m,j}(y) \Big| &\geq& \frac{1}{|D|}\, \Big| \Big(D \circ \bigcup_{j=1}^{N_m} \tilde{\cB}_{m,j}\Big)(y) \Big| \\
&\geq& \frac{1}{|D|} \, \Big( \Big| D \circ \bigcup_{j=1}^{N_m} \cB_{m,j}(y) \Big| 
-(1+\delta) \sum_{l > m} \sum_{C \in\mathfrak{S}_l} \abs{C} \Big). 
\end{eqnarray*}
With \eqref{eqn:covaux11} we obtain by taking into account \eqref{eqn:nothingtoprove}
\begin{eqnarray*}
\sum_{C \in \mathfrak{S}_m} \abs{C} &\geq&  \frac{1}{|D|} \, \Big| D \circ \bigcup_{j=1}^{N_m} \cB_{m,j}(y) \Big| - 
\frac{1+\delta}{|D|}\, \sum_{l > m} \sum_{C \in\mathfrak{S}_l} \abs{C}\\
&\geq& \big( 1- (1+\delta)(1-\delta) \big)\, \frac{1}{|D|}\, \min_{m \leq i \leq M} \Big|  D \circ \bigcup_{j=1}^{N_i} \cB_{i,j}(y) \Big| \\
&=& \frac{\delta^2}{|D|}\, \min_{m \leq i \leq M} \Big| D \circ \bigcup_{j=1}^{N_i} \cB_{i,j}(y) \Big|.
\end{eqnarray*}
Consequently,
\begin{eqnarray*}
\sum_{l \geq m} \sum_{C \in \mathfrak{S}_l} \abs{C} &\geq& \sum_{l > m} \sum_{C \in \mathfrak{S}_l} \abs{C}  + 
\frac{\delta^2}{|D|}\, \min_{m \leq i \leq M} \Big| D \circ \bigcup_{j=1}^{N_i} \cB_{i,j}(y) \Big| \\
&\geq& \Big( \big(M - (m+1) + 1 \big) \, \frac{\delta^2}{|D|} + \frac{\delta^2}{|D|}  \Big) \, \min_{m \leq i \leq M}\,\Big| D \circ \bigcup_{j=1}^{N_i} \cB_{i,j}(y) \Big|.
\end{eqnarray*} 
Hence, the proof of the inequality~\eqref{eqn:whattoshow} is complete. 
Finally, choosing $M$ large enough such that 
\[
\frac{(M-1)\delta^2}{|D|} \geq (1-\delta)
\]
concludes the proof of the lemma. 
\end{proof}

%\section{Extensions of hyperfinite relations by group actions} 
% cocycle extensions, weak mixing, ergodicity, the pointwise ergodic theorem, entropy,  information functions...........
%Integral convergence result, yields $h^{*}(\cP)$...Define base space entropy 

\section{Proof of the Shannon-McMillan-Breiman theorem} \label{sec:SMB}



The goal of this section is to prove Theorem~\ref{thm:MAIN_SMBhyp}, as
well as~Corollary~\ref{cor:MAIN_SML1}. To this end, we adapt the overall strategy given 
in \cite[Section 4]{Li01} to the situation of amenable equivalence relations.
%This section is devoted to a Shannon-McMillan-Breiman theorem for general countable 
%groups $\Gamma$. 
%For the proof, Lindenstrauss' overall strategy for
%the proof of the Shannon-McMillan-Breiman theorem for p.m.p.\@ actions of
%amenable goups (see Theorem~1.3 in \cite{Li01})
%can be modified to the setting of amenable equivalence relations. However, 
%instead of group actions, 
%we have to deal of groupoid translations in fibers over its unit space,
%given by some probability space. \\
%We keep the notation from the previous sections, where 
As usual, $\cR$ is p.m.p.\@ and hyperfinite, $\Gamma$ is countable and  $\alpha: \cR \to \Gamma$ a  class injective  measurable cocycle. Recall that for a
p.m.p.\@ group action $\Gamma \curvearrowright (X,\lambda)$, 
%we have defined the relation $\cR^X$ as the extension of the equivalence relation $\cR$ by the cocycle $\alpha$ and the $\Gamma$-action on $X$. 
the extended equivalence relation $\cR^X$ over
$\big( X \times Y, \lambda \times \nu \big)$ is defined by the condition 
\[
\big( (x,y), (x^\prime,y^\prime) \big) \in \cR^X \Longleftrightarrow y\cR y^\prime \,\,\text{ and } x=\alpha(y,y^\prime)x^\prime
\,,\]
%or equivalently by the condition
%\[
%\big( (x,y), (x^\prime,y^\prime) \big) \in \cR^X \Longleftrightarrow y\cR y^\prime \,\,\text {and } 
%\, \exists\, \gamma \in\Gamma:\,
%\gamma = \alpha(y,y^\prime), \,\, \gamma x^\prime = x.
%\]
%
When the measure $\nu$ is $\cR$-invariant, it follows that $ \lambda \times \nu$ is $\cR^X$-invariant,
since the $\Gamma$-action on $X$ preserves $\lambda$. The projection map $\pi : \cR^X\to \cR$ given by $(x,y)\to y$ is {\it injective} when restricted to the $\cR^X$-equivalence class of $(x,y)$, for almost all $(x,y)\in \cR^X$.  Further, it 
is well known that an extension of an amenable action is amenable, and thus in particular if $\cR$ is amenable, so is $\cR^X$. But since the extension is class-injective, in fact every hyperfinite exhaustion $(\cR_n)$ of $\cR$ can be canonically lifted to a hyperfinite exhaustion $(\cR_n^X)$ of $\cR^X$, via $\cR_n^X((x,y))=\set{(\alpha(z,y)x,z)\,;\, z\in \cR_n(y)}$. Note that if $(\cR_n)$ is a bounded hyperfinite exhuastion, then so is $(\cR_n^X)$, with the same bounds on the equivalence classes.  









For a given hyperfinite exhaustion $(\cR_n)$ for $\cR$, we define a set 
$\Phi \subseteq \operatorname{Aut}(\cR)$ satisfying the conclusions of Proposition~\ref{prop:finitephin}.
Then, every $\phi \in \Phi$ can be extended naturally to an inner automorphism
$\phi^X \in \operatorname{Aut}(\cR^X)$ by setting
\begin{equation}\label{phi^X}
\phi^{X} \big( (x,y) \big) := \big( \alpha(\phi(y),y)x, \phi(y) \big).
\end{equation}
For a subset $D \subseteq \Phi$, we write $D^X$ for the set 
$\big\{ \phi^X\,|\, \phi\in D \big\}$. Note that by definition, the set 
$\Phi^X:= \{ \phi^X\,|\, \phi \in \Phi\}$ is generating for the relation $\cR^X$.
Clearly, if $\phi$ preserves the classes of $\cR_n$ almost surely, then $\phi^X$ preserves the classes of $\cR_n^X$ almost surely. 



Let us now recall the concept of ergodicity for a
measure preserving equivalence relation. Let $\cZ$ be such a relation over a probability space $(Z,\eta)$. 
A subset
$A \subseteq Z$ will be called $\cZ$-{\em invariant} if 
\[
\big( A \times Z \big)\, \cap \cZ = A \times A. 
\]
The relation $\cZ$ is {\em ergodic} if every 
$\cZ$-invariant set $A \subseteq Z$ satisfies $\eta(A) \in \{0,1\}$.
From now on, we will always assume that the relation $\cR^X$ is {\em ergodic}.
For sufficient conditions guaranteeing this property, we refer to the discussion in \S \ref{sec:ergodicity} below. 


%\begin{Theorem}[SMB-Theorem for hyperfinite equivalence relations] \label{thm:SMBhyp}
%Let $\Gamma$ be a countable group which acts on a probability space $(X,\lambda)$
%by p.m.p. Assume further that $\cR$ is an amenable, measure preserving equivalence
%relation over a probability space $(Y,\nu)$ which gives rise to a measurable
%cocycle $\alpha: \cR \to \Gamma$ and such that the extended relation $\cR^X$ is ergodic. \\
%Then, for every pointwise approximate hyperfinite exhaustion $(\cF_n)$, 
%with the property that 
%\[
%\lim_{n \to \infty}|\cF_n(y)|/\log\,n = \infty \quad \quad \nu\mbox{- a.e. } y \in Y,
%\]
%we have the following convergence result.
%Given a finite partition $\cP$ of $X$, for $(\lambda \otimes \nu)$-almost every $(x,y) \in X \times Y$, 
%\[
%\lim_{n \to \infty} \frac{\cJ\big( \cP^{\cF_n(y)}(x) \big)}{|\cF_n(y)|} = h^{*}_{\cP},
%\]
%where $h^{*}_{\cP}$ is the cocycle entropy of the partition
%$\cP$. 
%\end{Theorem}

%\begin{Corollary}[$L^1$-convergence of the information function] \label{cor:SML1}
%The above convergence theorem also holds
%for $\nu$-almost every $y \in Y$ in $L^1(X,\lambda)$ and 
%in $L^1(X \times Y, \lambda \otimes \nu)$. 
%\end{Corollary}

One crucial ingredient for the proofs of Theorem~\ref{thm:MAIN_SMBhyp} and Corollary~\ref{cor:MAIN_SML1}
is the pointwise ergodic theorem. A general form of it being valid for suitable asymptotically invariant sequences of subset functions in an  amenable equivalence relation
was established in \cite[Thm.\@ 2.1]{BN13b}. 
Since we restrict our discussion here to hyperfinite exhaustions, we will state a less general but easily accessible special case which is sufficient for our purposes. Indeed, the following fact is an immediate consequence of the martingale convergence theorem.  

\begin{Theorem} \label{thm:ETequiv}
Let $\cZ$ be an ergodic, p.m.p.\@  equivalence relation 
over a probability space $(Z,\eta)$. Let $(\cZ_n)$ be a bounded hyperfinite exhaustion for $\cZ$. \\ 
Then for all $f \in L^1(Z,\eta)$, we have 
\begin{eqnarray*}
\lim_{n \to \infty} \big| \cZ_n(z) \big|^{-1}\,\sum_{w \in \cZ_n(z)} f(w)
 = \int_Z f(w)\, d\eta(w)
\end{eqnarray*}
for $\eta$-almost every $z \in Z$. 
\end{Theorem}

%\begin{proof}
%We observe that hyperfinite exhaustions are asymptotically invariant
%(as a consequence of Proposition~\ref{prop:finitephin})
%and regular (since all the $\cZ_n$ are equivalence relations) 
%in the sense of the setting of~\cite{BN13b}. 
%The theorem then is a consequence of Theorem~2.1 there.

%\end{proof}

We now turn to establish some preliminary 
lemmas that will be used below. 
The first step is a straightforward consequence of the ergodicity of
the relation. 

\begin{Lemma} \label{lemma:ergodicity}
Let $\cZ$ be an ergodic p.m.p. equivalence relation over a probability $(Z, \eta)$ along
with a countable set $\Phi$ of inner automorphisms generating $\cZ$.
Then, for every $\delta > 0$, and every set $A \subseteq Z$ with 
$\eta(A) > 0$, there is a finite set $D \subset \Phi$ such that 
\[
\eta\big( D \circ A \big) \geq 1 - \delta. 
\]
\end{Lemma}
%{\bf This is true in general, but it may be necessary to assume that $D$ is of finite rank, namely 
%$D\circ \cR_n =\cR_n$ for all sufficiently large $n$. This may have to come into the definition of $\Phi$ itself.} 
\begin{proof}
Let $A \subseteq Z$ be measurable with $\eta(A) > 0$. 
Assume that there is some $\delta_0 > 0$ such that 
for all finite collections $D \subset \Phi$, we have $\eta(D \circ A) < 1 - \delta_0$.
We define $\overline{A} := \Phi \circ A = \bigcup_{\phi \in \Phi} \phi(A)$. This 
set is invariant under the relation $\cZ$. To see this, consider $z \in Z$, $\phi \in \Phi$ and $a \in A$ such that 
$(\phi(a), z) \in \cZ$. Since $\Phi$ is generating, there is a further
inner automorphism $\phi^{\prime} \in \Phi$ such that $z = \phi^{\prime}\phi(a)$.
We conclude that $z \in \Phi \circ A$ and thus, 
\[
\big( \overline{A} \times Z \big) \cap \cZ = \overline{A} \times \overline{A}.
\]  
Since $\Phi$ is countable, it follows $\eta(\overline{A}) \le 1 -\delta_0$ by
our assumption. Now $\overline{A}$ is an invariant set, hence by ergodicity
of the relation, we obtain $\eta(\overline{A}) = 0$. However, since every
inner automorphism preserves the measure, we have $\eta(\overline{A}) \geq
\eta(A) > 0$. This is a contradiction. 
\end{proof}

We need the following combinatorial lemma which is the measured equivalence relation
analog of the version for F{\o}lner sequences, see Lemma~1 in \cite{OW83}
and Lemma~4.2 in \cite{Li01}.
%
\begin{Lemma} \label{lemma:comb}
Let $\cR$ be a p.m.p. equivalence relation over $(Y,\nu)$, and let $(\cR_n)$ be a bounded hyperfinite exhaustion satisfying  
$\lim_{n \to \infty} \operatorname{ess}\,\operatorname{inf}_y |\cR_n(y)| = \infty $.
%$\lim_{n \to \infty}|\cR_n(y)| = \infty$ for $\nu$-almost every $y \in Y$. 
%and assume that $\cR_n(y)\ge a_n$ for almost all $y$, with $a_n\nearrow \infty$ {\bf REALLY NEEDED HERE???}. \\
Then, for every $\eta > 0$, there is some $\ell \in \NN$ such that the following holds.

Suppose that to $y \in Y$  
there corresponds a finite increasing sequence $(k_i), 1 \leq i \leq r(y)$ of integers (depending on $y$)  with  $|\cR_{k_1}(z)| \geq \ell$
for almost every $z \in Y$. 
%subset functions $\bar{\cR}^y_1 \subseteq \bar{\cR}^y_2 \subseteq 
%\dots \subseteq \bar{\cR}^y_r$ such that each $\bar{\cR}^y_i$, $ 1\le i \le r$ is equal to some subset function 
%$\cR_{k_i}$ with $$ and such that $\bar{\cR}^y_i(z) \setminus \bar{\cR}^y_{i-1}(z) \neq \emptyset$ for all $1 \leq i \leq r$
%$and every $z \in [y]$
%(with the convention that here, we set $\bar{\cR}^y_0 = \emptyset$). 
Then, there is $n_0 \in \NN$ such that for all $n \geq n_0$, the number of 
possible disjoint subcollections $\mathfrak{S}(y)$ of the form 
\[
\mathfrak{S}(y) \subseteq \big\{{\cR}_{k_i}(c) \,|\, c \in \cR_n(y), 1 \leq i \leq r(y) \big\}
\]
is at most $2^{\eta|\cR_n(y)|}$.
%where $c> 0$ does not depend on $n$, $N$, $\beta$, $r$ or $y$. 
\end{Lemma}



\begin{proof}
Let $\eta > 0$. Consider $\ell \in \NN$, $y \in Y$ and
an increasing  sequence $(k_i), 1 \leq i \leq r(y)$ of integers (depending on $y$) with $|\cR_{k_1}(z)| \geq \ell$ 
for all $1 \leq i \leq r(y)$
and almost every $z \in Y$. 
%statement of the lemma, with $N$ replaced by $\ell > 2$ for the moment. 
%By assumption, for all $z \in [y]$ and each $1 \leq i \leq r$, 
%we can pick elements $d^y_i(z) \in \bar{\cR}^y_i(z) \setminus \bar{\cR}^y_{i-1}(z)$. 
Since we assume that $\mathfrak{S}(y)$ is a collection of disjoint ${\cR}_{k_i}$-classes contained
in $\cR_n(y)$, there is a center set $\cC(y) \subseteq \cR_n(y)$ such that 
\[
\mathfrak{S}(y) = \big\{ {\cR}_{k_{t(c)}}(c)\,|\, c \in \cC(y)\big\} \text{  with  }  1 \leq t(c) \leq r(y) .
\]
Given $\mathfrak{S}(y)$ (and hence $\cC(y)$), we define a set of points $\cN(y)$ as follows. 
For $c \in \cC(y)$, define $n(c):= i$ as the maximal $1 \leq i \leq t(c)$ such that
$\cR_{k_i}(c) \setminus \cR_{k_{i-1}}(c) \neq \emptyset$, where $\cR_0 = \emptyset$ by 
convention. Then add to the set $\cN(y)$ an arbitrary point $p(c) \in \cR_{k_i}(c) \setminus \cR_{k_{i-1}}(c)$.
%If there is no such point,  
%we have $\cR_{k(c)}(c) \setminus \cR_{k(c)-1}(c) \neq \emptyset$,
%then we add exactly one point out of this latter set to $\cN(y)$. If 
%$\cR_{k(c)}(c) = \cR_{k(c)-1}(c)$, then we leave $\cN(y)$ unchanged. 
By processing in
that way for all $c \in \cC(y)$, we obtain a set $\cN(y)$ with cardinality at most $\abs{\cC(y)}$. We claim that   
we can uniquely recover $\mathfrak{S}(y)$ from knowing the elements in both sets $\cC(y)$ and $\cN(y)$.
Indeed, given $\cC(y)$ and $\cN(y)$, by construction, for every $c \in \cC(y)$, 
there is a minimal (and thus unique) $1 \leq t(c) \leq r(y)$ such that ${\cR}_{k_{t(c)}}(c) \cap \cN(y) \neq \emptyset$. 
%Further, for possible center sets $\cC(y)$,
%we have $|\cC(y)| = |\cN(y)|$ due to the disjointness property of the subcollection. 
Hence, 
the number of possible disjoint subcollections must be bounded by the number of choices for
the two sets $\cC(y)$ and $\cN(y)$. 
Since the
sizes of all classes involved are uniformly bounded from below by $\ell$, the cardinality of 
every $\cC(y)$ and $\cN(y)$ is bounded from above by $\lceil |\cR_n(y)|/\ell \rceil$.
In light of that, we need to bound the expression
\begin{eqnarray*}
\big( \lceil|\cR_n(y)|/\ell\rceil + 1 \big)^2 \, \binom{|\cR_n(y)|}{\lceil |\cR_n(y)| /\ell \rceil}^2.
\end{eqnarray*}
To do so, we use the entropy formula for the Stirling approximation. 
For this purpose, we define 
 \[
E(\ell):= \frac{1}{\ell}\, \log\,\ell + \Big(1 - \frac{1}{\ell} \Big)\,\log\,\Big(1 - \frac{1}{\ell} \Big)^{-1}.  
\]
%We note that any disjoint subcollection of $\cR_n(y)$ by tiling sets of size at least
%$\ell$ (where $|\cR_n(y)|>> \ell$) contains at most $\lceil|\cR_n(y)|/\ell\rceil$ many tiles. 
%Further, there is an 
%injective map from the collection of all disjoint subcollections with their elements 
%having minimum size $\ell$ to the collection of all subsets of $\cR_n(y)$
%consisting of at most $\lceil|\cR_n(y)|/\ell\rceil$ elements. {\bf Injectivity seems to be obtained when for every  set in the disjoint subcollection one chooses a "center" $a\in \cR_n(y)$, and an index $j\le n$ (a "radius") so that the set $\bar{\cR}_i(a)$ is $\cR_j(a)$. }
%Hence, and since $\ell > 2$, 
%the number of these disjoint subcollections must be bounded by
%\begin{eqnarray*}
%\big( \lceil|\cR_n(y)|/\ell\rceil +1 \big)\, \binom{|\cR_n(y)|}{\lceil|\cR_n(y)|/\ell\rceil}.
%\end{eqnarray*}  
%(To see this, fix an element )
%of size at least $\ell$ {\bf the bound for size at least $\ell$ is the same as the bound for size exactly $\ell$ ??????} in a set $\cR_n(y)$ (where $|\cR_n(y)|/2 > \ell$) is bounded 
%by the value of the binomial coefficient $\binom{|\cR_n(y)|}{\lceil|\cR_n(y)|/\ell\rceil}$.
By Stirling's approximation (see e.g.\@ \cite{FS08}, Example~VIII.10), for large enough $n \in \NN$, we get 
\begin{eqnarray*}
\binom{|\cR_n(y)|}{\lceil|\cR_n(y)|/\ell\rceil}^2 \leq \left(\exp \Big( E(\ell/2)|\cR_n(y)| \Big)\right)^2 \leq 2^{4\,E(\ell/2)\,|\cR_n(y)|}. % c\,2^{E(\ell)\,|\cR_n(y)|}.
\end{eqnarray*}  
%for some constant $c^{\prime}$ independent of $n$, $\ell$ and $y$. 

%(The fact that $c^{\prime}$ does not depend on $y$ is due to the  
%uniform boundedness of the values $|\cR_n(y)|$ from below by the sequence $(a_n)$ which tends to infinity.) 
%Now choosing any $c > c^{\prime}$, 
Now increasing $n$ if necessary, we can make sure that
\[
\big( \lceil\cR_n(y)/\ell\rceil +1 \big)^2\, \binom{|\cR_n(y)|}{\lceil|\cR_n(y)|/\ell\rceil}^2 
\leq 2^{5\,E(\ell/2)\,|\cR_n(y)|}.
\]
Since $E(\ell) \to 0$
as $\ell \to \infty$, we can find some $\ell$ such that $E(\ell/2) < \eta/5$, and  
this completes the proof of the claim. 
\end{proof}


%ANOTHER POSSIBLE VERSION


%For the following lemma, we work with the following notion. Namely, for
%For $\beta > 0$, we define 
%\[
%E(\beta):= -\beta\, \log\,\beta - \Big(1 - \beta \Big)\,\log\,\Big(1 - \beta \Big).  
%\]
%It is immediate that $E(\beta) \to 0$ as $\beta \to 0$.  


%\begin{Lemma}[Combinatorial Lemma] \label{lemma:comb2}
%Let $(\cR_n)$ be a bounded hyperfinite exhaustion for $\cR$, and assume that $\cR_n(y)\ge a_n$ for almost all $y$, with $a_n\nearrow \infty$.
%Then, for every $\eta > 0$, there is some $N \in \NN$ such that the following holds.\\
%Suppose that for $y \in Y$, 
%there is a finite sequence of subset functions $\bar{\cR}^y_1 \subseteq \bar{\cR}^y_2 \subseteq 
%\dots \subseteq \bar{\cR}^y_r$ such that each $\bar{\cR}^y_i$, $ 1\le i \le r$ is equal to some subset function 
%$\cR_{k_i}$ with $|\cR_{k_i}(z)| \geq N$ for $\nu$-almost every $z \in Y$.
%%and such that $\bar{\cR}^y_i(z) \setminus \bar{\cR}^y_{i-1}(z) \neq \emptyset$ for all $1 \leq i \leq r$
%$\nu$-almost every $z \in Y$. 
%(with the convention that here, we set $\bar{\cR}^y_0 = \emptyset$). 
%Then, there is $n_0 \in \NN$ such that for all $n \geq n_0$, the number of 
%possible subcollections $\cS(y)$ consisting of disjoint translates $\bar{\cR}^y_i(c) \subseteq \cR_n(y)$,
%(where $c \in \cR_n(y)$) 
%and satisfying $\sum_{C \in \cS(y)}|C| \geq (1 -\eta)|\cR_n(y)|$ is at most $2^{(\eta + E(\eta))|\cR_n(y)|}$.
%\end{Lemma}

%\begin{proof}
%Let $\eta > 0$. For $\ell \in \NN$ and $y \in Y$, suppose we are given a disjoint subcollection as
%described in the statement of the lemma, with $N$ replaced by $\ell$ for the moment. 
%We show first that for a prescribed part $\cZ(y) \subseteq \cR_n(y)$ with 
%$|\cZ(y)| \geq (1-\eta)|\cR_n(y)|$, we can uniquely recover all subcollections $\cS(y)$ of 
%the relevant form which satisfy $\cup \cS(y) = \cZ(y)$ by looking at specific subsets of $\cZ(y)$.
%Note that there is a set $\cC(y) \subseteq \cZ(y)$ of centers such that 
%$\cS(y) = \{ \bar{\cR}^y_{i(c)} (c)\,|\, c\in \cC(y) \}$, where for each $c$, we assign an 
%%index $1 \leq i(c) \leq r$. We claim that this index is already determined by $c$ and 
%is equal to the maximal element $1 \leq i \leq r$
%such that $\bar{\cR}^y_{i}(c) \cap C(y)\setminus \{c\} = \emptyset$. Indeed, if 


%We now distinguish subcollections in terms of $\cZ(y)$ and
%$\cC(y)$. Namely, given $c \in \cC(y)$, we define $i(c)$ to be the maximal element $1 \leq i \leq r$
%such that $\bar{\cR}^y_{i}(c) \cap C(y)\setminus \{c\} = \emptyset$. This way, two center 
%sets must give rise to distinct covers of $\cZ(y)$. 
%Clearly, all possible covers arise in that
%form. 
%\end{proof}


We are ready to prove the main lemma of this section. 
It is motivated by Lemma~4.3 in \cite{Li01} and 
provides an analog of it for hyperfinite exhaustions. 

\begin{Lemma}%[Abstract dynamical covering]
 \label{lemma:dyncov1}
Let $(B_k)$ be a sequence of measurable sets in $X \times Y$ such that 
\begin{eqnarray*}
\lambda \times \nu \Bigg( \bigcap_{k=1}^{\infty} \bigcup_{j \geq k} B_j \Bigg) > 0. 
\end{eqnarray*}
Then, for every $\delta > 0$ and $\lambda \times \nu$-a.e.\@ $(x,y) \in X \times Y$,
there is $n(x,y) \in \NN$ for which the following holds true: for each $n \geq n(x,y)$,
there is a disjoint collection of subsets of $\cR_n(y)$ 
\[
\mathfrak{S} = \big\{ \cR_{k_i}(b_i)\,|\, 1 \leq i \leq r \big\}
\]
with $b_i \in \cR_n(y)$ and $1 \leq k_i < n(x,y)$, such that 
\begin{enumerate}[(i)]
\item $\big( \alpha(b_i, y)x, b_i \big) \in B_{k_i}$ for all $i$,
\item $\sum_{C \in \mathfrak{S}} \abs{C} \geq (1-\delta)\,|\cR_n(y)|$.
\end{enumerate}
\end{Lemma}



\begin{proof}
Let $\delta > 0$. We set 
\[
B^{*} := \bigcap_{k=1}^{\infty} \bigcup_{j \geq k} B_j.
\]
By assumption, $(\lambda \times \nu)(B^{*})> 0$. 
Given the hyperfinite exhaustion $(\cR_n)$, we fix  
$\Phi^{X} \subseteq \operatorname{Aut}(\cR^X)$ satisfying the conclusions of Proposition~\ref{prop:finitephin}.
Since the extended
relation $\cR^X$ is ergodic, and since the set $\Phi^X$
is generating for $\cR^X$,
 we can use Lemma~\ref{lemma:ergodicity}
in order to find a finite set $D\subset \Phi$ such that the lifted automorphisms $D^X \subseteq \Phi^{X}$
 satisfy 
\begin{eqnarray} \label{eqn:useergodicity}
\lambda \times \nu\big( D^X \circ B^{*} \big) \geq 1- \delta/10.
\end{eqnarray}
Thus clearly $\lambda \times \nu\big( B^{*} \big) \geq (1- \delta/10)/\abs{D}$.
%By making inverses $\phi^{-1}$, we make sure that $D$ is a symmetric set
%and we also assume that the identity automorphism is contained in $D$.
%{\bf We define $\delta^{\prime}:= \delta/(4|D|)$ : not necessary.}
Further,
choose $M \in \NN$ such that Lemma~\ref{lemma:abstrcomb} holds
true for the fixed parameters $\delta$ and $D$ for the relation $\cR$. 

 The next step is
to construct two finite increasing integer sequences $(m_i), (N_i)$, $1 \leq i \leq M$,
according to the following algorithm.
\begin{enumerate}[(1)]
\item Define $m_1:= 1$.  
% as the smallest integer such that for every $l \geq m_1$, we have $D \circ \cR_l \subseteq \cR_l$.
\item If $m_i$ has been chosen, then determine $N_i$ large enough such that 
\[
\lambda \times \nu \Big( B^{*} \setminus \bigcup_{j=m_i}^{m_i + N_i} B_j \Big)
< \frac{\delta\cdot\lambda\times \nu(B^\ast)}{10\,M\,|D|}.
\]
\item Further, if $N_i$ has been chosen, choose $m_{i+1}$ large enough such that for
every $l \geq m_{i+1}$, we get
\[
\bigcup_{j < m_i + N_i} D \circ \cR_j^{-1}\cR_{l} \subseteq \cR_l.
\]
%\[
%\Big|\bigcup_{j < m_i + N_i} D \circ \cR_j^{-1}\,\cR_l \Big| \leq (1+ \delta^{\prime}) \big|\cR_l\big|. 
%\]
This is possible by Proposition~\ref{prop:finitephin} and since $D\subset \Phi$.  
%by the choice of $\Phi$ and since $(\cR_n)$ consists of increasing equivalence relations, 
%we can even make sure that $m_{i+1}$ is large enough such that
%\[
%\bigcup_{j < m_i + N_i} D \circ \cR_j^{-1}\cR_{l} \subseteq \cR_l
%\]
%for $l \geq m_{i+1}$. 
%Note that $D\subset \Phi$ satisfies 
In fact $D\circ \cR_n =\cR_n$ for all sufficiently large $n$. So it suffices that $m_{i+1} > m_i+N_i$ and also that $\cR_{m_{2}}$ (and hence each $\cR_{m_{i+1}}$) is invariant under $D$. 



\end{enumerate}
With the sequences $(m_i), (N_i)$ at our disposal, we define
\[
\tilde{B}^{*} := \bigcap_{i=1}^M \bigcup_{j=m_i}^{m_i + N_i} B_j. 
\]
Clearly 
\[
\lambda \times \nu \big(   B^{*} \setminus  \tilde{B}^{*} \big) \leq
%< \frac{\delta\cdot\lambda\times \nu(B^\ast)}{10}.
\sum_{i=1}^M \lambda \times \nu \Big( B^{*} \setminus \bigcup_{j=m_i}^{m_i+ N_i} B_j \Big)
< \frac{\delta\cdot\lambda\times \nu(B^\ast)}{10\,|D|},
\]
and so 
\[
\lambda \times \nu \big( D^X \circ  B^{*} \setminus D^X \circ \tilde{B}^{*} \big) <
\frac{\delta\cdot\lambda\times \nu(B^\ast)}{10}.
\]
Consequently, 
\[
\lambda \times \nu\big( D^X \circ \tilde{B}^{*} \big) \geq \nu\big( D^X \circ {B}^{*} \big) - \frac{\delta\cdot\lambda\times \nu(B^\ast)}{10}
\geq 1-\frac{\delta}{10}- \frac{\delta\cdot\lambda\times \nu(B^\ast)}{10}\ge 1-\frac{\delta}{5}.   
\]
Now by the pointwise ergodic theorem (Theorem~\ref{thm:ETequiv}), for $\lambda \times \nu$-a.e.\@ 
$(x,y)$, we can define $n(x,y)$ as the smallest integer value greater than $m_M + N_M$
 such that for all $n \geq n(x,y)$,
we have 
\begin{eqnarray} \label{eqn:ErgThm}
\cA\big( n, \one_{D^X\circ \tilde{B}^{*}} \big)(x,y):= 
\big| \cR_n(y) \big|^{-1} \sum_{z \in \cR_n(y)} \one_{D^X \circ \tilde{B}^{*}}\big( (\alpha(z,y)x,z) \big) > 1 - \frac{\delta}{4}.
\end{eqnarray} 
We fix some pair $(x,y)$ satisfying this condition, as well as $n \geq n(x,y)$. For $1 \leq i \leq M$ and $1 \leq j \leq N_i$, set
${\cT}_{i,j}:= \cR_{m_i + j - 1}$ and 
\[
A_{i,j}(y) := \big\{ b \in \cR_n(y) \,|\, \big( \alpha(b,y)x, b \big) \in B_{m_i + j -1} \big\}.
\]
By step~(3) of the algorithm and since $n > m_M + N_M$, 
we have $D \circ \cR_n \subseteq \cR_n$. In particular, each $\phi\in D$ gives rise to bijections $\phi :\cR_n(y)\to \cR_n(y)$, for a.e. $y\in Y$. 


%using also the symmetry of the set $D$ (and 
%the trivial automorphism being contained in $D$), 
%we know that we can find a set $\cR_n^{\prime}(y) \subseteq \cR_n(y)$
%with $|\cR_n^{\prime}(y)| \geq (1-|D|\,\delta^{\prime})|\cR_n(y)|$ and at the same time 
%$D \circ \cR^{\prime}_n(y) \subseteq \cR_n(y)$.
Applying the transformations in the set $D$ to the sets $A_{i,j}(y)$, we then obtain 
%Since $\big|D \circ \cR_n \big| \leq (1+\delta^{\prime}) \big|\cR_n\big|$ 
%by the step (3) of the construction algorithm for the sequences $(m_i), (N_i)$
%and $n > m_M + N_M$,
%we obtain
\begin{eqnarray*}
\end{eqnarray*}
\begin{eqnarray*}
D \circ \bigcup_{j=1}^{N_i} A_{i,j}(y) &=& \bigcup_{\phi \in D} \big\{ \phi(b) \,|\, b \in \cR_n(y), \big( \alpha(b,y)x, b \big) 
\in \bigcup_{j=1}^{N_i} B_{m_i + j -1} \big\} \\
&=& \big\{ b \in \cR_n(y)\,|\,\exists\, \phi \in D,\, b^{\prime} \in \cR_n(y): \, b = \phi(b^{\prime}),\, \big( \alpha(b^{\prime}, y)x, b^{\prime} \big)
\in \bigcup_{j=1}^{N_i} B_{m_i + j -1} \big\} .
\end{eqnarray*}
Together with~\eqref{eqn:ErgThm}, this yields for all $1 \leq i \leq M$ :
%and the symmetry of $D$ (and the identity automorphism being 
%contained in $D$), this yields
\begin{eqnarray*}
\Big| \bigcup_{j=1}^{N_i} D \circ A_{i,j}(y) \Big| &=&  \big| \big\{ b\in \cR_n(y)\,|\, \exists\, \phi \in D:\,
 b^\prime = \phi^{-1}(b), \, \big( \alpha(b^{\prime}, y)x, b^{\prime} \big)
\in \bigcup_{j=1}^{N_i} B_{m_i + j -1}  \big\}\big|   \\ \text{(using (\ref{phi^X}))}
%&\quad& - \big|\cR_n(y) \setminus \cR^{\prime}_n(y) \big| \\
&=& \big| \big\{ b \in \cR_n(y)\,|\, \big( \alpha(b,y)x, b \big) \in D^X \circ \bigcup_{j=1}^{N_i} B_{m_i + j -1} \big\} \big| \\\text{(since $\tilde{B}^\ast$ is an intersection)}
%- \delta / 4\,\big| \cR_n(y) \big| \\
&\geq& \big| \cR_n(y) \big|\, \cA\big(n, \one_{D^X \circ \tilde{B}^{*}}  \big)(x,y) \\ %-  \delta/4\,\big| \cR_n(y) \big| \\
(\text{by the pointwise ergodic theorem}) 
&\geq&\Big( 1- \frac{\delta}{4} \Big) \,\big| \cR_n(y) \big|\,.
\end{eqnarray*}
  %(Here, all elements denoted by $b^{\prime}$ are elements in $\cR^{\prime}_n(y)$.)\\

We finally apply Lemma~\ref{lemma:abstrcomb} with $\delta/4$ instead of $\delta$ to the arrays ${\cT}_{i,j}$ and $A_{i,j}(y)$, where $1 \leq i \leq M$ and
$1 \leq j \leq N_i$. The assumption of the Lemma  is indeed satisfied, namely $\cT_{i,j}=\cR_{m_i+j-1}$ satisfy (5.1) by the construction of $m_2$, which guarantees $D\circ \cR_m=\cR_m$ for all $m \ge m_2$. Furthermore, we have just shown that 
$$\min_{1\le i \le M}\Big| \bigcup_{j=1}^{N_i} D \circ A_{i,j}(y) \Big|\ge \Big( 1- \frac{\delta}{4} \Big) \,\big| \cR_n(y) \big|$$
and hence 
Lemma~\ref{lemma:abstrcomb} implies that there is a disjoint subcollection
\[
\mathfrak{S}(y) \subseteq \big\{\cT_{i,j}(b) \,|\, b \in A_{i,j}(y) \big\}
\]
with 
\[
\sum_{C \in \mathfrak{S}(y)} \abs{C} \geq (1-\delta)\,\big| \cR_n(y) \big|,
\]
as desired. 
\end{proof}

The following is an immediate consequence from the previous 
lemma. 

\begin{Lemma}%[Dynamical covering]
 \label{lemma:dyncov2}
Let $(\cR_n)$ be a bounded hyperfinite exhaustion, and keep the assumptions of the previous lemma. Then, for the number
\begin{eqnarray*}
h:= \operatorname{ess-inf}_{x,y} \, \liminf_{n \to \infty} \frac{\cJ\big( \cP^{\cR_n(y)}(x)\big)}{|\cR_n(y)|},
\end{eqnarray*}
the following holds true. 
For every $\delta > 0$, each $N \in \NN$ and $\lambda \times \nu$-a.e.\@ $(x,y)$,
there is a number $n(x,y) \in \NN$ such that for all $n \geq n(x,y)$, we can find
a disjoint collection (depending on both $x$ and $y$)
\[
\mathfrak{S} = \big\{ \cR_{k_i}(b_i) \,|\, 1 \leq i \leq r \big\}
\]
such that all $b_i \in \cR_n(y)$ and $N \leq k_i < n(x,y)$ and further,
\begin{enumerate}[(i)]
\item for all $i$, 
\[
\frac{\cJ\big( \cP^{\cR_{k_i}(b_i)}\big( \alpha(b_i,y)x\big) \big)}{|\cR_{k_i}(b_i)|} \leq h + \delta, 
%\quad \frac{|\cR_{k_i}(b_i)\setminus \cR_{k_i}(b_i)|}{|\cR_{k_i}(b_i)|} < \delta,
\]
\item  $\sum_{C \in \mathfrak{S}} \abs{C}\geq (1-\delta)\,|\cR_n(y)|$.  
\end{enumerate}
\end{Lemma}

\begin{proof}
%By assumption, we have 
%\begin{eqnarray*}
%\lim_{n \to \infty} \frac{\big| \cF_n(y)\,\setminus\,\cR_n(y) \big|}{\big| \cR_n(y) \big|} = 0
%\end{eqnarray*}
%for all $y \in Y_0$, where $Y_0$ is a $\nu$-conull subset of $Y$.
We apply the previous Lemma~\ref{lemma:dyncov1} to the sequence 
$(\cR_n)_{n \geq N}$ and the sets 
\[
B_k := \Big\{ (x,y) \in X \times Y\, \big| \, \frac{\cJ\big( \cP^{\cR_{N+k}(y)}(x) \big)}
{|\cR_{N + k}(y)|} \leq h + \delta\,\, %\frac{|\cF_{N+k}(y) \setminus \cR_{N+k}(y)|}{|\cR_{N+k}(y)|} < \delta
 \Big\}.
\]
%The observation that $\cP^{\cR_{N+k}(b)}(x) = \cP^{\cR_{N+k}(y)}\big( \alpha(b,y) x\big)$ 
%for $b \in \cR_n(y)$
%concludes the proof of the lemma. 
\end{proof}

We are now in position to prove the Shannon-McMillan-Breiman Theorem. % ~\ref{thm:MAIN_SMBhyp}.
To do so, we combine the previous results of this section, motivated by  
the proof of Theorem~1.3 in~\cite{Li01}.

\begin{proof}[{Proof of Theorem~\ref{thm:MAIN_SMBhyp}}]
%Fix a pointwise approximate hyperfinite exhaustion $(\cF_n)$. 
%Then, by definition, there is a hyperfinite
%exhaustion $(\cR_n)$ with $\cF_n \supseteq \cR_n$ such that 
%\begin{eqnarray} \label{eqn:approxaux}
%\lim_{n \to \infty} \frac{\big| \cF_n(y)\,\setminus\,\cR_n(y) \big|}{\big| \cR_n(y) \big|} = 0
%\end{eqnarray}
%for all $y \in Y_0$, 
%%and $z \in [y]$, 
%where $Y_0$ is a $\nu$-conull subset of $Y$.  
As before, we set
\begin{eqnarray*}
h:= \operatorname{essinf}_{x,y} \, \liminf_{n \to \infty} \frac{\cJ\big( \cP^{\cR_n(y)}(x)\big)}{|\cR_n(y)|}.
\end{eqnarray*}
If $h= \infty$, then there is nothing left to show. So assume that $h < \infty$. 
We show that for a.e.\@ $(x,y)$,
\begin{eqnarray} \label{eqn:mainclaim}
\limsup_{n \to \infty} \frac{\cJ\big( \cP^{\cR_n(y)}(x) \big)}{|\cR_n(y)|} \leq h. 
\end{eqnarray}
To this end, fix $\delta > 0$. 
We find $N \in \NN$ large enough such that 
Lemma~\ref{lemma:comb} holds for $\eta = \delta$ and $\ell = N$.
%to pick $N_0$ large enough so that
%\begin{eqnarray} \label{eqn:growth}
%\big| \cR_{j}(y) \big| \geq \max\big\{ 8\delta^{-1}\,\big( \log\, j \big); N \big\}
%\end{eqnarray}
%for every $j \geq N_0$.
By the growth assumption on the $(\cR_n)$, we find $N_1 \in \NN$ such that for 
a full measure set of $y$'s, we have $|\cR_k(z)| \geq N$ for all $k \geq N_1$ and
each $z \in [y]$.   
We have seen in Lemma~\ref{lemma:dyncov2} that for 
$\lambda \times \nu$-almost every $(x,y)$, there is $n(x,y) \in \NN$, $n(x,y) > N_1$,
such that for $n > n(x,y)$, we obtain a special subcollection 
$\mathfrak{S}=\mathfrak{S}(x,y)$ of
subsets of $\cR_n(y)$. Namely, this collection 
\begin{itemize}
\item is disjoint, 
\item satisfies $\sum_{C \in \mathfrak{S}}\abs{C} \geq (1-\delta)\,|\cR_n(y)|$,
\item Each $C \in \mathfrak{S}$ is of the form $\cR_{k}(b)$ with $N_1 \leq k< n(x,y)$ and $b \in \cR_n(y)$, which 
by the choice of $N_1$ implies that $|\cR_k(b)| \geq N$ for all $k \geq N_1$, 
\item %for the sets $\cF_k(b)$ for which 
writing $C= \cR_k(b) \in \mathfrak{S}$, the following holds
\begin{eqnarray} \label{eqn:infaux}
\frac{\cJ\big( \cP^{\cR_k(b)}\big(\alpha(b,y)x\big) \big)}{|\cR_k(b)|} \leq h + \delta\,.
%\quad \frac{|\cF_k(b) \setminus C}{|C|} < \delta.
\end{eqnarray}
\end{itemize}
By increasing $n$ if necessary, due to $|\cR_n(y)| \to \infty$, we can assume that % by the growth condition on $(\cR_n)$ that 
\begin{eqnarray} 
\big| \cR_{n}(y) \big| \geq  8\delta^{-1}\,( \log\, n ). \label{eqn:growth} 
%\big| \cR_n(y) \big| &\leq& (1+\delta)\,\big| \cR_n(y)\big|. \label{eqn:growth2}
\end{eqnarray}
%{\bf This assumes there is a lower bound on the size of all classes, and so a bounded relation cannot just be extended trivially.  
Let us now fix $x$, $y$, $n=n(x,y)$ and $\mathfrak{S}(x,y)=\mathfrak{S}$ satisfying all the conditions stated above. 

We first note that since $\cR_n(y)$ is the disjoint union of the sets $C = \cR_k(b_C) \in \mathfrak{S}$ and  
$\cG_n(y):= \cR_n(y) \setminus \bigcup_{C \in \mathfrak{S}} \cR_k(b_C)$, it follows that the partition 
$\cP^{\cR_n(y)}$ is given by $\left(\bigvee_{C \in \mathfrak{S}} \cP^{\cR_k(b_C)}\right)\vee \cP^{\cG_n(y)}$. 
Therefore for any point $x\in  X$, its $\cP^{\cR_n(y)}$-name arises as the intersection of  
the $\cP^{\cR_k(b_C)}$-name of $\alpha(b_C,y)x$ where $C = \cR_k(b_C) \in \mathfrak{S}$,  and of the 
$\cP$-name of $\alpha(b,y)x$ for  every 
%{\bf what is $d$ here ??????????}  
$b \in \cG_n(y):= \cR_n(y) \setminus \bigcup_{C \in \mathfrak{S}} \cR_k(b_C)$. 



% the set of  atoms in the partition (of $X$) given by  $\cP^{\cR_n(y)}$, namely the set of all possible $\cP^{\cR_n(y)}$-names 
%for some $x$ with $\cR_n(y)$ being tiled in the way described above, cf.\@ inequality~\eqref{eqn:infaux}). 

%Using this observation, we show that 
%our collection $\cS$ allows for at most $2^{[(h+\delta)|\cR_n(y)| + \log|\cP|\delta]|\cR_n(y)|}$ 
%of the $\cP^{\cR_n(y)}$-names in $\cK_n(y)$. \\


Consider the partition $\cP^{\cR_n(y)}$ (which is finer than each partition $\cP^{\cR_k(b)}$ when $b\in \cR_n(y)$) and define a set of atoms in it which we denote by $K_n(x,y)$.
Namely, for our fixed $x$, we consider the disjoint collection $\mathfrak{S}(x,y)$ of subsets of $\cR_n(y)$, and we put an atom of $\cP^{\cR_n(y)}$ %containing $x$ (namely its $\cP^{\cR_n(y)}$-name)
 in $K_n(x,y)$ provided that it arises as the intersection of elements of the partitions $\cP^{\cR_k(b)}$ which satisfy inequality~\eqref{eqn:infaux} (where we allow any $\cR_k(b)=C \in \mathfrak{S}(x,y)$), with elements in the partition $\cP^{\cG_n(y)}$.
 % where $\cG_n(y):= \cR_n(y) \setminus \bigcup_{C \in \mathfrak{S}} \cR_k(b_C)$. 






Now, for every $\cR_k(b)=C \in \mathfrak{S}(x,y)$,  
%$\cP^{\cR_k(b)}$ is a partition of the 
%space $X$ satisfying the bound on its information function stated in 
inequality \eqref{eqn:infaux} gives a lower bound on the measure of some of the atoms in the partition, and hence an upper bound on their number. It follows that for each such $C$,
there are at most $2^{(h + \delta)|\cR_k(b)|}$ atoms of $\cP^{\cR_k(b)}$ (namely $\cP^{\cR_k(b)}$-names) for which the inequality~\eqref{eqn:infaux}
can hold true. 


Consequently, %for almost every $x$, 
the number of atoms of $\cP^{\cR_n(y)}$
which appear as elements in $K_n(x,y)$ 
%which are in accordance
%with the properties of $\mathfrak{S}$ stated above (depending on $x$ and $y$)
 is bounded by 
\[
\prod_{C \in \mathfrak{S}} 2^{(h+\delta)\,|\cR_k(b)|} \cdot |\cP|^{|\cG_n(y)|} =
\prod_{C \in \mathfrak{S}} 2^{(h + \delta)\,|C|} \cdot 
%\big|\cP\big|^{|\cR_n(y) \setminus \cR_n(y)| +
\abs{\cP}^{ |\cR_n(y) \setminus \bigcup \mathfrak{S}|}.
\]
Since $\cR_n(y)$ is disjointly $(1-\delta)$-covered by the elements in $\mathfrak{S}$, 
we conclude that for a fixed $x \in X$, 
%with the second inequality in~\eqref{eqn:infaux} and with~\eqref{eqn:growth2} 
there are at most 
\[
2^{(h+\delta) \,|\cR_n(y)|} \cdot |\cP|^{\delta\,|\cR_n(y)|}
\]
many elements in $K_n(x,y)$. % which are in accordance with the properties of $\mathfrak{S}(x,y)$.


$K_n(x,y)$ depends on both $x$ and $y$, having been constructed using the collection $\mathfrak{S}(x,y)$ of subsets on $\cR_n(y)$. Now, we define another collection of atoms of $\cP^{\cR_m(y)}$, which we denote by $K_m(y)$. It  consists of all the atoms in the sets $ K_m(x,y)$ as $x$ varies in $X$, provided that $n(x,y)$ satisfies the conditions stated above and in addition $n(x,y)\le m$.  By Lemma~\ref{lemma:comb} (and the choices for $N$ and $N_1$), for all $m\ge m(y)$ the number of possibilities for $\mathfrak{S}(x,y)$ is bounded 
by $2^{\delta|\cR_m(y)|}$. % for some universal constant ${c}>0$. 
Hence, we obtain 
\[
\big| K_m(y) \big| \leq 2^{( h + 2\delta + \delta\,\log|\cP| )\,|\cR_m(y)|}.
\] 
%with the estimate being dependent on $y$, but independent of $x$. 
We now consider the sets
\[
X_m(y) := \Big\{ x \in X\,\big| \, \frac{\cJ\big( \cP^{\cR_m(y)}(x)\big)}{|\cR_m(y)|} > h + 3 \delta + \delta\,\log|\cP|\Big\}.
\]

Consider $\bigcup K_m(y)$, the union of all $\cP^{\cR_m(y)}$-atoms which belong to $K_m(y)$. 
Then,
\[
\lambda\big( X_m(y) \cap \bigcup K_m(y) \big) \leq \abs{K_m(y)}\,2^{-(h + 3\delta + 
\delta\,\log|\cP|)\,|\cR_m(y)|}
\leq 2^{-\delta\,|\cR_m(y)|}.
\]
It follows from the growth condition~\eqref{eqn:growth} stated above that
\[
2^{-\delta\,|\cR_m(y)|} \leq 2^{\log\, m^{-8}} \leq \frac{1}{m^2}
\]
for large enough $m$. This implies that $\sum_{m=1}^{\infty} 2^{-\delta|\cR_m(y)|} < \infty$. 
Thus, for $\nu$-almost-every $y \in Y$, we can apply the Borel-Cantelli lemma and
obtain that for $\lambda \times \nu$-almost every $(x,y)$, 
$x \notin X_m(y) \cap \bigcup K_m(y) $ if $m$ is large enough. On the other hand, we deduce from
Lemma~\ref{lemma:dyncov2} that for large enough $m$ (depending on $x$ and $y$), 
$x \in \bigcup K_m(y)$. This implies that for $\lambda \times \nu$-a.e. $(x,y)$,
and $m$ large enough, we must have $x \notin X_m(y)$, which means by definition of $X_m(y)$ that
\[
\limsup_{m \to \infty} \frac{\cJ\big( \cP^{\cR_m(y)}(x) \big)}{|\cR_m(y)|} \leq h + 3\delta + \delta\,\log|\cP|. 
\]
Letting $\delta \to 0$ yields \eqref{eqn:mainclaim}. This  establishes almost everywhere
convergence, as stated. Since the integrals of $\cJ\big( \cP^{\cR_m(y)}(x)\big)/\abs{\cR_m(y)}$ over $X\times Y$ converge to the cocycle entropy by Theorem~\ref{thm:MAINcocycleentropy}, we have $h = h^{*}_{\cP}(\alpha)$, as claimed.
\end{proof}


We conclude this section with the proof of Corollary~\ref{cor:MAIN_SML1}.
The proof follows along the lines of the $L^1$-convergence case proved  in
\cite[Thm. 4.1]{Li01}.

\begin{proof}[Proof of Corollary~\ref{cor:MAIN_SML1}]
By Theorem~\ref{thm:MAIN_SMBhyp}, the normalized information function converges 
pointwise almost surely (w.r.t.\@ the measure $\lambda \times \nu$) 
to $h^{*}_{\cP} = h^{*}_{\cP}(\alpha)$. Fix $\varepsilon > 0$, and  
for $n \in \NN$ and fixed $y \in Y$, define 
\begin{eqnarray*}
\mathcal{C}_n(y) := \Big\{ x \,\big| \, h^{*}_{\cP} -
\varepsilon \leq \frac{\cJ\big( \cP^{\cR_n(y)}(x) \big)}{\big| \cR_n(y) \big|} 
\leq h^{*}_{\cP} + \varepsilon \Big\}.
\end{eqnarray*}
Integration over $X$ gives 
\begin{eqnarray*}
\big( h^{*}_{\cP} - \varepsilon \big)\,\lambda\big( \cC_n(y) \big) 
&\leq& \int_{\cC_n(y)} \frac{\cJ\big( \cP^{\cR_n(y)}(x) \big)}{|\cR_n(y)|}\,d\lambda(x) \\
&\leq& \int_X \frac{\cJ\big( \cP^{\cR_n(y)}(x) \big)}{|\cR_n(y)|}\,d\lambda(x) \\ 
&\leq& \big( h^{*}_{\cP} + \varepsilon \big)\,\lambda\big( \cC_n(y) \big) + 
\int_{X \setminus \cC_n(y)} \frac{\cJ\big( \cP^{\cR_n(y)}(x) \big)}{|\cR_n(y)|}\,d\lambda(x).
\end{eqnarray*}
We define a new measure $\lambda^{*}$ on $X\setminus \cC_n(y)$ by setting
\[
\lambda^{*}(A) := \frac{\lambda(A)}{\lambda(X \setminus \cC_n(y))}.
\]
Then, since $\cC_n(y)$ is a disjoint union of atoms of the partition $\cP^{\cR_n(y)}$ of $X$, 
\begin{eqnarray*}
\int_{X \setminus \cC_n(y)} \frac{\cJ\big( \cP^{\cR_n(y)}(x) \big)}{|\cR_n(y)|}\,d\lambda(x)
&=& \lambda\big( X \setminus  \cC_n(y) \big) \, \int_{X \setminus \cC_n(y)} \frac{\cJ^{*}\big( \cP^{\cR_n(y)}(x) \big)}
{\big| \cR_n(y) \big|}\,d\lambda^{*}(x) \\
&& \quad - \lambda\big( X \setminus \cC_n(y) \big)\, \log\big( \lambda(X \setminus \cC_n(y)) \big), 
\end{eqnarray*}
where $\cJ^{*}$ denotes the information function with respect to $\lambda^{*}$. 
Now since the integral on the right hand side in the inequality above is just
the Shannon entropy of the partition $\cP^{\cR_n(y)}$ with respect to the 
measure $\lambda^{*}$, we arrive at
\begin{eqnarray*}
\int_{X \setminus \cC_n(y)} \frac{\cJ\big( \cP^{\cR_n(y)}(x) \big)}{|\cR_n(y)|}\,d\lambda(x) \leq 
\lambda\big( X \setminus \cC_n(y) \big)\,\log|\cP| - \lambda\big( X \setminus \cC_n(y) \big)\, 
\log\big( \lambda(X \setminus \cC_n(y)) \big).
\end{eqnarray*}
Clearly, by Theorem~\ref{thm:MAIN_SMBhyp}, for $\nu$-almost every $y\in Y$, 
the latter expression tends to zero as $n \to \infty$. 
Now sending $\varepsilon \to 0$ yields the first assertion of the claimed statement. 
For the second statement note that due to the dominated convergence theorem
(with dominating function $g(y):= 2\,|\cP|$) 
we have 
\begin{eqnarray*}
\lim_{n \to \infty} \int_Y \int_X \Big| \frac{\cJ\big(
\cP^{\cR_n(y)}(x) \big)}{|\cR_n(y)|} - h^{*}_{\cP} \Big|\,d\lambda(x)\,d\nu(y) = 0. 
\end{eqnarray*}
This concludes the proof of the corollary. 
%{\bf The same argument works for $L^1$ convergence in $X$ a.e. $y$  and in $X\times Y$ ???????}
\end{proof}


\section{Amenable relations, injective cocycles and ergodic extensions}\label{sec:ergodicity}


\subsection{Groups admitting injective cocycles}\label{sec:injective} 

As usual, Let $(Y,\nu)$ be a probability space, and  let $\cR\subset Y\times Y$ be a p.m.p.\@ Borel equivalence relation with $\cR$-invariant probability measure $\tilde{\nu}$, such that $\cR=\bigcup_{n\in \NN}\cR_n$ is hyperfinite, or equivalently, $\cR$ is amenable in the sense of \cite{CFW81}.  
Let $(X,\lambda)$ be a p.m.p.\@ action of a countable group $\Gamma$ and let $\alpha : \cR\to  \Gamma$ be a measurable cocycle. Let $\cR^X$ denote the extended relation on $(X\times Y,\lambda\times \nu)$. 

%, we can define a equivalence relation $\cR^\alpha$ on $X\times Y \times X \times Y$ via 
%$(x^\prime,y^\prime)\cR^\alpha(x,y)$ if and only of $x^\prime\cR x $ and $y^\prime=\alpha(x^\prime, x)y$.  The extended relation $\cR^\alpha$ is  p.m.p. with respect to $\lambda\times \mu$.  The relation $\cR^\alpha$ is an extension of the relation $\cR$ via the map $\pi : (x,y)\mapsto x$, and this is a class-bijective extension which is also a measurable factor map, namely $\pi$ maps the measure $\lambda\times \mu$ to $\lambda$.  
%$\cR^\alpha$ is also amenable, with respect to suitable lifts of the F\o lner subset functions on $\cR$ and a suitable extension of the countable generating set of $\cR$ to $\cR^\alpha$.  
%Recall that the cocycle is called weak-mixing, namely 
%that for every p.m.p. ergodic action $(Y,\mu)$ of $\Gamma$,  the extended relation $\cR^\alpha$ is ergodic.
As noted in the proof of Theorem~\ref{thm:MAINcocycleentropy}, given the ingredients just listed, the following limit exists 
\begin{eqnarray*}
h_{\cP}^{*}(\alpha) := \lim_{n \to \infty} \int_Y \frac{h^{\cP}\big( \cR_n \big)(y)}{\big| \cR_n(y)\big|}\, d\nu(y).
\end{eqnarray*}
However, in this generality we cannot say too much about its properties. A meaningful entropy invariant arises when we assume that the hyperfinite exhaustion is by bounded finite relations, and the cocycle is injective. The first condition can always be satisfied, and thus cocycle entropy exists as an invariant with the properties stated 
for all countable groups $\Gamma$ admitting an injective cocycle defined on a hyperfinite relation. 
In particular, cocycle entropy then coincides with Rokhlin entropy and finitary entropy provided that the $\Gamma$-action on $X$ is ergodic and essentially free. It may be the case that every countable group admits an injective cocycle defined on an ergodic hyperfinite relation, but this remains to be seen. 

 To indicate briefly that the class of groups in question is extensive indeed, let us note the following constructions of cocycles on amenable actions.
 
 {\bf 1) Amenable groups.}
%Clearly, the previous construction fails to provide an injective cocycle on an amenable relation when the Poisson boundary is the trivial one point space. Every amenable group admits at least one random walk 
%with trivial Poisson boundary, and for nilpotent groups, all random walks have that property. However, 

Let $\Gamma$ be amenable, let $(Y,\nu)$ be any p.m.p.\@ action of $\Gamma$, and assume that the action is essentially free. For example, we can take $Y$ to the Bernoulli action of $\Gamma$ on $\set{0,1}^\Gamma$. 
  Let $\cR=\cO_\Gamma$ be the orbit relation of $\Gamma$ on $Y$, and then there is a cocycle $\alpha : \cR \to \Gamma$ given by $\alpha(\gamma y,y)=\gamma$. This cocycle is indeed an injective cocycle on a p.m.p.\@ amenable equivalence relation $\cR$, by amenability of $\Gamma$ and freeness of the action. Given a p.m.p.\@ action of $\Gamma$ on a space $(X,\lambda)$, clearly $\Gamma$ acts on the product $(X\times Y, \lambda\times \nu)$, and the orbit relation of $\Gamma$ in the product coincides with the extended relation $\cR^X$ on $X\times Y$ that we have used throughout the paper. Thus $\alpha :\cR\to \Gamma$ is an injective cocycle defined on a p.m.p.\@ amenable relation. 
  
 
 
 {\bf 2) The Maharam extension of the Poisson boundary.}  For every countable infinite group $\Gamma$, and every generating probability measure $\mu$ on $\Gamma$, the Poisson boundary $B=B(\Gamma,\mu)$ is an amenable action of $\Gamma$, in the sense defined by Zimmer \cite{Zi78}, or equivalently, in the sense of \cite{CFW81}. Let $\eta$ denote the stationary measure on $B$, and $r_\eta(\gamma,b)=\frac{d\eta\circ \gamma}{d\eta}(b)$ the Radon-Nikodym derivative cocycle of $\eta$, so that $r_\eta : \Gamma \times B \to \RR^\ast_+$. The Maharam extension of $B$ by the cocycle $r_\eta$ is the $\Gamma$-action on $B\times \RR$ given by 
$\gamma(b,t)=(\gamma b, t-\log r_\eta(\gamma,b))$, and this action preserves the measure $\eta\times \theta$, where $d\theta(t)=e^t dt$ and $dt$ is Lebesgue measure on $\RR$. This action is again an amenable action of $\Gamma$, being an extension of an amenable action.
% {\bf REFERENCE \cite{Zi78} ?????}. 
Let us define $Y=B\times (-\infty,0)$, and let $\nu$ be the restriction of $\eta\times L$ to $Y$, a finite 
%{\bf EXPLANATION WHY FINITE?} 
measure which we normalize to be a probability measure. Then $(Y,\nu)$ is a probability space, and we define the relation $\cR$ on it to be the restriction of the orbit relation $\cO_\Gamma$ defined by $\Gamma$ on $B\times \RR$ to the subset $Y$. Thus $\cR$ is a p.m.p.\@ Borel equivalence relation with countable classes, since $\Gamma$ is countable and preserves the measure $\eta\times L$. The orbit relation $\cO_\Gamma$ is an amenable relation, hence it is hyperfinite, and as a result so is its restriction $\cR$ to the subset $Y$. Finally, if the action of $\Gamma$ on its Poisson boundary $B$ is essentially free, then we can define a cocycle 
$\alpha : \cO_\Gamma \to \Gamma$, by the formula $\alpha(\gamma (b,t),(b,t))=\gamma$. This cocycle is well-defined since the elements in a $\Gamma$-orbit are in bijective correspondence to the elements of $\Gamma$, by essential freeness. Restricting $\alpha$ to $\cR$ we obtain a cocycle from an amenable p.m.p.\@ equivalence relation $\cR$ to $\Gamma$. Furthermore, this cocycle is injective in this case, since 
$\gamma(b,t)=(\gamma b, t-\log r_\eta(\gamma,b))$, so that if $\gamma\neq \gamma^\prime$ then 
$\alpha((b,t), \gamma(b,t))\neq \alpha((b,t), \gamma^\prime(b,t))$. 

Clearly, the class of groups admitting a random walk such that the action on the associated Poisson boundary is essentially free is extensive indeed.  In fact typically many different random walks on a given non-amenable group $\Gamma$ give rise to Poisson boundaries admitting an essentially free action. It is a remarkable feature of the construction of cocycle entropy that it gives one and the same value for the entropy of the $\Gamma$-action on $X$, provided only that this action is ergodic and essentially free, and the value is independent of which cocycle $\alpha : \cR \to \Gamma$ as above was chosen to calculate it. 


  

\subsection{Ergodicity of cocycle extensions}

The Shannon-McMillan-Breiman theorem stated in Theorem \ref{thm:MAIN_SMBhyp}, being a pointwise convergence result for cocycle entropy, requires an additional ergodicity assumption for its validity, beyond those sufficient to guarantee the existence of cocycle entropy itself. We note that an ergodicity assumption also plays a role in the Shannon-McMillan-Breiman theorem for amenable groups, see \cite{Li01}. 

Let us recall that in \cite[Def. 2.2]{BN15a} a notion of weak-mixing for a cocycle $\alpha$ on p.m.p.\@ relation $\cR$ on $(Y,\nu)$ was defined, as follows. A cocycle $\alpha :\cR \to \Gamma$ is weak-mixing if for {\it every} p.m.p.\@ ergodic action of $\Gamma$ on a spaces $(X,\lambda)$, the extended relation $\cR^X$ on $X\times Y$ is ergodic with respect to the product measure $\lambda \times \nu$. In particular, the relation $\cR$ itself must be ergodic. 
This definition is a natural extension of the notion of weak-mixing for group actions, where the action of $\Gamma$ on a space $(B,\eta)$ (with $\eta$ not necessarily invariant), is called weak-mixing if given any p.m.p.\@ ergodic action of $\Gamma$ on a space $(X,\lambda)$, the product action of $\Gamma$ on $(X\times B, \lambda\times\eta)$ is still ergodic.  
%We refer to \cite{BN13b} and \cite{BN15a} for a detailed discussion of the properties of the notion of weak-mixing cocycle and its variants, and here will only comment briefly on some examples. 


{\bf 1) Amenable groups.} Let us first consider the case where $\Gamma$ is amenable. Referring to the relation $\cR$ and the cocycle $\alpha$ defined in \S \ref{sec:injective}(1), if the $\Gamma$-action on $(Y,\nu)$ is weak-mixing (in the usual sense for group actions), then the $\Gamma$-action on $(X\times Y,\lambda\times\nu)$ is ergodic for {\it every}  ergodic p.m.p.\@ action of $\Gamma$. 
Thus, referring to \S \ref{sec:injective}(1), the cocycle $\alpha : \cR \to \Gamma$ defined there is a weak-mixing cocycle and so here the extended relation $\cR^X$ is ergodic.  



{\bf 2) Poisson boundaries. } When $\Gamma$ is non-amenable, the most important source (but not always the only one) of weak-mixing actions of a countable group $\Gamma$ is the set of its actions on Poisson boundaries $B=B(\Gamma,\mu)$ \cite{AL05}. These actions are amenable as noted above, and satisfy a stronger condition than weak-mixing, namely double ergodicity with coefficients, see \cite{Ka03}. If $\Gamma$ is a non-amenable group, then the ergodic action on a Poisson boundary $(B,\eta)$ is not measure-preserving, and thus of  type $III$. The action on the Maharam extension $(B\times \RR,\eta\times L)$ is in fact measure-preserving, on a $\sigma$-finite (but not finite) measure space, but the Maharam extension is not necessarily an ergodic action of $\Gamma$.  The possibilities for it are determined by the type of the $\Gamma$-action on $(B,\eta)$, and in particular, if it is $III_1$, then the Maharam extension is ergodic. In general, this does not imply that for any ergodic action of $\Gamma$ on $(X,\lambda)$, the Maharam extension of $(X\times B,\lambda \times\eta)$ is ergodic.  If indeed this is the case for {\it every} p.m.p.\@ action of $\Gamma$, then the action of $\Gamma$ on $(B,\eta)$ is defined in \cite{BN13b} to have stable type $III_1$. 

Assume that the $\Gamma$-action on $(B,\eta)$ is essentially free, and let $(Y,\nu)$, $\cR$ and $\alpha$ be as defined  \S \ref{sec:injective}(2). Then the cocycle $\alpha :\cR\to \Gamma$ is injective, defined on a p.m.p.\@ amenable relation, and if the type of the $\Gamma$-action on $(B,\eta)$ is $III_1$ it is ergodic. If, furthermore, the $\Gamma$-action on $(B,\eta)$ has stable type $III_1$, then the cocycle $\alpha$ is weak-mixing, and hence $\cR^X$ is ergodic for every p.m.p.\@ action of $\Gamma$. Thus all the assumptions in the Shannon-McMillan-Breiman theorem are verified in this case, for which examples will be provided below. 


{\bf 3) Non-trivial stable type.} If the type of the $\Gamma$-action on $(B,\eta)$ is $III_\tau$ for some $\tau > 0$ then the action of $\Gamma$ on the Maharam extension has a set of ergodic components admitting a free transitive action of  the circle group $\RR/\ZZ \cdot\log \tau $, which acts on the Maharam extension and which commutes with the $\Gamma$-action. If this is also the situation for the Maharam extension of all the spaces $(X\times B, \lambda\times \eta)$ for {\it every} ergodic p.m.p.\@ action of $\Gamma$, then the $\Gamma$-action on $(B,\eta)$ is defined in \cite{BN13b} to have stable type $\tau$. In that case, it is also possible to prove a version of the Shannon-McMillan-Breiman pointwise convergence theorem, which applies, rather than to the information functions we defined, to a further  average of them. In particular, this provides a proof of the Shannon-McMillan mean-convergence theorem in our context.  In the interest of brevity, however, we shall provide the details elsewhere. We refer to \cite{BN13b} and \cite{BN15a} for to a detailed discussion of type, stable type and Maharam extensions in the context of pointwise ergodic theorems for group actions. 


{\bf 4) Ergodicity and mixing conditions.} Given an injective cocycle $\alpha : \cR \to \Gamma$ on an ergodic p.m.p.\@ amenable relation and a p.m.p.\@ action of $\Gamma$ on $(X,\lambda)$, it is possible to develop criteria to show that the extended relation $\cR^X$ is ergodic,   provided that the $\Gamma$-action on $X$ satisfies additional ergodicity or mixing conditions. This implies that the Shannon-McMillan-Breiman theorem is valid for $\Gamma$-actions on a suitable class of p.m.p.\@ actions on $(X,\lambda)$. 
A simple example of this phenomenon arises for the p.m.p.\@ actions of  finitely generated non-abelian free groups $\FF_r$. 
Here, taking the boundary $(\partial \FF_r, \nu)$ with the uniform measure $\nu$, we construct a cocycle $\alpha : \cR \to \FF_r$, where $\cR$ is an amenable p.m.p.\@ relation on $\partial \FF_r$, which is not ergodic, but in fact has exactly two ergodic components. If $(X,\lambda)$ is a p.m.p.\@ ergodic action of $\FF_r$ for which the index $2$ subgroup of even length words is ergodic, then the extended relation $\cR^X$ is an ergodic relation, see \cite{BN13a} for a detailed exposition. Hence these actions of $\FF_r$ satisfy the 
Shannon-McMillan-Breiman theorem.  We will give a detailed exposition of this case, which will also demonstrate the geometric significance of the theorem, in the next section.  

%$\alpha  If the $\Gamma$-action on $(B,\eta)$ has type $III_\tau$ for some $\tau \neq 0$, then for every p.m.p. action of $\Gamma$ on $(X,\lambda)$, the extended relation $\cR^X$ is ergodic, provided that the action of $\Gamma$ on $(X,\lambda)$ is a mildly-mixing action. {\bf REFERENCE : \cite{BNxx} ????? Aaronson Lemanczyk}
%


To conclude this section let us  note the following results concerning type and stable type, which are relevant to the foregoing discussion. 
\begin{Examples}
\begin{enumerate}

%\item Let $\Gamma$ denote a countable group, let $\mu$ be a generating probability measure of finite entropy on $\Gamma$, and let $\nu$ denote the $\mu$-stationary probability 
%measure on the standard Borel $\Gamma$-space $(\cP(\Gamma,\mu), \nu)$ which is the Poisson boundary of $P(\Gamma, \mu)$. This action is non-singular and amenable, although not necessarily essentially free. It is  
%also ergodic and in fact weakly mixing  
%namely the product action of $\Gamma$ on $(\times X, \nu\times \mu)$ is $\Gamma$-ergodic, whenever $(X,\mu)$ is a p.m.p. ergodic action of $\Gamma$,
 %by double ergodicity, see........
 
\item Let $\Gamma$ be an irreducible  lattice in a connected semisimple Lie group with finite center and without compact factors. Then the action of $\Gamma$ on the maximal boundary $(G/P,m)$, where $P$ is a minimal parabolic subgroup and $m$ is the Lebesgue measure class, is amenable and has {\it stable type} $III_1$, see \cite{BN13b}. 
\item Let $\Gamma$ be a discrete non-elementary subgroup of isometries of real hyperbolic space. By \cite{Su78} \cite{Su82}, the type of the action of $\Gamma$ on the boundary of hyperbolic space with respect to the Lebesgue measure class is $III_1$. In \cite{Sp87} this result was proved for the action of fundamental groups of compact connected negatively curved manifolds acting on the visual boundary with the manifold measure class. 

\item Let $\Gamma$ be a finitely generated free group. By \cite{RR97}, the action of $\Gamma$ has non-trivial type with respect to harmonic measures, namely stationary measures of suitable random walks. For  $\Gamma$ being a word hyperbolic group, it was proved in \cite{INO08} that the action of $\Gamma$ on the Poisson boundary associated with a generating measure of finite support has non-trivial type. 
\item In \cite{Bo14} it was proved that for word hyperbolic groups (with an additional technical condition) the action on the Gromov boundary with respect to a quasi-conformal measure (and in particular the Patterson-Sullivan measure) measure has non-trivial {\it stable type}.\end{enumerate}
\end{Examples}
 

 
 
% The key to passing from a non-singular amenable weak-mixing action $(B,\nu)$ of $\Gamma$ to a weak-mixing $\Gamma$-valued cocycle defined on an ergodic amenable p.m.p. equivalence relation is the construction of the Maharam extension, and the key assumption regarding it is that the Radon-Nikodym cocycle of the $\Gamma$-action on $(B,\nu)$ has non-zero stable type. See............
%
%
%
%
%One sufficient condition in order
%to guarantee ergodicity of $\cR^X$ is {\em weak mixing} of the relation
%$\cR$ over $(Y,\nu)$. A sufficient condition for that is mild mixing of the $\Gamma$-action on $X$ {\bf ?????????? SOURCE/EXPLAIN, refer to the Markov groups paper}. 
%
%{\bf Convergence of information functions : Shannon-McMillan-Breiman Theorem}
% Now let $\cR$ be a p.m.p. ergodic amenable equivalence relation on $X$ with invariant measure $\lambda$, let $\alpha :\cR\to \Gamma$ be a weak-mixing cocycle, and let $(Y,\mu)$ be an ergodic p.m.p. action of $\Gamma$. 
%Consider the extended relation $\cR^\alpha$ on $X\times Y$ with invariant measure $\lambda\times \mu$, and let $\cF^\alpha_n$ be suitable lifts to $\cR^\alpha$ of a sequence of subset function $\cF_n$ for the relation $\cR$. {\bf to do : under suitable assumptions, when $\cF_n$ are admissible, it follows that $\cF_n^\alpha$ are also admissible}.  
%
%Fix any partition $P=\{A_1,\dots, A_k\}$ of $Y$, and any finite set $S\subset   [x]\subset X$. Choosing $x_0\in [x]$, for an element $x^\prime \in S$ we can consider the group elements $\alpha(x^\prime, x_0)\in \Gamma$. Now consider the refined partition $P(S)=\bigvee_{x^\prime\in S)}\alpha(x^\prime, x_0)^{-1}P$ of $Y$, and the entropy $h(P, S)$ of the refined partition. Note that the entropy is independent of the choice of $x_0\in [x]$. Indeed if $x_0^\prime$ is another such choice, then $\alpha(x^\prime, x_0)=\alpha(x^\prime, x_0^\prime)\alpha(x_0^\prime, x_0)$, so that the group elements obtained differ from the previous ones by multiplication by the fixed  element $\alpha(x_0^\prime, x_0)^{-1}$ and hence the entropy of the refined partition they generate is the same as before. We can now state the following 
%generalization of the Shannon-McMillan-Breiman Theorem. 
% 
% \begin{Theorem}\label{SMB}
%Under the assumptions stated above, for any partition $P$ of $Y$, the function $h_P$ on $2_{{fin}}(X)$ is subadditive. 
% Hence $h(P, \mathcal{F}_n((x))/|\mathcal{F}_n(x)|$ converges to a limit which we denote by $h(P)( [x] )$ for almost every $x \in X$. The limit is an $\cR$-invariant function, and under the assumption of ergodicity of $\cR$, the limit $h(P)$ is in fact almost surely constant. 
% \end{Theorem}
%\begin{proof}  Based on Theorem \ref{thm:subadditive}. 
%\end{proof} 
%
%\subsection{Shannon-McMillan-Breiman theorem for amenable groups}

%Our results include the generic case of amenable measure preserving 
%equivalence relations $\cR$
%%constructed from the $\Gamma$-action on the Poisson boundary $(\partial\Gamma, \nu_{\partial\Gamma})$ 
%of the group
%(SOURCE:?:?:?). Precisely, if $\Gamma \curvearrowright (\partial\Gamma, \nu_{\partial\Gamma})$
%is essentially free (uniformly bounded stabilizers!?!??), weakly mixing and of stable type 
%$III_{\lambda}$ for $\lambda \in (0,1]$ (::::SOURCES:::), one can construct
%ergodic, amenable measure preserving equivalence relations, cf.\@ \cite{BN13b}. 
%For a more detailed exposition 
%about Poisson boundaries and amenable equivalence relations,
%we refer the interested reader to e.g.\@ SOURCE:::.\\ 
%Recall from above that for a cocycle $\alpha:\cR \to \Gamma$, one can 
%naturally define the extended equivalence relation $\cR^X$
%on $\big( X \times Y, \lambda \otimes \nu \big)$ 
%(which is amenable and measure preserving). We show in 
%Theorem~\ref{thm:MAIN_SMBhyp} that if $\cR^X$ is ergodic, then 
%the information function converges pointwise 
%almost surely if averaged by pointwise approximate hyperfinite exhaustions. 
%The $L^1$-convergence of the information function is a consequence,
%cf.\@ Corollary~\ref{cor:MAIN_SML1}. 
%We point out here that this is a Shannon-McMillan-Breiman theorem for a large class
%of
%groups $\Gamma$ and all arising ergodic p.m.p.\@ actions $\Gamma \curvearrowright X$.  
%In particular, to the know\-ledge of the authors, 
%it is the first result of its kind for non-amenable groups. Hence, from the
%geometric point of view, the present result significantly supersedes the
%convergence theorems for amenable groups, cf.\@ \cite{Ki75, OW83, Li01}.
%The great generality comes with a price to pay.
%For most groups $\Gamma$, though being aware of their existence, 
%we do not know how to construct hyperfinite 
%exhaustions explicitly. This is different in the 
%classical setting involving actions of amenable groups. There, as also done
%in the above mentioned references, 
%one can work with 
%tempered F{\o}lner sequences which can be concretely described in many settings.
%However, there are large classes of non-amenable groups for which the picture is rather clear, i.e.\@
%for negatively curved countable groups. 
%In order to illustrate our result, we will make everything explicit for the free
%group in the next Section~\ref{sec:freegroup}.
%There, we give concrete examples for approximate hyperfinite 
%exhaustions arising from the action of the free group on its Gromov boundary. \\ 
%


\section{The free group} \label{sec:freegroup}



%In the present section we will compute the entropy associated with measure-preserving actions of free non-commutative groups. 
\subsection{The boundary of the free group}
We briefly describe the ingredients we need for our analysis, following the exposition of \cite{BN13a}. 

Let $\FF=\langle a_1,\dots ,a_r \rangle$ be the free group of rank $r\ge 2$, with $S=\{a_i^{\pm 1}:~1\le i \le r\}$ a free set of generators. The (unique) {\em reduced form} of an element $g\in \FF$ is the expression  $g=s_1\cdots s_n$ with $s_i \in S$ and $s_{i+1}\ne s_i^{-1}$ for all $i$. Define $|g|:=n$, the length of the reduced form of $g$. The distance function on $\FF$ is defined by $d(g_1,g_2):=|g_1^{-1}g_2|$. The Cayley graph associated with the generating set $S$ is a regular tree of valency $2r$, and $d$ coincides with its edge-path distance. 


The boundary of $\FF$ is the set of all sequences $\xi=(\xi_1,\xi_2,\ldots) \in S^\NN$ such that $\xi_{i+1} \ne \xi_i^{-1}$ for all $i\ge 1$. We denote it by $\partial \FF$. A metric $d_\partial$ on $\partial \FF$ is defined by $d_\partial\big( (\xi_1,\xi_2, \ldots), (t_1, t_2, \ldots) \big) = \frac{1}{n}$ where $n$ is the largest  natural number such that $\xi_i=t_i$ for all $i < n$. If $\{g_i\}_{i=1}^\infty$ is any sequence of elements in $\FF$ and $g_i:=t_{i,1}\cdots t_{i,n_i}$ is the reduced form of $g_i$ then $\lim_i g_i = (\xi_1,\xi_2,\ldots) \in \partial \FF$ if $t_{i,j}$ is eventually equal to $\xi_j$ for all $j$. If $\xi \in \partial \FF$ then we will write $\xi_i \in S$ for $i$-th element in the sequence  $\xi=(\xi_1,\xi_2,\xi_3,\ldots)$.


We define a probability measure $\nu$ on $\partial \FF$, by the requirement that every finite sequence $t_1,\ldots, t_n$ with $t_{i+1} \ne t_i^{-1}$ for $1\le i <n$, the following holds :
$$\nu\Big(\big\{ (\xi_1,\xi_2,\ldots) \in \partial \FF :~ \xi_i=t_i ~ 1\le i \le n\big\}\Big) := |S_n(e)|^{-1}=(2r-1)^{-n+1}(2r)^{-1}.$$
%By the Carath\'eodory extension theorem, this uniquely defines a Borel probability measure $\nu$ on $\partial \FF$.


There is a natural action of $\FF$ on $\partial \FF$ by 
$$(t_1\cdots t_n)\xi := (t_1,\ldots,t_{n-k},\xi_{k+1},\xi_{k+2}, \ldots)$$ where $t_1,\ldots, t_n \in S$,  $t_1\cdots t_n$ is in reduced form and $k$ is the largest number $\le n$ such that $\xi_i^{-1} = t_{n+1-i}$ for all $i\le k$.  Observe that if $g=t_1 \cdots t_n$ then the Radon-Nikodym derivative satisfies
$$\frac{d\nu \circ g}{d\nu}(\xi) = (2r-1)^{2k-n}.$$

%\subsection{The horospherical and synchronous tail relations}
\subsection{The horospherical relation and the fundamental cocycle}\label{sec:free group review}
Let $\cR$ be the equivalence relation on $\partial \FF$ given by $(\xi,\eta) \in \cR$ if and only when writing $\xi=(\xi_1,\xi_2,\ldots)$ and $\eta=(\eta_1,\eta_2,\ldots)$, there exists $n$ such that $\eta_i = \xi_i$ for all $i> n$. Thus $\eta\cR \xi$ if and only if $\eta$ and $\xi$ have the same synchronous tail, if and only if they differ by finitely many coordinates only. 


 Let $\cR_n$ be the equivalence relation given by $(\xi,\eta)\in \cR_n$ if and only if $\xi_i=\eta_i~\forall i > n$. Then $\cR$ is the increasing union of the finite subequivalence relations $\cR_n$. Thus $\cR$ is a hyperfinite relation.

Consider the relation $\cR^\prime$ on $\partial \FF$  such that $\eta\cR^\prime \xi$  if and only if there is a $g \in \FF$ such that $g\xi = \eta$ and $\frac{d\nu \circ g}{d\nu}(\xi) = 1$.  Note that the level set of the Radon-Nikodym derivative, namely $\set{g\in \FF\,;\, \frac{d\nu \circ g}{d\nu} (\xi) = 1}$ is the horosphere in the Cayley tree based 
at $\xi$ and passing through the identity in $\FF$. 
%Thus it is natural to call $\cR^\prime$ the horospherical equivalence relation : $\eta\cR^\prime \xi$ if and only if $\eta =g\xi$, where $g^{-1}$ belongs to the horosphere based at $\xi$ passing through the identity. 
 Note that in that case $\xi=g^{-1}\eta$, and $\frac{d\nu\circ g^{-1}}{d\nu}(\eta)=\left(\frac{d\nu\circ g}{d\nu}(\xi)\right)^{-1}=1$, so that the relation is indeed symmetric. The transitivity of the horospherical relation $\cR^\prime$ follows from the cocycle identity which the Radon-Nikodym derivative satisfies. 
Thus $\cR^\prime$ is an equivalence relation, and by definition, the measure $\nu$ is $\cR^\prime$-invariant. 

The relation $\mathcal{R}^{\prime}$ on $\partial\mathbb{F}_{r}$ can also be defined more concretely by the condition that $\left(\xi,\eta\right)\in\mathcal{R}^{\prime}$ iff there
exists $k$ s.t. $\eta=g\xi$ and $g=\eta_{1}\cdot...\cdot\eta_{k}\cdot\xi_{k}^{-1}\cdot...\cdot\xi_{1}^{-1}$.
It follows that $\eta=g\xi$ has the same synchronous tail as $\xi$
from the $k+1$-th letter onwards. Equivalently, $g^{-1}$ belongs
to the horosphere based at $\xi$ and passing through the identity
in $\mathbb{F}_{r}$, namely the geodesic from $g^{-1}$ to $\xi$
and the geodesic from $e$ to $\xi$ meet at a point which is equidistant
from $e$ and $g^{-1}$. Thus it is natural to call $\mathcal{R}^{\prime}$ the \emph{horospherical
relation}
%: $\left(\xi,\eta\right)\in\mathcal{R}^{\prime}$ iff $\eta=g\xi,$
%where $g^{-1}$ belongs to the horosphere based at $\xi$ passing
%through the identity. Similarly, it is natural to call 
and the equivalence
class of $\xi$ under $\mathcal{R}_{n}$ the horospherical ball of
radius $n$ based at $\xi$. Since $\xi$ and $\eta=g\xi$ have the
same synchronous tail, $\mathcal{R}^{\prime}$ coincides with the
synchronous tail relation $\mathcal{R}$. 

The {\it fundamental cocycle} of the tail relation is the measurable map 
$\alpha : \cR \to \FF_r$ given, for $\eta=(\eta_1,\dots,\eta_k,\dots)$ 
and $\xi=(\xi_1,\dots,\xi_k,\dots)$ (with $(\eta,\xi)\in \cR_k$), by  
$$\alpha(\eta,\xi)=\eta_{1}\cdot...\cdot\eta_{k}\cdot\xi_{k}^{-1}\cdot...\cdot\xi_{1}^{-1}$$
so that $\alpha(\eta,\xi)\xi=\eta$. The cocycle takes values in the subgroup $\FF_r^{e}$ consisting of words of even length, and more precisely for each $k$ the set of values $\alpha(\eta, \xi)$ for fixed $\eta$ and $\xi\in\cR_k(\eta)$ coincides with the intersection of the word metric ball $B_{2k}(e)$ with the horoball  based at $\eta$ and passing through $e$. This set is called the horospherical ball of radius $2k$ determined by $\eta$ and denoted by $B_{2k}^\eta$. 

Note further that the set of cocycle values $\alpha(\xi, \eta)$ for $\xi\cR_k \eta$ with $\eta$ fixed, namely the set $\left(B_{2k}^\eta\right)^{-1}$, is given by 
$$\alpha(\xi,\eta)=\xi_{1}\cdot...\cdot\xi_{k}\cdot \eta^{-1}_{k}\cdot...\cdot\eta^{-1}_{1}$$
and this set contains words of length at most $2k$ whose first $(k-1)$ letters can be specified arbitrarily. 

Finally, consider the set of values of the cocycles $\alpha(\eta,\xi)$ for all points $\eta\neq \xi$ which are $\cR_k$ equivalent to another, i.e.\@ as we go over all pairs of distinct points in some equivalence class of the form 
$\cR_k(\zeta)$. This set of values clearly contains all the even words in a ball of radius $2k-2$, i.e. 
$B_{2k-2}(e)\cap\FF_r^e$.  


The {\it finite order automorphisms} of $\cR$ is the subgroup $\Phi$ of $\cup_{n\in \NN} [\cR_n]$ generated by the transformations defined as follows.  Let $\pi_{n}:\partial\mathbb{F}_{r}\to S^{n}$ be the projection given by  $\pi_{n}\left(s_{1},s_{2},...\right)=\left(s_{1},s_{2},...,s_{n}\right)$.
We say that a map $\psi:\partial\mathbb{F}_{r}\to\partial\mathbb{F}_{r}$
has \emph{order $n$} if $\psi\left(\xi\right)=\psi\left(\xi^{\prime}\right)$
for any two boundary points $\xi,\xi^{\prime}\in\partial\mathbb{F}_{r}$
with $\pi_{n}\left(\xi\right)=\pi_{n}\left(\xi^{\prime}\right)$. 


For any $\left(\xi,\xi^{\prime}\right)\in\mathcal{R}$ there exists
a map $\phi\in \Phi$ such that $\phi\left(\xi\right)=\xi^{\prime}$
and $\phi$ has order $n$ for some $n\in\mathbb{N}$, see \cite{BN13a}. 
Thus the group of finite order automorphisms clearly generates the synchronous tail (i.e.\@ the horospherical) relation.
%If $\left(\xi,\xi^{\prime}\right)\in\mathcal{R}$ and $\xi=(\xi_{1},\xi_{2},...)$,
%$\xi^{\prime}=\left(\xi_{1}^{\prime},\xi_{2}^{\prime},...\right)$
%then by definition of $\mathcal{R}$, there is an $n$ such that $\xi_{i}=\xi_{i}^{\prime}\;\forall\, i\geq n$
%. Let $m>n$ and let $\beta:S^{m}\to S^{m}$ be a bijection such that:
%
%1. $\beta\left(\pi_{m}\left(\xi\right)\right)=\pi_{m}\left(\xi^{\prime}\right)$
%
%2. if $\beta\left(t_{1},t_{2},...t_{m}\right)=\left(s_{1},s_{2},...,s_{m}\right)$
%then $t_{m}=s_{m}$ 
%
%3. $\beta|_{\mathcal{Z}}:\mathcal{Z}\to\mathcal{Z}$ where $\mathcal{Z}=\left\{ \left(s_{1},s_{2},...,s_{m}\right)\::\: s_{i+1}\neq s_{i}^{-1}\;\mbox{for all \ensuremath{1\leq i\leq m-1}}\right\} $. 
%
%Define $\phi\in Inn\left(\mathcal{R}\left(\mathbb{F}_{r}\right)\right)$
%by $\phi\left(t_{1},t_{2},...,\right)=\left(s_{1},s_{2},...\right)$
%where $\beta\left(t_{1},t_{2},...t_{m}\right)=\left(s_{1},s_{2},...,s_{m}\right)$
%and $t_{i}=s_{i}$ for all $i>m$. Since $\beta$ is a bijection,
%$\phi$ is invertible, $\left(\phi\left(\xi\right),\xi\right)\in\mathcal{R}$
%

{\it  The extended horospherical relation.} %{\bf NOTE : the product here in this argument is reversed !!!!!!!!!!!!!!!}
Let $\FF$ act by measure-preserving transformations on a probability space $(X,\lambda)$. Let $\cR_n^X$ be the equivalence relation on $ X\times \partial \FF$ defined by $((x,\xi),(x^\prime,\xi^\prime)) \in \cR_n^X$ if and only if there exists $g\in \FF$ with $(gx,g\xi)=(x^\prime,\xi^\prime)$ and $(\xi,\xi') \in \cR_n$ (i.e., if $\xi=(\xi_1,\ldots)\in S^\NN$ and $\xi'=(\xi'_1,\ldots)\in S^\NN$ then $\xi_i=\xi'_i$ for all $i\ge n$).

Inspecting the definitions, we see that the extended horospherical relation on $X\times \partial \FF_r$ coincides with the extension of the horospherical (i.e.\@ synchronous tail) relation $\cR$ on $\partial \FF_r$ via the fundamental cocycle $\alpha : \cR \to \FF_r$ defined above.   


It is easy to see that for any $(y,\xi) \in  X\times \partial \FF$, 
$$\abs{\cR^X_n(y,\xi))} = \abs{\cR_n(\xi)} = (2r-1)^n.$$
%So
%\begin{eqnarray*}
%| \cR^X_n(x,\xi) \setminus \cR^X_{n-1}(x,\xi)| &=& | \cR_n(\xi) \setminus \cR_{n-1}(\xi)|= (2r-1)^{n-1}(2r-2)\\
%  &=& \frac{2r-2}{2r-1} | \cR_n(\xi)| = \frac{2r-2}{2r-1} | \cR^X_n(x,\xi)|.
%  \end{eqnarray*}
%%  \end{lem}
The relation $\cR^X=\bigcup_{n \ge 1} \cR^X_n$ is thus a hyperfinite measurable equivalence relation, it preserves the measure $\nu\times \lambda$, and it is {\it uniform}, namely for each $n \ge 1$ almost every equivalence class of the relation $\cR_n^X$ has the same cardinality.   



\subsection{Shannon-McMillan-Breiman theorem for the free groups}

Let $\mathcal{R}^{X}$ be the equivalence relation on $X\times \partial \FF_r$.
We may assume the action of $\mathbb{F}_{r}$ on $\left(X,\lambda\right)$
is ergodic, and we will use the following. 
\begin{Theorem}\cite{BN13a}\label{F^2-ergodic}. If $\mathbb{F}_{r}^{e}$ acts on $(X,\lambda)$
ergodically, then the diagonal action $\mathbb{F}_{r}^{e}\curvearrowright X\times \partial\mathbb{F}_{r}$
is ergodic. 
\end{Theorem}



\noindent{\bf The type of the boundary action.}
The type of the action $\FF_r\curvearrowright  (\partial \FF_r, \nu)$ is $III_\tau$ where $\tau=(2r-1)^{-1}$. It is follows from  \cite[Theorem 4.1]{BN13a} that the stable type of $\FF_r\curvearrowright  (\partial \FF_r, \nu)$ is $III_{\tau^2}$. In fact, if $\FF_r^e$ denotes the index 2 subgroup of $\FF$ consisting of all elements of even word length then $\FF_r^e\curvearrowright  (\partial \FF_r,\nu)$ is  of type $III_{\tau^2}$ and stable type $III_{\tau^2}$. It is also weakly mixing. Indeed, $(\partial \FF_r,\nu)$ is naturally identified with $B(\FF_r,\mu)$, the Poisson boundary of the random walk generated by the measure $\mu$ that is distributed uniformly on $S$.  By \cite{AL05}, the action of any countable group $\Gamma$ on the Poisson boundary $B(\Gamma,\kappa)$ is weakly mixing whenever the measure $\kappa$ is adapted. This shows that $\FF_r\curvearrowright  (\partial \FF_r,\nu)$ is weakly mixing. Moreover, if we denote $S^2 = \{st:~s, t \in S\}$, then $(\partial \FF_r,\nu)$ is naturally identified with the Poisson boundary $B(\FF_r^e, \mu_2)$ where $\mu_2$ is the uniform probability measure on $S^2$. So the action $\FF_r^e \curvearrowright  (\partial \FF,\nu)$ is also weakly mixing.

%Rather than form the Maharam extension of the action $\FF \cc (\partial \FF,\nu)$ it is more convenient and fruitful to consider the Maharam extension of the action $\FF_2 \cc (\partial \FF,\nu)$. 

Let $\FF_r$ act on $\partial \FF_r \times \ZZ$ by $g(b,t) = (gb, t-\log r_{\nu}(g,b))$, which gives the discrete Maharam extension in this case. 
%Although it is possible to use the previous Theorem to obtain an equivalence relation on $\partial \FF \times \{0,1\}$ and a cocycle, it is more fruitful to consider a slightly different construction. 
Let $\cR$ be the orbit-equivalence relation restricted to $\partial \FF_r \times \{0\}$, which we may, for convenience, identify with $\partial \FF_r$. In other words, $b \cR b'$ if and only if there is an element $g \in \FF_r$ such that $gb=b'$ and $\frac{d\nu \circ g}{d\nu}(b)=1$. As noted above, this is the same as the (synchronous) tail-equivalence relation on $\FF_r$. In other words, two elements $\xi=(\xi_1,\xi_2,\ldots)$, $\eta=(\eta_1,\eta_2,\ldots) \in \partial \FF_r$ are $\cR$-equivalent if and only if there is an $m$ such that $\xi_n=\eta_n$ for all $n \ge m$. But now note that if $b\cR (gb)$, then necessarily $g \in \FF_r^e$. So $\cR$ can also be regarded as the orbit-equivalence relation for the $\FF_r^e$-action on $\partial \FF_r \times \ZZ$ restricted to $\partial \FF_r \times \{0\}$. 

Let $\alpha:\cR \to \FF_r^e$ be the cocycle $\alpha(gb,b)=g$ for $g\in \FF_r^e, b \in \partial \FF_r$. This is well-defined almost everywhere because the action of $\FF_r^e$ is essentially free. Because $\FF_r^e \curvearrowright  (\partial \FF_r,\nu)$ has type $III_{\tau^2}$ and stable type $III_{\tau^2}$, this cocycle is weakly mixing for $\FF_r^e$. In other words, if $\FF_r^e \curvearrowright (X,\mu)$ is any ergodic p.m.p action, then the equivalence relation $\cR^X$ defined on $X\times \partial \FF_r $ by the cocycle extension is ergodic. Since the relation is hyperfinite and the cocycle is injective, we conclude that the action of $\FF_r$ on any ergodic p.m.p.\@ space satisfies the Shannon-McMillan-Breiman theorem, provided only that $\FF_r^e$ acts ergodically on $X$. 




\begin{thebibliography}{1000000}

\bibitem[AL05]{AL05} Aaronson, J. and Lemanczyk, M. \textit{Exactness of Rokhlin endomorphisms and weak mixing of Poisson boundaries. Algebraic and topological dynamics}. Contemp. Math. 
{\bf 385}, Amer. Math. Soc., Providence, RI (2005), pp. 77--87.

\bibitem[ADR]{ADR} Anantharaman-Delaroche, C. and Renault, J. \textit{Amenable groupoids}. L'Enseignement Math\'ematique, vol. {\bf 36}, 2000. 


\bibitem[Av10]{Av10} Avni, N., \emph{Entropy {T}heory for {C}ross {S}ections}, 
Geom. Func. Ana., vol. {\bf 19} (2010), pp. 1515-1538.

\bibitem[Br57]{Br57} Breiman, L., \emph{The individual ergodic theorem
of information theory}, Ann. Math. Stat. \textbf{28} (1957), pp. 809--911. Correction,
ibid. 31 (1960),  pp. 809--910.


\bibitem[Bo10a]{Bo10a} Bowen, L. \textit{Invariant measures on the space of horofunctions of a word-hyperbolic group}.  Ergodic Theory and Dynamical Systems, \textbf{30}, no. 1 (2010),
pp. 97--129. 


\bibitem[Bo10b]{Bo10b} Bowen, L. \textit{The ergodic theory of free group actions: entropy and the f-invariant}. Groups Geom. Dyn. \textbf{4}, no. 3 (2010), pp. 419-432.



\bibitem[Bo10c]{Bo10c} Bowen, L. \textit{Measure conjugacy invariants for actions of countable sofic groups}. J. Amer. Math. Soc. \textbf{23} (2010), 
pp. 217--245.



\bibitem[Bo12]{Bo12} Bowen, L. \textit{Sofic entropy and amenable groups}. Ergodic Theory Dynam. Systems \textbf{32}, no. 2 (2012), pp. 427--466. 

\bibitem[Bo14]{Bo14} Bowen, L. \textit{The type and stable type of the boundary of a Gromov hyperbolic group}. Geometriae Dedicata \textbf{172} (2014),
pp. 363--386.


\bibitem[BN13a]{BN13a} Bowen, L. and Nevo, A. \textit{Geometric covering arguments and ergodic theorems for free groups}.  L'Enseignement Math\'ematique, \textbf{59} (2013), pp. 133--164.

\bibitem[BN13b]{BN13b} Bowen, L. and Nevo, A. \textit{Pointwise ergodic theorems beyond amenable groups}. Ergod. Th. and Dynam. Sys. \textbf{33} (2013), pp. 777--820.

\bibitem[BN15a]{BN15a} Bowen, L. and Nevo, A. \textit{Amenable equivalence relations and the construction of ergodic averages for group actions}. 
To appear in Journal d'Analyse Math\'ematique,(2015).

\bibitem[BN15b]{BN15b} Bowen, L. and Nevo, A. \textit{von Neumann and Birkhoff ergodic theorems for negatively curved groups}.
Ann. Sci. Ec. Norm. Sup{\'e}r. \textbf{48} (2015), pp. 1113--1147.

%\bibitem[BNxx]{BNxx} Bowen, L. and Nevo, A. \textit{The ergodic theory of Markov groups}. In preparation. 

\bibitem[CCK14]{CCK14} Ceccherini-Silberstein, T., Coornaert, M. and Krieger, F. 
\textit{An analogue of Fekete's lemma for subadditive functions on cancellative amenable semigroups}. J. Ana. Math. \textbf{124} (2014), pp. 59--81.


\bibitem[CFS]{CFS} Cornfeld, I. P., Fomin, S. V., and Sinai, Ya. G.,\textit{Ergodic Theory}. A Series of Comprehensive Studies in Mathematics, vol. 245, Springer Verlag, 1982. 


\bibitem[CFW81]{CFW81} Connes, A., Feldman, J. and Weiss, B. \textit{An amenable equivalence relation is generated by a single transformation}.
Ergod. Th. and Dynam. Sys. \textbf{1} (1981), pp. 431--450. 


\bibitem[Da01]{Da01}  Danilenko, A., \textit{Entropy Theory from the Orbital Point of View}. Monatsh. Math. \textbf{134} (2001), pp. 121-141. 


\bibitem[DP02]{DP02} Danilenko, A. and Park, K. \textit{Generators and Bernoullian factors for amenable actions and cocycles on their orbits}.
Ergod. Th. and Dynam. Sys. (2002) \textbf{22}, pp. 1715--1745.

\bibitem[De74]{De74} Denker, M. \textit{Finite generators for ergodic, measure-preserving transformations}. Prob. Th. Rel. Fields \textbf{29} (1974), pp. 45--55. 

\bibitem[Dy59]{Dy59} Dye, H. \textit{On groups of measure transformations I}.
Amer. J. Math., vol. \textbf{81} (1959), pp. 119--159.

\bibitem[Em75]{Em75} Emerson, W. R. \textit{Averaging strongly subadditive set functions in unimodular amenable groups I}.  Pacific Journal of Mathematics \textbf{61} (1975), pp. 391--400.

\bibitem[FM77]{FM77} Feldman, J. and Moore, C. C. \textit{Ergodic equivalence relations, von Neumann algebras and cohomology.
I and II.} Trans. A.M.S. 234 (1977), pp. 289--324

\bibitem[FS08]{FS08} Flajolet, P. and Sedgewick, R. \textit{Analytic Combinatorics}. Cambridge University Press New York, 2008.

\bibitem[Ga00]{Ga00} Gaboriau, D. \textit{On orbit equivalence of measure preserving actions}. Rigidity in dynamics and geometry (Cambridge, 2000),
Springer Berlin 2002, pp. 167--186.  

\bibitem[Gr99]{Gr99} Gromov, M. \textit{Topological invariants of dynamical systems and spaces of holomorphic maps. I.} Math. Phys. Anal. Geom. \textbf{2} (1999), pp. 323--415.

\bibitem[INO08]{INO08} Izumi, M., Neshveyev, S. and Okayasu, R. \textit{The ratio set of the harmonic measure of a
random walk on a hyperbolic group}. Israel J. Math. 163 (2008), pp. 285--316.


\bibitem[Ka97]{Ka97} Kaimanovich, V. \textit{Amenability, hyperfiniteness and isoperimetric inequalities}. C. R. Acad. Sci. Paris S{\'e}r. I Math., \textbf{325} (1997), pp. 999--1004.

\bibitem[Ka02]{Ka02} Kaimanovich, V. \textit{Equivalence relations with amenable leaves need not be amenable}. In \textit{Topology, ergodic theory, real algebraic geometry} of Amer. Math. Soc. Trans. Ser. 2, \textbf{202} (2001), pp. 151--166. 

\bibitem[Ka03]{Ka03} Kaimanovich, V. A. \textit{Double ergodicity of the Poisson boundary and applications to bounded cohomology}. Geom. Funct. Anal. \textbf{13} (2003), no. 4, pp. 852--861.

\bibitem[Ki75]{Ki75} Kieffer, J. C. \textit{A ratio limit theorem for a strongly subadditive set function in a locally compact amenable group}.  Pacific Journal of Mathematics \textbf{61} (1975), pp. 183--190.


\bibitem[Ke13]{Ke13} Kerr, D. \textit{Sofic measure entropy via finite partitions}. Groups. Geom. Dyn. \textbf{7} (2013), pp. 617--632. 

\bibitem[KL11]{KL11} Kerr, D. and Li, H. \textit{Entropy and the variational principle for actions of sofic groups}. Invent. Math., \textbf{186} (2011), pp. 501--558.

\bibitem[KL13]{KL13} Lerr, D. and Li, H. \textit{Soficity, amenability, and dynamical entropy}. Amer. J. Math. \textbf{135} (2013), pp. 721�-761.

\bibitem[Ko58]{Ko58} Kolmogorov, A.N. \textit{New Metric Invariant of Transitive Dynamical Systems and Endomorphisms of Lebesgue Spaces}.
(Russian) Doklady of Russian Academy of Sciences \textbf{119} (1958), pp. 861--864.

\bibitem[Ko59]{Ko59} Kolmogorov, A.N. \textit{Entropy per unit time as a metric invariant for automorphisms}
(Russian) Doklady of Russian Academy of Sciences \textbf{124} (1959), pp. 754--755.

\bibitem[Kr70]{Kr70} Krieger, W. \textit{On entropy and generators of measure-preserving transformations}. Trans. Amer. Math. Soc. \textbf{149} (1970), pp. 453--464. 

\bibitem[Li01]{Li01} Lindenstrauss, E. \textit{Pointwise theorems for amenable groups}.
Invent. Math., \textbf{146} (2001) pp. 259--295. 

\bibitem[LW00]{LW00} Lindenstrauss, E. and Weiss, B. \textit{Mean topological dimension}. Israel J. Math. \textbf{115} (2000), pp. 1--24. 



\bibitem[Mc53]{Mc53} McMillan, B.,  \emph{The basic theorems of information
theory}. Ann. Math. Stat. \textbf{24} (1953), pp. 196--219.


\bibitem[Ne05]{Ne05} Nevo, A. \textit{Pointwise ergodic theorems
for actions of groups}. Handbook of Dynamical Systems,
vol. 1B, Eds. B. Hasselblatt and A. Katok, 2006, Elsevier, pp. 871-982.


\bibitem[Ol85]{Ol85} Ollagnier, J. \textit{Ergodic Theory and Statistical Mechanics}.  Lecture Notes in Mathematics \textbf{1115}, Springer-Verlag, Berlin, 1985.

\bibitem[OW80]{OW80} Ornstein, D. and Weiss, B. \textit{Ergodic theory of amenable
group actions. I. The {R}ohlin lemma.}, Bull. Amer. Math. Soc. vol. \textbf{2}
(1980), pp. 161--164. 

\bibitem[OW83]{OW83} Ornstein, D. and Weiss, B. \textit{The Shannon-McMillan-Breiman theorem for a class of amenable groups}. Israel J. Math. \textbf{44} (1983) pp. 53--60.


\bibitem[OW87]{OW87} Ornstein, D. and Weiss, B. \textit{Entropy and isomorphism theorems for actions of amenable groups}. Journal d'Analyse Math{\'e}matique \textbf{48} (1987), pp. 1--141.


\bibitem[Po14]{Po14} Pogorzelski, F. \textit{Convergence theorems for graph sequences}. Int. J. Alg. Comp. \textbf{24} (2014), pp. 1233--1251.

\bibitem[PS16]{PS16} Pogorzelski, F. and Schwarzenberger, F. \textit{A Banach space-valued ergodic 
theorem for amenable groups and applications}. Journal d'Analyse Math{\'e}matique \textbf{130} (2016), pp. 19--69.


\bibitem[RR07]{RR97} Ramagge, J. and Robertson, G. \textit{Factors from trees}. Proc. Amer. Math. Soc. \textbf{125} no. 7 (1997),  
pp. 2051--2055.

\bibitem[Ro67]{Ro67} Rokhlin, V.A. \textit{Lectures  on  the  entropy  theory  of  transformations  with  invariant  measure}.
Uspehi Mat. Nauk. \textbf{22} (1967), pp. 3--56. 

\bibitem[RW00]{RW00} Rudolph, D. and Weiss, B. \textit{Entropy and mixing for amenable group actions}. Ann. Math. \textbf{151} (2000), pp. 1119--1150.

\bibitem[Se12]{Se12} Seward, B. \textit{Ergodic actions of countable
groups and finite generating partitions}. 
Groups, Geometry, and Dynamics \textbf{9} no. 3 (2015), pp. 793--810. 

\bibitem[Se14]{Se15a} Seward, B. \textit{Krieger's finite generator theorem for ergodic actions of countable groups I}. 
{\tt http://arxiv.org/abs/1405.3604} (2014).

\bibitem[Se15]{Se15b} Seward, B. \textit{Krieger's finite generator
theorem for ergodic actions of countable groups II}. 
{\tt http://arxiv.org/abs/1501.03367} (2015).

\bibitem[Se16]{Se16} Seward, B. \textit{Weak containment and Rokhlin entropy}. 
{\tt http://arxiv.org/abs/1602.06680} (2016).

\bibitem[Sh48]{Sh48} Shannon, C. E., \textit{A mathematical theory
of communication}. Bell System Tech. J. \textbf{27} (1948), pp. 379--423, pp. 623--656.


\bibitem[Si59]{Si59} Sinai, Y. \textit{On the concept of entropy for a dynamical system}. Dokl. Akad. Nauk SSSR \textbf{124} (1959), pp. 768--771.

\bibitem[Sp87]{Sp87}  Spatzier, R. J. \textit{An example of an amenable action from geometry}. Ergodic Theory Dynam.
Systems \textbf{7} no. 2 (1987), pp. 289--293.

\bibitem[ST12]{ST12} Seward, B. and Tucker-Drob, R. \textit{Borel structurability on the 2-shift of a countable group}. Ann. Pure and Applied Logic
\textbf{167} (2016).  

\bibitem[Su78]{Su78} Sullivan, D. \textit{On the ergodic theory at infinity of an arbitrary discrete group of hyperbolic
motions}. Riemann surfaces and related topics: Proceedings of the 1978 Stony Brook
Conference (State Univ. New York, Stony Brook, N.Y., 1978), pp. 465--496, Ann. of
Math. Stud. \textbf{97}, Princeton Univ. Press, Princeton, N.J., 1981.


\bibitem[Su82]{Su82} Sullivan, D. \textit{Discrete conformal groups and measurable dynamics}. Bull. Amer. Math. Soc. (N.S.) \textbf{6} no. 1 (1982),
pp. 57--73.

\bibitem[We84]{W84} Weiss, B. \textit{Measurable dynamics}.  Contemp. Math. voL. \textbf{26} (1984), pp. 395-421.

\bibitem[We03]{W03} Weiss, B. \textit{Actions of amenable groups}.
 London Mathematical Society Lecture Note Series \textbf{310}, 226-262, (2003). 


\bibitem[Zi78]{Zi78} Zimmer, R. \textit{Amenable ergodic group actions and an application to Poisson boundaries of random walks}. J. Funct. Anal. \textbf{27} (1978), 350-372. 




\end{thebibliography}



\end{document}

