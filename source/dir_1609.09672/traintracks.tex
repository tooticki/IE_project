\chapter{Train tracks and pants distance}

\section{Train tracks: basics}\label{sec:traintracks}

\subsection{Definition. Tie neighbourhoods}\label{sub:traintrackdefin}

Our basic definitions are largely inspired by the ones of \cite{mms} and of \cite{mosher}, but they will not coincide entirely with those.
\begin{defin}\label{def:pretrack}
A \nw{pretrack} on a surface $S$ (resp. on $S^X$ where $X\subseteq S$ is a non-peripheral annulus) is a 1-complex $\tau$ smoothly, properly embedded in $S$ (resp. $S^X$) such that, for each of its vertices $v$, there are a tangent line $L\subset T_vS$ (resp. $T_vS^X$) and a compact neighbourhood $S$ (resp. $S^X$) $\supset U\ni v$, such that the following is true. $U$ is homeomorphic to a disc; the boundary of $U$ is piecewise smooth; $U$ includes no other vertex of $\tau$ and intersects no edge which is not incident with $v$; $\tau\cap\bar U$ is a union of smooth, properly embedded paths in $U$, with their endpoints on $\partial\bar U$, each passing through $v$ and such that its tangent line at $v$ is $L$. Vertices of $\tau$ are called \nw{switches}, and edges are called \nw{branches}. 

A pretrack is \nw{semigeneric} if, for any switch $v$, there is a neighbourhood $U$ as above, with the following extra property: there is a point $x\in\partial\bar U$ such that, for each of the aforementioned smooth paths which make up $\tau\cap U$, $x$ is one of its endpoints. A pretrack is \nw{generic} if each switch (vertex) is $3$-valent.

\ul{All pretracks in the present work are (at least) semigeneric}: we will not specify it again. We will use the adjective `semigeneric' only when we wish to stress that a pretrack is not necessarily generic.

Fix a branch $b$ of a pretrack $\tau$, and consider the family $F$ of all closed segments contained in $b$ such that exactly one of their endpoints is a switch of $\tau$. Moreover, consider the smallest equivalence relation on $F$ which contains the inclusion relation $\subseteq$: it partitions $F$ into two equivalence classes that, in a self-descriptive manner, are called \nw{branch ends}. Occasionally, we use the term `branch end' also in reference to a fixed element of $F$. 

Consider a switch $v$ of $\tau$ and the above construction of the neighbourhood $U$ and the point $x$. There is only one branch end that reaches $v$ from the direction of $x$: it will be called \nw{large}. All other branch ends at $v$ are called \nw{small}. A branch is large or small if both its ends are; it is \nw{mixed} if its ends are of opposite kinds.

We denote $\br(\tau)$ the set of all branches of $\tau$. A pretrack $\sigma$ is a \nw{subtrack} of $\tau$ if $\sigma\subseteq \tau$ as sets.
\end{defin}

Again, let $X\subseteq S$ be a non-peripheral annulus. Occasionally, we will use the term \nw{generalized pretrack} for $\tau=\bar \tau\cap S^X$ , where $\bar\tau$ is a 1-complex smoothly, properly embedded in $\ol{S^X}$ such that all its vertices $v$ \emph{which lie in $S^X$} have a tangent line $L$ and a compact neighbourhood $U$ as in the above definition. In a generalized pretrack, only vertices lying in $S^X$ will be called \emph{switches}, while all edges are still called \emph{branches}. The concept of \emph{subtrack} is easily extended: in particular, among the subtracks of a non-compact pretrack, there may be some which are only generalized pretracks.

Any semigeneric pretrack $\tau$ can be endowed with a \emph{tie neighbourhood}: the following construction is a variation of the one given in \cite{mms}.
 
\begin{defin}
Fix $\epsilon>0$ small. Given any representative $e$ of a branch end in a semigeneric pretrack $\tau$, let $v$ be the switch which serves as an endpoint of $e$: a \nw{branch end rectangle} is a smooth, orientation-preserving embedding $R_e: [a,1+\epsilon]\times[-1,+1]\rightarrow S$ (resp. $S^X$), where $-1<a<1$, such that:
\begin{itemize}
\item $R_b([a,1]\times\{0\})=e$, $R_b(1,0)=v$;
\item $R_b([1-\epsilon,1+\epsilon]\times[-1,1])$ is a compact neighbourhood of $v$ testifying (like the $U$ used in Definition \ref{def:pretrack}) that the graph $\tau$ satisfies the condition defining a semigeneric pretrack at the vertex $v$;
\item a branch of $\tau$ intersects the image of $R_e$ if and only if $v$ is one of its switches;
\item for all $a\leq t\leq 1+\epsilon$, the arc $\alpha_t^e\coloneqq R_e(\{t\}\times[-1,1])$ is transverse to any smooth path embedded in $\tau$ and intersecting $\alpha_t^e$.
\end{itemize}

Let now $b$ be a branch of $\tau$. A \nw{branch rectangle} for $b$ is a map $R_b:[-1-\epsilon,1+\epsilon]\times[-1,1]\rightarrow S$ (resp. $S^X$) such that, for two suitable numbers $-1<a_1<a_2<1$, the maps $R_{e_1}:[a_1,1+\epsilon]\times[-1,1]\rightarrow S$ (resp. $S^X$) defined by restriction of $R_b$, and $R_{e_2}:[-a_2,1+\epsilon]\times[-1,1]\rightarrow S$ (resp. $S^X$) defined by $R_{e_2}(x,y)\coloneqq R_b(-x,-y)$ are branch end rectangles relative to the branch end representatives $e_1\coloneqq R_b([a_1,1+\epsilon]\times\{0\})$ and $e_2\coloneqq R_b([-1-\epsilon,a_2]\times\{0\})$, respectively. A branch rectangle is \emph{not} an embedding exactly when the two switches that delimit the branch $b$ coincide: in this case, its image is not really diffeomorphic to a rectangle.

\begin{figure}
\def\svgwidth{.65\textwidth}
\begin{center}
\input{tienbh.pdf_tex}
\end{center}
\caption{\label{fig:tienbh}This is the local picture of how the images of the functions $R_e$ overlap in the neighbourhood of a switch of $\tau$. The dashed vertical lines are two examples of ties. The tie neighbourhood $\bar\nei(\tau)$ is coloured in grey.}
\end{figure}

A \nw{tie neighbourhood} for $\tau$, denoted $\bar\nei(\tau)$, is specified by a family of branch end rectangles $\{R_e\}_{e\in E}$ where $E$ is a family of branch end representatives such that $\tau=\bigcup_{e\in E} e$, no branch end has two distinct representatives in $E$ and, if the union of $e_1,e_2\in E$ is a branch, then $e_1\cap e_2$ consists of more than one point. We list now a number of `consistency' conditions that the branch end rectangles $R_e$ are required to meet. Taking care of avoiding confusion, here and in the rest of this work we use $R_e$ to denote both a branch end rectangle and its image; we do the same for branch rectangles.

\begin{itemize}
\item The intersection of any two branch end rectangles $R_e, R_{e'}$ ($e,e'\in E$) is either connected or empty. More precisely, if $e\cup e'$ is a branch of $\tau$, or $e,e'$ meet at a switch of $\tau$, then $R_e\cap R_{e'}\not=\emptyset$ (so this set is connected); if neither of these is true, we require $R_e\cap R_{e'}=\emptyset$.
\item When $R_e\cap R_{e'}\not=\emptyset$, we require that the set $\alpha_t\coloneqq R_e(\{t\}\times [-1,1])\cap R_{e'}$ is connected for all $t$ for which it is defined and not empty. Same for $\alpha'_t\coloneqq R_{e'}(\{t\}\times [-1,1])\cap R_{e}$. Furthermore, the families $\{\alpha_t\}_t$, $\{\alpha'_t\}_t$ are required to define the same foliation of $R_e\cap R_{e'}$, possibly with some leaves degenerating to single points.
\end{itemize}
Suppose now that, at a switch $v$ of $\tau$, a large branch end representative $e\in E$ is meeting a collection of small branch end representatives $e_1,\ldots,	e_k\in E$. The following requests ensure that the rectangles get assembled as shown in Figure \ref{fig:tienbh}.
\begin{itemize}
\item $R_e(\{1\pm\epsilon\}\times[-1,1])\supseteq R_{e_i}(\{1\mp\epsilon\}\times[-1,1])$ for all $i=1,\ldots, k$ (here the signs $\pm,\mp$ mean that this expression summarizes two equalities).
\item For $i\not=j$, the segments $R_{e_i}(\{1-\epsilon\}\times[-1,1])$ and $R_{e_j}(\{1-\epsilon\}\times[-1,1])$ are disjoint.
\item There are two indices $1\leq i+,i-\leq k$ such that $R_e([1-\epsilon,1+\epsilon]\times \{1\})=R_{e_{i+}}([1-\epsilon,1+\epsilon]\times \{-1\})$ and $R_e([1-\epsilon,1+\epsilon]\times \{-1\})=R_{e_{i-}}([1-\epsilon,1+\epsilon]\times \{1\})$.
\end{itemize}
Suppose, finally, that the union of $e_1,e_2\in E$ is a branch $b$ of $\tau$.
\begin{itemize}
\item There must exist a branch rectangle $R_b$ that restricts to $R_{e_1},R_{e_2}$ in the sense specified by the above definition.
\end{itemize}

Usually we identify $\bar\nei(\tau)$ with the union of all branch end rectangles that constitute it. We denote with $\nei(\tau)$ the interior of $\bar\nei(\tau)$: it is an open, regular neighbourhood of $\tau$.

An arc $\alpha \subseteq \partial\bar\nei(\tau)$ that, when intersected with any $R_e$ ($e\in E$), is the image of a vertical segment $\{t\}\times[-1,1]$ is called a \nw{tie}. All ties intersect $\tau$ transversally and together they specify a foliation of $\bar\nei(\tau)$.

The boundary $\partial\bar\nei(\tau)$ can be subdivided into $\partial_v\bar\nei(\tau)$ which consists of the smooth segments of boundary which are also segments of ties; and $\partial_h\bar\nei(\tau)$ which consists of the remaining segments.
\end{defin}

A tie neighbourhood for a generalized pretrack $\tau$ on $S^X$ may be defined with a slight generalization of this construction. In this case, if $b$ is a branch of $\tau$ which is not compact and is obtained as the intersection of an edge $\bar b$ of $\bar\tau$ with $S^X$, we define a branch rectangle for $b$ as an embedding $R_b:[a,1+\epsilon]\times[-1,1]\rightarrow \bar S^X$, with $R_b([a,1+\epsilon]\times\{0\})=\bar b$, satisfying similar conditions as branch end rectangles, plus the extra request that $R_b\cap \partial\ol{S^X}=R_b(\{a\}\times[-1,1])$. The generalized definition of tie neighbourhood continues by revisiting the above constructions in a natural way.

We now define a different version of `neighbourhood' of a pretrack $\tau$: rather than right-angled corners, this time we want all corners in the boundary to be cusps, and coinciding with the vertices of $\tau$. Note that the following definition does not give a genuine neighbourhood of $\tau$; moreover, if $\tau$ is not generic, the interior of this `neighbourhood' may have more connected components than $\tau$.

First of all, when $\tau$ is a generic pretrack, there is a natural bijection between connected components of $\partial_v\bar\nei(\tau)$ and switches of $\tau$. If $\tau$ is not generic, instead, for each connected component of $\partial_v\bar\nei(\tau)$ there is a canonical choice of an associated switch of $\tau$, but this choice is not injective.

Given any component $V$ of $\partial_v\bar\nei(\tau)$, associated to a switch $v$ of $\tau$, let $\eta_1(V),\eta_2(V)$ be the two components of $\partial_h\bar\nei(\tau)$ which share an endpoint with $V$ (they may coincide). Let $\theta_1(V),\theta_2(V)$ be two smooth arcs which connect each endpoint of $V$ to $v$, are transverse to all ties they encounter, meet $\tau$ only at $v$, and are chosen so that $\eta_j(V)\cup\theta_j(V)$ is a smooth arc for $j=1,2$. There is a triangle $T_V\subseteq\bar\nei(\tau)$ whose edges are $V,\theta_1(V),\theta_2(V)$. 

We define $\nei_0(\tau)\coloneqq \nei(\tau)\setminus\left(\bigcup_V T_V\right)$, where the union is over all the connected components $V$ of $\partial_v\bar\nei(\tau)$. So $\nei_0(\tau)$ includes the interiors of all branches of $\tau$ but not the switches, which are the corners of $\partial\bar\nei_0(\tau)$. The smooth segments of $\partial\bar\nei_0(\tau)$ biject naturally with the connected components of $\partial_h\bar\nei(\tau)$.

One may also define a retraction $c_\tau:\bar\nei(\tau)\rightarrow \tau$ as follows. If $p\in \bar\nei_0(\tau)$, define $c_\tau(p)$ to be the only point of $\tau$ contained in the tie along $p$. If $p\in T_V$ for a component $V$ of $\partial_v\bar\nei(\tau)$ as above, instead, we define $c_\tau(p)$ as $c_\tau(r_V(p))$: here $r_V: T_V\rightarrow \theta_1(V)\cup\theta_2(V)$ is a retraction whose fibres are transverse to the ties of $\bar\nei(\tau)$. The map $c_\tau$ is called a \nw{tie collapse} for $\tau$ (even though its actual definition is a bit more involved than what the name suggests).

Given a tie neighbourhood $\bar\nei(\tau)$ of a pretrack $\tau$ in $S$ (resp. $S^X$ where $X$ is a non-peripheral annulus), and $\sigma$ a subtrack of $\tau$, there exists a tie neighbourhood $\bar\nei(\sigma)\subset \bar\nei(\tau)$ (not unique) such that each tie of $\bar\nei(\sigma)$ is a sub-tie of $\bar\nei(\tau)$, with the property that: if $b\in\br(\tau)$ is contained in $\sigma$ (note that $b$ may not be a branch there) and has a branch rectangle $R_b$ in $\bar\nei(\tau)$, then $R_b([-1+\epsilon ,1-\epsilon]\times[-1,1])\subseteq \bar\nei(\sigma)$ and $R_b([-1+\epsilon,1-\epsilon]\times\{-1,1\})\subseteq \partial_h\bar\nei(\sigma)$: informally, this property means that $\partial \bar\nei(\sigma)$ contains (roughly) as much as possible of $\partial\bar\nei(\tau)$.

\begin{figure}
\def\svgwidth{.65\textwidth}
\begin{center}
\input{corners.pdf_tex}
\end{center}
\caption{\label{fig:cornertypes} The possible local pictures for a 2-submanifold $C\subseteq S$ around a point of $\partial C$ which is not smooth. $C$ is shaded in grey. We need to distinguish whether the angle $\partial C$ forms at that point is convex (a. and b.) or concave (c. and d.); and whether the two segments delimiting the angle meet tangentially (a. and d.) or transversely (b. and c.). These are the only two pieces of information from these local pictures which are invariant under smooth isotopies of $S$.}
\end{figure}

\begin{defin}
Let $C$ be a 2-submanifold of a surface (possibly with boundary) $S$, such that $\partial C$ is piecewise smooth. With reference to Figure \ref{fig:cornertypes}, we define the \nw{index} of $C$ as
\begin{eqnarray*}
\idx(C) & \coloneqq & \chi(C) - (\#\text{corners of type a})/2 - (\#\text{corners of type b})/4\\
 & & + (\#\text{corners of type c})/4 + (\#\text{corners of type d})/2.
\end{eqnarray*}
\end{defin}

The index is additive: given two submanifolds as above $C',C''$, which only share a portion of their boundaries, $\idx(C'\cup C'')=\idx(C')+\idx(C'')$. Note that the 2-submanifolds considered here include, for any pretrack $\tau$, all compact connected components of $\bar\nei(\tau)$ and of $S\setminus \nei(\tau)$, plus all closures of connected components of $\nei_0(\tau)$, and of $S\setminus\bar \nei_0(\tau)$, which are compact. The possibility of foliating $\bar\nei(\tau)$ into ties implies that the index of any of its connected components is $0$. Closures of connected components of $\nei_0(\tau)$ have zero index, too.

\begin{defin}
A \emph{compact} pretrack $\tau$ on a surface $S$, entirely contained in the compactification $S_\bullet$, is:
\begin{itemize}
\item an \nw{almost track} if each connected component of $S_\bullet\setminus\nei(\tau)$ either has negative index or is a peripheral annulus; and, for each peripheral annulus $P$ among these, $\partial P$ is isotopic to a closed geodesic of $S$ (i.e. $P$ gives a funnel in $S$);
\item a \nw{train track} if each connected component of $S_\bullet\setminus\nei(\tau)$ has negative index;
\item a \nw{cornered train track} if it is a train track and each connected component of $\partial\left(S\setminus\nei(\tau)\right)$ has a corner.
\end{itemize}
\end{defin}

Note that, on a surface $S$ with finite hyperbolic area, $\tau$ is an almost track if and only if it is a train track, because $S$ has no closed geodesic encircling a puncture. The reason why we require peripheral annuli in almost tracks to be funnels is to prevent lifts of a train track from behaving pathologically at infinity (see \S \ref{sub:induced}), and is the only reason why the hyperbolic structure on $S$ is relevant for us.

Here is another definition which is, in some sense, symmetrical:
\begin{defin}
Let $S$ be a surface, even one with boundary. When each connected component $C\subseteq \inte(S)\setminus\nei(\tau)$, for $\tau$ a pretrack is \emph{homeomorphic} to a disc or a once-punctured disc, we say that the pretrack \nw{fills $S$}.
\end{defin}

\subsection{Carrying}

We will have a particular care for arcs and curves `contained' in pretracks and train tracks:
\begin{defin}
A \nw{(bounded, infinite, biinfinite, periodic) train path} along a pretrack $\tau$ is a smooth immersion $f:A\rightarrow \tau$, where:
\begin{itemize}
\item $A=[m_1,m_2]$, $[0,+\infty)$, $\R$ or $\faktor{\mathbb R}{m_3\mathbb Z}$, respectively, according to the adjective\linebreak ($m_1,m_2,m_3\in\mathbb Z$, with $m_1<m_2$ and $m_3>0$);
\item $f^{-1}(\text{switches})=A\cap\mathbb Z$.
\end{itemize}
\end{defin}

\begin{defin}\label{def:carried}
Let $\tau$ be a pretrack on a surface $S$, with $\bar\nei(\tau)$ a tie neighbourhood; let $\sigma$ be another pretrack; let $\beta$ be a curve or a multicurve; let $\delta$ be a properly embedded arc in $\bar\nei(\tau)$, with $\partial\delta=\delta\cap \partial_v\bar\nei(\tau)$, to be considered up to isotopy leaving the endpoints fixed.

An inclusion map $f:\sigma$ (resp. $\beta,\delta$) $\hookrightarrow\bar\nei(\tau)$, with its image transverse to each tie it encounters, and ambient isotopic to $\sigma$ (resp. to $\beta$; or isotopic to $\delta$ with fixed endpoints) in $S$ is called a \nw{carried realization}.

The pretrack $\sigma$ (resp. the (multi)curve $\beta$ or arc $\delta$) is \nw{carried} by $\tau$ if it admits a carried realization. 

We will often talk, more loosely, of carried realization referring simply to the image of $f$.

The pretrack $\sigma$ (resp. curve/multicurve $\beta$ or arc $\delta$) \nw{traverses} a branch $b$ if $\mathrm{im}(f)\cap \alpha_t^b\not=\emptyset$ for all $t\in[-1,1]$. If a (multi)curve or arc $\beta$ traverses a branch $b$, the \nw{multiplicity} of the traversing is the number of points in $\mathrm{im}(f)\cap \alpha_t^b$ for any $t\in[-1,1]$ (we may use expressions like \emph{traverses once, twice\ldots}). The \nw{carrying image} of a carrying injection is the union of the branches of $\tau$ which are traversed by $\sigma$ (resp. $\beta$, $\delta$). We will denote it with $\tau.\sigma$ ($\tau.\beta$, $\tau.\delta$ resp.).

A pretrack $\sigma$ is \nw{fully carried} if it is carried and $\tau.\sigma=\tau$. It is \nw{suited} to $\tau$ if it is carried and $f:\sigma\hookrightarrow \bar\nei(\tau)$ is a homotopy equivalence.

We denote $\cc(\tau)\subseteq\cc(S)$ the set of isotopy classes of curves carried by $\tau$.
\end{defin}

\begin{defin}\label{def:trainpathrealization}
Let $f:\beta\hookrightarrow\bar\nei(\tau)$ be a carried realization of a (multi)curve $\beta\in \cc(S)$ in a pretrack $\tau$, as specified in Definition \ref{def:carried} above. Then $f$ may be homotoped, keeping each point along the same tie of $\bar\nei(\tau)$, to a map $f'$ whose image is entirely contained in $\tau$: this new map is not injective anymore, but it is still an immersion. A suitable reparametrization of $f'$, then, defines a (collection of) periodic train path(s), which we call a \nw{train path realization} of $\beta$. The image of a train path realization of $\beta$ is the carrying image $\tau.\beta$.

Let now $f:\delta\hookrightarrow\bar\nei(\tau)$ be a carried realization of an arc $\delta$ in a pretrack $\tau$. Let $V,W$ be the (possibly coinciding) components of $\partial_v\bar\nei(\tau)$ where the endpoints of $\delta$ lie; $v,w$ be the switches of $\tau$ associated with $V,W$ respectively; and $\alpha_v,\alpha_w$ be the ties of $\bar\nei(\tau)$ through $v,w$ respectively. Define $\delta_{trim}$ as the longest segment of $\delta$ such that $f$ maps its extremes to points of $\alpha_v$, $\alpha_w$ respectively; and let $f_{trim}$ be the restriction of $f$ to $\delta_{trim}$.

Similarly as above, $f_{trim}$ may be homotoped, keeping each point along the same tie of $\bar\nei(\tau)$, to a map $f'_{trim}$ whose image is entirely contained in $\tau$ and can be reparametrized to get a bounded train path along $\tau$. We call the latter, again, a \nw{train path realization} of $\delta$; its image is $\tau.\delta$.
\end{defin}

\begin{rmk}
If $\tau$ is a \emph{generic} pretrack, then $\tau$ is a deformation retract of $\bar\nei_0(\tau)$, which is in turn a deformation retract of $\bar\nei(\tau)$. This makes it possible to ask more --- and we will --- from a carrying injection $f:\sigma\hookrightarrow \bar\nei(\tau)$ (or $f:\beta\hookrightarrow \bar\nei(\tau)$, $f:\delta\hookrightarrow \bar\nei(\tau)$), up to altering $f$ via isotopies which still map each point of $\sigma$ (resp. $\beta$, $\delta$) along the same tie as $f$ does.
\begin{itemize}
\item For a pretrack $\sigma$ and for a (multi)curve $\beta$, we require the image of $f$ to be contained in $\bar\nei_0(\tau)$. This means that a train path realization of $\beta$ is obtained just by reparametrizing $c_\tau\circ f$, while for a pretrack $\sigma$ we have $\tau.\sigma=c_\tau\circ f(\sigma)$.
\item For an embedded arc $\delta$, we require $\mathrm{im}(f)\cap \bar\nei_0(\tau)$ to be connected. This implies that, given $f_{trim}$ as defined above, one gets a train path realization of $\delta$ by reparametrizing $c_\tau\circ f_{trim}$.
\end{itemize}
\end{rmk}

\begin{rmk}\label{rmk:idx_of_nei_diff}
Let $\tau,\sigma$ be pretracks such that the injection $\sigma\hookrightarrow\nei(\tau)$ is a carrying map; choose a tie neighbourhood $\bar\nei(\sigma)\subseteq \bar\nei(\tau)$. Then each compact component of $\bar\nei(\tau)\setminus\nei(\sigma)$ has zero index.

This is because $\bar\nei(\tau)\setminus\nei(\sigma)$ can be subdivided into a family of rectangles and triangles with two (outward) right corners and a cusp; and they both have zero index. The triangles, in particular, may arise when $\partial_h\bar\nei(\sigma)$ and $\partial_h\bar\nei(\tau)$ have segments in common.
\end{rmk}

In the case of a train track, there is little space available in deciding \emph{how} to get something carried: the following statement summarizes Propositions 3.5.2 and 3.6.2 from \cite{mosher}:
\begin{prop}\label{prp:carryingunique}
Let $\sigma,\tau$ be two train tracks on a surface $S$, with $\sigma$ determined up to isotopies of $S$. Then, given any two carrying injections $f_1,f_2:\sigma\hookrightarrow\bar\nei(\tau)$, $f_1$ and $f_2$ are homotopic through carrying injections. The same is true for two carrying injections of a curve $\alpha\in\cc(S)$ carried by $\tau$. In particular the carrying image $\tau.\sigma$ or $\tau.\alpha$ is uniquely determined, and so is the train path corresponding to $\alpha$, up to reparametrization.
\end{prop}

We will need a slight generalization:
\begin{coroll}\label{cor:carryingunique}
Let $\tau$ be an almost track on a surface $S$. Then any two carrying injections of another almost track $\sigma$, or of a curve $\alpha\in\cc(S)$, in $\bar\nei(\tau)$ are homotopic through carrying injections.
\end{coroll}
\begin{proof}
Fix two carrying injections $f_1,f_2:\sigma$ (or $\alpha$) $\rightarrow\bar\nei(\tau)$; then there is a map $h: \sigma \times [0,1]\rightarrow S$ such that $h|_{\sigma\times\{0\}}=f_1$ and $h|_{\sigma\times\{1\}}=f_2$. As the domain of $h$ is compact, its image also is. Hence there is a union $P$ of peripheral annuli for $S$, one for each puncture, disjoint from each other and from $\mathrm{im}(h)\cup\bar\nei(\tau)$.

Let $\Sigma \in P$ be a finite set of points, one for each connected component; let then $S'\coloneqq S\setminus \Sigma$. This is a surface on its own, with a hyperbolic metric which is not related with the one of $S$. However $\tau\subseteq S'$ is a train track, as this is a property which is independent of the metric.

For the case of a carried almost track $\sigma$: $f_1(\sigma)\subseteq S'$ is a train track because, if we pick a tie neighbourhood $\nei(f_1(\sigma))\subseteq\nei(\tau)$, then any connected component of $S'\setminus\nei(f_1(\sigma))$ is a gluing of some connected components of $S'\setminus\nei(\tau)$ (at least one of them) with some of $\bar\nei(\tau)\setminus \nei(f_1(\sigma))$; the latter have zero index because of Remark \ref{rmk:idx_of_nei_diff}. Hence $S'\setminus\nei(f_1(\sigma))$ has negative index.

For the case of a carried curve $\alpha$, $f_1(\alpha)\subset S'$ is still essential.

The map $h$ serves as a homotopy between $f_1$ and $f_2$ also in $S'$. So, with an application of the above proposition, $f_1$ and $f_2$ are actually homotopic through carrying injections; hence the same property is true in $S$.
\end{proof}

The set $\cc(\tau)$, for $\tau$ a train track, is `generated' by few curves. To understand this we need the following notion:
\begin{defin}\label{def:transversemeasure}
Let $\tau$ be an almost track on a surface $S$. A \nw{transverse measure} on $\tau$ is a map $\mu:\br(\tau)\rightarrow \R_{\geq0}$ with the following property. For each switch $v$ of $\tau$, if $b$ is the branch having a large end at $v$, and $b_1,\ldots,b_m$ are the branches having a small end there (we list any of those branches \emph{twice} if it has both ends there), then $\mu(b)=\sum_{i=1}^m \mu(b_i)$. We denote the set of such measures with $\mathcal M(\tau)$. A transverse measure can be equally seen as an element of $\R^{|\br(\tau)|}$. More precisely, the subset $C$ of $\R^{|\br(\tau)|}$ consisting of transverse measures is a cone with its summit at the origin. Also define ${\mathcal M}_{\mathbb Q}(\tau)\coloneqq {\mathcal M}(\tau)\cap \mathbb Q^{|\br(\tau)|}$, i.e. the set of transverse measures which assign a rational weight to each branch.

Given $\alpha\in\cc(\tau)$ and a train path realization $f:\alpha\rightarrow \tau$, we can define the transverse measure $\mu_\alpha$ by setting, for each branch $b$ of $\tau$, $\mu_\alpha(b)=$ number of connected components of $f^{-1}(b)$ (i.e. the number of times a carried realization of $\alpha$ traverses $b$). It is an almost immediate consequence of Proposition \ref{prp:carryingunique} that this measure depends only on the isotopy class of $\alpha$.
\end{defin}

The following is a simplified version of Theorem 3.7.1 from \cite{mosher}, or Theorem 1.7.7 from \cite{penner}.
\begin{prop}\label{prp:measurecurvecorresp}
Let $\tau$ be a train track on $S$. Define\footnote{$WMC$ stands for \emph{weighted multicurves}.}
$$WMC(\tau)\coloneqq \left\{\left((\gamma_1,a_1),\ldots,(\gamma_m, a_m)\right)\left|\begin{array}{l}
m\in\mathbb N; \\
\gamma_j\in\cc(\tau)\text{ for all } j \text{ and are pairwise disjoint};\\
a_j\in \mathbb Q_{>0}\text{ for all } j
\end{array}\right.\right\}.$$
Let also 

Then the map
\begin{eqnarray*}
WMC(\tau) & \longrightarrow & {\mathcal M}_{\mathbb Q}(\tau)\setminus \{0_\tau\} \\
\left\{(\gamma_1,a_1),\ldots,(\gamma_m, a_m)\right\} & \longmapsto & \sum_{j=1}^m a_j\mu_{\gamma_j}
\end{eqnarray*}
is a bijection ($0_\tau$ denotes the zero transverse measure).
\end{prop}
\begin{coroll}\label{cor:measurecurvecorresp}
Let $\tau$ be an almost track on $S$. Then the above map is injective.
\end{coroll}
\begin{proof}
If the given map is not injective, one may find two distinct collections\linebreak $\left\{(\gamma_1,a_1),\ldots,(\gamma_m, a_m)\right\}$ and $\left\{(\gamma'_1,a'_1),\ldots,(\gamma'_{m'}, a_{m'})\right\}$ such that $\sum_{j=1}^m a_j\mu_{\gamma_j}=\linebreak \sum_{j=1}^{m'} a'_j\mu_{\gamma'_j}$. Let $\ul{\gamma_1},\ldots,\ul{\gamma_m},\ul{\gamma'_1},\ldots,\ul{\gamma'_{m'}}$ be carried realizations of the isotopy classes $\gamma_1,\ldots,\gamma_m$, $\gamma'_1,\ldots,\gamma'_{m'}$. We repeat the construction seen in the proof of Corollary \ref{cor:carryingunique}: let $\Sigma\subset S$ be a finite set consisting of a point for each peripheral annulus among the components of $S\setminus\bar\nei(\tau)$, and let $S'\coloneqq S\setminus \Sigma$. Then $\tau$ is a train track in $S'$ and the curves $\ul{\gamma_i},\ul{\gamma'_j}$ are all essential in $S'$, and carried by $\tau$: we identify them with their respective classes in $\cc_{S'}(\tau)$ (i.e. the subset of $\cc^0(S')$ consisting of all curves carried by $\tau$).

For $\alpha\in \cc_{S'}(\tau)$, let $\mu'_\alpha$ be the measure it induces on $\tau$ as a train track in $S'$. Then $\sum_{j=1}^m a_j\mu'_{\ul{\gamma_j}}=\sum_{j=1}^{m'} a'_j\mu'_{\ul{\gamma'_j}}$, but this contradicts the above proposition.
\end{proof}

\begin{defin}
Let $\tau$ be a train track on a surface $S$. It is shown that the cone $C$ specified above is the convex hull of a bounded number of rays. Pick the smallest such set $\{r_1,\ldots,r_l\}$ of rays: each $r_j$ will correspond to the real multiples of a single $\mu_{\gamma_j}$, for a $\gamma_j\in\cc(\tau)$. We denote $V(\tau)=\{\gamma_1,\ldots,\gamma_l\}$ the \nw{vertex set} of $\tau$; an element of this set is called a \nw{vertex cycle}.

If $\tau$ is an almost track on $S$, then remove an extra point from each of the components of $S\setminus\nei(\tau)$ which are peripheral annuli, to get a surface $S'$. Now $\tau\subset S'$ is a train track, and has vertex set $\{\gamma_1,\ldots,\gamma_l\}\subseteq \cc(S')$. Up to changing their order, we may suppose that for an index $0\leq k\leq l$ the curves $\gamma_{k+1},\ldots \gamma_l$ are inessential in $S$, while the remaining ones define isotopy classes $[\gamma_1],\ldots,[\gamma_k]\in\cc(S)$. We define then $V(\tau)\coloneqq\{[\gamma_1],\ldots,[\gamma_k]\}$.
\end{defin}

\subsection{More about tracks and curves}

\begin{defin}
Let $\tau$ be any pretrack on a surface $S$. A carried curve or arc $\gamma$ is \nw{wide} if its carried realization may be given an orientation such that each branch $b$ is either: traversed by $\gamma$ at most once; or traversed twice, in such a way that each segment of $\gamma\cap R_b$ appears to the right of the other.

We denote $W(\tau)\subset \cc(\tau)$ the set of wide carried curves of $\tau$.
\end{defin}

Note that, for $\tau$ an almost track, $V(\tau)\subseteq W(\tau)$ because it is quite easy to decompose $\mu_\gamma$ into a sum of measures represented by simpler curves if $\gamma$ is not wide (Lemma 2.8 in \cite{mms}).

\begin{lemma}\label{lem:vertexsetbounds}
There are bounds $N_0, N_1, N_2, C_0$ depending on $S$ such that, for any almost track $\tau$, the set $W(\tau)$ has no more than $N_0$ elements and its diameter is no larger than $C_0$, $\tau$ has at most $N_1$ branches, and $S\setminus\nei(\tau)$ consists at most $N_2$ connected components.

For a matter of convenience, we suppose all these bounds to be increasing with $\xi(S)$ (by enlarging them when necessary).
\end{lemma}
\begin{proof}
Compactness of almost tracks implies that the number of switches and branches in any of them is finite; and so is the number of connected components of $S \setminus \nei(\tau)$. This means that $\idx(S_\bullet)$ is the sum of indices of the connected components of $S_\bullet\setminus \nei(\tau)$ (because $\idx\left(\bar\nei(\tau)\right)=0$, as it is a finite gluing of rectangles). These components all have negative index, except for some peripheral annuli (they are at most as many as the punctures of $S$). But this poses an upper bound first of all on their number, and then also on the number of smooth edges in their boundary. Eventually, this yields that there is a \emph{uniform} bound on the number of branches and of switches of $\tau$, in terms of the topology of $S$.

As a consequence, there are finitely many distinct almost tracks up to diffeomorphisms of $S$; and the sets of wide curves are equivariant under diffeomorphisms. The definition of wide curve poses combinatorial constraints yielding that, for each almost track $\tau$, they are finitely many isotopy classes of them.

These remarks are enough to prove all claims in the statement.
\end{proof}

Given a curve $\gamma$, carried by a pretrack $\tau$, a \nw{wide collar} $A_\gamma$ for $\gamma$ is an open annulus $A_\gamma\subseteq S$, with compact closure, such that $\tau.\gamma$ constitutes one of the two components of $\partial\bar A_\gamma$, and the other component is an embedding of $\mathbb S^1$ belonging to the isotopy class $\gamma$. On top of this, we can require that $A_\gamma$ does not contain any switch of $\tau$.

A curve $\gamma\in\cc(\tau)$ has a wide collar if and only if it is wide carried. Suppose first that $\gamma$ admits a wide collar $A_\gamma$. Then a suitable realization of the core of $A_\gamma$, close to $\tau.\gamma$, is a carried realization of $\gamma$ and shows that $\gamma$ is wide carried. Conversely, if $\gamma$ is wide carried, one can always arrange for a carried realization, $\ul\gamma$, to have the property that, for each $b\in\br(\tau)$ traversed by $\gamma$, the segment $\ul\gamma\cap R_b([-1,1]\times[-1,1])$ has $b$ to its right. Then $\tau.\gamma$ and $\ul\gamma$ bound together a wide collar, obtained as a union of sub-ties in $\nei(\tau)$. 

There is a notion of canonical realization even for curves which are not carried, as pointed out in \cite{mms}:
\begin{defin}\label{def:efficientposition}
A multicurve $\gamma$ on $S$ is in \nw{efficient position} with respect to a train track $\tau$ if $\gamma$ is embedded in $S$, transversely to $\tau$, with the following restrictions. Let $\nei(\gamma)$ be a regular neighbourhood of $\gamma$ containing no corners of $\partial \bar\nei(\tau)$.
\begin{itemize}
\item the chosen embedding of $\gamma$ does not include any switch of $\tau$;
\item for each branch $b$, the intersection of $\gamma\cap R_b$ (when nonempty) is a collection of ties, or it is transverse to all ties (in this latter case we say that $b$ is \nw{traversed} by $\gamma$, similarly as for carried multicurves);
\item for each connected component $C$ of $S_\bullet\setminus\nei(\tau)$, each connected component of $\ol{C\setminus\nei(\gamma)}$ either has negative index or is a rectangle.
\end{itemize}

Such a multicurve is \nw{wide} if, moreover, there is a choice for the orientation of its components with the following properties.
\begin{itemize}
\item If $b$ is a branch such that $\gamma$ traverses $R_b$ transversally to the ties, $\gamma\cap R_b$ is either a single segment, or two segments each appearing to the right of the other.
\item Fix any component $C$ of $S_\bullet\setminus\nei(\tau)$, and any two points $P_1,P_2$ of $\gamma\cap \partial C$ which are consecutive along $\partial C$. Let $\alpha_1,\alpha_2$ be the connected components of $\gamma\cap C$ the two points belong to. Then either $\alpha_1=\alpha_2$ or, with the given orientations on $\gamma$, $\alpha_1,\alpha_2$ appear to the right of each other.
\end{itemize}
\end{defin}

Carried realizations of curves and arcs are a particular case of efficient position; the opposite extreme case is the following. An arc or curve $\gamma\in\acc(S)$ is \nw{dual} to a pretrack $\tau$ if is (isotopic to one) in efficient position, with no branch $b$ such that $\mathrm{im}(R_b)$ is traversed transversally to the ties. We denote $\cc^*(\tau)\subset \cc(S)$ the set of curves in $S$ which are dual to $\tau$.

\begin{defin}\label{def:recurrent}
An almost track $\tau$ is \nw{recurrent} if for each branch $b$ of $\tau$ is traversed by a periodic train path in $\tau$. Note that, for $\tau$ a train track, this is equivalent to saying that each branch is traversed by some carried curve of $\tau$.

The track $\tau$ is \nw{transversely recurrent} if if for each branch $b$ of $\tau$ there is a dual curve $\gamma$ intersecting $b$.

The track is \nw{birecurrent} if it both recurrent and transversely recurrent.
\end{defin}

It can be shown (see \cite{penner}, \S 1.3) that a train track $\tau$ is recurrent if and only if there is a carried multicurve whose carrying image is the entire $\tau$. Similarly, $\tau$ is transversely recurrent if and only if there is a family of pairwise disjoint simple closed loops in $S$, dual to $\tau$, possibly including isotopic pairs, and such that, for each branch $b\in\br(\tau)$, one of the specified realizations intersects $b$.

One of the main theorems in \cite{mms} is that efficient position always exists under some mild conditions on the train track (Theorem 4.1):
\begin{theo}\label{thm:efficientpositionexists}
Let $\tau$ be a birecurrent, cornered train track on $S$; and let $\gamma\subset \cc(S)$ be a multicurve. Then $\gamma$ has an isotopy representative in efficient position with $\tau$.

Given two different realizations $\gamma_1,\gamma_2$ of $\gamma$ in efficient position, they can be transformed into each other via a finite sequence of elementary operations (described in the original statement).
\end{theo}

\subsection{Elementary moves}

Some alterations of pretracks are considered to be \nw{elementary moves}. We describe them in Figures \ref{fig:ttcombing} and \ref{fig:ttsplitting}, referring to \S 3.12 and 3.13 of \cite{mosher} for formal definitions.

\begin{figure}
\begin{center}
\def\svgwidth{.7\textwidth}
\input{combing.pdf_tex}

\vspace{1ex}\includegraphics[width=.7\textwidth]{sliding.pdf}
\end{center}
\caption{\label{fig:ttcombing}The upper picture represents a \nw{comb move} in a semigeneric pretrack; this move, in particular, shrinks a mixed branch to a point. The inverse of a comb move will be called \nw{uncomb}. The notion of comb move does not make sense in the setting of generic train track, where it is replaced by the one of \nw{slide}, depicted in the lower picture: given a mixed branch, its large end `moves past' the small end, and in so doing it replaces the old mixed branch with a new one. The effect of a slide can be cancelled, up to isotopy, with a further slide.}
\end{figure}
\begin{figure}
\begin{center}
\def\svgwidth{.7\textwidth}
\vspace{1ex}\includegraphics[width=.7\textwidth]{splitting.pdf}
\end{center}
\caption{\label{fig:ttsplitting} A \nw{split} is an elementary move acting on a large branch. The easiest case is the one of the \nw{central split} (middle arrow) which is intuitively understood as cutting along the large branch with a pair of scissor; the scissors enter a branch end from the gap between two small branches, and exit on the opposite side through a similar gap. The two copies of the split branch may be connected via a `diagonal' new branch: if the latter is placed as shown in the top arrow, we speak of a \nw{right split}; if it is as in the bottom arrow, it will be called a \nw{left split}. Left and right splits are collectively called \nw{parity splits}. In a more neutral way, we will refer to the property of a split being left, right or central also as its \nw{parity}; but we will not generate confusion.}
\end{figure}

\begin{defin}
A sequence of almost tracks $\bm\tau=(\tau_j)_{j=0}^N$, $N\in\mathbb N$, or $(\tau_j)_{j=0}^{+\infty}$, with all $\tau_j$ lying on the same surface $S$ is a \nw{splitting sequence} if, for all $0\leq j <N$, $\tau_{j+1}$ is obtained from $\tau_j$ via a comb, uncomb or split move. In the case of generic almost tracks, instead, we require each element of the sequence to be obtained from the previous one via a slide or a split (we may use the adjective \emph{generic} in reference to splitting sequences, too, in order to mark this difference). We denote with $|\bm\tau|$ the number of splits the splitting sequence $\bm\tau$ includes.

Given two indices $0\leq k\leq l$ ($\leq N$), we denote by $\bm\tau(k,l)=(\tau_j)_{j=k}^l$ the splitting sequence obtained from $\bm\tau$ by considering only the entries indexed by the indices between $k$ and $l$. When referring to splitting sequences, we use the term \nw{subsequence} only to refer to a sequence $\bm\tau(k,l)$ i.e. a subsequence never skips intermediate entries.

Given two splitting sequences $\bm\sigma$ and $\bm\tau$, with the last entry of $\bm\sigma$ isotopic to the first entry of $\bm\tau$, we denote by $\bm\sigma*\bm\tau$ the splitting sequence obtained by adjoining the entries of $\bm\tau$ at the end of $\bm\sigma$, in the given order and excluding the first one (so as not to have a repetition).
\end{defin}

Note that any semigeneric train track can be converted into a generic one via some uncomb moves. Actually, a splitting sequence of semigeneric train tracks can be converted into a splitting sequence of generic ones (i.e. not only each track may be made generic, but comb and uncomb moves may be replaced with slides in a suitable manner), and vice versa.

Combs, uncombs and slides are to be regarded as invertible and unessential moves, whereas splits in some sense `downgrade' the train track; this is made precise by the following statement, which collects Propositions 3.12.2 and 3.14.1 from \cite{mosher}:
\begin{prop}\label{prp:carriediffsplit}
Given two train tracks $\tau,\tau'$ on a surface $S$, $\tau'$ is fully carried by $\tau$ if and only if $\tau$ and $\tau'$ are, respectively, the beginning and the end of a splitting sequence.

Moreover, $\tau'$ is suited to $\tau$ if and only if $\tau$ and $\tau'$ are, respectively, the beginning and the end of a splitting sequence with no central splits.

Finally, $\tau'$ and $\tau$ carry each other if and only if $\tau$ and $\tau'$ are the extremes of a splitting sequence with no splits.
\end{prop}

\begin{rmk}\label{rmk:decreasingmeasures}
If a pretrack $\tau'$ is carried by another pretrack $\tau$, one can pick tie neighbourhoods $\nei(\tau')\subseteq \nei(\tau)$, with the ties of the former obtained as restriction of the ones of the latter. As a consequence, any curve that is carried by $\tau'$ is also carried by $\tau$: $\cc(\tau')\subseteq \cc(\tau)$. In the case of almost tracks, where uniqueness of carrying holds, given any $\alpha\in\cc(\tau')$ and the transverse measures $\mu'_\alpha$, $\mu_\alpha$ induced by it on $\tau'$, $\tau$ respectively, we have $\max_{b\in\br(\tau')}\left(\mu'_\alpha(b)\right)\leq \max_{b\in\br(\tau)}\left(\mu_\alpha(b)\right)$ and $\min_{b\in\br(\tau')}\left(\mu'_\alpha(b)\right)\leq \min_{b\in\br(\tau)}\left(\mu_\alpha(b)\right)$.

The above proposition, then, implies that the sequence of sets $\left(\cc(\tau_j)\right)_j$ along a splitting sequence is decreasing; and the equality $\cc(\tau_j)=\cc(\tau_{j+1})$ holds every time the two tracks are obtained from each other with an elementary move which is not a split.
\end{rmk}

\begin{rmk}\label{rmk:recurrence_at_extremes}
Let $\bm\tau=(\tau_j)_{j=0}^N$ be a train track splitting sequence.
\begin{itemize}
\item If $\tau_N$ is recurrent then so are all entries of $\bm\tau$.
\item If $\tau_0$ is transversely recurrent then so are all entries of $\bm\tau$ (cfr. \cite{penner}, Lemma 1.3.3(b)).
\end{itemize}
\end{rmk}

\begin{rmk}\label{rmk:pickparameters}
After introducing almost tracks and elementary moves, we can finally fix the parameters $k=k(S),\ell=\ell(S)$ employed in the definitions of $\ma(S)$ and the quasi-pants graph $\pa(S)$, as promised after Remark \ref{rmk:ell1}.

Here are the choices in detail for $\pa(S)$. We let $k'$ be the maximum self-intersection number of a collection $V(\tau)$ over all almost tracks $\tau$ on $S$ (which is finite because almost tracks are finitely many different ones up to diffeomorphisms of $S$; see also Lemma \ref{lem:vertexsetbounds}), and we let $k(S)\coloneqq k'$. This means that an almost track vertex set is a vertex of $\pa(S)$ if and only if it cuts $S$ as specified in either of the conditions a), b), c) of Definition \ref{def:quasipants}.

As for the choice of $\ell$, let $\ell_0$ be the maximum intersection number between $V(\tau)$ and $V(\sigma)$, for $\tau$ any almost track and $\sigma$ obtained from $\tau$ with a split or taking a subtrack (this is again finite because all possible pairs are finitely many up to diffeomorphism). Then define $\ell\coloneqq\max\{\ell_0,\ell_1\}$ for $\ell_1$ the one defined in Remark \ref{rmk:ell1}: this means that a splitting of almost tracks induces a displacement along an edge in $\pa(S)$.

When $X\subseteq S$ is a non-annular subsurface, we would like that, every time $V(\tau)$ is a vertex of $\pa(S)$, $\pi_X V(\tau)$ is a vertex of $\pa(X)$: so it will be understood that, when $X$ is regarded as a subsurface of $S$, then the bound on self-intersection number for vertices of $\pa(X)$ is not taken to be $k(X)$ but $k(X,S)\coloneqq \max\{k(X),4k(S)+4\}$ instead (this works, because of Remark \ref{rmk:subsurf_inters_bound}).

The choices for $\ma(S)$ shall be made with the same spirit: given any train track, if its vertex set fills $S$ then it must be a vertex of the graph. The vertex sets of any pair of train tracks related via an elementary move must be represented by vertices at distance $\leq 1$; and any vertex must have distance $\leq 1$ from a complete clear marking.

Again, we wish also that, when $X\subseteq S$ is a non-annular subsurface and $V(\tau)$ is a vertex of $\ma(S)$, then also $\pi_X V(\tau)$ is a vertex of $\ma(X)$. So, when $X$ is regarded as a subsurface of $S$, we pick a bound for the self-intersection number differently from the case in which $X$ is a stand-alone surface.

These choices for $\ma(S)$ and $\ma(X)$ do not coincide with the ones given at the beginning of \S 6 in \cite{mms} as they give higher constants, but they do not alter the validity of the results in that work, and in particular of their Theorem 6.1, as we will see.
\end{rmk}

\section{Train tracks: more constructions}\label{sec:traintracksmore}
\subsection{Lifting and inducing an almost track}\label{sub:induced}

If $\tau$ is an almost track on $S$, and $X\subseteq S$ is a subsurface, we denote with $\tau^X$ the pre-image of $\tau$ in $S^X$; it is a pretrack. The universal cover of $S$ will be denoted $\tilde S$; similarly, the notation $\tilde\tau$ will denote the pre-image of $\tau$ in $\tilde S$ via the covering map $\tilde S\rightarrow S$.

\begin{rmk}\label{rmk:negativeindexincover}
\textit{All connected components of $S^X\setminus \nei(\tau^X)$ or of $\tilde S\setminus\tilde\tau$ which are compact have negative index.}

The best way to see this (focusing on $S^X\setminus \nei(\tau^X)$, as the case of the universal cover is identical) is to take $\sigma$ an almost track which fills $S$, has $\tau$ as a subtrack, and has the property that each connected component of $S\setminus \nei(\sigma)$ which is a \emph{smooth} peripheral annulus is also a connected component of $S\setminus \nei(\tau)$.

Each compact connected component of $S^X\setminus \nei(\sigma^X)$ which is not a smooth peripheral annulus, then, is diffeomorphic to a suitable connected component of $S\setminus \nei(\sigma)$; as such, it has negative index.

So the compact connected components of $S^X\setminus \nei(\tau^X)$ are obtained from gluing a handful of negative index components of $S^X\setminus \nei(\sigma^X)$ with, possibly, some components as in Remark \ref{rmk:idx_of_nei_diff}. So they have negative index.
\end{rmk}

Some adaptations of the results from \S 3.3 in \cite{mosher} will be now given, to rule the behaviour of $\tilde\tau$ and $\tau^X$ at infinity.

\begin{prop}\label{prp:paths_in_univ_cover}
Let $\tau$ be an almost track on a surface $S$. Then
\begin{itemize}
\item all train paths along $\tilde\tau$ are embedded;
\item the images of any two train paths in $\tilde\tau$ intersect in a connected set.
\end{itemize}
\end{prop}

\begin{prop}
In the same setting as the above proposition,
\begin{itemize}
\item if $\rho, \rho'$ are two infinite train paths which have finite Haudorff distance in $\tilde S$, then they eventually coincide;
\item if $\rho, \rho'$ are two biinfinite train paths which have finite Haudorff distance in $\tilde S$, then they coincide entirely.
\end{itemize}
\end{prop}

The proof of these two propositions are exactly the same as Propositions 3.3.1 and 3.3.2 in \cite{mosher}.

\begin{prop}\label{prp:paths_quasi_geod}
Let $\tau$ be an almost track on a surface $S$. Then there exists $\lambda\geq 1$ (depending on $\tau$) such that every train path in $\tilde\tau$ is a $\lambda$-quasigeodesic in $\tilde S\cong\Hy^2$ with the hyperbolic metric.
\end{prop}
\begin{proof}[Sketch]
The proof of this statement is substantially the same as the proof of Proposition 3.3.3 in \cite{mosher}. Rather than repeating their proof, we give some instructions to adapt it to our setting.

Replace $\tau$ with an almost track which fills $S$, has the original $\tau$ as a subtrack, and chosen so that the collection of funnels with smooth boundary appearing as connected components of $S\setminus \nei(\tau)$ remains unvaried. By isotoping $\tau$, make sure that $\tau\subseteq \core(S)$ and that the closure of each connected component of $\core(S)\setminus \tau$ is a closed disc. This request is equivalent to demanding that $\tau\cup \partial\core(S)$ is a connected pretrack on $S$, entirely contained in $\core(S)$.

Let $\tilde K$ be the preimage of $\core(S)$ in the universal cover $\tilde S$: it is connected and convex, by definition of $\core(S)$. The same arguments as in the proof of Proposition 3.3.3 in \cite{mosher} can be used to show that:
\begin{itemize}
\item if $\core(S)$ is compact, then $\tilde\tau$ is quasi-isometric to $\tilde K$;
\item if $\core(S)$ has punctures, then there exists a pretrack $\sigma^\infty\subseteq \tilde K$ which contains $\tilde\tau$ as a subtrack and is quasi-isometric to $\tilde K$.
\end{itemize}

After that, one proves that any train path $\delta$ in $\tilde\tau$ is a quasigeodesic, following the original proof verbatim.
\end{proof}

As a consequence:
\begin{coroll}\label{cor:distinctends}
Let $\tau$ be an almost track on a surface $S$ and let $X\subseteq S$ be a subsurface. Every infinite train path in $\tau^X$ with noncompact image, and every infinite train path in $\tilde\tau$, has a unique limit point on $\partial\ol{S^X}$ ($\partial\ol{\tilde S}$, resp.).

If two paths as above have the same limit point, then they eventually coincide.

A biinfinite train path in $\tilde\tau$ has distinct endpoints on $\partial\ol{\tilde S}$, and there are no two distinct paths with the same pair of endpoints.
\end{coroll}

\begin{proof}
All the statements concerning train paths in $\tilde\tau$ are easy consequences of the above propositions.

As for an infinite path $\rho$ in $\tau^X$ with noncompact image: let $\pi_1(X)\cong\Gamma_X<\Isom(\tilde S)$ be the group such that $S^X=\tilde S/\Gamma_X$. Let $P$ be a disjoint, $\Gamma_X$-equivariant union of an open horoball for each point of $\partial\tilde S$ which is parabolic under the action of $\Gamma_X$, and with the property that $\tau^X\cap P=\emptyset$: $\left(H(L_{\Gamma_X})\setminus P\right)/\Gamma_X$ is a compact surface with boundary. 

Let $\tilde\rho$ be any lift of $\rho$ to $\tilde S$: $\tilde\rho$ is a quasigeodesic in $\tilde S$, as seen in the previous proposition. If its endpoint at infinity belongs to the limit set $L_{\Gamma_X}$ then there is an $\epsilon>0$ such that $\tilde\rho$, with the exception of an initial segment, falls entirely in the closed neighbourhood $N\coloneqq\bar\nei_\epsilon\left(H(L_{\Gamma_X})\right)$. Therefore $\rho$ lies eventually in the compact set $(N\setminus P)/ \Gamma_X$, which means that it is compact itself.

So the endpoint at infinity of $\tilde\rho$ belongs to the domain of discontinuity $D_{\Gamma_X}$; let $\tilde A$ be an open neighbourhood of this point such that $\tilde A\cap \tilde S$ gets mapped isometrically via the covering map $\tilde S\cup D_{\Gamma_X}\rightarrow \ol{S^X}$; and let $A$ be its image in $\ol{S^X}$. Then $\rho\cap A$, which excludes only an initial segment of $\rho$, is a quasigeodesic in $S^X$, and has a unique limit point on $\partial\ol{S^X}$.

If two infinite paths $\rho,\rho'$ in $\tau^X$ with noncompact image approach the same endpoint in $\partial\ol{S^X}$, then one may choose lifts $\tilde\rho,\tilde\rho'$ to $\tilde S$ which also approach the same point in $\partial\ol{\tilde S}$, and so they eventually coincide. But then, $\rho, \rho'$ also do.
\end{proof}

\begin{rmk}\label{rmk:limitsetconsistency}
If $\tau$ carries $\tau'$, then all train paths in $(\tau')^X$ are carried by $\tau^X$, too.

In particular, the set of points of $\partial S^X$ which are endpoints for some infinite, noncompact train path in $(\tau')^X$ is a subset of the similarly defined set for $\tau^X$.
\end{rmk}

The concept of subsurface projection finds a version for train tracks in \cite{mms}. Before we introduce it, we need a slight generalization of Definition \ref{def:carried}. If $X$ is an annular subsurface of a surface $S$ and $\beta\in \cc(X)$, a \emph{carried realization} $f:\beta \hookrightarrow \bar\nei(\tau^X)$ is specified by the same hypotheses as in the definition of carried realization of a curve, plus the extra one that the endpoints of $f(\beta)$ in $\partial\ol{S^X}$ are the same as the ones of $\beta$. This gives sense to the adjective \emph{carried}, hence to the notation $\cc(\tau^X)$ for the subset of $\cc^0(X)$ consisting of all arcs carried by $\tau^X$. We can adapt similarly the content of Definition \ref{def:trainpathrealization} to speak of a \emph{train path realization} of an arc $\beta$ carried by $\tau^X$.

\begin{defin}
Given $\tau$ an almost track on a surface $S$ and $X\subseteq S$ a subsurface, the \nw{track induced by $\tau$} on $X$ is $\tau|X$, the subtrack of $\tau^X$ consisting of those branches belonging to the carrying image of some element of $\cc(\tau^X)$.
\end{defin}

Note that $\cc(\tau|X)=\cc(\tau^X)$ and that, if $X$ is not an annulus, then $\cc(\tau|X)=\cc(\tau)\cap\cc^0(S^X)= \cc(\tau)\cap\cc^0(X)$. If $X$ is an annulus, then each element of $\cc(\tau)$ which intersects $X$ essentially lifts to an element of $\cc(\tau|X)$, but it is not true in general that an element of $\cc(\tau|X)$ projects to an element of $\cc(\tau)$.

Following \cite{mms}, we define $V(\tau|X)\subset \cc(\tau|X)$ to be the set of \emph{wide} carried arcs. Again, the definition of wide carried arc here is not any different from the definition of wide carried curve in an almost track.

\begin{lemma}\label{lem:inducedisonsurface}
If $X$ is not an annulus, then the induced track $\tau|X$ is an almost track on $S^X$. Moreover, if $r:\ol{S^X}\rightarrow \core(S^X)=X\cap\core(S)$ is a retraction, then $r(\tau|X)\eqqcolon\rho$ is a pretrack contained in $\inte(X)$ which is an almost track in $S$, too.
\end{lemma}

Let $p:S^X\rightarrow S$ be the covering map. In the statement we have already implicitly identified $p^{-1}(X)$ with $X$, and $\core(S^X)$ with $X\cap \core(S)$, using the fact that these pairs are isometric via $p$. As it is noted in \cite{mms}, Lemma 3.1 and subsequent remarks, $\tau|X$ need not be a train track when $\tau$ is.

\begin{proof}
We prove directly that $\rho$ is an almost track in $S^X$; clearly this will mean that $\tau|X$ is, too. Also, we may suppose that $X$ has geodesic boundary\footnote{If $X$ has two isotopic boundary components in $S$, their geodesic representatives end up coinciding. But this is no cause for concern, as the boundary components of $X\subseteq S^X$ are pairwise not isotopic.}. 

We claim that the compact connected components of $\core(S^X)\setminus\nei(\rho)$ either have negative index or are compact peripheral annuli in $X$. If one of these components, $C$ say, were not either of these two alternatives, then $\idx(C)\geq 0$ and $C$ would lie away from $\partial \core(S^X)$. But then, consider $r^{-1}(C)$: it would be contained in $S^X$, hence it would be a compact connected component of $S^X\setminus\nei(\tau^X)$. So $\idx\left(r^{-1}(C)\right)<0$, because it would be the gluing of patches as in Remark \ref{rmk:negativeindexincover} plus other ones ruled by Remark \ref{rmk:idx_of_nei_diff}. This is a contradiction.

We will now exclude that $\core(S^X) \setminus \nei(\rho)$ includes a connected component which is a cusp $P$ with smooth boundary $\partial P$. If there is one, then there are a periodic train path in $\rho$, and one in $\tau|X$, which (up to the due reparametrization) are isotopic to $\partial P$; necessarily, there is no closed geodesic in $S^X$ which is isotopic to $\partial P$. 

There is a chain of inclusions of closed subsets $p(P)\subseteq p\left(\core(S^X)\right)\subseteq \core(S)\subseteq S$ (again, remember that $p(P)\cong P$ and $p\left(\core(S^X)\right)\cong \core(S^X)$); note that there cannot be any closed geodesic in $S$ which is isotopic to $\partial p(P)$, either.

On the other hand, $\partial P=p(\partial P)$ is also an embedded loop in $S$, isotopic to a (reparametrized) train path along $p(\tau^X)\subseteq \tau$. However, since $\tau$ is an almost track, is does not admit periodic train paths encircling cusps, hence a contradiction.

The argument carried out so far proves that any smooth connected component of $\partial\bar\nei(\rho)$ which is inessential in $\core(S^X)$ (which is the same as saying, inessential in $S^X$) is parallel to a closed geodesic, one of the connected components of $\partial\core(S^X)$. This means that $\rho$ is an almost track in $S^X$, and so is $\tau|X$.

Now we prove that $p(\rho)$ (which can be identified with $\rho$ itself) is an almost track on $S$. Each connected component of $S\setminus p\left(\nei(\rho)\right)$ is either contained in $X$ and, in this case, it is a copy of a connected component of $p^{-1}(X)\setminus \nei(\rho)$; or it contains entirely a closed geodesic, a connected component of $\partial X$. None of these complementary components may transgress the conditions that make $p(\rho)$ an almost track in $S$.
\end{proof}

\begin{rmk}\label{rmk:annulusinducedbasics}
Let $\gamma\in\cc(S)$, let $X$ be a regular neighbourhood of $\gamma$ and let $\tau$ be an almost track on $S$. We list here a few basic facts about the induced track $\tau|X$.

\begin{enumerate}
\item If $\gamma$ admits a carried realization in $\tau|X$ then it admits one in $\tau$. Note, though, that the first one is certainly embedded as a train path (see below), whereas the latter need not be. The converse is not true in general: if $\gamma$ is carried by $\tau$ but $\cc(\tau^X)=\emptyset$ then $\gamma$ is not carried by $\tau|X$. On the other hand, as we note in point \ref{itm:gammacarriedininduced} below, $\gamma$ is carried by $\tau|X$ if $\cc(\tau^X)\not=\emptyset$.

\item\label{itm:embeddedcore} \textit{When $\gamma$ is carried by $\tau^X$, it is embedded as a train path. The realization as a train path is unique up to obvious reparametrizations.}

Uniqueness of carrying of $\gamma$ provided the existence of a carried realization is just a consequence of the uniqueness of carrying of $\gamma$ in $\tau$ (Corollary \ref{cor:carryingunique}). 

We prove that the train path realization of $\gamma$ is embedded by contradiction. Suppose first that $\tau^X.\gamma$ is not a simple curve: then (forgetting about it being a pretrack) it is a compact 1-complex with two or more distinct simple cycles. As $S^X$ is a planar surface, this means that $\ol{S^X}\setminus\nei(\tau^X.\gamma)$ consists of at least $3$ connected components. Considering that $\tau^X.\gamma$ keeps away from $\partial\ol{S^X}$, and that $\gamma$ is not nullhomotopic in $S^X$, two of these components include each one component of $\partial\ol{S^X}$ and are topological annuli, possibly with a bunch of outward right angles in their boundaries. So their index is not positive.

Any other connected component $C$ is also a compact connected component of $S^X\setminus\nei(\tau^X.\gamma)$, and it must have $\idx(C)<0$ because it is a gluing of regions as in Remarks \ref{rmk:negativeindexincover} and \ref{rmk:idx_of_nei_diff}. This means that $\ol{S^X}$ is a gluing of regions with index $\leq 0$, and one of them has certainly index $<0$. But this is impossible since $\idx(\ol{S^X})=0$.

So $\tau^X.\gamma$ is a simple curve, not nullhomotopic in $S^X$. If $\gamma$ traversed any branch there more than once, it would traverse at least twice all of them, and always keeping the same orientation. But this means that $\gamma$ would not be a generator for $\pi_1(S^X)$, thus it contradicts the definition of $X$.

Before we continue with more observations, we give a definition:
\begin{defin}\label{def:eorientation}
Let $\tau$ be an almost track, let $\gamma\in \cc(S)$ and let $X$ be a regular neighbourhood of $\gamma$. Suppose that $\gamma$ is carried by $\tau$ (and by $\tau^X)$ and that $e$ is a branch end of $\tau^X$ such that $e\cap \tau^X.\gamma=\{v\}$, where $v$ is the switch of $\tau^X$ which serves as endpoint of $e$. Consider a train path $\beta$ in $\tau^X$ which traverses the branch containing $e$ pointing towards $v$, and then follows $\gamma$ until it reaches $v$ again. The \nw{$e$-orientation} on $\gamma$ is the orientation specified by the path $\beta$.
\end{defin}

\item\label{itm:uniquecarrying} \textit{Uniqueness of carrying on $\tau^X$ for an element $\alpha\in \cc(\tau^X)$ holds, similarly as for train tracks and almost tracks.}

It is enough to show that there is only one train path realization of $\alpha$, up to reparametrization. Let then $\ul\alpha_1,\ul\alpha_2$ be two train path realizations of $\alpha$: they have the same endpoints on $\partial\ol{S^X}$. Consider each of these two realizations as the union of two infinite train paths approaching opposite components of $\partial\ol{S^X}$: then, by Corollary \ref{cor:distinctends}, they coincide outside a compact subset of $S^X$.

Since $\ul\alpha_1,\ul\alpha_2$ are homotopic arcs, there are lifts $\tilde{\ul\alpha}_1,\tilde{\ul\alpha}_2$ of the two on $\tilde S$ which approach the same pair of points at infinity. By Corollary \ref{cor:distinctends} $\tilde{\ul\alpha}_1,\tilde{\ul\alpha}_2$ must coincide, and so do $\ul\alpha_1,\ul\alpha_2$.

\item\label{itm:windaboutgamma} \textit{Let $\alpha\in\cc(\tau^X)$. If $\alpha$ traverses at least one of the branches of $\tau^X$ twice, then $\gamma$ is carried by $\tau^X$ and all branches traversed more than once by $\alpha$ must belong to $\tau^X.\gamma$.}

The carrying image $\tau^X.\alpha$ is a generalized pretrack. Among the connected components of $S^X\setminus\nei(\tau^X.\alpha)$ there is no topological closed disc: if there is one, which we call $D$, then lift $\alpha$ to an arc $\tilde\alpha$ in $\tilde S$. This is homotopic, with fixed extremes, to a biinfinite train path along $\tilde\tau$: let $\tilde\tau.\tilde\alpha$ be its image. $D$ lifts homeomorpically to a connected component of $\tilde S\setminus\nei(\tilde\tau.\tilde\alpha)$; but the presence of a topological disc among them is a contradiction to Proposition \ref{prp:paths_in_univ_cover}, according to which train paths along $\tilde\tau$ are embedded.

Encircle each component of $\partial\ol{S^X}$ with a smooth loop which intersects $\nei(\tau^X.\alpha)$ in exactly one tie. Let $Y$ be the compact annulus delimited by these two loops in $S^X$. As a consequence of the previous paragraph, each connected component of $Y\setminus \nei(\tau^X.\alpha)$ includes part of $\partial Y\setminus\nei(\tau^X.\alpha)$, which consists of exactly 2 components. The connected components of $Y\setminus \nei(\tau^X.\alpha)$, then, are at most 2.

If $Y\setminus \nei(\tau^X.\alpha)$ is connected, then $\idx\left(Y\setminus \nei(\tau^X.\alpha)\right)=0$, by index additivity since $\idx(Y)=\idx\left(\bar\nei(\tau^X.\alpha)\right)=0$. This means that $Y\setminus \nei(\tau^X.\alpha)$ is a rectangle i.e. $\partial\bar\nei(\tau^X.\alpha)$ consists of two smooth components, and they are both isotopic to $\alpha$: so $\alpha$ will be embedded.

If, instead, the components of $Y\setminus \nei(\tau^X.\alpha)$ are two, then each of them is necessarily homeomorphic to a disc. So $\left((\tau^X.\alpha)\cap Y\right)\cup\partial Y$ is a 3-valent graph, hence with the property that $2\#\text{(edges)}=3\#\text{(vertices)}$, which cuts the annulus $Y$ into two cells. 

Computation of the Euler characteristic yields that the graph has $6$ edges and $4$ vertices. This implies that $\tau^X.\alpha$ is a generalized pretrack with two switches, two compact branches, and two more branches heading towards the two opposite components of $\partial\ol{S^X}$.

\begin{figure}[h]
\centering
\includegraphics[width=.25\textwidth]{arcinannularcover.pdf}
\caption{\label{fig:arcinannularcover}The only possibility for $\tau^X.\alpha$ (Remark \ref{rmk:annulusinducedbasics}, point \ref{itm:windaboutgamma}) provided that $\alpha$ does not have an embedded realization as a train path, up to a diffeomorphism of $S^X$ (possibly an orientation-reversing one). The arc $\alpha$ may wind around the core curve an arbitrarily high number of times.}
\end{figure}

There is only one diffeomorphism type of generalized pretrack in $S^X$ with these properties and that carries a path from one component of $\partial\ol{S^X}$ to the other, traversing all branches of the pretrack: $\tau^X.\alpha$ must belong to this class (Figure \ref{fig:arcinannularcover}). So $\tau^X.\alpha$, in particular, carries $\gamma$.

Necessarily $\alpha$, after traversing the first infinite branch of $\tau^X.\alpha$, winds around $\tau^X.\gamma$ traversing each branch always in the same direction, and eventually leaves it to traverse the remaining infinite branch of $\tau^X.\alpha$.

The structure of an arc $\alpha\in \cc(\tau^X)$ when $\gamma$ is carried can therefore be summarized as follows.

\item \label{itm:horizontalstretch} \textit{Suppose that $\tau^X$ carries $\gamma$. Let $\alpha\in\cc(\tau^X)$ and let $\ul\alpha$ be a realization of $\alpha$ as a train path along $\tau^X$. Then $\ul\alpha$ consists of the concatenation of three segments $\rho_1,\beta,\rho_2$ such that: $\beta$ traverses branches of $(\tau|X).\gamma$, always in the same direction; $\rho_1$ and $\rho_2$ are embedded train paths, connecting $\tau^X.\gamma$ with distinct components of $\partial\ol{S^X}$ and not traversing any branch in $\tau^X.\gamma$. The branch ends of $\rho_1\rho_2$ which are adjacent to $\tau^X.\gamma$ induce opposite orientations on $\gamma$.}

We will denote $\beta=\hs(\tau,\alpha)$ and call it the \nw{horizontal stretch} of $\alpha$ in $\tau$. We have that $\alpha\in V(\tau^X)$ if and only if $\hs(\tau,\alpha)$ is embedded. In this case we may abusively identify $\hs(\tau,\alpha)$ with its image $\tau^X.\gamma\cap\tau^X.\alpha$.

The paths $\rho_1,\rho_2$ also deserve a name. When $\tau^X$ carries $\gamma$, we call an \nw{outgoing ramp} a train path $\rho:[0,+\infty)\rightarrow \tau^X$ such that $\rho(0)\in\tau^X.\gamma$, and converging to a point in $\partial\ol{S^X}$ without traversing any branch in $\tau^X.\gamma$. An \nw{ingoing ramp} is a train path $\rho:(-\infty,0]\rightarrow \tau^X$ such that the map $\rho'(x)\coloneqq \rho(-x)$ is an outgoing ramp.

\item\label{itm:dataforramp} \textit{Suppose two distinct outgoing ramps for $\tau^X$ diverge towards the same point of $\partial\ol{S^X}$. Then their initial branch ends give opposite orientations to $\gamma$.}

Let $\rho_1,\rho_2:[0,+\infty)\rightarrow \tau^X$ be the two ramps. By Corollary \ref{cor:distinctends} they eventually coincide so the subtrack $\sigma=(\tau^X.\gamma)\cup\mathrm{im}(\rho_1)\cup\mathrm{im}(\rho_2)$ has, among the components of $S^X\setminus\nei(\sigma)$, a compact component which is a compact topological disc. It has at least two outward corners in the boundary --- at the confluence of $\rho_1,\rho_2$ --- but, as it is a union of components as in Remarks \ref{rmk:idx_of_nei_diff} and \ref{rmk:negativeindexincover}, its index is negative and the corners in the boundary must be at least three more. This is possible only if it has corners both at $\rho_1(0)$ and at $\rho_2(0)$, which means that their initial branch ends induce opposite orientations on $\gamma$.

\item \textit{The elements of $V(\tau^X)$ are exactly the ones of $\cc(\tau^X)$ which are embedded as train paths and, if $\tau|X$ does not carry $\gamma$, then $V(\tau|X)=\cc(\tau|X)$} (cfr. Lemma 3.5 in \cite{mms}). \textit{If $\tau|X$ carries $\gamma$ then, for any $\alpha\in\cc(\tau|X)$, there exist $\beta\in V(\tau|X)$, $m\in\mathbb Z$, such that $\alpha=D_X^m(\beta)$. Here $D_X$ is the self-diffeomorphism \emph{of $S^X$} given by the Dehn twist about its core curve $\gamma$.}

The first sentence is just a direct consequence of point \ref{itm:windaboutgamma} above. As for the second one: consider the decomposition given in the above point. So $\alpha$ fails to be wide carried if and only if $\hs(\tau,\alpha)$ is not an embedded train path, and this occurs if and only if it traverses all branches of $\tau^X.\gamma$. But then one between $D_X(\ul\alpha)$ and $D_X^{-1}(\ul\alpha)$ traverses each branch of $\tau^X.\gamma$ once less than $\ul\alpha$. Repeat Dehn twisting until the arc thus obtained has an embedded horizontal stretch.

\item \label{itm:gammacarriedininduced} A similar argument as in the point above yields a weaker form for the missing implication in point 1: if $\gamma$ is carried by $\tau$ (hence by $\tau^X$, too) and $\cc(\tau^X)\not=\emptyset$ then $\gamma$ is carried by $\tau|X$. Given any $\alpha\in\cc(\tau^X)$ then, up to replacing $\alpha$ with the correct $D_X^{\pm 1}(\alpha)\in\cc(\tau^X)$, we can suppose that it traverses all branches of $\tau^X.\gamma$. So those branches are part of $\tau|X$.

\item \label{itm:diambound}\textit{Given any two $\alpha_1,\alpha_2\in V(\tau|X)$, $i\coloneqq i(\alpha_1,\alpha_2)\leq 4$ --- hence $d_{\cc(X)}(\alpha_1,\alpha_2)\leq 5$.}

Consider any two carried realizations $\ul\alpha_1,\ul\alpha_2$ which realize the intersection number $i$ between their isotopy classes; we identify them with their images in $S^X$. We may suppose that $i>0$. Perform a series of isotopies on $\ul\alpha_1,\ul\alpha_2$, transversely to the ties of $\nei(\tau|X)$, resulting in two new realizations $\ul\alpha'_1,\ul\alpha'_2$. These isotopies shall be performed in such a way that each transverse intersection between $\ul\alpha_1$ and $\ul\alpha_2$ turns into an entire \emph{segment} of intersection between $\ul\alpha'_1$ and $\ul\alpha'_2$, without introducing new connected components of intersection. Informally, we may say that the local picture around each component of $\ul\alpha'_1\cap\ul\alpha'_2$ is the same as the one around a large branch in a pretrack.

Let then $\sigma\coloneqq \ul\alpha'_1\cup\ul\alpha'_2$: it is a generalized pretrack, carried by $\tau|X$. It has no mixed branches: the existence of one would imply that one of the two arcs traverses some branch of $\tau|X$ more than once and this contradicts the fact that they are embedded as train paths. Fix a tie neighbourhood $\bar\nei(\sigma)\subseteq \nei(\tau|X)$: then each compact connected component $C$ of $S^X\setminus \nei(\sigma)$ has negative index because, as $\alpha_1,\alpha_2$ traverse each branch of $\tau|X$ at most once and we have chosen representatives which intersect minimally, $C$ includes at least one compact component of $S^X\setminus\nei(\tau|X)$; Remarks \ref{rmk:idx_of_nei_diff} and \ref{rmk:negativeindexincover} apply. So this means that, if a component of $S^X\setminus \sigma$ has compact closure, then it is a topological disc and contains at least three outward cusps in its boundary.

Now, $(\ul\alpha'_1\cup\ul \alpha'_2\cup \partial Y)\cap Y$ as a graph has $2i+4$ vertices (out of which $4$ lie along $\partial Y$ and the others are switches), and they are all trivalent. The edges of the graph are $i+2(i+1)+4$ (here they are counted as large branches + small branches + edges $\partial Y$ is cut into). By an Euler characteristic computation, this graph cuts $Y$ into $i+2$ regions, each homeomorphic to a disc. Out of these, exactly $4$ have an edge along $\partial Y$; so the remaining $i-2$ ones are identifiable with the connected components of $S^X\setminus\sigma$ with compact closure.

We have noted that each of these latter regions has at least $3$ outward cusps in its boundary; moreover, out of the four regions which are adjacent to $\partial Y$, exactly two must contain an outward cusp. In total, the number of outward cusps in $\bar\nei_0(\sigma)$ is $2i$, because each large branch of $\sigma$ provides two of them. Hence $2i\geq 3(i-2)+2$ and $i\leq 4$.
\end{enumerate}
\end{rmk}

\begin{lemma}\label{lem:decreasingfilling}
Let $\tau$ be an almost track on $S$. Then the subsurface $S'$ of $S$ --- not necessarily a connected one, and possibly including annular components --- filled by the curves in $V(\tau)$ is also the subsurface filled by $\cc(\tau)$.

Moreover, if $C^1,\ldots, C^k$ are the connected components of $S'$ then, for each $i$ such that $C^i$ is non-annular, $V(\tau|C^i)=V(\tau)\cap\cc(C^i)$.
\end{lemma}

\begin{proof}
\step{1} case of $\tau$ a train track.

Use the notation $\tau\|i$ to mean $\tau|C^i$ if $C^i$ is not an annulus, and the core curve of $S^{C^i}$ otherwise.

We identify each $C^i$ with its homeomorphic copy in $S^{C^i}$: this copy is a deformation retract of $\ol{S^{C^i}}$. With this convention, for each $v\in V(\tau)$ there is exactly one $i=i(v)$ such that $C^i$ contains a curve isotopic to $v$, and this curve either is essential there (for $C^i$ not an annulus) or is the core curve (for $C^i$ an annulus). Also, $v$ is carried by the respective $\tau\|i(v)$, enforcing the identification $\cc(S^{C^{i(v)}})=\cc(C^{i(v)})$.

For each $1\leq i\leq k$, let $Y^i\subset S^{C^i}$ be a surface (\emph{not} an essential subsurface of $S^{C^i}$ according to the definition used so far), defined as follows:
\begin{itemize}
\item if $C^i$ is not an annulus, fix a collection of peripheral annuli in $S^{C^i}$, one for each topological puncture, disjoint from each other and from $\bar\nei(\tau\|i)$; and remove one point from each of them (similarly as it is done in Corollary \ref{cor:carryingunique});
\item if $C^i$ is an annulus, remove one point from each connected component of $C^i\setminus \tau\|i$.
\end{itemize}
 
Let $\tau^i=\tau\|i$ setwise, but considered as a train track on $Y^i$. Then we may identify ${\mathcal M}_{\mathbb Q}(\tau^i)={\mathcal M}_{\mathbb Q}(\tau\|i)$ and there is a natural, linear map $f^i:{\mathcal M}_{\mathbb Q}(\tau^i)\rightarrow {\mathcal M}_{\mathbb Q}(\tau)$.

For each element $v \in V(\tau)$, $\tau^{i(v)}$ carries a unique curve $\tilde v$, essential in $Y^{i(v)}$, such that the chain of natural maps $Y^{i(v)}\hookrightarrow S^{C^{i(v)}} \twoheadrightarrow \inte(C^{i(v)})\hookrightarrow S$ sends $\tilde v$ to a loop homotopic to $v$.

For each $\alpha\in\cc(\tau)$ we denote $\mu_\alpha$ the corresponding measure in ${\mathcal M}_{\mathbb Q}(\tau)$; while, for $\beta\in\cc(\tau^i)$, we denote $\nu^i_\beta$ the corresponding measure in ${\mathcal M}_{\mathbb Q}(\tau^i)$ (see Definition \ref{def:transversemeasure}). For all $v\in V(\tau)$, $f^{i(v)}(\nu^{i(v)}_{\tilde v})=\mu_v$, implying in particular that $\tilde v\in V(\tau^{i(v)})$: if $\nu^{i(v)}_{\tilde v}$ could be written nontrivially as a sum in ${\mathcal M}_{\mathbb Q}(\tau^i)$, then also $\mu_v$ could be written nontrivially as a sum in in ${\mathcal M}_{\mathbb Q}(\tau)$. If $C^{i(v)}$ is not an annulus then, by definition of $i(v)$, $v\in \cc(\tau\|i(v))$ and $\nu^{i(v)}_v=\nu^{i(v)}_{\tilde v}$ under the identification ${\mathcal M}_{\mathbb Q}(\tau^i)={\mathcal M}_{\mathbb Q}(\tau\|i)$, so $v\in V(\tau\|i(v))$. In particular this proves the inclusion $V(\tau\|i)\supseteq V(\tau)\cap \cc(C^i)$ for all $1\leq i\leq k$ such that $C^i$ is not an annulus.

Fix now $\alpha\in \cc(\tau)$; then in ${\mathcal M}_{\mathbb Q}(\tau)$ there is an equality $\mu_\alpha=\sum_{v\in V(\tau)} a_v\mu_v$ for $a_v\in {\mathbb Q}_{\geq 0}$; note that $\mu_\alpha=\sum_{i=1}^k f^i(\nu^i)$ where $\nu^i\in {\mathcal M}_{\mathbb Q}(\tau^i)$, $\nu^i= \sum_{v\in V(\tau) \text{ s.t. } i(v)=i} a_v\nu_{\tilde v}$.

As a consequence of Proposition \ref{prp:measurecurvecorresp}, for each $1\leq i\leq k$ it is also possible to write $\nu^i=\sum_{j=1}^{s(i)} c^i_j \nu_{\beta^i_j}$ for $c^i_j\in{\mathbb Q}_{> 0}$ and $\{\beta^i_1,\ldots,\beta^i_s\}\subset \cc(\tau^i)$ a family of pairwise disjoint curves. But each train path realization $\ul\beta^i_j$ of $\beta^i_j$ in $\tau^i$ is also a train path in $\tau\|i$; as such, it is homotopic to an embedded loop $\hat \beta^i_j$, entirely contained in $C^i$, and possibly homotopic into a puncture of $C^i$.

The covering map $\hat p^i:S^{C^i}\rightarrow S$ turns this into a homotopy in $S$. So each $\hat p^i(\hat \beta^i_j)$, which can be considered to coincide with $\hat \beta^i_j$, is a simple closed curve in $S$, homotopic to the periodic train path $\hat p^i(\ul\beta^i_j)$ along $\tau^i$. Also, $\hat p^i(\hat \beta^i_j)$ is an essential curve in $S$, as no train path in $\tau$, which is a train track, may represent a null-homotopic or puncture-homotopic curve.

For each $i$, $j$, let $\alpha^i_j$ be the isotopy class of $\hat p(\hat\beta^i_j)$. This defines a family of isotopy classes of curves in $S$ which are admit pairwise disjoint realizations, and each of them may be supposed to lie entirely in the respective $C^i$. So $\mu_\alpha=\sum_{i=1}^k f^i(\nu^i) = \sum_{i=1}^k \sum_{j=1}^{s(i)} c^i_j \mu_{\alpha^i_j}$. The bijection guaranteed by Proposition \ref{prp:measurecurvecorresp}, then, implies that this double summation has only one entry, identical to the original $\mu_\alpha$. This implies that $\alpha$ is contained in $S'$, so $\cc(\tau)$ fills $S'$ as well as $V(\tau)$.

If $\alpha\in V(\tau\|i)$ for $C^i$ not an annulus, suppose that $\alpha\not\in V(\tau)$. Anyway $\alpha\in\cc(\tau)$, and the argument above proves, in particular, that in the expression $\mu_\alpha=\sum_{v\in V(\tau)} a_v\mu_v$ the coefficients $a_v\not=0$ all have the same $i(v)\eqqcolon i$. But then $\nu^i_\alpha=\sum_{v\in V(\tau)\cap \cc(C^i)} a_v\nu_v$ implying that only one $a_v\not=0$ because $\alpha$ is a vertex cycle for $\tau\|i$. So $V(\tau\|i)\subseteq V(\tau)\cap \cc(C^i)$.

\step{2} $\tau$ is only an almost track on $S$.

Similarly as above, let $T$ be the surface obtained from $S$ by picking a collection $P$ of peripheral annuli, one for each topological puncture, disjoint from each other and from $\bar\nei(\tau)$, and removing one point from each of them. Then $\tau_T=\tau$ is a train track on $T$, and $\cc(\tau_T)$ and $V(\tau_T)$ fill the same, possibly disconnected, subsurface $T'$ of $T$; up to isotopies, we may suppose that, every time a connected component of $\partial T'$ is isotopic to one of $\partial P$, they actually coincide. Let then $T''$ be the subsurface of $S$ consisting of all connected components of $T'\cup P$ which are not peripheral annuli. There is a natural bijection between the connected components of $T'$ and the ones of $T''$, and we claim that $T''$ is the subsurface of $S$ filled by both $V(\tau)$ and $\cc(\tau)$.

By definition, $V(\tau)$ may be identified with the subset of $V(\tau_T)$ consisting of all curves in this latter set which are not homotopically trivial in $S$; i.e. the ones that are not homotopic to a connected component of $\partial P$. An application of the same idea as in the proof of Corollary \ref{cor:carryingunique} guarantees that, among these elements of $V(\tau_T)$, no two distinct ones become isotopic in $S$. Fix an embedding in $T'$ for all elements in $V(\tau_T)$.

We prove first that, given any curve $\alpha\in\cc(T'')\subseteq \cc(S)$ there is an element of $V(\tau)$ that intersects it essentially. Isotope $\alpha$ so that it lies entirely in $T'$, and to be disjoint from all the chosen embeddings in $T'$ of elements in $V(\tau_T)\setminus V(\tau)$. Consider what happens in the surface $T$: since $V(\tau_T)$ fills $T'$, there exists an element of $V(\tau_T)$ which intersects $\alpha$ however $\alpha$ is isotoped within $T'$. This element is necessarily an element of $V(\tau)$, because we have already excluded the remaining elements of $V(\tau_T)$.

Now note that each element of $\cc(\tau)$ admits an embedding contained in $T'$, and is essential in $S$: so it has actually an embedding in $T''$. This means that $\cc(\tau)$ fills $T''$ again, because it cannot fill any higher complexity surface properly containing $T''$.

It has been proved above that, for any $Z'$ non-annular connected component of $T'$, $V(\tau_T|Z')=V(\tau_T)\cap \cc(Z')$. Let $Z''$ be the corresponding connected component of $T''$: then $(T^{Z'},\tau_T|Z')$ and $(S^{Z''},\tau|Z'')$ are easily identified. Discarding, in both sides of the given equality, any curve homotopic to a component of $\partial P$, we have $V(\tau|Z'')=V(\tau)\cap \cc(Z'')$ as required.
\end{proof}

\begin{rmk}
For $Y\subseteq X$ nested subsurfaces of $S$, $\tau$ an almost track on $S$, $\bm\tau=(\tau_j)_{j=1}^N$ a splitting sequence of almost tracks, the following facts, following partly from the above lemma, hold. Recall that the parameters for $\pa$ and $\ma$, chosen in Remark \ref{rmk:pickparameters}, are suited to `work well' with subsurface projections.

\begin{itemize}
\item If the collection of curves $\pi_X\left(V(\tau)\right)$ fills $X$, or the collection $V(\tau)$ does, then $\pi_Y\left(V(\tau)\right)$ fills $Y$. Also, if $V(\tau)$ or $\pi_X\left(V(\tau)\right)$ is a vertex of $\pa(X)$, then $\pi_Y\left(V(\tau)\right)$ is a vertex of $\pa(Y)$. These two facts are seen more easily using the version a) in Definition \ref{def:quasipants} and the definition of filling in terms of constraints on the intersection pattern between curves; the self-intersection number of the considered families of curves is no cause for concern, thanks to the choices made in Remark \ref{rmk:pickparameters}.
\item For $1\leq j\leq N$, the sets $\cc(\tau_j)$ make up a decreasing family (Remark \ref{rmk:decreasingmeasures}): so the lemma yields that, along the sequence $\bm\tau$, the subsurfaces of $S$ filled by $V(\tau_j)$ are also a decreasing family with respect to inclusion. If $X$ is a fixed subsurface of $S$, the same is true of the subsurfaces of $S^X$ (or of $X$ itself, via Lemma \ref{lem:inducedisonsurface}) filled by $V(\tau_j|X)$.
\item If, along a splitting sequence, for a fixed subsurface $X\subseteq S$, $\pi_X V(\tau_j)$ is a vertex of $\ma(X)$, then all $\pi_X V(\tau_{j'})$, for $j'<j$, are. And similarly with $\pa(S)$, using the version c) in Definition \ref{def:quasipants}.
\item The same as above is true for $V(\tau_j|X)$ instead of $\pi_X V(\tau_j)$. In particular the indices $j$ such that $V(\tau_j|X)$ is a vertex of $\pa(X)$ and of $\ma(X)$ include all the ones in the accessible interval $I_X$ (see \S \ref{sub:goodbehaviour} below).
\item The surface filled by $V(\tau|X)$ is a subsurface of the one filled by $\pi_X V(\tau_j)$.
\end{itemize}
\end{rmk}

The second statement of Lemma \ref{lem:decreasingfilling} may be slightly generalized: let $X\subseteq S$ be a subsurface such that $\partial X$ is essentially disjoint from $S'$. Then $V(\tau|X)=V(\tau)\cap\cc(X)$. The proof is the same as the one developed above, except that $X$ replaces all connected components of $S'$ which it contains.

\begin{lemma}\label{lem:induction_vertices_commute}
Let $X\subset S$ be a subsurface, and let $\tau$ be an almost track on $S$.
\begin{enumerate}
\item For $X$ not an annulus, let $\gamma\in W(\tau|X)$, $\delta\in \pi_X W(\tau)$, and suppose that $Y\subseteq X$ is a non-annular subsurface such that $\pi_Y(\gamma),\pi_Y(\delta)\not=\emptyset$: then 
$$d_{\cc(X)}(\gamma,\delta)\leq F(8 N_1(S^X))\quad\text{and}\quad d_Y(\gamma,\delta)\leq F\left(32 N_1(S^X) + 4\right)$$
where the function $F$ is the one defined by Lemma \ref{lem:cc_distance} and $N_1$ is defined in Lemma \ref{lem:vertexsetbounds}.
\item If $X$ is an annulus, then $\pi_X W(\tau)\subseteq V(\tau|X)$.
\item There is a bound $C_1=C_1(S)$ such that the following is true, for $X$ not an annulus. If $V(\tau|X)$ is a vertex of $\pa(X)$ then also $\pi_X V(\tau)$ is; and\linebreak $d_{\pa(X)}\left(\pi_X V(\tau),V(\tau|X)\right)\leq C_1$. Similarly for $\ma(X)$ instead of $\pa(X)$.
\end{enumerate}
\end{lemma}
In the statement of this lemma we are implicitly using the identification of $\cc(X)$ with $\cc(S^X)$. Moreover, we are identifying $\pa(S^X)=\pa(X)$ using the constants for $X$ a subsurface of $S$ (see Remark \ref{rmk:pickparameters}).

\begin{proof}
In order to prove Claim 1, let $X_0\subset S^X$ be a compact connected 2-submanifold with boundary such that $\bar\nei(\tau|X)\cup X\subset \inte(X_0)$ and the inclusions $X\subseteq X_0\subseteq S^X$ are homotopy equivalences. Let $p:S^X\rightarrow S$ be the covering map. Let $\gamma\in W(\tau|X), \beta\in W(\tau)$, and fix a carried realization of each of these two curves in the respective tie neighbourhoods (which are subsets of $S^X$ and of $S$, respectively).

One way to get a representative of $\pi_X(\beta)$ in $\cc(S^X)$ is as follows. We suppose that $\beta$ is not essentially disjoint from $X$, else Claim 1 is trivial: under this assumption, let $\beta^X$ be the union of all connected components of $p^{-1}(\beta)$ that intersect $X$ essentially: $\beta^X$ is either a single curve or a collection of essential arcs in $S^X$.

Then we can realize $\pi_X(\beta)$ as a family of curves in $X_0$: it will consist of all curves which arise as connected components of $X_0\cap\partial\left(\bar\nei(\beta')\cup \bar\nei(\partial X_0)\right)$ for $\beta'$ a connected component of $\beta^X$, and are essential in $S^X$. If the regular neighbourhoods specified in this expression is narrow enough, we have $\partial\bar\nei(\beta^X)\subseteq \bar\nei(\tau^X)$ and transverse to all ties it encounters; and $\bar\nei(\partial X_0)\cap \bar\nei(\tau|X)=\emptyset$. 

As a result, let $\delta\in\pi_X(\beta)$: the above realization of $\pi_X(\beta)$ gives a representative of $\delta$ which is transverse to all ties of $\bar\nei(\tau|X)$ it encounters; however, it is possible that $\delta$ has nonempty intersection with $\partial_h\bar\nei(\tau|X)$. Since $\beta$ is wide in $\tau$, a component of $\beta^X$ intersects at most twice any tie of $\bar\nei(\tau^X)$, and then, $\delta$ intersects at most $4$ times any tie in $\bar\nei(\tau|X)$.

Recall that $\gamma$ is wide carried in $\bar\nei(\tau|X)$: so we may suppose that, within each branch rectangle $\mathrm{im}(R_b)$ of $\bar\nei(\tau|X)$, there are no more than $8$ intersection points between $\gamma$ and $\delta$. No intersection point can be found outside $\bar\nei(\tau|X)$, so $i(\gamma,\delta)\leq 8N_1(S^X) \leq 8N_1(S)$. Recalling Lemma \ref{lem:cc_distance}, $d_{\cc(S^X)}(\gamma,\delta)\leq F\left( i(\gamma,\delta)\right)$.

By Remark \ref{rmk:subsurf_inters_bound}, for any $Y$ nonannular subsurface of $X$, if $\xi_1\in \pi_Y\left(V(\tau)\right)=\pi_Y\left(\pi_X\left(V(\tau)\right)\right)$ and $\xi_2\in \pi_Y\left(V(\tau|X)\right)$, then $i(\xi_1, \xi_2)\leq 32 N_1(S^X) + 4$, so
$$d_{\cc(Y)}\left(\pi_X V(\tau),V(\tau|X)\right)\leq F\left(32 N_1(S^X) + 4\right)\leq F\left(32 N_1(S) + 4\right)\eqqcolon f(S).$$

For Claim 2, in which $X$ is an annulus instead: just note that, every time $\beta$ is wide carried by $\tau$, $\pi_X(\beta)$ is wide carried by $\tau^X$. 

For Claim 3: apply the identification of $\tau|X$ with an almost track in $X$ as specified by Lemma \ref{lem:inducedisonsurface}, for ease of notation. If $V(\tau|X)$ is a vertex of $\pa(X)$ then $V(\tau|X)$ fills a --- possibly disconnected --- subsurface $X'$ of $X$ such that $X\setminus X'$ is a collection of pairs of pants (Definition \ref{def:quasipants}); but $V(\tau|X)\subseteq \cc(\tau)$, so there is a surface $X\subseteq W\subseteq S$ such that $\cc(\tau)$ fills a possibly disconnected subsurface $W'$ of $W$ with $W\setminus W'$ a collection of pairs of pants. $V(\tau)$ fills the same $W'$ (Lemma \ref{lem:decreasingfilling}), so it is a vertex of $\pa(W)$; and then, $\pi_X V(\tau)$ is a vertex of $\pa(X)$ (see Lemma \ref{lem:decreasingfilling} and the Remark following it).

Let $M> \max\{M_6(S),f(S)\}$. Then, by Theorem \ref{thm:mmprojectiondist} and Lemma \ref{lem:pantsquasiisom},
$$
d_{\pa(X)}\left(\pi_X V(\tau),V(\tau|X)\right)\leq 
e_0 \sum_{\substack{Y\subseteq X\\ Y\text{ not an annulus}}} \left[d_Y\left(\pi_X V(\tau), V(\tau|X)\right)\right]_M + e_1
$$
where $e_0,e_1$ are both functions of $M$ and $S$; but the summation is empty. This proves the existence of the claimed constant $C_1(S)$ holding for the case of $\pa(X)$.

One deals with $\ma(S)$ in an entirely similar way. One shows that $V(\tau|X)$ is a vertex of $\ma(X)$ similarly, and the distance estimate goes also along the same lines, except that the annular subsurface contributions also have to be taken into account, and this is done with the following observation.

For $Y$ an annular subsurface of $X$, making use of Claim 2, $\pi_Y\left(\pi_X V(\tau)\right)=\pi_Y V(\tau) \subseteq V(\tau|Y)$; and $\pi_Y V(\tau|X)\subseteq V((\tau|X)|Y)$. Therefore $d_Y\left(\pi_X V(\tau), V(\tau|X)\right)\leq d_{\cc(Y)} \left( V(\tau|Y), V((\tau|X)|Y)\right)$. However $(\tau|X)|Y$ is a subtrack of $\tau|Y$, so we can simply apply point \ref{itm:diambound} in Remark \ref{rmk:annulusinducedbasics}, which gives $d_{\cc(Y)} \left( V(\tau|Y), V((\tau|X)|Y)\right)\leq 5$.

Theorem \ref{thm:mmprojectiondist}, applied with $M> \max\{5,M_6(S),f(S)\}$, concludes.
\end{proof}

\begin{defin}
Let $\sigma,\tau$ be train tracks on a surface $S$, and let $X\subseteq S$ be a subsurface. Then the following notation will be employed:
\begin{eqnarray*}
d_{\cc}(\sigma,\tau) & \coloneqq & d_{\cc(S)}\left(V(\sigma),V(\tau)\right) \\
d_{X}(\sigma,\tau) & \coloneqq & d_X\left(V(\sigma),V(\tau)\right) = d_{\cc(X)}\left(\pi_X V(\sigma),\pi_X V(\tau)\right)
\end{eqnarray*}
provided that neither of the collections involved in the distance measurement is empty, and
\begin{eqnarray*}
d_{\ma}(\sigma,\tau) & \coloneqq & d_{\ma(S)}\left(V(\sigma),V(\tau)\right) \\
d_{\pa}(\sigma,\tau) & \coloneqq & d_{\pa(S)}\left(V(\sigma),V(\tau)\right) \\
d_{\ma(X)}(\sigma,\tau) & \coloneqq & d_{\ma(X)}\left(V(\sigma|X),V(\tau|X)\right) \\
d_{\pa(X)}(\sigma,\tau) & \coloneqq & d_{\pa(X)}\left(V(\sigma|X),V(\tau|X)\right)
\end{eqnarray*}
provided that the sets considered on the right hand side are actual vertices of the considered graphs.
\end{defin}

A closing remark for this subsection: in general, a train train track split reflects into an operation on induced train tracks which is either taking a subtrack, or performing splits (maybe more than one). This behaviour is controlled slightly better with a theorem from \cite{mms} which will be stated in a few pages.

\subsection{Multiple moves specified by an arc}
We now describe an operation on an almost track called an \emph{unzip}. Informally, an unzip for an almost track $\tau$ consists of cutting it open along a path that begins at a point of $\partial_v\bar\nei(\tau)$ and, proceeding transversally to the ties, ends at an interior point of $\nei(\tau)$. An example of unzip can be seen in Figure \ref{fig:multiplesplit}, e.g. from the first to the second picture. The effect on an unzip can always be expressed in terms of elementary moves, but a number of arguments in this work will be explained more conveniently by usage of this alternative formalism.

\begin{defin}\label{def:zipper}
Let $\tau$ be an almost track on a surface $S$. A \nw{zipper} in $\tau$ is a smooth embedding $\kappa:[-\epsilon,t]\rightarrow \bar\nei(\tau)$, for $t>0$ and $\epsilon>0$ small, such that the following are true.
\begin{itemize}
\item $\kappa(-\epsilon)$ lies along a connected component of $\partial_v \bar\nei(\tau)$ --- call this component $Z$.
\item $\kappa$ is transverse to all ties of $\bar\nei(\tau)$ it encounters.
\item $\kappa^{-1}(\text{ties through switches of }\tau)=\mathbb Z\cap [-\epsilon,t]$.
\item Let $\kappa_P\coloneqq c_\tau\circ\kappa|_{[0,t]}$; it is a smooth path along $\tau$, not necessarily embedded. If $\kappa(t)$ lies on the same tie as a switch of $\tau$, then the branch end $\kappa_P\left((t-\epsilon,t]\right)$ is small.
\end{itemize}

In order to describe the effect of an unzip note that, if $\kappa_P$ is an embedding itself, then there is an open set $\nei(\kappa)$, with $\mathrm{im}(\kappa_P)\subseteq \bar\nei(\kappa)\subseteq \bar\nei(\tau)$ and with the following properties:
\begin{itemize}
\item $\bar\nei(\kappa)$ is diffeomorphic to a triangle;
\item one edge of the triangle is contained in $Z$, and the opposite vertex is $\kappa_P(t)$;
\item the remaining two edges are transverse to all ties they encounter;
\item on the other hand, the interiors $s_1,s_2$ of each of these two edges intersects $\tau$ tangentially in a number of closed segments: i.e. in a neighbourhood of each intersection segment between $\tau$ and $s_i$, $\tau\cup s_i$ looks like a neighbourhood of a large branch in a pretrack.
\end{itemize}

These properties imply the following. Let $B$ be the connected component of $\partial\bar\nei(\kappa)\setminus \tau$ which intersects $Z$, and let $C\coloneqq\partial\bar\nei(\kappa)\setminus B$ --- in other words, $C$ is the maximum connected subset of $\partial\bar\nei(\kappa)$ which contains $\kappa_P(t)$ and has its extremes along $\tau$. Then, the tangential intersection property implies that $\tau\cup C$ is a pretrack whose complementary bigon regions are all adjacent to the image of $\kappa_P$.

So, again if $\kappa_P$ is an embedding, we define the \nw{unzip along $\kappa$} as $\tau''=(\tau\cup C)\setminus \nei(\kappa)$. It is an almost track and has a tie neighbourhood $\bar\nei(\tau'')\subseteq \bar\nei(\tau)$ with $\partial\bar\nei(\tau)\setminus Z\subseteq \partial\bar\nei(\tau'')$. Note, however, that there is no direct correlation between $\nei(\kappa)$ and $\nei(\tau'')$.

If $\kappa_P$ is not an embedding, the unzip is defined inductively on $\lceil t\rceil$. Let $t'=\zeta-2\epsilon$, for $\zeta\in \mathbb Z_+$ chosen such that $\kappa_P|_{[-\epsilon,t']}$ is an embedding instead. One can suppose, up to tie-transverse isotopies, that $\kappa|_{[-\epsilon,t']}=\kappa_P|_{[-\epsilon,t']}$. Let $\tau'$ be the unzip of $\tau$ along $\kappa|_{[-\epsilon,t']}$, and let $\kappa'=\kappa\cap\bar\nei(\tau')$, reparametrized so as to be a zipper for $\tau'$: the unzip $\tau''$ of $\tau$ along $\kappa$ is then defined as the unzip of $\tau'$ along $\kappa'$.
\end{defin}

\begin{defin}\label{def:multiplesplit}
A \nw{large multibranch} for an almost track $\tau$ in a surface $S$ is a carried realization $\beta$ of a carried arc (i.e. $\beta$ is \emph{not} to be considered up to isotopy) in $\bar\nei(\tau)$, whose endpoints belong to distinct components of $\partial_v\bar\nei(\tau)$.

A \nw{splitting arc} is an embedded large multibranch traversing exactly one large branch --- equivalently it does not traverse any small branch. It may traverse any number of mixed branches.

Let $\beta$ be a large multibranch. Specify a splitting parity (left, right, or center); if the parity is not center, also specify two distinct points $P_1,P_2$ lying along $\beta_{trim}$ (see Definition \ref{def:trainpathrealization}), but not contained in the same ties as a switch of $\tau$; they will be called \nw{anchors}. 

The \nw{multiple split} with respect to the specified data is given by the following construction.

\begin{itemize}
\item In case the specified parity is not center (see Figure \ref{fig:multiplesplit}): let $\kappa_j:[-\epsilon,t_j]\rightarrow \bar\nei(\tau)$, $j=1,2$, be two zippers, beginning at the opposite endpoints of $\beta$, following $\beta$ and ending at $P_1,P_2$ respectively; the pairing of the endpoints of $\beta$ with the anchors is to be done so that each the images of $(\kappa_1)_P$ and $(\kappa_2)_P$ overlap. 

Unzip $\tau$ along $\kappa_1$ to get a new almost track that we call $\tau_1$. Note that $P_1$ is a switch of $\tau_1$. We may suppose that the tie $\alpha_2$ of $\bar\nei(\tau)$ through $P_2$ is not entirely contained in $\bar\nei(\tau_1)$. 

Then it is possible to find two zippers $\kappa_{21},\kappa_{22}$ in $\bar\nei(\tau_1)$ with the following property. On the one hand, $\kappa_{21},\kappa_{22}$ are both isotopic to $\kappa_2$ via an isotopy which keeps each point along the same tie of $\bar\nei(\tau)$ (equivalently, $c_\tau\circ\kappa_{21}|_{[0,t_2]},c_\tau\circ\kappa_{22}|_{[0,t_2]}$ are both reparametrizations of $(\kappa_2)_P$); on the other hand, the two are \emph{not} isotopic to each other via a similar tie-transverse isotopy in $\bar\nei(\tau_1)$.

Finally, if the parity specified for the multiple split is right, let $\kappa'_2$ be the one between $\kappa_{21},\kappa_{22}$ that, after intersecting the tie through $P_1$, traverses the small branch end to the right of $P_1$. If the parity is left instead, let $\kappa'_2$ be the other of those two zippers. The multiple split of $\tau$ with the aforementioned data is, then, defined to be the unzip of $\tau_1$ along $\kappa'_2$.

\item In case the specified parity is center: $\beta$ traverses at least one large branch: choose one, $b$. Let $P_1,P_2$ be distinct points (also called \emph{anchors}, even if in this case they are part of the specified data) belonging to the same component of $\beta\cap R_b([-1,1]\times[-1,1])$. Let $\kappa_1,\kappa_2$ be two zippers, with each of them following $\beta$ from an endpoint to $P_1$ and $P_2$, respectively; this time, however, we choose the two zippers to be \emph{disjoint}. Unzip $\tau$ along $\kappa_1$ and along $\kappa_2$ to get a new track which we call $\tau'$. Here, $\beta\cap\bar\nei(\tau')$ traverses a single branch, which is large. The central multiple split of $\tau$ along $\beta$ is, then, the central split of $\tau'$ along this large branch.
\end{itemize}

If $\beta$ is a splitting arc such that $b$ is the only large branch $\beta$ traverses, the \nw{wide split} along $\beta$ with a given parity is defined to be the multiple split along $\beta$, according to the given parity, choosing any two anchors $P_1,P_2$ which lie along ties in $R_b([-1,1]\times[-1,1])$.
\end{defin}

\begin{rmk}
We have defined a large multibranch to be a fixed carried realization, rather than an isotopy class, of a carried arc. This has been convenient for the definitions of anchors, and for the description of what a multiple split is supposed to do. However, if one perturbs a large multibranch under an isotopy keeping each point along the same tie, and the anchors in particular, the result of the multiple split is the same.
\end{rmk}

\begin{figure}[htbp]
\begin{center}
\def\svgwidth{\textwidth}
\input{multisplitting.pdf_tex}
\end{center}
\caption{\label{fig:multiplesplit}An example of a left multiple split. Here the large multibranch $\beta$ is the one with endpoints specified in the first picture and traversing twice the branch $b$ (this information suffices to figure out what it looks like in $\bar\nei(\tau)$, up to isotopies which are inconsequential for the construction). The anchors are $P_1,P_2$; $P_2$ in the picture is meant to lie along the upper segment in $\beta\cap R_b$. In the second picture, the zipper $\kappa_1$ following $\beta$ from a suitable endpoint to $P_1$ has been unzipped, resulting in the definition of two points $P_{21},P_{22}$ in place of the old $P_2$. The points $P_{21}, P_{22}$ are endpoints for two substantially different zippers $\kappa_{21},\kappa_{22}$ which both begin at the remaining endpoint of $\beta$ and project to the old zipper $\kappa_2$ for $\tau$. In the third picture, the zipper $\kappa'_2=\kappa_{22}$ has been unzipped, too. The choice of $\kappa_{22}$ rather than $\kappa_{21}$ is determined by the specified parity.}
\end{figure}

\begin{rmk}\label{rmk:generic_move_as_unzip}
The given definition of wide split includes, as a special case, the one of a `standard' split as defined by Figure \ref{fig:ttsplitting}. If $b$ is the branch to split, one may choose a suitable splitting arc which is entirely contained in $R_b$.

The result of a wide split is always the same as a splitting sequence involving exactly one split.

If $\tau$ is a \emph{generic} almost track, then any elementary move on $\tau$ which is not a central split is also the result of an unzip along a \emph{single} zipper $\kappa: [-\epsilon, 1+\epsilon]\rightarrow \bar\nei(\tau)$: a \emph{large} zipper defined on this domain will produce a parity split, and a \emph{small} one will produce a slide (see below for the definitions).
\end{rmk}

\begin{rmk}\label{rmk:zipvssplit}
If $\tau'$ is the unzip of an almost track $\tau$ along a zipper $\kappa:[-\epsilon,t]\rightarrow \bar\nei(\tau)$, then $\tau'$ is itself an almost track, clearly carried by $\tau$ and therefore obtained from it via a splitting sequence (Proposition \ref{prp:carriediffsplit}). Some remarks will ease the incoming estimates for the number of splits necessary to get $\tau'$ from $\tau$. If $t\in (0,1)$ then $\tau'$ is isotopic to $\tau$; whereas, if $t\in[1,2)$, then:
\begin{itemize}
\item if $\kappa_P\left((1-\epsilon,1]\right)$ is a small branch end, then $\tau'$ is comb equivalent to $\tau$;
\item if $\kappa_P\left((1-\epsilon,1]\right)$ is a large branch end, and all small branch ends of $\tau$ incident to $\kappa_P(1)$ lie on the same side of $\kappa_P$, then $\tau'$ is obtained from $\tau$ with a parity split, plus possibly some comb/uncomb moves;
\item if $\kappa_P\left((1-\epsilon,1]\right)$ is a large branch end, and small branch ends of $\tau$ incident to $\kappa_P(1)$ lie on both sides of $\kappa_P$, then $\tau'$ is obtained from $\tau$ with two parity splits, plus possibly some comb/uncomb moves.
\end{itemize}

More generally, suppose that $\kappa:[-\epsilon,t]\rightarrow \bar\nei(\tau)$ is a zipper with no restriction on its length, but with the property that $\kappa_P\left((\zeta-\epsilon,\zeta]\right)$ is a small branch end for all integers $1\leq \zeta\leq t$ (we call it a \nw{small zipper}). Then the result of the unzipping is comb equivalent to $\tau$.

If one supposes instead that $\kappa_P\left((\zeta-\epsilon,\zeta]\right)$ is a large branch end for all integers $1\leq \zeta\leq t$ (we call it a \nw{large zipper}), then necessarily $\kappa_P$ is an embedding. A branch can be traversed twice only if, for some value of $\zeta$, $\kappa_P\left((\zeta-\epsilon,\zeta]\right)$ is a small branch end.

Similarly as before, there is a difference in behaviour depending on whether the branches incident to $\kappa_P([1,t])$ all lie on the same side of it or not. In the first case, the unzip along $\kappa$ is comb equivalent to an almost track obtained from $\tau$ with one parity wide split; in the second case, the number of wide splits needed is $2$. A particular case of the first behaviour is when there is only 1 branch sharing a switch with $\tau([1,t])$.

Given a zipper $\kappa:[-\epsilon,t]\rightarrow\bar\nei(\tau)$ which is neither large nor small itself, it is always possible to take the maximal $\zeta\in \mathbb N$ such that $\kappa|_{[-\epsilon,\zeta+\epsilon]}$ is either a large or a small zipper. Let $\tau'$ be the unzip of $\tau$ along $\kappa|_{[-\epsilon,\zeta+\epsilon]}$; $\kappa|_{[\zeta+\epsilon, t]}$ admits a natural reparametrization as $\kappa':[-\epsilon,t']$, for some $t'\leq t-\zeta$ which makes it into a zipper for $\tau'$: unzipping $\tau'$ along $\kappa'$ is the same as unzipping $\tau$ along $\kappa$. 

Recursively, one can decompose the unzip along $\kappa$ into a sequence of unzips along zippers that are alternatively small and large. An upper bound for the number of large zippers involved is given by the number of $\zeta\in \mathbb N$ such that $\kappa_P([\zeta,\zeta+1])$ is a large branch of $\tau$. We call this decomposition \nw{canonical}.
\end{rmk}

Our notion of wide split agrees with the one defined in \cite{mosher}, \S 3.13. We are interested in Proposition 3.13.3 in that work: it uses the fact that, given two comb equivalent train tracks, there is a canonical identification between the respective families of splitting arcs. This correspondence may be found with our definitions as well, so the said Proposition holds for us:

\begin{prop}\label{prp:combpersistence}
Let $\tau,\sigma$ be two comb equivalent train tracks, and $\alpha,\beta$ be splitting arcs, for $\tau$ and for $\sigma$ respectively, which correspond to each other under the equivalence of the two tracks. Let $\tau'$ be the train track obtained from $\tau$ by splitting along $\alpha$ according to a given parity; and let $\sigma'$ be the train track obtained from $\sigma$ by splitting along $\beta$ according to the same parity.

Then $\tau',\sigma'$ are comb equivalent.
\end{prop}

Also, a recursive application of Proposition 3.13.4 in that monograph gives the following:
\begin{prop}\label{prp:deleteslidings}
Given a train track splitting sequence $\bm\tau$, let $(j_r)_{r=1}^R$ be the indices such that the move from $\tau_{j_r-1}$ to $\tau_{j_r}$ is a split. Then there is a sequence $\bm\sigma=(\sigma_r)_{r=0}^R$ where $\sigma_0=\tau_0$, $\sigma_r$ is comb equivalent to $\tau_{j_r}$ and $\sigma_r$ is obtained from $\sigma_{r-1}$ via a wide split, of the same parity as the split between $\tau_{j_r-1}$ and $\tau_{j_r}$.
\end{prop}
A sequence of wide splits will be called a \nw{wide splitting sequence}.


\subsection{Cornerization of train tracks}
The purpose of the following discussion is to connect the most common version of train track found in the literature (which corresponds to the definition we have given) to the notion of cornered train track, which is the notion of train track given and developed in \cite{mms}.

\begin{defin}
Given a train track $\tau$ on a surface $S$, a train track $\tau'$ is a \nw{cornerization} of $\tau$ if:
\begin{itemize}
\item $\tau'$ is cornered;
\item $\tau$ is obtained from $\tau'$ via a sequence of central splits and comb equivalences;
\item if $\tau$ is transversely recurrent, then $\tau'$ also is.
\end{itemize}
\end{defin}
We have not required explicitly that if $\tau$ is recurrent, then $\tau'$ also has to be: this is indeed a direct consequence of the second bullet.

\begin{lemma}\label{lem:cornerization}
Any train track $\tau$ on $S$ admits a cornerization.
\end{lemma}
\begin{proof}
We focus on the case when $\tau$ is transversely recurrent. Let $\ul\gamma=\{\ul\gamma_1,\ldots,\ul\gamma_m\}$ be a set of essential curves in $S$ as specified after Definition \ref{def:recurrent}: pairwise disjoint, dual to $\tau$, such that for each branch $b$ of $\tau$ there is a $\ul\gamma_i$ intersecting $b$. It is not required that two of them are not isotopic.

Suppose that $\tau$ is not cornered. We will build a transversely recurrent train track $\tau'$ such that $\partial\bar\nei(\tau')$ has fewer smooth components than $\partial\bar\nei(\tau)$, and $\tau$ is obtained from $\tau'$ with a central split. This will prove our claim by induction.

We deal with a special case first: $\tau$ contains a connected component $\lambda$ which is an embedded loop and all curves $\ul\gamma_1,\ldots,\ul\gamma_m$ intersect $\lambda$ in at most one point. Then necessarily one of them, $\ul\gamma_1$ say, intersects $\lambda$ in exactly one point. In that case, $\lambda$ and $\ul\gamma_1$ together fill a handle $H$ of $S$. In $H$, replace $\lambda$ with a new component $\lambda'$ and $\ul\gamma_1$ with a new dual curve $\ul\gamma'_1$ as shown in Figure \ref{fig:cornerizehandle}.

\begin{figure}[h]
\centering \def\svgwidth{.8\textwidth}
\input{cornerizehandle.pdf_tex}
\caption{\label{fig:cornerizehandle}How to get rid of a loop component $\lambda$ of $\tau$ with a dual curve $\ul\gamma_1$ which intersects $\tau$ in a single point, lying along $\lambda$. The two lie in a handle $H$ of $S$. Note: $\lambda'$ shall differ from $\lambda$ only in a neighbourhood of $\ul\gamma_1$ whose intersection with any other $\ul\gamma_i$ is empty.}
\end{figure}

Define $\tau'\coloneqq(\tau\setminus\lambda)\cup\lambda'$. It is true that $\partial\bar\nei(\tau')$ has one smooth component less than $\partial\bar\nei(\tau)$ and that $\tau'$ is a train track. Furthermore it is tranversely recurrent: the new curve $\ul\gamma'_1$ may be isotoped to keep dual to $\tau'$ and intersect either of the two branches in $\lambda'$; while the other curves $\ul\gamma_2,\ldots,\ul\gamma_m$ are still in efficient position with respect to $\tau'$, and therefore dual: as it was done in Definition \ref{def:efficientposition}, fix a regular neighbourhood $\nei(\gamma_i')$ which contains no corners of $\partial\bar\nei(\tau')$. As one may see from the figure, $H\setminus\left(\nei(\tau')\cup \nei(\gamma_i')\right)$ will only consist of connected components with $\geq 4$ corners each, so $S\setminus\left(\nei(\tau')\cup \nei(\gamma_i')\right)$ consists only of negative index components or rectangles. For each branch $b$ of $\tau'\setminus\lambda'$ there is one curve among $\ul\gamma_2,\ldots,\ul\gamma_m$ which intersects it.

Now suppose that a component as above does not exist in $\tau$. Let $\alpha\subset\partial\bar\nei(\tau)$ be a smooth connected component: in particular $\alpha$ is isotopic to a wide carried curve of $\tau$, so it makes sense to talk about the carrying image $\tau.\alpha$. Let $C$ be the connected component of $\bar S\setminus\nei(\tau)$ such that $\alpha\subset\partial C$.

We wish to make sure that there is a curve $\ul\gamma_i$ intersecting $\alpha$, and intersecting $\tau$ in at least two points. Suppose such a curve does not exist: i.e. all curves $\underline{\gamma_i}$ intersecting $\alpha$ in fact intersect $\tau$ in only $1$ point. By our hypothesis, this excludes the possibility that $\tau.\alpha$ is a smooth loop consisting of an entire connected component of $\tau$.

Thus, in case $\tau.\alpha$ is a smooth loop (i.e. $\alpha$ has an embedded train path realization), we have that $\tau.\alpha$ includes a mixed or large branch $b$ of $\tau$. In case $\tau.\alpha$ is not a smooth loop, $\tau.\alpha$ surely includes a large branch $b$ of $\tau$.

In both cases, there is a $\ul{\gamma_i}$ such that the single point $\ul{\gamma_i}\cap \tau.\alpha$ lies along $b$. With a small isotopy on $\ul{\gamma_i}$, slide it towards one of the large ends/the large end (resp.) of $b$, and then past the relative switch. The curve thus obtained, $\ul{\gamma'_i}$, is still dual to $\tau$; moreover the single intersection of $\ul{\gamma_i}$ with $b$ has been replaced with an intersection point of $\ul{\gamma'_i}$ with each of the small branch ends which are incident at that switch.

So, it is always possible to assume, up to enlarging the collection $\ul\gamma$ of curves dual to $\tau$, that there exists a curve $\underline{\gamma_i}$ intersecting $\alpha$, and having more than $1$ intersection point with $\tau$. Fix a regular neighbourhood $\nei(\gamma_i)$ which contains no corners of $\partial\bar\nei(\tau)$. Let $p\in \ul{\gamma_i}\cap \tau.\alpha$, let $g$ be an arc of $\gamma$ which begins at $p$, proceeds towards $C$ through a half-tie, then through $C$, and finally follows a tie of $\tau$ so as to arrive at the nearest intersection point $q\in \ul{\gamma_i}\cap\tau$. Pinch a neighbourhood of $p$ in $\tau$ along $g$ and fold it with a neighbourhood of $q$, so as to create a new large branch $f$: this pretrack will be our $\tau'$ (Figure \ref{fig:cornerization}).

\begin{figure}
\begin{center}
\def\svgwidth{.9\textwidth}
\input{cornerization.pdf_tex}
\end{center}
\caption{\label{fig:cornerization}The elementary step to reduce the number of smooth components of $\partial\bar\nei(\tau)$ is depicted above. The dual curve $\ul{\gamma_i}$ shows the way to \emph{fold} $\tau$ (this is how the inverse of a split move is usually called) so as to reduce the total number of smooth boundary components.}
\end{figure}

We claim, first of all, that $\tau'$ is a train track. Note that the complementary regions of $\nei(\tau')$ in $S$ can be identified with the ones of $\nei(\tau)$, except $C$ which is replaced by either a single complementary component $C'$ or by two $C',C''$. In either case what we obtain is diffeomorphic to $C\setminus (\nei(\ul{\gamma_i})|_g)$, where by $\nei(\ul{\gamma_i})|_g$ we mean the component of $\nei(\ul{\gamma_i})\cap C$ which contains $g$. Since, by efficient position of $\ul{\gamma_i}$, components of $C\setminus\nei(\ul{\gamma_i})$ either have negative index or are rectangles, $C',C''$ will have the same property, because they are obtained as the gluing of some of such components, plus some rectangles, along boundary edges. But $g$ cannot cut a rectangle out of $C$: that would mean that $\alpha$ is not a smooth boundary component. This proves the claim.

Also, we claim that $\tau'$ is transversely recurrent. The given curves $\ul{\gamma_1},\ldots,\ul{\gamma_k}$ may not be enough to intersect all branches of $\tau'$ because $\ul{\gamma_i}$, in general, will intersect the new large branch $f$ born from folding, but not the four branches which are incident to its ends. However, if $\ul{\gamma_i}$ is altered via an isotopy that slides it past either switch of $f$, it turns into a curve, again dual to $\tau'$, and intersecting two of the four branch ends sharing their switch with $f$. So we may add two more curves, isotopic to $\ul\gamma_i$, to the collection $\ul\gamma$, to intersect all branches of $\tau'$.

In case $\tau$ is \emph{not} transversely recurrent, we apply a similar idea. We do not select a family of curves $\ul\gamma$; after choosing a connected component $C$ of $\bar S\setminus\nei(\tau)$ with a smooth boundary component $\alpha$, we may pick a properly embedded arc $g$ in $C$ such that: the endpoints of $g$ are distinct; at least of them lies along $\alpha$; and $g$ does not cut out a region of $C$ with nonnegative index. One builds $\tau'$ similarly as above and then shows that is a train track.
\end{proof}

We now describe how to cornerize consistently an entire splitting sequence: the first step is to make it cleaner by turning it into a wide splitting sequence.

\begin{rmk}\label{rmk:centralsplitbound}
Any (wide) splitting sequence involves at most $N_1$ central (wide) splits: given a train track $\tau$, each branch contributes at most $4$ corners in $\partial\bar\nei(\tau)$; and each central split decreases the total number of corners by $4$, whereas any other move keeps it unvaried.
\end{rmk}

\begin{lemma}[Postponing central splits]\label{lem:postpone}
Given a wide splitting sequence $\bm\sigma=(\sigma_0\ldots,\sigma_i,\sigma_{i+1},\sigma_{i+2},\ldots,\sigma_N)$, call $\beta_j$ (for $j=0,1$) the splitting arc for $\sigma_{i+j}$ that shall be split along to get $\sigma_{i+j+1}$. Suppose that the split along $\beta_0$ is central whereas the one along $\beta_1$ is parity. Then there is a wide splitting sequence $\bm{\sigma'}=(\ldots,\sigma'_i,\ldots,\sigma'_{i+k},\ldots)$, with $2\leq k\leq 7$, such that
\begin{itemize}
\item $\sigma_j=\sigma'_j$ for all $j\leq i$;
\item the wide split(s) between $\sigma'_i$ and $\sigma'_{i+k-1}$ is/are parity;
\item the wide split between $\sigma'_{i+k-1}$ and $\sigma'_{i+k}$ is central;
\item $\sigma_j$ is comb equivalent to $\sigma'_{j-2+k}$ for all $j\geq i+2$. (and in particular the all wide splits in the subsequence $(\sigma'_j)_{j\geq i+k}$ have the same parity as the corresponding ones in $(\sigma_j)_{j\geq i+2}$). 
\end{itemize}
\end{lemma}
\begin{proof}
We first build a splitting sequence $\bm\rho$ turning $\sigma_i$ into $\sigma_{i+2}$, and involving $1$ to $6$ parity splits, followed by $1$ central split.

Since the split along $\beta_0$ is central, a tie neighbourhood $\bar\nei(\sigma_{i+1})$ is contained in $\bar\nei(\sigma_i)\setminus \beta_0$, with ties obtained by restriction of the old ones; and $\partial_v\bar\nei(\sigma_{i+1})$ consists of $\partial_v\bar\nei(\sigma_i)$, deprived of two connected components. So we may suppose that, in $\bar\nei(\sigma_i)$, the arcs $\beta_0,\beta_1$ are properly embedded and disjoint. The endpoints of $\beta_0$ lie on distinct components of $\partial_v\bar\nei(\sigma_i)$, and so do the ones of $\beta_1$ because this property must hold when considering it as an arc in $\bar\nei(\sigma_{i+1})$. It is also impossible that the two splitting arcs have an extreme each on the same connected component of $\partial_v\bar\nei(\sigma_i)$, because the two connected components where $\partial\beta_0$ lie are not part of $\partial_v\bar\nei(\sigma_{i+1})$.

The arc $\beta_1$, in general, is not a splitting arc for $\sigma_i$, but it is a large multibranch. Also, since the wide central split along $\beta_0$ shall turn it into a splitting arc, the two arcs are embedded in $\bar\nei(\sigma_i)$ with this property: $\beta_1$ traverses each branch at most twice, and if $b$ is any branch traversed twice then $R_b\setminus \beta_0$ consists of two connected components, and the two segments of $\beta_1\cap R_b$ each lie in one.

Let $\delta\coloneqq \bigcup_{b\text{ traversed by }\beta_0} \beta_1\cap R_b$. If $B_0$ is the large branch of $\sigma_i$ traversed by $\beta_0$, then each component of $\delta$ traverses $B_0$ (here we use the term `traverse' in an obvious generalized setting). Let indeed $e$ be the first or last branch end traversed by a connected component $\delta_0$ of $\delta$: either $e$ is the first/last branch end traversed by $\beta_0$ or $\beta_1$; or $\beta_1$ and $\beta_0$ give two different train paths originating from $e$. So the branch ends at both extremities of $\delta_0$ are large, implying that $\delta_0$ must traverse a large branch: necessarily $B_0$. In particular, $\delta$ shall consist of $0$, $1$ or $2$ connected components.

Let $B_1$ be the large branch traversed by $\beta_1$ in $\sigma_{i+1}$: $c_{\sigma_i}(B_1)\subseteq \sigma_i$ is a union of branches, including a large one $b_1$. Place the two anchors $P_1,P_2$ for the wide split of $\sigma_{i+1}$ along $\beta_1$ within the same segment of $\beta_1\cap R_{b_1}([-1,1]\times[-1,1])$. This is not important for the wide split, as it only affects its result up to isotopy; but is relevant for the following argument.

Let $\rho''_i$ be the result of the multiple split of $\sigma_i$ along $\beta_1$, with the same parity as the wide split that $\sigma_{i+1}$ undergoes in the sequence $\bm\sigma$, and with anchors $P_1,P_2$. Call $\kappa_{11}$ the first zipper that gets unzipped, and $\rho'_i$ the result of the unzip; and call $\kappa_{12}'$ the second zipper, with $P'_2\in\rho'_i$ the endpoint of $\kappa_{12}'$. The arc $\beta_0$ is a large multibranch in $\rho''_i$, and the central multiple split along it gives back $\sigma_{i+2}$. This multiple split will be described as the unzip of a zipper $\kappa_{01}$ to obtain an almost track $\rho'''_i$, followed by the unzip along another zipper $\kappa_{02}$ to get $\rho''''_i$, and finally a central split.

According to Remarks \ref{rmk:generic_move_as_unzip} and \ref{rmk:zipvssplit}, the multiple split along $\beta_1$ can be canonically decomposed into unzips along small and large zippers. While unzips along small zippers give the same result as a series of comb/uncomb moves, the ones along large zippers give a track which is comb equivalent to the one obtained with one or two wide splits so their effect is obtained with a splitting sequence which involves one or two splits, even if \emph{only up to isotopy}. This specifies how to build a splitting sequence $\bm\rho^1$, turning $\sigma_i$ progressively into $\rho'_i$ and $\rho''_i$. Similarly, the subsequent multiple split along $\beta_0$ is the result of a splitting sequence $\bm\rho^0$ turning $\rho''_i$ into $\rho'''_i,\rho''''_i$, and finally $\sigma_{i+2}$. Also, let $\bm\rho=\bm\rho^1*\bm\rho^0$. The last split in this sequence is central, the others are parity.

We will now estimate how many splits are involved, in total, in the sequence $\bm\rho^1$, by analysing how many unzips along large zippers take place when canonically decomposing the unzips along $\kappa_{11}$ and $\kappa'_{12}$. There are several cases to consider, but the following consideration is always true: let $\eta$ be the second-to-last almost track obtained when applying the unzips in the canonical decomposition, and let $\kappa_l:[-\epsilon,t]\rightarrow \bar\nei(\eta)$ be the last zipper in the decomposition, which is a large one. Then $t\in(1,2)$, $P_1$ belongs to the tie through $\kappa_l(1)$, and the branch end $(\kappa_l)_P([1,t])$ shares its switch only with another branch end: so the unzip along $\kappa_l$ contributes only one (parity) split in the sequence $\bm\rho^1$.

Suppose that $P_1,P_2$ lie along $\delta$, and therefore along one same connected component $\delta_0$; in particular they lie along $\delta_0\cap R_{B_0}$. The first case to consider is the one when this supposition holds, and $\delta$ has only 1 connected component. In this case there are no large branches contained in the image of $(\kappa_{11})_P$, and the one of $(\kappa'_{12})_P$ contains only one. So the zipper $\kappa_{11}$ is small, and either the zipper $\kappa'_{12}$ is large, or the unzip along it can be subdivided into the unzip along a small zipper followed by one along a large zipper. In either case, with this configuration $\bm\rho^1$ only involves 1 split (due to the consideration in the above paragraph).

The second case is the one of $P_1,P_2$ lying again along $\delta$, but $\delta$ having two connected components. Then the component different from $\delta_0$ is included entirely in $\beta_1\setminus b_1$ --- thus it traverses branches of $\sigma_i$ that are completely contained in the image of either $(\kappa_{11})_P$, or $c_{\sigma_i}\circ(\kappa'_{12})_P$. In the first sub-case the unzip along $\kappa_{11}$ is canonically decomposed into at most $3$ unzips, with one occurring along a large branch and the other, or others, along one or two small ones; and the unzip along $\kappa'_{12}$, instead, is decomposed similarly as above; the large zippers involved are one or two, implying that the splits required in $\bm\rho^0$ are $1$ to $3$ ($4$ is to be excluded because of the previous consideration; in any case, they are all parity splits). In the second sub-case the zipper $\kappa_{11}$ is small, while the canonical decomposition of the unzip along $\kappa'_{12}$ involves $2$ unzips along large zippers, the second of which requires only $1$ split: there are again at most $3$ (parity) splits in $\bm\rho^1$.

The third and last case is the one of $P_1,P_2$ not lying along $\delta$. In this case $\delta$ (if nonempty) is entirely contained in the images of $(\kappa_{11})_P$ and $c_{\sigma_i}\circ(\kappa'_{12})_P$. Several cases are possible as $\delta$ may consist of up to $2$ components, and each of them may be traverse branches belonging to each of the images of the specified maps. But a similar analysis as above shows that the canonical decompositions of the two unzips together involve at most three unzips along large zippers, thus there are at most $5$ (parity) splits in $\bm\rho^1$.

Turning to $\bm\rho^0$: $\beta_0$ traverses each branch of $\rho''_i$ at most once. The restriction of the tie collapse $c_{\sigma_i}:\rho''_i\rightarrow \sigma_i$ maps all branch ends of $\rho''_i$ sharing a switch with $\rho''_i.\beta_0$ to branch ends of $\sigma_i$ sharing a switch with $\sigma_i.\beta_0$, except possibly for the one, $e$, whose endpoint $v$ lies along the same tie of $\bar\nei(\sigma_i)$ as $P_1$ or $P_2$: it may be the case that $c_{\sigma_i}(e)\subseteq \sigma_i.\beta_0$.

The multiple central split along $\beta_0$ is a wide central split, if $\beta_0$ traverses only one large branch of $\rho''_i$. If this is not true, the only possibility is that $\beta_0$ traverses $2$ large branches of $\rho''_i$, with one of them delimited by $v$; call the other one $b$. Without affecting the result of the multiple central split, one can suppose that the anchors $Q_1,Q_2$ (with $Q_j$ endpoint of $\kappa_{0j}$) lie in $\beta_0\cap R_b$, and that $Q_1$ is the one closer to $v$.

We can now apply a similar argument as before. The unzip along $\kappa_{01}$ can be canonically decomposed into at most $3$ unzips, only one of which takes place along a large zipper $\kappa_l$, defined on an appropriate interval $[-\epsilon,t]$. One sees that, necessarily, $t\in(1,2)$; $(\kappa_l)_P([1,t])$ contains no switches other than $v$ and is only incident to the branch end $e$. The zipper $\kappa_{02}$ is small instead. So the splitting sequence $\bm\rho^1$ requires at most $1$ parity split further than the last, central split.

Let $k$ be the total number of splits in $\bm\rho$: it has been proved above that $2\leq k\leq 7$, and that only the last split is central anyway.

Now that $\bm\rho$ has been defined, we build $\bm\sigma'$ from $\bm\sigma$ with the following steps. Call $\bm\sigma_-=\bm\sigma(0,i)$. The wide splitting sequence $\bm\sigma(i+2,N)$ can be turned into a (regular) splitting sequence, $\bm\sigma''_+$. According to Proposition \ref{prp:deleteslidings} there is a wide splitting sequence $\bm\sigma'_{0+}$, starting with $\sigma_i$, whose elements are orderedly comb equivalent to the ones of $\bm\rho*\bm\sigma''_+$. This sequence begins with with $k-1$ wide parity splits followed by a central one, and $(\sigma'_{0+})_k$ is comb equivalent to $\sigma_{i+2}$. Define $\bm\sigma'=\bm\sigma_-*(\bm\sigma'_{0+})$. 
\end{proof}

\begin{defin}\label{def:cornerizedseq}
Let $\tau=(\tau_j)_{j=0}^N$ be a train track splitting sequence. We build a splitting sequence $P\bm\tau$ (not uniquely defined) in the following way.

Let $\hat\tau_0$ be a cornerization of $\tau_0$. Let $\bm{\hat\tau}$ be a splitting sequence from $\hat\tau_0$ to $\tau_0$, and then continuing as $\bm\tau$.

If $\bm{\hat\tau}$ involves no central split, or all central splits are followed only by central splits and slides, the process ends here. Else replace $\bm{\hat\tau}$ with a wide splitting sequence $\bm\sigma^0$ according to Proposition \ref{prp:deleteslidings}, and define a list of wide splitting sequences $\bm\sigma^i=(\sigma^i_j)_j$ recursively this way: within the sequence $\bm\sigma^i$, let $\ell(i)$ be the highest index $j$ such that $\sigma^i_j,\sigma^i_{j+1},\sigma^i_{j+2}$ are a central wide split followed by a parity one: postpone the central split according to Lemma \ref{lem:postpone} above, and call $\bm\sigma^{i+1}$ the new sequence. Repeat the process as far as it is possible to define $\ell(i)$. Finally, replace the last constructed $\bm\sigma^R$ back with a (non-wide) splitting sequence: this will be $P\bm\tau$.

Let $\cnr\bm\tau$ be the initial subsequence of $P\bm\tau$ obtained by truncating just after the last parity split. It is called a \nw{cornerization} of $\bm\tau$.
\end{defin}

We summarize the properties of $\cnr\bm\tau$ in the below:
\begin{lemma}\label{lem:ctauproperties}
Given a splitting sequence $\bm\tau=(\tau_j)_{j=0}^N$ on a surface $S$, let $\cnr\bm\tau=(\cnr\tau_j)_{j=0}^M$ be a cornerization of it defined from $P\bm\tau=(P\tau_j)_{j=0}^{M+M'}$. Then the following properties hold:
\begin{enumerate}
\item if all elements of $\bm\tau$ are recurrent, then so are the ones of $\cnr\bm\tau$;
\item if all elements of $\bm\tau$ are transversely recurrent, then so are the ones of $\cnr\bm\tau$;
\item all elements of $\cnr\bm\tau$ are cornered;
\item there is an increasing function $f:[0,N]\cap\mathbb Z\rightarrow [0,M+M']\cap\mathbb Z$, with $f(0)=0$, such that $\cnr\tau_{f(j)}$ is comb equivalent to a cornerization of $\tau_j$ for all $0\leq j\leq N$, and if $N_0$ is the lowest index such that no parity split occurs in $\bm\tau$ after $\tau_{N_0}$ then $f(N_0)=M$;
\item $|\bm\tau|-N_1\leq|\cnr\bm\tau|\leq 6^{N_1}|\bm\tau|$, for $N_1$ defined as in Remark \ref{rmk:centralsplitbound}.
\end{enumerate}
\end{lemma}
\begin{proof}
\begin{enumerate}
\item $P\tau_{M+M'}$ is comb equivalent to $\tau_N$. If the latter is recurrent, then so is the former; but if the last element of a splitting sequence is recurrent then so are all the previous ones.

\item $\cnr\tau_0$ is defined to be a cornerization of $\tau_0$. If the latter is transversely recurrent, then so is the former; but if the first element of a splitting sequence is transversely recurrent then so are all the following ones.

\item $\cnr\tau_0$ is cornered and, as all wide splits in the sequence are parity splits, the complementary regions of each $\cnr\tau_j$ are diffeomorphic to the ones of $\cnr\tau_0$.

\item Recall the notation for the construction of $P\bm\tau$ set up in Definition \ref{def:cornerizedseq}. Suppose that $\bm{\hat\tau}=(\hat\tau_j)_{j=0}^{N'+N}$ and, for each $i$, $\bm\sigma^i=(\sigma^i_j)_{j=0}^{m_i}$. Let $h:[0,N]\rightarrow [0,N'+N]$ (here and onwards it is understood that intervals are always in $\mathbb Z$) be defined as $h(j)=N'+j$. Let $\omega:[0,N'+N]\rightarrow [0,m_1]$ be defined inductively as follows: $\omega(0)=0$; $\omega(j+1)=\omega(j)+1$ if $\hat\tau_{j+1}$ is obtained from $\hat\tau_j$ with a split, else $\omega(j+1)=\omega(j)$ (i.e. $\omega$ establishes the natural correspondence between the splits of $\bm{\hat\tau}$ and the corresponding wide splits of $\bm\sigma^0$). And let also $\rho: [0,m_R]\rightarrow [0,M+M']$ be defined inductively with these conditions: $\rho(0)=0$; $\rho(j+1)$ is the least index $s>\rho(j)$ such that $P\tau_{s+1}$ is obtained from $P\tau_s$ with a split (so $\rho$ establishes the natural correspondence between $\bm\sigma^R$ and $P\bm\tau$).

For $0\leq i<R-1$, let $f_i:[0,m_i]\rightarrow [0,m_{i+1}]$ be defined as follows. Suppose $\bm\sigma^i=(\ldots,\sigma_s,\sigma_{s+1},\sigma_{s+2},\ldots)$ and $\bm\sigma^{i+1}=(\ldots,\sigma'_s,\ldots,\sigma'_{s+k_i},\ldots)$, with a notation derived from the one used in the statement of Lemma \ref{lem:postpone}. Then we define $f_i(j)=j$ for $j\leq s$; $f_i(j)=j+k_i-2$ for $j\geq s+3$; $f_i(s+1)=s$; $f_i(s+2)=s+k_i-1$ (i.e. $f_i$ maps the indices before and after the parity wide split involved in the postponement to the indices before and after the parity wide split(s) introduced by the postponement, respectively). Note that, for all $j$, $\sigma^i_j$ is either obtained from $\sigma^{i+1}_{f_i(j)}$ with comb moves and a central wide split, or is  comb equivalent to it; whereas $\omega$ and $\rho$ establish a correspondence between tracks which are comb equivalent. So $f\coloneqq\rho\circ f_{R-1}\circ\cdots\circ f_0\circ\omega$ is such that $f(0)=0$, and also $f(m_0)=M$ because $M$ is the lowest index such that the following moves in $P\bm\tau$ are all combs or central splits, which is what one must get, due to the descriptions of the maps given. The map $f$ is increasing because all the composed maps are. Also, $\tau_j$ is obtained from $\cnr\tau_{f(j)}$ with a sequence of central splits and so, by properties 2. and 3., the latter is a cornerization of the former.

\item Lemma \ref{lem:postpone} and Remark \ref{rmk:centralsplitbound} together yield that $P\tau$ involves at least as many splits as $\tau$, and the number of central ones is unvaried: hence the first inequality. For the second one, consider for each $i$ the numbers $p(i)$ counting how many parity wide splits occur between $\sigma^i_{\ell(i)+1}$ and the end of the sequence $\bm\sigma^i$; and $c(i)$ counting the number of consecutive wide central splits at the end of the sequence $\bm\sigma^i$. For each $0\leq i <R$ one of the following is true: either $p(i+1)=p(i)-1$, and in this case $c(i)=c(i+1)$; or $p(i)=1,p(i+1)\geq 1$, and in this case $c(i+1)=c(i)+1$. Let $i_1,\ldots,i_s$ be the sequence of indices $i$ such that the second scenario occurs: necessarily $s\leq N_1$. For notational convenience, denote also $i_0=0$ and $i_{s+1}=R$. Between any $i_j,i_{j+1}$ the number $p(i)$ is strictly decreasing, and is undefined for $i=R$, yielding that $|\bm\sigma^{i_{j+1}}|\leq 6|\bm\sigma^{i_j}|$ for all $0\leq j\leq s$, because each old parity split has been replaced with at most six ones. Hence $|\bm\sigma^R|\leq 6^{N_1}|\bm\sigma^0|$.

Switching between wide and regular split sequences leaves the number of splits unaltered.
\end{enumerate}
\end{proof}

\subsection{Diagonal extensions}
\label{sub:diagext}
We will list some technical lemmas by Masur and Minsky, which require the following definitions:
\begin{defin}\label{def:diagext}
Let $\tau$ be a recurrent almost track that fills $S$. A \nw{diagonal extension} of $\tau$ is an almost track $\sigma$ of which $\tau$ is a subtrack, with the properties that
\begin{itemize}
\item if a branch $b$ of $\sigma$ is not included in $\tau$, then its interior lies entirely in a connected component of $S\setminus \bar\nei_0(\tau)$ (and its endpoints, in particular, are switches of $\tau$);
\item each peripheral annulus component in $S\setminus\nei(\sigma)$ is also a peripheral annulus component in $S\setminus\nei(\tau)$.
\end{itemize}

The set of all \emph{recurrent} diagonal extensions of $\tau$ will be denoted $\e(\tau)$. We will, furthermore, denote
$$\f(\tau)\coloneqq\bigcup_{\substack{\rho\text{ recurrent subtrack of }\tau\\ \rho\text{ fills }S}}\e(\rho)$$
(this differs from the notation employed in \cite{masurminskyi}, \cite{masurminskyq} where the same set is called $\mathrm{N}(\tau)$). Consequently it is possible to define $\ce(\tau)=\bigcup_{\rho\in \e(\tau)} \cc(\rho)$ and $\cf(\tau)=\bigcup_{\substack{\rho\text{ recurrent subtrack of }\tau\\ \rho\text{ fills }S}}\ce(\rho)$.

For $k\in\mathbb N$ and $\tau$ a recurrent almost track, let 
$$\ce_k(\tau)\coloneqq \bigcup_{\delta\in \e(\tau)} \{\gamma\in \cc(\delta)\mid \mu_\gamma(b)\geq k\text{ for each } b\in\br(\tau)\};$$
where, for each $\delta\in \e(\tau)$, $\mu_\gamma$ is understood to be the element corresponding to $\gamma$ in ${\mathcal M}_{\mathbb Q}(\delta)$: so $\ce_k(\tau)$ is the set of the curves $\gamma$ in $\ce(\tau)$ with the property that, among the diagonal extensions which carry it, there is one where $\gamma$ traverses each branch in the subtrack $\tau$ at least $k$ times.

Let also $\cf_k(\tau)\coloneqq \bigcup_{\substack{\rho\text{ recurrent subtrack of }\tau\\ \rho\text{ fills }S}} \ce_k(\rho)$.
\end{defin}

\begin{lemma}[\cite{masurminskyi}, Lemma 4.2]\label{lem:cf_decreasing}
Let $\tau,\tau'$ be recurrent train tracks on a surface $S$, and let $X\subseteq S$ be a subsurface. Suppose that $\tau'|X$ is carried by $\tau|X$, and that they both fill $S^X$ (or, equivalently, $X$). Then $\cf(\tau'|X)\subseteq \cf(\tau|X)$ and, if $\tau'|X$ is fully carried by $\tau|X$, then $\ce(\tau'|X)\subseteq \ce(\tau|X)$.

More specifically, if $\delta'\in \f(\tau'|X)$ is a diagonal extension of a subtrack $\sigma'$ of $\tau'|X$ which fills $X$, then $\delta'$ is fully carried by  $\delta\f(\tau|X)$ a diagonal extension of a subtrack $\sigma$ of $\tau|X$ which again fills $X$; and $\sigma'$ is fully carried by $\sigma$.
\end{lemma}
The original lemma has no reference to subsurfaces and induced tracks, but its proof works equally well to prove the above statement replacing occurrences of $S,\sigma,\tau$ there with $S^X,\tau'|X,\tau|X$ respectively, even if induced train tracks, being almost tracks, may have paths encircling some peripheral annuli.

Note that the notion of tie neighbourhood in \cite{masurminskyi} is slightly different from ours, as it is similar to our notion of $\bar\nei_0(\tau)$. Some terminology is different, too: note in particular that train tracks filling the surface they lie on are called \emph{large}, while the sets we denote $\ce,\cf$ correspond to the ones called $PE,PN$ there, respectively (they are, more generically, sets of \emph{transverse measures}, but it is not important). The last sentence in the statement above is not mentioned in the original statement, but is deducible from its proof.

\begin{lemma}[\cite{masurminskyi}, Lemma 4.5]\label{lem:diag_inters_control}
Let $\tau$ be a recurrent train track on a surface $S$, and let $X\subseteq S$ be a subsurface such that $\tau|X$ fills $S^X$. If $\alpha\in\ce(\tau|X)$ and $\beta\in\cc(X)$ is not carried by any diagonal extension of $\tau|X$, then $i(\alpha,\beta)\geq \min_{b\in\br(\tau)}\mu_\alpha(b)$.
\end{lemma}
Similarly as for the previous lemma, the original statement is not meant for induced train tracks. Again the proof goes through with the same modifications pointed out above, but it is worth highlighting some further points. In particular, from the beginning of the proof, we need that all biinfinite train paths along the complete lift of $\tau|X$ in $\Hy^2$ are uniformly quasi-geodesic: this is the content of Proposition \ref{prp:paths_quasi_geod}.

Another point that is worth marking concerns what sort of shapes $\Hy^2$ is cut into by $\widetilde{\tau|X}$. Since $\tau|X$ fills $S^X$, each of its complementary components in $S^X$ either has a polygon as its closure, or is peripheral (meaning that its closure in $\ol{S^X}$ includes an arc of $\partial\ol{S^X}$). Since polygons lift diffeomorphically, a similar property is true for $\widetilde{\tau|X}$ in $\Hy^2$. Any peripheral component $P$ of $\Hy^2\setminus \widetilde{\tau|X}$ must be part of $\mathcal H_\pm$, because the endpoints of $\tilde\beta$ cannot lie in $\partial\bar\Hy^2\cap P$: that would mean that $\beta$ is not a closed curve in $S^X$ but a properly embedded, and non-compact arc. These remarks ensure that the original proof works for our generalized statement.

\begin{lemma}[\cite{masurminskyi}, Lemma 4.4]\label{lem:ccnesting}
Let $\tau$ be a recurrent train track on a surface $S$, and let $X\subseteq S$ be a subsurface such that $\tau|X$ fills $X$. Then:
\begin{itemize}
\item if $X\cong S_{0,4}$ then $\nei_1(\ce_3(\tau|X))\subset \ce(\tau|X)$;
\item if $X\cong S_{1,1}$ then $\nei_1(\ce_2(\tau|X))\subset \ce(\tau|X)$;
\item if $X$ is of any other topological type, then $\nei_1(\ce_1(\tau|X))\subset \ce(\tau|X)$.
\end{itemize}
\end{lemma}
Here, with $\nei_k(\cdot)$ we mean the set of points at distance $\leq k$ from the given subset of $\cc(X)$.

\begin{proof}
Let us focus with the first case. According to the above lemma, for each essential closed curves $\alpha\in \ce_3(\tau|X)$ and $\beta\not\in \ce(\tau|X)$ we must have $i(\alpha,\beta)\geq 3$. So $d_\cc(\alpha,\beta)\geq 2$, and this proves the claim. The same argument proves the claim in the other cases.
\end{proof}

Recall that in \cite{masurminskyi}, for two train tracks $\sigma,\tau$ on a same surface, the authors define $d_\cc(\tau,\sigma)\coloneqq \min_{\beta\in V(\tau),\alpha\in V(\sigma)} d_\cc(\beta,\alpha)$.
\begin{lemma}[\cite{masurminskyi}, Lemma 4.7 (Nesting Lemma)]
There is a constant $D=D(S)$ such that if $\sigma$ and $\tau$ are two recurrent train tracks filling $S$, with $\sigma$ carried by $\tau$ and $d_\cc\left(V(\tau),V(\sigma)\right)>D$, then:
\begin{itemize}
\item for $S\cong S_{0,4}$, $\cf(\sigma)\subset \cf_3(\tau)$;
\item for $S\cong S_{1,1}$, $\cf(\sigma)\subset \cf_2(\tau)$;
\item for all other topological types, $\cf(\sigma)\subset \cf_1(\tau)$.
\end{itemize}
\end{lemma}
\begin{proof}
The third statement is the actual content of the lemma in \cite{masurminskyi}. For what concerns the other two statements, a simplified proof is possible. Recall, first of all, that $\cc(S_{0,4})$ and $\cc(S_{1,1})$ are isomorphic to the Farey graph (see for instance \cite{farb}, \S 4.1.1). In \cite{ibaraki}, Figure 10, there is a classification of the diffeomorphism types of train tracks on $S_{0,4}$ which are \emph{complete} i.e. are not subtracks of any other train track; whereas the only one in $S_{1,1}$ is given in Figure 4 there. This means that any other train track on these surfaces is diffeomorphic to a subtrack of the given ones; but it is easy to note that none of those fills the surface. So, for each track $\tau$ that fills $S_{0,4}$ or $S_{1,1}$ respectively, $\{\tau\}=\e(\tau)=\f(\tau)$.

For each of the given train tracks $\tau$ in $S_{0,4}$ (resp. in the only track given in $S_{1,1}$), the two vertex cycles $\alpha_1,\alpha_2$ intersect in 2 (resp. 1) points, so they are connected by an edge in $\cc(S_{0,4})$ (resp. $\cc(S_{1,1}$). It is therefore possible to enforce an identification of the curve complex with the Farey graph such that $\alpha_1,\alpha_2$ correspond to $0/1$ and $1/0$, respectively.

Let $\beta\in\cc(S)$ (where $S=S_{0,4}$ or $S_{1,1}$ accordingly) be a curve carried by $\sigma$, and for $j=1,2$ let $b_j$ the weight that $\beta$ assigns to a branch of $\tau$ which is traversed by $\alpha_j$ but not $\alpha_{3-j}$ (it will be the same for any such branch). Then, as an element of the Farey graph, $\beta$ corresponds to $\pm b_2/b_1$. If $\beta\not\in \cf_3(\tau)$ then either $b_1$ or $b_2$ is $\leq 2$. Suppose, without loss of generality, that $b_2\leq 2$. If $b_2=1$ then $d_{\mathrm{Farey}}(\pm b_1/b_2,1/0)=1$. If $b_2=2$ then $\lfloor b_1/2\rfloor/1$ lies at distance $1$ from both $\pm b_1/b_2$, $1/0$. This proves the claim.
\end{proof}

\begin{lemma}[\cite{masurminskyq}, Lemma 3.5]\label{lem:vertexnotinterior}
Let $\tau$ be a recurrent train track on a surface $S$, and let $X\subseteq S$ be a subsurface such that $\tau|X$ fills $X$. Then, if $\alpha \in V(\tau|X)$ then $\alpha\not\in \cf_1(\tau|X)$.
\end{lemma}
Again the original given proof works for induced train tracks, with the same replacement as the one pointed out for Lemma \ref{lem:cf_decreasing}. But, again, it is worth noting some points: first of all, $V(\tau|X)$ must consist of at least $2$ elements else it cannot fill $S^X$; but then, $\cc(\tau|X)$ does neither (by Lemma \ref{lem:decreasingfilling}). This replaces the argument used at the beginning of the original proof to show that the set of projective transverse measures has dimension $\geq 2$.

Secondly, the proof appeals to the injectivity of a map $P(\omega)\rightarrow\mathcal{ML}(S)$, basically the inverse map of the one mentioned in Proposition \ref{prp:measurecurvecorresp}. In order to suit better our setting, it may be more straightforward to say, to obtain the contradiction constructed in the original proof, that $v\in\cc(X)$ not having a unique carrying image in $\omega$ is a behaviour that transgresses Corollary \ref{cor:carryingunique}.

The assertion about positive generalized Euler characteristics (i.e. index) of the disc $D''$ is still a contradiction, even if an almost track gives a larger variety of possible complementary regions than in the original setting: this is because the original disc $D$ cannot intersect any lift of a peripheral annulus component of $S^X\setminus (\tau|X)$, anyway.

\section{Good behaviour of splitting sequences}\label{sub:goodbehaviour}

In the following sections we aim to prove the following statement.
\begin{theo}\label{thm:core}
Let $\bm\tau=(\tau_j)_{j=0}^N$ be a splitting sequence of semigeneric, birecurrent train tracks with their vertex sets $V(\tau_j)\in\pa(S)$ for all $0\leq j\leq N$. Then there is a constant $A>1$, depending only on $S$, such that
$$d_{\pa(S)}(V(\tau_0),V(\tau_N))=_A |\utw(\rar(\cnr\bm\tau))|.$$
\end{theo}
Here, $\rar$ and $\utw$ denote, respectively, a \emph{rearrangement} and an \emph{untwisting} of the splitting sequence $\bm\tau$, which we define in \S \ref{sub:rearrang} and in \S \ref{sec:traintrackconclusion}, respectively.

Our result about splitting sequences in a graph which is quasi-isometric to the pants graph comes after several other ones on the same line. The first one is a \emph{structure theorem} for cornered train track splitting sequences. We state it in a slighty altered way. Given $X\subseteq S$ a subsurface, and $\bm\tau=(\tau_j)_{j=0}^N$ a train track splitting sequence indexed by the interval $[0,N]$, let the \nw{accessible interval} for $X$ in $[0,N]$ be defined as follows.
\begin{itemize}
\item Suppose $X$ is not an annulus. Then define 
\begin{eqnarray*}
  m_X & \coloneqq &\min\left\{i\in [0,N]\mid \mathrm{diam}_X\left(\cc^*(\tau_i|X)\right)\geq 3\right\} \\
\text{and } n_X & \coloneqq & \max\left\{i\in [0,N]\mid \mathrm{diam}_X\left(\cc(\tau_i|X)\right)\geq 3\right\}.
\end{eqnarray*}
 If $m_X,n_X$ are both finite and $m_X\leq n_X$, then let $I_X=[m_X,n_X]$. Else set $I_X=\emptyset$.
\item Suppose $X$ is an annulus, with $\gamma$ its core curve. Then define $I_X$ to be the set of indices $i$ such that $\gamma$ is carried and is a twist curve in $\tau_i$. That the set $I_X$ is an actual interval is proved in Lemma \ref{lem:twistcurvebasics}. This definition of $I_X$ for annuli is different from the one of \cite{mms} thus the meaning of our statement is slightly different from the one in the in the original paper; but this new version is easily derived through Lemma \ref{lem:twistininduced}, as we will see. We will also use the notation $I_\gamma$ to mean $I_{\nei(\gamma)}$ where $\gamma\in\cc(S)$ and $\nei(\gamma)$ is a regular neighbourhood of a curve in the isotopy class $\gamma$.
\end{itemize}

\begin{theo}[\cite{mms}, Theorem 5.3]\label{thm:mmsstructure}
Given a splitting sequence $\bm\tau=(\tau_j)_{j=0}^N$ of generic, cornered, birecurrent train tracks on a surface $S$, there is a constant $\mathsf K_0=\mathsf K_0(S)$ such that, for each subsurface $X\subseteq S$, the following properties hold.

\begin{enumerate}
\item Let $[a,b]\subseteq [0,N]$ be an interval of indices, disjoint from $I_X$ except for, possibly, a single point. Then, if $\pi_X(\tau_b)\not=\emptyset$, then $d_X(V(\tau_a),V(\tau_b))\leq \mathsf K_0$. If $X$ is an annulus, also $d_X(\tau_a|X,\tau_b|X)\leq \mathsf K_0$.
\item If $X$ is an annulus, and $i\in I_X$, then $\tau_{i+1}|X$ is obtained from $\tau_i|X$ with slides (if this is the case between $\tau_{i+1}$ and $\tau_i$); or with at most 2 splits and/or taking a subtrack (if there is a split between $\tau_{i}$ and $\tau_{i+1}$).
\item If $X$ is not an annulus and $i\in I_X$, then $\partial X$, when put in efficient position with respect to $\tau_i$, is wide. Moreover $V(\tau_i|X)$ fills $S^X$. Finally, if $\tau_{i+1}|X\not=\tau_i|X$, then $\tau_{i+1}|X$ is obtained from $\tau_i|X$ with slides (if this is the case between $\tau_{i+1}$ and $\tau_i$); or with a split or taking a subtrack (if there is a split between $\tau_{i}$ and $\tau_{i+1}$).
\end{enumerate}
\end{theo}

The differences between this statement and the original one may be summarized into three points. The first one is the definition of accessible interval for an annulus, and this is cared after in Lemma \ref{lem:twistininduced} and the following observation. The other ones, instead, concern the first statement: the interval $[a,b]$ here may share an endpoint with $I_X$, which was not allowed in the original version. But if that statement is true then this one also is, possibly picking a larger value for $\mathsf K_0(S)$ --- see also Remark \ref{rmk:pickparameters} for the existence of a universal bound for the distance induced by a single split. A larger value of $\mathsf K_0$ is necessary also for the last sentence in this statement to be true; but it is possible to choose one, due to Lemmas \ref{lem:vertexsetbounds} and \ref{lem:induction_vertices_commute}.

\begin{rmk}
When a multicurve $\gamma$ is wide (not carried) with respect to a given train track $\tau$, in general, it is not necessarily true that all possible efficient positions for $\gamma$ comply with the conditions defining a wide curve in Definition \ref{def:efficientposition}: the definition of wide curve only asks for an efficient position as such to exist.

When $\gamma=\partial X$ as in the above theorem, anyway, \emph{any} efficient position with respect to $\tau_i$, $i\in I_X$, shows that $\partial X$ is wide. This is a consequence of Lemma 5.2 in \cite{mms}, where the proof works by contradiction, supposing that there exists any efficient position for $\partial X$ which does not show that it is wide.
\end{rmk}

A number of theorems prove that, in different measures, train track splitting sequences induce quasigeodesic in more than one of the graphs previously introduced.

\begin{theo}[\cite{masurminskyq}, Theorem 1.3]\label{thm:mm_cc_geodicity}
Given a train track splitting sequence $\bm\tau=(\tau_j)_{j=0}^N$ on a surface $S$, there is a constant $Q=Q(S)$ such that the set $\left(V(\tau_j)\right)_j$ is a $Q$-unparametrized quasi-geodesic in $\cc(S)$.
\end{theo}
The proof of this theorem employs all the lemmas listed in \S \ref{sub:diagext} above. We spend just a few words about a secondary issue: that is, such proof requires some slight adaptations in order to hold for $S_{0,4}$ and $S_{1,1}$, whose curve graph have specific definitions. In order to cope with those surfaces, we may employ the lemmas in \S \ref{sub:diagext}: i.e. read the theorem's proof replacing the lemmas employed there with the corresponding ones in \S\ref{sub:diagext}. Also, we need to replace each occurrence of  $\mathcal PN(\cdot)$ with $\cf(\cdot)$; and each occurrence of $int(\mathcal PN(\cdot)$ with $\cf_3(\cdot)$ or $\cf_2(\cdot)$ depending on whether we aim to prove the statement for $S_{0,4}$ or $S_{1,1}$, respectively.

The same theorem is true for subsurface projections (adding some technical hypotheses):
\begin{theo}[\cite{mms}, Theorem 5.5]\label{thm:mms_cc_geodicity}
Given a splitting sequence $\bm\tau=(\tau_j)_{j=0}^N$ of cornered, birecurrent train tracks on a surface $S$, there is a constant $Q=Q(S)$ such that, for each subsurface $X\subseteq S$ such that $\pi_X(V(\tau_N))\not=\emptyset$, the sequence $\left(\pi_XV(\tau_j)\right)_{j=0}^N$ is a $Q$-unparametrized quasi-geodesic in $\cc(X)$.
\end{theo}

And a stronger statement is true for the marking graph: it is the starting point for our Theorem \ref{thm:core}.
\begin{theo}[\cite{mms}, Theorem 6.1]\label{thm:mms_main}
Given a splitting sequence $\bm\tau=(\tau_j)_{j=0}^N$ of cornered, birecurrent train tracks on a surface $S$, whose vertex sets each fill $S$, there is a constant $Q=Q(S)$ such that the set $\left(V(\tau_j)\right)_{j\in J}$ is a $Q$-quasi-geodesic in $\ma(S)$.

Here $J\subseteq [0,N-1]$ is the set of indices $j$ such that $\tau_j$ splits to $\tau_{j+1}$
\end{theo}

Theorems \ref{thm:mmsstructure} and \ref{thm:mm_cc_geodicity} are employed in the proof of \ref{thm:mms_cc_geodicity}, and \ref{thm:mms_main} depends on all the previous three.