\chapter*{Preface}

Possibly the most interesting questions related to the topology and geometry of surfaces are the ones arising when looking for bonds among geometric properties of the many spaces and groups which may be associated to a surface. In particular one may think of the \emph{Teichm\"uller space} (with either the Teichm\"uller or the Weil-Petersson metric), the \emph{mapping class group}, and a number of graphs: the \emph{curve graph}, the \emph{arc graph}, the \emph{pants graph} and the \emph{marking graph}. A summary of the network of natural maps and quasi-isometries between these objects may be found in \cite{duchin}, for instance. As this network of maps helps understanding one object's geometry through another one, the mentioned graphs give combinatorial, more manageable models for the Teichm\"uller space and the mapping class group. The pants graph, in particular, has been shown by J. Brock (\cite{brock1}) to be quasi-isometric to the Teichm\"uller space with the Weil-Petersson metric.

In the setting described above \emph{train tracks}, introduced by W.P. Thurston, have been employed for several steps towards a concrete comprehension of the surface-related graphs. A train track on a surface is a 1-complex which, just like a railway network, may be travelled smoothly in infinitely many different ways, making different choices when a switch is met. Travelling along a train track one may describe loops, or infinite paths with bounded curvature which can be straightened to geodesics. The simplest loops one may describe when travelling along a train track are finitely many, and they are called \emph{vertex cycles}.

When a train track is repeatedly altered via \emph{elementary moves} to get a so-called \emph{splitting sequence}, the change produced on the set of vertex cycles seems to `proceed towards a definite direction' in the surface-related graphs listed above. While a summary of the results in this area is given in \S \ref{sec:role_train_tracks}, there in one in particular, Theorem 6.1 in \cite{mms}, which motivates this work (here it is stated imprecisely):
\begin{theono}
Given a splitting sequence $\bm\tau$ of train tracks on a surface $S$, whose vertex sets fill $S$, each of these sets is a vertex of $\mc(S)$, the marking graph of $S$. There is a constant $Q=Q(S)$ such that the sequence of vertex sets of tracks in $\bm\tau$ moves along a $Q$-quasi-geodesic in $\mc(S)$.
\end{theono}

The definition of marking graph used in \cite{mms} is not the most common definition, to be found in \cite{masurminskyii}, but it gives a quasi-isometric graph, and is fitted for families of curves arising as vertex sets. We may apply a similar trick for the pants graph $\pc(S)$, so that the vertex sets of a large family of train tracks are vertices of this quasi-isometric, different version of $\pc(S)$: we denote it $\pa(S)$. Actually, the vertices of $\pc(S)$ will inject to the vertices of $\pa(S)$.

One may think, then, that a similar result as the one above may hold in $\pa(S)$. But there must be some phenomenon which obstruct the vertex sets along a train track splitting sequence from giving a quasi-geodesic in this graph.

The key to understanding this obstruction is the hierarchy machinery, and Theorem 6.12 in particular, of \cite{masurminskyii}. It applies for estimating distances both in the pants graph and in the marking graph of a fixed surface $S$; however, while distances in the marking graph are estimated via a summation over all distances induced in the curve complexes of subsurfaces of $S$, \emph{including annuli}, annuli are to be excluded when estimating distances in the pants graph. 

This gap between the two summations suggests that, in order to generalize the result from \cite{mms} to a statement valid for the pants graph, one has to control the contribution given by annuli in the summation. This is the idea behind the main theorem of this work i.e. Theorem \ref{thm:core}, here stated in a simplified way:

\begin{theono}
Let $\bm\tau=(\tau_j)_{j=0}^N$ be a splitting sequence of train tracks with their vertex sets $V(\tau_j)\in\pa(S)$ for all $0\leq j\leq N$. Then there is a number $A>1$, depending on $S$, such that
$$\frac{1}{A}|\utw(\rar\bm\tau)|-A\leq d_{\pa(S)}(V(\tau_0),V(\tau_N))\leq A|\utw(\rar\bm\tau)|+A.$$
\end{theono}

In this theorem, $\rar$ and $\utw$ are operations which turn a splitting sequence into a new one, and $|\cdot|$ denotes the number of elementary moves which are \emph{splits} (i.e. the non-invertible kind of elementary move). Loosely speaking, the difference between $\rar\bm\tau$ and $\bm\tau$ is that the elementary moves are performed in a different order: this way, every time there is annulus in $S$ such that $\bm\tau$ spans a high distance in the annulus' curve complex, in $\rar\bm\tau$ this distance appears as the result of a splitting sequence which realizes a series of many Dehn twists in a row.

$\utw(\rar\bm\tau)$ is obtained from $\rar\bm\tau$ via removal of the majority of these Dehn twists, in such a way that a splitting sequence is obtained anyway. This operation kills all overly high annulus contributions in the summation of Theorem 6.12 of \cite{masurminskyii} but produces a new splitting sequence which retains several properties of the old one.

Not only is our theorem similar in spirit to the aforementioned one from \cite{mms} but, once the operations $\rar$ and $\utw$ are defined, it may be proved using essentially the same line of proof. A major adaptation is necessary as the proof in \cite{mms} makes use of local finiteness in $\mc(S)$, while neither $\pc(S)$ nor $\pa(S)$ has this property; however, our sequence $\utw(\rar\bm\tau)$ has the property that, if a subsequence of it gives bounded distance in $\pa(S)$, it gives bounded distance in $\ma(S)$, too: this is the aim of our constructions.

It must be stressed that the three steps necessary to get the pants distance --- i.e. $\rar$, $\utw$ and $|\cdot|$ --- may be computed algorithmically from $\bm\tau$ using the definitions and proofs included in this work. So we give an effective way to compute the distance covered in the pants graph by a splitting sequence. More generally, given any two vertices of $\pc(S)$, one may always define a splitting sequence whose endpoints in $\pa(S)$ lie close to the selected vertices.

This way, the machinery in this work will be useful to get distances in the Weil-Petersson metric, too, as noted above; but we do not develop this aspect. We consider, instead, an application of the above for computation of volumes of hyperbolic 3-manifolds. According to a theorem proven in \cite{brock2}, the hyperbolic volume of a mapping torus over a surface $S$, defined by a pseudo-Anosov $\psi:S\rightarrow S$, is given (up to constants) by the minimum displacement induced in $\pc(S)$ by $\psi$.

Although that result requires extracting the \emph{minimum} displacement, which is a hard operation in general, it is possible to derive a couple of interesting corollaries from our machinery: a step towards effective computation of volume of hyperbolic mapping tori. In \cite{agol_pa}, given $\psi$, a standard method is described to get an infinite splitting sequence which is `preperiodic up to application of $\psi$', i.e. it may be subdivided into a `preperiod' followed by chunks of equal length, such that all entries in a given chunk are obtained from the previous one applying $\psi$. In Theorem \ref{thm:agol_volume} we prove that the `period' $\bm\rho$ in this sequence is a splitting sequence such that the distance in $\pa(S)$ between its extremes is close to the requested minimum. So the hyperbolic volume of the mapping torus is, up to additive and multiplicative constants, given by $|\utw(\rar\bm\rho)|$.

We also sketch a way of estimating hyperbolic volume for complements of closed braids in a solid torus, which are a special case of mapping tori. In this case there is another splitting sequence to be considered, the one deriving from the \emph{trasmission} and \emph{relaxing} technique for \emph{interval identification systems}, as defined in \cite{dynnikovwiest}. We explain briefly that their formalism is equivalent to train tracks except that, in order to compute pants distance, no rearrangement operation on the same line as $\rar$ is needed; and we give a few considerations about how hyperbolic volume relates with this formalism.

An outline of the contents of this work follows.

Chapter 1 traces the background the present work is built on. After giving some definitions about surfaces and 3-manifolds, aimed mainly at fixing the most basic terminology, in \S \ref{sec:graphs} we sum up all that is necessary to know about the curve graph, the marking graph and the pants graph. In particular we describe what is a subsurface projection and we give the statement of Theorem 6.12 of \cite{masurminskyii}. The section also includes some original work, as we define $\pa(S)$ and prove that it is quasi-isometric to $\pc(S)$.

In \S \ref{sec:hypvolumehistoric} we outline some previous work about estimates of hyperbolic 3-volume. In particular we focus on the \emph{guts} approach by Agol, and on the way it reflects on the volume estimates for link complements found by Lackenby and by Futer--Kalfagianni--Purcell. Then we switch to the setting of mapping tori, and give the full statement of the volume estimate in terms of distance in the pants graph, as found by Brock. Finally, in \ref{sec:role_train_tracks} we give a very short outline of the results about surface-related graphs which involve train tracks.

Chapter 2 is concerned with the main content of this work: the proof of Theorem \ref{thm:core}. After giving the basic definitions about train tracks, carried curves, and elementary moves, in \S \ref{sec:traintracksmore} we give some more involved, but still general constructions. The first one is a `subsurface projection' for train tracks, the \emph{induction} as defined in \cite{mms}. We also prove a list many basic properties of induced train tracks, in particular the ones induced on an annular subsurface. The second construction is a different viewpoint on elementary moves: an elementary move may be considered as the result of cutting the track open along a \emph{zipper}. This viewpoint will be convenient to describe a number of rearrangement procedures of elementary moves. In the third subsection we explain how to reduce train tracks to the kind treated in \cite{mms}, where each connected component of the track complement has a boundary with a corner on each component. In the fourth subsection we recover some lemmas concerning diagonal extensions of train tracks from the work of Masur and Minsky, revisiting them in terms of almost tracks or induced train tracks, according to our needs.

Once all the necessary terminology is given, in \S \ref{sub:goodbehaviour} we list the results about train tracks and geodicity of splitting sequences in the curve graph and in the marking graph, which are relevant to the present work.

In \S \ref{sec:twistcurves} we perform a deep analysis of \emph{twist curves}, in order to define the rearrangement $\rar$. We have already mentioned before that, given a splitting sequence $\bm\tau$, and the collections $V(\tau_j)$ of vertex sets for each track in the sequence, we wish to control the growth of $d_X\left(V(\tau_j),V(\tau_{j'})\right)$ for $X\subset S$ an annulus. It turns out that this distance may be high only if the core curve $\gamma$ of $X$ is a twist curve at some point of the splitting sequence. We consider twist curves as curves that, move after move, are able to produce high powers of a Dehn twist (or of a Dehn twist inverse). However, the moves producing these Dehn twists may be very sparse along $\bm\tau$, so it is convenient, for us to have a control on them, to group them consecutively. This requires a fair amount of work, because it is necessary to analyse minutely all the possible dynamics a splitting sequence may show around $\gamma$. In particular we set up some conventions to avoid the ambiguities caused by having every elementary move defined only up to isotopies. Then we show that twist moves are determined by \emph{twist modelling functions}, and we analyze how the evolution of $\bm\tau$ causes a movement in $\cc(X)$, the annulus' `curve complex', which proceeds always in the same direction, when $\gamma$ is a twist curve.

We define the \emph{rotation number} to quantify `the number of entire Dehn twists taking place about a twist curve', however sparse they may be, and finally we give a procedure to concentrate almost all the rotation number in a chunk of the splitting sequence where nothing occurs but Dehn twists about $\gamma$. We then define $\rar\bm\tau$ applying this procedure to all annuli $X$ which show a high distance $d_X\left(V(\tau_0),V(\tau_N)\right)$ between the extremes of the splitting sequence.

The last subsection bounds the number of these annuli in terms of the pants distance covered by the splitting sequence. It is inspired by an idea employed in \cite{masurminskyq}. This bound is necessary only as a technical step, because Theorem \ref{thm:core} would result into a better (linear) bound as an immediate corollary.

In \S \ref{sec:traintrackconclusion} we introduce the \emph{untwisted sequence} $\utw(\rar\bm\tau)$: the idea in its definition is that it shall closely mimic $\rar\bm\tau$, except that the chunks of sequence expressing Dehn twists will have a capped length. The splitting sequences $\utw(\rar\bm\tau)$ and $\rar\bm\tau$ begin with the same track and end with different ones; however, we prove that they share several properties, and the distances they cover in $\pa(S)$ are bounded in terms of one another. Crucially, $\utw(\rar\bm\tau)$ covers distances in $\ma(S)$ which are bounded in terms of the ones covered in $\pa(S)$.

This allows us to proceed to the proof of Theorem \ref{thm:core}: in general we need to subdivide a splitting sequence $\bm\tau$ into chunks such that each vertex set fills the same subsurface of $S$. Then, with an interplay between $\rar\bm\tau$ and $\utw(\rar\bm\tau)$, we are finally able to revisit the proof of Theorem 6.1 in \cite{mms} to suit the pants graph setting.

In Chapter 3 we connect Theorem 6.1 with the problem of estimating the hyperbolic volume of mapping tori. We first prove Theorem \ref{thm:agol_volume} about Agol's maximal splitting sequence, as mentioned before; then, in \S \ref{sec:dw}, we turn to the simpler case of punctured discs, and mapping tori which may be described as the complement of a closed braid in a solid torus. We describe how to turn Dynnikov and Wiest's transmissions of interval identification systems into a train track splitting sequence and prove Corollaries \ref{cor:dwgivesdistance} about pants distance and \ref{cor:dwgivesvolume} about hyperbolic volume.

Then we sketch some further properties, including Proposition \ref{prp:futervolume}, which is a volume formula in terms of words in the braid group $B_n$ under a suitable generating set, and of which David Futer has an independent proof.