\section{Untwisted sequences. Proof of the main statement}\label{sec:traintrackconclusion}

\ul{Note:} Consistently with \S \ref{sec:twistcurves}, we work in the setting of \emph{generic} train tracks only. This is not restrictive, however, since a semigeneric splitting sequence can always be converted to a generic one, to which the main result (Theorem \ref{thm:main_full}) applies.

In this section we will explain how to deprive a splitting sequence of a high number of Dehn twist, and then we show that this sequence is suitable for application of the same techniques of proof of quasi-geodicity as in \cite{mms}, Theorem 6.12.

\subsection{The untwisted sequence}\label{sub:untwistedsequence}
\begin{defin}\label{def:untwistedsequence}
Let $S$ be a surface, and let $\bm\tau=(\tau_j)_{j=0}^N$ be a generic splitting sequence of train tracks evolving firmly in a subsurface $S'$, not necessarily connected. Let $\gamma_1,\ldots,\gamma_r\subseteq \cc(\tau_0)$ be curves all contained in $S'$, and suppose that $\bm\tau$ is $(\gamma_1,\ldots,\gamma_r)$-arranged. In particular the sequence $(\max DI_t)_{t=1}^r$ is increasing.

Let $m_t\coloneqq \rot_{\bm\tau}(\gamma_t;DI_t)\geq 2\mathsf{K}_0+4$, let $\epsilon_t$ be the sign of $\gamma_t$ as a twist curve in $\bm\tau$, and let $g_t\coloneqq \max DI_t(0)-\min DI_t(0)$ be the `period length' in the sequence $DI_t$.

Define recursively $\phi_0\coloneqq \mathrm{id}_S:S\rightarrow S$; and, for $1\leq t \leq r$, $\phi_t\coloneqq D_{\gamma_t}^{\epsilon_t(2\mathsf{K}_0+4-m_t)}\circ \phi_{t-1}$.

Let $NI_0\coloneqq[0,\min DI_1]$; $NI_t\coloneqq [\max DI_t,\min DI_{t+1}]$ for $1\leq t\leq r-1$; $NI_r\coloneqq [\max DI_r,N]$; $DK_t\coloneqq DI_t(0)\cup \ldots \cup DI_t(2\mathsf{K}_0+3)$ and $DL_t\coloneqq DI_t(m_t-2\mathsf{K}_0-4)\cup \ldots \cup DI_t(m_t-1)$ for $1\leq t\leq r$. Define the \nw{untwisting} of $\bm\tau$ as
\begin{align*}
\utw\bm\tau \coloneqq\  & \bm\tau(NI_0)*\\
 &  * \left(\phi_1\cdot \bm\tau(DL_1)\right) * \left(\phi_1\cdot\bm\tau(NI_1)\right) *\\
 & \ldots  \\
 & * \left(\phi_{r-1}\cdot\bm\tau(DL_{r-1})\right) * \left(\phi_{r-1}\cdot\bm\tau(NI_{r-1})\right) * \\
 & * \left(\phi_r\cdot\bm\tau(DL_r)\right) * \left(\phi_r\cdot\bm\tau(NI_r)\right).
\end{align*}

In the above notation, $\phi_t\cdot \bm\tau(\ldots)$ means the splitting sequence obtained from $\bm\tau(\ldots)$ via application of $\phi_t$ to all its entries.
\end{defin}

If $\utw\bm\tau=(\utw\tau_j)_{j=0}^{N'}$, the above definition provides a natural subdivision of $[0,N']$ into subintervals
$$
NI_0^\utw, DI_1^\utw,NI_2^\utw,DI_2^\utw,\ldots,NI_{r-1}^\utw,DI_r^\utw, NI_r^\utw
$$
where the maximum of each subinterval is the minimum of the following one. Each interval $DI_t^\utw$ has twist nature with respect to $\phi_t(\gamma_t)$, and may be subdivided into $DI_t^\utw(0),\ldots, DI_t^\utw(2\mathsf{K}_0+3)$. For all $1\leq t\leq r+1$, there is a natural bijection between $NI_t$ and $NI_t^\utw$; and, if $t\not=0$, between $DI_t(m_t-2\mathsf{K}_0-4+s)$ and $DI_t^\utw(s)$ for all $0\leq s \leq 2\mathsf{K}_0+3$. In order to keep the employed notation simple, let $DI_0=DI_0^\utw=DK_0=DL_0=\{0\}$. If $j\in NI_t$ (resp. $DL_t$) let $\dn j$ be the corresponding index in $NI_t^\utw$ (resp. $DI_t^\utw$). Extend $\dn:[0,N]\rightarrow [0,N']$ in the only way that gives a monotonic map. For $j'\in [0,N']$ let $\up j'$ be the least of the indices $j\in[0,N]$ such that $\dn j=j'$. Note that there is always a $t$ such that $\up j'\in DL_t\cup NI_t$.

For $j\in [0,N]$, let $t(j)$ be the least index $t$ such that $j\in DI_t$ or $j\in NI_t$. For $j'\in [0,N']$, let $\utw t(j')$ be the least index $t$ such that $j\in DI_t^\utw$ or $j\in NI_t^\utw$. Then, for all $j\in \bigcup_{t=0}^{r} (DL_t\cup NI_t)$, $\utw\tau_{\dn j}=\phi_{t(j)}(\tau_j)$; and for all $j'\in[0,N']$, $\utw\tau_{j'}=\phi_{\utw t(j')}(\tau_{\up j'})$.

Anyway note that each interval $DI^\utw_t$ can be made to correspond not only to $DL_t$, but to $DK_t$ as well: for all $1\leq t \leq r$, if $j\in DK_t$, and $j'\coloneqq \min DI_t^\utw + (j - \min DI_t)$, then $\utw\tau_{j'}=\phi_{t-1}(\tau_j)$. This means that, for all $0\leq t \leq r-1$,
$$
\utw\bm\tau(DI_t^\utw\cup NI_t^\utw\cup DI_{t+1}^\utw) = \phi_t\cdot  \bm\tau\left(DL_t\cup NI_t\cup DK_{t+1}\right).
$$

Also, for $0\leq t\leq r-1$, let $[t]\dn:[0,N]\rightarrow [0,N']$ be defined as a correspondence that exploits this identity: $[t]\dn j\coloneqq j-\min DK_{t+1}+\min DK_{t+1}^\utw$ if $j\in DK_{t+1}$; $[t]\dn j\coloneqq \max DI_{t+1}^\utw$ if $j\in DI_{t+1}\setminus DK_{t+1}$; and $[t]\dn j\coloneqq \dn j$ otherwise. With this correspondence, $\utw\tau_{[t]\dn j}=\phi_t(\tau_j)$ for all $j\in DL_t\cup NI_t\cup DK_{t+1}$.

Define also $[t]\up:[0,N']\rightarrow [0,N]$ by setting $[t]\up j'$ to be the least $j\in[0,N]$ such that $[t]\dn j=j'$. Define $[r]\dn\coloneqq \dn$ and $[r]\up\coloneqq\up$.

For $X\subseteq S$ a subsurface, denote $I_X^\utw$ the accessible interval of $X$ in $\utw\bm\tau$.

\ul{Note:} most of the time in this section we will deal with an arranged splitting sequence $\bm\tau$ and its respective untwisted sequence $\utw\bm\tau$, as above. In order to simplify notations for distances along these two splitting sequences we will adopt similar ones as in \cite{mms}.
\begin{itemize}
\item If $i,j\in [0,N]$ and $Y\subseteq S$ is a subsurface, $d_Y(i,j)\coloneqq d_Y(\tau_i,\tau_j)$, and similarly for $d_{\pa(Y)}$, $d'_{\pa(Y)}$ and other distances in graphs. If $I\subseteq [0,N]$ is a subinterval, $d_Y(I)\coloneqq d_Y(\min I, \max I)$ etc.
\item If $i,j\in [0,N']$ and $X\subseteq S$ is a subsurface, $d_X(i,j)^\utw\coloneqq d_X(\utw\tau_i,\utw\tau_j)$, and similarly for $d_{\pa(X)}(\ldots,\ldots)^\utw$, $d'_{\pa(X)}(\ldots,\ldots)^\utw$ and other distances in graphs. If $I\subseteq [0,N]$ is a subinterval, $d_Y(I)^\utw\coloneqq d_Y(\min I, \max I)^\utw$ etc.
\end{itemize}

In a bit we will need a version of untwisting for train track splitting sequences which \emph{do not} evolve firmly in any subsurface of $X$. Let $\bm\tau=(\tau_j)_{j=0}^N$ be a generic splitting sequence of cornered birecurrent train tracks: $\bm\tau$ can be seen as a concatenation $\bm\tau^1*\bm\epsilon^2*\bm\tau^2*\ldots*\bm\epsilon^w*\bm\tau^w$ where each $\bm\tau^i$ evolves firmly in a fixed subsurface of $S$ and each $\bm\epsilon^u$ is a single split, say from a track $\tau_j$ to $\tau_{j+1}$, such that $V(\tau_j)$ fills a surface strictly and essentially containing the one filled by $\tau_{j+1}$ --- thus the complexity of the former is higher than the one of the latter. This decomposition is possible because, as it has been pointed out after Lemma \ref{lem:decreasingfilling}, the subsurfaces filled by $V(\tau_j)$, for $0\leq j\leq N$, are a decreasing family with respect to the inclusion. Moreover, the number $w\leq \xi(S)$; and the single split in each $\bm\epsilon_u$ may induce at most $d_{\pa(X)}(\rar\tau_j,\rar\tau_{j+1})\leq 1$.

\begin{defin}\label{def:not_firmly}
For $\bm\tau$ as above, we define
$$
\rar\bm\tau\coloneqq \rar\bm\tau^1*\bm\epsilon^1*\rar\bm\tau^2*\ldots*\bm\epsilon^{w-1}*\rar\bm\tau^w.
$$

Suppose now that each $\bm\tau^u$ in the subdivision above is $(\gamma_1^u,\ldots,\gamma_{r(u)}^u)$-arranged for a suitable family of curves. The construction of each $\utw\bm\tau^u$ as in Definition \ref{def:untwistedsequence} above would give, in particular, a diffeomorphism $\phi_{r(u)}^u:S\rightarrow S$ (called simply $\phi_r$ there); and we denote here $\psi_{u+1}\coloneqq \phi_{r(u)}^u\circ\ldots\circ \phi_{r(1)}^1$ for all $1\leq u \leq w-1$; while $\psi_1\coloneqq \mathrm{id}_S$. Let then
\begin{align*}
\utw\bm\tau \coloneqq\  & \utw\bm\tau^1*\\
 &  * \left(\psi_2\cdot \bm\epsilon^2\right) * \left(\psi_2\cdot\utw\bm\tau^2\right) *\\
 & \ldots  \\
 & * \left(\psi_{w-1}\cdot \bm\epsilon^{w-1}\right) * \left(\psi_{w-1}\cdot\utw\bm\tau^{w-1}\right) * \\
 & * \left(\psi_w \cdot \bm\epsilon^w\right) * \left(\psi_w\cdot\utw\bm\tau^w\right).
\end{align*}
\end{defin}

\begin{lemma}[Unbroken accessible intervals]\label{lem:untwistedsubsurfaces}
Let $\bm\tau=(\tau_j)_{j=0}^N$ be a $(\gamma_1,\ldots,\gamma_r)$-arranged train track splitting sequence, evolving firmly in a subsurface $S'$, not necessarily connected. Let $X\subseteq S'$ be a subsurface of $S$. Then the following properties hold.
\begin{enumerate}
\item If $\gamma_t$ cuts $\partial X$, then $DI_t\not\subseteq I_X$; more precisely, $DI_t\cap I_X$ contains at most $2g_t$ indices;
\item If $\gamma_t$ does not cut $\partial X$, then either $DI_t\subseteq I_X$ or $DI_t\cap I_X=\emptyset$.
\item Let $t_-\coloneqq t(\min I_X)$; let $t_+$ be the highest $t$ such that $\max I_X\geq \max DI_t$, and suppose that there is no $t$ with $I_X\subsetneq DI_t$. Then, for all $t_-\leq t\leq t_+$, $\phi_t^{-1}\circ \phi_{t_-}$ fixes $X$  and each component of $\partial X$, up to isotopies of $S$.
\item In the setting specified above, $[t_+]\dn I_X=I_{\phi_{t_-}(X)}^\utw$.
\item Let $\utw t_- \coloneqq \utw t(\min I_X^\utw)$; let $\utw t_+$ be the highest $t$ such that $\max I_X^\utw \geq \max DI_t^\utw$, and suppose that there is no $t$ with $I_X^\utw\subsetneq DI_t^\utw$. Then
$$I_{\phi_{\utw t_-}^{-1}(X)}= \left[[\utw t_+]\up \min I_X^\utw, \max DI_{\utw t_+}\right]
\text{ or }
I_{\phi_{\utw t_-}^{-1}(X)}= \left[[\utw t_+]\up \min I_X^\utw,[\utw t_+]\up \max I_X^\utw\right]
$$ depending on whether $\max I_X^\utw=\max DI_{\utw t_+}^\utw$ or not.
\item If $X$ is not an annulus, $\gamma_t$ does not intersect $X$ essentially, and $j, j+g_t\in DI_t\subseteq I_X$, then $\tau_j|X=\tau_{j+g_t}|X$ (up to isotopy).

\item If $\bigcup_{t=t_0}^{t_1} (DI_t\cup NI_t)\subseteq (DI_{t_0}\cup NI_{t_0})\cup I_X$, and $\gamma_t$ does not intersect $X$ for $t_0+1\leq t \leq t_1$, then all the respective maps $\phi_t\circ \phi_{t_0}^{-1}$ have their restriction to $\phi_{t_0}(X)$ isotopic to the inclusion $\phi_{t_0}(X)\hookrightarrow S$; and all the $\hat\phi_t: S^X\rightarrow S^{\phi_t(X)}=S^{\phi_{t_0}(X)}$, lift of the respective $\phi_t$, are isotopic to $\hat\phi_{t_0}$.

As a consequence, $\utw\tau_{\dn j}|\phi_{t_0}(X)=\hat\phi_{t_0}(\tau_j|X)$ for all $j\in \bigcup_{t=t_0}^{t_1} (DI_t\cup NI_t)$. If, in addition, there is an interval $I\subseteq I_X$ such that $I\subseteq \left(\bigcup_{t=t_0}^{t_1} (DI_t\cup NI_t)\right)\cup DK_{t_1+1}$ (setting $DK_{r+1}=\{N\}$ for simplicity), then $\utw\tau_{[t_1]\dn j}|\phi_{t_0}(X)=\hat\phi_{t_0}(\tau_j|X)$ for all $j\in I$.
\end{enumerate}
\end{lemma}
\begin{proof}
When $X$ is an annulus, claim 1 is a direct consequence of Lemma \ref{lem:smallinterference}, according to which one must have $\rot_{\bm\tau}(\gamma_t; DI_t\cap I_X)=0$. For $X$ not an annulus, suppose for a contradiction that $\#\left(DI_t\cap I_X\right)\geq 2g_t+1$, which means that there are two indices $k,l\in DI_t\cap I_X$ such that $\tau_l=D_{\gamma_t}^{2\epsilon_t}(\tau_k)$. Since $k,l\in I_X$, $\cc(\tau_k|X),\cc(\tau_l|X)$ both have diameter $\geq 3$ in $\cc(X)$, so they fill $X$, and both sets must include a curve which essentially intersects $\gamma_t$ in $S$, as $\gamma_t$ essentially intersects $X$: i.e. $\pi_{\nei_t}\cc(\tau_k|X),\pi_{\nei_t}\cc(\tau_l|X)\not=\emptyset$ where $\nei_t$ is a regular neighbourhood of $\gamma_t$.

From Theorem \ref{thm:mmsstructure} and the subsequent remark, we know that any efficient position for $\partial X$ in $\tau_k$ or $\tau_l$ is wide. Therefore, when lifting an efficiently positioned $\partial X$ to $S^{\nei_t}$, it will not traverse the same branch of $\tau_k^{\nei_t}$, or of $\tau_l^{\nei_t}$, twice in the same verse.

We claim that $\pi_{\nei_t}\cc(\tau_k|X)\subseteq V(\tau_k^{\nei_t})\cup D_{\nei_t}^{\epsilon_t}\cdot V(\tau_k^{\nei_t})$, for $D_{\nei_t}$ denoting the Dehn twist in $S^{\nei_t}$ about its core. It is certainly true, to start with, that $\pi_{\nei_t}\cc(\tau_k|X)\subseteq \cc(\tau_k^{\nei_t})$; if there is an $\alpha\in \pi_{\nei_t}\cc(\tau_k|X) \cap \left(D_{\nei_t}^{\epsilon_t i}\cdot V(\tau_k^{\nei_t})\right)$ for $i\not=0,1$, then $i>1$ by Lemma \ref{lem:onerollingdirection}, implying that $\alpha$, when embedded in $\tau_k^{\nei_t}$, traverses thrice a branch $b$ contained in $\tau_k^{\nei_t}.\gamma_t$ (see point \ref{itm:hl_vs_multiplicity} after Definition \ref{def:horizontallength}). Now, $\alpha$ shall be essentially disjoint from all components of $\pi_{\nei_t}(\partial X)$: but then, any component of $\pi_{\nei_t}(\partial X)$, assuming $\partial X$ in efficient position, shall traverse $b$ at least twice; and this contradicts the fact that it is wide.

However, $\cc(\tau_l^{\nei_t}) \subseteq D_{\nei_t}^{2\epsilon_t} \cdot \cc(\tau_k^{\nei_t})= \bigcup_{i=2}^\infty D_{\nei_t}^{\epsilon_t i}\cdot V(\tau_k^{\nei_t})$ and this, together with the fact just proved, implies in particular $\pi_{\nei_t}\cc(\tau_k|X)\cap \pi_{\nei_t}\cc(\tau_l|X)=\emptyset$. On the other hand $\pi_{\nei_t}\cc(\tau_l|X)\subseteq \pi_{\nei_t}\cc(\tau_k|X)$ because $\tau_k|X$ carries $\tau_l|X$, and they are both nonempty. This is a contradiction.

As for claim 2: let now $[k,l]\coloneqq DI_t$, so that $\tau_l=D_{\gamma_t}^{\epsilon_t m_t}(\tau_k)$. There is a lift $\hat D: S^X\rightarrow S^X$ of $D_{\gamma_t}^{\epsilon_t m_t}$ which fixes $X$ and each component $\partial X$ up to isotopy in $S^X$. So $\tau_l^X= \hat D(\tau_k^X)$ and $\cc(\tau_l^X)= \hat D\cdot \cc(\tau_k^X)$, yielding that also $\tau_l|X=\hat D(\tau_k|X)$. 

Now, when $X$ is an annulus, $j\in I_X$ means that $\tau_j|X$ is combed, i.e. its core $\gamma$ is a twist curve of $\tau_j$ (Lemma \ref{lem:twistininduced}); and it is clear from the above that $\gamma$ is a twist curve for $\tau_k$ if and only if it is one for $\tau_l=D_{\gamma_t}^{\epsilon_t m_t}(\tau_k)$ and for all tracks in between, proving the claim in this case. When $X$ is not an annulus, $\cc(\tau_l|X)=\hat D^{\epsilon_t m_t}\cdot\cc(\tau_k|X)$ and $\cc^*(\tau_l|X)=\hat D^{\epsilon_t m_t}\cdot\cc^*(\tau_k|X)$. So the sets $\cc(\tau_k|X), \cc(\tau_l|X)$ have the same diameter in $\cc(X)$, and the same is true of $\cc^*(\tau_k|X), \cc^*(\tau_l|X)$. Again, this means that $k\in I_X$ if and only if $l\in I_X$.

For claim 3 note, first of all, that it were $t_->t_+$ then $\max DI_{t_-}> \max I_X$ leading to $I_X\subsetneq DI_{t_-}$, contradicting the hypothesis.  Then $t_-\leq t_+$. The claim is proved if one proves that, for each $t_-\leq t<t+1\leq t_+$, the diffeomorphism $\phi_{t+1}^{-1}\circ\phi_t=D_{\gamma_{t+1}}^{-\epsilon_{t+1}(2\mathsf{K}_0+4-m_{t+1})}$ fixes $X$ and and each component of $\partial X$, up to isotopies of $S$. For this value of $t$, $\min DI_{t+1}\geq \min I_X$ and $\max DI_{t+1}\leq \max I_X$, so claims 1 and 2 apply, and imply that $D_{\gamma_{t+1}}$ fixes $X$ and each of the connected components of $\partial X$ (again, up to isotopies of $S$): the same is true of any power of it.

To prove claim 4: let $I_X=[k,l]$. Suppose first that $X$ is not an annulus. 

Note that if $t_+<r$ then, from claims 1 and 2 above, it can be derived that not only $l<\max DI_{t_+ +1}$, but also $l<\max DK_{t_+ +1}$. Therefore (even for $t_+=r$), if $j\in I_X$ then $j'\coloneqq [t_+]\dn j$ has the property that $\utw\tau_{j'}=\phi_t(\tau_j)$ for a suitable $t-\leq t\leq t_+$: this is a consequence of the basic remarks and definitions given above. So $\mathrm{diam}_{\cc\left(\phi_t(X)\right)}\left(\cc(\utw\tau_{j'}|\phi_t(X)\right)$, $\mathrm{diam}_{\cc\left(\phi_t(X)\right)}\left(\cc^*(\utw\tau_{j'}|\phi_t(X)\right)\geq 3$ just by application of $\phi_t$ to $\tau_j$. But it has been just proved above that, for the considered values of $t$, $\phi_t(X)=\phi_{t_-}(X)$. This yields $j'\in I_{\phi_{t_-}(X)}^\utw$ i.e. one inclusion is proved for $X$ not an annulus.

For the opposite inclusion, suppose $l<N$: this implies that, if $t_+<r$, then $l+1\leq \max DK_{t_+ +1}$. Thus (even for $t_+=r$) not only $\utw\tau_{[t_+]\dn l}=\phi_{t_+}(\tau_l)$, but also $\utw\tau_{[t_+]\dn (l+1)}=\phi_{t_+}(\tau_{l+1})$. Since $\mathrm{diam}_{\cc(X)}\left(\cc(\tau_{l+1}|X)\right)<3$, also $\mathrm{diam}_{\cc(\phi_{t_+}(X))}\left(\cc(\utw\tau_{[t_+]\dn(l+1)}|\phi_{t_+}(X))\right)<3$. As $\phi_{t_+}(X)=\phi_{t_-}(X)$, this yields $[t_+]\dn(l+1)\not\in I_{\phi_{t_-}(X)}^\utw$.

Similarly, suppose that $k>0$: then $k-1\geq \min DI_{t_-}$, thus $\utw\tau_{[t_+]\dn k}=\phi_{t_-}(\tau_k)$ and $\utw\tau_{[t_+]\dn (k-1)}=\phi_{t_-}(\tau_{k-1})$. Since $\mathrm{diam}_{\cc(X)}\left(\cc^*(\tau_{k-1}|X)\right)<3$, also\linebreak $\mathrm{diam}_{\cc(\phi_{t_-}(X))}\left(\cc^*(\utw\tau_{[t_+]\dn(k-1)}|\phi_{t_-}(X))\right)<3$, and this means that $[t_+]\dn(k-1)\not\in I_{\phi_{t_-}(X)}^\utw$. To sum up, considering that the cases $l=N$ or $k=0$ are trivial, $I_{\phi_{t_-}(X)}^\utw\subseteq [t_+]\dn I_X$.

If $X$ is an annulus a similar argument applies: but, rather than looking at the preservation of the diameter of the considered sets under the respective diffeomorphisms, the preserved property is whether the core curve of $X$, which turns into the core of $\phi_{t_-}(X)$, is twist.

In order to prove claim 5, note that similarly as above one has $\utw t_-\leq \utw t_+$. Let $k\coloneqq [\utw t_-]\up(\min I_X^\utw)$, and let $Y\coloneqq \phi_{\utw t_-}^{-1}(X)$. Necessarily $k\in I_Y$, since $[\utw t_-]\dn k=\min I_X^\utw$ and $\utw\tau_{\min I_X^\utw}=\phi_{\utw t_-}(\tau_k)$. Suppose there is a $t$ such that $I_Y\subsetneq DI_t$. There are two cases to discern.

If $k=\max NI_{\utw t_-}=\min DI_{\utw t_-+1}$, then $t=\utw t_-+1$ and $\max I_Y< \max DI_t$; also $\max I_Y< \max DK_t$ by claim 1. But then $\utw\tau_{\max DI_t}=\phi_{\utw t_-}(\tau_{\max DK_t})$ implying, with arguments similar to the ones already seen above, that $\max DI_t^\utw$ and so that $I_X\subseteq \left[[\utw t_-]\dn k, \max DI_t^\utw-1\right]$. But necessarily $\dn k=\min DI_t^\utw$ resulting in $I_X\subsetneq DI_t^\utw$, a contradiction.

If $k$ is any other index, then $t=\utw t_-$, and $\min DI_t\not \in I_Y$. This implies, by definition of $k$ as $[\utw t_+]\up(\min I_X^\utw)$, which by $t_-\leq t_+$ is the same as $\up(\min I_X^\utw)$, that $k\in DL_t$, $k\not=\min DL_t$. Either $\max DI_t=N$, or $\max DI_t+1\not\in I_Y$ and $\utw\tau_{\max DI_t^\utw+1}=\phi_{\utw t_-}(\tau_{\max DI_t+1})$. Both scenarios imply $\max I_X\leq \max DI_t^\utw$, but on the other hand also $\min I_X=[\utw t_+]\dn k > \min DI_t^\utw$ by assumption. Hence the same contradiction as above: $I_X\subsetneq DI_t^\utw$.

Therefore there is no $t$ with $I_Y\subsetneq DI_t$ and claim 4 applies: $I_X=[\utw t_+]\dn I_Y$. Now if, for any fixed value of $t$, $I_Y\supseteq DK_t$ or $DL_t$, then $I_Y\supseteq DI_t$, by claims 1 and 2 above. This completes the proof of the claim.

In claim 6, note that $\tau_{j+g_t}=D_{\gamma_t}^{\epsilon_t}(\tau_j)$. Since $\gamma_t$ does not intersect $X$ essentially, $D_{\gamma_t}|_X:X\rightarrow S$ is isotopic to the inclusion map $X\hookrightarrow S$. The map has a lift $\hat D: S^X\rightarrow S^X$ whose restriction to $X=\core(S^X)$ is again isotopic to the inclusion map. This means that $\hat D$ is itself isotopic to $\mathrm{id}_{S^X}$. Hence $\tau_j^X,\tau_{j+g_t}^X=\hat D(\tau_j^X)$ are pretracks isotopic in $S$, and $\tau_j|X$,$\tau_{j+g_t}|X$ are, too.

For the first statement in claim 7, note that $\phi_t\circ \phi_{t_0}^{-1}$ is a number of Dehn twist about curves essentially disjoint from $X$, so it follows as an application of what has been said previously. As for the second statement, we know that for all $j\in \bigcup_{t=t_0}^{t_1} (DI_t\cup NI_t)$ we have $\utw\tau_{\dn j}=\phi_{t(j)}(\tau_j)$ and $\dn j=[t_1]\dn j$ while, for all $j\in I\setminus \bigcup_{t=t_0}^{t_1} (DI_t\cup NI_t)\subseteq DK_{t_1+1}$, we have $\utw\tau_{[t_1]\dn j}=\phi_{t_1}(\tau_j)$. Lifting, we find that $\utw\tau_{\dn j}|\phi_{t(j)}(X)=\hat\phi_{t(j)}(\tau_j|X)$ and $\utw\tau_{[t_1]\dn j}|\phi_{t_1}(X)=\hat\phi_{t_1}(\tau_j|X)$, respectively, in the two specified cases; but $\phi_{t(j)}(X)$ and $\phi_{t_1}(X)$ are isotopic to $\phi_{t_0}(X)$ in both scenarios.
\end{proof}

\begin{rmk}\label{rmk:pantsboundunderdt_cutting}
There is a constant $C'_2=C'_2(S)$ such that the following are true. In the setting of the above lemma, for $X$ not an annulus, suppose that $\gamma_t$ cuts $\partial X$ essentially, and let $k,l\in DI_t\cap I_X$. Then $d_{\pa(X)}(\tau_k|X,\tau_l|X)\leq C'_2$, and\linebreak $d_Y\left(V(\tau_k|X),V(\tau_l|X)\right)\leq C'_2$ for all $Y\subseteq X$ non-annular subsurfaces. Incidentally, it has been already noted in the remark after Lemma \ref{lem:decreasingfilling} that $V(\tau_k|X),V(\tau_l|X)$ are both vertices of $\pa(X)$.

\ul{Note:} the constants $C_2,C'_2$ will be merged into a single one $C_2$ after the present remark.

A straightforward consequence of claim 1 in the above lemma is that $\rot_{\bm\tau}(\gamma_t; k,l)\leq 2$ so, by Lemma \ref{lem:twistsplitnumber}, at most $5N_3^2$ splits --- which are all twist ones about $\gamma_t$ --- occur in $\bm\tau(k,l)$. A combinatorial finiteness argument, entirely similar to the one in the proof of Lemma \ref{lem:pantsboundunderdt}, gives an upper bound $K$, depending on $S$ only, for $i(\alpha,\beta)$ where $\alpha\in V(\tau_k)$ and $\beta\in V(\tau_l)$.

For each $Y\subseteq X$ non-annular subsurface, $\pi_Y V(\tau_k),\pi_Y V(\tau_l)$ are both nonempty; and the intersection number between any pair of elements, one from each set, is not greater than $4K+4$ (Remark \ref{rmk:subsurf_inters_bound}). Hence there is a bound on $d_Y\left(V(\tau_k), V(\tau_l)\right)$ (Lemma \ref{lem:cc_distance}), and one on $d_{\pa(X)}\left(V(\tau_k), V(\tau_l)\right)$ (Theorem \ref{thm:mmprojectiondist}, Lemma \ref{lem:pantsquasiisom}).

Lemma \ref{lem:induction_vertices_commute} commutes these bounds into ones for the required distances.
\end{rmk}

\begin{coroll}\label{cor:subsurface_bijection}
Let $\bm\tau=(\tau_j)_{j=0}^N$ be a $(\gamma_1,\ldots,\gamma_r)$-arranged train track splitting sequence, evolving firmly in a subsurface $S'$, not necessarily connected, of $S$. Let $\Sigma_1(\bm\tau)$ be the family of all connected components of $S'$, and let $\Sigma_2(\bm\tau)$ be the family of all (isotopy classes of) subsurfaces $X\subseteq S'$ of $S$ such that $I_X$ is not empty and not strictly contained in a single $DI_t$, $1\leq t \leq r$. Define $\Sigma_1(\utw\bm\tau)=\Sigma_1(\bm\tau)$, and $\Sigma_2(\utw\bm\tau)$ similarly as above (adding superscripts $^\utw$).

Also let $\Sigma(\bm\tau)\coloneqq \Sigma_1(\bm\tau)\cup \Sigma_2(\bm\tau)$ and $\Sigma(\utw\bm\tau)\coloneqq \Sigma_1(\utw\bm\tau)\cup \Sigma_2(\utw\bm\tau)$.

For each $X\in \Sigma_2(\bm\tau)$, let $\phi_X\coloneqq \phi_{t_-}$ for $t_-=t_-(X)$ defined as in claim 3 of Lemma \ref{lem:untwistedsubsurfaces} above.

Then the map $\Sigma(\bm\tau)\rightarrow \Sigma(\utw\bm\tau)$, defined by $X\mapsto\utw X\coloneqq X$ for $X\in \Sigma_1(S)$, and by $X\mapsto \utw X\coloneqq \phi_X(X)$ (up to isotopies in $S$) for $X\in\Sigma_2(S)$, is a bijection.
\end{coroll}
\begin{proof}
The two definitions are easily seen to agree for $X\in \Sigma_1(\bm\tau)\cap \Sigma_2(\bm\tau)$. Also, the fact that the map is bijective easily follows from its restriction $\Sigma_2(\bm\tau)\rightarrow\Sigma_2(\utw\bm\tau)$ being so.

We first prove injectivity. Given $X_1,X_2\in \Sigma_2(\bm\tau)$, they may have:
\begin{itemize}
\item $t_-(X_1)=t_-(X_2)$. In this case $\phi_{X_1}=\phi_{X_2}$; so $\utw X_1,\utw X_2$ are isotopic surfaces if and only if $X_1,X_2$ are.
\item $t_-(X_1)<t_-(X_2)$ (without loss of generality). In this case $\dn(\min I_{X_1})<\dn(\min I_{X_2})$ so $I_{X_1}^\utw=[t_+(X_1)]\dn I_{X_1}\not=[t_+(X_2)]\dn I_{X_2}=I_{X_2}^\utw$. This implies, in particular, $X_1,X_2$ not isotopic in $S$.
\end{itemize}

Surjectivity is a straightforward consequence of claim 5 in the lemma above.
\end{proof}

Note that, in particular, all regular neighbourhoods of curves $\nei(\gamma_t)\in \Sigma(\bm\tau)$, so there is a bijection between the family $\nei(\gamma_1),\ldots, \nei(\gamma_r)$ and the corresponding $\utw\left(\nei(\gamma_1)\right),\ldots, \utw\left(\nei(\gamma_r)\right)$. This means, in particular, that their core curves $\utw\gamma_1\coloneqq \phi_{\nei(\gamma_1)}(\gamma_1),\ldots, \utw\gamma_r\coloneqq \phi_{\nei(\gamma_r)}(\gamma_r)$ are all distinct, meaning that $\utw\bm\tau$ is $(\utw\gamma_1,\ldots,\utw\gamma_r)$-arranged and that $\utw(\utw\bm\tau)=\utw\bm\tau$.

\begin{prop}\label{prp:locallyfinite}
Let $S$ be a surface. There is a constant $C_5(S)$ such that the following is true.

Let $\bm\tau=(\tau_j)_{j=0}^N$ be a $(\gamma_1,\ldots,\gamma_r)$-arranged splitting sequence, evolving firmly in a subsurface $S'$, not necessarily connected, of $S$. Let $X\in \Sigma(\bm\tau)$, $X$ not an annulus, and let $[k,l]$ be an interval of indices for $\bm\tau$, with the condition that $[k,l]\subseteq I_X$ if $X$ is not a connected component of $S'$. Then
$$
d_{\pa(\utw X)}(\dn k,\dn l)^\utw \leq C_5 \left(d_{\pa(X)}(k,l)\right)^2
\text{ and }
d_{\pa(X)}(k,l) \leq C_5 \left(d_{\pa(\utw X)}(\dn k,\dn l)^\utw\right)^2.
$$

There are two increasing functions $\Psi_S, \Psi'_S:[0,+\infty)\rightarrow [0,+\infty)$ such that, if $\gamma_1,\ldots,\gamma_r$ are the effective twist curves of $\bm\tau$ and $\bm\tau$ is effectively arranged, then
$$
d_{\ma(\utw X)}(\dn k,\dn l)^\utw \leq \Psi_S\left(d_{\pa(\utw X)}(\dn k,\dn l)^\utw\right) \leq \Psi'_S\left( d_{\pa(X)}(k,l) \right).
$$
\end{prop}
\begin{proof}

\step{1} preparation.

The curves $\gamma_1,\ldots,\gamma_r$ come with the usual condition that $(\min DI_t)_t$ is an increasing sequence. Let $1\leq t_1<\ldots<t_q\leq r$ be the indices such that $DI_{t_s}\cap [k,l] \not=\emptyset$ and $\gamma_t$ is essentially \emph{not} disjoint from $X$. So each of these $\gamma_{t_s}$ may be essentially contained in $X$ or intersect $\partial X$ essentially; but, according to claims 1 and 2 in Lemma \ref{lem:untwistedsubsurfaces}, the latter case may occur only for $s=1,q$ because all the other $DI_{t_s}\subseteq [k,l]$.

We establish a parallel notation for intervals of indices of $\bm\tau$, one that resembles the one used so far but focuses only on $X$ and on the family $\gamma_{t_1},\ldots,\gamma_{t_q}$. First of all rename these curves as $\delta_1,\ldots,\delta_q$. Then, for $s=1,\ldots,q$, denote $XDI_s\coloneqq DI_{t_s}\cap [k,l]$, and $XDL_s \coloneqq DL_{t_s}\cap [k,l]$. For $s=1,\ldots, q-1$, moreover, define $XNI_s\coloneqq [\max XDI_s,\min XDI_{s+1}]$. Define also $XNI_0\coloneqq [k, \min XDI_1]$, $XNI_q\coloneqq [\max XDI_q,l]$. Each $XNI_s$ is a union of intervals $NI_t$, and of intervals $DI_t$ for some $\gamma_t$ not intersecting $X$ essentially. In order to improve compatibility with the formulas, let also $XDI_0\coloneqq XDL_0\coloneqq \{k\}$.

We define also intervals related to the sequence $\utw\bm\tau$: $XDI_s^\utw\coloneqq \dn XDI_s$ for all $s=1,\ldots, q$, $XNI_s^\utw\coloneqq \dn XNI_s$ for all $s=0,\ldots, q$.

The collection $V(\tau_l|X)$ is a vertex of $\pa(X)$: if $X$ is strictly contained in a connected component of $S'$, this derives from the hypothesis $l\in I_X$. Otherwise we know from Lemma \ref{lem:decreasingfilling} that $V(\tau_l|X)$ consists of all elements of $V(\tau_l)$ which are essentially contained in $X$ and not isotopic to connected components of $\partial X$: these elements fill $X$ and abide by the mutual intersection bound established in accordance with Remark \ref{rmk:pickparameters}.

Now, $\bm\tau$ is, in particular, $(\delta_1,\ldots,\delta_q)$-arranged: so, by Proposition \ref{prp:tcbound}, simplifying its statement, there is a bound $q\leq c\cdot d_{\pa(X)}(k,l)$ for a $c=c(S)$. On the other hand: $\utw\bm\tau$ is $(\utw\gamma_{t_1},\ldots,\utw\gamma_{t_q})$-arranged; these curves are all contained in $\utw X$ except for possibly $\utw\gamma_{t_1}, \utw\gamma_{t_q}$; and either $\dn l \in I_{\utw X}^\utw$ or $\utw X=X$ is a connected component of $S'$. This implies that $V(\utw\tau_{\dn l}|\utw X)$ is a vertex of $\pa(\utw X)$ so, again according to Proposition \ref{prp:tcbound}, it is also true that $q\leq c \cdot d_{\pa(X)}(\dn k,\dn l)^\utw$.

Denote, for simplicity, $t_0\coloneqq 0$ and $t_{q+1}\coloneqq r+1$. The following argument is straightforward if we are working on $X\in \Sigma_1(\bm\tau)$, and is motivated by claim 7 in Lemma \ref{lem:untwistedsubsurfaces} if $X\in \Sigma_2(\bm\tau)$. For $0\leq s \leq q$, if $t_s\leq t < t_{s+1}$ then $\phi_t|_X=\phi_{t_s}|_X$; also, if $\hat \phi_t: S^X\rightarrow S^{\phi_t(X)}$ is the lift of $\phi_t$, then the sequence $\left(\utw\tau_j|\utw X\right)_{j\in XDI_s^\utw \cup XNI_s^\utw}$ is obtained from $(\tau_j|X)_{j\in XDL_s\cup XNI_s}$ applying $\hat\phi_{t_s}$ to each entry, and removing some repetitions of tracks in the Dehn intervals $DI_t\subset XNI_s$, where $\gamma_t$ does not intersect $X$ essentially. So $d_{\pa(\utw X)}(XDI_s^\utw\cup XNI_s^\utw)^\utw=d_{\pa(X)}(XDL_s\cup XNI_s)$. 

\step{2} reciprocal bounds for distances in the pants graph.

Fix an index $s$, and denote simply $a\coloneqq \min XDL_s$; $b\coloneqq \max XDI_s$; $\phi\coloneqq \phi_{t(\min XDL_s)}$. Take $\mathsf{R}_0=\max_{Y\subseteq S \text{ subsurface}} \mathsf{R}_0(Y,Q)$, for $Q$ the quasi-isometry constant introduced in Theorem \ref{thm:mms_cc_geodicity}, as defined in Lemma \ref{lem:reversetriangle}. By those theorem and lemma, we have that for any non-annular subsurface $Y\subseteq X$, $d_Y\left(\tau_a|X,\tau_b|X\right) \leq  d_Y(\tau_k|X,\tau_l|X)+ 2\mathsf{R}_0$.

Let $M'=M+2\mathsf{R}_0$ --- for the definition of $M$, see the beginning of \S \ref{sub:twistcurvebound} --- we have
$$
\sum_{\substack{Y\subset X\text{ essential} \\ \text{and non-annular}}} [d_Y(\tau_a|X,\tau_b|X)]_{M'} =
\sum_{\substack{Y\subset X\text{ essential} \\ \text{and non-annular}}} \left([d_Y(\tau_k|X,\tau_l|X)]_M+ \mathsf{r}(Y)\right)
$$

where $\mathsf{r}(Y)=2\mathsf{R}_0$ if it comes alongside a nonzero summand, and $0$ otherwise. So $[d_Y(\tau_k,\tau_l)]_M+ \mathsf{r}(Y)\leq \frac{M'}{M} [d_Y(\tau_k|X,\tau_l|X)]_M$. Therefore, by Theorem \ref{thm:mmprojectiondist} and Lemma \ref{lem:pantsquasiisom} there are constants $c_0, c_1$ depending on $S$ (and $M$) such that $d_{\pa(X)}(\tau_a,\tau_b)\leq c_0 d_{\pa(X)}(\tau_k,\tau_l) + c_1$.

Similarly, we get $d_{\pa(\utw X)}(\utw\tau_{\dn a},\utw\tau_{\dn b})\leq c_0 d_{\pa(\utw X)}(\utw\tau_{\dn k},\utw\tau_{\dn l}) + c_1$.

So, on the one hand,
\begin{eqnarray*}
 & d_{\pa(\utw X)}(\dn k,\dn l)^\utw \leq \sum_{s=0}^q d_{\pa(\utw X)}(XDI_s^\utw\cup XNI_s^\utw)^\utw = & \\
 &  \sum_{s=0}^q d_{\pa(X)}(XDL_s\cup XNI_s) \leq (q+1) \left(c_0 d_{\pa(X)}(k,l)+c_1\right) & 
\end{eqnarray*}
and, as already pointed out, $q\leq c\cdot d_{\pa(X)}(k,l)$. This, together the considerations at the beginning of \S \ref{sub:twistcurvebound}, proves the existence of a $C_5(S)$ such that $d_{\pa(\utw X)}(\dn k,\dn l)^\utw\leq C_5 \left(d_{\pa(X)}(k,l)\right)^2$.

On the other hand,
\begin{eqnarray*}
 & d_{\pa(X)}(k, l) \leq \sum_{s=0}^q \left(d_{\pa(X)}(\min XDI_s,\min XDL_s) + d_{\pa(X)}(XDL_s\cup XNI_s) \right). & 
\end{eqnarray*}

For $s=1,\ldots, q$ (neglect the dummy index $s=0$), $\bm\tau(\min XDI_s,\min XDL_s)$ has twist nature about $\delta_s$ which is either contained in $X$ or cutting $\partial X$ essentially. So the sequence falls in the kind treated in either the last bullet of Lemma \ref{lem:pantsboundunderdt}, or Remark \ref{rmk:pantsboundunderdt_cutting}, implying that $d_{\pa(X)}(\min XDI_s,\min XDL_s)\leq C_2(S)$. Continuing the chain of inequalities:
\begin{eqnarray*}
\ldots & \leq \sum_{s=0}^q \left(C_2 + c_0 d_{\pa(\utw X)}(XDI_s^\utw\cup XNI_s^\utw)^\utw +c_1\right) \leq & \\
 & \leq (q+1)\left(C_2+c_0 d_{\pa(\utw X)}(\dn k,\dn l)^\utw + c_1\right) & 
\end{eqnarray*}

Again as already pointed out, $q\leq c \cdot d_{\pa(\utw X)}(\dn k,\dn l)^\utw$ therefore $C_5(S)$ can be taken so that $d_{\pa(X)}(k,l)\leq C_5 \left(d_{\pa(\utw X)}(\dn k,\dn l)^\utw\right)^2$.

\step{3} bounds for distances in annular subsurfaces, when $\bm\tau$ is effectively arranged.

Let $\alpha\in \cc(X)$ be a curve and, for ease of notation, let simply $\nei=\nei(\alpha)$ be a regular neighbourhood of $\alpha$ in $X$. We wish to prove the existence of constants depending on $S$ such that, for all $0\leq s\leq q$, and all $j,j'\in XDL_s\cup XNI_s$, $d_\nei(\tau_j|X,\tau_{j'}|X)$ is bounded by this constant. Note that each $V\left((\tau_j|X)^\nei\right)$ is a subset of the respective $V(\tau_j^\nei)$.

\begin{itemize}
\item If $\alpha$ is none of the $\delta_s$ then, as it follows immediately from the definition of effective twist curve, $d_\nei(\tau_0|X,\tau_N|X)< 4\mathsf{K}_0+19$; and consequently, $d_\nei(\tau_j|X,\tau_{j'}|X)< 4\mathsf{K}_0+2\mathsf{R}_0+19$.
\item If $\alpha=\delta_u$ for $u\not=s$, then $[j,j']\subseteq G_{t_u-}\cup I_{t_u-}$ or $[j,j']\subseteq I_{t_u+}\cup G_{t_u+}$ --- see Definition \ref{def:arranged}. The first case implies that $d_\nei(\tau_j|X,\tau_{j'}|X)\leq \mathsf{\mathsf{K}_0}+2\mathsf{R}_0+9$, and the second one that $d_\nei(\tau_j|X,\tau_{j'}|X)\leq \mathsf{K}_0+6$.
\item If $\alpha=\delta_s$ then $[j,j']\subseteq DL_s\cup I_{s+}\cup G_{s+}$. If $[j,j']\subseteq DL_s$ then $\rot_{\bm\tau}(\delta_s;j,j')\leq 2\mathsf{K}_0+4$ hence $d_\nei(\tau_j|X,\tau_j'|X)\leq 2\mathsf{K}_0+8$ by point \ref{itm:rot_vs_dist} in Remark \ref{rmk:rotbasics}. If $[j,j']\subseteq I_{s+}\cup G_{s+}$ then same as above applies, and if $j,j'$ range in the union of the three intervals then $d_\nei(\tau_j|X,\tau_{j'}|X)$ is bounded by the sum of the two previous bounds.
\end{itemize}

This proves the existence of the desired bound, which we call $c'$. As for the splitting sequence $\utw\bm\tau$, let $\alpha\in\cc(\utw X)$.  

Fix a value $0\leq s \leq q$, and recall the conclusions of Step 1. If $t_s\leq t < t_{s+1}$ then $\phi_t|X=\phi_{t_s}|X$ and in particular there is a curve $\beta_s\in \cc(X)$ such that $\phi_t(\beta_s)=\phi_{t_s}(\beta_s)=\alpha$. If $\hat{\hat\phi}_t: S^{\nei(\beta_s)}\rightarrow S^{\nei(\alpha)}$ is the lift of $\phi_t$, then the sequence $\left((\utw\tau_j|X)^{\nei(\alpha)}\right)_{j\in XDI_s^\utw \cup XNI_s^\utw}$ is obtained from $\left((\utw\tau_j|X)^{\nei(\beta_s)}\right)_{j\in XDL_s\cup XNI_s}$ applying $\hat{\hat\phi}_{t_s}(\min XDL_s)$ to each entry, and removing some repetitions of entries in the intervals $DI_t\subset XNI_s$, where $\gamma_t$ does not intersect $X$ essentially. So if $[a,b]=XDL_s\cup XNI_s$ then $[\dn a,\dn b]=XDI_s^\utw\cup XNI_s^\utw$ and  $d_{\nei(\alpha)}(\utw\tau_{\dn a}|\utw X,\utw\tau_{\dn b}|\utw X)=d_{\nei(\beta_s)}(\tau_a|X,\tau_b|X)\leq c'$.

\step{4} bound for $d_{\ma(\utw X)}(k,l)^\utw$, when $\bm\tau$ is effectively arranged.

The distance bounds seen above prove immediately that\linebreak $d_{\nei(\alpha)}(\utw\tau_{\dn k}|\utw X,\tau_{\dn l}|\utw X)\leq c'(q+1)$ for all $\alpha\in \cc(\utw X)$.

Let then $M''\coloneqq \max \{M,c'(q+1)+1\}$. We have
$$
\sum_{Y\subset \utw X\text{ essential}} [d_Y(\utw\tau_{\dn k}|X,\tau_{\dn l}|X)]_{M''} =
\sum_{\substack{Y\subset \utw X\text{ essential} \\ \text{and non-annular}}} [d_Y(\utw\tau_{\dn k}|X,\tau_{\dn l}|X)]_{M''}
$$
and this implies, via the usual Theorem \ref{thm:mmprojectiondist} and Lemma \ref{lem:pantsquasiisom}, that\linebreak $d_{\ma(\utw X)}(\dn k,\dn l)^\utw\leq f_0 d_{\pa(\utw X)}(\dn k,\dn l)^\utw + f_1$. Here, $f_0$ and $f_1$ derive from $e_0,e_1$ cited in Theorem \ref{thm:mmprojectiondist} and may be supposed, by taking looser bounds, to depend on $S$ and $M''$ only, and be increasing functions in the variable $M''$.

Using a simplified notation $d_\utw\coloneqq d_{\pa(\utw X)}(\dn k,\dn l)^\utw$ and $d\coloneqq d_{\pa(X)}(k,l)$:
\begin{eqnarray*}
 & d_{\ma(\utw X)}(\dn k,\dn l)^\utw\leq f_0\left(S,c'' d_\utw\right) d_\utw + f_1\left(S,c'' d_\utw\right) & \\
 &\leq C_5 f_0\left(S,c'' C_5 d^2\right) d^2 + f_1\left(S, c'' C_5 d^2\right) & 
\end{eqnarray*}
where $c''$ is a further constant, deriving from the previous ones $c,c'$; and we are using the fact that $q\leq c d_\utw$, $d_\utw\leq d^2$. This proves our claim.
\end{proof}

\subsection{Proof of the main statement}\label{sub:conclusion}

In this subsection we prove Theorem \ref{thm:core}.

\begin{defin}
Let $\bm\tau=(\tau_j)_{j=0}^N$ be a splitting sequence of birecurrent train tracks on $S$, and let $X\subseteq S$ be a non-annular subsurface. Let $0\leq j <N$ be an index such that $\tau_j$ splits to $\tau_{j+1}$. We say that this split move is \nw{visible} in $X$ if, in order to turn $\tau_j|X$ into $\tau_{j+1}|X$, a split is required (i.e. it does not suffice to apply comb equivalences and/or take subtracks).

We denote with $|\bm\tau|_X$ the number of splits in $\bm\tau$ which are visible in $X$.
\end{defin}

%The last part of the proof of Theorem 5.3 in \cite{mms} (which is quoted in this work as Theorem \ref{thm:mmsstructure}) deals with the setting described in the first sentence of the above Definition. In that setting, we have that $\tau_{j+1}|_X$ is comb equivalent to $\tau_j|X$, a subtrack of it, or obtained from it with a single split.

Paraphrasing the statement of Theorem \ref{thm:core}, we want to show that the distance induced in the pants graph by a splitting sequence is, up to constants, the number of splits in the untwisted sequence: this is easy in one direction.

\begin{prop}\label{prp:easyttbound}
There is a constant $C_6=C_6(S)$ such that the following is true. Let $\bm\tau=(\tau_j)_{j=0}^N$ be a generic splitting sequence of recurrent, cornered train tracks on $S$. Let $X$ be a subsurface of $S$ with $X\supseteq S'$ the subsurface, not necessarily connected, filled by $V(\tau_0)$, and suppose that $V(\tau_N)$ is a vertex of $\pa(S)$. Then
$$d_{\pa(X)}\left(\tau_0,\tau_N\right)\leq C_6 |\utw(\rar\bm\tau)|.$$
\end{prop}
\begin{proof}
Suppose first that $\bm\tau$ evolves firmly in a (possibly disconnected) subsurface $S'$. Define a strictly increasing sequence of indices $(j_i)_{i=0}^{K}$ for the sequence $\rar\bm\tau=(\rar\tau_j)_{j=0}^{N'}$ such that one of the following holds --- using the standard notation of Definition \ref{def:arranged} and of Proposition \ref{prp:rearrang2}.
\begin{itemize}
\item $[j_i,j_{i+1}]=DI^\rar_s$, or consists of $DI^\rar_s$ plus some slide moves, for some $s$; in this case $d_{\pa(S)}(\rar\tau_{i_j},\rar\tau_{i_{j+1}})\leq C_2$ by Lemma \ref{lem:pantsboundunderdt}.
\item $[j_i,j_{i+1}]\cap DI^\rar_s$ is not more than a single index for each $s$, and there is exactly one split move in $(\rar\bm\tau)(j_i,j_{i+1})$. Then $d_{\pa(S)}(\rar\tau_{i_j},\rar\tau_{i_{j+1}})\leq 1$ by the choice of parameters in Remark \ref{rmk:pickparameters} and, in particular, $i\left(V(\rar\tau_{i_j}),V(\rar\tau_{i_{j+1}})\right)\leq \ell$. This bounds uniformly $i\left(\pi_Y V(\rar\tau_{i_j}),\pi_Y V(\rar\tau_{i_{j+1}})\right)$ and then $d_Y\left(\rar\tau_{i_j}|X,\rar\tau_{i_{j+1}}|X\right)$, for each non-annular subsurface $Y\subseteq X$ (with an application of Remark \ref{rmk:subsurf_inters_bound}, Lemma \ref{lem:cc_distance}, Lemma \ref{lem:induction_vertices_commute} successively). Theorem \ref{thm:mmprojectiondist} with Lemma \ref{lem:pantsquasiisom}, then, gives a uniform bound $C_6$ for $d_{\pa(X)}(\rar\tau_{i_j},\rar\tau_{i_{j+1}})$. Suppose $C_6\geq C_2$, in order to make notation simpler in the following chain of inequalities.
\end{itemize}

Now, $|\utw(\rar\bm\tau)|\geq K\geq d_{\pa(S)}(\rar\tau_0,\rar\tau_{N'})/C_6$. The second inequality follows directly from concatenating the ones above, while the first one is due to the fact that each $\left(\utw(\rar\bm\tau)\right)(\dn j_i, \dn j_{i+1})$ contains at least one split. This proves the claim in the specified special case.

More generally, view $\bm\tau$ as a concatenation $\bm\tau^1*\bm\epsilon^1*\bm\tau^2*\ldots*\bm\epsilon^{w-1}*\bm\tau^w$ as pointed out just before Definition \ref{def:not_firmly}. For each $1\leq u\leq w$, let $J^u$ be the interval of indices for $\bm\tau$ coming from $\bm\tau^u$. So
\begin{eqnarray*}
 & |\utw(\rar\bm\tau)|= \sum_{u=1}^w \left(|\psi_u\cdot\utw(\rar\bm\tau^u)|+1\right)-1 \geq & \\
 & \sum_{u=1}^w \left(d_{\pa(X)}\left(\psi_u(\rar\tau_{\min J^u})|X, \psi_u(\rar\tau_{\max J^u}|X)\right)/C_6+1\right) -1 = & \\
 & \sum_{u=1}^w \left(d_{\pa(X)}\left(\rar\tau_{\min J^u}|X, \rar\tau_{\max J^u}|X\right)/C_6+1\right) -1\geq d_{\pa(X)}(\rar\tau_0|X,\rar\tau_{N'}|X)/C_6 &
\end{eqnarray*} 
using the special case proved above combined with the triangle inequality.
\end{proof}

We state Claim 6.14 from \cite{mms} in a slightly more general setting:
\begin{lemma}\label{lem:mms614}
Let $\bm\tau=(\tau_j)_{j=0}^N$ be a splitting sequence of birecurrent train tracks on $S$, and let $X\subseteq S$ be a subsurface. Then there is a constant $N_4=N_4(X)$ such that, if $V(\tau_k|X)=V(\tau_l|X)$, then $|\bm\tau(k,l)|_X\leq N_4$.
\end{lemma}

The proof is the same as the original claim: in that setting, the splitting sequence had some more properties than in the above statement, but they are not actually employed.

\begin{prop}\label{prp:hardttbound}
There is a constant $C_7(S)$ such that the following is true. Let $\bm\tau=(\tau_j)_{j=0}^N$ be a generic splitting sequence of birecurrent, cornered train tracks, which evolves firmly in a subsurface $S'$ of $S$, not necessarily connected; then, for each non-annular connected component $T$ of $S'$,
$$d_{\pa(T)}(V(\tau_0),V(\tau_N))\geq_{C_7}|\utw(\rar\bm\tau)|_T.$$
\end{prop}

The proof of this statement will be very similar to the one of Theorem 6.1 in \cite{mms}, quoted in this work as Theorem \ref{thm:mms_main}. Large portions of our proof, actually, would be verbatim repetitions of pieces of the proof of the mentioned result, up to a replacement of a few names of objects.

The original splitting sequence $\bm\tau$ is never needed in this proof: so we may well identify $\rar\bm\tau=\bm\tau$ and, in particular, suppose that $\bm\tau$ is effectively arranged. Index $\utw\bm\tau=(\utw\tau_j)_{j=0}^{N'}$. In this proof, in order to avoid introducing \emph{ad hoc} notations and arguments, when $T$ is a connected component of $S'$ redefine $I_T\coloneqq[0,N]$ and $I_T^\utw\coloneqq [0,N']$. 

If $X\in \Sigma(\bm\tau)$ is not an annulus and $[p,q]=I\subset I_X$, let $\mathcal T_X(p,q)=\mathcal T_X(I)$ be the set of indices $\dn p\leq j\leq (\dn q)-1$ (in particular $j\in \dn I\subseteq I_{\utw X}^\utw$) such that $\utw\tau_{j+1}$ is obtained from $\utw\tau_j$ via a split which is visible in $\utw X$. Note that for $X\not= T$ Theorem \ref{thm:mmsstructure} holds, and yields that in this case only one split is required. 

The main step in our proof is a tailored version of Proposition 6.9 from \cite{mms}, to be proved by induction on the complexity of subsurfaces of $S'$. We will refer to it as the key claim:

\begin{claim}
Let $X\in \Sigma(\bm\tau)$. There is a constant $C_7(X,S)$ such that for any $J_X\subset I_X$ we have
$$|\mathcal T_X(J_X)|\leq_{C_7} d_{\pa(X)}(J_X).$$
\end{claim}

Proposition \ref{prp:hardttbound} will then follow taking $X=T$ and $J_T=[0,N]$.

Before we start the proof, we make some observations. First of all, there is a difference between our hypotheses and the ones of \cite{mms}: the latter include the request that the vertex cycles of each entry in $\bm\tau$ fill the entire surface, whereas here we restrict to a connected component of the subsurface $S'$ they fill. 

The proof of our result, just as the one given in \cite{mms}, works by induction on $\xi(X)$. In general, however, our argument will exclude annular subsurfaces while they are explicitly considered in \cite{mms}. In particular, annuli cannot serve as an induction basis: this role will be covered by those subsurfaces which are homeomorphic to $S_{0,4}$ or $S_{1,1}$; but it is not necessary to distinguish the induction basis from the rest. It is worth noting that the re-definition of $I_T$ given above invalidates Theorem \ref{thm:mmsstructure}. This causes no harm, since that theorem, during the present proof, will be used only on \emph{proper} subsurfaces of the $X$ fixed in the above claim. The statements of Lemma \ref{lem:untwistedsubsurfaces} are still true with the new definition instead, and their proofs are either easier or trivial.

To start the proof, fix two constants $\mathsf{T}_0(X),\mathsf{T}_1(X)$ with the following conditions:
\begin{eqnarray*}
\max\{6K_1+2K_2+2\mathsf{K}_0(X)+2,2\mathsf{R}_0, M_6(X), & & \\
 C_2+2F(32N_1(S^X)+4)+2\mathsf{K}_0(X)+1\} & \leq & \mathsf{T}_0(X) \\
\max\{\mathsf{T}_0(X)+2\mathsf{R}_0, & & \\
N_1K_2 + C_2 + 2 F\left(8 N_1(S^X)\right)+2\mathsf{K}_0\} & \leq & \mathsf{T}_1(X)
\end{eqnarray*}

Here, $K_1$ is the constant denoted $\mathsf{N}_1$ in \cite{mms}: an upper bound for $d_Y(\alpha,\beta)$, for $Y$ a subsurface of $S$ and $\alpha,\beta$ wide curves for one same almost track on $S$. Its existence is proved with a bound on $i(\pi_Y(\alpha),\pi_Y(\beta))$, similarly as in Lemma \ref{lem:induction_vertices_commute} and Remark \ref{rmk:subsurf_inters_bound}. $K_2$ is the constant denoted $\mathsf{N}_2$ in \cite{mms}: an upper bound for $d_Y(V(\tau|Z),V(\tau'|Z))$, where $Y,Z$ are subsurfaces of $S$; $\tau,\tau'$ are almost tracks with the latter obtained from the former as a subtrack, or with a split. $K_2$ exists again by a similar argument as in Lemma \ref{lem:induction_vertices_commute}. Again we take $\mathsf{R}_0=\max_{Y\subseteq S \text{ subsurface}} \mathsf{R}_0(Y,Q)$, for $Q$ the quasi-isometry constant introduced in Theorem \ref{thm:mms_cc_geodicity}, as defined in Lemma \ref{lem:reversetriangle}.

The constant $M_6(X)$ is introduced in Theorem 6.12 of \cite{masurminskyii}, also stated as Theorem \ref{thm:mmprojectiondist} here, in the light of \S 8 in that work, which allows to ignore annular subsurfaces. Finally, $C_2$ has been defined in Lemma \ref{lem:pantsboundunderdt} and re-defined in Remark \ref{rmk:pantsboundunderdt_cutting}; $N_1$ is defined in Lemma \ref{lem:vertexsetbounds}; $F$ is the function introduced in Lemma \ref{lem:induction_vertices_commute}.

\begin{defin}
A proper, non-annular subsurface $Y\subsetneq X$ is an \nw{inductive subsurface} if $d_Y(J_X)\geq \mathsf{T}_0(X)$. The related \nw{inductive subinterval} is then $J_Y\coloneqq I_Y\cap J_X$. For any non-inductive $Y$, just set $J_Y\coloneqq \emptyset$.

A subinterval $I\subseteq J_X$ is \nw{straight} if, for all non-annular, proper subsurfaces $Y\subsetneq X$, we have $\mathrm{diam}_Y(I)\leq\mathsf{T}_1(X)$. Here $\mathrm{diam}_Y(I)\coloneqq \mathrm{diam}_{\cc(Y)}\left(\bigcup_{j\in I}\pi_Y\left(V(\tau_j)\right)\right)$.
\end{defin}
Note that if $Y$ is an inductive subsurface then $Y\in \Sigma_2(\bm\tau)$. This is because, for any $Y\not\in \Sigma_2(\bm\tau)$ with $I_Y\not=\emptyset$, there is an index $s$ such that $I_Y\subsetneq DI_s$, necessarily for a $\gamma_s$ intersecting $\partial Y$ (claims 1, 2 in Lemma \ref{lem:untwistedsubsurfaces}) and then, for any $i,j\in I_Y$, one has $d_Y(i,j)\leq C_2+2F(32N_1(S^X)+4)$ by Remark \ref{rmk:pantsboundunderdt_cutting} combined with Lemma \ref{lem:induction_vertices_commute}. Hence one may subdivide $J_X=[\min J_X, \min I_Y]\cup (J_X\cap I_Y) \cup [\max I_Y, \max J_X]$ (one or more intervals may be empty, as we may have defined an upper bound lower than the lower bound), and get $d_Y(J_X)\leq 2\mathsf{K_0} + C_2+F(32 N_1(S^X)+4)< \mathsf{T}_0(X)$ with an application of Theorem \ref{thm:mmsstructure}.

\begin{lemma}\label{lem:mms612} \emph{(Plays the role of \cite{mms}, Lemma 6.12)}
If $I\subseteq J_X$ is disjoint from all inductive subintervals of $J_X$, then $I$ is straight for $X$.
\end{lemma}
The proof is the one of Lemma 6.12 in \cite{mms}, verbatim.

In our setting Claim 6.14 from \cite{mms} (i.e. Lemma \ref{lem:mms614} above) reads:
\begin{claim}
If for a pair of indices $k,l\in J_X$ we have $V(\utw\tau_{\dn k}|\utw X)=V(\utw\tau_{\dn l}|\utw X)$, then $\mathcal T_X(k,l)\leq N_4$.
\end{claim}

Similarly as noted at the beginning of \S \ref{sub:twistcurvebound},  for all $C\geq M_6(X)$ there are constants $e_0(S,C)$, $e_1(S,C)$ such that, for all $k,l\in J_X$,
\begin{equation}\label{eqn:mmsubsurfaceproj_mixed}
e_0^{-1} d_{\pa(X)}\left(\tau_k, \tau_l\right) - e_1 -C_1 \leq
\sum_{\substack{Y\subset X\text{ essential} \\ \text{and non-annular}}} [d_Y(\tau_k,\tau_l)]_C \leq
e_0 d_{\pa(X)}\left(\tau_k, \tau_l\right) + e_1 + C_1
\end{equation}
where $C_1=C_1(X)$.

\begin{lemma}\label{lem:mms613}\emph{(Plays the role of \cite{mms}, Lemma 6.13)}
Let $I\subseteq J_X$ be a straight subinterval. Then $|\mathcal T_X(I)|\leq_A d_X(I)$, for a constant $A=A(X)$.
\end{lemma}
\begin{proof}
Let $[p,q]=I$. Fix $C\coloneqq 1+\max \{M_6(X),\mathsf{T}_1(X)\}$ and let\linebreak $\mathsf{R}_1\coloneqq \max\{e_0(X,C),e_1(X,C)+C_1(X)\}+1$. Define a map $\rho:[0,M]\rightarrow I$, starting with $\rho(0)\coloneqq p$, and recursively setting $\rho(n+1)$ to be the smallest element in $[\rho(n),q]$ with $d_{\pa(X)}\left(\rho(n),\rho(n+1)\right)=\mathsf{R}_1$, until this is no longer defined: when this stage is reached, set $M\coloneqq n+1$ and $\rho(M)=q$.

Then for all $0\leq n \leq M-1$, and all $\rho(n)\leq j\leq \rho(n+1)$, we get that $d_{\pa(X)}(\tau_{\rho(n)},\tau_j)\leq \mathsf{R}_1$, so $d_{\ma(\utw X)}(\dn \rho(n), \dn j)^\utw \leq \Psi'_S(\mathsf R_1)$ according to Proposition \ref{prp:locallyfinite}.

We define $\mathsf V$ to be the maximum cardinality for a ball of radius $\Psi'_S(\mathsf R_1)$ in $\ma(X)$: then the sequence $\utw\bm\tau(\dn\rho(n),\dn\rho(n+1))$ has all the respective entries $V(\tau_j|X)$ lying within one of these balls. So, via the above claim, $|\mathcal T_X(\rho(n),\rho(n+1)|\leq N_4\mathsf V$ and therefore $|\mathcal T_X(I)|\leq N_4 \mathsf V \cdot M$.

Claim 6.15 in \cite{mms} holds for us (just replace $\mathcal M(X)$ with $\pa(X)$ and employ formula \ref{eqn:mmsubsurfaceproj_mixed} above; in particular, perform the summation over non-annular subsurfaces $Y$ only):
\begin{claim}
For $0\leq n\leq M-2$, $d_X\left(\rho(n),\rho(n+1)\right)\geq \mathsf{R}_0+1$.
\end{claim}

The remainder of the proof of the present lemma is the same the final part in the proof of Lemma 6.13 in \cite{mms}, where $M \leq d_X(I)+1$ is obtained. Hence the bound on the number of splits.
\end{proof}

When $X\cong S_{1,1}$ or $S_{0,4}$, the proof of the key claim ends here because, as $X$ has no subsurfaces, all intervals can be supposed to be straight. Theorem \ref{thm:mmprojectiondist} and Lemma \ref{lem:pantsquasiisom}, applied for these subsurfaces, assert that $\pa(X)$ is quasi-isometric to $\cc(X)$. So the key claim, for these subsurfaces, coincides with the statement of the above lemma. We now continue for the induction step.

All that is said in \cite{mms} from Definition 6.17 to Lemma 6.20 applies completely in our setting. Our proof continues with that chunk of proof, replacing all occurrences of $\mathcal S_X$ there with $\mathcal T_X$; the constants $\mathsf{N}_1,\mathsf{N}_2$ there with $K_1,K_2$ respectively; references to Lemmas 6.12 and 6.13 there with our Lemmas \ref{lem:mms612}, \ref{lem:mms613} respectively.

\begin{defin}
\nw{Assign} an index $r\in \mathcal T_X(J_X)$ to a subsurface $Y\in \mathsf{Ind}'$ if $r\in J_Y$, if the split from $\utw\tau_r$ to $\utw \tau_{r+1}$ is visible in $\utw Y$, and there is no other $Z\in \mathsf{Ind}$, $Z\subsetneq Y$, with these properties.

For $I\subseteq J_X$, denote the set of indices in $\mathcal T_X(I)$ which are assigned to $Y$ by $\mathsf{AI}_Y(I)\subseteq \mathcal T_Y(J_Y)$.
\end{defin}

\begin{lemma}\label{lem:mms616}
Let $Y\in\mathsf{Ind}$. There is a constant $B=B(X)$ such that, if $I\subseteq J_Y$ has all indices in $\mathcal T_X(I)$ assigned to $X$, then $|\mathcal T_X(I)|\leq B$.
\end{lemma}
\begin{proof}
Let $[p,q]=I$. First we follow the approach of Lemma 6.16 in \cite{mms} to prove the existence of a bound, depending on $X$ only, for $d_{\cc(X)}\left(V(\tau_p|X),V(\tau_q|X)\right)$. Since $p,q\in J_Y\subseteq I_Y$, Theorem \ref{thm:mmsstructure} asserts that $\partial Y$ is wide when put in efficient position with respect to $\tau_p$ or to $\tau_q$; and, via lift, also when put in efficient position with respect to $\tau_p^X$ or $\tau_q^X$. Let $\alpha\in V(\tau_p|X)$ and identify it with a carried realization, which we may suppose to realize the intersection number of $\alpha$ with $\partial Y$ --- if it does not, one may fix this with the bigon removal technique employed in the proof of Corollary 4.3 from \cite{mms}. 

As $\partial Y$ is wide in $\tau_p$, it crosses each of the branch rectangles in $\bar\nei(\tau_p)$ --- which are at most $N_1$: see Lemma \ref{lem:vertexsetbounds} --- in at most two connected components each. So it intersects at most $2N_1$ branch rectangles of $\tau_p^X$. For each branch rectangle $R_b$ of $\tau_p^X$, the intersections $\alpha\cap R_b$ and $\partial Y\cap R_b$ consist of at most two arcs each; so $\alpha\cap\partial Y\cap R_b$ is at most $4$ points, else there is a bigon. This bounds $i\left(\alpha,\partial Y\right)$; and, via Lemma \ref{lem:cc_distance}, also $d_X\left(\alpha,\partial Y\right)$.

The same bound applies for all $\alpha\in V(\tau_q|X)$. So, via triangle equality, we also have a bound for $d_{\cc(X)}\left(V(\tau_p| X),V(\tau_q|X)\right)$; and, via Lemma \ref{lem:induction_vertices_commute}, for $d_X(p, q)$.

Now we prove that $I$ is straight i.e. we look for a bound for $d_Z\left(J\right)$, for $Z\in \mathsf{Ind}$ and $\emptyset\not=J\subseteq I$ a subinterval. First we find a bound for $d_Z(J\cap I_Z)$, provided that the given interval $J\cap I_Z=[k,l]$ is nonempty. For ease of notation, let also $I'\coloneqq I\cap I_Z$.

Since $\dn [k,l]$ contains only indices of $\mathcal T_X(J_X)$ assigned to $X$, and not to any proper, inductive subsurface, $(\utw\tau_{j'}|\utw Z)_{j'=\dn k}^{\dn l}$ includes no split move. Define $\tau_0\coloneqq t(\min I')$, and $t_1$ as the highest $t$ such that $\max I' \geq \max DI_t$. Then $\bigcup_{t=t_0}^{t_1} (DI_t\cup NI_t)\subseteq DI_{t_0}\cup NI_{t_0}\cup I'$ and $I'\subseteq \left(\bigcup_{t=t_0}^{t_1} (DI_t\cup NI_t)\right)\cup DK_{t_1+1}$ (by convention, $DK_{r+1}=\{N\}$).

We know from claim 1 in Lemma \ref{lem:untwistedsubsurfaces} that, for all $t_0<t\leq t_1$: $\gamma_t$ does not cut $\partial Z$; moreover, the splits in $\utw\bm\tau$, indexed by $DI_t^\utw$, shall be invisible when inducing the tracks to $\utw Z$. This means that $\utw\gamma_t$ does not intersect $\utw Z$ at all, and that $\gamma_t$ does not intersect $Z$.

Claim 7 in Lemma \ref{lem:untwistedsubsurfaces} applies for $I'$, and therefore for $J$. The entries of $\bm\tau(I')$ which do not reflect in an entry of $\utw\bm\tau(\dn I')$ are contained in some $DI_t\setminus DL_t$ for $t_0+1\leq t\leq t_1$ or in $DK_{t_1+1}$. Set $J''\coloneqq \left(DK_{t_1+1}\cap [k,l]\right)\cup \{l\}$ (so that $l\in J''$ anyway). By said claim, the sequence $(\utw\tau_{j'}|\utw Z)_{j'\in \dn J}$ is obtained from $(\tau_j|Z)_{j=k}^{\min J''}$ by application of $\hat \phi_{t_0}:S^Z\rightarrow S^{\utw Z}$ to each entry, and possibly removal of some repeated ones. If $\tau_j|Z$ splits to $\tau_{j+1}|Z$ for $j,j+1\in [k,l]$, then neither of the two belongs to $DI_t\setminus DL_t$, for $t_0+1\leq t\leq t_1$, otherwise disjointness of $\gamma_t$ from $Z$ would be contradicted. So in that case $j,j+1\in DK_{t_1+1}\cap [k,l]$.

When one induces to $Z$ the subsequence $\bm\tau(k, \min J'')$, only comb equivalences and subtrack extractions are seen. While a subtrack always has less non-mixed branches than the almost track it is taken from, comb equivalences are unable to alter the count of these branches. So $d_{\cc(Z)}\left(V(\tau_k|Z),V(\tau_{\min J''}|Z)\right)\leq N_1K_2$.

And the distance spanned in $J''$, when this interval is not trivial, is cared after with Lemma \ref{lem:pantsboundunderdt} and Remark \ref{rmk:pantsboundunderdt_cutting}\footnote{These two exclude the case of $\gamma_{t_1+1}$ disjoint from $Z$, but the way to deal with it is obvious.}: $d_{\cc(Z)}\left(V(\tau_{\min J''}|Z),V(\tau_l|Z)\right)\leq C_2$. Combining the two estimates, and using Lemma \ref{lem:induction_vertices_commute}, we have $d_Z(k,l)\leq N_1K_2 + C_2 + 2 F\left(8N_1(S^X)\right)$.

In general, we may need to add a contribution for the distance spanned in $[\min J,k]$ and in $[l,\max J]$ (if either, or both, are nonempty): so $d_Z(J)\leq N_1K_2 + C_2 + 2 F\left(8N_1(S^X)\right)+2\mathsf{K_0}$, by Theorem \ref{thm:mmsstructure}. This last bound holds also if $J\cap I_Z$ is empty.

This proves that $I$ is a straight interval, and there is a bound for $d_X(I)$: but then, Lemma \ref{lem:mms613} ensures that there is a bound $|\mathcal T_X(I)|\leq B$, too.
\end{proof}

\begin{lemma}\emph{(Plays the role of \cite{mms}, Lemma 6.22)}
There is a constant $A=A(X)$ such that
$$
\left|\bigcup_{Y\in \mathsf{Ind}}\mathcal T_X(J_Y)\right|\leq_A |\mathsf{Ind}|+ \sum_{Y\in \mathsf{Ind}} d_Y(J_X).
$$
\end{lemma}
\begin{proof}
Given any maximal interval $I$ among the ones treated in the lemma above, we have just shown that $|\mathcal T_X(I)|\leq B$. This implies that, for a suitable constant $A'$,
$$\left|\bigcup_{Y\in \mathsf{Ind}}\mathcal T_X(J_Y)\right|\leq_{A'}  \left|\bigcup_{Y\in \mathsf{Ind}}\mathsf{AI}_Y(J_Y)\right|\leq \sum_{Y\in\mathsf{Ind}} |\mathcal T_Y(J_Y)|.$$
We now apply the inductive hypothesis given by the key claim: $|\mathcal T_Y(J_Y)|\leq_A d_{\pa(Y)}(J_Y)$ for a constant $A=A(X)$.

Let $C\coloneqq 1+\max\{ M_6(Y), \mathsf{T}_0(X)+2\mathsf{R}_0\}$. Then, as shown in formula \ref{eqn:mmsubsurfaceproj_mixed} above,
$$
d_{\pa(Y)}(J_Y)\leq_{\mathsf{E}} \sum_{Z\subseteq Y\text{ non-annular}} [d_Z(J_Y)]_C
$$
for a suitable constant $\mathsf E(S,C)>1$. The proof ends as the one of Lemma 6.22 of \cite{mms}, verbatim.
\end{proof}

The proof ends for us the same way as the proof of Proposition 6.9 in \cite{mms} after proving Lemma 6.22. It just suffices to replace occurrences of $\mathcal S_X$ there with $\mathcal T_X$; the ones of $\mathcal M(X)$ with $\pa(S)$; and references to lemmas previously proved there with the lemmas above. 

It is just worth marking that the final estimate $|\mathsf{Ind}'|\leq_A d_{\pa(X)}(J_X)$ may be proved simply by saying that
$$
|\mathsf{Ind}'|\leq \frac{1}{\mathsf T_0}\sum_{Y\in\mathsf{Ind'}} [d_Y(J_X)]_{\mathsf T_0} \leq_{\mathsf E(X;\mathsf T_0)} \frac{1}{\mathsf T_0}d_{\pa(X)}(J_X)
$$
again by application of formula \ref{eqn:mmsubsurfaceproj_mixed}.
\cvd

\begin{coroll}\label{cor:hardttbound}
There is a constant $C_8=C_8(S)$ such that the following is true. Let $\bm\tau=(\tau_j)_{j=0}^N$ be a generic splitting sequence of birecurrent, cornered train tracks on $S$. Let $X$ be a subsurface of $S$ with $X\supseteq S'$ the subsurface, not necessarily connected, filled by $V(\tau_0)$, and suppose that $V(\tau_N|X)$ is a vertex of $\pa(X)$: then
$$d_{\pa(X)}\left(V(\tau_0|X),V(\tau_N|X)\right)\geq_{C_8} |\utw(\rar\bm\tau)|.$$
\end{coroll}

\begin{proof}
Subdivide $\bm\tau=\bm\tau^1*\bm\epsilon^2*\bm\tau^2*\ldots*\bm\epsilon^w*\bm\tau^w$ as it is done before Definition \ref{def:not_firmly}. To simplify notations, we may replace each $\bm\tau^u$ with the respective $\rar\bm\tau^u$. For $1\leq u\leq w$, let $J^u$ be the interval of indices in $\bm\tau$ supplied by $\bm\tau^u$. The subsequence $\bm\tau^u$ will evolve firmly in a (possibly disconnected) subsurface $S^u\subset S$ and so will do $\utw\tau^u$; let $S^u=T_1^u\sqcup\ldots\sqcup T_{k(u)}^u$ be the decomposition into connected components.

\step{1} for each $1\leq u\leq w$, $|\utw\bm\tau^u|\leq \frac{1}{N_0N_4+1} \sum_{i=1}^{k(u)} |\utw\bm\tau^u|_{T_i^u}$. When using this notation, we agree that $|\utw\bm\tau^u|_{T_i^u}=0$ when $T_i^u$ is an annulus.

Let $p,q\in \dn J^u$ be two indices such that no split in $\utw\bm\tau(p,q)$ is visible in any of the non-annular connected components of $S^u$. Let $T^u_i$ be non-annular. Our assumption on the interval of indices $[p,q]$ implies that, within this interval, the induced tracks $\utw\tau_j|T^u_i$ may change only under comb equivalences and subtrack extractions. Consequently, if $j'>j$, then $V(\utw\tau_{j'}|T^u_i)\subseteq V(\utw\tau_j|T^u_i)$. In Lemma \ref{lem:decreasingfilling} we have shown that $V(\utw\tau_j|T^u_i)=V(\utw\tau_j)\cap \cc(T^u_i)$.

So we have $V(\utw\tau_j)=\Xi^u\cup\bigcup_{T^u_i\text{ non-annular}} V(\utw\tau_j|T^u_i)$, where $\Xi^u$ is the collection of all core curves of the $T^u_i$ which are annuli; therefore this set is also decreasing as $j$ increases in $[p,q]$. As a consequence of Lemma \ref{lem:vertexsetbounds}, then, $V(\utw\tau_j)$ changes at most $N_0$ times within the interval $[p,q]$. By Lemma \ref{lem:mms614} applied on the entire surface $S$, then, $|\utw\bm\tau^u(p,q)|\leq N_4N_0$.

In other words, among every $N_0N_4+1$ consecutive splits in $\utw\bm\tau^u$, at least one of them must be visible in one of the connected components of $S^u$.

\step{2} we claim that there is a constant $A=A(S)$ such that, for each $1\leq u\leq w$,
$$
|\utw\bm\tau^u|\leq_A \sum_{\substack{Y\subset X\text{ essential} \\ \text{and non-annular}}} [d_Y(J^u)]_M
$$
where $M\geq \max \left(\{M_6(X)|X\text{ subsurface of }S\}\cup\{2\}\right)$ is fixed.

According to formula \ref{eqn:mmsubsurfaceproj_mixed} above, for all $Z\subseteq X$, $Z$ subsurface of $S$, there is a constant $e=e(S,M)$ such that:
$$
d_{\pa(Z)}(J^u)=_{e} \sum_{\substack{Y\subset Z\text{ essential} \\ \text{and non-annular}}} [d_Y(J^u)]_M \eqqcolon \mathrm{sum}(Z,M,u)
$$
for a suitable constant $M$ which we may suppose to be depending only on $S$. The formula may be given sense also for $Z$ an annulus: in that case the summation is empty, and the $d_{\pa(Z)}(J^u)$ can also be set to $0$.

Clearly (see the Remark following Lemma \ref{lem:decreasingfilling}), $S^u\subseteq S'\subseteq X$. The family of all non-annular subsurfaces $Y\subset X$ can be partitioned into:
\begin{itemize}
\item the sub-families of subsurfaces $Y\subset T_i^u$ --- one for each $1\leq i\leq k(u)$;
\item the sub-family of all subsurfaces $Y$ which are cut by $\partial S^u$;
\item the sub-family of all subsurfaces $Y$ essentially disjoint from $S^u$.
\end{itemize}
If $Y$ is any of the subsurfaces as in the second bullet and $j,j'\in J^u$, then any $\alpha\in \pi_Y(V(\tau_j)), \beta\in \pi_Y(V(\tau_{j'}))$  will not intersect $\pi_Y(\partial S^u)\not=\emptyset$. So $d_Y(\alpha,\beta)\leq 2\leq M$. The surfaces as in the third bullet, instead, just do not exist, because $V(\tau_N|X)\in \pa^0(X)$ implies $V(\tau_j|X)\in \pa^0(X)$ for all $j\in J^u$, too. So $\ol{X\setminus S^u}$ consists of discs, punctured discs, annuli and pairs of pants.

This yields that only the subsurfaces as in the first bullet count in the summation $\mathrm{sum}(X,M,u)$, which is thus equal to $\sum_{i=1}^{k(u)} \mathrm{sum}(T_i^u,M,u)$.

From Step 1 above and Proposition \ref{prp:hardttbound} we have

\begin{center}
$|\utw\bm\tau^u| \leq \frac{1}{N_0 N_4+1} \sum_{i=1}^{k(u)} |\utw\bm\tau^u|_{T_i} \leq_{C_7(N_0 N_4+1)} \sum_{i=1}^{k(u)} d_{\pa(T_i)}(J^u).$
\end{center}

According to equalities we have established previously, then, there is a constant $A=A(S)$ such that

\begin{center}
$|\utw\bm\tau^u|\leq_A \sum_{i=1}^{k(u)}\mathrm{sum}(T_i^u,M,u) = \mathrm{sum}(X,M,u).$
\end{center}

\step{3} proof of the statement.

Recall the constant $K_2$ introduced in the proof of Proposition \ref{prp:hardttbound}. Fix\linebreak $M\geq \max\left( \{M_6(X)|X\text{ subsurface of }S\}\cup\{2\}\right) + 2(\xi(S)-1)\mathsf{R}_0$.

Note that, for each subsurface $Y\subset X$, $d_Y(\tau_0,\tau_N)\geq \sum_{u=1}^w d_Y(J^u)-2(w-1)\mathsf{R}_0$ by repeated application of Lemma \ref{lem:reversetriangle}.

Let $E\coloneqq \max_{0\leq j\leq\xi(S)}e\left(S,M-2j\mathsf{R}_0\right)$; and let $M'\coloneqq M-2(w-1)\mathsf{R}_0$. Then
\begin{eqnarray*}
 & d_{\pa(X)}(0,N)=_E \sum_{\substack{Y\subset X\text{ essential} \\ \text{and non-annular}}} [d_Y(0,N)]_{M'} & \\
 & \geq \sum_{\substack{Y\subset S\text{ essential} \\ \text{and non-annular}}} \left([d_Y(J^1) + \ldots + d_Y(J^w)]_M-\mathsf r(X))\right)
\end{eqnarray*}
where $\mathsf r(X)=2(w-1)\mathsf{R}_0$ if the other summand is nonzero, and $0$ otherwise. A simple computation shows that $[x]_M-\mathsf r(X)\geq \left(1-\frac{2(\xi(S)-1)\mathsf{R}_0}{M}\right)[x]_M$ (the bracketed term is positive). So, if $E'=E\left(1-\frac{2(\xi(S)-1)\mathsf{R}_0}{M}\right)^{-1}$, then
\begin{eqnarray*}
 & d_{\pa(X)}(0,N) \geq_{E'} \sum_{\substack{Y\subset X\text{ essential} \\ \text{and non-annular}}} [d_Y(J^1) + \ldots + d_Y(J^w)]_M & \\
 & \geq \sum_Y \left([d_Y(J^1)]_M + \ldots + \sum_X [d_Y(J^w)]_M\right) =\sum_{u=1}^w \mathrm{sum}(X,M,u). &
\end{eqnarray*}

Step 1 above gives $\sum(X,M,u)\geq_A |\utw\bm\tau^u|=|\psi_u\cdot \utw\bm\tau^u|$: therefore\linebreak $\sum_{u=1}^w \mathrm{sum}(X,M,u)\geq_A |\utw\bm\tau|-\xi(S)+1$. This concludes the proof.
\end{proof}

From Proposition \ref{prp:easyttbound} and Corollary \ref{prp:hardttbound} we easily derive:
\begin{coroll}\label{cor:ttsamelength}
Let $\bm\tau=(\tau_j)_{j=0}^N$ be a generic splitting sequence of birecurrent, cornered train tracks on $S$. Let $X$ be a subsurface of $S$ with $X\supseteq S'$ the subsurface, not necessarily connected, filled by $V(\tau_0)$, and suppose that $V(\tau_N|X)$ is a vertex of $\pa(X)$. Let $\bm\tau'$ be another splitting sequence, beginning and ending with the same train tracks as $\bm\tau$. Then
$$
|\utw(\rar\bm\tau)| =_{C_6C_8} |\utw(\rar\bm\tau')|.
$$
\end{coroll}

We may now prove a slightly generalized statement for Theorem \ref{thm:core}:

\begin{theo}\label{thm:main_full}
Let $\bm\tau=(\tau_j)_{j=0}^N$ be a generic splitting sequence of birecurrent train tracks on $S$, \emph{not necessarily cornered ones}. Let $X$ be a subsurface of $S$ with $X\supseteq S'$ the subsurface, not necessarily connected, filled by $V(\tau_0)$, and suppose that $V(\tau_N|X)$ is a vertex of $\pa(X)$. Then, for a constant $C_9=C_9(S)$ independent of $\bm\tau$,
$$d_{\pa(X)}(0,N)=_{C_9} |\utw(\rar(\cnr\bm\tau))|.$$
\end{theo}

\begin{proof}
Consider a cornerization $\cnr\bm\tau=(\cnr\tau_j)_{j=0}^M$ of the splitting sequence $\bm\tau$. Point 4 in Lemma \ref{lem:ctauproperties} gives, in particular, that $\cnr\tau_0$ is a cornerization of $\tau_0$ and $\cnr\tau_M$ is a cornerization of $\tau_N$. In particular, as $V(\tau_0|X)\subseteq \cc(\cnr\tau_0|X)$ and $V(\tau_N|X)\subseteq \cc(\cnr\tau_M|X)$, an application of Lemma \ref{lem:decreasingfilling} and following observations gives that $V(\cnr\tau_0|X)$, $V(\cnr\tau_M|X)$ are vertices of $\pa(X)$.

By Remark \ref{rmk:centralsplitbound} the number of splits turning $\cnr\tau_0$ into $\tau_0$, and $\cnr\tau_M$ into $\tau_N$, is bounded in terms of the topology of $S$. Therefore $d_\pa(\tau_0,\cnr\tau_0)$ and $d_\pa(\cnr\tau_M,\tau_N)$ are also bounded.

Finally,
$$
d_{\pa(X)}(\cnr\tau_0,\cnr\tau_M)=_{\max\{C_6,C_8\}} |\utw(\rar(\cnr\bm\tau))|,
$$
by a combination of Proposition \ref{prp:easyttbound} and of Corollary \ref{cor:subsurface_bijection}.

The triangle inequality completes the proof.
\end{proof}

\begin{coroll}
Let $\bm\tau=(\tau_j)_{j=0}^N$ be a generic splitting sequence of birecurrent train tracks on $S$. Let $X$ be a subsurface of $S$ with $X\supseteq S'$ the subsurface, not necessarily connected, filled by $V(\tau_0)$, and suppose that $V(\tau_N|X)$ is a vertex of $\pa(X)$. Then $(\sigma_j)_{j=0}^M=\bm\sigma\coloneqq \rar(\cnr\bm\tau)$ describes an unparametrized quasi-geodesic in the pants graph.

If $J$ is the sequence of indices $0\leq j< M$ such that $j,j+1 \in DL_t\cup NI_t$ for some $0\leq t \leq q$ (see Definition \ref{def:untwistedsequence} and following constructions) and $\sigma_j$ splits to $\sigma_{j+1}$, then $\left(V(\sigma_j)\right)_{j\in J}$ describes a $\max\{C_6,C_8\}$-quasi-geodesic in $\pa(X)$.
\end{coroll}
\begin{proof}
With minor adaptations to Proposition \ref{prp:easyttbound}, Proposition \ref{prp:hardttbound}, Corollary \ref{cor:hardttbound}, one proves that for any two indices $0\leq k\leq l\leq M$,
$$
|(\utw\bm\sigma)(\dn k,\dn l)|\leq_{C_8} d_{\pa(X)}\left(V(\sigma_k),V(\sigma_l)\right)\leq_{C_6} |(\utw\bm\sigma)(\dn k,\dn l)|
$$
which is just a restatement of our second claim.

In order to prove the first claim, it is sufficient to prove that\linebreak $d_{\pa(X)}\left(V(\sigma_a),V(\sigma_b)\right)\leq C_2$ for all $a,b$ comprised between two consecutive indices in $J$. For any two indices as such, there is a $t$ such that, for all $j\in [a,b]\setminus DI_t$, $\sigma_{j+1}$ is obtained from $\sigma_j$ with a slide. We conclude with Lemma \ref{lem:pantsboundunderdt} and Remark \ref{rmk:pantsboundunderdt_cutting}, combined.
\end{proof}