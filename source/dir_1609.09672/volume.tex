\chapter{Hyperbolic volume estimates}

\section{Pseudo-Anosov mapping tori}

In all this section, $S$ is a \emph{closed} surface and $\psi:S\rightarrow S$ is a pseudo-Anosov diffeomorphism. We give an application of our main theorem and of Theorem \ref{thm:brockmappingtori} proved by Brock. Recall the introductory notions given in \S \ref{sub:mappingtori}.

Given a measured lamination $(\lambda,\mu)$ and a train track $\tau$ on $S$, one may say that $\lambda$ is \nw{carried} by $\tau$ if $\lambda$ can be ambient-isotoped to lie in $\nei(\tau)$, in a way that is transverse to all ties: we are just repeating Definition \ref{def:carried} to suit this setting. Uniqueness of carrying (Lemma 1.7.11 in \cite{penner}, previously simplified in Proposition \ref{prp:carryingunique} here) still holds. The bijection between rational transverse measures on a train track and weighted multicurves, described in Proposition \ref{prp:measurecurvecorresp} here, is a simplified version of Theorem 1.7.12 in \cite{penner}, establishing a similar bijection between \emph{real} transverse measures and carried measured laminations.

We say that $\tau$ is \nw{suited} to $(\lambda,\mu)$ if $\lambda$ is fully carried by $\tau$ and $\bar\nei(\tau)$ does not admit any carried, properly embedded arc which is disjoint from some carried realization of $\lambda$ --- this is equivalent to the definition given in \cite{papadopoulos} in terms of foliations, via the correspondence explained in \cite{levitt}. Note that, when such an arc exists instead, it is impossible that its ends lie along the same component of $\partial_v\bar\nei(\tau)$. We have that
\begin{claim}
Every measured lamination $(\lambda,\mu)$ has a birecurrent train track $\tau$ which is suited to it.
\end{claim}
This is a consequence of Corollary 1.7.6 in \cite{penner}: its statement only guarantees the existence of a birecurrent $\tau$ which fully carries $(\lambda,\mu)$; but, if $\tau$ is not suited to $\lambda$, it is sufficient to perform a multiple central split (see Definition \ref{def:multiplesplit}) along any of the carried arcs of $\tau$ which are disjoint from the carried realization of $\lambda$: this keeps the train track transversely recurrent, and also recurrent because the new track still fully carries $(\lambda,\mu)$ and we may apply Proposition 1.3.1 in \cite{penner}. Repeat until there are no more carried arcs disjoint from the carried realization of $\lambda$: this takes a finite number of steps.

In \cite{agol_pa} it described how, starting from $(\lambda,\mu)$ and a generic track $\tau_0$ suited to it, one may define the \nw{maximal splitting sequence} $\bm\tau=(\tau_j)_{j=0}^{+\infty}$: if $\tau_j$ has been defined, split simultaneously all branches of $\tau_j$ which are given maximal weight by $(\lambda,\mu)$, with the only parity which keeps $(\lambda,\mu)$ carried by the new track $\tau_{j+1}$. This sequence will not feature any central split. And in Theorem 3.5 of that work, which improves Theorem 4.1 in \cite{papadopoulos}, it is proved that
\begin{theo}
Let $(\lambda^s,\mu^s)$ be the stable lamination of $\psi$, and let $\tau_0$ be a train track suited to $(\lambda^s,\mu^s)$. The maximal splitting sequence $\bm\tau$ built from $\tau_0$ has two associated numbers $m,n$ such that, for all $j\geq m$,
$$
\tau_{j+n}=\psi(\tau_j).
$$
Moreover, if $\mu_j\in\mathcal M(\tau)$ is the transverse measure induced by $(\lambda^s,\mu^s)$ on $\tau_j$, then $\mu_{j+n}=c^{-1}\psi_*(\mu_j)$ where $c>1$ is the constant associated to the pseudo-Anosov diffeomorphism $\psi$.
\end{theo}

We prove the following
\begin{theo}\label{thm:agol_volume}
Let $(\lambda^s,\mu^s)$ be the stable lamination of $\psi$, and let $\tau_0$ be a birecurrent train track suited to $(\lambda^s,\mu^s)$. Let $\bm\rho\coloneqq \bm\tau(m,m+n)$, using the notation of the above theorem. Let $M\coloneqq \faktor{S\times [0,1]}{\sim_\psi}$ be the mapping torus built from $S$ and $\psi$ --- which is hyperbolic. Then there is a constant $C_{10}$, only depending on $S$, such that
$$
\vol(M)=_{C_{10}} |\utw(\rar(\bm\rho))|.
$$
\end{theo}

Note: for the purposes of this theorem and in order to connect it with the previously developed theory, $\bm\rho$ may be considered as a sequence where exactly one split occurs at each move: if more are split simultaneously, we just insert more intermediate steps.

\begin{lemma}
All train tracks $\tau_j$ in a splitting sequence $\bm\tau$ as in Theorem \ref{thm:agol_volume} are cornered and birecurrent, and $\bm\tau$ evolves firmly in $S$.
\end{lemma}
\begin{proof}
Each $\tau_j$ is cornered: if $\partial\bar\nei(\tau_j)$ includes a smooth component, then necessarily $\lambda^s$ includes a component which is a closed geodesic; but this contradicts minimality of the stable lamination.

Since $\tau_0$ is transversely recurrent, all $\tau_j$ are. Also, they are recurrent because they all fully carry the measured lamination $(\lambda^s,\mu^s)$, so Proposition 1.3.1 in \cite{penner} applies.

Let $S'$ be the subsurface of $S$, possibly a disconnected one, which is filled by $\cc(\tau_m)$ --- and by $V(\tau_m)$, by Lemma \ref{lem:decreasingfilling}. Then $\cc(\tau_{m+n})=\psi\cdot\cc(\tau_m)$ fills $\psi(S')$; moreover, up to isotopies, $\psi(S')\subseteq S'$, because of the decreasing filling properties of splitting sequences, stated after Lemma \ref{lem:decreasingfilling}. On the other hand, $\xi(S')=\xi\left(\psi(S')\right)$\footnote{We may just define $\xi(S')$ as the sum of the complexities of each connected component.}, so $S',\psi(S')$ are isotopic in $S$. If $S'\subsetneq S$, then $\psi$ fixes $\partial S'$: and this contradicts the fact that $\psi$, being pseudo-Anosov, is in particular irreducible.

So each set $\cc(\tau_{m+in})$ or $V(\tau_{m+in})$ for $i\geq 0$ fills $S$. Since the filled surface decreases along a splitting sequence, all $V(\tau_j)$, $j\geq 0$, fill $S$.
\end{proof}

\begin{proof}[of Theorem \ref{thm:agol_volume}]
Note that $M$ is hyperbolic because of Theorem \ref{thm:mappingtorushyperbolic}.

\step{1} we prove that is sufficient to show the existence of a constant $A=A(S)$ such that, for all $z\in \mathbb Z_{>0}$,\footnote{Here we use again the notation $d_{\pa(S)}(m,m+zn)$ when the splitting sequence is understood, as it was introduced in \S \ref{sub:untwistedsequence}.}
$$
d_{\pa(S)}(m,m+zn)\geq_A z|\utw(\rar(\bm\rho))|.
$$

Suppose that this condition is true, and note that actually $d_{\pa(S)}(m,m+zn)=\linebreak d_{\pa(S)}\left(V(\tau_m),\psi^z\cdot V(\tau_m)\right)$. Lemma \ref{lem:pantsquasiisom} proves that the inclusion $\pc(S)\hookrightarrow \pa(S)$ is a quasi-isometry, and this implies that there exists a pants decomposition $p$ of $S$ such that $d_{\pc(S)}(p, \psi^z(p))\geq_{A'} z|\utw(\rar(\bm\rho))|$ for another constant $A'=A'(S)$, and for all $z\in \mathbb Z_{>0}$. So the stable translation distance in $\pc(S)$ is $|\psi|^{st} \geq_{A'} |\utw(\rar(\bm\rho))|$.

On the other hand, Proposition \ref{prp:easyttbound} gives that $d_{\pa(S)}(m,m+n)\leq_{A''} |\utw(\rar(\bm\rho))|$, and this proves that also $|\psi|\leq_{A'''} |\utw(\rar(\bm\rho))|$ (again here $A''$, $A'''$ depend on $S$ only). Remark \ref{rmk:stable_dist_pc} and Brock's Theorem \ref{thm:brockmappingtori} conclude the argument.

Let $\bm\omega\coloneqq\rar\bm\rho$, indexed as $(\omega_j)_{j=0}^N$; for $z\in \mathbb Z_{>0}$, let $\bm\omega^{*z}\coloneqq \bm\omega*(\psi\cdot \bm\omega)*\ldots*(\psi^{z-1}\cdot \bm\omega)$: $\bm\omega^{*z}$ begins and ends with the same train tracks as $\bm\tau(m,m+zn)$.

\step{2} given any curve $\gamma\in \cc(S)$, the indices $0\leq i \leq z-1$ such that $\gamma$ is a twist curve for at least one entry in the splitting sequence $\bm\omega^i\coloneqq \psi^i\cdot\bm\omega =\bm\omega^{*z}\left(iN,(i+1)N\right)$ are at most $N_0+1$, and they are all consecutive. $N_0$ was introduced in Lemma \ref{lem:vertexsetbounds}.

Let $i$, $i'$ be two indices such that $\gamma$ is an effective twist curve in $\bm\rho^i$, $\bm\rho^{i'}$. Suppose, for a contradiction, that $i'-i\geq N_0+2$. Then $\gamma$ is a twist curve for at least one entry of $\bm\rho^i$ and one of $\bm\rho^{i'}$. Necessarily (Lemma \ref{lem:twistcurvebasics} applied to a sufficiently long initial segment of the sequence $\bm\tau$), it is a twist curve in all entries in $\bm\rho^j$ for all $i<j<i'$; and in particular $\gamma$ is a twist curve in $\tau_{m+jn}=\psi^j(\tau_m)$ for all $i<j\leq i'$.

Therefore $\psi^{-j}(\gamma)$ is a twist curve in $\tau_m$ for all $i<j\leq i'$. No two curves in this family are isotopic, because all $\psi^{-j}$ are pseudo-Anosov (see Remark \ref{rmk:power_pa}). They also all belong to $W(\tau_m)$, whose size is $\leq N_0$. This is a contradiction.

\step{3} we define a generalized version of the untwisted sequence, fitted to our scenario, and prove some basic properties.

If $\gamma_1,\ldots,\gamma_r$ are the effective twist curves of $\bm\omega$ (and of $\bm\rho$), it is possible to subdivide $[0,N]$ into $NI_0,DI_1,NI_1,\ldots, DI_r, NI_r$ as shown in \S \ref{sub:untwistedsequence}. Since $\bm\omega^{*z}$ is the concatenation of $z$ `copies' of $\bm\omega$, each transformed under a suitable power of $\psi$, one may similarly subdivide $[0,zN]$ into
$$NI^*_0,DI^*_1,NI^*_1,\ldots, DI^*_{zr}, NI^*_{zr}$$
with the maximum of each interval coinciding with the minimum of the following one. Note that each $NI^*_{ir}$, $0< i < z$, is the concatenation of a `copy' of $NI_r$ and a `copy' of $NI_0$. One defines also $DK^*_t, DL^*_t\subseteq DI^*_t$ just as seen in \S \ref{sub:untwistedsequence}.

For $1\leq t \leq r$, $1\leq i \leq z-1$, define $\gamma_{ri+t}=\psi^i(\gamma_t)$: then $\bm\omega^{*z}(DI^*_t)$ has twist nature about $\gamma_t$ for all $1\leq t\leq zr$. The curves in $(\gamma_t)_{t=1}^{zr}$ are \emph{not} necessarily all distinct, but we have proved in Step 2 that each of them occurs at most $N_0+1$ times in this enumeration.

If they were all distinct, then $\bm\omega^{*z}$ would be $(\gamma_t)_{t=1}^{zr}$-arranged: so we may say that the above notation is a `variation' of the notation used to describe arranged sequences, in a more general case. We will see now how to generalize the constructions developed in \S \ref{sec:traintrackconclusion}.

For notational convenience, let $\eta_1,\ldots, \eta_Q$ be an enumeration of the $(\gamma_t)_{t=1}^{zr}$ such that no curve occurs twice. The splitting sequence $\bm\omega^{*z}$ is $(\eta_1,\ldots, \eta_Q)$-arranged, even if in general each $\eta_u$ admits more than one choice for a Dehn interval; and one may need to change the order in which these curves are listed, in order to have $\left(\min DI_{\eta_u}\right)_{u=1}^Q$ increasing.

The definition of $\utw\bm\omega=(\utw\omega_j)_{j=0}^{N'}$ from $\bm\omega$ in Definition \ref{def:untwistedsequence} involves application of diffeomorphisms $\phi_t$ for $1 \leq t \leq r$. Let $\Psi\coloneqq \phi_r^{-1}\circ \psi$, and let
$$
\utw^{*z}\bm\omega\coloneqq \utw\bm\omega*(\Psi\cdot \utw\bm\omega)*\ldots*(\Psi^{z-1}\cdot \utw\bm\omega).
$$

$\utw^{*z}\bm\omega=(\utw^{*z}\omega_j)_{j=0}^{zN'}$ serves as a generalization of the concept of untwisted sequence for $\bm\omega^{*z}$: informally, the latter sequence is `arranged except that the sequence $(\gamma_t)$ may include repetitions of the same curve'. The construction to get $\utw^{*z}\bm\omega$ from $\bm\omega^{*z}$ is indeed exactly the same as in \S \ref{sub:untwistedsequence}, except that here we do not require the curves $(\gamma_t)_{t=1}^{zr}$ to be distinct. A sequence of subintervals in $[0, zN']$:
$$NI^{\utw*}_0,DI^{\utw*}_1,NI^{\utw*}_1,\ldots, DI^{\utw*}_{zq}, NI^{\utw*}_{zr}$$
is naturally defined. The function $\dn:[0,zN]\rightarrow [0,zN']$, which, for all $1\leq t \leq zr$, maps $DL^*_t$ onto the respective $DI^{\utw*}_t$ and $NI^*_t$ onto the respective $NI^{\utw*}_t$, is defined just as in \S \ref{sub:untwistedsequence}, and so are the maps $\up$, $[t]\dn$, $[t]\up$. For $X\subseteq S$ a subsurface, denote $I^*_X, I^{\utw *}_X$ its accessible interval with respect to the splitting sequences $\bm\omega^{*z}$, $\utw^{*z}\bm\omega$ respectively.

For $j\in [0,zN]$, let $t(j)$ be the least index $t$ such that $j\in DI^*_t$ or $j\in NI^*_t$. For $j'\in [0,zN']$, let $\utw^* t(j')$ be the least index $t$ such that $j\in DI_t^{\utw *}$ or $j\in NI_t^{\utw *}$. There are diffeomorphisms $\phi^*_t$ for $1\leq t\leq zq$ such that, for all $j\in \bigcup_{t=0}^{zr} (DL^*_t\cup NI^*_t)$, $\utw^{*z}\omega_{\dn j}=\phi_{t(j)}(\omega^{*z}_j)$; and for all $j\in[0,N']$, $\utw^{*z}\omega_{j'}=\phi_{\utw^* t(j')}(\omega^{*z}_{\up j'})$. 

All claims in Lemma \ref{lem:untwistedsubsurfaces} work also in this setting (replace $\bm\tau$ with $\bm\omega^{*z}$, $\utw\bm\tau$ with $\utw^{*z}\bm\omega$, all index intervals with the starred versions defined here): this is because the proof of that lemma makes no use of the fact that the curves $\gamma_t$ were distinct in its original setting. Similarly, the terminology introduced in Corollary \ref{cor:subsurface_bijection} is immediately adapted so that it will work here.

And using that terminology, if one defines the sequence $\utw^*\gamma_1\coloneqq \phi^*_{\nei(\gamma_1)}(\gamma_1)$, \ldots, $\utw^*\gamma_{zr}\coloneqq \phi^*_{\nei(\gamma_{zr})}(\gamma_{zr})$, each $\utw^{*z}\bm\omega(DI^{\utw*}_t)$ has twist nature with respect to $\utw^*\gamma_t$. Two $\gamma_t,\gamma_{t'}$ coincide (up to isotopy) if and only if $\utw^*\gamma_t,\utw^*\gamma_{t'}$ do. So $\utw^{*z}\bm\omega$ is $(\utw^*\eta_1,\ldots,\utw^*\eta_Q)$-arranged.

\step{4} we claim a modified version of Proposition \ref{prp:locallyfinite}:
\begin{claim}
There is a constant $C'_5(S)$ such that the following is true.

Let $X\in \Sigma(\bm\omega^{*z})$, $X$ not an annulus, and let $[k,l]\subseteq [0,zN]$, with $[k,l]\subseteq I^*_X$ if $X\not=S$. Then
$$
d_{\pa(\utw^* X)}(\dn k,\dn l)^{\utw*} \leq C'_5 \left(d_{\pa(X)}(k,l)\right)^2
\text{ and }
d_{\pa(X)}(k,l) \leq C'_5 \left(d_{\pa(\utw^* X)}(\dn k,\dn l)^{\utw*}\right)^2.
$$

There are two increasing functions $\Psi''_S, \Psi'''_S:[0,+\infty)\rightarrow [0,+\infty)$ such that
$$
d_{\ma(\utw^* X)}(\dn k,\dn l)^{\utw*} \leq \Psi''_S\left(d_{\pa(\utw^* X)}(\dn k,\dn l)^{\utw*}\right) \leq \Psi'''_S\left( d_{\pa(X)}(k,l) \right).
$$
\end{claim}

With the notation $d_{\pa(X)}(\cdot,\cdot)$, this time, we measure pants distances along the sequence $\bm\omega^{*z}$; and with $d_{\pa(\utw^* X)}(\cdot,\cdot)^{\utw*}$, distances along $\utw^{*z}\bm\omega$.

The proof of this fact follows the proof of Proposition \ref{prp:locallyfinite}, with few modifications. As explained above, the intervals which index the sequences $\bm\omega^{*z}$ and $\utw^{*z}\bm\omega$ admit a subdivision with respect to the sequences of curves $(\gamma_t)_{t_1}^{zr}$, $(\utw^*\gamma_t)_{t=1}^{zr}$, respectively, similarly to arranged splitting sequences.

In Step 1 of the original proof several definition were given, which we repeat without substantial modifications here. The only thing that needs to be clarified is why the number $q$ of curves to be considered is bounded by the pants distance, since those curve may appear multiple times in the current setting.

For each $1\leq u\leq Q$ let $\langle u\rangle$ be a choice of $t$ such that $\gamma_t=\eta_u$ and, if possible, such that $DI^*_{\langle u\rangle}\subseteq [k,l]$. Consider $\bm\omega$ as $(\eta_1,\ldots,\eta_Q)$-arranged (possibly reindexing these curves). 
As a consequence of Proposition \ref{prp:tcbound}, then, the number of indices $u$ such that $\eta_u\subseteq X$ and $DI^*_{\langle u\rangle}\subseteq [k,l]$ is bounded from above by $C_3 d_{\pa(X)}(k,l)+ C_4$.

So, as we define $\delta_1=\gamma_{t_1},\ldots,\delta_q=\gamma_{t_q}$ to be the curves (here listed with repetitions allowed) such that $DI^*_{t_s}\cap [k,l] \not=\emptyset$ and $\gamma_t$ intersects $X$ essentially, we find that the ones such that $DI^*_{t_s}\subseteq [k,l]$ are disjoint from $\partial X$ (by claim 1 in Lemma \ref{lem:untwistedsubsurfaces} adapted to this setting). Therefore they are exactly the curves $\eta_u$ with $DI^*_{\langle u\rangle}\subset[k,l]$, each counted at most $N_0+1$ times. Moreover, it is only for $s=1,q$ one may have simultaneously $DI^*_{t_s} \setminus [k,l], DI^*_{t_s}\cap [k,l]\not=\emptyset$. So $q\leq c \cdot d_{\pa(X)}(k,l)$ for a suitable $c=c(S)$.

The same ideas may be applied to bound $q$ in terms of pants distance in $\utw^{*z}\bm\omega$. Under the correspondence $\utw^*$ given by Corollary \ref{cor:subsurface_bijection} in this modified setting, the indices $t_s$, $1 \leq s \leq q$, turn out to be almost exactly the values of $t$ such that $\utw^*\gamma_t$ intersects $\utw^* X$, $DI^{\utw*}_{t_s}\cap [\dn k,\dn l] \not=\emptyset$. We say `almost', because the only exception to this last sentence is that $\utw^*\gamma_{t_q}$ may have $DI^{\utw*}_{t_q}\cap [\dn k,\dn l] =\emptyset$. So it is also true that $q\leq c \cdot d_{\pa(\utw^* X)}(\dn k,\dn l)^{\utw*}$, if $c=c(S)$ is chosen suitably.

After these modifications, reconstructing Step 2 in the proof of Proposition \ref{prp:locallyfinite} is straightforward.

Following Step 3 of that proof, we wish to prove the existence of constants depending on $S$ such that, however one picks $\alpha\in\cc(X)$ with $\nei=\nei(\alpha)$ a regular neighbourhood, for all $0\leq s\leq q$, and all $j,j'\in XDL^*_s\cup XNI^*_s$, $d_\nei(\omega^{*z}_j|X,\omega^{*z}_{j'}|X)$ is bounded by this constant. Let $I^*_\nei$ be the accessible interval for $\nei$ in the sequence $\bm\omega^{*z}$.

For each $0\leq i\leq z$, let $[a(i),b(i)]=H_i\coloneqq I^*_\nei\cap [j,j']\cap [iN, (i+1)N]$. Step 2 of this proof implies immediately that there are at most $N_0+1$ values of $i$ such that $H_i\not=\emptyset$, and they are consecutive --- let $\Lambda$ be their family. Then one may write $[j,j']=J_-\cup \left(\bigcup_{i\in\Lambda}H_i\right)\cup J_+$ where $J_-, J_+$, if nonempty, are intervals sharing only one element with $I^*_\nei$.

As all $\bm\omega^i$ are effectively arranged, the original proof of Proposition \ref{prp:locallyfinite} provides bounds for $d_\nei(\omega^{*z}_{a(i)}|X,\omega^{*z}_{b(i)}|X)$, for all $i\in \Lambda$. For what concerns $J_-\eqqcolon[j,b]$, Theorem \ref{thm:mmsstructure} gives that $d_\nei(\omega^{*z}_{j},\omega^{*z}_{b})\leq \mathsf{K}_0$. Similarly for $J_+$. So we have a bound for the distance covered in each of the pieces in which we have split $[j,j']$, and these pieces are at most $N_0+3$: hence the existence of a constant, $c'$, bounding $d_\nei(\omega^{*z}_j|X,\omega^{*z}_{j'}|X)$ from above.

The remainder of Step 3, and Step 4, of Proposition \ref{prp:locallyfinite} may be applied here with no substantial modifications, thus completing the proof of our claim.

\step{5} The sufficient condition declared in Step 1 above holds.

Step 3 above acts in place of Proposition \ref{prp:locallyfinite} to prove the following version of Proposition \ref{prp:hardttbound}, using exactly the same line of proof.
\begin{claim}
There is a constant $C'_7(S)$ such that
$$d_{\pa(S)}(V(\omega^{*z}_0),V(\omega^{*z}_0)))\geq_{C'_7}|\utw^{*z}\bm\omega|.$$
\end{claim}

Finally, just note that $V(\omega^{*z}_0)=V(\rho_0)=V(\tau_m)$; $V(\omega^{*z}_0)=V(\tau_{m+zn})$;\linebreak $|\utw^{*z}\bm\omega|=z|\utw\bm\omega|= z|\utw(\rar\bm\rho)|$.
\end{proof}

\section{Braids in the solid torus}\label{sec:dw}

Among mapping tori, the ones coming from \emph{braids} in the way we are about to describe admit an application of the train track machinery different from the one given in the previous section. See \cite{farb}, \S 9.1 for details about braids and interpretations of braid groups.

Recall that the \nw{braid group} on $n$ strands,
\begin{equation}\label{eqn:braidgroup}
B_n\coloneqq \left\langle\sigma_1,\ldots,\sigma_{n-1}\left|
\begin{array}{lr}
\sigma_i\sigma_j=\sigma_j\sigma_i & \text{for }|i-j|\geq2;\\
\sigma_i\sigma_{i+1}\sigma_i=\sigma_{i+1}\sigma_i\sigma_{i+1} & \text{for }1\leq i\leq n-1
\end{array}
\right.\right\rangle
\end{equation}
has a natural identification with $\mcg(D^2_n)$, where with $D^2_n$ we mean the closed disk, punctured $n$ times. Here we identify $D^2_n$ with 
$$
\left\{z\in \R^2\left| \|z\|\leq 1\right.\right\}\setminus \left\{\left.\left(-1+\frac{2}{n+1}j,0\right) \right|j=1,\ldots,n\right\}.
$$
This model, in particular, places all punctures along the horizontal axis $\R\times\{0\}$; and, for $1\leq i\leq n-1$, the generator $\sigma_i$ of $B_n$ corresponds to a \emph{half-twist} in $D^2_n$, swapping the $i$-th and the $(i+1)$-th punctures counting from the left.

From now on, it will always be assumed that $n\geq 3$, which implies $\xi\left(\inte(D^2_n)\right)\geq 4$ i.e. $\inte(D^2_n)$ is a surface in the sense we have stuck with in all the previous work. For simplicity, we will identify $\R$ with $\R\times\{0\}$ and similarly for points and intervals in the two sets.

If $\psi\in B_n\cong \mcg(D^2_n)$ has a restriction to $\inte(D^2_n)$ which is pseudo-Anosov, then we know that the mapping torus $M\coloneqq \faktor{\inte(D^2_n)\times [0,1]}{\sim_\psi}$ is hyperbolic (see Theorem \ref{thm:mappingtorushyperbolic}). This mapping torus actually admits a (diffeomorphic) embedding in $\R^3$ as follows: if $w$ is any word representing $\psi$ in $B_n$, let $\ul{\ul w}$ be a braid representation of the word $w$, embedded in $\inte(D^2)\times [0,1]$ with the property that $\ul{\ul w}\cap \left(\inte(D^2)\times\{0,1\}\right)= \left\{\left.\left(-1+\frac{2}{n+1}j,0\right) \right|j=1,\ldots,n\right\}\times\{0,1\}$.

The identification of each point of $\inte(D^2)\times\{0\}$ with the corresponding point of $\inte(D^2)\times\{1\}$ (via the identity map $\inte(D^2)\times\{0\}\rightarrow \inte(D^2)\times\{1\}$) produces a solid torus $T\cong \inte(D^2)\times\mathbb S^1$, which admits an embedding $T\hookrightarrow \R^3$ (so we identify it with a chosen embedding). Through this quotient, $\ul{\ul w}$ projects to a \nw{closed braid} $\ul w$ in $\R^3$ which is fitted to $T$: each strand of $\ul{\ul w}$ will project to a path or loop which intersects the image of $\inte(D^2)\times\{t\}$ only once for each $0<t<1$. Our mapping torus $M$ is seen to be diffeomorphic to $T\setminus\ul w$.


\subsection{Strip decompositions}

We give here a construction very similar to the one introduced in \cite{dynnikovwiest}, even if our definitions will have a more `visual' flavour which will make it easier to relate them with train track splitting sequences (see Figure \ref{fig:stripdecomposition}).

\begin{figure}
\def\svgwidth{.65\textwidth}
\centering{\input{stripdecomp.pdf_tex}}
\caption{\label{fig:stripdecomposition}The basic construction of a strip decomposition, with a marker $O$ and a cutter $H$. A puncture of $D^2_n$ lies along the cutter. Strips are filled in light grey.}
\end{figure}

Let $p$ be (the union of all curves in) a pants decomposition on $D^2_n$; suppose that the curves of $p$ are realized in a way that intersects $\R$ transversely, and such that $p\cup\R$ bounds no bigons. Fix $O\in \R$, which we call a \nw{marker}; and let $H\coloneqq (-\infty,O]$ be the corresponding \nw{cutter}. Fix $\nei(\R)\coloneqq\left(\R \times (-\epsilon,\epsilon)\right)$, where $\epsilon>0$ is chosen so that $p\cap \R \times (-2\epsilon,2\epsilon)$ consists of a set of arcs joining the two opposite boundary components $\R\times\{-\epsilon\}$, $\R\times\{\epsilon\}$. We can suppose, up to isotopies, that these arcs are all vertical. Define also $\nei(H)\coloneqq H \times (-\epsilon, \epsilon)$.

The \nw{strip decomposition} $\beta(p,O)$ is then defined as follows.

Let $AC(p, H)$ be the set of all connected components of $p\setminus\nei(H)$. In general, $AC(p,O)$ will consist of arcs and loops: we subdivide $AC(p,O)=A(p,O)\sqcup C(p,O)$ accordingly. We say that two arcs $\alpha_1,\alpha_2\in A(p,O)$ are \nw{consecutive} if there is a closed region $R=R(\alpha_1,\alpha_2)\subseteq D^2_n$, diffeomorphic to a rectangle, such that $\alpha_1,\alpha_2$ are two opposite sides of $\partial R$, the other two sides are two intervals along $\partial\bar\nei(\R)$, and $\inte(R)\cap p=\emptyset$.

We say that two arcs in $A(p,O)$ are \nw{parallel} if they are in the same class under the equivalence relation generated by consecutiveness: an equivalence class for parallelism will be called a \nw{strip}. The \nw{width} of a strip is its size as a set. The strip decomposition $\beta(p,O)$ is then the disjoint union of $C(p,0)$ with the set of all strips in $A(p,O)$.

If $s\in \beta(p,O)$ is a strip, then 
$$
R(s)\coloneqq \left(\bigcup_{\alpha\in s}\alpha\right)\cup \left(\bigcup_{\substack{\alpha_1,\alpha_2\in s\\\text{consecutive}}}R(\alpha_1,\alpha_2)\right)
$$
is again diffeomorphic to a rectangle, and the elements of $s$ are then uniquely defined from $R(s)$ as the connected components of $R(s)\cap p$. With this in mind, we may confuse a strip $s$ with the corresponding $R(s)$.

In particular, each strip $s$ has two \nw{bases}, i.e. the two intervals $I_1,I_2\in \R$ such that $I_1\times\{\epsilon_1\}$ and $I_2\times\{\epsilon_2\}$ are the two connected components of $R(s)\cap \partial\bar\nei(H)$ for a suitable (unique) choice of $\epsilon_1,\epsilon_2\in\{\pm\epsilon\}$, and two \nw{ends}, i.e. the two connected components of $R(s)\cap (H\times\left(-2\epsilon,2\epsilon)\right)$. 

We say that two strip ends $e_1,e_2$, belonging to either the same strip in $\beta(p,O)$ or distinct ones, \nw{overlap} if the corresponding strip bases $I_1,I_2$ have $I_1\cap I_2\not=\emptyset$. The two ends are said to \nw{completely overlap} if $I_1=I_2$. If $e_1,e_2$ completely overlap and belong to the same strip $s$, then $s$ consists of a single arc which, together with a vertical arc in $\nei(H)$, makes up a curve in $p$: we cannot have more than one arc, else $p$ contains a pair of isotopic curves. In all other cases, the bases of $s$ can be distinguished into a \nw{left} and a \nw{right} one, according to the relative order of their minima in $\R$. The same terminology applies to the two ends of $s$.

A \nw{strip cut} is an elementary move on a strip decomposition, defined as follows.

There are exactly two strip ends $e_1,e_2$ which are `closest' to the marker $O$, i.e. such that the corresponding bases $I_1,I_2$ have $\max I_1=\max I_2$ and $p\cap\R \cap (\max I_1,O]=\emptyset$. Place the indices so that $I_1$ is shorter, or equal, to $I_2$. Let $O'\coloneqq \max \left((\R\cap p)\setminus I_1\right)$: then $O'<O$. We define the strip cut of $\beta(p,O)$ to be the strip decomposition $\beta(p,O')$. A \nw{strip cutting sequence} is a sequence of strip decompositions, each obtained from the previous with a strip cut: an example is given in Figure \ref{fig:stripcut}.

\begin{figure}
\includegraphics[width=\textwidth]{stripstt_involved.pdf}
\caption{\label{fig:stripcut}In the upper line is an example of strip cutting sequence obtained from the pants decomposition drawn in the leftmost picture. In the lower line are the train tracks obtained accordingly. Note how the cutter (the horizontal line) in the upper pictures gets shorter and shorter. (This figure has been derived from Figure 1 in \cite{dynnikovwiest}. Many thanks to Bert Wiest for having kindly agreed to its reuse.)}
\end{figure}

It is convenient to have a closer look at what happens with a strip cut. When $e_1,e_2$ belong to one same strip $s$, then necessarily $I_1=I_2$ and they must consist of a single point, else $p$ includes two isotopic curves --- similarly to what has been noted above. In this case, $\beta(p,O)$ differs from $\beta(p,O')$ only in that a strip in the former, consisting of a single arc, has been replaced with a loop.

When the strips $s_1,s_2$ to which $e_1,e_2$ belong are different, it may still be the case that $I_1=I_2$ i.e. $s_1,s_2$ contain the same number of arcs. Then the move's effect is that $\beta(p,O)\setminus\{s_1,s_2\}=\beta(p,O')\setminus\{s'\}$ for a strip $s'$ such that $R(s)=R(s_1)\cup R(s_2)\cup \left(I_1\times (-\epsilon,\epsilon)\right)$. We say that $s_1$ and $s_2$ \nw{merge} to $s'$.

If $I_1\subsetneq I_2$ instead, there are two (unique) strips $s'_1$, $s'_2$ such that $\beta(p,O)\setminus\{s_1,s_2\}=\beta(p,O')\setminus\{s'_1,s'_2\}$ and the following is true. Assign each arc $\alpha\in s_2$ to a family $s_{21}$ or $s_{22}$ according to whether $\alpha\cap e_2$ is contained in $I_1\times (-2\epsilon,2\epsilon)$ or not, respectively. Define $R(s_{21}),R(s_{22})$ exactly at it has been done above for a strip: they are again two rectangles. Then $R(s'_1)=R(s_1)\cup R(s_{21})\cup \left(I_1\times (-\epsilon,\epsilon)\right)$ while $s'_2=s_{22}$ --- and in particular $R(s'_2)=R(s_{22})$. We say that $s_1$ \nw{stretches} to $s'_1$ while $s_2$ \nw{shrinks} to $s'_2$.

%In a strip cutting sequence $\bm\beta=(\beta_j)_{j=0}^N$, a selection of strips $(s_j\in\beta_j)_{j=k}^l$ for $0\leq k\leq l\leq N$ is a \nw{line of descent} if, for all $k\leq j < l$, $s_j$ is equal, stretches, shrinks or merges with another strip to $\beta_{j+1}$.

%If $J=[k,l]\subseteq [0,N]$, for a strip cutting sequence $\bm\beta$ we adopt the notation $\bm\beta(k,l)=\bm\beta(I)=(\beta_k,\ldots,\beta_l)$ as done for splitting sequences.

%A \nw{transmission} in $\bm\beta$ is a subsequence $\bm\beta(k,l)$ if it features a line of descent $(s_j)_{j=k}^l$ such that either:
%\begin{itemize}
%\item $s_j$ shrinks to $s_{j+1}$ for all $k\leq j <l$;
%\item $s_j$ shrinks to $s_{j+1}$ for all $k\leq j <l-1$ and $s_{l-1}$ merges, with another strip, to $s_l$
%\end{itemize}
%and it is maximal i.e. there is no subsequence $\bm\beta(k',l')$, for $0\leq k'\leq k \leq l\leq l'\leq N$ and $l-k<l'-k'$, with the same property. This notion of transmission coincides with the one given in \cite{dynnikovwiest}.

Given a pants decomposition $p$, there is a \nw{canonical} strip cutting sequence $\bm\beta^p$ defined from $\beta(p,1)$ and performing strip cuts until it is no longer possible (i.e. there is no strip left).

\subsection{Strip decompositions turn into train tracks}

Strip cutting sequences are, morally, a particular case of train track splitting sequences. Given a strip decomposition $\beta=\beta(p,O)$, we define a semigeneric train track $\trk\beta$ as follows: let $G\subseteq AC(p,O)$ be a collection consisting of all elements of $C(p,O)$, and exactly one arc $a(s)$ for each strip $s\in\beta(p,O)$. If $e$ is a strip end, let $a(e)$ be the only endpoint of $a(s)$ which is contained in $e$. Let $E$ be the family of all pairs of overlapping strip ends. Finally, let $\Gamma\subset D^2_n$ be a $1$-complex obtained as the union of the elements of $G$, plus a straight segment $a(e_1,e_2)$ joining $a(e_1)$ to $a(e_2)$, for each pair $(e_1,e_2)\in E$. If two of these new segments intersect each other, or one intersects an element of $G$, then the point they share is an endpoint for both arcs. Finally, homotope $\Gamma$ to make sure that, for each strip $s$, each end $e$ of $s$, and each pair $(e,e')\in E$ for $e'$ another strip end, $a(s)\cup a(e,e')$ is smoothly embedded.

Call $\trk\beta$ the result of this operation: it is a pretrack. 

\begin{lemma}\label{lem:birecurrent}
If $\beta=\beta(p,O)$ is a strip decomposition, then $\trk\beta$ is a birecurrent train track.
\end{lemma}
\begin{proof}[Sketch]
We only give a sketch of the argument proving that $\trk\beta$ is a train track. One has to make sure that $S\setminus \nei_0(\trk\beta)$ includes no connected component which is a:
\begin{itemize}
\item disc with smooth boundary: the existence of one would imply that $p$ includes a homotopically trivial curve;
\item 1-punctured disc with smooth boundary: the existence of one would imply that $p$ includes a curve homotopic into a puncture;
\item monogon: the existence of one would imply that $p$ forms a bigon with $\R$, which was excluded at the beginning of the construction;
\item bigon: the existence of one would imply that $p$ has two distinct, isotopic components.
\end{itemize}

$\trk\beta$ is recurrent because each branch is traversed by a connected component of $p$. As for transverse recurrence, it suffices to exhibit a collection of curves in $\trk\beta$ such that, for each branch $\in\br(\trk\beta)$, there is a curve in the collection which can be put in dual position with respect to $\trk\beta$, intersecting $b$. Recall the notation $a(s), a(e_1,e_2)$ used above for smooth segments in $\trk\beta$ --- which are parts of branches, but not necessarily \emph{entire} ones.

\begin{itemize}
\item For each pair of consecutive punctures $\left(-1+\frac{2}{n+1}j,0\right), \left(-1+\frac{2}{n+1}(j+1),0\right)$, $0\leq j\leq n-1$, include in the collection the round curve encircling them. There is a realization of this curve, dual to $\trk\beta$, which intersect all segments $a(e_1,e_2)$ for $e_1$, $e_2$ strip ends located between the two punctures. And there is a different realization which will intersect all segments $a(s)$ for $s$ a strip with an end located between the two punctures. Both realizations intersect the elements of $C(p,O)$ which pass between the two given punctures.
\item Include the curve encircling all punctures but the leftmost one $\left(-1+\frac{2}{n+1},0\right)$. More precisely, realize it in a way that encircles all $D^2_n$ except for a small neighbourhood of $\partial D^2_n$ and of $\left[-1, -1+\frac{2}{n+1}+\epsilon\right]\times \{0\}$. Depending on what particular realization of the curve has been chosen, the curve will be dual to $\trk\beta$ and will intersect all segments $a(e_1,e_2)$ for $e_1$, $e_2$ strip ends located to the left of the leftmost puncture; or all segments $a(s)$ for $s$ a strip with an end in the same locations. Both realizations all the elements of $C(p,O)$ which pass to the left of the leftmost puncture.
\item Similarly, include the curve encircling all punctures but the rightmost one $\left(-1+\frac{2}{n+1}n,0\right)$.
\end{itemize}

Figure \ref{fig:dualcurves} gives a local picture for these curves. In order to show that these curves are actually in efficient position with respect to $\trk\beta$, one may apply an argument similar as the one applied above to show that $\trk\beta$ is a train track.
\end{proof}

\begin{figure}[h]
\centering
\includegraphics[width=.7\textwidth]{braidtransverserecurrent.pdf}
\caption{\label{fig:dualcurves}The curves showing transverse recurrence of $\trk\beta$ as in Lemma \ref{lem:birecurrent}. The picture to the left shows, dashed, a round curve encircling two consecutive punctures of $D^2_n$, and how it can be isotoped to be dual to $\trk\beta$, and intersect any of the branches located `between' the two punctures. The picture to the right shows that a similar property holds for the round curve encircling all punctures of $D^2_n$  but the leftmost one --- embedded in two different ways, as prescribed in the proof of that lemma.}
\end{figure}

\begin{figure}
\centering
\def\svgwidth{.8\textwidth}
\input{cutissplit.pdf_tex}
\caption{\label{fig:cutissplit}When $\beta$ strip cuts to $\beta'$ and the strip cut causes a strip $s_1$ to stretch and another one, $s_2$ to shrink, either $\trk\beta'$ is comb equivalent to $\trk\beta$ (line above), or there is a sequence of elementary moves, which are all comb/uncomb moves except for one split, turning $\trk\beta$ to $\trk\beta'$. The first case occurs when the left base of $s_2$ is entirely contained in the base of another strip end, and the same case otherwise. The split parity, in this latter case, is determined by the strip widths before and after the cut.}
\end{figure}

How do $\trk\beta$ and $\trk\beta'$ relate, when $\beta'$ is a strip decomposition obtained from $\beta$ with a split?

When the strip cut causes the replacement of a strip containing a single arc with a loop, or causes two strips to merge, $\trk\beta=\trk\beta'$. Otherwise, there is a strip $s_1$ which stretches and another one, $s_2$, which shrinks. Let $e_1,e_2$ be the right ends of the two. The strip cut reflects on $\trk\beta$ the following way: first of all, let $\kappa$ be a zipper in $\trk\beta$ such that $\kappa_P$ begins at $a(e_2)$ and runs along $a(s_2)$. Unzip $\trk\beta$ along $\kappa$: this gives a track which is comb equivalent to $\trk\beta$. Now, in order to get $\trk\beta'$, another elementary move is needed, which is either a comb or a split. Some more detail is given in Figure \ref{fig:cutissplit}.

This means that, for a strip cutting sequence $\bm\beta=(\beta_j)_{j=0}^N$, the corresponding sequence $(\trk\beta_j)_{j=0}^N$ may be completed to a splitting sequence by deleting any redundant entry and then possibly inserting, immediately after each $\trk\beta_j$, a train track with is obtained from it with an uncomb move. We call $\trk\bm\beta$ the splitting sequence thus obtained.

\begin{defin}
A pants decomposition $p$ of $D^2_n$ is \nw{round} if each connected component in the pants decomposition intersects $\R$ in exactly 2 points.
\end{defin}

\begin{coroll}\label{cor:dwgivesdistance}
There is a constant $A_1=A_1(n)$ such that, if $r$ is a round pants decomposition in $D^2_n$ and $\psi\in B_n\cong \mcg(D^2_n)$, then
$$d_{\pa(D^2_n)}\left(r,\psi(r)\right)=_{A_1}|\utw\left(\rar\left(\cnr(\trk\bm\beta^{\psi(r)})\right)\right)|.$$
\end{coroll}

Note that, in this statement, we are letting it be understood that $\cnr(\trk\bm\beta^{\psi(r)})$ must be turned into a generic splitting sequence in order for $\rar$ and $\utw$ to make sense (at least if we wish to rely on the approach of \S \ref{sec:twistcurves}, \ref{sec:traintrackconclusion}, which was meant for generic splitting sequences only).

\begin{proof}
For simplicity, let $\trk\bm\beta^{\psi(r)}\eqqcolon \bm\tau =(\tau_j)_{j=0}^N$. Theorem \ref{thm:main_full} gives
$$|\utw\left(\rar\left(\cnr\bm\tau\right)\right)|=_{C_9} d_{\pa(D^2_n)}\left(V(\tau_0),\psi(r)\right).$$
Note that the number of strips in $\beta\left(\psi(r),1\right)$ is bounded in terms of $n$ --- a strip $s$ is uniquely determined by whether it lies above or below $\R$, and by the punctures of $D^2_n$ the two ends of $s$ lie between. This implies that the number of distinct possibilities for $\tau_0$ is bounded in terms of $n$, so there is a bound for $d_{\pa(D^2_n)}\left(r,V(\tau_0)\right)$. This proves our claim.
\end{proof}

\begin{coroll}\label{cor:dwgivesvolume}
There is a constant $A_2=A_2(n)$ such that the following is true. Let $r$ be a round pants decomposition in $D^2_n$ and $\psi\in B_n\cong \mcg(D^2_n)$ such that $\psi$ defines a pseudo-Anosov mapping class on $\inte(D^2_n)$. Let $M\coloneqq \faktor{\inte(D^2_n)\times [0,1]}{\sim_\psi}$ be the related mapping torus. Then, defining $r(m)\coloneqq \psi^m(r)$,
$$
\vol(M) =_{A_2}\limsup_{m\rightarrow+\infty} \frac{1}{m} |\utw\left(\rar\left(\cnr(\trk\bm\beta^{r(m)})\right)\right)|
$$
and also
$$
\vol(M) =_{A_2} \min_{\phi\in \mathrm{Conj}(\psi)} |\utw\left(\rar\left(\cnr(\trk\bm\beta^{\phi(r)})\right)\right)|
$$
where $\mathrm{Conj}(\psi)$ is the conjugacy class of $\psi$ in $B_n$.
\end{coroll}
The same considerations as the ones after the statement of Corollary \ref{cor:dwgivesdistance} apply.
\begin{proof}
We use a simplified notation $s(\nu)\coloneqq |\utw\left(\rar\left(\cnr(\trk\bm\beta^{\nu(r)})\right)\right)|$ for $\nu\in\mcg(D^2_n)$.

As a consequence of Corollary \ref{cor:dwgivesdistance} plus Lemma \ref{lem:pantsquasiisom}, there is a constant $A'_1=A'_1(n)$ such that, for each $m>0$, one has $d_{\pc(S)}\left(r,r(m)\right)=_{A'_1}s(\psi^m)$.

So $|\psi|^{st}=_{A'_1} \limsup_{m\rightarrow+\infty} \frac{1}{m} s(\psi^m)$ and, via Remark \ref{rmk:stable_dist_pc}, there is a further constant $A''_1=A''_1(n)$ such that $|\psi|=_{A''_1} \limsup_{m\rightarrow+\infty} \frac{1}{m} s(\psi^m)$.

Moreover, let $p$ be a pants decomposition such that $d_{\pc(S)}\left(p,\psi(p)\right)=|\psi|$. There is a $\lambda\in \mcg(D^2_n)$ such that $\lambda(p)$ is round. The pants decompositions $\lambda(p),r$ both belong to the finite set in $\pc(D^2_n)$ consisting of round pants decompositions, so there is an upper bound $d_{\pc(S)}\left(\psi(p),\psi\circ \lambda^{-1}(r)\right)=d_{\pc(S)}\left(p,\lambda^{-1}(r)\right)=d_{\pc(S)}\left(\lambda(p),r\right)\leq c$ depending only on the number of strands $n$.

Thus $d_{\pc(S)}\left(r,\lambda\circ \psi \circ \lambda^{-1} (r)\right)= d_{\pc(S)}\left(\lambda^{-1}(r), \psi \circ \lambda^{-1} (r)\right)\leq d_{\pc(S)}\left(\lambda^{-1}(r), p\right) + \linebreak d_{\pc(S)}\left(p, \psi(p)\right) + d_{\pc(S)}\left(\psi(p), \psi \circ \lambda^{-1} (r)\right) \leq |\psi|+2c$, while $d_{\pc(S)}\left(r,\lambda\circ \psi \circ \lambda^{-1} (r)\right) =_{A'_1}
s(\lambda\circ \psi \circ \lambda^{-1})$ from Corollary \ref{cor:dwgivesdistance} above.

On the other hand, $|\psi|\leq d_{\pc(S)}\left(\nu^{-1}(r), \psi \circ \nu^{-1} (r)\right)$ however $\nu\in \mcg(D^2_n)$ is chosen. Combining the two bounds, $|\psi|=_{A''_1} \min_{\phi\in \mathrm{Conj}(\psi)} s(\phi)$. Here we choose, for simplicity, an $A''_1=A''_1(n)$ such that both this coarse equality and the one proved above hold.

Theorem \ref{thm:brockmappingtori} concludes.
\end{proof}

\subsection{Bounds without train tracks (sketch)}

The splitting sequences arising as $\trk\bm\beta$ have several good properties, which we state in Proposition \ref{prp:dwcutnumber}. Due to these properties, it is not really necessary to switch to the train track formalism and employ a derived splitting sequence such as\linebreak $|\utw\left(\rar\left(\cnr(\trk\bm\beta^{\psi(r)})\right)\right)|$ to have Corollaries \ref{cor:dwgivesdistance} and \ref{cor:dwgivesvolume}. As a complete proof of Proposition \ref{prp:dwcutnumber} would go partly beyond the scope of this thesis, we only outline it.

We need a definition. Let a strip decomposition $\beta(p,M)$ be given, with the property that the strip cut on this decomposition will result in a strip $s_1$ to stretch, and another one $s_2$ to shrink to a strip $s'_2$. Let $e$ be the right end of $s_2$, $e'$ be the left one. If the two strip ends of $s_2$ partially overlap and so do the ones of $s'_2$, then the move is a \nw{spiralling} of $s_2$. Let $I$ be the left base of $s_2$, and let $M'\coloneqq \max I$. Successive strip cuts will turn $\beta(p,M)$ into $\beta(p,M')$, and generate a strip cutting sequence $\bm\varsigma$ with a sequence $(s_2^{(j)})_{j=1}^k$ of strips, one for each decomposition in $\bm\varsigma$, such that $s_2^{(j)}$ shrinks to $s_2^{(j+1)}$ for all $1\leq j<k$. If each cut in $\bm\varsigma$ is a spiralling of the respective $s_2^{(j)}$ --- i.e. if the ends of $s_2^{(k)}$ still overlap --- then we say that $\bm\varsigma$ is a \nw{complete spiralling} of $s_2$. See Figure \ref{fig:complete_spiralling}.

Let $\gamma$ be the only curve of $V(\trk\beta(p,M))$ which traverses only $a(s_2)$ and $a(e,e')$, both exactly once. Then $\gamma$ is a twist curve for $\trk\beta(p,M)$. A spiralling move reflects into an elementary move which is twist\footnote{We have not defined what this means in a semigeneric setting. We may say that an elementary move on a track $\tau$ with respect to a twist curve $\gamma$ is twist if, given $A_\gamma$ a twist collar, the move is the result of unzipping along a zipper $\kappa:[-\epsilon, t]\rightarrow\bar\nei(\tau)$, which does not intersect $\tau.\gamma$ and such that $\kappa([0,t])\subseteq A_\gamma$. Also, we may define the sign of $\gamma$ as the sign $\gamma$ has in any generic track which is comb equivalent to $\tau$.} with respect to $\gamma$, and if $\bm\varsigma$ is a complete spiralling turning $\beta(p,M)$ into $\beta(p,M')$, then $\trk\beta(p,M')=D_\gamma^\epsilon(\trk\beta(p,M))$, with $\epsilon$ the sign of $\gamma$ as a twist curve. We call $\gamma$ the \nw{spiralled curve}.

Given a strip cutting sequence $\bm\beta$, the subsequence $\bm\beta(k,l)$ is a \nw{maximal spiralling} if there is a sequence $(s_2^{(j)})_{j=k}^l$ of strips, one for each decomposition in $\bm\beta(k,l)$, such that $s_2^{(j)}$ shrinks to $s_2^{(j+1)}$ for all $k\leq j<l$, each strip cut in $\beta(k,l)$ is a spiralling of the respective $s_2^{(j)}$, and $\bm\beta(k,l)$ is not contained in any longer subsequence of $\bm\beta$ with the same properties. It may not be possible to subdivide a maximal spiralling into complete spirallings, but note that $\trk\left(\bm\beta(k,l)\right)$ has twist nature about one same spiralled curve $\gamma$.

\begin{figure}
\def\svgwidth{.5\textwidth}\centering
\input{complete_spiralling.pdf_tex}
\caption{\label{fig:complete_spiralling}The successive cuts occurring in a complete spiralling.}
\end{figure}

We will sketch the proof of the following Proposition.
\begin{prop}\label{prp:dwcutnumber}
Let $p$ be a pants decomposition of $D^2_n$, and let $(\tau_j)_{j=0}^N=\bm\tau\coloneqq \trk\bm\beta^p$: in particular, $V(\tau_N)=p$. Then there is a constant $A_3=A_3(S)$ such that
$$d_{\pa(D^2_n)}(\cnr\tau_0,\cnr\tau_N),\ d_{\pa(D^2_n)}(\tau_0,\tau_N),\ d_{\pc}(r,p)=_{A_3}\|\bm\beta^p\|,$$
where $\|\bm\beta^p\|$ is the number of strip cuts in the sequence $\bm\beta^p$, renormalized so that each maximal spiralling is counted as only $1$; and $r$ is any round pants decomposition of $D^2_n$.
\end{prop}
A restatement of Corollary \ref{cor:dwgivesvolume} follows from the above Proposition:
\begin{coroll}\label{cor:dwgivesvolume_cutnumber}
There is a constant $A_4=A_4(n)$ such that the following is true. Let $r$ be a round pants decomposition in $D^2_n$ and $\psi\in B_n\cong \mcg(D^2_n)$ be such that $\psi$ defines a pseudo-Anosov mapping class on $\inte(D^2_n)$. Let $M\coloneqq \faktor{\inte(D^2_n)\times [0,1]}{\sim_\psi}$ be the related mapping torus. Then, defining $r(m)\coloneqq \psi^m(r)$,
$$
\vol(M) =_{A_4}\limsup_{m\rightarrow+\infty} \frac{1}{m} \|\bm\beta^{r(m)})\|
$$
and also
$$
\vol(M) =_{A_4} \min_{\phi\in \mathrm{Conj}(\psi)} \|\bm\beta^{\phi(r)})\|
$$
where $\mathrm{Conj}(\psi)$ is the conjugacy class of $\psi$ in $B_n$.
\end{coroll}

The important property of spiralling moves is that they encode almost all elementary moves affecting twist curves:

\begin{lemma}\label{lem:nonspiralbound}
Let $\bm\beta$ be a strip cutting sequence $D^2_n$, and let $\bm\tau\coloneqq \trk\bm\beta$. Let $\gamma\in\cc(D^2_n)$ be a twist curve at some stage along the splitting sequence $\bm\tau$; more precisely let $I_\gamma$ be the accessible interval of $\nei(\gamma)$, and suppose that $k,l$ are the indices such that $\bm\tau(I_\gamma)=\trk\left(\bm\beta(k,l)\right)$.

Then there is a number $A_5=A_5(n)\geq 3$ such that, if more than $A_5$ strip cuts in $\bm\beta(k,l)$ reflect into moves on a train track level which are \emph{not} far from $\gamma$, then the ones after the $A_5$-th are all spirallings with $\gamma$ their spiralled curve (in particular they all occur consecutively).
\end{lemma}

The proof of this lemma is based on the fact that, every time a strip is cut, at least one branch end of the respective $\tau_j$ disappears, or is replaced with a new one located to the left of the old one. Whilst $j$ increases within the interval $[k,l]$, and branch ends get concentrated to the left hand side of $D^2_n$, the number of strips $s$ such that $a(s)\subseteq \tau_j.\gamma$ decreases. When there is only one such $s$, $\gamma$ is certainly a combed curve in the respective $\tau_j$. And, provided that $\gamma$ stays carried, it takes a bounded number of cuts, affecting the respective $\tau_j.\gamma$'s, before we reach that stage.

Some further properties hold:
\begin{itemize}
\item After a maximal spiralling with $\gamma$ its spiralled curve, the next strip cut reflects into $\gamma$ not being carried any longer by the new train track;
\item A spiralling cannot change the subsurface filled by the set of vertex cycles of the related train tracks.
\item Let $\bm\tau=\trk\bm\beta$ for $\bm\beta$ a strip cutting sequence. Then there is a number $A_6$ such that, if a curve $\gamma\in\cc(D^2_n)$ and two indices $j,j'$ are such that $d_{\nei(\gamma)}(\tau_j,\tau_{j'})\geq A_6$, then not only $\gamma$ needs become a twist curve at some stage between $j$ and $j'$, but it undergoes elementary moves which are produced by spiralling.
\item If $\trk\bm\beta(k,l)$ consists of $m>1$ complete spiralling with $\gamma$ the spiralled curve, and $\bm\sigma(k',l')=\cnr\left(\trk\bm\beta(k,l)\right)$, then $\sigma_{l'}=D_\gamma^{\epsilon (m-1)}\sigma_i$ for an index $k'\leq i<l'$ such that $\gamma$ is a twist curve in $\sigma_i$.
\end{itemize}

All this means that:
\begin{lemma}
There is a constant $A_7=A_7(n)$ such that the following is true.

Let $\bm\beta$ be a strip cutting sequence in $D^2_n$, let $(\tau_j)_{j=0}^N=\bm\tau\coloneqq \cnr\left(\trk\bm\beta\right)$, and let $\gamma\in\cc(D^2_n)$ be a curve such that two indices $0\leq k<l\leq N$ exist with $d_{\nei(\gamma)}(\tau_k,\tau_l)\geq A_7$. Then there is an interval $DI_\gamma=[p,q]\subseteq [0,N]$ such that $\gamma$ is a twist curve for all $\tau_j$, $j\in DI_\gamma$, with some sign $\epsilon$, $\tau_q=D_\gamma^{\epsilon m}(\tau_p)$ for some $m\geq 2\mathsf K_0+4$, and the moves in $\bm\tau(DI_\gamma)$ all derive from spirallings where $\gamma$ is the spiralled curve.

Moreover, if $[j,j']\subset [0,N]$ intersects $DI_\gamma$ at an endpoint at most, then\linebreak $d_{\nei(\gamma)}(\tau_j,\tau_{j'})\leq A_6$.

A similar property is true for $\trk\bm\beta$ instead of $\bm\tau$.
\end{lemma}

This means, in particular, that \emph{any subsequence of $\bm\tau$ which evolves firmly in some (not necessarily connected) subsurface of $D^2_n$ is effectively arranged} (see Definition \ref{def:arranged}), possibly changing the constants involved in the definition of effectively arranged sequence.

Subdivide $\bm\tau= \bm\tau^1*\bm\epsilon^2*\bm\tau^2*\ldots*\bm\epsilon^w*\bm\tau^w$ as seen in \S \ref{sub:untwistedsequence}: one proves, with no substantial changes from \S \ref{sub:twistcurvebound} and \S \ref{sec:traintrackconclusion}, that
\begin{claim}
There is a constants $A_8=A_8(n)$, such that the following is true.

Let $p$ be a pants decomposition on $D^2_n$, and let $r$ be a round pants decomposition. Let $(\tau_j)_{j=0}^N=\bm\tau\coloneqq\cnr(\trk\bm\beta^p)$. Then
$$
d_{\pc}(r,p),\ d_{\pa}(r,p),\ d_{\pa}(\tau_0,\tau_N)=_{A_8} \left|\utw\bm\tau\right|.
$$

Here, $\utw$ is defined piece-by-piece as in Definition \ref{def:not_firmly}.
\end{claim}

Note that counting the number of splits in $\utw\bm\tau$ gives roughly the same number as counting their number in $\bm\tau$, but assigning a fixed weight to each interval $DI_\gamma$, \emph{no matter how many splits it includes}. And if one counts splits in this latter way in $\trk\bm\beta$ obtains again the same number, roughly (see Lemma \ref{lem:ctauproperties} and the last bullet above).

The only passage left to prove that this last number is roughly $\|\bm\beta^p\|$ and complete the proof of Proposition \ref{prp:dwcutnumber} is the following Lemma:

\begin{lemma}
There is a bound $A_9(n)$ such that, if $\bm\beta$ is a strip cutting sequence on $D^2_n$, and $\trk\left(\bm\beta(k,l)\right)$ consists of comb equivalences only, then $l-k\leq A_9$.
\end{lemma}

As a conclusion to this work, we note that Corollary \ref{cor:dwgivesvolume_cutnumber} implies a closed formula for hyperbolic volume, of which David Futer has an independent proof, unpublished at the time of writing. A set of generators for the $B_n$, larger than the standard one given in the presentation (\ref{eqn:braidgroup}), is given by
$$\bm\Delta\coloneqq \{\Delta_{ij}\coloneqq (\sigma_i\cdot\cdots\cdot\sigma_{j-1})(\sigma_i\cdot\cdots\cdot\sigma_{j-2})\ldots\sigma_i|0\leq i<j\leq n\}$$
which, rather than representing a half-twist switching two consecutive punctures of $D^2_n$, give a half twist reversing the position of all punctures from the $i$-th to the $j$-th. These generators are the ones used in \cite{dynnikovwiest}.

For $\psi\in B_n$, denote
$$g_\Delta(\psi) \coloneqq \min \{l|\exists\ a_1,\ldots,a_l \mbox{ so that } \psi=\delta_1^{a_1}\ldots\delta_l^{a_l}\mbox{ for some choice of }\delta_k\in \bm\Delta\}.$$

\begin{prop}\label{prp:futervolume}
There is a constant $A=A(n)$ such that the following is true. Let $\psi\in B_n\cong \mcg(D^2_n)$ be such that $\psi$ defines a pseudo-Anosov mapping class on $\inte(D^2_n)$. Let $M\coloneqq \faktor{\inte(D^2_n)\times [0,1]}{\sim_\psi}$ be the related mapping torus. Then
$$
\vol(M) =_{A}\limsup_{m\rightarrow+\infty} \frac{1}{m} g_\Delta(\psi^m)
$$
and
$$
\vol(M) =_{A} \min_{\phi\in \mathrm{Conj}(\psi)} g_\Delta(\phi)
$$
where $\mathrm{Conj}(\psi)$ is the conjugacy class of $\psi$ in $B_n$.
\end{prop}

Given Corollary \ref{cor:dwgivesvolume_cutnumber}, one produces the $\geq_A$ part of these two statements by constructing a braid word $w=\delta_1^{a_1}\ldots\delta_l^{a_l}$ for $\psi^m$ (resp. for $\phi\in \mathrm{Conj}(\psi)$), with $\|\bm\beta^{r(m)}\|$ (resp. $\|\bm\beta^{\phi(r)}\|$) $\geq l/3-(2n+1)$. More precisely, one constructs $w^{-1}$ adapting the process described in \cite{dynnikovwiest}, \S 2.4 to \emph{relax} the strips of the strip decomposition. Given the strip cutting sequence $\bm\beta^{r(m)}$ (resp. $\bm\beta^{\phi(r)}$) for $r$ a round pants decomposition, the process describes how to apply, to each entry $\beta_j$ of the sequence, an element of $\mcg(D^2_n)$ which will turn it into a strip decomposition $\beta'_j$ for a different pants decomposition, with the property that if $\alpha\in \beta'_j$ is a loop then $\#(\alpha\cap\R)\leq 2$ while, if $s\in \beta'_j$ is a strip, then $R(s)\cap\R$ is empty or connected. The diffeomorphism $\lambda$ to be applied to the last entry of $\bm\beta^{r(m)}$ (resp. of $\bm\beta^{\phi(r)}$), which it just $r(m)$ (resp. $\phi(r)$), will make it round. This does not mean that $\lambda=\psi^{-m}$ (resp. that $\lambda=\phi^{-1}$), but note that the process gives a word $w'$ for $\lambda$ having $w=\delta_1^{a_1}\ldots\delta_{l'}^{a_{l'}}$ for $l' \leq 3\|\bm\beta^{r(m)}\|$ (resp. $3\|\bm\beta^{\phi(r)}\|$). Moreover, it is easy to realize that there is a further $\nu\in\mcg(D^2_n)$, with $g_\Delta(\nu)\leq 2n+1$, such that $\nu\circ\lambda=\psi^{-m}$ (resp. $\nu\circ\lambda=\phi^{-1}$).

The $\leq_A$ inequality does not depend on Corollary \ref{cor:dwgivesvolume_cutnumber}, but rather on a Gromov norm argument (see \cite{thurstonnotes}, \S 6 and in particular \S 6.5), based on a construction very similar to the one in \cite{lackenby}, \S 2. Given any $\phi\in\mathrm{Conj}(\psi)$, one has $\faktor{\inte(D^2_n)\times [0,1]}{\sim_\psi}\cong \faktor{\inte(D^2_n)\times [0,1]}{\sim_\phi}$; while $\faktor{\inte(D^2_n)\times [0,1]}{\sim_{\psi^m}}$ is an $m$-fold cyclic cover of $\faktor{\inte(D^2_n)\times [0,1]}{\sim_\psi}$, hence the ratio between the respective volumes is $m$. So one may prove that, if $\faktor{\inte(D^2_n)\times [0,1]}{\sim_\nu}$ for $\nu$ pseudo-Anosov, then $\vol(M) \leq_{A} g_\Delta(\nu)$. Both statements of the Corollary will follow, varying $\nu$ suitably.

Let $w=\delta_1^{a_1}\ldots \delta_l^{a_l}$ be a word in $B_n$ which realizes $g_\Delta(\nu)$. Note that $M\cong T\setminus\ul w$ is also diffeomorphic to $\mathbb S^3\setminus (\ul w\cup L_0)$, for $L_0$ any meridian circle of $\partial\bar T$. One may regard $\ul w$ as subdivided into a number of braids, each corresponding to a factor $\delta_j^{a_j}$ ($1\leq j\leq l$): encircle each of these braids with a further \emph{augmenting} loop $L_j$, similarly to what is done in \cite{lackenby}.

Given a disc $D_j$ bounded by $L_j$ in $\mathbb S^3$, cut $\mathbb S^3\setminus (\ul w\cup L_0\cup\ldots \cup L_l)$ along $D_j$, twist one of the two copies of $D_j$ by a multiple of $2\pi$ and attach it back onto the other: this is a homeomorphism. This property gives, in particular, $\mathbb S^3\setminus (\ul w\cup L_0\cup\ldots \cup L_l)\cong \mathbb S^3\setminus (\ul{w'}\cup L_0\cup\ldots \cup L_l)$, where $w'=\delta_1^{\epsilon_1}\ldots \delta_l^{\epsilon_l}$ with each $\epsilon_j\in\{0,1\}$.

When $W\coloneqq \ul{w'}\cup L_0\cup\ldots \cup L_l$ is isotoped close to the equator of $\mathbb S^3$ to give a link diagram, the decomposition of $\mathbb S^3\setminus W$ into two 3-cells with ideal vertices, as shown in \cite{benedetti}, \S E.5-iv, may be refined, without adding any new vertices or ideal vertices, into a triangulation of this manifold, as it is done in \cite{lackenby}: the number of triangles employed is linear in $l$, therefore the Gromov norm of $\mathbb S^3\setminus W$ is also at most linear in $l$. But then so is the Gromov norm of $\mathbb S^3\setminus (\ul w \cup L_0)$, which is obtained from the former manifold by Dehn surgery (this is immediately implied by Proposition 6.5.2, Corollary 6.5.3 and Lemma 6.5.4 of \cite{thurstonnotes}): and the latter is proportional to $\vol\left(\mathbb S^3\setminus (\ul w \cup L_0)\right)$. This ends the sketch of proof.