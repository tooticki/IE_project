We give a method of computing distances between certain points in the pants graph of a surface $S$, up to multiplicative and additive constants. More precisely, we consider splitting sequences of train tracks on $S$, such that the vertex set of each track in the sequence subdivides $S$ into pieces which are pairs of pants, or simpler than that. It is possible to regard the sequence given by the vertex set of each track as a path along the edges of a graph, which is naturally quasi-isometric to the pants graph of $S$: and we show how to estimate the distance between two points along this path.

The present work is inspired by a result of Masur, Mosher and Schleimer according to which, if the vertex sets along a splitting sequence fill $S$, then they give a quasi-geodesic path in the marking graph; and the distance between the extremes of this path is given, up to constants, by the number of \emph{splits} occurring in the sequence.

However, their result cannot hold for the pants graph: it may well be that a high number of splits in the splitting sequence make the vertex sets span a high distance in some annular projection; and, despite this, these sets cover no similarly high distances in the pants graph.

We work to treat this discrepancy: we describe a machinery that, given a train track splitting sequence, produces first a new one where the moves only contributing to annular distance are grouped altogether; and then a further one, the \emph{untwisted sequence}. This latter sequence resembles the former, but the distance it spans in any annular subsurface projection is controlled by the pants graph distance. After these constructions, we prove a distance formula by showing that the untwisted sequence is suitable for application of the same arguments conceived by Masur, Mosher and Schleimer.

Thanks to a result of J.F.~Brock, our distance estimates in the pants graph reflect into hyperbolic volume estimates for pseudo-Anosov mapping tori. We give a couple of results in this area: the first one uses I.~Agol's \emph{maximal splitting sequence}, the second one revisits I.~Dynnikov and B.~Wiest's \emph{interval identifications systems} and their \emph{transmission} to give an estimate of the hyperbolic volume for a solid torus minus a closed braid. We also sketch how one may regard this latter result as independent of the train track machinery.