\chapter{Background}\label{cha:background}

\section{Essentials on surfaces and 3-manifolds}

\subsection{Surfaces}\label{sub:surfaces}
These lines are conceived to fix once and for all the most basic objects treated in this work. When the term is used with no specification, a \nw{surface} $S$ is an oriented, connected, differentiable 2-manifold, possibly non-compact, without boundary unless otherwise specified. Even when any of these implicit specifications is not met (and in that case we will always make it explicit), the 2-manifolds we deal with are always of \nw{finite type}, i.e. homeomorphic to a \emph{compact} 2-manifold (possibly) with boundary, (possibly) with finitely many points removed (from its interior, if it has any boundary). For a surface $S$ with no extra specifications, we also require that the \nw{complexity} $\xi(S)\coloneqq 3(\text{genus}-1)+ (\#\text{punctures})$ is $\geq 1$.

\begin{defin} \label{def:hyperbolic}
Let $M$ be an orientable, connected, finite-type, smooth $n$-manifold ($n\geq 2$), with no boundary. A \nw{hyperbolic structure} on $M$ is an atlas of charts $\phi_i:U_i\rightarrow \Hy^n$, where each $U_i\subseteq M$ is an open subset, such that each change of chart $\phi_j\circ\phi_i^{-1}:\phi_i(U_i\cap U_j)\rightarrow \phi_j(U_i\cap U_j)$ is the restriction of an element of $\Isom^+(\Hy^n)$. This atlas is also required to cover the whole $M$, and to be maximal.
\end{defin}

Any surface will be understood to be endowed with a complete hyperbolic metric, \emph{not necessarily with finite area}, which makes it isometric to the quotient $\Hy^2/\Gamma$ for a subgroup $\Gamma<\mathrm{Isom}^+(\Hy^2)$, $\Gamma\cong \pi_1(S)$, acting freely and properly discontinuously. A \emph{puncture}, in this work, shall be considered as a purely topological concept, unrelated with the hyperbolic metric. Given a surface $S$ and one of its punctures, we say that $S$ has a \nw{cusp} there if the puncture has a neighbourhood with finite area. Note that the area of $S$ is infinite if and only if $S$ has a puncture which is not a cusp: then we say that $S$ has a \nw{funnel} at that puncture.

By \nw{curve} on a surface $S$ we mean an embedding ${\mathbb S}^1\hookrightarrow S$ which, unless otherwise specified, is defined \emph{only up to isotopies}. A curve is \nw{essential} if it can be homotoped neither into a disc, nor into a peripheral annulus. A \nw{multicurve} is a collection of pairwise disjoint and non-isotopic essential curves: it is a well-known fact that a multicurve comprises at most $\xi(S)$ different curves.

Given any two essential curves $\alpha_1,\alpha_2$ on a surface $S$, their \nw{intersection number} $i(\alpha_1,\alpha_2)$ is the minimum number of intersection points between $\alpha_1,\alpha_2$ attained when deforming both curves within their respective isotopy classes --- if $\alpha_1,\alpha_2$ are isotopic, then $i(\alpha_1,\alpha_2)=0$. If $A_1,A_2$ are two finite collections of essential curves on $S$, then $i(A_1,A_2)\coloneqq \sum_{\alpha_1\in A_1,\alpha_2\in A_2} i (\alpha_1,\alpha_2)$.

Given $S$ a surface, a \nw{subsurface} $X$ is either the entire $S$ or a connected 2-submanifold \emph{with boundary}, of finite type, with the following requests:
\begin{itemize}
\item each puncture of $X$ is also a puncture of $S$;
\item $\partial X$ consists of a collection of components of $\partial S$ and smooth essential curves in $S$;
\item $\mathrm{int}(X)$ is not homemorphic to a pair of pants $S_{0,3}$.
\end{itemize}
Note that this definition of subsurface includes closed annuli and complements of open annuli in $S$, as their boundary contains a pair of isotopic curves in $S$. In general, we allow two components of $\partial X$ to be isotopic curves (but, even in this case, we assume that distinct components have distinct realizations in $S$).

Subsurfaces, similarly as curves, are to be considered up to isotopies in $S$. In a few occasion we will drop the connectedness condition, and in those cases we make this explicit.

In sections 8.1--8.3, \cite{thurstonnotes}, the \emph{limit set} $L_\Gamma\subseteq \partial\ol{\Hy^2}$ of $\Gamma$ is defined; together with the \emph{domain of discontinuity} $D_\Gamma\coloneqq \partial\ol{\Hy^2}\setminus L_\Gamma$. Define the \nw{convex core} of $S$ as $\core(S)\coloneqq H(L_\Gamma)/\Gamma$, where $H(L_\Gamma)$ is the convex hull of $L_\Gamma$ in $\Hy^2$. It is a hyperbolic surface with totally geodesic boundary, and finite area. Also, define the \nw{funnel compactification} of $S$ as the quotient $\bar S\coloneqq (\Hy^2\cup D_\Gamma)/\Gamma$. Since it is shown that the action of $\Gamma$ on $\Hy^2\cup D_\Gamma$ is free and properly discontinuous (Proposition 8.2.3 in the cited work), $\bar S$ is a surface, with boundary unless $\bar S=S$. Note that $\bar S$ may still have punctures (the cusps of $S$), so it is \emph{not} compact in general.

There is a natural diffeomorphism $r:\bar S\rightarrow \core(S)$, isotopic to $\mathrm{id}_{\bar S}$. It restricts to a diffeomorphism $S\rightarrow \inte\left(\core(S)\right)$. Both $\bar S,\core(S)$ are not compact in general because they add a circular boundary to a given puncture of $S$ only if all its neighbourhoods have infinite area --- or, equivalently, if $S$ has a closed geodesic encircling the puncture; again equivalently, if and only if the puncture cannot be identified with the quotient of a point in $L_\Gamma$ (a parabolic one).

A \nw{peripheral annulus} of $S$ is a 2-submanifold diffeomorphic to $\mathbb S^1\times [0,1)$, bounded by an inessential curve, which serves as a closed neighbourhood for a puncture of $S$. It is \emph{not} a subsurface of $S$.

The definition of $\bar S$ depends, even topologically, on the hyperbolic metric on $S$. The \nw{compactification} of $S$, instead, is taken to be a compact surface with boundary $S_\bullet \coloneqq S\setminus \inte(P)$, where $P$ is a collection of disjoint peripheral annuli, one for each topological puncture of $S$. In particular, when a topological puncture is a funnel for the hyperbolic metric, one can choose a peripheral annulus bounded by an inessential closed geodesic: with this choice we get $\partial\core(S)\subseteq \partial S_\bullet$. So, $S_\bullet$ is a compact surface with boundary where `each puncture of $S$ is turned into a boundary component'. It is possible to identify $r_\bullet:S\rightarrow \inte(S_\bullet)$ via a diffeomorphism, homotopic to $\mathrm{id}_S$. No preferred metric is to be considered on $S_\bullet$.

A \nw{(properly embedded) arc} on $S$ is a smooth embedding $\rho:[0,1]\hookrightarrow S_\bullet$, with $\rho^{-1}(\partial S_\bullet)=\{0,1\}$; an arc is \nw{essential} if no homotopy relative to $\partial S_\bullet$ turns it into a point, or a path contained in $\partial S_\bullet$. We identify arcs with their restriction in $S$; more precisely, we identify $\rho$ with $r_\bullet^{-1}\circ \rho|_{(0,1)}:(0,1)\rightarrow S$. Similarly as curves, arcs are usually considered to be defined up to isotopies relative to $\partial S_\bullet$. A collection of pairwise disjoint and non-isotopic (rel. $\partial S_\bullet$) arcs in $S_\bullet$ (which is the same as saying, in $S$), consists of at most $3|\chi(S)|$ arcs (see Remark 1.1 in \cite{przytycki}).

With an abuse of notation, sometimes subsurfaces $X$ will be identified with the corresponding $\inte(X)$. A distinction between the two concepts will be done only when it is relevant and not self-evident.

Note that, for $X$ a subsurface of $S$, the natural map $\pi_1(X)\rightarrow \pi_1(S)$ is an injection. If $S\cong \Hy^2/\Gamma$, let $\Gamma_X<\Gamma$ be the subgroup which is identified with $\pi_1(X)$ under the isomorphism $\pi_1(S)\cong\Gamma$, and denote $S^X\coloneqq \Hy^2/\Gamma_X$, which is clearly a covering space for $S$: it is a surface if $X$ is non-annular, and diffeomorphic to $\R\times\mathbb S^1$ otherwise.

There is a natural, isometric inclusion $X\hookrightarrow S^X$, which will be enforced to consider $X$ both as a subsurface of $S$ and of $S^X$. If $\partial X\subseteq \core(S)$, then the subsurfaces $\core(S^X)$ and $X\cap\core(S)$ are isotopic in $S^X$. If $X$ has geodesic boundary (and still consisting of distinct curves), then the two actually coincide.

\subsection{Hyperbolic 3-manifolds}\label{sec:hyp3mflds}

By \nw{3-manifold} in this work we mean, unless explicitly stated otherwise, an oriented, connected, differentiable 3-manifold, possibly non-compact, without boundary unless otherwise specified, and always of \emph{finite type}, i.e.\ homeomorphic to a compact 3-manifold (possibly) with boundary, (possibly) with a 2-manifold (with or without boundary) removed from its boundary.

We refer to \cite{bonahon2002geometric} for a more precise discussion of most of the facts we are about to state. In general, (as found by H.~Kneser in 1929 and improved later), a 3-manifold $M$ [possibly with boundary, with some extra conditions: in the following sentences we use square brackets to refer to the adaptations due to cover the case of manifolds with boundary] admits a subdivision as a connect sum $M_1\#\ldots \# M_k$ of \nw{prime} 3-manifolds, i.e. 3-manifolds which do not admit any non-trivial further subdivision as a connect sum of 3-manifolds; and the factors of this subdivision are uniquely determined. Prime manifolds are, in particular, \nw{irreducible} i.e. they contain no embedded, essential 2-sphere.

Any irreducible manifold $M$, different from $\mathbb S^2\times\mathbb S^1$, contains a canonical collection $F$ of essentially embedded tori [and annuli with their boundary along $\partial M$] such that each connected component of $M\setminus F$ is one of the following (not pairwise exclusive): \nw{Seifert-fibred} (i.e. a $\mathbb S^1$-bundle over a 2-orbifold); [an interval bundle over a surface;] atoroidal [and anannular] (i.e. contains no essentially embedded torus [nor annulus with its boundary along $\partial M$]). This is called the \nw{JSJ decomposition} of $M$, after W.~Jaco, P.~Shalen and K.~Johannson.

W.P.~Thurston's Geometrization Conjecture, now a theorem by G. Perelman (see \cite{cao2006complete}; \cite{bonahon2002geometric}, Conjecture 4.1; \cite{friedl}, \S 6), implies that each connected component in the JSJ decomposition of an irreducible manifold admits a \emph{complete}, homogeneous Riemannian metric (i.e. a \emph{geometric structure}) with totally geodesic boundary.

In particular, one has:
\begin{claim}
Let $M$ be a 3-manifold which is closed, or is the interior of a compact 3-manifold whose boundary consists of tori. Suppose $M$ is irreducible, atoroidal, and not Seifert-fibered. Then $M$ admits a complete hyperbolic metric.
\end{claim}

A proof of this fact, with an extra hypothesis, was found by Thurston himself (\cite{thurstonhaken}). This overview should be sufficient to communicate the dominant role played by hyperbolic geometry in 3-manifolds, which Thurston correctly foresaw.

Also, in the statement of the Geometrization Theorem, one may require that each of the connected components of $M\setminus F$ has \emph{finite volume} under the specified geometric structure: there is only a handful of exceptional manifolds whose JSJ components cannot satisfy this further request; and none of them is hyperbolic, anyway. So, from now on, we say that a 3-manifold is \nw{hyperbolic} if it may be endowed with a finite-volume, complete hyperbolic structure.

Mostow's Rigidity Theorem in its classical form (Theorem 5.7.2 in \cite{thurstonnotes}) is a strong statement of uniqueness for these structures:
\begin{claim}
If $M_1^n$ and $M_2^n$ are two complete hyperbolic $n$-manifolds, for $n\geq 3$, with finite volume, and $\phi:\pi_1(M_1^n)\rightarrow \pi_1(M_2^n)$ is an isomorphism, then there exists a unique isometry $\psi: M_1^n\rightarrow M_2^n$ inducing $\phi$.
\end{claim}

\section{Graphs attached to a surface}\label{sec:graphs}
\subsection{Coarse geometry}
We are going to deal with several (in)equalities up to multiplicative and additive constants. Given four numbers $x,y\geq 0$, $Q\geq 1$ and $q\geq 0$, we write $x\leq_{(Q,q)} y$ (or $y\geq_{(Q,q)} x$) to mean $x\leq Qy+q$; and $x=_{(Q,q)} y$ to mean $x\leq_{(Q,q)} y \leq_{(Q,q)} x$. We also write $x\leq_Q y$, $x=_Q y$ to mean $x\leq_{(Q,Q)} y$, $x=_{(Q,Q)} y$ respectively.

Let $\mathbf G$ be a graph, with $\mathbf G^0$ the set of its vertices. $\mathbf G^0$ is turned into a geodesic metric space by assigning length one to each of its edges and defining, for each pair of vertices $x,y\in \mathbf G^0$, their distance $d_{\mathbf G}(x,y)$ to be the length of the shortest edge path connecting them. When $A,B \subseteq \mathbf G^0$ are non-empty, we define $d_{\mathbf G}(A,B)\coloneqq\mathrm{diam}_{\mathbf G} (A\cup B)$.

A map $g:I\rightarrow\mathbf G^0$, where $I=J\cap\mathbb Z$ for $J\subseteq \R$ an interval (possibly $J=\R$), is a \nw{$Q$-quasigeodesic} if $d_{\mathbf G}(g(a),g(b))=_Q |a-b|$ for all $a,b\in I$.

A map $g$ as above is a \nw{$Q$-unparametrized quasigeodesic} if there is an increasing map $\rho:I'\rightarrow I$, where $I'=J'\cap \mathbb Z$ for $J'\subseteq \R$ again an interval, such that: $\min \rho(I')=\min I,\max \rho(I')=\max I$; $g\circ \rho$ is a $Q$-quasigeodesic; if $a,b\in I$ are such that $\rho(\iota)\leq a\leq b\leq \rho(\iota+1)$ then $d_{\mathbf G}(g(a),g(b))\leq Q$.

In both definitions, $g$ is also allowed to be a \emph{multi-valued function}, i.e. a function with values in the power set, $I\rightarrow \mathcal P(\mathbf G^0)$; but in this case we require, in addition, $\mathrm{diam}_{\mathbf G}(g(i))\leq Q$ for all $i\in I$.

Note that in this work, when referring to a vertex $v$ of one of the graphs $\mathbf G$ we are going to define, we write $v\in \mathbf G$ (when we really mean $v\in\mathbf G^0$, as above). More generally, every time $\mathbf G$ is implicitly regarded as a set, we mean the set of its vertices, $\mathbf G^0$.

\subsection{The curve complex}

The parent of all graphs attached to a surface $S$ is the curve graph, extensively studied in \cite{masurminskyi}, \cite{masurminskyii} and further work. The graphs, albeit defined for $S$ a surface, can also be considered for surfaces with boundary, via $\cc(S)\coloneqq \cc\left(\mathrm{int}(S)\right)$ and similar identifications.

\begin{defin}
The \nw{curve complex} of $S$, denoted $\cc(S)$, is a simplicial complex whose vertices correspond to isotopy classes of essential curves in $S$.
\begin{itemize}
\item If $\xi(S)> 4$, there is an edge between two vertices if and only if the corresponding curves can be isotoped to be disjoint.
\item If $S\cong S_{1,1}$, there is an edge between two vertices if and only if the corresponding curves, when isotoped into minimal position, intersect in 1 point.
\item If $S\cong S_{0,4}$, there is an edge between two vertices if and only if the corresponding curves, when isotoped into minimal position, intersect in 2 points.
\end{itemize}
A collection of $n+1$ vertices in $\cc(S)$ spans a $n$-simplex if and only if any two of them are connected by an edge.

For $X$ a subsurface of $S$ which is not an annulus, the curve complex is defined just as if it were a stand-alone surface: $\cc(X)\coloneqq \cc\left(\inte(X)\right)$. The obvious map $\cc^0(X)\rightarrow \cc^0(S)$ is well defined and is an injection.

A special definition is needed when $X$ is an annular subsurface of $S$. Each vertex of $\cc(X)$ will represent an isotopy class, with fixed endpoints, of arcs (paths) properly embedded into $\ol{S^X}$ with an endpoint on each component of $\partial\ol{S^X}$. In this graph, too, there is an edge between two vertices if and only if the corresponding arcs can be isotoped, \emph{fixing their endpoints}, to have no intersection in $S^X$. The same construction as above applies for higher-dimensional skeleta of this graph.
\end{defin}

The most direct descendant of the curve graph is the \emph{arc graph}:

\begin{defin}
Suppose $S$ has punctures. The \nw{arc complex} of $S$, denoted $\ac(S)$, is a simplicial complex whose vertices correspond to classes of essential, properly embedded arcs in $S_\bullet$ under the equivalence relation given by isotopies fixing $\partial S_\bullet$ setwise; and there is an edge between any two vertices if the corresponding arcs can be isotoped to be disjoint. A collection of $n+1$ vertices in $\ac(S)$ spans a $n$-simplex if and only if any two of them are connected by an edge.

The \nw{arc and curve complex} $\acc(S)$ is defined as follows. Its $1$-skeleton is obtained from the $1$-skeleton of the disjoint union $\ac(S)\sqcup\cc(S)$ by adding an edge between a pair of vertices $\alpha\in\ac(S)$ and $\gamma\in\cc(S)$ every time $\alpha,\gamma$, may be seen as an arc and a curve, respectively, in $S_\bullet$, which can be isotoped (relatively to the boundary, resp.) to be disjoint. Again, a collection of $n+1$ vertices in $\acc(S)$ spans a $n$-simplex if and only if any two of them are connected by an edge.
\end{defin}

\begin{defin}\label{def:subsurfaceproj}
For $X\subseteq S$ a subsurface, the \nw{subsurface projection} $\pi_X:\cc(S)\rightarrow \mathcal P(\cc(X))$ is defined as follows (see \cite{masurminskyii}, \S 2.3, 2.4). 

\begin{itemize}
\item If $X$ is not an annulus: first of all, there is a natural map $\psi_X:\acc^0(X)\rightarrow \mathcal{P}\left(\cc^0(X)\right)$ which maps each $\alpha\in\cc^0(X)$ to $\{\alpha\}$ and each $\beta\in\ac^0(X)$ to the set of all isotopy classes of connected components of $\partial\nei(\beta\cup \partial X)$ which are essential in $X$ (they are 1 or 2). Here $\nei(\beta\cup \partial X)$ is a narrow, regular neighbourhood in $X$. One extends $\psi_X:\mathcal{P}\left(\acc^0(X)\right)\rightarrow \mathcal{P}\left(\cc^0(X)\right)$ naturally by defining $\psi_X(A)$ as the union of $\psi_X(a)$ over $a\in A$.

There is also a natural map $\pi'_X:\cc^0(S)\rightarrow \mathcal P\left(\acc^0(X)\right)$. Given $\alpha\in \cc^0(S)$: if $\alpha\in\cc^0(X)$ then, simply, $\pi'_X(\alpha)\coloneqq \{\alpha\}$. If $\alpha$ can be isotoped to lie completely out of $X$, then $\pi'_X(\alpha)\coloneqq\emptyset$. Otherwise $\alpha$ intersects $\partial X$ essentially; in this case, identify $\alpha$ with a representative of its isotopy class minimizing the number of intersection points with $\partial X$, and set $\pi'_X(\alpha)\coloneqq\alpha\cap X$, considered as a subset of $\ac^0(X)$ (minimality of the number of intersection points implies that $\alpha\cap X$ consists of essential arcs only).

Now, for $\alpha\in\cc^0(S)$, let $\pi_X(\alpha)\coloneqq \psi_X\left(\pi'_X(\alpha)\right)$. For $A\subseteq\cc^0(S)$ we define again $\pi_X(A)\coloneqq\bigcup_{\alpha\in A} \pi_X(\alpha)$.

\item If $X$ is an annulus: given $\alpha\in \cc^0(S)$, set $\pi_X(\alpha)=\emptyset$ if $\alpha$ does not intersect the core curve of $X$ essentially (including the case of $\alpha$ \emph{being} the core curve of $X$). Else, consider the preimage $\tilde\alpha$ of $\alpha$ in $S^X$ under the covering map $S^X\rightarrow S$: $\tilde\alpha$ consists of an infinite family of disjoint, quasi-geodesic paths. In particular each of them has two well-defined endpoints on $\partial\ol{S^X}$. Let then $\pi_X(\alpha)$ be the set of all connected components in $\tilde\alpha$ which connect the two connected components of $\partial\ol{S^X}$.
\end{itemize}

For $A,B\subseteq \cc^0(S)$ we use the shorthand notation $d_X(A,B)\coloneqq d_{\cc(X)}(\pi_X(A),\pi_X(B))$.
\end{defin}

Note that the subsurface projection of any curve onto any subsurface, if nonempty, has always diameter $\leq 1$.

Some literature, including \cite{mms} which will be employed many times in the present work, define the subsurface projection to be the map denoted $\pi'_X$ above. However, $\acc(X)$ includes $\cc(X)$ quasi-isometrically and these two versions of projection commute with that inclusion (again up to quasi-isometries): this is the only relevant aspect for us so we can stick with our definition, even if this might require changing some of the quasi-equality constants in the statements that will be quoted.

We recall a couple of useful lemmas about distances in curve complexes:

\begin{lemma}[Lemma 1.2, \cite{bowditch}]\label{lem:cc_distance}
If $S$ is a surface and $\alpha_1,\alpha_2\in \cc(S)$, then
$$
d_{\cc(S)}(\alpha_1,\alpha_2)\leq F\left(i(\alpha_1,\alpha_2)\right)
$$
for a function $F:\mathbb N\rightarrow\mathbb N$ with $F(n)=O(\log n)$, and independent of $S$.
\end{lemma}

\begin{rmk}\label{rmk:subsurf_inters_bound}
An observation which will be useful in several occasions when applying Lemma \ref{lem:cc_distance} above is the following: if $X\subseteq S$ is a non-annular subsurface and $\alpha_1,\alpha_2$ are curves in $\cc(S)$ with $\pi_X(\alpha_1),\pi_X(\alpha_2)\not=\emptyset$, then for any $\alpha'_1\in\pi_X(\alpha_1),\alpha'_2\in\pi_X(\alpha_2)$ the intersection number $i(\alpha'_1,\alpha'_2)\leq 4 i(\alpha_1,\alpha_2)+4$.

We may identify $\alpha_1,\alpha_2$ with two representatives that intersect transversely, realize the intersection number between their isotopy classes, and minimize the number of intersection points with $\partial X$. Let $\nei_1(\partial X)$, $\nei_2(\partial X)$ be two narrow, regular neighbourhoods of $\partial X$ in $S$, with $\bar\nei_1(\partial X) \subseteq \nei_2(\partial X)$. By definition of subsurface projection, for $i=1,2$, we may consider $\alpha'_i$ to be realized as the union of a set $a_i$ consisting of one or two parallel `copies' of a connected component of $(\alpha_i\cap X)\setminus\nei_i(\partial X)$; and a set $b_i$ which, if nonempty, consists of one or two segments of $\left(\partial\bar \nei_i(\partial X)\right)\cap X$. The set $b_i$ is empty (and $a_i=\alpha'_i$) exactly when $\alpha'_i=\alpha_i$. For the purposes of the following discussion, we may suppose that, for each choice of $i,j\in\{1,2\}$, each connected component of $a_i\cap \nei_j(\partial X)$ contains a point of $a_i\cap\partial X$.

Each intersection point between $\alpha'_1$ and $\alpha'_2$ is either:
\begin{itemize}
\item an intersection point between $a_1$ and $a_2$, which is to say, a `copy' of an intersection point between $\alpha_1$ and $\alpha_2$ which is contained in $X$; considering that each of $a_1$, $a_2$ consists of at most two parallel `copies' of an essential arc or curve in $X$, each intersection point between $\alpha_1$, $\alpha_2$ admits at most 4 `copies' among the intersection points between $a_1$, $a_2$.
\item an intersection point between $a_1$ and $b_2$: $a_1\cap b_2$ consists of at most $4$ points, each close to a different extremity of an arc constituting $a_1$.
\end{itemize}

Necessarily, $b_1\cap b_2=b_1\cap a_2=\emptyset$. Hence our claim.
\end{rmk}

\begin{lemma}[\S 2.4, \cite{masurminskyii}]\label{lem:annulus_distance}
If $X\subset S$ is an annulus and $\alpha_1,\alpha_2\in\cc^0(X)$ ($\alpha_1\not=\alpha_2$), then $d_{\cc(X)}(\alpha_1,\alpha_2)=1+i(\alpha_1,\alpha_2)$. Moreover $\cc(X)$ is quasi-isometric to $\mathbb Z$.
\end{lemma}
If $\alpha_1,\alpha_2\in \cc(X)$ for $X\subset S$ an annular subsurface, $i(\alpha_1,\alpha_2)$ is defined to be the minimum number of intersection points between $\alpha_1,\alpha_2$ attained when deforming both arcs within their respective isotopy class with fixed endpoints. Extreme points on $\partial\ol{S^X}$ do not count as intersection points.

Theorem 1.1 from \cite{masurminskyi} asserts that there is a $\delta=\delta(S)>0$ such that $\cc(S)$ is $\delta$-hyperbolic. It has been proved later (see e.g. \cite{bowditch2014uniform}, Theorem 1.1, or \cite{hensel20151}, Theorem 1.1) that there exists a universal value $\delta>0$ such that $\cc(S)$ is $\delta$-hyperbolic for all surfaces $S$. In Lemma \ref{lem:annulus_distance} above we have recalled that, for $X\subseteq S$ an annular subsurface, $\cc(X)$ is quasi-isometric to $\mathbb Z$. This implies, as recalled by Lemma 6.6 from \cite{mms}, that a reverse triangle inequality will hold for distances in it. Here we rephrase it in a way that refers to this setting only:
\begin{lemma}\label{lem:reversetriangle}
For any surface $S$ as above and any $Q>0$ there is a constant $\mathsf{R}_0=\mathsf{R}_0(S,Q)$ such that, if $X\subseteq S$ is a subsurface and $f:[l,m]\rightarrow \mathcal P\left(\cc^0(X)\right)$ is a $Q$-unparametrized quasi-geodesic, then for any $l\leq a\leq b\leq c\leq m$, if $\alpha=f(a),\beta=f(b),\gamma=f(c)$ then
$$
d(\alpha,\beta)+d(\beta,\gamma)\leq d(\alpha,\gamma)+\mathsf{R}_0.
$$
\end{lemma}
Uniform hyperbolicity of curve complexes implies that, in this statement, $\mathsf{R}_0$ may well be considered as depending on $Q$ only.

\subsection{Pants and marking graphs}

Two of the most immediate descendants of the curve complex are the pants graph and the marking graph. We define them as following \cite{masurminskyii}, \S 2, \S 6 and \S 8.

A \nw{pants decomposition} for a surface $S$ with $\xi(S)\geq 4$ is a maximal collection of essential, pairwise disjoint isotopy classes of curves $\{\alpha_1,\ldots,\alpha_n\}$ in $S$. Equivalently, once a set of pairwise disjoint representatives for $\{\alpha_1,\ldots,\alpha_n\}$ is chosen, its complement in $S$ will consist of a number of pairs of pants $\cong S_{0,3}$.

We say, instead, that a \emph{possibly infinite} collection $A\subseteq \cc(S)$ \nw{fills} $S$ if any other essential curve on $S$ intersects a curve in $A$. If $A=\{\alpha_1,\ldots,\alpha_n\}$ a finite set, then this is equivalent to saying that, once a set of representatives has been chosen for $\{\alpha_1,\ldots,\alpha_n\}$, with pairwise minimal number of intersection points, its complement in $S$ consists of a number of topological open discs $\cong S_{0,1}$ and 1-punctured discs $S_{0,2}$.

Given a collection of curves on $S$, either finite or infinite, there is always a subsurface of $S$ they fill; and it is unique \emph{up to isotopies in $S$}. So when we speak of the subsurface filled by the given collection, we mean its isotopy class or any representative of that class.

A \nw{complete marking} is a collection of pairs $\{p_1=(\alpha_1,t_1),\ldots,p_n=(\alpha_n,t_n)\}$ such that $\{\alpha_1,\ldots,\alpha_n\}$ is a pants decomposition of $S$ and, for each $i$, $t_i\subset \cc\left(\nei(\alpha_i)\right)$ (called \nw{transversal}) is a nonempty set of diameter $1$. A complete marking is \nw{clean} if, for each $i$, a curve $\beta_i\subset S$ exists such that: $\nei(\alpha_i\cup\beta_i)$ is either a 1-punctured torus or a 4-punctured sphere; $\pi_{\alpha_i}(\beta_i)=t_i$; and $\beta_i$ is disjoint from $\alpha_j$ for all $j\not=i$. If such $\beta_i$'s exists, they are unique.

If a complete marking is not clean, a clean marking which is \nw{compatible} with it is one whose base pants decomposition is the same, while each transversal set has the minimum distance possible from the original one, among all the clean complete markings with the same base pants decomposition. For a subdomain $Y\subset S$, the projection $\pi_Y(\mu)$ of a marking $\mu$ is defined as follows: when $Y$ is an annulus whose core is one of the $\alpha_i$, then $\pi_Y(\mu)=t_i$. When $Y$ is any other subsurface (including other annuli), $\pi_Y(\mu)=\bigcup_i \pi_Y(\alpha_i)$. A simplified version of this definition is given for pants decomposition.

\begin{defin}
The \nw{marking graph} $\mc(S)$ of a surface $S$ is a graph whose vertices correspond to all possible clean complete markings on $S$. There is a vertex between each couple of markings $\left((\alpha_1,\pi_{\alpha_1}(\beta_1)),\ldots,(\alpha_n,\pi_{\alpha_n}(\beta_n))\right)$ and\linebreak $\left((\alpha'_1,\pi_{\alpha'_1}(\beta'_1)),\ldots,(\alpha'_n,\pi_{\alpha'_n}(\beta'_n))\right)$ that are obtained from each other with one of these moves:
\begin{itemize}
\item twist: the only difference between the two markings is that $\beta'_i$ is obtained from $\beta_i$ by performing a twist or a half twist (depending on $\#\alpha_i\cap\beta_i$) around $\alpha_i$;
\item flip: all the couples are the same, except for one where $\alpha_i$ and $\beta_i$ have had their roles swapped, and the complete marking thus obtained has been then replaced with a compatible clean one.
\end{itemize}

The \nw{pants graph} $\pc(S)$ of $S$ is a graph whose vertices correspond to all possible pants decompositions $S$. Two vertices are joined by an edge if and only if the corresponding pants decompositions can be completed to clean complete markings which are obtained from each other with a flip move.
\end{defin}

Note that, if $S\cong S_{0,4}$ or $S_{1,1}$, then $\pc(S)$ coincides with the $1$-skeleton of $\cc(S)$.

Distances in the pants and the marking graphs are usually investigated via subsurface projections (Theorem 6.12 from \cite{masurminskyii}): consistently with Definition \ref{def:subsurfaceproj}, the subsurface projection of a pants decomposition is just the union of the projections of all curves constituting it.

\begin{theo}\label{thm:mmprojectiondist}
There exists a constant $M_6(S)$ such that, given $M>M_6$, there are constants $e_0, e_1$ only depending on $M$ and $S$ such that, for any pair of complete clean markings $\mu_I, \mu_T$ on $S$:
$$
\sum_{X\subseteq S} [d_X(\mu_I, \mu_T)]_M =_{(e_0,e_1)} d_{\mc}(\mu_I, \mu_T);
$$
and, for any pair of pants decompositions $p_I, p_T$,
$$
\sum_{\substack{X\subseteq S\\ X\text{ is not an annulus}}} [d_X(p_I, p_T)] =_{(e_0,e_1)} d_{\pc}(p_I, p_T).
$$

Here $[x]_M\coloneqq x$ if $x\geq M$, and $0$ otherwise. The summations shall be meant over $X\subseteq S$ subsurfaces, where each isotopy class of subsurfaces is counted only once.
\end{theo}

\subsection{The quasi-pants graph}
We will never really employ the given definition of marking graph: we will use a construction given in \cite{mms} instead. Let $k_1$ be the maximum self-intersection number of a complete clean marking on $S$, and let $\ell_1$ be the maximum intersection number between any two complete clean markings on $S$ obtained from each other via an elementary move. Fix any $k\geq k_1, \ell\geq \ell_1$. Then we can redefine $\ma(S)$ to be the graph whose vertices consist of collections of essential, distinct isotopy classes of curves on $S$ which fill the surface and have self-intersection number $\leq k$; and there is an edge between any two vertices corresponding to collections of curves intersecting in at most $\ell$ points. This graph, depending on the parameters $k,\ell$, is shown to be quasi-isometric to $\mc(S)$ (with constants depending on $k,\ell$). And the first formula in Theorem \ref{thm:mmprojectiondist} holds for estimating distances in $\ma(S)$ as well, if one chooses suitable $M\geq M_6(S,k,\ell)$; $e_j=e_j(S,M,k,\ell)$ ($j=0,1$) and the functions $M_6,e_0,e_1$ are suitably defined.

With the pants graph, we perform a similar construction:
\begin{defin}\label{def:quasipants}
A \nw{quasi-pants graph} $\pa(S)$ is defined as follows. Fix two parameters $k,\ell\geq 0$.
\begin{itemize}
\item Each vertex of the graph represents a collection $\{\alpha_1,\ldots,\alpha_m\}$ of essential, distinct isotopy classes of curves of $S$ such that, when a set of representatives minimizing the number of mutual intersection is chosen, this number is $\leq k$; and with one of the following, equivalent, properties:
\begin{enumerate}[label=\alph*)]
\item each isotopy class in $\cc(S)$ either is one of the $\alpha_j$, or intersects one of the $\alpha_j$ (however the representatives are chosen);
\item given a set of representatives for the $\{\alpha_1,\ldots,\alpha_m\}$, minimizing the number of mutual intersection, its complement in $S$ is a collection of topological open discs, 1-punctured discs, and pairs of pants;
\item Let $X$ be the subsurface of $S$ filled by $\{\alpha_1,\ldots,\alpha_m\}$ (possibly one which is not connected and with annular components); then $S\setminus X$ is a collection of (closed) pairs of pants.
\end{enumerate}
In particular, vertices include all pants decompositions of $S$.
\item There is an edge between two vertices $\mu$ and $\nu$ if and only if $i(\mu,\nu)\leq\ell$.
\end{itemize}
We only consider values of $\ell$ for which the graph is connected and each vertex has distance at most $1$ from a pants decomposition.
\end{defin}

\begin{rmk}\label{rmk:ell1}
There is a number $\ell_1\geq 0$ such that all values $\ell>\ell_1$ are acceptable for the last sentence of the above definition. Two pants decompositions which are adjacent in the pants graph have mutual intersection number $1$ or $2$: thus, for $\ell\geq 2$, the subgraph of $\pa(S)$ having pants decompositions as vertices is connected, because among its edges there are all the ones of $\pc(S)$, which is connected.

Moreover, note that orbits under the action of $\mcg(S)$ on $\pa(S)$ are finitely many, so there is a finite bound for $m=\max_{\mu\in\pa(S)}\min_{p\in\pc(S)} i(\mu,p)$. A suitable value for $\ell_1$ is then just $\max\{2,m\}$.
\end{rmk}

The parameters $k,\ell$ for $\ma(S)$ and $\pa(S)$ will be fixed once and for all after having introduced train tracks (see Remark \ref{rmk:pickparameters}). Meanwhile we prove the following:
\begin{lemma}\label{lem:pantsquasiisom}
The injection $\iota:\pc^0(S)\rightarrow\pa^0(S)$ of the vertex set of the first graph into the second one is a quasi-isometry, with constants depending on $S,k,\ell$. In particular, the second formula in Theorem \ref{thm:mmprojectiondist} holds for distance estimation in $\pa(S)$ as well, if we choose $M\geq M_6(S,k,\ell)$; $e_j=e_j(S,M,k,\ell)$ ($j=0,1$) for suitably defined functions $M_6,e_0,e_1$.
\end{lemma}

In order for the mentioned formula to make sense, we are extending naturally to vertices of $\pa(S)$ our notion of subsurface projection, and the notation $d_X$. A similar statement concerning the graphs we have denoted $\ma(S)$ and $\mc(S)$ is implicitly used in \cite{mms}, \S 6.

\begin{proof}
It suffices to show that $\iota$ is a quasi-isometric embedding because, by definition of $\pa(S)$, the $1$-neighbourhood of $\iota\left(\pc(S)\right)$ is the entire $\pa(S)$.

The inequality $d_\pa(\iota(p),\iota(q))\leq d_\pc(p,q)$ holds for any $p,q\in\pc(S)$: if two vertices of $\pc(S)$ are connected by an edge then so are their images in $\pa(S)$, by Remark \ref{rmk:ell1} above.

We only need to prove an inequality in the opposite direction. To do so, we build a map $\Phi:\pa(S)\rightarrow\pc(S)$ which will turn out to be a quasi-inverse of $\iota$: for $\mu\in \pa(S)$ let $X(\mu)$ be the (possibly disconnected) subsurface of $S$ filled by $\mu$. Set $\Phi(\mu)$ to be a pants decomposition including all curves of $\partial X(\mu)$ and chosen so that the intersection number between it and $\mu$ is minimal among all pants decompositions with the specified condition. We may suppose that $\Phi$ is $\mcg(S)$-equivariant: $\Phi(\psi\cdot\mu)=\psi\cdot\Phi(\mu)$ for all $\psi$ self-homeomorphism of $S$. As the orbits of $\mcg(S)$ in $\pa^0(S)$ are finitely many, there are two numbers $a,a'$ such that $i(\mu,\Phi(\mu))\leq a$ and $d_\pa(\mu,\Phi(\mu))\leq a'$ for all $\mu$. Note that $\Phi\circ\iota=\mathrm{id}_{\pc(S)}$.

Take any $\mu,\nu\in\pa(S)$ with $d_\pa(\mu,\nu)=1$; then $i(\mu,\nu)\leq\ell$. Let $X$ be the (possibly disconnected) subsurface of $S$ filled by $\mu$. The bound on the intersection number yields that there is a finite family $A(\mu)\subseteq \pa(S)$ such that $\nu$ is obtained from an element of $A(\mu)$ by Dehn twisting about some components of $\partial X$. The association $\mu\mapsto A(\mu)$ can be supposed to be equivariant under homeomorphisms of $S$. Again by finiteness of the number of orbits, there is a global upper bound $b$ for the size of $A(\mu)$, and also a bound $b'$ for $d_\pc(\Phi(\mu),\Phi(\nu))$ for any $\mu\in\pa(S),\nu\in A(\mu)$; and therefore for any $\mu,\nu$ at distance $1$ from each other.

This means that, more generally, $d_\pc(\Phi(\mu),\Phi(\nu))\leq b' d_\pa(\mu,\nu)$ for any $\mu,\nu$: given a geodesic connecting the two, we get an upper bound for the length of a path joining the images of the vertices under $\Phi$. In particular, if $p,q\in\pc(S)$, then $d_\pc(p,q)\leq b' d_\pa(\iota(p),\iota(q))$. The bound $b'$ depends only on $S,k,\ell$.

We now prove the second part of the lemma's statement. %For simplicity, we replace the previously defined map $\Phi$ with a new $\Phi':\pa(S)\rightarrow\pc(S)$ satisfying slightly different requests: $\Phi'$ shall be $\mcg(S)$-equivariant, fix all vertices of $\pa(S)$ which are pants decompositions, and map each vertex which is not a pants decomposition to a pants decomposition which is adjacent to it in $\pa(S)$.
Given any $\mu\in\pa(S)$ and $X\subseteq S$ a non-annular subsurface, $i(\mu,\Phi(\mu))\leq a$ implies that, given any $\alpha'_1\in \pi_X(\mu)$, $\alpha'_2\in \pi_X(\Phi(\mu))$, we have $i(\alpha'_1,\alpha'_2)\leq 4a+4$ by Remark \ref{rmk:subsurf_inters_bound}, therefore $d_X\left(\mu,\Phi(\mu)\right)\leq \max \left\{F(k),F(4a+4)\right\}$ by Lemma \ref{lem:cc_distance}. Hence there is number $\beta=\beta(S)$ such that, for any $\mu,\nu\in\pa(S)$, $|d_X(\mu,\nu)-d_X(\Phi(\mu),\Phi(\nu))|\leq \beta$.

Given $M\geq \max\{M_6(S),\beta(S)\}$, where $M_6$ is defined as in Theorem \ref{thm:mmprojectiondist}, we get the following chain of inequalities. We neglect indices on summations, as they will always range over all isotopy classes of non-annular subsurfaces $X\subset S$; and we write simply $d_X$ for $d_X(\mu,\nu)$ and $d_X\Phi$ for $d_X(\Phi(\mu),\Phi(\nu))$.
\begin{eqnarray*}
 & d_\pa(\mu,\nu)\leq d_\pa(\Phi(\mu),\Phi(\nu)) +2a' \leq d_\pc(\Phi(\mu),\Phi(\nu)) +2a'\leq  & \\
 & e_0(M)\sum [d_X\Phi]_M + e_1(M)+2a' \leq e_0(M)\sum [d_X+\beta]_M + e_1(M)+2a' \leq  & \\
 & e_0(M)\sum \left([d_X]_{M-\beta}+\beta_X\right) + e_1(M)+2a' \leq e_0(M)(1+\beta)\sum [d_X]_{M-\beta} + e_1(M)+2a'. & 
\end{eqnarray*}
Here we have denoted with $\beta_X$ a quantity which is equal to $\beta$ only if $[d_X]_{M-\beta}\not=0$; else it is also $0$. For the other inequality:
\begin{eqnarray*}
 & d_\pa(\mu,\nu) \geq d_\pa(\Phi(\mu),\Phi(\nu)) -2a' \geq (b')^{-1} d_\pc(\Phi(\mu),\Phi(\nu)) -2a' \geq  & \\
 & (b'e_0(M))^{-1}\sum [d_X\Phi]_M -e_1(M)-2a' \geq (b'e_0(M))^{-1}\sum [d_X-\beta]_M -e_1(M)-2a' \geq & \\
 & (b'e_0(M))^{-1} \frac{M}{M+\beta}\sum [d_X]_{M+\beta} -e_1(M)-2a'. & 
\end{eqnarray*}

So, we have that a formula as in Theorem \ref{thm:mmprojectiondist} holds for distances in $\pa(S)$, too. Denoting with a $'$ the quantities related with $\pa(S)$ rather than $\pc(S)$, we choose $M_6'(S)=\max\{M_6(S),\beta(S)\}+\beta(S)$; $e_0'(M)=e_0(M)\max\left\{1+\beta,b'\frac{M+\beta}{M}\right\}$, $e_1'(M)=e_1(M)+2$. Dependence of these new parameters from $k,\ell$ is hidden in the constants $a'$, $\beta$ and $b'$.
\end{proof}

\section{Hyperbolic volume estimates}\label{sec:hypvolumehistoric}

In theory, given a hyperbolic 3-manifold --- we know from Mostow Rigidity that no 3-manifold is hyperbolic in two different ways --- one may work out its volume by triangulating it with ideal tetrahedra, and then looking for a solution for Thurston's gluing consistency equations (see \cite{thurstonnotes}, Chapter 4). But, other than a precise computation of volume from a given, fixed hyperbolic 3-manifold, it is interesting to understand if one may work out the volume from some topological `parameter' in a given class of 3-manifold, albeit not precisely but only up to multiplicative and additive constants. Here we give two examples of this.

\subsection{Links complements in $\mathbb S^3$, and the guts approach}

Let $K\subset \mathbb S^3$ be a link (a smooth embedding of a number of disjoint copies of $\mathbb S^1$). It is a result of W.P. Thurston (see for instance Theorem 10.5.1, \cite{kawauchi}) that every \emph{knot} $K$ in $\mathbb S^3$ satisfies one and only one of the following: it is a torus knot; it is a satellite knot (including composite knots); its complement admits a complete hyperbolic metric with finite volume.

So, when a knot is prime, it is `likely' to fall in the third case (from now on: we say, simply, that the knot, or link, is \nw{hyperbolic}). There is no classification for \emph{links} that works as well as the one given above for knots, but from the underlying idea that a `majority' of link is hyperbolic is still true, via Thurston's Geometrization (see \S\ref{sec:hyp3mflds}). Many explicit classes of hyperbolic links are known (see e.g. \cite{adams} for an account), and it is a matter of interest how the hyperbolic volume of the complement is related with properties which be readable from a link diagram.

Results in this field have been established by Lackenby --- for links which admit an alternating diagram, see \cite{lackenby} --- and by Futer--Kalfagianni--Purcell --- for Montesinos links and for a subclass of the closed braids called \emph{positive}, see \cite{futerkalfapurcell}. They all descend from a theorem of Ian Agol (\cite{agol}, made sharper in \cite{agolimproved}), which reads as follows. If $S$ is a surface bounded by $K$, which `does not admit simplifications' in $\mathbb S^3\setminus K$, and $M\coloneqq \mathbb S^3\setminus \nei(S)$, then $\vol(\mathbb S^3\setminus K)\geq a\chi(\mathrm{guts}(M))$, for $a$ a constant and $\mathrm{guts}(M)$ being the union of the atoroidal and anannular components arising from the JSJ decomposition.

Lackenby's result is that, if a hyperbolic link $K$ has a prime, alternating diagram $D$ then $\vol(\mathbb S^3\setminus K)=_A t(D)$ for $A$ a constant and $t(D)$ the number of \emph{twist regions} in the diagram, i.e. maximal concatenations of bigons in $\mathbb S^2\setminus D$. A crossing of $D$ which is not adjacent to a bigon counts as a crossing. The theorem is established via a careful choice of a surface $S$ bounding $K$ and obtained from $D$ and then using Agol's theorem for the lower bound, and a clever triangulation for the upper bound. The result of Futer--Kalfagianni--Purcell is established via a heavy generalization of Lackenby's machinery.

\subsection{Mapping tori, and pants distance}\label{sub:mappingtori}

Given $S$ a surface and $\psi:S\rightarrow S$ is a suitable homeomorphism (or mapping class), the corresponding \nw{mapping torus} is a 3-manifold $M \cong S\times I/\sim_\psi$, where $I=[0,1]$, and $\sim_\psi$ is the equivalence relation that identifies each point $(x,0)$, $x\in S$, with $(\psi(x),1)$. A mapping torus is in particular a fibre bundle over $\mathbb S^1$, with fibre $S$. The map $\psi$ is called the \nw{monodromy} of the mapping torus.

We distinguish among three kinds of behaviour for $\psi$:
\begin{itemize}
\item $\psi$ has \nw{finite order} if some $n>0$ exists such that $\psi^n$ is isotopic to $\mathrm{id}_S$;
\item $\psi$ is \nw{reducible} if there is a multicurve on $S$ which is fixed by $\psi$, up to isotopy;
\item $\psi$ is \nw{pseudo-Anosov} if neither it has finite order nor it is reducible.
\end{itemize}

It is worth recalling some basic concepts: we follow the summary given in \cite{hyperbfibermfld}. Further details, with partially different conventions, may be found in \cite{cassonbleiler}, chapters 3--6 or in \cite{penner}, \S 1.6, \S 1.7, Chapter 3. A \nw{geodesic lamination} on a surface $S$ is a closed subset of $S$ which is a disjoint union of geodesics, called \nw{leaves} of the lamination.

Let $\lambda$ be a geodesic lamination; consider a function $\mu:T(\lambda)\rightarrow\R_{\geq 0}$, where $T(\lambda)$ is the set of all compact 1-manifolds embedded in $S$ and intersecting the leaves of $\lambda$ transversely (the 1-manifolds' boundaries, in particular, are disjoint from $\lambda$). We say that $\mu$ is a \nw{transverse measure} on $\lambda$ if it has the following properties. The function $\mu$ is $\sigma$-additive, meaning that, for each countable family $\{\alpha_i\}_{i\in\mathbb N}\subseteq T(\lambda)$ such that $(i\not=j\Rightarrow \alpha_i\cap\alpha_j=\partial\alpha_i\cap\partial\alpha_j)$ and that $\alpha\coloneqq \bigcup\alpha_i\in T(\lambda)$, we have $\mu(\alpha)=\sum \mu(\alpha_i)$. Given $\alpha_0,\alpha_1\in T(\lambda)$ two manifolds which are isotopic via a continuous family of $\alpha_t\in T(\lambda)$, $t\in [0,1]$, we have $\mu(\alpha_0)=\mu(\alpha_1)$. If $\alpha\in T(\lambda)$ is actually disjoint from $\lambda$, then $\mu(\alpha)=0$.

A \nw{measured lamination} is a pair $(\lambda,\mu)$ where $\lambda$ is a lamination and $\mu$ is a tranverse measure for $\lambda$, \emph{with full support} i.e. $\mu(\alpha)\not=0$ for all $\alpha\in T(\lambda)$ not disjoint from $\lambda$. We say that a lamination is \nw{minimal} if each half-leaf of $\lambda$ is dense in $\lambda$ and that it \nw{fills} $S$ if $S\setminus\lambda$ consists of a number of contractible connected components and peripheral annuli. Rephrasing a classical theorem of Thurston (cf. \cite{thurston_pa}, Theorem 4; \cite{flp}, Theorem 1.6; or \cite{hyperbfibermfld}, Theorem 2.5):

\begin{theo}
The map $\psi:S\rightarrow S$ is a pseudo-Anosov homeomorphism if and only if there exist a pair of minimal measured laminations $(\lambda^s,\mu^s)$, $(\lambda^u,\mu^u)$ filling $S$, a constant $c>1$ and a homeomorphism $\psi'$, isotopic to $\psi$, such that
$$
(\psi(\lambda^s),\psi_*\mu^s)=(\lambda^s,c^{-1}\mu^s)\quad\text{and}\quad
(\psi(\lambda^u),\psi_*\mu^u)=(\lambda^u,c\mu^u).
$$
The constant $c$ is unique, and the two measured laminations are unique up to scaling of the assigned transverse measure. They are called \nw{stable lamination} and \nw{unstable lamination}, respectively.
\end{theo}

\begin{rmk}\label{rmk:power_pa}
As an immediate consequence, if $\psi$ is pseudo-Anosov and $n\not=0$, then $\psi^n$ is also pseudo-Anosov.
\end{rmk}

In \cite{hyperbfibermfld} it is proved that each of the three possibilities listed above has a precise consequence on the JSJ characterization of a mapping torus:
\begin{theo}\label{thm:mappingtorushyperbolic}
Let $M$ be a mapping torus, equal to $\faktor{S\times I}{\sim_\psi}$ where $S$ is a surface, and $\psi$ is a self-homeomorphism of $S$. Then
\begin{itemize}
\item $M$ is Seifert-fibered if and only if $\psi$ has finite order;
\item $M$ contains an essential embedded torus if and only if $\psi$ is reducible;
\item $M$ is hyperbolic if and only if $\psi$ is pseudo-Anosov.
\end{itemize}
\end{theo}

In \cite{brock1}, \cite{brock2}, Brock proves (in particular) the following result.
\begin{theo}\label{thm:brockmappingtori}
Let $S$ be a surface. Two constants $e_2=e_2(S), e_3=e_3(S)$ exist such that, if $\psi:S\rightarrow S$ is a pseudo-Anosov homeomorphism and $M=S\times I/\sim_\psi$ is the corresponding mapping torus (which is hyperbolic), then
$$
\vol(M) =_{(e_2,e_3)} |\psi|.
$$
Here, $|\psi|$ denotes the translation distance of the map induced by $\psi$ on $\pc(S)$.
\end{theo}

Given a homeomorphism $\psi:S\rightarrow S$, the \nw{translation distance} induced by $\psi$ in $\pc(S)$ is the quantity $|\psi|_{\pc(S)}\coloneqq\min \{d(v,\psi\cdot v)|v\in\text{vertices of }\pc(S)\}$. The \nw{stable translation distance} induced by $\psi$ is, instead, $|\psi|_{\pc(S)}^{st}\coloneqq\lim_{n\rightarrow\infty} d(v,\psi^n\cdot v)/n$ where $v$ is any fixed vertex of $\pc(S)$. This latter quantity is well-defined and not depending on $v$, for general facts about metric spaces --- see \cite{bridson}, Chapter II.6, \S 6.6. Consequently, via triangle inequality,  one has $|\psi|_{\pc(S)}^{st}\leq |\psi|_{\pc(S)}$.

\begin{rmk}\label{rmk:stable_dist_pc}
We prove that there is a constant $e_4=e_4(S)$ such that, if $\psi\in\mcg(S)$ is pseudo-Anosov, then
$$|\psi|_{\pc(S)}=_{e_4} |\psi|_{\pc(S)}^{st}$$.

Theorem 3.2 in \cite{brock1} shows that there is a map $Q:\pc^0(S)\rightarrow \mathrm{Teich}(S)$, equivariant under the action of $\mcg(S)$ on the two metric spaces. Here $\mathrm{Teich}(S)$ is the Teichm\"uller space of $S$, equipped with the Weil-Petersson metric. Let $|\psi|_{WP}$, $|\psi|_{WP}^{st}$ be the translation distance and the stable translation distance, respectively, induced by $\psi$ in $\mathrm{Teich}(S)$; they are defined in an entirely similar way as in the pants graph. The theorem we have just mentioned implies that $|\psi|_{WP}$, $|\psi|_{\pc(S)}$ are equal up to multiplicative and additive errors; and the same is true of $|\psi|_{WP}^{st}$, $|\psi|_{\pc(S)}^{st}$.

In addition to what we have already noted earlier, we have to prove that $|\psi|_{\pc(S)}\leq_{e_4} |\psi|_{\pc(S)}^{st}$. We claim that $|\psi|_{WP}\leq |\psi|_{WP}^{st}$, which implies the desired inequality.

By Theorem 1.1 in \cite{daskalopoulos}, there is a unique $\psi$-equivariant, complete geodesic $g$ in $\mathrm{Teich}(S)$ i.e.\ an axis for the action of $\psi$. Let $x\in g$: then $|\psi|_{WP}\leq d_{WP}(x,\psi\cdot x)$, and note that $d_{WP}(x,\psi^n\cdot x)=n\cdot d_{WP}(g,\psi\cdot g)$ because the distance between $x$, $\psi^n\cdot x$ is realized as the segment of $g$ between these two points. Therefore also $|\psi|_{WP}^{st}=d_{WP}(g,\psi\cdot g)$ and the claim is proved.

An equality similar to the one just shown holds for translation distance and stable translation distance in $\cc(S)$, due to the reverse triangle inequality in Lemma \ref{lem:reversetriangle}.
\end{rmk}

\section{The role of train tracks}\label{sec:role_train_tracks}

\emph{Train tracks} on a surface $S$ will be properly defined in \S \ref{sec:traintracks}. Informally, a train track is a 1-complex on $S$ with the property that each vertex is `smoothed out' i.e. there is one selected edge incident to the vertex, such that one may proceed smoothly from this edge to any other one. Train tracks were introduced by Thurston (see \cite{thurstonnotes}, \S 8.9) to study geodesic laminations: informally, given a lamination, it is always possible to `squeeze' bands consisting of parallel segments of its leaves, and turn the lamination into a train track (\cite{penner}, Theorem 1.6.5);  as a particular case, one may do this for a (multi)curve. We say, then, that the train track \emph{carries} a lamination or a (multi)curve when one may draw a family of smooth paths along the track which is isotopic to the lamination or (multi)curve.

A track $\tau$, in general, will carry a huge quantity of different laminations and (multi)curves. The set $\cc(\tau)\subseteq \cc(S)$ of the curves carried by $\tau$, in particular, includes ones which travel along the edges of $\tau$ a high number of times: but there is a finite family $V(\tau)\subseteq\cc(\tau)$ collecting the simplest ones. A \emph{split} is a move on a train track which turns it into a new one, $\tau'$, with $\cc(\tau')\subseteq\cc(\tau)$; and in general, each curve in $\cc(\tau')$ traverses the branches of $\tau'$ fewer times than the ones in $\tau$.

What makes train tracks particularly interesting, then, is that a \emph{splitting sequence} of train tracks i.e. a sequence $(\tau_j)_{j\geq 0}$ of iterated splits on a train track, will change the set $V(\tau_j)$ so that, by following them, we move through $\cc(S)$ keeping, roughly, always the same direction. To start with, splitting sequences were used to prove the hyperbolicity of the curve complex in \cite{masurminskyi}. Some formal statements are given in \S \ref{sub:goodbehaviour}, but we stress here that Masur and Minsky proved in \cite{masurminskyq} that $\left(V(\tau_j)\right)_{j\geq 0}$ is an unparametrized quasi-geodesic in the curve complex. This property is complemented by the fact that $\cc(S)\setminus\cc(\tau_0)$ is quasi-convex, as shown in \cite{notcarried}. Recent work \cite{mms} of Masur, Mosher and Schleimer --- the one that motivates this thesis --- has shown that if some extra, mild hypotheses are met, then this sequence is a \emph{true} quasi-geodesic in $\ma(S)$. 

Furthermore, it is worth pointing out that the reason which motivated the introduction of train tracks is also reflected into the behaviour at infinity of a splitting sequence. The sets of all laminations carried by $\tau_j$ are also a decreasing family in $j$. The monograph \cite{mosher} of Mosher explores how, and in what circumstances, the splitting sequence `converges' to a lamination (Mosher actually uses the language of \emph{foliations} instead). Also notably, the paper \cite{hamenstadt} of Hamenst\"adt uses train track splitting sequences to construct a natural identification of the Gromov boundary of $\cc(S)$ with a suitable subspace of the space of geodesic laminations.

Train tracks and their splitting sequences, then, provide a combinatorial, concrete way of understanding some aspects of the geometry of the surface-related graphs defined in \S \ref{sec:graphs} and of their closest relatives, e.g. Teichm\"uller spaces. The present thesis follows this current.