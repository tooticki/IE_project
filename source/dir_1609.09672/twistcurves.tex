\section{All about twist curves}\label{sec:twistcurves}

\subsection{Terminology and basics}

When a train track splitting sequence spans long distances in annulus subsurface projections we find that, morally, they are caused by application of high powers of Dehn twist; they may be produced by elementary moves which are hidden and sparse in the sequence, and the key to track them down are \emph{twist curves}. We need some work to make this sentence precise.

First of all, suppose that $\tau$ is an almost track and that $\gamma$ is a \emph{wide carried} curve for $\tau$. Let $A_\gamma$ be a wide collar and let $X$ be a regular neighbourhood of $\gamma$ in $S$. Let $\hat p:S^X\rightarrow S$ be the covering map. We have seen in point \ref{itm:embeddedcore} of Remark \ref{rmk:annulusinducedbasics} that the core curve of $S^X$ (which we identify with $\gamma$ itself) is carried by $\tau^X$ and is embedded as a train path; $A_\gamma$ has a homeomorphic lift to $S^X$, which is an annulus having $\tau^X.\gamma$ as a boundary component.

\begin{defin}
Let $e$ be a small branch end of $\tau^X$ such that $e\cap\tau^X.\gamma$ is a switch of $\tau^X$. If $e\cap A_\gamma\not=\emptyset$, we say that $e$ \nw{hits} $A_\gamma$; else, $e$ \nw{avoids} $A_\gamma$. This terminology does not apply to branch ends of $\tau^X$ such that $e\cap(\tau^X.\gamma)$ is empty, or consists of more than one point.

Given a small branch end $e'$ of $\tau$, we say that it \emph{hits} $A_\gamma$ if $e'=\hat p(e)$ for $e$ a branch end of $\tau^X$ hitting $A_\gamma$; we say that $e'$ \emph{avoids} $A_\gamma$ if $e'$ does not hit $A_\gamma$ and $e'=\hat p(e)$ for $e$ a branch end of $\tau^X$ avoiding $A_\gamma$. With this definition we are including, among the small branch ends of $\tau$ avoiding $A_\gamma$, the ones which are part of $\tau.\gamma$, too. However, this terminology does not apply to branch ends of $\tau$ which are disjoint from $\tau.\gamma$.

The set of all branch ends hitting $A_\gamma$ and the set of all branches avoiding it will be called the two \nw{sides} of $\gamma$. 
\end{defin}

\begin{defin}\label{def:twistcurve}
Let $\tau$ be an almost track on a surface $S$, let $\gamma\in\cc(S)$ be wide carried by $\tau$ and let $X$ be a regular neighbourhood of $\gamma$. The curve $\gamma$ is a \nw{twist curve} for $\tau^X$ and for $\tau$ if the two following conditions hold.
\begin{itemize}
\item There is a wide collar $A_\gamma$ (to be viewed in $S^X$) such that, for each branch end $e$ of $\tau^X$ hitting $A_\gamma$, the $e$-orientation (see Definition \ref{def:eorientation}) on $\gamma$ is always the same. The collar $A_\gamma$ (regarded as either $\subseteq S$ or $\subseteq S^X$) will be called a \nw{twist collar}, and the orientation given to $\gamma$ the \nw{$A_\gamma$-orientation}.
\item Among the branches of $\tau^X$ included in $\tau^X.\gamma$, there is at least a large one.
\end{itemize}
A curve $\gamma$ whose realization as a train path in $\tau$ is embedded, and which may be regarded as a twist curve after picking a collar on \emph{either} of its sides, is called \nw{combed}.
\end{defin}

An example of twist curve for an almost track $\tau$ is given in Figure \ref{fig:twistcurve}. The second bullet above is equivalent to saying that there is a branch end $f$ in $\tau^X$ which avoids $A_\gamma$, and such that the $f$-orientation is opposite to the $A_\gamma$-orientation. In particular, if $\gamma$ is combed, the orientations given to $\gamma$ by twist collars on opposite sides are opposite. So, this condition implies that $\cc(\tau^X)\not=\emptyset$ and that $\gamma$ is carried by $\tau|X$, by point \ref{itm:gammacarriedininduced} in Remark \ref{rmk:annulusinducedbasics}.

\begin{figure}
\centering{\includegraphics[width=.9\textwidth]{twistcurve.pdf}}
\caption{\label{fig:twistcurve}\emph{Left:} An example of twist curve which is not embedded as a train path. A possible choice for a twist collar $A_\gamma$ is marked in grey, and the arrows describe the $A_\gamma$-orientation. The extremities of splitting arcs for a wide spurious and a wide bispurious split (see Definition \ref{def:twistmoves}) are drawn with a dashed and a dotted line, respectively. \emph{Right:} The effect of a wide bispurious split. The twist curve has a twist collar that necessarily differs from the previous one. In Remark \ref{rmk:permanenceconventions} we show how to make the two collars correspond under a suitable homotopy equivalence $\E$ based on the tie collapse.}
\end{figure}

\begin{defin}
Let $\tau$ be an almost track on a surface $S$, let $\gamma\in\cc(S)$ be a twist curve for $\tau$ with a fixed regular neighbourhood $X$ and a fixed twist collar $A_\gamma$. Let $e$ be a branch end of $\tau^X$ sharing a switch with $\tau^X.\gamma$ and avoiding $A_\gamma$. If the $e$-orientation on $\gamma$ is opposite to the $A_\gamma$-orientation we call $e$ an \nw{$A_\gamma$-favourable} branch end; else we say that $e$ is \nw{$A_\gamma$-adverse}.
\end{defin}

For $\tau$, $\gamma$, $X$ as specified above, we say that $\gamma$ is a \emph{twist curve}, or it is \emph{combed}, in $\tau|X$ if it is a carried curve there and satisfies similar hypotheses as the ones that have been laid out for $\gamma$ in $\tau^X$. Lemma \ref{lem:twistininduced} specifies how being a twist curve in $\tau$ relates with being one in $\tau|X$.

\vspace{1ex}
\ul{Note}: all pretracks and splitting sequences in the present \S \ref{sec:twistcurves} will be \emph{generic}, as we will specify in our definitions and statements. Only in \S \ref{sub:twistcurvebound}, for technical reasons, we need to employ some semigeneric ones (see the note at the beginning of that subsection).

\vspace{1ex}
Note first of all that, if $\tau$ is a generic almost track with a wide curve $\gamma$, which has $A_\gamma$ as a wide collar and $X$ as a regular neighbourhood of the latter, then a small branch end $e'$ that hits (resp. avoids) $A_\gamma$ in $\tau$ has only one lift to $S^X$ which also hits (resp. avoids) $A_\gamma$\footnote{This is not true in the semigeneric setting: a branch end of $\tau$ which avoids $A_\gamma$ and shares its switch with two small branch ends in $\tau.\gamma$ has two distinct lifts that give opposite orientations to $\gamma$.}. So some of the definitions given above for $\tau^X$ descend to $\tau$.
\begin{itemize}
\item If $e'$ is a small branch end of $\tau$ which hits (resp. avoids) $A_\gamma$, let $e$ be the lift of $e'$ to $\tau^X$ which hits (resp. avoids) $A_\gamma$, and define the \emph{$e'$-orientation} on $\gamma$ to be the $e$-orientation.
\item If $\gamma$ is a twist curve with $A_\gamma$ a twist collar, and $e'$ is a small branch end of $\tau$ which avoids $A_\gamma$, let $e$ be the lift of $e'$ which avoids $A_\gamma$. We say that $e'$ is \emph{$A_\gamma$-favourable} (resp. \emph{adverse}) if $e$ is $A_\gamma$-favourable (resp. adverse).
\end{itemize}

\begin{rmk}\label{rmk:twistparam}
We set up here several pieces of notation referring to regular neighbourhoods of twist curves, to be used in the present section. Given $\gamma$ a twist curve of a generic almost track $\tau$, let $A_\gamma$ be a twist collar and $X$ be a regular neighbourhood of $\gamma$. For simplicity, we will suppose that $A_\gamma$ is a connected component of $\nei(\tau.\gamma)\setminus \tau.\gamma$ which is suitable to serve as twist collar. Also, we will identify it with its diffeomorphic lift to $S^X$, which serves as a twist collar for $\gamma\subset S^X$, regarded as a twist curve of $\tau^X$. Call $\Hy^2\stackrel{p}{\rightarrow} S^X\stackrel{\hat p}{\rightarrow} S$ the (metric) covering maps between these surfaces. 

Extend $p$ to the metric universal cover $\bar\Hy^2_X\stackrel{p}{\rightarrow}\ol{S^X}$ of $\ol{S^X}$ ($\bar\Hy^2_X$ is homeomorphic to a closed strip). We set up a parametrization as specified by the following commutative diagram:

\begin{equation*}
\begindc{\commdiag}[260]
\obj(1,3)[strip]{$\R\times[-2,2]$}
\obj(4,3)[hyclosed]{$\bar\Hy^2_X$}
\obj(1,1)[cylinder]{$\sph^1\times[-2,2]$}
\obj(4,1)[annulusclosed]{$\ol{S^X}$}
\mor{strip}{hyclosed}{$\tilde\upsilon$}
\mor{cylinder}{annulusclosed}{$\upsilon$}
\mor{hyclosed}{annulusclosed}{$p$}[\atleft,\surjectivearrow]
\mor{strip}{cylinder}{$e$}[\atleft,\surjectivearrow]
\mor{strip}{annulusclosed}{$q$}[\atleft,\surjectivearrow]
\enddc
\end{equation*}

Here, $e(x,t)=(e^{2\pi i x},t)$ and $\upsilon,\tilde\upsilon$ are built to be two diffeomorphisms, which are not required to be orientation-preserving (see below for a determination of whether they preserve orientations), and are not unique; however, we specify some requests below. To start with, we require that $\tau^X.\gamma=\upsilon(\sph^1\times\{0\})$ and $A_\gamma=\upsilon\left(\sph^1\times(0,1)\right)$.

Define $q\coloneqq\upsilon\circ e$; $\tilde\gamma\coloneqq \tilde\upsilon(\R\times\{0\})$; and $\tilde A_\gamma\coloneqq\tilde\upsilon\left(\R\times(0,1)\right)$. Also, orient the two components of $\partial\ol{S^X}$ consistently with the $A_\gamma$-orientation on $\gamma$; their lifts are to be oriented accordingly.

We require that the map $q$ complies with the following request. Let $b$ be a branch of $\tau^X$ hitting $A_\gamma$, and consider a connected component $\tilde b$ of $q^{-1}(b)\cap (\R\times [0,1])$: we require that, for all $t\in [0,1]$, $\tilde b\cap (\R\times\{t\})$ is a unique point that we denote $\tilde b(t)$, and that, if $\tilde b:[0,1]\rightarrow \R\times [0,1]$ is parametrized so that $\tilde b(1)$ lies along $\R\times\{0\}$, then the second coordinate of $\tilde b$ is decreasing.

This request means that, in $\R\times[0,1]$, the preimages of branches hitting $A_\gamma$ are directed from the upper left to the lower right. The $A_\gamma$-orientation on $\gamma$ together with the above request on $q$, then, determine whether $\upsilon$ is orientation-preserving or -reversing. In the first case we say that the twist curve $\gamma$ has \nw{positive sign}, in the second one that it has \nw{negative} one. This property does not depend on the twist collar $A_\gamma$ chosen; even in case $\gamma$ is combed, and we take two different twist collar which intersect opposite sides of $\gamma$, the sign of $\gamma$ with respect to either collar is the same.

A natural map $\R\times[0,1]\rightarrow \bar A_\gamma$, where the closure of $A_\gamma$ is to be meant in $S$, is obtained as $\hat p\circ q$; but, with an abuse of notation, we will denote it again as $q$, taking care of avoiding misunderstandings.

A number $x\in\R$ will be called an \nw{upper obstacle} for $\tau^X$ if $(x,0)\in q^{-1}(v)$ where $v$ is a switch of $\tau^X$ along $\tau^X.\gamma$, and incident to a branch end hitting $A_\gamma$; a \nw{lower obstacle} if $(x,0)\in q^{-1}(v)$ where $v$ is a switch of $\tau^X$ along $\tau^X.\gamma$, and incident to a branch end avoiding $A_\gamma$ and favourable; a \nw{fake obstacle} if $(x,0)\in q^{-1}(v)$ where $v$ is a switch of $\tau^X$ along $\tau^X.\gamma$, and incident to a branch end avoiding $A_\gamma$ and adverse. 

A ramp $\rho$ in $\tau^X$ will be said to be \nw{hitting} or \nw{avoiding} $A_\gamma$, to be \nw{$A_\gamma$-favourable} or \nw{adverse} in accordance with the properties of the only branch end adjacent to $\tau^X.\gamma$ and traversed by $\rho$.
\end{rmk}

\begin{rmk}\label{rmk:sameorientationforarcs}
Let $\gamma$ be a twist curve for a generic almost track $\tau$, with a twist collar $A_\gamma$; and let $\alpha\in\cc\left(\nei(\gamma)\right)$. 

The decomposition of a train path realization of $\alpha$ into $\rho_1,\beta,\rho_2$ given in point \ref{itm:horizontalstretch} of Remark \ref{rmk:annulusinducedbasics} gets a further property in this case: since one of between $\rho_1,\rho_2$ (say $\rho_1$) terminates with a branch end hitting $A_\gamma$, the last branch end of $\rho_2$, which avoids $A_\gamma$ must be favourable: it has been already noted, indeed, that the two branch ends must give opposite orientations to $A_\gamma$. If $\alpha$ is oriented so that $\rho_1$ is its first segment, then the segment $\beta$ is swept according to the $A_\gamma$-orientation.
\end{rmk}

\begin{lemma}\label{lem:twistininduced}
Let $\tau$ be a generic, recurrent almost track on a surface $S$, $\gamma\in \cc(S)$ be a curve carried by $\tau$, and $X$ be a regular neighbourhood of $\gamma$. Then $\gamma$ is a twist curve for $\tau$ if and only if it is a curve carried by, and combed in, $\tau|X$.

More specifically, if $\gamma$ is a twist curve, then $\tau|X$ contains all branches of $\tau^X$ hitting $A_\gamma$ in $S^X$ (which biject naturally with the ones of $\tau$ hitting $A_\gamma$ in $S$) and branch ends which avoid $A_\gamma$ and are favourable. Moreover, for each pair $e,e'$ where $e$ is a branch end hitting $A_\gamma$ and $e'$ is one which avoids $A_\gamma$ and is favourable, there is an element of $V(\tau^X)$ which traverses both.

The branches avoiding $A_\gamma$ and adverse to it do not belong to $\tau|X$.
\end{lemma}
\begin{proof}
Recall points \ref{itm:embeddedcore} and \ref{itm:uniquecarrying} in Remark \ref{rmk:annulusinducedbasics}, in particular that $\gamma$ is embedded as a train path in $\tau^X$.

No matter whether $\gamma$ is a twist curve for $\tau$, if it is one for $\tau|X$ then it is actually combed: for in Remark \ref{rmk:sameorientationforarcs} it is shown that, if $e$ is a branch end traversed by some $\alpha\in\cc(X)$ and avoiding $A_\gamma$, then $e$ is favourable.

Note moreover the following behaviour. Let $e,e'$ be two branch ends sharing a switch, $v,v'$, respectively, with $\tau^X.\gamma$, located on opposite sides of $\tau^X.\gamma$, and such that the $e$- and the $e'$-orientations on $\gamma$ are opposite. One can always find an incoming ramp $\rho$ traversing $e$ and an outgoing one $\rho'$ traversing $e'$. The switches $v,v'$ will cut $\tau^X.\gamma$ into two segments, only one of which, $\beta$, has at its extremes two large branch ends. The train route obtained from the concatenation of $\rho,\beta,\rho'$ is then a properly embedded arc, hence it is a realization of some element of $V(\tau^X)$ which traverses both $e, e'$.

Suppose now that $\gamma$ is a twist curve with a twist collar $A_\gamma$: then it is a twist curve in $\tau^X$ too and, since it is embedded there, the branches it traverses in $\tau^X$ include a large one only if a favourable branch end $e'$ exists. Pick a branch end $e$ of $\tau^X$ hitting $A_\gamma$: the pair $e,e'$, however the two are chosen, complies with the hypotheses above; thus they are traversed by one same element of $V(\tau^X)$, and in particular they are part of $\tau|X$; so all branches hitting $A_\gamma$ and all favourable ones belong to $\tau|X$.

Let $\alpha\in V(\tau|X)$ be any of the wide arcs constructed in the way specified above: it certifies that $\cc(\tau^X)\not=\emptyset$ and therefore, by point \ref{itm:gammacarriedininduced} in Remark \ref{rmk:annulusinducedbasics}, that $\tau^X.\gamma\subset \tau|X$. Moreover $(\tau|X).\gamma$ must necessarily include a large branch of $\tau|X$, else it is impossible for the arc $\alpha$ to enter $(\tau|X).\gamma$ and leave it after travelling through some of its branches.

Clearly, there is a side of $\gamma$ such that all branch ends of $\tau|X$ sharing a switch with $(\tau|X).\gamma$ and approaching it from that side will give $\gamma$ the same orientation, so $\gamma$ is a twist curve in $\tau|X$. Due to the argument at the beginning of this proof, it is combed; equivalently, no branch ends avoiding $A_\gamma$ and adverse belong to $\tau|X$. One implication of the first part the lemma's statement is thus proved, together with the part of the statement which specifies which branch ends around $\tau^X.\gamma$ belong to $\tau|X$.

We only have left to prove the remaining implication. Suppose that $\gamma$ is combed in $\tau|X$ --- in particular it is carried, thus so it is by $\tau^X$ and by $\tau$. The presence of a large branch of $\tau^X$ within $\tau^X.\gamma$ is proved with the same argument as in the paragraph above; hence there must be a large branch of $\tau$ within $\tau.\gamma$. 

We claim now that $\gamma$ is wide in $\tau$. Let $\ul\gamma$ be a carried realization: if it is not wide then --- regardless of the orientation that we put on it --- there are two branches $b_1,b_2$ of $\tau$, traversed at least twice, such that in $R_{b_1}$ one segment $s_1$ of $\ul\gamma$ sees another segment $t_1$ to its left, and in $R_{b_2}$ one segment $s_2$ of $\ul\gamma$ sees another segment $t_2$ to its right. Let $\ul{\hat\gamma}$ be the homeomorphic lift of $\ul\gamma$ to $S^X$. Call $\hat s_1,\hat s_2$ the lifts of $s_1,s_2$ which are contained in $\ul{\hat\gamma}$, and $\hat t_1,\hat t_2$ the lifts of $t_1,t_2$ which are located in the same branch rectangles as $\hat s_1,\hat s_2$. The segments $\hat t_1,\hat t_2$ belong to other lifts $\ul{\hat\gamma}_1,\ul{\hat\gamma}_2$ of $\ul\gamma$; they do not cross $\ul{\hat\gamma}$. So $\ul{\hat\gamma}_1$ must traverse two branch ends $e_{11},e_{12}$, each sharing a switch with $\tau^X.\gamma$, located on the same side of $\gamma$, and impressing on it opposite orientations. Similarly $\ul{\hat\gamma}_2$ must traverse branch ends $e_{21},e_{22}$ with the same properties. Suppose that the $e_{11},e_{21}$-orientations on $\gamma$ are opposite: then, by the observation at the beginning of the present proof, there is an element $\alpha_j\in V(\tau^X)$ traversing both $e_{1j},e_{2j}$ for $j=1,2$. Thus $e_{11},e_{12}$ both belong to $\tau|X$ contradicting the assumption that $\gamma$ is combed there.

Now we prove that, among the wide collars of $\gamma$ in $\tau^X$, there must be at least one twist collar (meaning, one such that all branch ends intersecting it give the same orientation to $\gamma$). Suppose not: then, on each side of $\gamma$ in $\tau^X$, we can find a pair of branch ends sharing a switch with $\tau^X.\gamma$ and impressing opposite orientations on $\gamma$. Call $e_{11},e_{12}$ the ones on one given side and $e_{21},e_{22}$ the others on the other one. The same contradiction as above occurs.

Note that a twist collar $A_\gamma$ for $\gamma$ in $\tau^X$ projects to a wide collar for $\gamma$ in $\tau$; and it will be a twist collar, too. This completes the proof.
\end{proof}

The above lemma is the reason why we have been able to state Theorem \ref{thm:mmsstructure} without problems due to having employed a definition of accessible interval for annular subsurfaces which is different from the one given in \cite{mms}. As for the first statement: by looking at the original statement, we see that our definition comprises an interval larger than the original one, and this does not affect the statement's validity.

The second statement of Theorem \ref{thm:mmsstructure} may instead seem stronger than what is proved in the original paper. But it is not in reality, because the proof only uses the property that the induced track under exam is combed.

\begin{lemma}\label{lem:twistcurvetrees}
Let $\gamma$ be a twist curve for a generic, recurrent almost track $\tau$, and let $X$ be a regular neighbourhood of $\gamma$. Then $\ol{\tau|X\setminus (\tau|X).\gamma}$ (the closure is meant in $\ol{S^X}$ here) is a disjoint union of trees: each of them intersects $(\tau|X).\gamma$ in a single point which serves as a root, and consists only of mixed branches of $\tau|X$. 

In particular, every train path between two endpoints of such a component is a ramp. If such a ramp $\rho$ is outgoing, then $\rho$ enters each traversed branch from its small end.
\end{lemma}
\begin{proof}
\step{1} given any two distinct outgoing ramps $\rho_1,\rho_2$ in $\tau|X$, their images either are disjoint or their intersection is a bounded, initial sub-train path of both.

Suppose that the images of the two ramps are not disjoint. Then, by Lemma \ref{lem:twistininduced}, either they both hit $A_\gamma$ or both avoid it and are favourable. Suppose the first alternative holds, the other being entirely similar. As a consequence of point \ref{itm:windaboutgamma} in Remark \ref{rmk:annulusinducedbasics}, both $\rho_1,\rho_2$ are embedded train paths. Even if $\rho_1,\rho_2$ do not begin at the same switch along $\tau^X.\gamma$, then (up to reversing indices) it is possible to have an embedded train path $\rho'_1$ which begins at the same switch as $\rho_2$, follows part of $(\tau|X).\gamma$ and then continues as $\rho_1$. The claim is proved if we show that the intersection between the images of $\rho'_1$ and $\rho_2$ is connected.

Suppose it is not. Then necessarily $\rho_1,\rho_2$ together bound a topological disc $B$, with 1 or 2 cusps along its boundary. Since $\tau|X\subseteq \tau^X$, $B$ is a union, along their respective boundaries, of regions as in Remarks \ref{rmk:idx_of_nei_diff} and \ref{rmk:negativeindexincover}, so it must have negative index: and this is a contradiction.

\step{2} the bounded branches of $\tau|X$ not belonging to $(\tau|X).\gamma$ are all mixed.

An immediate consequence of the previous step is that no branch in $\tau|X$ is traversed by distinct ramps in opposite directions.

If $b\in\br(\tau|X)$ is a large branch, not contained in $(\tau|X).\gamma$, then fix $\rho$ an outgoing ramp which traverses $\rho$, and let $\rho'$ be any other ramp which also does so. Then, since $b$ belongs to the intersection between the images of $\rho$ and $\rho'$, $\rho$ and $\rho'$ are not only traversing $b$ in the same direction but they are also coming from the same branch end adjacent to $b$. This means that one of the branch ends adjacent to $b$, is actually not traversed by any outgoing ramp in $\tau|X$: and this contradicts the definition itself of $\tau|X$.

If $b\in\br(\tau|X)$ is a small branch, not contained in $(\tau|X).\gamma$, then fix $\rho$ an outgoing ramp which traverses $\rho$, and let $e$ be the branch end $\rho$ traverses just after leaving $b$. Let $\rho'$ be an outgoing ramp traversing $e$ but not $b$. Then $\rho,\rho'$ traverse $e$ in the same direction, and their routes must coincide up to the point when they enter $e$. But this would require $\rho'$ to traverse $b$, too: and this is our contradiction.

\step{3} $\ol{\tau|X\setminus (\tau|X).\gamma}$ is a graph with no cycles.

If $C\subseteq \ol{\tau|X\setminus (\tau|X).\gamma}$ is a cycle, it consists entirely of mixed branches in $\tau|X$. But then it is smooth i.e. it represents a curve carried by $\tau|X$. This curve cannot be nullhomotopic, so it is necessarily homotopic to $\gamma$. But then, $C$ and $(\tau|X).\gamma$ project to $S$ and give two distinct carried realizations of $\gamma$, a contradiction.

Note that any outgoing ramp $\rho$ in $\tau|X$ will enter its first branch from a small end. Since the branch is mixed, $\rho$ leaves it through its large end; necessarily, $\rho$ shall traverse all its branches in a similar fashion. This ends the proof.
\end{proof}

\subsection{Splitting sequences seen at a twist curve}
\begin{defin}\label{def:twistmoves}
Let $\tau$ be a generic almost track on $S$ and let $\gamma$ be a twist curve for $\tau$. A \nw{twist split} about $\gamma$ is a parity split of a large $b\in\br(\tau)$ included in $\tau.\gamma$ and sharing one switch with a branch end hitting $A_\gamma$, where the parity is chosen so that $\gamma$ is still carried and stays a twist curve after the split.

A \nw{twist slide} about $\gamma$ is a slide as the one explicated by the second drawing in Figure \ref{fig:ttcombing}, provided that the horizontal line consists of branches in $\tau.\gamma$; and either the collar $A_\gamma$ is under such line and the slide is represented by the rightward arrow; or the collar $A_\gamma$ is above such line and the slide is represented by the leftward arrow. Note that the inverse move of a twist slide is not a twist slide.

A slide, split, or wide split on $\tau$ is \nw{far from $\gamma$} if there is a regular neighbourhood $\nei(\tau.\gamma)$ such that $\tau\cap\nei(\tau.\gamma)$ is not altered by the move. 

A move that is neither a twist one nor a far one from $\gamma$, but keeps $\gamma$ a twist curve, is called a \nw{spurious} move. A spurious (wide) split affecting a large branch traversed twice by $\gamma$ will be called \nw{bispurious}.

A \nw{wide twist split} about $\gamma$ is a wide split corresponding to a sequence of elementary moves where the only split is a twist split about $\gamma$. Similar definitions hold for a \nw{wide far}, \nw{wide spurious} or \nw{wide bispurious} split. Note that a wide split is far from $\gamma$ if and only if $\nei(\tau.\gamma)$ is not altered by the wide split as a whole.

We will say that a splitting sequence $\bm\tau=(\tau_j)_{j=0}^k$ has \nw{twist nature} about $\gamma$ if it consists only of twist splits and twist slides (\emph{not} wide twist splits) about $\gamma$.
\end{defin}

Note that the property of being a twist curve is never altered by a slide. A twist move always involves a branch $b$ along $\tau.\gamma$ with exactly one switch shared with a branch end hitting $A_\gamma$. Informally, one may view a twist move as moving this latter branch end forward along $\tau.\gamma$, according to the $A_\gamma$-orientation, so that its endpoint moves past the other switch of $b$. This idea is exploited with the introduction of \emph{modelling functions} for sequences of twist nature, which will be explained in Remark \ref{rmk:twistnaturemodelling}. However, note that the definition of twist nature is not easily adapted to a wide split setting: when realizing a wide twist split as a sequence of elementary moves, in particular, it may be necessary to employ slides which are not twist.

\begin{rmk}\label{rmk:spurious}
Let $\gamma$ be a twist curve for a generic almost track $\tau$. Suppose that a splitting arc $\alpha$ or a zipper $\kappa$ describe a move that leaves $\gamma$ carried. Then how is $\alpha$, or $\kappa$, placed with respect to $\tau.\gamma$? And with respect to a carried realization $\ul\gamma$ of $\gamma$? The following considerations are true also for `standard' (non-wide) splits, as a special case.

If the wide split along $\alpha$ is far from $\gamma$, then $\alpha$ is disjoint from both $\tau.\gamma$ and $\ul\gamma$. If the wide split is twist, then, however one picks a lift $\hat\alpha$ of $\alpha$ to $S^X$, its extremes lie in distinct components of $S^X\setminus\tau^X.\gamma$.

If the given wide split is spurious, then $\alpha$ and $\gamma$ admit disjoint carried realizations in $\nei(\tau)$. The examples of spurious and bispurious splitting arcs given in Figure \ref{fig:twistcurve}, left, should be able to communicate intuitively why this is true. Note that, when $\gamma$ is an embedded in $\tau$, $\ul\gamma$ can be taken to be a train path realization of $\gamma$. When it is not, there is no guarantee that $\alpha$ can be realized disjointly from $\tau.\gamma$: it cannot if and only if the wide spurious split is actually bispurious.

As for the case of a zipper $\kappa$, similar consideration apply. Namely, if all elementary moves specified by unzipping $\kappa$ are far, then $\kappa$ is disjoint from both $\tau.\gamma$ and $\ul\gamma$. More generally, $\kappa$ can always be made disjoint from $\ul\gamma$ (not from $\tau.\gamma$). Also, if $A_\gamma$ is a twist collar for $\gamma$ in $\tau$, the following alternative can be assumed: either the last point of $\kappa$ lies along $\tau.\gamma$, or the image of $\kappa$ does not intersect $A_\gamma$. In other words, the unzip $\kappa$ will never create a switch inside $A_\gamma$. This assumption does not pose any actual limitation as to the possible result of an unzip: the only restriction following from it concerns the realization of the result of the unzip within its isotopy class.

Note that a spurious split or wide split is possible only if $\gamma$ is not combed; and that the splitting arc corresponding to a spurious split must traverse one of the large branches in $\tau.\gamma$.
\end{rmk}

\begin{rmk}\label{rmk:consistentcollars}
Suppose that $\gamma$ is a twist curve in a generic almost track $\tau$ with a twist collar $A_\gamma$.

Let $\tau$ undergo a (wide) split which turns it into $\tau'$, but keeps $\gamma$ a twist curve: then, once a carried realization $\tau'\hookrightarrow \nei(\tau)$ is fixed, let $c:\tau'\rightarrow \tau$ be the restriction to $\tau'$ of the tie collapse $c_\tau:\bar\nei(\tau)\rightarrow \tau$. Then $c(\tau'.\gamma)=\tau.\gamma$; moreover one may find an extension $c:S\rightarrow S$, isotopic to $\mathrm{id}_S$, and a twist collar $A'_\gamma$ for $\gamma$ in $\tau'$, such that $A_\gamma= c(A'_\gamma)$. A more concrete way of choosing twist collars, consistently with a splitting sequence, will be described in Remark \ref{rmk:permanenceconventions}.

Meanwhile we give the following definition: let $T$ be a subset of the family of almost tracks fully carried by $\tau$, with the property that $\gamma$ is a twist curve for each $\sigma\in T$ (e.g. $T$ may be the family of the entries of a splitting sequence preserving $\gamma$). For each $\sigma\in T$, let $A_\gamma(\sigma)$ be a twist collar for $\gamma$. We say that the twist collars $\{A_\gamma(\sigma)\mid\sigma\in T\}$ (more appropriately, the pairs $\{(\sigma,A_\gamma(\sigma))\mid \sigma\in T\}$) form an \nw{$A_\gamma$-family} if, for each pair $\sigma,\sigma'\in T$ such that $\sigma$ carries $\sigma'$, the two twist collars $A_\gamma,A_\gamma'$ are related in the way described above.

The $A_\gamma(\sigma)$-orientations for $\gamma$ in the respective almost tracks are all the same, and we may as well just call then the $A_\gamma$-orientation. Similarly we may speak of a branch end of $\sigma$ which hits or avoids $A_\gamma$, when we actually mean that it hits or avoids $A_\gamma(\sigma)$.
\end{rmk}

\begin{lemma}\label{lem:twistcurvebasics}
Given a generic splitting sequence of almost tracks $\bm\tau$ and a curve $\gamma$, let $\Sigma$ be the set of indices $j$ such that $\gamma$ is a twist curve for $\tau_j$: then $\Sigma$ is an interval. If all entries in the splitting sequence are recurrent, then $\gamma$ can only cease to be a twist curve by ceasing to be carried.

Given a fixed $J\in\Sigma$ and $A_\gamma$ a twist collar for $\tau_J$, there is an $A_\gamma$-family of twist collars for $\bm\tau(J,\max \Sigma)$.
\end{lemma}

This means that, along a \emph{recurrent} splitting sequence, an element of $\cc(S)$ evolves through a subsequence of these stages:
\begin{enumerate}[label=\alph*.]
\item carried, not wide;
\item wide, not twist;
\item twist, not combed;
\item combed;
\item not carried.
\end{enumerate}

\begin{proof}
Let $j_0$ be the lowest of the indices $j$ such that $\gamma$ is a twist curve for $\tau_j$, with a twist collar $A_\gamma(j_0)$, and let $j_1$ be the highest index such that $\gamma$ is carried by $\tau_{j_1}$. Then $\gamma$ is carried by all $\bm\tau(j_0,j_1)$ because the set of carried curves can only shrink along a splitting sequence (Remark \ref{rmk:decreasingmeasures}). Also, $\gamma$ is wide in this sequence, as a family of wide collars $A_\gamma(j)$ for $\tau_j.\gamma$ ($j\in[j_0,j_1]$) can be defined recursively, with the condition that a suitable homotopy equivalence $S\rightarrow S$ mapping $\tau_j$ to $\tau_{j-1}$, as in Remark \ref{rmk:consistentcollars}, will also map $A_\gamma(j)$ to $A_\gamma(j-1)$.

Suppose that $\gamma$ is a twist curve in $\tau_j$ for a fixed index $j_0\leq j< j_1$, with $A_\gamma(j)$ a twist collar; so $\gamma$ is carried by $\tau_{j+1}$. If the elementary move on $\tau_j$ is a slide or there is neighbourhood of $\tau_j.\gamma$ not affected by the move, then clearly $\gamma$ is a twist curve in $\tau_{j+1}$ with $A_\gamma(j+1)$ a twist collar. So suppose that the elementary move operated on $\tau_j$ is a split which affects the train track close to $\tau_j.\gamma$. Then the branch being split is traversed by $\gamma$; the split can be considered to be a wide split along a splitting arc $\alpha$ that traverses one branch only.

Since $\gamma$ is a twist curve, at least one of the ends of $\alpha$ must lie on the side of $\tau_j.\gamma$ opposite to $A_\gamma(j)$. If exactly one does, then the only parity that keeps $\gamma$ carried after the split is the one specifying a twist split. If both endpoints lie opposite $A_\gamma(j)$, then any splitting parity keeps $\gamma$ carried, and the branch ends of $\tau_{j+1}$ hitting $A_\gamma(j+1)$ all give $\gamma$ the same orientation, similarly as before. So the only way $\gamma$ may fail being a twist curve in this last case is the absence of any large branch of $\tau_{j+1}$ among the ones traversed by $\gamma$. But if this is the case, then no further split may affect all neighbourhoods of $\tau_{j'}.\gamma$ for $j'\geq j$: therefore $\gamma$ will not return a twist curve at any later stage in the splitting sequence $\bm\tau$ --- but it may still be carried.

This proves that $\Sigma$, as defined in the statement, is an interval; also, $\{A_\gamma(j)\}_{j\in\Sigma}$ is an $A_\gamma(0)$-family. Moreover, if $\bm\tau$ is recurrent, the last kind of split described above, which keeps $\gamma$ carried but does not keep it a twist curve, cannot occur. In that event, indeed, $\tau_{j+1}$ is not recurrent, because any train path which traverses a branch end $e$ hitting $A_\gamma(j+1)$ is forced to remain within $\tau_{j+1}.\gamma$ and cannot get back to traversing $e$. So, in case $\bm\tau$ is recurrent, $\Sigma=[j_0,j_1]$ i.e. $\gamma$ ceases being a twist curve by ceasing being carried.

As for the last claim the lemma's statement, just restrict the arguments in this proof to the subsequence $\bm\tau(J,\max \Sigma)$: let $A_\gamma(J)\coloneqq A_\gamma$ given in the statement, and recursively construct the $A_\gamma$-family of twist collars.
\end{proof}

\begin{rmk}\label{rmk:permanenceconventions}
In the present Remark we describe how to choose consistent representatives of train tracks and twist collars in a splitting sequence such that a fixed curve is a twist curve for all tracks in the sequence.

Train tracks and almost tracks are are generally understood to be regarded up to isotopy, but it is convenient to fix some conventions concerning the way we regard almost tracks to change along a generic (possibly wide) splitting sequence $\bm\tau=(\tau_j)_{j=0}^N$, when our focus is on a curve $\gamma$ which stays a twist curve all along the sequence. This step is necessary because most of this section analyses how to discern the effect of twist moves from the other ones. Since all the following conventions are only a limitation when choosing an almost track within its isotopy class, they will be used in the proofs, but do not affect the validity of the statements. For a matter of convenience, we allow the sequence $\bm\tau$ to include moves which consist of isotopies only.

For all $0\leq j\leq N$, we will use the notation $A_\gamma(j)$ to mean a twist collar for $\gamma$ in $\tau_j$ such that $\{A_\gamma(j)\}_{j=0}^N$ is an $A_\gamma(0)$-family; and we denote $q_j$ the map $q:\R\times[-2,2]\rightarrow S^X$ (or $\rightarrow S$) built as specified in Remark \ref{rmk:twistparam}, but with specific reference to the almost track $\tau_j$.

It is convenient to communicate the idea behind the adopted conventions before they are described in detail. For each entry in the sequence $\left(q_j^{-1}(\tau_j^X)\right)_j$, i.e. the sequence $\bm\tau$ lifted to $\R\times[-2,2]$, the following entry $q_{j+1}^{-1}(\tau_{j+1}^X)$ shall not only include the equator $\R\times\{0\}$, but also appear naturally as a carried realization in $q_j^{-1}\left(\nei(\tau_j^X)\right)$.

Unfortunately it is not possible to keep the same map $q$ for all entries in the sequence, because one has to take into account the change of $\tau_j.\gamma$ under a bispurious split, for instance (see Figure \ref{fig:twistcurve}, right): in that case, adjusting $\tau_j$ within its own isotopy class will not suffice for us to keep the same map $q$ for all entries, and expect it to behave in compliance with Remark \ref{rmk:twistparam}. And the same problem arises when, albeit $\tau_j.\gamma,\tau_{j+1}.\gamma$ are isotopic, the isotopy between the two does not keep each point anchored along its tie in $\nei(\tau_j)$.

\begin{figure}
\centering
\def\svgwidth{\textwidth}
\input{permanenceconventions.pdf_tex}
\caption{\label{fig:permanenceconventions} An illustration of the conventions set up in Remark \ref{rmk:permanenceconventions}, for a bispurious parity split, and in particular of its effects seen in $\R\times[-2,2]$. The pictures show only the changes for a region of $q_j^{-1}(\tau_j)$ and are not meant to be accurate. Three pieces of $q_j^{-1}(A_\gamma(j))$ and of $q_{j+1}^{-1}(A_\gamma(j+1))$ are shown in light grey. The split in $S$ is equivalent to unzipping a zipper, that lifts to an infinite family including two ones along $\R\times\{0\}$: call $\kappa$ one of these two. The main purpose of the set of established conventions is to make sure that, along a splitting sequence, for all $j$, $\tau_j.\gamma=q_j(\R\times\{0\})$ and $A_\gamma(j)=q_j\left(\R\times(0,1)\right)$ while each elementary move leaves the new pretrack naturally carried by the old one. Note indeed that, in these pictures, as $\kappa$ is unzipped, the result of the operation is not embedded as Definition \ref{def:zipper} would describe: it is slightly altered with an isotopy so that $\R\times\{0\}$ stays a lift for $\tau_{j+1}.\gamma$. No particular care is needed for any other unzips, far from $\R\times\{0\}$. The map $\E=\E_j$ defined in Remark \ref{rmk:permanenceconventions} lifts, via $q_j,q_{j+1}$, to a map of $\R\times[-2,2]$ which crushes the dark grey regions to a single line, and is a diffeomorphism on the complement of the grey regions (including the ones which are not drawn). If the split were only spurious, no region would be crushed: $\E_j$ would give a diffeomorphism of $\R\times[-2,2]$ --- and we may assume that it would give the identity. Finally, in these pictures two lifts of ramps, $\rho_j,\rho_{j+1}$, have been highlighted with a thicker line: they begin at the same point along $\R\times\{-2\}$ and are both $A_\gamma$-adverse. Note that the extremity $(x_j,0)$ of $\rho_j$ along $\R\times\{0\}$ --- here: a fake obstacle --- moves along $\R\times\{0\}$ in the direction suggested by the last branch end $\rho_j$ traverses; and reaches $(x_{j+1},0)$, which is the last point of the zipper $\kappa$. This is because any lift of $c_{\tau_j^X}\circ\rho_{j+1}$ via $q_j$ includes a segment along $\R\times\{0\}$, which is not part of $\rho_j$. Should $\rho'_{j+1}$ be such that $c_{\tau_j^X}\circ\rho'_{j+1}$ does not include a segment as above, the corresponding $\rho'_j$ would end at the same point of $\R\times\{0\}$ as $\rho'_{j+1}$. Note that, if a lift of an incoming ramp for $\tau_{j+1}^X$ begins at a given point $a$ of $\R\times\{\pm 2\}$, then necessarily $\tau_j^X$ also has an incoming ramp with one of its lifts beginning at $a$; but the inverse implication does not hold.
}
\end{figure}

We start describing the conventions from the following basic case (see also Figure \ref{fig:permanenceconventions}): let $\tau$ be an almost track where $\gamma$ is a twist curve with twist collar $A_\gamma$; and let $\tau''$, which carries $\gamma$, be obtained via the unzip of $\tau$ along a zipper $\kappa: [-\epsilon, t)\rightarrow \bar\nei(\tau)$ such that $\kappa_P$ is embedded.

Recall the notation used in Definition \ref{def:zipper}. Isotope $\tau''$ so that, if a switch lies in the set called $C$ there, then there is a switch of $\tau$ lying along the same tie of $\nei(\tau)$.

Let $R_\kappa$ be the union of the ties of $\nei(\tau)$ intersecting $\kappa$. Consider the restriction $c_\tau|_{\tau''.\gamma}$ of the tie collapse $c_\tau$ (see \S \ref{sub:traintrackdefin}), and extend it to a continuous map $\E:(S,\tau''.\gamma)\rightarrow (S,\tau.\gamma)$ with the following properties:
\begin{itemize}
\item $\E$ is surjective;
\item $\E|_{S\setminus R_\kappa}=\mathrm{id}_{S\setminus R_\kappa}$;
\item $\E$ maps each tie segment in $\bar\nei(\tau)$ to a segment or point of the same tie;
\item if $C\cap\tau_{j+1}.\gamma$ consists of exactly two \emph{smooth} components\footnote{Either two connected components which are smooth, or a single connected component consisting of two smooth paths joined at a cusp. It is exactly when one of these two scenarios occurs that we cannot keep the parametrization $q$ constant along the sequence $\bm\tau$: and in this case $\E$ shall describe the `crush' of the two pieces of $C$ to a single one.} note that, with the conditions set up so far, $\E(\bar\nei(\kappa))=\mathrm{im}(\kappa_P)$: then take $\E$ to be a diffeomorphism between $(S\setminus \bar\nei(\kappa),(\tau''.\gamma)\setminus C)$ and $(S\setminus\mathrm{im}(\kappa_P),(\tau.\gamma) \setminus\mathrm{im}(\kappa_P))$;
\item if not, then take $\E$ to be a diffeomorphism $(S,\tau''.\gamma)\rightarrow (S,\tau.\gamma)$.
\end{itemize}

Another equally basic case is the one of two almost tracks $\tau,\tau''$, both carrying $\gamma$, with $\tau''$ obtained from $\tau$ via a central split, along a branch $b$. Identify $R_b$ with its image. Also, identify $\tau''$ with a carried realization of it in $\nei_0(\tau)$, with the property that $\tau\setminus R_b=\tau''\setminus R_b$. Similarly as above, extend $c_\tau|_{\tau''.\gamma}$ to a map $\E$ with the same properties as above, except that we replace $R_\kappa$ with $R_b$, define $C$ as the union of the two `copies' of $b$ produced with the central split, and replace $\nei(\kappa)$ with $K$ the union of the open tie segments in $R_b$ each delimited by two points in $C$.

In all the above described scenarios, call $\F$ the inverse \emph{diffeomorphism} of $\E$ --- its domain and image will depend on the particular case as described above. The map $\E$ gives rise to a unique lift $\hat\E:S^X\rightarrow S^X$, via the covering map $\hat p$, with the property that it extends to the identity map on $\partial\ol{S^X}$.

The twist collar for $\gamma$ in $\tau''$ will be supposed to be $A_\gamma''= \F(A_\gamma)$: this is possible if the map $\E$ is realized appropriately, and/or $\nei(\tau'')$ is chosen wisely (recall indeed that we want $A_\gamma''$ to be one of the components of $\nei(\tau''.\gamma)\setminus(\tau''.\gamma)$. Given the parametrization $q:\R\times [-2,2]\rightarrow S^X$ for $\tau^X$, the corresponding parametrization $q''$ for ${\tau''}^X$ shall be taken to comply with the equality $q=\hat\E\circ q''$.

If all the elementary moves produced by unzipping $\kappa$ are far from $\gamma$, it is possible to take $\E=\F=\mathrm{id}_S$, $A_\gamma''=A_\gamma$, $q=q''$, and suppose that $\tau\cap\nei(\tau.\gamma)=\tau''\cap\nei(\tau.\gamma)$.

If $\kappa$ decomposes into two shorter zippers $\kappa_1,\kappa_1$, then the respective maps relate via $\E(\kappa)=\E(\kappa_2)\circ \E(\kappa_1)$ and $\F(\kappa)=\F(\kappa_1)\circ \F(\kappa_2)$. This allows us to give sense to the maps $\E,\F$ and to $A_\gamma''$ even if $\kappa_P$ is not an embedded path.

Back to the case of a (possibly wide) splitting sequence $\bm\tau$, each move can be regarded as the combination of one or more unzips. Remark \ref{rmk:generic_move_as_unzip}, and Remark \ref{def:multiplesplit} for wide splitting sequences, give some guidelines to convert a splitting sequence into a sequence of unzips and central splits, albeit not in a unique way. Exceptionally we will also allow sequences with trivial moves, i.e. unzips that only result in isotopies.

For each index $j$, then, a continuous surjection ${\mathcal E}_j:(S,\tau_{j+1}.\gamma)\rightarrow(S,\tau_j.\gamma)$ is defined by composing the maps $\E$ related with each of the unzips and central splits used to perform the move turning $\tau_j$ into $\tau_{j+1}$. Let $\F_j$ be its inverse, defined as the composition of the maps $\F$ seen above where they are defined.

If $b$ is the large branch, mixed branch or carried realization of the splitting arc involved in the move, then $\E_j$, restricted to $S\setminus \E_j^{-1}(b)$, is a diffeomorphism with its image; so $S\setminus b$ is always included in the domain of $\F_j$. Every time the elementary move or wide split is far from $\gamma$, $\E_j=\F_j=\mathrm{id}_S$. 

Denote $\hat{\E}_j:S^X\rightarrow S^X$ the lift of $\E_j$ as above. As a direct consequence of what established, the twist collars correspond under $A_\gamma(j+1)=\F_j\left(A_\gamma(j)\right)$, and in particular $\{A_\gamma(j)\}_{J=0}^N$ is an $A_\gamma(0)$-family. Meanwhile, the parametrization maps are be subject to the condition $q_j=\hat\E_j\circ q_{j+1}$. If the elementary move/wide split between $\tau_j$ and $\tau_{j+1}$ is far from $\gamma$, then $\tau_j\cap\nei(\tau_j.\gamma)=\tau_{j+1}\cap\nei(\tau_j.\gamma)$.

Remark \ref{rmk:limitsetconsistency} will be used multiple times. In particular, combined with the above conventions, it gives a constraint on the following construction. Fix $i<j$, and let $\rho_j:(-\infty,0]\rightarrow \tau_j^X$ be any incoming ramp; then there is a unique incoming ramp $\rho_i:(-\infty,0]\rightarrow \tau_i^X$ coming from the same point of $\partial\ol{S^X}$ and of the same kind as $\rho_j$ (by `kind' we mean hitting, favourable, or adverse; for the uniqueness, see point \ref{itm:dataforramp} in Remark \ref{rmk:annulusinducedbasics}). Actually, as a consequence of the conventions established here, $\rho_i$ is an initial segment of (a reparametrization of) $c_{\tau_i^X} \circ \rho_j$. Note that this last expression would not even be well defined, if a carried realization of $\tau_j$ in $\nei(\tau_i)$ were not fixed as we have done in the present Remark.

So, let $\hat\rho_j: (-\infty,0]\rightarrow \R\times[-2,2]$ be any lift of $\rho_j$ via $q_j$: it begins at a point $a\in\R\times\{-2,2\}$ and ends at an obstacle $x_j\in \R\times\{0\}$. There will be a connected component of $q_i^{-1}(\mathrm{im}(\rho_i))$ starting at the same point $a$; it ends at an obstacle $x_i$ for $\tau_i^X$, of the same kind (upper, lower, or fake) as $x_j$ for $\tau_j^X$.

The constraint, then, is that $x_j\geq x_i$ if they are upper or fake obstacles i.e. if $\rho_i,\rho_j$ are ramps hitting $A_\gamma$ or $A_\gamma$-adverse; and $x_j\leq x_i$ if they are lower obstacles i.e. if $\rho_i,\rho_j$ are ramps avoiding $A_\gamma$ and favourable. The inequalities are strict exactly when the image of $c_{\tau_i} \circ \rho_j$ has more than one point along $\tau_i^X.\gamma$, implying that one needs to trim it in order to get $\rho_j$.

Figure \ref{fig:permanenceconventions} summarizes part of the ideas behind these conventions. Another, simpler and more restrictive convention, which concerns sequences of twist nature only, will be described in Remark \ref{rmk:twistnaturemodelling}, and it clashes with parts of the assumptions made above. Therefore in the rest of this section, whenever necessary, we will clarify which of the two conventions we are about to use.
\end{rmk}

\begin{defin}
A \nw{twist modelling function} is a smooth map $h:\R\times[0,1]\rightarrow \R$ with the following properties. For all $x,t$, it holds that
$$
h(x+2\pi,t)=h(x,t)+2\pi\text{; }\frac{\partial}{\partial x}h(x,t)>0 \text{ and }\frac{\partial}{\partial t} h(x,t)\leq 0.$$
Moreover there exists and $\epsilon>0$ such that, for all $x$ and for $1-\epsilon \leq t<1$, the map satisfies $h(x,t)=x$.
\end{defin}

\begin{rmk}\label{rmk:twistnaturemodelling}
In the present Remark we describe how to choose consistent representatives of train tracks in splitting sequences of twist nature.

Consider a generic splitting sequence of almost tracks $\bm\tau=(\tau_j)_{j=0}^N$, of twist nature about a twist curve $\gamma$, and let $X$ be a regular neighbourhood of the latter in $S$. We use here the notations given in Remark \ref{rmk:twistparam}. 

As announced in Remark \ref{rmk:permanenceconventions} above, one may apply a different convention from the one seen there, when it comes to find a concrete realization of $\tau''$ obtained by applying a twist split or a twist slide on an almost track $\tau$, about a twist curve $\gamma$. Under this restriction, indeed, it is always possible to make $\tau.\gamma$ and $\tau''.\gamma$ coincide under an isotopy that keeps each point along its tie.

If $b\in\br(\tau)$, with $b\subseteq \tau.\gamma$, is the large/mixed branch that is being split/slid, then exactly one of its endpoints is adjacent to a branch end $e$ hitting $A_\gamma$: as it has been already noted, the result of the elementary move may be visualized as a shift of this branch end beyond the other endpoint of $b$.

In order to attain this visualization, the move may be realized with an unzip along a zipper $\kappa:[-\epsilon, t]\rightarrow \bar\nei(\tau)$ such that $\kappa\left([0,t)\right)$ is contained in $A_\gamma$. In this scenario, the map $\E$ defined in Remark \ref{rmk:permanenceconventions} is a diffeomorphism (isotopic to $\mathrm{id}_S$). But in this set of conventions, rather than keeping track of the map $\E$, the new track $\tau''$ is immediately replaced with $\F(\tau'')$. This is convenient as $A_\gamma$ is a twist collar for $\gamma$ in $\tau''$, too; and $\tau\setminus A_\gamma=\tau''\setminus A_\gamma$.

In other words, it becomes useless to appeal to $A_\gamma$-families, because there is no need to adapt the twist collar $A_\gamma$. Similarly, in this context the maps $\tilde\upsilon$, $\upsilon$, $q$ of Remark \ref{rmk:twistparam} are taken to be the same when applying the construction to $\tau$ or to $\tau''$. 

One may apply this convention for an entire sequence of twist nature $\bm\tau$. However, rather than consider each elementary move as the result of a \emph{single} unzip, it is better to keep the model less rigid: an elementary move may as well be the result of unzipping \emph{more} than a zipper as described above, posing the condition that all unzips but one result into isotopies of the almost track. This makes sure that constructions like the one in Lemma \ref{lem:functiongivestwist} work properly.

%Once a family of realizations $\bm\tau^{twist}$ for the entries of $\bm\tau$ has been defined, in compliance with this convention, it may be easily converted to a family of realizations $\bm\tau^{general}$ as in Remark \ref{rmk:permanenceconventions}. It suffices, indeed, \emph{not} to apply the isotopies which keep the sets $\tau_j.\gamma$ fixed as $j$ varies, and keep trace of the maps $\E_j$ instead. This means also that the covering maps $q_j$ for entries of $\bm\tau^{general}$ will be distinct: this compensates the fact that, when lifting the sequences $\bm\tau^{twist}$ to $\R\times[-2,2]$ via $q$, and each entry of $\bm\tau^{general}$ via the respective $q_j$, the two sequences of pretracks in $\R\times[-2,2]$ coincide.
The argument concerning endpoints of ramps, developed in Remark \ref{rmk:permanenceconventions}, applies in this set of conventions, too; but the twist nature of the sequence limits the variety of possible behaviours. Given an almost track $\tau$ where we unzip a zipper $\kappa$ as specified above, there is exactly one coset $x+2\pi\mathbb Z\subset \R$, where $x$ is an \emph{upper} obstacle, such that all obstacles (whether upper, lower or fake) of $\tau^X$ which do not belong to this coset are found also as obstacles of $(\tau'')^X$.

Define the obstacles $x_i,x_j$ similarly as in that paragraph. Then $x_j\geq x_i$ if they are upper obstacles, and $x_j=x_i$ if they are lower or fake obstacles. The first inequality is strict only when $x_i$ belongs to one of the families of upper obstacles $x+\mathbb Z$ which are affected by the unzips along $\bm\tau(i,j)$.

Moreover, if one defines similarly two pairs $(x_i,x_j)$ and $(x'_i,x'_j)$, then $x_i=x'_i$ if and only if $x_j=x'_j$. In words, two ramps in $\tau_i^X$ have their images partly overlapping if and only if their counterparts in $\tau_j^X$ partly overlap (possibly they only have one common endpoint).

This establishes a correspondence between sequences of twist nature and twist modelling functions, as we explain below.

Given a twist modelling function $h$, one can use it to define a self-diffeomorphism $\tilde h$ of $\R\times[0,1]$, by sending $(x,t)\mapsto(h(x,t),t)$; and it can also be extended to a map $\tilde h:\R\times [-2,2]\rightarrow \R\times [-2,2]$, by setting it to the identity outside $\R\times[0,1]$. This map $\tilde h$ is not continuous, as it fails along $\R\times \{0\}$. We call $\tilde H\coloneqq \tilde\upsilon\circ \tilde h \circ\tilde\upsilon^{-1}$ the corresponding self-bijection of $\bar\Hy^2_X\rightarrow \bar\Hy^2_X$.

Since $\tilde h$ is equivariant under horizontal translation by $2\pi$, it descends to a self-bijection of $\sph^1\times[-2,2]$ on $\ol{S^X}$ and, via conjugation by $\upsilon$, the latter is turned into a self-bijection $\hat H:\ol{S^X}\rightarrow \hat H$.

Finally, a map $H:S\rightarrow S$ is obtained by setting $H|_{A_\gamma}=\hat H|_{A_\gamma}$ and $H|_{S\setminus A_\gamma}\coloneqq \mathrm{id}_{S\setminus A_\gamma}$. This is again not continuous, but it is a self-diffeomorphism of $S\setminus \tau_0.\gamma$ which does not permute its connected components. Note a detail: $H$ fixes $\tau_0.\gamma$ pointwise, but $\hat H$ and $\tilde H$ fix $\tau_0^X.\gamma$ and $\tilde\gamma$, respectively, only setwise.

One can see (cfr. \cite{mosher}, p. 215) that, after modelling $\bm\tau$ under the conventions given above, for each pair of indices $0\leq k\leq l\leq N$, there exists a twist modelling function $h_k^l:\R\times[0,1]\rightarrow \R$ such that $\tau_l= H_k^l(\tau_k)$, where $H_k^l$ is defined with the process above; and we can also suppose that, given three indices $0\leq k\leq l \leq r \leq N$, it holds that $h_k^r(x,t)=h_l^r\left(k_k^l(x,t),t\right)$ i.e. $\tilde h_k^r=\tilde h_l^r\circ\tilde h_k^l$ and similar composition rules hold for the maps defined above. We will say that $h_k^l$ is a twist modelling function \nw{associated with} the splitting sequence $\bm\tau(k,l)$.
\end{rmk}

\begin{lemma}\label{lem:functiongivestwist}
Let $\tau_0$ be a generic almost track on a surface $S$ with $\gamma$ a twist curve, and $A_\gamma$ a fixed twist collar. Let $h:\R\times[0,1]\rightarrow \R$ be a twist modelling function. Define a bijection $H:S\rightarrow S$ as in Remark \ref{rmk:twistnaturemodelling}.

Suppose that $\tau_1\coloneqq H(\tau_0)$ is a (generic) almost track; equivalently that, for any $x$ upper obstacle for $\tau_0^X$, $h(x,0)$ is not a lower or fake obstacle of $\tau_0^X$. Then there exists a splitting sequence $\bm\tau$ turning $\tau_0$ into $\tau_1$, and having twist nature about $\gamma$ with twist collar $A_\gamma$. Moreover, $\bm\tau$ has $h$ as an associated twist modelling function.
\end{lemma}
\begin{proof}
Consider the smooth map $\Phi:\R\times[0,1]\times[0,1]\rightarrow \R\times [0,1]$ defined by the linear combination $\Phi(x,t,u)=(1-u)x+ u h(x,t)$. For all $u\in[0,1]$ the map $h_u(x,t)\coloneqq\Phi(x,t,u)$ is a twist modelling function; let $\tilde h_u$ be the corresponding self-map of $\R\times[-2,2]$, defined as prescribed in Remark \ref{rmk:twistnaturemodelling}. We have $\frac{\partial}{\partial u}\Phi(x,t,u)\geq 0$ for all $x,t,u$ and $\frac{\partial}{\partial u}\Phi(x,t,u)= 0$ if and only if $h_1(x,t)=x$.

For all $u$ let $H_u$ be the self-map induced by $h_u$ on $S$. Up to small perturbations of $\Phi$ not affecting the previously listed properties, one can also suppose that for every fixed $u\in[0,1]$, one of the following is true:
\begin{itemize}
\item there is one coset $x_0(u)+2\pi\mathbb Z$ such that, if $x$ is an upper obstacle for $\tau_0^X$ while $h_u(x,0)$ is a lower or fake obstacle for $\tau_0^X$, then $x\in x_0(u)+2\pi\mathbb Z$;
\item if $x$ is an upper obstacle for $\tau_0^X$, then $h_u(x,0)$ is neither a lower nor fake obstacle for $\tau_0^X$.
\end{itemize}

Let $0<u_1<\ldots<u_k<1$ be the values of $u$ as in the first bullet. Define, for $u\in[0,1]$, $\tau_u=H_u(\tau_0)$: it is a generic almost track exactly when $u$ is none of the aforementioned values.

Given two values $0\leq u<u'\leq 1$: if there is a $j$ such that $u_j<u<u'<u_{j+1}$ then $\tau_u,\tau_{u'}$ are turned into each other with an isotopy of $S$. If there is exactly one value of $j$ such that $u<u_j<u'$ then $\tau_{u'}$ is obtained from $\tau_u$ by shifting exactly one branch end hitting $A_\gamma$ beyond a branch end that avoids it (plus an isotopy); which is the same as saying: via a twist slide or a twist splitting.

So if we define another sequence $v_0,\ldots,v_k$ such that $0\leq v_0<u_1<v_1<u_2<\ldots<v_{k-1}<u_k<v_k\leq 1$, then $\bm\tau=(\tau_{v_j})_{j=0}^k$ is the desired splitting sequence, with twist nature.
\end{proof}

\begin{rmk}\label{rmk:twistfunctionsflexibility}
\begin{enumerate}
\item Given any increasing function $\eta:\R\rightarrow \R$ such that $\eta(x)\geq x$ for all $x$, there is a twist modelling function $h: \R\times[0,1]\rightarrow \R$ such that $h|_{\R\times\{0\}}=\eta$.

\item Fix a generic almost track $\tau$ carrying a twist curve $\gamma$; fix $A_\gamma$ and the other complementary constructions given in Remark \ref{rmk:twistparam}. Given any two twist modelling functions $h,h'$ and the relative maps $H,H':S\rightarrow S$ built from them, suppose that $h|_{\R\times\{0\}}=h'|_{\R\times\{0\}}$ and that $H(\tau)$ is a train track. Then $H'(\tau)$ is also a train track, isotopic to the former one: the isotopy is made explicit by the $1$-parameter family $H_u(\tau)$, where $u\in[0,1]$ and $H_u$ is the self-map of $S$ obtained from the twist modelling function $h_u=(1-u)h+uh'$.

\item A particular case of twist modelling function is given by the ones giving $h(x,1)=x+2\pi m$ for some fixed $m\in\mathbb N$. We consider these maps in the setting specified in Remark \ref{rmk:twistnaturemodelling}. Due to the $2\pi$-periodicity in the $x$ variable, the corresponding maps $\hat H$ and $H$ are continuous; and, due to the point above, one may as well suppose that they are smooth, up to isotopy. If $\epsilon$ is the sign of $\gamma$ as a twist curve, then $\hat H$ is isotopic to $D_X^{\epsilon m}$, where $D_X$ is the Dehn twist of $\ol{S^X}$ about its core curve; and $H$ is isotopic to $D_\gamma^{\epsilon m}$, where $D_\gamma$ is the Dehn twist of $S$ about $\gamma$.
\end{enumerate}
\end{rmk}

\subsection{The wide arc set at a twist curve}
We introduce a few more notations and immediate remarks about the arcs carried by an induced track in an annulus. We use here the parametrization of Remark \ref{rmk:twistparam}.

This definition integrates the definition of horizontal stretch given in point \ref{itm:horizontalstretch} in Remark \ref{rmk:annulusinducedbasics}:
\begin{defin}\label{def:horizontallength}
Let $\tau$ be a generic almost track where a specified curve $\gamma$ is a twist curve with a specified twist collar $A_\gamma$, and let $X$ be a regular neighbourhood of $\gamma$. For $\alpha\in\cc(\tau^X)$, let $\ul\alpha_\tau$ be a realization of $\alpha$ as a train path along $\tau^X$, and let $\ul{\tilde\alpha}_\tau$ be a lift of it to the universal cover $\bar\Hy^2_X$ of $\ol{S^X}$. The train path $\ul{\tilde\alpha}_\tau$ is necessarily embedded in $\tilde\tau$, and traverses one or more branches belonging to $\tilde\tau.\tilde\gamma$, which we already know to be consecutive.

An \nw{interval stretch} is an interval $[x,y]$ such that $[x,y]\times\{0\}= \tilde \upsilon^{-1}(\tilde\gamma\cap\ul{\tilde\alpha}_\tau)= (\R\times\{0\})\cap \tilde\upsilon^{-1}(\ul{\tilde\alpha}_\tau)$; there is one for each choice of $\ul{\tilde\alpha}_\tau$, so they are a family of intervals obtained from one another via addition of a $\mathbb Z$-multiple of $2\pi$. Necessarily, $x$ is an upper obstacle for $\tau^X$ and $y$ is a lower one (see Remark \ref{rmk:sameorientationforarcs}).

Finally, we call the \nw{horizontal length} of $\alpha$ the length of any of these intervals: $\hl(\tau,\alpha)\coloneqq y-x$ (this is independent of the choice of $\ul{\tilde\alpha}_\tau$).

If $\bm\tau=(\tau_j)_{j=0}^N$ is a splitting sequence of almost tracks such that $\gamma$ stays a twist curve throughout, and $\alpha\in\cc(\tau_i^X)$ we may denote, leaving that sequence implicit, $\hl_\alpha(i)= \hl(\tau_i,\alpha)$; $\ul\alpha_i=\ul\alpha_{\tau_i}$. In that case we choose the family $\ul{\tilde\alpha}_j$, for $0\leq j\leq i$, to be consisting of arcs all having the same pair of endpoints along $\partial\bar\Hy^2$. We denote $[x_j,y_j]$ the corresponding interval stretches, and $\hl_\alpha(j)=\hl(\tau_j,\alpha)$.
\end{defin}

We list some basic properties of the horizontal length:
\begin{enumerate}
\item With $x,y$ as in the above definition, consider the two segments of $\tilde\upsilon^{-1}(\ul{\tilde\alpha}_\tau)\setminus \left((x,y)\times\{0\}\right)$, i.e. the two lifts of ramps of $\ul\alpha_\tau$. The one having an endpoint at $(x,0)$ is contained in $\R\times[0,2]$ i.e. it projects to a ramp hitting $A_\gamma$ in $S^X$; while the other one is contained in $\R\times[-2,0]$ so it projects to a ramp which avoids $A_\gamma$ and is favourable.

\item \textit{For any $\alpha\in\cc(\tau)$, $\hl(\tau,\alpha)$ is never a multiple of $2\pi$.} If it were, that would mean that $\hs(\tau,\alpha)$ begins and ends at the same switch along $\tau^X.\gamma$. But this is impossible, because one of the two ramps composing $\tau^X.\alpha\setminus\tau^X.\gamma$ hits $A_\gamma$ and the other one avoids it: by genericness of $\tau$, they do not meet $\tau^X.\gamma$ at the same switch.

\item\label{itm:hl_vs_multiplicity} \textit{$\lfloor \hl(\tau,\alpha)/2\pi\rfloor$ is the minimum number of times $\alpha$ travels along a fixed branch within $\tau^X.\gamma$.} In particular this quantity is independent of metric properties and realization of $\tau$ within a given isotopy class. Moreover there is at least one branch traversed $\lfloor \hl(\tau,\alpha)/2\pi\rfloor+1$ times.

\item\label{itm:hlvertex} \textit{An arc $\alpha\in\cc(\tau^X)$ is wide carried if and only if $\hl(\tau,\alpha)<2\pi$}: this bound on horizontal length is equivalent to saying that $\hs(\tau,\alpha)$ does not traverse all branches in $\tau^X.\gamma$. Remark \ref{rmk:annulusinducedbasics}, point \ref{itm:windaboutgamma}, concludes our argument.

\item\label{itm:farisininfluent} \textit{Let $\bm\tau$ be a splitting sequence as in Definition \ref{def:horizontallength}, and fix $0\leq j<N$. Suppose the move between $\tau_j$ and $\tau_{j+1}$ is far from $\gamma$. Then, if the conventions of Remark \ref{rmk:permanenceconventions} are used then, for any $\alpha\in\cc(\tau_{j+1}^X)$, $\hs_\alpha(j+1)\subseteq \hs_\alpha(j)$. If the move between $\tau_j$ and $\tau_{j+1}$ is far from $\gamma$, then $\hs_\alpha(j)=\hs_\alpha(j+1)$.}

As a consequence of point \ref{itm:horizontalstretch}, a train path realization $\ul\alpha_{j+1}$ of $\alpha$ in $\tau_{j+1}^X$ is the concatenation of $\rho^h_{j+1},\hs_\alpha(j+1),\rho^f_{j+1}$ for $\rho^h_{j+1}$ an incoming ramp for $\tau_{j+1}^X$ hitting $A_\gamma$, and $\rho^f_{j+1}$ an outgoing ramp avoiding $A_\gamma$ and favourable.

A train path realization $\ul\alpha_j=c_{\tau_j^X}(\ul\alpha_{j+1})$ --- here we are neglecting any necessary reparametrization --- is the concatenation of $c_{\tau_j^X}(\rho^h_{j+1})$, $c_{\tau_j^X}(\hs_\alpha(j+1))=\linebreak\hs_\alpha(j+1)$, $c_{\tau_j^X}(\rho^f_{j+1})$: hence $\hs_\alpha(j+1)\subseteq \hs_\alpha(j)$.

When the move that takes place between $\tau_j,\tau_{j+1}$ is far from $\gamma$, the conventions in Remark \ref{rmk:permanenceconventions} imply that neither $c_{\tau_j^X}(\rho^h_{j+1})$ nor $c_{\tau_j^X}(\rho^f_{j+1})$ include any segment along $\tau_j^X.\gamma$. So $\hs_\alpha(j)=\hs_\alpha(j+1)$.

\item\label{itm:twistnaturehl} \textit{Let $\bm\tau$ be a splitting sequence as in Definition \ref{def:horizontallength}, with twist nature about $\gamma$, and fix two indices $0\leq k\leq l\leq N$. Model $\bm\tau$ as said in Remark \ref{rmk:twistnaturemodelling}, and let $h$ be the twist modelling function $h$ associated with $\bm\tau(k,l)$. Let $\alpha\in V(\tau_l)$, and let $[x_k,y_k],[x_l,y_l]$ be two consistent choices of interval stretches for $\alpha$ in $\tau_k$ and $\tau_l$, respectively; then $y_k=y_l$ and $x_l=h(x_k,0)\geq x_k$; and, in particular, $\hl_\alpha(l)\leq \hl_\alpha(k)$.}

Under the conventions of Remark \ref{rmk:twistnaturemodelling}, $\ul{\tilde\alpha}_l\setminus p^{-1}(\bar A_\gamma)= \ul{\tilde\alpha}_k\setminus p^{-1}(\bar A_\gamma)$, and this implies that $\ul{\tilde\alpha}_l\setminus(\R\times\{0\})=\tilde H\left(\ul{\tilde\alpha}_l\setminus(\R\times\{0\})\right)$. This gives immediately $y_k=y_l$, and $x_l=h(x_k,0)$.
\end{enumerate}

\begin{lemma}\label{lem:onerollingdirection}
Let $\tau$ be a generic almost track, and let $\gamma$ be a twist curve with sign $\epsilon$ and $X$ a regular neighbourhood of it. Then the following are true. 
\begin{itemize}
\item For each $m\in\mathbb N$,
$$
D_X^{\epsilon m}\cdot V(\tau^X)=\{\alpha\in\cc(\tau^X)\mid 2\pi m<\hl(\tau,\alpha)<2\pi(m+1)\}
$$
where $D_X$ is the Dehn twist about $\gamma$ in $S^X$. In particular, for all $\alpha\in V(\tau^X)$, $\hl(\tau,D_X^{\epsilon m}(\alpha))=\hl(\tau,\alpha)+2\pi m$; and
$$
\cc(\tau^X)=\bigcup_{j\geq 0} D_X^{\epsilon j}\cdot V(\tau^X).
$$
\item If $m\in\mathbb N,\alpha\in V(\tau^X), \beta\in D_X^{\epsilon m}\cdot V(\tau^X)$ then $m-1\leq i(\alpha,\beta)\leq m+1$. In particular, $\mathrm{diam} \left(V(\tau^X)\right)\leq 2$.
\end{itemize}
\end{lemma}
\begin{proof}
The first equality in the first bullet has been shown, in the particular case $m=0$, in point \ref{itm:hlvertex} of the above list. For $m>0$: given $\alpha\in V(\tau^X)$, construct $\ul{\tilde\alpha}_\tau$ and the corresponding interval stretch $[x,y]$ as prescribed in Definition \ref{def:horizontallength}.

Let $\tilde\rho_+=\ul{\tilde\alpha}_\tau\cap(\R\times(0,2])$ and $\tilde\rho_-=\ul{\tilde\alpha}_\tau\cap(\R\times[-2,0))$. Define then $\rho'_-\coloneqq\{(x'+2\pi m,y')\in\R\times[-2,0)|(x',y')\in \rho_-\}$, and $\ul{\tilde\beta}_\tau\coloneqq\rho_+\cup [x,y+2\pi m]\times\{0\}\cup \rho'_-$. The path specified by $\ul\beta=q(\ul{\tilde\beta}_\tau)\subset\ol{S^X}$ is then a train path along $\tau$, realizing some $\beta\in \cc(\tau^X)$. By construction, $\hl(\tau,\beta)=\hl(\tau,\alpha)+2\pi m\in (2\pi m, 2\pi(m+1))$; and $\beta=D^{\epsilon m}(\alpha)$. Since all $\beta\in D_X^{\epsilon m}\cdot V(\tau^X)$ may be realized this way, this construction proves the $\subseteq$ inclusion in the given equality.

To show the opposite inclusion, we can pick any $\alpha\in\cc(\tau^X)$ with $2\pi m<\hl(\tau,\alpha)<2\pi(m+1)$ and define $\tilde\rho_+,\tilde\rho_-$ as above. Then we define $\rho_-'=\{(x'-2\pi m,y')\in\R\times[-2,0)|(x',y')\in \rho_-\}$, and  $\ul{\tilde\beta}_\tau=\rho_+\cup [x,y-2\pi m]\times\{0\}\cup \rho'_-$ (note that, due to our hypothesis on $\alpha$, $y>x+2\pi m$). Then, similarly as above, this gives $\beta\in \cc(\tau^X)$ with $\alpha=D_X^{\epsilon m}(\beta)$. Moreover $0<\hl(\beta)<2\pi$, hence $\beta\in V(\tau^X)$.

The second equality of the first bullet is immediate.

A preliminary discussion for the second bullet: we have seen in Lemma \ref{lem:twistcurvetrees} that $\tau|X$ consists of $\tau^X.\gamma$ plus a forest $\sigma$ of trees each having its root along $\tau^X.\gamma$. Let $P\subset \R$ be the set such that $P\times \{0\}=q^{-1}(\text{switches of }\tau^X\text{ along }\tau^X.\gamma)$; and let $\nu>0$ be small, to be constrained more precisely in a bit. In the conventions set up in Remark \ref{rmk:twistparam}, we may add the further request that
$$\tilde\upsilon^{-1}(\sigma)\subseteq \bigcup_{a\in P} (a-\nu,a+\nu)\times[-2,2].$$
Then, let $\alpha\in\cc(\tau^X)$ and $\ul{\tilde\alpha}_\tau$ be defined as above. Also, let $\alpha_-=\ul{\tilde\alpha}_\tau\cap \R\times\{-2\}, \alpha_+=\ul{\tilde\alpha}_\tau\cap \R\times\{2\}$ be the two endpoints of $\ul{\tilde\alpha}_\tau$. Let $\seg(\alpha)$ be defined as the straight line segment joining $\alpha_-$ with $\alpha_+$: since $\seg(\alpha)$ and $\ul{\tilde\alpha}_\tau$ are isotopic relatively to their endpoints, $q\left(\seg(\alpha)\right)$ is a representative of the isotopy class $\alpha\in\cc(X)$.

Given any $X\subset \R\times [-2,2], j\in \mathbb Z$, denote $X+j=\{(a+2\pi j,b)|(a,b)\in X\}$; and $\mathcal O(X)=\bigcup_{j\in\mathbb Z}(X+j)$.

However we pick another $\alpha\not=\beta\in \cc(\tau^X)$, the collection of segments $\mathcal O\left(\seg(\alpha)\right)\cup \mathcal O\left(\seg(\beta)\right)$ does not bound any bigon, so $q\left(\seg(\alpha)\cup\seg(\beta)\right)$ does neither.  Hence, $i(\alpha,\beta)$ is the number of points of $\mathcal O\left(\seg(\alpha)\right)\cap \seg(\beta)$; which is the same as saying, the number of $j\in\mathbb Z$ such that $\seg(\alpha)+j$ intersects $\seg(\beta)$.

Now take $m\in\mathbb N,\alpha\in V(\tau^X), \beta\in D_X^{\epsilon m}\cdot V(\tau^X)$ as in the second bullet in the statement. Let $[x,y]$ be the interval stretch for $\alpha$ corresponding to a definite choice of $\ul{\tilde\alpha}_\tau$, and $[z,w]$ the one for a choice of a lift $\ul{\tilde\beta}_\tau$. Also, let $\alpha_+=(x',2),\alpha_-=(y',-2), \beta_+=(z',2),\beta_-=(w',-2)$: we have $|x'-x|<\nu$ and similarly for the other letters. If $\nu$ is small, then $y'>x'$ and $z'>w'$.

We claim that, without loss of generality, one may suppose $y'-x'<w'-z'$. Recall indeed that $0<y-x<2\pi$ and $2\pi m<z-w<2\pi(m+1)$. So, if $m>0$ and $\nu$ is small enough, then also $y'-x'<w'-z'$; whereas, if $m=0$, the roles of $\alpha$ and $\beta$ can be swapped, so the same supposition can be done by symmetry after discarding the only case left out: $y'-x'=w'-z'$. In this special case, indeed $\seg(\alpha)$ and $\seg(\beta)+j$ are parallel for any $j$, so $i(\alpha,\beta)=0$ consistently with our statement.

Now, $\seg(\alpha)+j$ intersects $\seg(\beta)$ if and only if their respective endpoints along $\R\times\{2\}$ come in the reverse order with respect to the ones on $\R\times\{-2\}$. And, as $y'-x'<w'-z'$, this condition is verified if and only if $[x'+2\pi j,y'+2\pi j]\subseteq (z',w')$ i.e. $j\in \frac{1}{2\pi}(z'-x',w'-y')$: we estimate how many integer $j$ lie within this interval.

From an algebraic manipulation of the above inequalities involving $x,y,z,w$ we get, on the one hand, $w-y>z-x +2\pi(m-1)$. If $\nu$ is small enough, we have also $w-y-2\nu>z-x+2\pi(m-1)+2\nu$, hence $(z'-x',w'-y')\supseteq (z-x+2\nu,w-y-2\nu)\supseteq (z-x+2\nu,z-x+2\nu+2\pi(m-1))$ --- we agree that an interval is empty if its supremum is lower than its infimum. In this last interval there are exactly $m-1$ elements of $2\pi\mathbb Z$, so the possible values of $j$ are at least as many: $i(\alpha,\beta)\geq m-1$.

On the other hand, a similar manipulation gives $w-y<z-x +2\pi(m+1)$ and, again for $\nu$ small enough, $w-y+2\nu<z-x+2\pi(m-1)-2\nu$. Hence $(z'-x',w'-y')\subseteq (z-x-2\nu,w-y+2\nu)\subseteq (z-x-2\nu,z-x-2\nu+2\pi(m+1))$. Since in this last interval there are exactly $m+1$ elements of $2\pi\mathbb Z$, we get $i(\alpha,\beta)\leq m+1$.

The last claim in the statement is a direct application of Lemma \ref{lem:annulus_distance}.
\end{proof}

We introduce the following definition to measure how many \emph{complete} Dehn twists are induced by a twist splitting sequence:

\begin{defin}\label{def:rot}
Let $\bm\tau=(\tau_j)_{j=0}^N$ be a generic splitting sequence of almost tracks such that a twist curve $\gamma$ stays a twist curve throughout, with a given $A_\gamma$-family of twist collars.

We define the \nw{rotation number} about $\gamma$ of a subsequence $\bm\tau(k,l)$ as
$$
\rot_{\bm\tau}(\gamma,k,l)\coloneqq \min_{\alpha\in\cc(\tau_l|X)}\left(\left\lfloor\frac{\hl_\alpha(k)}{2\pi}\right\rfloor-\left\lfloor\frac{\hl_\alpha(l)}{2\pi}\right\rfloor\right).
$$
When some of the data is understood from the context we will use a lighter notation such as $\rot(\gamma,k,l)$ or $\rot(k,l)$; or $\rot_{\bm\tau}(\gamma),\rot_{\bm\tau}$ when the rotation number is computed on the entire splitting sequence rather than a subsequence.
\end{defin}

\begin{rmk}\label{rmk:rotbasics}
We give some basic facts related with the rotation number, using the same notations already set up in the definition above. 

\begin{enumerate}
\item \label{itm:concatrot_below} \textit{If $0\leq k\leq r\leq l\leq N$, then $\rot(k,l)\geq\rot(k,r)+\rot(r,l)$.}
This is just because $\rot(k,l)\geq \min_{\alpha\in\cc(\tau_l|X)}\left(\left\lfloor\frac{\hl_\alpha(k)}{2\pi}\right\rfloor-\left\lfloor\frac{\hl_\alpha(r)}{2\pi}\right\rfloor\right) + \min_{\alpha\in\cc(\tau_l|X)}\left(\left\lfloor\frac{\hl_\alpha(r)}{2\pi}\right\rfloor-\left\lfloor\frac{\hl_\alpha(l)}{2\pi}\right\rfloor\right)\linebreak \geq \rot(k,r)+\rot(r,l)$. The second inequality is explained by replacing the minimum in the first summand with the one taken on the larger set $\cc(\tau_r|X)$.

\item\label{itm:rotwithvertexonly} \textit{An alternative definition is $\rot_{\bm\tau}(\gamma,k,l)= \min_{\alpha\in V(\tau_l|X)}\left\lfloor\frac{\hl_\alpha(k)}{2\pi}\right\rfloor$. In particular the rotation number is always nonnegative.}

We can replace the minimum over $\cc(\tau_l|X)$ with the one over $V(\tau_l|X)$ because of Lemma \ref{lem:onerollingdirection}.

\item \label{itm:rotofdehn} \textit{If $\bm\tau(k,l)$ is a splitting sequence of twist nature, associated with a twist modelling function $h$ having $h(x,0)=x+2\pi m$ for some $m\in\mathbb N$ (i.e. a neighbourhood of $\gamma$ in $S$ differs, from $\tau_k$ to $\tau_l$, according to the self-map $H=D_\gamma^{\epsilon m}:S\rightarrow S$) then $\rot(k,l)=m$.} This is a simple consequence of point \ref{itm:twistnaturehl} after Definition \ref{def:horizontallength}.

\item \textit{If $0\leq k\leq r\leq l\leq N$ and $\bm\tau(r,l)$ is a sequence consisting of slides only then $\rot(k,r)=\rot(k,l)$.} Since the inverse of a slide move is also a slide move, from point \ref{itm:concatrot_below} in this list we get $\rot(k,r)\leq\rot(k,l)\leq \rot(k,r)$.

\item \textit{The rotation number is not affected by changing the train tracks within their isotopy class and changing the choice for the parametrizations defined in Remark \ref{rmk:twistparam}.}

\item \label{itm:rot_vs_dt_vertices} \textit{If $m=\rot(k,l)$, then
$$V(\tau_l^X)\subseteq \left(D_X^{\epsilon m}\cdot V(\tau_k^X)\right)\cup \left(D_X^{\epsilon(m+1)}\cdot V(\tau_k^X)\right)\cup \left(D_X^{\epsilon(m+2)}\cdot V(\tau_k^X)\right).$$}
This is a consequence of Lemma \ref{lem:onerollingdirection}. By definition of rotation number, $\hl_\alpha(k)>2\pi m$ for all $\alpha \in V(\tau_l^X)$, and so $V(\tau_l^X)\subseteq \bigcup_{j\geq m} D_X^{\epsilon j}\cdot V(\tau_k^X)$. Fix $\alpha$ realizing the minimum in the definition of $\rot(k,l)$: then $\alpha= D_X^{\epsilon m}(\alpha')$ for some $\alpha'\in V(\tau_k^X)$.

Given any $\beta\in V(\tau_l^X)\cap \left(D_X^{\epsilon(m+j)}\cdot V(\tau_k^X)\right)$ for a fixed $j\geq 0$, write correspondingly $\beta=D_X^{\epsilon m}(\beta')$ for $\beta'\in D_X^{\epsilon j}\cdot V(\tau_k^X)$. Then said Lemma implies that $i(\alpha,\beta)=i(\alpha',\beta')\geq j-1$ while, on the other hand, $i(\alpha,\beta)\leq 1$ because these two arcs both belong to $V(\tau_l^X)$. So $j\leq 2$, as required.

\item \label{itm:rot_vs_dist}\textit{If $m=\rot(k,l)$, then $m\leq d_{\cc(X)}\left(V(\tau_k^X),V(\tau_l^X)\right)\leq m+4$.}

This is a consequence of the point above and Lemma \ref{lem:onerollingdirection}. The core of the argument is that, if $\alpha\in V(\tau_k^X)$ and $\beta\in V(\tau_l^X)$ are distinct, then $m-1\leq i(\alpha,\beta)\leq m+3$ and Lemma \ref{lem:annulus_distance} gives $m\leq d_{\cc(X)}(\alpha,\beta)\leq m+4$. 

\item \textit{If $\alpha\in\cc(\tau_l^X)$ then $\left\lfloor \frac{\hl_\alpha(k)}{2\pi}\right\rfloor-\left\lfloor \frac{\hl_\alpha(l)}{2\pi}\right\rfloor\leq  \rot(k,l)+2$.}

Point \ref{itm:rot_vs_dt_vertices} together with Lemma \ref{lem:onerollingdirection} give this for $\alpha\in V(\tau_l^X)$ first, and then for all $\alpha\in\cc(\tau_l^X)$.

\item \label{itm:concatrot_above} \textit{If $0\leq k\leq r\leq l\leq N$, then $\rot(k,l)\leq \rot(k,r)+\rot(r,l)+2$.}
If $\alpha\in \cc(\tau_l^X)$ realizes the minimum defining $\rot(r,l)$ then
$\rot(k,l)\leq \left(\left\lfloor \frac{\hl_\alpha(k)}{2\pi}\right\rfloor-\left\lfloor \frac{\hl_\alpha(r)}{2\pi}\right\rfloor\right) + \rot(r,l)$
and we use the point above.

\item \label{itm:tmfbeyondrot} \textit{If $\bm\tau(k,l)$ is a splitting sequence of twist nature, associated with a twist modelling function $h$, then $h(x,0)<x+2\pi(\rot(k,l)+3)$ for all $x\in\R$ which are upper obstacles for $\tau_k^X$.}

For each $x$ upper obstacle for $\tau_k^X$ there are some $\alpha\in V(\tau_l^X)$ and $y>x$ such that $[x,y]$ is an interval stretch for $\alpha$ with respect to $\tau_k$; from point \ref{itm:rot_vs_dt_vertices} above, $y-x=\hl_\alpha(k)<2\pi(\rot(k,l)+3)$; on the other hand, because of point \ref{itm:twistnaturehl} after Definition \ref{def:horizontallength}, $y-h(x)=\hl_\alpha(l)\in(0,2\pi)$; therefore $h(x)-x\leq 2\pi(\rot(k,l)+3)$.
\end{enumerate}
\end{rmk}

\begin{lemma}[Dehn twists with remainder]\label{lem:dehn+remainder}
Let $\bm\tau=(\tau_j)_{j=0}^N$ be a generic splitting sequence of almost tracks on a surface $S$ which has twist nature about a fixed curve $\gamma$ with sign $\epsilon$. Suppose the sequence is modelled according to Remark \ref{rmk:twistnaturemodelling}, with a fixed twist collar $A_\gamma$. Let $m\coloneqq \rot_{\bm\tau}(\gamma,0,N)$.

Then there is another splitting sequence $\bm\tau'=(\tau_j)_{j=0}^{N''}$ with twist nature about $\gamma$; there are two indices $0<N'\leq N''$ and a factorization $N''-N'=mk$, such that:
\begin{itemize}
\item $\tau'_0=\tau_0$, and $\tau'_{N''}$ is obtained from $\tau_N$ with slides only;
\item $\rot_{\bm\tau'}(\gamma,0,N')=0$;
\item for any choice of indices $N'\leq j < j+k\leq N''$, $\tau'_{j+k}=D_\gamma^{\epsilon}(\tau'_j)$, where $D_\gamma$ is the Dehn twist about $\gamma$ in $S$. In particular, $\tau'_{N''}=D_\gamma^{\epsilon m}(\tau_{N'})$.
\end{itemize}
\end{lemma}
\begin{proof}
Let $X$ be a regular neighbourhood of $\gamma$, and $h$ be the twist modelling function associated with $\bm\tau(0,N)$.

\step{1} make sure that $h(x,0),0\geq x+2\pi m$, possibly operating further slides.

This step is devoted to defining recursively the entries of an extension $\bm\tau(N,N''')$ of the splitting sequence $\bm\tau$. This extension will also be of twist nature, \emph{with no splits}, but with the possibility that some $\tau_{j+1}$ is isotopic to $\tau_j$.

The extension will be associated with a twist modelling function $h'''$, such that $h'''(h(x,0),0)\geq x+2\pi m$ for all $x\in\R$. Actually, it will suffice to show that this inequality holds for every $x$ which is an upper obstacle for $\tau_0^X$. If this is true, one will be able to adapt $h'''$ so that the inequality holds for all $x\in\R$, without changing $\tau_{N'''}$.

We fix some notation for the entries $\bm\tau(N,N''')$, which we are about to build. For $0\leq j \leq N'''-N$, label $\beta_j^0,\ldots,\beta_j^r$ the segments of $(\tau_{N+j}|X).\gamma$ delimited by two switches, and such that their extremities are located at large branch ends. Due to Lemma \ref{lem:twistininduced}, these segments are exactly the ones which can be obtained as $\hs_\alpha(N+j)$ for some $\alpha\in V(\tau_{N+j}^X)$; of course, all the $\alpha$ defining the same segment have the same $\hl_\alpha(N+j)$, and we call it $\hl(\beta_j^i)$.

Note that, as $\bm\tau$ is a splitting sequence of twist nature, two elements $\alpha,\alpha'\in V(\tau_{N+j}^X)$ have $\hs_\alpha(N+j)=\hs_{\alpha'}(N+j)$ if and only if $\hs_\alpha(N+j')=\hs_{\alpha'}(N+j')$ for each $j'\leq j$: it is just another way of phrasing the behaviour of obstacles along a sequence which was noted in Remark \ref{rmk:twistnaturemodelling}. This observation establishes a natural bijection between the collections $(\beta_j^i)_i$, $(\beta_{j'}^i)_i$ for $j< j'$: i.e. two segments $\beta_j^i$ and $\beta_{j'}^{i'}$ may be supposed to comply with the condition $i=i' \Longleftrightarrow$ there is an element $\alpha\in V(\tau_{N+j'}^X)$ such that $\beta_j^i=\hs_\alpha(N+j)$ and $\beta_{j'}^{i'}=\hs_\alpha(N+j')$.

For each $i$, then, fix an $\alpha^i\in V(\tau_N^X)=V(\tau_{N+1}^X)=\ldots=V(\tau_{N'''}^X)$ such that $\hs_{\alpha^i}(\tau_{N+j})=\beta_j^i$ for all $j$.

Furthermore, we assign superscripts so that $\beta_j^i\subset \beta_j^{i'}\Rightarrow i<i'$. In order to get this condition one may impose, for instance, that the sequence $\left(\hl(\beta_0^i)\right)_i$ is increasing. The slides along the sequence will then force $\beta_j^i\subset \beta_j^{i'} \Rightarrow \beta_{j+1}^i\subset \beta_{j+1}^{i'}$ for all $0\leq j < N'''-N$.

During the course of our recursion, each almost track $\tau_{N+j}$ will have the following property: 
\begin{center}
$\hl(\beta_j^i)\leq \hl_{\alpha^i}(0)-2\pi m$ for all $i\leq j$.
\end{center}

The sequence will then stop at the index $N'''=N+r$. At that point, recall point \ref{itm:twistnaturehl} after Definition \ref{def:horizontallength}: given any upper obstacle $x$ for $\tau_0^X$, there is an $\alpha^i$ such that two consistent choices for interval stretches of $\alpha^i$ with respect to $\tau_0$ and $\tau_{N'''}$ turn out to be $[x,y]$ and $[h'''(h(x,0),0),y]$ respectively, for some $y\in\R$ lower obstacle for both $\tau_0^X,\tau_{N'''}^X$. This implies $\hl_{\alpha^i}(j)\leq \hl_{\alpha^i}(0)-2\pi m\Rightarrow h'''(h(x,0),0)>x+2\pi m$, as desired.

We now enter the recursive process: suppose that $\tau_{N+j}$ has been defined for some $0\leq j <N'''-N$: we construct $\tau_{N+j+1}$. If $\hl_{\alpha^{j+1}}(N+j)<\hl_{\alpha^{j+1}}(0)-2\pi m$, then it is fine to set $\tau_{N+j+1}=\tau_{N+j}$. Else, denote $[x_0,y]$, $[\xi,y]$ two consistent choices for interval stretches of $\alpha^{j+1}$ with respect to $\tau_0$ and $\tau_{N+j}$, respectively. By assumption, $\xi<x_0+2\pi m$.

We claim the following:
\begin{itemize}
\item there exists a twist modelling function $h_j$ such that $h_j(\xi+2\pi \zeta,1)=x_0+2\pi(m+\zeta)$ for all $\zeta\in\mathbb Z$, and $h(x,1)=x$ for any $x\not\in\xi+2\pi\mathbb Z$ which is an upper obstacle for $\tau_{N+j}^X$; 
\item given $H_j$ the self-map of $S$ derived from $h_j$ with the construction given in Remark \ref{rmk:twistnaturemodelling}, $\tau_{N+j+1}\coloneqq H_j(\tau_{N+j})$ is isotopic, or obtained with twist slides only, from $\tau_{N+j}$.
\end{itemize}

If it is possible to define $\tau_{N+j+1}$ this way, then $\hl(\beta_{j+1}^{j+1})=y-h_j(\xi,0)= y-x_0-2\pi m= \hl_{\alpha^j}(0)-2\pi m$; and for, $i\leq j$, $\hl(\beta_{j+1}^i)\leq \hl(\beta_j^i) \leq \hl_{\alpha^i}(0)-2\pi m$ as required by the property claimed above.

The claim in the first bullet is true if and only if the interval $(\xi,x_0+2\pi m]$ contains no upper obstacles for $\tau_{N+j}^X$ (see also Remark \ref{rmk:twistfunctionsflexibility}), and the one in the second bullet is true if and only if $(\xi,x_0+2\pi m]$ contains no lower obstacles for $\tau_{N+j}^X$.

Suppose that an upper or lower obstacle $\bar\xi$ exists in the specified segment, for a contradiction. Then one between $[\xi,\bar\xi]$ and $[\bar\xi, y]$ is a connected component of $q^{-1}(\beta_j^i)$, for some $i$; clearly $i\leq j$, because by construction $\beta_j^i\subseteq \beta_j^{j+1}$. Therefore $\hl(\beta_j^i) \leq \hl_{\alpha^i}(0)-2\pi m$.

If $\bar\xi$ is a lower obstacle, then $[\xi,\bar\xi]$ is the one of the two segments which makes a connected component of $q^{-1}(\beta_j^i)$. But then $\hl(\beta_j^i) \leq \hl_{\alpha^i}(0)-2\pi m$ implies $\xi\geq x_0+2\pi m$, contrary to the assumption. If $\bar\xi$ is an upper obstacle, then $[\bar\xi, y]$ is a connected component of $q^{-1}(\beta_j^i)$. Let $[\bar x_0,y]$ be an interval stretch for $\alpha^i$ in $\tau_0^X$, consistent with the fixed interval stretch $[\bar\xi, y]$ related to $\tau_{N+j}^X$. The hypothesis $\hl(\beta_j^i)=\hl_{\alpha^i}(N+j)\leq \hl_{\alpha^i}(0)-2\pi m$ translates into $\bar x_0 \leq \bar\xi -2\pi m$. On the other hand, since $\bar\xi\leq x_0+2\pi m$ by definition, we get $\bar x_0\leq x_0$; and this is impossible as it would imply $\beta_0^i\supseteq \beta_0^{j+1}$ while $\beta_{N+j}^i\subsetneq \beta_{N+j}^{j+1}$.

This concludes the recursion argument. The claimed twist modelling function for $\bm\tau(N,N''')$ is, of course, $h'''\coloneqq h_{N'''-N-1}\circ \cdots \circ h_0$.

\step{2} proof of the lemma.

We can now change the notation partially: remove from $\bm\tau(N,N''')$ any $\tau_j$ such that $\tau_{j+1}$ is isotopic to it (the value of $N'''$ decreases accordingly). Moreover, since the original $h$ will not be needed, for a simpler notation let $h$ be the twist modelling map associated with the entire sequence $\bm\tau(0,N''')$. Call $H$ the self-bijection of $S$ that one obtains from $h$. Furthermore, consider a twist modelling function $h_D$ (and, consequently, a map $H_D:S\rightarrow S$) defined with the condition that $h_D(x,0)=x+2\pi\eqqcolon \eta_D(x)$. According to what noted in Remark \ref{rmk:twistfunctionsflexibility}, such a map $h_D$ exists, and $H_D$ is a diffeomorphism of $S$, in the isotopy class of $D_\gamma^\epsilon$. 

Consider now the function $\eta':\R\rightarrow\R$ defined by $\eta'(x)=h(x,0)-2\pi m$. Clearly $\eta'$ is a strictly increasing function; also, $\eta(x)\geq x$ since, by construction, $h(x,0)\geq x+2\pi m$. So, again, it is possible to find a twist modelling function $h':\R\times[0,1]\rightarrow \R$ with $h'|_{\R\times\{0\}}=\eta'$ (with the relative map $H':S\rightarrow S$).

The composition $(\eta_D)^{m}\circ \eta'$ agrees with $h|_{\R\times\{0\}}$. So, according to what noted in Remark \ref{rmk:twistfunctionsflexibility}, $(H_D)^{m}\circ H'(\tau_0)$ is isotopic to $H(\tau)=\tau_{N'''}$.

The new sequence $\bm\tau'$ is then the concatenation of $m+1$ sequences, each obtained from an application of Lemma \ref{lem:functiongivestwist}. The first one, $\bm\tau'(0,N')$, is a splitting sequence of twist nature built from $\tau_0$ and $h'$: the Lemma's hypotheses are met, as by definition $H'(\tau_0)=D_\gamma^{-\epsilon m}\circ H(\tau_0)$ is a generic almost track.

Let $\alpha\in V(\tau_{N'''}^X)$ be an arc realizing the minimum in the definition of $\rot_{\bm\tau}(0,N''')$; then $\hl_\alpha(0)\in (2\pi m,2\pi(m+1))$, so if $[x_0,y]$ and $[x_{N'''},y]$ are interval stretches for $\alpha$ in $\tau_0$ and $\tau_{N'''}$ respectively, then $x_0\in (y-2\pi(m+1),y-2\pi m)$ and $x_{N'''}=h(x_0,0) \in (y-2\pi,y)$. An interval stretch for $\alpha$ in $\tau'_{N'}$ is instead $[\xi,y]$ where $\xi=\eta'(x_0)= h(x_0,0)-2\pi m \in (y-2\pi(m+1),y-2\pi m)$. So, in the sequence $\bm\tau'$, both $\hl_\alpha(0), \hl_\alpha(N')\in (2\pi m,2\pi (m+1))$ and this implies that $\rot_{\bm\tau'}(0,N')=0$.

Another subsequence $\bm\tau'(N',N'+k)$ will turn $\tau_{N'}$ into $H_D(\tau_{N'})$; and, for $1\leq j<m$, one can consistently define $\bm\tau'(N'+kj)= (H_D)^j\cdot\bm\tau'(N',N'+k)$ --- i.e. transforming each entry of $\bm\tau'(N',N'+k)$ under $(H_D)^j$. Set $N''\coloneqq N'+km$: by definition, $\tau'_{N''}=\tau_{N'''}$.
\end{proof}

\begin{lemma}\label{lem:twistsplitnumber}
Let $N_3=N_3(S)$ be a constant such that, given any generic almost track $\tau$ on $S$, and any twist curve $\gamma$ for $\tau$, for each side of $\tau^{\nei(\gamma)}.\gamma$ and any orientation on $\gamma$ there are at most $N_3$ branch ends sharing a switch with $\tau^{\nei(\gamma)}.\gamma$, located on the specified side and giving $\gamma$ the required orientation.

If $\bm\tau=(\tau_j)_{j=0}^N$ is a splitting sequence of almost tracks and $\bm\tau(k,l)$ has twist nature about $\gamma\in W(\tau_k)$, then the number of twist splits in $\bm\tau$ is bounded by $N_3^2\left(\rot(k,l)+3\right)$.
\end{lemma}

\begin{proof}
The existence of the constant $N_3$ is just a consequence of more general bounds on combinatorics of almost tracks (Lemma \ref{lem:vertexsetbounds}).

Model the sequence $\bm\tau(k,l)$ in accordance with Remark \ref{rmk:twistnaturemodelling}. Every time $\tau_{j+1}$ is obtained from $\tau_j$ with a twist split, then $\tau_{j+1}^X$ is obtained from $\tau_j^X$ with infinitely many splits, only one of which is a twist split along $\gamma$. So we count the number of those twist splits instead. For $k\leq j<j'\leq l$, let $h_j^{j'}$ be the twist modelling function associated with $\bm\tau(j,j')$.

Denote $\ldots y^{-1}, y^0, y^1,\ldots$ the ordered biinfinite sequence in $\R$ of lower obstacles for $\tau_k^X$ --- suppose for simplicity that $y^0=0$; and let $x_k^1,\ldots,x_k^r$ be the upper obstacles for $\tau_k^X$ lying within the interval $(0,2\pi)$; these have a natural bijection with the branch ends in $\tau_k^X$ hitting $A_\gamma$. 

Note that, according to the conventions as in Remark \ref{rmk:twistnaturemodelling}, $\ldots y^{-1}, y^0, y^1,\ldots$ are lower obstacles for $\tau_j^X$ for all $j\geq k$, too. For each $k<j\leq l$ we define instead $x_j^i\coloneqq h_k^j(x_k^i)$ for $i=1,\ldots,r$. Each $x_j^i$ is an upper obstacle for the respective $\tau_j^X$. So each sequence $(x_j^i)_j$ describes the alterations of a branch end in $\tau_k$, hitting $A_\gamma$, along the sequence $\bm\tau(k,l)$: the position of its endpoint along the carrying image of $\gamma$ changes along the sequence, moving long a direction specified by the $A_\gamma$-orientation on $\gamma$. As already specified in Remark \ref{rmk:twistnaturemodelling}, $j'>j\Rightarrow x_{j'}^i\geq x_j^i$.

The elementary move between $\tau_j$ and $\tau_{j+1}$, then, is a (twist) splitting if and only if there are two indices $1\leq a\leq r$, $b\geq 0$ such that $x_j^a<y^b<x_{j+1}^a$. In this case the choice for the indices $a, b$ is unique.

As a consequence of point \ref{itm:tmfbeyondrot} in Remark \ref{rmk:rotbasics}, for each $i=1,\ldots,r$ we have $x_l^i<x_k^i +2\pi(\rot(k,l)+3)$. In each interval $\left[2\pi \zeta,2\pi(\zeta+1)\right)$, $\zeta\in\mathbb N$, there can be at most $N_3$ lower obstacles of any fixed $\tau_j^X$, because they must be lifts of endpoints of distinct branch ends avoiding $A_\gamma$ and favourable. Hence, in the splitting sequence $\bm\tau$, between each $x_k^i$ and the corresponding $x_l^i$ there at most $N_3\left(\rot(k,l)+3\right)$ lower obstacles (of any of the tracks in the sequence). The total number of splits in $\bm\tau$ is bounded by total number of indices $j$ such that a choice of $a,b$ as above exists: since also $r\leq N_3$, a bound is $N_3^2\left(\rot(k,l)+3\right)$.
\end{proof}

\begin{lemma}\label{lem:vertexsetnontwist}
Suppose that, in a generic splitting sequence $\bm\tau$ of almost tracks on a surface $S$, a curve $\gamma$ is a twist curve for some $\tau_j,\tau_{j+1}$ and the elementary move between these two train tracks is \emph{not} a twist split about $\gamma$. Let $X$ be a regular neighbourhood of $\gamma$. Then $V(\tau_{j+1}|X)\subseteq V(\tau_{j}|X)$.
\end{lemma}
\begin{proof}
It suffices to prove the statement in the case the move occurring is a split which is spurious or far from $\gamma$. Suppose that $\alpha\in V(\tau_{j+1}|X)$ fails to be wide in $\tau_j|X$, i.e. $\hl_{\alpha}(j)>2\pi$ while $\hl_{\alpha}(j+1)<2\pi$: by point \ref{itm:farisininfluent} after Definition \ref{def:horizontallength}, we have that the split may only be spurious, because $\lfloor \hl_{\alpha}(j)/2\pi\rfloor$ and $\lfloor \hl_{\alpha}(j+1)/2\pi\rfloor$ are distinct quantities, and both well-defined up to isotopies. 

Let $b\subseteq \tau_j.\gamma$ be the branch that is about to be split. We use the conventions set up in Remark \ref{rmk:permanenceconventions} and repeat the notation used in the first paragraph there. Also, let $\tilde{\ul\alpha}_j, \tilde{\ul\alpha}_{j+1}$ be two lifts of $\ul\alpha_j, \ul\alpha_{j+1}$ via the maps $q_j,q_{j+1}$ respectively, with the same endpoints on $\R\times[-2,2]$. Let $[x_j,y_j]$ be the interval stretch for $\alpha$ in $\tau_j^X$ consistent with the choice of the lift $\tilde{\ul\alpha}_j$, and define $[x_{j+1},y_{j+1}]$ similarly, replacing all occurrences of the index $j$ with $j+1$. As a consequence of Remark \ref{rmk:sameorientationforarcs}, the points $x_j,x_{j+1}$ are upper obstacles while $y_j,y_{j+1}$ are lower ones.

Since the split is spurious, $\hat p^{-1}(R_b)$ has 1 or 2 components intersecting $\tau_j^X.\gamma$ (as many as the number of times $b$ is traversed by $\gamma$). 

In the rest of the proof, square brackets will enclose adaptations that apply for the case of $\gamma$ traversing $b$ twice. Let $b_1$[, $b_2$] be the [two] lift[s] of $b$ that lie along $\tau_j^X.\gamma$; and let $e_1$[, $e_2$] be the only favourable branch end[s] sharing a switch with $b_1$[, $b_2$, respectively]. 

As seen in point \ref{itm:horizontalstretch} of Remark \ref{rmk:annulusinducedbasics}, $\ul\alpha_{j+1}$ is the concatenation of train paths $\rho_{j+1}^h,\hs_\alpha(j+1),\rho_{j+1}^f$ where the first is an incoming ramp for $\tau_{j+1}^X$ hitting $A_\gamma$, while the last is an outgoing ramp avoiding $A_\gamma$ and favourable. A similar decomposition $\rho_j^h,\hs_\alpha(j),\rho_j^f$ for $\ul\alpha_j$ holds.

Using the conventions of Remark \ref{rmk:permanenceconventions}, $\tau_{j+1}$ is obtained from $\tau_j$ unzipping a single zipper, defined on an interval $[-\epsilon,t]$ for $1<t<2$. This lifts to an infinite family of zippers for $\tau_j^X$, and only one [two, resp.] of these intersects $\bar\nei(\tau^X.\gamma)$: we call it [them, resp.] $\kappa_1$[, $\kappa_2$ according to which of $b_1,b_2$ they traverse].

Note that, since unzipping $\kappa_1$ [and $\kappa_2$] does not realize a twist split, this [these two, resp.] zipper[s] cannot intersect $A_\gamma$. Therefore $\rho_j^h=c_{\tau_j^X}\left(\rho_{j+1}^h\right)$ i.e. the tie collapse cannot create a segment lying along $\tau_j^X.\gamma$, and in particular $x_j=x_{j+1}$. The only way to have $\hl_\alpha(j+1)<2\pi<\hl_\alpha(j)$, then, is that $y_{j+1}<y_j$.

If $\rho_j^f$ does not begin with $e_1$ [nor with $e_2$], then also $\rho_j^f=c_{\tau_j^X}\left(\rho_{j+1}^f\right)$. But this would imply that ${\ul\alpha}_j,{\ul\alpha}_{j+1}$ have the same horizontal stretch, so $\hl_\alpha(j)=\hl_\alpha(j+1)$ leading to a contradiction.

Thus, [without loss of generality] $\rho_j^f$ begins at $e_1$; and this means that there is a fake obstacle $x_j<w_j<y_j$ such that $[w_j,y_j]\times\{0\}$ is one of the connected components of $q_j^{-1}(b_1)$; in particular no obstacles lie in $(w_j,y_j)$. In order to have the hypothesized shortening in horizontal length, $\kappa_1$ shall begin along the component of $\partial_v\bar\nei(\tau_j^X)$ which lies close to $q_j(y_j,0)$. Then the conventions of Remark \ref{rmk:permanenceconventions} imply that $y_{j+1}<w_j$ and, for all obstacles $z<w_j$ for $\tau_j^X$, $y_{j+1}>z$. 

Note that the upper obstacle $x_j+2\pi\in (x_j,y_j)$ because $\hl_\alpha(j)>2\pi$. Necessarily, then $w_j>x_j+2\pi$ and, from the above paragraph, also $y_{j+1}>x_j+2\pi$. This implies $\hl_{\alpha}(j+1)=y_{j+1}-x_j>2\pi$, contrarily to our assumption.
\end{proof}

\begin{lemma}[Three ramps criterion]\label{lem:threeramps}
Let $\bm\tau$ be a generic almost track splitting sequence such that a curve $\gamma\in\cc(S)$ stays a twist curve in the subsequence $\bm\tau(k,l)$. Suppose the following:
\begin{itemize}
\item $\gamma$ is not combed in $\tau_l$;
\item $\bm\tau(k,l)$ consists of subsequences which have been alternatively modelled according to the conventions of Remark \ref{rmk:permanenceconventions} --- possibly including moves which consist of trivial unzips --- and of Remark \ref{rmk:twistnaturemodelling} (the latter subsequences, of course, must be of twist nature);
\item under this fixed model, there is an incoming ramp $\rho_l^h: (-\infty,0]\rightarrow \tau_l^X$, hitting $A_\gamma$, such that $\rho_k^h\coloneqq c_{\tau_k^X}\circ \rho_l^h$ also intersects $\tau_l.\gamma$ only in the point $c_{\tau_k^X}\circ \rho_l^h(0)$.
\end{itemize}

Then, if $\alpha\in\cc(\tau_l^X)$ has a train path realization which includes $\rho_l^h$ then $\hl_\alpha(k)> \hl_\alpha(l)-2\pi$; in particular $\rot_{\bm\tau}(\gamma;k,l)\leq 1$.
\end{lemma}
\begin{proof}
Given a subsequence $\bm\tau(r,r')$ of twist nature, and a model of it after the prescriptions given in Remark \ref{rmk:twistnaturemodelling}, the latter may easily be converted into a model $\bm\sigma$ after Remark \ref{rmk:permanenceconventions}, except that some unzips may be trivial. In order to get this alternative model, do \emph{not} apply the isotopies that, after each unzip in $\bm\tau(r,r')$, make sure that $\tau_j.\gamma$ stays unchanged; and keep trace of changes of $\tau_j.\gamma$ via the maps $\E_j$ instead, as explained in Remark \ref{rmk:permanenceconventions}. This means, in particular, that rather than having a single map $q:\R\times[-2,2]\rightarrow \ol{S^X}$, in the new model we have individual maps $q_j$ for each entry in $\bm\sigma$. Anyway, for each $r\leq j\leq r'$ one may find an index $j'$ such that $q^{-1}(\tau_j)=q_{j'}^{-1}(\sigma_{j'})$ (the indices do not correspond exactly because the trivial unzips appearing along $\bm\tau(r,r')$ have to be inserted as individual moves). Moreover, if $\bm\tau(r,r')$ is replaced with $\bm\sigma$, then the last bullet in the statement still holds.

So, ultimately, we can simplify the proof by supposing that the entire sequence $\bm\tau(k,l)$ is modelled after Remark \ref{rmk:permanenceconventions}, possibly with trivial moves.

Since $\gamma$ is not combed in $\tau_l$, $\tau_l^X$ features two branch ends $e^f,e^a$ which avoid $A_\gamma$ and are favourable and adverse, respectively. There are two outgoing ramps $\rho_l^f,\rho_l^a:[0,+\infty)\rightarrow \tau_l^X$ which begin at $e^f,e^a$ respectively.

Let $\tilde\rho_l^h$ be a lift of $\rho_l^h$ to $\R\times[-2,2]$ via $q_l$: $x_l^h\coloneqq \tilde\rho_l^h(0)$ is an upper obstacle for $\tau_l^X$. The hypothesis on $\rho_k^h$ implies that it is an incoming ramp for $\tau_k^X$ and, if one lifts it to a $\tilde\rho_k^h$ in $\R\times[-2,2]$ via $q_k$, so that $\tilde\rho_k^h,\tilde\rho_l^h$ depart from the same point of $\R\times\{-2,2\}$, then the upper obstacle $x_k^h\coloneqq\tilde\rho_k^h(0)=x_l^h$.

Let $\rho_k^f,\rho_k^a$ be outgoing ramps for $\tau_k^X$ obtained from $c_{\tau_k^X}\circ\rho_l^f,c_{\tau_k^X}\circ\rho_l^a$ respectively by trimming their initial subpaths lying along $\tau_k^X.\gamma$. Choose lifts $\tilde\rho_l^a,\tilde\rho_l^f$ of $\rho_l^a,\rho_l^f$ via $q_l$ so that, if $x_l^a\coloneqq\tilde\rho_l^a(0),x_l^f\coloneqq\tilde\rho_l^f(0)$, then $x_l^a<x_l^f<x_l^a+2\pi$ and $x_l^h<x_l^f$.

Let $\tilde\rho_k^a,\tilde\rho_k^f$ be lifts of $\rho_k^a,\rho_k^f$ whose endpoints on $\R\times\{-2\}$ are the same as $\tilde\rho_l^a,\tilde\rho_l^f$ respectively. Let $x_k^a\coloneqq\tilde\rho_k^a(0),x_k^f\coloneqq\tilde\rho_k^f(0)$.

Then, by Remark \ref{rmk:permanenceconventions}, $x_k^a\leq x_l^a$ and $x_k^f\geq x_l^f$. Also, note that the order of $x_l^a,x_l^f,x_l^a+2\pi$ along $\R\times\{0\}$ is the same as the order of the extremities of $\tilde\rho_l^a$, $\tilde\rho_l^f$, $\tilde\rho_l^a+(2\pi,0)$ along $\R\times\{-2\}$, which are also extremities of $\tilde\rho_k^a$, $\tilde\rho_k^f$, $\tilde\rho_k^a+(2\pi,0)$ (these paths are all disjoint); therefore also $x_k^a,x_k^f,x_k^a+2\pi$ have the same order.

So, joining the inequalities, $x_k^f< x_k^a+2\pi\leq x_l^a+2\pi < x_l^f+2\pi$. Let $\alpha\in\cc(\tau_l^X)$ be the arc defined by the train path $\ul\alpha$ having, among its lifts to $\R\times[-2,2]$, the concatenation of $\tilde\rho_l^h$, $[x_l^h,x_l^f]\times\{0\}$, $\tilde\rho_l^f$. In particular $\hl_{\alpha}(l)=x_l^f-x_l^h$ and, via a similar realization in $\tau_k^X$, $\hl_{\alpha}(k)=x_k^f-x_k^h < x_l^f+2\pi-x_l^h=\hl_{\alpha}(l)+2\pi$.

This construction covers all $\alpha\in\cc(\tau_l^X)$ whose realization includes $\rho_l^h$: it suffices to choose an appropriate $\rho_l^f$ at the beginning of this proof. In particular $\rot_{\bm\tau}(\gamma;k,l)\leq 1$, as claimed.
\end{proof}

\subsection{Twist split grouping}\label{sub:rearrang}
In our incoming estimates of distance in $\pa(S)$ induced by a train track splitting sequence, a big annoyance will be the potentially high number of twist splits that do not result in an any accordingly large contribution to the pants distance, as it is shown in Lemma \ref{lem:pantsboundunderdt}. This is the reason why we shall work to identify exactly how they alter the computation.

\begin{prop}[Split rearrangement for a twist curve]\label{prp:rearrang1}
Suppose that $\bm\tau=(\tau_j)_{j=0}^N$ is a generic train track splitting sequence on a surface $S$, such that a fixed curve $\gamma$ stays a twist curve, with one same twist collar $A_\gamma$ and sign $\epsilon$, throughout. Let $X$ be a regular neighbourhood of $\gamma$, and let $m\coloneqq\rot_{\bm\tau}(0,N)$.% and let $s_{far},s_{spu}$ be the numbers of split moves along $\bm\tau$ which are far from $\gamma$ and spurious, respectively.

Then there are another splitting sequence $\bm{\tau'}=(\tau'_j)_{j=0}^{N(5)}$ whose first and last entries are the same as $\bm\tau$, and three indices $0\leq N(1)\leq N(2)\leq N(3)\leq N(4) \leq N(5)$, with the following properties.
\begin{enumerate}
\item $\rot_{\bm\tau'}\left(0,N(2)\right)\leq 3$; there is an increasing map $f: [0,N]\rightarrow [0,N(2)]$ with $f(0)=0$, $f(N)=N(2)$ and such that, for all $0\leq j \leq N$, $\tau_j$ is obtained from $\tau'_{f(j)}$ with a splitting sequence of twist nature about $\gamma$ followed by a comb equivalence.
\item If $0<N(1)<N(2)$ then $\gamma$ is not combed in $\bm\tau'\left(0,N(1)-1\right)$ and it is in $\bm\tau'\left(N(1),N(2)\right)$; if these inequalities are not strict, instead, then $\gamma$ is either combed or not combed in the entire sequence $\bm\tau'\left(0,N(2)\right)$.
\item $\bm\tau'\left(N(2),N(3)\right)$ and $\bm\tau'\left(N(4),N(5)\right)$ have twist nature about $\gamma$, while\linebreak $\bm\tau'(N(3),N(4))$ consists of slides only.
\item Let $m'\coloneqq \rot_{\bm\tau'}(N(2),N(4))$ Then $m'\geq m-5$, while $\rot_{\bm\tau'}(N(2),N(3))=0$ and there is a factorization $N(5)-N(4)=m'k$ such that $\tau'_{k+j}=D_\gamma^\epsilon(\tau'_j)$ for all $N(4)\leq j\leq N(5)-k$.
\end{enumerate}
\end{prop}

A note of warning: the sequence $\bm\tau'$ will be built with the possibility that some entry is obtained from the previous one via isotopies only. It is clear, anyway, that one may delete the repeated entries in the sequence (forgetting about any model according to Remarks \ref{rmk:permanenceconventions} and \ref{rmk:twistnaturemodelling} that the sequence is given).

\begin{proof}
\step{1} replacement of $\bm\tau$ with a splitting sequence with control on the number of branches hitting $A_\gamma$.

\begin{figure}[h]
\centering
\includegraphics[width=.4\textwidth]{pushpull.pdf}
\caption{\label{fig:pushpull}A pull (above) and a push (below) as the result of an unzip.}
\end{figure}

We define two particular kinds of slide around $\gamma$: a \emph{push} is a slide which increases the number of branch ends hitting $A_\gamma$; a \emph{pull} is one which decreases said number (see Figure \ref{fig:pushpull}). We also define a \emph{f-push} and a \emph{f-pull} accordingly, but replacing the words `hitting $A_\gamma$' with `avoiding $A_\gamma$ and favourable'.

In this step a new splitting sequence $\bm\sigma=(\sigma_j)_{j=0}^M$ will be constructed, with the same beginning track as $\bm\tau$ and with the elements of the sequence orderedly comb equivalent to the ones of $\bm\tau$ (in particular the last entries of the two sequences will be comb equivalent), but with the property that no pulls occur; and that, if $\bm\sigma(M',M)$ is the maximal subsequence of $\bm\sigma$ where $\gamma$ is combed, then no f-pull occurs there, either. Recall from the observation before Lemma \ref{lem:twistcurvebasics} that, if $\gamma$ is combed in a track along a splitting sequence, than $\gamma$ will remain combed in the sequence, as long as it is a twist curve.

Let $\bm\sigma^0$ be a wide splitting sequence obtained from $\bm\tau$ via Proposition \ref{prp:deleteslidings}; and let $\ul{\bm\sigma}^0$ be the conversion of it into another regular splitting sequence, by decomposing each wide split into elementary moves. 

We wish to define recursively a family of splitting sequences $\ul{\bm\sigma}^i$ whose entries keep orderedly comb equivalent to the ones of $\bm\tau$: in particular there is an $A_\gamma$-family of twist collars for all entries of all these sequences. We require that $\ul{\bm\sigma}^i$ is obtained from adjoining $\ul{\bm\sigma}^i_-*\ul{\bm\sigma}^i_+$, where $\ul{\bm\sigma}^i_-$ includes no pull, and no f-pull where $\gamma$ is combed; and $\ul{\bm\sigma}^i_+$ is the translation of a wide splitting sequence $\bm\sigma^i_+$ into a regular one --- for $i=0$, $\ul{\bm\sigma}^0_-$ is empty while $\ul{\bm\sigma}^0_+=\ul{\bm\sigma}^0$. The subsequence of $\ul{\bm\sigma}^i_+$ accounting for the $j$-th wide split in $\bm\sigma^i_+$ will be called $\ul{\bm\sigma}^i_+(j)$.

Also, we require $|\ul{\bm\sigma}^i|$ to be the same for all $i$ while $|\ul{\bm\sigma}^i_+|$ is strictly decreasing as $i$ increases: we stop the recursion when $|\ul{\bm\sigma}^i_+|=0$ (or earlier), so that $\ul{\bm\sigma}^i_-$ is the sequence $\bm\sigma$ that we desired to build. 

Suppose $\ul{\bm\sigma}^0,\ldots, \ul{\bm\sigma}^i$ have been defined; we proceed further to $\ul{\bm\sigma}^{i+1}$. Let $(\alpha^i_j)_{j=0}^{s(i)}$ be the sequence of the splitting arcs employed along the sequence $\bm\sigma^i_+$. If in $\ul{\bm\sigma}^i_+$ there is no pull nor f-pull where $\gamma$ is combed, then the recursion stops here with $\bm\sigma=\ul{\bm\sigma}^i$.

Else, the first pull in the sequence, or f-pull while $\gamma$ is combed, occurs as part of a $\ul{\bm\sigma}^i_+(j)$ such that the corresponding wide split in $\ul{\bm\sigma}^i_+$ is twist: far wide splits, indeed, fail to produce a pull or a f-pull, and a spurious wide split can only produce a f-pull, and only before $\gamma$ gets combed. Denote $t_1,t_2$ the indices such that $\ul{\bm\sigma}^i_+(j)=\ul{\bm\sigma}^i(t_1,t_2)$; but call $\xi_1\coloneqq \ul\sigma^i_{t_1}$, $\xi_2\coloneqq \ul\sigma^i_{t_2}$, for simplicity, the train tracks before and after this wide twist split.

Orient the splitting arc $\alpha^i_j$, embedded in $\bar\nei(\xi_1)$, so that it traverses branches of $\gamma$ according to the $A_\gamma$-orientation. 

Define $B_1,B_2\subseteq \br(\xi_1)$ as the two collections of branches such that $\bigcup B_1,\bigcup B_2$ give the two connected components of $\ol{\xi_1.\alpha^i_j\setminus\xi_1.\gamma}$ --- either, or both, may empty if the specified set is connected or empty. Let $R_1,R_2$ be the unions of the branch rectangles $R_b$ for $b\in B_1$, $b\in B_2$ respectively; and let $\alpha^i_{j1}=\alpha^i_j\cap R_1$, $\alpha^i_{j2}=\alpha^i_j\cap R_2$. They are the images of two disjoint small zippers (if either is empty, just ignore it): unzipping $\xi_1$ along $\alpha^i_{j1}$ and $\alpha^i_{j2}$, one gets a sequence of slides, none of which is a pull or a f-pull, turning $\xi_1$ into $\xi_1'$. In this latter almost track there is a splitting arc $\alpha^i_{j3}$ which corresponds to $\alpha^i_j$ (via Proposition \ref{prp:combpersistence}), and it traverses only branches contained in $\xi_1'.\gamma$.

With a series of twist splits, rearrange the branch ends with their endpoints lying along $\alpha^i_{j3}$: move all the ones hitting $A_\gamma$ past the ones that avoid it. The result, $\xi_2'$, is comb equivalent to the wide split along $\alpha^i_{j3}$, that is $\xi_2$.

We define $\ul{\bm\sigma}^{i+1}_-$ by adjoining $\ul{\bm\sigma}^i(0,t_1)$ to the elementary moves seen above which turn $\xi_1$ into $\xi_1'$ and then $\xi_2'$.

If $\xi_2$ is the last entry of the sequence $\bm\sigma^i_+$ then $\bm\sigma^{i+1}_+$ can be considered to be a trivial sequence with the entry $\xi_2'$ only, and the recursion ends here. If it is not then $\xi_2'$, being comb equivalent to $\xi_2$, carries a splitting arc which corresponds to $\alpha^i_{j+1}$; by Proposition \ref{prp:combpersistence} the wide split of it, with the same parity as $\alpha^i_{j+1}$ in $\xi_2$, gives an almost track which is comb equivalent to the entry of $\bm\sigma^i_+$ which succeeds $\xi_2$.

Continuing this way one gets a wide splitting sequence $\bm\sigma^{i+1}$ whose entries are comb equivalent to the entries of $\bm\sigma^i$ from $\xi_2$ on. As $\xi_2$ is not the first entry of $\bm\sigma^i_+$, $|\bm\sigma^i_+|>|\bm\sigma^{i+1}_+|$. This concludes the description of the recursive construction.

\step{2} grouping together the moves which are not twist; definition of the subsequence $\bm\tau'(0,N(2))$.

The argument used in this step is going to have some lines in common with the one of Lemma \ref{lem:postpone} about central split postponement. The underlying idea is: scan the sequence $\bm\sigma$ neglecting all twist moves, and performing all the other ones instead. At a later stage, the twist moves forgotten here will be performed altogether to give rise to the subsequence $\bm\tau'(N(2),N(5))$.

Consider the sequence of indices $0=j_0\leq j_1<\ldots <j_{2r-1}\leq j_{2r}=M$ such that, for all $i$ such that the following expressions make sense, the sequence $\bm\sigma(j_{2i},j_{2i+1})$ has twist nature whereas no split or slide in $\bm\sigma(j_{2i+1},j_{2i+2})$ is a twist one. We will have $j_1=0$ if and only if $\bm\sigma$ does \emph{not} begin with a twist split or slide, and similarly $j_{2r-1}=M$ if and only if $\bm\sigma$ ends with a twist split or slide.

While the sequences $\bm\sigma(j_{2i+1},j_{2i+2})$ will be modelled with the conventions explained in Remark \ref{rmk:permanenceconventions}, the sequences of twist nature $\bm\sigma(j_{2i},j_{2i+1})$ will be realized with the conventions of \ref{rmk:twistnaturemodelling}. In particular, along each sequence $\bm\sigma(j_{2i},j_{2i+1})$, the carrying images $\sigma_j.\gamma$ and the twist collar $A_\gamma(j)$, for $j_{2i}\leq j\leq j_{2i+1}$, are all the same subset of $S$; and each of these sequences shall be regarded as one only affecting the picture within $A_\gamma(j)$, associated to a twist modelling function $h_i$ and, correspondingly, to a bijection $H_i:S\rightarrow S$.

Let $q_j:\R\times[-2,2]\rightarrow S^X$ be the parametrization of $S^X$ which has been set up for $\sigma_j$, in compliance of Remark \ref{rmk:twistparam} and of either Remark \ref{rmk:permanenceconventions} or \ref{rmk:twistnaturemodelling} as specified above: it is handy to add to $q_j$ the request that all upper obstacles for $\sigma_j^X$ are points of $2\pi\mathbb Q\times\{0\}$; while all lower and fake obstacles are points of $2\pi(\R\setminus\mathbb Q)\times\{0\}$. With this request one may as well suppose that all $h_i$ have $h_i(2\pi\mathbb Q,0)=2\pi \mathbb Q$, and this will be helpful when removing the twist moves, as we will see.

In accordance with what seen in Remark \ref{rmk:generic_move_as_unzip}, for each index $j\in[j_{2i+1},j_{2i+2}-1]$ for some $i$, if the elementary move performed on $\sigma_j$ is not a central split, then it is the result of unzipping a zipper $\kappa_j$. For ease of notation, if the move is a central split, let $\kappa_j$ be a splitting arc traversing the branch that is being split. In order to comply with the further request on parametrizations specified above, necessarily the following statement shall hold, for any $j$ such that the $j$-th move is not a central split and for any $x\in\R$ such that $q_j(x,0)$ lies on the same tie as the last point of $\kappa_j$: $x\in 2\pi(\R\setminus\mathbb Q)$ if $\kappa_j$ intersects $A_\gamma$ (since twist moves have to be excluded, this means that its unzip describes necessarily a push); while $x\in 2\pi\mathbb Q$ otherwise.

Define, for all $i=0,\ldots r-1$, for all $j_{2i+1}< j \leq j_{2i+2}$, $\E_{(j)}\coloneqq  \E_{j_{2i+1}}\circ\ldots\circ\E_{j-1}: (S,\sigma_j)\rightarrow (S,\sigma_{j_{2i+1}})$. Here the maps $\E_j$ are relative to the sequence $\bm\sigma$, and defined as in Remark \ref{rmk:permanenceconventions}. Set then, for notational convenience, $\E_{(j_{2i+1})}=\mathrm{id}_S$, and $\E_{[i]}=\E_{(j_{2i+2})}$. Also, define $\F_{(j)}\coloneqq  \F_{j-1}\circ\ldots\circ\F_0$, which is defined on some subset of $S$ which we do not make explicit: for our purposes, it is important only that it includes the twist collar $A_\gamma(0)$. Note that there is a discrepancy between the definitions of $\F_{(j)}$ and $\E_{(j)}$.

Now, for all $j_{2i+1}\leq j < j_{2i+2}$, define $\phi_j:S\rightarrow S$ by
$$\phi_j|_{A_\gamma(j)}\coloneqq \F_{(j)}\circ H_0^{-1}\circ\E_{[0]}\circ\ldots\circ H_{i-1}^{-1}\circ \E_{[i-1]}\circ H_i^{-1}\circ \E_{(j)}\quad\text{and}\quad \phi_j|_{S\setminus A_\gamma(j)}=\mathrm{id}_{S\setminus A_\gamma(j)}.$$

This map is, in words: a bijection of $S$ which fixes $A_\gamma(j)$, is discontinuous only along $\sigma_j.\gamma$, and `undoes' the twist splits and slides performed in $\bm\sigma(0,j)$: while each such move pushes the branch ends hitting $A_\gamma$ forwards, this map moves them backwards by the same length. So define $\rho^i_j\coloneqq \phi_j(\sigma_j)$. We see that $\rho^i_j$ is a generic almost track, because we have kept upper obstacles in $2\pi\mathbb Q$ and lower and fake ones in $2\pi(\R\setminus\mathbb Q)$. Clearly, the existing parametrization $q_j$ can be used with respect to $(\rho_j^i)^X$, too, respecting the requests of Remark \ref{rmk:twistparam}.

Let then $\bm\rho^i=(\rho^i_j)_{j_{2i+1}}^{j_{2i+2}}$: this is a sequence resembling $\bm\sigma({j_{2i+1}},{j_{2i+2}})$ except that the effects of all the twist splits occurred earlier along $\bm\sigma$ have been cancelled. More precisely, the twist modelling function
\begin{equation}\label{eqn:h_tot}
h^*_i\coloneqq h_i(h_{i-1}(\cdots h_0(x,t)\cdots))
\end{equation}
induces a map $H^*_i$ such that $H^*_i(\rho^i_j)=\sigma_j$, for all $j_{2i+1}\leq j \leq j_{2i+2}$.

For each fixed ${j_{2i+1}}\leq j<{j_{2i+2}}$, the zipper/splitting arc $\kappa_j$ is still a zipper/multiple branch for $\rho^i_j$ and the unzip/central multiple split of it gives $\rho^i_{j+1}$. But, in general, if $\kappa_j$ is a zipper, its unzip may possibly describe \emph{not} a single elementary move, but more than one; or even just an isotopy. Similarly, if $\kappa_j$ is a splitting arc for $\sigma_j$, it may be the case that $\kappa_j$ is not a splitting arc in $\rho^i_j$. So, $\bm\rho^i$ in general is \emph{not} a splitting sequence, but each $\rho^i_{j+1}$ ($j_{2i+1}\leq j < j_{2i+2}$) is carried by the previous $\rho^i_j$. Also, note that $\rho^i_{j_{2i+2}}=\rho^{i+1}_{j_{2i+3}}$ for all $0\leq i\leq r-2$. Let $\bm\rho\coloneqq (\bm\rho^0)*\cdots*(\bm\rho^{r-1})$ (which is not a splitting sequence either, but the concatenation makes sense).

Each unzip may be subdivided into several unzips along shorter zippers, each of which give a single elementary move (or an isotopy). With this subdivision a splitting sequence $(\bm\rho^i)'$ is built, which touches orderedly all almost tracks in the sequence $\bm\rho^i$ and respects the conventions of Remark \ref{rmk:permanenceconventions}. Patch together all these pieces to get a new splitting sequence (possibly with trivial moves) $\bm\rho'\coloneqq (\bm\rho^0)'*\cdots*(\bm\rho^{r-1})'$. Note that, if $\xi$ is one of the almost tracks inserted between $\rho^i_j$ and $\rho^i_{j+1}$ (for suitable $i,j$) then $H^*_i(\xi)$ is isotopic to either $\sigma_j$ or $\sigma_{j+1}$; which, in turn, is comb equivalent to $\tau_{j'}$ for a suitable $j'$.

Let $(\bm\rho')^u$ be the maximal subsequence of $\bm\rho'$ where $\gamma$ is not combed, and let $(\bm\rho')^c$ the maximal subsequence where $\gamma$ is combed. Either of these may be empty or trivial. Index $\bm\rho'=(\rho'_j)_{j=0}^{N(2)}$ and let $N(1)\in [1,N(2)]$ be an index such that:
\begin{itemize}
\item either $(\bm\rho')^u$ is empty, or $(\bm\rho')^u=\bm\rho'(0,N(1)-1)$;
\item either $(\bm\rho')^c$ is empty, or $(\bm\rho')^u=\bm\rho'(N(1), N(2))$.
\end{itemize}

By construction, one has an increasing map $f_\sigma:[0,N]\rightarrow[0,M]$ such that $f_\sigma(0)=0$, $f_\sigma(N)=M$ and $\sigma_{f_{\sigma}(j)}$ comb equivalent to $\tau_j$ for all $j$. Also, for $0\leq j\leq M$, $\sigma_j$ is obtained from $\rho^i_j$ (for the correct $i$) via a splitting sequence of twist nature, as seen. This is enough to build the claimed map $f:[0,N]\rightarrow [0,N(2)]$.

%If the move turning $\sigma_j$ into $\sigma_{j+1}$ is far, then $\rho^i_j$ is also turned into $\rho^i_j$ with a single, far elementary move. If the move from $\sigma_j$ to $\sigma_{j+1}$ is spurious instead, more than one move may be required from $\rho^i_j$ to $\rho^i_{j+1}$: switching from one to the other requires either the multiple split with respect to $\alpha_j$ with anchors $P_1,P_2$ and the parity that was previously specified, or the unzip of $\kappa_j$. It is possible to decompose these moves into elementary moves, each of which complies 

%\begin{figure}
%\centering
%\def\svgwidth{.85\textwidth}
%\input{rearrang_multisplit.pdf_tex}
%\caption{\label{fig:rearrang1}In the splitting sequence $\bm\sigma$, the splitting arc $\alpha_j$ specifies a wide left split, spurious (but not bispurious) with respect to the twist curve $\gamma$. In $\bm\rho$, the multiple split specified by the same data is not a wide split instead, and shall be subdivided into elementary moves in order to get a splitting sequence. A multiple split is specified by unzipping two suitable zippers. The picture exemplifies why unzipping the first one results into a sequence of slides, while acting on the second one is a sequence of wide splits which are all spurious but one which is twist.}
%\end{figure}

%More precisely, $\rho^i_j$ is turned into $\rho^i_{j+1}$ via a multiple split, plus an isotopy, if $j$ is a critical index, and via a wide splitting otherwise. In the critical case, then, we can expect that more than one wide split is needed to turn $\rho^i_j$ into $\rho^i_{j+1}$. We examine more closely how many take place: in order to do this we regard the picture only up to isotopy, and therefore we only consider the multiple split.

%The splitting data is the same as in the sequence $\bm\sigma$: we multiply/widely split along $\alpha_j$ with the same parity specified in $\bm\sigma$; and if the parity is not central, with respect to the anchors $P_j^{(1)},P_j^{(2)}$ previously specified.

%Suppose the specified parity is not central (see Figure \ref{fig:rearrang1}). This multiple split consists of unzipping along a zipper $\kappa_1$, and then unzipping the newly obtained almost track $\rho^i_{j+1/2}$ along another zipper $\kappa'_2$. Each of the two unzips factors into a wide splitting sequence, as specified in Remark \ref{rmk:zipvssplit}. Due to the way we have given names to $P_j^{(1)},P_j^{(2)}$, $(\kappa_1)_P$ does not include any endpoint of favourable branch ends. This means that $\kappa_1$ is a small zipper, and that its unzip is equivalent to a sequence of slides.

%As for the unzip along $\kappa'_2$: let $\kappa_2^1,\ldots,\kappa_2^s$ be the \emph{large} zippers occurring the canonical decomposition of $\kappa'_2$, enumerated in their right order (so we forget about the small zippers). All $(\kappa_2^{s'})_P$, for $1\leq s'\leq s$, lie along the carrying image of $\gamma$. Then $\kappa_2^s$ will contain the point $P_j^{(1)}$; moreover, one can see, with the help of Remark \ref{rmk:zipvssplit}, that the unzips about $\kappa_2^1,\ldots,\kappa_2^{s-1}$ are each given by a single wide twist split, while the unzip about $\kappa_2^s$ may correspond to a wide spurious split, or a pair of wide splits one of which is spurious and the other twist. Moreover, since $(\kappa'_2)_P$ is an embedding with its image contained in $\rho^i_{j+1/2}.\gamma$, necessarily $s\leq N_3$.

%Suppose now that the multiple splitting under analysis is central instead. Then it is the result of two (simultaneous) unzips along two zippers $\kappa_1,\kappa_2$ with $(\kappa_1)_P,(\kappa_2)_P$ disjoint; this is followed by a central split. Similarly as in the case examined above but more easily, each large zipper occurring in the canonical decomposition of $\kappa_1$ and $\kappa_2$ gives a twist split. The final central split is spurious.

%Note that some care is needed when considering the compliance with the conventions set up in Remark \ref{rmk:permanenceconventions} here. The parametrization of $\rho^i_{j+1}$ with respect to the one of $\rho^i_j$, and the difference between the two as subsets of $S$, has been already made precise when they have been defined from $\sigma_j,\sigma_{j+1}$. This means that $\rho^i_{j+1}$ is \emph{not} a setwise realization of the prescribed multiple split on $\rho^i_j$; and not even the standard supposition that a twist split only changes the picture in $A_\gamma$ is respected. But we will not use those hypotheses here.

% involves at most $N_3s_{spu}$ twist splits; and, similarly as the original $\bm\sigma$, $s_{far}$ far splits and $s_{spu}$ spurious splits.

\step{3} estimation of $\rot_{\bm\rho'}\left(0,N(2)\right)$.

Let $X$ be an annular neighbourhood of $\gamma$ in $S$. We only cover the case $0<N(1)<N(2)$, as the other ones are simple adaptations. 

We claim that there is an incoming, hitting ramp $\delta^h_{N(2)}$ in $(\rho'_{N(2)})^X$ such that $c_{(\rho'_0)^X}\circ \delta^h_{N(2)}$ is an hitting ramp for $(\rho'_0)^X$.

Recall, first of all, that each zipper involved in each of the subsequences\linebreak $\bm\sigma(j_{2i+1},j_{2i+2})$ induces a single elementary move; let $\sigma_j,\sigma_{j+1}$ be the almost tracks before and after the move.

If the move is a push: all the upper obstacles for $\sigma_j^X$ are also upper obstacles for $\sigma_{j+1}^X$. If $\delta:(-\infty,0]\rightarrow \sigma_{j+1}^X$ is a ramp hitting $A_\gamma(j+1)$, and $q_{j+1}^{-1}\left(\delta(0)\right)$ is a collection of upper obstacles for $\sigma_j^X$, too, then $c_{\sigma_j^X}\circ \delta$ is also a ramp hitting $A_\gamma(j)$ --- i.e. no segment of $c_{\sigma_j^X}\circ \delta$ lies along $\sigma_j^X.\gamma$. This can be understood from Figure \ref{fig:pushpull}, even though it is simplified.

The zipper $\kappa_j$ unzipped in this move is the same employed to turn $\rho^i_j$ into $\rho^i_{j+1}$; and, despite the result of this unzip not being necessarily a single elementary move, it is seen that the upper obstacles for $(\rho^i_j)^X$ are a subset of the ones for $(\rho^i_{j+1})^X$ and that a similar property as above holds: i.e. if $\delta:(-\infty,0]\rightarrow (\rho^i_{j+1})^X$ is an incoming ramp hitting $A_\gamma(j+1)$, with $q_{j+1}^{-1}\left(\delta(0)\right)$ a collection of upper obstacles for $(\rho^i_j)^X$, then $c_{(\rho^i_j)^X}\circ\delta$ is a ramp for $(\rho^i_j)^X$ hitting $A_\gamma(j)$.

If the elementary move from $\sigma_j$ to $\sigma_{j+1}$ is not a push, recall that it is not a pull or a twist move, either. The unzip that realizes it keeps $q_j^{-1}(\sigma_j^X)\cap\left(\R\times[0,1]\right)= q_{j+1}^{-1}(\sigma_{j+1}^X)\cap\left(\R\times[0,1]\right)$, and so also $q_j^{-1}(\rho_j^X)\cap\left(\R\times[0,1]\right)= q_{j+1}^{-1}(\rho_{j+1}^X)\cap\left(\R\times[0,1]\right)$. This implies that, for \emph{any} incoming ramp $\delta:(-\infty,0]\rightarrow (\rho^i_{j+1})^X$ hitting $A_\gamma(j+1)$, $c_{(\rho^i_j)^X}\circ\delta$ is a ramp for $(\rho^i_j)^X$ hitting $A_\gamma(j)$.

Considering that $\bm\rho'$ only inserts intermediate stages between the entries of $\bm\rho$, then, it is possible to pick $\delta^h_{N(2)}$ an incoming ramp for $(\rho'_{N(2)})^X$ with $(q'_{N(2)})^{-1}\left(\delta^h_{N(2)}(0)\right)$ a subset of the upper obstacles of $(\rho'_0)^X$. The argument seen above yields that both $c_{(\rho'_{N(1)})^X}\circ\delta^h_{N(2)}$ and $c_{(\rho'_0)^X}\circ\delta^h_{N(2)}$ are incoming ramps hitting $A_\gamma$ in $(\rho'_{N(1)})^X$, $(\rho'_0)^X$ respectively.

So the sequence $\bm\rho'\left(0,N(1)-1\right)$, if nonempty, complies with the hypotheses of Lemma \ref{lem:threeramps}; hence $\rot_{\bm\rho'}\left(\gamma;0,N(1)-1\right)\leq 1$.

As for the sequence $(\bm\rho')^c$, recall that no f-pulls occur here. So an argument entirely similar to the above shows that there is an outgoing, favourable ramp $\delta^f_{N(2)}$ for $\rho'_{N(2)}$ with $c_{(\rho'_{N(1)})^X}\circ\delta^h_{N(2)}$ a favourable ramp for $(\rho'_{N(1)})^X$. This means that, if $\alpha\in V(\rho'_{N(2)})$ has a train path realization which includes $\delta^h_{N(2)}$ and $\delta^f_{N(2)}$ then $\hl\left(\rho'_N(1),\alpha\right)=\hl\left(\rho'_N(2),\alpha\right)$ i.e. $\alpha\in V(\rho'_{N(1)})$. The move turning $\rho'_{N(1)-1}$ into $\rho'_{N(1)}$ makes $\gamma$ into a combed curve, so it must be a spurious split: hence, by Lemma \ref{lem:vertexsetnontwist}, $V(\rho'_{N(1)})\subseteq V(\rho'_{N(1)-1})$. The presence of the element $\alpha$ both in the latter set and in $V(\rho'_{N(2)})$ implies that $\rot_{\bm\rho'}\left(\gamma;N(1)-1, N(2)\right)=0$.

By point \ref{itm:concatrot_above} in Remark \ref{rmk:rotbasics}, $\rot_{\bm\rho'}\left(0,N(2)\right)\leq \rot_{\bm\rho'}\left(0,N(1)-1\right)+\linebreak \rot_{\bm\rho'}\left(N(1)-1,N(2)\right)+2 =3$.

\step{4} definition of the subsequence $\bm\tau'(N(2),N(5))$ and conclusion.

Let $h_{tot}\coloneqq h^*_{r-1}$ (see (\ref{eqn:h_tot}) above) and let $H_{tot}$ be the corresponding self-map of $S$. Note that, by construction, $\sigma_M=H_{tot}(\rho'_{N'})$. Applying Lemma \ref{lem:functiongivestwist} to the twist modelling function $h_{tot}$ is therefore possible to build a splitting sequence $\bm\omega$, having twist nature about $\gamma$, beginning with $\rho'_{N(2)}$ and ending with $\sigma_M$. Let $m'$ be the rotation number of the sequence $\bm\omega$.

As a consequence of Lemma \ref{lem:dehn+remainder}, $\bm\omega$ can be replaced by a new sequence $\bm\omega'=(\omega'_j)_{j=0}^{Q}$ of twist nature, with $\omega'_0=\rho'_{N(2)}$, $\omega'_Q$ comb equivalent to $\sigma_M$ (and to $\tau_N$) such that, for suitable integers $Q', k'$ we have $Q=Q'+k'm'$ and $\omega'_{j+k'}= D_\gamma^\epsilon(\omega'_j)$ for all $Q'\leq j\leq Q-k'$; while $\rot_{\bm\omega'}(\gamma;0,Q')=0$.

Let $\bm\xi$ be a series of slides that turns $\omega'_Q$ into $\tau_N$; so the sequence $D_\gamma^{-\epsilon m'}\cdot\bm\xi$ (i.e. the application of $D_\gamma^{-\epsilon m'}$ to all elements in the sequence $\bm\xi$) turns $\omega'_{Q'}$ into $D_\gamma^{-\epsilon m'}(\tau_N)$. Lemma \ref{lem:functiongivestwist} gives a new splitting sequence $\bm\omega''=(\omega''_j)_{j=0}^{m'k}$, with twist nature about $\gamma$, with $\omega''_{j+k}= D_\gamma^\epsilon(\omega''_j)$ for all $0\leq j\leq m'(k-1)$, and such that $\omega''_0=D_\gamma^{-\epsilon m'}(\tau_N)$ and $\omega''_{m'k}=\tau_N$.

Define finally $\bm\tau'=\bm\rho'*\bm\omega'(0,Q')*(D_\gamma^{-\epsilon m'}\cdot\bm\xi)*\bm\omega''$. This splitting sequence satisfies all requirements in the statement with $N(3)=N(2)+Q'$ and $N(4)=N(3)+ (\text{length of }\bm\xi)-1$. In particular, due to point \ref{itm:concatrot_above} in Remark \ref{rmk:rotbasics}, $m'\geq \rot_{\bm\tau'}(0,N(5))-\rot_{\bm\tau'}(0,N(2))-2 \geq m-5$.
\end{proof}

\begin{rmk}
The rearrangement procedure has a good behaviour with respect to the following properties --- meaning that if all tracks in $\bm\tau$ have the specified property then all tracks in $\bm\tau'$ have the same property, too.
\begin{itemize}
\item \textit{Recurrence and transverse recurrence.} In Remark \ref{rmk:recurrence_at_extremes} we have noted that all train tracks in a splitting sequence are recurrent if the last track in the sequence is. Moreover, they are all transversely recurrent if the first track in the sequence is. And $\bm\tau,\bm\tau'$ have the same endpoints.
\item \textit{Being cornered.} For each train track $\tau'_j$ there is a train track $\tau_i$ with a natural correspondence between the components of $S\setminus\nei(\tau'_j)$ and the ones of $S\setminus\nei(\tau_i)$, under which they are pairwise diffeomorphic.
\end{itemize}
\end{rmk}

\begin{rmk}\label{rmk:twist_disjointness}
We report a property of twist curves noted in \cite{mosher}, p. 215:
\begin{claim}
Let $\Gamma\subseteq \cc^0(S)$ be a family of essential curves which are pairwise disjoint up to isotopies, and are all twist curves for a given almost track $\tau$. Then, even if the carried images of these curves are not necessarily pairwise disjoint, it is possible to take a family of pairwise disjoint twist collars $A_\gamma$, for all $\gamma\in\Gamma$. For each of the $\gamma\in\Gamma$ which are combed in $\tau$, one may also choose what side of $\gamma$ the collar $A_\gamma$ must lie.
\end{claim}
\end{rmk}

\begin{lemma}[Small interference of twist curves]\label{lem:smallinterference}
Let $\bm\tau=(\tau_j)_{j=0}^N$ be a generic splitting sequence of almost tracks such that a curve $\gamma\in \cc^0(S)$ remains a twist curve throughout a subsequence $\bm\tau(k,l)$. If either of the following is true, then $\rot_{\bm\tau}(\gamma;k,l)=0$.
\begin{itemize}
\item There is a curve $\gamma_1$, intersecting $\gamma$ essentially, that also remains a twist curve throughout $\bm\tau(k,l)$.
\item There is a family of curves $\gamma_1,\ldots,\gamma_m$, all disjoint up to isotopy from $\gamma$ (not necessarily from each other), such that $\bm\tau(k,l)$ consists of subsequences of twist nature, each with respect to one of the curves $\gamma_j$.
\end{itemize}
\end{lemma}

\begin{proof}
Let $X$ be a regular neighbourhood of $\gamma$ in $S$.

In the first scenario: we claim that, for all $k\leq j\leq l$, $\pi_X(\gamma_1)\subseteq \cc(\tau_j^X)$ is actually a subset of $V(\tau_j^X)$. Note, first of all, that it is surely it is not empty.

If the claim is false, one of the branches in $\tau_j^{X}.\gamma$ is traversed twice, in the same direction, by an arc in the family $\pi_{X}(\gamma_1)$; and, if this is true, then also $\gamma_1$ traverses twice and in the same direction one of the branches in $\tau_j.\gamma$. But this would contradict the fact that, as a twist curve, $\gamma_1\in W(\tau_j)$.

This means that $V(\tau_k^{X})\cap V(\tau_l^{X})\supseteq \pi_{X}(\gamma_1)\not=\emptyset$. By point \ref{itm:rot_vs_dt_vertices} in Remark \ref{rmk:rotbasics}, this implies that $\rot_{\bm\tau}(\gamma,k,l)=0$.

In the second scenario, call $\bm\tau(k_i,l_i)$ the subsequence of $\bm\tau(k,l)$ which has twist nature with respect to $\gamma_i$; and model it according to Remark \ref{rmk:twistnaturemodelling}, applied to the twist curve $\gamma_i$. As the curves $\gamma_i$ are all disjoint from $\gamma$, the $j$-th elementary move in $\bm\tau(k,l)$ changes $\tau_{k+j-1}$ only within the relevant $A_{\gamma_i}(j-1)=A_{\gamma_i}(j)$, thus it does not affect $A_\gamma(j)$ nor $\tau_{k+j}.\gamma$, because of the disjointness property (Remark \ref{rmk:twist_disjointness} above). In other words, for $k_i\leq j < l_i$, $c_{\tau_j^X}|_{\tau_{j+1}^X}$ is the identity map out of $A_{\gamma_i}$. This means that we can employ the same parametrization $q:\R\times[-2,2]\rightarrow S^X$, in compliance with Remark \ref{rmk:twistparam}, and focused on the twist curve $\gamma$, for all almost tracks in the sequence $\bm\tau(k,l)$.

There are now two sub-cases to be considered. If none of the $\gamma_i$ traverses one same branch of the corresponding $\tau_{k_i}$ as $\gamma$, then all splits in the sequence $\bm\tau(k,l)$ are far from $\gamma$ and so $V(\tau_l^X)\subseteq V(\tau_k^X)$ by Lemma \ref{lem:vertexsetnontwist}. So $\rot_{\bm\tau}(\gamma,k,l)=0$ because the two vertex sets are not disjoint (point \ref{itm:rot_vs_dt_vertices} in Remark \ref{rmk:rotbasics}).

Now suppose that at least one of the curves $\gamma_i$ traverses a branch of $\tau_{k_i}.\gamma$. Then in $\tau_{k_i}^X$ there is necessarily an $A_\gamma$-adverse branch, also traversed by $\gamma_i$; and thus, there is one also in $\tau_k^{X}$. Let then $B\subseteq \tau_k^X.\gamma$ be a union of consecutive branches whose extremities are both small branch ends, whose endpoints are switches incident to a $A_\gamma$-favourable and an $A_\gamma$-adverse branch end, respectively, and such that all switches in $\inte(B)$ are incident to branch ends hitting $A_\gamma$. Then, if a twist curve $\delta$ for $\tau_k$ has no essential intersection with $\gamma$, then no lift of it to $S^X$ traverses any branch in $B$; nor any lift of a twist collar $A_\delta\subseteq S$ to $S^X$ may intersect $B$.

This implies, by recursion, that $B$ remains delimited by a pair of branch ends which are favourable and adverse, respectively, after each elementary move in the sequence $\bm\tau(k,l)$. Let $f$ be the favourable one of these two branch ends: by what has been said so far, $c_{\tau_j^X}(f)=f$ for all $k\leq j\leq l$.

Pick any $\alpha\in V(\tau_l^X)$ which traverses $f$. Due to the decomposition specified in point \ref{itm:horizontalstretch} of Remark \ref{rmk:annulusinducedbasics}, a train path realization $\ul\alpha_l$ of $\alpha$ will include an incoming, favourable ramp $\rho_l^f$ ending with $f$, followed by an embedded stretch $\hs(\tau_l,\alpha)$ along $\tau_l^X.\gamma$, and finally an outgoing ramp $\rho_l^h$ hitting $A_{\gamma}$, and starting at a branch end $e$.

The arc $\alpha$ also belongs to $\cc(\tau_k^X)$. A train path realization of $\alpha$ in $\tau_k^{X}$ is $c_{\tau_k^X}\circ\ul\alpha_l$ (possibly with a reparametrization). Note that both $c_{\tau_k^X}\circ\rho_l^f$, $c_{\tau_k^X}\circ\rho_l^h$ are ramps in $\tau_k^X$ because $c_{\tau_k^X}$ is the identity on both $f,e$. 

Therefore also $\hs(\tau_k,\alpha)=\hs(\tau_l,\alpha)$, hence $\alpha\in V(\tau_k^{X})$ and $\rot(\gamma;k,l)=0$.
\end{proof}

Now we are going to use some machinery that was already set up in \cite{mms}, so our hypotheses on the considered train track splitting sequences become more restrictive. Also, we will use shorthand notations like $\bm\tau(I)\coloneqq \bm\tau(\min I,\max I)$, where $I$ is an interval in $\mathbb Z$ and $\bm\tau$ is a splitting sequence indexed by a superinterval of $I$.

\begin{defin}
A train track splitting sequence $\bm\tau=(\tau_j)_{j=0}^N$ on a surface $S$ \nw{evolves firmly} in a possibly disconnected subsurface $S'$ of $S$ if, for each $0\leq j\leq N$, $V(\tau_j)$ fills exactly the subsurface $S'$.
\end{defin}

%\begin{rmk}\label{rmk:paritykeepsnonempty}
%In a parity splitting sequence $\bm\sigma=(\sigma_j)_{j=0}^N$ of cornered recurrent train tracks, all tracks fill the same surface $S'$. This is because, if no central split is involved, then the open sets $S\setminus \sigma_j$ are the diffeomorphic for any two values of $j$.

%Moreover, given any $\gamma\in\cc(S)$, if $\sigma_j|\nei(\gamma)\not=\emptyset$ for a given $j$ then $\sigma_{j'}|\nei(\gamma)\not=\emptyset$ for all $0\leq j'\leq N$.

%Suppose that this second claim is false for a specific index $j'$. As $\sigma_{j'}$ is recurrent, each branch $b\in\br(\sigma_{j'}^{\nei(\gamma)})$ is part of a train path with both endpoints on $\partial \bar S^{\nei(\gamma)}$; also, this train path shall have both endpoints on the same component of $\partial \bar S^{\nei(\gamma)}$ in order for $V(\sigma_{j'}^{\nei(\gamma)})$ to be empty. If $b$ is part of two different train paths, and the endpoints of the two lie on opposite components of $\partial \bar S^{\nei(\gamma)}$, then parts of the two paths can be patched to get a representative of an element of  $V(\sigma_{j'}^{\nei(\gamma)})$, a contradiction. 

%This means that $\br(\sigma_{j'}^{\nei(\gamma)})$ can be partitioned into two subsets, associating to each branch $b$ the component of $\partial\bar S^{\nei(\gamma)}$ where the endpoints of a train path through $b$ lie. A consequence of this is that $S^{\nei(\gamma)}\setminus \sigma_{j'}^{\nei(\gamma)}$ includes a topological annulus among its connected components. But, as $\sigma_{j'}^{\nei(\gamma)}$ is obtained from $\sigma_j^{\nei(\gamma)}$ via parity splittings or folds plus slides, $S^{\nei(\gamma)}\setminus \sigma_{j'}^{\nei(\gamma)}$ and $S^{\nei(\gamma)}\setminus \sigma_{j}^{\nei(\gamma)}$ are diffeomorphic. And this is impossible, since there an annulus among $S^{\nei(\gamma)}\setminus \sigma_{j}^{\nei(\gamma)}$ would mean $V(\sigma_{j}^{\nei(\gamma)})=\emptyset$.
%\end{rmk}

\begin{defin}\label{def:etc}
Let $\bm\tau=(\tau_j)_{j=0}^N$ be a generic train track splitting sequence on $S$, evolving firmly in a subsurface $S'$, not necessarily connected.

A curve $\gamma\in\cc(S)$, and essential in one of the non-annular connected components of $S'$, is an \nw{effective twist curve} for $\bm\tau$ if 
$$d_{\nei(\gamma)}(\tau_0,\tau_N)\geq 4\mathsf{K}_0+ 19,$$
where $\nei(\gamma)$ is a regular neighbourhood of $\gamma$ in $S$, and $\mathsf{K}_0$ is the constant defined in Theorem \ref{thm:mmsstructure}.
\end{defin}

Note that the given definition does not require that $\pi_{\nei(\gamma)}\left(V(\tau_0)\right), \pi_{\nei(\gamma)}\left(V(\tau_N)\right)\not=\emptyset$, because this is automatic by the request that all vertex cycles fill the same $S'$. Also, note that for an effective twist curve $\gamma$ in a splitting sequence $\bm\tau$ necessarily $I_\gamma\not=\emptyset$, by the first point of Theorem \ref{thm:mmsstructure}; in other words, an effective twist curve is, in particular, a twist curve for some tracks in the sequence $\bm\tau$. Moreover, $d_{\nei(\gamma)}(\tau_{\min I_\gamma},\tau_{\max I_\gamma})\geq 2\mathsf{K}_0+ 19$.

\begin{defin}\label{def:arranged}
Let $\bm\tau=(\tau_j)_{j=0}^N$ be a generic splitting sequence of cornered birecurrent train tracks on a surface $S$, evolving firmly in some subsurface $S'$, not necessarily connected. Let $\gamma_1,\ldots,\gamma_r\in\cc(\tau_0)$, and for each $1\leq t \leq r$ let $I_t$ be the accessible interval of $\nei_t\coloneqq \nei(\gamma_t)$ a regular neighbourhood of $\gamma_t$.

We say that the splitting sequence $\bm\tau$ is \nw{$(\gamma_1,\ldots,\gamma_r)$-arranged} if the following conditions hold. For each $t=1,\ldots,r$, there is a \nw{Dehn interval} for $\gamma_t$: a subinterval $DI_t\subset I_t$ such that $\bm\tau(DI_t)$ has twist nature with respect to $\gamma_t$ with $\rot_{\bm\tau}(\gamma_t;DI_t)\geq 2\mathsf{K}_0+4$, and is arranged into Dehn twists as prescribed in Lemma \ref{lem:dehn+remainder}, \emph{with no remainder}. Given any two intervals $DI_t$, for distinct values of $t$, they may intersect in at most one point; the curves are listed with the condition that the sequence $(\max DI_s)_{s=1}^r$ is increasing. Also, let $G_{t-}\coloneqq[0,\min I_t]$; $G_{t+}\coloneqq[\max I_t,N]$; $I_{t-}\coloneqq[\min I_t,\min DI_t]$; $I_{t+}\coloneqq [\max DI_t,\max I_t]$.

As an additional piece of notation, we subdivide each interval $DI_t$ into subintervals $DI_t(0), \ldots, DI_t(m_t-1)$, where $m_t=\rot_{\bm\tau}(\gamma_t;DI_t)$, dividing Dehn twists from one another. More precisely the maximum of each subinterval is also the minimum of the following one; if we call $a_t(i)=\min DI_t(i)$ for $i=0,\ldots,m_t-1$ and $a_t(m_t)=\max DI_t(m_t-1)$, then the sequence $a_t(0),\ldots,a_t(m_t)$ is an arithmetic progression and, for each $i,j$ such that $a_t(0) \leq a_t(i) +j \leq a_t(m_t)$, we have $\tau_{a_t(i)+j}=D_{\gamma_t}^{\epsilon i}\left(\tau_{a_t(0)+j}\right)$.

Let now $\gamma_1,\ldots,\gamma_r\in\cc(\tau_0)$ be the effective twist curves of $\bm\tau$. We say that $\bm\tau$ is \nw{effectively arranged} if it is $(\gamma_1,\ldots,\gamma_r)$-arranged and the following holds. For each $1\leq t \leq r$, if $i,j\in G_{t-}\cup I_{t-}$, then $d_{\nei_t}(\tau_i,\tau_j)\leq \mathsf{K}_0+2\mathsf{R}_0+9$; while, if $i,j\in I_{t+}\cup G_{t+}$, then $d_{\nei_t}(\tau_i,\tau_j)\leq \mathsf{K}_0+6$ --- and if $i,j\in I_{t+}$ then $d_{\nei_t}(\tau_i,\tau_j)\leq 6$. %The number of twist splits about $\gamma_t$ occurring in $R\bm\tau(I_{t-}^\rar)$ is bounded by $2N_4(2N_4+2+2\mathsf{R}_0)$.

Here $\mathsf{K}_0$ is as in Definition \ref{def:etc}, and $\mathsf{R}_0=\mathsf{R}_0(S,Q)$ is as defined in Lemma \ref{lem:reversetriangle}, where $Q$ is the quasi-isometry constant introduced in Theorem \ref{thm:mms_cc_geodicity}.
\end{defin}

\begin{prop}[Effective rearrangement]\label{prp:rearrang2}
Let $\bm\tau=(\tau_j)_{j=0}^N$ be a generic splitting sequence of cornered birecurrent train tracks, which evolves firmly in some (possibly disconnected) subsurface $S'$ of a surface $S$. Let $\gamma_1,\ldots,\gamma_r\in\cc(\tau_0)$ be the effective twist curves of $\bm\tau$, listed so that the sequence $(\max I_{\gamma_s})_{s=1}^r$ is increasing.

Then there is a $(\gamma_1,\ldots,\gamma_r)$-effectively arranged splitting sequence $\rar\bm\tau=(\rar\tau_j)_{j=0}^{N'}$ which begins and ends with the same train tracks as $\bm\tau$. The two splitting sequences, in particular, have the same family of effective twist curves. Moreover, if $I^\rar_s$ is the accessible interval of $\gamma_s$ in $\rar\bm\tau$, then the sequence $(\max I^\rar_s)_{s=1}^m$ is also increasing.
%The number of splits which are not twist splits about any of the effective twist curves $\gamma_1,\ldots,\gamma_q$ in the sequence $\bm\tau$ and in the sequence $\rar\bm\tau$ is the same.
\end{prop}

\begin{proof}
We will define recursively (decreasing the indices) a sequence of splitting sequences $\bm\tau=\bm\tau^{r+1},\bm\tau^r,\ldots,\bm\tau^1=\rar\rm\tau$ on the surface $S$ --- each of those will be indexed as $\bm\tau^s=(\tau^s_j)_{j=0}^{N^s}$ --- with the following properties. Their entries will all be cornered birecurrent train tracks; the first and last entries in each of these sequences will always be the same as in $\bm\tau$; each $\bm\tau^s$, informally speaking, is `partially' effectively arranged: it satisfies the requests in the definition of effectively arranged only for the curves $\gamma_t$ with $t\geq s$.

More precisely: for all $1\leq t \leq r$, denote $I^s_t$ the accessible interval relative to the curve $\gamma_t$ in the splitting sequence $\bm\tau^s$; and $G_{t-}^s\coloneqq[0,\min I_t^s]; G_{t+}^s\coloneqq[\max I_t^s,N^s]$. For the indices $s\leq t\leq r$ an interval $DI^s_t\subseteq I^s_t$ will be provided, together with $I^s_{t-}\coloneqq[\min I^s_t,\min DI^s_t]$; $I^s_{t+}\coloneqq [\max DI^s_t,\max I^s_t]$. The following claim will be true for each $s=r+1,r,\ldots,1$:
\begin{claim}
In the splitting sequence $\bm\tau^s$, for all $s\leq t\leq r$:
\begin{enumerate}
\item $i,j\in I^s_{t+}\cup G^s_{t+}\Longrightarrow d_{\nei_t}(\tau^s_i,\tau^s_j)\leq \mathsf{K}_0+6$; and $\rot_{\bm\tau^s}(\gamma_t;I^s_{t+})\leq 2$;
\item $d_{\nei_t}(\tau^s_0,\tau^s_{\min DI^s_t})\leq \mathsf{K}_0+9$, and $i,j\in G^s_{t-}\cup I^s_{t-}\Longrightarrow d_{\nei_t}(\tau^s_i,\tau^s_j)\leq \mathsf{K}_0+2\mathsf{R}_0+9$;
\item $DI^s_t$ is a Dehn interval for $\gamma_t$: $\bm\tau^s(DI^s_t)$ is a sequence of twist nature with respect to $\gamma_t$, with $\rot_{\bm\tau^s}(\gamma_t;DI^s_t)\geq 2\mathsf{K}_0+4$, arranged in Dehn twists with no remainder.
\end{enumerate}
\end{claim}
Eventually, it will suffice to define $\rar\bm\tau\coloneqq \bm\tau^1$. 

\step{1} recursive construction of the sequences $\bm\tau^s$.

Fix $1\leq s\leq r$. Suppose that all $\bm\tau^i$ for $i\geq s+1$ have been defined, together with intervals $DI^i_t, I^i_{t-}, I^i_{t+}\subseteq I^i_t$ for $t$ in the range $i,\ldots,r$. For each fixed $i$, the intervals $DI^i_t$ are pairwise disjoint except possibly for a common endpoint.

We now build $\bm\tau^s$ from $\bm\tau^{s+1}$ with an application of Proposition \ref{prp:rearrang1}, with respect to $\gamma_s$, on a suitable subsequence of $\bm\tau^{s+1}$. We define $I^{s+1}_{s+}$ as follows: the idea is that $I^{s+1}_{s+}$ is an interval we do not want to apply Proposition \ref{prp:rearrang1} on, because it is already structured in twists with respect to other curves.
\begin{itemize}
\item If $\max I^{s+1}_s$ is contained in $DI^{s+1}_t$ for a $t\geq s+1$, let $J$ be the maximal concatenation of intervals $DI^{s+1}_u$, $r\geq u \geq s+1$, such that the maximum of one is the minimum of another, and $DI^{s+1}_t$ is part of the union. Let then $I^{s+1}_{s+}= I^{s+1}_s\cap J$.
\item If $\max I^{s+1}_s$ is not contained in any $DI^{s+1}_t$, define $I^{s+1}_{s+}\coloneqq \{\max I^{s+1}_s\}$.
\end{itemize}

Set now $\bm\sigma\coloneqq\bm\tau^{s+1}(\min I^{s+1}_s,\min I^{s+1}_{s+})$. Let $\bm\sigma'$ be the splitting sequence obtained from $\bm\sigma$ by application of Proposition \ref{prp:rearrang1}; and define $\bm\tau^s\coloneqq \bm\tau^{s+1}(G^{s+1}_{s-})*\bm\sigma'*\bm\tau^{s+1}(I^{s+1}_{s+}\cup G^{s+1}_{s+})$. As entries in $\bm\tau^s$ are given indices in the interval $[0,N^s]$, the three sequences which compose it are indexed by subintervals which we call $[0,a^s]$, $[a^s,b^s]$, $[b^s,N^s]$, respectively. With this indexing, the subsequence $\bm\tau^{s+1}(I^{s+1}_{s+})$ is copied to a subsequence of $\bm\tau^s$ indexed by an interval which we call $I^s_{s+}$; and it has $\min I^s_{s+}=b^s$.

Note that, as a consequence of the construction, the sequence $(\max I^s_t)_{t=1}^r$ is increasing. Given an index $1\leq i\leq r+1$, let $P(i)$ be the following property: ``for all $i \leq t \leq r$, $\max DI^i_t$ coincides with either $\min DI^i_{u}$ for an index $u\geq t$, or with $\max I^i_t$; and the sequences $(\min DI^i_t)_{t=i}^r, (\max DI^i_t)_{t=i}^r$ are increasing''.

We prove the following: 
\begin{claim}
If $P(i)$ is true for a given $i> 1$, then $DI^i_t\subseteq I^i_{(i-1)+}\cup G^i_{(i-1)+}$ for all $i\leq t\leq r$.
\end{claim}
\begin{proof}
There are two cases to consider: if $\max DI^i_i=\max I^i_i$, which is $\geq \max I^i_{i-1}$, then clearly, by construction of $I^i_{(i-1)+}$, $DI^{i}_i\subseteq I^{i}_{(i-1)+}\cup G^{i}_{(i-1)+}$; and for $i \leq t \leq r$, $DI^{i}_t\subseteq I^{i}_{(i-1)+}\cup G^{i}_{(i-1)+}$ is true because of the last sentence in $P(i)$.

If $\max DI^i_i=\min DI^i_u$ for some $u\geq i$ then, by the monotonicity of $(\min DI^{i}_t)_{t=i}^r$, necessarily $u=i+1$. Let then $J'$ be the maximal interval which contains $DI^i_i$ and is obtained as a union of intervals $DI^i_t$, $i\leq t\leq r$. Let $i\leq t'\leq r$ be the index such that $\max J'=\max DI^i_{t'}$. Suppose for a contradiction that there exists an index $i\leq t''\leq r$ with $DI^i_{t''}\not\subseteq I^i_{(i-1)+}\cup G^i_{(i-1)+}$; then, again by monotonicity, also $DI^i_i\not\subseteq I^i_{(i-1)+}\cup G^i_{(i-1)+}$.

Considering the way $I^i_{(i-1)+}$ has been defined in the sequence $\bm\tau^i$, in order for this to happen it must be $\max I^i_{i-1}>\max J'$. But in this case, also $\max I^i_{t'}>\max J'$ for all $i\leq t \leq r$, and in particular $\max I^i_{t'}>\max DI^i_{t'}$. However, by definition of $J'$ and $t'$, there is no index $u$ such that $\max DI^i_{t'}=\min DI^i_u$. So the index $t'$ is a contradiction to $P(i)$. 
\end{proof}

Now, $P(r+1)$ is voidly true; we assume $P(s+1)$, and further ahead we prove that $P(s)$ follows: this is necessary to legitimate the recursive construction of $\bm\tau^s$.

As said above, $DI^{s+1}_t\subseteq I^{s+1}_{s+}\cup G^{s+1}_{s+}$ for all $s+1\leq t\leq r$. So the construction of $\bm\tau^s$ causes the subsequences $\bm\tau^{s+1}(I^{s+1}_{t+})$ and $\bm\tau^{s+1}(DI^{s+1}_t$) to be copied to subsequences inside $\bm\tau^s\left(b^s,N^s\right)$. They will be indexed by intervals which we call $I^s_{t+}, DI^s_t$ respectively.

In $\bm\sigma'$, following the notation given in Proposition \ref{prp:rearrang1}, there is a subsequence indexed by the subinterval $[N(4), N(5)]$, and we call it $\bm\sigma''$: $\bm\sigma''$ has twist nature with respect to $\gamma_s$ and is arranged into Dehn twists with no remainder. When inserting $\bm\sigma'$ as the subsequence $\bm\tau^s(a^s,b^s)$ of $\bm\tau^s$, $\bm\sigma''$ will be given indices in a subinterval of $[a^s,b^s]$: we call it $DI^s_s$. Define, for $t\geq s$, $I^s_{t-}\coloneqq [\min I^s_t,\min DI^s_t]$.

With these definitions, $P(s)$ is `almost true': it is true that ``for all $s+1 \leq t \leq r$, $\max DI^s_t$ coincides with either $\min DI^s_{u}$ for an index $u\geq t$, or with $\max I^s_t$; and the sequences $(\min DI^s_t)_{t=s+1}^r, (\max DI^s_t)_{t=s+1}^r$ are increasing''. This is because, since all the relevant intervals $DI^{s+1}_t$ ($s+1 \leq t \leq r$) are contained in $I^{s+1}_{s+}\cup G^{s+1}_{s+}$, the corresponding $DI^s_t$ are contained in $[b^s,N^s]$, and this family of subintervals is just a translation of the corresponding family $DI^{s+1}_t$ in $[0,N^{s+1}]$.

But the construction forces $\max DI^s_s$ to be a lower bound for all intervals $DI^s_t$, $s+1 \leq t \leq r$; and we have $\max DI^s_s=\max I^s_s$ if $I^s_s$ is disjoint from all $DI^s_t,t>s$, while $\max DI^s_s=\min_{t>s}\left(\min DI^s_t\right)$ otherwise. So $P(s)$ is true.

\step{2} the properties claimed above for $\bm\tau^s$ hold.

Those properties are empty for $s=r+1$. Supposing that they hold for $\bm\tau^{r+1},\ldots,\linebreak \bm\tau^{s+1}$, we prove that they hold for $\bm\tau^s$, establishing an inductive argument.

Note, first of all, that if $t\geq s$ then $i,j\in G^s_{t-}\Rightarrow d_{\nei_t}\left(\tau^s_i,\tau^s_j\right)\leq \mathsf{K}_0$ by point 1 of Theorem \ref{thm:mmsstructure}. Same for $i,j\in G^s_{t+}$. Hence $d_{\nei_t}\left(\tau^s_{\min I^s_t},\tau^s_{\max I^s_t}\right)\geq d_{\nei_t}(\tau^s_0,\tau^s_{N^s})-2\mathsf{K}_0=2\mathsf{K}_0+19$; and $\rot_{\bm\tau^s}(\gamma_t;I^s_t)\geq 2\mathsf{K}_0+15$, by point \ref{itm:concatrot_above} in Remark \ref{rmk:rotbasics}.

Properties 1 and 3 for $t>s$: since $DI^{s+1}_t, I^{s+1}_{t+}\subseteq I^{s+1}_{s+}\cup G^{s+1}_{s+}$, we have that $\bm\tau^s(DI^s_t),\bm\tau^s(I^s_{t+})$ are copies of $\bm\tau^{s+1}(DI^{s+1}_t),\bm\tau^{s+1}(I^{s+1}_{t+})$, respectively. Hence, by inductive hypothesis, they have $\rot_{\bm\tau^s}(\gamma_t; DI^s_t)\geq 2\mathsf{K}_0+ 4$; and $\rot_{\bm\tau^s}(\gamma_t,I^s_{t+})\leq 2$ which yields $d_{\nei_t}\left(\tau^s_{\min I^s_{t+}}, \tau^s_{\max I^s_{t+}}\right)\leq 6$ via point \ref{itm:rot_vs_dist} of Remark \ref{rmk:rotbasics}.

Property 1 for $t=s$: we claim that $\rot_{\bm\tau^{s+1}}(\gamma_s;I^{s+1}_{s+})\leq 2$ --- and $\rot_{\bm\tau^s}(\gamma_s;I^s_{s+})\leq 2$, because the two rotation numbers are computed on two copies of the same sequence --- therefore $d_{\nei_s}\left(\tau^s_{\min I^s_{s+}}, \tau^s_{\max I^s_{s+}}\right)\leq 6$.

The claim is obvious if $I^{s+1}_{s+}=\{\max I^{s+1}_s\}$. If $\max I^{s+1}_s\in DI^s_t$ for a fixed $t>s$, call $A=I^{s+1}_{s+}\cap DI^s_t; B=[\min I^{s+1}_{s+},\min A]$. Then $\rot_{\bm\tau^{s+1}}(\gamma_s;A)=0$ because of Lemma \ref{lem:smallinterference} (both in case $\gamma_s,\gamma_t$ intersect and in case they do not). According to the same lemma, when $B$ is not a single point, necessarily all curves $\gamma_u$ with $DI^{s+1}_u\subseteq B$ must be essentially disjoint from $\gamma_s$, because $\rot_{\bm\tau^{s+1}}(\gamma_u;DI^{s+1}_u)\geq 2\mathsf{K}_0+ 4$. But then the sequence $\bm\tau^s(B)$ falls into the case covered in the second point of Lemma \ref{lem:smallinterference}, which yields $\rot_{\bm\tau^{s+1}}(\gamma_s;B)=0$. So $\rot_{\bm\tau^{s+1}}(\gamma_s,I^{s+1}_{s+})\leq 2$ by point \ref{itm:concatrot_above} in Remark \ref{rmk:rotbasics}. 

Property 2 for $t>s$: again because $\max I^{s+1}_{t-}= \min DI^{s+1}_t \in I^{s+1}_{s+}\cup G^{s+1}_{s+}$, the sequence $\bm\tau^s(G^s_{t-}\cup I^s_{t-})$ begins and ends with the same train tracks as $\bm\tau^s(G^{s+1}_{t-}\cup I^{s+1}_{t-})$. So $d_{\nei_t}(\tau^s_0,\tau^s_{\min DI^s_t})\leq \mathsf{K}_0+9$ by inductive hypothesis. According to Theorem \ref{thm:mms_cc_geodicity} the sequence $\left(\pi_{\nei_t}(V(\tau^s_j))\right)_{j\in G^s_{t-}\cup I^s_{t-}}$ is a $Q$-unparametrized quasi-geodesic in $\cc(\nei_t)$, so the reverse triangle inequality in Lemma \ref{lem:reversetriangle} gives, for $i,j\in G^s_{t-}\cup I^s_{t-}$, $d_{\nei_t}(\tau^s_i,\tau^s_j)\leq \mathsf{K}_0+2\mathsf{R}_0+9$ as required.

Property 2 for $t=s$: Proposition \ref{prp:rearrang1} above guarantees that $\rot_{\bm\tau^s}(\gamma_s,I^s_{s-})\leq 5$, as $\bm\tau^s(I^s_{s-})$ indeed corresponds, using the notation given in that Proposition, to the subsequence $\bm\tau'\left(0,N(4)\right)$ of the output sequence $\bm\tau'$. So, for all pairs $i,j\in I^s_{s-}$, we have $d_{\nei_s}(\tau^s_i,\tau^s_j)\leq 9$ (see point \ref{itm:rot_vs_dist} in Remark \ref{rmk:rotbasics}). Combine it with the previously noted bounds for $i,j\in G^s_{s-}$ to complete the proof that property 2 holds.

Property 3 for $t=s$: by point \ref{itm:concatrot_above} in Remark \ref{rmk:rotbasics}, $\rot_{\bm\tau^s}(\gamma_s;DI^s_s)\geq \rot_{\bm\tau^s}(\gamma_s;I^s_s) - \rot_{\bm\tau^s}(\gamma_s;I^s_{s-})- \rot_{\bm\tau^s}(\gamma_s;I^s_{s+})-4\geq 2\mathsf{K}_0+4$.
\end{proof}

\subsection{A bound on the number of highly twisting curves}
\label{sub:twistcurvebound}

\ul{Note:} In this subsection we still deal mostly with generic almost tracks. However, we make use of the diagonal extension machinery from \S \ref{sub:diagext}; so we have to consider some semigeneric almost tracks as well. The adjective `generic' will be made explicit when appropriate, anyway.

Given a surface $S$, recall that in Remark \ref{rmk:pickparameters} we have fixed the parameters $k,\ell$ involved in the definition of $\pa(S)$ and of $\pa(X')$ for $X'$ a subsurface of $S$. Let $M\coloneqq \max_{X'\subseteq S} M_6(X',k(X',S),\ell(X'))$, where the maximum is taken over all $X'\subseteq S$ non-annular subsurfaces, and $M_6(X',k,\ell)$ is defined as in Lemma \ref{lem:pantsquasiisom}.

Given a non-annular subsurface $X'$ and two train tracks $\sigma,\tau$ on $S$, suppose that $\pi_Y\left(V(\sigma)\right),\pi_Y\left(V(\tau)\right)\not=\emptyset$ for all $Y\subseteq X'$ non-annular subsurfaces. In this case define
$$d'_{\pa(X')}(\sigma,\tau)\coloneqq \sum_{\substack{Y\subset X'\text{ essential} \\ \text{and non-annular}}} [d_Y(\sigma,\tau)]_M.$$

And, if non-emptyness holds also for all projections onto annuli $Y\subseteq X'$, we may also define
$$d'_{\ma(X')}(\sigma,\tau)\coloneqq \sum_{Y\subset X'\text{ essential}} [d_Y(\sigma,\tau)]_M.$$

Similarly as in Theorem \ref{thm:mmprojectiondist}, the summations shall be meant over $Y\subset X'$ subsurfaces, counting only one representative for each isotopy class in $S$. Restating that theorem, also in the light of Lemma \ref{lem:pantsquasiisom} one has
\begin{eqnarray*}
d'_{\pa(X')}(\sigma,\tau)  & =_{(e_0,e_1)} &
d_{\pa(X')}(\pi_{X'} V(\sigma), \pi_{X'} V(\tau)); \\
d'_{\ma(X')}(\sigma,\tau)  & =_{(e_0,e_1)} & 
d_{\ma(X')}(\pi_{X'} V(\sigma), \pi_{X'} V(\tau)).
\end{eqnarray*}
in the two respective cases, for suitable constants $e_0(X',M, k,\ell), e_1(X',M, k,\ell)$. 

Suppose now that $V(\sigma|X')$ and $V(\tau|X')$ are vertices of $\pa(X')$ (resp. $\ma(X)$). Then $\pi_{X'} V(\sigma)$ and $\pi_{X'} V(\tau)$ are vertices there, too, i.e. the above formulas make sense for them. This implies, via Lemma \ref{lem:induction_vertices_commute}, that
\begin{eqnarray*}
d'_{\pa(X')}(\sigma,\tau) & =_{(e_0,e_1+ C_1)} &
d_{\pa(X')}\left(\sigma, \tau\right); \\
d'_{\ma(X')}(\sigma,\tau) & =_{(e_0,e_1+ C_1)} &
d_{\ma(X')}\left(\sigma, \tau\right).
\end{eqnarray*}

The aim of this subsection is to prove the following
\begin{prop}\label{prp:tcbound}
Let $\bm\tau=(\tau_j)_{j=0}^N$ be a generic, recurrent train track splitting sequence on a surface $S$ which evolves firmly in some subsurface $S'$ --- not necessarily a connected one. Let $X$ be (another) non-annular subsurface of $S$; let $\gamma_1,\ldots,\gamma_q\subseteq \cc(\tau_0)$ be curves all contained, and essential, in $X$; and suppose that $\bm\tau$ is $(\gamma_1,\ldots,\gamma_q)$-arranged (see Definition \ref{def:arranged}; in particular the sequence $(\max DI_t)_{t=1}^q$ is increasing).

Fix $0\leq k\leq l\leq N$ with $V(\tau_l|X)\in\pa^0(X)$ and such that $DI_t \subseteq [k,l]$ for all $1\leq t \leq q$. 

Then there are constants $C_3, C_4$, only depending on $S$, such that
$$q\leq C_3 d'_{\pa(X)}(\tau_k,\tau_l) + C_4.$$
\end{prop}

Before we start, anyway, we prove a lemma which will be of use in the following sections, too.

\begin{lemma}\label{lem:pantsboundunderdt}
Let $S$ be a surface, $X$ be a non-annular subsurface of $S$ (possibly $X=S$), $\gamma\in \cc(X)$.

Let $\tau$ be a generic almost track and $\bm\tau=(\tau_j)_{j=0}^N$ be a generic splitting sequence of almost tracks on $S$, with $\bm\tau(k,l)$ a sequence of twist nature about $\gamma$.

\begin{enumerate}
\item $\left(D_\gamma(\tau)\right)|X$ and $D_\gamma(\tau|X)$ are isotopic (here $D_\gamma:X\rightarrow X$, so it can be extended trivially to both $S$ and $S^X$, where the almost tracks lie).
\item If $V(\tau_k|X)$ is a vertex of $\pa(X)$ (resp. of $\ma(X)$), then $V(\tau_j|X)$ is one, too, for all $k\leq j\leq l$.
\item In the sequence $(\tau_j|X)_{j=k}^l$, each entry is fully carried by the previous one.
\item There is a bound $C_2(S)$ such that, if $V(\tau_k|X)$ is a vertex of $\pa(X)$, then $d_{\pa(X)}(\tau_k|X,\tau_l|X)\leq C_2$, and $d_Y\left(\tau_k|X,\tau_l|X\right)\leq C_2$ for all $Y\subseteq X$ non-annular subsurfaces.
\end{enumerate}
\end{lemma}
\begin{proof}
To prove claim 1, note that the map $D_\gamma:S\rightarrow S$ has a lift $\hat D: S^X\rightarrow S^X$ whose restriction to $\core(X)=X\cap \core(S^X)$ concides with the restriction of $D_\gamma:X\rightarrow X$. Therefore $\hat D: S^X\rightarrow S^X$ is isotopic to $D_\gamma:S^X\rightarrow S^X$. The claim follows from $\hat D(\tau|X)=\left(D_\gamma(\tau)\right)|D_\gamma(X)=\left(D_\gamma(\tau)\right)|X$.

For claims 2 and 3, model $\bm\tau(k,l)$ according to Remark \ref{rmk:twistnaturemodelling}. If $h$ is a twist modelling function associated with $\bm\tau(k,l)$, then by point \ref{itm:tmfbeyondrot} in Remark \ref{rmk:rotbasics}, one may assume $h(x,0)<x+2\pi\left(\rot(k,l)+3\right)$ for all $x\in\R$ which means that there is a twist modelling function $h'$ such that $h'(h(x,0),0)=x+2\pi(\rot(k,l)+3)$. This is associated with a splitting sequence of twist nature, which we call $\bm\sigma$, turning $\tau_l$ into $D_\gamma^{\epsilon(\rot(k,l)+3)}(\tau_k)$, where $\epsilon$ is the sign of $\gamma$ as a twist curve.

It is clear that $V\left(D_\gamma^{\epsilon(\rot(k,l)+3)}(\tau_k|X)\right)= D_\gamma^{\epsilon(\rot(k,l)+3)}\cdot V(\tau_k|X)$: here, $D_\gamma$ shall be meant as the Dehn twist about $\gamma$ as a diffeomorphism $S^X\rightarrow S^X$ or $X\rightarrow X$. So $V\left(D_\gamma^{\epsilon(\rot(k,l)+3)}(\tau_k|X)\right)$ is a vertex of $\pa(S)$ (resp. of $\ma(X)$) if and only if $V(\tau_k|X)$ is one, too; but the remark following Lemma \ref{lem:decreasingfilling}, applied to $\bm\sigma$, yields that in this case $V(\tau_l|X)$ is also a vertex of $\pa(X)$ (resp. $\ma(X)$). Moreover in the sequence $\bm\tau*\bm\sigma$, induced on $X$, each entry carries the following one, and the first one $\tau_k|X$ fully carries the last one $D_\gamma^{\epsilon(\rot(k,l)+3)}(\tau_k|X)$. So the carrying must be a full one at any intermediate stage of the sequence.

For claim 4: define, from $\bm\tau(k,l)$, a splitting sequence $\bm\tau'=(\tau'_j)_{j=0}^{N''}$ arranged in Dehn twists plus remainder, as in Lemma \ref{lem:dehn+remainder}. We adopt the notation used in the statement of that lemma. In particular, $V(\tau'_0)=V(\tau_k)$ and $D_\gamma^{\epsilon m}\cdot V(\tau'_{N'})=V(\tau'_{N''})=V(\tau_l)$. Since $\rot_{\bm\tau'}(\gamma;0,N')=0$, at most $3N_3^2$ splits occur in $\bm\tau'(0,N')$ by Lemma \ref{lem:twistsplitnumber}.

But, once the surface $S$ is fixed, the possible pairs $(\tau_0,\gamma)$ as in the statement are finitely many up to the action of $\mcg(S)$ (cfr. Lemma \ref{lem:vertexsetbounds}). The finiteness of possible choices implies that there is a bound $k_1$, depending on $S$ only, on $i(\alpha,\beta)$, for $\alpha\in V(\tau'_0),\beta\in V(\tau'_{N'})$. 

Let $p(\tau_{N'})$ be a pants decomposition of $S$ including the curve $\gamma$, chosen so that $\max_{\alpha \in p(\tau_{N'}), \beta \in V(\tau_{N'})} i(\alpha,\beta)$ is minimal among all pants decompositions with this property. So, again by finiteness of possible configurations up to $\mcg(S)$, a constant $k_2=k_2(S)$ exists with $\max_{\alpha \in p(\tau_{N'}), \beta \in V(\tau_{N'})} i(\alpha,\beta)\leq k_2$. Using the fact that $D_\gamma^{\epsilon m}\cdot p(\tau_{N'})=p(\tau_{N'})$ and $D_\gamma^{\epsilon m}\cdot V(\tau'_{N'})=V(\tau'_{N''})$, also $\max_{\alpha \in p(\tau_{N'}), \beta \in V(\tau_{N''})} i(\alpha,\beta)\leq k_2$.

We now use arguments similar to the ones in Lemma \ref{lem:induction_vertices_commute}. For each $Y\subseteq X$ non-annular subsurface, $\pi_Y V(\tau'_0),\pi_Y V(\tau'_{N'}), \pi_Y\left(p(\tau'_{N'})\right), \pi_Y V(\tau'_{N'})$ are all non-empty. Using Remark \ref{rmk:subsurf_inters_bound}, given $\alpha\in \pi_Y V(\tau'_0)$, $\beta\in\pi_Y V(\tau'_{N'})$ one has $i(\alpha,\beta)\leq 4k_1+4$, and a similar bound $4k_2+4$ holds for $i(\alpha,\beta)$ if $\alpha\in \pi_Y\left(p(\tau'_{N'})\right)$ and $\beta\in \pi_Y V(\tau'_{N'})$ or $\in \pi_Y V(\tau'_{N''})$.

Appealing to Lemma \ref{lem:cc_distance},
$$
d_Y\left(V(\tau'_0), V(\tau'_{N''})\right)\leq F(4k_1+4)+ 2 F(4k_2+4).
$$
This implies also that $d_Y\left(\tau'_0|X, \tau'_{N''}|X\right)$ is bounded, by the first statement in Lemma \ref{lem:induction_vertices_commute}.

Let $M> \max\{M_6(S),F(4k_1+4)+ 2 F(4k_2+4)\}$. Then, by Theorem \ref{thm:mmprojectiondist} and Lemma \ref{lem:pantsquasiisom} applied with the specified value $M$,
$$d_{\pa(X)}\left(\pi_X V(\tau'_0),\pi_X V(\tau'_{N''})\right)\leq e_1(X,M,k(X,S),\ell(X)).$$
Lemma \ref{lem:induction_vertices_commute}, finally, gives our claim.
\end{proof}

Rather than proving Proposition \ref{prp:tcbound} directly in the form given, we will lean on the following definition and lemma.\footnote{Many thanks to Saul Schleimer for having suggested the proof of this lemma.}

\begin{defin}Given a surface $S$ and a sequence $\alpha_1,\ldots,\alpha_r$ of distinct isotopy classes of essential simple closed curves on $S$, we call an increasing subsequence $\alpha_{j_1},\ldots,\alpha_{j_s}$ a \nw{chain} if, for all $1\leq i < s$, $\alpha_{j_{i+1}}$ intersects $\alpha_{j_i}$ essentially.
\end{defin}

\begin{lemma}\label{lem:chainbound}
Let $S$ be a surface and let $\alpha_1,\ldots,\alpha_r$ be a sequence of distinct isotopy classes of essential simple closed curves on $S$. Suppose that any chain in this sequence has at most $c$ elements: then $r\leq \xi(S)c$.
\end{lemma}
\begin{proof}
We define a partition $A_1,\ldots,A_t$ of $\{1,\ldots, r\}$ inductively, as follows --- the number $t\geq 1$ will be determined by the construction. First, assign $1\in A_1$.

Suppose now that all indices $1,\ldots,i$ ($i<r$) have been assigned to some set in the partition. Let then $i+1\in A_{u+1}$, where $u$ is the highest of all $v\geq 0$ such that, for each $1\leq v'\leq v$, there exists $j\in A_{v'}$, $j\leq i$, such that $\alpha_{i+1}$ intersects $\alpha_j$ essentially. In particular, if $\alpha_{i+1}$ is disjoint from all curves $\alpha_j$ with $j\leq i$ and $j\in A_1$, then we assign $i+1\in A_1$, too. Eventually, we define $t$ to be the highest index $u$ such that some $1\leq i\leq r$ has been assigned to $A_u$.

As a consequence of the construction, if $j\in A_u$ for $u>1$ then there exists an index $1\leq l(j)< j$ such that $l(j)\in A_{u-1}$. Moreover, all indices in a specified set $A_u$ of the partition correspond to pairwise disjoint curves: as such, they are part of a pants decomposition for $S$, so there are at most $\xi(S)$ indices in $A_u$. Therefore $t\geq r/\xi(S)$.

Let now $j\in A_t$. The indices $l^{(t-1)}(j),l^{(t-2)}(j),\ldots,l(j),j$ belong to $A_1,A_2,$ \ldots, $A_{t-1},A_t$ respectively (here $l^{(i)}$ denotes the iteration of $l$ for $i$ times), and the subsequence of $\alpha_1,\ldots,\alpha_r$ specified by those indices is a chain. Therefore $t\leq c$; and $r\leq \xi(S)c$ as claimed.
\end{proof}

As our argument to prove Proposition \ref{prp:tcbound} is based on the diagonal extension machinery, we wish to make sure that diagonal extensions work smoothly with respect to Dehn twists.

\begin{lemma}\label{lem:weightsaftertwist}
Let $\bm\tau=(\tau_j)_{j=0}^N$ be a generic, recurrent train track splitting sequence. Consider the induced train tracks $\rho_j\coloneqq \tau_j|X$ on a non-annular essential subsurface $X$ of $S$. Suppose that a subsequence $\bm\tau(k,l)$ has twist nature with respect to a curve $\gamma\in\cc(X)$, and that $\rot_{\bm\tau}(\gamma;k,l)=m\geq 1$.

If $\alpha\in \cc(\rho_N)$ intersects $\gamma$ essentially, then $\alpha$ traverses each branch of $\rho_0$ contained in $\rho_0.\gamma$ at least $m$ times.

Suppose now that each $\rho_j$ fills $S^X$; let $\alpha$ be a curve essentially intersecting $\gamma$, with $\alpha\in \cc(\delta_N)$ for some $\delta_N\in \f(\rho_N)$. Then there is a $\delta_0\in \f(\rho_0)$ carrying $\alpha$, built from a recurrent subtrack of $\rho_0$ that fills $S^X$ and includes $\rho_0.\gamma$; and $\alpha$ traverses all branches in $\delta_0.\gamma$ at least $m-1$ times.
\end{lemma}
\begin{proof}
We prove the first statement. The set $\pi_{\nei(\gamma)}(\alpha)$ is nonempty and contained in $\cc(\rho_l^{\nei(\gamma)})$. By definition of rotation number, each arc $\alpha'\in \cc(\rho_l^{\nei(\gamma)})$ has $\hl_{\alpha'}(k)>2\pi m$ so it traverses each branch in $\rho_k^{\nei(\gamma)}.\gamma$ at least $m$ times (see point \ref{itm:hl_vs_multiplicity} after Definition \ref{def:horizontallength}). So also $\alpha$ shall traverse each branch of $\rho_k.\gamma$ at least $m$ times; and the same must be true in $\rho_0.\gamma$ (see Remark \ref{rmk:decreasingmeasures}). In particular this is true if one picks $\alpha\in \cc(\rho_N)\subseteq \cc(\rho_l)$.

We work now on the second of the two statements. Without altering the truth of the statement we can suppose, from this point on, that $\bm\tau(k,l)$ has been already replaced with the subdivision into Dehn twists with remainder guaranteed by Lemma \ref{lem:dehn+remainder}. In particular we are replacing the original $\tau_l$ with a train track which is comb equivalent to it; and therefore we are also operating some comb/uncomb moves after the new $\tau_l$, in order to continue smoothly with the original splitting sequence $\bm\tau(l,N)$.

This means that there is an index $k\leq r< l$ with $\tau_l=D_\gamma^{\epsilon m}(\tau_r)$ where $\epsilon$ is the sign of $\gamma$ as a twist curve. By Lemma \ref{lem:pantsboundunderdt}, also $\rho_l=D_{\gamma}^{\epsilon m}(\rho_r)$; and clearly $\f(\rho_l)=D_{\gamma}^{\epsilon m}\cdot \f(\rho_r)$, so similar relations hold for the sets of carried curves: $\cc(\rho_l)=D_{\gamma}^{\epsilon m}\cdot\cc(\rho_r)$; $\cf(\rho_l)=D_{\gamma}^{\epsilon m}\cdot \cf(\rho_r)$.

Rather than the original statement, we will prove this other one:
\begin{claim}
Let $\alpha$ be a curve essentially intersecting $\gamma$, carried by $\delta_l\in \f(\rho_l)$ which is a diagonal extension of a recurrent subtrack $\sigma_l$ of $\rho_l$ that fills $S^X$. Then there exists a $\delta'_r\in \f(\rho_r)$ which is a diagonal extension of the recurrent subtrack $\sigma_r=D_{\gamma}^{-\epsilon m}(\sigma_l)$ of $\rho_r$ and contains $\rho_r.\gamma$; and $\alpha$ traverses each branch in $\delta'_r.\gamma$ at least $m-1$ times.
\end{claim}

This implies the desired statement as follows. According to Lemma \ref{lem:cf_decreasing},\linebreak $\cf(\rho_N)\subseteq \cf(\rho_l)$ so the above statement is true for $\alpha\in\cf(\rho_N)$ in particular. Also, the last statement in that lemma ensures that not only $\cf(\rho_r)\subseteq \cf(\rho_0)$, but also there is a $\delta'_0\in \f(\rho_0)$ fully carrying $\delta'_r$. In particular $\delta'_0$ will carry both $\alpha$ and $\gamma$, and $\alpha$ traverses at least $m-1$ times any branch in $\delta'_0.\gamma$.

Let $\delta_r\coloneqq D_\gamma^{-\epsilon m}(\delta_l)$, and let $\alpha_-\coloneqq D_{\gamma}^{-\epsilon m}(\alpha)\in \cc(\delta_r)$. Without loss of generality, we may suppose that $\sigma_r$ complies with the following maximality property: any almost track $\sigma_r\subsetneq \xi\subseteq \rho_r$ has the property that a branch of $\xi$ intersects one of $\delta_r$ at a point that is interior for both branches; and this property remains true when changing $\xi$ up to isotopies fixing $\sigma_r$.

Our $\delta'_r$ is going to be a diagonal extension of $\sigma'_r\coloneqq \sigma_r\cup\rho_r.\gamma$ --- which is a generic almost track, is recurrent as $\sigma_r$ is, and fills $S^X$. When $\sigma_r=\sigma'_r$, one just takes $\delta'_r\coloneqq \delta_r$ and proves the desired claim using the first claim of the present lemma. When $\sigma_r\not=\sigma'_r$ instead, $\sigma'_r$ and $\delta_r$ are not subtracks of a common almost track, hence $\alpha_-$ may not be carried by any diagonal extension of $\sigma'_r$; nevertheless we will show that $D_\gamma^\epsilon(\alpha_-)$ is carried by a suitable one, and this will allow us to conclude.

\step{1} finding an `efficient position' for the extra branches in $\br(\delta_r)\setminus\br(\sigma_r)$ (similar to what Definition \ref{def:efficientposition} requires for curves): this will be necessary in order to avoid monogons in the construction of $\delta'_r$. The construction has some points in common with \S 4.2 in \cite{mms}.

Let $a_1,\ldots,a_s$ be an enumeration of the branches in $\br(\delta_r)\setminus\br(\sigma_r)$ that are traversed by $\alpha_-$. For each $i=1,\ldots,s$, let $Q_i$ be the closure in $S$ of the connected component of $S\setminus \sigma_r$ which contains $a_i$. Also, if $\ul\gamma$ is a train path realization of $\gamma$ in $\rho_r$ (and in $\sigma'_r$), let $\beta_i^1,\ldots,\beta_i^{u(i)}$ be the maximal segments of this path which lie in $Q_i$ and are not contained entirely in its boundary. Any two of them are not necessarily disjoint but, since $\gamma$ is wide in $\rho_i$, each branch of $\sigma'_r$ is not traversed more than twice in total, with multiplicities, by this collection.

For each of the $\beta_i^j$, let $\ul\beta_i^j$ be a smooth embedded path in $\bar\nei(\sigma'_r)$ which is transverse to all ties and has its extremes at two cusps in $\partial Q_i$: $\ul\beta_i^j$ shall be constructed to be parallel to $\beta_i^j$ but, at each extremity of $\beta_i^j$, if it is not a corner of $\partial Q_i$, $\ul\beta_i^j$ shall continue traversing branches in $\partial Q_i$, until a corner is reached. Also, we choose $\ul\beta_i^j$ to sit inside $\inte(Q_i)$, except for its endpoints; and all $\ul\beta_i^j$ to be disjoint from each other except possibly for their endpoints.

Fix a branch $b\in\br(\sigma'_r)$ with $b\subseteq Q_i$, such at least one of the following is true:
\begin{itemize}
\item $b\subseteq\partial Q_i$ and $b$ is traversed by some $\ul\beta_i^j$ at least once;
\item $b$ is traversed at least twice in total by the family $\{\ul\beta_i^j\}_{j=1}^{u(i)}$.
\end{itemize}
If $b\subseteq \partial Q_i$, define the \emph{fold} at $b$ as the union of all sub-ties which are contained in $R_b([-1,1]\times[-1,1])$ and have one endpoint along $b$ and the other along some $\ul\beta_i^j$. Otherwise, define the fold at $b$ as the union of all sub-ties which are contained in $R_b([-1,1]\times[-1,1])$ and have both endpoints along some $\ul\beta_i^j$. Informally, the fold at $b$ is a triangle or rectangle containing all chunks of the arcs $\ul\beta_i^j$ which traverse $b$: the ones that one could sensibly fold together or fold to $b$.

Define a \emph{folding area} as a maximal union of folds whose interior is connected. For each folding area $A$, the boundary $\partial A$ includes open subsets of a finite number of ties of $\nei(\sigma'_r)$. Cut $A$ along each tie that has an open subset in $\partial A$, to get a collection of \emph{folding rectangles} and \emph{triangles}. For $A'$ a folding rectangle or triangle, call $\partial_h A'=\mathrm{int}\left(\partial A'\cap(\sigma_r\cup \ul\beta_i^1\ldots \cup\ul\beta_i^{u(i)})\right)$ and $\partial_v A'=\inte(\partial A'\setminus \partial_h A')\cup \left(A'\cap (\text{corners of }\partial Q_i)\right)$. The two components of $\partial_h A'$ are two tie-transverse arcs in $\nei(\sigma'_r)$. They have the same image under the tie collapse $c:\bar\nei_0(\sigma'_r)\rightarrow \sigma'_r$: we call it the \emph{crush} of $\partial_h A'$; its closure is a bounded train path in $\sigma'_r$ (up to reparametrization). Figure \ref{fig:diagext_efficientpos}, left, explains this construction.

We need to adjust each branch $a_i$ with respect to $\sigma'_r$ via isotopies which leave their endpoints fixed and keep the inclusion $a_i\subseteq Q_i$, in order to attain an analogous condition to efficient position for curves (Definition \ref{def:efficientposition}); in particular, we wish that the newly obtained smooth paths $a''_i$ are still embedded in $S^X$ individually; but they need not be pairwise disjoint.

More specifically, we require that, for each $i$:
\begin{enumerate}
\item $a_i''\setminus \sigma_r$ is connected and $a''_i\cap\sigma'_r$ is a union of branches of $\sigma'_r$ plus isolated points;
\item if $i'\not=i$ but $Q_{i'}=Q_i$, then $a''_i\cap a''_{i'}$ shall be exactly one of the following: empty, one common endpoint, or a union of branches of $\sigma'_r$; moreover, $a''_{i'}$ cannot intersect more than 1 component of $Q_i\setminus a''_i$;
\item the association of each connected component of $a''_i\cap\sigma'_r$ with the connected component of $a''_i\cap\bar\nei(\sigma'_r)$ which contains it is a bijection; 
\item given a connected component of $a''_i\cap\bar\nei(\sigma'_r)$, it is either a single tie, or it is transverse to all ties of $\nei(\sigma'_r)$ it encounters; in this second case, each of its endpoints either is a corner of $\partial Q_i$ or lies along $\partial_v\bar\nei(\sigma'_r)$;
\item if a connected component of $a''_i\cap\sigma'_r$ is an isolated point, then either
\begin{itemize}
\item it is contained in $\partial Q_i$ and there is a component of $a''_i\cap\bar\nei_0(\sigma'_r)$ which consists of exactly that point,
\item or it is contained in a component of $a''_i\cap\bar\nei(\sigma'_r)$ which is a tie of $\bar\nei(\sigma'_r)$;
\end{itemize}
\item if a given connected component of $a''_i\cap\sigma'_r$ is a union $b$ of branches of $\sigma'_r$, let $B$ be the corresponding connected component of $a''_i\cap\bar\nei(\sigma'_r)$: then $B\setminus b$ is entirely contained in $\bar\nei(\sigma'_r)\setminus \bar\nei_0(\sigma'_r)$;
\item however one chooses a finite-length smooth, embedded, path $\rho$ along $\sigma'_r\cap Q_i$ (not necessarily a train path), and a subarc $a$ of $a''_i$, which intersect exactly at their endpoints, the region they bound together has negative index.
\end{enumerate}

Note that, if one can choose two paths $\rho$ and $a$ as in condition 7 and they satisfy it, then they can only delimit a 1-punctured bigon. If they exist but do not satify the condition then, even admitting that $a$ may degenerate to a single point, the region they bound cannot be a zero-gon, monogon, or 1-punctured zero-gon/monogon.

\begin{figure}
\centering
\begin{minipage}[c]{.5\textwidth}
\def\svgwidth{\textwidth}
\input{foldingareas.pdf_tex}
\end{minipage}\hspace{5em}
\begin{minipage}[c]{.35\textwidth}
\def\svgwidth{\textwidth}
\input{foldingareas_adaptation.pdf_tex}
\end{minipage} 
\caption{\label{fig:diagext_efficientpos}\textit{Left:} The construction of the arcs $\ul\beta_i^j$ given $\beta_i^j$ in a complementary region $Q_i$ of $\sigma_r$ (dotted), and the related folding rectangles and triangles (marked with black stripes). The twist collar $A_\gamma$ is marked in grey. \textit{Right:} Adaptation of the arcs $a_i$ (dashed) with respect to a folding rectangle or triangle (here shown with its foliation into ties). In short: every time $a_i$ does \emph{not} enter and exit the folding triangle/rectangle from opposite edges, we move it off.}
\end{figure}
\begin{figure}
\centering
\def\svgwidth{.8\textwidth}
\input{diagext_furtheradjust.pdf_tex}
\caption{\label{fig:diagext_furtheradjust}Ways to remove one of the innermost bigons between $\bigcup_{i=1}^{s} \phi(a_i)$ and $\sigma'_r$ --- say one bounded by $(\rho,a)$. The picture a) depicts the case of a bigon whose corners are both tangential intersections of some $a_i$ (dashed) with a switch of $\sigma'_r$: the bigon can be crushed with $a$ into $\sigma'_r$. The pictures b1), b2), b3) depict the case of a bigon with one corner being a transverse intersection between $\phi(a_i)$ and $\sigma'_r$, and the other a tangential one. The idea is to move $a$ across $\rho$ in order to remove completely the transverse intersection point. In b1) a new tangential intersection point is introduced at a switch along the $\sigma'_r$-edge of the cancelled bigon; in b2) a tangential intersection point is introduced at a switch not along that edge; in b3) none is introduced at all. None of these moves increases the number of innermost bigons, but it may leave it unvaried: in b3) another arc $\phi(a_{i'})$ is marked in grey to show how it may happen. In any case the number $x$ decreases. The arc $\phi(a_{i'})$ depicted also shows how condition 2 is not violated when moving $\phi(a_i)$.}
\end{figure}

\begin{figure}
\centering
\def\svgwidth{.85\textwidth}
\input{wisediagext.pdf_tex}
\caption{\label{fig:diagextconstruction}An example of the way the construction of $\delta'_r$ works in one of the complementary regions of $\sigma_r$, in this case $Q_1=Q_2=Q_3$. The arcs $\beta_\cdot^\cdot$ are part of $\rho_r.\gamma$ (in this picture they share a branch traversed twice by $\gamma$), and a twist collar for the latter is marked in grey. Branches $a_1,a_2,a_3\in\br(\delta_r)\setminus\br(\sigma'_r)$ (picture to the left) are drawn with a dashed line. The arrows around the $\beta_\cdot^\cdot$ show the instructions to `bend' the segments $e_j$ to get the corresponding $e'_j$ in $\nei(\sigma'_r.\gamma)$. The picture to the right shows the new branches that are added to $\sigma'_r$ to get $\sigma''_r$. A cross marks a new branch that forms a bigon with another one, thus is to be excluded from the definition of $\delta'_r$.}
\end{figure}

In practice, the position described above can be attained with the process we are about to describe. Start with isotoping the arcs $a_i$ so that their interiors remain contained in the respective $\mathrm{int}(Q_i)$, pairwise disjoint, and each intersects each arc $\ul\beta_i^j$ transversally, without forming bigons. Then, via slight perturbations, for each folding triangle/rectangle $A$, make sure that no $a_i$ includes points of $\partial A\setminus (\partial_h A\cup\partial_v A)$ (i.e. the corners of the folding triangle/rectangle which are not cusps in $\partial Q_i$).

Let $\mathcal S$ be the set of all connected components of the intersections of the form $a_i\cap A$, for some $1\leq i\leq s$ and $A$ a folding rectangle or triangle. First, we make sure that, for every $a_i$ and every folding rectangle or triangle $A$, every connected component of $a_i\cap A$ has its endpoints either on the two opposite components of $\partial_h A$, or the two opposite components of $\partial_v A$ (remember: one of the latter may be a single point). This is done as specified in Figure \ref{fig:diagext_efficientpos}: the process comes to an end, because moving one or more portions of the arcs $a_i$ off a folding rectangle/triangle results into making the cardinality of $\mathcal S$ strictly lower. 

When performing each of these moves, we keep the interiors of the $a_i$ pairwise disjoint, thus they comply with a stricter version of condition 2 in the list above. When nothing is left to move off, they can also be supposed to intersect $\sigma'_r$ only in isolated points, thus complying with a stricter version of condition 1. We can also suppose that conditions 3, 4 are satisfied by all $a_i$ provided that the tie neighbourhood is chosen wisely. 

Let now $\phi:S^X\rightarrow S^X$ be a smooth map complying with the following requests. On the points belonging $\sigma'_r$, to any of the $\ul\beta_i^j$, or to any folding area --- denote $Y$ the set of all these points --- $\phi$ coincides with the tie collapse $c:\bar\nei_0(\sigma'_r)\rightarrow \sigma'_r$. $\phi$ is then defined to be homotopic to $\mathrm{id}_{S^X}$, with $\phi|_{S^X\setminus Y}$ injective, $\phi|_{\bar\nei(\sigma'_r)}$ keeping each point along the same tie, and $\phi(S^X\setminus Y)\cap \sigma'_r=\emptyset$.

The family $\{\phi(a_i)\}$ is seen to comply with all conditions 1--6 that the family $\{a''_i\}$ is required to satisfy (possibly adapting the tie neighbourhood $\bar\nei(\sigma'_r)$), but not necessarily condition 7. However, it is impossible that a segment of one of the $\phi(a_i)$ bounds, together with a smooth embedded path along $\sigma'_r$, a bigon whose corners are each a transverse intersection point between some $\phi(a_i)$ and some branch of $\sigma'_r$. If there were one such bigon, then necessarily there would be a bigon bounded by the corresponding $a_i$ and one of the $\ul\beta_i^j$.

Now, let $x$ be the number of pairs $(\rho,a)$ which transgress condition $7$: $\rho$ is a smooth path along $\sigma'_r$, $a$ is a subarc of one of the $\phi(a_i)$, $\rho$ and $a$ intersect exactly at their endpoints, and they bound a bigon. In Figure \ref{fig:diagext_furtheradjust} it is shown how to set up a procedure that recursively considers an innermost bigon bounded by some $(\rho,a)$, and moves $\phi(a_i)$ off the bigon to decrease $x$ strictly. At each stage of the recursion, one shall probably perform isotopies to make sure that the position with respect to $\bar\nei(\sigma'_r)$ keeps respecting conditions 3--6. When it is impossible to proceed with the recursion any further, condition 7 will also be satisfied; while conditions 1 and 2 will be still satisfied, too. Define the family $\{a''_i\}$ to be the set of arcs eventually attained.

Finally, define $\delta''_r\coloneqq \sigma_r\cup\left(\bigcup_{i=1}^s a''_i\right)$. It is clear that $\delta''_r$ carries $\ul\alpha_-$.

\step{2} Construction of $\delta'_r$.

Before we start with this step, we announce that the construction of $\delta'_r$ will consist of removing any transverse intersection between the family $\{a''_i\}$ and $\sigma'_r$: the arcs $a''_i$ will be `broken' at each transverse intersection point, and then bent as $D_\gamma^\epsilon$ would do, in order to get attached to a switch of $\sigma'_r$. In Step 3 we will see why $\delta'_r$ thus constructed carries $\alpha$.

The curve $\gamma$ is a twist curve in $\sigma'_r$: the first condition in Definition \ref{def:twistcurve} is clearly verified; the second one is, too, because $\sigma'_r$ is recurrent. In particular $\gamma$ inherits the twist collar $A_\gamma$ from $\rho_r$. As it was already done previously, we suppose that the component of $\Lambda$ of $\partial\bar A_\gamma$ which is not part of $\sigma'_r.\gamma$ coincides with a component of $\partial\bar\nei(\sigma'_r.\gamma)$.

Enumerate (in any way) $e_1,\ldots,e_t$ the `half-ties' contained in $\left(\bigcup_{i=1}^s a''_i\right)\cap\bar\nei(\sigma'_r.\gamma)$, i.e., for each connected component $e$ of this intersection which is a tie, the two opposite segments of it, from $e\cap\sigma'_r.\gamma$ to $\partial\bar\nei(\sigma'_r)$, are two of the elements of the list. Let then $P_j$ be the endpoint of the corresponding $e_j$ that lies along $\partial\bar\nei(\sigma'_r)$; and let $i(j)$ be the only value of $1\leq i\leq s$ such that $e_j\subset a''_i$. We wish to define, for each $j$, a new arc $e'_j$ contained in $\bar\nei_0(\sigma'_r)$ whose endpoints are $P_j$ and a suitable switch of $\sigma'_r$. 

To do so, we perform the following construction for each $e_j$, one after another according to their order --- see also Figure \ref{fig:diagextconstruction}. Supposing that $e'_1,\ldots,e'_{j-1}$ have been constructed, consider the closures of the connected components of $(\bar\nei_0(\sigma'_r)\cap Q_{i(j)})\setminus\sigma'_r$: each of them is bounded by a smooth component of $\partial\bar\nei_0(\sigma'_r)$ and a bounded train path along $\sigma'_r.\gamma$, and therefore is a bigon, with its edges transverse to the ties of $\sigma'_r$, and corners (cusps) at switches of $\sigma'_r$. Let then $E_j\subseteq Q_{i(j)}$ be the region among these which contains $e_j$.

Define $e'_j$ to be an arc in $E_j$ with the following properties:
\begin{itemize}
\item $e'_j\cap\partial E_j$ consists exactly of the endpoints of $e'_j$, which shall be $P_j$ and one of the cusps of $\partial E_j$;
\item $e'_j$ is transverse to all ties it meets;
\item $e'_j$ runs in parallel with (part) of one of the arcs $\ul\beta_{i(j)}^{\ldots}$; moreover it proceeds in the verse consistent with the $A_\gamma$-orientation if $e_j\subseteq \ol{A_\gamma}$, and in the opposite verse otherwise;
\item $e'_j$ is part of a smooth path that begins at a point of $\left(\bigcup_{i=1}^s a''_i\right)\setminus \bar\nei(\sigma'_r.\gamma)$, enters $e'_j$ from $P_j$ and, after leaving it, continues entering a branch in $\sigma'_r$ and following it;
\item $e'_j$ is disjoint from all $e'_{j'}$ for $j'<j$.
\end{itemize}
Note that these conditions determine uniquely what is the endpoint of $e'_j$ other than $P_j$.

We claim that $e'_j\subseteq\bar\nei(\sigma'_r.\gamma)$. This is clear if $e_j\subseteq\ol{A_\gamma}$: as $e'_j$ proceeds in the $A_\gamma$-orientation, eventually it must travel between $\sigma'_r.\gamma$ and a branch end hitting $A_\gamma$ (there is one necessarily, because $\sigma'_r.\gamma$ intersects more than one of the regions $Q_i$), and is forced to reach the switch between the two. Since this switch belongs to a branch end hitting $A_\gamma$, it must necessarily be located along $\partial Q_{i(j)}$.

In case $e_j\cap A_\gamma=\emptyset$, the analysis is slightly more involved. Suppose that $e'_j$, starting at $P_j$, and proceeding along an arc $\ul\beta$ among the $\ul\beta_{i(j)}^{\ldots}$, oppositely to the $A_\gamma$-orientation, leaves $\nei(\sigma'_r.\gamma)$. Then this event must occur just after $e'_j$ has traversed a large branch end in $\sigma'_r.\gamma$. The first branch end $\eta_1$ traversed which is not in $\sigma'_r.\gamma$ is then, necessarily, one that avoids $A_\gamma$; and it is adverse, due to the orientation of $e'_j$. Also, $\eta_1$ is included in some smooth edge $\lambda$ of $\partial Q_i$. Necessarily, $\ul\beta$ does not travel along the entire length of $\lambda$ but only a portion.

This means that $\gamma$ traverses a large branch $b$ of $\sigma'_r$, contained in the \emph{interior} of $\lambda$, and then gets out of $\lambda$ on the side opposite to $Q_{i(j)}$. Let $\eta_2$ be the branch end of $\sigma'_r$, contained in $\lambda$, which shares with $b$ the switch other than the one $b$ shares with $\eta_1$. Then $\eta_1,\eta_2$ are branch ends giving $\gamma$ opposite orientations; so, as $\eta_2$ is a branch end hitting $A_\gamma$, $\eta_1$ is necessarily a favourable one, a contradiction.

When this process is done, define $\sigma''_r\coloneqq \sigma'_r\cup \left(\bigcup_{i=1}^s \left(a''_i\setminus \nei_0(\sigma'_r)\right)\right)\cup\left(\bigcup_{j=1}^t e'_j\right)$. In general $\sigma''_r$ is not a train track as there may be bigons among the complementary regions of $S\setminus \sigma''_r$. Monogons, nullgons and 1-punctured nullgons are to be excluded instead, as they would require either $a''_i$ to bound a bigon with $\sigma'_r$, or $\sigma''_r$ to have more switches than $\sigma'_r$.

Define $\delta'_r$ to be a maximal subtrack of $\sigma''_r$ which includes $\sigma'_r$ and is a train track. Practically speaking, define $\delta'_r$ by deleting any branch in $\br(\sigma''_r)\setminus\br(\sigma'_r)$ which bounds a bigon together with a train path along $\sigma'_r$; and furthermore, every time there are two branches in $\br(\sigma''_r)\setminus\br(\sigma'_r)$ surviving after this operation, and bounding a bigon together, delete one of them. As each of these operations affects only one of the regions $\{Q_i\}$, and any innermost bigon has to be entirely contained in one of these regions, the suggested procedure is able to remove the extra branches without leaving any bigon in $\delta'_r$. 

The curve $\gamma$ is carried by $\delta'_r$ and is again a twist curve there.

We consider a tie neighbourhood for $\delta''_r$ built with the following constraints: $\bar\nei(\delta''_r)$ and $\bar\nei(\sigma'_r)$ induce one same tie neighbourhood $\nei(\sigma_r)$ for their common subtrack $\sigma_r$; each transverse intersection point between $\delta''_r$ and $\sigma'_r$ is contained in a connected component of $\bar\nei(\delta''_r)\cap\bar\nei(\sigma'_r)$ which is diffeomorphic to a square, and the tie foliations given by the two tie neighbourhoods are each parallel to one of the two pairs of opposite edges of the square.

From this, we build a tie neighbourhood for $\sigma''_r$ too, with the property that $\bar\nei(\sigma''_r)\subseteq \bar\nei(\sigma'_r.\gamma)\cup \bar\nei(\delta''_r)$; $\bar\nei(\sigma''_r)\setminus \bar\nei(\sigma'_r.\gamma)= \bar\nei(\delta''_r)\setminus \bar\nei(\sigma'_r.\gamma)$; and all ties in each rectangle $R_b$, for $b$ a branch of $\sigma''_r$ which is part of $\sigma''_r.\gamma$, are sub-ties of $R_{b'}$, for $b'\in\br(\sigma'_r)$ a branch contained in $\sigma'_r.\gamma$ ($b,b'$ are actually the same branch, regarded as part of the two different almost tracks). This means that the branch end rectangles $R_{e'_j}$ have their image contained in $\bar\nei(\sigma'_r.\gamma)$, albeit with a new specification for their ties. This definition of tie neighbourhood induces a tie neighbourhood for $\delta'_r$, and again one for $\sigma_r$, which \emph{does not} coincide with the $\nei(\sigma_r)$ previously defined. An example of these constructions is given in Figure \ref{fig:diagonalnbhs}.

\begin{figure}
\centering{\includegraphics[width=.9\textwidth]{diagonalnbhs.pdf}}
\caption{\label{fig:diagonalnbhs}A choice of tie neighbourhoods related to a portion of the previous Figure \ref{fig:diagextconstruction}. The picture on the left shows how $\nei(\sigma'_r)$ (chequerboard-coloured) and $\nei(\delta''_r)$ (grey) intersect and define on their common subtrack $\sigma_r$ the same tie neighbourhood $\nei(\sigma_r)$ (black). The picture on the right shows a choice of $\nei(\sigma''_r)$, contained in $\sigma'_r$ (painted in grey here) and coinciding with it outside $\bar\nei(\sigma'_r.\gamma)$.}
\end{figure}

We will now use, several times, the notation $\nei(\sigma'_r.\gamma)$ to mean specifically the tie neighbourhood induced by $\nei(\sigma'_r)$: it is necessary to clarify this, because $\sigma''_r.\gamma$ (or $\delta'_r.\gamma$), albeit coinciding with $\sigma'_r.\gamma$ setwise, inherits a different tie neighbourhood from $\sigma''_r$ (or $\delta'_r$). Similarly we will use $\nei(\sigma''_r.\sigma'_r)$, $\nei(\sigma''_r.\sigma_r)$ to mean the tie neighbourhoods inherited from $\sigma''_r$, as opposed to $\nei(\sigma'_r)$ the `native' tie neighbourhood and $\nei(\sigma_r)$ the tie neighbourhood that $\sigma_r$ inherits from $\sigma'_r$ or, equivalently, from $\delta''_r$. Note that not only $\bar\nei(\sigma''_r.\sigma'_r)\subseteq \bar\nei(\sigma'_r)$ but there is a diffeomorphism $f:\bar\nei(\sigma'_r)\rightarrow \bar\nei(\sigma''_r.\sigma'_r)$ sending each tie of $\bar\nei(\sigma'_r)$ to a subset of it.

For each $\beta_i^j$, let $\mathcal X_i^j$ be the connected component of $\bar\nei(\sigma'_r)\setminus \nei(\sigma_r)$ which intersects it. 

\step{3} $D_\gamma^\epsilon(\alpha_-)$ is carried by $\delta'_r$, and conclusion.

Let $\ul\alpha_-$ be a carried realization of $\alpha_-$ in $\bar\nei(\delta''_r)$ and $\ul\gamma$ be a carried realization of $\gamma$ in $\bar\nei(\sigma_r)$ (therefore it is one in $\bar\nei(\sigma'_r)$ and in $\bar\nei(\delta''_r)$, too), with the properties that: they intersect transversely, minimally among the pairs of curves in the respective homotopy classes; each of the connected components of $\ul\alpha_-\cap\bar\nei(\sigma'_r)$ is either of the following:
\begin{itemize}
\item transverse to all ties, with its endpoints on $\partial_v\bar\nei(\sigma'_r)$;
\item a single tie of $\bar\nei(\sigma'_r)$.
\end{itemize}
This property may be required because of the previously specified intersection pattern between $\bar\nei(\sigma'_r)$ and $\bar\nei(\delta''_r)$. Similarly, we require that each of the connected components of $\ul\gamma\cap\bar\nei(\delta''_r)$ is either:
\begin{itemize}
\item transverse to all ties of $\bar\nei(\delta''_r)$ (but this time its endpoints may lie on $\partial_h\bar\nei(\delta''_r)$, too);
\item a single tie of $\bar\nei(\delta''_r)$.
\end{itemize}

We are about to construct a carried realization of $\alpha_+\coloneqq D_\gamma^\epsilon(\alpha_-)$. What we will do, informally, is realize the Dehn twist on $\ul\alpha_-$ by bending each transverse intersection of $\alpha_-$ in $\bar\nei(\delta''_r)$ to follow the branches of $\sigma''_r$, which have been introduced as a bending of the branches of $\delta''_r$, specifically for this idea to work; and by making the carried portions of $\ul\alpha_-$ in $\sigma'_r$ wind once more about $\sigma'_r.\gamma$. Once a carried realization of $D_\gamma^\epsilon(\alpha_-)$ in $\sigma''_r$ is proved to exist, it will be clear that there exists one in $\delta'_r$, too.

Let $\mathcal T$ be a (narrow) regular neighbourhood of $\ul\gamma$ in $\bar\nei(\sigma'_r.\gamma)$, and let $D_{\mathcal T}$ be a concrete realization of the Dehn twist about $\ul\gamma$, chosen to be the identity map outside $\mathcal T$.

Let $\Xi\coloneqq \ul\alpha_-\cap\bar\nei(\sigma'_r.\gamma)$. It is legitimate to suppose that there is a bijection between the connected components of $\Xi\cap \mathcal T$ and the points of $\Xi\cap\ul\gamma$.

Recall that $\Lambda$ is the connected component of $\partial\bar A_\gamma$ which is also a connected component of $\partial\bar\nei(\sigma'_r.\gamma)$. Given a connected component $\xi$ of $\Xi$, we have that $\xi\cap \ul\gamma\not=\emptyset$ if and only if $\xi$ has one endpoint along $\Lambda$, and the other endpoint along some other component of $\partial\bar\nei(\sigma'_r.\gamma)$. If this is not the case, any intersection point between $\ul\gamma$ and $\ul\alpha_-$, located along $\xi$, could be deleted by moving $\ul\gamma$ via isotopies, and this would reduce their total amount, a contradiction.

So, if a connected component $\xi$ intersects $\ul\gamma$ and is not a tie, orient it from its endpoint along $\Lambda$ towards the other one: then we have that $\xi$ traverses the ties according to the $A_\gamma$-orientation. This is the case because, at the point where $\xi$ begins, either $\ul\alpha_-$ is entering $\bar\nei(\sigma'_r.\gamma)$ by traversing a branch end of $\sigma'_r$ which hits $A_\gamma$; or $\ul\alpha_-$ has just entered $\bar\nei(\sigma'_r)$ via a component of $\partial_v\bar\nei(\sigma'_r)$ which has an endpoint along $\Lambda$.

Define $\Xi'=D_{\mathcal T}^\epsilon(\Xi)$: then the curve $\ul\alpha_+\coloneqq (\ul\alpha_-\setminus\Xi)\cup\Xi'$ is a loop in $S$ in the isotopy class of $\alpha_+$. Moreover $\ul\alpha_+\setminus \nei(\sigma'_r.\gamma)$ is transverse to all ties of $\bar\nei(\delta''_r)$, and therefore of $\bar\nei(\sigma''_r)$, it encounters. We wish to isotope $\Xi'$ so that it is entirely contained in $\bar\nei(\sigma''_r)\cap\bar\nei(\sigma'_r.\gamma)$, and is transverse to the ties of $\bar\nei(\sigma''_r)$. After the end of this modification, the resulting $\ul\alpha_+$ is a carried realization of $\alpha_+$ in $\bar\nei(\sigma''_r)$.

Regardless of what are the properties of the single connected components $\Xi'$, after a small perturbation leaving fixed the endpoints of each connected component and not altering the intersection pattern, is transverse to all ties of $\bar\nei(\sigma'_r.\gamma)$. Moreover, given any component of $\Xi'$ with an endpoint along $\Lambda$, if one orients it from this endpoint towards the other one, then one actually gets the $A_\gamma$-orientation on it.

For each $\mathcal X_i^j$, each connected component $\zeta$ of $\Xi'\cap \ol{\mathcal X}_i^j$ falls into one of the two following cases:
\begin{itemize}
\item $\zeta$ has one endpoint along $\partial_h\bar\nei(\sigma'_r)$. In this case $\zeta$ is part of $D_{\mathcal T}^\epsilon(\xi)$, for $\xi$ a connected component of $\Xi$ which is a tie of $\bar\nei(\sigma'_r.\gamma)$. The other endpoint of $\zeta$ lies necessarily along $\partial\bar\nei(\sigma_r)$. Furthermore, there is one of the branch ends $e'_u$ defined in Step 2 such that the first endpoint of $\zeta$ lies actually along the tie $R_{e'_u}(\{0\}\times[-1,1])$ of $\bar\nei(\sigma''_r)$, and such that $R_{e'_u}$ intersects the same connected component of $\ol{\mathcal X}_i^j\cap \partial\bar\nei(\sigma_r)$ on which the second endpoint of $\zeta$ lies. Informally, $\zeta$ traverses, in the same order, the ties of $\bar\nei(\sigma'.\gamma)$ that $e'_u$ traverses.
\item both endpoints of $\zeta$ are away from $\partial_h\bar\nei(\sigma'_r)$. This includes the case of $\zeta$ being part of $D_{\mathcal T}^\epsilon(\xi)$, for $\xi$ a connected component of $\Xi$ which does not intersect $\ul\gamma$ --- so that, actually, $D_{\mathcal T}^\epsilon(\xi)=\xi$.
\end{itemize}

Recall the diffeomorphism $f:\bar\nei(\sigma'_r)\rightarrow \bar\nei(\sigma''_r.\sigma'_r)$ defined above, and define $\Xi'_f$ as follows. Given any connected component $\xi'$ of $\Xi'$, attach to each extreme point of $f(\xi')$ not lying along $\partial_h\bar\nei(\sigma'_r.\gamma)$ a segment of tie in $\bar\nei(\sigma'_r.\gamma)$, so as to obtain a path in $\bar\nei(\sigma'_r.\gamma)$ with both endpoints lying along $\partial\bar\nei(\sigma'_r.\gamma)$. Let then $\Xi'_f$ be the union of all paths thus obtained: $(\ul\alpha_-\setminus\Xi)\cup\Xi'_f$ is again a representative of $\alpha_+$. Now the components $\zeta$ of $\Xi'_f\cap \ol{\mathcal X}_i^j$ behave again as in one of the bullets above, but the ones in the second bullet are entirely contained in $\bar\nei(\sigma''_r)$, and the ones in the first bullet may be isotoped, leaving their endpoints fixed, so that they end up entirely contained in the relevant $R_{e'_u}$, transversely to its ties.

This gives the announced, desired realization of $\Xi'$; and we have proved that $\alpha_+$ is carried by $\sigma''_r$ and, as it was anticipated above, this is enough to say that it is carried by $\delta'_r$, too: if $\ul\alpha_+$ traverses any branch of $\sigma''_r$ which is not found in $\delta'_r$, there is another branch or union of branches with the same endpoints, which is is subset of $\delta'_r$ instead: so the carried realization of $\alpha_+$ may be adjusted to traverse this other branch instead.

Note that $\delta'_r$ is recurrent, as each branch is traversed by either $\alpha_+$ or by an element of $\cc(\sigma'_r)$. Let $\eta$ be a generic almost track which is comb equivalent to $\delta'_r$: then $\alpha_+\in\cc(\eta)$ and there is a generic splitting sequence of twist nature which turns $\eta$ into $D_\gamma^{\epsilon(m-1)}(\eta)$; its rotation number is $m-1$ (see Lemma \ref{lem:functiongivestwist}, and point \ref{itm:rotofdehn} of Remark \ref{rmk:rotbasics}). Note that, by definition, $\alpha=D_\gamma^{\epsilon(m-1)}(\alpha_+)\in\cc\left(D_\gamma^{\epsilon(m-1)}(\eta)\right)$: so, by the first statement of the present lemma, $\alpha$ traverses each branch in $\eta.\gamma$ at least $m-1$ times. This property is not affected by comb equivalences: so the same is true for the carrying image of $\alpha$ in $\delta'_r$.
\end{proof}

Given an almost track $\rho$ (in $S^X$, say), we give a generalized version of the definitions specified in \S \ref{sub:diagext}. If $\alpha_1,\ldots,\alpha_s\in\cc(\rho)$ and $k\geq 1$, denote:
$$
\cc_k(\rho;\alpha_1,\ldots,\alpha_s)\coloneqq\left\{\alpha\in \cc(\rho)\,\left|\,\parbox{.5\textwidth}{$\alpha$ traverses at least $k$ times each branch contained in one of the $\rho.\alpha_j$}\right.\right\}.
$$
Denote subsequently $\ce_k(\rho;\alpha_1,\ldots,\alpha_s) \coloneqq \bigcup_{\delta\in \e(\rho)}  \cc_k(\delta;\alpha_1,\ldots,\alpha_s)$; and\linebreak $\cf_k(\rho;\alpha_1,\ldots,\alpha_s)\coloneqq \bigcup_\omega \ce_k(\omega;\alpha_1,\ldots,\alpha_s)$, where the union is performed over all $\omega$ subtracks of $\rho$ which fill $S^X$, and include $\rho.\alpha_j$ for all $j$.

\begin{lemma}\label{lem:tcboundsimplest}
In the setting of Proposition \ref{prp:tcbound}, suppose a subsequence $(\gamma_{t_j})_{j=1}^r$ of the sequence of curves $(\gamma_j)$ fills a subsurface $X\subset S$ which is homeomorphic to $S_{0,4}$ or $S_{1,1}$.

Fix $a_-\leq\min DI_{t_1}$, and $a_+\geq\max DI_{t_r}$ such that $V(\tau_{a_+}|X)\not=\emptyset$. Then, for any $\alpha_-\in V(\tau_{a_-}|X), \alpha_+\in V(\tau_{a_+}|X)$,
$$
d_X\left(\alpha_-,\alpha_+\right)\geq \lfloor (r-1)/3\rfloor.
$$
\end{lemma}
\begin{proof}
This proof is based on the ideas used to prove Theorem 1.3 in \cite{masurminskyq} (stated as Theorem \ref{thm:mm_cc_geodicity} in the present work). We will use the notation $\rho_j=\tau_j|X$ and, in order to avoid introducing further notation, we denote our $\gamma_{t_j}$'s simply as $\gamma_j$'s, forgetting about the other effective twist curves. We do the same with the notation for the intervals $DI_j$ and similar ones. For each $j$, let $m_j\coloneqq \rot(\gamma_j;DI_j)$. Also, for $1\leq j\leq r-1$ set $a_j\coloneqq\min DI_j$ and, asymmetrically, set also $a_r\coloneqq\max DI_{r-1}$.

Note that each pair of distinct curves $\gamma_j,\gamma_{j'}$ intersect and fill $X$ --- and $S^X$ --- because $X$ is a 4-holed sphere or a 1-holed torus. So, for all $1\leq j\leq r$, $\rho_{a_j}$ fills $S^X$ as it carries two of these curves.

Suppose $3\leq j \leq r$. Then $a_{j-2}<\max DI_{j-2}\leq a_{j-1}<\max DI_{j-1}\leq a_j$. Lemma \ref{lem:weightsaftertwist}, applied to $\bm\tau(a_{j-2},a_j)$ with respect to the twist curve $\gamma_{j-1}$, yields that $\gamma_j\in \cc(\rho_{\alpha_j})$ traverses at least $2\mathsf{K}_0+4$ times each branch of $\rho_{a_{j-2}}$ contained in $\rho_{a_{j-2}}.\gamma_{j-1}$. The same lemma, applied with respect to the twist curve $\gamma_{j-2}$, yields also that $\gamma_j$ traverses at least $2\mathsf{K}_0+4$ times each branch of $\rho_{a_{j-2}}$ contained in $\rho_{a_{j-2}}.\gamma_{j-2}$.

Let now $4 \leq j \leq r$, and fix $\alpha\in \cf(\rho_{a_j})$: we want to show that $\alpha\in \cf_3(\rho_{a_{j-3}})$. If $\alpha=\gamma_{j-1}$ a completely similar argument to the one just applied, considered for the sequence $\bm\tau(a_{j-3}, a_j)$ with respect to the two twist curves $\gamma_{j-3},\gamma_{j-2}$, yields $\gamma_{j-1}\in \cc_3(\rho_{a_{j-3}}; \gamma_{j-2},\gamma_{j-3})\subset\cf_3(\rho_{a_{j-3}})$. (Actually we might as well conclude that $\gamma_{j-1}\in \cc_{2\mathsf{K}_0+4}(\rho_{a_{j-3}}; \gamma_{j-2},\gamma_{j-3})$, but we do not need it; also in the following inclusions, we will only care about branches being traversed thrice.) The last inclusion is due to the fact that, as $\gamma_{j-2},\gamma_{j-3}$ fill $S^X$, also $\rho_{a_{j-3}}.\gamma_{j-2}\cup \rho_{a_{j-3}}.\gamma_{j-3}$ is an almost track filling $S^X$, and $\cc_3(\rho_{a_{j-3}}; \gamma_{j-2},\gamma_{j-3})\subseteq \ce_3(\rho_{a_{j-3}}.\gamma_{j-2}\cup \rho_{a_{j-3}}.\gamma_{j-3})$. 

If $\alpha$ is any other curve, it will intersect $\gamma_{j-1}$. The following chain of inclusions holds:
\begin{itemize}
\item $\alpha\in \cf_3(\rho_{a_{j-1}};\gamma_{j-1})$, by Lemma \ref{lem:weightsaftertwist} applied to the sequence $\bm\tau(a_{j-1},a_j)$ with respect to the twist curve $\gamma_{j-1}$.
\item $\cf_3(\rho_{a_{j-1}};\gamma_{j-1})\subset \cf_3(\rho_{a_{j-3}};\gamma_{j-1})$, because of Lemma \ref{lem:cf_decreasing}: first of all, clearly $\rho_{a_{j-1}}$ is carried by $\rho_{a_{j-3}}$ and, as noted above, both almost tracks fill $S^X$. If a curve $\beta$ is carried by $\delta'$, diagonal extension of $\sigma'$ which is a subtrack of $\rho_{a_{j-1}}$ which fills $S^X$ and contains $\rho_{a_{j-1}}.\gamma_{j-1}$, and $\beta$ traverses all branches in $\sigma'.\gamma_{j-1}$ at least thrice, then consider the almost tracks $\delta$ and $\sigma$ given by the last statement of said Lemma, with $\sigma$ a subtrack of $\rho_{a_{j-3}}$: necessarily $\sigma\supset \rho_{a_{j-3}}.\gamma_{j-1}$ and, by composition of carrying maps, $\beta$ in $\delta$ traverses each branch in $\rho_{a_{j-3}}.\gamma_{j-1}$ at least thrice.
\item $\cf_3(\rho_{a_{j-3}};\gamma_{j-1}) \subset\cf_3(\rho_{a_{j-3}};\gamma_{j-1},\gamma_{j-2},\gamma_{j-3})\subset \cf_3(\rho_{a_{j-3}})$: the first of these two inclusions is just due to the argument above (with translated indices) that $\rho_{a_{j-3}}.\gamma_{j-1}$ will traverse all branches carrying $\gamma_{j-2},\gamma_{j-3}$. The second one is due to the fact that $\cf_3(\rho_{a_{j-3}};\gamma_{j-1},\gamma_{j-2},\gamma_{j-3})=\ce_3(\rho_{a_{j-3}}.\gamma_{j-1}\cup \rho_{a_{j-3}}.\gamma_{j-2}\cup \rho_{a_{j-3}}.\gamma_{j-3})$.
\end{itemize}

Lemma \ref{lem:ccnesting}, together with this chain of inclusions, yields that $\nei_1(\cf(\rho_{a_j}))\subseteq \nei_1(\cf_3(\rho_{a_{j-3}}))\subseteq \cf(\rho_{a_{j-3}})$. And, nesting these last found inclusions for different values of $j$, for any pair of indices $1\leq i<i'\leq r$ such that $3|(i'-i)$ we get
$$
\mathcal N_{(i'-i)/3}\left(\cf(\rho_{a_{i'}})\right)\subseteq \cf(\rho_{a_i}).
$$

Denote $\hat r\coloneqq \lfloor(r-1)/3\rfloor$. If, for any $\alpha_+\in V(\rho_{a+})\subseteq \cf(\rho_{a_{3\hat r+1}}),\alpha_-\in V(\rho_{a-})$, we have $d_X(\alpha_-,\alpha_+)< \hat r$, then
$$\alpha_- \in \nei_{\hat r - 1}(\cf(\rho_{a_{3\hat r+1}}))\subseteq \cf(\rho_{a_4}) \subseteq \cf_3(\rho_{a_1}),$$
as a consequence of the inclusions proved above. Also, $\cf_3(\rho_{a_1})\subseteq \cf_3(\rho_{a_-})$, by an argument entirely similar to the inclusion shown in the second bullet above.

But $\alpha_- \in \cf_3(\rho_{a_-})$ cannot be true because of Lemma \ref{lem:vertexnotinterior}. So $d_X(\alpha_-,\alpha_+)\geq \hat r$, and this proves the claim.
\end{proof}

\begin{lemma}\label{lem:subsurfacesdontrepeat}
Let $\bm\tau=(\tau_j)_{j=0}^N$ be a splitting sequence of generic, recurrent almost tracks on a surface $S$, such that $\bm\tau(k,l)$ has twist nature about a curve $\gamma$, with $\rot_{\bm\tau}(\gamma;k,l)\geq 2\mathsf{K}_0+4$. Let $Y\subset S$ be a subsurface with $\gamma$ essentially intersecting $\partial Y$.

Let $0\leq a_-\leq k<l\leq a_+\leq N$, and let $\alpha_-\in W(\tau_{a-}),\alpha_+\in W(\tau_{a+})$ with the properties that
\begin{itemize}
\item $\alpha_-\subset Y$ (when choosing appropriate representatives in their isotopy classes);
\item $\alpha_+\cap Y\not=\emptyset$ (however the representative of $\alpha_+$ is chosen);
\item both $\alpha_-,\alpha_+$ intersect $\gamma$ essentially.
\end{itemize}

Then $\alpha_+$ intersects $\partial Y$ essentially.
\end{lemma}
\begin{proof}
Let $X$ be a regular neighbourhood of $\gamma$, let $m\coloneqq \rot_{\bm\tau}(\gamma;k,l)$ and let $c_-=\max\{\min I_\gamma,a_-\}$, $c_+=\min\{\max I_\gamma,a_+\}$. Then $c_-\leq k < l \leq c_+$.

As $\alpha_-\in W(\tau_{a-})$, then also $\pi_X(\alpha_-)\subseteq V(\tau_{a_-}^X)$: if any arc $\pi_X(\alpha_-)$ traverses a branch of $\tau_{a_-}^X$ twice in the same direction (this is the only way it may not be wide: see point \ref{itm:windaboutgamma} in Remark \ref{rmk:annulusinducedbasics}), then also $\alpha_-$ traverses a branch of $\tau_{a_-}$ twice in the same direction. Similarly, $\pi_X(\alpha_+)\subseteq V(\tau_{a_+}^X)$.%\subseteq \cc(\tau_l^X)$. Lemma \ref{lem:onerollingdirection} and point \ref{itm:rot_vs_dt_vertices} of Remark \ref{rmk:rotbasics} yield that $\cc(\tau_l^X)\subseteq D_{\gamma}^{\epsilon m}\cdot \cc(\tau_k^X)\subseteq D_{\gamma}^{\epsilon m}\cdot \cc(\tau_{a_-}^X)$.

From the statement 1 of Theorem \ref{thm:mmsstructure} we have $d_{\cc(X)}\left(\tau_{a_-}|X, \tau_{c_-}|X\right)\leq \mathsf{K}_0$ and $d_{\cc(X)}\left(V(\tau_{c_+}|X), \pi_X(\alpha_+)\right)\leq \mathsf{K}_0.$

Note that $\pi_X(\partial Y)\not=\emptyset$, and as $\partial Y,\alpha_-$ do not intersect, $d_X(\partial Y,\alpha_-)=1$. The triangle inequality holds for $d_{\cc(X)}$ even when its arguments are \emph{sets}, so\linebreak $d_{\cc(X)}\left(\pi_X(\partial Y), V(\tau_{c_-}|X)\right)\leq \mathsf{K}_0+1$. Subsequently,
\begin{eqnarray*}
 & d_X\left(\partial Y,\alpha_+\right) \geq d_{\cc(X)}\left(V(\tau_{c_-}|X),V(\tau_{c_+}|X)\right)- d_{\cc(X)}\left(\pi_X(\partial Y), V(\tau_{c_-}|X)\right) + & \\
 & - d_{\cc(X)}\left(V(\tau_{c_+}|X), \pi_X(\alpha_+)\right) \geq d_{\cc(X)}\left(V(\tau_{c_-}|X),V(\tau_{c_+}|X)\right)-2\mathsf{K}_0-1 & 
\end{eqnarray*}
again by the triangle inequality; and $d_{\cc(X)}\left(V(\tau_{c_-}|X),V(\tau_{c_+}|X)\right)\geq \rot_{\bm\tau}(\gamma;c_-,c_+)\geq m$ as seen in point \ref{itm:rot_vs_dist} of Remark \ref{rmk:rotbasics}.

Since $m\geq 2\mathsf{K}_0+4$, we have $d_X\left(\partial Y,\alpha_+\right)\geq 3$, meaning in particular that $\pi_X(\alpha_+)$ intersects $\pi_X(\partial Y)$. Thus also $\alpha_+$ intersects $\partial Y$ as required.
\end{proof}

\begin{coroll}\label{cor:subsurfacesdontrepeat}
In the setting of Proposition \ref{prp:tcbound}, let $1\leq t_1<t_2<t_3\leq q$ be three indices, and let $Y\subseteq S$ be a subsurface, with the following properties: $\gamma_{t_2}$ essentially intersects both $\gamma_{t_1},\gamma_{t_3}$; $\gamma_{t_1}\subset Y$, $\gamma_{t_2}$ essentially intersects $\partial Y$; and $\gamma_{t_3}$ cannot be realized disjointly from $Y$ as a curve on $S$.

Then $\gamma_{t_3}$ intersects $\partial Y$ essentially, too.
\end{coroll}
This statement may be read as follows: once a chain subsequence breaks out of a given subsurface $Y$, no subsequent entry will enter it again.
\begin{proof}
Apply the previous lemma with $\gamma=\gamma_{t_2}$,  $[k,l]=DI_{t_2}$, $a_-=\max DI_{t_1}$, $a_+=\min DI_{t_3}$, $\alpha_-=\gamma_{t_1}$, $\alpha_+=\gamma_{t_3}$. We have $\rot_{\bm\tau}(\gamma_{t_2};DI_{t_2})\geq 2\mathsf{K}_0+4$ by definition of Dehn interval.
\end{proof}

Proposition \ref{prp:tcbound} will be proved employing an auxiliary statement, by induction on the complexity $\xi(X')$ of subsurfaces $X'\subseteq X\subseteq S$. Its proof will be based on the ideas used to prove Theorem 1.3 in \cite{masurminskyq}.

Fix any chain subsequence $\bm\delta=(\delta_1,\ldots,\delta_r)$ of $(\gamma_1,\ldots,\gamma_q)$; let $X'\subseteq S$ be the subsurface which is filled by this chain subsequence. We say that $X'$ is \nw{$0$-good} with respect to the given chain. Note that, as $\bm\delta$ is a chain, $X'$ is connected.

We give a recursive definition of two functions $\tl,\tr:[1,r]\rightarrow [1,r]$, which are auxiliary to $\bm\delta$. The idea is that, for each index $1\leq i\leq r$, the following holds: $i\in [\tl i,\tr i]$; the curves in $\bm\delta$ indexed by $[\tl i,\tr i]$ fill a proper subsurface of $X'$; and it is impossible to pick a larger interval with the same property. However, we would also like that, every time two indices $i\not=i''$ have $[\tl i,\tr i]\not= [\tl i'',\tr i'']$ and these two intervals both give families of curves not filling the entire $X'$, there is an index $i< i'< i''$ such that the curves indexed by $[\tl i',\tr i']$ do fill $X'$ instead.

Start by setting $\tl 1\coloneqq 1$, and let $\tr 1$ be the highest index $L$ such that the curves $\delta_1,\ldots,\delta_L$ do not fill $X'$. Now, given an index $1<i_0\leq r$, suppose we have defined $\tl i,\tr i$ for all $i< i_0$: if $\tr(i_0-1)\geq i_0$, set $\tl i_0\coloneqq\tl(i_0-1), \tr i_0\coloneqq\tr(i_0-1)$. Else let $\tl i_0\coloneqq\tl(i_0-1)$ and $\tr i_0\coloneqq i_0$, so that the curves $\delta_{\tl i_0},\ldots,\delta_{\tr i_0}$ fill $X'$; furthermore, let $\tl(i_0+1)\coloneqq i_0+1$, and let $\tr(i_0+1)$ be the highest index $L$ such that $\delta_{i_0+1},\ldots,\delta_L$ do not fill $X'$.

For each $i$, let $\bm\delta(i)=(\delta_{\tl i},\ldots,\delta_{\tr i})$, and let $Y(i)$ be the subsurface of $X'$ filled by the curves in $\bm\delta(i)$. We call elements of the family $\{Y(i)|1\leq i\leq r\}\setminus\{X'\}$ the \nw{$1$-good} subsurfaces.

Again with a recursive definition, for $k>1$ we say that $Z\subseteq S$ is a \nw{$k$-good} subsurface with respect to $\bm\delta$ if there is an index $1\leq i\leq r$ such that $Y(i)\not=X'$ and $Z$ is a $(k-1)$-good subsurface of $Y(i)$, computed with respect to the chain $\bm\delta(i)$.

Also, a subsurface of $S$ is \nw{good} with respect to $\bm\delta$ if it is $k$-good for some $k\in\mathbb N$. We prove a fact about good subsurfaces:

\begin{claim}
Given $Z\subsetneq X'$ a good subsurface (with respect to $\bm\delta$), there is at most one $1$-good subsurface $Y$ such that $Z\subseteq Y$. Moreover two subsurfaces $Y(i),Y(i')$ which are $\subsetneq X'$, but such that $Y(i'')=X'$ for some $i<i''<i'$, are certainly distinct.
\end{claim}

\begin{proof}
We prove the two claims together, with an an argument by contradiction that works for both. Supposing that either of the two is false may be restated as the following hypothesis: there is a good subsurface $Z$ (for the second statement, $Z\coloneqq Y(i)=Y(i')$), contained in two $1$-good subsurfaces $Y(i),Y(i')$, not necessarily distinct but with an index $i<i''<i'$ with $Y(i'')=X'$. 

This means that $Z$, when appearing as a good subsurface of $Y(i)$ with respect to the chain $\bm\delta(i)$, gets filled by some curves $\delta_{\iota_-},\ldots,\delta_{\iota_+}$, for $\tl i\leq \iota_-< \iota_+ \leq \tr i$. Similarly with respect to the index $i'$: $Z$ is filled by $\delta_{\iota'_-},\ldots,\delta_{\iota'_+}$, for $\tl i'\leq \iota'_-< \iota'_+ \leq \tr i'$.

But, by definition, $\{\delta_{\iota_-},\ldots,\delta_{\iota'_+}\}\supseteq \{\delta_{\tl i''},\ldots,\delta_{\tr i''}\}$: so these families both fill $X'$. Therefore there is some $\delta_{\iota(2)}$ (for $\iota_+< \iota(2)< \iota'_-$) intersecting $\partial Z$; moreover there are $\iota_-\leq\iota(1)\leq\iota_+$, $\iota'_-\leq\iota(3)\leq\iota'_+$ such that both $\delta_{\iota(1)}$ and $\delta_{\iota(3)}$ intersect $\delta_{\iota(2)}$: this is because we are picking from two families of curves that fill $Z$. Since the curves $\delta_{\iota(1)}, \delta_{\iota(2)}, \delta_{\iota(3)}$, in the given order, are taken from the sequence $(\gamma_1,\ldots,\gamma_q)$ with respect to which $\bm\tau$ is arranged, this contradicts Corollary \ref{cor:subsurfacesdontrepeat} above.
\end{proof}

The definition of $\tl$, $\tr$ and the second statement of this claim together imply what follows. Let $1<x_1<\ldots<x_\eta\leq r$ are the indices $i$ such that $Y(i)=X'$ (in particular, for these indices, $\tr i=i$); note that this collection includes no two consecutive values of $i$. If an interval $I$ of consecutive indices $i$ does not include any of these ones, then the corresponding $Y(i)$ are always the same surface. If $1\leq i< x_j < i' \leq r$ for some $1\leq j\leq\eta$, and $Y(i),Y(i')\not=X'$, then $Y(i)\not= Y(i')$. So the number $\eta'$ of $1$-good subsurfaces of $X'$ equals either $\eta$ or $\eta+1$. The quantities $\eta,\eta'$ will be recalled below.

Let $Y_1,\ldots,Y_{\eta'}$ be an enumeration of these subsurfaces, in the same order as the sequence $\left(Y(i)\right)_i$ but avoiding repetitions and occurrences of $X'$. And, for each $1\leq u\leq \eta'$, if $Y_u=Y(i)$, let $\bm\delta_u=\bm\delta(i)$ be the chain sequence, subsequence of $\bm\delta$, certifying that $Y_u$ is a good subsurface. This is independent of the choice of $i$.

Now define, for $0\leq j,j'\leq N$,
$$d''_{\bm\delta}(\tau_j,\tau_{j'})\coloneqq \sum_{Y \subseteq X' \text{ good w.r.t. } \bm\delta} [d_Y(\tau_j,\tau_{j'})]_M.$$

Clearly, $d''_{\bm\delta}(\tau_j,\tau_{j'})\leq d'_{\pa(X')}(\tau_j,\tau_{j'})$. We prove the following claim: 
\begin{claim}
Given the arranged sequence $\bm\tau=(\tau_j)_{j=0}^N$ as in the statement of Proposition \ref{prp:tcbound}, let $\bm\delta=(\delta_1\coloneqq\gamma_{t_1},\ldots,\delta_r\coloneqq\gamma_{t_r})$ be a chain subsequence of the $(\gamma_t)_{t=1}^q$ ($t_1<\ldots<t_r$), and call $X' \subseteq S$ ($X'=S$ is allowed here) the essential, non-annular subsurface filled by the curves in $\bm\delta$. Let $a_-\leq \min DI_{t_1}, a_+\geq \max DI_{t_r}$ be two indices along the sequence $\bm\tau$, with $\pi_{X'}\left(V(\tau_{a_+})\right) \in \pa^0(X')$.

Then there are two constants $c_3, c_4$, only depending on the topological types of $S$ and $X'$, such that $r\leq c_3 d''_{\bm\delta}(\tau_{a_-},\tau_{a_+})+c_4$ (which is in turn $\leq c_3 d'_{\pa(X')}(\tau_{a_-},\tau_{a_+})+c_4$).
\end{claim}
\begin{proof}\label{prf:tcbound}

\step{1} Set-up of the induction on $\xi(X')$.

As previously noted (Lemma \ref{lem:decreasingfilling} and subsequent observations), the condition $\pi_{X'}\left(V(\tau_{a_+})\right) \in \pa^0(X')$ implies that, for all $a_-\leq j\leq a_+$ and all subsurfaces $Y\subseteq X'$, $\pi_Y\left(V(\tau_j)\right) \in \pa^0(Y)$.

For ease of notation, for $i=1,\ldots,r$ let $a_i \coloneqq\min DI_{t_i}$, %$a_{i+}\coloneqq\max DI_{t_i}$;
and $m_i\coloneqq\rot_{\bm\tau}(\delta_i,DI_{t_i})$. Let moreover $\rho_j\coloneqq \tau_j|X'$ for all $j$.

The induction basis is for $X'\cong S_{0,4}$ or $\cong S_{1,1}$. In that case 
$$d''_{\pa(X')}\left(\tau_{a_-},\tau_{a_+}\right)=d_{X'}\left(V(\tau_{a_-}),V(\tau_{a_+})\right)\geq d_{\cc(X')}\left(V(\rho_{a_-}),V(\rho_{a_+})\right)-2\tilde n_1$$
where $\tilde n_1\coloneqq 2F(8 N_1(S))$ (see Lemma \ref{lem:induction_vertices_commute}). Lemma \ref{lem:tcboundsimplest}, applied with reference to the family $\delta_1,\ldots,\delta_r$, yields that $d_{\cc(X')}\left(V(\rho_{a_-}),V(\rho_{a_+})\right) \geq r/3-1$. So the claim is proved with $c_3(X')= 3, c_4(X') = 3(1+2\tilde n_1)$.

We get now to the induction step, i.e. prove that the statement holds for $X'$, provided that $X'$ is not homeomorphic to $S_{0,4}$ or $S_{1,1}$ and that the statement holds for any essential, non-annular $Y\subsetneq X'$.

\step{2} For any $\alpha_-\in V(\tau_{a_-})$ and $\alpha_+\in V(\tau_{a_+})$, $d_{\cc({X'})}(\alpha_-,\alpha_+)\geq \lfloor (\eta-1)/2\rfloor$.

For $1\leq j \leq \eta$ consider the previously defined $x_j$ and let $\sigma_j\coloneqq\rho_{a_{\tl x_j}}$. As already noted, for any fixed $1\leq j \leq \eta$, the chain sequence $\bm\delta(x_j)= \{\delta_{\tl x_j},\ldots,\delta_{x_j}\}$ fills $X'$. As all curves in each sequence $\bm\delta(x_j)$ are carried by the respective $\sigma_j$, each $\sigma_j$ fills $X'$; and, by Lemma \ref{lem:cf_decreasing}, $\cf(\sigma_j)\supseteq \cf(\sigma_{j+1})$.

If a curve $\delta_k$ belongs to the sequence $\bm\delta(x_{j+1})$ (for $1\leq j<r$), then it belongs to the set $\cc_1\left(\sigma_j;\bm\delta(x_j)\right)$: this is because, for each $i \in [(\tl x_j)+1, k]$, $\delta_i\in \cc(\rho_{a_i})$ intersects the previous $\delta_{i-1}$; so, according to the first statement of Lemma \ref{lem:weightsaftertwist} applied to the sequence $\bm\tau(a_{\tl x_j}, a_i)$, we have $\sigma_j.\delta_i\supseteq \sigma_j.\delta_{i-1}$. Therefore $\sigma_j.\delta_k$ includes all $\sigma_j.\delta$ for $\delta\in \bm\delta(x_j)$, and this is exactly what has just been claimed.

Now, for $1\leq j\leq \eta-2$ consider any curve $\alpha\in \cf(\sigma_{j+2})$: $\alpha$ will intersect some $\delta\in \bm\delta(x_{j+1})$ because that collection fills $X'$. So Lemma \ref{lem:weightsaftertwist}, applied along the sequence $\bm\tau(a_{\tl x_j},a_{\tl x_{j+2}})$ with respect to the twist curve $\delta$, yields that there is a diagonal extension $\omega_j$ of a subtrack of $\sigma_j$ which fills $X'$, where $\alpha$ is carried with $\omega_j.\alpha\supseteq \omega_j.\delta=\sigma_j.\delta$; and, as $\sigma_j.\delta$ contains all the carrying images in $\sigma_j$ of the curves in $\bm\delta(x_j)$, it is a subtrack filling $X'$: $\alpha\in \cf_1(\sigma_j)$.

Lemma \ref{lem:ccnesting} gives then $\cf(\sigma_{j+2})\subseteq \cf_1(\sigma_j)\subseteq \nei_1\left(\cf(\sigma_j)\right)$. Now the argument goes as in Lemma \ref{lem:tcboundsimplest}. Nesting these inclusions, for all pairs of indices $1\leq j<j'\leq \eta$ such that $2|(j'-j)$, we get
$$
\mathcal N_{(j'-j)/2}(\cf(\sigma_{j'}))\subseteq \cf(\sigma_j).
$$

Denote $\hat \eta\coloneqq \lfloor(\eta-1)/2\rfloor$. If, for any $\alpha_+\in V(\rho_{a_+})\subseteq \cf(\sigma_{a_{2\hat \eta}+1})$, $\alpha_-\in V(\rho_{a_-})$, we have $d_{X'}(\alpha_-,\alpha_+)< \hat\eta$, then
$$\alpha_- \in \mathcal N_{\hat\eta - 1}(\cf(\sigma_{a_{2\hat \eta}+1}))\subseteq \cf(\sigma_{3}) \subseteq \cf_1(\sigma_1),$$
as a consequence of the inclusions proved above. Also, $\cf_1(\sigma_1)\subseteq \cf_1(\rho_{a_-})$, due to an argument already employed, based on the last statement of Lemma \ref{lem:cf_decreasing}: fix any $\xi\in \cf_1(\sigma_1)$. Then there is a $\omega_1\in \e(\sigma'_1)\subseteq \f(\sigma_1)$, where $\sigma'_1$ is a subtrack of $\sigma_1$ that fills $X'$; and $\xi$ is carried by $\omega_1$ and traverses all branches of $\sigma'_1$. By said Lemma, there is a $\omega_-\in \e(\rho_{a_-}')\subseteq \f(\rho_{a_-})$, with $\rho_{a_-}'$ a subtrack of $\rho_{a_-}$ filling $X'$, such that $\omega_-$ fully carries $\omega_1$, which carries $\xi$; moreover $\rho_{a_-}'$ fully carries $\sigma'_1$ so, by composition of carrying maps, $\xi$ will traverse in $\omega_-$ all branches belonging to $\rho_{a_-}'$.

But $\alpha_- \in \cf_1(\rho_{a_-})$ cannot be true because of Lemma \ref{lem:vertexnotinterior}. So $d_{X'}(\alpha_-,\alpha_+)\geq \hat \eta$, proving the claim of this step.

\step{3} Proof of the key claim for Proposition \ref{prp:tcbound}.

Write
$$d''_{\bm\delta}(\tau_{a_-},\tau_{a_+})= [d_{X'}(\tau_{a_-},\tau_{a_+})]_M + \sum_{\substack{Y\subset {X'}\text{ }k \text{-good w.r.t. }\bm\delta \\ (k\geq 1)}} [d_Y(\tau_{a_-},\tau_{a_+})]_M.$$

The fact proved above the beginning of the present proof implies that the set of the $k$-good subsurfaces for ${X'}$, $k\geq 1$, can be partitioned into families: one for each $1\leq u\leq \eta'$, consisting of the good subsurfaces of $Y_u$ which are good with respect to $\bm\delta_u$. So the above summation can be split accordingly. As for the first term instead, from Step 2 we get that $[d_{X'}(\tau_{a_-},\tau_{a_+})]_M\geq d_{X'}(\tau_{a_-},\tau_{a_+}) - M \geq \eta/2 - 1 -M$. So
\begin{eqnarray*}
 & d''_{\bm\delta}(\tau_{a_-},\tau_{a_+}) \geq \left(\sum_{u=1}^{\eta'} d''_{\bm\delta_u}(\tau_{a_-},\tau_{a_+})\right) + \eta/2 - 1 -M & \\
 & \geq \sum_{u=1}^{\eta'} \left(1/2 + d''_{\bm\delta_u}(\tau_{a_-},\tau_{a_+})\right) - 3/2 -M & 
\end{eqnarray*}
(where we are also using the fact that $\eta\leq \eta'\leq \eta+1$). Now we apply the induction hypothesis and get that, if $r(u)$ is the length of the chain $\bm\delta_u$, the last expression is
$$
\geq \sum_{u=1}^{\eta'} \left(1/2 + [r(u)/\hat c_3 - \hat c_4]_0\right) - 3/2 -M,
$$
where $\hat c_3, \hat c_4$ are upper bounds for the constants $c_3(Y), c_4(Y)$ over all topological types of subsurfaces $Y$ of ${X'}$; and the notation $[\cdot]_0$ indicates that we consider this summand only if it is positive. There will be a constant $c'$ (depending on ${X'}$) such that $1/2 + [r(u)/\hat c_3 - \hat c_4]_0\geq 1/3 +c' r(u)$, so the expression is 
$$
\geq \eta'/3 + c'\sum_{u=1}^{\eta'} r(u) - 3/2 -M \geq \min\{1/3,c'\}\left(\eta'+\sum_{u=1}^{\eta'} r(u)\right) -3/2 -M \geq r -3/2 -M.
$$

The last inequality is due to the following argument: the sequence $\bm\delta$ consists of the junction of the sequences, and elements, $\bm\delta_1,\delta_{x_1},\bm\delta_2,\ldots,\bm\delta_\eta,\delta_{x_\eta},[\bm\delta_{\eta+1}]$ (the last sequence may not exist). So its length is $r=\eta+\sum_{u=1}^{\eta'} r(u)$.

This concludes the proof of the key claim.
\end{proof}

We now prove Proposition \ref{prp:tcbound}. Let $\bm\delta=(\delta_1\coloneqq\gamma_{t_1},\ldots,\delta_r\coloneqq\gamma_{t_r})$ be a chain subsequence \emph{with maximal length} of $\gamma_1,\ldots,\gamma_q$ as defined in the statement. Let $r$ be the length of this chain, and let $X'$ be the subsurface of $X$ (and $S$) filled by the curves in $\bm\delta$. Then
$$r\leq c_3(X') d'_{\pa(X')}(\tau_k,\tau_l)+c_4(X')\leq c_3(X') d'_{\pa(X)}(\tau_k,\tau_l)+c_4(X')$$
($d'_{\pa(X)}$ is indeed a summation involving all summands already present in $d'_{\pa(X')}$). Lemma \ref{lem:chainbound} gives then
$$
q\leq \xi(X)r \leq \xi(X)\left( c_3(X') d'_{\pa(X)}(\tau_k,\tau_l)+c_4(X') \right).
$$

Let now $\tilde c_3(S)\coloneqq \max_{Y\subseteq S} c_3(Y)$, $\tilde c_4(S)\coloneqq \max_{Y\subseteq S} c_4(Y)$. Since\linebreak $\xi(S) = \max_{Y\subseteq S} \xi(Y)$ we have
$$
q \leq \xi(S)\left( \tilde c_3(S) d'_{\pa(X)}(\tau_k,\tau_l)+\tilde c_4(S) \right).
$$
which defines the required constants $C_3(S),C_4(S)$. This completes the proof.