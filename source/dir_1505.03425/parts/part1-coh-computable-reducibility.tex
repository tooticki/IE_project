
Before proving that~$\coh$ does not imply $\amt$ over~$\rca$,
we illustrate the key features of our construction by 
showing that $\amt$ does not reduce to~$\coh$ in one step.
This one-step reducibility is known as \emph{computable reducibility}~\cite{DzhafarovStrong,Hirschfeldt2016notions,Patey2016weakness}.
The general construction will consist of iterating this 
one-step diagonalization to construct a Turing ideal
whose functions are dominated by a single $\Delta^0_2$ function.

\begin{definition}[Computable reducibility]
A principle $\Psf$ is \emph{computably reducible} to another principle~$\Qsf$ (written $\Psf \leq_c \Qsf$)
if every~$\Psf$-instance~$I$ computes a~$\Qsf$-instance~$J$ such that for every solution~$X$ to~$J$,
$X \oplus I$ computes a solution to~$I$.
\end{definition}

The remainder of this section is devoted to the proof of the following theorem.

\begin{theorem}\label{thm:amt-comp-reduc-coh}
$\amt \not \leq_c \coh$
\end{theorem}

In order to prove Theorem~\ref{thm:amt-comp-reduc-coh},
we need to construct a $\Delta^0_2$ function $f$ such that 
for every uniformly computable sequence of sets~$\vec{R} = R_0, R_1, \dots$,
there is an $\vec{R}$-cohesive set~$G$ such that every~$G$-computable
function is dominated by~$f$. Thankfully, Jockusch and Stephan~\cite{Jockusch1993cohesive}
proved that for every such sequence of sets~$\vec{R}$, 
every p-cohesive set computes an infinite $\vec{R}$-cohesive set.
The sequence of all primitive recursive sets is therefore called a \emph{universal instance}.
Hence we only need to build a $\Delta^0_2$ function~$f$ and a p-cohesive set~$G$
such that every $G$-computable function is dominated by~$f$ to obtain Theorem~\ref{thm:amt-comp-reduc-coh}.

Given some uniformly computable sequence of sets~$\vec{R} = R_0, R_1, \dots$,
the usual construction of an $\vec{R}$-cohesive set~$G$ is done by a computable Mathias forcing.
The forcing conditions are pairs $(F,X)$, where~$F$ is a finite set representing the finite
approximation of~$G$ and~$X$ is an infinite, computable reservoir such that~$max(F) < min(X)$.
The construction of the~$\vec{R}$-cohesive set is obtained by building
an infinite, decreasing sequence of Mathias conditions, starting with~$(\emptyset, \omega)$
and interleaving two kinds of steps.
Given some condition~$(F,X)$,
\begin{itemize}
	\item[(S1)] the \emph{extension} step consists of taking an element $x$ from $X$ and adding it to~$F$,
	thereby forming the extension $(F \cup \{x\}, X \setminus [0,x])$;
	\item[(S2)] the \emph{cohesiveness} step consists of deciding which one of $X \cap R_i$
	and $X \cap \overline{R}_i$ is infinite, and taking the chosen one as the new reservoir.
\end{itemize}
The first step ensures that the constructed set~$G$ will be infinite, whereas
the second step makes the set $G$ $\vec{R}$-cohesive.
Looking at the effectiveness of the construction, the step (S1) is computable,
assuming we are given some Turing index of the set~$X$.
The step (S2), on the other hand, requires to decide which one of two computable sets
is infinite, knowing that at least one of them is. This decision
requires the computational power of a PA degree relative to~$\emptyset'$ (see \cite[Lemma 4.2]{Cholak2001strength}).
Since we want to build a $\Delta^0_2$ function~$f$ dominating every $G$-computable function,
we would like to make the construction of~$G$ $\Delta^0_2$. Therefore the step (S2) has to be revised.

\subsection{Effectively constructing a cohesive set}

The above construction leads to two observations.
First, at any stage of the construction, the reservoir~$X$ of the Mathias condition~$(F, X)$
has a particular shape. Indeed, after the first application of stage~(S2), 
the set $X$ is, up to finite changes, of the form $\omega \cap R_0$
or $\omega \cap \overline{R_0}$. After the second application of (S2), it is in one of the following forms: $\omega \cap R_0 \cap R_1$,
$\omega \cap R_0 \cap \overline{R}_1$, $\omega \cap \overline{R}_0 \cap R_1$,
$\omega \cap \overline{R}_0 \cap \overline{R}_1$, and so on. More generally, given some string~$\sigma \in 2^{<\omega}$,
we can define~$R_\sigma$ inductively as follows:
First, $R_\varepsilon = \omega$, and then, if $R_\sigma$ has already been defined for some string $\sigma$ of length~$i$,
$R_{\sigma 0} = R_\sigma \cap \overline{R}_i$ and~$R_{\sigma 1} = R_\sigma \cap R_i$.
By the first observation, we can replace Mathias conditions by pairs ~$(F, \sigma)$, where $F$ is a finite set
and $\sigma \in 2^{<\omega}$. The pair~$(F, \sigma)$ denotes the Mathias condition~$(F, R_\sigma \setminus [0, max(F)])$.
A pair $(F, \sigma)$ is \emph{valid} if $R_\sigma$ is infinite.
The step (S2) can be reformulated as choosing, given some valid condition $(F, \sigma)$, which one of $(F, \sigma 0)$
and $(F, \sigma 1)$ is valid.

%\begin{center}
%\begin{tikzpicture}[x=2.6cm, y=2cm, 
%		node/.style={}, 
%		arrow/.style={black, thick,->},
%		continue/.style={loosely dotted, thick},
%		hidden/.style={opacity=0.2}
%]

%  \node[node] (a0) at (1, 1) {$(\emptyset, \varepsilon)$};
%	\node[node] (a00) at (2, 1) {$(\{x\}, \varepsilon)$};
%	\node[node] (a000) at (3, 1.5) {$(\{x\}, 0)$};
%	\node[node] (a001) at (3, 0.5) {$(\{x\}, 1)$};
%	\node[node] (a0000) at (4, 1.5) {$(\{x, y\}, 0)$};
%	\node[node] (a0010) at (4, 0.5) {$(\{x, z\}, 1)$};
%	\node[node] (a00000) at (5, 2) {$(\{x, y\}, 00)$};
%	\node[node] (a00001) at (5, 1.5) {$(\{x, y\}, 01)$};
%	\node[node] (a00100) at (5, 0.5) {$(\{x, z\}, 10)$};
%	\node[node] (a00101) at (5, 0) {$(\{x, z\}, 11)$};

	% Finite transitive set F
%	\draw[arrow] (a0) to  node [above, black] {\tiny (S1)} (a00);
%	\draw[arrow] (a00) -- (a000);
%	\draw[arrow] (a00) -- (a001);
%	\draw[arrow] (a001) to  node [below] {\tiny (S1)} (a0010);
%	\draw[arrow] (a000) to  node [above] {\tiny (S1)} (a0000);
%	\draw[arrow] (a0010) -- (a00100);
%	\draw[arrow] (a0010) to  node [below] {\tiny (S2)} (a00101);
%	\draw[arrow] (a0000) to  node [above] {\tiny (S2)} (a00000);
%	\draw[arrow] (a0000) -- (a00001);

%	\node[node] (s20) at (2.5,1) {\tiny (S2)};

%	\draw[continue] (5.5, 2) -- (6, 2);
%	\draw[continue] (5.5, 1.5) -- (6, 1.5);
%	\draw[continue] (5.5, 0.5) -- (6, 0.5);
%	\draw[continue] (5.5, 0) -- (6, 0);
%\end{tikzpicture}
%\end{center}

Second, we do not actually need to decide which one of~$R_{\sigma 0}$ and~$R_{\sigma 1}$ is infinite
assuming that~$R_{\sigma}$ is infinite. Our goal is to dominate every~$G$-computable function with a $\Delta^0_2$ function $f$.
Therefore, given some $G$-computable function~$g$, it is sufficient to find a finite set $S$ of candidate values for~$g(x)$ 
and make~$f(x)$ be greater than the maximum of $S$. Instead of choosing which one of~$R_{\sigma 0}$ and~$R_{\sigma 1}$ is infinite,
we will explore both cases in parallel. The step (S2) will split some condition $(F, \sigma)$
into two conditions~$(F, \sigma 0)$ and $(F, \sigma 1)$. Our new forcing conditions are therefore tuples $(F_\sigma : \sigma \in 2^n)$
which have to be thought of as $2^n$ parallel Mathias conditions $(F_\sigma, \sigma)$ for each~$\sigma \in 2^n$.
Note that $(F_\sigma, \sigma)$ may not denote a valid Mathias condition in general since $R_\sigma$ may be finite.
Therefore, the step (S1) becomes~$\Delta^0_2$, since we first have to check whether $R_\sigma$ is non-empty
before picking an element in~$R_\sigma$. The whole construction is $\Delta^0_2$ and yields a $\Delta^0_2$ infinite
binary tree $T$. In particular, any degree PA relative to $\emptyset'$ bounds an infinite path though $T$ and therefore bounds a $G$-cohesive set. However, the degree of the set $G$ is not sensitive in our argument. We only care about the effectiveness of
the tree~$T$.

\subsection{Dominating the functions computed by a cohesive set}

We have seen in the previous section how to make the construction of a cohesive set more effective
by postponing the choices between forcing~$G \subseteq^{*} R_i$ and~$G \subseteq^{*} \overline{R}_i$
to the end of the construction. We now show how to dominate every $G$-computable function
for every infinite path~$G$ through the $\Delta^0_2$ tree constructed in the previous section.
To do this, we will interleave a third step deciding whether $\Phi^G_e(n)$ halts, and if so, collecting the 
candidate values of~$\Phi^G_e(n)$.
Given some Mathias precondition~$(F, X)$ (a precondition is a condition 
where we do not assume that the reservoir is infinite) and some $e,x \in \omega$, we can $\Delta^0_2$-decide 
whether there is some set $E \subseteq X$ such that~$\Phi^{F \cup E}_e(x) \downarrow$.
If this is the case, then we can effectively find this a finite set~$E \subseteq X$ and
compute the value~$\Phi^{F \cup E}_e(x)$. If this is not the case, then for every infinite set~$G$ satisfying
the condition~$(F, X)$, the function~$\Phi^G_e$ will not be defined on input~$x$. In this case,
our goal is vacuously satisfied since $\Phi^G_e$ will not be a function and therefore
we do not need do dominate~$\Phi^G_e$. 
Let us go back to the previous construction.
After some stage, we have constructed a condition~$(F_\sigma : \sigma \in 2^n)$ inducing a finite tree of depth~$n$. 
The step (S3) acts as follows for some~$x \in \omega$:
\begin{itemize}
	\item[(S3)] Let~$S = \{0\}$. For each~$\sigma \in 2^n$ and each~$e \leq x$, decide whether 
	there is some finite set~$E \subseteq R_\sigma \setminus [0, max(F_\sigma)]$ such that~$\Phi^{F_\sigma \cup E}_e(x) \downarrow$.
	If this is the case, add the value of $\Phi^{F_\sigma \cup E}_e(x)$ to $S$ and set~$\tilde{F}_\sigma = F_\sigma \cup E$, otherwise set $\tilde{F}_\sigma = F_\sigma$.
	Finally, set~$f(x) = max(S)+1$ and take~$(\tilde{F}_\sigma : \sigma \in 2^n)$ as the next condition.
\end{itemize}
Note that the step (S3) is $\Delta^0_2$-computable uniformly in the condition~$(F_\sigma : \sigma \in 2^n)$.
The whole construction therefore remains~$\Delta^0_2$ and so does the function~$f$.
Moreover, given some $G$-computable function~$g$, there is some Turing index~$e$ such that~$\Phi^G_e = g$.
For each $x \geq e$, the step (S3) is applied at a finite stage and decides whether~$\Phi^G_e(x)$ halts or not
for every set satisfying one of the leaves of the finite tree. In particular, this is the case for the set $G$ and 
therefore $\Phi^G_e(x) \in S$. By definition of $f$, $f(x) \geq max(S) \geq \Phi^G_e(x)$. Therefore $f$ dominates the function~$g$.

\subsection{The formal construction}

Let~$\vec{R} = R_0, R_1, \dots$ be the sequence of all primitive recursive sets.
We define a $\Delta^0_2$ decreasing sequence of conditions~$(\emptyset, \varepsilon) \geq c_0 \geq c_1 \dots$
such that for each~$s \in \omega$
\begin{itemize}
	\item[(i)] $c_s = (F^s_\sigma : \sigma \in 2^s)$ and~$|F^s_\sigma| \geq s$ if~$R_\sigma \setminus [0, max(F^s_\sigma)] \neq \emptyset$.
	\item[(ii)] For every~$e \leq s$ and every~$\sigma \in 2^s$, either $\Phi^{F^s_\sigma}_e(s) \downarrow$
	or $\Phi^G_e(s) \uparrow$ for every set~$G$ satisfying $(F^s_\sigma, R_\sigma)$.
\end{itemize}
Let~$P$ be a path through the tree $T = \{ \sigma \in 2^{<\omega} : R_\sigma \mbox{ is infinite} \}$
and let~$G = \bigcup_s F^s_{P \restr s}$. By (i), for each~$s \in \omega$, $|F^s_{P \restr s}| \geq s$
since~$R_{P \restr s}$ is infinite. Therefore the set~$G$ is infinite.
Moreover, for each~$s \in \omega$, the set~$G$ satisfies the condition~$(F^{s+1}_{P \restr s+1}, R_{P \restr s+1})$,
so~$G \subseteq^{*} R_{P \restr s+1} \subseteq R_s$ if~$P(s) = 1$ and
$G \subseteq^{*} R_{P \restr s+1} \subseteq \overline{R}_s$ if~$P(s) = 0$. 
Therefore~$G$ is $\vec{R}$-cohesive.

For each~$s \in \omega$, let~$f(s) = 1 + max(\Phi^{F^s_\sigma}_e(s) : \sigma \in 2^s, e \leq s)$.
The function~$f$ is $\Delta^0_2$. We claim that it dominates every $G$-computable function.
Fix some~$e$ such that~$\Phi^G_e$ is total. For every~$s \geq e$, let~$\sigma = P \restr s$. By (ii), either
$\Phi^{F^s_\sigma}_e(s) \downarrow$ or $\Phi^G_e(s) \uparrow$ for every
set~$G$ satisfying $(F^s_\sigma, R_\sigma)$. Since $\Phi^G_e(s) \downarrow$,
the first case holds. By definition of~$f$, $f(s) \geq \Phi^{F^s_\sigma}_e(s) = \Phi^G_e(s)$.
Therefore~$f$ dominates the function $\Phi^G_e$. This completes the proof of Theorem~\ref{thm:amt-comp-reduc-coh}.




