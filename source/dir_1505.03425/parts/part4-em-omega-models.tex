
Now we have settled the domination framework, it suffices to implement
the abstract module to obtain $\omega$-structures which do not satisfy $\amt$.
We have illustrated the notion of module by implementing one for $\coh$.
An immediate consequence is the existence of an $\omega$-model of $\coh$
which is not a model of~$\amt$. 
In this section, we shall extend this separation to the Erd\H{o}s-Moser theorem.
As noted before, every $\omega$-model of~$\emo$ which is not a model of~$\amt$
is also a model of~$\coh$.
This section is devoted to the proof of the following theorem.

\begin{theorem}\label{thm:emo-not-amt}
There exists an $\omega$-model of~$\emo$ which is not a model of~$\amt$.
\end{theorem}

At first sight, the forcing notion introduced in section~\ref{sect:emo-computable-reducibility} seems
to have a direct mapping to the abstract notion of forcing defined in the domination framework.
However, unlike cohesiveness where the module implementation was immediate,
the Erd\H{o}s-Moser theorem raises new difficulties:
\begin{itemize}
	\item The Erd\H{o}s-Moser theorem is not known to admit a universal instance.
	We will therefore need to integrate the information about the instance in the notion of condition.
	Moreover, the $\iniop$ operator will have to choose accordingly some new instance of~$\emo$
	at every iteration level. We need to make~$\iniop$ computable, but the collection of every infinite computable tournament functionals
	is not even computably enumerable.

	\item The notion of EM condition introduced in section~\ref{sect:emo-computable-reducibility} contains a $\Pi^{0,R}_1$ property
	ensuring extensibility. Since the tournament $R$ depends on the previous iteration which is being constructed,
	we have only access to a finite part of~$R$. We need therefore to ensure that whatever the extension of the finite
	tournament is, the condition will be extendible.
\end{itemize}

We shall address the above-mentioned problems one at a time
in subsections~\ref{subsect:enum-emo-infinite} and~\ref{subsect:emo-new-condition}.

\subsection{Enumerating the infinite tournaments}\label{subsect:enum-emo-infinite}

In section~\ref{sect:emo-computable-reducibility}, we were also confronted to the problem
of enumerating all infinite tournaments
and solved it by relativizing the construction to a low subuniform degree
in order to obtain a low sequence of infinite tournaments containing at least
every infinite computable tournament. We cannot apply the same
trick to handle the construction of an $\omega$-model of~$\emo$
as solutions to some computable tournaments may bound new tournaments
and so on. However, as we shall see, we can restrict ourselves
to primitive recursive tournaments to generate an $\omega$-model of~$\emo$.

Given a sequence of sets $X_0, X_1, \dots$, define $\Mcal_{\vec{X}}$ to be the $\omega$-structure
whose second-order part is the Turing ideal generated by $\vec{X}$, that is,
$$
\{ Y \in 2^\omega : (\exists i)[ Y \leq_T X_0 \oplus \dots \oplus X_i ]\}
$$

\begin{lemma}\label{lem:em-uniform-model}
There exists a uniformly computable sequence of infinite, primitive recursive tournament functionals
$T_0, T_1, \dots$ such that for every sequence of sets $X_0, X_1, \dots$ such that
$X_i$ is an infinite transitive subtournament of $T_i^{X_0 \oplus \dots \oplus X_{i-1}}$
for each $i \in \omega$,
$$
\Mcal_{\vec{X}} \not \models \amt \imp \Mcal_{\vec{X}} \models \emo \wedge \coh
$$
\end{lemma}
\begin{proof}
As $\rca \vdash \semo \wedge \coh \imp \emo$,
it suffices to prove that for every set $X$, 
\begin{itemize}
	\item[(i)] for every stable, infinite, $X$-computable tournament~$R$, there exists an infinite $X$-p.r.\ tournament~$T$
	such that every infinite $T$-transitive subtournament $X$-computes an infinite $R$-transitive subtournament.
	\item[(ii)] for every $X$-computable complete atomic theory~$T$ and every
uniformly $X$-computable sequence of sets $\vec{R}$, there exists an infinite $X$-p.r.\ tournament
such that every infinite transitive subtournament $X$-computes either an $\vec{R}$-cohesive set
or an atomic model of~$T$.
\end{itemize}

(i)
Fix a set $X$ and a stable, infinite, $X$-computable tournament~$R$.
Let $\tilde{f} : \omega \to 2$ be the $X'$-computable function defined by $\tilde{f}(x) = 0$
if $(\forall^\infty s) R(s, x)$ and $\tilde{f}(x) = 1$ if $(\forall^\infty s) R(x, s)$.
By Schoenfield's limit lemma~\cite{Shoenfield1959degrees}, there exists an $X$-p.r.
function $g : \omega^2 \to 2$ such that $\lim_s g(x, s) = \tilde{f}(x)$ for every $x \in \omega$.
Considering the $X$-p.r. tournament $T$ such that $T(x,y)$ holds iff $x < y$ and $g(x, y) = 1$
or $x > y$ and $g(x, y) = 0$, every infinite $T$-transitive subtournament $X$-computes an infinite
$R$-transitive subtournament.

(ii) Jockusch and Stephan proved in~\cite{Jockusch1993cohesive} that for every set~$X$,
and every uniformly $X$-computable sequence of sets~$\vec{R}$, 
every p-cohesive set relative to~$X$ computes an $\vec{R}$-cohesive set.
The author proved in~\cite{Patey2015Somewhere} that for every $X$-computable complete atomic theory~$T$,
there exists an $X'$-computable coloring $f : \omega \to \omega$ such that every infinite
set $Y$ \emph{thin} for $f$ (i.e.\ such that $f(Y) \neq \omega$) $X$-computes an atomic model of~$T$.
He also proved that for every such $X'$-computable coloring $f : \omega \to \omega$,
there exists an infinite, $X$-p.r. tournament~$R$ such that every infinite transitive subtournament
is either p-cohesive, or $X$-computes an infinite set thin for~$f$.
\end{proof}

We can therefore fix this computable enumeration $T_0, T_1, \dots$
of tournament functionals, and make $\iniop(n)$ return an empty condition
paired with $T_n$. Thus, taking at each iteration an infinite set satisfying one of the parts,
we obtain an $\omega$-model of~$\emo$.

\subsection{The new Erd\H{o}s-Moser conditions}\label{subsect:emo-new-condition}

Fix some primitive recursive tournament functional $R$.
According to the analysis of the Erd\H{o}s-Moser presented in section~\ref{sect:emo-computable-reducibility},
we would like to define the forcing conditions to be tuples~$(\vec{F}, \Ccal)$ where
\begin{itemize}
	\item[(a)] $\Ccal$ is a non-empty $\Pi^{0,D}_1$ $k$-cover class of $[t, +\infty)$ 
	for some $k, t \in \omega$
	\item[(b)] $F_\nu \cup \{x\}$ is $R^D$-transitive for every $Z_0 \oplus \dots \oplus Z_{k-1} \in \Ccal$,
	every $x \in Z_\nu$ and each $\nu < k$
	\item[(c)] $Z_\nu$ is included in a minimal $R^D$-interval of $F_\nu$
	for every $Z_0 \oplus \dots \oplus Z_{k-1} \in \Ccal$ and each~$\nu < k$.
\end{itemize}

However, at a finite stage, we have only access to a finite part of~$D$,
and therefore we cannot express the properties (a-c). Indeed,
we may have made some choices about the $F$'s such that $F_\nu \cup \{x\}$
is not $R^D$-transitive for every part~$\nu$, every $D$ satisfying the previous iterations and cofinitely many $x \in \omega$.
We need therefore to choose the $F$'s carefully enough so that whatever the extension of the finite tournament
to which we have access, we will be able to extend at least one of the $F$'s.

The initial condition $(\{\emptyset\}, \{\omega\})$ satisfies the properties (a-c) no matter what $D$ is, since $\{\omega\}$
does not depend on $D$.
Let us have a closer look at the question $Q2$ asked in section~\ref{sect:emo-computable-reducibility}.
For the sake of simplification, we will consider that the question is asked below the unique 
part of the initial condition. It therefore becomes:

\smallskip
{\itshape
Q3: Is there a finite set $E \subseteq \omega$ such that for every 2-partition $\tuple{E_0, E_1}$ of~$E$,
there exists an $R^D$-transitive subset $F_1 \subseteq E_i$ for some $i < 2$ such that $\varphi(D, F_1)$ holds?
}
\smallskip

Notice that this is a syntactic question since it depends on the purely formal variable $D$
representing the effective join of the sets constructed in the previous iterations.
Thanks to the usual query process, we are able to transform it into a concrete $\Sigma^0_1$
formula getting rid of the formal parameter~$D$, 
and obtain some answer that the previous layers guarantee to hold for every set $D$
satisfying the previous iterations.

If the answer is negative, then by compactness, for every set $D$ satisfying the previous iterations,
there is a 2-partition $Z_0 \cup Z_1 = \omega$ such that for every $i < 2$ and every $R^D$-transitive
subset $G \subseteq Z_i$, $\varphi(D, G)$ does not hold. For every set $D$,
the $\Pi^{0,D}_1$ class $\Ccal$ of such 2-partitions $Z_0 \oplus Z_1$ is therefore guaranted to be non-empty.
Note again that since $D$ is a syntactic variable, the class $\Ccal$ is also syntactic,
and purely described by finite means.

If the answer is positive, then we are given some finite set $E \subseteq \omega$ witnessing it.
Moreover, we are guaranted that for every set $D$ satisfying the previous iterations,
for every 2-partition $\tuple{E_0, E_1}$ of~$E$, there exists an $R^D$-transitive subset 
$F_1 \subseteq E_i$ for some $i < 2$ such that $\varphi(D, F_1)$ holds.
In we knew the set $D$, we would choose one ``good'' 2-partition $\tuple{E_0, E_1}$
as we do in section~\ref{sect:emo-computable-reducibility}. However, this choice
depends on infinitely many bits of information of $D$. We will need therefore to
try every 2-partition in parallel.

There is one more difficulty.
With this formulation, we are not able to find the desired extension, since $D$ is syntactic,
and therefore we do not know how to identify the color~$i$ and the actual set~$F_1$ given some 2-partition $\tuple{E_0, E_1}$.
Thankfully, we can slightly modify the question to ask to provide the witness $F_1$ for each such a partition
in the answer.

\smallskip
{\itshape
Q4: Is there a finite set $E \subseteq \omega$ and a finite function $g$ such that for every 2-partition $\tuple{E_0, E_1}$ of~$E$,
$g(\tuple{E_0, E_1}) = F_1$ for some $i < 2$ and some $R^D$-transitive subset $F_1 \subseteq E_i$ such that $\varphi(D, F_1)$ holds?
}
\smallskip

The question $Q4$ is equivalent to the question $Q3$, but provides a constructive witness~$g$
in the case of a positive answer as well. We can even formulate the question so that we know the relation $R^D$
over the set~$F_1$. Thus we are able to talk about minimal $R^D$-intervals of $F_1$.

Now, we can extend the initial condition $(\{\emptyset\}, \{\omega\})$
into some condition $(\vec{F}, \Ccal)$ as follows:
For each 2-partition $\tuple{E_0, E_1}$ of $E$, letting $F_1 = g(\tuple{E_0, E_1})$, 
for every minimal $R^D$-interval $I$, we create a part~$\nu = \tuple{E_0, E_1, I}$
and set $F_\nu = F_1$. Take some $t' > max(\vec{F})$
and let $\Ccal$ be the $\Pi^{0,D}_1$ class of covers $\bigoplus_\nu Z_\nu$ of $[t', +\infty)$ such that
for every part $\nu = \tuple{E_0, E_1, I}$
\begin{itemize}
	\item[(b')] $F_\nu \cup \{x\}$ is $R^D$-transitive for every $x \in Z_\nu$
	\item[(c')] $Z_\nu$ is included in the minimal $R^D$-interval $I$
\end{itemize}
Fix some set $D$ satisfying the previous iterations. We claim that $\Ccal$ is non-empty.
Any element $x \in [t', +\infty)$ induces a 2-partition $g(x) = \tuple{E_0, E_1}$ of $E$ by
setting $E_0 = \{ y \in E : R^D(y, x) \}$ and $E_1 = \{y \in E : R^D(x, y)\}$.
On the other hand, for every 2-partition $\tuple{E_0, E_1}$ of $E$,
we can define a partition of $[t', +\infty)$ by setting
$Z_{\tuple{E_0, E_1}} = \{ x \in [t',+\infty) : g(x) = \tuple{E_0, E_1} \}$.
By definition, $E_0 \to_{R^D} Z_{\tuple{E_0, E_1}}  \to_{R^D} E_1$.
Therefore, the cover $\bigoplus_\nu Z_\nu$ of $[t', +\infty)$ defined by
\[
Z_\nu = \cond{
	Z_{\tuple{E_0, E_1}} & \mbox{ if } \nu = \tuple{E_0, E_1, I}, I = (max(F_\nu), +\infty) \mbox{ and } F_\nu \subseteq E_0\\
	Z_{\tuple{E_0, E_1}} & \mbox{ if } \nu = \tuple{E_0, E_1, I}, I = (-\infty, min(F_\nu)) \mbox{ and } F_\nu \subseteq E_1\\
	\emptyset & \mbox{ otherwise}\\
}
\]
is in $\Ccal$ and witnesses the non-emptiness of $\Ccal$.

The problem of having access to only a finite part of the class~$\Ccal$ appears
more critically when considering the question below some part~$\nu$ of an arbitrary condition $c = (\vec{F}, \Ccal)$.
The immediate generalization of the question $Q4$ is the following.

\smallskip
{\itshape
Q5: For every cover $X_0 \oplus \dots \oplus X_{k-1} \in \Ccal$, is there a finite set $E \subseteq X_\nu$ and a finite function~$g$ such that for every 2-partition $\tuple{E_0, E_1}$ of~$E$,
$g(\tuple{E_0, E_1})$ is a finite $R^D$-transitive subset of some $E_j$ such that $\varphi(D, F_\nu \cup g(\tuple{E_0, E_1}))$ holds?
}
\smallskip

As usual, although this question is formulated in a $\Pi^0_2$ manner, it can be turned into a $\Sigma^{0,D}_1$ query 
using a compactness argument.

\smallskip
{\itshape
Q5': Is there some $r \in \omega$, a finite sequence of finite sets $E^0, \dots, E^{r-1}$ 
and a finite sequence of functions $g^0, \dots, g^{r-1}$ such that
\begin{itemize}
	\item[(1)] for every $X_0 \oplus \dots \oplus X_{k-1} \in \Ccal$, there is some $i < r$ such that $E^i \subseteq X_\nu$
	\item[(2)] for every $i < r$ and every 2-partition $\tuple{E_0, E_1}$ of~$E^i$,
	$g^i(\tuple{E_0, E_1})$ is a finite $R^D$-transitive subset of some $E_j$ such that $\varphi(D, F_\nu \cup g^i(\tuple{E_0, E_1}))$ holds?
\end{itemize}
}
\smallskip

In the case of a negative answer, we can apply the standard procedure consisting in refining the $\Pi^{0,D}_1$ class $\Ccal$
into some $\Pi^{0,D}_1$ class~$\Dcal$ forcing $\varphi(D, G)$ not to hold on every part refining the part~$\nu$ in $c$.
The class $\Dcal$ is non-empty since we can construct a member of it from a witness of failure of $Q5$.
The problem appears when the answer is positive. We are given some finite sequence $E^0, \dots, E^{r-1}$ 
and a finite sequence of functions $g^0, \dots, g^{r-1}$ satisfying (i) and (ii).
For every $D$, there is some $X_0 \oplus \dots \oplus X_{k-1} \in \Ccal$ and some $i < r$
such that $E^i \subseteq X_\nu$, but this $i$ may depend on~$D$. We cannot choose some~$E^i$
as we used to do in section~\ref{sect:emo-computable-reducibility}.

Following our moto, if we are not able to make a choice, we will try every possible case in parallel.
The idea is to define a condition $d = (\vec{E}, \Dcal)$ and a refinement function $f$ forking the part~$\nu$ into various parts, each one representing a possible scenario. For every part $\mu$ of $c$ which is different from $\nu$, 
create a part $\mu$ in $d$ and set $E_\mu = F_\mu$. For every $i < r$ and every 2-partition $\tuple{E_0, E_1}$ of $E^i$,
create a part~$\mu = \tuple{i, E_0, E_1}$ in $d$ refining $\nu$ and set $E_\mu = F_\nu \cup g^i(\tuple{E_0, E_1})$.
Accordingly, let $\Dcal$ be the $\Pi^{0,D}_1$ class of covers $\bigoplus_\mu Y_\mu$ of $[t, +\infty)$
such that there is some $i < r$ and some cover $X_0 \oplus \dots \oplus X_{k-1} \in \Ccal$ satisfying first
$E^i \subseteq X_\nu$, second $Y_\mu \subseteq X_{f(\mu)}$ for each part~$\mu$ of $d$
and third $Y_\mu = \emptyset$ if $\mu = \tuple{j, E_0, E_1}$ for some $j \neq i$.

The class $\Dcal$ $f$-refines $\Ccal$, but does not $f$-refine $\Ccal^{[\nu, E^i]}$ for some fixed $i < r$.
Because of this, the condition $d$ does not extends the condition $c$ in the sense of section~\ref{sect:emo-computable-reducibility}.
We shall therefore generalize the operator $\cdot \mapsto \Ccal^{[\nu, \cdot]}$ to define it over tuples of sets.

\smallskip
\emph{Restriction of a cover class}. Given some cover class $(k, Y, \Ccal)$,
some part~$\nu$ of $\Ccal$ and some $r$-tuple $E^0, \dots, E^{r-1}$ of finite sets, we denote by $\Ccal^{[\nu, \vec{E}]}$
the cover class $(k+r-1, Y, \Dcal)$ such that $\Dcal$ is the collection of 
\[
X_0 \oplus \dots \oplus X_{\nu-1} \oplus Z_0 \oplus \dots \oplus Z_{r-1} \oplus X_{\nu+1} \oplus \dots \oplus X_{k-1}
\]
such that $X_0 \oplus \dots \oplus X_{k-1} \in \Dcal$ and there is some $i < r$ such that
$E^i \subseteq X_\nu$, $Z_i = X_\nu$ and $Z_j = \emptyset$ for every $j \neq i$.
In particular, $\Ccal^{[\nu, \vec{E}]}$ refines $\Ccal$ with some refinement function $f$ which forks the part~$\nu$
into~$r$ different parts. Such a function $f$ is called the \emph{refinement function witnessing the restriction}.

We need to define the notion of extension between conditions accordingly.
A condition $d = (\vec{E}, \Dcal)$ \emph{extends} a condition $c = (\vec{F}, \Ccal)$
(written $d \leq c$) if there is a function $f : parts(\Dcal) \to parts(\Ccal)$ such that the following holds:
\begin{itemize}
	\item[(i)] $(E_\nu, dom(\Dcal))$ Mathias extends $(F_{f(\nu)}, dom(\Ccal))$ for each $\nu \in parts(\Dcal)$ 
	\item[(ii)] Every $\bigoplus_\mu Y_\mu \in \Dcal$ $f$-refines some $\bigoplus_\nu X_\nu \in \Ccal$
	such that for each part~$\mu$ of $d$, either $E_\mu \setminus H_{f(\mu)} \subseteq X_{f(\mu)}$,
	or $Y_\mu = \emptyset$.
\end{itemize}

Note that this notion of extension is coarser than the one defined in section~\ref{sect:emo-computable-reducibility}.
Unlike with the previous notion of extension, there may be from now on some part~$\mu$ of $d$ refining the part~$\nu$ of $c$,
such that $(E_\mu, Y_\mu)$ does not Mathias extend $(F_\nu, X_\nu)$ for some $\bigoplus_\mu Y_\mu \in \Dcal$
and every $\bigoplus_\nu X_\nu \in \Ccal$, but in this case, we make $(E_\mu, Y_\mu)$ non-extendible by ensuring that $Y_\mu = \emptyset$.

\subsection{Implementing the Erd\H{o}s-Moser module}

We are now ready to provide a concrete implementation of a module support and a module for~$\emo$.
Define the tuple $\Sb^{\emo} = \tuple{\Pb, \Ub, \parop, \iniop, \satop}$ as follows:
$\Pb$ is the collection of all conditions~$(\vec{F}, \Ccal, R)$ where $R$ is a primitive recursive
tournament functional and
\begin{itemize}
	\item[(a)] $\Ccal$ is a non-empty $\Pi^{0,D}_1$ $k$-cover class of $[t, +\infty)$ 
	for some $k, t \in \omega$
	\item[(b)] $F_\nu \cup \{x\}$ is $R^D$-transitive for every $Z_0 \oplus \dots \oplus Z_{k-1} \in \Ccal$,
	every $x \in Z_\nu$ and each $\nu < k$
	\item[(c)] $Z_\nu$ is included in a minimal $R^D$-interval of $F_\nu$
	for every $Z_0 \oplus \dots \oplus Z_{k-1} \in \Ccal$ and each~$\nu < k$.
\end{itemize}
Once again, $\Ccal$ is actually a $\Pi^{0,D}_1$ formula denoting a non-empty $\Pi^{0,D}_1$ class.
A condition $d = (\vec{E}, \Dcal, T)$ \emph{extends} 
$c = (\vec{F}, \Ccal, R)$
(written $d \leq c$) if $R = T$ and
there exists a function $f : parts(\Dcal) \to parts(\Ccal)$ such that the properties (i) and (ii)
mentioned above hold.

Given some condition $c = (\vec{F}, \Ccal, R)$,
$\parop(c) = \{\tuple{c, \nu} : \nu \in parts(\Ccal)\}$.
Define $\Ub$ as $\bigcup_{c \in \Pb} \parop(c)$, that is, the set of all pairs $\langle (\vec{F}, \Ccal, R), \nu \rangle$
where $\nu \in parts(\Ccal)$. The operator $\iniop(n)$ returns the condition $(\{\emptyset\}, \{\omega\}, R_n)$
where $R_n$ is the $n$th primitive recursive tournament functional.
Last, define $\satop(\tuple{c, \nu})$ to be the collection of all $R^D$-transitive subtournaments
satisfying the Mathias precondition $(F_\nu, X_\nu)$ where $X_\nu$ is \emph{non-empty} for some $\bigoplus_\nu X_\nu \in \Ccal$.
The additional non-emptiness requirement of $X_\nu$ in the definition of the $\satop$ operator
enables us to ``disable'' some part by setting $X_\nu = \emptyset$. Without this requirement,
the property (i) of a module support would not be satisfied. Moreover, since every cover class has an
acceptable part, there is always one part~$\nu$ in $\Ccal$ such that $\satop(\tuple{c,\nu}) \neq \emptyset$.

\begin{lemma}
The tuple $\Sb^{\emo}$ is a module support.
\end{lemma}
\begin{proof}
We must check that if~$d \leq_\Pb c$ for some~$c, d \in \Pb$, then there is a function~$g : \parop(d) \to \parop(c)$
such that $\satop(\nu) \subseteq \satop(g(\nu))$ for each~$\nu \in \parop(d)$.
Let $d = (\vec{E}, \Dcal, R)$ and $c = (\vec{F}, \Ccal, R)$ be such that $d \leq_\Pb c$.
By definition, there is a function $f : parts(\Dcal) \to parts(\Ccal)$ satisfying
the properties (i-ii). Let $g : \parop(d) \to \parop(c)$ be defined by $g(\tuple{d, \nu}) = \tuple{c, f(\nu)}$.
We claim that $g$ is a refinement function witnessing $d \leq_\Pb c$.
Let $G$ be any set in $\satop(\tuple{d, \nu})$. We will prove that $G \in \satop(\tuple{c, f(\nu)})$.
The set $G$ is an $R^D$-transitive subtournament
satisfying the Mathias condition $(E_\nu, X_\nu)$ where $X_\nu \neq \emptyset$ for some $\bigoplus_\nu X_\nu \in \Dcal$.
By (ii), since $X_\nu$ is non-empty, there is some $\bigoplus_\mu Y_\mu \in \Ccal$ 
such that $E_\nu \setminus F_{f(\nu)} \subseteq Y_{f(\nu)}$
and $X_\nu \subseteq Y_{f(\nu)}$. It suffices to show that $(F_\nu, X_\nu)$
Mathias extends $(F_{f(\nu)}, Y_{f(\nu)})$ to deduce that $G$ satisfies the Mathias condition $(F_{f(\nu)}, Y_{f(\nu)})$
and finish the proof. By (i), $F_{f(\nu)} \subseteq E_\nu$.
Since $E_\nu \setminus F_{f(\nu)} \subseteq Y_{f(\nu)}$ and $X_\nu \subseteq Y_{f(\nu)}$, we are done.
\end{proof}

We next define an implementation of the module $\Mb^{\emo} = \langle \Sb^{\emo}, \boxop, \unboxop, \progop \rangle$ as follows.
Given some condition $c = (\vec{F}, \Ccal, R)$, some $\nu \in parts(\Ccal)$ and some
$\Sigma^0_1$ formula $\varphi(D, G)$, $\unboxop(\tuple{c,\nu}, \varphi)$ returns the $\Sigma^0_1$ formula $\psi(D)$ which holds
if there is a finite sequence of finite sets $E^0, \dots, E^{r-1}$ and a finite sequence of functions $g^0, \dots, g^{r-1}$ such that
\begin{itemize}
	\item[(1)] for every $X_0 \oplus \dots \oplus X_{k-1} \in \Ccal$, there is some $i < r$ such that $E^i \subseteq X_\nu$
	\item[(2)] for every $i < r$ and every 2-partition $\tuple{E_0, E_1}$ of~$E^i$,
	$g^i(\tuple{E_0, E_1})$ is a finite $R^D$-transitive subset of some $E_j$ such that $\varphi(D, F_\nu \cup g^i(\tuple{E_0, E_1}))$ holds.
\end{itemize}

If the answer to $\psi(D)$ is $\tuple{\no}$, $\unboxop(\tuple{c,\nu}, \tuple{\no})$ returns
the tuple~$\tuple{d, f, b}$ where $d = (\vec{E}, \Dcal, R)$ is a condition such that $d \leq_f c$
and defined as follows. For every part $\mu \neq \nu$ of $c$, create
a part~$\mu$ in $d$ and set $E_\mu = F_\mu$. Furthermore, fork the part~$\nu$ into two parts $\nu_0$
and $\nu_1$ in $d$ and set $E_{\nu_i} = F_\nu$ for each~$i < 2$. 
Define $\Dcal$ to be the $\Pi^{0,D}_1$ class of all covers $\bigoplus_\mu Y_\mu$
$f$-refining some cover $\bigoplus_\nu X_\nu \in \Ccal$ and such that for every $i < 2$
and every finite $R^D$-transitive set $E \subseteq Y_{\nu_i}$, $\varphi(D, F_\nu \cup E)$ does not hold.
Moreover, $b : \parop(c) \to \ansop[D,G]$ is the constant function $\tuple{\no}$.

Suppose now that the answer to $\psi(D)$ is $a = \tuple{\yes, r, E^0, \dots, E^{r-1}, f^0, \dots, f^{r-1}, a'}$
where $a'$ is a function which on every $i < r$ and every 2-partition $\tuple{E_0, E_1} = i$,
returns an answer to $\varphi(D, F_\nu \cup g^i(\tuple{E_0, E_1}))$.
The function $\unboxop(\tuple{c,\nu}, a)$ returns the tuple~$\tuple{d, f, b}$ where $d$ is a condition such that $d \leq_f c$
and whose definition has been described in subsection~\ref{subsect:emo-new-condition}.
The function $b : \parop(d) \to \ansop[D, G]$ returns on every part~$\mu = \tuple{i, E_0, E_1}$ 
the tuple $\tuple{\yes, a'(i, E_0, E_1)}$.

Last, given some condition~$c = (\vec{F}, \Ccal, R)$ and some $\nu \in parts(\Ccal)$, 
$\progop(\tuple{c,\nu})$ is the query $\varphi(D, G) = (\exists n)[n \in G \wedge n > max(F_\nu)]$.
Note that we cannot force $\neg \varphi(D, G)$ on every part~$\tuple{c,\nu}$,
since every cover class has an acceptable part. Applying the query lemma infinitely many times
on the progress operator ensures that if we take any path through the infinite tree of the acceptable parts,
the resulting $R^D$-transitive subtournament will be infinite.

\begin{lemma}
The tuple $\Mb^{\emo}$ is a module.
\end{lemma}
\begin{proof}
We need to ensure that given some part~$\nu$ of some condition~$c = (\vec{F}, \Ccal, R)$
and some answer~$a$ to a $\Sigma^0_1$ formula $\psi(D) = \boxop(\tuple{c,\nu}, \varphi)$ where $\varphi(D, G)$ is a $\Sigma^0_1$ formula, 
$\unboxop(\tuple{c,\nu}, a)$ outputs a tuple $\tuple{d, f, b}$ where $d = (\vec{E}, \Dcal, R)$ is a condition such that 
$d \leq_f c$ where $f$ forks only part $\nu$ of~$c$,
and for every part~$\mu$ of~$d$ such that~$f(\tuple{d,\mu}) = \tuple{c,\nu}$, 
and every set~$G \in \satop(\tuple{d,\mu})$, $b(\tuple{d,\mu})$ is an answer to~$\varphi(D, G)$.

Suppose that $a = \tuple{\no}$.
By definition of $\satop(\tuple{d,\mu})$ and by construction of $d$, $G$ is $R^D$-transitive and satisfies the Mathias condition
$(E_{\nu_i}, Y_{\nu_i})$ for some $i < 2$ and some cover $\bigoplus_\mu Y_\mu \in \Dcal$.
In particular, $E_{\nu_i} = F_\nu$ and $Y_{\nu_i}$ is such that for every finite $R^D$-transitive set $E \subseteq Y_{\nu_i}$,
$\varphi(D, F_\nu \cup E)$ does not hold. In particular, taking $E = G \setminus F_\nu$, $\varphi(D, G)$ does not hold.

Suppose now that $a = \tuple{\yes, r, E^0, \dots, E^{r-1}, f^0, \dots, f^{r-1}, a'}$
where $a'$ is a function which on every $i < r$ and every 2-partition $\tuple{E_0, E_1} = i$,
returns an answer to $\varphi(D, F_\nu \cup g^i(\tuple{E_0, E_1}))$.
By definition of $\satop(\tuple{d,\mu})$ and by construction of $d$, $G$ is $R^D$-transitive and satisfies the Mathias condition
$(E_\mu, Y_\mu)$ for some cover $\bigoplus_\mu Y_\mu \in \Dcal$, where $\mu = \tuple{i, E_0, E_1}$.
By construction of $d$, $E_\mu = F_\nu \cup g^i(\tuple{E_0,E_1})$, and by definition of $g^i$,
$\varphi(D, F_\nu \cup g^i(\tuple{E_0,E_1}))$.  Since $G$ satisfies $(E_\mu, Y_\mu)$,
$F_\nu \cup g^i(\tuple{E_0,E_1}) \subseteq G$ and $G \setminus E_\mu \subseteq Y_\mu$. 
Therefore $\varphi(D, G)$ holds.
\end{proof}


\subsection{The separation}

We have defined a module $\Mb^{\emo}$ for the Erd\H{o}s-Moser theorem.
In this subsection, we explain how we create an $\omega$-model of $\emo$ which is not a model of $\amt$
from the infinite sequence of stage trees constructed in subsection~\ref{subsect:framework-construction}.
Given the uniform enumeration $R_0, R_1, \dots$ of all primitive recursive tournament functionals,
we shall define an infinite sequence of sets $X_0, X_1, \dots$ together with a $\Delta^0_2$ function $f$ such that
for every $s$,
\begin{itemize}
	\item[1.] $X_{s+1}$ is an infinite, transitive subtournament of $R^{X_0 \oplus \dots \oplus X_s}$
	\item[2.] $f$ dominates every $X_0 \oplus \dots \oplus X_s$-computable function.
\end{itemize}
By 2, any $\Delta^0_2$ approximation $\tilde{f}$ of the function $f$ 
is a computable instance of the escape property with no solution in $\Mcal_{\vec{X}}$,
that is, such that no function in $\Mcal_{\vec{X}}$ escapes~$f$.
By the computable equivalence between the escape property and the atomic model theorem (see subsection~\ref{subsect:dominating-amt}), 
$\Mcal_{\vec{X}} \not \models \amt$.
By Lemma~\ref{lem:em-uniform-model}, $\Mcal_{\vec{X}} \models \emo \wedge \coh$.

Start with $X_0 = \emptyset$ and the $\Delta^0_2$ enumeration $T_0 \geq T_1 \geq \dots$
of stage trees constructed in subsection~\ref{subsect:framework-construction},
and let $c_0 \geq c_1 \geq \dots$ be the sequence of their roots.
The set $U$ of their parts form an infinite, finitely branching tree,
whose structure is given by the refinement functions.
Moreover, by the construction of the sequence $T_0, T_1, \dots$, for every $s$, 
there is some part~$\nu$ in $c_{s+1}$ refining some part~$\mu$
in $c_s$ and which forces $\progop(\mu)$. Call such a part~$\nu$ a \emph{progressing part}. We may also
consider that every part of~$c_0$ is a progressing part, for the sake of uniformity.
By the implementation of $\progop$, if $\nu$ is a progressing part which refines some part~$\mu$,
$\mu$ is also a progressing part. Therefore, the set the progressing parts forms an infinite subtree $U_1$ of $U$.

Let $\nu_0, \nu_1, \dots$ be an infinite path through $U_1$.
Notice that $\satop(\nu_s) \neq \emptyset$.
Indeed, if $\satop(\nu_s) = \emptyset$, then the part~$\nu_s$ is empty in $\Ccal_s$, where $c_s = (\vec{E}_s, \Ccal_s)$,
and therefore we cannot find some progressing part~$\nu_{s+1}$ refining $\nu_s$.
Therefore, the set $\bigcap_s \satop(\nu_s)$ is non-empty. Let $X_1 \in \bigcap_s \satop(\nu_s)$.
By definition of $\satop(\nu_s)$, $X_1$ is a transitive subtournament of $R^{X_0}$.
By definition of $\progop$, for every $s$ and every set $G \in \satop(\nu_s)$, there is some~$n \in G$ such that $n > s$.
Therefore, the set $X_1$ is infinite, so the property 1 is satisfied.

Repeat the procedure with the sequence of stage trees $T_1^{[\nu_1]} \geq T_2^{[\nu_2]} \geq \dots$ and so on.
We obtain an infinite sequence of sets $X_0, X_1, \dots$ satisfying the property 1.
Let $f$ be the $\Delta^0_2$ function which on input $x$, returns $max(U_x)+1$ where
$U_x$ is the finite set stated in the domination lemma (Lemma~\ref{lem:domination-lemma}) for stage trees of depth $x$.
Fix some Turing index $e$ such that $\Phi^{X_0 \oplus \dots \oplus X_i}_e$ is total.
By the domination lemma, for every $x \geq max(e,i)$, $\Phi^{X_0 \oplus \dots \oplus X_i}_e(x) \in U_x < f(x)$.
Therefore the function~$f$ dominates every $X_0 \oplus \dots \oplus X_i$-computable function.
This finishes the proof of Theorem~\ref{thm:emo-not-amt}.

