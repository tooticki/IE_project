
Reverse mathematics is a mathematical program whose goal is
to classify theorems in terms of their provability strength.
It uses the framework of subsystems of second-order arithmetic,
with the base theory $\rca$, standing for Recursive Comprehension Axiom.
$\rca$ is composed of the basic first-order Peano axioms,
together with $\Delta^0_1$-comprehension and $\Sigma^0_1$-induction schemes.
$\rca$ is usually thought of as capturing \emph{computational mathematics}.
This program led to two important observations:
First, most ``ordinary'' (i.e.\ non set-theoreric) theorems require only very weak set existence axioms.
Second, many of those theorems are actually \emph{equivalent}
to one of five main subsystems over $\rca$, known as the ``Big Five''~\cite{Montalban2011Open}.

However, Ramsey theory is known to provide a large class of theorems escaping this phenomenon.
Indeed, consequences of Ramsey's theorem for pairs ($\rt^2_2$) usually belong to their own subsystem.
Therefore, they received a lot of attention from the reverse mathematics 
community~\cite{Cholak2001strength,Hirschfeldt2008strength,Hirschfeldt2007Combinatorial,Seetapun1995strength}.
This article focuses on Ramseyan principles below the arithmetic comprehension axiom~($\aca$). 
See Soare~\cite{Soare2016Turing} for a general introduction to computability theory,
and Hirschfeldt~\cite{Hirschfeldt2015Slicing} for a good introduction
to the reverse mathematics below~$\aca$.

\subsection{Cohesiveness}

Cohesiveness is a statement playing a central role in the analysis of Ramsey's theorem for pairs~\cite{Cholak2001strength}.
It can be seen as a sequential version of Ramsey's theorem for singletons and admits
characterizations in terms of degrees whose jump computes a path through a $\Pi^{0,\emptyset'}_1$ class~\cite{Jockusch1993cohesive}.
The decomposition of $\rt^2_2$ in terms of $\coh$ and stable Ramsey's theorem for pairs ($\srt^2_2$)
has been reused in the analysis of many consequences of Ramsey's theorem~\cite{Hirschfeldt2007Combinatorial}.
The link between cohesiveness and~$\srt^2_2$ is still an active research subject
\cite{Chong2014metamathematics,DzhafarovStrong,Hirschfeldt2016notions,Patey2016weakness}.

\begin{definition}[Cohesiveness]
An infinite set $C$ is $\vec{R}$-cohesive for a sequence of sets $\vec{R} = R_0, R_1, \dots$
if for each $i \in \omega$, $C \subseteq^{*} R_i$ or $C \subseteq^{*} \overline{R_i}$.
A set $C$ is \emph{p-cohesive} if it is $\vec{R}$-cohesive where
$\vec{R}$ is an enumeration of all primitive recursive sets.
$\coh$ is the statement ``Every uniform sequence of sets $\vec{R}$
has an $\vec{R}$-cohesive set.''
\end{definition}

Jockusch and Stephan~\cite{Jockusch1993cohesive} studied the degrees of unsolvability of cohesiveness
and proved that~$\coh$ admits a universal instance whose solutions
are the p-cohesive sets. They characterized their degrees as those whose
jump is PA relative to~$\emptyset'$.
The author extended this analysis to every computable instance of~$\coh$ and studied their degrees
of unsolvability~\cite{Patey2016weakness}.
Cholak, Jockush and Slaman~\cite{Cholak2001strength} proved that $\rt^2_2$ is computably equivalent to $\srt^2_2+\coh$.
Mileti~\cite{Mileti2004Partition} and Jockusch and Lempp [unpublished]
formalized this equivalence over $\rca$.
Hirschfeldt, Jockusch, Kjos-Hanssen, Lempp and Slaman~\cite{Hirschfeldt2008strength} proved that~$\coh$ contains a model
with no diagonally non-computable function, thus $\coh$ does not imply~$\srt^2_2$ over~$\rca$.
Cooper~\cite{Cooper1973Minimal} proved that every degree above~$\mathbf{0'}$ is the jump of a minimal degree.
Therefore there exists a p-cohesive set of minimal degree.

\subsection{The Erd\H{o}s-Moser theorem}

The Erd\H{o}s-Moser theorem is a principle coming from graph theory.
It provides together with the ascending descending principle~($\ads$) an alternative proof of
Ramsey's theorem for pairs ($\rt^2_2$). Indeed, every coloring~$f : [\omega]^2 \to 2$
can be seen as a tournament~$R$ such that~$R(x,y)$ holds if~$x < y$ and~$f(x,y) = 1$, or~$x > y$ and~$f(y, x) = 0$.
Every infinite transitive subtournament induces a linear order whose infinite ascending or descending
sequences are homogeneous for~$f$.

\begin{definition}[Erd\H{o}s-Moser theorem]
A tournament $T$ on a domain $D \subseteq \omega$ is an irreflexive binary relation on~$D$ such that for all $x,y \in D$ with $x \not= y$, exactly one of $T(x,y)$ or $T(y,x)$ holds. A tournament $T$ is \emph{transitive} if the corresponding relation~$T$ is transitive in the usual sense. A tournament $T$ is \emph{stable} if $(\forall x \in D)[(\forall^{\infty} s) T(x,s) \vee (\forall^{\infty} s) T(s, x)]$.
$\emo$ is the statement ``Every infinite tournament $T$ has an infinite transitive subtournament.''
$\semo$ is the restriction of $\emo$ to stable tournaments.
\end{definition}

Bovykin and Weiermann~\cite{Bovykin2005strength} introduced the Erd\H{o}s-Moser theorem in reverse mathematics
and proved that $\emo$ together with the chain-antichain principle ($\cac$) is equivalent to $\rt^2_2$ over $\rca$. 
This was refined into an equivalence between $\emo+\ads$ and $\rt^2_2$ by Montalb\'an (see~\cite{Bovykin2005strength}),
and the equivalence still holds between the stable versions of the statements.
Lerman, Solomon and Towsner~\cite{Lerman2013Separating} proved that~$\emo$ 
is strictly weaker than~$\rt^2_2$ by constructing an $\omega$-model
of~$\emo$ which is not a model of the stable ascending descending sequence~($\sads$). $\sads$
is the restriction of~$\ads$ to linear orders of order type~$\omega+\omega^{*}$~\cite{Hirschfeldt2007Combinatorial}. 
The author noticed in~\cite{Patey2013note} that their construction can be adapted to obtain a separation of~$\emo$ from the stable thin set theorem for pairs~($\sts(2)$).
Wang strengthened this separation by constructing in~\cite{Wang2014Definability} a standard model of many theorems,
including~$\emo$, $\coh$ and weak K\"onig's lemma ($\wkl$) which is neither a model of~$\sts(2)$
nor a model of~$\sads$. The author later refined in~\cite{Patey2015Iterative,Patey2016weakness} the forcing technique of Lerman, Solomon and Towsner and showed that it is strong enough to obtain the same separations as Wang.

On the lower bounds side, Lerman, Solomon and Towsner~\cite{Lerman2013Separating}\ showed that~$\emo$ 
implies the omitting partial types principle ($\opt$)
over~$\rca + \bst$ and Kreuzer proved in~\cite{Kreuzer2012Primitive} that $\semo$ implies
$\bst$ over $\rca$. The statement $\opt$ can be thought of as a stating 
for every set~$X$ the existence of a set hyperimmune relative to~$X$.
Finally, the author proved in~\cite{Patey2015Somewhere} that $\rca \vdash \emo \imp [\sts(2) \vee \coh]$.
In particular, every model of~$\emo$ which is not a model of~$\sts(2)$ is also a model of $\coh$.
This fact will be reused in this paper since $\sts(2)$ implies the atomic model theorem over~$\rca$~\cite{Patey2015Somewhere}.

\subsection{Domination and the atomic model theorem}\label{subsect:dominating-amt}

The atomic model theorem is a statement coming from model theory. It has been
introduced by Hirschfeldt, Shore and Slaman~\cite{Hirschfeldt2009atomic} in the settings of reverse mathematics.

\begin{definition}[Atomic model theorem]
A formula $\varphi(x_1, \dots, x_n)$ of $T$ is an \emph{atom} of a theory $T$ if for each formula $\psi(x_1, \dots, x_n)$
  we have $T \vdash \varphi \imp \psi$ or $T \vdash \varphi \imp \neg \psi$ but not both.
  A theory $T$ is \emph{atomic} if, for every formula $\psi(x_1, \dots, x_n)$ consistent with $T$,
  there is an atom $\varphi(x_1, \dots, x_n)$ of $T$ extending it, i.e., one such that $T \vdash \varphi \imp \psi$.
  A model $\Acal$ of $T$ is \emph{atomic} if every $n$-tuple from $\Acal$ satisfies an atom of $T$. 
$\amt$ is the statement ``Every complete atomic theory has an atomic model''.
\end{definition}

This strength of the atomic model theorem received a lot of attention from the reverse mathematics community 
and was subject to many refinements.
On the upper bound side, Hirschfeldt, Shore and Slaman~\cite{Hirschfeldt2009atomic} 
proved that~$\amt$ is a consequence of~$\sads$ over~$\rca$. 
The author~\cite{Patey2015Somewhere} proved that the stable thin set theorem for pairs ($\sts(2)$) implies~$\amt$ over~$\rca$.

On the lower bound side, Hirschfeldt, Shore and Slaman~\cite{Hirschfeldt2009atomic} proved that~$\amt$
implies the omitting partial type theorem ($\opt$) over~$\rca$. 
Hirschfeldt and Greenberg, and independently Day, Dzhafarov and Miller, 
strengthened this result by proving that $\amt$ implies the finite intersection property ($\fip$) over~$\rca$ (see~\cite{Hirschfeldt2015Slicing}).
The principle $\fip$ was first introduced by Dzhafarov and Mummert~\cite{Dzhafarov2013strength}. 
Later, Downey, Diamondstone, Greenberg and Turetsky~\cite{Downey2012Finite} and 
Cholak, Downey and Igusa~\cite{Cholak2015Any} proved that~$\fip$ is equivalent to the principle asserting, for every set $X$,
the existence of a 1-generic relative to~$X$. In particular, every model of $\amt$ contains 1-generic reals.

The computable analysis of the atomic model theorem revealed the genericity flavor of the statement.
More precisely, the atomic model theorem admits a pure computability-theoretic characterization
in terms of hyperimmunity relative to a fixed $\Delta^0_2$ function. 

\begin{definition}[Escape property]
For every $\Delta^0_2$ function $f$, there exists a function~$g$ such that $f(x) < g(x)$ for infinitely many $x$.
\end{definition}

The escape property is a statement in between hyperimmunity relative to~$\emptyset'$ and hyperimmunity.
The atomic model theorem is computably equivalent to the escape property, that is, for every complete atomic theory $T$,
there is a $\Delta^{0,T}_2$ function $f$ such that for every function $g$ satisfying the escape property for~$f$,
$T \oplus g$ computes an atomic model of~$T$. Conversely, for every $\Delta^0_2$ approximation $\tilde{f}$
of a function $f$, there is a $\tilde{f}$-computable complete atomic theory such that for every atomic model $\Mcal$,
$\tilde{f} \oplus \Mcal$ computes a function satisfying the escape property for~$f$.
In particular, the $\omega$-models satisfying $\amt$
are exactly the ones satisfying the escape property. However the formalization of this equivalence requires
more than the $\Sigma^0_1$ induction scheme.
It was proven to hold over~$\rca + \ist$ but not~$\rca + \bst$~\cite{Hirschfeldt2009atomic,Conidis2008Classifying}, 
where~$\ist$ and~$\bst$ are respectively the~$\Sigma^0_2$ induction scheme and
the $\Sigma^0_2$ bounding scheme.

Hirschfeldt, Shore and Slaman~\cite{Hirschfeldt2009atomic} asked the following question.

\begin{question}
Does the cohesiveness principle imply the atomic model theorem over~$\rca$?
\end{question}

Note that $\amt$ is not computably reducible to~$\coh$, since there exists a cohesive set of minimal degree~\cite{Cooper1973Minimal},
and a computable atomic theory whose computable atomic models bound 1-generic reals~\cite{Hirschfeldt2015Slicing}, 
but no minimal degree bounds a 1-generic real~\cite{Yu2006Lowness}.

In this paper, we answer this question negatively.
We shall take advantage of the characterization of $\amt$ by the escape property to create an $\omega$-model $\Mcal$ of $\emo$, $\wkl$
and $\coh$ simultaneously, together with a $\Delta^0_2$ function $f$ dominating every function in~$\Mcal$.
Therefore, any $\Delta^0_2$ approximation $\tilde{f}$ of the function $f$ 
is a computable instance of the escape property belonging to $\Mcal$, but with no solution in $\Mcal$.
The function $f$ witnesses in particular that~$\Mcal \not \models \amt$.
Our main theorem is the following.

\begin{theorem}[Main theorem]\label{thm:main-theorem}
$\coh \wedge \emo \wedge \wkl$ does not imply $\amt$ over~$\rca$.
\end{theorem}

The proof techniques used to prove the main theorem will be introduced
progressively by considering first computable non-reducibility, and then generalizing
the diagonalization to Turing ideals by using an effective iterative forcing.

\subsection{Definitions and notation}

\emph{String, sequence}.
Fix an integer $k \in \omega$.
A \emph{string} (over $k$) is an ordered tuple of integers $a_0, \dots, a_{n-1}$
(such that $a_i < k$ for every $i < n$). The empty string is written $\varepsilon$. A \emph{sequence}  (over $k$)
is an infinite listing of integers $a_0, a_1, \dots$ (such that $a_i < k$ for every $i \in \omega$).
Given $s \in \omega$,
$k^s$ is the set of strings of length $s$ over~$k$ and
$k^{<s}$ is the set of strings of length $<s$ over~$k$. Similarly,
$k^{<\omega}$ is the set of finite strings over~$k$
and $k^{\omega}$ is the set of sequences (i.e. infinite strings)
over~$k$. 
Given a string $\sigma \in k^{<\omega}$, we denote by $|\sigma|$ its length.
Given two strings $\sigma, \tau \in k^{<\omega}$, $\sigma$ is a \emph{prefix}
of $\tau$ (written $\sigma \preceq \tau$) if there exists a string $\rho \in k^{<\omega}$
such that $\sigma \rho = \tau$. Given a sequence $X$, we write $\sigma \prec X$ if
$\sigma = X \uh n$ for some $n \in \omega$, where $X \uh n$ denotes the restriction of $X$ to its first $n$ elements.
A \emph{binary string} is a \emph{string} over~$2$.
A \emph{real} is a sequence over~$2$.
We may identify a real with a set of integers by considering that the real is its characteristic function.

\emph{Tree, path}.
A tree $T \subseteq k^{<\omega}$ is a set downward-closed under the prefix relation.
A \emph{binary} tree is a tree $T \subseteq 2^{<\omega}$.
A sequence $P \in k^\omega$ is a \emph{path} though~$T$ if for every $\sigma \prec P$,
$\sigma \in T$. A string $\sigma \in k^{<\omega}$ is a \emph{stem} of a tree $T$
if every $\tau \in T$ is comparable with~$\sigma$.
Given a tree $T$ and a string $\sigma \in T$,
we denote by $T^{[\sigma]}$ the subtree $\{\tau \in T : \tau \preceq \sigma \vee \tau \succeq \sigma\}$.

\emph{Sets, partitions}.
Given two sets $A$ and $B$, we denote by $A < B$ the formula
$(\forall x \in A)(\forall y \in B)[x < y]$
and by $A \subseteq^{*} B$ the formula $(\forall^{\infty} x \in A)[x \in B]$,
meaning that $A$ is included in $B$ \emph{up to finitely many elements}.
Given a set~$X$ and some integer~$k$, a~\emph{$k$-cover of~$X$}
is a $k$-uple $A_0, \dots, A_{k-1}$ such that~$A_0 \cup \dots \cup A_{k-1} = X$.
We may simply say~\emph{$k$-cover} when the set~$X$ is unambiguous. 
A \emph{$k$-partition} is a $k$-cover whose sets are pairwise disjoint.
A \emph{Mathias condition} is a pair $(F, X)$
where $F$ is a finite set, $X$ is an infinite set
and $F < X$.
A condition $(F_1, X_1)$ \emph{extends } $(F, X)$ (written $(F_1, X_1) \leq (F, X)$)
if $F \subseteq F_1$, $X_1 \subseteq X$ and $F_1 \setminus F \subset X$.
A set $G$ \emph{satisfies} a Mathias condition $(F, X)$
if $F \subset G$ and $G \setminus F \subseteq X$.
We refer the reader to Chapter 2 in Hirschfeldt~\cite{Hirschfeldt2015Slicing} for a gentle introduction
to effective forcing.



