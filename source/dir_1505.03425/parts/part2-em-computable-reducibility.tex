
We now strengthen the analysis of the previous section by
proving that the atomic model theorem is not computably reducible to the Erd\H{o}s-Moser theorem.
Theorem~\ref{thm:amt-comp-reduc-coh} is an immediate consequence of this result since
$[\amt \vee \coh] \leq_c \emo$ (see~\cite{Patey2015Somewhere}).
After this section, we will be ready to iterate the construction
in order to build an $\omega$-model of~$\emo \wedge \coh$ which is not a model of~$\amt$.

\begin{theorem}\label{thm:amt-comp-reduc-em-coh}
$\amt \not \leq_c \emo$
\end{theorem}

Before proving Theorem~\ref{thm:amt-comp-reduc-em-coh},
we start with an analysis of the combinatorics of the Erd\H{o}s-Moser theorem.
Just as we did for cohesiveness, we will show how to build solutions to $\emo$ through $\Delta^0_2$ constructions, 
postponing the $\Pi^0_2$ choices to the end.

\subsection{The combinatorics of the Erd\H{o}s-Moser theorem}\label{subsect:combi-em}

The standard way of building an infinite object by forcing consists of defining an increasing
sequence of finite approximations, and taking the union of them. Unlike
$\coh$ where every finite set can be extended to an infinite cohesive set,
some finite transitive subtournaments may not be extensible to an infinite one.
We therefore need to maintain some extra properties which will guarantee that
the finite approximations are extendible.
The nature of these properties constitue the core of the combinatorics of $\emo$.

Lerman, Solomon and Towsner~\cite{Lerman2013Separating} proceeded to an analysis
of the Erd\H{o}s-Moser theorem.
They showed in particular that it suffices to ensure that the finite transitive subtournament~$F$
has infinitely many \emph{one-point extensions}, that is, infinitely many elements~$x$ such that
$F \cup \{x\}$ is transitive, to extend~$F$ to an infinite transitive subtournament (see~\cite[Lemma 3.4]{Lerman2013Separating}).
This property is sufficient to add elements one by one to the finite approximation.
However, when adding elements by block, we shall maintain a stronger invariant. We will require that
the reservoir is included in a minimal interval of the finite approximation~$F$.
In this section, we reintroduce the terminology of Lerman, Solomon and Towsner~\cite{Lerman2013Separating}
and give a presentation of the combinatorics of the Erd\H{o}s-Moser theorem
motivated by its computational analysis.

\begin{definition}[Minimal interval]
Let $R$ be an infinite tournament and $a, b \in R$
be such that $R(a,b)$ holds. The \emph{interval} $(a,b)$ is the
set of all $x \in R$ such that $R(a,x)$ and $R(x,b)$ hold.
Let $F \subseteq R$ be a finite transitive subtournament of $R$.
For $a, b \in F$ such that $R(a,b)$ holds, we say that $(a,b)$
is a \emph{minimal interval of $F$} if there is no $c \in F \cap (a,b)$,
i.e., no $c \in F$ such that $R(a,c)$ and $R(c,b)$ both hold.
\end{definition}

Fix a computable tournament~$R$, and consider a pair $(F, X)$ where
\begin{itemize}
	\item[(i)] $F$ is a finite $R$-transitive set representing the \emph{finite approximation}
	of the infinite $R$-transitive subtournament we want to construct
	\item[(ii)] $X$ is an infinite set disjoint from $F$, included in a minimal interval of~$F$
	and such that $F \cup \{x\}$ is $R$-transitive for every $x \in X$. In other words,
	$X$ is an infinite set of one-point extensions.
	Such a set $X$ represents the \emph{reservoir}, that is, a set of candidate
	elements we may add to~$F$ later on.
\end{itemize}

\begin{figure}[htbp]
\begin{center}
\begin{tikzpicture}[x=1.5cm, y=1.2cm, 
		node/.style={circle, draw, fill=lightblue, inner sep=0pt, minimum size=1.5em}, 
		arrow/.style={black,thick,->,-triangle 45},
		good/.style={fill=white},
		hiddenarrow/.style={black,thick,->, opacity=0.2},
		hidden/.style={opacity=0.2}
]

  \node[node] (a) at (1, 2) {$a$};
	\node[node] (b) at (1, 1) {$b$};
	\node[node] (c) at (1, 0) {$c$};
	\node[node,good,hidden] (d) at (3, -0.5) {$d$};
	\node[node,good] (e) at (4, 0.5) {$e$};
	\node[node,good,hidden] (f) at (5, 2.5) {$f$};

	\node at (7, 2.5) {$(-\infty, a)$};
	\node at (7, 1.5) {$(a, b)$};
	\node at (7, 0.5) {$(b, c)$};
	\node at (7, -0.5) {$(c, +\infty)$};

	% Finite transitive set F
	\draw[arrow] (a) -- (b);
	\draw[arrow] (b) -- (c);
	\draw[arrow] (a)  .. controls (0.4,1) .. (c);

	% D
	\draw[hiddenarrow] (a) -- (d);
	\draw[hiddenarrow] (c) -- (d);
	\draw[hiddenarrow] (b) -- (d);

	% E
	\draw[arrow] (a) -- (e);
	\draw[arrow] (e) -- (c);
	\draw[arrow] (b) -- (e);

	% F
	\draw[hiddenarrow] (f) -- (a);
	\draw[hiddenarrow] (f) -- (c);
	\draw[hiddenarrow] (f) -- (b);

	\draw[very thick, loosely dotted] (4.5,0.5) -- (5.5,0.5);

	\draw[loosely dashed] (2, 2) -- (6, 2);
	\draw[loosely dashed] (2, 1) -- (6, 1);
	\draw[loosely dashed] (2, 0) -- (6, 0);
\end{tikzpicture}
\end{center}
\caption{In this figure, $F = \{a, b, c\}$ is a transitive set,
$X = \{d, e, f, \dots \}$ a set of one-point extensions, $(b, c) = \{e, \dots \}$ a minimal interval of~$F$ 
and $(F, X \cap (b, c))$ an EM condition. The elements~$d$ and~$f$ are not part of the minimal interval~$(b, c)$.} 
\end{figure}


The infinite set~$X$ ensures extensibility of the finite set~$F$ into an infinite $R$-transitive
subtournament. Indeed, by applying the Erd\H{o}s-Moser theorem to $R$ over the domain $X$, there exists an infinite $R$-transitive
subtournament $H \subseteq X$. One easily checks that $F \cup H$ is $R$-transitive.
The pair $(F, X)$ is called an Erd\H{o}s-Moser condition in~\cite{Patey2015Degrees}.
A set~$G$ \emph{satisfies} an EM condition~$(F, X)$ if it is $R$-transitive and satisfies the Mathias condition~$(F, X)$.
In order to simplify notation, given a tournament $R$ and two sets~$E$ and~$F$,
we denote by $E \to_R F$ the formula $(\forall x \in E)(\forall y \in F) R(x,y)$.

Suppose now that we want to add a finite number of elements of~$X$ into $F$ to obtain
a finite $T$-transitive set $\tilde{F} \supseteq F$,
and find an infinite subset $\tilde{X} \subseteq X$ such that $(\tilde{F}, \tilde{X})$
has the above mentioned properties. We can do this in a few steps:

\begin{itemize}
	\item[1.] Choose a finite (not necessarily $R$-transitive) set $E \subset X$.
	\item[2.] Any element $x \in X \setminus E$ induces a 2-partition $\tuple{E_0, E_1}$ of $E$
	by setting $E_0 = \{y \in E : R(y, x) \}$ and $E_1 = \{y \in E : R(x, y)\}$.
	Consider the coloring $f$ which associates to any element of $X \setminus E$ the corresponding 2-partition $\tuple{E_0, E_1}$ of $E$.
	\item[3.]
	As~$E$ is finite, there exists finitely many 2-partitions of~$E$, so $f$ colors each element of $X \setminus E$ into
	finitely many colors. By Ramsey's theorem for singletons applied to~$f$, there exists a 2-partition $\tuple{E_0, E_1}$ of $E$
	together with an infinite subset $\tilde{X} \subseteq X \setminus E$ such that for every $x \in \tilde{X}$, $f(x) = \tuple{E_0, E_1}$.
	By definition of~$f$ and~$E_i$, $E_0 \to_R \tilde{X} \to_R E_1$.
	
	\item[4.] Take any $R$-transitive subset $F_1 \subseteq E_i$ for some~$i < 2$ and set $\tilde{F} = F \cup F_1$.
	The pair $(\tilde{F}, \tilde{Y})$ satisfies the required properties (see~\cite[Lemma 5.9]{Patey2015Degrees} for a proof).
\end{itemize}

From a computational point of view, if we start with a computable condition~$(F, X)$, that is, where~$X$ is a computable set,
we end up with a computable extension~$(\tilde{F}, \tilde{Y})$.
Remember that our goal is to define a $\Delta^0_2$ function~$f$ which will dominate every $G$-computable function
for some solution~$G$ to~$R$. For this, we need to be able to~$\emptyset'$-decide
whether~$\Phi^G_e(n) \downarrow$ or $\Phi^G_e(n) \uparrow$ for every solution~$G$ to~$R$ satisfying 
some condition~$(F, X)$. 
More generally, given some~$\Sigma^0_1$ formula $\varphi$, we focus on the computational power required to decide a question of the form

\smallskip
{\itshape
Q1: Is there an $R$-transitive extension $\tilde{F}$ of $F$ in $X$ such that $\varphi(\tilde{F})$ holds?
}
\smallskip

Trying to apply naively the algorithm above requires a lot of computational power.
In particular, step 3 requires to choose a true formula among finitely many $\Pi^{0, X}_2$ formulas.
Such a step needs the power of PA degree relative to the jump of~$X$.
We shall apply the same trick as for cohesiveness, consisting in not trying to choose a true $\Pi^{0,X}_2$ formula,
but instead parallelizing the construction. Given a finite set $E \subset X$, 
instead of finding an infinite subset $\tilde{Y} \subset X \setminus E$
whose members induce a 2-partition of $E$, we will construct
as many extensions of $(F, X)$ as there are 2-partitions of~$E$. The question now becomes

\smallskip
{\itshape
Q2: Is there a finite set $E \subseteq X$ such that for every 2-partition $\tuple{E_0, E_1}$ of~$E$,
there exists an $R$-transitive subset $F_1 \subseteq E_i$ for some $i < 2$ such that $\varphi(F \cup F_1)$ holds?
}
\smallskip

This question is $\Sigma^{0,X}_1$, which is good enough for our purposes.
If the answer is positive, we will try the witness $F_1$ associated to each 2-partition of $E$ in parallel.
Note that there may be some 2-partition $\tuple{E_0, E_1}$ of~$E$
such that the set $Y = \{ x \in X \setminus E : E_0 \to_R \{x\} \to_R E_1 \}$ is finite,
but this is not a problem since there is \emph{at least}
one good 2-partition such that the corresponding set is infinite. 
The whole construction yields again a tree of pairs~$(F, X)$.

If the answer is negative, we want to ensure that
$\varphi(\tilde{F})$ will not hold at any further stage of the construction.
For each~$n \in \omega$, let $H_n$ be the set of the $n$ first elements of~$X$.
Because the answer is negative, for each~$n \in \omega$, there exists a 2-partition $\tuple{E_0, E_1}$
of~$H_n$ such that for every $R$-transitive subset $F_1 \subseteq E_i$ for any $i < 2$, $\varphi(F \cup F_1)$ does not hold.
Call such a 2-partition an \emph{avoiding} partition of $H_n$. 
Note that if $\tuple{E_0, E_1}$ is an avoiding partition of $H_{n+1}$, then $\tuple{E_0 \uh n, E_1 \uh n}$
is an avoiding partition of $H_n$. So the set of avoiding 2-partitions of some $H_n$
forms an infinite tree~$T$. Moreover, the predicate ``$\tuple{E_0, E_1}$ is an avoiding partition of $H_n$''
is $\Delta^{0, H_n}_1$ so the tree $T$ is $\Delta^{0,X}_1$. The collection of the infinite paths through $T$
forms a non-empty $\Pi^{0,X}_1$ class $\Ccal$ defined as the collection of 2-partitions $Z_0 \cup Z_1 = X$
such that for every $i < 2$ and every $R$-transitive subset $F_1 \subseteq Z_i$, $\varphi(F \cup F_1)$
does not hold.

The natural next step would be to apply weak K\"onig's lemma to obtain a 2-partition of~$X$
such that for every finite $R$-transitive subset $F_1$ of any of its parts, $\varphi(F \cup F_1)$ does not hold.
By the low basis theorem, we could take the 2-partition to be low over~$X$ and the whole construction would remain~$\Delta^0_2$.
However, when iterating the construction, we will be given only finite pieces of tournaments
since the tournament may depend on an oracle being constructed at a previous iteration. In this setting,
it will be impossible to compute a member of the $\Pi^{0,X}_1$ class $\Ccal$ of 2-partitions, since 
we will have access to only a finite piece of the corresponding tree~$T$.
In order to get progressively prepared to the iterated forcing, we will not apply $\wkl$
and will work with $\Pi^0_1$ classes of 2-partitions.
Therefore, if the answer is negative, we duplicate the finite $R$-transitive $F$ into two sets $F_0 = F_1 = F$,
and commit~$F_i$ to take from now on its next elements from~$X_i$ for some 2-partition
$X_0 \cup X_1 = X$ belonging to the $\Pi^0_1$ class~$\Ccal$ of 2-partitions witnessing the negative answer.
Iterating the process by asking several questions leads to tuples $(F_0, \dots, F_{k-1}, \Ccal)$
where $F_i$ is a finite $R$-transitive set taking its elements from the $i$th part of the class~$\Ccal$ of $k$-partitions.
This notion of forcing will be defined formally in a later section.

\subsection{Enumerating the computable infinite tournaments}

Proving that some principle~$\Psf$ does not computably reduce to~$\Qsf$
requires to create a $\Psf$-instance~$X$ such that \emph{every} $X$-computable $\Qsf$-instance
has a solution~$Y$ such that~ $Y \oplus X$ does not compute a solution to~$X$.
In the case of $\amt \not \leq_c \coh$, we have been able to restrict ourselves to only one instance of $\coh$,
since Jockusch and Stephan~\cite{Jockusch1993cohesive} showed it admits a universal instance.
It is currently unknown whether the Erd\H{o}s-Moser theorem admits a universal instance, that is, a computable infinite tournament
such that for every infinite transitive subtournament $H$ and for every computable infinite tournament $T$,
$H$ computes an infinite transitive $T$-subtournament. See~\cite{Patey2015Degrees} for an extensive study of the existence
of universal instances for principles in reverse mathematics.

Since we do not know whether $\emo$ admits a universal instance, we will need to diagonalize against
the solutions to every computable $\emo$-instance. In fact, we will prove a stronger result. We will construct
a $\Delta^0_2$ function~$f$ and an infinite set~$G$ which is eventually transitive simultaneously for every computable infinite tournament,
and such that $f$ dominates every $G$-computable function. There exists no computable sequence of sets
containing all computable sets. Therefore it is not possible to computably enumerate every infinite computable tournament.
However, one can define an infinite, computable, binary tree such that every infinite path
computes such a sequence. 
See the notion of sub-uniformity defined by Mileti in~\cite{Mileti2004Partition} for details.
By the low basis theorem, there exists a low set bounding a sequence containing, 
among others, every infinite computable tournament.
As we shall prove below,  for every set~$C$ and every uniformly $C$-computable sequence of infinite tournaments~$\vec{R}$,
there exists a set~$G$ together with a $\Delta^{0, C}_2$ function $f$ such that
\begin{itemize}
	\item[(i)] $G$ is eventually $R$-transitive for every $R \in \vec{R}$
	\item[(ii)] If $\Phi^{G \oplus C}_e$ is total, then it is dominated by $f$ for every $e \in \omega$.
\end{itemize}
Thus it suffices to choose~$C$ to be our low set and $\vec{R}$ to be a uniformly $C$-computable sequence
of infinite tournaments containing every computable tournament to deduce the existence of a set~$G$ together
with a $\Delta^0_2$ function $f$ such that 
\begin{itemize}
	\item[(i)] $G$ is eventually $R$-transitive for every infinite, computable tournament $R$
	\item[(ii)] If $\Phi^{G \oplus C}_e$ is total, then it is dominated by $f$ for every $e \in \omega$
\end{itemize}

By the computable equivalence between $\amt$ and the escape property,
there exists a computable atomic theory $T$ such that every atomic model computes
a function~$g$ not dominated by~$f$. If $\amt \leq_c \emo$, then there exists
an infinite, computable tournament~$R$ such that every infinite $R$-transitive subtournament 
computes a model of~$T$, hence computes a function~$g$ not dominated by~$f$.
As the set~$G$ is, up to finite changes, an infinite $R$-transitive subtournament,
$G$ computes such a function~$g$, contradicting our hypothesis. Therefore $\amt \not \leq_c \emo$.

\subsection{Cover classes}

In this part, we introduce some terminology about classes of $k$-covers.
Recall that a $k$-cover of some set $X$ is a $k$-uple $A_0, \dots, A_{k-1}$ such that~$A_0 \cup \dots \cup A_{k-1} = X$.
In particular, the sets are not required to be pairwise disjoint.

\smallskip
\emph{Cover class}.
We identify a $k$-cover $Z_0 \cup \dots \cup Z_{k-1}$ of some set $X$ with the $k$-fold join of its parts
$Z = \bigoplus_{i < k} Z_i$, and refer this as a \emph{code} for the cover.
A \emph{$k$-cover class} of some set~$X$ is a tuple $\tuple{k, X, \Ccal}$
where $\Ccal$ is a collection of codes of $k$-covers of~$X$. 
We will be interested in $\Pi^0_1$ $k$-cover classes.
A \emph{part} of a $k$-cover class $\tuple{k, X, \Ccal}$ is a number $\nu < k$. Informally, a part $\nu$
represents the collection of all $Z_\nu$, where $Z_0 \oplus \dots \oplus Z_{k-1} \in \Ccal$.
For the simplicity of notation, we may use the same letter~$\Ccal$ to denote both a $k$-cover class~$(k, X, \Ccal)$
and the actual collection of $k$-covers~$\Ccal$. We then write $dom(\Ccal)$ for $X$
and $parts(\Ccal)$ for~$k$.

\smallskip
\emph{Restriction of a cover}. Given some $k$-cover $Z = Z_0 \oplus \dots \oplus Z_{k-1}$ of some set~$X$ and given some set~$Y \subseteq X$, we write $Z \restr Y$ for the $k$-cover $(Z_0 \cap Y) \oplus \dots \oplus (Z_{k-1} \cap Y)$ of~$Y$.
Similarly, given some cover class~$(k, X, \Ccal)$ and some set~$Y \subseteq X$, we denote by $\Ccal \restr Y$
the cover class~$(k, Y, \Dcal)$ where $\Dcal = \{ Z \restr Y : Z \in \Ccal \}$.
Given some part~$\nu$ of $\Ccal$ and some set~$E$, we write~$\Ccal^{[\nu, E]}$
for the cover class~$(k, X, \Dcal)$ where 
$\Dcal = \{ Z_0 \oplus \dots \oplus Z_{k-1} \in \Ccal : E \subseteq Z_\nu \}$.

\smallskip
\emph{Refinement}. The collection of cover classes can be given a natural partial order as follows.
Let~$m \geq k$ and $f : m \to k$. An $m$-cover $V_0 \oplus \dots \oplus V_{m-1}$ of $Y$ \emph{$f$-refines}
a $k$-cover $Z_0 \oplus \dots \oplus Z_{k-1}$ of $X$ if $Y \subseteq X$ and $V_\nu \subseteq Z_{f(\nu)}$ for each~$\nu < m$.
Given two cover classes $(k, X, \Ccal)$ and~$(m, Y, \Dcal)$
and some function $f : m \to k$, we say that $\Dcal$ \emph{$f$-refines} $\Ccal$
if for every $V \in \Dcal$, there is some~$Z \in \Ccal$ such that $V$ $f$-refines $Z$.
In this case, we say that \emph{part $\nu$ of $\Dcal$ refines part~$f(\nu)$ of~$\Ccal$}.

\smallskip
\emph{Acceptable part}. 
We say that part $\nu$ of $\Ccal$ is \emph{acceptable} if there exists some $Z_0 \oplus \dots \oplus Z_{k-1} \in \Ccal$
such that $Z_\nu$ is infinite. Part $\nu$ of $\Ccal$ is \emph{empty} if 
for every $Z_0 \oplus \dots \oplus Z_{k-1} \in \Ccal$, $Z_\nu = \emptyset$.
Note that if $\Ccal$ is non-empty and $dom(\Ccal)$ is infinite, then $\Ccal$ has at least one acceptable part.
Moreover, if~$\Dcal \leq_f \Ccal$ and part~$\nu$ of~$\Dcal$ is acceptable, then so is part~$f(\nu)$ of $\Ccal$.
The converse does not hold in general.

\subsection{The forcing notion}

We now get into the core of our forcing argument by defining
the forcing notion which will be used to build an infinite set eventually
transitive for every infinite computable tournament.
Fix a set $C$ and a uniformly $C$-computable sequence of infinite tournaments $R_0, R_1, \dots$
We construct our set~$G$ by a forcing whose conditions are tuples $(\alpha, \vec{F}, \Ccal)$ where
\begin{itemize}
	\item[(a)] $\Ccal$ is a non-empty $\Pi^{0,C}_1$ $k$-cover class of $[t, +\infty)$ 
	for some $k, t \in \omega$ ; $\alpha \in t^{<\omega}$
	\item[(b)] $(F_\nu \setminus [0, \alpha(i))) \cup \{x\}$ is $R_i$-transitive for every $Z_0 \oplus \dots \oplus Z_{k-1} \in \Ccal$,
	every $x \in Z_\nu$, every $i < |\alpha|$ and each $\nu < k$
	\item[(c)] $Z_\nu$ is included in a minimal $R_i$-interval of $F_\nu \setminus [0, \alpha(i))$
	for every $Z_0 \oplus \dots \oplus Z_{k-1} \in \Ccal$, every $i < |\alpha|$ and each $\nu < k$.
\end{itemize}
A condition $(\beta, \vec{E}, \Dcal)$ \emph{extends} 
$(\alpha, \vec{F}, \Ccal)$
(written $(\beta, \vec{E}, \Dcal) \leq (\alpha, \vec{F}, \Ccal)$) if $\beta \succeq \alpha$
and there exists a function $f : parts(\Dcal) \to parts(\Ccal)$ such that the following holds:
\begin{itemize}
	\item[(i)] $(E_\nu, dom(\Dcal))$ Mathias extends $(F_{f(\nu)}, dom(\Ccal))$ for each $\nu < parts(\Dcal)$ 
	\item[(ii)] $\Dcal$ $f$-refines $\Ccal^{[f(\nu), E_\nu \setminus F_{f(\nu)}]}$ for each $\nu < parts(\Dcal)$
\end{itemize}

One may think of a condition $(\alpha, \vec{F}, \Ccal)$ with, say, $parts(\Ccal) = k$,
as $k$ parallel Mathias conditions which are,
up to finite changes, Erd\H{o}s-Moser conditions simultaneously for the tournaments $R_0, \dots, R_{|\alpha|-1}$.
Given some $i < |\alpha|$, the value $\alpha(i)$ indicates at which point the sets $\vec{F}$
start being $R_i$-transitive.
More precisely, for every part $\nu < k$ and every $k$-cover $Z_0 \oplus \dots \oplus Z_{k-1} \in \Ccal$,
$(F_\nu \setminus [0, \alpha(i)), Z_\nu)$ is an Erd\H{o}s-Moser condition for $R_i$ for each $i < |\alpha|$.
Indeed, because of clause~(i), the elements $E_\nu \setminus F_{f(\nu)}$ added to $E_\nu$ come from $dom(\Ccal)$
and because of clause~(ii), these elements must come from the part $f(\nu)$ of the class~$\Ccal$,
otherwise $\Ccal^{[f(\nu), E_\nu \setminus F_{f(\nu)}]}$ would be empty and so would be $\Dcal$.

Of course, there may be some parts~$\nu$ of $\Ccal$ which are non-acceptable, that is, such that $Z_\nu$ is finite
for every $k$-cover $Z_0 \oplus \dots \oplus Z_{k-1} \in \Ccal$. However, by the infinite pigeonhole principle, 
$Z_\nu$ must be infinite for at least one $\nu < k$.
Choosing $\alpha$ to be in $t^{<\omega}$ instead of $\omega^{<\omega}$
ensures that all elements added to~$\vec{F}$ will have to be $R_i$-transitive
simultaneously for each~$i < |\alpha|$, as the elements are taken from $dom(\Ccal)$
and therefore are greater than the threshold $\alpha(i)$ for each $i < |\alpha|$.
A \emph{part} of a condition $c = (\alpha, \vec{F}, \Ccal)$ is a pair $\tuple{c, \nu}$,
where $\nu < parts(\Ccal)$. For the simplicity of notation, we may identify a part $\tuple{c, \nu}$
of a condition with the part $\nu$ of the corresponding cover class $\Ccal$. It must however be clear
that a part depends on the condition~$c$.

We start with a few basic lemmas reflecting the combinatorics described in 
the subsection~\ref{subsect:combi-em}.
They are directly adapted from the basic properties of an Erd\H{o}s-Moser condition
proven in~\cite{Patey2015Degrees}.
The first lemma states that each element of the finite transitive tournaments $\vec{F}$ behaves
uniformly with respect to the elements of the reservoir, that is, is beaten by every element
of the reservoir or beats all of them.

\begin{lemma}\label{lem:em-comp-reduc-uniform-behaviour}
For every condition~$c = (\alpha, \vec{F}, \Ccal)$,
every $Z_0 \oplus \dots \oplus Z_{k-1} \in \Ccal$, 
every part $\nu$ of $\Ccal$, every $i < |\alpha|$ and every $x \in F_\nu \setminus [0, \alpha(i))$, 
either $\{x\} \to_{R_i} Z_\nu$ or $Z_\nu \to_{R_i} \{x\}$.
\end{lemma}
\begin{proof}
By property (c) of the condition~$c$, there exists a minimal $R_i$-interval
$(u, v)$ of $F_\nu \setminus [0, \alpha(i))$ containing $Z_\nu$.
Here, $u$ and $v$ may be respectively $-\infty$ and $+\infty$.
By definition of an interval, $\{u\} \to_{R_i} Z_\nu \to_{R_i} \{v\}$.
By definition of a minimal interval, $R_i(x, u)$ or $R_i(v, x)$ holds.
Suppose the former holds. By transitivity of $F_\nu \setminus [0, \alpha(i))$,
for every $y \in Z_\nu$, $R_i(x, y)$ holds, since both $R_i(x, u)$ and~$R_i(u, y)$ hold. 
Therefore $\{x\} \to_{R_i} Z_\nu$. In the latter case, by symmetry, $Z_\nu \to_{R_i} \{x\}$.
\end{proof}

The second lemma is the core of the combinatorics of the Erd\H{o}s-Moser theorem. It provides
sufficient properties to obtain a valid extension of a condition. Properties (i) and (ii)
are simply the definition of an extension. Properties (iii) and (iv) help to propagate
properties (b) and (c) from a condition to its extension. We shall see empirically that 
properties (iii) and (iv) are simpler to check than (b) and (c), 
as the former properties match exactly the way we add elements to our finite tournaments $\vec{F}$. 
Therefore, ensuring that these properties
are satisfied usually consists of checking that we followed the standard process of adding elements
to~$\vec{F}$. 

\begin{lemma}\label{lem:em-comp-reduc-sufficient-cond-ext}
Fix a condition~$c = (\alpha, \vec{F}, \Ccal)$ where $\Ccal$ is a $k$-cover class of~$[t, +\infty)$. 
Let $E_0, \dots, E_{m-1}$ be finite sets, $\Dcal$ be a non-empty $\Pi^{0,C}_1$ $m$-cover class of $[t', +\infty)$
for some~$t' \geq t$ and $f : m \to k$ be a function such that for each~$i < |\alpha|$ and $\nu < m$,
\begin{itemize}
	\item[(iii)] $E_\nu$ is $R_i$-transitive
	\item[(iv)] $V_\nu \to_{R_i} E_\nu$ or $E_\nu \to_{R_i} V_\nu$ for each $V_0 \oplus \dots \oplus V_{m-1} \in \Dcal$
\end{itemize}
Set $H_\nu = F_{f(\nu)} \cup E_\nu$ for each $\nu < m$.
If properties (i) and (ii) of an extension are satisfied for~$d = (\alpha, \vec{H}, \Dcal)$ with witness $f$,
then~$d$ is a valid condition extending~$c$.
\end{lemma}
\begin{proof}
All we need is to check properties (b) and (c) for~$d$ in the definition of a condition.
We prove property (b). Fix an $i < |\alpha|$, some part $\nu$ of $\Dcal$, and an $x \in V_\nu$
for some $V_0 \oplus \dots \oplus V_{m-1} \in \Dcal$. In order to prove that 
$(F_{f(\nu)} \cup E_\nu) \setminus [0, \alpha(i)) \cup \{x\}$
is $R_i$-transitive, it is sufficient to check that the set contains no 3-cycle.
Fix three elements $u < v < w \in (F_{f(\nu)} \cup E_\nu) \setminus [0, \alpha(i)) \cup \{x\}$.
\begin{itemize}
	\item Case 1: $\{u, v, w\} \cap F_{f(\nu)} \setminus [0, \alpha(i)) \neq \emptyset$. 
	Then $u \in F_{f(\nu)} \setminus [0, \alpha(i))$ as $F_{f(\nu)} < E_\nu < \{x\}$ and $u < v < w$.
	By property (ii), there is some $Z_0 \oplus \dots \oplus Z_{k-1} \in \Ccal$ such that $E_\nu \cup \{x\} \subseteq Z_{f(\nu)}$.
	If $v \in F_{f(\nu)}$, then by property (b) of the condition~$c$ on~$Z_{f(\nu)}$, $\{u, v, w\}$ is $R_i$-transitive.
	If $v \not \in F$, then by Lemma~\ref{lem:em-comp-reduc-uniform-behaviour}, $\{u\} \to_{R_i} Z_{f(\nu)}$
	or $Z_{f(\nu)} \to_{R_i} \{u\}$, so $\{u, v, w\}$ is $R_i$-transitive since~$v, w \in Z_{f(\nu)}$.

	\item Case 2: $\{u, v, w\} \cap  F_{f(\nu)} \setminus [0, \alpha(i)) = \emptyset$. 
	Then at least $u, v \in E_\nu$ because $E_\nu < \{x\}$.
	If $w \in E_\nu$ then $\{u, v, w\}$ is $R_i$-transitive by $R_i$-transitivity of $E_\nu$.
	In the other case, $w = x \in V_\nu$. As $E_\nu \to_{R_i} V_\nu$ or $V_\nu \to_{R_i} E_\nu$,
	$\{u, v\} \to_{R_i} \{w\}$ or $\{w\} \to_{R_i} \{u, v\}$ and $\{u, v, w\}$ is $R_i$-transitive.
\end{itemize}

We now prove property (c) for $d$. Fix some $V_0 \oplus \dots \oplus V_{m-1} \in \Dcal$, 
some part $\nu$ of~$\Dcal$ and some $i < |\alpha|$.
By property (ii), there is some~$Z_0 \oplus \dots \oplus Z_{k-1} \in \Ccal$ such that $E_\nu \cup V_\nu \subseteq Z_{f(\nu)}$.
By property (c) of the condition~$c$, $Z_{f(\nu)}$ (and so $V_\nu$) is included in a minimal $R_i$-interval $(u, v)$ of 
$F_{f(\nu)} \setminus [0, \alpha(i))$.
Here again, $u$ and $v$ may be respectively $-\infty$ and $+\infty$. 
By assumption, either $E_\nu \to_{R_i} V_\nu$ or $V_\nu \to_{R_i} E_\nu$. As $E_\nu$ is a finite $R_i$-transitive set,
it has a minimal and a maximal element, say~$x$ and~$y$. If $E_\nu \to_{R_i} V_\nu$
then $V_\nu$ is included in the $R_i$-interval $(y, v)$.
Symmetrically, if $V_\nu \to_{R_i} E_\nu$ then 
$V_\nu$ is included in the $R_i$-interval $(u, x)$.
To prove minimality for the first case, assume that some $w$ is in the interval $(y, v)$.
Then $w \not \in F_{f(\nu)} \setminus [0, \alpha(i))$ by minimality of the interval $(u, v)$ with respect to 
$F_{f(\nu)} \setminus [0, \alpha(i))$, and $w \not \in E_\nu$ by maximality of~$y$.
Minimality for the second case holds by symmetry.
\end{proof}

Now we have settled the necessary technical lemmas, we start proving
lemmas which will be directly involved in the construction of the transitive subtournament.
The following simple progress lemma states that we can always find an extension of a condition
in which we increase both the finite approximations corresponding to the acceptable parts 
and the number of tournaments for which we are transitive simultaneously.
Moreover, this extension can be found uniformly.

\begin{lemma}[Progress]\label{lem:em-comp-reduc-ext}
For every condition~$c = (\alpha, \vec{F}, \Ccal)$ and every $s \in \omega$,
there exists an extension $d = (\beta, \vec{E}, \Dcal)$ such that $|\beta| \geq s$ and
$|E_\nu| \geq s$ for every acceptable part $\nu$ of~$\Dcal$.
Furthermore, such an extension can be found $C'$-effectively, uniformly in~$c$ and~$s$.
\end{lemma}
\begin{proof}
Fix a condition $c = (\alpha, \vec{F}, \Ccal)$.
First note that for every $\beta \succeq \alpha$ such that $\beta(i) > max(F_\nu : \nu < parts(\Ccal))$
whenever $|\alpha| \leq i < |\beta|$, $(\beta, \vec{F}, \Ccal)$ is a condition extending~$c$.
Therefore it suffices to prove that for every such condition~$c$ and every part $\nu$ of $\Ccal$,
we can $C'$-effectively find a condition~$d = (\alpha, \vec{H}, \Dcal)$ refining~$c$
with witness~$f : parts(\Dcal) \to parts(\Ccal)$ such that $f$ forks only parts refining part $\nu$ of $\Ccal$,
and either every such part $\mu$ of $\Dcal$ is empty or $|H_\mu| > |F_\nu|$.
Iterating the process finitely many times enables us to conclude.

Fix some part $\nu$ of $\Ccal$ and let~$\Dcal$ be the collection of $Z_0 \oplus \dots \oplus Z_{k-1} \in \Ccal$
such that $Z_\nu = \emptyset$. We can $C'$-decide whether or not $\Dcal$ is empty.
If $\Dcal$ is non-empty, then $(\alpha, \vec{F}, \Dcal)$ is a valid extension of~$c$
with the identity function as witness and such that part $\nu$ of $\Dcal$ is empty.
If $\Dcal$ is empty, we can $C'$-computably find some $Z_0 \oplus \dots \oplus Z_{k-1} \in \Ccal$
and pick some~$x \in Z_\nu$.
Consider the $C$-computable $2^{|\alpha|}$-partition $(X_\rho : \rho \in 2^{|\alpha|})$ of $\omega$ defined by
$$
X_\rho = \{ y \in \omega : (\forall i < |\alpha|)[R_i(y, x) \leftrightarrow \rho(i) = 1] \} 
$$
Let $\tilde{\Dcal}$ be the cover class refining $\Ccal^{[\nu, x]}$ such that
part $\nu$ of $\tilde{\Dcal}$ has $2^{|\alpha|}$ forks induced by the
$2^{|\alpha|}$-partition~$\vec{X}$. Define $\vec{H}$ by
$H_\mu = F_\mu$ if $\mu$ refines a part different from $\nu$,
and $H_\mu = F_\nu \cup \{x\}$ if $\mu$ refines part $\nu$ of~$\Ccal$.
The forking according to~$\vec{X}$ ensures that property (iv) of Lemma~\ref{lem:em-comp-reduc-sufficient-cond-ext} holds.
By Lemma~\ref{lem:em-comp-reduc-sufficient-cond-ext}, $d = (\alpha, \vec{H}, \tilde{\Dcal})$ is a valid extension of~$c$.
\end{proof}

\subsection{The strategy}

Thanks to Lemma~\ref{lem:em-comp-reduc-ext}, we can define an infinite, $C'$-computable
decreasing sequence of conditions $(\varepsilon, \emptyset, \{\omega\}) \geq c_0 \geq c_1 \geq \dots$
such that for each~$s \in \omega$, 
\begin{itemize}
	\item[1.] $|\alpha_s| \geq s$.
	\item[2.] $|F_{s, \nu}| \geq s$ for each acceptable part~$\nu$ of~$\Ccal_s$
\end{itemize}
where $c_s = (\alpha_s, \vec{F}_s, \Ccal_s)$.
As already noticed, if some acceptable part $\mu$ of $\Ccal_{s+1}$ refines some part $\nu$ of~$\Ccal_s$,
part $\nu$ of~$\Ccal_s$ is also acceptable.
Therefore, the set of acceptable parts forms an infinite, finitely branching $C'$-computable tree~$\Tcal$.
Let $P$ be any infinite path through~$\Tcal$. 
The set $H(P) = (\bigcup_s F_{s, P(s)})$ is infinite,
and $H(P) \setminus [0, \alpha_{i+1}(i))$ is $R_i$-transitive for each $i \in \omega$.

Our goal is to build a $C'$-computable function dominating every function computed
by $H(P)$ for at least one path $P$ trough~$\Tcal$. However, it requires too much
computational power to distinguish acceptable parts from non-acceptable ones,
and even some acceptable part may have only finitely many extensions. Therefore,
we will dominate the functions computed by~$H(P)$ for \emph{every} path $P$ trough~$\Tcal$.
 
At a finite stage, a condition contains finitely many parts, each one representing
the construction of a transitive subtournament.
As in the construction of a cohesive set, it suffices to check one by one whether 
there exists an extension of our subtournaments which will
make terminate a given functional at a given input.
In the next subsection, we develop the framework necessary to decide such a termination
at a finite stage.

\subsection{Forcing relation}

As a condition $c = (\alpha, \vec{F}, \Ccal)$ corresponds to the construction of multiple
subtournaments $F_0, F_1, \dots$ at the same time, the forcing relation will depend on which
subtournament we are considering. In other words, the forcing relation depends on the part $\nu$ of~$\Ccal$
we focus on.

\begin{definition}\label{def:em-comp-reduc-forcing-relation}
Fix a condition $c = (\alpha, \vec{F}, \Ccal)$, a part~$\nu$ of~$\Ccal$ and two integers~$e$, $x$.
\begin{itemize}
	\item[1.] $c \Vdash_\nu \Phi_e^{G \oplus C}(x) \uparrow$ if $\Phi_e^{(F_\nu \cup F_1) \oplus C}(x) \uparrow$
	for all $Z_0 \oplus \dots \oplus Z_{k-1} \in \Ccal$ and all subsets $F_1 \subseteq Z_\nu$
	such that $F_1$ is $R_i$-transitive simultaneously for each $i < |\alpha|$.
	\item[2.] $c \Vdash_\nu \Phi_e^{G \oplus C}(x) \downarrow$ if $\Phi_e^{F_\nu \oplus C}(x) \downarrow$.
\end{itemize}
\end{definition}

The forcing relations defined above satisfy the usual forcing properties.
In particular, let $c_0 \geq c_1 \geq \dots$ be an infinite decreasing
sequence of conditions. This sequence induces an infinite, finitely branching tree of acceptable parts~$\Tcal$.
Let~$P$ be an infinite path trough~$\Tcal$. If 
$c_s \Vdash_{P(s)} \Phi_e^{G \oplus C}(x) \uparrow$ (respectively $c_s \Vdash_{P(s)} \Phi_e^{G \oplus C}(x) \downarrow$)
at some stage~$s$, then $\Phi_e^{H(P) \oplus C}(x) \uparrow$ (respectively $\Phi_e^{H(P) \oplus C}(x) \downarrow$).

Another important feature of this forcing relation is that we can decide $C'$-uniformly in its parameters
whether there is an extension forcing~$\Phi^{G \oplus C}_e(x)$ to halt or to diverge. 
Deciding this relation with little computational power is useful because our $C'$-computable dominating function will
need to decide termination $\Gamma^{G \oplus C}(x)$ to check whether it has to dominate the value 
outputted by $\Gamma^{G \oplus C}(x)$.

\begin{lemma}\label{lem:em-comp-reduc-force-dense}
For every condition~$c = (\alpha, \vec{F}, \Ccal)$ and every pair of integers $e, x \in \omega$,
there exists an extension~$d = (\alpha, \vec{H}, \Dcal)$ such that for each part~$\nu$ of~$\Dcal$
$$
d \Vdash_\nu \Phi_e^{G \oplus C}(x) \uparrow \hspace{10pt} \vee \hspace{10pt} d \Vdash_\nu \Phi_e^{G \oplus C}(x) \downarrow
$$
Furthermore, such an extension can be found $C'$-effectively, uniformly in~$c$, $e$ and~$x$.
\end{lemma}
\begin{proof}
Given a condition~$c$ and two integers $e, x \in \omega$,
let $I_{e,x}(c)$ be the set of parts $\nu$ of~$c$
such that $c \not \Vdash_\nu \Phi_e^{G \oplus C}(x) \downarrow$ and $c \not \Vdash_\nu \Phi_e^{G \oplus C}(x) \uparrow$.
Note that $I_{e,x}(c)$ is $C'$-computable uniformly in~$c$, $e$ and~$x$.
It suffices to prove that given such a condition~$c$ and a part~$\nu \in I_{e,x}(c)$, one can $C'$-effectively
find an extension~$d$ with witness $f$ such that $f(I_{e,x}(d)) \subseteq I_{e,x}(c) \setminus \{\nu\}$.
Applying iteratively the operation enables us to conclude.

Fix a condition~$c = (\alpha, \vec{F}, \Ccal)$ where $\Ccal$ is a $k$-cover class, and fix some part~$\nu \in I_{e,x}(c)$.
The strategy is the following: either we can fork part~$\nu$ of $\Ccal$ into enough parts so that we 
force~$\Phi_e^{G \oplus C}(x)$ to diverge
on each forked part, or we can find an extension forcing $\Phi_e^{G \oplus C}(x)$ to converge on part~$\nu$ without forking.
Hence, we ask the following question.

\smallskip
{\itshape
Q2: Is it true that for every $k$-cover~$Z_0 \oplus \dots \oplus Z_{k-1} \in \Ccal$,
for every~$2^{|\alpha|}$-partition $\bigcup_{\rho \in 2^\alpha} X_\rho = Z_\nu$,
there is some~$\rho \in 2^{|\alpha|}$ and some finite set~$F_1$ which is $R_i$-transitive
for each~$i < |\alpha|$ simultaneously, and such that~$\Phi_e^{(F_\nu \cup F_1) \oplus C}(x) \downarrow$?
}
\smallskip

If the answer is no, then by forking the part~$\nu$ of~$\Ccal$ into $2^{|\alpha|}$ parts,
we will be able to force~$\Phi_e^{G \oplus C}(x)$ to diverge.
Let~$m = k+2^{|\alpha|}-1$ and define the function~$f : m \to k$ by $f(\mu) = \mu$ 
if $\mu < k$ and $f(\mu) = \nu$ otherwise.
Let~$\Dcal$ be the collection of all~$m$-covers $V_0 \oplus \dots \oplus V_{m-1}$ which $f$-refine
some $Z_0 \oplus \dots \oplus Z_{k-1} \in \Ccal$ and such that for every part $\mu$ of $\Dcal$ 
$f$-refining part~$\nu$ of $\Ccal$ and every subset $F_1 \subseteq V_\mu$
which is $R_i$-transitive simultaneously for each~$i < |\alpha|$, $\Phi_e^{F_\nu \cup F_1}(x) \uparrow$.
Note that~$\Dcal$ is a $\Pi^{0,C}_1$ $m$-cover class $f$-refining $\Ccal$. Moreover~$\Dcal$
is non-empty since the answer to~{\itshape Q2} is no.
Let~$\vec{E}$ be defined by $E_\mu = F_\mu$ if $\mu < k$
and $E_\mu = F_\nu$ otherwise. The condition~$d = (\alpha, \vec{E}, \Dcal)$ extends~$c$
with witness~$f$. For every part~$\mu$ of $\Dcal$ $f$-refining part $\nu$ of $\Ccal$, $d \Vdash_\mu \Phi_e^{G \oplus C}(x) \uparrow$,
therefore $f(I_{e,x}(d)) \subseteq I_{e,x}(c) \setminus \{\nu\}$.

Suppose now that the answer is yes. By compactness, we can $C'$-effectively find a finite set~$E \subseteq Z_\nu$
for some~$Z_0 \oplus \dots \oplus Z_{k-1} \in \Ccal$ such that for every $2^{|\alpha|}$-partition $(E_\rho : \rho \in 2^{|\alpha|})$
of $E$, there is some $\rho \in 2^{|\alpha|}$ and some set $F_1 \subseteq E_\rho$ which is $R_i$-transitive
simultaneously for each $i < |\alpha|$ and such that $\Phi_e^{(F_\nu \cup F_1) \oplus C}(x) \downarrow$.
There are finitely many $2^{|\alpha|}$-partitions of $E$. Let~$n$ be the number of such partitions. 
These partitions induce a finite $C$-computable $n$-partition of~$dom(\Ccal)$
defined for each $(E_\rho : \rho \in 2^{|\alpha|})$ by
$$
X_{\tuple{E_\rho : \rho \in 2^{|\alpha|}}} = \left\{ y \in dom(\Ccal) : (\forall i < |\alpha|) 
	\cond{
	\mbox{ if } \rho(i) = 0 \mbox{ then } E_\rho \to_{R_i} \{y\} \\ 
	\mbox{ if } \rho(i) = 1 \mbox{ then } \{y\} \to_{R_i} E_\rho} \right\}
$$

Let~$\tilde{\Dcal}$ be the $\Pi^{0,C}_1$ $(k+n-1)$-cover class refining $\Ccal^{[\nu, E]}$
and such that part~$\nu$ of~$\Ccal^{[\nu, E]}$ is refined accordingly to the above partition of~$dom(\Ccal)$.
Let~$f : k+n-1 \to k$ be the refining function witnessing it. 
Define~$\vec{H}$ as follows. For every part~$\mu$ of $\Dcal$, refining part~$\nu$ of $\Ccal^{[\nu, E]}$,
by definition of~$\tilde{\Dcal}$, there is some~$2^{|\alpha|}$-partition $\tuple{E_\rho : \rho \in 2^{|\alpha|}}$ of~$E$
such that for every $V_0 \oplus \dots V_{k+n-2} \in \tilde{\Dcal}$, $V_\mu \subseteq X_{\tuple{E_\rho : \rho \in 2^{|\alpha|}}}$.
By choice of~$E$, there exists some set $F_1 \subseteq E_\rho$ for some~$\rho \in 2^{|\alpha|}$
which is $R_i$-transitive simultaneously for each $i < |\alpha|$ and such that
$\Phi_e^{(F_\nu \cup F_1) \oplus C}(x) \downarrow$.
This set $F_1$ can be found $C'$-effectively. Set $H_\mu = F_\nu \cup F_1$.
For every part~$\mu$ of $\tilde{\Dcal}$ which refines some part~$\xi$ of $\Ccal^{[\nu, E]}$ different from~$\nu$,
set~$H_\mu = F_\xi$.
By Lemma~\ref{lem:em-comp-reduc-sufficient-cond-ext}, $d = (\alpha, \vec{H}, \tilde{\Dcal})$ is a valid condition
extending~$c$. Moreover, for every part $\mu$ of~$\tilde{\Dcal}$ refining part~$\nu$ of $\Ccal$,
$d \Vdash_\mu \Phi_e^{G \oplus C}(x) \downarrow$. Therefore $f(I_{e,x}(d)) \subseteq I_{e,x}(c) \setminus \{\nu\}$.
\end{proof}


\subsection{Construction}

We are now ready to construct our infinite transitive subtournament~$H(P)$ together
with a $C'$-computable function~$f$ dominating every~$H(P) \oplus C$-computable function.
Thanks to Lemma~\ref{lem:em-comp-reduc-ext} and Lemma~\ref{lem:em-comp-reduc-force-dense}, we can $C'$-compute an infinite
descending sequence of conditions $(\epsilon, \emptyset, 1^{<\omega}) \geq c_0 \geq c_1 \geq \dots$
such that at each stage $s \in \omega$,
\begin{itemize}
	\item[1.] $|\alpha_s| \geq s$
	\item[2.] $|F_{s, \nu}| \geq s$ for each acceptable part~$\nu$ of~$\Ccal_s$
	\item[3.] $c_s \Vdash_\nu \Phi_e^{G \oplus C}(x) \downarrow$ or $c_s \Vdash_\nu \Phi_e^{G \oplus C}(x) \uparrow$
	for each part~$\nu$ of~$\Ccal_s$ if $\tuple{e, x} = s$
\end{itemize}
where $c_s = (\alpha_s, \vec{F}_s, \Ccal_s)$.
Property 1 ensures that the resulting set with be eventually transitive
for every tournament in~$\vec{R}$. Property~2 makes the subtournaments infinite.
Last, property 3 enables us to $C'$-decide at a finite stage whether a functional terminates on a given
input, with the transitive subtournament as an oracle.

Define the $C'$-computable function $f : \omega \to \omega$ as follows:
On input~$x$, the function~$f$ looks at all stages~$s$ such that $s = \tuple{e,x}$ for
some $e \leq x$. For each such stage~$s$, and each part~$\nu$ in~$\Ccal_s$,
the function $C'$-decides whether $c_s \Vdash_\nu \Phi^{G \oplus C}_e(x) \downarrow$
or $c_s \Vdash_\nu \Phi^{G \oplus C}_e(x) \uparrow$. 
In the first case, $f$ computes the value $\Phi^{F_{s, \nu} \oplus C}_e(x)$.
Having done all that, $f$ returns a value greater than the maximum of the computed values.

Fix any infinite path~$P$ trough the infinite tree $\Tcal$ of the acceptable parts induced
by the infinite descending sequence of conditions. 
We claim that $f$ dominates every function computed by~$H(P) \oplus C$.
Fix any Turing index $e \in \omega$ such that $\Phi_e^{H(P) \oplus C}$ is total.
Consider any input~$x \geq e$ and the corresponding stage $s = \tuple{e,x}$. 
As $\Phi_e^{H(P) \oplus C}$ is total, $c_s \not \Vdash_{P(s)} \Phi_e^{G \oplus C}(x) \uparrow$,
hence by property 3, $c_s \Vdash_{P(s)} \Phi_e^{G \oplus C}(x) \downarrow$.
By construction, $f(x)$ computes the value of $\Phi_e^{F_{s,P(s)} \oplus C}(x)$ and returns
a greater value. As $F_{s,P(s)}$ is an initial segment of $H(P)$, 
$\Phi_e^{F_{s,P(s)} \oplus C}(x) = \Phi_e^{H(P) \oplus C}(x)$
and therefore $f(x) > \Phi_e^{H(P) \oplus C}(x)$.
This completes the proof of~$\amt \not \leq_c \emo$.

We identify a $k$-cover $Z_0 \cup \dots \cup Z_{k-1}$ of some set $X$ with the $k$-fold join of its parts 
