
In this last section, we give a short history of the zoo related to the Erd\H{o}s-Moser theorem.
In Figure~\ref{fig:local-zoo}, we present the various implications proven
between the principles $\emo$, $\sts^2$, $\sads$ and $\amt$. An arrow
denotes an implication over $\rca$. A dotted arrow from a principle $\Psf$
to a principle $\Qsf$ denotes the existence of an $\omega$-model of $\Psf$ which is not a model of~$\Qsf$.

\begin{figure}[htbp]
\begin{center}
\begin{tikzpicture}[x=2cm, y=1.7cm, 
	node/.style={minimum size=2em},
	impl/.style={draw,very thick,-latex},
	strict/.style={impl}, %{draw, thick, -latex, double distance=2pt},
	nonimpl/.style={draw, very thick, dotted, -latex},
	edgelabel/.style={inner sep=0pt}]

	\node[node] (coh+em+wkl) at (2, 3) {$\coh+\emo+\wkl$};
	\node[node] (em) at (2, 2)  {$\emo$};
	\node[node] (sts2) at (1, 1) {$\sts(2)$};
	\node[node] (sads) at (3, 1) {$\sads$};
	\node[node] (amt) at (2, 0) {$\amt$};

	\draw[strict] (coh+em+wkl) -- (em);
	\draw[strict] (sts2) -- (amt) node [edgelabel, midway, above right=5pt] {\tiny (2)};
	\draw[strict] (sads) -- (amt) node [edgelabel, midway, above left=5pt] {\tiny (1)};
	
	\draw[nonimpl] (em) -- (sads) node [edgelabel, midway, below left=2pt] {\tiny (3)};
	\draw[nonimpl] (em) -- (sts2) node [edgelabel, midway, below right=2pt] {\tiny (4)};
	\draw[nonimpl] (coh+em+wkl) -- (sts2) node [edgelabel, midway, below right=1pt] {\tiny (5)};
	\draw[nonimpl] (coh+em+wkl) -- (sads) node [edgelabel, midway, below left=1pt] {\tiny (5)};
	\draw[nonimpl] (coh+em+wkl) .. controls (0,1) .. (amt) node [edgelabel, midway, right=3pt] {\tiny (6)};;
\end{tikzpicture}
\end{center}
\caption{Evolution of the zoo}\label{fig:local-zoo}
\end{figure}

Justification of the arrows:
\begin{itemize}
	\item[(1)] Hirschfeldt, Shore and Slaman~\cite{Hirschfeldt2009atomic} proved that~$\amt$ is a consequence of~$\sads$ over~$\rca$. 
	\item[(2)] The author proved in~\cite{Patey2015Somewhere} that $\sts(2)$ implies $\amt$ over~$\rca$ using a similar argument.
	\item[(3)] Lerman, Solomon and Towsner~\cite{Lerman2013Separating} separated $\emo$ from~$\sads$ using an iterated forcing construction.
	\item[(4)] The author noticed in~\cite{Patey2013note} that the forcing of Lerman, Solomon and Towsner
	could be adapted to separate $\emo$ from $\sts(2)$ over $\rca$.
	\item[(5)] Wang~\cite{Wang2014Definability} used the notion of preservation of $\Delta^0_2$ definitions
	to separate $\coh+\emo+\wkl$ from $\sads$ and $\sts(2)$ over $\rca$.
	\item[(6)] This is the main result of the current paper.
\end{itemize}


