

\documentclass[options]{amsart}

\usepackage[utf8x]{inputenc}
\usepackage{latexsym,ifthen,amssymb}
\usepackage{latexsym,ifthen,amssymb}
%\usepackage{natbib}
\usepackage[toc,page,title,titletoc,header]{appendix}
\usepackage{multirow}
 \usepackage{longtable}
 \usepackage{rotating}
 \usepackage{graphicx, amsmath, amsthm, amssymb,float,setspace,color, multirow}
\usepackage{enumitem}
\usepackage{graphicx}
\usepackage{bm}
\usepackage{subfigure}
\usepackage{float}
\usepackage{rotating}
\usepackage{epstopdf}
%\usepackage[]{caption2}
\usepackage{latexsym,ifthen,amssymb}
%\usepackage{natbib}
\usepackage{algorithm}
\usepackage{algorithmicx}
\usepackage{algpseudocode}
\usepackage[toc,page,title,titletoc,header]{appendix}
\usepackage{multirow}
 \usepackage{longtable}
 \usepackage{rotating}
 \usepackage{graphicx, amsmath, amsthm, amssymb,float,setspace,color, multirow}
\usepackage{tikz}
\usepackage{hyperref}
\usepackage{enumitem}
\usepackage[margin=1cm]{caption}

\newcommand{\ludovic}[1]{\sethlcolor{lightred}\hl{#1}}
\newcommand{\benoit}[1]{\sethlcolor{yellow}\hl{#1}}
\newcommand{\liu}[1]{\sethlcolor{lightblue}\hl{#1}}

\hypersetup{
	pdfborder={0 0 0}
}
\usetikzlibrary{decorations.markings}


% Added by Ludovic
\usepackage{url}
\usepackage[margin=1in]{geometry}
\usepackage{xcolor}	
\usepackage{soul}

% to change the margins.  Default for 12pt is 1.5in
%\usepackage[margin=1.1in]{geometry}

\usepackage{amsmath,amssymb,amsthm}
\usepackage{MnSymbol}
\usepackage{graphicx,subfigure}
\usepackage[usenames,dvipsnames]{xcolor} % for more colors


% references

\newcommand{\secref}[1]{Section~\ref{sec:#1}}
\newcommand{\figref}[1]{Figure~\ref{fig:#1}}


% operator

\DeclareMathOperator{\co}{co}
\DeclareMathOperator{\diag}{diag}
\DeclareMathOperator{\dist}{dist}
\DeclareMathOperator*{\argmin}{argmin}
\DeclareMathOperator*{\argmax}{argmax}
\DeclareMathOperator{\vol}{vol}
\DeclareMathOperator{\linspan}{span}
\DeclareMathOperator{\id}{id}


% editing
\usepackage{color}
\newcommand{\comment}[1] {\textbf{\color{red} [#1]}}
\newcommand{\change} [1] {{\color{blue} #1}}
\newcommand{\changeme}[1]{{\sc \color{Orange} [#1]}}


%commands

\newcommand{\dby}[1]{\partial_{#1}}
\newcommand{\vint}[1]{\langle #1 \rangle}
\newcommand{\vvint}[1]{\llangle #1 \rrangle}
\newcommand{\Vint}[1]{\left\langle #1 \right\rangle}
\newcommand{\VVint}[1]{\left\llangle #1 \right\rrangle}
\newcommand{\sig}[1]{\sigma_{\mathrm{#1}}}
\newcommand{\itembf}[1]{\item \textbf{#1}}


% for mean-value integrals (use \dashint or \ddashint):
\def\Xint#1{\mathchoice
{\XXint\displaystyle\textstyle{#1}}%
{\XXint\textstyle\scriptstyle{#1}}%
{\XXint\scriptstyle\scriptscriptstyle{#1}}%
{\XXint\scriptscriptstyle\scriptscriptstyle{#1}}%
\!\int}
\def\XXint#1#2#3{{\setbox0=\hbox{$#1{#2#3}{\int}$ }
\vcenter{\hbox{$#2#3$ }}\kern-.6\wd0}}
\def\ddashint{\Xint=}
\def\dashint{\Xint-}


% shorthands

\def\Diag{\operatorname{Diag}}
\def\minimize{\operatorname*{minimize}}
\def\st{\operatorname*{subject~to}}
\def\quand{\quad \mbox{and} \quad}
\def\p{\partial}
\def\grad{\nabla}
\def\veps{\varepsilon}
\def\vphi{\varphi}
\def\root3{\sqrt{3}}
\def\SN{\text{S}_N}
\def\PN{\text{P}_N}
\def\MN{\text{M}_N}
\def\dt{\Delta t}
\def\dx{\Delta x}
\def\dy{\Delta y}
\def\dz{\Delta z}
\def\siga{\sig{a}}
\def\sigs{\sig{s}}
\def\sigt{\sig{t}}

\def\divg{\grad \cdot}
\def\divx{\grad_x \cdot}
\def\curl{\grad \times}

\def\nin{\not \in}

\def\intd{\, d}
\def\intdv{\intd v}
\def\intdw{\intd w}
\def\intdx{\intd x}
\def\intdS{\intd S(\Omega)}

\def\St{\text{St}}
\def\Kn{\text{Kn}}

\def\ba{\mathbf{a}}
\def\bb{\mathbf{b}}
\def\bc{\mathbf{c}}
\def\bd{\mathbf{d}}
\def\be{\mathbf{e}}
\def\bff{\mathbf{f}}
\def\bg{\mathbf{g}}
\def\bh{\mathbf{h}}
\def\bi{\mathbf{i}}
\def\bj{\mathbf{j}}
\def\bk{\mathbf{k}}
\def\bl{\mathbf{l}}
\def\bm{\mathbf{m}}
\def\bn{\mathbf{n}}
\def\bo{\mathbf{o}}
\def\bp{\mathbf{p}}
\def\bq{\mathbf{q}}
\def\br{\mathbf{r}}
\def\bs{\mathbf{s}}
\def\bt{\mathbf{t}}
\def\bu{\mathbf{u}}
\def\bv{\mathbf{v}}
\def\bw{\mathbf{w}}
\def\bx{\mathbf{x}}
\def\by{\mathbf{y}}
\def\bz{\mathbf{z}}

\def\bA{\mathbf{A}}
\def\bB{\mathbf{B}}
\def\bC{\mathbf{C}}
\def\bD{\mathbf{D}}
\def\bE{\mathbf{E}}
\def\bF{\mathbf{F}}
\def\bG{\mathbf{G}}
\def\bH{\mathbf{H}}
\def\bI{\mathbf{I}}
\def\bJ{\mathbf{J}}
\def\bK{\mathbf{K}}
\def\bL{\mathbf{L}}
\def\bM{\mathbf{M}}
\def\bN{\mathbf{N}}
\def\bO{\mathbf{O}}
\def\bP{\mathbf{P}}
\def\bQ{\mathbf{Q}}
\def\bR{\mathbf{R}}
\def\bS{\mathbf{S}}
\def\bT{\mathbf{T}}
\def\bU{\mathbf{U}}
\def\bV{\mathbf{V}}
\def\bW{\mathbf{W}}
\def\bX{\mathbf{X}}
\def\bY{\mathbf{Y}}
\def\bZ{\mathbf{Z}}

\def\cA{\mathcal{A}}
\def\cB{\mathcal{B}}
\def\cC{\mathcal{C}}
\def\cD{\mathcal{D}}
\def\cE{\mathcal{E}}
\def\cF{\mathcal{F}}
\def\cG{\mathcal{G}}
\def\cH{\mathcal{H}}
\def\cI{\mathcal{I}}
\def\cJ{\mathcal{J}}
\def\cK{\mathcal{K}}
\def\cL{\mathcal{L}}
\def\cM{\mathcal{M}}
\def\cN{\mathcal{N}}
\def\cO{\mathcal{O}}
\def\cP{\mathcal{P}}
\def\cQ{\mathcal{Q}}
\def\cR{\mathcal{R}}
\def\cS{\mathcal{S}}
\def\cT{\mathcal{T}}
\def\cU{\mathcal{U}}
\def\cV{\mathcal{V}}
\def\cW{\mathcal{W}}
\def\cX{\mathcal{X}}
\def\cY{\mathcal{Y}}
\def\cZ{\mathcal{Z}}

\def\bbA{\mathbb{A}}
\def\bbB{\mathbb{B}}
\def\bbC{\mathbb{C}}
\def\bbD{\mathbb{D}}
\def\bbE{\mathbb{E}}
\def\bbF{\mathbb{F}}
\def\bbG{\mathbb{G}}
\def\bbH{\mathbb{H}}
\def\bbI{\mathbb{I}}
\def\bbJ{\mathbb{J}}
\def\bbK{\mathbb{K}}
\def\bbL{\mathbb{L}}
\def\bbM{\mathbb{M}}
\def\bbN{\mathbb{N}}
\def\bbO{\mathbb{O}}
\def\bbP{\mathbb{P}}
\def\bbQ{\mathbb{Q}}
\def\bbR{\mathbb{R}}
\def\bbS{\mathbb{S}}
\def\bbT{\mathbb{T}}
\def\bbU{\mathbb{U}}
\def\bbV{\mathbb{V}}
\def\bbW{\mathbb{W}}
\def\bbX{\mathbb{X}}
\def\bbY{\mathbb{Y}}
\def\bbZ{\mathbb{Z}}

\def\bsalpha{\boldsymbol{\alpha}}
\def\bsbeta{\boldsymbol{\beta}}
\def\bsgamma{\boldsymbol{\gamma}}
\def\bsdelta{\boldsymbol{\delta}}
\def\bsvarepsilon{\boldsymbol{\varepsilon}}
\def\bszeta{\boldsymbol{\zeta}}
\def\bseta{\boldsymbol{\eta}}
\def\bstheta{\boldsymbol{\theta}}
\def\bsvarteta{\boldsymbol{\vartheta}}
\def\bsiota{\boldsymbol{\iota}}
\def\bskappa{\boldsymbol{\kappa}}
\def\bslambda{\boldsymbol{\lambda}}
\def\bsmu{\boldsymbol{\mu}}
\def\bsnu{\boldsymbol{\nu}}
\def\bsxi{\boldsymbol{\xi}}
\def\bspi{\boldsymbol{\pi}}
\def\bsvarpi{\boldsymbol{\varpi}}
\def\bsrho{\boldsymbol{\rho}}
\def\bsvarrho{\boldsymbol{\varrho}}
\def\bssigma{\boldsymbol{\sigma}}
\def\bsvarsigma{\boldsymbol{\varsigma}}
\def\bstau{\boldsymbol{\tau}}
\def\bsupsilon{\boldsymbol{\upislon}}
\def\bsphi{\boldsymbol{\phi}}
\def\bsvarphi{\boldsymbol{\varphi}}
\def\bschi{\boldsymbol{\chi}}
\def\bspsi{\boldsymbol{\psi}}
\def\bsomega{\boldsymbol{\omega}}

\def\bsGamma{\boldsymbol{\Gamma}}
\def\bsDelta{\boldsymbol{\Delta}}
\def\bsTheta{\boldsymbol{\Theta}}
\def\bsLambda{\boldsymbol{\Lambda}}
\def\bsXi{\boldsymbol{\Xi}}
\def\bsPi{\boldsymbol{\Pi}}
\def\bsSigma{\boldsymbol{\Sigma}}
\def\bsUpsilon{\boldsymbol{\Upsilon}}
\def\bsPhi{\boldsymbol{\Phi}}
\def\bsPsi{\boldsymbol{\Psi}}
\def\bsOmega{\boldsymbol{\Omega}}


\def\lu#1{\sethlcolor{orange}\hl{#1}}
\def\lufn#1{\sethlcolor{orange}\hl{\footnote{\hl{#1}}}}
\def\ludovic#1{\sethlcolor{green}\hl{#1}}
\def\ludovicfn#1{\sethlcolor{green}\hl{\footnote{\hl{#1}}}}
% End added


\DeclareMathOperator{\dom}{dom}
\newtheorem{theorem}{Theorem}[section]
\newtheorem{lemma}[theorem]{Lemma}
\newtheorem{fact}[theorem]{Fact}
\newtheorem{claim}[theorem]{Claim}
\newtheorem{corollary}[theorem]{Corollary}
\newtheorem{proposition}[theorem]{Proposition}

\theoremstyle{definition}
\newtheorem{definition}[theorem]{Definition}
\newtheorem{statement}[theorem]{Statement}
\newtheorem{example}[theorem]{Example}
\newtheorem{xca}[theorem]{Exercise}


\theoremstyle{remark}
\newtheorem{remark}[theorem]{Remark}
\newtheorem {question}[theorem]{Question}
\numberwithin{equation}{section}

\newtheoremstyle{noparens}%
  {}{}%
{}{}%
{\bfseries}{.}%
{ }%
{\thmname{#1}\thmnumber{ #2}\thmnote{ #3}}
\theoremstyle{noparens}
\newtheorem*{question*}{Question}
\newtheorem*{theorem*}{Theorem}

\newcommand{\range}{\operatorname{ran}}

\title{A computable analysis of variable words theorems}



\author{Lu Liu}
\address{Department of Mathematics\\
Central South University\\
ChangSha 410083\\
People’s Republic of China}
%196 Auditorium Road\\ Storrs, Connecticut 06269 U.S.A.}
%\curraddr{}
\email{g.jiayi.liu@gmail.com}

\author{Benoit Monin}
\address{LACL, Département d'Informatique \\
Faculté des Sciences et Technologie \\
61 avenue du Général de Gaulle \\
94010 Créteil Cedex}
%196 Auditorium Road\\ Storrs, Connecticut 06269 U.S.A.}
%\curraddr{}
\email{benoit.monin@computability.fr}

\author{Ludovic Patey}
\address{Institut Camille Jordan\\
Université Claude Bernard Lyon 1\\
43 boulevard du 11 novembre 1918\\
F-69622 Villeurbanne Cedex}
%196 Auditorium Road\\ Storrs, Connecticut 06269 U.S.A.}
%\curraddr{}
\email{ludovic.patey@computability.fr}



%\revauthor{Damir D.Dzhafarov, Denis Hirschfeldt}




%\subjclass[2010]{}
\thanks{}


\def\t{\tilde}
\def\propertyL{\textcolor{red}{2-hyperimmunity}}
\def\hyper{\textcolor{red}{2-hyperimmune}}
%\def\IA{\textcolor{red}{increasingly approximable}}
\def\msf{\mathsf}
%\def\immune{\textcolor{red}{immune}}
\def\mcal{\mathcal}
\def\EF{\t{E}_n,\t{F}_n,n\in\omega}
%\def\CPFB{CPFB}
\def\TBA{\textcolor{green}{TBA}}
\begin{document}

\begin{abstract}
The Carlson-Simpson lemma is a combinatorial statement occurring in the proof of the Dual Ramsey theorem. Formulated in terms of variable words, it informally asserts that given any finite coloring of the strings, there is an infinite sequence with infinitely many variables such that for every valuation, some specific set of initial segments is homogeneous. Friedman, Simpson, and Montalban asked about its reverse mathematical strength. We study the computability-theoretic properties and the reverse mathematics of this statement, and relate it to the finite union theorem. In particular, we prove the Ordered Variable word for binary strings in $\aca_0$. 
\end{abstract}

\maketitle



\section{Introduction}

Let $(\Nb)^k$ and $(\Nb)^\infty$ denote the set of partitions of $\Nb$ into exactly $k$ and infinitely many non-empty pieces, respectively. For $X \in (\Nb)^\infty$, $(X)^k$ is the set of all $Y \in (\Nb)^k$ which are coarser than $X$.

\begin{statement}[Dual Ramsey theorem]
	$\drt^k$ is the statement ``If $(\Nb)^k$ is colored with finitely many Borel colors, then there is some $X \in (\Nb)^\infty$ such that $(X)^k$ is monochromatic''.
\end{statement}

The Dual Ramsey theorem was proven by Carlson and Simpson~\cite{Carlson1984dual},
and studied from a reverse mathematical viewpoint by Slaman~\cite{Slaman1997note}, Miller and Solomon~\cite{Miller2004Effectiveness} and Dzhafarov et al.~\cite{Dzhafarov2017Effectiveness}.
In this paper, we shall focus on a combinatorial lemma used by Carlson and Simpson to prove the Dual Ramsey theorem. This lemma can be formulated in terms of \emph{variable words}.

\begin{definition}[Variable word]
An \emph{infinite variable word} on a finite alphabet $A$
is an $\omega$-sequence $W$ of elements of $A \cup \{x_i : i \in \Nb\}$
in which all variables occur at least once, and finitely often. Moreover, the first occurrence of $x_i$ comes before the first occurrence of $x_{i+1}$.
A \emph{finite variable word} is an initial segment of an infinite variable word. A finite or infinite variable word is \emph{ordered} if moreover all occurences of $x_i$ come before any occurrence of $x_{i+1}$.
Given $\bar a = a_0a_1 \dots a_{k-1} \in A^{<\omega}$, we let $W(\bar a)$
denote the finite $A$-string obtained by replacing $x_i$ with $a_i$ in $W$ and then truncating the result
just before the first occcurence of $x_k$.
\end{definition}

\begin{statement}[Variable word theorem]
$\vw{n}{r}$ is the statement ``If $A^{<\omega}$ is colored with $r$ colors for some alphabet $A$
of cardinality $n$, there exists
an infinite variable word $W$ such that $\{W(\bar a) : \bar a \in A^{<\omega}\}$ is monochromatic.
$\ovw{n}{r}$ is the same statement as $\vw{n}{r}$ but for ordered variable words.
\end{statement}

In this paper, we study the computability-theoretic properties of the variable word theorems using the framework of reverse mathematics.\footnote{The authors thank Damir Dzhafarov, Stephen Flood, Reed Solomon and Linda Brown Westrick for bringing the attention of the authors to the Carlson-Simpson lemma, and for  numerous discussions. The authors are also thankful to Denis Hirschfeldt and Barbara Csima for showing them how to use Lovasz Local Lemma to prove lower bounds to combinatorial theorems.}

\subsection{Reverse mathematics}
Reverse mathematics is a vast foundational program aiming to determine the optimal axioms to prove ordinary theorems. It uses the framework of second-order arithmetic, with a base theory $\rca$ consisting of the axioms of Robinson arithmetic, the $\Sigma^0_1$ induction scheme and the $\Delta^0_1$ comprehension scheme. The system $\rca$ arguably captures \emph{computable mathematics}. Starting from a proof-theoretic perspective, modern reverse mathematics tends to be seen as a framework to analyse the computability-theoretic features of theorems. Among the distinguished statements, let us mention weak K\"onig's lemma ($\wkl$), asserting that every infinite binary tree has an infinite path, the arithmetic comprehension axiom ($\aca$), and the $\Pi^1_1$ comprehension axiom ($\piooca$), consisting of the comprehension scheme restricted to arithmetic and $\Pi^1_1$ formulas, respectively. See Simpson~\cite{Simpson2009Subsystems} for reference book on classical reverse mathematics.

The statements studied within this framework are mainly of the form $(\forall X)[\Phi(X) \rightarrow (\exists Y)\Psi(X, Y)]$, where $\Phi$ and $\Psi$ are arithmetic formulas with set parameters, and can be considered as \emph{problems}. Given a statement $\Psf$ of this form,
	a set $X$ such that $\Phi(X)$ holds is an \emph{instance} of $\Psf$, and a set $Y$ such that $\Psi(X, Y)$ holds is a \emph{solution} to the $\Psf$-instance $X$. In this paper, we shall consider exclusively statements of this kind.

Friedman and Simpson~\cite{Friedman2000Issues}, and later Montalban~\cite{Montalban2011Open}, asked about the reverse mathematical strength of the ordered variable word. The statement $\ovw{k}{\ell}$ is known to be provable in $\rca + \piooca$. Our main result is a direct combinatorial proof of $\ovw{2}{\ell}$ in $\rca + \aca$.

\begin{theorem}\label{thm:aca-ovw2}
	For every $\ell \geq 2$, $\rca + \aca \vdash \ovw{2}{\ell}$.
\end{theorem}

On the lower bound hand, Miller and Solomon~\cite{Miller2004Effectiveness} constructed a computable instance $c$ of $\ovw{2}{2}$ with no $\Delta^0_2$ solution, and deduced that $\rca + \wkl$ does not prove $\vw{2}{2}$. Indeed, seeing the instance $c$ of $\ovw{2}{2}$ as an instance of $\vw{2}{2}$, and noticing that the jump of a solution to $\vw{2}{2}$ gives a solution to $\ovw{2}{2}$, one can deduce that $c$ has no low $\vw{2}{2}$-solution. In this paper, we improve their lower bound by constructing a computable instance of $\ovw{2}{2}$ whose solutions are of DNC degree relative to $\emptyset'$.

\subsection{Organization of the paper}
In section~\ref{sect:ht-ovw}, we shall give a simple proof of the ordered variable word for binary strings ($\ovw{2}{\ell}$) using the finite union theorem. Then, in section~\ref{sect:ovw-aca}, we provide a direct combinatorial proof of the same statement over $\rca + \aca$.
Finally, in section~\ref{sect:ovw-lower-bounds}, we give a new lower bound on the strength of $\ovw{2}{\ell}$ using a computable version of Lovasz Local Lemma.
%Finally, in section~\ref{sect:fut-lower-bounds}, we improve a lower bound of Townser about a combinatorial lemma in his proof of the finite union theorem.

\subsection{Notation}

Given two sets $A$ and $B$, we write $A < B$ for the formula $(\forall x \in A)(\forall y \in B) x < y$.
Given a set $A$, we write $A^{<\omega}$ for the set of finite $A$-valued strings. In particular, $2^{<\omega}$ is the set of binary strings. We denote by $\Pcal_{fin}(\Nb)$ the collection of finite \emph{non-empty} subsets of $\Nb$. Given two strings $\sigma, \tau \in A^{<\omega}$, $\sigma*\tau$ denotes their concatenation. We may also write $\sigma\tau$ when there is no ambiguity. Given a string or a sequence $X$ and some $n \in \omega$, we write $X \uhr n$ for the initial segment of $X$ of length $n$. In particular, $X \uhr 0$ is the empty string, written $\varepsilon$.


\section{A simple proof of the Ordered Variable Word theorem from the Finite Union Theorem}\label{sect:ht-ovw}

Simpson first noted a relation between Hindman's theorem and the Carlson-Simpson lemma~\cite{Carlson1984dual}. In this section, we give a formal counterpart to his observation by giving a simple proof of $\ovw{2}{\ell}$ using the Finite Union Theorem, a statement known to be equivalent to Hindman's theorem. A variation of the proof below was used by Dzhafarov et al.~\cite{Dzhafarov2017Effectiveness} to give an upper bound to the Open Dual Ramsey's theorem. A direct combinatorial proof of $\ovw{2}{\ell}$ in $\rca + \aca$ will be given in the next section. 

\begin{definition}
An \emph{IP collection} is an infinite collection of finite sets $\Ical \subseteq \Pcal_{fin}(\Nb)$
which is closed under \emph{non-empty} finite unions and contains an infinite subcollection
of pairwise disjoint sets.
\end{definition}

Note that any IP collection $\Ical$ necessarily contains an infinite $\Ical$-computable sequence $S_0 < S_1 < \dots$.

\begin{statement}[Finite union theorem]
For every $\ell \in \Nb$, $\fut_\ell$ is the statement ``For every coloring $c : \Pcal_{fin}(\Nb) \to \ell$,
there is a monochromatic IP collection''. $\wfut^2_\ell$ is the statement ``For every coloring $c : \Pcal_{fin}(\Nb) \times \Nb \to \ell$,
there is an IP collection $\Ical$ and a color $i < \ell$ such that $c(S, \min T) = i$ for every $S < T \in \Ical$.''
\end{statement}

\begin{theorem}
$\rca \vdash \forall \ell(\fut_\ell \to \wfut^2_\ell)$.
\end{theorem}
\begin{proof}
Assume $\ell \geq 2$, the other cases being trivial.
Let $f : \Pcal_{fin}(\Nb) \times \Nb \to \ell$ be an instance of $\wfut^2_\ell$.
Note that over $\rca$, $\fut_\ell \imp \aca$ and $\aca \imp \coh$.
Let $\vec{R}$ be a sequence of set defined for every $S \in \Pcal_{fin}(\Nb)$ and $i < \ell$
by $R_{S,i} = \{ n \in \Nb : f(S, n) = i \}$.
Apply $\coh$ to $\vec{R}$ to obtain an infinite $\vec{R}$-cohesive set $C$.
In particular, for every $S \in \Pcal_{fin}(\Nb)$, $\lim_{n \in C} f(S, n)$ exists.

Let $h : \omega \to C$ be a computable bijection.
Let $\tilde{f} : \Pcal_{fin}(\Nb) \to \ell$ be defined by $\tilde{f}(S) = \lim_{n \in C} f(h[S], n)$.
$\tilde{f}$ is a $\Delta^{0,f \oplus C}_2$ instance of $\fut_\ell$, so by the finite union theorem, there is an IP collection $\Ical \subseteq \Pcal_{fin}(\Nb)$.
and a color $i < \ell$ such that 
for every $S \in \Ical$, $\tilde{f}(S) = \lim_{n \in C} f(h[S], n) = i$. Note that for every $S \in \Ical$, $\min h[S] \in C$.
Therefore, by $f$-computably 
%\benoit{$f$-computably should be $f$-$\Delta^0_2$ ?} 
thinning-out the set $\Ical$, we obtain an IP collection $\Jcal \subseteq \Ical$
such that for every $S < T \in \Jcal$, $f(h[S], \min h[T]) = i$.
The set $\{h[S] : S \in \Jcal\}$ is a solution to $f$.
%
%Let $\Ical = \{ S \in \Pcal_{fin}(\Nb) : \min S \in C \}$. Note that $\Ical$ is an IP collection.
%Let $\tilde{f} : \Ical \to \ell$ be defined by $\tilde{f}(S) = \lim_{n \in C} f(S, n)$.
%$\tilde{f}$ is a $\Delta^{0,f \oplus C}_2$ instance of $\fut_\ell$, so by the finite union theorem, there is
%an IP collection $\Jcal \subseteq \Ical$ 
%%\benoit{Why do we have $\Jcal \subseteq \Ical$ ?} 
%and a color $i < \ell$ such that 
%for every $S \in \Jcal$, $\tilde{f}(S) = \lim_{n \in C} f(S, n) = i$. Note that for every $S \in \Jcal$, $\min S \in C$.
%Therefore, by $f$-computably 
%%\benoit{$f$-computably should be $f$-$\Delta^0_2$ ?} 
%thinning-out the set $\Jcal$, we obtain an IP collection $\Kcal \subseteq \Jcal$
%such that for every $S < T \in \Kcal$, $f(S, \min T) = i$.
\end{proof}

\begin{theorem}
$\rca \vdash \forall \ell(\wfut^2_\ell \to \ovw{2}{\ell})$.
\end{theorem}
\begin{proof}
Let $f : 2^{<\omega} \to \ell$ be an instance of $\ovw{2}{\ell}$.
Define an instance $g : \Pcal_{fin}(\Nb) \times \Nb \to \ell$ of $\wfut^2_\ell$ as follows:
Given some $S \in \Pcal_{fin}(\Nb)$ and $n \in \Nb$, if $\max S < n$, then set $g(S, n) = f(\sigma)$,
where $\sigma$ is the binary string of length $n$ defined by $\sigma(i) = 1$ iff $i \in S$.
If $n \leq \max S$, set $g(S, n) = 0$. By $\wfut^2_\ell$, there is an IP collection $\Ical$
and a color $i < \ell$ such that $g(S, \min T) = i$ for every $S < T \in \Ical$.
Compute from $\Ical$ an infinite increasing sequence of pairwise disjoint finite sets
$F_0 < F_1 < \dots$ Let $W$ be the infinite variable word defined by 
$$
W(n) = \left\{\begin{array}{ll}
	1 & \mbox{ if } n \in F_0\\
	x_i & \mbox{ if } n \in F_i \mbox{ for some } i \geq 1\\
	0 & \mbox{ otherwise}	
\end{array}\right.
$$
The variable word $W$ and the sequence of the $F$'s is a solution to the instance $f$ of $\ovw{2}{\ell}$.
\end{proof}

%Note that if we impose infinite variable words to contains only the bit $0$ or variables that take values in $\{0,1\}$, as it is the case for the instance constructed in the previous theorem, then $\ovw{2}{\ell}$ is equivalent to Hindman's theorem over~$\rca$.

\begin{corollary}
$\rca \vdash \aca^{+} \to \forall \ell \ovw{2}{\ell}$.
\end{corollary}
\begin{proof}
Immediate since $\aca^{+} \imp \forall \ell \fut_\ell \imp \forall \ell \wfut^2_\ell \imp \forall \ell \ovw{2}{\ell}$ over~$\rca$.
\end{proof}


\section{A proof of the Ordered Variable Word theorem in ACA}\label{sect:ovw-aca}

The proof of the previous section gave a very coarse computability-theoretic upper bound of the Ordered Variable Word theorem in terms of $\omega$-jumps. In this section, we give a direct combinatorial proof of $\ovw{2}{\ell}$ in $\rca + \aca$. Actually, every PA degree relative to $\emptyset'$ is sufficient to compute a solution of a computable instance of $\ovw{2}{\ell}$.
	We thereby answer a question of Miller and Solomon~\cite{Miller2004Effectiveness}.

\begin{theorem}\label{ovwth2}
For every $\ell \in\omega$,
every computable instance $c$ of $\ovw{2}{\ell}$,
every PA degree over $\emptyset'$ computes a solution to $c$.
\end{theorem}

A formalization of Theorem~\ref{ovwth2} yields
a proof of Theorem~\ref{thm:aca-ovw2}.

\begin{proof}[Proof of Theorem~\ref{thm:aca-ovw2}]
The proof of Theorem~\ref{ovwth2} can be
formalized within $\rca + \aca$. Indeed, the arguments require only arithmetical induction to be carried out, and
every model of $\rca + \aca$ is a model of
the statement ``For every set $X$, there is a set of PA degree over the jump of $X$.''
\end{proof}

Let us first introduce some notation.
For a finite set $F$ and a string $\sigma \in 2^{<\omega}$
let $\sigma_F$ be the binary string of length $|\sigma|$
defined by $\sigma_F(i) = \sigma(i)$ if $i \not \in F$, and $\sigma_F(i) = 1-\sigma(i)$ otherwise. Let $\leq_{lex}$ denote the shortlex order
on  $\omega^{<\omega}$, that is, the order with the shortest length first, and with the strings of same length sorted lexicographically.

In what follows, fix a coloring $\t{c}:2^{<\omega}\rightarrow \ell$,
and a string $\rho\in 2^{<\omega}$.

The main combinatorial lemma we use is
 Lemma~\ref{ovwprop1}.
As a warm up, we first prove the following
lemma \ref{ovwprop0}, which is a consequence of Lemma~\ref{ovwprop1}
and the proof is somehow similar but
much simpler. In the following lemma, one may think of $\rho_{P'}$ as a finite variable word, where the positions at $\t{P}$ are replaced by a same variable kind.


\begin{lemma}\label{ovwprop0}
For any
$P\subseteq \{0,\cdots,|\rho|-1\}$ with
$(\forall n\in P)[\rho(n) = 0]
\wedge |P|\geq \ell$,
there exist two
subsets $P'<\t{P}$ of $P$
with $\t{P}\ne\emptyset$
such that $\t{c}(\alpha) = \t{c}(\alpha_{\t{P}})$ where $\alpha = \rho_{P'}$.
\end{lemma}
\begin{proof}
Suppose $P = \{p_0 < \cdots < p_{m-1}\}$.
Let $\ell_0, \dots, \ell_m$ be defined by
$\ell_i = \t{c}(\rho_{\{p_0, \dots, p_{i-1}\}})$. In particular, $\ell_0 = \t{c}(\rho)$.
Since $|P| = m \geq \ell$,
so among $\ell_0,\cdots, \ell_m$,
there must exists $i<j$ such
that $\ell_i=\ell_j$.
Let $P' = \{p_0,\cdots,p_{i-1}\}$
(if $i=0$ then $P' = \emptyset$), and
$\t{P} = \{p_i,\cdots,p_{j-1}\}$,
let $\alpha = \rho_{P'}$.
Clearly $P'<\t{P}$ and $\t{P}\ne\emptyset$.
It is also easy to see that
$\t{c}(\alpha) = \ell_i = \ell_j = \t{c}(\alpha_{\t{P}})$.
\end{proof}

We now prove a technical lemma used in the proof of our main combinatorial
lemma (Lemma~\ref{ovwprop1}). The sequence in the following lemma is obtained by a simple greedy algorithm, with finitely many resets.

\begin{lemma}\label{ovwclaim1}
There exists a nonempty set
of colors $L\subseteq \{0,1,\cdots,\ell-1\}$,
$|L|+1$ many sets of binary strings
$\Gamma_0 = \{\tau^{\eta}\}_{\eta\in L},
\Gamma_1=\{\tau^\eta\}_{\eta\in L^2},
\cdots,\Gamma_{|L|}=\{\tau^\eta\}_{\eta\in L^{|L|+1}}$,
such that, letting
$$
\t{\eta} =
 \underbrace{\max L *\max L *\cdots*\max L}_{|L|+1\text{ many }}
$$
 and letting $\t{\rho} = \tau^{\t{\eta}}*0$, the following holds:
\begin{enumerate}
\item $\rho\prec\Gamma_0$ and $
\tau^{\eta}\prec\tau^\beta\Leftrightarrow
\eta<_{lex}\beta$;
\item $\t{\rho}(|\tau|) = 0$
for all $\tau\in \Gamma_i,i\leq |L|$;

\item for all $i\leq |L|$, $\eta\in L^{i+1}$,
let $\eta_0\prec\eta_1\prec\cdots\prec\eta_{i-1}$ denote
all nonempty predecessors of $\eta$, let
$Q = \big\{|\tau^{\eta_0}|,|\tau^{\eta_1}|,
\cdots,|\tau^{\eta_{i-1}}|\big\}$
(if $i=0$ then $Q=\emptyset$),
then $\t{c}(\tau^{\eta}_Q) = \eta(i)$;

\item let $P = \{|\tau^\eta|\}_{\eta\in L^{\leq |L|+1}}$,
for all subset $Q$ of $P$, all $\tau\succeq\t{\rho}$,
$\t{c}\big(\tau_Q\big)
\in L$.
\end{enumerate}
%\benoit{For (4) it seems that more is nedeed in the proof of the next lemma : not only $\t{c}(\tau)$ needs to an element of $L$ for $\tau \succeq \t{\rho}$, but also $\t{c}(\tau_P)$ for some set of positions $P$ (which is the case of course).}
Moreover,
$\Gamma_i,i\leq |L|$ is  computable in the jump of $\t{c}$,
 uniformly in $\rho$.


\end{lemma}
\begin{proof}
We firstly show how to find $\Gamma_0$.
Start with $L = \{0,1,\cdots,\ell-1\}$.
At step 1, try to find a string $\tau\in 2^{<\omega}$ such that
$\t{c}(\rho\tau) =  0$ and
let $\tau^0 = \rho\tau$. Then
try to find a $\tau$ such that
$\t{c}(\tau^00\tau) = 1$ and
let $\tau^1 = \tau^00\tau$. Generally, after $\tau^j$ is found,
try to find $\tau$ such that
$\t{c}(\tau^j0\tau)=j+1$
and let $\tau^{j+1} = \tau^j0\tau$ if
$\tau$ is found.
If during the above process, after $\tau^{j-1}$ is
defined ( $\tau^{-1} = \rho$ ),
there is no $\tau$ such that
$\t{c}(\tau^{j-1}0\tau) = j$,
then we start all over again
with $\rho$ replaced by $\rho_1 =
\tau^j0$ and with $L$ replaced by $ L \setminus \{j\}$.
%, i.e.,
%we try to find a sequence of string
%$\tau^i\succ\rho_1,i\in \{0,\cdots,l-1\}-\{j+1\}$
% with
%$\t{c}(\tau^i) = i$ and $
%i<k\Rightarrow\tau^i\prec \tau^k\wedge
%\tau^k(|\tau^i|) = 0$.

%Let $L_1$ denote the
%colors remaining after the search of $\Gamma_0$.
%Without loss of generality, suppose $L_1 = \{0,\cdots,k^*\}$.
%To find $\Gamma_1$,
%\begin{itemize}
%\item Try to find $\tau^{0i},i\in L_1$ such that
%$\t{c}(\tau^{0i}_{\{|\tau^0|\}}) = i$ and
%$i<k\Rightarrow \tau^{0i}\prec\tau^{0k}
%\wedge \tau^{0k}(|\tau^{0i}|) = 0$.
%
%\item Generally,
%at step $i+1\in L_1$, try to find
%$\tau^{ik},k\in L_1$ with $\tau^{ik}\succ\tau^{(i-1)k^*}0$ such that
%$\t{c}(\tau^{ik}_{\{|\tau^i|\}}) = k$ and
%$j<k\Rightarrow \tau^{ij}\prec\tau^{ik}
%\wedge \tau^{i(k+1)}(|\tau^{ik}|) = 0$.
%\end{itemize}
%
%If during the process, after $\tau^{ik}$ is found,
%$\tau^{i(k+1)}$ can not be found,
%i.e., there exists no $\tau$ such that
%$\t{c}((\tau^{ik}0\tau)_{\{|\tau^i|\}}) = k+1$,
%then let $\rho_2 = (\tau^{ik}0)_{\{|\tau^i|\}}$
%and start all over  again with
%$\rho$ replaced by $\rho_2$ to rebuild $\Gamma_0,\Gamma_1,\cdots$.

Generally, given a set of colors $L$ and after $\tau^{\beta}$ is found,
let $\eta$
be the immediate successor (with respect to $\leq_{lex}$ order restricted to $L$-strings)
 of $\beta$,
let $\eta_0\prec\eta_1\prec\cdots\prec\eta_{i-1}$ denote
all nonempty predecessors of $\eta$, let
$Q = \big\{|\tau^{\eta_0}|,|\tau^{\eta_1}|,
\cdots,|\tau^{\eta_{i-1}}|\big\}$
(if $i=0$ then $Q=\emptyset$),
we try to find $\tau$ such that
$\t{c}((\tau^{\beta}0\tau)_{Q}) = \eta(|\eta|-1)$.
If such a string $\tau$ does not exists then we
start  all over again with $\rho $
replaced by  $\tau^\beta0_{Q}$ and $L$ replaced by $L \setminus \{\eta(|\eta|-1)\}$.
If such $\tau$ exists then let $\tau^\eta =
\tau^\beta0\tau$.

Note that we have to
start over for at most $\ell-1$ times before we
ultimately succeed since
there are $\ell$ colors in total.
It is plain to check all the four items.
Also note that the sequence $\Gamma_0,\cdots,\Gamma_{|L|}$ is
$\t{c}'$-computable since we only need to
use the jump of $\t{c}$ to know whether the next $\tau^{\eta}$ can
 be found.



\end{proof}

\begin{lemma}\label{ovwprop1}

There exists a string $\t{\rho}\succ\rho$
 and a finite set $P\subseteq
 \big\{|\rho|,\cdots, |\t{\rho}|-1\big\}$ with
 $(\forall i\in P)[\t{\rho}(i) = 0]$
 such that
 for all $\sigma\succeq \t{\rho}$ there exists
 two subsets $P'<\t{P}$ of $P$ with $\t{P}\ne\emptyset$
 such that, letting $\alpha = \sigma_{P'}$,
$\t{c}(\alpha) = \t{c}(\alpha_{\t{P}}) = \t{c}(\alpha
 \uhr {\min \t{P} })$.
 Moreover, $|P|< \ell^{\ell+2}$, and $\t{\rho}, P$,
are computable in the jump of $\t{c}$, uniformly in $\rho$.
\end{lemma}
\begin{proof}
Let $L$ and $\t{\rho}$ satisfy Lemma~\ref{ovwclaim1}.
We claim that $\t{\rho}$ and $P = \{|\tau^\eta|\}_{\eta\in L^{\leq |L|+1}}$
satisfy the current lemma.
It is clear
by item 1 of Lemma~\ref{ovwclaim1} that $\t{\rho}\succ\rho$
and by item 2 of  Lemma~\ref{ovwclaim1} that
$(\forall i\in P)[\t{\rho}(i) = 0]$.

Fix an arbitrary $\sigma\succeq\t{\rho}$.
We now describe how to construct $P'$ and $\t{P}$.
Define $\ell_0, \dots, \ell_{|L|}$
and $p_0, \dots, p_{|L|}$ inductively by
$\ell_0 = \t{c}(\sigma)$,
$\ell_{i+1} = \t{c}(\sigma_{\{p_0,p_1,\cdots,p_i\}})$, and
$p_{i} =|\tau^{\ell_0\cdots \ell_{i}}|$ (where $\tau^{\ell_0\cdots \ell_{i}} \in \Gamma_i$).
%To construct $P'$ and $\t{P}$,
%consider the following  flipping bit process:
%
%
%\begin{itemize}
%\item Suppose $\t{c}(\sigma) = \ell_0$. By item 4
%of Lemma~\ref{ovwclaim1}, $\ell_0\in L$. Therefore
%$\tau^{\ell_0}$ is defined.
%Then, let
%$p_0 = |\tau^{\ell_0}|$, at step 1 imagine
%we flip the
%bit at position $p_0$ and $\sigma$ becomes
%$\sigma_{\{p_0\}}$.
%
%\item Suppose
%$\t{c}(\sigma_{\{p_0\}}) = \ell_1$.
%Then, let $p_1 = |\tau^{\ell_0\ell_1}|$, at step 2
%imagine we flip the bit  at position $p_1$ and
%$\sigma_{\{p_0\}}$ becomes $\sigma_{\{p_0,p_1\}}$.
%
%\item Suppose $\t{c}(\sigma_{\{p_0,p_1\}}) = \ell_2$.
%Then, let $p_2=|\tau^{\ell_0\ell_1\ell_2}|$, at  step 3
%imagine we flip the bit at position $p_2$
%and $\sigma_{\{p_0,p_1\}}$ becomes
%$\sigma_{\{p_0,p_1,p_2\}}$.
%
%\item In the $(i+1)^{th}$ step where $1\leq i\leq |L|$, suppose
%$\t{c}(\sigma_{\{p_0,p_1,\cdots,p_{i-1}\}}) = \ell_i$.
%Then, let $p_{i} =|\tau^{\ell_0\cdots \ell_{i}}|$
% imagine we flip the bit at position $p_{i}$
% and $\sigma_{\{p_0,\cdots,p_{i-1}\}}$ becomes
%$\sigma_{\{p_0,\cdots,p_{i-1},p_{i}\}}$.
%
%\end{itemize}
Since $\ell_0,\cdots,\ell_{|L|}\in L$ (by item 4 of Lemma~\ref{ovwclaim1}), there is some $i<j\leq |L|$
such that $\ell_i=\ell_j $.
Let $P' = \{p_0,\cdots,p_{i-1}\}$
(if $i=0$ then $P' = \emptyset$),
$\t{P}= \{p_i,\cdots,p_{j-1}\}$,
and let $\alpha = \sigma_{P'}$. We claim that
$\t{c}(\alpha) = \t{c}(\alpha_{\t{P}}) = \t{c}(\alpha\uhr \min \t{P})$.
Note that $\min \t{P}  =p_i= |\tau^{\ell_0\cdots \ell_i}|$.
Therefore
$\alpha\uhr \min \t{P} = \tau^{\ell_0\cdots \ell_i}_{P'}$.
By item 3 of Lemma~\ref{ovwclaim1},
we have $\t{c}(\tau^{\ell_0\cdots \ell_i}_{P'}) = \ell_i $.
Meanwhile, by definition of $\ell_i$,
$\t{c}(\sigma_{P'}) =
\t{c}(\alpha) = \ell_i$.
By definition of $\ell_j$,
$\t{c}(\sigma_{P'\cup \t{P}}) =
\t{c}(\alpha_{\t{P}}) = \ell_j$.
Thus, $\t{c}(\alpha) = \t{c}(\alpha_{\t{P}})
 = \t{c}(\alpha\uhr \min \t{P} )$.


\end{proof}

We say that $(\t{\rho}, P)$ is \emph{$\t{c}$-valid}
if $P$ and $\t{\rho}$ satisfy Lemma~\ref{ovwprop1}.
We say that $(P',\t{P})$ \emph{witnesses $\t{c}$-validity of $(\t{\rho}, P)$ for $\sigma \succeq \t{\rho}$} if $P' < \t{P}  \subseteq P$, and letting $\alpha = \sigma_{P'}$,
$\t{c}(\alpha) = \t{c}(\alpha_{\t{P}}) = \t{c}(\alpha
 \uhr {\min \t{P} })$.
Before proving Theorem~\ref{ovwth2}, we start with
the following simpler version.

\begin{theorem}\label{ovwth1}
For every $\ell\in\omega$, every computable instance  $c:2^{<\omega}\rightarrow
\ell$ of $\ovw{2}{\ell}$,
every $PA$ degree over $\emptyset''$ computes
a solution to $c$.
\end{theorem}
\begin{proof}
It suffices to compute, given a PA degree relative to $\emptyset''$,
an infinite binary sequence $Y\in 2^\omega$
together with a sequence of finite
sets $\t{P}_0<\t{P}_{1}<\cdots$ with
$(\forall i\in\omega)(\forall n\in \t{P}_i)
[Y(n) = 0]$ such that the following holds:
\begin{quote}
Let $Position = \big\{\min \t{P}_i : i \geq 1 \big\}$.
There is some $\t{\ell}<\ell$ such that for all subset $J$ of $\omega$, letting
 $\t{P}_J = \bigcup\limits_{i\in J}\t{P}_i$, then we have,
 $
(\forall p\in Position)\big[
c(Y_{\t{P}_J}\uhr p) = \t{\ell}\ ].
$ 	
\end{quote}


Using Lemma~\ref{ovwprop1}, we first construct
a $\emptyset'$-computable sequence of strings
 $\t{\rho}_0 \prec \t{\rho}_1 \prec\cdots$,
a sequence of finite sets $P_i\subseteq \big\{
|\t{\rho}_{i-1}|,\cdots, |\t{\rho}_i|-1
\big\}$ and a sequence of colorings $c_i:[\t{\rho}_i]^\preceq\rightarrow
L_i$ inductively as follows.
$\t{\rho}_0 = \varepsilon$ and $c_0 = c$.
Given $\t{\rho}_i$ and $c_i:[\t{\rho}_i]^\preceq\rightarrow
L_i$, let $\t{\rho}_{i+1} \succeq \t{\rho}_i$
and $P_i\subseteq
\big\{ |\t{\rho}_i|,\cdots,|\t{\rho}_{i+1}|-1\big\}$ be such that
$(\t{\rho}_{i+1}, P_i)$- is $c_i$-valid,
and let $c_{i+1}$ be the coloring of $[\t{\rho}_{i+1}]^{\preceq}$ which on $\sigma \succeq \t{\rho}_{i+1}$
associates $\langle P', \t{P}, j \rangle$ such that
$(P',\t{P})$ witnesses $c_i$-validity of $(\t{\rho}_{i+1}, P_i)$
for $\sigma$, and $c_i(\sigma_{P'}) = j$. If there are multiple such tuples, take the least one, in some arbitrary order. Note that the range of $c_i$ is some finite set $L_i$.
%
% as following:
%\begin{itemize}
%\item Let $\t{\rho}_1$, $P_0\subseteq
%\big\{ |\t{\rho}_0|,\cdots,|\t{\rho}_1|-1\big\}$ be such that:
%$(\forall n\in P_0)[\t{\rho}_1(n) = 0]$,
%for all $\sigma\succeq \t{\rho}_1$ there exists
%subsets $P'<\t{P}$ of $P_0$ with $\t{P}\ne\emptyset$ such that
%let $\alpha = \sigma_{P'}$, $c_0(\alpha) = c_0(\alpha_{\t{P}})$.
%By Lemma~\ref{ovwprop0} such $\t{\rho}_1, P_0$ exists.
%
%\item Now we define $c_1$ according to profile of $c_0$.
%For $\sigma\succeq\t{\rho}_1$,
%$c_1(\sigma) = (P',\t{P},j)$ if:
%$P'<\t{P}$ are subsets of $P_0$, $\t{P}\ne\emptyset$, and
%let $\alpha = \sigma_{P'}$, $c_0(\alpha) = c_0(\alpha_{\t{P}})=j$.
%If multiple $(P',\t{P},j)$ satisfy the above condition,
%then simply let $c_1(\sigma)$ be the "smallest" such $(P',\t{P},j)$.
%Here "smallest" is an arbitrary order.
%Clearly the range of $c_1$, namely $L_1$, is finite
%and by definition of $\t{\rho}_1, P_0$, $c_1$ is well defined
%on $[\t{\rho}_1]^\preceq$.
%
%\item Note that $c_1$ is computable since
%$c_0$ is computable. Therefore, apply proposition \ref{ovwprop1} on
%$c_1,\t{\rho}_1$, we can $\mbf{0'}$-compute $\t{\rho}_2\succ\t{\rho}_1$
%and $P_1\subseteq \big\{|\t{\rho}_1|+1,\cdots, |\t{\rho}_2|-1\big\}$
% with $(\forall n\in P_1)[\t{\rho}_2(n) = 0]$ such that:
%for all $\sigma\succeq \t{\rho}_2$ there exists
%subsets $P'<\t{P}$ of $P_1$ with $\t{P}\ne\emptyset$ such that
%let $\alpha = \sigma_{P'}$, $c_1(\alpha) = c_1(\alpha_{\t{P}})
% = c_1(\alpha\uhr_{\min\{\t{P}\}})$.
%
% \item Define $c_2$ according to profile
% of $c_1$. For $\sigma\succeq \t{\rho}_2$,
% $c_2(\sigma) = (P',\t{P},j)$ if:
%$P'<\t{P}$ are subsets of $P_1$, $\t{P}\ne\emptyset$, and
%let $\alpha = \sigma_{P'}$, $c_1(\alpha) = c_1(\alpha_{\t{P}})=
%c_1(\alpha\uhr_{\min\{\t{P}\}}) = j$.
%
%\item In this way, we guarantee that for all $i$:
%(a) $(\forall n\in P_{i-1})[\t{\rho}_i(n) = 0]$;
%(b) for any $\sigma\succeq \t{\rho}_i$,
% $c_i(\sigma) = (P',\t{P},j)$ if
% $P'<\t{P}$ are subsets of $P_{i-1}$, $\t{P}\ne\emptyset$, and
%let $\alpha = \sigma_{P'}$, $c_{i-1}(\alpha) = c_{i-1}(\alpha_{\t{P}})=
%c_{i-1}(\alpha\uhr_{\min\{\t{P}\}}) = j$;
%and (c) $c_i$ is well defined with finite range $L_i$.
%
%\end{itemize}

We now analyze for $\sigma\succeq \t{\rho}_i$
 what $c_i(\sigma)= \langle P',\t{P},j \rangle$ means.
Note that elements of $L_i,i\in\omega$
admit a natural partial order $\lhd$ as follows:
for
$\langle P'_{0},\t{P}_0,j_0 \rangle\in L_{i},
\langle P'_{1},\t{P}_1,j_1\rangle \in L_{i+1}$,
$\langle P'_{1},\t{P}_1,j_1 \rangle $ is an immediate
successor of $\langle P'_{0},\t{P}_0,j_0 \rangle$
if and only if $j_1 = \langle P'_{0},\t{P}_0,j_0\rangle$.
Clearly every $j\in L_i$ admit a unique
immediate predecessor.

\begin{claim}\label{ovwclaim2}
Fix some $n \geq 1$, and let
$
\t{\ell} \lhd \langle P'_0,\t{P}_0,j_0\rangle \lhd \dots \lhd \langle P'_{n-1},\t{P}_{n-1},j_{n-1}\rangle = c_n(\sigma)
$,
Let $P' = \bigcup_{i\leq n-1}P'_i$ and
$\alpha = \sigma_{P'}$.
Then for any subset $J$ of $\{0,\cdots, n-1\}$,
$$(\forall p\in \big\{ \min \t{P}_j : 1 \leq j \leq n-1\big\}\cup\{|\alpha|\})
\big[\ c(\alpha_{\t{P}_J}\uhr p)
 = \t{\ell}\ \big].$$
\end{claim}
\begin{proof}
First we prove the claim for $n=1$.
By definition of $c_1(\sigma) = \langle P'_0,\t{P}_0,j_0 \rangle$,
letting $\beta = \sigma_{P'_0}$,
$c_0(\beta) = c_0(\beta_{\t{P}_0})= j_0=\t{\ell}$.
In other words,
 for any subset $J\subseteq \{0\}$,
$$(\forall p\in \big\{ \min\{\t{P}_j : 1 \leq j \leq 0\}\big\} \cup\{|\beta|\})
\big[\ c(\beta_{\t{P}_J}\uhr p)
 = \t{\ell}\ \big].$$
So the claim holds for $n=1$.
Suppose now the claim
 holds for $n-1$.

Suppose $c_n(\sigma) = \langle P'_{n-1},\t{P}_{n-1},j_{n-1} \rangle$. Let $\beta = \sigma_{P'_{n-1}}$. We have $c_{n-1}(\beta) =c_{n-1}(\beta_{\t{P}_{n-1}}) = c_{n-1}(\beta\uhr \min \t{P}_{n-1} ) = j_{n-1} = \langle P'_{n-2},\t{P}_{n-2},j_{n-2}\rangle$. As $c_{n-1}(\beta) = \langle P'_{n-2},\t{P}_{n-2},j_{n-2}\rangle$ and as $\t{\ell} \lhd \langle P'_{n-2},\t{P}_{n-2},j_{n-2}\rangle$, by induction hypothesis, for any subset $J$ of $\{0,\cdots,n-2\}$ we have:
\begin{align}\label{ovweq1}
&c(\beta_{(\cup_{i\leq n-2} P_i')\cup \t{P}_J}) = \t{\ell}.
\end{align}

Let $\beta' = \beta_{\t{P}_{n-1}}$. As $c_{n-1}(\beta') = \langle P'_{n-2},\t{P}_{n-2},j_{n-2}\rangle$ and as $\t{\ell} \lhd \langle P'_{n-2},\t{P}_{n-2},j_{n-2}\rangle$, by induction hypothesis, for any subset $J$ of $\{0,\cdots,n-2\}$ we have:
\begin{align}\label{ovweq2}
&c(\beta'_{(\cup_{i\leq n-2} P_i')\cup \t{P}_J}) = \t{\ell}.
\end{align}

As $c_{n-1}(\beta\uhr \min \t{P}_{n-1} ) = \langle P'_{n-2},\t{P}_{n-2},j_{n-2}\rangle$ and as $\t{\ell} \lhd \langle P'_{n-2},\t{P}_{n-2},j_{n-2}\rangle$, by induction hypothesis, for any subset $J$ of $\{0,\cdots,n-2\}$ we have:
\begin{align}\label{ovweq4}
&(\forall p \in\big\{ \min \t{P}_j : 1\leq j\leq n-2\big\}\cup \big\{|\beta\uhr \min\t{P}_{n-1}|\big\})
\big[\ c(\beta_{(\cup_{i\leq n-2} P_i')\cup\t{P}_J}\uhr p) = \t{\ell}\ \big].
\end{align}
But $|\beta\uhr \min\t{P}_{n-1}| = \min \t{P}_{n-1}$. So (\ref{ovweq4}) means for any subset $J$ of $\{0,\cdots,n-2\}$ we have:
\begin{align}\nonumber
&(\forall p \in\big\{ \min \t{P}_j : 1\leq j\leq n-1\big\})
\big[\ c(\beta_{(\cup_{i\leq n-2} P_i')\cup\t{P}_J}\uhr p) = \t{\ell}\ \big].
\end{align}
Or equivalently, for any subset $J$ of $\{0,\cdots,n-1\}$ we have:
\begin{align}\label{ovweq3}
&(\forall p \in\big\{ \min \t{P}_j : 1\leq j\leq n-1\big\})\big[\ c(\beta_{(\cup_{i\leq n-2} P_i')\cup\t{P}_J}\uhr p) = \t{\ell}\ \big].
\end{align}

Now from \ref{ovweq1}, \ref{ovweq2} and \ref{ovweq3} we deduce that for any subset $J$ of $\{0,\cdots,n-1\}$ we have:
$$(\forall p\in \big\{ \min \t{P}_j : 1 \leq j \leq n-1\big\}\cup\{|\beta|\})\big[\ c(\beta_{(\cup_{i\leq n-2} P_i')\cup\t{P}_J}\uhr p) = \t{\ell}\ \big]$$
which completes the proof of the claim since
$\beta_{\cup_{i\leq n-2} P_i'} = \alpha$.

%Note that if $c_n(\sigma) = \langle P'_{n-1},\t{P}_{n-1},j_{n-1} \rangle$,
%then, letting $\beta = \sigma_{P'_{n-1}}$, we have
%$c_{n-1}(\beta) =c_{n-1}(\beta_{\t{P}_{n-1}}) =
%c_{n-1}(\beta\uhr \min \t{P}_{n-1} ) = j_{n-1}
% = \langle P'_{n-2},\t{P}_{n-2},j_{n-2}\rangle$.
%Let
% $P'' = \bigcup_{i\leq n-2}P'_i$,
% $\t{\beta} = \beta_{P''}$, $\beta' =
%  \t{\beta}\uhr \min \t{P}_{n-1}$,
%since
% $\t{\ell}\in L_0$ is also predecessor of $\langle P'_{n-2},\t{P}_{n-2},j_{n-2}\rangle$,
% by induction hypothesis,
% for any subset
% $J$ of $\{0,\cdots,n-2\}$,
%\begin{align}\label{ovweq1}
%&(\forall p \in\big\{ \min \t{P}_j : 1\leq j\leq n-2\big\}\cup\{|\t{\beta}|\})
%\big[\ c(\t{\beta}_{\t{P}_J}\uhr p) = \t{\ell}\
%\big],
%\\ \nonumber
%&(\forall p \in\big\{ \min \t{P}_j : 1\leq j\leq n-2\big\}\cup\{|\t{\beta}|\})
%\big[\
%c(\t{\beta}_{\t{P}_{n-1}\cup \t{P}_J}\uhr p)
%=\t{\ell}\ \big],
%\\ \nonumber
%&\ c(\beta'_{ \t{P}_J})
%=\t{\ell}.
%\end{align}
%But (\ref{ovweq1}) clearly implies
%for any subset $J$ of $\{0,\cdots,n-1\}$,
%$$
%(\forall p\in \big\{ \min \t{P}_j : 1\leq j\leq n-1\big\}\cup\{|\t{\beta}|\})
%\big[
%\ c(\t{\beta}_{\t{P}_J}\uhr p) = \t{\ell}
%\big].
%$$
%Since $\t{\beta} = \alpha$ thus the proof of the claim is completed.
\end{proof}

Let $\mcal{T}_0$ be the $\emptyset'$-computable set of all $\gamma$ such that
$(\forall i\leq |\gamma|)[\gamma(i)\in L_i]$,
$\gamma(i) \lhd \gamma(i+1)$ and $\gamma(|\gamma|-1) = c_{|\gamma|-1}(\t{\rho}_{|\gamma|})$.
Then, let $\mcal{T}$ be the downward closure of the set $\mcal{T}_0$
by the prefix relation. The tree $\mcal{T}$ is infinite by construction of the strings $\t{\rho_i}$, the colors $c_i$ and the sets $P_i$ : a witness for the $c_i$-validity of $(\t{\rho}_{i+1}, P_{i+1})$ for $\rho_{i+1}$ yields a node of $\mcal{T}_0$ of length $i+2$. The tree $\mcal{T}$ is also $\emptyset'$-computably bounded, and $\emptyset''$-computable.
Let $j_0*\langle P'_0,\t{P}_0,j_0\rangle *\langle P'_1,\t{P}_1,j_1\rangle *\cdots$
be an infinite path through $\mcal{T}$ computed by any PA degree over $\emptyset''$.
By construction, $\langle P'_i,\t{P}_i,j_i\rangle \lhd \langle P'_{i+1},\t{P}_{i+1},j_{i+1} \rangle$.
Let $X = \bigcup_{i\in\omega}\t{\rho}_i$,
$P' = \bigcup_{i\in\omega} P'_i$
and let $Y = X_{P'}$.
Clearly  $(\forall i \forall n\in \t{P}_i)[Y(n) = 0]$
and $Y$ is computable in the given PA degree relative to $\emptyset''$.
Therefore,
letting $Position = \big\{\min \t{P}_i : i \geq 1\big\}$, it suffices
to show that
for all subsets $J$ of $\omega$,
$$
(\forall p\in Position)\big[
c(Y_{\t{P}_J}\uhr p) = j_0\ ].
$$

Without loss of generality,
suppose $p = \min \t{P}_n$
and $J\subseteq \{0,\cdots,n-1\}$.
Since $j_0*\langle P'_0,\t{P}_0,j_0\rangle*\langle P'_1,\t{P}_1,j_1\rangle*\cdots
 \langle P'_n,\t{P}_n,j_n\rangle$ is an initial segment of some element in
 $\mathcal{T}_0$,
there must exist some
$N>n$
 such that
 $c_N(\t{\rho}_{N+1}) = \langle P'_{N-1},\t{P}_{N-1},j_{N-1}\rangle$.
 Let $\sigma = \t{\rho}_{N+1}, \alpha =\sigma_{P'}$. Clearly
 $\alpha\prec Y\wedge |\alpha|>p$. Moreover,
 by Claim~\ref{ovwclaim2},
 $c(\alpha_{\t{P}_J}\uhr p) = j_0$.
 Thus $c(Y_{\t{P}_J}\uhr p) = j_0$.

\end{proof}

Finally, we slightly modify the proof of Theorem~\ref{ovwth1}
to derive Theorem~\ref{ovwth2}.

\begin{proof}[Proof of Theorem~\ref{ovwth2}]
The main point is to make the tree
$\mcal{T}$ $\emptyset'$-computable.
To ensure this, after we obtain
$\t{\rho}_i,c_i$, we do not
directly go to $\t{\rho}_{i+1}$.
Instead, we $\emptyset'$-compute
$\t{\rho}_i^{0}\prec\t{\rho}_i^1\prec\cdots
\prec\t{\rho}_i^{r_i}$
such that $\t{\rho}_i^0\succ\t{\rho}_i$ and
$c_i\big(
\{\tau:\tau\succeq \t{\rho}_i^{r_i} \}\big)
 \subseteq c_i\big(\big\{\t{\rho}_i^0,\cdots,\t{\rho}^{r_i}_i \big\}
 \big)$. Then we $\emptyset'$-compute $\t{\rho}_{i+1}\succ
 \t{\rho}_i^{r_i}$ as in the proof of Theorem~\ref{ovwth1}.
 Note that this indeed can be
 achieved using $\emptyset'$ since
 $c_i$ is computable.
 Define $\mcal{T}$ to be
 the set of all $\gamma$ such that
 $(\forall i\leq |\gamma|)[\gamma(i)\in L_i]$,
$\gamma(i) \lhd \gamma(i+1)$,
 and either $|\gamma| = 1\wedge \gamma\in L_0$ or
 there exists $\t{\rho}_{|\gamma|-1}^u$
 with $c_{|\gamma|-1}(\t{\rho}_{|\gamma|-1}^u)
 =\gamma(|\gamma|-1)$.
 It is easy to see
that $\mcal{T}$ is $\emptyset'$-computable
since $c_i$ is computable for all $i$ and
the sequences $\langle c_i : i \in \omega \rangle$ and
$\langle \t{\rho}_i^v : i\in\omega, v\leq r_i \rangle$
are $\emptyset'$-computable.

Now we show that $\mcal{T}$ is a tree.
Suppose $\gamma\in \mcal{T}$,
$|\gamma| = n+1$ with $n \geq 1$, and  $c_n(\t{\rho}_n^u)
 =\gamma(n) = \langle P',\t{P},j \rangle \in L_n$.
 We claim that $\gamma \uhr n \in \mcal{T}$.
 If $n = 1$, then $\gamma \uhr 1 \in L_0 \subseteq \mcal{T}$.
 Otherwise, let
$\langle Q',\t{Q},k \rangle \in L_{n-1}$ be the predecessor
  of $\langle P',\t{P},j \rangle$.
We need to show that
there exists $\t{\rho}_{n-1}^v$ such that
$c_{n-1}(\t{\rho}_{n-1}^v) = \langle Q',\t{Q},k \rangle$.
$c_n(\sigma) = \langle P',\t{P},j \rangle$ implies
that, letting $\alpha = \sigma_{P'}$,
$c_{n-1}(\alpha) = c_{n-1}(\alpha\uhr \min \t{P})
 = j = \langle Q',\t{Q},k \rangle$.
Note that $\alpha\succeq \t{\rho}_{n-1}^{r_{n-1}}$
since $P'> |\t{\rho}_{n-1}^{r_{n-1}}|$.
But $c_{n-1}\big( \{\tau:\tau\succeq \t{\rho}_{n-1}^{r_{n-1}}\}\big)
\subseteq c_{n-1}\big(\{\t{\rho}_{n-1}^0,\cdots,\t{\rho}_{n-1}^{r_{n-1}}\} \big)$.
Therefore there exists $\t{\rho}_{n-1}^u$ such that
$c_{n-1}(\t{\rho}_{n-1}^u) = \langle Q',\t{Q},k \rangle$.
It follows that $\gamma \uhr n \in \mcal{T}$ and that $\mcal{T}$ is a tree.
%Reapply the argument with $\sigma$ replaced
%by $\t{\rho}^u_{n-1}$, $c_n$ replaced by $c_{n-1}$
%and $(P',\t{P},j)$ replaced by $(Q'_{n-1},\t{Q}_{n-1},k_{n-1})$,
%since $\t{\rho}^u_{n-1}\succ \t{\rho}^{r_{n-2}}_{n-2}$,
%we have that for some $\t{\rho}^r_{n-2}$,
% $c_{n-2}(\t{\rho}^r_{n-2})
% = (Q'_{n-2},\t{Q}_{n-2},k_{n-2})$.
% Reapplying the argument $n-m$ times produces
% a $\t{\rho}_{m}^v$ with
% $c_m(\t{\rho}_m^v) = (Q',\t{Q},k)$.
Any PA degree relative to $\emptyset'$ computes an infinite path through $\mcal{T}$. The rest of the proof goes exactly the same
 as Theorem~\ref{ovwth1}.
\end{proof}

We now give an alternative proof of Theorem~\ref{ovwth2} based on the definitional complexity of the solutions of~$c$.

\begin{proof}[Second proof of Theorem~\ref{ovwth2}]
Let $P_0, P_1, \dots$ be the $\emptyset'$-computable sequence defined in the proof of Theorem~\ref{ovwth1}. We have seen that there exists an infinite ordered variable word such that the $n$th variable kind appears before the position $\max P_n$. Let $\mcal{T}$ be the tree of all finite ordered variable words which are finite solutions to $c$ and such that the $n$th variable appears before the position $\max P_n$. By the previous observation, the tree is infinite, $\emptyset'$-computable, and $\emptyset'$-computably bounded. Any PA degree relative to $\emptyset'$ computes an infinite variable word which, by construction of $\mcal{T}$, is a solution to $c$. This completes the proof of Theorem~\ref{ovwth2}.
\end{proof}

Note that the above proof can be slightly modified to obtain a proof of a sequential version of the ordered variable word.

\begin{statement}
$\mathsf{Seq}\ovw{n}{\ell}$ is the statement ``If $c_0, c_1, \dots$ is a sequence of $\ell$-colorings of a fixed alphabet $A$ of cardinality $n$, there exists a variable word $W$ such that for every $i \in \omega$ and every $\bar b \in A^i$, $\{W(\bar b\bar a) : \bar a \in A^{<\infty}\}$ is monochromatic for $c_i$.''
\end{statement}

\begin{theorem}
For every computable instance $c_0, c_1, \dots$ of $\mathsf{Seq}\ovw{2}{\ell}$,
every PA degree relative to $\emptyset'$ computes a solution to $\bar c$.
\end{theorem}
\begin{proof}
	The proof is similar to Theorem~\ref{ovwth2}. Using Lemma~\ref{ovwprop1}, we first construct a $\emptyset'$-computable sequence of strings
 $\t{\rho}_0 \prec \t{\rho}_1 \prec\cdots$,
a sequence of finite sets $P_i\subseteq \big\{
|\t{\rho}_{i-1}|,\cdots, |\t{\rho}_i|-1
\big\}$ and a sequence of colorings $d_i:[\t{\rho}_i]^\preceq\rightarrow
L_i$ inductively as follows.
$\t{\rho}_0 = \varepsilon$ and $d_0 = c_0$.
Given $\t{\rho}_i$ and $d_i:[\t{\rho}_i]^\preceq\rightarrow
L_i$, let $\t{\rho}_{i+1} \succeq \t{\rho}_i$
and $P_i\subseteq
\big\{ |\t{\rho}_i|,\cdots,|\t{\rho}_{i+1}|-1\big\}$ be such that
$(\t{\rho}_{i+1}, P_i)$- is $d_i$-valid,
and let $d_{i+1}$ be the coloring of $[\t{\rho}_{i+1}]^{\preceq}$ which on $\sigma \succeq \t{\rho}_{i+1}$
associates $\langle P', \t{P}, j, k \rangle$ such that
$(P',\t{P})$ witnesses $d_i$-validity of $(\t{\rho}_{i+1}, P_i)$
for $\sigma$, $d_i(\sigma_{P'}) = j$ and $c_{i+1}(\sigma_{P'}) = k$. Note that the main difference with the previous construction is that we handle more and more colorings among $c_0, c_1, \dots$ at each level. The remainder of the proof is the same as in Theorem~\ref{ovwth2}.
\end{proof}

The theorem above is optimal, in that we can obtain the following reversal.

\begin{theorem}
There is a computable instance $c_0, c_1, \dots$ of $\mathsf{Seq}\ovw{2}{2}$,
such that every solution is of PA degree relative to $\emptyset'$.
\end{theorem}
\begin{proof}
Let $R_0, R_1, \dots$ be a uniformly computable sequence of sets
such that for every $e$,
if $\Phi^{\emptyset'}_e(e) \downarrow = 0$ then $R_e$ is finite,
and if $\Phi^{\emptyset'}_e(e) \downarrow = 1$ then $R_e$ is cofinite.
In particular, any function $f : \omega \to 2$ such that $f(e)$ gives
a side of $R_e$ which is infinite, is DNC$_2$ relative to $\emptyset'$,
hence of PA degree relative to $\emptyset'$.
Let $c_i : 2^{<\infty} \to 2$ be defined by $c_i(\sigma) = 1$ iff $|\sigma| \in R_i$, and let $W$ be a solution to $\bar c$, that is, a variable word $W$ such that for every $i \in \omega$ and every $\bar b \in A^i$, $\{W(\bar b\bar a) : \bar a \in A^{<\infty}\}$ is monochromatic for $c_i$.
We claim that $W$ computes such a function $f$.
Given $e$, let $f(e) = c_e(W(\bar b))$, where $\bar b  \in 2^e$ is arbitrary (this is well-defined, since $c_e(\bar b)$ depends only on the length of $\bar b$). By definition of $W$, $\{W(\bar b\bar a) : \bar a \in A^{<\infty}\}$ is monochromatic for $c_e$, the color of $c_e(W(\bar b))$ appears infinitely often in $R_e$. Therefore, $W$ is of PA degree relative to $\emptyset'$. This completes the proof.
\end{proof}



\section{A difficult instance of the Ordered Variable Word theorem}\label{sect:ovw-lower-bounds}

Miller and Solomon~\cite{Miller2004Effectiveness} constructed a computable instance of $\ovw{2}{2}$ with no $\Delta^0_2$ solution. In this section, we strengthen their proof by constructing a computable instance of $\ovw{2}{2}$ such that every solution is of DNC degree relative to $\emptyset'$, using a significantly simpler argument.

The proof makes an essential use of a computable version of Lovasz Local Lemma proven by Rumyantsev and Shen~\cite{Rumyantsev2014Probabilistic}. The idea of using Lovasz Local Lemma to analyse the computability-theoretic strength of problems in reverse mathematics comes from Csima and Dzhafarov, Hirschfeldt, Jockusch, Solomon and Westrick~\cite{Csima2018reverse}, who proved that a version of Hindman's theorem for subtractions is not computably true.


\begin{definition}
	Fix a countable set of variables $x_0, x_1, \dots$
	A \emph{(disjunctive) clause} $C$ is a tuple of the form $(x_{n_1} = i_1 \vee \dots \vee x_{n_k} = i_k)$, with $i_1, \dots, i_k < 2$. The \emph{length} of $C$ is the integer $k$. An \emph{infinite CNF formula} is an infinite conjunction of disjunctive clauses. An infinite CNF formula $\bigwedge_n C_n$ is \emph{computable} if the function which given $n$ outputs a code for $C_n$ is computable, and the set of $n$ such that $C_n$ contains the variable $x_j$ is uniformly computable in $j$.
\end{definition}

\begin{theorem}[{Rumyantsev and Shen~\cite{Rumyantsev2014Probabilistic}}]\label{thm:lll-computable}
For every $\alpha \in (0,1)$, there exists some $N \in \omega$ such that every computable infinite CNF where each variable appears in at most $2^{\alpha n}$  clauses of size $n$ (for every n) and all clauses have size at least $N$, has a computable satisfying assignment.
\end{theorem}

%\begin{theorem}
%There is a computable instance $c$ of $\ovw{2}{2}$ such that any solution $W$ to $c$ is complete.
%\end{theorem}
%\begin{proof}
%We shall build a computable instance $c$ of $\ovw{2}{2}$ such that for any solution $W$ to $c$, the $W$-computable function $f$ which to $n$ associates the smallest $t$ such that the first occurrence of the $n$-th variable appear at position $t$, bounds the function $n \mapsto \min t \text{ s.t. }\emptyset'[t] \upharpoonright n = \emptyset' \upharpoonright n$.
%
%Fix $\alpha = 0.5$, and let $N$ be the threshold of Theorem~\ref{thm:lll-computable}. For a finite variable word $w$, let $T_w$ be the tree corresponding to every possible instantiation of the variables (in particular every leaf of $T_w$ has length $|w|$). For any $1 \leq n \leq s \in \omega$, let us fix in advance a list $L^n_s$ of the variable words of length smaller $t \leq s$, with exactly $n-1$ distinct variables. Let us define the computable function $g$ which to $s$ associates $\sup_{1 \leq n \leq s} |L^n_s|$.
%
%We can suppose without loss of generality that the enumeration of $\emptyset'$ is delayed so that at most one element is enumerated at a given stage, and whenever some element if enumerated in $\emptyset'$ at stage $s$, then no element is enumerated in $\emptyset'$ between stage $s+1$ and stage $g(s)$.
%
%Let us define the following computable predicate $P \subseteq \omega \times \omega \times 2^{<\omega}$ : we have $P(n,s,\tau)$ if there exists $t,k \in \omega$ such that $s=t+k$ with:
%\begin{enumerate}
%\item $t < t+k \leq g(t)$ 
%\item $\emptyset'[t-1] \upharpoonright n \neq \emptyset'[t] \upharpoonright n$
%\item $|\tau|+|w^n_s| = s$, where $w^n_s$ is the $k$-th element of $L^n_t$.\\
%\end{enumerate}
%
%\textit{Claim : For any $1 \leq n \leq s$ and any $\tau$, if $P(n,s,\tau)$ holds, there are unique elements $t,k$ such that (1) and (2) holds above. In particular $w^n_s$ is well-defined in function of $1 \leq n \leq s$.}\\
%
%Proof : The claim holds by assumption on the delayed enumeration of elements in $\emptyset'$. Suppose $P(n,s,\tau)$ holds and let $t,k$ such that $s=t+k$ with $t < t+k \leq g(t)$ and $\emptyset'[t-1] \upharpoonright n \neq \emptyset'[t] \upharpoonright n$. Consider $\t{t}$ and $\t{k}$ such that $\t{t} < \t{t}+\t{k} \leq g(\t{t})$ and $\emptyset'[\t{t}-1] \upharpoonright n \neq \emptyset'[\t{t}] \upharpoonright n$. Suppose $\t{t} < t$. By assumption on the delayed enumeration of $\emptyset'$ we must have $\emptyset'[\t{t}] = \emptyset'[g(\t{t})]$. As an element is enumerated in $\emptyset'$ at stage $t$, it follows that $\t{t}+\t{k} < t$ which implies $\t{t}+\t{k} \neq t + k = s$. Similarly if $\t{t} > t$ we must have $\t{t} > g(t)$ and then $\t{t}+\t{k} \neq t + k = s$. Thus if $\t{t}+\t{k} = t + k = s$ we must have $\t{t} = t$ and therefore that $\t{k} = k$. This shows the claim.\\
%
%
%For any $1 \leq n \leq s \in \omega$ and any $\tau$ such that $P(n,s,\tau)$, for any $i \in \{0,1\}$, we define $C_{n,s,\tau,i}$ to be the clause consisting of disjunction of the form ``$x_{\sigma\tau} = i$'' for any leaf $\sigma$ of the tree $T_{w^n_s}$, whose paths are exactly the possible instantiations of the finite word variable $w^n_s$. Formally:
%$$
%\bigvee \{x_{\sigma \tau} = i : \sigma \in T_{w^n_s} \}.
%$$
%And let $C$ be the conjunction:
%$$
%\bigwedge\limits_{N \leq n \leq s \in \omega, \tau \in 2^{<\omega}, i < 2}
% \{ C_{n,s,\tau, i} : P(n,s,\tau)\}.
%$$
%
%This infinite CNF formula is clearly computable as the predicate $P(n,s,\tau)$ is. Clearly $C_{n,s,\tau, i}$ has length $2^n$, that is, the number of leaves of $T_{w^n_s}$.
%
%Let us show that for any $m$, any $\rho \in 2^{<\omega}$, the variable $x_\rho$ appears in at most $2$ clauses of $C$ of length $m$. As $|C_{n,s,\tau, i}| = 2^n$, the only clauses of length $m$ in which $x_\rho$ could appear are of the form $C_{n,s,\tau, i}$  for $2^n = m$. Now for $n$ fixed, suppose $x_\rho$ appears in some clause $C_{n,s_1,\tau_1, i_1}$ and in some clause $C_{n,s_2,\tau_2, i_2}$ for some $s_1,s_2, \tau_1,\tau_2, i_1,i_2$. In particular $P(n,s_1,\tau_1)$ and thus by (3) we have $|\rho| = s_1$. As also $P(n, s_2,\tau_2)$, still by (3) we have $|\rho| = s_2$. This implies $s_1=s_2$. But then $w^{n_1}_{s_1} = w^{n_2}_{s_2}$ and then clearly we must have $\tau_1=\tau_2$. It follows that for any $\rho$, the variable $x_\rho$ appears in at most two clause of length $m$ : $C_{n,s,\tau, 0}$ and $C_{n, s,\tau, 1}$ for $n$ such that $2^n = m$ and for some $s,\tau$.\\
%
%Note also that in $C$ we consider only disjunctions $C_{n,s,\tau, i}$ for $n \geq N$. In particular any clause in $C$ has length bigger than $N$. It follows that we are in the condition of Theorem \ref{thm:lll-computable} and then that there exists a computable color $c$ which satisfies every clause of $C$. 
%
%Let us show that any solution to $c$ computes the jump. Let $W$ be a solution to $c$. Suppose for contradiction that there exists $n \geq N$ such that the first occurrence of the $n$-th variable appear in $W$ at a position $s$ with $s < \min t\text{ s.t. } \emptyset'[t] \upharpoonright n = \emptyset' \upharpoonright n$. In particular there exists $r \geq s$ such that $\emptyset_{r-1}' \upharpoonright n \neq \emptyset_{r}' \upharpoonright n$. As $|W \upharpoonright s| \leq r$ and as $W \upharpoonright s$ contains $n-1$ distinct variables, there must exists $k$ such that $r \leq r+k < g(r)$ and such that $W \upharpoonright s = w^n_{r+k}$, that is, the $k$-th element of $L^n_r$, the list of variable words of length smaller than $r$, with exactly $n-1$ distinct variables. But by construction $w^n_{r+k}$ cannot be an initial segment of a solution~:~This is because for any string $\tau$ such that $|\tau| + |w^n_{r+k}| = r+k$, there is one leaf $\sigma_1$ of $T_{w^n_{r+k}}$ of such that $c(\sigma_1\tau) = 0$ and one leaf $\sigma_2$ of $T_{w^n_{r+k}}$ such that $c(\sigma_2\tau) = 1$. Therefore no extension of $w^n_{r+k}$ of length $r+k$ can be a solution.
%
%It follows that for every $n \geq N$, the $t$ is the smallest position such that the first occurrence of the $n$-th variable occurs in $W$, then we must have $\emptyset'[t] \upharpoonright n = \emptyset' \upharpoonright n$. It follows that $W$ computes the jump.
%\end{proof}

\begin{theorem}\label{thm:ovw-delta2}
There is a computable instance $c$ of $\ovw{2}{2}$ and a computable function $h : \omega \to \omega$ such that if $\Phi_e^{\emptyset'}$ outputs a finite variable word in which the first $h(e)$ variable kinds occur, then $\Phi_e^{\emptyset'}$ is not extendible into an infinite solution to $c$.
\end{theorem}
\begin{proof}
Fix $\alpha = 0.5$, and let $N$ be the threshold of Theorem~\ref{thm:lll-computable}. For every index $e$ and stage $s$, we interpret $\Phi^{\emptyset'}_e[s]$ as a finite variable word $W_{e,s}$ with exactly $N+e$ variable kinds, and where a new variable occurs right after $W_{e,s}$. Such a variable word induces a binary tree $T_{e,s}$ with $2^{N+e}$ leaves. Let $L_{e,s}$ be the set of leaves of $T_{e,s}$, that is, the set of all instantiations of the variable word $W_{e,s}$. Moreover, all the leaves of $T_{e,s}$ have the same length $n_{e,s}$.

The idea is the following: since the variable word is ordered and a new variable kind occurs right after $W_{e,s}$, no variable among the first $N+e$ variables can occur after $W_{e,s}$. If $W$ is a solution to $c$ with initial segment $W_e = \lim_s W_{e,s}$ for some color $i$, then $W$ must be homogeneous for $c$ for every instance of the variables, so in particular when setting all the variables after the $N+e$ first ones to 0. Hence, there must be infinitely many strings $\tau$ such that for every $\sigma \in \lim_s L_{e,s}$, $c(\sigma\tau) = i$. By ensuring that for cofinitely many $\tau$, there is some  $\sigma \in L_{e,|\tau|}$ such that $c(\sigma\tau) \neq i$, we force $W_e$ not to be a solution to $c$ for color $i$.


Fix a countable collection of variables $(x_\rho : \rho \in 2^{<\omega})$. Each variable $x_\rho$ corresponds to the color of the string $\rho$.
Given some $s\in\omega,\tau \in 2^{<\omega}$ and some $i < 2$,
if $n_{e,s}+|\tau| = s $, then
let $C_{e,s,\tau,i}$ be the disjunctive $2^{N+e}$-clause
$$
\bigvee \{x_{\sigma\tau} = i : \sigma \in L_{e,s} \}.
$$
And let $C$ be the conjunction
$$
\bigwedge\limits_{n_{e,s}+|\tau| = s }
 \{ C_{e,s,\tau, i} : e \in \omega, \tau \in 2^{<\omega}, i < 2\}.
$$
This infinite CNF formula is clearly computable. Clearly $C_{e,s,\tau,i}$ has length $2^{N+e}$. Note that for every $\rho,e$, there exists at most one $\tau$ such that $(\exists \sigma\in L_{e,|\rho|})[\sigma\tau = \rho]$. Therefore, each variable $x_\rho$ appears in at most $2$ clauses of length $2^{N+e}$, namely, $C_{e, |\rho|,\tau, 0}$ and $C_{e, |\rho|, \tau, 1}$, where $\tau$ is such that $(\exists \sigma\in L_{e,|\rho|})[\sigma\tau = \rho]$. Therefore, this formula satisfies the conditions of Theorem~\ref{thm:lll-computable}, and has a computable assignment $c : 2^{<\omega} \to 2$. By construction, letting $h(e) = N+e+1$, the formula ensures that if $\Phi_e^{\emptyset'}$ outputs a finite variable word in which the first $h(e)$ variables kinds occur, then $\Phi_e^{\emptyset'}$ is not extendible into an infinite solution to $c$.
\end{proof}

\begin{definition}
	A function $f : \omega \to \omega$ is \emph{diagonally non-computable relative to $X$} (or $X$-dnc) if for every $e$,
	$f(e) \neq \Phi_e^X(e)$.
\end{definition}

\begin{corollary}
	There is a computable instance $c$ of $\ovw{2}{2}$ such that every solution is of $\emptyset'$-dnc degree.
\end{corollary}
\begin{proof}
	Let $c$ and $h$ be as in Theorem~\ref{thm:ovw-delta2}. For every $e$, let $\alpha_e$ be a computable bijection from the finite variable words in which the first $h(e)$ variable kinds occur, to the set of the integers.
	By Kleene's fixpoint theorem, there is a computable function $g : \omega \to \omega$ such that for every $e$, $\Phi^{\emptyset'}_{g(e)} = \alpha^{-1}_{g(e)}(\Phi^{\emptyset'}_e(e))$.	
	
	Let $W$ be a solution to $c$, that is, an infinite variable word. Let $f$ be the $W$-computable function defined by $f(e) = \alpha_{g(e)}(w_e)$, where $w_e$ is the first initial segment of $W$ in which the first $h(g(e))$ variable kinds occur. We claim that $f$ is $\emptyset'$-dnc. Indeed, given $e \in \omega$, $w_e \neq \Phi^{\emptyset'}_{g(e)}$, so
	$$
	f(e) = \alpha_{g(e)}(w_e) \neq \alpha_{g(e)}(\Phi^{\emptyset'}_{g(e)})  = \Phi^{\emptyset'}_e(e)
	$$
	This completes our proof.
\end{proof}

We conclude this section with a small computational observation about $\vw{2}{2}$ based on the syntactical form of the statement.

\begin{definition}
	A function $g : \omega \to \omega$ \emph{dominates}
	$f : \omega \to \omega$ if $(\forall x)f(x) < g(x)$.
	A function $f : \omega \to \omega$ is \emph{hyperimmune}
	if it is not dominated by any computable function.
	A Turing degree is \emph{hyperimmune-free} if it does not contain any hyperimmune function.
\end{definition}

\begin{lemma}[Folklore]\label{lem:pi01-hi-pa}
Let $\Psf$ be a statement of the form $(\forall X)[\Phi(X) \rightarrow (\exists Y)\Psi(X, Y)]$ where $\Phi$ is an arbitrary predicate, and $\Psi$ is a $\Pi^0_2$ predicate. For every computable instance $I$ of $\Psf$, if $I$ has a solution of hyperimmune-free degree, then every PA degree computes a solution to $I$.
\end{lemma}
\begin{proof}
	Say $\Psi(X, Y) \equiv (\forall x)(\exists y)\Theta(X \uh y, Y \uh y, x, y)$, where $\Theta$ is a decidable predicate.
	Let $I$ be a computable $\Psf$-instance with a solution $S$ of hyperimmune-free degree. 
	Let $h : \omega \to \omega$ be the $S$-computable function such that for every $x$, $\Theta(I, S, x, h(x))$ holds. In particular, there is a computable function $g : \omega \to \omega$ such that $(\forall x)\max (h(x), S(x)) < g(x)$. Let $T \subseteq \omega^{<\omega}$ be the computably bounded tree defined by
	$$
	T = \left\{ \sigma \in \omega^{<\omega} : \begin{array}{l}
 		(\forall x < |\sigma|)\sigma(x) < g(x)) \wedge  \\
 		(\forall x < |\sigma|)[g(x) < |\sigma| \rightarrow (\exists y < |\sigma|)\Theta(I \uh y, \sigma \uh y, x, y)]
 \end{array} \right\}
	$$
	In particular, $S \in [T]$, so the tree is infinite. Moreover,
	any $R \in [T]$ is a solution to $I$, and any PA degree computes a member
	of $[T]$. This completes the proof.
\end{proof}

\begin{corollary}
	There is a computable instance of $\vw{2}{2}$ such that
	every solution is of hyperimmune degree.
\end{corollary}
\begin{proof}
	First, note that the statement $\vw{2}{2}$ is of the form of Lemma~\ref{lem:pi01-hi-pa}.
	Let $c : 2^{<\omega} \to 2$ be the computable
	instance of $\vw{2}{2}$ with no low solution constructed by Miller and Solomon~\cite{Miller2004Effectiveness} or by Theorem~\ref{thm:ovw-delta2}. Letting $\dbf$ be a low PA degree, $\dbf$ computes no solution to~$c$, hence by Lemma~\ref{lem:pi01-hi-pa}, every solution to~$c$ is of hyperimmune degree.
\end{proof}

It is still unknown whether there is a computable instance of $\ovw{2}{2}$ such that every solution is PA over $\emptyset'$, or even just computes $\emptyset'$. In particular the following questions remain open:

\begin{question}
	Does $\vw{2}{2}$ or $\ovw{2}{2}$ imply $\aca$ over $\rca$?
\end{question}

\begin{question}
	Is there a computable instance of $\vw{2}{2}$ or $\ovw{2}{2}$ such that the measure of oracles computing a solution to it is null?
\end{question} 

%\section{Lower bounds on the Finite Union Theorem}\label{sect:fut-lower-bounds}
%
Towsner~\cite{Towsner2012simple} gave a combinatorially simple proof of the Finite Union Theorem in $\aca^{+}$. He introduced the notion of full-match, and proved its existence over $\aca$ in~\cite[Lemma 2.7]{Towsner2012simple}.  However, the existence of a full-match is not known to be equivalent to $\aca$. This notion is the cornerstone of Towsner's proof, as having low${}_n$ full-matches for some fixed $n$ would be sufficient to obtain arithmetical solutions to the Finite Union Theorem. In this section, we improve the lower bound of the existence of a full-match by showing that it cannot be proven in $\wkl$.

\begin{definition}
Fix a coloring $c : \Pcal_{fin}(\Nb) \to r$.
Let $\Bcal \subseteq \Pcal_{fin}(\Nb)$ be a finite collection,
and let $\Ical \subseteq \Pcal_{fin}(\Nb) - \Bcal$ be an IP collection.
\begin{itemize}
	\item[(i)] $\Bcal$ \emph{left-matches} $\Ical$ if for every $S \in \Ical$, there is some $B \in \Bcal$ such that $c(B) = c(B \cup S)$. 
	\item[(ii)] $\Bcal$ \emph{right-matches} $\Ical$ if for every $S \in \Ical$, there is some $B \in \Bcal$ such that $c(S) = c(B \cup S)$. 
	\item[(iii)] $\Bcal$ \emph{full-matches} $\Ical$ if for every $S \in \Ical$, there is some $B \in \Bcal$ such that $c(B) = c(B \cup S) = c(S)$. 
\end{itemize}
\end{definition}

Note that in Towsner's paper~\cite{Towsner2012simple}, a right-match is called a half-match. A full-match is both a left-match and a right-match, but the converse is not true in general.

\begin{statement}
	$\lmfut_r$ denotes the statement ``For every coloring $c :\Pcal_{fin}(\Nb) \to r$, there is a finite collection $\Bcal \subseteq \Pcal_{fin}(\Nb)$  and an IP collection $\Ical \subseteq \Pcal_{fin}(\Nb) - \Bcal$ such that $\Bcal$ left-matches $\Ical$.
	The statements $\rmfut_r$ and $\rmfut_r$ are defined accordingly for the notions of right-match and full-match.
\end{statement}

Towsner~\cite[Lemma 2.5]{Towsner2012simple} proved $\rmfut_r$ over $\rca$. He also constructed in~\cite[Theorem 3.8]{Towsner2012simple} a computable instance of $\lmfut_2$ with no computable solution, therefore showing that $\rca \nvdash \lmfut_2$. We now adapt his proof to show that $\rca + \wkl \nvdash \lmfut_2$, by proving that $\lmfut_2$ implies the existence of hyperimmune functions.

The following theorem combines the techniques of Towsner~\cite[Theorem 3.8]{Towsner2012simple} and Csima and Mileti~\cite[Theorem 4.1]{Csima2009strength}. A familiarity with the mentioned proofs will simplify drastically the understanding of the proof of Theorem~\ref{thm:lmfut-opt}.



\begin{theorem}\label{thm:lmfut-opt}
There is a computable instance of $\lmfut_2$ such 
that every solution is of hyperimmune degree.
\end{theorem}
\begin{proof}
The strategy is the following: We will build a computable coloring $c : \Pcal_{fin}(\Nb) \to 2$ such that for every finite collection $\Bcal$ which left-matches an IP collection $\Ical$ we satisfy the following requirement for each $e \in \Nb$:
\begin{quote}
$\Rcal_e$: If $\Phi_e$ is total, then
there is an input $k_e$ such that 
$\Ical \cap [k_e, \Phi_e(k_e)] = \emptyset$.
\end{quote}
Here, $[a, b] = \{a, a+1, \dots, b \}$.
Suppose we have constructed $c$, and let $(\Bcal, \Ical)$ be such that $\Bcal$ left-matches~$\Ical$.
Let $p_\Ical$ be the $\Ical$-computable function which on input
$n$ searches for a set $S \in \Ical$ such that $\min S > n$ and outputs $\max S$. For every $e \in \Nb$, $p_\Ical(k_e) > \Phi_e(k_e)$.
Therefore $p_\Ical$ is not dominated by any computable function, hence is hyperimmune. We now explain how to build the coloring $c$.

Using a movable marker procedure, define for every $e \in \Nb$ and at every stage $s$ two inputs $x_{e,s}, y_{e,s} > e$ ordered as follows
$$
x_{0,s} < y_{0,s} < x_{1,s} < y_{1,s} < \dots 
$$
Moreover, if $\Phi_{e,s}(x_{e,s})\downarrow$, then $\Phi_{e,s}(x_{e,s}) < y_{e,s}$, and if $\Phi_{e,s}(y_{e,s})\downarrow$, then $\Phi_{e,s}(y_{e,s}) < x_{e+1,s}$. We can show by induction that each marker is moved at most finitely many times, and therefore $x_e = \lim_s x_{e,s}$ and $y_e = \lim_s y_{e,s}$ exist for every $e$. For every $e, s \in \Nb$, let $X_{e,s} = [x_{e,s}, \Phi_{e,s}(x_{e,s})]$ if $\Phi_{e,s}(x_{e,s})\downarrow$, and $X_{e,s} = \emptyset$ otherwise. The set $Y_{e,s}$ is defined accordingly.
	According to the previous observation, $X_e = \lim_s X_{e,s}$ and $Y_e = \lim_s Y_{e,s}$ exist for every $e \in \Nb$, and satisfy
$$
X_{0,s} < Y_{0,1} < X_{1,s} < Y_{1,s} < \dots
$$
The construction will work as follows. Using the combinatorics of Towsner~\cite[Theorem 3.8]{Towsner2012simple} (to be explained later), for every $e$, we will pick two sets $S \subseteq X_{e,s}$ and $T \subseteq Y_{e,s}$, and will ensure that $\{S, T, U\} \not \subseteq \Ical$ for every solution $(\Bcal, \Ical)$ to $c$, and cofinitely many finite sets $U$. If $\Phi_e$ is partial, our requirement is vacuously satisfied. If $\Phi_e$ is total, at some finite stage $t$, there will be some input $k_{S,T} \in \Nb$ such that $\Phi_{e,t}(k_{S,T})\downarrow$ and $\{S, T, U\} \not \subseteq \Ical$ for every $U \subseteq [k_{S,T}, \Phi_{e,t}(k_{S,T})]$. This way, we will have ensured that if $S$ and $T$ both belong to $\Ical$, then $\Ical \cap [k_{S,T}, \Phi_{e,t}(k_{S,T})] = \emptyset$. Since this is checked at a finite stage $t$, we then pick the next pair $S' \subseteq X_{e,t}$ and $T' \subseteq Y_{e,t}$ and do the same procedure, and so on until we have consumed all the pairs over $X_{e,t}$ and $Y_{e,t}$.
Therefore, we will have ensured that either $\Ical \cap X_e = \emptyset$, or $\Ical \cap Y_e = \emptyset$, or there is some $S \subseteq \Ical \cap X_e$ and $T \in \Ical \cap Y_e$
such that $\Ical \cap [k_{S,T}, \Phi_{e,t}(k_{S,T})] = \emptyset$. 

We say that a requirement $\Rcal_e$ is \emph{ready} at stage $s$ if 
$\Phi_{e,s}(x_{e,s})\downarrow$ and $\Phi_{e,s}(y_{e,s})\downarrow$.
Given a pair of finite sets $S^0 < S^1$, $\Rcal_e$ is \emph{$(S^0, S^1)$-satisfied for $\Rcal_e$} at stage $s$
if there is some $k > S^1$ such that $\Phi_{e,s}(k)\downarrow$
and for every set $A < S^0$, there is some $u < 2$ such that
for every $B \subseteq [k, \Phi_{e,s}(k)]$, $c(A) \neq c(A \cup S^u \cup B)$. In other words, $\Rcal_e$ is $(S^0,S^1)$-satisfied at stage $s$ if for every solution $(\Bcal, \Ical)$ to $c$ such that $S^0, S^1 \in \Ical$, $\Ical \cap [k, \Phi_{e,s}(k)] = \emptyset$. Note that a requirement may be ready at some stage, but not a later stage, whereas if $S^0 < S^1$ is satisfied for $\Rcal_e$ at some stage, then will always remain satisfied.
A requirement $\Rcal_e$ is \emph{satisfied} at stage $s$ if either $\Phi_e$ is partial, or it is $(S^0, S^1)$-satisfied for every pair $S^0 \subseteq X_{e,s}$ and $S^1 \subseteq Y_{e,s}$.

At any stage $s$ of the construction and for every $e < s$ such that $\Rcal_e$ is ready, we will have distinguished two sets $S^0_{e,s} \subseteq X_{e,s}$ and $S^1_{e,s} \subseteq Y_{e,s}$ which currently receive attention. They are chosen two be the least pair (in an arbitrary fixed order) which is not yet satisfied for $\Rcal_e$. The following two definitions are defined by Towsner~\cite[Theorem 3.8]{Towsner2012simple}.

A \emph{primary $s$-decomposition} of a set $B$, where $s = \max B$ is a tuple $i, u, Z, D$ such that $B = Z \cup S^u_{e,s} \cup D$, $Z < S^u_{e,s} < D$,
neither $Z$ nor $D$ contains $S^{1-u}_{i,s}$ as a subsequence, and there is no primary $s$-decomposition of $D$. Clearly, there is at most one primary $s$-decomposition of $B$.

We say that $B$ \emph{contains $e$ with polarity $v$} if there is a primary $\max B$-decomposition $i, u, Z, D$ of $B$ with either $e = i$ and $v = u$,
or $e$ contained in $Z$ with polarity $|v-u|$. Observe that whenever $B$ contains $e$, $B = Z \cup S^u_{e,t} \cup D$ for some $t \leq \max B$.

We now define our coloring by stages. At stage $s$, supposed we have already decided $c(B')$ whenever $\max B' < s$. Let $B$ be such that $\max B = s$. If $B$ has a primary $s$-decomposition $B = Z \cup S^u_{e,s} \cup D$, we set $c(B) = c(Z)$ if $u = 0$ and $c(B) \neq c(Z)$ if $u = 1$. If there is no primary $s$-decomposition of $B$, we set $c(B) = 0$. This completes the construction. We now turn to the verification.


\begin{claim}
For a total Turing functional $\Phi_e$, all $S \subseteq X_e$ and all $T \subseteq Y_e$, $\Rcal_e$ is $(S,T)$-satisfied at some stage. 
\end{claim}
\begin{proof}
Suppose not. Let $S \subseteq X_e$ and $T \subseteq Y_e$ be the least pair (in the previously fixed order) such that $\Rcal_e$ is not $(S,T)$-satisfied at any stage. Let $s$ be a stage such that for all $t \geq s$, $X_{e,t} = X_e$, $Y_{e,t} = Y_e$, and $\Rcal_e$ is $(S',T')$-satisfied for all the previous pairs $(S', T')$. By construction, $S^0_{e,t} = S$ and $S^1_{e,t} = T$.

It is easy to see that for any $B \geq s$, there is a $v_B$ such that
$A \cup S^u_e \cup B$ contains $e$ with polarity $|v_B - u|$ for all $A < \in \Bcal$. We now prove by induction on the length of $B$ that for all $B > s_1$ and $A < S^0_{e,t}$, $c(A \cup S^{v_B}_e \cup B) = c(A)$ and $c(A \cup S^{1-v_B}_e \cup B) \neq c(A)$. Given $u < 2$, let $D = A \cup S^u_e \cup B$. In particular, $D$ admits a primary $\max B$-decomposition $Z \cup S^{u'}_i \cup B'$. If we just have $e = i$, then $v_B = 0$ and the claim follows immediately from the decomposition of the coloring. Otherwise, we have two cases.
In the first case, $u' = 0$. Then $c(D) = c(Z)$ and $Z$ contains $e$ with polarity $|v_B - u|$. By induction hypothesis applied to $Z \setminus (A \cup S^u_e)$, $c(D) = c(Z) = c(A)$ if $u = v_B$ and $c(D) = c(Z) \neq c(A)$ if $u \neq v_B$. In the second case, $u' = 1$. Then $c(D) \neq c(Z)$ and $Z$ contains $e$ with polarity $1-|v_B-u|$. By induction hypothesis applied to $Z \setminus (A \cup S^u_e)$, $c(D) \neq c(Z) \neq c(A)$ if $u = v_B$, so $c(D) = c(A)$, and $c(D) \neq c(Z) = c(A)$ if $u \neq v_B$. This completes the induction and the proof of the claim.
\end{proof}

Let $(\Bcal, \Ical)$ be a solution to $c$.

\begin{claim}
	For every total Turing functional $\Phi_e$, there is an input $k$
	such that $\Ical \cap [k, \Phi_e(k)] = \emptyset$.
\end{claim}
\begin{proof}
By the padding lemma, we can assume that for all $A \in \Bcal$,
$A < e$. 
If $\Ical \cap [x_e, \Phi_e(x_e)] = \emptyset$ or $\Ical \cap [y_e, \Phi_e(y_e)] = \emptyset$, the we are done. Otherwise, let $S \subseteq X_e$ and $T \subseteq Y_e$ be such that $S, T \in \Ical$.
By assumption, $x_{e,s}, y_{e,s} > e$ for every $s \in \Nb$.
It follows that $A < e < S$. By the previous claim, $\Rcal_e$ is $(S,T)$-satisfied at some stage. Unfolding the definition, there is some $k > T$
such that $\Phi_e(k)\downarrow$ and for every $A < S$ (and in particular for every $A \in \Bcal$), either for every $B \subseteq [k, \Phi_e(k)]$, $c(A) \neq c(A \cup S \cup B)$,
or for every $B \subseteq [k, \Phi_e(k)]$, $c(A) \neq c(A \cup T \cup B)$.
In any case, $\Ical \cap [k, \Phi_e(k)] = \emptyset$. This completes the proof of the claim.
\end{proof}

As explained above, it follows that $\Ical$ is of hyperimmune-degree.
This completes the proof of Theorem~\ref{thm:lmfut-opt}.
\end{proof}

\begin{corollary}
$\rca + \wkl \nvdash \lmfut_2$
\end{corollary}
\begin{proof}
	By the relativized Hyperimmune-free Basis Theorem~\cite{Jockusch197201},
	this is a model of $\rca + \wkl$ with only hyperimmune-free degrees.
	By Theorem~\ref{thm:lmfut-opt}, this is not a model of $\lmfut_2$.
\end{proof}

Blass, Hirst and Simpson built a simple computable instance of the finite union theorem such that every solution computes the halting set: Given a finite set $S = \{x_0 < x_1 < \dots < x_n\}$, a gap $(x_i, x_{i+1})$ is \emph{large} if $\emptyset'\uh x_i = \emptyset'_{x_{i+1}} \uh x_i$,
	and is \emph{small} otherwise. A gap $(x_i, x_{i+1})$ is \emph{very small in $S$} if $\emptyset'_{x_n} \uh x_i \neq \emptyset'_{x_n} \uh x_{i+1}$.
	Letting $SG(S)$ and $VSG(G)$ denote the number of small gaps and of very small gaps in $S$, respectively, their coloring is simply defined by $c(S) = VSG(S) \mod 2$. Given an IP collection $\Ical$ homogeneous for $c$, they proved that $SG(S)$ is even for every $S \in \Ical$, and that the gap $(\max S, \min T)$ is large for every $S < T \in \Ical$. However, for this same instance, one can easily construct a computable full-match $(\Bcal, \Ical)$ with $|\Bcal| = 2$ as follows. Let $\Bcal = \{B_0, B_1\}$, where $SG(B_0) \mod 2 = SG(B_1) \mod 2 = 0$, $VSG(B_0)$ is odd, and $VSG(B_1)$ is even. Then $\Bcal$ full-matches $\Pcal_{fin}(\Nb) - \Bcal$. This motivates the following question.

\begin{question}
	Does $\lmfut_2$ or $\fmfut_2$ imply $\aca$ over $\rca$?
\end{question}


\bibliographystyle{plain}
\bibliography{bibliography}

\appendix

%\section{Previous proofs}
%\section{Numerical results for $\bar d_{n,2}(X)^2$}\label{app:num}
In order to support the heuristic argumentation on the intrinsic and rate optimal growth limitation of the truncation error $\Probb\big(X\notin C_{n}\big)$ induced by the extended dual quantization error modulus,
we consider the two dimensional random variable 
\[
	X = (W_T, \sup_{0\leq t \leq T}W_t),
\]
where $(W_t)_{0\leq t \leq T}$ is a Brownian Motion.

This example is motivated by the pricing of exotic options, where this joint distribution plays an important role.

Using a variant of the CVLQ algorithm (see \cite{dualStat}) adapted for the dual quantization modulus inside $C_n$ and the nearest neighbor mapping outside, we have computed a sequence of optimal grids together with the squared dual quantization error $\bar d_{n,2}(X)^2$ and the truncation error $\Probb\big(X\notin C_{n}\big)$.

These results are reported in Table~\ref{tab:num}. 

\begin{table}[h!]
\centering
\label{tab:num}
\caption{Numerical results for the dual quantization $X$}
\begin{tabular}{l|c|c}
$n$ & $\bar d_{n,2}(X)^2$ & $\Probb\big(X\notin C_{n}\big)$ \\
\hline
50 & 0.04076 & 0.01784\\
100 & 0.01966 & 0.00795 \\
150 & 0.01236 & 0.00412 \\
200 & 0.00931 & 0.00141 \\
\end{tabular}
\end{table}

Furthermore we see in figure \ref{fig:num} a log-log plot for the convergence of the two rates $\bar d_{n,2}(X)^2$ and $\Probb\big(X\notin C_{n}\big)$.

\begin{figure}[h!]
\includegraphics{BMjointSupDistortionTruncationLog.pdf}
\caption{log-log plot of $\bar d_{n,2}(X)^2$ and $\Probb\big(X\notin C_{n}\big)$ with respect to the grid size $n$}
\label{fig:num}
\end{figure}

The distortion rate $\bar d_{n,2}(X)^2$ shows here an absolute stable convergence rate (least-squares fit of exponent yields $-1.07192$) which is consistent with the theoretical optimal rate of $n^{-\frac{2}{d}}$.
Moreover, the truncation error $\Probb\big(X\notin C_{n}\big)$ outperforms also in this case the heuristically derived rate of $n^{-1}$ and also outperforms the squared "inside" quantization error, which means that also for such an un-symmetric and non-spherical distribution of the Brownian motion and its supremum, an second order rate can be achieved.

This confirms again the motivation of the extended dual quantization error as the correction penalization constraint on growth of the convex hull in order to preserve second order stationarity.





\end{document}


%An  ISPFS, $A_i,B_i,i\in\omega$ is \IA if there exists an array
%of ISPFS, $A_i^{(m)},B_i^{(m)},i\in\omega,m\in\omega$
%such that: $(\forall i) \big(\ \lim\limits_{m\rightarrow\infty}
%A_i^{(m)}=A_i,\ B_i^{(m)}=B_m\big)$;
%for all $m$, $A^{(m)}_i,B^{(m)}_i$ is an ISPFS;
