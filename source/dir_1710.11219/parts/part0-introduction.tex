
Let $(\Nb)^k$ and $(\Nb)^\infty$ denote the set of partitions of $\Nb$ into exactly $k$ and infinitely many non-empty pieces, respectively. For $X \in (\Nb)^\infty$, $(X)^k$ is the set of all $Y \in (\Nb)^k$ which are coarser than $X$.

\begin{statement}[Dual Ramsey theorem]
	$\drt^k$ is the statement ``If $(\Nb)^k$ is colored with finitely many Borel colors, then there is some $X \in (\Nb)^\infty$ such that $(X)^k$ is monochromatic''.
\end{statement}

The Dual Ramsey theorem was proven by Carlson and Simpson~\cite{Carlson1984dual},
and studied from a reverse mathematical viewpoint by Slaman~\cite{Slaman1997note}, Miller and Solomon~\cite{Miller2004Effectiveness} and Dzhafarov et al.~\cite{Dzhafarov2017Effectiveness}.
In this paper, we shall focus on a combinatorial lemma used by Carlson and Simpson to prove the Dual Ramsey theorem. This lemma can be formulated in terms of \emph{variable words}.

\begin{definition}[Variable word]
An \emph{infinite variable word} on a finite alphabet $A$
is an $\omega$-sequence $W$ of elements of $A \cup \{x_i : i \in \Nb\}$
in which all variables occur at least once, and finitely often. Moreover, the first occurrence of $x_i$ comes before the first occurrence of $x_{i+1}$.
A \emph{finite variable word} is an initial segment of an infinite variable word. A finite or infinite variable word is \emph{ordered} if moreover all occurences of $x_i$ come before any occurrence of $x_{i+1}$.
Given $\bar a = a_0a_1 \dots a_{k-1} \in A^{<\omega}$, we let $W(\bar a)$
denote the finite $A$-string obtained by replacing $x_i$ with $a_i$ in $W$ and then truncating the result
just before the first occcurence of $x_k$.
\end{definition}

\begin{statement}[Variable word theorem]
$\vw{n}{r}$ is the statement ``If $A^{<\omega}$ is colored with $r$ colors for some alphabet $A$
of cardinality $n$, there exists
an infinite variable word $W$ such that $\{W(\bar a) : \bar a \in A^{<\omega}\}$ is monochromatic.
$\ovw{n}{r}$ is the same statement as $\vw{n}{r}$ but for ordered variable words.
\end{statement}

In this paper, we study the computability-theoretic properties of the variable word theorems using the framework of reverse mathematics.\footnote{The authors thank Damir Dzhafarov, Stephen Flood, Reed Solomon and Linda Brown Westrick for bringing the attention of the authors to the Carlson-Simpson lemma, and for  numerous discussions. The authors are also thankful to Denis Hirschfeldt and Barbara Csima for showing them how to use Lovasz Local Lemma to prove lower bounds to combinatorial theorems.}

\subsection{Reverse mathematics}
Reverse mathematics is a vast foundational program aiming to determine the optimal axioms to prove ordinary theorems. It uses the framework of second-order arithmetic, with a base theory $\rca$ consisting of the axioms of Robinson arithmetic, the $\Sigma^0_1$ induction scheme and the $\Delta^0_1$ comprehension scheme. The system $\rca$ arguably captures \emph{computable mathematics}. Starting from a proof-theoretic perspective, modern reverse mathematics tends to be seen as a framework to analyse the computability-theoretic features of theorems. Among the distinguished statements, let us mention weak K\"onig's lemma ($\wkl$), asserting that every infinite binary tree has an infinite path, the arithmetic comprehension axiom ($\aca$), and the $\Pi^1_1$ comprehension axiom ($\piooca$), consisting of the comprehension scheme restricted to arithmetic and $\Pi^1_1$ formulas, respectively. See Simpson~\cite{Simpson2009Subsystems} for reference book on classical reverse mathematics.

The statements studied within this framework are mainly of the form $(\forall X)[\Phi(X) \rightarrow (\exists Y)\Psi(X, Y)]$, where $\Phi$ and $\Psi$ are arithmetic formulas with set parameters, and can be considered as \emph{problems}. Given a statement $\Psf$ of this form,
	a set $X$ such that $\Phi(X)$ holds is an \emph{instance} of $\Psf$, and a set $Y$ such that $\Psi(X, Y)$ holds is a \emph{solution} to the $\Psf$-instance $X$. In this paper, we shall consider exclusively statements of this kind.

Friedman and Simpson~\cite{Friedman2000Issues}, and later Montalban~\cite{Montalban2011Open}, asked about the reverse mathematical strength of the ordered variable word. The statement $\ovw{k}{\ell}$ is known to be provable in $\rca + \piooca$. Our main result is a direct combinatorial proof of $\ovw{2}{\ell}$ in $\rca + \aca$.

\begin{theorem}\label{thm:aca-ovw2}
	For every $\ell \geq 2$, $\rca + \aca \vdash \ovw{2}{\ell}$.
\end{theorem}

On the lower bound hand, Miller and Solomon~\cite{Miller2004Effectiveness} constructed a computable instance $c$ of $\ovw{2}{2}$ with no $\Delta^0_2$ solution, and deduced that $\rca + \wkl$ does not prove $\vw{2}{2}$. Indeed, seeing the instance $c$ of $\ovw{2}{2}$ as an instance of $\vw{2}{2}$, and noticing that the jump of a solution to $\vw{2}{2}$ gives a solution to $\ovw{2}{2}$, one can deduce that $c$ has no low $\vw{2}{2}$-solution. In this paper, we improve their lower bound by constructing a computable instance of $\ovw{2}{2}$ whose solutions are of DNC degree relative to $\emptyset'$.

\subsection{Organization of the paper}
In section~\ref{sect:ht-ovw}, we shall give a simple proof of the ordered variable word for binary strings ($\ovw{2}{\ell}$) using the finite union theorem. Then, in section~\ref{sect:ovw-aca}, we provide a direct combinatorial proof of the same statement over $\rca + \aca$.
Finally, in section~\ref{sect:ovw-lower-bounds}, we give a new lower bound on the strength of $\ovw{2}{\ell}$ using a computable version of Lovasz Local Lemma.
%Finally, in section~\ref{sect:fut-lower-bounds}, we improve a lower bound of Townser about a combinatorial lemma in his proof of the finite union theorem.

\subsection{Notation}

Given two sets $A$ and $B$, we write $A < B$ for the formula $(\forall x \in A)(\forall y \in B) x < y$.
Given a set $A$, we write $A^{<\omega}$ for the set of finite $A$-valued strings. In particular, $2^{<\omega}$ is the set of binary strings. We denote by $\Pcal_{fin}(\Nb)$ the collection of finite \emph{non-empty} subsets of $\Nb$. Given two strings $\sigma, \tau \in A^{<\omega}$, $\sigma*\tau$ denotes their concatenation. We may also write $\sigma\tau$ when there is no ambiguity. Given a string or a sequence $X$ and some $n \in \omega$, we write $X \uhr n$ for the initial segment of $X$ of length $n$. In particular, $X \uhr 0$ is the empty string, written $\varepsilon$.
