
The proof of the previous section gave a very coarse computability-theoretic upper bound of the Ordered Variable Word theorem in terms of $\omega$-jumps. In this section, we give a direct combinatorial proof of $\ovw{2}{\ell}$ in $\rca + \aca$. Actually, every PA degree relative to $\emptyset'$ is sufficient to compute a solution of a computable instance of $\ovw{2}{\ell}$.
	We thereby answer a question of Miller and Solomon~\cite{Miller2004Effectiveness}.

\begin{theorem}\label{ovwth2}
For every $\ell \in\omega$,
every computable instance $c$ of $\ovw{2}{\ell}$,
every PA degree over $\emptyset'$ computes a solution to $c$.
\end{theorem}

A formalization of Theorem~\ref{ovwth2} yields
a proof of Theorem~\ref{thm:aca-ovw2}.

\begin{proof}[Proof of Theorem~\ref{thm:aca-ovw2}]
The proof of Theorem~\ref{ovwth2} can be
formalized within $\rca + \aca$. Indeed, the arguments require only arithmetical induction to be carried out, and
every model of $\rca + \aca$ is a model of
the statement ``For every set $X$, there is a set of PA degree over the jump of $X$.''
\end{proof}

Let us first introduce some notation.
For a finite set $F$ and a string $\sigma \in 2^{<\omega}$
let $\sigma_F$ be the binary string of length $|\sigma|$
defined by $\sigma_F(i) = \sigma(i)$ if $i \not \in F$, and $\sigma_F(i) = 1-\sigma(i)$ otherwise. Let $\leq_{lex}$ denote the shortlex order
on  $\omega^{<\omega}$, that is, the order with the shortest length first, and with the strings of same length sorted lexicographically.

In what follows, fix a coloring $\t{c}:2^{<\omega}\rightarrow \ell$,
and a string $\rho\in 2^{<\omega}$.

The main combinatorial lemma we use is
 Lemma~\ref{ovwprop1}.
As a warm up, we first prove the following
lemma \ref{ovwprop0}, which is a consequence of Lemma~\ref{ovwprop1}
and the proof is somehow similar but
much simpler. In the following lemma, one may think of $\rho_{P'}$ as a finite variable word, where the positions at $\t{P}$ are replaced by a same variable kind.


\begin{lemma}\label{ovwprop0}
For any
$P\subseteq \{0,\cdots,|\rho|-1\}$ with
$(\forall n\in P)[\rho(n) = 0]
\wedge |P|\geq \ell$,
there exist two
subsets $P'<\t{P}$ of $P$
with $\t{P}\ne\emptyset$
such that $\t{c}(\alpha) = \t{c}(\alpha_{\t{P}})$ where $\alpha = \rho_{P'}$.
\end{lemma}
\begin{proof}
Suppose $P = \{p_0 < \cdots < p_{m-1}\}$.
Let $\ell_0, \dots, \ell_m$ be defined by
$\ell_i = \t{c}(\rho_{\{p_0, \dots, p_{i-1}\}})$. In particular, $\ell_0 = \t{c}(\rho)$.
Since $|P| = m \geq \ell$,
so among $\ell_0,\cdots, \ell_m$,
there must exists $i<j$ such
that $\ell_i=\ell_j$.
Let $P' = \{p_0,\cdots,p_{i-1}\}$
(if $i=0$ then $P' = \emptyset$), and
$\t{P} = \{p_i,\cdots,p_{j-1}\}$,
let $\alpha = \rho_{P'}$.
Clearly $P'<\t{P}$ and $\t{P}\ne\emptyset$.
It is also easy to see that
$\t{c}(\alpha) = \ell_i = \ell_j = \t{c}(\alpha_{\t{P}})$.
\end{proof}

We now prove a technical lemma used in the proof of our main combinatorial
lemma (Lemma~\ref{ovwprop1}). The sequence in the following lemma is obtained by a simple greedy algorithm, with finitely many resets.

\begin{lemma}\label{ovwclaim1}
There exists a nonempty set
of colors $L\subseteq \{0,1,\cdots,\ell-1\}$,
$|L|+1$ many sets of binary strings
$\Gamma_0 = \{\tau^{\eta}\}_{\eta\in L},
\Gamma_1=\{\tau^\eta\}_{\eta\in L^2},
\cdots,\Gamma_{|L|}=\{\tau^\eta\}_{\eta\in L^{|L|+1}}$,
such that, letting
$$
\t{\eta} =
 \underbrace{\max L *\max L *\cdots*\max L}_{|L|+1\text{ many }}
$$
 and letting $\t{\rho} = \tau^{\t{\eta}}*0$, the following holds:
\begin{enumerate}
\item $\rho\prec\Gamma_0$ and $
\tau^{\eta}\prec\tau^\beta\Leftrightarrow
\eta<_{lex}\beta$;
\item $\t{\rho}(|\tau|) = 0$
for all $\tau\in \Gamma_i,i\leq |L|$;

\item for all $i\leq |L|$, $\eta\in L^{i+1}$,
let $\eta_0\prec\eta_1\prec\cdots\prec\eta_{i-1}$ denote
all nonempty predecessors of $\eta$, let
$Q = \big\{|\tau^{\eta_0}|,|\tau^{\eta_1}|,
\cdots,|\tau^{\eta_{i-1}}|\big\}$
(if $i=0$ then $Q=\emptyset$),
then $\t{c}(\tau^{\eta}_Q) = \eta(i)$;

\item let $P = \{|\tau^\eta|\}_{\eta\in L^{\leq |L|+1}}$,
for all subset $Q$ of $P$, all $\tau\succeq\t{\rho}$,
$\t{c}\big(\tau_Q\big)
\in L$.
\end{enumerate}
%\benoit{For (4) it seems that more is nedeed in the proof of the next lemma : not only $\t{c}(\tau)$ needs to an element of $L$ for $\tau \succeq \t{\rho}$, but also $\t{c}(\tau_P)$ for some set of positions $P$ (which is the case of course).}
Moreover,
$\Gamma_i,i\leq |L|$ is  computable in the jump of $\t{c}$,
 uniformly in $\rho$.


\end{lemma}
\begin{proof}
We firstly show how to find $\Gamma_0$.
Start with $L = \{0,1,\cdots,\ell-1\}$.
At step 1, try to find a string $\tau\in 2^{<\omega}$ such that
$\t{c}(\rho\tau) =  0$ and
let $\tau^0 = \rho\tau$. Then
try to find a $\tau$ such that
$\t{c}(\tau^00\tau) = 1$ and
let $\tau^1 = \tau^00\tau$. Generally, after $\tau^j$ is found,
try to find $\tau$ such that
$\t{c}(\tau^j0\tau)=j+1$
and let $\tau^{j+1} = \tau^j0\tau$ if
$\tau$ is found.
If during the above process, after $\tau^{j-1}$ is
defined ( $\tau^{-1} = \rho$ ),
there is no $\tau$ such that
$\t{c}(\tau^{j-1}0\tau) = j$,
then we start all over again
with $\rho$ replaced by $\rho_1 =
\tau^j0$ and with $L$ replaced by $ L \setminus \{j\}$.
%, i.e.,
%we try to find a sequence of string
%$\tau^i\succ\rho_1,i\in \{0,\cdots,l-1\}-\{j+1\}$
% with
%$\t{c}(\tau^i) = i$ and $
%i<k\Rightarrow\tau^i\prec \tau^k\wedge
%\tau^k(|\tau^i|) = 0$.

%Let $L_1$ denote the
%colors remaining after the search of $\Gamma_0$.
%Without loss of generality, suppose $L_1 = \{0,\cdots,k^*\}$.
%To find $\Gamma_1$,
%\begin{itemize}
%\item Try to find $\tau^{0i},i\in L_1$ such that
%$\t{c}(\tau^{0i}_{\{|\tau^0|\}}) = i$ and
%$i<k\Rightarrow \tau^{0i}\prec\tau^{0k}
%\wedge \tau^{0k}(|\tau^{0i}|) = 0$.
%
%\item Generally,
%at step $i+1\in L_1$, try to find
%$\tau^{ik},k\in L_1$ with $\tau^{ik}\succ\tau^{(i-1)k^*}0$ such that
%$\t{c}(\tau^{ik}_{\{|\tau^i|\}}) = k$ and
%$j<k\Rightarrow \tau^{ij}\prec\tau^{ik}
%\wedge \tau^{i(k+1)}(|\tau^{ik}|) = 0$.
%\end{itemize}
%
%If during the process, after $\tau^{ik}$ is found,
%$\tau^{i(k+1)}$ can not be found,
%i.e., there exists no $\tau$ such that
%$\t{c}((\tau^{ik}0\tau)_{\{|\tau^i|\}}) = k+1$,
%then let $\rho_2 = (\tau^{ik}0)_{\{|\tau^i|\}}$
%and start all over  again with
%$\rho$ replaced by $\rho_2$ to rebuild $\Gamma_0,\Gamma_1,\cdots$.

Generally, given a set of colors $L$ and after $\tau^{\beta}$ is found,
let $\eta$
be the immediate successor (with respect to $\leq_{lex}$ order restricted to $L$-strings)
 of $\beta$,
let $\eta_0\prec\eta_1\prec\cdots\prec\eta_{i-1}$ denote
all nonempty predecessors of $\eta$, let
$Q = \big\{|\tau^{\eta_0}|,|\tau^{\eta_1}|,
\cdots,|\tau^{\eta_{i-1}}|\big\}$
(if $i=0$ then $Q=\emptyset$),
we try to find $\tau$ such that
$\t{c}((\tau^{\beta}0\tau)_{Q}) = \eta(|\eta|-1)$.
If such a string $\tau$ does not exists then we
start  all over again with $\rho $
replaced by  $\tau^\beta0_{Q}$ and $L$ replaced by $L \setminus \{\eta(|\eta|-1)\}$.
If such $\tau$ exists then let $\tau^\eta =
\tau^\beta0\tau$.

Note that we have to
start over for at most $\ell-1$ times before we
ultimately succeed since
there are $\ell$ colors in total.
It is plain to check all the four items.
Also note that the sequence $\Gamma_0,\cdots,\Gamma_{|L|}$ is
$\t{c}'$-computable since we only need to
use the jump of $\t{c}$ to know whether the next $\tau^{\eta}$ can
 be found.



\end{proof}

\begin{lemma}\label{ovwprop1}

There exists a string $\t{\rho}\succ\rho$
 and a finite set $P\subseteq
 \big\{|\rho|,\cdots, |\t{\rho}|-1\big\}$ with
 $(\forall i\in P)[\t{\rho}(i) = 0]$
 such that
 for all $\sigma\succeq \t{\rho}$ there exists
 two subsets $P'<\t{P}$ of $P$ with $\t{P}\ne\emptyset$
 such that, letting $\alpha = \sigma_{P'}$,
$\t{c}(\alpha) = \t{c}(\alpha_{\t{P}}) = \t{c}(\alpha
 \uhr {\min \t{P} })$.
 Moreover, $|P|< \ell^{\ell+2}$, and $\t{\rho}, P$,
are computable in the jump of $\t{c}$, uniformly in $\rho$.
\end{lemma}
\begin{proof}
Let $L$ and $\t{\rho}$ satisfy Lemma~\ref{ovwclaim1}.
We claim that $\t{\rho}$ and $P = \{|\tau^\eta|\}_{\eta\in L^{\leq |L|+1}}$
satisfy the current lemma.
It is clear
by item 1 of Lemma~\ref{ovwclaim1} that $\t{\rho}\succ\rho$
and by item 2 of  Lemma~\ref{ovwclaim1} that
$(\forall i\in P)[\t{\rho}(i) = 0]$.

Fix an arbitrary $\sigma\succeq\t{\rho}$.
We now describe how to construct $P'$ and $\t{P}$.
Define $\ell_0, \dots, \ell_{|L|}$
and $p_0, \dots, p_{|L|}$ inductively by
$\ell_0 = \t{c}(\sigma)$,
$\ell_{i+1} = \t{c}(\sigma_{\{p_0,p_1,\cdots,p_i\}})$, and
$p_{i} =|\tau^{\ell_0\cdots \ell_{i}}|$ (where $\tau^{\ell_0\cdots \ell_{i}} \in \Gamma_i$).
%To construct $P'$ and $\t{P}$,
%consider the following  flipping bit process:
%
%
%\begin{itemize}
%\item Suppose $\t{c}(\sigma) = \ell_0$. By item 4
%of Lemma~\ref{ovwclaim1}, $\ell_0\in L$. Therefore
%$\tau^{\ell_0}$ is defined.
%Then, let
%$p_0 = |\tau^{\ell_0}|$, at step 1 imagine
%we flip the
%bit at position $p_0$ and $\sigma$ becomes
%$\sigma_{\{p_0\}}$.
%
%\item Suppose
%$\t{c}(\sigma_{\{p_0\}}) = \ell_1$.
%Then, let $p_1 = |\tau^{\ell_0\ell_1}|$, at step 2
%imagine we flip the bit  at position $p_1$ and
%$\sigma_{\{p_0\}}$ becomes $\sigma_{\{p_0,p_1\}}$.
%
%\item Suppose $\t{c}(\sigma_{\{p_0,p_1\}}) = \ell_2$.
%Then, let $p_2=|\tau^{\ell_0\ell_1\ell_2}|$, at  step 3
%imagine we flip the bit at position $p_2$
%and $\sigma_{\{p_0,p_1\}}$ becomes
%$\sigma_{\{p_0,p_1,p_2\}}$.
%
%\item In the $(i+1)^{th}$ step where $1\leq i\leq |L|$, suppose
%$\t{c}(\sigma_{\{p_0,p_1,\cdots,p_{i-1}\}}) = \ell_i$.
%Then, let $p_{i} =|\tau^{\ell_0\cdots \ell_{i}}|$
% imagine we flip the bit at position $p_{i}$
% and $\sigma_{\{p_0,\cdots,p_{i-1}\}}$ becomes
%$\sigma_{\{p_0,\cdots,p_{i-1},p_{i}\}}$.
%
%\end{itemize}
Since $\ell_0,\cdots,\ell_{|L|}\in L$ (by item 4 of Lemma~\ref{ovwclaim1}), there is some $i<j\leq |L|$
such that $\ell_i=\ell_j $.
Let $P' = \{p_0,\cdots,p_{i-1}\}$
(if $i=0$ then $P' = \emptyset$),
$\t{P}= \{p_i,\cdots,p_{j-1}\}$,
and let $\alpha = \sigma_{P'}$. We claim that
$\t{c}(\alpha) = \t{c}(\alpha_{\t{P}}) = \t{c}(\alpha\uhr \min \t{P})$.
Note that $\min \t{P}  =p_i= |\tau^{\ell_0\cdots \ell_i}|$.
Therefore
$\alpha\uhr \min \t{P} = \tau^{\ell_0\cdots \ell_i}_{P'}$.
By item 3 of Lemma~\ref{ovwclaim1},
we have $\t{c}(\tau^{\ell_0\cdots \ell_i}_{P'}) = \ell_i $.
Meanwhile, by definition of $\ell_i$,
$\t{c}(\sigma_{P'}) =
\t{c}(\alpha) = \ell_i$.
By definition of $\ell_j$,
$\t{c}(\sigma_{P'\cup \t{P}}) =
\t{c}(\alpha_{\t{P}}) = \ell_j$.
Thus, $\t{c}(\alpha) = \t{c}(\alpha_{\t{P}})
 = \t{c}(\alpha\uhr \min \t{P} )$.


\end{proof}

We say that $(\t{\rho}, P)$ is \emph{$\t{c}$-valid}
if $P$ and $\t{\rho}$ satisfy Lemma~\ref{ovwprop1}.
We say that $(P',\t{P})$ \emph{witnesses $\t{c}$-validity of $(\t{\rho}, P)$ for $\sigma \succeq \t{\rho}$} if $P' < \t{P}  \subseteq P$, and letting $\alpha = \sigma_{P'}$,
$\t{c}(\alpha) = \t{c}(\alpha_{\t{P}}) = \t{c}(\alpha
 \uhr {\min \t{P} })$.
Before proving Theorem~\ref{ovwth2}, we start with
the following simpler version.

\begin{theorem}\label{ovwth1}
For every $\ell\in\omega$, every computable instance  $c:2^{<\omega}\rightarrow
\ell$ of $\ovw{2}{\ell}$,
every $PA$ degree over $\emptyset''$ computes
a solution to $c$.
\end{theorem}
\begin{proof}
It suffices to compute, given a PA degree relative to $\emptyset''$,
an infinite binary sequence $Y\in 2^\omega$
together with a sequence of finite
sets $\t{P}_0<\t{P}_{1}<\cdots$ with
$(\forall i\in\omega)(\forall n\in \t{P}_i)
[Y(n) = 0]$ such that the following holds:
\begin{quote}
Let $Position = \big\{\min \t{P}_i : i \geq 1 \big\}$.
There is some $\t{\ell}<\ell$ such that for all subset $J$ of $\omega$, letting
 $\t{P}_J = \bigcup\limits_{i\in J}\t{P}_i$, then we have,
 $
(\forall p\in Position)\big[
c(Y_{\t{P}_J}\uhr p) = \t{\ell}\ ].
$ 	
\end{quote}


Using Lemma~\ref{ovwprop1}, we first construct
a $\emptyset'$-computable sequence of strings
 $\t{\rho}_0 \prec \t{\rho}_1 \prec\cdots$,
a sequence of finite sets $P_i\subseteq \big\{
|\t{\rho}_{i-1}|,\cdots, |\t{\rho}_i|-1
\big\}$ and a sequence of colorings $c_i:[\t{\rho}_i]^\preceq\rightarrow
L_i$ inductively as follows.
$\t{\rho}_0 = \varepsilon$ and $c_0 = c$.
Given $\t{\rho}_i$ and $c_i:[\t{\rho}_i]^\preceq\rightarrow
L_i$, let $\t{\rho}_{i+1} \succeq \t{\rho}_i$
and $P_i\subseteq
\big\{ |\t{\rho}_i|,\cdots,|\t{\rho}_{i+1}|-1\big\}$ be such that
$(\t{\rho}_{i+1}, P_i)$- is $c_i$-valid,
and let $c_{i+1}$ be the coloring of $[\t{\rho}_{i+1}]^{\preceq}$ which on $\sigma \succeq \t{\rho}_{i+1}$
associates $\langle P', \t{P}, j \rangle$ such that
$(P',\t{P})$ witnesses $c_i$-validity of $(\t{\rho}_{i+1}, P_i)$
for $\sigma$, and $c_i(\sigma_{P'}) = j$. If there are multiple such tuples, take the least one, in some arbitrary order. Note that the range of $c_i$ is some finite set $L_i$.
%
% as following:
%\begin{itemize}
%\item Let $\t{\rho}_1$, $P_0\subseteq
%\big\{ |\t{\rho}_0|,\cdots,|\t{\rho}_1|-1\big\}$ be such that:
%$(\forall n\in P_0)[\t{\rho}_1(n) = 0]$,
%for all $\sigma\succeq \t{\rho}_1$ there exists
%subsets $P'<\t{P}$ of $P_0$ with $\t{P}\ne\emptyset$ such that
%let $\alpha = \sigma_{P'}$, $c_0(\alpha) = c_0(\alpha_{\t{P}})$.
%By Lemma~\ref{ovwprop0} such $\t{\rho}_1, P_0$ exists.
%
%\item Now we define $c_1$ according to profile of $c_0$.
%For $\sigma\succeq\t{\rho}_1$,
%$c_1(\sigma) = (P',\t{P},j)$ if:
%$P'<\t{P}$ are subsets of $P_0$, $\t{P}\ne\emptyset$, and
%let $\alpha = \sigma_{P'}$, $c_0(\alpha) = c_0(\alpha_{\t{P}})=j$.
%If multiple $(P',\t{P},j)$ satisfy the above condition,
%then simply let $c_1(\sigma)$ be the "smallest" such $(P',\t{P},j)$.
%Here "smallest" is an arbitrary order.
%Clearly the range of $c_1$, namely $L_1$, is finite
%and by definition of $\t{\rho}_1, P_0$, $c_1$ is well defined
%on $[\t{\rho}_1]^\preceq$.
%
%\item Note that $c_1$ is computable since
%$c_0$ is computable. Therefore, apply proposition \ref{ovwprop1} on
%$c_1,\t{\rho}_1$, we can $\mbf{0'}$-compute $\t{\rho}_2\succ\t{\rho}_1$
%and $P_1\subseteq \big\{|\t{\rho}_1|+1,\cdots, |\t{\rho}_2|-1\big\}$
% with $(\forall n\in P_1)[\t{\rho}_2(n) = 0]$ such that:
%for all $\sigma\succeq \t{\rho}_2$ there exists
%subsets $P'<\t{P}$ of $P_1$ with $\t{P}\ne\emptyset$ such that
%let $\alpha = \sigma_{P'}$, $c_1(\alpha) = c_1(\alpha_{\t{P}})
% = c_1(\alpha\uhr_{\min\{\t{P}\}})$.
%
% \item Define $c_2$ according to profile
% of $c_1$. For $\sigma\succeq \t{\rho}_2$,
% $c_2(\sigma) = (P',\t{P},j)$ if:
%$P'<\t{P}$ are subsets of $P_1$, $\t{P}\ne\emptyset$, and
%let $\alpha = \sigma_{P'}$, $c_1(\alpha) = c_1(\alpha_{\t{P}})=
%c_1(\alpha\uhr_{\min\{\t{P}\}}) = j$.
%
%\item In this way, we guarantee that for all $i$:
%(a) $(\forall n\in P_{i-1})[\t{\rho}_i(n) = 0]$;
%(b) for any $\sigma\succeq \t{\rho}_i$,
% $c_i(\sigma) = (P',\t{P},j)$ if
% $P'<\t{P}$ are subsets of $P_{i-1}$, $\t{P}\ne\emptyset$, and
%let $\alpha = \sigma_{P'}$, $c_{i-1}(\alpha) = c_{i-1}(\alpha_{\t{P}})=
%c_{i-1}(\alpha\uhr_{\min\{\t{P}\}}) = j$;
%and (c) $c_i$ is well defined with finite range $L_i$.
%
%\end{itemize}

We now analyze for $\sigma\succeq \t{\rho}_i$
 what $c_i(\sigma)= \langle P',\t{P},j \rangle$ means.
Note that elements of $L_i,i\in\omega$
admit a natural partial order $\lhd$ as follows:
for
$\langle P'_{0},\t{P}_0,j_0 \rangle\in L_{i},
\langle P'_{1},\t{P}_1,j_1\rangle \in L_{i+1}$,
$\langle P'_{1},\t{P}_1,j_1 \rangle $ is an immediate
successor of $\langle P'_{0},\t{P}_0,j_0 \rangle$
if and only if $j_1 = \langle P'_{0},\t{P}_0,j_0\rangle$.
Clearly every $j\in L_i$ admit a unique
immediate predecessor.

\begin{claim}\label{ovwclaim2}
Fix some $n \geq 1$, and let
$
\t{\ell} \lhd \langle P'_0,\t{P}_0,j_0\rangle \lhd \dots \lhd \langle P'_{n-1},\t{P}_{n-1},j_{n-1}\rangle = c_n(\sigma)
$,
Let $P' = \bigcup_{i\leq n-1}P'_i$ and
$\alpha = \sigma_{P'}$.
Then for any subset $J$ of $\{0,\cdots, n-1\}$,
$$(\forall p\in \big\{ \min \t{P}_j : 1 \leq j \leq n-1\big\}\cup\{|\alpha|\})
\big[\ c(\alpha_{\t{P}_J}\uhr p)
 = \t{\ell}\ \big].$$
\end{claim}
\begin{proof}
First we prove the claim for $n=1$.
By definition of $c_1(\sigma) = \langle P'_0,\t{P}_0,j_0 \rangle$,
letting $\beta = \sigma_{P'_0}$,
$c_0(\beta) = c_0(\beta_{\t{P}_0})= j_0=\t{\ell}$.
In other words,
 for any subset $J\subseteq \{0\}$,
$$(\forall p\in \big\{ \min\{\t{P}_j : 1 \leq j \leq 0\}\big\} \cup\{|\beta|\})
\big[\ c(\beta_{\t{P}_J}\uhr p)
 = \t{\ell}\ \big].$$
So the claim holds for $n=1$.
Suppose now the claim
 holds for $n-1$.

Suppose $c_n(\sigma) = \langle P'_{n-1},\t{P}_{n-1},j_{n-1} \rangle$. Let $\beta = \sigma_{P'_{n-1}}$. We have $c_{n-1}(\beta) =c_{n-1}(\beta_{\t{P}_{n-1}}) = c_{n-1}(\beta\uhr \min \t{P}_{n-1} ) = j_{n-1} = \langle P'_{n-2},\t{P}_{n-2},j_{n-2}\rangle$. As $c_{n-1}(\beta) = \langle P'_{n-2},\t{P}_{n-2},j_{n-2}\rangle$ and as $\t{\ell} \lhd \langle P'_{n-2},\t{P}_{n-2},j_{n-2}\rangle$, by induction hypothesis, for any subset $J$ of $\{0,\cdots,n-2\}$ we have:
\begin{align}\label{ovweq1}
&c(\beta_{(\cup_{i\leq n-2} P_i')\cup \t{P}_J}) = \t{\ell}.
\end{align}

Let $\beta' = \beta_{\t{P}_{n-1}}$. As $c_{n-1}(\beta') = \langle P'_{n-2},\t{P}_{n-2},j_{n-2}\rangle$ and as $\t{\ell} \lhd \langle P'_{n-2},\t{P}_{n-2},j_{n-2}\rangle$, by induction hypothesis, for any subset $J$ of $\{0,\cdots,n-2\}$ we have:
\begin{align}\label{ovweq2}
&c(\beta'_{(\cup_{i\leq n-2} P_i')\cup \t{P}_J}) = \t{\ell}.
\end{align}

As $c_{n-1}(\beta\uhr \min \t{P}_{n-1} ) = \langle P'_{n-2},\t{P}_{n-2},j_{n-2}\rangle$ and as $\t{\ell} \lhd \langle P'_{n-2},\t{P}_{n-2},j_{n-2}\rangle$, by induction hypothesis, for any subset $J$ of $\{0,\cdots,n-2\}$ we have:
\begin{align}\label{ovweq4}
&(\forall p \in\big\{ \min \t{P}_j : 1\leq j\leq n-2\big\}\cup \big\{|\beta\uhr \min\t{P}_{n-1}|\big\})
\big[\ c(\beta_{(\cup_{i\leq n-2} P_i')\cup\t{P}_J}\uhr p) = \t{\ell}\ \big].
\end{align}
But $|\beta\uhr \min\t{P}_{n-1}| = \min \t{P}_{n-1}$. So (\ref{ovweq4}) means for any subset $J$ of $\{0,\cdots,n-2\}$ we have:
\begin{align}\nonumber
&(\forall p \in\big\{ \min \t{P}_j : 1\leq j\leq n-1\big\})
\big[\ c(\beta_{(\cup_{i\leq n-2} P_i')\cup\t{P}_J}\uhr p) = \t{\ell}\ \big].
\end{align}
Or equivalently, for any subset $J$ of $\{0,\cdots,n-1\}$ we have:
\begin{align}\label{ovweq3}
&(\forall p \in\big\{ \min \t{P}_j : 1\leq j\leq n-1\big\})\big[\ c(\beta_{(\cup_{i\leq n-2} P_i')\cup\t{P}_J}\uhr p) = \t{\ell}\ \big].
\end{align}

Now from \ref{ovweq1}, \ref{ovweq2} and \ref{ovweq3} we deduce that for any subset $J$ of $\{0,\cdots,n-1\}$ we have:
$$(\forall p\in \big\{ \min \t{P}_j : 1 \leq j \leq n-1\big\}\cup\{|\beta|\})\big[\ c(\beta_{(\cup_{i\leq n-2} P_i')\cup\t{P}_J}\uhr p) = \t{\ell}\ \big]$$
which completes the proof of the claim since
$\beta_{\cup_{i\leq n-2} P_i'} = \alpha$.

%Note that if $c_n(\sigma) = \langle P'_{n-1},\t{P}_{n-1},j_{n-1} \rangle$,
%then, letting $\beta = \sigma_{P'_{n-1}}$, we have
%$c_{n-1}(\beta) =c_{n-1}(\beta_{\t{P}_{n-1}}) =
%c_{n-1}(\beta\uhr \min \t{P}_{n-1} ) = j_{n-1}
% = \langle P'_{n-2},\t{P}_{n-2},j_{n-2}\rangle$.
%Let
% $P'' = \bigcup_{i\leq n-2}P'_i$,
% $\t{\beta} = \beta_{P''}$, $\beta' =
%  \t{\beta}\uhr \min \t{P}_{n-1}$,
%since
% $\t{\ell}\in L_0$ is also predecessor of $\langle P'_{n-2},\t{P}_{n-2},j_{n-2}\rangle$,
% by induction hypothesis,
% for any subset
% $J$ of $\{0,\cdots,n-2\}$,
%\begin{align}\label{ovweq1}
%&(\forall p \in\big\{ \min \t{P}_j : 1\leq j\leq n-2\big\}\cup\{|\t{\beta}|\})
%\big[\ c(\t{\beta}_{\t{P}_J}\uhr p) = \t{\ell}\
%\big],
%\\ \nonumber
%&(\forall p \in\big\{ \min \t{P}_j : 1\leq j\leq n-2\big\}\cup\{|\t{\beta}|\})
%\big[\
%c(\t{\beta}_{\t{P}_{n-1}\cup \t{P}_J}\uhr p)
%=\t{\ell}\ \big],
%\\ \nonumber
%&\ c(\beta'_{ \t{P}_J})
%=\t{\ell}.
%\end{align}
%But (\ref{ovweq1}) clearly implies
%for any subset $J$ of $\{0,\cdots,n-1\}$,
%$$
%(\forall p\in \big\{ \min \t{P}_j : 1\leq j\leq n-1\big\}\cup\{|\t{\beta}|\})
%\big[
%\ c(\t{\beta}_{\t{P}_J}\uhr p) = \t{\ell}
%\big].
%$$
%Since $\t{\beta} = \alpha$ thus the proof of the claim is completed.
\end{proof}

Let $\mcal{T}_0$ be the $\emptyset'$-computable set of all $\gamma$ such that
$(\forall i\leq |\gamma|)[\gamma(i)\in L_i]$,
$\gamma(i) \lhd \gamma(i+1)$ and $\gamma(|\gamma|-1) = c_{|\gamma|-1}(\t{\rho}_{|\gamma|})$.
Then, let $\mcal{T}$ be the downward closure of the set $\mcal{T}_0$
by the prefix relation. The tree $\mcal{T}$ is infinite by construction of the strings $\t{\rho_i}$, the colors $c_i$ and the sets $P_i$ : a witness for the $c_i$-validity of $(\t{\rho}_{i+1}, P_{i+1})$ for $\rho_{i+1}$ yields a node of $\mcal{T}_0$ of length $i+2$. The tree $\mcal{T}$ is also $\emptyset'$-computably bounded, and $\emptyset''$-computable.
Let $j_0*\langle P'_0,\t{P}_0,j_0\rangle *\langle P'_1,\t{P}_1,j_1\rangle *\cdots$
be an infinite path through $\mcal{T}$ computed by any PA degree over $\emptyset''$.
By construction, $\langle P'_i,\t{P}_i,j_i\rangle \lhd \langle P'_{i+1},\t{P}_{i+1},j_{i+1} \rangle$.
Let $X = \bigcup_{i\in\omega}\t{\rho}_i$,
$P' = \bigcup_{i\in\omega} P'_i$
and let $Y = X_{P'}$.
Clearly  $(\forall i \forall n\in \t{P}_i)[Y(n) = 0]$
and $Y$ is computable in the given PA degree relative to $\emptyset''$.
Therefore,
letting $Position = \big\{\min \t{P}_i : i \geq 1\big\}$, it suffices
to show that
for all subsets $J$ of $\omega$,
$$
(\forall p\in Position)\big[
c(Y_{\t{P}_J}\uhr p) = j_0\ ].
$$

Without loss of generality,
suppose $p = \min \t{P}_n$
and $J\subseteq \{0,\cdots,n-1\}$.
Since $j_0*\langle P'_0,\t{P}_0,j_0\rangle*\langle P'_1,\t{P}_1,j_1\rangle*\cdots
 \langle P'_n,\t{P}_n,j_n\rangle$ is an initial segment of some element in
 $\mathcal{T}_0$,
there must exist some
$N>n$
 such that
 $c_N(\t{\rho}_{N+1}) = \langle P'_{N-1},\t{P}_{N-1},j_{N-1}\rangle$.
 Let $\sigma = \t{\rho}_{N+1}, \alpha =\sigma_{P'}$. Clearly
 $\alpha\prec Y\wedge |\alpha|>p$. Moreover,
 by Claim~\ref{ovwclaim2},
 $c(\alpha_{\t{P}_J}\uhr p) = j_0$.
 Thus $c(Y_{\t{P}_J}\uhr p) = j_0$.

\end{proof}

Finally, we slightly modify the proof of Theorem~\ref{ovwth1}
to derive Theorem~\ref{ovwth2}.

\begin{proof}[Proof of Theorem~\ref{ovwth2}]
The main point is to make the tree
$\mcal{T}$ $\emptyset'$-computable.
To ensure this, after we obtain
$\t{\rho}_i,c_i$, we do not
directly go to $\t{\rho}_{i+1}$.
Instead, we $\emptyset'$-compute
$\t{\rho}_i^{0}\prec\t{\rho}_i^1\prec\cdots
\prec\t{\rho}_i^{r_i}$
such that $\t{\rho}_i^0\succ\t{\rho}_i$ and
$c_i\big(
\{\tau:\tau\succeq \t{\rho}_i^{r_i} \}\big)
 \subseteq c_i\big(\big\{\t{\rho}_i^0,\cdots,\t{\rho}^{r_i}_i \big\}
 \big)$. Then we $\emptyset'$-compute $\t{\rho}_{i+1}\succ
 \t{\rho}_i^{r_i}$ as in the proof of Theorem~\ref{ovwth1}.
 Note that this indeed can be
 achieved using $\emptyset'$ since
 $c_i$ is computable.
 Define $\mcal{T}$ to be
 the set of all $\gamma$ such that
 $(\forall i\leq |\gamma|)[\gamma(i)\in L_i]$,
$\gamma(i) \lhd \gamma(i+1)$,
 and either $|\gamma| = 1\wedge \gamma\in L_0$ or
 there exists $\t{\rho}_{|\gamma|-1}^u$
 with $c_{|\gamma|-1}(\t{\rho}_{|\gamma|-1}^u)
 =\gamma(|\gamma|-1)$.
 It is easy to see
that $\mcal{T}$ is $\emptyset'$-computable
since $c_i$ is computable for all $i$ and
the sequences $\langle c_i : i \in \omega \rangle$ and
$\langle \t{\rho}_i^v : i\in\omega, v\leq r_i \rangle$
are $\emptyset'$-computable.

Now we show that $\mcal{T}$ is a tree.
Suppose $\gamma\in \mcal{T}$,
$|\gamma| = n+1$ with $n \geq 1$, and  $c_n(\t{\rho}_n^u)
 =\gamma(n) = \langle P',\t{P},j \rangle \in L_n$.
 We claim that $\gamma \uhr n \in \mcal{T}$.
 If $n = 1$, then $\gamma \uhr 1 \in L_0 \subseteq \mcal{T}$.
 Otherwise, let
$\langle Q',\t{Q},k \rangle \in L_{n-1}$ be the predecessor
  of $\langle P',\t{P},j \rangle$.
We need to show that
there exists $\t{\rho}_{n-1}^v$ such that
$c_{n-1}(\t{\rho}_{n-1}^v) = \langle Q',\t{Q},k \rangle$.
$c_n(\sigma) = \langle P',\t{P},j \rangle$ implies
that, letting $\alpha = \sigma_{P'}$,
$c_{n-1}(\alpha) = c_{n-1}(\alpha\uhr \min \t{P})
 = j = \langle Q',\t{Q},k \rangle$.
Note that $\alpha\succeq \t{\rho}_{n-1}^{r_{n-1}}$
since $P'> |\t{\rho}_{n-1}^{r_{n-1}}|$.
But $c_{n-1}\big( \{\tau:\tau\succeq \t{\rho}_{n-1}^{r_{n-1}}\}\big)
\subseteq c_{n-1}\big(\{\t{\rho}_{n-1}^0,\cdots,\t{\rho}_{n-1}^{r_{n-1}}\} \big)$.
Therefore there exists $\t{\rho}_{n-1}^u$ such that
$c_{n-1}(\t{\rho}_{n-1}^u) = \langle Q',\t{Q},k \rangle$.
It follows that $\gamma \uhr n \in \mcal{T}$ and that $\mcal{T}$ is a tree.
%Reapply the argument with $\sigma$ replaced
%by $\t{\rho}^u_{n-1}$, $c_n$ replaced by $c_{n-1}$
%and $(P',\t{P},j)$ replaced by $(Q'_{n-1},\t{Q}_{n-1},k_{n-1})$,
%since $\t{\rho}^u_{n-1}\succ \t{\rho}^{r_{n-2}}_{n-2}$,
%we have that for some $\t{\rho}^r_{n-2}$,
% $c_{n-2}(\t{\rho}^r_{n-2})
% = (Q'_{n-2},\t{Q}_{n-2},k_{n-2})$.
% Reapplying the argument $n-m$ times produces
% a $\t{\rho}_{m}^v$ with
% $c_m(\t{\rho}_m^v) = (Q',\t{Q},k)$.
Any PA degree relative to $\emptyset'$ computes an infinite path through $\mcal{T}$. The rest of the proof goes exactly the same
 as Theorem~\ref{ovwth1}.
\end{proof}

We now give an alternative proof of Theorem~\ref{ovwth2} based on the definitional complexity of the solutions of~$c$.

\begin{proof}[Second proof of Theorem~\ref{ovwth2}]
Let $P_0, P_1, \dots$ be the $\emptyset'$-computable sequence defined in the proof of Theorem~\ref{ovwth1}. We have seen that there exists an infinite ordered variable word such that the $n$th variable kind appears before the position $\max P_n$. Let $\mcal{T}$ be the tree of all finite ordered variable words which are finite solutions to $c$ and such that the $n$th variable appears before the position $\max P_n$. By the previous observation, the tree is infinite, $\emptyset'$-computable, and $\emptyset'$-computably bounded. Any PA degree relative to $\emptyset'$ computes an infinite variable word which, by construction of $\mcal{T}$, is a solution to $c$. This completes the proof of Theorem~\ref{ovwth2}.
\end{proof}

Note that the above proof can be slightly modified to obtain a proof of a sequential version of the ordered variable word.

\begin{statement}
$\mathsf{Seq}\ovw{n}{\ell}$ is the statement ``If $c_0, c_1, \dots$ is a sequence of $\ell$-colorings of a fixed alphabet $A$ of cardinality $n$, there exists a variable word $W$ such that for every $i \in \omega$ and every $\bar b \in A^i$, $\{W(\bar b\bar a) : \bar a \in A^{<\infty}\}$ is monochromatic for $c_i$.''
\end{statement}

\begin{theorem}
For every computable instance $c_0, c_1, \dots$ of $\mathsf{Seq}\ovw{2}{\ell}$,
every PA degree relative to $\emptyset'$ computes a solution to $\bar c$.
\end{theorem}
\begin{proof}
	The proof is similar to Theorem~\ref{ovwth2}. Using Lemma~\ref{ovwprop1}, we first construct a $\emptyset'$-computable sequence of strings
 $\t{\rho}_0 \prec \t{\rho}_1 \prec\cdots$,
a sequence of finite sets $P_i\subseteq \big\{
|\t{\rho}_{i-1}|,\cdots, |\t{\rho}_i|-1
\big\}$ and a sequence of colorings $d_i:[\t{\rho}_i]^\preceq\rightarrow
L_i$ inductively as follows.
$\t{\rho}_0 = \varepsilon$ and $d_0 = c_0$.
Given $\t{\rho}_i$ and $d_i:[\t{\rho}_i]^\preceq\rightarrow
L_i$, let $\t{\rho}_{i+1} \succeq \t{\rho}_i$
and $P_i\subseteq
\big\{ |\t{\rho}_i|,\cdots,|\t{\rho}_{i+1}|-1\big\}$ be such that
$(\t{\rho}_{i+1}, P_i)$- is $d_i$-valid,
and let $d_{i+1}$ be the coloring of $[\t{\rho}_{i+1}]^{\preceq}$ which on $\sigma \succeq \t{\rho}_{i+1}$
associates $\langle P', \t{P}, j, k \rangle$ such that
$(P',\t{P})$ witnesses $d_i$-validity of $(\t{\rho}_{i+1}, P_i)$
for $\sigma$, $d_i(\sigma_{P'}) = j$ and $c_{i+1}(\sigma_{P'}) = k$. Note that the main difference with the previous construction is that we handle more and more colorings among $c_0, c_1, \dots$ at each level. The remainder of the proof is the same as in Theorem~\ref{ovwth2}.
\end{proof}

The theorem above is optimal, in that we can obtain the following reversal.

\begin{theorem}
There is a computable instance $c_0, c_1, \dots$ of $\mathsf{Seq}\ovw{2}{2}$,
such that every solution is of PA degree relative to $\emptyset'$.
\end{theorem}
\begin{proof}
Let $R_0, R_1, \dots$ be a uniformly computable sequence of sets
such that for every $e$,
if $\Phi^{\emptyset'}_e(e) \downarrow = 0$ then $R_e$ is finite,
and if $\Phi^{\emptyset'}_e(e) \downarrow = 1$ then $R_e$ is cofinite.
In particular, any function $f : \omega \to 2$ such that $f(e)$ gives
a side of $R_e$ which is infinite, is DNC$_2$ relative to $\emptyset'$,
hence of PA degree relative to $\emptyset'$.
Let $c_i : 2^{<\infty} \to 2$ be defined by $c_i(\sigma) = 1$ iff $|\sigma| \in R_i$, and let $W$ be a solution to $\bar c$, that is, a variable word $W$ such that for every $i \in \omega$ and every $\bar b \in A^i$, $\{W(\bar b\bar a) : \bar a \in A^{<\infty}\}$ is monochromatic for $c_i$.
We claim that $W$ computes such a function $f$.
Given $e$, let $f(e) = c_e(W(\bar b))$, where $\bar b  \in 2^e$ is arbitrary (this is well-defined, since $c_e(\bar b)$ depends only on the length of $\bar b$). By definition of $W$, $\{W(\bar b\bar a) : \bar a \in A^{<\infty}\}$ is monochromatic for $c_e$, the color of $c_e(W(\bar b))$ appears infinitely often in $R_e$. Therefore, $W$ is of PA degree relative to $\emptyset'$. This completes the proof.
\end{proof}
