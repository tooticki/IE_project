
Towsner~\cite{Towsner2012simple} gave a combinatorially simple proof of the Finite Union Theorem in $\aca^{+}$. He introduced the notion of full-match, and proved its existence over $\aca$ in~\cite[Lemma 2.7]{Towsner2012simple}.  However, the existence of a full-match is not known to be equivalent to $\aca$. This notion is the cornerstone of Towsner's proof, as having low${}_n$ full-matches for some fixed $n$ would be sufficient to obtain arithmetical solutions to the Finite Union Theorem. In this section, we improve the lower bound of the existence of a full-match by showing that it cannot be proven in $\wkl$.

\begin{definition}
Fix a coloring $c : \Pcal_{fin}(\Nb) \to r$.
Let $\Bcal \subseteq \Pcal_{fin}(\Nb)$ be a finite collection,
and let $\Ical \subseteq \Pcal_{fin}(\Nb) - \Bcal$ be an IP collection.
\begin{itemize}
	\item[(i)] $\Bcal$ \emph{left-matches} $\Ical$ if for every $S \in \Ical$, there is some $B \in \Bcal$ such that $c(B) = c(B \cup S)$. 
	\item[(ii)] $\Bcal$ \emph{right-matches} $\Ical$ if for every $S \in \Ical$, there is some $B \in \Bcal$ such that $c(S) = c(B \cup S)$. 
	\item[(iii)] $\Bcal$ \emph{full-matches} $\Ical$ if for every $S \in \Ical$, there is some $B \in \Bcal$ such that $c(B) = c(B \cup S) = c(S)$. 
\end{itemize}
\end{definition}

Note that in Towsner's paper~\cite{Towsner2012simple}, a right-match is called a half-match. A full-match is both a left-match and a right-match, but the converse is not true in general.

\begin{statement}
	$\lmfut_r$ denotes the statement ``For every coloring $c :\Pcal_{fin}(\Nb) \to r$, there is a finite collection $\Bcal \subseteq \Pcal_{fin}(\Nb)$  and an IP collection $\Ical \subseteq \Pcal_{fin}(\Nb) - \Bcal$ such that $\Bcal$ left-matches $\Ical$.
	The statements $\rmfut_r$ and $\rmfut_r$ are defined accordingly for the notions of right-match and full-match.
\end{statement}

Towsner~\cite[Lemma 2.5]{Towsner2012simple} proved $\rmfut_r$ over $\rca$. He also constructed in~\cite[Theorem 3.8]{Towsner2012simple} a computable instance of $\lmfut_2$ with no computable solution, therefore showing that $\rca \nvdash \lmfut_2$. We now adapt his proof to show that $\rca + \wkl \nvdash \lmfut_2$, by proving that $\lmfut_2$ implies the existence of hyperimmune functions.

The following theorem combines the techniques of Towsner~\cite[Theorem 3.8]{Towsner2012simple} and Csima and Mileti~\cite[Theorem 4.1]{Csima2009strength}. A familiarity with the mentioned proofs will simplify drastically the understanding of the proof of Theorem~\ref{thm:lmfut-opt}.



\begin{theorem}\label{thm:lmfut-opt}
There is a computable instance of $\lmfut_2$ such 
that every solution is of hyperimmune degree.
\end{theorem}
\begin{proof}
The strategy is the following: We will build a computable coloring $c : \Pcal_{fin}(\Nb) \to 2$ such that for every finite collection $\Bcal$ which left-matches an IP collection $\Ical$ we satisfy the following requirement for each $e \in \Nb$:
\begin{quote}
$\Rcal_e$: If $\Phi_e$ is total, then
there is an input $k_e$ such that 
$\Ical \cap [k_e, \Phi_e(k_e)] = \emptyset$.
\end{quote}
Here, $[a, b] = \{a, a+1, \dots, b \}$.
Suppose we have constructed $c$, and let $(\Bcal, \Ical)$ be such that $\Bcal$ left-matches~$\Ical$.
Let $p_\Ical$ be the $\Ical$-computable function which on input
$n$ searches for a set $S \in \Ical$ such that $\min S > n$ and outputs $\max S$. For every $e \in \Nb$, $p_\Ical(k_e) > \Phi_e(k_e)$.
Therefore $p_\Ical$ is not dominated by any computable function, hence is hyperimmune. We now explain how to build the coloring $c$.

Using a movable marker procedure, define for every $e \in \Nb$ and at every stage $s$ two inputs $x_{e,s}, y_{e,s} > e$ ordered as follows
$$
x_{0,s} < y_{0,s} < x_{1,s} < y_{1,s} < \dots 
$$
Moreover, if $\Phi_{e,s}(x_{e,s})\downarrow$, then $\Phi_{e,s}(x_{e,s}) < y_{e,s}$, and if $\Phi_{e,s}(y_{e,s})\downarrow$, then $\Phi_{e,s}(y_{e,s}) < x_{e+1,s}$. We can show by induction that each marker is moved at most finitely many times, and therefore $x_e = \lim_s x_{e,s}$ and $y_e = \lim_s y_{e,s}$ exist for every $e$. For every $e, s \in \Nb$, let $X_{e,s} = [x_{e,s}, \Phi_{e,s}(x_{e,s})]$ if $\Phi_{e,s}(x_{e,s})\downarrow$, and $X_{e,s} = \emptyset$ otherwise. The set $Y_{e,s}$ is defined accordingly.
	According to the previous observation, $X_e = \lim_s X_{e,s}$ and $Y_e = \lim_s Y_{e,s}$ exist for every $e \in \Nb$, and satisfy
$$
X_{0,s} < Y_{0,1} < X_{1,s} < Y_{1,s} < \dots
$$
The construction will work as follows. Using the combinatorics of Towsner~\cite[Theorem 3.8]{Towsner2012simple} (to be explained later), for every $e$, we will pick two sets $S \subseteq X_{e,s}$ and $T \subseteq Y_{e,s}$, and will ensure that $\{S, T, U\} \not \subseteq \Ical$ for every solution $(\Bcal, \Ical)$ to $c$, and cofinitely many finite sets $U$. If $\Phi_e$ is partial, our requirement is vacuously satisfied. If $\Phi_e$ is total, at some finite stage $t$, there will be some input $k_{S,T} \in \Nb$ such that $\Phi_{e,t}(k_{S,T})\downarrow$ and $\{S, T, U\} \not \subseteq \Ical$ for every $U \subseteq [k_{S,T}, \Phi_{e,t}(k_{S,T})]$. This way, we will have ensured that if $S$ and $T$ both belong to $\Ical$, then $\Ical \cap [k_{S,T}, \Phi_{e,t}(k_{S,T})] = \emptyset$. Since this is checked at a finite stage $t$, we then pick the next pair $S' \subseteq X_{e,t}$ and $T' \subseteq Y_{e,t}$ and do the same procedure, and so on until we have consumed all the pairs over $X_{e,t}$ and $Y_{e,t}$.
Therefore, we will have ensured that either $\Ical \cap X_e = \emptyset$, or $\Ical \cap Y_e = \emptyset$, or there is some $S \subseteq \Ical \cap X_e$ and $T \in \Ical \cap Y_e$
such that $\Ical \cap [k_{S,T}, \Phi_{e,t}(k_{S,T})] = \emptyset$. 

We say that a requirement $\Rcal_e$ is \emph{ready} at stage $s$ if 
$\Phi_{e,s}(x_{e,s})\downarrow$ and $\Phi_{e,s}(y_{e,s})\downarrow$.
Given a pair of finite sets $S^0 < S^1$, $\Rcal_e$ is \emph{$(S^0, S^1)$-satisfied for $\Rcal_e$} at stage $s$
if there is some $k > S^1$ such that $\Phi_{e,s}(k)\downarrow$
and for every set $A < S^0$, there is some $u < 2$ such that
for every $B \subseteq [k, \Phi_{e,s}(k)]$, $c(A) \neq c(A \cup S^u \cup B)$. In other words, $\Rcal_e$ is $(S^0,S^1)$-satisfied at stage $s$ if for every solution $(\Bcal, \Ical)$ to $c$ such that $S^0, S^1 \in \Ical$, $\Ical \cap [k, \Phi_{e,s}(k)] = \emptyset$. Note that a requirement may be ready at some stage, but not a later stage, whereas if $S^0 < S^1$ is satisfied for $\Rcal_e$ at some stage, then will always remain satisfied.
A requirement $\Rcal_e$ is \emph{satisfied} at stage $s$ if either $\Phi_e$ is partial, or it is $(S^0, S^1)$-satisfied for every pair $S^0 \subseteq X_{e,s}$ and $S^1 \subseteq Y_{e,s}$.

At any stage $s$ of the construction and for every $e < s$ such that $\Rcal_e$ is ready, we will have distinguished two sets $S^0_{e,s} \subseteq X_{e,s}$ and $S^1_{e,s} \subseteq Y_{e,s}$ which currently receive attention. They are chosen two be the least pair (in an arbitrary fixed order) which is not yet satisfied for $\Rcal_e$. The following two definitions are defined by Towsner~\cite[Theorem 3.8]{Towsner2012simple}.

A \emph{primary $s$-decomposition} of a set $B$, where $s = \max B$ is a tuple $i, u, Z, D$ such that $B = Z \cup S^u_{e,s} \cup D$, $Z < S^u_{e,s} < D$,
neither $Z$ nor $D$ contains $S^{1-u}_{i,s}$ as a subsequence, and there is no primary $s$-decomposition of $D$. Clearly, there is at most one primary $s$-decomposition of $B$.

We say that $B$ \emph{contains $e$ with polarity $v$} if there is a primary $\max B$-decomposition $i, u, Z, D$ of $B$ with either $e = i$ and $v = u$,
or $e$ contained in $Z$ with polarity $|v-u|$. Observe that whenever $B$ contains $e$, $B = Z \cup S^u_{e,t} \cup D$ for some $t \leq \max B$.

We now define our coloring by stages. At stage $s$, supposed we have already decided $c(B')$ whenever $\max B' < s$. Let $B$ be such that $\max B = s$. If $B$ has a primary $s$-decomposition $B = Z \cup S^u_{e,s} \cup D$, we set $c(B) = c(Z)$ if $u = 0$ and $c(B) \neq c(Z)$ if $u = 1$. If there is no primary $s$-decomposition of $B$, we set $c(B) = 0$. This completes the construction. We now turn to the verification.


\begin{claim}
For a total Turing functional $\Phi_e$, all $S \subseteq X_e$ and all $T \subseteq Y_e$, $\Rcal_e$ is $(S,T)$-satisfied at some stage. 
\end{claim}
\begin{proof}
Suppose not. Let $S \subseteq X_e$ and $T \subseteq Y_e$ be the least pair (in the previously fixed order) such that $\Rcal_e$ is not $(S,T)$-satisfied at any stage. Let $s$ be a stage such that for all $t \geq s$, $X_{e,t} = X_e$, $Y_{e,t} = Y_e$, and $\Rcal_e$ is $(S',T')$-satisfied for all the previous pairs $(S', T')$. By construction, $S^0_{e,t} = S$ and $S^1_{e,t} = T$.

It is easy to see that for any $B \geq s$, there is a $v_B$ such that
$A \cup S^u_e \cup B$ contains $e$ with polarity $|v_B - u|$ for all $A < \in \Bcal$. We now prove by induction on the length of $B$ that for all $B > s_1$ and $A < S^0_{e,t}$, $c(A \cup S^{v_B}_e \cup B) = c(A)$ and $c(A \cup S^{1-v_B}_e \cup B) \neq c(A)$. Given $u < 2$, let $D = A \cup S^u_e \cup B$. In particular, $D$ admits a primary $\max B$-decomposition $Z \cup S^{u'}_i \cup B'$. If we just have $e = i$, then $v_B = 0$ and the claim follows immediately from the decomposition of the coloring. Otherwise, we have two cases.
In the first case, $u' = 0$. Then $c(D) = c(Z)$ and $Z$ contains $e$ with polarity $|v_B - u|$. By induction hypothesis applied to $Z \setminus (A \cup S^u_e)$, $c(D) = c(Z) = c(A)$ if $u = v_B$ and $c(D) = c(Z) \neq c(A)$ if $u \neq v_B$. In the second case, $u' = 1$. Then $c(D) \neq c(Z)$ and $Z$ contains $e$ with polarity $1-|v_B-u|$. By induction hypothesis applied to $Z \setminus (A \cup S^u_e)$, $c(D) \neq c(Z) \neq c(A)$ if $u = v_B$, so $c(D) = c(A)$, and $c(D) \neq c(Z) = c(A)$ if $u \neq v_B$. This completes the induction and the proof of the claim.
\end{proof}

Let $(\Bcal, \Ical)$ be a solution to $c$.

\begin{claim}
	For every total Turing functional $\Phi_e$, there is an input $k$
	such that $\Ical \cap [k, \Phi_e(k)] = \emptyset$.
\end{claim}
\begin{proof}
By the padding lemma, we can assume that for all $A \in \Bcal$,
$A < e$. 
If $\Ical \cap [x_e, \Phi_e(x_e)] = \emptyset$ or $\Ical \cap [y_e, \Phi_e(y_e)] = \emptyset$, the we are done. Otherwise, let $S \subseteq X_e$ and $T \subseteq Y_e$ be such that $S, T \in \Ical$.
By assumption, $x_{e,s}, y_{e,s} > e$ for every $s \in \Nb$.
It follows that $A < e < S$. By the previous claim, $\Rcal_e$ is $(S,T)$-satisfied at some stage. Unfolding the definition, there is some $k > T$
such that $\Phi_e(k)\downarrow$ and for every $A < S$ (and in particular for every $A \in \Bcal$), either for every $B \subseteq [k, \Phi_e(k)]$, $c(A) \neq c(A \cup S \cup B)$,
or for every $B \subseteq [k, \Phi_e(k)]$, $c(A) \neq c(A \cup T \cup B)$.
In any case, $\Ical \cap [k, \Phi_e(k)] = \emptyset$. This completes the proof of the claim.
\end{proof}

As explained above, it follows that $\Ical$ is of hyperimmune-degree.
This completes the proof of Theorem~\ref{thm:lmfut-opt}.
\end{proof}

\begin{corollary}
$\rca + \wkl \nvdash \lmfut_2$
\end{corollary}
\begin{proof}
	By the relativized Hyperimmune-free Basis Theorem~\cite{Jockusch197201},
	this is a model of $\rca + \wkl$ with only hyperimmune-free degrees.
	By Theorem~\ref{thm:lmfut-opt}, this is not a model of $\lmfut_2$.
\end{proof}

Blass, Hirst and Simpson built a simple computable instance of the finite union theorem such that every solution computes the halting set: Given a finite set $S = \{x_0 < x_1 < \dots < x_n\}$, a gap $(x_i, x_{i+1})$ is \emph{large} if $\emptyset'\uh x_i = \emptyset'_{x_{i+1}} \uh x_i$,
	and is \emph{small} otherwise. A gap $(x_i, x_{i+1})$ is \emph{very small in $S$} if $\emptyset'_{x_n} \uh x_i \neq \emptyset'_{x_n} \uh x_{i+1}$.
	Letting $SG(S)$ and $VSG(G)$ denote the number of small gaps and of very small gaps in $S$, respectively, their coloring is simply defined by $c(S) = VSG(S) \mod 2$. Given an IP collection $\Ical$ homogeneous for $c$, they proved that $SG(S)$ is even for every $S \in \Ical$, and that the gap $(\max S, \min T)$ is large for every $S < T \in \Ical$. However, for this same instance, one can easily construct a computable full-match $(\Bcal, \Ical)$ with $|\Bcal| = 2$ as follows. Let $\Bcal = \{B_0, B_1\}$, where $SG(B_0) \mod 2 = SG(B_1) \mod 2 = 0$, $VSG(B_0)$ is odd, and $VSG(B_1)$ is even. Then $\Bcal$ full-matches $\Pcal_{fin}(\Nb) - \Bcal$. This motivates the following question.

\begin{question}
	Does $\lmfut_2$ or $\fmfut_2$ imply $\aca$ over $\rca$?
\end{question}
