\documentclass{amsart}

%\usepackage[cp866]{inputenc}
%\usepackage[russian,english]{babel}

\usepackage{amssymb}
\usepackage{mathrsfs}
%\usepackage{amsbib}

\usepackage{cite}

%=================================================
\theoremstyle{plain}
\newtheorem{theorem}{Theorem}
\newtheorem{lemma}{Lemma}
\newtheorem*{corollary}{Corollary}
\theoremstyle{definition}
\newtheorem*{remark}{Remark}

\numberwithin{equation}{section}
%=================================================

\begin{document}

\author[D.~A.~Popov]{D.~A.~Popov}

\title[Selberg Formula for Cofinite Groups]%
{Selberg Formula for Cofinite Groups\\ and the Roelke Conjecture}

\maketitle

%\setcounter{tocdepth}{2}
\tableofcontents
%---------------------------------------------------


\section{Introduction}
\label{sec1}

Let $H$~be the upper half-plane with the Poincar\'e metric
$ds^2=y^{-2}(dx^2+dy^2)$. A~\textit{cofinite group} is a discrete
group $\Gamma\subset\operatorname{PSL}(2,\mathbb{R})$ with
noncompact fundamental domain~$F$ whose area~$|F|$ with respect to
the invariant measure $d\mu=y^{-2}\,dx\,dy$ is finite. In what
follows, we only deal with cofinite groups~$\Gamma$.

The Laplace operator $\Delta=y^2(\partial_x^2+\partial_y^2)$
extends to be a self-adjoint operator on~$L^2(F,d\mu)$ with
continuous spectrum covering the interval $[1/4,\infty)$ and with
discrete spectrum~$\{\lambda_n\}$
($\Delta\varphi_n+\lambda_n\varphi_n=0$,
$0=\lambda_0<\lambda_1\leqslant\lambda_2\leqslant\cdots$,
$\varphi_n\in L^2(F,d\mu)$). Little is known about the structure of
the discrete spectrum. In particular, it is not known for what
groups~$\Gamma$ this spectrum is infinite.

Selberg posed the following question: what cofinite groups~$\Gamma$
satisfy the Weyl formula
\begin{equation}
\label{eq1.1}
N_\Gamma\biggl(T^2+\frac14\biggr)=\#\biggl\{n|\lambda_n\leqslant
T^2+\frac14\biggr\}\simeq \frac{|F|}{4\pi}\,T^2\qquad (T\to\infty)?
\end{equation}
Nowadays, such groups are said to be \textit{essentially cuspidal}.
Formula~\eqref{eq1.1} has been proved for a number of groups
(see~\cite{1,2,3,4}), in particular, for the congruence subgroups
of~$\operatorname{SL}(2,\mathbb{Z})$. However, all these groups
correspond to nongeneric points in the Teichm\"uller space. Roelke
(e.g., see~\cite{1}) conjectured that $N_\Gamma(T^2+1/4)\to\infty$
as $T\to\infty$.

The interest in these questions arose in connection with the
papers~\cite{5,6}. These papers, as well as~\cite{7,8,9}, provide a
number of sufficient conditions for the Weyl law~\eqref{eq1.1} to
be violated. Based on these results, Sarnak~\cite{2} conjectured
that neither the Weyl law nor even the Roelke conjecture holds for
generic cofinite groups~$\Gamma$.

One main approach to studying the spectrum~$\{\lambda_n\}$ is based
on the Selberg formula, and all the results of the present paper
are corollaries of this formula.

The Selberg formula for cofinite groups is given in
Sec.~\ref{sec2}. Symbolically, it can be written as
\begin{equation}
\label{eq1.2}
\sum_{n \geqslant 0}h(r_n)=
\Phi_\Gamma[h|\{\lambda_n\},\{N(P)\},\varphi] \qquad
\forall\,h\in\{h\}_S \quad
\biggl(\lambda_n=r_n^2+\frac14\biggr),
\end{equation}
where $\Phi_\Gamma$~is a functional on the space~$\{h\}_S$ (see
Sec.~\ref{sec2}). The functional~$\Phi_\Gamma$ depends on the
spectrum~$\{\lambda_n\}$, the set~$\{N(P)\}$ of norms of hyperbolic
conjugacy classes, the function~$\varphi$ defined by
$\varphi(s)=\det\Phi(s)$ (where $\Phi(s)$~is the scattering
matrix), and finitely many parameters such as~$|F|$, the number of
elliptic and parabolic classes, $\operatorname{tr}\Phi(1/2)$, etc.
The Selberg formula for a cocompact group~$\Gamma$ reads
\begin{equation}
\label{eq1.3}
\sum_{n \geqslant 0}h(r_n)=\Phi_\Gamma[h|\{\lambda_n\},\{N(P)\}].
\end{equation}
It was shown in~\cite{10} that, for strictly hyperbolic groups, the
Selberg formula (supplemented by an additional condition on the
function~$h$) implies that the spectrum~$\{\lambda_n\}$ satisfies
equations of the form
\begin{equation}
\label{eq1.4}
\sum_{n \geqslant 0}h(r_n)=\widetilde\Phi_\Gamma[h|\{\lambda_n\}\}].
\end{equation}
One aim of the present paper is to generalize this result to
arbitrary cofinite groups. We show (Theorem~\ref{th1}) that the
following analog of formula~\eqref{eq1.4} holds:
\begin{equation}
\label{eq1.5}
\sum_{n \geqslant 0}h(r_n)=
\widetilde\Phi_\Gamma[h|\{\lambda_n\},\{s_\alpha\}].
\end{equation}
Here $\{s_\alpha\}$~is the set of poles
$s_\alpha=\beta_\alpha+i\gamma_\alpha$ of~$\varphi$ such that
$\beta_\alpha<1/2$ and $\gamma_\alpha\ne0$. This set will be called
the \textit{resonance spectrum}.

We take various functions~$h$ and obtain relations that should be
satisfied by the discrete spectrum and the resonance spectrum. In
Sec.~\ref{sec6}, we compute the asymptotics
of~$\widetilde\Phi_\Gamma[h|\{\lambda_n\},\{s_\alpha\}]$ as $t \to
0$ for the case of $h(r)=e^{-tr^2}(r^2+p^2)^{-1}$ and use this
asymptotics to prove the Roelke conjecture (Theorem~\ref{th2}).

Note that if $\Gamma=\operatorname{SL}(2,\mathbb{Z})$, then
$s_\alpha=\rho_\alpha/2$, where the~$\rho_\alpha$ are the
nontrivial zeros of the Riemann zeta function, and
formulas~\eqref{eq1.5} specify relations between the
sets~$\{\lambda_n\}$ and~$\{\rho_\alpha\}$.

A preliminary version of Theorem~\ref{th1}, containing a number of
inaccuracies, was published in~\cite{11}. The second theorem
in~\cite{11}, which gave some sufficient conditions for the Roelke
conjecture to be true, was based on the assumption that the series
$D_\Gamma=\sum_{s_\alpha}|s_\alpha|^{-2}$ converges for any
group~$\Gamma$. I am grateful to the referee of~\cite{11} for
pointing out that this assumptions has never been proved and is
most likely false.



\section{Preliminaries}
\label{sec2}

This section provides some insight into the Selberg formula for
cofinite groups and introduces notation to be used in the paper.
All the information given here can be found
in~\cite{1,3,12,13,14,15,16,17}.

Throughout the paper, we assume that the functions~$h(\,\cdot\,)$
belong to the class~$\{h\}_S$, that is, satisfy the following
conditions:

1.~$h(r)=h(-r)$.

2.~The function~$h$ is holomorphic in the strip
$\{|\operatorname{Im}r|\leqslant 1/2+\varepsilon,\
\varepsilon>0\}$.

3.~In this strip, one has $|h(r)|=O(1+|r|^2)^{-1-\varepsilon}$
($|r|\to\infty$).

By~$g$ we denote the Fourier transform of~$h$,
\begin{equation}
\label{eq2.1}
g(y)=\frac{1}{2\pi}\int_{-\infty}^\infty h(r)e^{-iry}\,dr,\qquad
h(r)=\int_{-\infty}^\infty e^{iry}g(y)\,dy.
\end{equation}
The numbers~$r_n$, $s_n$, and~$\overline{s}_n$ with a Latin
subscript are defined by the formulas
\begin{equation}
\label{eq2.2}
\lambda_n=s_n\overline{s}_n,\quad
\overline{s}_n=1-s_n,\quad
s_0=1,\quad
s_n=\frac12+ir_n,\quad
\lambda_n=\frac14+r_n^2.
\end{equation}
The eigenvalues $\lambda_n$ in the interval $0\leqslant
\lambda_n<1/4$ are said to be exceptional; the number~$M$ of such
eigenvalues is finite. For the exceptional eigenvalues, one has
$$
r_n=-i\biggl(\frac14-\lambda_n\biggr)^{1/2},\qquad
r_0=-\frac{i}2.
$$
The eigenvalues $\lambda_n\geqslant 1/4$ will be numbered by a
subscript~$j$, so that $\lambda_j=1/4+r_j^2$, $r_j\geqslant0$.

When considering the Selberg formula, we restrict ourselves to the
case of the trivial one-dimensional representation
$\chi\colon\Gamma\to\mathbb{C}$, $\chi(\gamma)=1$,
$\forall\,\gamma\in\Gamma$. Then the Selberg formula reads
\begin{equation}
\label{eq2.3}
\sum_{n\geqslant0}h(r_n)=H[h]+S_R[h]+S_P[g]+\mathscr{P}[h|\varphi]
\end{equation}
for every $h\in\{h\}_S$ and determines the form of the
functional~$\Phi_\Gamma$~\eqref{eq1.2}.

Let us give the definitions of the objects occurring on the
right-hand side in~\eqref{eq2.3}. First,
\begin{equation}
\label{eq2.4}
H[h]=\frac{|F|}{4\pi}\int_{-\infty}^\infty r\tanh\pi rh(r)\,dr.
\end{equation}
Second, $S_R[h]$ is the contribution of the conjugacy classes
(in~$\Gamma$) of elliptic elements (the contribution of elliptic
conjugacy classes), and
\begin{equation}
\label{eq2.5}
S_R[h]=\sum_{\{R\}}\,\sum_{k=1}^{p-1}\frac{1}{p\sin\pi k/p}
\int_{-\infty}^\infty h(r)\frac{e^{-2\pi kr/p}}{1+e^{-2\pi r}}\,dr,
\end{equation}
where the sum is over the set~$\{R\}$ of primitive elliptic
conjugacy classes and $p=p(R)$ is the order of a class~$R$. The
number~$|\{R\}|$ of elliptic conjugacy classes and their maximum
order are finite.

The third term on the right-hand side in~\eqref{eq2.3} is the
contribution of hyperbolic conjugacy classes; it is given by the
formula
\begin{equation}
\label{eq2.6}
S_P[g]=\sum_{\{P_0\}}\,\sum_{k=1}^\infty
\frac{\ln N(P_0)}{N(P_0)^{k/2}-N(P_0)^{-k/2}}\,g(k\ln N(P_0)),
\end{equation}
where the sum is over the set of primitive classes~$P_0$ and
$N(P_0)$ is the norm of a class~$P_0$. Recall that every hyperbolic
element $\gamma\in\Gamma$ is conjugate
in~$\operatorname{SL}(2,\mathbb{R})$ to the transformation $z\to
N(P)z$, $N(P)>1$, $P=P_0^k$ ($k\geqslant1$), where $P$~is an
arbitrary hyperbolic class and $N(P)=N(P_0)^k$. In what follows, we
write
\begin{equation}
\label{eq2.7}
B_0=\min_{\{P\}}N(P),\quad
B_0>1,\qquad
b_0=\ln B_0.
\end{equation}
The last term~$\mathscr{P}[h|\varphi]$ on the right-hand side
in~\eqref{eq2.3} is the contribution of parabolic conjugacy classes
(the contribution of the continuous spectrum), and
\begin{equation}\label{eq2.8}
\begin{split}
\mathscr{P}[h|\varphi]&=\frac{1}{4\pi}\int_{-\infty}^\infty h(r)\,
\frac{\varphi'}{\varphi}\biggl(\frac{1}{2}+ir\biggr)\,dr-
\frac{n}{2\pi}\int_{-\infty}^\infty
h(r)\,\frac{\Gamma'}{\Gamma}(1+ir)\,dr
\\
&\qquad-\frac{h(0)}{4}\biggl(n-\operatorname{tr}\Phi\biggl(\frac12\biggr)\biggr)-ng(0)\ln2.
\end{split}
\end{equation}
Here $\Phi(s)$~is the $n\times n$ matrix of free terms in the
Eisenstein series (the scattering matrix),
\begin{equation}
\label{eq2.9}
\varphi(s)=\det\Phi(s),
\end{equation}
$n$~is the number of primitive parabolic conjugacy classes (the
number of pairwise nonequivalent parabolic points
in~$\overline{F}$), and $\Gamma(\,\cdot\,)$ is the gamma function.

The properties of~$\varphi$ are described in~\cite{1,3,12,17}. This
is a meromorphic function satisfying the functional equations
\begin{equation}
\label{eq2.10}
\varphi(s)\varphi(1-s)=1,\qquad
\varphi(s)=\widetilde{\varphi(\tilde s)},
\end{equation}
where the tilde stands for complex conjugation. The
function~$\varphi$ is holomorphic in the half-plane
$\operatorname{Re}s>1/2$ except for finitely many poles on the
interval $(1/2,1]$. The poles~$s_\alpha$ of~$\varphi$ with
$\operatorname{Re}s_\alpha<1/2$ lie in the strip
$1-\mu_0<\operatorname{Re}s<1/2$ symmetrically with respect to the
real axis, and
\begin{equation}
\label{eq2.11}
\sum_\alpha\biggl(\frac{1}{2}-\beta_\alpha\biggr)|s_\alpha|^{-2}=
C_\Gamma<\infty,\qquad
\sum_{0<\gamma_\alpha\leqslant x}1\leqslant A_\Gamma x^2.
\end{equation}
Throughout the following, Greek subscripts are used to number the
poles of~$\varphi$.

The Selberg zeta function~$Z(\,\cdot\,)$ for cofinite groups is
defined in the same way as for cocompact groups; for
$\operatorname{Re}s>1$, one has
$$
Z(s)=\prod_{\{P_0\}}\,\prod_{k=1}^\infty[1-N(P_0)^{-k-s}].
$$
This function has an analytic continuation into the entire plane of
the variable $s=\sigma+it$ as a meromorphic function and satisfies
a functional equation of the form $Z(1-s)=A(s)Z(s)$. An explicit
expression for the factor~$A(s)$ can be found in the
papers~\cite{1,12}, which give a complete description of all the
zeros and poles of~$Z$, their multiplicities being indicated. Let
us present this description and simultaneously introduce a
numbering to be used for the nontrivial zeros of the Selberg zeta
function.

\textit{The nontrivial zeros of~$Z(\,\cdot\,)$} are

1.~The zeros~$s_j$ on the critical line $\operatorname{Re}s=1/2$.
They are arranged symmetrically with respect to the real axis, and
one has the corresponding eigenvalues
$$
\lambda_j=s_j(1-s_j),\qquad s_j=1/2+r_j\qquad (j\geqslant0).
$$

2.~The zeros~$s_m\in(0,1)$, $m=1,\dots,M_1$. They are arranged
symmetrically with respect to the point $s=1/2$, and one has the
corresponding eigenvalues
$$
\lambda_m=s_m(1-s_m),\qquad s_m=\sigma_m.
$$

3.~The zeros
$$
s_\alpha=\beta_\alpha+i\gamma_\alpha,\qquad
1-\mu_0<\beta_\alpha<1/2,
$$
at the poles of the function~$\varphi$~\eqref{eq2.9}. These zeros
are arranged symmetrically with respect to the real axis.

4.~The zeros~$s_\nu=\sigma_\nu$, $1/2<\sigma_\nu\leqslant1$,
$\nu=0,1,\dots,M_2-1$, at the poles of~$\varphi$. One has the
corresponding eigenvalues $\lambda_\nu=\sigma_\nu(1-\sigma_\nu)$
($\nu\ne0$) and $\lambda_0=0$ ($\sigma_0=1$). The poles
of~$Z(\,\cdot\,)$ lie at the points $s=-l+1/2$, $l=0,1,\dots$\,,
and the \textit{trivial zeros of~$Z(\,\cdot\,)$} lie at the points
$s=-l$, $l=0,1,\dots$\,.

The numbers~$\lambda_m$,~$\lambda_\nu$, and~$\lambda_j$ exhaust the
whole discrete spectrum, and so
$\{\lambda_n\}=\{\lambda_m\}\cup\{\lambda_\nu\}\cup\{\lambda_j\}$.





\section{Explicit formula for~$\boldsymbol{S_P[g]}$}
\label{sec3}

In analytic number theory, the term \textit{explicit formulas}
refers to formulas representing the object of study by a series
over zeros and poles of the corresponding analytic function. The
main example is given by the explicit formula representing the
Chebyshev function~$\Psi(\,\cdot\,)$ by a series over the
nontrivial zeros of the Riemann zeta function. In our setting, the
function (see~\cite{14,15})
\begin{equation}
 \label{eq3.1}
\Lambda^\Gamma(P)=\frac{\ln N(P_0)}{1-N(P)^{-1}}
\end{equation}
is an analog of the Mangoldt function~$\Lambda(\,\cdot\,)$, and
\begin{equation}
 \label{eq3.2}
\Psi^\Gamma(x)=\sum_{B_0\leqslant N(P)\leqslant x}\Lambda^\Gamma(P)
\end{equation}
is the corresponding analog of the Chebyshev function. Based on the
results in~\cite{15,16}, let us present a definitive version of an
explicit formula for the function
\begin{equation}
\label{eq3.3}
\Psi_1^\Gamma(x)=\int_{B_0}^x\Psi^\Gamma(\xi)\,d\xi.
\end{equation}
This formula reads
\begin{equation}
\label{eq3.4}
\begin{gathered}
\Psi_1^\Gamma(x)=\Sigma_{R,\Delta}(x)+\Sigma_{R,\varphi}(x)+
\Psi^\Gamma_{1,0}+\Delta_R(x),
\\
\Delta_R(x)=O\biggl(\frac{x^2\ln x}R\biggr),\qquad
x\geqslant x_0,\quad
R\to\infty,
\end{gathered}
\end{equation}
The functions~$\Sigma_{R,\Delta}(x)$ and~$\Sigma_{R,\varphi}(x)$ on
the right-hand side in~\eqref{eq3.4} are given by the formulas
\begin{align}
\Sigma_{R,\Delta}(x)&=\sum_{0\leqslant r_j\leqslant R}
\frac{x^{1+s_j}}{s_j(1+s_j)}+
\frac{x^{1+\tilde s_j}}{\tilde s_j(1+\tilde s_j)}
\nonumber
\\
&\qquad+\sum_{1/2<s_m<1}\biggl(\frac{x^{1+s_m}}{s_m(1+s_m)}+
\frac{x^{1+\overline s_m}}{\overline s_m(1+\overline s_m)}\biggr)
+\sum_{\nu=0}^{M_2-1}\frac{x^{1+s_\nu}}{s_\nu(1+s_\nu)}\,,
\label{eq3.5}
\\
%\end{align}
%\begin{equation}
\Sigma_{R,\varphi}(x)&=\sum_{(\alpha,R)}
\frac{x^{1+s_\alpha}}{s_\alpha(1+s_\alpha)}+
\frac{x^{1+\tilde s_\alpha}}{\tilde s_\alpha(1+\tilde s_\alpha)}\qquad
(s_\alpha=\beta_\alpha+i\gamma_\alpha).
\label{eq3.6}
\end{align}
%\end{equation}
Just as above, the tilde stands for complex conjugation,
$\overline{s}=1-s$, and the summation in~$\sum_{(\alpha,R)}$ is
over all the poles~$s_\alpha$ of~$\varphi$ such that
\begin{equation}
\label{eq3.7}
\beta_\alpha<1/2,\quad
\gamma_\alpha>0,\quad
0<\gamma_\alpha<R.
\end{equation}
Throughout the following, we assume that
\begin{equation}
\label{eq3.8}
B_0\geqslant x_0.
\end{equation}
Note that $B_0=(1/4)(3+\sqrt5\,)^2\simeq6,8541$ if
$\Gamma=\operatorname{SL}(2,\mathbb{Z})$.

Formula~\eqref{eq3.4} can be proved by standard methods of analytic
number theory based on the integral representation~\cite{15}
$$
\Psi_1^\Gamma(x)=\frac{1}{2\pi i}\int_{\sigma_1-i\infty}^{\sigma_1+i\infty}
\frac{x^{s+1}}{s(s+1)}\,\frac{Z'}{Z}(s)\,ds,\qquad
\sigma_1>1.
$$
An analog of formula~\eqref{eq3.4} for cocompact groups was proved
in~\cite{15}; in this case, one can take $x_0=2$. Note that it is
sufficient for us to have the coarse estimate
\begin{equation}
 \label{eq3.9}
|\Delta_R(x)|=O\biggl(\frac{x^k\ln x}{v(R)}\biggr),\qquad
v(R)\to\infty\quad (R\to\infty).
\end{equation}
The first two terms on the right-hand side in~\eqref{eq3.4} are the
sum of residues of the integrand in the rectangular domain with
vertices $\sigma_1\pm iR$ and $-A\pm iR$ ($A\to\infty$). The
residues at the points $s=0,-1$ must be considered separately
(see~\cite{15}), and their contribution is included
in~$\Psi_{1,0}^\Gamma(x)$. This function includes the contributions
of the poles and trivial zeros of~$Z(\,\cdot\,)$, and
\begin{equation}
\label{eq3.10}
\Psi^\Gamma_{1,0}(x)=-\frac{2}{3}x^{3/2}(n-\operatorname{tr}\Phi(1/2))+
\widetilde\Psi^\Gamma_{1,0}(x).
\end{equation}
The first term on the right-hand side in this formula is the
contribution of the pole at $s=1/2$. The
function~$\Psi^\Gamma_{1,0}(x)$ can be explicitly evaluated based
on the results in~\cite{1,12}, and
$\widetilde\Psi^\Gamma_{1,0}(\,\cdot\,)$ is a function
differentiable for $x\geqslant2$ and such that
\begin{equation}
\widetilde\Psi^\Gamma_{1,0}(x)=C_1^\Gamma x\ln x+O(x),\qquad
\frac{d\widetilde\Psi_{1,0}^\Gamma}{dx}(x)=C_1^\Gamma\ln x+
C_2^\Gamma+O(x^{-1}).
\label{eq3.11}
\end{equation}
For example, if $\Gamma=\operatorname{SL}(2,\mathbb{Z})$, then
\begin{align*}
\widetilde\Psi^\Gamma_{1,0}(x)&=C_1x\ln x+C_2x+C_3+C_4x\ln(1-x^{-1})+
C_5\ln(1-x^{-1})
\\
&\qquad+C_6x^{-1/2}+C_7(1+x^{-1})\ln\frac{1+x^{-1/2}}{1-x^{-1/2}}\,,
\end{align*}
where the~$C_i$ are some constants. Let us introduce the function
\begin{equation}
\label{eq3.12}
F(x)=\frac{d}{dx}\,\frac{g(\ln x)}{\sqrt{x}}\,.
\end{equation}

\begin{lemma}[{\rm(en explicit formula for~$S_P[g]$)}]
\label{lem1} Let $B_0\geqslant x_0$~\eqref{eq3.8}, and let
$h\in\{h\}_S$ be a function such that
\begin{equation}
\label{eq3.13}
\int_{b_0}^\infty e^{y/2}y[|g(y)|+|g^{(1)}(y)|+|g^{(2)}(y)|]\,dy=
C_\Gamma[g]<\infty,\qquad
g^{(3)}\in L^1(b_0,\infty).
\end{equation}
Then
\begin{equation}
\label{eq3.14}
S_P[g]=S_P^\infty[g]+S_{\rm ex}[g]+S_0[g]
\end{equation}
for every $B\geqslant B_0$, where
\begin{align}
S_0[g]&=-\int_B^\infty F(x)\,\frac{d}{dx}\,\Psi^\Gamma_{1,0}(x)\,dx-
\int_{B_0}^B\Psi^\Gamma(x)F(x)\,dx,
\label{eq3.15}
\\
S_{\rm ex}[g]&=-\sum_{1/2<s_m<1}\int_B^\infty\biggl(\frac{x^{s_m}}{s_m}+
\frac{x^{\overline s_m}}{\overline s_m}\biggr)F(x)\,dx-
\sum_{\nu=0}^{M_2-1}\int_B^\infty\frac{x^{s_\nu}}{s_\nu}\,F(x)\,dx,
\label{eq3.16}
\\
\label{eq3.17}
S_P^{\infty}[g]&=-\sum_{(\alpha)}\int_B^\infty
\biggl(\frac{x^{s_\alpha}}{s_\alpha}+
\frac{x^{\tilde s_\alpha}}{\tilde s_\alpha}\biggr)F(x)\,dx
-\sum_{r_j\geqslant0}\int_B^\infty\biggl(\frac{x^{s_j}}{s_j}+
\frac{x^{\overline s_j}}{\overline s_j}\biggr)F(x)\,dx,
\end{align}
and the symbol~$\sum_{(\alpha)}$ stands for a sum over the poles
$s_\alpha=\beta_\alpha+i\gamma_\alpha$ of~$\varphi$ such that
$\beta_\alpha<1/2$ and $\gamma_\alpha>0$ \textup(see~\eqref{eq3.7}\textup). Here
and in what follows, $\sum_{r_j\geqslant0}+\sum_{(\alpha)}=
\lim_{R\to\infty}\bigl(\,\sum_{r_j\leqslant
R}+\sum_{|\gamma_\alpha|\leqslant R}\bigr)$.
\end{lemma}

\begin{proof}
An analog of Lemma~\ref{lem1} for strictly hyperbolic groups was
proved in~\cite{10}. Let us rewrite~\eqref{eq2.6} in the form
$$
S_P[g]=\sum_{\{P\}}\Lambda^\Gamma(P)f(N(P)),\qquad
f(x)=\frac{g(\ln x)}{\sqrt x}\,.
$$
By Abel's partial summation formula,
$$
\sum_{B_0\leqslant N(P)\leqslant x}\Lambda^\Gamma(P)f(N(P))=
\Psi^\Gamma(x)f(x)-\int_{B_0}^x\Psi^\Gamma(\xi)F(\xi)\,d\xi.
$$
In view of the estimates (see~\cite{15,16})
$$
\Psi^\Gamma(x)=O(x),\quad
\Psi^\Gamma_1(x)=O(x^2),\qquad
x\to\infty,
$$
and condition~\eqref{eq3.13}, we obtain
$$
S_P[g]=-\int_{B_0}^\infty\Psi^\Gamma(x)F(x)\,dx=
-\int_B^\infty\frac{d\Psi^\Gamma_1}{dx}(x)F(x)\,dx-
\int_{B_0}^B\Psi^\Gamma(x)F(x)\,dx.
$$
Further, we integrate by parts and find that
\begin{equation}
\label{eq3.18}
S_P[g]=\Psi^\Gamma_1(B)F(B)-\int_{B_0}^B\Psi^\Gamma(x)F(x)\,dx+
\int_B^\infty\Psi_1^\Gamma(x)\,\frac{dF}{dx}(x)\,dx.
\end{equation}
Let us use formula~\eqref{eq3.4}. We obtain
\begin{align*}
S_{P}[g]&=\int_B^\infty[\Sigma_{R,\Delta}(x)+\Sigma_{R,\varphi}(x)+
\Psi^\Gamma_{1,0}(x)]\,\frac{dF}{dx}(x)\,dx
\\
&\qquad+\Psi_1^\Gamma(B)F(B)-\int_{B_0}^B\Psi^\Gamma(x)F(x)\,dx+
\int_B^\infty\frac{dF}{dx}(x)\Delta_R(x)\,dx,
\end{align*}
whence it follows that
\begin{align*}
S_{P}[g]&=-\Psi_1^\Gamma(B)F(B)+\Delta_R(B)F(B)
\\
&\qquad-\int_B^\infty\frac{d}{dx}[\Sigma_{R,\Delta}(x)+\Sigma_{R,\varphi}(x)+
\Psi^\Gamma_{1,0}(x)]F(x)\,dx
\\
&\qquad+\Psi_1^\Gamma(B)F(B)+\int_{B_0}^B\Psi^\Gamma(x)F(x)\,dx+
\int_B^\infty\frac{dF}{dx}(x)\Delta_R(x)\,dx.
\end{align*}
We differentiate, change the order of summation and integration,
and then pass to the limit as $R\to\infty$, thus obtaining the
desired result. The proof of Lemma~\ref{lem1} is complete.
\end{proof}

Formally, one can obtain Eq.~\eqref{eq3.14} by substituting an
explicit function for~$\Psi_1^\Gamma(x)$ (see~\cite{16})
into~\eqref{eq3.18} with subsequent integration by parts and
changing the order of integration and summation.

Note that if we use the estimate~\eqref{eq3.9}, then the entire
derivation remains valid except that~$e^{y/2}$ must be replaced
with~$e^{(k-3/2)y}$ in condition~\eqref{eq3.13}.

\setcounter{equation}{19}
\begin{lemma}
 \label{lem2}
Let the assumptions of Lemma~\ref{lem1} be satisfied. Then
\begin{equation}
\sum_{n\geqslant0}h(r_n)=H[h]+S_R[h]+S_P^\infty[g]+S_{\rm ex}[g]+
S_0[g]+\mathscr{P}[h|\varphi]. \label{eq3.20}
\end{equation}
Here
\begin{equation}
\label{eq3.21}
\begin{aligned}
S_P^\infty[g]&=-\sum_{r_j\geqslant0}\frac{1}{r_j^2+1/4}
\int_b^\infty(\cos r_jy+2r_j\sin r_jy)f(y)\,dy
\\
&\qquad-\sum_{(\alpha)}\int_B^\infty\biggl(\frac{x^{s_\alpha}}{s_\alpha}+
\frac{x^{\tilde s_\alpha}}{\tilde s_\alpha}\biggr)F(x)\,dx,
\\
f(y)&=-\frac{1}{2}g(y)+g^{(1)}(y),\qquad
b=\ln B,
\end{aligned}
\end{equation}
and the remaining terms on the right-hand side in~\eqref{eq3.20}
have been defined above.
\end{lemma}

\begin{proof}
The proof amounts to the substitution of the
expression~\eqref{eq3.14} into the Selberg formula~\eqref{eq2.3}
with regard to the remark that
$$
\biggl(\frac{x^{s_j}}{s_j}+\frac{x^{\tilde s_j}}{\tilde
s_j}\biggr)F(x)\,dx= \frac{1}{1/4+r^2_j}(\cos r_jy+2r_j\sin
r_jy)f(y)\,dy\qquad (\tilde s_j=\overline s_j)
$$
for $y=\ln x$. The proof of Lemma~\ref{lem2} is complete.
\end{proof}




\section{Explicit formula for~$\boldsymbol{\mathscr{P}[h|\varphi]}$}
\label{sec4}

We introduce the notation
\begin{align}
\label{eq4.1}
J[h|\varphi]&=\frac{1}{4\pi}\int_{-\infty}^\infty h(r)\,
\frac{\varphi'}\varphi\biggl(\frac{1}{2}+ir\biggr)\,dr,
\\
\label{eq4.2} \Delta
\mathscr{P}[h|\varphi]&=-\frac{n}{2\pi}\int_{-\infty}^\infty h(r)\,
\frac{\Gamma'}{\Gamma}(1+ir)\,dr+
\frac{h(0)}{4}(n-\operatorname{tr}\Phi(1/2))-ng(0)\ln 2
\end{align}
and represent~$\mathscr{P}[h|\varphi]$~\eqref{eq2.8} in the form
\begin{equation}
\label{eq4.3} \mathscr{P}[h|\varphi]=J[h|\varphi]+\Delta
\mathscr{P}[h|\varphi].
\end{equation}
As was already noted, the main properties of the
function~$\varphi$~\eqref{eq2.9} are indicated in~\cite{1,3,12}.
According to these papers,
\begin{equation}
\label{eq4.4}
\varphi(s)=\sqrt\pi\biggl(\frac{\Gamma(s-1/2)}{\Gamma(s)}\biggr)^n
\sum_{n=1}^\infty\frac{a_n}{b_n^{2s}}\,,\qquad
a_1\ne0,\quad
0<b_1<b_2<\cdots\,,
\end{equation}
for $\sigma=\operatorname{Re}s>1$. By the functional
equation~\eqref{eq2.10}, if $s_\alpha$~is a pole of~$\varphi$, then
so is~$\tilde s_\alpha$, while $1-s_\alpha$~and~$1-\tilde s_\alpha$
are zeros of~$\varphi$. The logarithmic derivative of~$\varphi$
occurring in~\eqref{eq4.1} has the simple fraction
decomposition~\cite{17}
\begin{equation}
\label{eq4.5}
\frac{\varphi'}\varphi(s)=\sum_{\mu}\biggl(\frac{1}{s-1+s_\mu}-
\frac{1}{s-s_\mu}\biggr)+\sum_{\beta_\alpha<1/2}
\biggl(\frac{1}{s-1+\tilde s_\alpha}-\frac{1}{s-s_\alpha}\biggr)-2\ln b_1.
\end{equation}
The summation in~$\sum_\mu$ is over all poles of~$\varphi$ such
that $1/2<s_\mu\leqslant1$ ($s_\mu=\sigma_{\mu})$). If $s=1/2+ir$,
then ($r\in\mathbb{R}$)
\begin{equation}
\label{eq4.6}
\begin{aligned}
\frac{1}{s-1+s_\mu}-\frac{1}{s-s_\mu}&=
-(1-2s_\mu)\,\frac{1}{r^2+(s_\mu-1/2)^2}>0\qquad
(1/2<s_\mu\leqslant1),
\\
\frac{1}{s-1+\tilde s_\alpha}-\frac{1}{s-s_\alpha}&=
\frac{1}{ir-a_\alpha^{(1)}}-\frac{1}{ir-a_\alpha^{(2)}}=
-\frac{1-2\beta_\alpha}{(r-\gamma_\alpha)^2+(\beta_\alpha-1/2)^2}<0.
\end{aligned}
\end{equation}
In the last formula,
\begin{equation}
\label{eq4.7}
a_\alpha^{(1)}=s/2-\tilde s_\alpha,\qquad
a_\alpha^{(2)}=s_\alpha-1/2.
\end{equation}
It was proved in~\cite{1,12} that the series
$$
\sum_{\beta_\alpha<1/2}
\frac{1-2\beta_\alpha}{(s-\gamma_\alpha)^2+(\beta_\alpha-1/2)^2}
$$
converges uniformly on compact sets. We substitute the
expressions~\eqref{eq4.5} and~\eqref{eq4.6} into~\eqref{eq4.1},
change the order of integration and summation, and obtain
\begin{equation}
\label{eq4.8}
J[h|\varphi]=J_0[h|\varphi]+J_1[h|\varphi],
\end{equation}
where
\begin{align}
\label{eq4.9}
J_0[h|\varphi]&=-\sum_{\mu}(1-2s_\mu)\frac{1}{4\pi}\int_{-\infty}^\infty
\frac{h(r)\,dr}{r^2+(s_\mu-1/2)^2}-2g(0)\ln b_1,
\\
\label{eq4.10}
J_1[h|\varphi]&=\sum_{\beta_\alpha<1/2}\bigl(I(a_\alpha^{(1)})-
I(a_\alpha^{(2)})\bigr),
\end{align}
and $I(a)$~is the integral
$$
I(a)=\frac{1}{4\pi}\int_{-\infty}^\infty\frac{h(r)}{ir-a}\,dr.
$$
By Parseval's identity,
$$
I(a)=\frac{1}{2}\int_{-\infty}^\infty g(y)\hat{f}(-y)\,dy,\qquad
\hat{f}(-y)=\frac{1}{2\pi i}\int_{-\infty}^\infty\frac{e^{iry}}{r+ia}\,dr.
$$
The last integral can be evaluated using residue theory.

As a result, we obtain
\begin{equation}
\label{eq4.11}
I(a_\alpha^{(1)})-I(a_\alpha^{(2)})=-\frac{1}{2}\int_0^\infty g(y)
[e^{-y/2+\tilde s_\alpha y}+e^{-y/2+s_\alpha y}]\,dy.
\end{equation}
We substitute this expression into~\eqref{eq4.10}, take into
account our definition of~$\sum_{(\alpha)}$
($\sum_{(\alpha)}=\sum_{\beta_\alpha<1/2,\ |\gamma_\alpha|>0}$),
and arrive at the relation
\begin{equation}
\label{eq4.12}
J_1[h|\varphi]=-\sum_{(\alpha)}\int_0^\infty g(y)[e^{-y/2+\tilde s_\alpha y}+
e^{-y/2+s_\alpha y}]\,dy.
\end{equation}

\begin{lemma}[{\rm(an explicit formula for~$\mathscr{P}[h|\varphi]$)}]
 \label{lem3}
For every cofinite group~$\Gamma$, one has
\begin{equation}
\label{eq4.13}
\mathscr{P}[h|\varphi]=J_1[h|\varphi]+\Delta\mathscr{P}[h|\varphi]+
J_0[h|\varphi].
\end{equation}
Here~$J_1[h|\varphi]$ is defined in~\eqref{eq4.12}, $\Delta
\mathscr{P}[h|\varphi]$~is defined in~\eqref{eq4.2},
and~$J_0[h|\varphi]$ is defined in~\eqref{eq4.9}.
\end{lemma}

\begin{proof}
The proof amounts to the substitution of the right-hand sides
of~\eqref{eq4.8} and~\eqref{eq4.12} into~\eqref{eq4.3}. The
series~$J_1[h|\varphi]$ converges in view of~\eqref{eq4.13}.
\end{proof}




\section{Main theorem}
\label{sec5}

Theorem~\ref{th1} proved in the section specifies the explicit form
of the functional~$\widetilde\Phi_\Gamma$ in formula~\eqref{eq1.5}.
Let us preliminarily transform the expression~\eqref{eq3.21}
for~$S_P^\infty[g]$.

\begin{lemma}
 \label{lem4}
Let the assumptions of Lemma~\ref{lem1} be satisfied, and let
$\lambda_n \ne 1/4$. Then
\begin{equation}
\label{eq5.1}
S_P^\infty[g]=W[g]+S_P^1[g|\Delta]+S_P^2[g|\varphi]+S_P^3[g|\varphi]-
J_1[h|\varphi].
\end{equation}

Here~$J_1[h|\varphi]$ is defined in~\eqref{eq4.12}, and
\begin{align}
\label{eq5.2}
W[g]&=-2f(b)\biggl[\,\sum_{r_j \geqslant 0}\frac{\cos r_jb}{r_j^2+1/4}+
\sum_{(\alpha)}\frac{\cos b\gamma_\alpha}{\gamma_\alpha^2}
e^{(\beta_\alpha-1/2)b}\biggr],
\\
\nonumber
S_P^1[g|\Delta]&=\sum_{r_j > 0}\frac{1}{(r_j^2+1/4)r_j}\biggl\{\sin r_jb
\biggl[-\,\frac{1}{2}g(b)+2g^{(2)}(b)\biggr]+
\\
\label{eq5.3}
&\qquad+\int_b^\infty \sin r_jy\biggl[-\,\frac{1}{2}g^{(1)}(y)+
2g^{(3)}(y)\biggr]\,\biggr\}dy,
\\
\nonumber
S_P^2[g|\varphi]&=2\sum_{(\alpha)}\biggl[g(b)e^{(\beta_\alpha-1/2)b}
\sin\gamma_\alpha b\biggl(\frac{\gamma_\alpha}{\beta_\alpha^2+\gamma_\alpha^2}
-\frac{1}{\gamma_\alpha}\biggr)
\\
&\qquad+\frac{(\beta_\alpha-1/2)}{\gamma_\alpha^2}g(0)-g(b)
\frac{\beta_\alpha^3\cos b\gamma_\alpha}
{\gamma_\alpha^2(\beta_\alpha^2+\gamma_\alpha^2)}\,
e^{(\beta_\alpha-1/2)b}\biggr],
\label{eq5.4}
\\
\label{eq5.5}
S_P^3[g|\varphi]&=2\sum_{(\alpha)}\gamma_\alpha^{-2}\int_0^b
\frac{d^2}{dy^2}\bigl(e^{(\beta_\alpha-1/2)y}g(y)\bigr)
\cos\gamma_\alpha y\,dy.
\end{align}
If $\lambda_n=1/4$, then one must add $-4k\displaystyle\int_0^b
f(y)\,dy$ to the right-hand side of~\eqref{eq5.1}, where $k$~is the
multiplicity of the eigenvalue $\lambda_n=1/4$.
\end{lemma}

\begin{proof}
Consider the integral
$$
A_\alpha=\int_B^\infty\biggl(\frac{x^{s_\alpha}}{s_\alpha}+
\frac{x^{\tilde s_\alpha}}{\tilde s_\alpha}\biggr)F(x)\,dx
$$
occurring on the right-hand side in~\eqref{eq3.21}. By
definition~\eqref{eq3.12} of the function~$F$,
$$
A_\alpha=-\frac{2g(b)}{s_\alpha\tilde s_\alpha}e^{(\beta_\alpha-1/2)b}
(\beta_\alpha \cos b\gamma_\alpha+\gamma_\alpha\sin b\gamma_\alpha)-
2\int_b^\infty e^{(\beta_\alpha-1/2)y}\cos(\gamma_\alpha y)g(y)\,dy.
$$
One derives Eq.~\eqref{eq5.1} from~\eqref{eq3.20} by twice
integrating by parts. The absolute convergence of the
series~$S_P^1[g|\Delta]$ and~$S_P^2[g|\varphi]$ follows
from~\eqref{eq2.11} and the convergence of the series
$\sum_{(\alpha)}\gamma_\alpha^{-3}$ and $\sum_{r_j>0}r_j^{-3}$. To
prove the convergence of the series~$S_P^3[g|\varphi]$, it suffices
to integrate by parts. Now the convergence of the series~$W[g]$
follows from formula~\eqref{eq5.1}.

The proof of Lemma~\ref{lem4} is complete.
\end{proof}

\begin{corollary}
For any $b>b_0$ and any cofinite group~$\Gamma$ such that
$B_0\geqslant x_0$, the series in the definition of~$W[g]$
converges; i.e.,
$$
\sum_{r_j \geqslant 0}\frac{\cos r_j b}{r_j^2+1/4}+
\sum_{(\alpha)}\frac{\cos b\gamma_\alpha}{\gamma_\alpha^2}\,
e^{(\beta_\alpha-1/2)b}=C_\Gamma(b)<\infty.
$$
\end{corollary}
Now we are in a position to prove Theorem~\ref{th1}.


\begin{theorem}
 \label{th1}
Let a function $h\in\{h\}_S$ satisfy condition~\eqref{eq3.13}. Then
\begin{equation}
\label{eq5.6}
\sum_{n\geqslant0}h(r_n)=H[h]+G[h]+S_P^1[g|\Delta]+S_P^2[g|\varphi]
+S_P^3[g|\varphi]+M[g]
\end{equation}
for every cofinite group~$\Gamma$ such that $B_0\geqslant x_0$.

Here
\begin{align}
\label{eq5.7}
G[h]&=-\frac{n}{2\pi}\int_{-\infty}^{+\infty}h(r)\psi(1+ir)\,dr\qquad
\biggl(\psi(x)=\frac{\Gamma'}{\Gamma}(x)\biggr),
\\
\nonumber
M[g]&=W[g]+S_{\rm ex}[g]+S_R[h]+S_0[g]-\sum_\mu\frac{1-2s_\mu}{4}
\int_{-\infty}^{+\infty}\frac{h(r)dr}{r^2+(s_\mu-1/2)^2}+
\\
\label{eq5.8}
&\qquad+\biggl(n-\operatorname{tr}\Phi\biggl(\frac12\biggr)\biggr)h(0)-g(0)(n\ln 2+2b_{1}),
\end{align}
where the summation in~$\sum_\mu$ is over the poles of~$\varphi$
such that $1/2 < s_\mu \leqslant 1$ $(s_\mu=\sigma_\mu)$.
\end{theorem}

\begin{proof}
The proof amounts to the substitution of~\eqref{eq5.1}
and~\eqref{eq4.13} into~\eqref{eq3.20}. The proof of
Theorem~\ref{th1} is complete.
\end{proof}

Note that it follows from~\eqref{eq5.3},~\eqref{eq5.4},
and~\eqref{eq5.5} that the contribution of the discrete spectrum
$\{\lambda_n\}$ to $\sum_{n \geqslant 0}h(r_n)$ is determined by
the behavior of~$g(y)$ for $y>b$, while the contribution of the
resonance spectrum~$\{s_\alpha\}$ is determined by the behavior
of~$g(y)$ for $y \leqslant b$.




\section{Proof of the Roelke conjecture}
\label{sec6}

The proof of the Roelke conjecture is based on an analysis of the
asymptotics as $t\to 0$ of the expressions on the right-hand side
in~\eqref{eq5.6} for the case in which
\begin{equation}
\label{eq6.1} h(r)=h(r,p)=\frac{e^{-tr^2}}{r^2+p^2}\qquad (p>1/2).
\end{equation}
Throughout the following, $h(r)=h(r,p)$ and $g(y)=g(y,p)$
in~\eqref{eq2.1}. The dependence of the objects in question on~$p$
will be indicated explicitly. By~$C_i$ we denote various constants
independent of~$p$. We write $A(t,p)=O_p(t^k)$ if
$A(t,p)=C(p)O(t^k)$ ($t\to 0$) and $A(t,p)=R(t,p)$ if
$A(t,p)=C_0(p)+C_1(p)t+C_2(p)t^2+\cdots$ \,($t\to 0$). By~$F_i(t)$
we denote functions independent of~$p$ such that $F_i(t)\leqslant
C_i$. The constants~$C_i$ and~$C_i(p)$, as well as the
functions~$F_i(t)$, may be different in different formulas.

\begin{lemma}
 \label{lem5}
For every cofinite group~$\Gamma$ such that $B_0\geqslant x_0$, one
has the following asymptotic formula as $t\to 0$\textup:
\begin{multline}\label{eq6.2}
%\nonumber
\sum_{n\geqslant 0}\frac{e^{-tr_n^2}}{r_n^2+p^2}+
\sum_{(\alpha)}\gamma_\alpha^{-2}e^{-t\gamma_\alpha^2}=
\frac{|F|}{4\pi}e^{tp^2}\ln\frac{1}{t}-\frac{n}{\pi}I(t,p)+p^{2}e^{tp^2}S(t)+B(t,p), \\
B(t,p)= e^{tp^2}\bigl[C_{0}(p)+C_{1}(p)\sqrt{t}F_{1}(t)+C_{2}(p)tF_{2}(t)+C_{3}(p)t^{3/2}F_{3}(t)+\\
+C_{4}(p)p^{2}F_{4}(t)+C_{5}t^{5/2}F_{5}(t)+O_{p}(t^{3})\bigr],
\end{multline}
where
\begin{align}
\label{eq6.3}
I(t,p)&=\int_0^\infty\frac{e^{-tr^2}\ln r}{r^2+p^2}\,dr,\quad S(t)=\sum\limits_{(\alpha)}\gamma_{\alpha}^{-4}e^{-t\gamma_{\alpha}^{2}},
\\
\label{eq6.4}
\bigl|F_{i}(t)\bigr|&\leqslant C_{i}.
\end{align}
\end{lemma}

\begin{proof}
We use the standard notation~\cite{20}
$$
\operatorname{erf}(x)=\frac{2}{\sqrt{\pi}}\int_0^x e^{-\xi^2}\,d\xi,\qquad
\operatorname{erfc} x=1-\operatorname{erf}x
$$
and the following properties of the probability
integral~$\operatorname{erf}x$:
\begin{gather*}
\operatorname{erf}x=-\operatorname{erf}(-x)=\frac{2}{\sqrt{\pi}}
\sum_{n=0}^\infty\frac{(-1)^nx^{2n+1}}{n!(2n+1)}\,,
\\
\frac{\sqrt{\pi}}{2}\,\frac{e^{-x^2}}{x+\sqrt{x^2+2}}\leqslant
\operatorname{erfc} x\leqslant\frac{\sqrt{\pi}}{2}\,
\frac{e^{-x^2}}{x+\sqrt{x^2+4/\pi}}\qquad
(x\geqslant 0)
\\
\operatorname{erfc} z=\frac{e^{-z^2}}{\sqrt{\pi}\,z}\biggl(1+\sum_{m=1}^\infty
\frac{(-1)^m1\cdot 3\cdots(2m-1)}{(2z^2)^m}\biggr)\qquad
\biggl(|z|\to\infty,\
|{\arg z}|\leqslant\frac{\pi}{2}\biggr).
\end{gather*}
The function
$$
g(y,p)=\frac{1}{\pi}\int_0^{+\infty}
\frac{e^{-tr^2}\cos ry}{r^2+p^2}\,dr
$$
and its derivatives can be expressed via the probability
integral~\cite{19,20},
\begin{equation}
\label{eq6.5}
g(y)=\frac{1}{4p}e^{tp^2}\biggl[2e^{-yp}-e^{-yp}\operatorname{erfc}
\biggl(\frac{y}{2\sqrt{t}}-p\sqrt{t}\biggr)+
e^{yp}\operatorname{erfc}\biggl(\frac{y}{2\sqrt{t}}+p\sqrt{t}\biggr)\biggr].
\end{equation}
Thus,
\begin{equation}
\label{eq6.6}
g(y)=\frac{1}{2p}e^{tp^2-yp}+\Delta g(y),\qquad
|\Delta g(y)|\leqslant Ce^{-y^2/(4t)}.
\end{equation}
Likewise,
\begin{align}
\nonumber
g^{(1)}(y)&=-\frac{1}\pi\int_0^\infty\frac{re^{-tr^2}}{r^2+p^2}\sin ry\,dr
\\
\label{eq6.7}
&=\frac{1}{4}e^{tp^2}\biggl[-2e^{-yp}+e^{-yp}\operatorname{erfc}
\biggl(\frac{y}{2\sqrt{t}}-p\sqrt{t}\biggr)+
e^{yp}\operatorname{erfc}\biggl(\frac{y}{2\sqrt{t}}+p\sqrt{t}\biggr)\biggr]
\end{align}
and hence
\begin{equation}
\label{eq6.8}
g^{(1)}(y)=-\frac{1}{2}e^{tp^2-yp}+\Delta g^{(1)}(y),\qquad
|\Delta g^{(1)}(y)|\leqslant Ce^{-y^2/(4t)}.
\end{equation}
Finally,
\begin{equation}
\label{eq6.9}
g^{(2)}(y)=-\frac{1}{\pi}\int_0^\infty
\frac{r^2\cos ry\,e^{-tr^2}}{p^2+r^2}\,dr=p^2g(y)-
\frac{1}{\sqrt{4\pi t}}\,e^{-y^2/(4t)}.
\end{equation}
These results mean that the assumptions under which
Theorem~\ref{th1} applies hold for $p>1/2$ and $B_0>x_0$.

According to definition~\eqref{eq2.4},
\begin{align*}
H[h]&=H_0[h]+H_1[h],
\\
H_0[h]&=\frac{|F|}{2\pi}\int_0^\infty\frac{re^{-tr^2}}{p^2+r^2}\,dr,\qquad
H_1[h]=-\frac{|F|}{\pi}\int_0^\infty\frac{re^{-tr^2}}{p^2+r^2}\,
\frac{e^{-2\pi r}}{1+e^{-2\pi r}}\,dr,
\end{align*}
and hence
$$
H_1[h]=R_1(t,p).
$$
On the other hand (see~\cite{19}),
\begin{gather*}
H_0[h]=\frac{|F|}{4\pi}e^{tp^2}(-\operatorname{Ei}(-a)),\qquad
a=p^2t,
\\
-\operatorname{Ei}(-a)=\int_a^\infty e^{-t}t^{-1}\,dt,
\end{gather*}
and we can use the relation
$$
\operatorname{Ei}(-a)=\gamma+\ln a+ \sum_{n=1}^\infty
\frac{(-a)^n}{n\cdot n!}
$$
to obtain
\begin{equation}
\label{eq6.10}
H[h]=\frac{|F|}{4\pi}e^{tp^2}\ln\frac{1}{t}+R_2(t,p).
\end{equation}

The next term on the right-hand side in~\eqref{eq5.6} is
\begin{equation}
\label{eq6.11}
G[h]\equiv G(t,p)=-\frac{n}{2\pi}\int_{-\infty}^{+\infty}
\frac{e^{-tr^2}}{r^2+p^2}\psi(1+ir)\,dr\qquad
\biggl(\psi(z)=\frac{\Gamma'}{\Gamma}(z)\biggr).
\end{equation}
Since the function $\operatorname{Re}\psi(1+ir)$ is even and
$\operatorname{Im}\psi(1+ir)$ is odd~\cite{20}, we have
\begin{align}
\label{eq6.12}
G(t,p)&=G_1(t,p)+G_2(t,p),
\\
\label{eq6.13}
G_1(t,p)&=-\frac{n}{\pi}\int_{0}^{1}\frac{e^{-tr^2}}{r^2+p^2}
\operatorname{Re}\psi(1+ir)\,dr,
\\
\label{eq6.14}
G_2(t,p)&=-\frac{n}{\pi}\int_{1}^{\infty}\frac{e^{-tr^2}}{r^2+p^2}
\operatorname{Re}\psi(1+ir)\,dr.
\end{align}

We use the relation
$$
\operatorname{Re}\psi(1+ir)=-\gamma+r^2\sum_{n=1}^\infty
\frac{1}{n(n^2+r^2)}
$$
to obtain
$$
G_1(t,p)=R_3(t,p).
$$
To analyze~$G_2(t,p)$, we use the asymptotic relation
$$
\operatorname{Re}\psi(1+ir)=\ln r+\frac{1}{12r^2}+\frac{1}{120r^4}+
\frac{1}{252r^6}+\cdots\qquad (r\to \infty).
$$
This relation implies that
$$
G_{2}(t,p)=C_{0}(p)+C_{1}(p)t+C_{2}(p)t^{3/2}+C_{3}(p)t^{2}+O_{p}\bigl(t^{5/2}\bigr)
$$
and hence
\begin{equation}
\label{eq6.15}
G(t,p)=C_0(p)+C_1(p)t-\frac{n}{\pi}I(t,p)+C_{2}(p)t^{3/2}+C_{3}(p)t^{2}+O_{p}\bigl(t^{5/2}\bigr),
\end{equation}
where the integral~$I(t,p)$ is defined~\eqref{eq6.3}. Using the
above-mentioned properties of~$g(y)$ and~$g^{(1)}(y)$
(see~\eqref{eq6.6},~\eqref{eq6.8}
and~\eqref{eq5.3},~\eqref{eq5.4}), we have
\begin{equation}
\label{eq6.16}
\begin{aligned}
S_p^1[g|\Delta]&=R(t,p),
\\
S_P^2[g|\varphi]&=R(t,p) + Cg(0).
\end{aligned}
\end{equation}
Since
\begin{align}
\nonumber
g(0)&=\frac{1}{2p}e^{tp^2}\bigl(1-\operatorname{erf}(p\sqrt{t}\,)\bigr)=
\\
\label{eq6.17}
&=\frac{1}{2p}+C_0\sqrt{t}+C_1p^2t^{3/2}+O_p(t^{5/2}),
\end{align}
(see~\eqref{eq6.5}), it follows that
\begin{equation}
\label{eq6.18}
S_P^2[g|\varphi]=C(p)+C_0(\sqrt{t}+C_1(p)t+C_2(p)t^{3/2}+C_{3}(p)t^{2}+O_p(t^{5/2})\qquad
(t\to 0).
\end{equation}
We represent~$S_P^3[g|\varphi]$~\eqref{eq5.5} in the form
\begin{equation}
\label{eq6.19}
S_P^3[g|\varphi]=2\sum_{(\alpha)}\frac{(\beta_\alpha-1/2)^2}{\gamma_\alpha^2}
I_\alpha^{(0)}(t,p)+4\sum_{(\alpha)}\frac{(\beta_\alpha-1/2)}{\gamma_\alpha^2}
I_\alpha^{(1)}(t,p)+2\sum_{(\alpha)}\frac{1}{\gamma_\alpha^2}
I_\alpha^{(2)}(t,p),
\end{equation}
where
\begin{equation}
\label{eq6.20}
I_\alpha^{(i)}(t,p)=\int_0^b e^{(\beta_\alpha-1/2)y}
\cos(\gamma_\alpha y)g^{(i)}(y,p)\,dy.
\end{equation}

Consider the integrals~$I_\alpha^{(0)}(t,p)$. To this end, we
represent~$g(y,p)$~\eqref{eq6.5} in the form
\begin{align}
\label{eq6.21}
g(y,p)&=\frac{1}{2p}e^{tp^2-yp}+\frac{e^{tp^2}}{4}f_0(y,p),
\\
\label{eq6.22}
f_0(y,p)&=\frac{1}{p}\biggl[e^{yp}\operatorname{erfc}
\biggl(\frac{y}{2\sqrt{t}}+p\sqrt{t}\,\biggr)-
e^{-yp}\operatorname{erfc}\biggl(\frac{y}{2\sqrt{t}}-
p\sqrt{t}\,\biggr)\biggr].
\end{align}

Note that
$$
f_0(y,-p)=f_0(y,p).
$$
We introduce the notation
$$
\xi=\frac{y}{2\sqrt{t}}\,,\quad
(\operatorname{erfc}\xi)^{(n)}=-\Phi^{(n)}(\xi)\qquad
(n \geqslant 1).
$$
Then
$$
|\Phi^{(n)}(\xi)|=\frac{2}{\sqrt{\pi}}\,e^{-\xi^2}|H_{n-1}(\xi)|,
$$
where $H_n(\xi)$ is the Hermite polynomial.

Let us expand~$f_0(y,p)$ in a Taylor series in powers
of~$p\sqrt{t}$\,. This expansion has the form
\begin{align*}
f_0(y,p)&=\operatorname{erfc}\xi\frac{2\sinh yp}{p}-
\frac{\Phi^{(1)}(\xi)}{1!}p\sqrt{t}\,\frac{2\cosh yp}{p}+
\frac{\Phi^{(2)}(\xi)}{2!}p^2t\frac{2\sinh yp}{p}
\\
&\qquad-\frac{\Phi^{(3)}(\xi)}{3!}p^3t^{3/2}\frac{2\cosh yp}{p}+
\frac{\Phi^{(4)}(\xi)}{4!}p^4t^2\frac{2\sinh yp}{p}+\cdots\,.
\end{align*}
Further, we expand~$\sinh yp$ and~$\cosh yp$ in Taylor series and
obtain
\begin{multline}\label{eq6.23}
f_0(y,p)=4\xi\operatorname{erfc}(\xi)\sqrt{t}+\frac{4}{\sqrt{\pi}}e^{-\xi^{2}}\sqrt{t}+\\
+\operatorname{erfc}(\xi)\bigl[c_{1}t^{3/2}\xi^{3/2}p^{2}+c_{2}t^{5/2}\xi^{5/2}p^{4}+\ldots\bigr] \sum_{k=1}^\infty t^{k+1/2}p^{2k}e^{-\xi^2}Q_{2k+1}(\xi)\qquad
(t \to 0),
\end{multline}
where $Q_{2k+1}(\xi)$ is a polynomial of degree~$2k+1$.

We substitute~\eqref{eq6.23} into~\eqref{eq6.21}, take into account
definition~\eqref{eq6.20}, and find the desired expansion
of~$I_\alpha^{(0)}(t,p)$,
\begin{equation}
\label{eq6.24}
I_\alpha^{(0)}(t,p)=e^{tp^2}[C_\alpha^{(0)}(p)+tF_{1,\alpha}^{(0)}(t)+
t^2p^2F_{2,\alpha}^{(0)}(t)+t^3p^3F_{3,\alpha}^{(0)}(t)+\cdots]\qquad
(t \to 0).
\end{equation}
where
\begin{equation}
\label{eq6.25}
|F_{k,\alpha}^{(0)}(t)| \leqslant C_k^0.
\end{equation}
Since
\begin{multline}
\label{eq6.26}
g^{(1)}(y)\,=\,-\frac{1}{2}e^{-tp^{2}-yp}+\frac{1}{4}e^{tp^{2}}f_{1}(y,p),\\
f_{1}(y,p)\,=\,2\operatorname{erfc}\xi+\operatorname{erfc}\xi(C_{1}t\xi^{2}p^{2}+C_{2}t^{2}\xi^{4}p^{4}+\ldots)+\\
+\,e^{-\xi^{2}}(C_{3}t\xi p^{2}+C_{4}t^{2}Q_{5}(\xi)p^{4}+C_{3}t^{3}Q_{7}(\xi)p^{6} +\ldots),
\end{multline}
similarly to (\ref{eq6.24}), we obtain
\begin{equation}
\label{eq6.27} I_\alpha^{(1)}(t,p)=e^{tp^2}[C_\alpha^{(1)}(p)+
\sqrt{t}\,F_{0,\alpha}^{(1)}(t)+t^{3/2}p^2F_{1,\alpha}^{(1)}(t)+
t^{5/2}p^4F_{2,\alpha}^{(1)}(t)+\cdots],
\end{equation}
where
\begin{equation}
\label{eq6.28}
|F_{k,\alpha}^{(1)}(t)| \leqslant C_k^{(1)}.
\end{equation}
Since the series
$$
\sum_{(\alpha)}\frac{(\beta_\alpha-1/2)^2}{\gamma_\alpha^2}\,,\qquad
\sum_{(\alpha)}\frac{(\beta_\alpha-1/2)}{\gamma_\alpha^2}
$$
converge and the constants~$C_k^{(i)}$ are independent of~$\alpha$
and~$p$, we find from~\eqref{eq6.19} that
\begin{align}
\nonumber
S_P^3[g,\varphi]&=e^{tp^2}[C_0(p)+\sqrt{t}\,F_{1}(t)+
tF_{2}(t)+p^2t^{3/2}F_{3}(t)
\\
\label{eq6.29}
&\qquad+p^2t^2F_{4}(t)+p^4t^{5/2}F_{5}(t)+p^4t^3F_{6}(t)+\cdots]+A(t,p).
\end{align}

In this formula, $A(t,p)$~is the last term on the right-hand side
in~\eqref{eq6.19}; i.e.,
$$
A(t,p)=2\sum_{(\alpha)}\frac{1}{\gamma_\alpha^2}I_\alpha^{(2)}(t,p),
$$
where the integral~$I_\alpha^{(2)}(t,p)$ is defined
in~\eqref{eq6.20}. Using~\eqref{eq6.9}, we obtain
\begin{align}
\label{eq6.30}
A(t,p)&=A_1(t)+A_2(t,p),
\\
\label{eq6.31}
A_1(t)&=-\frac{1}{\sqrt{\pi t}}\sum_{(\alpha)}\gamma_\alpha^{-2}
\int_0^b e^{(\beta_\alpha-1/2)y}\cos(\gamma_\alpha y)e^{-y^2/4t}\,dy,
\\
\label{eq6.32}
A_2(t,p)&=2p^2\sum_{(\alpha)}\gamma_\alpha^{-2}\int_0^b
e^{(\beta_\alpha-1/2)y}\cos(\gamma_\alpha y)g(y,p)\,dy.
\end{align}
We point out that~$A_1(t)$ is independent of~$p$.
Consider~$A_2(t,p)$. We expand~$e^{(\beta_\alpha-1/2)y}$ in a
Taylor series and write
\begin{align*}
A_2(t,p)&=2p^2\sum_{(\alpha)}\frac{1}{\gamma_\alpha^{3}}\biggl[g(b)
\sin(\gamma_\alpha b)-\int_0^b g^{(1)}(y,p)\sin(\gamma_\alpha y)\,dy\biggr]
\\
&\qquad+2p^2\sum_{(\alpha)}\biggl[\frac{(\beta_\alpha-1/2)}{\gamma_\alpha^{2}}
\int_0^by\cos(\gamma_\alpha y) g(y)\,dy
\\
&\qquad+2p^2\sum_{(\alpha)}
\frac{(\beta_\alpha-1/2)^2}{\gamma_\alpha^{2}}\int_0^b \frac{y^2}{2!}
\cos(\gamma_\alpha y) g(y)\,dy+\cdots\biggr].
\end{align*}
The series on the right-hand side in this relation converge
absolutely, and the integrals can be estimated by the same scheme
as~$I_\alpha^{(0)}$ and~$I_\alpha^{(1)}$. As a result, we obtain
the expansion
\begin{multline}\label{eq6.33}
A_2(t,p)=e^{tp^2}[C_0(p)+p^{2}S(t) +C_{1}(p)\sqrt{t}F_{1}(t)+C_{2}(p)tF_{2}(t)+C_{3}(p)t^{3/2}F_{3}(t)+\\
+C_{4}(p)t^{2}F_{4}(t)+C_{5}t^{5/2}F_{5}(t)+O_{p}(t^{3})+\cdots]
\end{multline}
Now consider~$A_1(t)$~\eqref{eq6.31}. We have
$$
A_1(t)=-\frac{1}{\sqrt{\pi t}}\sum_{(\alpha)}\int_0^b\cos(\gamma_\alpha y)
e^{-y^2/4t}\biggl[1+\frac{(\beta_\alpha-1/2)}{1!}y+
\frac{(\beta_\alpha-1/2)^2}{2!}y^2+\cdots\biggr].
$$
It follows that
\begin{gather}
\label{eq6.34}
A_1(t)=-\frac{1}{\sqrt{\pi t}}\int_0^\infty\cos(\gamma_\alpha y)
e^{-\frac{y^2}{4t}}\,dy+\Delta A_1(t)
\\
\label{eq6.35}
|\Delta A_1(t)|\leqslant Ct^{1/2}.
\end{gather}

Since
$$
\int_0^\infty e^{-\beta x^2}\cos bx\,dx=\frac{1}{2}
\sqrt{\frac{\pi}{\beta}}\,e^{-\frac{b^2}{4\beta}}\qquad
(\operatorname{Re}\beta>0)
$$
(see~\cite{18}), we obtain
\begin{equation}
\label{eq6.36}
A_1(t)=-\sum_{(\alpha)}\gamma_\alpha^{-2}e^{-t\gamma_\alpha^2}+
\Delta A_1(t).
\end{equation}
Now the desired estimate for~$S_P^3[g|\varphi]$ follows
from~\eqref{eq6.29},~\eqref{eq6.36}, and~\eqref{eq6.33}; it has the
form
\begin{multline}\label{eq6.37}
S_P^3[g|\varphi]=e^{tp^2}[C_0(p)+p^{2}S(t)+C_{1}(p)\sqrt{t}F_{1}(t)+C_{2}(p)tF_{2}(t)+C_{3}(p)t^{3/2}F_{3}(t)+ \\
+\,C_{4}(p)t^{2}F_4(t)+C_{5}(p)t^{5/2}F_5(t)+O_{p}(t^{3})+\cdots]-\\
-\,\sum_{(\alpha)}\gamma_\alpha^{-2}e^{-t\gamma_\alpha^2}+\Delta A_1(t).
\end{multline}
where~$\Delta A_1(t)$ satisfies the estimate~\eqref{eq6.35}. The
sum
$$
B(t)=\sum_{(\alpha)}\gamma_\alpha^{-2}e^{-t\gamma_\alpha^2}
$$
in this expression can be estimated using~\eqref{eq2.11}. The Abel
transform (summation by parts formulas) gives the estimate
$$
|B(t)| \leqslant CA\ln\frac{1}{t}\,.
$$
It remains to consider the expressions on the right-hand side
in~\eqref{eq5.8}. In view of the properties of~$g(\,\cdot\,,p)$, we
have
\begin{equation}
\label{eq6.38}
\begin{gathered}
W[g]=R(t,p),\quad
S_R[h]=R[t,p],\quad
S_0[g]=R(t,p),
\\
\biggl(n-\operatorname{tr}\Phi\biggl(\frac{1}{2}\biggr)\biggr)
h(0)=R(t,p),\qquad
S_{\rm ex}[g]=R(t,p).
\end{gathered}
\end{equation}
Since~$g(0)$ has already been considered~\eqref{eq6.17}, it remains
to find the asymptotics of the finite sum
\begin{equation}
\label{eq6.39}
S_{\rm ex}^1[h]=-\sum_\mu\frac{1-2s_\mu}{4\pi}\int_{-\infty}^{+\infty}
\frac{h(r,p)}{r^2+(s_\mu-1/2)^2}\,dr.
\end{equation}
occurring in the expression~\eqref{eq5.8} for~$M[g]$.

This sum can be represented in the form
\begin{align*}
S_{\rm ex}^1[g]&=\sum_\mu C_\mu J_\mu(t,p),
\\
J_\mu(t,p)&=\int_0^\infty
\frac{e^{-tr^2}dr}{(r^2+a_\mu^2)(r^2+p^2)}\qquad
\biggl(|a_\mu|<\frac12\biggr).
\end{align*}
We use the formula~\cite{19}
$$
\int_0^\infty \frac{e^{-\lambda^2x^2}}{x^2+\beta^2}\,dx=
\bigl(1-\operatorname{erf}(\beta\lambda)\bigr)
\frac{\pi}{2\beta}e^{\beta^2\lambda^2}
$$
and obtain
\begin{gather}
\label{eq6.40}
S_{\rm ex}^1[g]=e^{tp^2}\sqrt{t}\,[C_0+C_1 p^2t+C_2p^4t^2+\cdots]+F(t),
\\
\nonumber
|F(t)|<C\sqrt{t}\,.
\end{gather}

Now we gather all the estimates and arrive at~\eqref{eq6.2}. The
proof of Lemma~\ref{lem5} is complete.
\end{proof}

Let us proceed to the proof of the Roelke conjecture.
V.~A.~Bykovskii noticed to me that, given Eq.~\eqref{eq6.2}, to
prove the Roelke conjecture it suffices to consider the difference
\begin{equation}
\label{eq6.41}
Q(t,p_{1},p_{2})= \sum_{n \geqslant 0}h(r,p_1)-\sum_{n \geqslant 0}h(r,p_2)\qquad
\biggl(p_2>p_1>\frac12\biggr).
\end{equation}

\begin{theorem}[\rm proof of the Roelke conjecture]
\label{th2} For every cofinite group~$\Gamma$ such that $B_0
\geqslant x_0$, one has
\begin{equation}
\label{eq6.41}
N_\Gamma\biggl(T^2+\frac14\biggr)=\#\{n|r_n \leqslant T\}\to\infty,\qquad
T\to\infty.
\end{equation}
\end{theorem}

\begin{proof}
Consider the function $Q(t,p_{1},p_{2})$. The relation~\eqref{eq6.2} implies that the equality
\begin{multline}
\label{eq6.43}
\frac{p_{2}^{2}-p_{1}^{2}}{p_{1}^{2}e^{tp_{1}^{2\mathstrut}}-p_{2}^{2}e^{tp_{2}^{2\mathstrut}}}
\sum\limits_{m\geqslant 0}\frac{e^{-tr_{m}^{2}}}{(r_{m}^{2}+p_{1}^{2})(r_{m}^{2}+p_{2}^{2})}\,=\\
=\,S(t)+\frac{e^{tp_{1}^{2}}-e^{tp_{2}^{2}}}{p_{1}^{2}e^{tp_{1}^{2\mathstrut}}-p_{2}^{2}e^{tp_{2}^{2\mathstrut}}}\,\frac{|F|}{4\pi}\,\ln{\frac{1}{t}}\,-\\
-\,\frac{n}{\pi}\,\frac{I(t,p_{1})-I(t,p_{2})}{p_{1}^{2}e^{tp_{1}^{2\mathstrut}}-p_{2}^{2}e^{tp_{2}^{2\mathstrut}}}\,+\,
\frac{B(t,p_{1})-B(t,p_{2})}{p_{1}^{2}e^{tp_{1}^{2\mathstrut}}-p_{2}^{2}e^{tp_{2}^{2\mathstrut}}}
\end{multline}
holds for any $p_{1}, p_{2}$ ($p_{2}>p_{1}$). M.A.~Korolev noticed that the results from \cite{21} imply the following asymptotic expansion
for the integrak $I(t,p)$ defined by \eqref{eq6.3}
\begin{equation}
\label{eq6.44}
I(t,p)\,=\,\frac{\pi}{2}\ln{p}\,\sum\limits_{n=0}^{\infty}\frac{p^{2n-1}}{n!}\,t^{n}+
\sum\limits_{n=0}^{\infty}p^{2n}t^{n+1/2}d_{n}' + \ln{\frac{1}{t}}\sum\limits_{n=0}^{\infty}p^{2n}t^{n+1/2}d_{n}.
\end{equation}
Suppose that the spectrum $\{r_{m}\}$ is finite. Then we have
\begin{equation}
\label{eq6.45}
\sum\limits_{m\geqslant 0}\frac{e^{-tr_{m}^{2}}}{(r_{m}^{2}+p_{1}^{2})(r_{m}^{2}+p_{2}^{2})} = \sum\limits_{m = 0}^{M}\frac{e^{-tr_{m}^{2}}}{(r_{m}^{2}+p_{1}^{2})(r_{m}^{2}+p_{2}^{2})}
= \sum\limits_{k=0}^{\infty}a_{k}(p_{1},p_{2})t^{k}.
\end{equation}
In this case, the left hand side of~\eqref{eq6.43} becomes an analytic function of $t$.
At the same tine, both the formula~\eqref{eq6.2} for $B(t,p)$ and~\eqref{eq6.44} imply that the asymptotic expansion of the right hand side of~\eqref{eq6.43}
contains the terms
\[
-\,\frac{n}{\pi}\,t^{5/2}\ln{\frac{1}{t}}(p_{1}^{2}+p_{2}^{2})(d_{2}-d_{1}).
\]
The explicit expressions for the coefficients $d_{n}$ in~\eqref{eq6.44} can be derived, and it follows that $d_{2}>d_{1}$.
Thus, the hypothesis about the finiteness of the spectrum leads to the contradiction.
Theorem~\ref{th2} is proved.
\end{proof}

If we consider the case
$$
h(r)=e^{-tr^2},\qquad
g(y)=\frac{1}{\sqrt{4\pi t}}\,e^{-y^2/4t},
$$
then~\eqref{eq5.6} implies the asymptotic relation
\begin{equation}
\label{eq6.46}
\sum_{n\geqslant 0}e^{-tr_n^2}+\sum_{(\alpha)}e^{-t\gamma_\alpha^2}=
\frac{|F|}{4\pi}\,\frac{1}{t}+\frac{n\ln t}{4\sqrt{\pi t}}+
\frac{C}{\sqrt{t}}+C_0+C_1\sqrt{t}+O(t).
\end{equation}
We omit its proof. We only note that it approves with the result of~\cite[Theorem~11.1]{17}.

\begin{remark}
We point out that the left-hand side of~\eqref{eq6.43} is the sum
$\sum_{n\geqslant 0}h(r_n)+\sum_{(\alpha)}h(\gamma_\alpha)$. For
$r_n \gg p$, the same sum is on the left-hand side
in~\eqref{eq6.2} (see also~\eqref{eq6.43} when $r_{m}\to +\infty$).. Such a symmetry ($\{r_n\}\leftrightarrow
\{\gamma_\alpha\}$) apparently holds for a fairly broad class of
functions $h \in \{h\}_S$. This symmetry is related to the fact
that the sum $S_P^3[g|\varphi]$~\eqref{eq5.5} contains a term of
the form
$$
2\gamma_\alpha^{-2}\int_0^b \cos\gamma_\alpha yg^{(2)}(y)\,dy\simeq
h(\gamma_\alpha),
$$
provided that~$g^{(2)}(y)$ decays sufficiently rapidly as $|y|\to
\infty$.
\end{remark}

I am grateful to the anonymous referees, whose remarks helped me
eliminate a number of inaccuracies contained in earlier versions of
the paper and M.A.~Korolev for the help.

\begin{thebibliography}{99}

\bibitem{1}
A.~B.~Venkov,
``Spectral theory of automorphic functions,''
\textit{Trudy Mat. Inst. Steklov}
\textbf{153}, 3--171
(1981).
English transl.:
\textit{Proc. Steklov Inst. Math.}
\textbf{153},
1--163 (1982).


\bibitem{2}
P.~Sarnak,
``Spectra on hyperbolic surfaces,''
\textit{Bull. Amer. Math. Soc.}
\textbf{40}
(4), 441--478
(2003).

\bibitem{3}
A.~Selberg,
\textit{Harmonic Analysis}, Teil 2,
Vorlesungniederschrift,
G\"ottingen,
1954.

\bibitem{4}
A.~B.~Venkov,
``On essentially cuspidal noncongruence subgroups
of $\operatorname{PSL}(2,\mathbb{Z})$,''
\textit{J. Funct. Anal.}
\textbf{92}, 1--7
(1990).

\bibitem{5}
J.~M.~Deshouillers, H.~Iwaniec, R.~S.~Phillips, and P.~Sarnak,
``Maass cusp forms,''
\textit{Proc. Nat. Acad. Sci. USA}
\textbf{82}, 3533--3534
(1985).

\bibitem{6}
R.~S.~Phillips and P.~Sarnak,
``On cusp forms for co-finite subgroups
of $\operatorname{PSL}(2,\mathbb{R})$,''
\textit{Invent. Math.}
\textbf{80}, 339--364
(1985).

\bibitem{7}
S.~A.~Wolpert,
``Spectral limits for hyperbolic surfaces {\rm I,~II},''
\textit{Invent. Math.}
\textbf{108}, 67--89, 91--129
(1992).

\bibitem{8}
S.~A.~Wolpert,
``Disappearance of cusp forms in special families,''
\textit{Ann. Math.}
\textbf{139}, 239--291
(1994).

\bibitem{9}
W.~Luo,
``Nonvanishing of $L$-values and the Weyl law,''
\textit{Ann. Math.}
\textbf{154}, 477--502
(2001).

\bibitem{10}
D.~A.~Popov,
``On the Selberg trace formula for strictly hyperbolic groups,''
\textit{Funktsional. Anal. Prilozhen.}
\textbf{47}
(4), 53--66
(2014).
English transl.:
\textit{Funct. Anal. Appl.}
\textbf{47}
(4), 290--301
(2013).

\bibitem{11}
D.~A.~Popov,
``Selberg formula for cofinite groups and the Roelke conjecture,''
\texttt{http://arxiv.org/abs/1506.08886}v1.

\bibitem{12}
A.~B.~Venkov,
``Remainder term in the Weyl--Selberg asymptotic formula,''
\textit{Zap. Nauchn. Semin. Leningr. Otd. Mat. Inst. Steklova (LOMI)}
\textbf{91}, 5--21
(1979).
English transl.:
\textit{J. Soviet Math.}
\textbf{17}
(5), 2083--2097
(1981).

\bibitem{13}
A.~B.~Venkov,
``Spectral theory of automorphic functions,
       the Selberg zeta-function, and some problems
       of analytic number theory and mathematical physics,''
\textit{Uspekhi Mat. Nauk}
\textbf{36} (3),
69--135
(1979).
English transl.:
\textit{Russian Math. Surveys}
\textbf{34} (3), 79--153
(1979).


\bibitem{14}
D.~A.~Hejhal,
``The Selberg trace formula and the Riemann zeta function,''
\textit{Duke Math. J.}
\textbf{43}
(3), 441--481
(1976).

\bibitem{15}
D.~A.~Hejhal,
``The Selberg trace formula for $\operatorname{PSL}(2,\mathbb{R})$,
 vol.~1,'' \textit{Lect. Notes in Math.}
\textbf{548},
Springer-Verlag, Berlin--New York,
1976.

\bibitem{16}
D.~A.~Hejhal,
``The Selberg trace formula for $\operatorname{PSL}(2,\mathbb{R})$,
vol.~2,'' \textit{Lect. Notes in Math.}
\textbf{1001},
Springer-Verlag, Berlin,
1983.

\bibitem{17}
H.~Iwaniec,
\textit{Introduction to the Spectral Theory of Automorphic Forms},
Revista Math. Iber.,
Madrid, 1995.

\bibitem{18}
I.~S.~Gradshtein and I.~M.~Ryzhik,
\textit{Tables of Integrals, Series, and Products},
Academic Press, New York--London,
1965.


\bibitem{19}
H.~Bateman and A.Erd\'elyi,
\textit{Higher Transcendental Functions},
vol.~2, McGraw--Hill, New York--Toronto--London,
1953.

\bibitem{20}
\textit{Handbook of Mathematical Functions with Formulas, Graphs, and
Mathematical Tables}, M.~Abramowitz and I.~Stegun, eds.,
National Bureau of Standards, Washington, D.C., 1964.

\bibitem{21}
M.F.~Fedoryuk, \textit{Saddle-point method} (in Russian), Moscow, Nauka, 1977.

\end{thebibliography}
\end{document}
