%!TEX root = AppM2_MultiD.tex

\section{Preliminaries}\label{sec:preliminary}
The specific intensity $I(t,\br,\nu,\bOmega)$ is governed by the
radiative transfer equation
\begin{equation}\label{eq:rt}
  \dfrac{1}{c}\pd{I}{t}+\bOmega\cdot\nabla I
  = \cC(I),   
\end{equation}
where $c$ is the speed of light. The variables in the equation are
time $t \in \mathbb{R}^+$, the spatial coordinates
$\br = (x, y, z)\in \mathbb{R}^3$, the angular variables $\bOmega = \left(
\Omega_x, \Omega_y, \Omega_z \right) \in \bbS^2$,
and frequency $\nu\in\mathbb{R}^+$. The right-hand side
$\cC(I)$ describes the interactions between photons and the background medium
and are not the focus of this paper. A typical right-hand side takes the form 
\[
  \cC(I) = -\sigma_a I(\bsOmega) - \sigma_s
  \left(I(\bsOmega) - \frac1{4\pi}\int_{\bbS^2} I(\bsOmega)
  \dd\bsOmega \right),
\]
where $\sigma_a$ and $\sigma_s$ are constant parameters. 
We introduce the moment method in the context of second order models. Let
\begin{equation}\label{eq:v-def}
\begin{array}{rllll}
  \bv = [ &1, &&& \\ [2mm]
  & (\bOmega\cdot\be_x), & (\bOmega\cdot\be_y), & (\bOmega\cdot\be_z),
  &\\ [2mm]
  & (\bOmega\cdot\be_x)^2,&(\bOmega\cdot\be_x)(\bOmega\cdot\be_y),
  &(\bOmega\cdot\be_x)(\bOmega\cdot\be_z), & \\
  && (\bOmega\cdot\be_y)^2,&(\bOmega\cdot\be_y)(\bOmega\cdot\be_z) &]^T.
\end{array}
\end{equation}
Use $\be_x$, $\be_y$ and $\be_z$ to denote the unit vectors along the
coordinate axes. Define
$$
\Vint{ \psi } := \int_{\bbS^2} \psi(\nu, \bOmega)\, \dd\bOmega.
$$
Multiplying equation \eqref{eq:rt} by the vector $\bv$ defined in \eqref{eq:v-def}
and integrating over the angular variables give
\begin{equation}\label{eq:moment-eq}
  \dfrac{1}{c}\pd{\left\langle \bv I\right\rangle}{t}
  + \pd{\Vint{\Omega_x \bv I}}{x} + \pd{\Vint{\Omega_y \bv I}}{y}
  + \pd{\Vint{\Omega_z \bv I}}{z}
  = \left\langle\bv \mathcal{C}(I)\right\rangle.
\end{equation}
In system \eqref{eq:moment-eq}, the time evolution of second-order moments
relies on third-order moments. Therefore \eqref{eq:moment-eq}
is not a closed system. If we approximate the third-order moments in \eqref{eq:moment-eq}
using lower order moments, we could get a closed system. 
Let\footnote{
The notation $a \simeq b$ means `$a$ is an 
approximation of $b$.'
}
\[
  E^0 \simeq \Vint{I}, \quad
  \bE^1 \simeq \Vint{\bOmega I}, \quad
  \bbrE^2 \simeq \Vint{\bOmega\otimes\bOmega I}, \quad
  \bbrE^3 \simeq \Vint{\bOmega\otimes\bOmega\otimes\bOmega
  I}.
\]
A closed system of equations has the form
\begin{equation}\label{eq:moment-closure}
  \begin{split}
    & \dfrac1c\pd{E^0}{t} + \nabla\cdot\bE^1 = r^0(E^0, \bE^1, \bbrE^2), \\
    & \dfrac1c \pd{\bE^1}{t} + \nabla \cdot \bbrE^2 = 
    \br^1(E^0, \bE^1, \bbrE^2), \\
    & \dfrac1c \pd{\bbrE^2}{t} + \nabla \cdot \left[\bbrE^3(E^0, \bE^1,
  \bbrE^2)\right] = 
    \bbr^2(E^0, \bE^1, \bbrE^2).
  \end{split}
\end{equation}
The choice of $\bbrE^3$, $r^0$, $\br^1$, and $\bbr^2$ specify a
\emph{closure}.  The system \eqref{eq:moment-closure} is a second
order moment model.  The following properties of a moment model
concern us the most, which were frequently discussed in the
literature.
\begin{itemize}
\item[] {\bf Rotational invariance: }
Consider a conservation law in multi-dimensions,
\begin{equation}\label{eq:csv}
  \pd{\bU}{t} + \sum\limits_{i=1}^D 
  \pd{\bF_i(\bU)}{x_i} = 0.
\end{equation}
It satisfies rotational invariance if for any unit
vector $\bn=(n_1, \cdots, n_D)^T \in\bbR^D$, there exists a matrix
$\mathbb{T}$ depending on $\bn$, such that
\[
  \sum\limits_{i=1}^D n_i \bF_i(\bU) = 
  \mathbb{T}^{-1} \bF_1(\mathbb{T} \bU).
\]

\item[] {\bf Hyperbolicity:}
Let $\boldsymbol{\mathrm{J}}_i$ be the Jacobian matrix
of the flux function $\bF_i$ in equation \eqref{eq:csv}.
The system \eqref{eq:csv} is hyperbolic if for any
unit vector $\bn\in\bbR^D$, 
$\sum\limits_{i=1}^D n_i \boldsymbol{\mathrm{J}}_i$
is real diagonalizable.

\item[] {\bf Realizability:}
The realizability domain is defined as moments which 
could be generated by a nonnegative distribution
function \cite{junk2000maximum}. A closure is said
to be realizable if the higher order moments it
closes belong to the realizability domain.

For one dimensional problem, \cite{CurFial91} gives necessary and
sufficient conditions for realizability. Its results cover moments of
arbitrary order.  For multi-dimensional case, only the conditions for
the first and second order models are currently known
\cite{kershaw1976flux}, while the conditions for moments of higher
order remain open problems.
\end{itemize}

The maximum entropy models are equipped with all the properties
mentioned above.  For detailed discussions we refer to
\cite{junk2000maximum, levermore1996moment,Dubroca-Feugas-1999}.  We
review the principles for deriving the maximum entropy models by
taking the second order case as an example. It is called the
$M_2$ model.  Solve the following constrained variational minimization
problem
\begin{equation}
\label{eq:mn-opt}
\begin{split}
  \minimize \:\:  & H(I)\\
 \st \:\: & \Vint{ I }= E^0, \Vint{\bOmega I} = \bE^1, 
 \text{and } \Vint{\bOmega \otimes \bOmega I} = \bbrE^2
\end{split}
\end{equation}
where $H(I)$ is the Bose-Einstein entropy
\begin{equation}\label{eq:be-entropy}
 H(I) := \Vint{\frac{2 k_B \nu^2}{c^3} \left(
  \chi I \log \left( \chi I \right)
  - \left(\chi I + 1 \right) \log \left( \chi I + 1 \right)
  \right)},
\end{equation}
where $\chi = \dfrac{c^2}{2 \hbar \nu^3}$.  This gives us an ansatz
\begin{equation}\label{eq:be-ansatz}
 \mansatz(\nu, \bOmega) =
  \dfrac{1}{\chi} \left( \exp \left( -\dfrac{\hbar \nu}{k_B}
    \bsalpha\cdot\bv \right)-1 \right)^{-1},
\end{equation}
where $\bsalpha \cdot \bv $ is a second order polynomial of
$\bOmega \in \bbS^2$.  The parameters $\bsalpha$ is the unique vector
such that
$$\Vint{ \mansatz} = E^0, \quad 
\Vint{\bOmega \mansatz} = \bE^1,
\text{and} \quad 
\Vint{\bOmega \otimes \bOmega \mansatz} = \bbrE^2.
$$
The $M_2$ method is defined by taking
\begin{equation}\label{eq:mn-flux-and-production}
  \begin{array}{ll}
    \bbrE^3 := \Vint{\bOmega \otimes \bOmega \otimes \bOmega \mansatz},
    \quad & r^0 := \Vint{ \cC (\mansatz)},\\
    \br^1 := \Vint{ \bOmega \cC (\mansatz)}, \quad
    & \bbr^2 := \Vint{ \bOmega \otimes \bOmega \cC
    (\mansatz)}.
  \end{array}
\end{equation}
in \eqref{eq:moment-closure}.

However, the $M_2$ closure is not given explicitly, so
\eqref{eq:mn-flux-and-production} has to be computed by solving the
optimization problem \eqref{eq:mn-opt} numerically. The numerical
optimization at each time step for all spatial grid is extremely 
expensive.

Recent work \cite{pichard2016approximation} proposes an approximation
of the $M_2$ method in multi-dimensions by directly approximating its
closure relation, though the corresponding ansatz to the closure is
not clarified. We adopt the approach of constructing an ansatz to
approximate the $M_2$ ansatz, then the closure relation is given
naturally as in \eqref{eq:mn-flux-and-production}.

In a previous work \cite{alldredge2016approximating}, we examined the
properties of second order extended quadrature method of moments
(EQMOM) proposed in \cite{vikas2013radiation} in slab geometry, and
the model was referred as the $B_2$ model. In EQMOM, the ansatz
$\hat I$ is reconstructed by a combination of beta distributions. The
beta distribution on $[-1, 1]$ is given by
$$
\mathcal{F}(\mu; \gamma, \delta) = \frac1{2 \Beta(\xi, \eta)}
 \left(\frac{1 + \mu}2 \right)^{\xi - 1}
 \left(\frac{1 - \mu}2 \right)^{\eta-1},
\quad 
 \xi  = \frac\gamma\delta, \quad
 \eta = \frac{1 - \gamma}\delta.
$$
where $\Beta(\xi, \eta)$ is the beta function. For the $B_2$ model in
1D slab geometry, the ansatz is taken as
\[
w \mathcal{F}(\mu; \gamma, \delta)
\]
where the parameters $w$, $\gamma$, and $\delta$ are given by
consistency to the known moments.

We found that the 1D $B_2$ model shares the key features of the $M_2$
model in slab geometry, including existence of non-negative ansatz and
therefore realizability, as well as global hyperbolicity. It is the
focus of this paper to extend the 1D $B_2$ model to three-dimensional
case.

Our motivation to this extension is based on observing a common
attribute between the $B_2$ and the $M_2$ ansatz in 1D slab
geometry. Both ansatz can exactly recover the isotropic distribution,
and at the same time, give a combination of Dirac functions on the
boundary of the realizability domain. Dirac functions could not be
recovered by the standard spectral method which has a polynomial as an
ansatz. It has been pointed out that the inability to capture
anisotropy is a drawback of the standard spectral method
\cite{frank2006partial}.

In three-dimensional space, the anisotropy of the specific intensity
could come in orthogonal directions. For example, we consider a setup
similar to the double beam problem discussed in
\cite{pichard2016approximation} \footnote{ In
  \cite{pichard2016approximation}, this example is used to demonstrate
  the advantage of the $M_2$ model over its first-order counterpart,
  the $M_1$ model.  }.  For the region $[x,y]\in[-1,1]\times[-1,1]$,
consider equation \eqref{eq:rt} with the right-hand side chosen as
isotropic scattering (which means $\sigs$ is a non-negative constant):
\[
  \cC(I) = \sigs \left( -I + \dfrac{1}{4\pi} \Vint{I} 
  \right).
\]
Laser beams are imposed as boundary inflow from orthogonal 
directions: $I = \delta(\bsOmega\cdot\be_x-1)$ on the 
boundary $x = -1$, and $I = \delta(\bsOmega\cdot\be_y - 1)$
on the boundary $y = -1$. 


For the extreme case when the medium
is vacuum and $\sigs = 0$, the exact solution for any $ct > 2$
is
\begin{equation}\label{eq:crossing-beam-distribution}
  I = \delta(\bsOmega\cdot\be_x - 1) + 
  \delta(\bsOmega\cdot\be_y - 1).
\end{equation}
It is pointed out in \cite{pichard2016approximation} that 
the $M_2$ ansatz is able to exactly reproduce the distribution in
\eqref{eq:crossing-beam-distribution} from the moments. We aim to construct an
ansatz that can capture anisotropy in  
orthogonal directions, like the $M_2$ ansatz.

For non-vanishing scattering, the steady-state solution of the above 
problem is an isotropic distribution. For any period before steady-state
is reached, the exact specific intensity $I$ should be somewhere
between double beams, as in \eqref{eq:crossing-beam-distribution},
and isotropic. The ansatz of the $M_2$ model provides
a smooth interpolation between these two extremes, giving it an advantage 
in simulating such problems. We aim to propose an
ansatz with similar features. This will be discussed in the following sections.

%%% Local Variables: 
%%% mode: latex
%%% TeX-master: "AppM2_MultiD.tex"
%%% End: 
