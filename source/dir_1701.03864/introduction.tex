%!TEX root = AppM2_MultiD.tex


\section{Introduction}

The radiative transfer equations describe the transportation of light in a medium
\cite{pomraning1973equations}. They are \emph{kinetic equations}, and the unknown 
is the specific intensity of photons. The specific intensity is a function of time, 
spatial coordinates, frequency, and angular variables. The moment method is an 
efficient approach for reducing the computation cost brought about by the high-dimensionality of 
variables of kinetic equations.

Moments are obtained by integrating the specific intensity against
monomials of the angular variables. In many applications, the
quantities of interest are the few lowest order moments. Therefore
moments are good choices for discretizing the angular variables.
However, moment systems are not closed. Closing the system by
specifying a constitutive relationship is known as the
\emph{moment-closure problem}. One approach towards moment-closure is
to recover the angular dependence of the specific intensity from the
known moments. The reconstructed specific intensity is called an
\emph{ansatz}. Ideally, the ansatz should be non-negative for all
moments which can be generated by a non-negative distribution. Also,
one would like the system to be hyperbolic since hyperbolicity is
necessary for the local well-posedness of Cauchy problem.  Other
natural requirements include that the ansatz satisfies rotational
invariance and reproduces the isotropic distribution at equilibrium.
Numerous forms of ans\"atze have been studied in the literature
\cite{pomraning1973equations, Lewis-Miller-1984}.  Yet, in
multi-dimensional cases, the maximum entropy method, referred to as
the $M_n$ model, is perhaps the
only method known so far to have both realizability and global
hyperbolicity \cite{Dubroca-Feugas-1999}. However, the flux functions
of the maximum entropy method are generally not explicit
\footnote{
With the first order maximum entropy model for the grey equations as the only exception
\cite{Dubroca-Feugas-1999}.
}, so
numerically computing such models involve solving highly
nonlinear and probably ill-conditioned optimization problems
frequently. There have been continuous efforts on speeding up the
computation process \cite{alldredge2012high, alldredge2014adaptive,
  garrett2015optimization}. Recently, there are also attempts in
deriving closed-form approximations of the maximum entropy closure in
order to avoid the expensive computations. For 1D cases, an
approximation to the $M_n$ models using the Kershaw closure is given
in \cite{schneider2016kershaw}.  For multi-dimensional cases, a model
based on directly approximating the closure relations of the $M_1$
and $M_2$ methods is proposed in \cite{pichard2016approximation}.  Our
work in this paper also aims at constructing closed-form
approximations of the maximum entropy model. Like
\cite{pichard2016approximation}, we seek a closed-form approximation
to the $M_2$ method in 3D. But unlike \cite{pichard2016approximation},
we derive our model from an ansatz with some similarity to that of the
$M_2$ model.

In a previous study \cite{alldredge2016approximating}, we analyzed the
second order extended quadrature method of moments
(EQMOM) introduced in \cite{vikas2013radiation} which we call the $B_2$
model.  
In this work, we propose an approximation of the $M_2$ model in
3D space by extending the $B_2$ model studied in
\cite{alldredge2016approximating} to 3D. The reason for this approach
is that the $B_2$ ansatz shares the following properties with the $M_2$ ansatz:
\begin{enumerate}
\item it interpolates smoothly between isotropic and Dirac
  distribution functions;
\item it captures anisotropy in opposite directions. 
\end{enumerate}
The $B_2$ closure in \cite{alldredge2016approximating} is for slab
geometries. Preserving rotational invariance when extending it to 3D
space is non-trivial. We use the sum of the axisymmetric $B_2$
ans\"atze in three mutually orthogonal directions as the ansatz for a
second order moment model in 3D space. This new model is referred to
as the {\it 3D $B_2$ model}. The consistency of known moments requires
the three mutually orthogonal directions to be the three eigenvectors
of the second-order moment matrix. We point out that there are three free
parameters in the ansatz of the 3D $B_2$ model after the consistency of 
known moments is fulfilled. These parameters are
specified as functions of the first-order moments and the eigenvalues
of the second-order moment matrix. We prove that the 3D $B_2$ model is
rotationally invariant.  The region where the model possesses a
non-negative ansatz is illustrated, as well as the hyperbolicity region of the model
with vanished first-order moment. Though far from
perfect, the 3D $B_2$ model shares some important features of the
$M_2$ closure. Also, the model has explicit flux functions, making it
very convenient for numerical simulations.


The rest of this paper is organized as follows. In Section 2 we recall
the basics of moment models, and briefly, introduce the $M_2$ method
as well as the $B_2$ model for 1D slab geometry. In Section 3 we
propose the 3D $B_2$ model. In Section 4 we analyze its
properties. Finally, in Section 5 we summarize and discuss future
work.

%%% Local Variables: 
%%% mode: latex
%%% TeX-master: "AppM2_MultiD.tex"
%%% End: 
