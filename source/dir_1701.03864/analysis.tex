%!TEX root = AppM2_MultiD.tex

\section{Model Properties}\label{sec:properties}

In this section, we will study the rotational invariance, realizability,
and hyperbolicity of the 3D $B_2$ model proposed.

The proof of rotational invariance is almost straightforward for our
model. This is because all the parameters $\bR_i$, $w_i$,
$\gamma_i$, and $\delta_i$ in the ansatz $\bansatz$ are given as
functions of known moments $E^0$, $\bE^1$, and $\bbrE^2$. Consequently,
the ansatz is rotationally invariant, so we conclude that the
moment system produced by $\bansatz$ has rotational
invariance. More precisely, we have 
\begin{theorem}\label{lem:rotational_invariant}
  The 3D $B_2$ model \eqref{eq:moment_model} is rotationally invariant.
\end{theorem}
\begin{proof}
  For any unit vector $\bn = (n_x, n_y, n_z) \in \mathbb{R}^3$, there
  exists a rotation to transform $\bn$ to the $x$-axis. Let 
  $\bn = \bbrT \be_x$, where $\bbrT$ is the rotation matrix. The
  rotated velocity is denoted by $\tilde{\bOmega} = \bbrT^T \bOmega$.
  We denote $\tilde{\be}_x = \bn = \bbrT \be_x$,
  $\tilde{\be}_y = \bbrT \be_y$ and $\tilde{\be}_z = \bbrT \be_z$.
  After the rotation, the known moments are denoted by $\tilde{\bE}$,
  and we write the ansatz before and after the rotation with
  explicit dependence on the known moments by
  $\bansatz(\bOmega; \bE)$ and $\bansatz(\bOmega; \tilde{\bE})$. We
  use $\tilde{E}^0$, $\tilde{\bE}^1$, and $\tilde{\bbrE}^2$ to denote
  the corresponding moments after the rotation, respectively. Let us
  define $\tilde{\bv}$ as
  \[
  \begin{array}{rllll}
    \tilde{\bv} = [ & 1, &&& \\ [2mm]
                    & (\bOmega\cdot\tilde{\be}_x), 
                         & (\bOmega\cdot\tilde{\be}_y), 
                           & (\bOmega\cdot\tilde{\be}_z), & \\ [2mm]
                    & (\bOmega\cdot\tilde{\be}_x)^2, 
                         & (\bOmega\cdot\tilde{\be}_x)(\bOmega\cdot\tilde{\be}_y), 
                           & (\bOmega\cdot\tilde{\be}_x)(\bOmega\cdot\tilde{\be}_z), & \\
                    && (\bOmega\cdot\tilde{\be}_y)^2, 
                         &(\bOmega\cdot\tilde{\be}_y)(\bOmega\cdot\tilde{\be}_z) & ]^T.
  \end{array}
  \]
  It is clear there exists a transformation matrix $\mathbb{T}$
  which depends only on $\bbrT$ such that
  \[
    \tilde{\bv} = \mathbb{T} \bv,
  \]
  where $\bv$ is defined in \eqref{eq:v-def}.
  Thus, the known moments satisfy $\tilde{\bE} = \mathbb{T} \bE$ and
  \[ 
  \tilde{E}^0 = E^0, \quad \tilde{\bE}^1 = \bbrT \bE^1, \quad
  \tilde{\bbrE}^2 = \bbrT^T \bbrE^2 \bbrT.
  \]
  Consequently, the eigenvectors $\tilde{\bR}_i$ of $\tilde{\bbrE}^2$
  are $\tilde{\bR}_i = \bbrT^T \bR_i$, and thus,
  \[
  \begin{aligned}
    &\tilde{\bOmega} \cdot \tilde{\bR}_i = \bOmega \cdot \bR_i, \\
    &\tilde{F}_i = \tilde{\bE}^1 \cdot \tilde{\bR}_i = \bE^1 \cdot
    \bR_i = F_i.
  \end{aligned}
  \]
  The given closure for $w_i$, $\gamma_i$, and $\delta_i$
  are functions of the eigenvalues of $\bbrE^2$ and $F_i$, $i=1,2,3$. Thus,
  these parameters are exactly the same before and after the
  rotation. Therefore, the ansatz after the rotation satisfies 
  \[
  \begin{aligned}
    \bansatz(\tilde{\bOmega}; \mathbb{T} \bE) &= \sum_{i=1}^3
    \dfrac1{2\pi} w_i f(\tilde{\bOmega} \cdot \tilde{\bR}_i; \gamma_i,
    \delta_i) \\
    &= \sum_{i=1}^3 \dfrac1{2\pi} w_i f(\bOmega \cdot \bR_i; \gamma_i,
    \delta_i) = \bansatz(\bOmega; \bE).
  \end{aligned}
  \]
  Meanwhile, notice that we have the relation
  \[
  \begin{array}{rllll}
    \tilde{\bv} = [ & 1, &&& \\ [2mm]
                    & (\tilde{\bOmega}\cdot \be_x), 
                         & (\tilde{\bOmega}\cdot \be_y), 
                           & (\tilde{\bOmega}\cdot \be_z), & \\ [2mm]
                    & (\tilde{\bOmega}\cdot\be_x)^2, 
                         & (\tilde{\bOmega}\cdot\be_x)(\tilde{\bOmega}\cdot\be_y), 
                           & (\tilde{\bOmega}\cdot\be_x)(\tilde{\bOmega}\cdot\be_z), & \\
                    && (\tilde{\bOmega}\cdot\be_y)^2, 
                         &(\tilde{\bOmega}\cdot\be_y)(\tilde{\bOmega}\cdot\be_z)
                           & ]^T \\
  \end{array}
  \]
  Therefore,
  \[
  \begin{split}
    n_x \bff_x(\bE) + n_y \bff_y(\bE) + n_z \bff_z(\bE) = &
    \int_{\bbS^2}(n_x \Omega_x + n_y \Omega_y + n_z \Omega_z)
    \bv \bansatz (\bOmega; \bE) \dd\bOmega \\
    = & \int_{\bbS^2}(\bOmega \cdot \tilde{\be}_x)~ (\mathbb{T}^{-1}
    \tilde{\bv})~
    \bansatz(\tilde{\bOmega}; \mathbb{T} \bE) ~ | \bbrT | ~\dd \tilde{\bOmega} \\
    = & ~\mathbb{T}^{-1} \int_{\bbS^2} (\tilde{\bOmega} \cdot \be_x)~
    \tilde{\bv}
    \bansatz(\tilde{\bOmega}; \mathbb{T} \bE) \dd \tilde{\bOmega} \\
    = & ~\mathbb{T}^{-1} \int_{\bbS^2} (\bOmega \cdot \be_x)~ \bv
    \bansatz(\bOmega; \mathbb{T} \bE) \dd \bOmega \\
    = & ~\mathbb{T}^{-1} \bff_x(\mathbb{T} \bE).
  \end{split}
  \]
  This gives us rotational invariance \footnote{We note that the
    proof is not at all dependent on whether the function $f$ in the ansatz is
    assigned as a beta distribution.}.
\end{proof}


Let us turn to the realizability of our model. First, we point out
that the 3D $B_2$ model provides a non-negative ansatz even for some
moments on the boundary of the realizability domain. For example,
the moments satisfying $|F_i| = \lambda_i$, $\forall i = 1, 2, 3$, 
correspond to ans\"atze of the form
\[
  \bansatz = \sum\limits_{i=1}^3
  \left[\alpha^+_i
    \delta(\bOmega\cdot\bR_i-1)+
    \alpha^-_i
  \delta(\bOmega\cdot\bR_i+1)\right].
\]
We recall the following results from Lemma \ref{lem:realizable-bound}: if
$\lambda_i$ are distinct positive values, then the eight vertices of 
the rectangular box $|F_j| \leq \lambda_j$, $j = 1,2,3$, are the only points 
on the boundary of the realizability domain where a non-negative 
ansatz for $\bansatz$ may exist.
Moreover, the ansatz contains the equilibrium distribution.
Moments satisfying $\lambda_i = \dfrac{E^0}{3}$, $i=1,2,3$, and
$\bE^1 = \boldsymbol{0}$ reproduce $\bansatz = \dfrac{E^0}{4\pi}$.


Recall that 
\eqref{eq:condition-positive-ansatz} is a sufficient condition for
\eqref{eq:B2-ansatz} to give a non-negative ansatz. It is equivalent
to
\begin{equation}\label{eq:condition-realizable-v1}
  0 \leq \sigma_i \leq w_i,\quad
  \text{and}\quad \sigma_i w_i \geq F_i^2,\quad
  i = 1,2,3.
\end{equation}
We examine this condition to check the realizability of our
model. Define the following discriminant
\begin{equation}\label{eq:discriminant}
  \Delta \triangleq \min \left\{w_1 \sigma_1 - F_1^2, w_2 \sigma_2 - F_2^2, w_3
  \sigma_3 - F_3^2 \right\}.
\end{equation}
Instantly, we have
\begin{theorem}\label{thm:condition-nonnegative-ansatz}
  For $|F_j| \leq \lambda_j \neq 0$, $j = 1,2,3$, the 3D $B_2$ model has
  a non-negative ansatz $\bansatz$ if 
  \begin{equation}\label{eq:cond-sigma-positive}
    3 \lambda_i^2 + \lambda_i (\lambda_j + \lambda_k) - \lambda_j
    \lambda_k > 0, \quad \forall~i,j,k,~~\text{mutually different},
  \end{equation}
  and
  \[ 
  \Delta \geq 0.
  \]
\end{theorem}
\begin{proof}
  We first prove $\sigma_1 \leq w_1$. Notice that 
  \[
    w_1 - \sigma_1 = 2 g(\lambda_2,\lambda_3; F_2,F_3)
    = \dfrac{4 \left(\lambda_2 - \dfrac{F_2^2}{\lambda_2}\right)
      \left(\lambda_3-\dfrac{F_3^2}{\lambda_3}\right)
      \left(\lambda_2 - \dfrac{F_2^2}{\lambda_2} +
    \lambda_3 - \dfrac{F_3^2}{\lambda_3}\right)}{3
    (\lambda_2 + \lambda_3)^2}.
  \]
  Also, if $|F_j| \leq \lambda_j$, we have 
  \[
    \lambda_j - \dfrac{F_j^2}{\lambda_j} \geq 0.
  \]
  Therefore, inside the rectangular box $|F_j| \leq \lambda_j$, $j = 1,2,3$,
  we have $w_1 - \sigma_1 \geq 0$. Similarly, we could prove
  $\sigma_2 \leq w_2$ and $\sigma_3 \leq w_3$.

  We now discuss the condition for $\sigma_i \geq 0$, $i = 1,2,3$.
  We begin by examining $\sigma_1$. From \eqref{eq:sigma-formula}, we see that 
  for fixed $\lambda_i$, $i = 1,2,3$, the function $\sigma_1$ monotonically increases
  for any $|F_j|$. Therefore, if $\sigma_1 \geq 0$ holds for $\bE^1 = \boldsymbol{0}$, 
  then it is valid for the whole rectangular box
  $|F_j| \leq \lambda_j$, $j = 1,2,3$. So, the problem becomes seeking
  $(\lambda_1, \lambda_2,\lambda_3)$ for which
  $\left.\sigma_1\right|_{F_1 = F_2 = F_3 = 0} \geq 0$ holds. As
  \[
    \left.\sigma_1\right|_{F_1=F_2=F_3=0} = 
    \dfrac{\lambda_1(3\lambda_1^2 + \lambda_1\lambda_2 + \lambda_1 \lambda_3
    - \lambda_2\lambda_3)}{3(\lambda_1 + \lambda_2)(\lambda_1 + \lambda_3)},
  \]
  the necessary and sufficient condition for $\sigma_1 > 0$ is
  \begin{equation}\label{eq:cond-sigma-positive-1}
    3\lambda_1^2 + \lambda_1(\lambda_2 + \lambda_3) - \lambda_2\lambda_3 
    > 0,
  \end{equation}
which completes our proof.
\end{proof}
From the proof of Theorem \ref{thm:condition-nonnegative-ansatz}
we have the following corollary.
\begin{corollary}
  Let $\bE^1 = \boldsymbol{0}$.  If \eqref{eq:cond-sigma-positive} is
  valid and $\lambda_i \neq 0$, $\forall i = 1,2,3$, the 3D $B_2$
  model has a non-negative ansatz.
\end{corollary}
\begin{proof}
  In the case of $\bE^1 = \boldsymbol{0}$, $\Delta > 0$ is automatically
  valid under the conditions specified in the corollary.
\end{proof}

Given $\lambda_i$ and $F_i$, $i=1,2,3$, we could use the 
condition placed on the discriminant $\Delta$ in Theorem 
\ref{thm:condition-nonnegative-ansatz} to verify whether 
a non-negative ansatz exists. For each fixed 
$(\lambda_1, \lambda_2, \lambda_3)$, we sample for  
the whole region within the rectangular box $|F_j| \leq \lambda_j$, 
$j = 1,2,3$. It is found that if 
$\dfrac{\lambda_i}{E^0}\geq\dfrac17$, $i = 1,2,3$, then for any
$(F_1, F_2, F_3)$ belonging to the region $|F_j| \leq \lambda_j$,
$j = 1,2,3$, the 3D $B_2$ model has a non-negative ansatz. Note
that the realizability domain for $F_j$ is the ellipsoid given in Lemma
\ref{proposition-realizability}, and the rectangular box $|F_j| \leq \lambda_j$,
$j = 1,2,3$, is contained within the ellipsoid, with its eight
vertices touching the domain boundary. 
Figure \ref{fig:realizable-region} illustrates the region that is found to
admit a non-negative ansatz.
\begin{figure}
  \subfigure[$\left(\dfrac{\lambda_1}{E^0},\dfrac{\lambda_2}{E^0},
    \dfrac{\lambda_3}{E^0}\right)$ are taken as barycentric coordinates within the 
    triangle. The outer triangle is the realizability domain. The curves
    correspond to the outer boundary of the constraints \eqref{eq:cond-sigma-positive}.
    The blue region gives non-negative ansatz for 3D $B_2$ model
    for all $\bE^1$ satisfying $|F_j|\leq\lambda_j$, $j = 1,2,3$. 
  ]{
  \includegraphics[width=0.4\textwidth]{./images/positive_ansatz_region}}  
  \hfill
  \subfigure[The sphere correspond to the realizability domain of $\bE^1$ when
    $\lambda_1 = \lambda_2 = \lambda_3$.The rectangle within the sphere is
  the region for $\bE^1$ when the 3D $B_2$ model has a non-negative ansatz.]{
  \includegraphics[width=0.35\textwidth]{./images/Region_F}} 
  \hfill
  \caption{Region which correspond to a non-negative ansatz for 3D $B_2$ model.}
  \label{fig:realizable-region}
\end{figure}

\begin{remark}
  By Lemma \ref{lem:E3}, for $\bE^1=\boldsymbol{0}$, the third-order
  moments given by the 3D $B_2$ ansatz is a zero tensor, equal to that given by
  $M_2$. For this particular case, even when there is no non-negative ansatz, 
  the closure relation is still realizable.
\end{remark}

We proceed to study the hyperbolicity of the model. Due to the extreme
complexity of the formula, we restrict our discussions to the case that
$\bE^1 = \boldsymbol{0}$. We first prove the following facts:
\begin{lemma}\label{lem:fact-bound-sigma-w}
  In the interior of the realizability domain $\cM$, if
  $\bE^1 = \boldsymbol{0}$, we have
  \[
  w_i > 0,\quad \sigma_i + w_i > 0,\quad i = 1,2,3.
  \]
\end{lemma}
\begin{proof}
  Take $i = 1$ for example. First, note
  \[
    g(x,y; 0,0) = \dfrac{2 q(x,y; 0,0)(x + y -1 -r(x,y;0,0))}{3(x+y)^2} 
    = \dfrac{2 xy}{3(x+y)}.
  \]
  Therefore,
  \[
    \begin{split}
      w_1 =    & \lambda_1 - g(\lambda_1,\lambda_2;0,0) -g(\lambda_1, \lambda_3;0,0)
      + 2 g(\lambda_2,\lambda_3;0,0) \\
      = & \dfrac13\left(3\lambda_1 - \dfrac{2\lambda_1 \lambda_2}
      {\lambda_1 + \lambda_2} - \dfrac{2\lambda_1 \lambda_3}
      {\lambda_1 + \lambda_3} + \dfrac{4 \lambda_2 \lambda_3}
    {\lambda_2 + \lambda_3} \right) \\
    = & \dfrac13\left(\lambda_1 - \dfrac{2\lambda_1 \lambda_2}
    {\lambda_1 + \lambda_2} + 2 \lambda_1 - \dfrac{2 \lambda_1 \lambda_3}
    {\lambda_1 + \lambda_3}
  + \dfrac{4 \lambda_2 \lambda_3}{\lambda_2 + \lambda_3}\right) \\
  = & \dfrac13\left(\lambda_1 - \dfrac{2\lambda_1 \lambda_2}
  {\lambda_1 + \lambda_2} + \dfrac{2 \lambda_1(\lambda_1 + \lambda_3 - \lambda_3)}
  {\lambda_1 + \lambda_3}
+ \dfrac{4 \lambda_2 \lambda_3}{\lambda_2 + \lambda_3}\right) \\
= & \dfrac13\left(\lambda_1 - \dfrac{2\lambda_1 \lambda_2}
{\lambda_1 + \lambda_2} + \dfrac{2 \lambda_1^2}{\lambda_1 + \lambda_3}
    + \dfrac{4 \lambda_2 \lambda_3}{\lambda_2 + \lambda_3}\right). 
  \end{split}
\]
We need to prove $w_1 \geq 0$ for two cases: 
  \begin{enumerate}
  \item $\lambda_1 \geq \lambda_2$ or $\lambda_1 \geq \lambda_3$.
    Because $w_1$ is symmetric with respect to $\lambda_2$ and
    $\lambda_3$, we only need to discuss the case 
    $\lambda_1 \geq \lambda_2$.
  \item $\lambda_1 < \lambda_2$ and $\lambda_1 < \lambda_3$.
    Due to $w_1$ being symmetric about $\lambda_2$ and $\lambda_3$,
    we only need to discuss the case $\lambda_1 < \lambda_2 \leq
    \lambda_3$.
  \end{enumerate}
  The proof is as follows:
  \begin{enumerate}
  \item If $\lambda_1 \geq \lambda_2$, then
    \[
    -\dfrac{2\lambda_1 \lambda_2}{\lambda_1 + \lambda_2}
    \geq -\dfrac{2\lambda_1 \lambda_2}{\lambda_2 + \lambda_2}
    = -\lambda_1,
    \]
    therefore
    \[
    w_1 = \dfrac13\left(\lambda_1 - \dfrac{2\lambda_1 \lambda_2}
      {\lambda_1 + \lambda_2} + \dfrac{2 \lambda_1^2}{\lambda_1 + \lambda_3}
      + \dfrac{4 \lambda_2 \lambda_3}{\lambda_2 + \lambda_3}\right)
    \geq \dfrac13\left(\dfrac{2 \lambda_1^2}{\lambda_1 + \lambda_3}
      + \dfrac{4 \lambda_2 \lambda_3}{\lambda_2 + \lambda_3}\right) > 0.
    \]
  \item If $\lambda_1 < \lambda_2 \leq \lambda_3$, then
    \[
    -\dfrac{2\lambda_1 \lambda_2}{\lambda_1 + \lambda_2}
    \geq -\dfrac{2\lambda_1 \lambda_2}{\lambda_1 + \lambda_1}
    = -\lambda_2,
    \]
    and
    \[
    \dfrac{4\lambda_2 \lambda_3}{\lambda_2 + \lambda_3} \geq
    \dfrac{4\lambda_2 \lambda_3}{\lambda_3 + \lambda_3}
    = 2\lambda_2.
    \]
    Therefore
    \[
    w_1 = \dfrac13\left(\lambda_1 - \dfrac{2\lambda_1 \lambda_2}
      {\lambda_1 + \lambda_2} + \dfrac{2 \lambda_1^2}{\lambda_1 + \lambda_3}
      + \dfrac{4 \lambda_2 \lambda_3}{\lambda_2 + \lambda_3}\right)
    \geq \dfrac13\left(\lambda_1 - \lambda_2 + 
      \dfrac{2 \lambda_1^2}{\lambda_1 + \lambda_3} + 2\lambda_2
    \right)
    > 0.
    \]
  \end{enumerate}
  This proves $w_1 > 0$. Similarly, $w_j > 0$, $j = 2,3$.

  Next, we prove $\sigma_1 + w_1 > 0$. We have
  \[
    \sigma_1 + w_1   = \dfrac23\left(3\lambda_1 - \dfrac{2\lambda_1 \lambda_2}
      {\lambda_1 + \lambda_2} - \dfrac{2\lambda_1 \lambda_3}
      {\lambda_1 + \lambda_3} + \dfrac{2 \lambda_2 \lambda_3}
    {\lambda_2 + \lambda_3} \right)
    = \dfrac23\left(\lambda_1 - 
      \dfrac{2\lambda_1 \lambda_2}{\lambda_1 + \lambda_2}
      + \dfrac{2\lambda_1^2}{\lambda_1 + \lambda_3} + 
      \dfrac{2\lambda_2 \lambda_3}{\lambda_2 + \lambda_3}
    \right).
  \]
  Similar to discussions on $w_1$, we have
  \begin{enumerate}
  \item If $\lambda_1 \geq \lambda_2$, then
    \[
    \sigma_1 + w_1 =   \dfrac23\left(\lambda_1 - 
      \dfrac{2\lambda_1 \lambda_2}{\lambda_1 + \lambda_2}
      + \dfrac{2\lambda_1^2}{\lambda_1 + \lambda_3} + 
      \dfrac{2\lambda_2 \lambda_3}{\lambda_2 + \lambda_3}
    \right)
    \geq 
    \dfrac23\left(\dfrac{2\lambda_1^2}{\lambda_1 + \lambda_3} + 
      \dfrac{2\lambda_2 \lambda_3}{\lambda_2 + \lambda_3}
    \right).
    \]
  \item If $\lambda_1 < \lambda_2 \leq \lambda_3$, then
    \[
    \sigma_1 + w_1 = \dfrac23\left(\lambda_1 - 
      \dfrac{2\lambda_1 \lambda_2}{\lambda_1 + \lambda_2}
      + \dfrac{2\lambda_1^2}{\lambda_1 + \lambda_3} + 
      \dfrac{2\lambda_2 \lambda_3}{\lambda_2 + \lambda_3}
    \right)
    \geq \dfrac23\left(\lambda_1 - \lambda_2 + 
      \dfrac{2 \lambda_1^2}{\lambda_1 + \lambda_3} + \lambda_2
    \right)
    > 0.
    \]
  \end{enumerate}
  Similarly, $\sigma_i + w_i > 0$, $i = 2,3$.
\end{proof}

To study hyperbolicity, we start with calculating the Jacobian
matrix of the flux $\bff_x$, $\bff_y$, and $\bff_z$. Due to the rotational
invariance of the 3D $B_2$ model, it could be assumed without loss of
generality that $\bbrE^2$ is diagonal, $\bR_1$ is parallel to
the $x$-axis, $\bR_2$ is parallel to the $y$-axis, and $\bR_3$ is
parallel to the $z$-axis, respectively. The most involving part in
calculating the Jacobian matrix is the derivatives of third-order
moments. We first note that, by Lemma \ref{lem:E3}, fixing
$\bE^1 = \boldsymbol{0}$ makes the value of all third-order moments
zero, no matter what the values of the other moments are. Therefore,
\[
\pd{E^3_{ijk}}{E^0} = 0, \quad \pd{E^3_{ijk}}{E^2_{lm}} = 0, 
\quad\forall i,j,k,l,m = 1,2,3.
\]
So we only need to compute $\pd{E^3_{ijk}}{E^1_l}$.
First, we have
\[
\begin{split}
  \pd{E^3_{123}}{E^1_l} = & \pd{\Vint{(\bOmega\cdot\bR_i) (\bOmega\cdot\bR_j)
      (\bOmega\cdot\bR_k) \bansatz}}{E^1_l} R_{i1} R_{j2} R_{k3}\\
  = & \pd{\Vint{(\bOmega\cdot\bR_1) (\bOmega\cdot\bR_2) (\bOmega\cdot\bR_3) \bansatz}}{E^1_l} 
  =  0.
\end{split}
\]
For the terms $\pd{E^3_{iij}}{E^1_k}$, we have
\[
\pd{E^3_{iij}}{E^1_k} = \pd{\Vint{(\bOmega\cdot\bR_l) (\bOmega\cdot\bR_m)
    (\bOmega\cdot\bR_n) \bansatz}}{E^1_k} R_{li} R_{mi} R_{nj} 
= \pd{\Vint{(\bOmega\cdot\bR_i)^2 (\bOmega\cdot\bR_j) \bansatz}}{E^1_k}.
\]
And by
\[
F_i = \bE^1\cdot\bR_i = E^1_1 R_{1i} + E^1_2 R_{2i} + E^1_3 R_{3i},
\]
we get $\pd{F_i}{E^1_k} = \delta_{ik}$, which is used below in
computing $\pd{E^3_{iij}}{E^1_k}$.

If $i = j$ and $k \not= i$,
\[
\pd{\Vint{(\bOmega\cdot\bR_i)^3 \bansatz}}{E^1_k} 
=  F_i \dfrac{\partial}{\partial E^1_k}\left( \dfrac{ \sigma_i^2 
    + 2 F_i^2 - 3 w_i \sigma_i }
  {2 F_i^2 - w_i \sigma_i - w_i^2}\right) 
= 0.
\]
And if $i \not= j$ and $k \not= j$,
\[
\pd{\Vint{(\bOmega\cdot\bR_i)^2 (\bOmega\cdot\bR_j)
    \bansatz}}{E^1_k} 
=  \dfrac{ F_j }{2}\dfrac{\partial}{\partial E^1_k} 
\left(1-\dfrac{ \sigma_j^2 
    + 2 F_j^2 - 3 w_j \sigma_j }
  { 2 F_j^2 - w_j \sigma_j - w_j^2 }
\right)
=  0.
\]
Therefore, the non-zero entries in the Jacobian matrix can be
$\pd{E^3_{iij}}{E^1_j}$ only. By rotational invariance of the model,
we need only study the Jacobian matrix in the $x$-direction,
$\pd{\bff_x}{\bE}$, which is
\begin{equation}\label{eq:Jacobi-of-ApproxM2-sx}
  \renewcommand\arraystretch{2.}
  \boldsymbol{\mathrm{J}}_x = 
  \left(
    \begin{array}{ccccccccc}
      0 & 1 & 0 & 0 & 0 & 0 & 0 & 0 & 0 \\
      0 & 0 & 0 & 0 & 1 & 0 & 0 & 0 & 0 \\
      0 & 0 & 0 & 0 & 0 & 1 & 0 & 0 & 0 \\
      0 & 0 & 0 & 0 & 0 & 0 & 1 & 0 & 0 \\
      0 & \pd{E^3_{111}}{E^1_1} & 0 & 
                                      0 & 0 & 0 & 0 & 0 & 0\\
      0 & 0 & \pd{E^3_{112}}{E^1_2} & 
                                      0 & 0 & 0 & 0 & 0 & 0\\
      0 & 0 & 0 & 
                  \pd{E^3_{113}}{E^1_3} & 0 & 0 & 0 & 0 & 0\\
      0 & \pd{E^3_{122}}{E^1_1} & 0 & 
                                      0 & 0 & 0 & 0 & 0 & 0\\
      0 & 0 & 0 & 0 & 0 & 0 & 0 & 0 & 0
    \end{array}
  \right).
\end{equation}
For the non-zero entries in $\boldsymbol{\mathrm{J}}_x$, we have the
following bounds:
\begin{lemma}\label{lem:E3_11k-bound}
  In the interior of the realizability domain $\cM$, if
  $\bE^1 = \boldsymbol{0}$, we have
  \begin{enumerate}
    \item $0 < \pd{E^3_{11k}}{E^1_k} < \dfrac12$, for $k = 2,3$;
    \item $0 < \pd{E^3_{111}}{E^1_1} \leq 1$ if and only if $\sigma_1 > 0$.
  \end{enumerate}
\end{lemma}
\begin{proof}
  For the first item, we only need to verify for $k=2$. By Lemma \ref{lem:E3},
  one has
  \[
  \begin{split}
    \pd{E^3_{112}}{E^1_2} = & \pd{\Vint{(\bOmega\cdot\bR_1)^2
        (\bOmega\cdot\bR_2) \bansatz}}{E^1_2} \\ = & \dfrac12\left(1 -
      \dfrac{\sigma_2^2 + 2 F_2^2 - 3 w_2 \sigma_2}{2 F_2^2 - w_2
        \sigma_2 -w_2^2}\right) \\
    = & \dfrac12 \dfrac{(w_2 - \sigma_2)^2}{w_2 (\sigma_2 + w_2)}.
  \end{split}
  \]
  By Lemma \ref{lem:fact-bound-sigma-w} we have $w_2 > 0$ and 
  $\sigma_2 + w_2 > 0$, thus, $\pd{E^3_{112}}{E^1_2} > 0$. 
  In addition, from the proof of Theorem \ref{thm:condition-nonnegative-ansatz},
  we have $\sigma_2 \leq w_2$, therefore $\pd{E^3_{112}}{E^1_2} < \dfrac12$.

  For the second item, we have by Lemma \ref{lem:E3},
 \[
 \begin{aligned}
   \pd{E^3_{111}}{E^1_1} = & \pd{\Vint{(\bOmega\cdot\bR_1)^3
       \bansatz}}{E^1_1} 
   = \dfrac{\sigma_1^2 + 2 F_1^2 - 3 w_1 \sigma_1}{2 F_1^2 - w_1
     \sigma_1 -w_1^2} \\
   = &\dfrac{\sigma_1 (3 w_1 - \sigma_1)}{w_1 (\sigma_1 + w_1)}
   = 1 - \dfrac{(w_1 - \sigma_1)^2}{w_1 (\sigma_1 + w_1)} \leq 1,
 \end{aligned}
 \]
 And $\pd{E^3_{111}}{E^1_1} > 0$ is equivalent to
 $\sigma_1 (3 w_1 - \sigma_1) > 0$. As $\sigma_1 \leq w_1$,
 we have $3 w_1 - \sigma_1 \geq 2 w_1 > 0$, implying that
 $\pd{E^3_{111}}{E^1_1} > 0$ is equivalent to $\sigma_1 > 0$.
\end{proof}
We now give the condition for the real diagonalizability of the Jacobian
matrix $\boldsymbol{\mathrm{J}}_x$ as follows:
\begin{theorem}\label{lem:Jx-invertible}
  The Jacobian matrix $\boldsymbol{\mathrm{J}}_x$ defined in
  \eqref{eq:Jacobi-of-ApproxM2-sx} is real diagonalizable
  if and only if $\sigma_1 > 0$.
\end{theorem}
\begin{proof}
The characteristic polynomial of $\boldsymbol{\mathrm{J}}_x$ is
\begin{equation}
  \left| \lambda \bI - \boldsymbol{\mathrm{J}}_x \right| = \lambda^3 \left(\lambda^2-\pd{E^3_{111}}{E^1_1}\right) \left(\lambda^2-
    \pd{E^3_{112}}{E^1_2}\right) \left(\lambda^2-\pd{E^3_{113}}{E^1_3}
  \right),
\end{equation}
thus zero is a multiple eigenvalue of $\boldsymbol{\mathrm{J}}_x$. The
corresponding eigenvectors are
\[
(0,0,0,0,0,0,0,0,1)^T,\quad
(0,0,0,0,0,0,0,1,0)^T,\quad
(1,0,0,0,0,0,0,0,0)^T.
\]
In the case that
\[
\pd{E^3_{111}}{E^1_1} \geq 0, \quad 
\pd{E^3_{112}}{E^1_2} \geq 0, \quad
\pd{E^3_{113}}{E^1_3} \geq 0,
\]
the corresponding eigenvalues of the matrix are
\[
\lambda_1^{\pm} = \pm \sqrt{\pd{E^3_{111}}{E^1_1}}, \quad
\lambda_2^{\pm} = \pm \sqrt{\pd{E^3_{112}}{E^1_2}}, \quad
\lambda_3^{\pm} = \pm \sqrt{\pd{E^3_{113}}{E^1_3}}, \quad
\]
and the corresponding eigenvectors are
\[
  \begin{array}{cccccc}
    \left(\begin{array}{c}
        1 \\
        \lambda_1^{\pm} \\
        0 \\
        0 \\
        |\lambda_1^{\pm}|^2 \\
        0 \\
        0 \\
        \pd{E^3_{122}}{E^1_1} \\
        0
    \end{array} \right), \qquad
    \left(\begin{array}{c}
        0 \\
        0 \\
        -1 \\
        0 \\
        0 \\
        \lambda_2^{\pm} \\
        0 \\
        0 \\
        0
    \end{array}\right), \qquad
    \left(\begin{array}{c}
        0 \\
        0 \\
        0 \\
        -1 \\
        0 \\
        0 \\
        \lambda_3^{\pm} \\
        0 \\
        0
    \end{array}\right).
  \end{array}
\]
It could be verified directly that if any of the eigenvalues
$\lambda_i^{\pm}$, $i = 1,2,3$, equals zero, the Jacobian matrix is
not real diagonalizable. If we have
\begin{equation}\label{eq:cond-jacobi}
\pd{E^3_{111}}{E^1_1} > 0, \quad 
\pd{E^3_{112}}{E^1_2} > 0, \quad
\pd{E^3_{113}}{E^1_3} > 0,
\end{equation}
by the linear independence of the eigenvectors, one concludes that the
Jacobian matrix is real diagonalizable. Then the proof is finished by Lemma
\ref{lem:E3_11k-bound}.
\end{proof}
As a direct consequence of Theorem \ref{lem:Jx-invertible}, 
the 3D $B_2$ model is hyperbolic at equilibrium. This can be proved by the following
arguments. Let $\bR_j$, $j = 1,2,3$ be the three eigenvectors of $\bbrE^2$. Denote
the $k$-th component of the vector $\bR_j$ to be $R_{kj}$. Define the
Jacobian matrix of the 3D $B_2$ model \eqref{eq:moment_model} along $\bR_j$, $j = 1,2,3$
to be $\sum\limits_{k = 1}^3 R_{kj} \bJ_k$. Theorem \ref{lem:Jx-invertible}
shows that for the cases $\bE^1 = \boldsymbol{0}$, condition \eqref{eq:cond-sigma-positive} 
is the necessary and sufficient condition for the Jacobian matrix along $\bR_j$,
$\forall j = 1,2,3$ to be real diagonalizable. The above result holds
because for any given $\bR_j$, $j = 1,2,3$, we could always rotate the
coordinate system, such that $\bR_j$ is aligned with the
$x$-axis. Theorem \ref{thm:condition-nonnegative-ansatz}
gives \eqref{eq:cond-sigma-positive-1}
as the necessary and sufficient condition for $\sigma_1 > 0$, and rotation of coordinates can permute
the indices in \eqref{eq:cond-sigma-positive-1}, which results in \eqref{eq:cond-sigma-positive}.
Notice that at equilibrium, $\bbrE^2$ is a scalar matrix, so any direction is an eigenvector of
$\bbrE^2$. Therefore, the 3D $B_2$ model is hyperbolic at equilibrium. 

For given moments, we could always choose a
coordinate system such that $\bbrE^2$ is a diagonal matrix. The
system is hyperbolic if and only if for an arbitrary
$\bn \not= \boldsymbol{0}$, we always have
$n_x \bJ_x + n_y \bJ_y + n_z \bJ_z$ to be real diagonalizable.
For $\bE^1 = \boldsymbol{0}$, we sample for all possible
$(\lambda_1, \lambda_2, \lambda_3)$ and all unit vectors $\bn$, to
check if the matrix is real diagonalizable.
There is a hyperbolicity region around equilibrium for $\bE^1 =
\boldsymbol{0}$ as in Figure \ref{fig:hyperbolic_region}. 
The hyperbolicity region is a smaller
region than that enclosed by \eqref{eq:cond-sigma-positive}. However, it does cover a
neighborhood of the equilibrium.
\begin{figure}  
  \centering
  \includegraphics[width=0.48\textwidth]{./images/hyperbolic_region_original.pdf}
  \caption{Region of hyperbolicity when $\bE^1 = \boldsymbol{0}$.
    $\left(\dfrac{\lambda_1}{E^0},\dfrac{\lambda_2}{E^0},
    \dfrac{\lambda_3}{E^0}\right)$ are taken as barycentric coordinates within the 
    triangle. The outer triangle is the realizability domain. The curves
    correspond to the outer boundary of the constraints \eqref{eq:cond-sigma-positive}.
    The 3D $B_2$ model is found to be hyperbolic within the dotted blue region. 
  }
  \label{fig:hyperbolic_region}
\end{figure}

Finally, we point out that although the 3D $B_2$ model is aimed at 
approximating the $M_2$ model, there is an interesting difference 
between them. This difference arises from the fact that
the ansatz is assumed to be the form $\bansatz$ in \eqref{eq:B2-ansatz}, 
and is independent of choice for the function $f(\mu; \gamma, \delta)$. 
When the given moments satisfy
\begin{equation} \label{eq:moments-with-axisymmetric-M2-ansatz}
  \exists i\not=j, \text{ such that } \lambda_i = \lambda_j,
  ~~\text{and}~~F_i = F_j = 0,
\end{equation}
the corresponding ansatz in the $M_2$ model is an axisymmetric
function. This includes the equilibrium distribution. Exactly at the
equilibrium, the 3D $B_2$ ansatz $\bansatz$ is isotropic, and, thus,
axisymmetric. However, even in neighbourhoods of the
equilibrium, moments corresponding to an axisymmetric ansatz in the
$M_2$ model would usually not reproduce an axisymmetric ansatz for the 3D
$B_2$ model. In other words, for arbitrary $\epsilon > 0$, there exist 
moments in the set
\[
\mathcal{A}_\epsilon = \left\{ (E^0,\bE^1,\bbrE^2) \in \cM ~\Big|~
  \bE^1 = \boldsymbol{0}, \sum_{i=1}^3 \left| \lambda_i -
    \dfrac{E^0}{3} \right|^2 < \epsilon, \text{ and
    \eqref{eq:moments-with-axisymmetric-M2-ansatz} is valid} \right\},
\]
for which the 3D $B_2$ ansatz $\bansatz$ is not axisymmetric; otherwise, 
the closure relation may lose the necessary regularities. More precisely, 
we claim:
\begin{theorem} 
  There are no functions
  $w_i(\lambda_1, \lambda_2, \lambda_3; F_1, F_2, F_3)$, $i=1,2,3$, in
  the 3D $B_2$ ansatz $\bansatz$ satisfying both items below:
  \begin{enumerate}
  \item $w_i$, $i=1,2,3$, are differentiable at the equilibrium state.
  \item The ansatz $\bansatz$ is axisymmetric for any moments in
    $\mathcal{A}_\epsilon$.
  \end{enumerate}
\end{theorem}
\begin{proof}
  We prove by contradiction. Suppose that \eqref{eq:B2-ansatz} is an
  axisymmetric distribution. Without losing generality we assume the
  corresponding moments satisfy $\lambda_2 = \lambda_3$, therefore the
  symmetric axis is aligned to $\bR_1$, and $F_2 = F_3 = 0$. To get
  axisymmetry in \eqref{eq:B2-ansatz}, the contributions from
  $w_2 f(\bOmega\cdot\bR_2;\gamma_2,\delta_2)$ and
  $w_3 f(\bOmega\cdot\bR_3;\gamma_3,\delta_3)$ have to be either zero
  or constant functions, hence $\sigma_2 = \dfrac{w_2}{3}$ and
  $\sigma_3 = \dfrac{w_3}{3}$, giving
  \[
  \sigma_1 + \sigma_2 + \sigma_3 = \lambda_1.
  \]
  Similar relations could be obtained when the symmetric 
  axis is aligned to $\bR_2$ or $\bR_3$.
  Consider the case when $\bE^1 = \boldsymbol{0}$. 
  Let 
  \[
  \sigma(\lambda_1,\lambda_2,\lambda_3) = 
  \sigma_1(\lambda_1,\lambda_2,\lambda_3;0,0,0)
  + \sigma_2(\lambda_1,\lambda_2,\lambda_3;0,0,0)
  + \sigma_3(\lambda_1,\lambda_2,\lambda_3;0,0,0).
  \]
  Based on the above arguments, we have
  \begin{equation}\label{eq:h-along-axis}
    \renewcommand\arraystretch{1.5}
    \sigma(\lambda_1,\lambda_2,\lambda_3) = \left \{
      \begin{array}{l}
        \lambda_1,\quad\text{if}\quad \lambda_2 = \lambda_3 
        = \frac12(E^0-\lambda_1),\\
        \lambda_2,\quad\text{if}\quad \lambda_1 = \lambda_3
        = \frac12(E^0-\lambda_2),\\
        \lambda_3,\quad\text{if}\quad \lambda_1 = \lambda_2
        = \frac12(E^0-\lambda_3).
      \end{array}\right.
  \end{equation}
  If all $w_i$, $i=1,2,3$, are differentiable, then all
  $\sigma_i$, $i=1,2,3$, are differentiable, so $\nabla \sigma$
  should be a continuous function for all realizabile
  moments. Let
  \[
  \bn_1 = (1, -\frac12, -\frac12),\quad
  \bn_2 = (-\frac12, 1, -\frac12),\quad
  \bn_3 = (-\frac12, -\frac12, 1),
  \]
  then $\nabla \sigma \cdot(\bn_1 + \bn_2 + \bn_3) = 0$. On the other
  hand, $\nabla \sigma \cdot\bn_1$ is equivalent to taking the
  derivative of $\sigma$ along
  $\lambda_2 = \lambda_3 = \frac12(E^0-\lambda_1)$, and we have
  similar relationships for $\bn_2$ and $\bn_3$.  So evaluating
  $\nabla \sigma \cdot\bn_j$ at $\lambda_j = \dfrac{E^0}{3}$,
  $j=1,2,3$ and $\bE^1 = \boldsymbol{0}$ according to
  \eqref{eq:h-along-axis} gives
  $\nabla \sigma \cdot(\bn_1 + \bn_2 + \bn_3) = 3$, leading to a
  contradiction. Therefore the two items can not be satisfied
  simultaneously.
\end{proof}
Notice that the proof of this lemma does not make use of the
specific form of the function $f$ in \eqref{eq:B2-ansatz}. In fact,
it can be seen from the proof that this inconsistency is due to the 
fact that the ansatz is a linear combination of three axisymmetric 
distributions. However, although the new model does not reproduce an 
axisymmetric ansatz for moments corresponding
to an axisymmetric ansatz in the $M_2$ model,
in such cases the closure of the new model retain the same structure 
as the $M_2$ closure. Without loss of generality consider the case when
$\lambda_2 = \lambda_3 = 0$ and $F_2 = F_3 = 0$. From
\eqref{eq:B2-closure}, we have
\begin{equation}\label{eq:axisymmetric-structure}
  \begin{split}
    & \Vint{ (\bOmega\cdot\bR_1)^2 (\bOmega\cdot\bR_2) \bansatz } =
    \Vint{ (\bOmega\cdot\bR_1)^2 (\bOmega\cdot\bR_3) \bansatz } =
    0,\\
    & \Vint{ (\bOmega\cdot\bR_1) (\bOmega\cdot\bR_2)^2 \bansatz } =
    \Vint{ (\bOmega\cdot\bR_3) (\bOmega\cdot\bR_2)^2 \bansatz }
    = \dfrac12\left(F_1-\Vint{ (\bOmega\cdot\bR_1)^3 }\right),
  \end{split}
\end{equation}
satisfying the same equalities as that given by an $M_2$ ansatz with
$\bR_1$ as the symmetric axis. 
 
Define
\[
  E_1 = \dfrac{\|\bE^1\|}{E^0},\quad
  E_2 = \dfrac{1}{(E^0)^3} (\bE^1)^T \bbrE^2 \bE^1, \quad
  E_3 = \dfrac{1}{(E^0)^4} \Vint{ \left(\bOmega\cdot\bE^1\right)^3 
  \hat{I}}.
\]
Then $E_i$, $i = 1,2,3$ would be the scaled first, second and
third-order moments for slab geometry cases. We compare the contour of
$E_3$ between the 3D $B_2$ model and the $M_2$ model for slab
geometry\footnote{The figure for the slab geometry was reproduced
  based on the data used to plot the corresponding figure in
  \cite{alldredge2016approximating}, and the computation was carried
  out by Dr. Alldredge using his own code during our collaboration
  therein.} in Figure \ref{fig:compare-contour}.  It is shown in
Figure \ref{fig:compare-contour} that the 3D $B_2$ model provides
realizable closure which is qualitatively similar to that of $M_2$
closure for most of realizable moments.
\begin{figure}
  \subfigure[The value of $E_3$  in slab geometry for 
  normalized realizable moments using
  the 3D $B_2$ closure.]{
    \label{fig:imageB2} 
    \includegraphics[width=0.48\textwidth]{./images/E3B2}} \hfill
  \subfigure[The value of $E_3$ for normalized realizable moments
  using the maximum entropy closure in slab geometry for the
  monochromatic case.]{
    \label{fig:imageM2mono} 
    \includegraphics[width=0.48\textwidth]{./images/E3M2}}%E3M2}} 
  \hfill
  \caption{Comparing the value of the closing moment $E_3$ for $M_2$
    for the monochromatic case and for the 3D $B_2$ model in slab
    geometry.}
  \label{fig:compare-contour}
\end{figure}

%%% Local Variables: 
%%% mode: latex
%%% TeX-master: "AppM2_MultiD.tex"
%%% End: 
