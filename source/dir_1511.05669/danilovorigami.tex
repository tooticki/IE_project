\documentclass[a4paper,12pt]{amsart}
%\usepackage{amsmath}
\usepackage{amssymb}
\usepackage{mathrsfs}
%\usepackage{theorem}
\usepackage{color}
\usepackage{epsfig}
\usepackage{graphicx}
\usepackage[all]{xy}
%\usepackage{showkeys}%label��\��
%   package{amstheorem} not used.

%\pagestyle{myheadings}
%\markright{{\footnotesize {\the\year /\the\month /\the\day}%header�̓��t��\��
%}}

\setlength{\textheight}{23cm}
\setlength{\textwidth}{16cm}
\setlength{\oddsidemargin}{0cm}
\setlength{\evensidemargin}{0cm}
\setlength{\topmargin}{0cm}


%\setlength\oddsidemargin{-0.9cm}
%\setlength\evensidemargin{-0.9cm}
%\setlength\topmargin{-1.3cm}
%\setlength\textheight{25.6cm}
%\setlength\textwidth{14.9cm}
%\renewcommand{\theenumi}{\roman{enumi}}
%\renewcommand{\labelenumi}{{\upshape(\theenumi)}}
%\renewcommand{\theenumii}{\alph{enumii}}
%\renewcommand{\labelenumii}{{\upshape(\theenumii)}}
% \newtheorem
%\theorembodyfont{\itshape}
\theoremstyle{plain}
\newtheorem{theorem}{Theorem}[section]
\newtheorem{prop}[theorem]{Proposition}
\newtheorem{corollary}[theorem]{Corollary}
\newtheorem{lemma}[theorem]{Lemma}
\newtheorem{claim}[theorem]{Claim}
%\theorembodyfont{\upshape}
\theoremstyle{definition}
\newtheorem{remark}[theorem]{Remark}
%\newtheorem{property}[theorem]{Property}
\newtheorem{conjecture}[theorem]{Conjecture}
\newtheorem{definition}[theorem]{Definition}
\newtheorem{example}[theorem]{Example}
\newtheorem{fact}[theorem]{Fact}
\newtheorem{assump}[theorem]{Assumption}


\newtheorem{assumption}[theorem]{Assumption}


\newcommand{\C}{\mathbb{C}}
\newcommand{\R}{\mathbb{R}}
\newcommand{\Z}{\mathbb{Z}}
\newcommand{\N}{\mathbb{N}}
\newcommand{\Q}{\mathbb{Q}}
\newcommand{\CP}{\mathbb{C}\mathrm{P}}
\newcommand{\RP}{\mathbb{R}\mathrm{P}}
\newcommand{\id}{\mathop{\mathrm{id}}\nolimits}
\newcommand{\diff}{\mathop{\mathit{diff}}\nolimits}
\newcommand{\Diff}{\mathop{\mathrm{Diff}}\nolimits}
\newcommand{\g}{\mathfrak g}
\newcommand{\Tt}{\mathfrak t}
\newcommand{\h}{\mathfrak h}
\newcommand{\SF}{\mathscr F}
\newcommand{\CO}{\mathcal O}
\newcommand{\vep}{\varepsilon}
\newcommand{\ind}{{\rm ind}}
\newcommand{\bD}{\mathbf D}
\newcommand{\U}{\mathcal U}


\renewcommand{\tilde}{\widetilde}
\renewcommand{\setminus}{\smallsetminus}
\newcommand{\nin}{/\kern-2.1ex\in}
\newcommand{\abs}[1]{\lvert#1\rvert}
\newcommand{\norm}[1]{\lVert#1\rVert}

%\newcommand{\system}{\color{red} {[compatible $+$ acyclic] system}}



\def\pr{\operatorname{pr}}
\def\<{\left\langle}
\def\>{\right\rangle}
\def\End{\operatorname{End}}
\def\Ker{\operatorname{Ker}}
\def\ind{\operatorname{ind}}
\def\Aut{\operatorname{Aut}}
\def\supp{\operatorname{supp}}

%\definecolor{brown}{cmyk}{0.00, 0.90, 1.00, 0.30}

%\def\bol#1{{\mbox{\boldmath $#1$}}}

%\renewcommand{\refname}%
%{\begin{center}\normalsize\mdseries\scshape%
%{References}\end{center}}


\numberwithin{equation}{section}

\title[A Danilov-type formula for toric origami manifolds]{A Danilov-type formula for toric origami manifolds \\ via localization of index}
\date{}
\author{Hajime Fujita}


%%%% DEDICATION %%%%


%%%%% FOOTNOTES%%%%%%%
\subjclass[2010]{Primary 53D50 ; Secondary 53C27, 58J20, 57S25} 
\keywords{origami manifold, symplectic toric manifold, equivariant index, localization.}
\thanks{$^1$Partly supported by Grant-in-Aid for Young Scientists (B) 26800045.}

%%%%% AFFILIATION %%%%%%%
\address{Department of Mathematical and Physical Sciences Japan Women's University 2-8-1 Mejirodai, Bunkyo-ku Tokyo, 112-8681 Japan}
\email{fujitah@fc.jwu.ac.jp}


\begin{document}

\maketitle

\begin{abstract}
We give a direct geometric proof of a Danilov-type formula for toric origami manifolds by using the localization of Riemann-Roch number. 
\end{abstract}



\tableofcontents
%%%%%%%%%%%%%%%%%%%%%%%%%%%%%%%%
\section{Introduction}
A {\it symplectic toric manifold} is a symplectic manifold on which a half dimensional torus $T$ acts in an effective Hamiltonian way. A famous theorem of Delzant \cite{Delzant} says that there is one-to-one correspondence between the set of (compact connected) symplectic toric manifolds and the set of simple polytopes called {\it Delzant polytopes} (see \cite{Guilleminbook}) via moment maps. Therefore, several properties of symplectic toric manifolds, such as the symplectic volume and the ring structure of the (equivariant) cohomology and so on, can be detected from the Delzant polytopes. In view of the geometric quantization of symplectic manifolds we are interested in the {\it Riemann-Roch numbers}. The Riemann-Roch number $RR(M,L)$ is an invariant of a compact symplectic manifold $(M,\omega)$ with a {\it pre-quantizing line bundle} $(L,\nabla)$, a pair consisting of a Hermitian line bundle $L$ and a Hermitian connection $\nabla$ whose curvature form is equal to $-\sqrt{-1}\omega$, which is defined as follows.  We fix an $\omega$-compatible almost complex structure and then it determines a spin$^c$-structure of $M$ and we have a spin$^c$-Dirac operator $D$ with coefficients in $L$. We define an integer $RR(M,L)$ as the analytic index of the spin$^c$-Dirac operator: 
$$
RR(M,L):=\ind(D). 
$$
If a compact Lie group $G$ acts on $M$ preserving all the data, $\omega$, $(L,\nabla)$ and $D$, then the index becomes a virtual representation of $G$, an element of the character ring $R(G)$. In this case the Riemann-Roch number is called the {\it Riemann-Roch character} or the {\it $G$-equivariant Riemann-Roch number} and is denoted by $RR_G(M,L)$.
Such a procedure is called {\it spin$^c$-quantization} (\cite{Silva-Karshon-Tolman}\cite{Fuchs}\cite{Paradanspinc}) nowadays and considered as a quantization of spin$^c$-manifolds. 
When $(X,\omega)$ is a symplectic toric manifold with the action a torus $T$ we can choose an almost complex structure so that it is integrable and invariant under the action of the torus $T$. Then $L$ has a structure of a holomorphic line bundle and the Riemann-Roch number is equal to the dimension of $H^0(X,L)$, the space of holomorphic sections of $L$. Moreover when we consider a lift of the torus action to the pre-quantizing line bundle, $RR_T(X,L)=H^0(X,L)$ becomes a representation of the torus $T$.  Classical theorem of Danilov \cite{Danilov} says that the representation $RR_T(X,L)$ can be described in terms of the integral points in the Delzant polytope. Precisely we have 
\begin{equation}\label{originaldanilov}
RR_T(X,L)=\bigoplus_{\xi\in\mu(M)\cap{\Tt}_{\Z}^*}\C_{(\xi)}, 
\end{equation}
where $\mu$ is the moment map,  $\Tt_{\Z}^*$ is the integral weight lattice in the dual of the Lie algebra of $T$ and $\C_{(\xi)}$ is the representation of the torus associated with the integral weight $\xi\in{\Tt}_{\Z}^*$. Though Danilov's original proof was based on an algebraic geometric setting, a proof in the symplectic geometric setting is also known. See \cite{Hamiltontoric} for example.   

A {\it folded symplectic manifold} introduced by Cannas da Silva, Guillemin and Woodward in \cite{SilvaGuilleminWoodward} is a pair consisting of an even-dimensional smooth manifold and a closed 2-form which may degenerate in a transverse way and it is called the {\it folded symplectic form}. When the degenerate locus (which becomes a hypersurface and called the {\it fold}) has a structure of a circle bundle whose vertical tangent bundle coincides with the degenerate direction of the folded symplectic form, the folded symplectic manifold is called an {\it origami manifold}. By definition a folded symplectic manifold (resp. origami manifold) is a generalization of a symplectic manifold, and several notions and studies in symplectic geometry are generalized to the folded symplectic (resp. origami) case, such as pre-quantizing line bundle,  Hamiltonian group action, moment map, convexity property and so on. It is known that a folded symplectic manifold is not orientable in general, and hence it does not admit an almost complex structure, however, if it is orientable, then it admits a stable almost complex structure as shown in \cite[Theorem~2]{SilvaGuilleminWoodward}. Since the stable almost complex structure determines a spin$^c$-structure, we can define its spin$^c$-quantization by the index of spin$^c$-Dirac operator. If the folded symplectic manifold is equipped with a Hamiltonian group action, then it becomes a virtual representation and is also called the Riemann-Roch character.  In particular the spin$^c$-quantization of a toric origami manifold is a virtual representation of the torus. 

In this paper we give a proof of the following generalization of Danilov's formula (\ref{originaldanilov}) for spin$^c$-quantization of toric origami manifolds by making use of the localization theorem of index developed in \cite{Fujita-Furuta-Yoshida1, Fujita-Furuta-Yoshida2}. 

\medskip

\noindent
{\bf Theorem} (Theorem~\ref{origamiDanilov}). \ 
{\it Let $(M,\omega)$ be an oriented toric origami manifold with the action of a torus $T$ and a $T$-equivariant pre-quantizing line bundle $(L,\nabla)$. Then we have 
$$
RR_T(M,L)=\bigoplus_{\xi\in\mu(M^+)\cap{\Tt}_{\Z}^*}\C_{(\xi)}-\bigoplus_{\xi\in\mu(M^-)\cap{\Tt}_{\Z}^*}\C_{(\xi)}
$$as elements in the character ring of $T$. }

\medskip

\noindent
Precise statement and notations are explained in the subsequent sections. 
The formula itself can be obtained as a consequence of the cobordism theorem \cite[Theorem~4.1]{Silva-Guillemin-Pires} and Danilov's formula (\ref{originaldanilov}) for symplectic toric manifolds. There is an another possible approach which uses the theory of {\it multi-fans} introduced by Hattori and Masuda \cite{Hattori-Masuda1}. Masuda and Park showed in \cite{Masuda-Park} that one can associate a multi-fan for each oriented toric origami manifold. In view of the theory of multi-fans the above formula can be considered as a special case of the equivariant index formula \cite[Theorem~11.1]{Hattori-Masuda1}, which is based on the fixed point formula. In contrast to these proofs, our proof is direct and geometric, which detects the contribution of each lattice point directly.  Once we construct a geometric structure which we call an {\it acyclic compatible system} on an open subset of the manifold, then the index of Dirac operator is localized at the complement of the open subset by the localization formula in \cite{Fujita-Furuta-Yoshida2}. In this paper we construct an acyclic compatible system on the complement of the inverse image of the lattice points and the fold for toric origami manifolds. It implies that the Riemann-Roch character is equal to the sum of contributions of the lattice points and the fold. We show that the contribution of the lattice point $\xi$ is equal to $\C_{(\xi)}$ with sign determined by the orientation and the contribution of the fold is zero. Our proof does not rely on neither the original Danilov's formula nor the fixed point formula. In fact, as a special case, our proof gives a new direct proof of Danilov's formula for symplectic toric manifolds. Note that there is an another generalization of the formula (\ref{originaldanilov}) by Karshon and Tolman \cite{KarshonTolman}. They gave a formula for toric manifolds with a torus invariant {\it presymplectic form}. Though their proof is based on the holomorphic structure of toric manifolds, our proof does not use such rigid structure and it is topological and flexible. 

This paper is organized as follows. In Section~\ref{Folded symplectic forms and  toric origami manifolds} we summarize several known facts about folded symplectic manifolds, origami manifolds and toric origami manifolds, which we use in this paper. The convexity theorem for toric origami manifolds (Theorem~\ref{origamiconvexity}) is essential for us. 
In Section~\ref{Stable almost complex structure and Clifford module bundle} we discuss stable almost complex structures on folded symplectic manifolds. We construct a $\Z/2$-graded Clifford module bundle in terms of the stable almost complex structure. 
In Section~\ref{Compatible fibration on toric origami manifolds} we construct a structure of {\it (good) compatible fibration} on toric origami manifolds,  which is a family of torus fibrations (foliations) with specific compatibility condition introduced in \cite{Fujita-Furuta-Yoshida2}.  The construction is based on an open covering of the convex polytope associated with the natural stratification of the polytope with respect to the dimension of the faces. Strictly speaking there exist {\it cracks} on which we can not extend the compatible fibration keeping the compatibility condition. Though the crack causes an extra contribution to the Riemann-Roch character, we show that it is equal to 0. In Section~\ref{Compatible system on toric origami manifolds} we construct a {\it compatible system} on the compatible fibration of the toric origami manifolds, which is a family of Dirac-type operators along the fibers of the compatible fibration with specific anti-commutativity. In \cite{Fujita-Furuta-Yoshida2} the authors had already constructed compatible system for Hamiltonian torus manifolds, and our construction for the complement of the fold is based on that. On the other hand a neighbourhood of the fold has a structure of a quotient of the product of the fold and the cylinder with the standard folded symplectic structure by a natural $S^1$-action. We use this structure to define the Dirac-type operator along fibers near the fold. To discuss the localization it is essential to investigate the {\it acyclicity} of the compatible system. The fundamental property of the moment map says that it is acyclic outside the lattice points and the fold. In Section~\ref{Localization formula and Danilov type formula} we explain the localization formula of the Riemann-Roch character by making use of the acyclic compatible system. 
In Section~\ref{Computation of the local contribution} we compute the local contribution of the crack, lattice points and the fold. We first consider the symplectic toric case, i.e., origami manifolds with the empty fold, and compute the local contribution. We use a decomposition of a neighbourhood of the fiber, the inverse image of the lattice point, into the product of the cotangent bundle of the fiber and the normal direction of the symplectic submanifold containing the fiber. We apply the product formula (\cite[Theorem~8.8]{Fujita-Furuta-Yoshida2}) to the neighbourhood of the fiber. 
%The computation for the crack resolves itself into the toric case by embedding the crack into a compact symplectic toric manifold. 
The vanishing of the contribution from the fold follows from the product structure of a neighbourhood of the fold. 
The last three sections are appendixes. In Appendix~\ref{Acyclic compatible systems and their local indices} we give a brief summary of the theory of local index following \cite{Fujita-Furuta-Yoshida2, Fujita-Furuta-Yoshida3} and \cite{Fujitacobinv}. In Appendix~\ref{A formula of local indeices of vector spaces} we show a useful formula of local indices of vector spaces, which will be essential in the proof of Lemma~\ref{locinddisc} and Lemma~\ref{locindcylinder}. In Appendix~\ref{A computation of local index of the folded cylinder} we give a direct computation of the local index of the folded cylinder and show that it is equal to $0$. We use this result to show that vanishing of the contribution from the fold. 

%\subsection{Notations}

%%%%%%%%%%%%%%%%%%%%%%%%%%%%%%%%
\section{Folded symplectic forms and  toric origami manifolds}
\label{Folded symplectic forms and  toric origami manifolds}
\subsection{Folded symplectic forms and  origami manifolds}
In this section we recall basic definitions and facts on folded symplectic manifolds and origami manifolds. 
Details can be found in \cite{Silva-Guillemin-Pires}, \cite{SilvaGuilleminWoodward}, \cite{Holm-Pires} and \cite{Masuda-Park}. 

A folded symplectic form $\omega$ on a smooth $2n$-dimensional manifold $M$ is a closed 2-form whose top power $\omega^n$ vanishes transversally on a submanifold $Z$ and whose restriction to $Z$ has maximal rank. In this case $Z$ is a hypersurface in $M$ and is called the {\it folding hypersurface} or {\it fold}. 
The pair $(M,\omega)$ is called a {\it folded symplectic manifold} and the $2$-form $\omega$ is called a {\it folded symplectic form}. 
Let $i_Z:Z\hookrightarrow M$ be the inclusion of $Z$ into $M$. The restriction $i_Z^*\omega$ determines a line field on $Z$, called the {\it null foliation}, whose fiber at $z\in Z$ is $\ker(i_Z^*\omega_z)$. 
%Let $E\to Z$ be the real rank 2 bundle over $Z$ whose fiber at $z\in Z$ is $\ker(\omega_z)$. Then we have $R=E\cap TZ$. 

Suppose that $(M,\omega)$ is an oriented folded symplectic manifold with non-empty fold $Z$. Then $M\setminus Z$ is not connected and has a decomposition $M\setminus Z=M_+\sqcup M_-$, where $M_+$ (resp. $M_-$) is the union of connected components such that $\omega^n|_{M_+}$ agrees (resp. disagrees) with the given orientation of $M$. 

\begin{definition}
A folded symplectic manifold $(M,\omega)$ is called  an {\it origami manifold} if the null foliation $\ker(i_Z^*\omega)$ is the vertical tangent bundle of a principal $S^1$-bundle structure 
$\pi:Z\to B$ over $Z$ with a compact base $B$. 
\end{definition}
Note that since $B$ is compact the total space $Z$ is also compact. As in the symplectic reduction procedure, there is the unique symplectic form $\omega_B$ on $B$ satisfying $\pi^*\omega_B=i_Z^*\omega$. 
An analogue of Darboux's theorem for folded symplectic forms says that near any point $p\in Z$ there exists a coordinate chart centered at $p$ where the folded symplectic form $\omega$ can be written as  
$$
x_1dx_1\wedge dy_1+dx_2\wedge dy_2+\cdots + dx_n\wedge dy_n. 
$$
In this local description, the fold $Z$ is given by the equation $x_1=0$ and the null foliation is the line field spanned by $\frac{\partial}{\partial y_1}$.  
This local description has a global variant. 

\begin{theorem}[Theorem~1 in \cite{SilvaGuilleminWoodward} ]\label{Morsermodel} 
Let $(M,\omega)$ be an oriented origami manifold with fold $Z\to B$. 
Fix a connection 1-form $\alpha$ of $Z\to B$. Then there exists a neighbourhood ${\mathcal U}$ of $Z$ and an orientation preserving diffeomorphism $\varphi:Z\times(-\varepsilon,\varepsilon)\to {\mathcal U}$ such that 
$$
\varphi\circ\iota_0=\iota_Z
$$and 
$$
\varphi^*\omega=p_Z^*\iota^*_Z\omega+d(t^2p_Z^*\alpha), 
$$where $\iota_0: Z \to Z\times(-\varepsilon,\varepsilon)$ is the inclusion 
$z\mapsto (z,0)$ and $p_Z:Z\times(-\varepsilon,\varepsilon)\to Z$ is the natural projection. 
\end{theorem}

\begin{example}\label{exsphere2}
For a positive integer $n$ let $S^{2n}$ be the unit sphere in $\R^{2n}\oplus\R=\C^{n}\oplus \R$ with coordinates $x_1, y_1, \cdots, x_n, y_n, h$. Let $\omega$ be the restriction to $S^{2n}$ of the 2-form  $dx_1\wedge dy_1+\cdots+dx_n\wedge dy_n$ on $\R^{2n}\oplus \R$. Then $\omega$ is a folded symplectic form on $S^{2n}$ with the fold $S^{2n-1}$, the equator sphere given by $h=0$. The Hopf fibration $S^1\hookrightarrow S^{2n-1}\to \CP^{n-1}$ gives a structure of origami manifold on $(S^{2n}, \omega)$. 
\end{example}

%%%%%%%%%%%%%%%%%%%%%%%%%%%%%%%%%%%%%%%%%%%%%%%%%%%%%%%%%%%%%%%%%%%%%%%%%
\subsection{Hamiltonian torus actions and toric origami manifolds}

The action of a compact Lie group $G$ on an origami manifold $(M,\omega)$ is called {\it Hamiltonian} if it admits a moment map $\mu$, that is, a map $\mu:M\to {\mathfrak g}^*={\rm Lie}(G)^*$ satisfying the conditions : 
\\
$\bullet$ $\mu$ is equivariant with respect to the given action of $G$ on $M$ and the coadjoint action of $G$ on ${\mathfrak g}^*$. 
\\
\noindent 
$\bullet$ for any $v\in {\mathfrak g}$ we have $d\langle \mu, v \rangle = \iota({v^M})\omega$, where  $\langle \cdot, \cdot \rangle$ is the pairing between ${\mathfrak g}^*$ and ${\mathfrak g}$ and $\iota({v^M})\omega$ is the contraction of $\omega$ by  the induced fundamental vector field $v^M$. 

\begin{definition}
A Hamiltonian torus origami manifold $(M,\omega, T,\mu)$ (or $M$ for short) is a connected origami manifold $(M,\omega)$ equipped with an effective Hamiltonian action of a torus $T$ with a choice of a corresponding moment map $\mu$. If the dimension of the torus $T$ is half of that of $M$, then we call $(M,\omega, T,\mu)$ a {\it toric origami manifold}. 
\end{definition}

If the fold $Z$ is empty, a Hamiltonian torus origami manifold is a Hamiltonian  torus manifold in the usual sense. 
%\begin{definition}
%Let $\Delta_i$ $(i=1,2)$ be a polytope in $\R^n$. 
%If $F_i$ is a face of a polytope $\delta_i$ ($i=1,2$), we say that $\Delta_1$ near $F_1$ {\it agrees} with $\Delta_2$ near $F_2$ when $F_1=F_2$ and there is an open subset ${\mathcal U}$ of $\R^n$ containing $F_1$ such that ${\mathcal U}\cap\Delta_1={\mathcal U}\cap \Delta_2$. 
%\end{definition}
The following is an origami analogue of the famous convexity theorem for Hamiltonian torus manifolds. 

\begin{theorem}[Theorem~3.2 in \cite{Silva-Guillemin-Pires}]\label{origamiconvexity}
Let $(M,\omega,T,\mu)$ be  a connected compact origami manifold with null fibration $\pi:Z\to B$ and a Hamiltonian torus action of a torus $T$ with moment map $\mu$. Then : \\
{\rm (a)} The image $\mu(M)$ is the union of a finite number of convex polytopes $\Delta_1,\cdots, \Delta_N$ in the dual of the Lie algebra ${\mathfrak t}^*$, each of which is the image of the moment map restricted to the closure of a connected component of $M\setminus Z$. 
\\
{\rm (b)} Over each connected component $Z'$ of $Z$, the null fibration is given by a subgroup of $T$ if and only if $\mu(Z')$ is a facet of each of the one or two polytopes corresponding to the neighbourhood(s) of $M\setminus Z$, and when those are two polytopes $\Delta_1$ and $\Delta_2$ there exists an open subset $\tilde \Delta_{Z'}$ containing $\mu(Z')$ such that ${\tilde \Delta_{Z'}}\cap\Delta_1={\tilde \Delta_{Z'}}\cap \Delta_2$. 

We call such images $\mu(M)$ origami polytopes. 
\end{theorem}

\begin{example}\label{exsphere3}
Consider the origami manifold $(S^{2n}, \omega)$ given in Example~\ref{exsphere2}. Let $T:=(S^1)^n$ be the $n$-dimensional torus. Then the action of $T$ on $S^{2n}$ given by 
$$ 
(t_1,\ldots, t_n)\cdot (z_1, \ldots, z_1, h):=(t_1z_1,\ldots, t_nz_n, h)
$$for $(t_1,\ldots, t_n)\in T$ and $(z_1, \ldots, z_1,h)\in S^{2n}\subset \C^n\oplus \R$ is Hamiltonian (in fact, toric) action with the moment map $\mu:S^{2n}\to \R^n$, 
$$
\mu(z_1,\ldots,z_n,h):=\frac{1}{2}(|z_1|^2, \ldots, |z_n|^2). 
$$The image of $\mu$ is the union of two copies of the $n$-simplex, $\xi_1,\cdots,\xi_n\geq 0$, $\xi_1+\cdots+\xi_n\leq 1/2$,  and the image of fold $S^{2n-1}$ is the \lq\lq hypotenuse\rq\rq, $\xi_1+\cdots +\xi_n=1/2$. See Figure~\ref{pics4} for the case of $n=2$. 

\begin{figure}[h]
%WinTpicVersion4.28b
{\unitlength 0.1in
\begin{picture}( 45.6800, 11.3600)( 16.0000,-27.3600)
% POLYGON 2 0 3 0 Black White
% 4 1600 1600 1600 2720 2880 2720 1600 1600
% 
{\color[named]{Black}{%
\special{pn 8}%
\special{pa 1600 1600}%
\special{pa 1600 2720}%
\special{pa 2880 2720}%
\special{pa 1600 1600}%
\special{pa 1600 2720}%
\special{fp}%
}}%
% STR 2 0 3 0 Black White
% 4 2832 2160 2832 2240 2 0 0 0
% $=$
\put(28.3200,-22.4000){\makebox(0,0)[lb]{$=$}}%
% LINE 2 0 3 0 Black White
% 48 2232 2152 1664 2720 2280 2200 1760 2720 2336 2240 1856 2720 2384 2288 1952 2720 2432 2336 2048 2720 2488 2376 2144 2720 2536 2424 2240 2720 2592 2464 2336 2720 2640 2512 2432 2720 2688 2560 2528 2720 2744 2600 2624 2720 2792 2648 2720 2720 2848 2688 2816 2720 2176 2112 1600 2688 2128 2064 1600 2592 2080 2016 1600 2496 2024 1976 1600 2400 1976 1928 1600 2304 1920 1888 1600 2208 1872 1840 1600 2112 1824 1792 1600 2016 1768 1752 1600 1920 1720 1704 1600 1824 1664 1664 1600 1728
% 
{\color[named]{Black}{%
\special{pn 8}%
\special{pa 2232 2152}%
\special{pa 1664 2720}%
\special{fp}%
\special{pa 2280 2200}%
\special{pa 1760 2720}%
\special{fp}%
\special{pa 2336 2240}%
\special{pa 1856 2720}%
\special{fp}%
\special{pa 2384 2288}%
\special{pa 1952 2720}%
\special{fp}%
\special{pa 2432 2336}%
\special{pa 2048 2720}%
\special{fp}%
\special{pa 2488 2376}%
\special{pa 2144 2720}%
\special{fp}%
\special{pa 2536 2424}%
\special{pa 2240 2720}%
\special{fp}%
\special{pa 2592 2464}%
\special{pa 2336 2720}%
\special{fp}%
\special{pa 2640 2512}%
\special{pa 2432 2720}%
\special{fp}%
\special{pa 2688 2560}%
\special{pa 2528 2720}%
\special{fp}%
\special{pa 2744 2600}%
\special{pa 2624 2720}%
\special{fp}%
\special{pa 2792 2648}%
\special{pa 2720 2720}%
\special{fp}%
\special{pa 2848 2688}%
\special{pa 2816 2720}%
\special{fp}%
\special{pa 2176 2112}%
\special{pa 1600 2688}%
\special{fp}%
\special{pa 2128 2064}%
\special{pa 1600 2592}%
\special{fp}%
\special{pa 2080 2016}%
\special{pa 1600 2496}%
\special{fp}%
\special{pa 2024 1976}%
\special{pa 1600 2400}%
\special{fp}%
\special{pa 1976 1928}%
\special{pa 1600 2304}%
\special{fp}%
\special{pa 1920 1888}%
\special{pa 1600 2208}%
\special{fp}%
\special{pa 1872 1840}%
\special{pa 1600 2112}%
\special{fp}%
\special{pa 1824 1792}%
\special{pa 1600 2016}%
\special{fp}%
\special{pa 1768 1752}%
\special{pa 1600 1920}%
\special{fp}%
\special{pa 1720 1704}%
\special{pa 1600 1824}%
\special{fp}%
\special{pa 1664 1664}%
\special{pa 1600 1728}%
\special{fp}%
}}%
% POLYGON 2 0 3 0 Black White
% 4 3264 1608 3264 2728 4544 2728 3264 1608
% 
{\color[named]{Black}{%
\special{pn 8}%
\special{pa 3264 1608}%
\special{pa 3264 2728}%
\special{pa 4544 2728}%
\special{pa 3264 1608}%
\special{pa 3264 2728}%
\special{fp}%
}}%
% LINE 2 0 3 0 Black White
% 48 3896 2160 3328 2728 3944 2208 3424 2728 4000 2248 3520 2728 4048 2296 3616 2728 4096 2344 3712 2728 4152 2384 3808 2728 4200 2432 3904 2728 4256 2472 4000 2728 4304 2520 4096 2728 4352 2568 4192 2728 4408 2608 4288 2728 4456 2656 4384 2728 4512 2696 4480 2728 3840 2120 3264 2696 3792 2072 3264 2600 3744 2024 3264 2504 3688 1984 3264 2408 3640 1936 3264 2312 3584 1896 3264 2216 3536 1848 3264 2120 3488 1800 3264 2024 3432 1760 3264 1928 3384 1712 3264 1832 3328 1672 3264 1736
% 
{\color[named]{Black}{%
\special{pn 8}%
\special{pa 3896 2160}%
\special{pa 3328 2728}%
\special{fp}%
\special{pa 3944 2208}%
\special{pa 3424 2728}%
\special{fp}%
\special{pa 4000 2248}%
\special{pa 3520 2728}%
\special{fp}%
\special{pa 4048 2296}%
\special{pa 3616 2728}%
\special{fp}%
\special{pa 4096 2344}%
\special{pa 3712 2728}%
\special{fp}%
\special{pa 4152 2384}%
\special{pa 3808 2728}%
\special{fp}%
\special{pa 4200 2432}%
\special{pa 3904 2728}%
\special{fp}%
\special{pa 4256 2472}%
\special{pa 4000 2728}%
\special{fp}%
\special{pa 4304 2520}%
\special{pa 4096 2728}%
\special{fp}%
\special{pa 4352 2568}%
\special{pa 4192 2728}%
\special{fp}%
\special{pa 4408 2608}%
\special{pa 4288 2728}%
\special{fp}%
\special{pa 4456 2656}%
\special{pa 4384 2728}%
\special{fp}%
\special{pa 4512 2696}%
\special{pa 4480 2728}%
\special{fp}%
\special{pa 3840 2120}%
\special{pa 3264 2696}%
\special{fp}%
\special{pa 3792 2072}%
\special{pa 3264 2600}%
\special{fp}%
\special{pa 3744 2024}%
\special{pa 3264 2504}%
\special{fp}%
\special{pa 3688 1984}%
\special{pa 3264 2408}%
\special{fp}%
\special{pa 3640 1936}%
\special{pa 3264 2312}%
\special{fp}%
\special{pa 3584 1896}%
\special{pa 3264 2216}%
\special{fp}%
\special{pa 3536 1848}%
\special{pa 3264 2120}%
\special{fp}%
\special{pa 3488 1800}%
\special{pa 3264 2024}%
\special{fp}%
\special{pa 3432 1760}%
\special{pa 3264 1928}%
\special{fp}%
\special{pa 3384 1712}%
\special{pa 3264 1832}%
\special{fp}%
\special{pa 3328 1672}%
\special{pa 3264 1736}%
\special{fp}%
}}%
% POLYGON 2 0 3 0 Black White
% 4 4888 1616 4888 2736 6168 2736 4888 1616
% 
{\color[named]{Black}{%
\special{pn 8}%
\special{pa 4888 1616}%
\special{pa 4888 2736}%
\special{pa 6168 2736}%
\special{pa 4888 1616}%
\special{pa 4888 2736}%
\special{fp}%
}}%
% LINE 2 0 3 0 Black White
% 48 5520 2168 4952 2736 5568 2216 5048 2736 5624 2256 5144 2736 5672 2304 5240 2736 5720 2352 5336 2736 5776 2392 5432 2736 5824 2440 5528 2736 5880 2480 5624 2736 5928 2528 5720 2736 5976 2576 5816 2736 6032 2616 5912 2736 6080 2664 6008 2736 6136 2704 6104 2736 5464 2128 4888 2704 5416 2080 4888 2608 5368 2032 4888 2512 5312 1992 4888 2416 5264 1944 4888 2320 5208 1904 4888 2224 5160 1856 4888 2128 5112 1808 4888 2032 5056 1768 4888 1936 5008 1720 4888 1840 4952 1680 4888 1744
% 
{\color[named]{Black}{%
\special{pn 8}%
\special{pa 5520 2168}%
\special{pa 4952 2736}%
\special{fp}%
\special{pa 5568 2216}%
\special{pa 5048 2736}%
\special{fp}%
\special{pa 5624 2256}%
\special{pa 5144 2736}%
\special{fp}%
\special{pa 5672 2304}%
\special{pa 5240 2736}%
\special{fp}%
\special{pa 5720 2352}%
\special{pa 5336 2736}%
\special{fp}%
\special{pa 5776 2392}%
\special{pa 5432 2736}%
\special{fp}%
\special{pa 5824 2440}%
\special{pa 5528 2736}%
\special{fp}%
\special{pa 5880 2480}%
\special{pa 5624 2736}%
\special{fp}%
\special{pa 5928 2528}%
\special{pa 5720 2736}%
\special{fp}%
\special{pa 5976 2576}%
\special{pa 5816 2736}%
\special{fp}%
\special{pa 6032 2616}%
\special{pa 5912 2736}%
\special{fp}%
\special{pa 6080 2664}%
\special{pa 6008 2736}%
\special{fp}%
\special{pa 6136 2704}%
\special{pa 6104 2736}%
\special{fp}%
\special{pa 5464 2128}%
\special{pa 4888 2704}%
\special{fp}%
\special{pa 5416 2080}%
\special{pa 4888 2608}%
\special{fp}%
\special{pa 5368 2032}%
\special{pa 4888 2512}%
\special{fp}%
\special{pa 5312 1992}%
\special{pa 4888 2416}%
\special{fp}%
\special{pa 5264 1944}%
\special{pa 4888 2320}%
\special{fp}%
\special{pa 5208 1904}%
\special{pa 4888 2224}%
\special{fp}%
\special{pa 5160 1856}%
\special{pa 4888 2128}%
\special{fp}%
\special{pa 5112 1808}%
\special{pa 4888 2032}%
\special{fp}%
\special{pa 5056 1768}%
\special{pa 4888 1936}%
\special{fp}%
\special{pa 5008 1720}%
\special{pa 4888 1840}%
\special{fp}%
\special{pa 4952 1680}%
\special{pa 4888 1744}%
\special{fp}%
}}%
% STR 2 0 3 0 Black White
% 4 4392 2152 4392 2232 2 0 0 0
% $\bigcup$
\put(43.9200,-22.3200){\makebox(0,0)[lb]{$\bigcup$}}%
% LINE 0 0 3 0 Black White
% 2 1600 1608 2880 2720
% 
{\color[named]{Black}{%
\special{pn 20}%
\special{pa 1600 1608}%
\special{pa 2880 2720}%
\special{fp}%
}}%
% STR 2 0 3 0 Black White
% 4 2190 1760 2190 1860 2 0 0 0
% $\Delta$
\put(21.9000,-18.6000){\makebox(0,0)[lb]{$\Delta$}}%
% STR 2 0 3 0 Black White
% 4 3900 1770 3900 1870 2 0 0 0
% $\Delta_1$
\put(39.0000,-18.7000){\makebox(0,0)[lb]{$\Delta_1$}}%
% STR 2 0 3 0 Black White
% 4 5550 1800 5550 1900 2 0 0 0
% $\Delta_2$
\put(55.5000,-19.0000){\makebox(0,0)[lb]{$\Delta_2$}}%
\end{picture}}%

\caption{An origami polytope for $S^4$}\label{pics4}
\end{figure}

\end{example}


%%%%%%%%%%%%%%%%%%%%%%%%%%%%%%%%%%%%%%%%%%%%%%%%%%%%%%%%%%%%%%%%%%%%%%
\section{Stable almost complex structure and Clifford module bundle}
\label{Stable almost complex structure and Clifford module bundle}
Let $(M,\omega)$ be a $2n$-dimensional oriented folded symplectic manifold with fold $Z$ and ${\mathcal U}$ an open neighbourhood of $Z$ as in Theorem~\ref{Morsermodel}. 
Let $M_+$ (resp. $M_-$) be the union of connected components of $M\setminus Z$ such that $\omega^n|_{M_+}$ agrees (resp. disagrees) with the given orientation of $M$. 
In \cite{SilvaGuilleminWoodward}, it was shown that $M$ has a stable almost complex structure. More precisely the following holds. 
\begin{theorem}[Theorem~2 in \cite{SilvaGuilleminWoodward}]\label{stablealmostcomplex}
There exixts an almost complex structure $\tilde J$ on the real $(2n+2)$-dimensional vector bundle $TM\oplus\R^2$ , and a $\C$-linear isomorphism 
$$
(TM\oplus\R^2)|_{M\setminus\U}\cong T(M\setminus \U)\oplus \C. 
$$Moreover, $TM\oplus\R^2$ has a symplectic structure $\tilde\omega$ which is canonical up to homotopy, and the homotopy class of $\tilde J$ is unique provided $\tilde J$ is compatible with the natural symplectic structure on $TM\oplus \R^2$. 
\end{theorem}

\begin{remark}\label{stablecomplexstrrem}
One can see in the proof of \cite[Theorem~2]{SilvaGuilleminWoodward} that the above $\tilde J$ has the following properties. 
\begin{enumerate}
\item Let $J$ be an almost complex structure on $M\setminus Z$ which is compatible with $\omega|_{M\setminus Z}$. Then one can construct $\tilde J$ so that the following equality holds. 
\begin{equation}\label{stablestd}
\begin{cases}
 \ \tilde J|_{M_+\setminus {\mathcal U}}=J|_{M_+\setminus {\mathcal U}}\oplus (\sqrt{-1}) \\ 
 \ \tilde J|_{M_-\setminus {\mathcal U}}=J|_{M_-\setminus {\mathcal U}}\oplus (-\sqrt{-1}). 
\end{cases}
\end{equation}
\item By using a connection of the principal $S^1$-bundle $Z\to B$ we have the splitting of the tangent bundle $TZ\cong \pi^*TB\oplus T_{\pi}Z$, where $T_{\pi}Z$ is the tangent bundle along the fiber, which is a real line bundle over $Z$. Since $T\U$ is oriented, and hence, $TZ$ is also oriented, the fact that $B$ is a symplectic manifold implies that $T_{\pi}Z$ is an orientable.  In particular, $T_{\pi}Z$ is trivial real line bundle. Under these identifications the almost complex structure $\tilde J|_{\U}$ in Theorem~\ref{stablealmostcomplex}  on $T\U\oplus \R^2\cong \pi^*TB\oplus T_{\pi}Z\oplus\R\oplus\R^2$ can be taken as the direct sum of almost complex structures on the symplectic vector bundle $\pi^*TB$ and the trivial bundle $T_{\pi}Z\oplus\R\oplus\R^2$ of real rank 4. 
\item If a compact Lie group $G$ acts on $(M,\omega)$, then we can take $\tilde J$ to be $G$-invariant. In fact we will use such an invariant $\tilde J$ in the subsequent sections. 
\end{enumerate}
\end{remark}

By using $\tilde J$ and $\tilde \omega$, we have a Riemannian metric on $TM\oplus \R^2$, and $TM$ is equipped with the metric as a subbundle of $TM\oplus\R^2$. 
Moreover the stable almost complex structure induces a spin$^c$-structure on $M$.  Now we construct a Clifford module bundle over $TM$ in terms of this stable almost complex structure. 



We first explain the construction for the vector space case. 
Let $E$ be an even dimensional Euclidean vector space. 
Suppose that a complex structure $J_{\tilde E}$ on $\tilde E:=E\oplus\R^e$ which preserves the metric on $\tilde E$ is given for a non-negative (even) integer $e$. 
By using $J_{\tilde E}$ we have a $\Z/2$-graded $Cl(\tilde E)=Cl(E)\otimes Cl(\R^e)$-module $W_{\tilde E}:=\wedge^{\bullet}_{\C}\tilde E$, the exterior product algebra of the Hermitian vector space $\tilde E$. 
The Clifford action of $Cl(\tilde E)$ is defined by the wedge product and the interior product. We define $W_E$ as the set of all linear maps from an irreducible representation $W_e$ of the Clifford algebra $Cl_e:=Cl(\R^e)$ to $W_{\tilde E}$ which commute with the Clifford action of $Cl_e$, 
$$
W_E:={\rm Hom}_{Cl_e}(W_e,W_{\tilde E}), 
$$where $Cl_e$ acts on $W_{\tilde E}$ by using the inclusion $Cl_e\hookrightarrow Cl(E\oplus\R^e)$. Note that $W_E$ is equipped with the Clifford action of $Cl(E)$ by 
$$
\alpha\cdot\phi : v\mapsto \alpha \phi(v)
$$for $\alpha\in Cl(E)$ and $v\in W_e$ using the inclusion $Cl(E)\hookrightarrow Cl(E\oplus\R^e)$. 

\begin{lemma}
$W_E$ is an irreducible $\Z/2$-graded $Cl(E)$-module. 
\end{lemma}
\begin{proof}
Suppose that $E$ is equipped with an almost complex structure $J_E$ and $J_{\tilde E}$ is the direct sum of $J_E$ and the standard complex structure $\sqrt{-1}$ on $\R^e=\C^{e/2}$ (for a specific order of the basis of $\R^e$). 
In this case, one can see that $\wedge^{\bullet}_{\C}{\tilde E}=\wedge^{\bullet}_{\C}{E}\otimes\wedge_{\C}^{\bullet}\R^{e}$ and 
$$
W_E={\rm Hom}_{Cl_e}(W_e,\wedge^{\bullet}_{\C}{\tilde E})=\wedge_{\C}^{\bullet}E\otimes{\rm Hom}_{Cl_e}(W_e, \wedge_{\C}^{\bullet}\R^{e})=\wedge_{\C}^{\bullet}E.
$$
%by fixing a basis of $E$ 
It implies that $W_E$ is an irreducible $Cl(E)$-module. 
Since any complex structure on $\tilde E$ is homotopic to the direct sum $J_E\oplus {\sqrt{-1}}$ and the irreducible representation of $Cl(E)$ is unique, we complete the proof. 
\end{proof}
By applying the above construction for an almost complex structure on $TM\oplus\R^2$ we have the $\Z/2$-graded $Cl(TM)$-module bundle 
\begin{equation}\label{Clifford}
W:={\rm Hom}_{Cl_2}(W_2, \wedge_{\C}^{\bullet}(TM\oplus\R^2)) 
\end{equation}
over $M$. Note that we have $W|_{M_{\pm}\setminus{\mathcal U}}\cong\wedge_{\C}^{\bullet}T(M_{\pm}\setminus{\mathcal U})$ by (\ref{stablestd}), which is the standard $Cl(T(M_{\pm}\setminus {\mathcal U}))$-module bundle of $M_{\pm}\setminus {\mathcal U}$. 
For any Hermitian line bundle $L$ we have an another $\Z/2$-graded $Cl(E)$-module bundle $W_L:=W\otimes L$. 



\begin{definition}\label{origamiRR}
For a compact oriented origami manifold $(M,\omega)$ without boundary and a Hermitian line bundle $L$ over $M$ the {\it Riemann-Roch number} $RR(M,L)$ is defined as the index of spin$^c$-Dirac operator which acts on the smooth sections of the Clifford module bundle $W_L$: 
$$
RR(M,L):=\ind(W_L). 
$$
\end{definition}
\begin{remark}
%Strictly speaking the index $\ind(W_L)$ is defined as the analytic index of a Dirac-type operator $D$ which acts on the smooth sections of $W_L$. 
Since any two Dirac-type operators can be joined in the space of Dirac-type operators the index $RR(M,L)=\ind(W_L)$ does not depend on the choice of the Dirac-type operators by the homotopy invariance of the analytic index. 
\end{remark}


%%%%%%%%%%%%%%%%%%%%%%%%%%%%%%%%%%%%%%%%%%%%%%%%%%%%%%%%%%%%%%%%%%%%%%
\section{Compatible fibration on toric origami manifolds}
\label{Compatible fibration on toric origami manifolds}
In this section we construct a structure of {\it good compatible fibration} on toric origami manifolds. 
The notion of good compatible fibration is a family of torus fibrations (or more generally foliations) over an open covering of the manifold with some compatibility condition and is introduced in \cite{Fujita-Furuta-Yoshida2}.  
See also Definition~\ref{goodcompatifib}. 

\begin{assump}\label{assump}
In this section we consider a toric origami manifold $(M,\omega, T,\mu)$ satisfying the following assumptions.
\begin{itemize}
\item $M$ is connected, oriented, and compact without boundary. 
\item $(M,\omega, T,\mu)$ satisfies the condition (b) in Theorem~\ref{origamiconvexity}. Namely, the null foliation is given by a subgroup of $T$. 
\end{itemize}
\end{assump}
Suppose that $\dim M=2n$.
Let $\mu(M)=\bigcup_i\Delta_i$ be the union of convex polytopes associated with the moment map $\mu:M\to \Tt^*$. 
For each $i$ let $\Delta_{i}=\Delta_Z\cup\bigcup_{j=0}^n\bigcup_{k=1}^{m_j}\Delta_{i,k}^{(j)}$ be the stratification of $\Delta_i$, where we put\footnote{Strictly speaking we consider each connected component of $Z$. } $\Delta_Z:=\mu(Z)$ and $\{\Delta_{i,1}^{(j)},\cdots, \Delta_{i,m_j}^{(j)}\}$ is the set of all $j$-dimensional faces of $\Delta_i$ for each $j\in\{0,\cdots,n\}$. 
%We take a small open neighbourhood $\tilde \Delta_Z$ of $\Delta_Z$ as in Theorem~\ref{origamiconvexity}~(b). 
We take and fix a neighbourhood $\U:=Z\times (-\vep,\vep)$ of $Z$ in $M$ as in Theorem~\ref{Morsermodel} for some small $\vep>0$, and we may assume that $\overline{\U}=Z\times[-\vep,\vep,]$ and an open neighbourhood $\tilde\Delta_Z$ in Theorem~\ref{origamiconvexity}(b) has the form $\tilde \Delta_Z=\mu(\U)$. 

The construction of the good compatible fibration is divided into two parts, 
fibrations near the fold and fibrations outside the fold. 

\subsection{Torus actions near the fold}
\label{Torus actions near the fold}
We set $\U':=Z\times (-\frac{\vep}{2}, \frac{\vep}{2})$. We take $\vep>0$ small enough so that the $S^1$-aciton on $Z$ can be extended to a free $S^1$-action on $\U'$. By using this $S^1$-action we have an $S^1$-bundle structure on $\U'$ with the base space $B\times(-\frac{\vep}{2}, \frac{\vep}{2})$.


\subsection{Torus actions outside the fold}
\label{Torus actions outside the fold}
We construct a family of torus actions on $M\setminus\overline{\U}$. 
We put $\Delta_{i,Z}':=\Delta_i\setminus\mu(\overline{\U})$ and 
we first construct an open covering  
$$\Delta_{i,Z}'=\left(\bigcup_{j=0}^{n}\bigcup_{k=1}^{m_j}\tilde\Delta_{i,k}^{(j)}\right)$$ 
by the following procedure. See also Figure~\ref{Open covering outside the imge of the fold}. 

\begin{figure}[h]
%WinTpicVersion4.26
\unitlength 0.1in
\begin{picture}( 31.3100, 23.1000)( 31.8900,-41.0000)
% POLYLINE 1 0 3 0 Black White
% 4 3200 3990 6200 3990 6200 1790 6200 1790
% 
{\color[named]{Black}{%
\special{pn 13}%
\special{pa 3200 3990}%
\special{pa 6200 3990}%
\special{pa 6200 1790}%
\special{fp}%
}}%
% BOX 2 0 3 0 Black White
% 2 6150 3950 6207 4017
% 
{\color[named]{Black}{%
\special{pn 8}%
\special{pa 6150 3950}%
\special{pa 6208 3950}%
\special{pa 6208 4018}%
\special{pa 6150 4018}%
\special{pa 6150 3950}%
\special{pa 6208 3950}%
\special{fp}%
}}%
% LINE 0 0 3 1 Black White
% 12 6210 3970 6167 4020 6207 3954 6153 4017 6190 3954 6153 3997 6173 3954 6153 3977 6210 3990 6184 4020 6210 4010 6201 4020
% 
{\color[named]{Black}{%
\special{pn 20}%
\special{pa 6210 3970}%
\special{pa 6168 4020}%
\special{fp}%
\special{pa 6208 3954}%
\special{pa 6154 4018}%
\special{fp}%
\special{pa 6190 3954}%
\special{pa 6154 3998}%
\special{fp}%
\special{pa 6174 3954}%
\special{pa 6154 3978}%
\special{fp}%
\special{pa 6210 3990}%
\special{pa 6184 4020}%
\special{fp}%
\special{pa 6210 4010}%
\special{pa 6202 4020}%
\special{fp}%
}}%
% STR 2 0 3 0 Black White
% 4 6290 4080 6290 4180 2 0 0 0
% $\Delta_{i,k}^{(j)}$
\put(62.9000,-41.8000){\makebox(0,0)[lb]{$\Delta_{i,k}^{(j)}$}}%
% STR 2 0 3 0 Black White
% 4 6320 2650 6320 2750 2 0 0 0
% $\Delta_{i,k'}^{(j+1)}$
\put(63.2000,-27.5000){\makebox(0,0)[lb]{$\Delta_{i,k'}^{(j+1)}$}}%
% STR 2 0 3 0 Black White
% 4 4540 4160 4540 4260 2 0 0 0
% $\Delta_{i,k''}^{(j+1)}$
\put(45.4000,-42.6000){\makebox(0,0)[lb]{$\Delta_{i,k''}^{(j+1)}$}}%
% LINE 3 0 3 0 Black White
% 60 4510 1790 3200 3100 4690 1790 3200 3280 4870 1790 3200 3460 5050 1790 3200 3640 5230 1790 3200 3820 5410 1790 3220 3980 5590 1790 3400 3980 5770 1790 3580 3980 5950 1790 3760 3980 6130 1790 3940 3980 6190 1910 4120 3980 6190 2090 4300 3980 6190 2270 4480 3980 6190 2450 4660 3980 6190 2630 4840 3980 6190 2810 5020 3980 6190 2990 5200 3980 6190 3170 5380 3980 6190 3350 5560 3980 6190 3530 5740 3980 6190 3710 5920 3980 6190 3890 6130 3950 6130 3950 6100 3980 4330 1790 3200 2920 4150 1790 3200 2740 3970 1790 3200 2560 3790 1790 3200 2380 3610 1790 3200 2200 3430 1790 3200 2020 3250 1790 3200 1840
% 
{\color[named]{Black}{%
\special{pn 4}%
\special{pa 4510 1790}%
\special{pa 3200 3100}%
\special{fp}%
\special{pa 4690 1790}%
\special{pa 3200 3280}%
\special{fp}%
\special{pa 4870 1790}%
\special{pa 3200 3460}%
\special{fp}%
\special{pa 5050 1790}%
\special{pa 3200 3640}%
\special{fp}%
\special{pa 5230 1790}%
\special{pa 3200 3820}%
\special{fp}%
\special{pa 5410 1790}%
\special{pa 3220 3980}%
\special{fp}%
\special{pa 5590 1790}%
\special{pa 3400 3980}%
\special{fp}%
\special{pa 5770 1790}%
\special{pa 3580 3980}%
\special{fp}%
\special{pa 5950 1790}%
\special{pa 3760 3980}%
\special{fp}%
\special{pa 6130 1790}%
\special{pa 3940 3980}%
\special{fp}%
\special{pa 6190 1910}%
\special{pa 4120 3980}%
\special{fp}%
\special{pa 6190 2090}%
\special{pa 4300 3980}%
\special{fp}%
\special{pa 6190 2270}%
\special{pa 4480 3980}%
\special{fp}%
\special{pa 6190 2450}%
\special{pa 4660 3980}%
\special{fp}%
\special{pa 6190 2630}%
\special{pa 4840 3980}%
\special{fp}%
\special{pa 6190 2810}%
\special{pa 5020 3980}%
\special{fp}%
\special{pa 6190 2990}%
\special{pa 5200 3980}%
\special{fp}%
\special{pa 6190 3170}%
\special{pa 5380 3980}%
\special{fp}%
\special{pa 6190 3350}%
\special{pa 5560 3980}%
\special{fp}%
\special{pa 6190 3530}%
\special{pa 5740 3980}%
\special{fp}%
\special{pa 6190 3710}%
\special{pa 5920 3980}%
\special{fp}%
\special{pa 6190 3890}%
\special{pa 6130 3950}%
\special{fp}%
\special{pa 6130 3950}%
\special{pa 6100 3980}%
\special{fp}%
\special{pa 4330 1790}%
\special{pa 3200 2920}%
\special{fp}%
\special{pa 4150 1790}%
\special{pa 3200 2740}%
\special{fp}%
\special{pa 3970 1790}%
\special{pa 3200 2560}%
\special{fp}%
\special{pa 3790 1790}%
\special{pa 3200 2380}%
\special{fp}%
\special{pa 3610 1790}%
\special{pa 3200 2200}%
\special{fp}%
\special{pa 3430 1790}%
\special{pa 3200 2020}%
\special{fp}%
\special{pa 3250 1790}%
\special{pa 3200 1840}%
\special{fp}%
}}%
% STR 2 0 3 0 Black White
% 4 3910 2210 3910 2310 2 0 0 0
% $\Delta_{i,1}^{(n)}$
\put(39.1000,-23.1000){\makebox(0,0)[lb]{$\Delta_{i,1}^{(n)}$}}%
% ELLIPSE 1 2 3 0 Black White
% 4 6200 3980 7230 4965 6200 2995 4615 4005
% 
{\color[named]{Black}{%
\special{pn 13}%
\special{pa 5170 3996}%
\special{pa 5170 3984}%
\special{fp}%
\special{pa 5170 3958}%
\special{pa 5172 3946}%
\special{fp}%
\special{pa 5172 3920}%
\special{pa 5174 3908}%
\special{fp}%
\special{pa 5176 3882}%
\special{pa 5178 3870}%
\special{fp}%
\special{pa 5180 3844}%
\special{pa 5182 3832}%
\special{fp}%
\special{pa 5186 3806}%
\special{pa 5188 3796}%
\special{fp}%
\special{pa 5194 3770}%
\special{pa 5196 3760}%
\special{fp}%
\special{pa 5204 3734}%
\special{pa 5206 3724}%
\special{fp}%
\special{pa 5214 3700}%
\special{pa 5216 3688}%
\special{fp}%
\special{pa 5226 3664}%
\special{pa 5230 3652}%
\special{fp}%
\special{pa 5238 3628}%
\special{pa 5242 3618}%
\special{fp}%
\special{pa 5252 3594}%
\special{pa 5258 3584}%
\special{fp}%
\special{pa 5268 3560}%
\special{pa 5274 3550}%
\special{fp}%
\special{pa 5286 3528}%
\special{pa 5290 3518}%
\special{fp}%
\special{pa 5304 3494}%
\special{pa 5310 3484}%
\special{fp}%
\special{pa 5324 3462}%
\special{pa 5330 3454}%
\special{fp}%
\special{pa 5346 3432}%
\special{pa 5352 3422}%
\special{fp}%
\special{pa 5368 3400}%
\special{pa 5376 3392}%
\special{fp}%
\special{pa 5392 3370}%
\special{pa 5400 3362}%
\special{fp}%
\special{pa 5418 3340}%
\special{pa 5424 3332}%
\special{fp}%
\special{pa 5444 3312}%
\special{pa 5452 3304}%
\special{fp}%
\special{pa 5470 3286}%
\special{pa 5478 3278}%
\special{fp}%
\special{pa 5498 3260}%
\special{pa 5508 3252}%
\special{fp}%
\special{pa 5528 3234}%
\special{pa 5536 3228}%
\special{fp}%
\special{pa 5558 3210}%
\special{pa 5568 3204}%
\special{fp}%
\special{pa 5588 3188}%
\special{pa 5598 3182}%
\special{fp}%
\special{pa 5620 3166}%
\special{pa 5630 3160}%
\special{fp}%
\special{pa 5652 3146}%
\special{pa 5662 3140}%
\special{fp}%
\special{pa 5684 3128}%
\special{pa 5694 3122}%
\special{fp}%
\special{pa 5718 3110}%
\special{pa 5728 3104}%
\special{fp}%
\special{pa 5752 3094}%
\special{pa 5762 3090}%
\special{fp}%
\special{pa 5786 3078}%
\special{pa 5796 3074}%
\special{fp}%
\special{pa 5820 3064}%
\special{pa 5830 3062}%
\special{fp}%
\special{pa 5854 3052}%
\special{pa 5866 3048}%
\special{fp}%
\special{pa 5890 3042}%
\special{pa 5902 3038}%
\special{fp}%
\special{pa 5926 3032}%
\special{pa 5938 3028}%
\special{fp}%
\special{pa 5962 3022}%
\special{pa 5974 3020}%
\special{fp}%
\special{pa 6000 3014}%
\special{pa 6010 3012}%
\special{fp}%
\special{pa 6036 3008}%
\special{pa 6048 3006}%
\special{fp}%
\special{pa 6074 3002}%
\special{pa 6086 3002}%
\special{fp}%
\special{pa 6112 3000}%
\special{pa 6124 2998}%
\special{fp}%
\special{pa 6150 2996}%
\special{pa 6162 2996}%
\special{fp}%
\special{pa 6188 2996}%
\special{pa 6200 2996}%
\special{fp}%
}}%
% LINE 2 0 3 1 Black White
% 22 5710 3115 5710 3965 5800 3065 5800 3965 5890 3045 5890 3965 5980 3015 5980 3965 6070 2995 6070 3965 6160 2995 6160 3965 5620 3165 5620 3965 5530 3235 5530 3965 5440 3325 5440 3965 5350 3425 5350 3965 5260 3585 5260 3965
% 
{\color[named]{Black}{%
\special{pn 8}%
\special{pa 5710 3116}%
\special{pa 5710 3966}%
\special{fp}%
\special{pa 5800 3066}%
\special{pa 5800 3966}%
\special{fp}%
\special{pa 5890 3046}%
\special{pa 5890 3966}%
\special{fp}%
\special{pa 5980 3016}%
\special{pa 5980 3966}%
\special{fp}%
\special{pa 6070 2996}%
\special{pa 6070 3966}%
\special{fp}%
\special{pa 6160 2996}%
\special{pa 6160 3966}%
\special{fp}%
\special{pa 5620 3166}%
\special{pa 5620 3966}%
\special{fp}%
\special{pa 5530 3236}%
\special{pa 5530 3966}%
\special{fp}%
\special{pa 5440 3326}%
\special{pa 5440 3966}%
\special{fp}%
\special{pa 5350 3426}%
\special{pa 5350 3966}%
\special{fp}%
\special{pa 5260 3586}%
\special{pa 5260 3966}%
\special{fp}%
}}%
% STR 2 0 3 2 Black White
% 4 5490 3585 5490 3685 2 0 0 0
% $\tilde\Delta_{i,k}^{(j)}$
\put(54.9000,-36.8500){\makebox(0,0)[lb]{$\tilde\Delta_{i,k}^{(j)}$}}%
% ELLIPSE 1 2 3 0 Black White
% 4 3230 4000 5560 3350 5880 4000 3080 1600
% 
{\color[named]{Black}{%
\special{pn 13}%
\special{pa 3190 3350}%
\special{pa 3202 3350}%
\special{fp}%
\special{pa 3228 3350}%
\special{pa 3240 3350}%
\special{fp}%
\special{pa 3268 3350}%
\special{pa 3280 3350}%
\special{fp}%
\special{pa 3306 3350}%
\special{pa 3318 3350}%
\special{fp}%
\special{pa 3346 3352}%
\special{pa 3358 3352}%
\special{fp}%
\special{pa 3384 3352}%
\special{pa 3396 3352}%
\special{fp}%
\special{pa 3424 3352}%
\special{pa 3436 3352}%
\special{fp}%
\special{pa 3462 3354}%
\special{pa 3474 3354}%
\special{fp}%
\special{pa 3502 3354}%
\special{pa 3514 3356}%
\special{fp}%
\special{pa 3540 3356}%
\special{pa 3552 3356}%
\special{fp}%
\special{pa 3580 3358}%
\special{pa 3592 3358}%
\special{fp}%
\special{pa 3618 3360}%
\special{pa 3630 3360}%
\special{fp}%
\special{pa 3658 3362}%
\special{pa 3668 3362}%
\special{fp}%
\special{pa 3696 3364}%
\special{pa 3708 3364}%
\special{fp}%
\special{pa 3734 3366}%
\special{pa 3746 3366}%
\special{fp}%
\special{pa 3774 3368}%
\special{pa 3786 3370}%
\special{fp}%
\special{pa 3812 3372}%
\special{pa 3824 3372}%
\special{fp}%
\special{pa 3852 3374}%
\special{pa 3862 3374}%
\special{fp}%
\special{pa 3890 3378}%
\special{pa 3902 3378}%
\special{fp}%
\special{pa 3928 3380}%
\special{pa 3940 3382}%
\special{fp}%
\special{pa 3968 3384}%
\special{pa 3980 3386}%
\special{fp}%
\special{pa 4006 3388}%
\special{pa 4018 3388}%
\special{fp}%
\special{pa 4044 3392}%
\special{pa 4056 3392}%
\special{fp}%
\special{pa 4084 3396}%
\special{pa 4096 3396}%
\special{fp}%
\special{pa 4122 3400}%
\special{pa 4134 3402}%
\special{fp}%
\special{pa 4160 3404}%
\special{pa 4172 3406}%
\special{fp}%
\special{pa 4200 3410}%
\special{pa 4210 3412}%
\special{fp}%
\special{pa 4238 3414}%
\special{pa 4250 3416}%
\special{fp}%
\special{pa 4276 3420}%
\special{pa 4288 3422}%
\special{fp}%
\special{pa 4314 3426}%
\special{pa 4326 3428}%
\special{fp}%
\special{pa 4354 3430}%
\special{pa 4366 3432}%
\special{fp}%
\special{pa 4392 3438}%
\special{pa 4404 3438}%
\special{fp}%
\special{pa 4430 3444}%
\special{pa 4442 3446}%
\special{fp}%
\special{pa 4468 3450}%
\special{pa 4480 3452}%
\special{fp}%
\special{pa 4508 3456}%
\special{pa 4518 3458}%
\special{fp}%
\special{pa 4546 3464}%
\special{pa 4556 3466}%
\special{fp}%
\special{pa 4584 3472}%
\special{pa 4596 3474}%
\special{fp}%
\special{pa 4622 3480}%
\special{pa 4634 3482}%
\special{fp}%
\special{pa 4660 3488}%
\special{pa 4672 3490}%
\special{fp}%
\special{pa 4698 3496}%
\special{pa 4710 3498}%
\special{fp}%
\special{pa 4736 3504}%
\special{pa 4748 3508}%
\special{fp}%
\special{pa 4774 3514}%
\special{pa 4786 3516}%
\special{fp}%
\special{pa 4812 3522}%
\special{pa 4824 3526}%
\special{fp}%
\special{pa 4850 3532}%
\special{pa 4860 3536}%
\special{fp}%
\special{pa 4886 3544}%
\special{pa 4898 3546}%
\special{fp}%
\special{pa 4924 3554}%
\special{pa 4936 3558}%
\special{fp}%
\special{pa 4962 3566}%
\special{pa 4972 3570}%
\special{fp}%
\special{pa 4998 3578}%
\special{pa 5010 3582}%
\special{fp}%
\special{pa 5036 3590}%
\special{pa 5046 3594}%
\special{fp}%
\special{pa 5072 3602}%
\special{pa 5084 3606}%
\special{fp}%
\special{pa 5108 3616}%
\special{pa 5120 3620}%
\special{fp}%
\special{pa 5144 3630}%
\special{pa 5156 3634}%
\special{fp}%
\special{pa 5180 3644}%
\special{pa 5192 3650}%
\special{fp}%
\special{pa 5216 3660}%
\special{pa 5226 3664}%
\special{fp}%
\special{pa 5250 3676}%
\special{pa 5262 3682}%
\special{fp}%
\special{pa 5286 3694}%
\special{pa 5296 3700}%
\special{fp}%
\special{pa 5320 3712}%
\special{pa 5330 3718}%
\special{fp}%
\special{pa 5352 3732}%
\special{pa 5362 3738}%
\special{fp}%
\special{pa 5384 3752}%
\special{pa 5394 3758}%
\special{fp}%
\special{pa 5416 3776}%
\special{pa 5424 3782}%
\special{fp}%
\special{pa 5446 3798}%
\special{pa 5454 3806}%
\special{fp}%
\special{pa 5474 3826}%
\special{pa 5482 3834}%
\special{fp}%
\special{pa 5500 3854}%
\special{pa 5508 3862}%
\special{fp}%
\special{pa 5522 3884}%
\special{pa 5528 3894}%
\special{fp}%
\special{pa 5542 3918}%
\special{pa 5546 3928}%
\special{fp}%
\special{pa 5554 3952}%
\special{pa 5556 3962}%
\special{fp}%
\special{pa 5560 3988}%
\special{pa 5560 4000}%
\special{fp}%
}}%
% STR 2 0 3 0 Black White
% 4 3680 3680 3680 3780 2 0 0 0
% $\tilde\Delta_{i,k''}^{(j+1)}$
\put(36.8000,-37.8000){\makebox(0,0)[lb]{$\tilde\Delta_{i,k''}^{(j+1)}$}}%
% LINE 3 0 3 0 Black White
% 40 3870 4000 3240 3370 3990 4000 3350 3360 4110 4000 3480 3370 4230 4000 3600 3370 4350 4000 3730 3380 4470 4000 3850 3380 4590 4000 3990 3400 4710 4000 4120 3410 4830 4000 4260 3430 4950 4000 4400 3450 5070 4000 4540 3470 5190 4000 4690 3500 5310 4000 4850 3540 5430 4000 5020 3590 5550 4000 5220 3670 3750 4000 3230 3480 3630 4000 3230 3600 3510 4000 3230 3720 3390 4000 3230 3840 3270 4000 3230 3960
% 
{\color[named]{Black}{%
\special{pn 4}%
\special{pa 3870 4000}%
\special{pa 3240 3370}%
\special{fp}%
\special{pa 3990 4000}%
\special{pa 3350 3360}%
\special{fp}%
\special{pa 4110 4000}%
\special{pa 3480 3370}%
\special{fp}%
\special{pa 4230 4000}%
\special{pa 3600 3370}%
\special{fp}%
\special{pa 4350 4000}%
\special{pa 3730 3380}%
\special{fp}%
\special{pa 4470 4000}%
\special{pa 3850 3380}%
\special{fp}%
\special{pa 4590 4000}%
\special{pa 3990 3400}%
\special{fp}%
\special{pa 4710 4000}%
\special{pa 4120 3410}%
\special{fp}%
\special{pa 4830 4000}%
\special{pa 4260 3430}%
\special{fp}%
\special{pa 4950 4000}%
\special{pa 4400 3450}%
\special{fp}%
\special{pa 5070 4000}%
\special{pa 4540 3470}%
\special{fp}%
\special{pa 5190 4000}%
\special{pa 4690 3500}%
\special{fp}%
\special{pa 5310 4000}%
\special{pa 4850 3540}%
\special{fp}%
\special{pa 5430 4000}%
\special{pa 5020 3590}%
\special{fp}%
\special{pa 5550 4000}%
\special{pa 5220 3670}%
\special{fp}%
\special{pa 3750 4000}%
\special{pa 3230 3480}%
\special{fp}%
\special{pa 3630 4000}%
\special{pa 3230 3600}%
\special{fp}%
\special{pa 3510 4000}%
\special{pa 3230 3720}%
\special{fp}%
\special{pa 3390 4000}%
\special{pa 3230 3840}%
\special{fp}%
\special{pa 3270 4000}%
\special{pa 3230 3960}%
\special{fp}%
}}%
% SPLINE 1 2 3 0 Black White
% 42 5670 1790 5670 1855 5670 1920 5671 1984 5674 2048 5678 2112 5681 2176 5686 2239 5690 2300 5696 2362 5703 2422 5711 2483 5719 2540 5728 2598 5737 2654 5749 2709 5760 2762 5772 2815 5785 2866 5798 2915 5811 2961 5826 3006 5840 3050 5856 3091 5872 3130 5888 3168 5906 3202 5924 3237 5942 3268 5959 3296 5978 3323 5997 3346 6016 3368 6036 3387 6057 3404 6075 3418 6095 3430 6116 3438 6136 3445 6156 3449 6177 3450 6177 3450
% 
{\color[named]{Black}{%
\special{pn 13}%
\special{pn 13}%
\special{pa 5670 1790}%
\special{pa 5670 1802}%
\special{fp}%
\special{pa 5670 1830}%
\special{pa 5670 1840}%
\special{fp}%
\special{pa 5670 1868}%
\special{pa 5670 1880}%
\special{fp}%
\special{pa 5670 1906}%
\special{pa 5670 1918}%
\special{fp}%
\special{pa 5670 1946}%
\special{pa 5670 1956}%
\special{fp}%
\special{pa 5672 1984}%
\special{pa 5672 1996}%
\special{fp}%
\special{pa 5674 2022}%
\special{pa 5674 2034}%
\special{fp}%
\special{pa 5676 2062}%
\special{pa 5676 2072}%
\special{fp}%
\special{pa 5678 2100}%
\special{pa 5678 2112}%
\special{fp}%
\special{pa 5680 2138}%
\special{pa 5680 2150}%
\special{fp}%
\special{pa 5682 2176}%
\special{pa 5684 2188}%
\special{fp}%
\special{pa 5686 2216}%
\special{pa 5686 2228}%
\special{fp}%
\special{pa 5688 2254}%
\special{pa 5688 2266}%
\special{fp}%
\special{pa 5690 2292}%
\special{pa 5690 2304}%
\special{fp}%
\special{pa 5694 2332}%
\special{pa 5696 2342}%
\special{fp}%
\special{pa 5696 2370}%
\special{pa 5698 2382}%
\special{fp}%
\special{pa 5702 2408}%
\special{pa 5704 2420}%
\special{fp}%
\special{pa 5706 2446}%
\special{pa 5708 2458}%
\special{fp}%
\special{pa 5712 2484}%
\special{pa 5714 2496}%
\special{fp}%
\special{pa 5718 2524}%
\special{pa 5720 2534}%
\special{fp}%
\special{pa 5724 2562}%
\special{pa 5724 2574}%
\special{fp}%
\special{pa 5728 2600}%
\special{pa 5730 2612}%
\special{fp}%
\special{pa 5734 2638}%
\special{pa 5736 2650}%
\special{fp}%
\special{pa 5742 2676}%
\special{pa 5746 2688}%
\special{fp}%
\special{pa 5750 2714}%
\special{pa 5752 2726}%
\special{fp}%
\special{pa 5758 2752}%
\special{pa 5762 2762}%
\special{fp}%
\special{pa 5766 2790}%
\special{pa 5770 2800}%
\special{fp}%
\special{pa 5776 2826}%
\special{pa 5778 2838}%
\special{fp}%
\special{pa 5784 2864}%
\special{pa 5788 2876}%
\special{fp}%
\special{pa 5794 2902}%
\special{pa 5798 2914}%
\special{fp}%
\special{pa 5804 2940}%
\special{pa 5808 2950}%
\special{fp}%
\special{pa 5816 2976}%
\special{pa 5820 2986}%
\special{fp}%
\special{pa 5830 3012}%
\special{pa 5832 3024}%
\special{fp}%
\special{pa 5840 3050}%
\special{pa 5844 3060}%
\special{fp}%
\special{pa 5854 3086}%
\special{pa 5858 3096}%
\special{fp}%
\special{pa 5870 3122}%
\special{pa 5874 3132}%
\special{fp}%
\special{pa 5884 3156}%
\special{pa 5888 3168}%
\special{fp}%
\special{pa 5902 3192}%
\special{pa 5908 3202}%
\special{fp}%
\special{pa 5920 3226}%
\special{pa 5924 3236}%
\special{fp}%
\special{pa 5938 3260}%
\special{pa 5944 3270}%
\special{fp}%
\special{pa 5958 3294}%
\special{pa 5964 3302}%
\special{fp}%
\special{pa 5980 3324}%
\special{pa 5986 3334}%
\special{fp}%
\special{pa 6004 3354}%
\special{pa 6012 3362}%
\special{fp}%
\special{pa 6030 3382}%
\special{pa 6040 3390}%
\special{fp}%
\special{pa 6060 3408}%
\special{pa 6070 3414}%
\special{fp}%
\special{pa 6094 3428}%
\special{pa 6104 3434}%
\special{fp}%
\special{pa 6128 3442}%
\special{pa 6140 3446}%
\special{fp}%
\special{pa 6166 3450}%
\special{pa 6178 3450}%
\special{fp}%
}}%
% LINE 3 0 3 1 Black White
% 22 6174 2785 5699 2356 6188 2654 5686 2202 6188 2511 5673 2047 6188 2368 5673 1904 6188 2226 5725 1809 6188 2083 5897 1821 6188 1940 6056 1821 6174 2928 5712 2511 6174 3070 5752 2690 6174 3213 5791 2868 6174 3356 5858 3070
% 
{\color[named]{Black}{%
\special{pn 4}%
\special{pa 6174 2786}%
\special{pa 5700 2356}%
\special{fp}%
\special{pa 6188 2654}%
\special{pa 5686 2202}%
\special{fp}%
\special{pa 6188 2512}%
\special{pa 5674 2048}%
\special{fp}%
\special{pa 6188 2368}%
\special{pa 5674 1904}%
\special{fp}%
\special{pa 6188 2226}%
\special{pa 5726 1810}%
\special{fp}%
\special{pa 6188 2084}%
\special{pa 5898 1822}%
\special{fp}%
\special{pa 6188 1940}%
\special{pa 6056 1822}%
\special{fp}%
\special{pa 6174 2928}%
\special{pa 5712 2512}%
\special{fp}%
\special{pa 6174 3070}%
\special{pa 5752 2690}%
\special{fp}%
\special{pa 6174 3214}%
\special{pa 5792 2868}%
\special{fp}%
\special{pa 6174 3356}%
\special{pa 5858 3070}%
\special{fp}%
}}%
% STR 2 0 3 0 Black White
% 4 5750 1961 5750 2080 2 0 0 0
% $\tilde\Delta_{i,k'}^{(j+1)}$
\put(57.5000,-20.8000){\makebox(0,0)[lb]{$\tilde\Delta_{i,k'}^{(j+1)}$}}%
\end{picture}%

\caption{Open covering outside the image of the fold}
\label{Open covering outside the imge of the fold}
\end{figure}

\begin{itemize}
\item[(0)] For each $k\in\{1,\cdots, m_0\}$ take a small open neighbourhood $\tilde\Delta_{i,k}^{(0)}$ of $\Delta_{i,k}^{(0)}$ in $\Delta_{i,Z}'$ so that 
%$\tilde\Delta_{i,k}^{(0)}\cap \tilde F=\emptyset$ and 
$\tilde\Delta_{i,k}^{(0)}\cap\tilde\Delta_{i,k'}^{(0)}=\emptyset$ if $k\neq  k'$. 
\item[(1)] For each $k\in\{1,\cdots, m_1\}$ take a small open neighbourhood $\tilde\Delta_{i,k}^{(1)}$ of 
$$
\left(\Delta_{i,Z}'\setminus\bigcup_{k_0=1}^{m_0}\tilde\Delta_{i,k_0}^{(0)}\right)\cap \Delta_{i,k}^{(1)}
$$ in $\Delta_{i,Z}'$ so that $\tilde\Delta_{i,k}^{(1)}\cap\tilde\Delta_{i,k'}^{(1)}=\emptyset$ if $k\neq  k'$. 
\item[(2)] For each $k\in\{1,\cdots, m_2\}$ take a small open neighbourhood $\tilde\Delta_{i,k}^{(2)}$ of 
$$
\left(\Delta_{i,Z}'\setminus\bigcup_{j=0}^{1}\bigcup_{k_j=1}^{m_j}\tilde\Delta_{i,k_j}^{(j)}\right)\cap \Delta_{i,k}^{(2)}
$$ in $\Delta_{i,Z}'$ so that $\tilde\Delta_{i,k}^{(2)}\cap\tilde\Delta_{i,k'}^{(2)}=\emptyset$ if $k\neq  k'$. 
\\ 
$\vdots$
\\
\item[(n-1)] For each $k\in\{1,\cdots, m_{n-1}\}$ take a small open neighbourhood $\tilde\Delta_{i,k}^{(n-1)}$ of 
$$
\left(\Delta_{i,Z}'\setminus\bigcup_{j=0}^{n-2}\bigcup_{k_j=0}^{m_j}\tilde\Delta_{i,k_j}^{(j)}\right)\cap \Delta_{i,k}^{(n-1)}
$$ in $\Delta_{i,Z}'$ so that $\tilde\Delta_{i,k}^{(n-1)}\cap\tilde\Delta_{i,k'}^{(n-1)}=\emptyset$ if $k\neq  k'$. 
\item[(n)] We set $\tilde\Delta_{i,1}^{(n)}:={\rm int}(\Delta_i)={\rm int}(\Delta_i^{(n)})$. 
\end{itemize}

\medskip

For a toric manifold  $M\setminus Z$ it is well-known that for each $i,j,k$ there exists a subtorus $T_{i,k}^{(j)}$ of $T$ such that $\dim T_{i,k}^{(j)}=n-j$ and for any $x\in \mu^{-1}({\rm int}(\Delta_{i,k}^{(j)})) \setminus Z$ the stabilizer subgroup at $x$ is equal to $T_{i,k}^{(j)}$. 
Particularly we have $T_{i,1}^{(n)}=\{e\}$.  
%acts on $\mu^{-1}({\rm int}(\Delta_{i}^{(n)}))$ freely. 
We take and fix a rational metric of the Lie algebra ${\mathfrak t}$ so that for each subspace ${\mathfrak h}$ in ${\mathfrak t}$ spanned by rational vectors one can associate the orthogonal complement subgroup ${\rm exp}({\mathfrak h}^{\perp})$ as a compact subgroup of $T$. 
Let $G_{i,k}^{(j)}$ be the orthogonal complement subgroup associated with (the Lie algebra of) the stabilizer subgroup $T_{i,k}^{(j)}$.  Note that we have $G^{(n)}_{i,1}=T$. Define an open subset of $M$ by  $M_{i,k}^{(j)}:=\mu^{-1}(\tilde\Delta_{i,k}^{(j)})$, which has the natural $G_{i,k}^{(j)}$-action and the following properties. 

\begin{itemize}
\item Each $G_{i,k}^{(j)}$ acts on $M_{i,k}^{(j)}$, and all orbits of $G_{i,k}^{(j)}$-action have the maximal dimension $\dim G_{i,k}^{(j)}$.  
\item If $\tilde\Delta_{i,k}^{(j)}\cap \tilde\Delta_{i,k'}^{(j')}\neq \emptyset$, then we have $G_{i,k}^{(j)}\subset G_{i,k'}^{(j')}$ or $G_{i,k}^{(j)}\supset G_{i,k'}^{(j')}$.  
\end{itemize}
%Note that the orbits of the action of $G_{i,k}^{(j)}$ on $M_{i,k}^{(j)}$ have $\dim (G_{i,k}^{(j)})$. 





\subsection{Good compatible fibration on toric origami manifolds}
\label{Good compatible fibration on toric origami manifolds}
By taking each open subset small enough we may assume that 
$\U'\cap M_{i,k}^{(j)}=\emptyset$ for all $i,j,k$ with $j\neq n$. 
The union $\displaystyle\U'\cup\bigcup_{i,j,k}M_{i,k}^{(j)}$ is not an open covering of the whole $M$. There exist a family of compact sets, which we call the {\it crack} $C_{i,k}^Z$ defined by 
$$
C_{i,k}^{Z}:=\mu^{-1}\left(\mu(\overline\U\setminus\U')\cap\Delta_{i,k}^{(n-1)}\right). 
$$


\begin{figure}[h]
\input{crack.tex}
\caption{Crack near the fold}
\label{Crack near the fold}
\end{figure}

Though we do not know the way to extend the good compatible fibration across the crack, we have the following.
%\end{remark}

\begin{prop}\label{crackgoodcompatifib}
A family of open subsets $\{\U', M_{i,k}^{(j)}\}_{i,j,k}$ defines a structure of good compatible fibration (Definition~\ref{goodcompatifib}) on the complement $M\setminus \bigcup_{i,k}C_{i,k}^{Z}$. 
\end{prop}

\begin{example}\label{exshpere5}
Consider the toric origami manifold $S^4$ with the moment map $\mu:S^4\to \R^2$ whose origami polytope is the union of two copies of the triangle, $\mu(S^4)=\Delta=\Delta_1\cup\Delta_2$.  
The open covering $\{\U', M_{i,k}^{(j)}\}_{i,j,k}$ consists of the inverse images of the following two copies of 5 open subsets of $\Delta_1$ ($=\Delta_2$) for any small $\vep >0$ : 
\begin{itemize}
\item $\tilde\Delta_Z$ : small open neighbourhood of the hypotenuse $\xi_1+\xi_2=1/2$. 
\item $\tilde\Delta^{(0)}_1=\tilde\Delta^{(0)}_2$ : small open ball of radius $\vep >0$ centered at $(0,0)$. 
\item $\tilde\Delta^{(1)}_{1,1}=\tilde\Delta^{(1)}_{2,1}$ : small open neighbourhood of the line segment,  $0\leq \xi_1<\vep, \vep/2\leq \xi_2 \leq 1-\vep$. 
\item $\tilde\Delta^{(1)}_{1,2}=\tilde\Delta^{(1)}_{2,2}$ : small open neighbourhood of the line segment $0\leq \xi_2<\vep, \vep/2\leq \xi_1 \leq 1-\vep$. 
\item ${\rm int}\Delta_1={\rm int}\Delta_2$. 
\end{itemize}
In this case the cracks consist of the inverse images of two compact subsets ${c_{1,1}^Z}={c_{2,1}^Z}$ and ${c_{1,2}^Z}={c_{2,2}^Z}$ defined by  
$$
{c_{1,1}^Z}\left(={c_{2,1}^Z}\right)  \ : \ \xi_1=0, 1-\vep\leq \xi_2 \leq 1-\vep/2 
$$ and 
$$
{c_{1,2}^Z}\left(={c_{2,2}^Z}\right) \ : \ \xi_2=0, 1-\vep\leq \xi_1\leq 1-\vep/2. 
$$

\begin{figure}[h]
%WinTpicVersion4.26
\unitlength 0.1in
\begin{picture}( 67.6800, 12.4000)(  8.0000,-28.4000)
% POLYGON 2 0 3 0 Black White
% 4 800 1600 800 2720 2080 2720 800 1600
% 
{\color[named]{Black}{%
\special{pn 8}%
\special{pa 800 1600}%
\special{pa 800 2720}%
\special{pa 2080 2720}%
\special{pa 800 1600}%
\special{pa 800 2720}%
\special{fp}%
}}%
% POLYGON 2 0 3 0 Black White
% 4 2664 1608 2664 2728 3944 2728 2664 1608
% 
{\color[named]{Black}{%
\special{pn 8}%
\special{pa 2664 1608}%
\special{pa 2664 2728}%
\special{pa 3944 2728}%
\special{pa 2664 1608}%
\special{pa 2664 2728}%
\special{fp}%
}}%
% POLYGON 2 0 3 0 Black White
% 4 4488 1616 4488 2736 5768 2736 4488 1616
% 
{\color[named]{Black}{%
\special{pn 8}%
\special{pa 4488 1616}%
\special{pa 4488 2736}%
\special{pa 5768 2736}%
\special{pa 4488 1616}%
\special{pa 4488 2736}%
\special{fp}%
}}%
% POLYGON 2 0 3 0 Black White
% 4 6288 1616 6288 2736 7568 2736 6288 1616
% 
{\color[named]{Black}{%
\special{pn 8}%
\special{pa 6288 1616}%
\special{pa 6288 2736}%
\special{pa 7568 2736}%
\special{pa 6288 1616}%
\special{pa 6288 2736}%
\special{fp}%
}}%
% LINE 2 2 3 0 Black White
% 2 800 1820 1810 2720
% 
{\color[named]{Black}{%
\special{pn 8}%
\special{pa 800 1820}%
\special{pa 1810 2720}%
\special{dt 0.045}%
}}%
% LINE 2 0 3 0 Black White
% 38 1690 2390 1570 2510 1760 2440 1640 2560 1820 2500 1700 2620 1890 2550 1760 2680 1950 2610 1840 2720 2010 2670 1960 2720 1630 2330 1510 2450 1570 2270 1450 2390 1500 2220 1380 2340 1440 2160 1320 2280 1370 2110 1260 2220 1310 2050 1190 2170 1250 1990 1130 2110 1180 1940 1070 2050 1120 1880 1000 2000 1050 1830 940 1940 990 1770 880 1880 930 1710 810 1830 860 1660 800 1720
% 
{\color[named]{Black}{%
\special{pn 8}%
\special{pa 1690 2390}%
\special{pa 1570 2510}%
\special{fp}%
\special{pa 1760 2440}%
\special{pa 1640 2560}%
\special{fp}%
\special{pa 1820 2500}%
\special{pa 1700 2620}%
\special{fp}%
\special{pa 1890 2550}%
\special{pa 1760 2680}%
\special{fp}%
\special{pa 1950 2610}%
\special{pa 1840 2720}%
\special{fp}%
\special{pa 2010 2670}%
\special{pa 1960 2720}%
\special{fp}%
\special{pa 1630 2330}%
\special{pa 1510 2450}%
\special{fp}%
\special{pa 1570 2270}%
\special{pa 1450 2390}%
\special{fp}%
\special{pa 1500 2220}%
\special{pa 1380 2340}%
\special{fp}%
\special{pa 1440 2160}%
\special{pa 1320 2280}%
\special{fp}%
\special{pa 1370 2110}%
\special{pa 1260 2220}%
\special{fp}%
\special{pa 1310 2050}%
\special{pa 1190 2170}%
\special{fp}%
\special{pa 1250 1990}%
\special{pa 1130 2110}%
\special{fp}%
\special{pa 1180 1940}%
\special{pa 1070 2050}%
\special{fp}%
\special{pa 1120 1880}%
\special{pa 1000 2000}%
\special{fp}%
\special{pa 1050 1830}%
\special{pa 940 1940}%
\special{fp}%
\special{pa 990 1770}%
\special{pa 880 1880}%
\special{fp}%
\special{pa 930 1710}%
\special{pa 810 1830}%
\special{fp}%
\special{pa 860 1660}%
\special{pa 800 1720}%
\special{fp}%
}}%
% LINE 0 0 3 0 Black White
% 2 6290 1950 6290 1810
% 
{\color[named]{Black}{%
\special{pn 20}%
\special{pa 6290 1950}%
\special{pa 6290 1810}%
\special{fp}%
}}%
% STR 2 0 3 0 Black White
% 4 1420 1970 1420 2070 2 0 0 0
% $\tilde\Delta_Z$
\put(14.2000,-20.7000){\makebox(0,0)[lb]{$\tilde\Delta_Z$}}%
% STR 2 0 3 0 Black White
% 4 2530 2850 2530 2950 2 0 0 0
% $\tilde\Delta_{i}^{(0)}$
\put(25.3000,-29.5000){\makebox(0,0)[lb]{$\tilde\Delta_{i}^{(0)}$}}%
% STR 2 0 3 0 Black White
% 4 4990 2900 4990 3000 2 0 0 0
% $\tilde\Delta_{i,2}^{(1)}$
\put(49.9000,-30.0000){\makebox(0,0)[lb]{$\tilde\Delta_{i,2}^{(1)}$}}%
% STR 2 0 3 0 Black White
% 4 7170 2880 7170 2980 2 0 0 0
% $c_{i,2}^Z$
\put(71.7000,-29.8000){\makebox(0,0)[lb]{$c_{i,2}^Z$}}%
% LINE 0 0 3 0 Black White
% 2 7140 2730 7340 2730
% 
{\color[named]{Black}{%
\special{pn 20}%
\special{pa 7140 2730}%
\special{pa 7340 2730}%
\special{fp}%
}}%
% STR 2 0 3 0 Black White
% 4 6010 1850 6010 1950 2 0 0 0
% $c_{i,1}^Z$
\put(60.1000,-19.5000){\makebox(0,0)[lb]{$c_{i,1}^Z$}}%
% POLYGON 2 2 3 0 Black White
% 5 4750 2570 5310 2570 5310 2730 4750 2730 4750 2570
% 
{\color[named]{Black}{%
\special{pn 8}%
\special{pn 8}%
\special{pa 4750 2570}%
\special{pa 4758 2570}%
\special{fp}%
\special{pa 4784 2570}%
\special{pa 4792 2570}%
\special{fp}%
\special{pa 4818 2570}%
\special{pa 4826 2570}%
\special{fp}%
\special{pa 4852 2570}%
\special{pa 4860 2570}%
\special{fp}%
\special{pa 4886 2570}%
\special{pa 4894 2570}%
\special{fp}%
\special{pa 4922 2570}%
\special{pa 4928 2570}%
\special{fp}%
\special{pa 4956 2570}%
\special{pa 4962 2570}%
\special{fp}%
\special{pa 4990 2570}%
\special{pa 4996 2570}%
\special{fp}%
\special{pa 5024 2570}%
\special{pa 5030 2570}%
\special{fp}%
\special{pa 5058 2570}%
\special{pa 5064 2570}%
\special{fp}%
\special{pa 5092 2570}%
\special{pa 5098 2570}%
\special{fp}%
\special{pa 5126 2570}%
\special{pa 5132 2570}%
\special{fp}%
\special{pa 5160 2570}%
\special{pa 5168 2570}%
\special{fp}%
\special{pa 5194 2570}%
\special{pa 5202 2570}%
\special{fp}%
\special{pa 5228 2570}%
\special{pa 5236 2570}%
\special{fp}%
\special{pa 5262 2570}%
\special{pa 5270 2570}%
\special{fp}%
\special{pa 5296 2570}%
\special{pa 5304 2570}%
\special{fp}%
\special{pa 5310 2590}%
\special{pa 5310 2598}%
\special{fp}%
\special{pa 5310 2624}%
\special{pa 5310 2632}%
\special{fp}%
\special{pa 5310 2658}%
\special{pa 5310 2666}%
\special{fp}%
\special{pa 5310 2692}%
\special{pa 5310 2700}%
\special{fp}%
\special{pa 5310 2726}%
\special{pa 5306 2730}%
\special{fp}%
\special{pa 5280 2730}%
\special{pa 5272 2730}%
\special{fp}%
\special{pa 5246 2730}%
\special{pa 5238 2730}%
\special{fp}%
\special{pa 5212 2730}%
\special{pa 5204 2730}%
\special{fp}%
\special{pa 5178 2730}%
\special{pa 5170 2730}%
\special{fp}%
\special{pa 5144 2730}%
\special{pa 5136 2730}%
\special{fp}%
\special{pa 5110 2730}%
\special{pa 5102 2730}%
\special{fp}%
\special{pa 5076 2730}%
\special{pa 5068 2730}%
\special{fp}%
\special{pa 5042 2730}%
\special{pa 5034 2730}%
\special{fp}%
\special{pa 5008 2730}%
\special{pa 5000 2730}%
\special{fp}%
\special{pa 4972 2730}%
\special{pa 4966 2730}%
\special{fp}%
\special{pa 4938 2730}%
\special{pa 4932 2730}%
\special{fp}%
\special{pa 4904 2730}%
\special{pa 4898 2730}%
\special{fp}%
\special{pa 4870 2730}%
\special{pa 4864 2730}%
\special{fp}%
\special{pa 4836 2730}%
\special{pa 4830 2730}%
\special{fp}%
\special{pa 4802 2730}%
\special{pa 4796 2730}%
\special{fp}%
\special{pa 4768 2730}%
\special{pa 4762 2730}%
\special{fp}%
\special{pa 4750 2714}%
\special{pa 4750 2706}%
\special{fp}%
\special{pa 4750 2680}%
\special{pa 4750 2672}%
\special{fp}%
\special{pa 4750 2646}%
\special{pa 4750 2638}%
\special{fp}%
\special{pa 4750 2612}%
\special{pa 4750 2604}%
\special{fp}%
\special{pa 4750 2578}%
\special{pa 4750 2570}%
\special{fp}%
}}%
% LINE 3 0 3 1 Black White
% 22 5180 2570 5020 2730 5120 2570 4960 2730 5060 2570 4900 2730 5000 2570 4840 2730 4940 2570 4780 2730 4880 2570 4750 2700 4820 2570 4750 2640 5240 2570 5080 2730 5300 2570 5140 2730 5310 2620 5200 2730 5310 2680 5260 2730
% 
{\color[named]{Black}{%
\special{pn 4}%
\special{pa 5180 2570}%
\special{pa 5020 2730}%
\special{fp}%
\special{pa 5120 2570}%
\special{pa 4960 2730}%
\special{fp}%
\special{pa 5060 2570}%
\special{pa 4900 2730}%
\special{fp}%
\special{pa 5000 2570}%
\special{pa 4840 2730}%
\special{fp}%
\special{pa 4940 2570}%
\special{pa 4780 2730}%
\special{fp}%
\special{pa 4880 2570}%
\special{pa 4750 2700}%
\special{fp}%
\special{pa 4820 2570}%
\special{pa 4750 2640}%
\special{fp}%
\special{pa 5240 2570}%
\special{pa 5080 2730}%
\special{fp}%
\special{pa 5300 2570}%
\special{pa 5140 2730}%
\special{fp}%
\special{pa 5310 2620}%
\special{pa 5200 2730}%
\special{fp}%
\special{pa 5310 2680}%
\special{pa 5260 2730}%
\special{fp}%
}}%
% STR 2 0 3 0 Black White
% 4 4180 2140 4180 2240 2 0 0 0
% $\tilde\Delta_{i,1}^{(1)}$
\put(41.8000,-22.4000){\makebox(0,0)[lb]{$\tilde\Delta_{i,1}^{(1)}$}}%
% POLYGON 2 2 3 0 Black White
% 5 4480 2490 4480 1930 4640 1930 4640 2490 4480 2490
% 
{\color[named]{Black}{%
\special{pn 8}%
\special{pn 8}%
\special{pa 4480 2490}%
\special{pa 4480 2484}%
\special{fp}%
\special{pa 4480 2456}%
\special{pa 4480 2450}%
\special{fp}%
\special{pa 4480 2422}%
\special{pa 4480 2416}%
\special{fp}%
\special{pa 4480 2388}%
\special{pa 4480 2380}%
\special{fp}%
\special{pa 4480 2354}%
\special{pa 4480 2346}%
\special{fp}%
\special{pa 4480 2320}%
\special{pa 4480 2312}%
\special{fp}%
\special{pa 4480 2286}%
\special{pa 4480 2278}%
\special{fp}%
\special{pa 4480 2252}%
\special{pa 4480 2244}%
\special{fp}%
\special{pa 4480 2218}%
\special{pa 4480 2210}%
\special{fp}%
\special{pa 4480 2184}%
\special{pa 4480 2176}%
\special{fp}%
\special{pa 4480 2150}%
\special{pa 4480 2142}%
\special{fp}%
\special{pa 4480 2116}%
\special{pa 4480 2108}%
\special{fp}%
\special{pa 4480 2082}%
\special{pa 4480 2074}%
\special{fp}%
\special{pa 4480 2048}%
\special{pa 4480 2040}%
\special{fp}%
\special{pa 4480 2012}%
\special{pa 4480 2006}%
\special{fp}%
\special{pa 4480 1978}%
\special{pa 4480 1972}%
\special{fp}%
\special{pa 4480 1944}%
\special{pa 4480 1938}%
\special{fp}%
\special{pa 4500 1930}%
\special{pa 4508 1930}%
\special{fp}%
\special{pa 4534 1930}%
\special{pa 4542 1930}%
\special{fp}%
\special{pa 4568 1930}%
\special{pa 4576 1930}%
\special{fp}%
\special{pa 4602 1930}%
\special{pa 4610 1930}%
\special{fp}%
\special{pa 4636 1930}%
\special{pa 4640 1934}%
\special{fp}%
\special{pa 4640 1962}%
\special{pa 4640 1968}%
\special{fp}%
\special{pa 4640 1996}%
\special{pa 4640 2002}%
\special{fp}%
\special{pa 4640 2030}%
\special{pa 4640 2036}%
\special{fp}%
\special{pa 4640 2064}%
\special{pa 4640 2070}%
\special{fp}%
\special{pa 4640 2098}%
\special{pa 4640 2104}%
\special{fp}%
\special{pa 4640 2132}%
\special{pa 4640 2138}%
\special{fp}%
\special{pa 4640 2166}%
\special{pa 4640 2172}%
\special{fp}%
\special{pa 4640 2200}%
\special{pa 4640 2208}%
\special{fp}%
\special{pa 4640 2234}%
\special{pa 4640 2242}%
\special{fp}%
\special{pa 4640 2268}%
\special{pa 4640 2276}%
\special{fp}%
\special{pa 4640 2302}%
\special{pa 4640 2310}%
\special{fp}%
\special{pa 4640 2336}%
\special{pa 4640 2344}%
\special{fp}%
\special{pa 4640 2370}%
\special{pa 4640 2378}%
\special{fp}%
\special{pa 4640 2404}%
\special{pa 4640 2412}%
\special{fp}%
\special{pa 4640 2438}%
\special{pa 4640 2446}%
\special{fp}%
\special{pa 4640 2472}%
\special{pa 4640 2480}%
\special{fp}%
\special{pa 4624 2490}%
\special{pa 4616 2490}%
\special{fp}%
\special{pa 4590 2490}%
\special{pa 4582 2490}%
\special{fp}%
\special{pa 4556 2490}%
\special{pa 4548 2490}%
\special{fp}%
\special{pa 4522 2490}%
\special{pa 4514 2490}%
\special{fp}%
\special{pa 4488 2490}%
\special{pa 4480 2490}%
\special{fp}%
}}%
% LINE 3 0 3 1 Black White
% 22 4480 2060 4640 2220 4480 2120 4640 2280 4480 2180 4640 2340 4480 2240 4640 2400 4480 2300 4640 2460 4480 2360 4610 2490 4480 2420 4550 2490 4480 2000 4640 2160 4480 1940 4640 2100 4530 1930 4640 2040 4590 1930 4640 1980
% 
{\color[named]{Black}{%
\special{pn 4}%
\special{pa 4480 2060}%
\special{pa 4640 2220}%
\special{fp}%
\special{pa 4480 2120}%
\special{pa 4640 2280}%
\special{fp}%
\special{pa 4480 2180}%
\special{pa 4640 2340}%
\special{fp}%
\special{pa 4480 2240}%
\special{pa 4640 2400}%
\special{fp}%
\special{pa 4480 2300}%
\special{pa 4640 2460}%
\special{fp}%
\special{pa 4480 2360}%
\special{pa 4610 2490}%
\special{fp}%
\special{pa 4480 2420}%
\special{pa 4550 2490}%
\special{fp}%
\special{pa 4480 2000}%
\special{pa 4640 2160}%
\special{fp}%
\special{pa 4480 1940}%
\special{pa 4640 2100}%
\special{fp}%
\special{pa 4530 1930}%
\special{pa 4640 2040}%
\special{fp}%
\special{pa 4590 1930}%
\special{pa 4640 1980}%
\special{fp}%
}}%
% CIRCLE 2 2 3 0 Black White
% 4 2660 2720 3020 2730 3560 2720 2670 1140
% 
{\color[named]{Black}{%
\special{pn 8}%
\special{pa 2662 2360}%
\special{pa 2670 2360}%
\special{fp}%
\special{pa 2696 2362}%
\special{pa 2704 2364}%
\special{fp}%
\special{pa 2728 2366}%
\special{pa 2736 2368}%
\special{fp}%
\special{pa 2760 2374}%
\special{pa 2766 2376}%
\special{fp}%
\special{pa 2790 2384}%
\special{pa 2796 2388}%
\special{fp}%
\special{pa 2820 2398}%
\special{pa 2826 2400}%
\special{fp}%
\special{pa 2848 2414}%
\special{pa 2854 2416}%
\special{fp}%
\special{pa 2876 2432}%
\special{pa 2880 2436}%
\special{fp}%
\special{pa 2900 2452}%
\special{pa 2906 2458}%
\special{fp}%
\special{pa 2924 2476}%
\special{pa 2930 2482}%
\special{fp}%
\special{pa 2946 2502}%
\special{pa 2950 2508}%
\special{fp}%
\special{pa 2964 2528}%
\special{pa 2968 2534}%
\special{fp}%
\special{pa 2982 2556}%
\special{pa 2984 2562}%
\special{fp}%
\special{pa 2994 2586}%
\special{pa 2998 2592}%
\special{fp}%
\special{pa 3006 2616}%
\special{pa 3008 2622}%
\special{fp}%
\special{pa 3014 2648}%
\special{pa 3014 2654}%
\special{fp}%
\special{pa 3018 2680}%
\special{pa 3018 2686}%
\special{fp}%
\special{pa 3020 2714}%
\special{pa 3020 2720}%
\special{fp}%
}}%
% LINE 3 0 3 0 Black White
% 10 2920 2480 2680 2720 2970 2550 2800 2720 3010 2630 2920 2720 2860 2420 2660 2620 2780 2380 2660 2500
% 
{\color[named]{Black}{%
\special{pn 4}%
\special{pa 2920 2480}%
\special{pa 2680 2720}%
\special{fp}%
\special{pa 2970 2550}%
\special{pa 2800 2720}%
\special{fp}%
\special{pa 3010 2630}%
\special{pa 2920 2720}%
\special{fp}%
\special{pa 2860 2420}%
\special{pa 2660 2620}%
\special{fp}%
\special{pa 2780 2380}%
\special{pa 2660 2500}%
\special{fp}%
}}%
\end{picture}%

\caption{Covering of the $S^4$}
\label{Covering of the $S^4$}
\end{figure}

\end{example}


%%%%%%%%%%%%%%%%%%%%%%%%%%%%%%%%%%%%%%%%%%%%%%%%%%%%%%%%%%%%%%%%%%%%%%%%%
\section{Compatible system on toric origami manifolds}
\label{Compatible system on toric origami manifolds}
In this section we construct a {\it compatible system (of Dirac-type operators)}  on toric origami manifolds. 
The notion of compatible system is introduced in \cite{Fujita-Furuta-Yoshida2}, which is a family of Dirac-type operators along leaves of compatible fibration and satisfies some anti-commutativity. See also Definition~\ref{compatible system}. 

\begin{assump}\label{assump2}
In this section we consider a toric origami manifold $(M,\omega,T,\mu)$ satisfying the following assumption. 
\begin{itemize}
\item $(M,\omega,T,\mu)$ satisfies Assumption~\ref{assump}. 
\item The de Rham cohomology class $[\omega]$ has an integral lift in $H^2(M,\Z)$. 
\item A $T$-equivariant pre-quantizing line bundle $(L,\nabla)$ is fixed. 
Namely, $L$ is a $T$-equivariant Hermitian line bundle over $M$ and $\nabla$ is a $T$-invariant Hermitian connection whose curvature form is equal to $-\sqrt{-1}\omega$. 
\end{itemize}
\end{assump}

Together with the assumptions we may choose a stable almost complex structure $\tilde J$ as in Theorem~\ref{stablealmostcomplex} so that the tangent bundle of each symplectic submanifold $\mu^{-1}({\rm int}(\Delta_{i,k}^{(j)}))$ is preserved by $\tilde J$ for all $i,j$ and $k$. 
%
Under the above assumption we use the $\Z/2$-graded Clifford module bundle $W_L$ as in the end of Section~\ref{Stable almost complex structure and Clifford module bundle}. 
As it is shown in Section~\ref{Compatible fibration on toric origami manifolds}, $M\setminus \bigcup_{i,k}C_{i,k}^{Z}$ has a structure of good compatible fibration $\{\U', M_{i,k}^{(j)}\}_{i,j,k}$. Since $\{M_{i,k}^{(j)}\}_{i,j,k}$ is a good compatible fibration on an open toric manifold $M\setminus \overline{\U}$, we have a compatible system $\{D_{i,k}^{(j)}\}_{i,j,k}$ on it as in \cite[Theorem~5.1]{Fujita-Furuta-Yoshida2}. Namely for each $i,j,k$ we have the following. 
% $\{D_{i,k}^{(j)}\}_{i,j,k}$ is the family of differential operators which satisfies the following. 
\begin{itemize}
\item $D_{i,k}^{(j)}$ is a first order formally self-adjoint differential operator of degree-one, which acts on the space of smooth sections of $W_{L}|_{M_{i,k}^{(j)}}$. 
\item $D_{i,k}^{(j)}$ contains only the differentials along the $G_{i,k}^{(j)}$-orbits. 
\item For each $x\in M_{i,k}^{(j)}$, the restriction of $D_{i,k}^{(j)}$ to the orbit $G_{i,k}^{(j)}\cdot x$ is a Dirac-type operator on the $\Z/2$-graded $Cl(T(G_{i,k}^{(j)}\cdot x))$-module bundle $W_L|_{G_{i,k}^{(j)}\cdot x}$. 
\item Let $\tilde u$ be a $G_{i,k}^{(j)}$-invariant section of the normal bundle to the orbit $G_{i,k}^{(j)}\cdot x$. Then $D_{i,k}^{(j)}$ anti-commutes with the Clifford multiplication $c(\tilde u)$ of $\tilde u$ : 
\begin{equation}\label{anti-commutatibity}
D_{i,k}^{(k)}c(\tilde u)+c(\tilde u)D_{i,k}^{(k)}=0.
\end{equation}
%\item The same conditions hold for a family of operators $\{D_{i,k'}^{(j'), Z}\}_{i,j',k'}$ on $\{M_{i,k'}^{(j'), Z}\}_{i,j',k'}$. 
\end{itemize}
Now we construct a differential operator $D_{Z}$ along the $S^1$-orbits on $\U$. We first study the product structure of $W|_{\U}$. 
%Fix a connection $\alpha$ of the principal $S^1$-bundle $\pi:Z\to B$ so that we have the diffeomorphism $\U\cong Z\times(-\vep,\vep)$ as in Theorem~\ref{Morsermodel}. 
Hereafter we use the identification $\U=Z\times (-\vep, \vep)\cong(Z\times S^1\times(-\vep,\vep))/S^1$ with respect to the diagonal $S^1$-action. 
%, and use the same notation $\pi:\U\to B$ for the projection as the $S^1\times (-\vep,\vep)$-bundle.  
%By using the connection of the principal $S^1$-bundle $Z\to B$ we have the splitting of the tangent bundle $TZ\cong \pi^*TB\oplus T_{\pi}Z$, where $T_{\pi}Z$ is the tangent bundle along the fiber, which is a real line bundle over $Z$. Since $T\U$ is oriented, and hence, $TZ$ is also oriented, the fact that $B$ is a symplectic manifold implies that $T_{\pi}Z$ is an orientable.  In particular, $T_{\pi}Z$ is trivial real line bundle. Under these identifications we may assume that the almost complex structure $\tilde J|_{\U}$ in Theorem~\ref{stablealmostcomplex}  on $T\U\oplus \R^2\cong \pi^*TB\oplus T_{\pi}Z\oplus\R\oplus\R^2$ is the direct sum of almost complex structures on the symplectic vector bundle $\pi^*TB$ and the trivial bundle $T_{\pi}Z\oplus\R\oplus\R^2$ of real rank 4. 
According to Remark~\ref{stablecomplexstrrem}(2) we may assume that the almost complex structure $\tilde J|_{\U}$ in Theorem~\ref{stablealmostcomplex}  on $T\U\oplus \R^2\cong \pi^*TB\oplus T_{\pi}Z\oplus\R\oplus\R^2$ is the direct sum of almost complex structures on the symplectic vector bundle $\pi^*TB$ and the trivial bundle $T_{\pi}Z\oplus\R\oplus\R^2$ of real rank 4. Then we have 
$$
W|_{\U}={\rm Hom}_{Cl_2}(W_2, \wedge_{\C}^{\bullet}(T\U\oplus\R^2))=\pi^*(\wedge
_{\C}^{\bullet}TB)\otimes{\rm Hom}_{Cl_2}(W_2, \wedge_{\C}^{\bullet}(T_{\pi}Z\oplus\R^3)). 
$$
On the other hand we have the commutative diagram of bundle maps 
\[
\xymatrix{ 
T_{\pi}Z \ar[d] & \ar[l]  {q}^*(T_{\pi}Z)\cong p^*(TS^1)
 \ar[r] \ar[d] & TS^1 \ar[d]\\
Z & \ar[l]^{q}  Z \times S^1 \ar[r]_p & S^1,  } 
\]
where $p:Z\times S^1\to S^1$ is the projection to the $S^1$-factor and $q:Z\times S^1\to (Z\times S^1)/S^1\cong Z$, $(z,t)\mapsto zt^{-1}$ is the quotient map with respect to the diagonal action of $S^1$. 
The isomorphism in the middle column is given by the differential of the map $S^1\to Z$, $t\mapsto zt^{-1}$ for $z\in Z$. The commutative diagram implies that the vector bundle $T_{\pi}Z\oplus\R^3\to\U\cong (Z\times S^1\times(-\vep,\vep))/S^1$ can be obtained as a quotient bundle of $p^*(TS^1)\oplus \R^3\to Z\times S^1\times (-\vep,\vep)$. 
In particular ${\rm Hom}_{Cl_2}(W_2, \wedge_{\C}^{\bullet}(T_{\pi}Z\oplus\R^3))\to\U$ can be obtained as a quotient bundle of ${\rm Hom}_{Cl_2}(W_2, \wedge_{\C}^{\bullet}(TS^1\oplus\R^3))\to Z^1\times S^1\times (-\vep,\vep)$, 
where the complex structure on $TS^1\oplus\R^3$ is given by the same formula for $B_t$ as in the proof of \cite[Theorem~2]{SilvaGuilleminWoodward} under a trivialization. Note that ${\rm Hom}_{Cl_2}(W_2, \wedge_{\C}^{\bullet}(TS^1\oplus\R^3))$ has a structure of $\Z/2$-graded $Cl(TS^1\oplus \R)$-module bundle over $S^1\times(-\vep,\vep)$. 

Now we decompose the line bundle $L$ over $\U$. 
Let $(L_0, \nabla)\to S^1\times(-\vep, \vep)$ be the pre-quantizing line bundle over the folded cylinder as in Appendix~\ref{A computation of local index of the folded cylinder}. 

\begin{prop}\label{product structure of L}
If we take $\vep$ small enough, then 
the diffeomorphism $\varphi:\U\stackrel{\cong}{\to} Z\times (-\vep,\vep)$ as in Theorem~\ref{Morsermodel} can be lifted to an isomorphism between $L|_{\U}\to \U$ and $(L|_Z\boxtimes L_0)/S^1\to (Z\times S^1\times(-\vep,\vep))/S^1=Z\times(-\vep, \vep)$.  
\end{prop}
\begin{proof}
Note that there exists the canonical isomorphism $\tilde\varphi_0$ between $\iota_Z^*L$ and $\iota_0^*\left((L|_Z\boxtimes L_0)/S^1\right)$. Fix a Hermitian connection of $(L|_Z\boxtimes L_0)/S^1$. Then the we have the required isomorphism by using $\tilde\varphi_0$ and the parallel transport. 
\end{proof}

Summarising we have the following. 

\begin{prop}\label{product structure of Cl}
Let $W_{B,L_B}:=\wedge^{\bullet}_{\C}TB\otimes (L|_Z/S^1)$ be a $\Z/2$-graded $Cl(TB)$-module bundle over $B$.  Let $W_{0,L_0}:={\rm Hom}_{Cl_2}(W_2, \wedge_{\C}^{\bullet}(TS^1\oplus\R^3))\otimes L_0$ be a $\Z/2$-graded $Cl(TS^1\oplus\R)$-module bundle over $S^1\times(-\vep,\vep)$ as in the above construction. 
The $\Z/2$-graded Clifford module bundle $W_L|_{\U}\to \U$ is isomorphic to the quotient bundle of the tensor product $\pi^*W_{B,L_B}\otimes p^*W_{0,L_0}\to Z\times S^1\times(-\vep,\vep)$ with respect to the diagonal $S^1$-action, where $\pi:Z\times S^1\times(-\vep,\vep)\to B$ and $p:Z\times S^1\times(-\vep,\vep)\to S^1\times(-\vep,\vep)$ are natural projections. 
\end{prop}

Let $D_{S^1}$ be a Dirac-type operator along the $S^1$-orbits in $S^1\times(-\vep,\vep)$, 
which acts on the space of smooth sections of $W_{0,L_0}$. See Appendix~\ref{A computation of local index of the folded cylinder} for the explicit description of $D_{S^1}$. Let $\epsilon_{B}$ be the map representing the $\Z/2$-grading of $W_{B,L_B}$, i.e., $\epsilon_B(v)=(-1)^{{\rm deg}(v)}(v)$ for $v\in W_{B,L_B}$.  The product of operators $\epsilon_B\otimes D_{S^1}$ is $S^1$-invariant, and it induces a differential operator $D_{Z}$ acting on the smooth sections of $W|_{\U}$ through the isomorphism in Proposition~\ref{product structure of L}. 
Since the $S^1$-action on $Z$ is given by a subgroup of $T$,   $D_{Z}$ is a differential operator along the $S^1$-orbits and satisfies the anti-commutativity as in (\ref{anti-commutatibity}). 

\begin{prop}
The family of differential operators $\{D_{Z}, D_{i,k}^{(j)}\}_{i,j,k}$ is a compatible system on the compatible fibration defined by the torus actions $\{S^1\curvearrowright{\mathcal U}', \ G_{i,k}^{(j)}\curvearrowright M_{i,k}^{(j)}\}_{i,j,k}$. 
\end{prop}

%%%%%%%%%%%%%%%%%%%%%%%%%%%%%%%%%%%%%%%%%%%%%%%%%%%%%%%%%%%%%%%%%%%%%%%%
\subsection{Acyclicity of the compatible system}
\label{Acyclicity of the compatible system}
In this section we determine the condition for the compatible system $\{D_Z,D_{i,k}^{(j)}\}_{i,j}$  to be {\it acyclic} (\cite[Definition~6.10]{Fujita-Furuta-Yoshida2} or Definition~\ref{strongly acyclic}).  
%We first prepare several notations. 


Let ${\mathfrak g}_{i,k}^{(j)*}$ be the dual of the Lie algebra of the subtorus $G_{i,k}^{(j)}$ and $({\mathfrak g}_{i,k}^{(j)*})_{\Z}$ the integral weight lattice of ${\mathfrak g}_{i,k}^{(j)*}$. Let $\iota_{i,k}^{(j)}:\g_{i,k}^{(j)}\to \g$ be the inclusion of the Lie subalgebra. Note that the composition $\mu_{i,k}^{(j)}:=(\iota_{i,k}^{(j)*})\circ \mu : M_{i,k}^{(j)} \to {\mathfrak g}_{i,k}^{(j)*}$ is the moment map for the Hamiltonian $G_{i,k}^{(j)}$-action on $M_{i,k}^{(j)}$. We put $M_{i,k}^{(j)\circ}:=M_{i,k}^{(j)}\setminus (\mu_{i,k}^{(j)})^{-1}((\g_{i,k}^{(j)*})_{\Z})$. 


\begin{prop}\label{acyclic1} 
For each $x\in M_{i,k}^{(j)\circ}$, we have $\ker(D_{i,k}^{(j)}|_{G_{i,k}^{(j)}\cdot x})=0$.  
\end{prop}
\begin{proof}
Note that for each $x\in M_{i,k}^{(j)}$ the kernel of $D_{i,k}^{(j)}|_{G_{i,k}^{(j)}\cdot x}$ vanishes if and only if there are no non-trivial global parallel sections of $L|_{G_{i,k}^{(j)}\cdot x}$. The proposition follows from the fact that if there exists a global parallel section, then we have $\mu_{i,k}^{(j)}(x)=\iota_{i,k}^{(j)*}(\mu(x))$ lies in the integral weight lattice $({\mathfrak g}_{i,k}^{(j)*})_{\Z}$. 
\end{proof}

We may take $\vep>0$ small enough so that $\mu(\U)=\mu(Z\times (-\vep,\vep))$ does not contain any integral lattice points outside $\mu(Z)=\Delta_Z$. Then we have the following by the same argument as that for Proposition~\ref{acyclic1}. 

\begin{prop}\label{acyclic2} 
For each $x\in \U'\setminus Z$, we have $\ker(D_Z|_{S^1\cdot x})=0$.  
\end{prop}

%Let $\U'=Z\times(-\vep/2, \vep/2)$ be the open neighbourhood of $Z$ as in Section~\ref{Torus actions near the fold}. 
We put $V:=\left(\U'\cup\bigcup_{i,j,k}M_{i,k}^{(j)\circ}\right)\setminus \left(Z\cup\bigcup_{i,k}C_{i,k}^{Z}\right)$. Then $M\setminus V$ is compact. Since $\{S^1\curvearrowright{\mathcal U}', \ G_{i,k}^{(j)}\curvearrowright M_{i,k}^{(j)}\}_{i,j,k}$ is a good compatible fibration one can see that the following four types of the anti-commutators on the intersections are non-negative. 
\begin{itemize}
\item $D_{i,k}^{(j)}D_{i,k'}^{(j')}+D_{i,k'}^{(j')}D_{i,k}^{(j)}$ on $M_{i,k}^{(j)}\cap M_{i,k'}^{(j')}$, 
%\item $D_{i,k}^{(j),Z}D_{i',k'}^{(j'),Z}+D_{i',k'}^{(j'),Z}D_{i,k}^{(j),Z}$ on $ M_{i,k}^{(j),Z}\cap M_{i,k'}^{(j'),Z}$, 
%\item $D_ZD_{i,k}^{(j),Z}+D_{i,k}^{(j),Z}D_Z$ on $\U\cap M_{i,k}^{(j),Z}$,\\
and 
\item  $D_ZD_{i,1}^{(n)}+D_{i,1}^{(n)}D_Z$ on $\U'\cap M_{i,1}^{(n)}$. 
\end{itemize}
See \cite[Proposition~5.8, Lemma~5.9]{Fujita-Furuta-Yoshida2} for example. Together with Proposition~\ref{acyclic1} this fact implies the following.  
\begin{prop}\label{acyclic3} 
The compatible system $\{D_{Z}, D_{i,k}^{(j)}\}_{i,j,k}$ is acyclic over $V$. 
\end{prop}


%%%%%%%%%%%%%%%%%%%%%%%%%%%%%%%%%%%%%%%%%%%%%%%%%%%%%%%%%%%%%%%%%%%
\subsection{Localization formula and Danilov-type formula}
\label{Localization formula and Danilov type formula}
As in Definition~\ref{origamiRR}, the Riemann-Roch number $RR(M,L)$ is defined for any origami manifold $(M,\omega)$ with pre-quantizing line bundle $(L,\nabla)$. If $(M,\omega)$ is a toric origami manifold with the action of a torus $T$, then the resulting index is an element of the character ring $R(T)$ of $T$. In this case we call the index the {\it equivariant Riemann-Roch number} or {\it Riemann-Roch character} and is denoted by $RR_T(M,L)$. 

We use notations in the previous sections and assume Assumption~\ref{assump2}. For each $i,j(\neq n),$ and $k$ we may assume that 
$$
\tilde\Delta_{i,k}^{(j)}\cap{\rm int}\Delta_i\cap \Tt_{\Z}^*=\emptyset,   
$$ and we take and fix a $T$-invariant small open neighbourhood $V_{i,k}^{(j)}$ of $(\mu_{i,k}^{(j)})^{-1}((\g_{i,k}^{(j)*})_{\Z})$ for each $i,j$ and $k$. By the above assumption one has that if $j\neq n$,  then  $V_{i,k}^{(j)}\cap \mu^{-1}(\Tt_{\Z}^*)$ consists of the inverse image of lattice points in the boundary $\partial \Delta_i=\Delta_i\setminus {\rm int}\Delta_i$. 
We also take and fix a small open neighbourhood $V_{i,k}^Z$ of the crack $C_{i,k}^Z$ so that it does not contain any integral points for each $i$ and $k$. Note that each open subset $V_{i,k}^{(j)}\cap V$ (resp. $V_{i,k}^{Z}\cap V$) with compact complement $V_{i,k}^{(j)}\setminus V_{i,k}^{(j)}\cap V=(\mu_{i,k}^{(j)})^{-1}((\g_{i,k}^{(j)*})_{\Z})(\supset M_{i,k}^{(j)}\cap \mu^{-1}(\g_{\Z}^*))$ (resp. $V_{i,k}^{Z}\setminus V_{i,k}^{Z}\cap V=C_{i,k}^{Z})$ is equipped with an acyclic compatible system by Proposition~\ref{acyclic3}, and hence,  the $T$-equivariant local index $\ind_T(V_{i,k}^{(j)}, V_{i,k}^{(j)}\cap V)$ (resp. $\ind_T(V_{i,k}^Z,V_{i,k}^Z\cap V)$) is defined (Theorem~\ref{def of local ind}) . As in the same way one can define the $T$-equivariant local index for the fold, $\ind_T(\U',\U'\setminus Z)$, is defined. 


The localization formula (Theorem~\ref{localizationprototype}) implies that the Riemann-Roch character is localized at $\mu^{-1}(\g_{\Z}^*)\cup Z\cup \bigcup_{i,k}C_{i,k}^Z\subset M\setminus V$ as follows. 
%Precisely we have the following localization formula. 
\begin{theorem}\label{localization of RR}
Under Assumption~\ref{assump2} we have the localization formula of $T$-equivariant index 
$$
RR_T(M,L)=\ind_T(\U',\U'\setminus Z)+\sum_{i,j,k}\ind_T(V_{i,k}^{(j)}, V_{i,k}^{(j)}\cap V)+\sum_{i,k}\ind_T(V_{i,k}^Z, V_{i,k}^Z\cap V). 
$$
\end{theorem}
By computing the contributions $\ind_T(\U',\U'\setminus Z)$ 
(Theorem~\ref{foldind=0}), $\ind_T(V_{i,k}^{(j)}, V_{i,k}^{(j)}\cap V)$ 
(Theorem~\ref{positive contribution}, Theorem~\ref{negative contribution}) 
and $\ind_T(V_{i,k}^Z, V_{i,k}^Z\cap V)$ (Theorem~\ref{crackind=0}) in the subsequent section, we have the following Danilov-type formula. 

\begin{theorem}\label{origamiDanilov}
%Suppose that all vertices of the Delzant polytopes $\{\Delta_i\}_i$ are integral points. 
Under Assumption~\ref{assump2} we have the following equality as elements in the character ring $R(T)$.\begin{equation}\label{origamiDanilovformula}
RR_T(M,L)=\sum_{\xi_+\in \mu(M^+)\cap\Tt^*_{\Z}}\C_{(\xi_+)}-\sum_{\xi_-\in \mu(M^-)\cap\Tt^*_{\Z}}\C_{(\xi_-)},  
\end{equation}
where for each $\xi \in \Tt_{\Z}^*$ we denote by $\C_{(\xi)}$ the irreducible representation of $T$ whose weight is given by $\xi$.  
\end{theorem}



To compute the local contributions in the subsequent sections, we will use the following notations. 
%Let $\{\Delta_i\}_{i=1,\cdots,N}$ be the collection of convex polytopes associated with the moment map $\mu:M\to \Tt^*$. 
We divide the collection of Delzant polytopes $\{\Delta_i\}_{i=1,\cdots,N}$ into two subsets, 
$$
\{\Delta_i\}_{i=1,\cdots, N}=\{\Delta_{i}^+\}_{i=1,\cdots, N_+}\cup\{\Delta_{i}^-\}_{i=1,\cdots, N_-}, 
$$where $N_++N_-=N$ and the sign is determined by the condition $\displaystyle\mu(M^{\pm})=\bigcup_{i=1}^{N_{\pm}}\Delta_i^{\pm}$. In a similar way we also use notations $\Delta_{i,k}^{(j)\pm}$,  $V_{i,k}^{(j)\pm}$, $\mu_{i,k}^{(j)\pm}$ and $\g_{i,k}^{(j)\pm}$. 

%Theorem~\ref{origamiDanilov} follows from Theorem~\ref{localization of RR} and the computations of . 

In terms of this notations the formula (\ref{origamiDanilovformula}) can be rewritten as 
$$
RR_T(M,L)=\sum_{i,j,k}\left(\sum_{\xi_+\in {\rm int}\Delta_{i,k}^{(j)+}\cap\Tt^*_{\Z}}\C_{(\xi_+)}-\sum_{\xi_-\in {\rm int}\Delta_{i,k}^{(j)-}\cap\Tt^*_{\Z}}\C_{(\xi_-)}\right)
$$


%\begin{remark}
%The formula itself is valid without the integrality condition for $\Delta_i$. In fact it can be proved without the integrality condition by an altenative proof based on the cobordism theorem \cite[Theorem~4.1]{Silva-Guillemin-Pires}. See the subsequent subsection. 
%, however, it is technically essential in the present paper to compute the contribution from the crack, $\ind_T(V_{i,k}^Z, V_{i,k}^Z\cap V)$. See Section~\ref{Contribution from the crack}.  
%\end{remark}

\begin{example}\label{exsphere6}
Consider the toric origami manifold $(S^{2n},\omega)$, the unit sphere, with the moment map $\mu:S^{2n}\to \R^n$ as in Example~\ref{exsphere3}, whose origami polytope is the union of two copies of the $n$-simplex, $\mu(S^{2n})=\Delta=\Delta_1\cup\Delta_2$. Since $\mu((S^{2n})^+)\cap\Tt^*_{\Z}=\mu((S^{2n})^-)\cap\Tt^*_{\Z}$,  one has 
$RR_T(S^{2n}, L)=0$ for any $T$-equivariant pre-quantizing line bundle $L$. 

Note that if we use the folded symplectic form $k\omega$ for any positive constant $k$, then the origami polytope for $(S^{2n}, k\omega)$ is the similar extension with ratio $k$ of the original origami polytope. In this case one also has $RR_T(S^{2n},L_k)=0$ for any $T$-equivariant pre-quantizing line bundle $L_k$. 
\end{example}

\subsection{Comments on another possible approaches}
The formula (\ref{origamiDanilovformula}) in Theorem~\ref{origamiDanilov} itself can be obtained as a consequence of the cobordism theorem \cite[Theorem~4.1]{Silva-Guillemin-Pires} and Danilov's theorem for symplectic toric manifolds. 

There is an another possible approach which uses the theory of {\it multi-fans} introduced in \cite{Hattori-Masuda1}. The equivariant index formula \cite[Theorem~11.1]{Hattori-Masuda1}, which is based on the fixed point formula, would be available to the left hand side of (\ref{origamiDanilovformula}). In fact as it is shown in \cite{Masuda-Park} one can associate a multi-fan for each oriented toric origami manifold.  

It would be possible to show the formula (\ref{origamiDanilovformula}) by using the theory of {\it transverse index} in \cite{Braverman}\cite{ParadanVergne}. In \cite{Braverman} it was shown that the Riemann-Roch character $RR_T(M,L)$ can be realized as a perturbation of Dirac operator by the Clliford multiplication of the Kirwan vector field of the moment map. By considering the perturbation $RR_T(M,L)$ is localized at the zero locus of the Kirwan vector field, i.e., the fixed point set $M^{T}$. Under Assumption~\ref{assump}, the fold has a free $S^1$-action, and hence, there are no contributions of the fold to $RR_T(M,L)$. In particular $RR_T(M,L)$ is the sum of contributions of the vertices of the image of the moment map $\mu(M\setminus Z)$.  As in \cite[Example~13]{Vergne3} the contribution from a fixed point is infinite sum of one dimensional representations of $T$ in general. It implies that $RR_T(M,L)$ is expressed as a cancellation of infinite sum of one dimensional representations. See also \cite{HajimeS1}  for the infinite dimensional nature of the transverse index and the finite dimensional nature of the index theory in \cite{Fujita-Furuta-Yoshida1, Fujita-Furuta-Yoshida2}.  

In contrast to these approaches our proof is direct and geometric, which detects the contribution of each lattice point directly and contains a new proof of original Danilov's theorem as a special case.  
%%%%%%%%%%%%%%%%%%%%%%%%%%%%%%%%%%%%%%%%%%%%%%%%%%%
\section{Computation of the local contribution} 
\label{Computation of the local contribution}
\subsection{Toric case} 
\label{Toric case}
In this subsection we consider the symplectic toric case, i.e., toric origami manifolds with empty fold. We first summarize the set-up and notations. 

Let $X$ be a $2n$-dimensional symplectic manifold equipped with a Hamiltonian torus action of an $n$-dimensional torus $G$. We assume that there exists a $G$-equivariant pre-quantizing line bundle $L_X\to X$.  Let $\mu_X:X\to {\mathfrak g}^*={\rm Lie}(G)^*$ and $\Delta_X=\mu_X(X)$ be the corresponding moment map and the Delzant polytope.   We take and fix an $m$-dimensional face $\Delta'$ of $\Delta_X$ and a point $\xi$ in the relative interior ${\rm int}(\Delta')$. Let $F:=\mu_X^{-1}(\xi)$ be the $m$-dimensional isotropic torus in $X$ and $X':=\mu_X^{-1}(\Delta')$ be the $2m$-dimensional symplectic submanifold of $X$. We take and fix a point $x\in F\subset X'$. Let $H$ be the stabilizer subgroup at $x$ with respect to $G$-action and $H^{\perp}$ the complementary orthogonal subtorus of $H$ in $G$ with respect to a rational metric of $\g$. Note that $H$ (resp. $H^{\perp}$) is  an $n-m$-dimensional (resp. $m$-dimensional) subtorus of $G$. We denote the inclusion map of Lie-algebra and its dual by $\iota_{H}:{\rm Lie}(H)=\h\to \g$ and  $\iota_{H}^*:\g^*\to\h^*$ respectively.   



We first give following comments. 
\begin{itemize}
\item Since the computation is purely local, we do not need the compactness of $\Delta_X$.  In fact we only use a part of the Delzant condition near $\xi$.    
\item We fix a $G$-invariant $\omega$-compatible almost complex structure on $X$ so that it also induces a $G$-invariant $\omega$-compatible almost complex structure on the inverse image of each face of $\Delta_X$. 
%\item We use the Riemannian metric on $X$ defined by the almost complex structure.
\item $F$ is a Lagrangian torus in the symplectic submanifold $X'$. 
\item $F$ can be described as the orbit $F=G\cdot x=H^{\perp}\cdot x$. 
\item The intersection $H\cap H^{\perp}$ is a finite Abelian group. 
\item Since $x$ is a fixed point with respect the $H$-action, the moment map image $(\iota_H^*\circ\mu)(x)=\iota_H^*(\xi)$ of $x$ with respect to the $H$-action is an element in the weight lattice $\h_{\Z}^*$. 
\item The argument below still holds when there exists a finite subgroup of $G$ which acts trivially on $X$. In fact in the proof of Lemma~\ref{locinddisc} we deal with the symplectic toric manifold $X_1$ for which such a subgroup $H\cap H_1\cap H_1^{\perp}$ may exist. 
\end{itemize}

If $Y$ is a smooth manifold and $Y'$ is its smooth submanifold, then we denote the normal bundle of $Y'$ in $Y$ by $\nu_Y(Y')$. We also denote the fiber at $y\in Y'$ by $\nu_Y(Y')_y$.  
%Let $\nu_{X}(F)$ be the normal bundle of $F$ in $X$ and $\nu_{X}(F)_x$ its fiber at $x$. 
There exists a $G$-invariant tubular neighbourhood $N_F$ of $F$ and $G$-equivariant diffeomorphism 
$$
N_F\cong(\nu_X(F)_x\times G)/H
=(\nu_X(F)_x\times H^{\perp})/{H\cap H^{\perp}}, 
$$where we use the $G$-action on the right hand side through the identification 
$G=H\cdot H^{\perp}=(H\times H^{\perp})/H\cap H^{\perp}$ arising from the exact sequence 
\[
\begin{split}
H\cap H^{\perp}\to H\times H^{\perp}\to H\cdot H^{\perp}=G \\ 
h\mapsto (h,h^{-1}), (h_1, h_2)\mapsto h_1h_2. 
\end{split}
\]
Since $F$ is a Lagrangian torus in $X'$ we have 
$$
\nu_X(F)_x\times H^{\perp}=
\nu_X(X')_x\times\nu_{X'}(F)_x\times H^{\perp}
=\nu_X(X')_x\times T^*_x(H^{\perp}\cdot x)\times H^{\perp}
=\nu_X(X')_x\times T^*H^{\perp},
$$ and hence, we have a $G$-equivariant isomorphism 
\begin{equation}\label{nbdofF}
N_F\cong (\nu_X(X')_x\times T^*H^{\perp})/H\cap H^{\perp}. 
\end{equation}


Now we describe the restriction $L_X|_{N_F}$. We first define an $H$-equivariant line bundle $L_1:=\nu_X(X')_x\times L_X|_x\to \nu_X(X')_x$, where we regarded $L_X|_x$ as a representation of $H$. Note that $\nu_X(X')_x$ has a natural symplectic structure and $L_1$ is equipped with a structure of pre-quantizing line bundle with respect to the symplectic structure. 
Let $L_2$ be the pull-back of $L_X|_{N_F}$ with respect to the natural map 
$T^*H^{\perp}\to (\nu_X(X')_x\times T^*H^{\perp})/H\cap H^{\perp}$, which is an $H\times H^{\perp}$-equivariant line bundle over $T^*H^{\perp}$. Note that though $H$-action on $T^*H^{\perp}$ is trivial, the action on $L_2$ is non-trivial in general.  We define an $H^{\perp}$-equivariant line bundle $\hat L_2\to T^*H^{\perp}$ by $\hat L_2:={\rm Hom}(L_X|_x,L_2)$. Then $\hat L_2$ is isomorphic to $L_2$ as $H^{\perp}$-equivariant line bundle and the induced $H$-action on $\hat L_2$ is trivial. We have two line bundles with connection $(L_1\boxtimes \hat L_2)/H\cap H^{\perp}$ and $L_X|_{N_F}$ over $(\nu_X(X')_x\times T^*H^{\perp})/H\cap H^{\perp}=N_F$. The restrictions of these two line bundles to the zero-section $F$ in $N_F$ are isomorphic to each other as line bundles with connection. The Darboux type theorem (\cite[Proposition~7.11]{Fujita-Furuta-Yoshida3}) implies that the $G$-equivariant isomorphism can be extended to a $G$-invariant neighbourhood of $F$. 


\begin{remark}
Strictly speaking we have to consider the data on sufficiently small neighbourhoods of the origin in $\nu_X(X')_x$ and the zero section $H^{\perp}$ in $T^*H^{\perp}$ as a  Lagrangian torus to consider the above isomorphisms and  the local indices  in the subsequent argument, though, we use the same notations $\nu_X(X')$ and $T^*H^{\perp}$ to simplify the notations. 
\end{remark}

Let $\Delta_1, \ldots, \Delta_{n-m}$ be codimension one faces of $\Delta_X$ such that $\Delta'$ is the intersection of them,  $\Delta'=\Delta_1\cap \cdots \cap \Delta_{n-m}$. For each $l=1,2,\ldots, n-m$, let $H_l$ be the circle subgroup of $H$ which acts trivially on the symplectic submanifold $X_l:=\mu_X^{-1}(\Delta_l)$ and $H_l^{\perp}$ the orthogonal complement of $H_l$. If we choose any members $\Delta_{l_1}, \ldots, \Delta_{l_\alpha}$, then we have a locally free action of the intersection $H_{l_1}^{\perp}\cap \cdots \cap H_{l_\alpha}^{\perp}=(H_{l_1}\cdot\cdots \cdot	H_{l\alpha})^{\perp}$ on a small neighbourhood of the inverse image of the complement of a neighbourhood of the boundary $\partial(\Delta_{l_1}\cap\cdots\cap\Delta_{l_\alpha})$ in $\Delta_{l_1}\cap\cdots\cap\Delta_{l_\alpha}$. Such a family of torus actions determines a good compatible fibration as in Section~\ref{Good compatible fibration on toric origami manifolds}.  For each $H_{l}$ we have the decomposition $H_l^{\perp}=(H\cap H_l^{\perp})\cdot H^{\perp}$.
On the other hand there exists a natural action of the product $H\times H^{\perp}$ on $N_F$ under the identification (\ref{nbdofF}). 
Then the above good compatible fibration is induced from the action of the subgroup $(H\cap H^{\perp}_l)\times H^{\perp}$ in $H\times H^{\perp}$. 

The $G$-equivariant local index $\ind_G(N_F,N_F\setminus F)$ is defined by using these structures and it is equal to the $H\cap H^{\perp}$-invariant part of the $H\times H^{\perp}$-equivariant local index $\ind_{H\times H^{\perp}}(\nu_X(X')_x\times T^*H^{\perp}, \nu_X(X')_x\times T^*H^{\perp}\setminus \{0\}\times H^{\perp})$. 
%, where we consider the zero-section $H^{\perp}\subset T^*H^{\perp}$. 
%Hereafter we regard an element of the integral weight lattice of a Lie group $G$ as a representation of $G$. Moreover if $\rho$ is a representation of $G$, then  we use the same notation $\rho$ for the corresponding element of the character ring  $R(G)$. 
%Let $\iota_{H}:{\rm Lie}(H)=\h\to \Tt$ be the inclusion and $\iota^*_{H}$ its dual. Note that since $x$ is a fixed point with respect the $H$-action the moment map image $(\iota_H^*\circ\mu)(x)=\iota_H^*(\xi)$ of $x$ with respect to the $H$-action is an element in the weight lattice $\h_{\Z}^*$, and it corresponds to the representation $L_X|_x$ of $H$ by the Kostant formula. 
%
For simplicity we use the following type of notations for the equivariant local indices: 
$$
{RR}_H(\nu_X(X')_x):=\ind_{H}(\nu_X(X')_x, \nu_X(X')_x\setminus \{0\})
$$and 
$$
RR_{H^{\perp}}(T^*H^{\perp}):=\ind_{H^{\perp}}(T^*H^{\perp}, T^*H^{\perp}\setminus H^{\perp}). 
$$

\begin{lemma}\label{locinddisc}
${RR}_H(\nu_X(X')_x)=\C_{(\iota_H^*(\xi))}=L_X|_x\in R(H)$.%  where $\iota_{H}^*$ is the dual of the inclusion of $\h$ into $\g$ and we regard $\iota_H^*(\xi)$ as a representation . 
\end{lemma}
\begin{proof}
The second equality follows from the property of the moment map and the Kostant formula. 
We show the first equality by induction on  $n-m=\dim(\nu_X(X')_x)/2$. If $n-m=1$, then the equality follows from the direct computation. See \cite[Example~2.3]{yoshidasymplecticcut} for example. Suppose that $n-m$ is grater than 1 and the statement holds for any situation with codimension $n-m-1$. We consider the decomposition $\nu_X(X')=\nu_{X}(X_1)\oplus\nu_{X_1}(X')$ and $H=H_1\cdot(H\cap H_1^{\perp})$. According to the decomposition the $H$-action on $\nu_X(X')$ factors the action of the product of $H_1$-action on $\nu_X(X_1)$ and $H\cap H_1^{\perp}$-action on $\nu_{X_1}(X')$. By Proposition~\ref{cobandprod} we have that $RR_{H}(\nu_X(X')_x)$ is equal to the $H_1\cap(H\cap H_1^{\perp})$-invariant part of the product $RR_{H_1}(\nu_X(X_1)_x)\otimes RR_{H\cap H_1^{\perp}}(\nu_{X_1}(X')_x)$. Note that $H_1^{\perp}$-action on $X_1$ gives a structure of a symplectic toric manifold whose momentum polytope is $\iota^*_{H^{\perp}_1}(\Delta_1)$. By the assumption of the induction we have $RR_{H_1^{\perp}}(\nu_{X_1}(X')_x)=\C_{(\iota_{H_1^{\perp}}^*(\xi))}$. 
By considering the subgroup $H\cap H_1^{\perp}$ we have $RR_{H\cap H_1^{\perp}}(\nu_{X_1}(X')_x)=\C_{(\iota_{H\cap H_1^{\perp}}^*(\xi))}$, and hence,  
%gives a symplectic toric manifold structure whose moment map image is $\Delta_1$ and $\Delta'$ is its codimension $n-m-1$ face in $\Delta_1$, the assumption of the induction implies that 
$$
RR_{H_1}(\nu_X(X_1)_x)\otimes RR_{H\cap H_1^{\perp}}(\nu_{X_1}(X')_x)=\C_{(\iota_{H_1}^*(\xi))}\otimes\C_{(\iota_{H\cap H_1^{\perp}}^*(\xi))}=\C_{(\iota_{H_1}^*(\xi)\oplus\iota_{H\cap H_1^{\perp}}^*(\xi))}. 
$$Note that under the natural isomorphism ${\rm Lie}(H_1)^*\oplus {\rm Lie}(H\cap H_1^{\perp})^*  \cong \h^*$ we have $\iota_{H_1}^*(\xi)\oplus\iota_{H\cap H_1^{\perp}}^*(\xi)=\iota_H^*(\xi)$. As we noted in the beginning of this section $\iota_H^*(\xi)$ is an element of the weight lattice $\h_{\Z}^*$, the $H_1\times (H\cap H_1^{\perp})$-representation $\C_{\iota_{H_1}^*(\xi)\oplus\iota_{H\cap H_1^{\perp}}^*(\xi)}$ induces an $H$-representation $\C_{(\iota_H^*(\xi))}$, and hence,  it implies that $RR_{H_1}(\nu_X(X_1)_x)\otimes RR_{H\cap H_1^{\perp}}(\nu_{X_1}(X')_x)$ decsends to an $H$-representation. 
In particular 
the index $RR_{H_1}(\nu_X(X_1)_x)\otimes RR_{H\cap H_1^{\perp}}(\nu_{X_1}(X')_x)$ is $H_1\cap H\cap H_1^{\perp}$-invariant, and we complete the proof. 
\end{proof}




Let $\iota_{H^{\perp}}:\h^{\perp}\to \g$ be the inclusion and $\iota^*_{H^{\perp}}$ its dual. We may assume that the moment map image $(\iota_{H^{\perp}}^*\circ\mu)(x)=\iota_{H^{\perp}}^*(\xi)$ of $x$ with respect to the $H^{\perp}$-action is an element in the weight lattice $(\h^{\perp})_{\Z}^*$. Otherwise the compatible system on $T^*H$ is acyclic, and hence, the local index $RR_{H^{\perp}}(T^*H^{\perp})$ is zero. 

\begin{lemma}\label{locindcylinder}
%$\ind_{H^{\perp}}(T^*H^{\perp}, T^*H^{\perp}\setminus H^{\perp})
$RR_{H^{\perp}}(T^*H^{\perp})=\C_{(\iota_{H^{\perp}}^*(\xi))}\in R(H^{\perp})$. 
\end{lemma}
\begin{proof}
Since the $H^{\perp}$-action on $T^*H^{\perp}$ is free, the induced good compatible fibration(system) on $TH^{\perp}$ consists of two open subsets, a small open neighbourhood of the zero-section $H^{\perp}$ and its complement. On the other hand by fixing a decomposition $H^{\perp}=(S^1)^{m}$, we have a product structure of compatible fibration and compatible system, where the $S^1$-equivariant data is determined by the inclusion $\iota_i : S^1\hookrightarrow (S^1)^{m}=H^{\perp}$ to the $i$th fator for $i=1,\cdots, m$. 
By applying Proposition~\ref{cobandprod} the local index $RR_{H^{\perp}}(T^*H^{\perp})$ is equal to the product of $RR_{S^1}(T^*S^1)$ defined the structure induced form $\iota_i$'s. Then the lemma follows from the computation of $RR_{S^1}(T^*S^1)$ (See \cite[Proposition~5.3]{Fujita-Furuta-Yoshida3} for example.). 
\end{proof}

Together with the product formula, Lemma~\ref{locinddisc} and Lemma~\ref{locindcylinder} imply the following. 
\begin{prop}\label{product HHperp}
We have the equality 
$$
RR_{H\times H^{\perp}}(\nu_X(X')_x\times T^*H^{\perp})=\C_{(\iota_H^*(\xi)\oplus\iota_{H^{\perp}}^*(\xi))}\in R(H\times H^{\perp}).  
$$ 
%as representations of $H\times H^{\perp}$.  
\end{prop}

\begin{theorem}\label{toriclocalcontribution}
$\ind_G(N_F,N_F\setminus F)\neq 0$ if and only if $\xi\in\g_{\Z}^*$, 
and if $\xi\in \g_{\Z}^*$, then we have $\ind_G(N_F,N_F\setminus F)=\C_{(\xi)}$. 
\end{theorem}
\begin{proof}
As we explained, $\ind_G(N_F,N_F\setminus F)$ is equal to the $H\cap H^{\perp}$-invariant part of $RR_{H\times H^{\perp}}(\nu_X(X')_x\times T^*H^{\perp})$ which is represented by a one-dimensional representation of $H\times H^{\perp}$. Suppose that the invariant part is non-zero. Then the one-dimensional representation $RR_{H\times H^{\perp}}(\nu_X(X')_x\times T^*H^{\perp})$ descends to a representation of $G$. Since $\h^*\oplus\h^{{\perp}*}$ is isomorphic to $\g^*$ by $\iota_H^*\oplus\iota_{H^{\perp}}^*$, the invariant part is equal to the point $\iota_H^*(\xi)\oplus\iota_{H^{\perp}}^*(\xi)=\xi\in \g_\Z^*$ by Proposition~\ref{product HHperp}. Conversely if $\xi\in \g_{\Z}^*$, then $RR_{H\times H^{\perp}}(\nu_X(X')_x\times T^*H^{\perp})$ represents a point in $\g_{\Z}^*$, and hence, a representation of $G$. In particular we have $\ind_G(N_F,N_F\setminus F)\neq 0$ as the $H\cap H^{\perp}$-invariant part. 
\end{proof}

\begin{definition}
%Let $X$ be a symplectic manifold with an Hamiltonian action of a torus $T$. Suppose that there exists an $T$-equivariant pre-quantizing line bundle $(L,\nabla)$. 
A $G$-orbit $F$ is called a {\it Bohr-Sommerfeld orbit} ({\it BS-orbit} for short) if there exists a non-trivial global parallel section on the restriction $(L_X,\nabla)|_F$. 
\end{definition}

\begin{prop}
A $G$-orbit $F$ is BS-orbit if and only if $\ind_G(N_F,N_F\setminus F)\neq 0$. 
\end{prop}	
\begin{proof}
We fix the decomposition $H^{\perp}=(S^1)^{m}$ as in the proof of Lemma~\ref{locindcylinder}. The computation in \cite[Remark~6.10]{Fujita-Furuta-Yoshida1} says that $RR_{S^1}(T^*S^1)$ is isomorphic to the space of parallel sectoins $\Gamma^{\rm par}(S^1, \iota_i^*\hat L_2|_{S^1})$ for each $i=1,\cdots,m$. By the product structure of $(TH^{\perp}, \hat L_2)$ near the zero section, the space of parallel sections $\Gamma^{\rm par}(H^{\perp}, \hat L_2|_{H^{\perp}})$ is generated by a constant section and isomorphic to the product of $\Gamma^{\rm par}(S^1, \iota_i^*\hat L_2|_{S^1})\cong RR_{S^1}(T^*S^1)$. It implies that $\Gamma^{\rm par}(H^{\perp}, \hat L_2|_{H^{\perp}})$ is isomorphic to $RR_{H^{\perp}}(T^*H^{\perp})$ as $H^{\perp}$-representation. 
Similarly by considering the restriction to the origin, we have that the one-dimensional representation $RR_H(\nu_X(X')_x)$ is isomorphic to $L_X|_x$ as $H$-representation. Then we have that $RR_H(\nu_X(X')_x)\otimes RR_{H^{\perp}}(T^*H^{\perp})$ is isomorphic to $\Gamma^{\rm par}(\{0\}\times H^{\perp}, L_X|_x\otimes \hat L_2|_{H^{\perp}})$ as $H\times H^{\perp}$-representation. 

If $F$ is a BS-orbit, then there exists a non-trivial global parallel section $s_F:F\to L|_F$. By considering the pull-back we have a non-trivial global parallel section $\tilde s_F:H^{\perp}\to L_X|_x\otimes \hat L_2|_{H^{\perp}}$, which is $H\cap H^{\perp}$-invariant, and hence,  
%Since there exists an $H\cap H^{\perp}$-invariant section $\tilde s_F$ on $H^{\perp}=H^{\perp}\times\{0\}$ we have 
it implies $\ind_G(N_F,N_F\setminus F)\neq 0$. 

Conversely suppose that $\ind_G(N_F,N_F\setminus F)\neq 0$. Then by the isomorphism $RR_H(\nu_X(X')_x)\otimes RR_{H^{\perp}}(T^*H^{\perp}) \cong\Gamma^{\rm par}(\{0\}\times H^{\perp}, L_X|_x\otimes \hat L_2|_{H^{\perp}})$ there exists an $H\cap H^{\perp}$-invariant non-trivial global parallel sections of $L_X|_x\otimes\hat L_2|_{H^{\perp}}$. It induces a non-trivial global parallel section $s_F$ of $L_X|_F$ by the natural map $H^{\perp}\to H^{\perp}\cdot x=F$. 
\end{proof}

When we consider the situation in Section~\ref{Localization formula and Danilov type formula} we have 
$$
\sum_{\xi\in\Delta_i}\ind_T(N_F, N_F\setminus F)=\sum_{j,k}\ind_T(V_{i,k}^{(j)+}, V_{i,k}^{(j)+}\cap V). 
$$
As a particular case we have a proof of Danilov's theorem for symplectic toric manifolds. 
\begin{theorem}\label{Danilov}
If $X$ is a closed symplectic toric manifold with pre-quantizing line bundle $L$, then we have the following equality of the $G$-equivariant Riemann-Roch number. 
$$
RR_G(X,L)=\sum_{\xi\in (\Delta_{X})_{\Z}}\C_{(\xi)}. 
$$
\end{theorem}

\subsection{Contribution from the fold} 
In the subsequent subsections we consider the toric origami case as in Theorem~\ref{origamiDanilov}. In this subsection we compute the contribution from the folded part, $\ind_T(\U',\U'\setminus Z)$. 
\begin{theorem}\label{foldind=0}
We have 
$$
\ind_T(\U',\U'\setminus Z)=0
$$as a $T$-equivariant index. 
\end{theorem}
\begin{proof}
As it is showed in Proposition~\ref{product structure of Cl} and by definition of $D_{Z}$, the acyclic compatible system on $\U'\setminus Z$ has a natural product structure between them on $B$ and $S^1\times(-\vep/2,\vep/2)$, and hence, its local index $\ind_T(\U',\U'\setminus Z)$ is equal to the product of them in the sense of the product formula \cite[Theorem~8.8]{Fujita-Furuta-Yoshida2}. On the other hand the compatible system on $S^1\times (-\vep/2,\vep/2)$ is the one associated with the natural folded structure on it, and it will be shown in Appendix~\ref{A computation of local index of the folded cylinder} that its local index is equal to $0$. See Proposition~\ref{vanishingoffoldedcylinder}. These facts imply $\ind_T(\U',\U'\setminus Z)=0$. 
\end{proof}





\subsection{Contribution from the positive unfolded part}
We compute the contribution from the unfolded part of the positive orientation, $\ind_T(V_{i,k}^{(j)+}, V_{i,k}^{(j)+}\setminus (\mu_{i,k}^{(j)+})^{-1}((\g_{i,k}^{(j)*})_{\Z}))$. Since $V_{i,k}^{(j)+}$ is away from the fold $Z$, the local situation is same as that for the genuine toric case, and hence,  we can apply Theorem~\ref{toriclocalcontribution}. 

\begin{theorem}\label{positive contribution}
We may choose $\tilde\Delta_{i,k}^{(j)+}$ small enough so that $\tilde\Delta_{i,k}^{(j)+}\cap\Tt_{\Z}^*={\rm int}\Delta_{i,k}^{(j)+}\cap\Tt_{\Z}^*$. 
%there exists the unique integral point $\xi \in \Tt_{\Z}^*$ such that $\iota_{i,k}^{(j)*}(\xi)=\mu_{i,k}^{(j)+}(x)\in (\g_{i,k}^{(j)*})_{\Z}$ for some $x\in M_{i,k}^{(j)+}$. 
Then we have 
$$
\ind_T(V_{i,k}^{(j)+}, V_{i,k}^{(j)+}\setminus (\mu_{i,k}^{(j)+})^{-1}((\g_{i,k}^{(j)*})_{\Z}))=\sum_{\xi\in{\rm int}\Delta_{i,k}^{(j)+}\cap\Tt_{\Z}^*}\C_{(\xi)}. 
$$
\end{theorem}
\begin{proof}
Since the compatible system $\{D_{Z}, D_{i,k}^{(j)}\}_{i,j,k}$ is acyclic on $V$, the complement of the inverse images of lattice points, the excision formula implies that the $T$-equivariant local index $\ind_T(V_{i,k}^{(j)+}, V_{i,k}^{(j)+}\setminus (\mu_{i,k}^{(j)+})^{-1}((\g_{i,k}^{(j)*})_{\Z}))$ is equal to the sum of contributions of the inverse image of the lattice point which is contained in $V_{i,k}^{(j)+}$. Each inverse image has a neighborhood of the form $N_F$ as in Subsection~\ref{Toric case}, and hence, the contribution of the lattice point $\xi$ is the representation corresponding to the lattice point $\C_{(\xi)}$. 
\end{proof}


\subsection{Contribution from the negative unfolded part}
We compute the contribution from the unfolded part of the negative orientation, $\ind_T(V_{i,k}^{(j)-}, V_{i,k}^{(j)-}\setminus (\mu_{i,k}^{(j)-})^{-1}((\g_{i,k}^{(j)*})_{\Z}))$. The situation is same as that for the positive unfolded part up to the orientation. 
%Since $V_{i,k}^{(j)-}$ is away from the fold $Z$, the local situation is same as that for the genuine toric case up to orientation. 
The difference appears only in the $\Z/2$-grading of the Clifford module bundle. Namely the $\Z/2$-grading in the negative case is opposite to the positive case, and hence, the resulting index has the opposite sign. The proof of the following theorem can be shown by the similar way for the proof of Theorem~\ref{positive contribution}. 

\begin{theorem}\label{negative contribution}
We may choose $\tilde\Delta_{i,k}^{(j)-}$ small enough so that $\tilde\Delta_{i,k}^{(j)-}\cap\Tt_{\Z}^*={\rm int}\Delta_{i,k}^{(j)-}\cap\Tt_{\Z}^*$. 
%there exists the unique integral point $\xi \in \Tt_{\Z}^*$ such that $\iota_{i,k}^{(j)*}(\xi)=\mu_{i,k}^{(j)+}(x)\in (\g_{i,k}^{(j)*})_{\Z}$ for some $x\in M_{i,k}^{(j)+}$. 
Then we have 
$$
\ind_T(V_{i,k}^{(j)-}, V_{i,k}^{(j)-}\setminus (\mu_{i,k}^{(j)-})^{-1}((\g_{i,k}^{(j)*})_{\Z}))=-\sum_{\xi\in{\rm int}\Delta_{i,k}^{(j)-}\cap\Tt_{\Z}^*}\C_{(\xi)}. 
$$
\end{theorem}


\subsection{Contribution from the crack}\label{Contribution from the crack}
We compute the contribution from the crack, $\ind_T(V_{i,k}^{Z}, V_{i,k}^{Z}\cap V)=\ind_T(V_{i,k}^{Z}, V_{i,k}^{Z}\setminus C_{i,k}^Z)$, and show that it is equal to 0. Note that each $V_{i,k}^Z$ has two components $V_{i,k}^Z\cap M^+$ and $V_{i,k}^Z\cap M^-$.  Then the open subsets $V_{i,k}^Z\cap M^+$ and $V_{i,k}^Z\cap M^-$ are isomorphic to each other as open symplectic toric manifolds up to their orientations. 
%It implies that $\ind_T(V_{i,k}^{Z}, V_{i,k}^{Z}\setminus C_{i,k}^Z)=\ind_T(V_{i',k'}^{Z}\cap M^+, V_{i',k'}^{Z}\cap M^+\setminus C_{i',k'}^Z)+\ind_T(V_{i',k'}^{Z}\cap M^-, V_{i',k'}^{Z}\cap M^-\setminus C_{i',k'}^Z)$. In particular we have the following.  
\begin{theorem}\label{crackind=0} 
We have the equality 
$$
\ind_T(V_{i,k}^{Z}, V_{i,k}^{Z}\setminus C_{i,k}^Z)=0
$$ as $T$-equivariant indices for each $i$ and $k$.  
\end{theorem}
\begin{proof}
Since $V_{i,k}^Z\cap M^+$ and $V_{i,k}^Z\cap M^-$ are isomorphic up to their orientations we have 
$$
\ind_T(V_{i,k}^{Z}, V_{i,k}^{Z}\setminus C_{i,k}^Z)=\ind_T(V_{i,k}^{Z}\cap M^+, V_{i,k}^{Z}\cap M^+\setminus C_{i,k}^{Z})+\ind_T(V_{i',k'}^{Z}\cap M^-, V_{i,k}^{Z}\cap M^-\setminus C_{i,k}^{Z})=0. 
$$
\end{proof}


%%%%%%%%%%%%%%%%%%%%%%%%%%%%%%%%%%%%%%%%%%%%%%%%%%%%%%%%%%%%%%%
\appendix

%%%%%%%%%%%%%%%%%%%%%%%%%%%%%%%%%%%%%%%%%%%%%%%%%%%%%%%%%%%%%%%%%%%
\section{Acyclic compatible systems and their local indices}
\label{Acyclic compatible systems and their local indices} 
In this appendix we give a brief summary of some definitions 
of compatible fibration, acyclic compatible system and 
their local indices following \cite{Fujita-Furuta-Yoshida2, Fujita-Furuta-Yoshida3} and \cite{Fujitacobinv}. 
We adopt combinations of definitions in \cite{Fujita-Furuta-Yoshida2} and \cite{Fujita-Furuta-Yoshida3}. 
%\subsection{Compatible fibration}
Let $V$ be a smooth  manifold. 
%{\color{red}
\begin{definition}\label{compatible fibration}
A {\it compatible fibration on $V$} is a collection of the data 
$\{V_{\alpha}, {\SF}_{\alpha}\}_{\alpha\in A}$ consisting of 
an open covering $\{V_{\alpha}\}_{\alpha\in A}$ of $V$ and 
a foliation $\SF_{\alpha}$ on $V_{\alpha}$ with compact leaves 
which  satisfies the following properties.
\begin{enumerate}
\item The holonomy group of each leaf of $\SF_\alpha$ is finite. 
\item\label{correspondence between foliation and pi_alpha}For each $\alpha$ and $\beta$, if a leaf $L\in \SF_\alpha$ has non-empty intersection $L\cap V_\beta\neq \emptyset$, then, $L\subset V_\beta$. 
%\item\label{foliation on overlap}For each $\alpha$ and $\beta$, the set 
%\[
%\SF_{\alpha\beta}=\{ L_\alpha\cap L_\beta \mid L_\alpha\in \SF_\alpha ,\ L_\beta \in \SF_\beta \} 
%\]
%of the intersections of leaves of $\SF_\alpha$ and $\SF_\beta$ is a foliation 
%on $V_{\alpha}\cap V_{\beta}$. 
%%\item {\color{red}The holonomy group of each leaf of $\SF_{\alpha\beta}$ is finite.} 
\end{enumerate}
\end{definition}
%}

%Let $\{ V_\alpha ,\SF_\alpha\}_{\alpha \in A}$ be a compatible fibration on $V$. 

\begin{definition}\label{goodcompatifib}
A compatible fibration $\{V_{\alpha}, {\SF}_{\alpha}\}_{\alpha\in A}$ on $V$ is called {\it good} if for all $\alpha$ and $\beta$ with $V_{\alpha}\cap V_{\beta}\neq \emptyset$ the following condition (i) or (ii) holds. 
\begin{itemize}
\item[(i)] For each leaf $L_{\alpha}\in {\SF}_{\alpha}$, there exists a leaf $L_{\beta}\in{\SF}_{\beta}$ such that $L_{\alpha}\subset L_{\beta}$. 
\item[(ii)] For each leaf $L_{\beta}\in {\SF}_{\beta}$, there exists a leaf $L_{\alpha}\in{\SF}_{\alpha}$ such that $L_{\beta}\subset L_{\alpha}$. 
\end{itemize}
\end{definition}

%%%%%%%%%%%%%%%%%%%%%%%%%%%%%%%
%\subsection{Compatible system and its acyclicity}\label{Compatible system and its acyclicity}
Let $(V,g)$ be a Riemannian manifold, $W$ a $Cl(TV)$-module bundle over $V$. Suppose that $V$ is equipped with a compatible fibration $\{V_{\alpha}, {\SF}_{\alpha}\}_{\alpha\in A}$. 
We impose the following conditions on the Riemannian metric $g$. 
\begin{assumption}\label{assumption for Riemannian metric}
%\begin{itemize}
%\item 
Let $\nu_\alpha=\{ u\in TV_\alpha \mid g(u,v)=0\ \text{for all }v\in T\SF_\alpha \}$ be the normal bundle of $\SF_\alpha$. Then, $g|_{\nu_\alpha}$ is invariant under holonomy, and gives a transverse invariant metric on $\nu_\alpha$.
%\item On each $V_\alpha\cap V_\beta\neq \emptyset$, for $i=\alpha$, $\beta$, let $\nu_{\alpha\beta}^i$ be the normal bundle of $\SF_{\alpha\beta}$ in $T\SF_i|_{V_\alpha\cap V_\beta}$ which is defined by 
%\[
%\nu_{\alpha\beta}^i=\{ u\in T\SF_i|_{V_\alpha\cap V_\beta}\mid g(u,v)=0\ \text{for all }v\in T\SF_{\alpha\beta} \}
%\]
%Then, $g|_{\nu_{\alpha\beta}^i}$ gives a transverse invariant metric on $\nu_{\alpha\beta}^i$. 
%\item $\nu_{\alpha\beta}^\alpha$ and $\nu_{\alpha\beta}^\beta$ are perpendicular to each other with respect to $g|_{V_\alpha\cap V_\beta}$. 
%\end{itemize}
\end{assumption}
%\begin{remark}
%The above Riemannian metric is an orbifold version of a compatible Riemannian metric which is actually used in \cite{Fujita-Furuta-Yoshida2}. 
%\end{remark}

\begin{definition}\label{compatible system}
A {\it compatible system} on $(\{V_{\alpha}, \SF_{\alpha}\}, W)$ is a data $\{ D_{\alpha}\}_{\alpha \in A}$ satisfying the following properties. 
\begin{enumerate}
\item $D_{\alpha}\colon \Gamma (W|_{V_{\alpha}})\to \Gamma (W|_{V_{\alpha}})$ is an order-one formally self-adjoint differential operator.
\item $D_{\alpha}$ contains only the derivatives along leaves of $\SF_{\alpha}$. 
\item $D_{\alpha}$ is a Dirac-type operator along leaves. 
Namely 
the principal symbol of $D_{\alpha}$ is given by the composition of 
the dual of the natural inclusion $\iota_{\alpha}\colon T\SF_{\alpha}\to TV_{\alpha}$ and the Clifford multiplication 
$c\colon T^*\SF_{\alpha}\cong T\SF_{\alpha}\subset TV_{\alpha} \to \End (W|_{V_{\alpha}})$ . 
%If $W$ is $\Z/2$-graded, then $D_{\alpha}$ is of degree-one. 
%$\sigma (D_{\alpha})=c\circ p_{\alpha}\circ \iota_{\alpha}^*\colon T^*V_{\alpha}\to \End (W|_{V_{\alpha}})$, where  is  from the tangent bundle along leaves of $\SF_\alpha$ to $TV_\alpha$, $p_{\alpha}\colon T^*\SF_{\alpha}\to T\SF_{\alpha}$ is the isomorphism induced by the Riemannian metric and is the  
\item For a leaf $L\in \SF_\alpha$ let $\tilde u\in \Gamma (\nu_\alpha|_L)$ be a section of $\nu_\alpha|_L$ parallel along $L$. 
%For $b\in U_{\alpha}$ and $u\in T_bU_{\alpha}$, let $\tilde{u}\in \Gamma(TV_{\alpha}|_{\pi^{-1}_{\alpha}(b)})$ be the horizontal lift of $u$ with respect to the Riemannian metric $g|_{V_\alpha}$. 
$\tilde{u}$ acts on $W|_L$ by the Clifford multiplication $c(\tilde{u})$. Then $D_{\alpha}$ and $c(\tilde{u})$ anti-commute each other, i.e. 
\[
0=\{ D_{\alpha},c(\tilde{u}) \}:=
D_{\alpha}\circ c(\tilde{u})+c(\tilde{u})\circ D_{\alpha}
\]
as an operator on $W|_L$. 
%\item If $V_{\alpha}\cap V_{\beta}\neq \emptyset$, then the anti-commutator $\{D_{\alpha},D_{\beta}\}:=D_{\alpha}\circ D_{\beta}+D_{\beta}\circ D_{\alpha}$ is a differential operator along leaves of $\SF_{\alpha\beta}$ of order at most two. 
%\item $D_{\alpha}$ anti-commutes with $Cl_{p,q} (\subset Cl(TM\oplus \R^{p,q}))$-action. 
\end{enumerate}
\end{definition}



As in \cite[Lemma~3.4]{Fujita-Furuta-Yoshida2} for each leaf $L\in\SF_{\alpha}$ 
we have a small 
open tubular neighbourhood $V_L$ of $L$ and the 
finite covering  $q_L:\tilde V_L\to V_L$ such that the induced foliation on 
$\tilde V_L$ is a bundle foliation with the projection $\pi_L:\tilde V_L\to \tilde U_L$. 


\begin{definition}\label{strongly acyclic}
A compatible system $\{ D_\alpha\}_{\alpha \in A}$ on $(\{V_{\alpha}, \SF_{\alpha}\}, W)$ is said to be {\it acyclic} if %there exists a collection of data $\{q_{L_\alpha}:\tilde V_{L_\alpha}\to V_{L_\alpha}, \ \pi_{L_\alpha}:\tilde V_{L_\alpha}\to \tilde U_{L_\alpha} \ | \ \alpha\in A, L_{\alpha}\in \SF_{\alpha}\}$ 
it satisfies the following conditions.  
\begin{enumerate}
\item The Dirac-type operator 
$q^*_{L}D_{\alpha}|_{\pi_L^{-1}(\tilde b)}$ has zero kernel 
for each $\alpha\in A$, leaf $L\in \SF_{\alpha}$ and 
$\tilde b\in \tilde U_{L}$. 
\item If $V_\alpha\cap V_\beta\neq \emptyset$, then the anti-commutator 
$\{D_{\alpha},D_{\beta}\}:=D_{\alpha}D_{\beta}+D_{\beta}D_{\alpha}$ is a non-negative operator on $V_\alpha\cap V_\beta$. %$W|_{L_{\alpha\beta}}$ for each $L_{\alpha\beta}\in \SF_{\alpha\beta}$. 
\end{enumerate}
\end{definition}

As in \cite[Section~5]{Fujita-Furuta-Yoshida2} we can construct such structures, good compatible fibration and compatible system, on Hamiltonian torus manifolds. Though the good compatible fibrations form a nice class, we have to generalize it to treat the product of such structures. 

%%%%%%%%%%%%%%%%%%%%%%%%%%%%%%%%%%
%\subsection{Definition of the local index}\label{ind(M,V,W)}
%In \cite{Fujita-Furuta-Yoshida2} Furuta, Yoshida and the author showed that when a manifold whose end is equipped with an acyclic compatible system with some technical conditions, the {\it local index} can be defined. 
%In this section we first introduce a class of compatible fibration which gives a sufficient condition to define the {local index}.

\begin{definition}\label{tangential}
Suppose that a compact Lie group $G$ 
acts on a Riemannian manifold $V$ in an isometric way. 
Let $\{V_{\alpha}, \SF_{\alpha}\}_{\alpha\in A}$ be a compatible fibration on $V$. 
If the following conditions are satisfied, then we call the compatible fibration a {\it $G$-tangential compatible fibration} (or {\it tangential compatible fibration} for short). 
\begin{itemize}
\item $\{V_{\alpha}\}_{\alpha\in A}$ is a $G$-invariant open covering of $V$. 
\item Each leaf $L$ of $\SF_{\alpha}$ has 
positive dimension for all $\alpha\in A$. 
\item For each leaf $L$ of $\SF_{\alpha}$ there exists some $x\in V_{\alpha}$ 
such that $L$ is contained in the $G$-orbit $G\cdot x$. 
\end{itemize}

A compatible system on a $G$-tangential compatible fibration is 
called {\it $G$-tangential compatible system} 
(or {\it tangential compatible system} for short). 
\end{definition}

%\begin{example}
%In \cite[Definition~6.7]{Fujita-Furuta-Yoshida2} a class of compatible fibration which is called {\it good compatible fibration} is defined. 
Any non-trivial torus action induces a good compatible fibration, which  
is a tangential compatible fibration. 
Moreover the product of two such good compatible fibrations 
is a tangential compatible fibration which is not good in general. 

%Though the notion of $G$-tangential compatible fibration can be defined for arbitrary compact Lie group $G$, we do not know examples of tangential compatible fibration for non-Abelian group.  
%\end{example}

%Let $M$ be a Riemannian manifold and $W$ a $\Z/2$-graded $Cl(TM)$-module bundle. 
%As in the same way in the proof Theorem~7.2 and Proposition~7.3 in \cite{Fujita-Furuta-Yoshida2}, we have the following.  

\begin{theorem}[Theorem~7.2 and Proposition~7.3 in \cite{Fujita-Furuta-Yoshida2},Theorem~3.7 in \cite{Fujitacobinv}]\label{def of local ind}
Suppose that $V$ is an open subset of $M$ whose complement is compact.  
If $V$ is equipped with a $G$-tangential acyclic compatible system $\{V_{\alpha}, \SF_{\alpha}, D_{\alpha}\}_{\alpha\in A}$, then  
we can define the 
local index $$\ind(M, \{V_{\alpha}, \SF_{\alpha}, D_{\alpha}\}_{\alpha\in A},W)=
\ind(M,V,W)=\ind(M,V)\in\Z,$$ which satisfies the excision formula, sum formula and product formula. 
\end{theorem}

Let us briefly recall the definition of the local index $\ind(M,V,W)$.  
Let $D:\Gamma(W)\to \Gamma(W)$ be a Dirac-type operator. 
We consider 
the perturbation $D_t:=D+t\sum_{\alpha\in A}\rho_{\alpha}D_{\alpha}\rho_{\alpha}$ for $t\gg 1$, where $\{\rho_{\alpha}\}_{\alpha\in A}$ is a family of smooth cut-off functions 
which is constant along leaves of $\SF_{\alpha}$ and satisfies 
some estimates as in \cite[Subection~4.1]{Fujita-Furuta-Yoshida2}. 
Such a perturbation $D_t$ gives a Fredholm operator on the space of $L^2$-sections of $W$. The local index $\ind(M,V)$ is defined as the analytic index of $D_t$ for $t\gg 1$. The excision formula implies the following localization formula of Dirac-type operator. 

\begin{theorem}\label{localizationprototype}
Suppose that $M$ is compact without boundary and an open subset $V$ of $M$ is equipped with a $G$-tangential acyclic compatible system. Then the index of any Dirac-type operator $\ind(W)$ is localized at the complement $M\setminus V$. Namely we have 
$$
\ind(W)=\ind(M,V). 
$$
\end{theorem}

\section{A formula of local indices of vector spaces }
\label{A formula of local indeices of vector spaces}

In this appendix we give a formula of equivariant local indices of vector spaces. For $l=1,2$ let $G_l$ be an $m_l$-dimensional torus and $R_l$ an $m_l$-dimensional Hermitian vector space on which $G_l$ acts unitary and effective way. 
%Let $G_2$ be $(m-1)$-dimensional torus and $R_2$ a $(m-1)$-dimensional Hermitian vector space on which $G_2$ acts unitary and effective way. 
We put the following assumptions for $l=1,2$. 

\begin{assump}
%$(1)$ A $G_1$-equivariant acyclic compatible system on $R_1^{\times}:=R_1\setminus\{0\}$ is given, where we consider the compatible fibration defined by the $G_1$ action on $R_1^{\times}$. 
%\\
$(1)$ A $G_l$-tangential equivariant compatible fibration (Definition~\ref{tangential}) on $R_{l}^{\times}:=R_l\setminus\{0\}$ is given. 
\\
$(2)$ For the compatible fibration in (1), a $G_l$-tangential equivariant acyclic compatible system on $R_l^{\times}$ is given. 
\end{assump}
By the assumption we have two equivariant local indices 
$\ind_{G_1}(R_1,R_1^{\times})$  and $\ind_{G_2}(R_2,R_2^{\times})$. 
Now we fix $\vep>0$ small enough and define two compatible fibrations and acyclic compatible systems on the product $R:=R_1\times R_2$. 

Define two subsets $R'$ and $R''$ of $R$ by 
$$
R':=\{(v_1, v_2)\in R \ | \ |v_1|>\vep, \ |v_2|<\vep\}, 
$$and 
$$
R'':=\{(v_1, v_2)\in R \  | \  |v_1|<\vep, \  |v_2|>\vep \}. 
$$
%We define a circle bundle structure on $R'$ by using the $G_1$-action on the first factor.  
We consider a structure of $G_1$-tangential (resp. $G_2$-tangential) compatible fibration on $R'$ (resp. $R''$) induced from the first (resp. second) factor. We also define a subset $R_{\infty}$ of $R$ by 
$$
R_{\infty}:=\{(v_1, v_2)\in R \ | \ |v_1|>\vep/2, \ |v_2|>\vep/2\},  
$$which is also equipped with a compatible fibration and compatible system arising from the product structure. 
Then the union $\tilde{R}_{\infty}:=R'\cup R'' \cup R_{\infty}$ gives an open covering of the complement of a compact neighbourhood of the origin of $R$. Note that the above compatible fibration and compatible system define a $G_1\times G_2$-tangential equivariant compatible fibration and acyclic compatible system on $\tilde R_{\infty}$, and hence, we have the equivariant local index 
$$
\ind_{G_1\times G_2}(R, \tilde R_{\infty}). 
$$

\begin{figure}[h]
%WinTpicVersion4.28b
{\unitlength 0.1in
\begin{picture}( 27.6000, 22.2000)( 23.1000,-26.5000)
% VECTOR 2 0 3 0 Black White
% 2 2600 2600 2600 600
% 
{\color[named]{Black}{%
\special{pn 8}%
\special{pa 2600 2600}%
\special{pa 2600 600}%
\special{fp}%
\special{sh 1}%
\special{pa 2600 600}%
\special{pa 2580 668}%
\special{pa 2600 654}%
\special{pa 2620 668}%
\special{pa 2600 600}%
\special{fp}%
}}%
% VECTOR 2 0 3 0 Black White
% 2 2600 2600 5000 2600
% 
{\color[named]{Black}{%
\special{pn 8}%
\special{pa 2600 2600}%
\special{pa 5000 2600}%
\special{fp}%
\special{sh 1}%
\special{pa 5000 2600}%
\special{pa 4934 2580}%
\special{pa 4948 2600}%
\special{pa 4934 2620}%
\special{pa 5000 2600}%
\special{fp}%
}}%
% STR 2 0 3 0 Black White
% 4 5070 2550 5070 2650 2 0 0 0
% $R_1$
\put(50.7000,-26.5000){\makebox(0,0)[lb]{$R_1$}}%
% STR 2 0 3 0 Black White
% 4 2460 460 2460 560 2 0 0 0
% $R_2$
\put(24.6000,-5.6000){\makebox(0,0)[lb]{$R_2$}}%
% STR 2 0 3 0 Black White
% 4 3040 2680 3040 2780 2 0 0 0
% $\vep/2$
\put(30.4000,-27.8000){\makebox(0,0)[lb]{$\vep/2$}}%
% STR 2 0 3 0 Black White
% 4 3670 2660 3670 2760 2 0 0 0
% $\vep$
\put(36.7000,-27.6000){\makebox(0,0)[lb]{$\vep$}}%
% STR 2 0 3 0 Black White
% 4 2380 1610 2380 1710 2 0 0 0
% $\vep$
\put(23.8000,-17.1000){\makebox(0,0)[lb]{$\vep$}}%
% STR 2 0 3 0 Black White
% 4 2310 2090 2310 2190 2 0 0 0
% $\vep/2$
\put(23.1000,-21.9000){\makebox(0,0)[lb]{$\vep/2$}}%
% LINE 2 2 3 0 Black White
% 2 3150 2590 3150 610
% 
{\color[named]{Black}{%
\special{pn 8}%
\special{pa 3150 2590}%
\special{pa 3150 610}%
\special{dt 0.045}%
}}%
% LINE 2 2 3 0 Black White
% 2 3710 2600 3710 620
% 
{\color[named]{Black}{%
\special{pn 8}%
\special{pa 3710 2600}%
\special{pa 3710 620}%
\special{dt 0.045}%
}}%
% LINE 2 2 3 0 Black White
% 2 2590 2150 4880 2150
% 
{\color[named]{Black}{%
\special{pn 8}%
\special{pa 2590 2150}%
\special{pa 4880 2150}%
\special{dt 0.045}%
}}%
% LINE 2 2 3 0 Black White
% 2 2610 1640 4900 1640
% 
{\color[named]{Black}{%
\special{pn 8}%
\special{pa 2610 1640}%
\special{pa 4900 1640}%
\special{dt 0.045}%
}}%
% LINE 3 0 3 0 Black White
% 54 4980 2100 4480 2600 4810 2150 4360 2600 4690 2150 4240 2600 4570 2150 4120 2600 4450 2150 4000 2600 4330 2150 3880 2600 4210 2150 3760 2600 4090 2150 3710 2530 3970 2150 3710 2410 3850 2150 3710 2290 4980 2220 4600 2600 4980 2340 4720 2600 4980 2460 4840 2600 4980 1980 4810 2150 4980 1860 4690 2150 4980 1740 4570 2150 4960 1640 4450 2150 4840 1640 4330 2150 4720 1640 4210 2150 4600 1640 4090 2150 4480 1640 3970 2150 4360 1640 3850 2150 4240 1640 3730 2150 4120 1640 3710 2050 4000 1640 3710 1930 3880 1640 3710 1810 3760 1640 3710 1690
% 
{\color[named]{Black}{%
\special{pn 4}%
\special{pa 4980 2100}%
\special{pa 4480 2600}%
\special{fp}%
\special{pa 4810 2150}%
\special{pa 4360 2600}%
\special{fp}%
\special{pa 4690 2150}%
\special{pa 4240 2600}%
\special{fp}%
\special{pa 4570 2150}%
\special{pa 4120 2600}%
\special{fp}%
\special{pa 4450 2150}%
\special{pa 4000 2600}%
\special{fp}%
\special{pa 4330 2150}%
\special{pa 3880 2600}%
\special{fp}%
\special{pa 4210 2150}%
\special{pa 3760 2600}%
\special{fp}%
\special{pa 4090 2150}%
\special{pa 3710 2530}%
\special{fp}%
\special{pa 3970 2150}%
\special{pa 3710 2410}%
\special{fp}%
\special{pa 3850 2150}%
\special{pa 3710 2290}%
\special{fp}%
\special{pa 4980 2220}%
\special{pa 4600 2600}%
\special{fp}%
\special{pa 4980 2340}%
\special{pa 4720 2600}%
\special{fp}%
\special{pa 4980 2460}%
\special{pa 4840 2600}%
\special{fp}%
\special{pa 4980 1980}%
\special{pa 4810 2150}%
\special{fp}%
\special{pa 4980 1860}%
\special{pa 4690 2150}%
\special{fp}%
\special{pa 4980 1740}%
\special{pa 4570 2150}%
\special{fp}%
\special{pa 4960 1640}%
\special{pa 4450 2150}%
\special{fp}%
\special{pa 4840 1640}%
\special{pa 4330 2150}%
\special{fp}%
\special{pa 4720 1640}%
\special{pa 4210 2150}%
\special{fp}%
\special{pa 4600 1640}%
\special{pa 4090 2150}%
\special{fp}%
\special{pa 4480 1640}%
\special{pa 3970 2150}%
\special{fp}%
\special{pa 4360 1640}%
\special{pa 3850 2150}%
\special{fp}%
\special{pa 4240 1640}%
\special{pa 3730 2150}%
\special{fp}%
\special{pa 4120 1640}%
\special{pa 3710 2050}%
\special{fp}%
\special{pa 4000 1640}%
\special{pa 3710 1930}%
\special{fp}%
\special{pa 3880 1640}%
\special{pa 3710 1810}%
\special{fp}%
\special{pa 3760 1640}%
\special{pa 3710 1690}%
\special{fp}%
}}%
% LINE 3 0 3 0 Black White
% 54 3150 1230 2600 680 3150 1110 2610 570 3150 990 2720 560 3150 870 2840 560 3150 750 2960 560 3150 630 3080 560 3710 1070 3200 560 3710 1190 3150 630 3710 1310 3150 750 3710 1430 3150 870 3710 1550 3150 990 3680 1640 3150 1110 3560 1640 3150 1230 3440 1640 3150 1350 3320 1640 3150 1470 3200 1640 3150 1590 3710 950 3320 560 3710 830 3440 560 3710 710 3560 560 3710 590 3680 560 3150 1350 2600 800 3150 1470 2600 920 3150 1590 2600 1040 3080 1640 2600 1160 2960 1640 2600 1280 2840 1640 2600 1400 2720 1640 2600 1520
% 
{\color[named]{Black}{%
\special{pn 4}%
\special{pa 3150 1230}%
\special{pa 2600 680}%
\special{fp}%
\special{pa 3150 1110}%
\special{pa 2610 570}%
\special{fp}%
\special{pa 3150 990}%
\special{pa 2720 560}%
\special{fp}%
\special{pa 3150 870}%
\special{pa 2840 560}%
\special{fp}%
\special{pa 3150 750}%
\special{pa 2960 560}%
\special{fp}%
\special{pa 3150 630}%
\special{pa 3080 560}%
\special{fp}%
\special{pa 3710 1070}%
\special{pa 3200 560}%
\special{fp}%
\special{pa 3710 1190}%
\special{pa 3150 630}%
\special{fp}%
\special{pa 3710 1310}%
\special{pa 3150 750}%
\special{fp}%
\special{pa 3710 1430}%
\special{pa 3150 870}%
\special{fp}%
\special{pa 3710 1550}%
\special{pa 3150 990}%
\special{fp}%
\special{pa 3680 1640}%
\special{pa 3150 1110}%
\special{fp}%
\special{pa 3560 1640}%
\special{pa 3150 1230}%
\special{fp}%
\special{pa 3440 1640}%
\special{pa 3150 1350}%
\special{fp}%
\special{pa 3320 1640}%
\special{pa 3150 1470}%
\special{fp}%
\special{pa 3200 1640}%
\special{pa 3150 1590}%
\special{fp}%
\special{pa 3710 950}%
\special{pa 3320 560}%
\special{fp}%
\special{pa 3710 830}%
\special{pa 3440 560}%
\special{fp}%
\special{pa 3710 710}%
\special{pa 3560 560}%
\special{fp}%
\special{pa 3710 590}%
\special{pa 3680 560}%
\special{fp}%
\special{pa 3150 1350}%
\special{pa 2600 800}%
\special{fp}%
\special{pa 3150 1470}%
\special{pa 2600 920}%
\special{fp}%
\special{pa 3150 1590}%
\special{pa 2600 1040}%
\special{fp}%
\special{pa 3080 1640}%
\special{pa 2600 1160}%
\special{fp}%
\special{pa 2960 1640}%
\special{pa 2600 1280}%
\special{fp}%
\special{pa 2840 1640}%
\special{pa 2600 1400}%
\special{fp}%
\special{pa 2720 1640}%
\special{pa 2600 1520}%
\special{fp}%
}}%
% LINE 3 0 3 0 Black White
% 38 5010 1520 3140 1520 5010 1609 3140 1609 5010 1699 3140 1699 5010 1787 3140 1787 5010 1876 3140 1876 5010 1965 3140 1965 5010 2054 3140 2054 5010 2143 3140 2143 5010 1432 3140 1432 5010 1342 3140 1342 5010 1254 3140 1254 5010 1165 3140 1165 5010 1076 3140 1076 5010 987 3140 987 5010 898 3140 898 5010 810 3140 810 5010 720 3140 720 5010 632 3140 632 5010 543 3140 543
% 
{\color[named]{Black}{%
\special{pn 4}%
\special{pa 5010 1520}%
\special{pa 3140 1520}%
\special{fp}%
\special{pa 5010 1610}%
\special{pa 3140 1610}%
\special{fp}%
\special{pa 5010 1700}%
\special{pa 3140 1700}%
\special{fp}%
\special{pa 5010 1788}%
\special{pa 3140 1788}%
\special{fp}%
\special{pa 5010 1876}%
\special{pa 3140 1876}%
\special{fp}%
\special{pa 5010 1966}%
\special{pa 3140 1966}%
\special{fp}%
\special{pa 5010 2054}%
\special{pa 3140 2054}%
\special{fp}%
\special{pa 5010 2144}%
\special{pa 3140 2144}%
\special{fp}%
\special{pa 5010 1432}%
\special{pa 3140 1432}%
\special{fp}%
\special{pa 5010 1342}%
\special{pa 3140 1342}%
\special{fp}%
\special{pa 5010 1254}%
\special{pa 3140 1254}%
\special{fp}%
\special{pa 5010 1166}%
\special{pa 3140 1166}%
\special{fp}%
\special{pa 5010 1076}%
\special{pa 3140 1076}%
\special{fp}%
\special{pa 5010 988}%
\special{pa 3140 988}%
\special{fp}%
\special{pa 5010 898}%
\special{pa 3140 898}%
\special{fp}%
\special{pa 5010 810}%
\special{pa 3140 810}%
\special{fp}%
\special{pa 5010 720}%
\special{pa 3140 720}%
\special{fp}%
\special{pa 5010 632}%
\special{pa 3140 632}%
\special{fp}%
\special{pa 5010 544}%
\special{pa 3140 544}%
\special{fp}%
}}%
% STR 2 0 3 0 Black White
% 4 4030 2410 4030 2510 2 0 0 0
% $R'$
\put(40.3000,-25.1000){\makebox(0,0)[lb]{$R'$}}%
% STR 2 0 3 0 Black White
% 4 2760 1130 2760 1230 2 0 0 0
% $R''$
\put(27.6000,-12.3000){\makebox(0,0)[lb]{$R''$}}%
% STR 2 0 3 0 Black White
% 4 4030 1140 4030 1240 2 0 0 0
% $R_{\infty}$
\put(40.3000,-12.4000){\makebox(0,0)[lb]{$R_{\infty}$}}%
\end{picture}}%

\caption{Open covering $\tilde R_{\infty}$.}
\label{ldomein}
\end{figure}

For $l=1,2$ define open  subsets $R_{l,0}$ and $R_{l,\infty}$ of $R_l$ by 
$$
R_{l,0}:=\{v\in R_l \ | \ |v|<\vep\}, 
$$and 
$$
R_{l,\infty}:=\{v\in R_l \ | \ |v|>\vep/2|\}. 
$$We set $R_{\infty}^{\rm prod}:=(R_{1,\infty}\times R_{2,0})\cup (R_{1,0}\times R_{2,\infty})\cup (R_{1,\infty}\times R_{2,\infty})$, which gives an open covering of a complement of a compact neighbourhood of the origin of $R$.
We consider the trivial fibration on $R_{l,0}$ and the $G_l$-tangential compatible fibration on $R_{l,\infty}$. The product of these structures induces a $G_1\times G_2$-tangential equivariant compatible fibration and acyclic compatible system on $R_{\infty}^{\rm prod}$, and hence, we have the equivariant local index 
$$
\ind_{G_1\times G_2}(R, R_{\infty}^{\rm prod}). 
$$ 


\begin{figure}[h]
%WinTpicVersion4.28b
{\unitlength 0.1in
\begin{picture}( 27.2000, 22.5000)( 23.1000,-26.7000)
% VECTOR 2 0 3 0 Black White
% 2 2600 2600 2600 600
% 
{\color[named]{Black}{%
\special{pn 8}%
\special{pa 2600 2600}%
\special{pa 2600 600}%
\special{fp}%
\special{sh 1}%
\special{pa 2600 600}%
\special{pa 2580 668}%
\special{pa 2600 654}%
\special{pa 2620 668}%
\special{pa 2600 600}%
\special{fp}%
}}%
% VECTOR 2 0 3 0 Black White
% 2 2600 2600 5000 2600
% 
{\color[named]{Black}{%
\special{pn 8}%
\special{pa 2600 2600}%
\special{pa 5000 2600}%
\special{fp}%
\special{sh 1}%
\special{pa 5000 2600}%
\special{pa 4934 2580}%
\special{pa 4948 2600}%
\special{pa 4934 2620}%
\special{pa 5000 2600}%
\special{fp}%
}}%
% STR 2 0 3 0 Black White
% 4 5030 2570 5030 2670 2 0 0 0
% $R_1$
\put(50.3000,-26.7000){\makebox(0,0)[lb]{$R_1$}}%
% STR 2 0 3 0 Black White
% 4 2470 450 2470 550 2 0 0 0
% $R_2$
\put(24.7000,-5.5000){\makebox(0,0)[lb]{$R_2$}}%
% STR 2 0 3 0 Black White
% 4 3040 2700 3040 2800 2 0 0 0
% $\vep/2$
\put(30.4000,-28.0000){\makebox(0,0)[lb]{$\vep/2$}}%
% STR 2 0 3 0 Black White
% 4 3670 2660 3670 2760 2 0 0 0
% $\vep$
\put(36.7000,-27.6000){\makebox(0,0)[lb]{$\vep$}}%
% STR 2 0 3 0 Black White
% 4 2410 1610 2410 1710 2 0 0 0
% $\vep$
\put(24.1000,-17.1000){\makebox(0,0)[lb]{$\vep$}}%
% STR 2 0 3 0 Black White
% 4 2310 2090 2310 2190 2 0 0 0
% $\vep/2$
\put(23.1000,-21.9000){\makebox(0,0)[lb]{$\vep/2$}}%
% LINE 2 2 3 0 Black White
% 2 3150 2590 3150 610
% 
{\color[named]{Black}{%
\special{pn 8}%
\special{pa 3150 2590}%
\special{pa 3150 610}%
\special{dt 0.045}%
}}%
% LINE 2 2 3 0 Black White
% 2 3710 2600 3710 620
% 
{\color[named]{Black}{%
\special{pn 8}%
\special{pa 3710 2600}%
\special{pa 3710 620}%
\special{dt 0.045}%
}}%
% LINE 2 2 3 0 Black White
% 2 2590 2150 4880 2150
% 
{\color[named]{Black}{%
\special{pn 8}%
\special{pa 2590 2150}%
\special{pa 4880 2150}%
\special{dt 0.045}%
}}%
% LINE 2 2 3 0 Black White
% 2 2610 1640 4900 1640
% 
{\color[named]{Black}{%
\special{pn 8}%
\special{pa 2610 1640}%
\special{pa 4900 1640}%
\special{dt 0.045}%
}}%
% STR 2 0 3 0 Black White
% 4 4030 2410 4030 2510 2 0 0 0
% $R_{1,\infty}\times R_{2,0}$
\put(40.3000,-25.1000){\makebox(0,0)[lb]{$R_{1,\infty}\times R_{2,0}$}}%
% STR 2 0 3 0 Black White
% 4 2690 1050 2690 1150 2 0 0 0
% $R_{1,0}\times R_{2,\infty}$
\put(26.9000,-11.5000){\makebox(0,0)[lb]{$R_{1,0}\times R_{2,\infty}$}}%
% STR 2 0 3 0 Black White
% 4 4030 1030 4030 1130 2 0 0 0
% $R_{1,\infty}\times R_{2,\infty}$
\put(40.3000,-11.3000){\makebox(0,0)[lb]{$R_{1,\infty}\times R_{2,\infty}$}}%
% LINE 3 0 3 0 Black White
% 42 3560 2150 2610 1200 3680 2150 2610 1080 3710 2060 2610 960 3710 1940 2610 840 3710 1820 2610 720 3710 1700 2620 610 3710 1580 2730 600 3710 1460 2850 600 3710 1340 2970 600 3710 1220 3090 600 3710 1100 3210 600 3710 980 3330 600 3710 860 3450 600 3710 740 3570 600 3440 2150 2610 1320 3320 2150 2610 1440 3200 2150 2610 1560 3080 2150 2610 1680 2960 2150 2610 1800 2840 2150 2610 1920 2720 2150 2610 2040
% 
{\color[named]{Black}{%
\special{pn 4}%
\special{pa 3560 2150}%
\special{pa 2610 1200}%
\special{fp}%
\special{pa 3680 2150}%
\special{pa 2610 1080}%
\special{fp}%
\special{pa 3710 2060}%
\special{pa 2610 960}%
\special{fp}%
\special{pa 3710 1940}%
\special{pa 2610 840}%
\special{fp}%
\special{pa 3710 1820}%
\special{pa 2610 720}%
\special{fp}%
\special{pa 3710 1700}%
\special{pa 2620 610}%
\special{fp}%
\special{pa 3710 1580}%
\special{pa 2730 600}%
\special{fp}%
\special{pa 3710 1460}%
\special{pa 2850 600}%
\special{fp}%
\special{pa 3710 1340}%
\special{pa 2970 600}%
\special{fp}%
\special{pa 3710 1220}%
\special{pa 3090 600}%
\special{fp}%
\special{pa 3710 1100}%
\special{pa 3210 600}%
\special{fp}%
\special{pa 3710 980}%
\special{pa 3330 600}%
\special{fp}%
\special{pa 3710 860}%
\special{pa 3450 600}%
\special{fp}%
\special{pa 3710 740}%
\special{pa 3570 600}%
\special{fp}%
\special{pa 3440 2150}%
\special{pa 2610 1320}%
\special{fp}%
\special{pa 3320 2150}%
\special{pa 2610 1440}%
\special{fp}%
\special{pa 3200 2150}%
\special{pa 2610 1560}%
\special{fp}%
\special{pa 3080 2150}%
\special{pa 2610 1680}%
\special{fp}%
\special{pa 2960 2150}%
\special{pa 2610 1800}%
\special{fp}%
\special{pa 2840 2150}%
\special{pa 2610 1920}%
\special{fp}%
\special{pa 2720 2150}%
\special{pa 2610 2040}%
\special{fp}%
}}%
% LINE 3 0 3 0 Black White
% 44 4230 1640 3270 2600 4110 1640 3160 2590 3990 1640 3150 2480 3870 1640 3150 2360 3750 1640 3150 2240 3630 1640 3150 2120 3510 1640 3150 2000 3390 1640 3150 1880 3270 1640 3150 1760 4350 1640 3390 2600 4470 1640 3510 2600 4590 1640 3630 2600 4710 1640 3750 2600 4830 1640 3870 2600 4940 1650 3990 2600 4950 1760 4110 2600 4950 1880 4230 2600 4950 2000 4350 2600 4950 2120 4470 2600 4950 2240 4590 2600 4950 2360 4710 2600 4950 2480 4830 2600
% 
{\color[named]{Black}{%
\special{pn 4}%
\special{pa 4230 1640}%
\special{pa 3270 2600}%
\special{fp}%
\special{pa 4110 1640}%
\special{pa 3160 2590}%
\special{fp}%
\special{pa 3990 1640}%
\special{pa 3150 2480}%
\special{fp}%
\special{pa 3870 1640}%
\special{pa 3150 2360}%
\special{fp}%
\special{pa 3750 1640}%
\special{pa 3150 2240}%
\special{fp}%
\special{pa 3630 1640}%
\special{pa 3150 2120}%
\special{fp}%
\special{pa 3510 1640}%
\special{pa 3150 2000}%
\special{fp}%
\special{pa 3390 1640}%
\special{pa 3150 1880}%
\special{fp}%
\special{pa 3270 1640}%
\special{pa 3150 1760}%
\special{fp}%
\special{pa 4350 1640}%
\special{pa 3390 2600}%
\special{fp}%
\special{pa 4470 1640}%
\special{pa 3510 2600}%
\special{fp}%
\special{pa 4590 1640}%
\special{pa 3630 2600}%
\special{fp}%
\special{pa 4710 1640}%
\special{pa 3750 2600}%
\special{fp}%
\special{pa 4830 1640}%
\special{pa 3870 2600}%
\special{fp}%
\special{pa 4940 1650}%
\special{pa 3990 2600}%
\special{fp}%
\special{pa 4950 1760}%
\special{pa 4110 2600}%
\special{fp}%
\special{pa 4950 1880}%
\special{pa 4230 2600}%
\special{fp}%
\special{pa 4950 2000}%
\special{pa 4350 2600}%
\special{fp}%
\special{pa 4950 2120}%
\special{pa 4470 2600}%
\special{fp}%
\special{pa 4950 2240}%
\special{pa 4590 2600}%
\special{fp}%
\special{pa 4950 2360}%
\special{pa 4710 2600}%
\special{fp}%
\special{pa 4950 2480}%
\special{pa 4830 2600}%
\special{fp}%
}}%
% LINE 3 0 3 0 Black White
% 26 4980 1590 3140 1590 4980 1710 3140 1710 4980 1830 3140 1830 4980 1950 3140 1950 4980 2070 3140 2070 4980 1470 3140 1470 4980 1350 3140 1350 4980 1230 3140 1230 4980 1110 3140 1110 4980 990 3140 990 4980 870 3140 870 4980 750 3140 750 4980 630 3140 630
% 
{\color[named]{Black}{%
\special{pn 4}%
\special{pa 4980 1590}%
\special{pa 3140 1590}%
\special{fp}%
\special{pa 4980 1710}%
\special{pa 3140 1710}%
\special{fp}%
\special{pa 4980 1830}%
\special{pa 3140 1830}%
\special{fp}%
\special{pa 4980 1950}%
\special{pa 3140 1950}%
\special{fp}%
\special{pa 4980 2070}%
\special{pa 3140 2070}%
\special{fp}%
\special{pa 4980 1470}%
\special{pa 3140 1470}%
\special{fp}%
\special{pa 4980 1350}%
\special{pa 3140 1350}%
\special{fp}%
\special{pa 4980 1230}%
\special{pa 3140 1230}%
\special{fp}%
\special{pa 4980 1110}%
\special{pa 3140 1110}%
\special{fp}%
\special{pa 4980 990}%
\special{pa 3140 990}%
\special{fp}%
\special{pa 4980 870}%
\special{pa 3140 870}%
\special{fp}%
\special{pa 4980 750}%
\special{pa 3140 750}%
\special{fp}%
\special{pa 4980 630}%
\special{pa 3140 630}%
\special{fp}%
}}%
\end{picture}}%

\caption{Open covering $R_{\infty}^{\rm prod}$.}
\label{proddomein}
\end{figure}

\begin{prop}\label{cobandprod}
We have the following equality among equivariant local indices. 
$$
\ind_{G_1\times G_2}(R, \tilde R_{\infty})=\ind_{G_1\times G_2}(R, R_{\infty}^{\rm prod})=
\ind_{G_1}(R_1,R_1^{\times})\otimes\ind_{G_2}(R_2,R_2^{\times})\in R(G_1\times G_2).  
$$
\end{prop}
\begin{proof}
The first equality follows from the cobordism invariance of local index (\cite[Theorem~7.1]{Fujitacobinv}). In fact the union of these two acyclic compatible systems on $\tilde R_{\infty}$ and $R_{\infty}^{\rm prod}$ is also $G_1\times G_2$-tangential acyclic compatible system. The second equality follows from the product formula (\cite[Theorem~8.8]{Fujita-Furuta-Yoshida2}). 
\end{proof}
%%%%%%%%%%%%%%%%%%%%%%%%%%%%%%%%%%%%%%%%%%%%%%%%%%%%%%%%%%%%%%%%
\section{Local index of folded cylinder}
\label{A computation of local index of the folded cylinder}

In this appendix we consider a natural folded symplectic structure on the cylinder and several geometric structures on it, which plays important role in the study of local property of the neighbourhood of the fold in a folded symplectic manifold. We consider a perturbation of the Dirac operator and give the direct computation of the $L^2$-kernel of the perturbed Dirac operator. We show that the $L^2$-kernel is trivial, in particular, the local index is equal to $0$. 

For any $\vep >0$, a folded symplectic structure on a cylinder (of finite length) $M_{\vep}:=(-\vep,\vep)\times S^1$ is given by a closed 2-form $2rdr\wedge d\theta$, where $(r,\theta)$ is a coordinate function on $M_{\vep}$. Here we use the opposite orientation of the cylinder as that in Section~\ref{Compatible system on toric origami manifolds} and subsequent argument for conventional reason. The standard $S^1$-action on the $S^1$-factor is Hamiltonian (in fact it is toric origami) with the moment map $(r,\theta)\mapsto r^2$. Moreover the trivial line bundle $L_0$ with connection $d-2\pi\sqrt{-1}r^2d\theta$ and the trivial lift of the $S^1$-action to the fiber direction gives an $S^1$-equivariant pre-quantizing line bundle over $M_{\vep}$. To give a computation of the local index of this toric origami manifold, we need a Clifford module bundle, Dirac-type operator along the $S^1$-orbits over a completion of $M_{\vep}$ as a Riemannian manifold.  
We summarize the set-up as follows. 

\medskip


{\bf SET-UP.}
\begin{itemize}
\item $M:=\R\times S^1$ : cylinder of infinite length
\item $(r,\theta)$ : coordinate function on $M$
\item $g:=dr^2+d\theta^2$ : Riemannian metric on $M$
\item $\rho : \R\to \R$ : smooth function with 
$$
\rho(r)=\left\{ 
\begin{array}{lll}
r^2 \quad (|r| < 1/4)) \\ 
 1/2  \quad (|r|>1/2) 
\end{array}\right.
$$
\item $\omega:=\rho'(r)dr\wedge d\theta$ : closed 2-form on $M$
\item $J:\partial_r\mapsto\partial_{\theta}, \ \partial_{\theta}\mapsto -\partial_r$ : almost complex structure on $M$
\item $TM_{\C}=(TM,J)$ : complex tangent bundle with frame $\partial_{\theta}$
\item $W^+:=M\times\C$, $W^-:=TM_{\C}$, $W:=W^{+}\oplus W^{-}$ : $\Z/2$-graded vector bundle
\item $c:T^*M\to {\rm End}(W)$ : Clifford action on $W$ defined by 
$$c(dr)=
\begin{pmatrix}
0 & -\sqrt{-1} \\ 
-\sqrt{-1} & 0 
\end{pmatrix}, \quad 
c(d\theta)=
\begin{pmatrix}
0 & -1 \\ 
1 & 0
\end{pmatrix} 
$$
\item $\nabla^W=d-2\pi\rho(r)
\begin{pmatrix}
1 & 0 \\ 
0 & 1
\end{pmatrix}d\theta$ : 
Clifford connection of $W$ 
\item 
$D=D^++D^-:\Gamma(W)\to \Gamma(W)$ : Dirac operator, 
\begin{eqnarray*}
D&=&c(\partial_r)\nabla^W_{\partial_r}+c(\partial_{\theta})\nabla^W_{\partial_{\theta}}=D^++D^- \\ 
%c(d\theta)\nabla_{\partial_{\theta}}+c(dr)\nabla_{\partial_r}=
&=&\begin{pmatrix}
0 &  -\partial_{\theta}-\sqrt{-1}\partial_r+2\pi\sqrt{-1}\rho \\
\partial_{\theta}-\sqrt{-1}\partial_r-2\pi\sqrt{-1}\rho & 0
\end{pmatrix}
\end{eqnarray*}
\item Let $S^1$ acts on $M$ in the standard way, and   
we take a lift of 
the $S^1$-action on $W$ so that the action on the fiber direction is trivial. 
%All the data are preserved by the $S^1$-action.  

\item $D_{S^1}=D_{S^1}^++D_{S^1}^-:\Gamma(W) \to \Gamma(W)$ : Dirac operator along the $S^1$-orbits,  
$$
D_{S^1}=c(\partial_{\theta})\nabla_{\partial_{\theta}}^W=
%c(d\theta)\nabla_{\partial_{\theta}}=
\begin{pmatrix}
0 &  -\partial_{\theta}+2\pi\sqrt{-1}\rho \\
\partial_{\theta}-2\pi\sqrt{-1}\rho & 0
\end{pmatrix}
$$
%\item $\mu:=-2\pi\rho:M\to \R$
%\item $\mu^M=-2\pi\rho\partial_{\theta}\in \Gamma(TM)$ : induced vector field 
%\item $f:M\to \R_+$ : smooth positive function on $M$ such that $f(r)=|r|$ for $|r-m|>1/2$ 
\end{itemize}

%\begin{remark}
%For conventional reason, we use the opposite orientation of $M$ as that in Section~\ref{Compatible system on Hamiltonian torus origami manifolds} and subsequent sections.  
%\end{remark}

\begin{remark}
When we consider the restriction to an $S^1$-invariant small open neighbourhood  $M_{\vep}$ of $\{0\}\times S^1=S^1$ in $M$, the closed 2-form $\omega$ is the folded symplectic form on $M_{\vep}$ and the $\Z/2$-graded Clifford module bundle $W$ is the one associated with the pre-quantizing line bundle $L_0$, the trivial line bundle with the connection $d-2\pi\sqrt{-1}\rho d\theta$. Note that $M_{\vep}$ has a unique spin$^c$-structure and the Clifford module bundle $W$ which is isomorphic to $W_{0,L_0}:={\rm Hom}_{Cl_2}(W_2, \wedge_{\C}^{\bullet}(TS^1\oplus\R^3))\otimes L_0$ as in Proposition~\ref{product structure of Cl}.  
\end{remark}




By using this data we have a compatible system on $M_{\vep}$ and can define the $S^1$-equivariant local index $\ind_{S^1}(S^1\times(-\vep,\vep), S^1\times(-\vep,\vep)\setminus S^1)$. The index is defined by the following perturbation of the Dirac operator:  
$$
D_t=D_t^++D_t^-,  \ D_t^+:=D^++tD_{S^1}^+,  \ D_t^-:=D^-+tD_{S^1}^-, 
$$
$$
D_t^+=(1+t)(\partial_{\theta}-2\pi\sqrt{-1}\rho)-\sqrt{-1}\partial_r, 
$$and 
$$
D_t^-=-(1+t)(\partial_{\theta}-2\pi\sqrt{-1}\rho)-\sqrt{-1}\partial_r.   
$$

\begin{prop}\label{vanishingoffoldedcylinder}
We have $\ker_{L^2}(D_t^+)=\ker_{L^2}(D_t^-)=0$ for any $t\geq 0$. In particular we have  $\ind_{S^1}(S^1\times(-\vep,\vep), S^1\times(-\vep,\vep)\setminus S^1)=0$, for any $\vep>0$. 
\end{prop}

\begin{proof}
By using the Fourier expansion $\displaystyle\phi(r,\theta)=\sum_{m\in\Z}a_m(r)e^{2\pi\sqrt{-1}m\theta}$ for smooth section $\phi$ of $W^+$, the equation $D_t^+\phi=0$ can be rewritten as a series of differential equations 
$$
a'_m(r)=2\pi(1+t)(m-\rho(r))a_m(r) \quad (m\in\Z).
$$ Each of these equations has solutions 
$$
a_m(r)=\alpha_m\exp\left(2\pi(1+t)\int_0^r(m-\rho(r))dr\right),
$$ where $\alpha_m\in\C$ is constant. 
Suppose that the solution $\phi$ is an $L^2$-section. Since $\rho\equiv 1/2$ on $\pm r\gg 0$ we have $\alpha_m=0$ for all $m\in\Z$. 
%When we consider $r\gg 0$ if the solution $\phi$ is an $L^2$-section, then we have only if $m-\rho(r)=m-1/2>0$. On the other hand when we consider $-r \gg 0$ if the solution $\phi$ is an $L^2$-section only if $m-\rho(r)=m-1/2<0$. 
In particular there are no non-trivial $L^2$-solutions of $D_t^+\phi=0$ for any $t$. As in the same way the equation $D_t^-\phi=0$ for $\displaystyle\phi(r,\theta)=\sum_{m\in\Z}b_m(r)e^{2\pi\sqrt{-1}m\theta}$ has solutions 
%for any constants $\beta_m\in\C$ and 
$$
b_m(r)=\beta_m\exp\left(-2\pi(1+t)\int_0^r(m-\rho(r))dr\right) \quad (m\in\Z), 
$$ for any constant $\beta_m\in\C$ and one can see that there are no non-trivial $L^2$-solutions. 
\end{proof}

\begin{remark}
The vanishing of the index can be deduced from the existence of 
an orientation reversing isomorphism of $S^1\times (-\vep, \vep)$ defined by $(\theta,t)\mapsto (\theta, -t)$.  
\end{remark}


%\section*{Appendix C. Local index of cotangent bundle of a torus}


\vspace{1cm}


\noindent{\bf Acknowledgements.}
The author would like to thank Mikio Furuta and Takahiko Yoshida for stimulating  conversations. Especially the argument in Section~\ref{Toric case} is based on the discussion with them. 
The auothor is grateful to the anonymous referee for his/her comments and pointing out some mistakes on the previous version. Due to referee's comments on the proof of Theorem~\ref{crackind=0}, the author could simplify it. 

\bibliographystyle{amsplain}
\bibliography{reference}


\end{document}


