\usepackage{amsmath}
\usepackage{amssymb}
\usepackage{amscd}
\usepackage{amsthm}
%\usepackage[francais,english]{babel}

\catcode`\@=11 

%\newcommand{\l@ng}[2]{\iflanguage{francais}{#2}{#1}}
\newcommand{\l@ng}[2]{#1}

\newcommand{\C}{{\mathbb C}}
\newcommand{\Q}{{\mathbb Q}}
\newcommand{\R}{{\mathbb R}}
\newcommand{\Z}{{\mathbb Z}}
\newcommand{\N}{{\mathbb N}}
\newcommand{\sln}{{\mathfrak{sl}}}
\newcommand{\gln}{{\mathfrak{gl}}}
\newcommand{\spn}{{\mathfrak{sp}}}
\newcommand{\son}{{\mathfrak{so}}}
\newcommand{\nilcone}{\mathcal{N}}
\newcommand{\orb}{{\mathcal O}}

\newcommand{\DomWeights}{{\Lambda_+}}
\newcommand{\Weights}{\Lambda}
\newcommand{\Roots}{\Delta}
\newcommand{\PosRoots}{{\Delta_+}}
\newcommand{\SimpRoots}{\Pi}
\newcommand{\Sections}{\Gamma}
\newcommand{\irrep}{\widehat}
\newcommand{\Lie}{\mathfrak}

\theoremstyle{plain}
\newtheorem{thm}{\l@ng{Theorem}{Th\'eor\`eme}}[section]
\newtheorem{prop}[thm]{Proposition}
\newtheorem{lem}[thm]{\l@ng{Lemma}{Lemme}}
\newtheorem{cor}[thm]{\l@ng{Corollary}{Corollaire}}
\newtheorem{clm}[thm]{\l@ng{Claim}{Pr\'etention}}
\newtheorem{conj}[thm]{Conjecture}
\newtheorem{des}[thm]{Desideratum}
\newtheorem{thm-}{\l@ng{Theorem}{Th\'eor\`eme}}
\newtheorem{prop-}[thm-]{Proposition}
\newtheorem{lem-}[thm-]{\l@ng{Lemma}{Lemme}}
\newtheorem{cor-}[thm-]{\l@ng{Corollary}{Corollaire}}
\newtheorem{clm-}[thm-]{\l@ng{Claim}{Pr\'etention}}
\newtheorem{conj-}[thm-]{Conjecture}
\newtheorem{des-}[thm-]{Desideratum}
\newtheorem*{conj*}{Conjecture}
\newtheorem*{thm*}{Theorem}

\theoremstyle{definition}
\newtheorem{defn}[thm]{\l@ng{Definition}{D\'efinition}}
\newtheorem{defn-}[thm-]{\l@ng{Definition}{D\'efinition}}
\newtheorem{conv}[thm]{Convention}
\newtheorem{conv-}[thm-]{Convention}
\newtheorem{cond}[thm]{Condition}
\newtheorem{cond-}[thm-]{Condition}

\theoremstyle{remark}
\newtheorem{rmk}[thm]{\l@ng{Remark}{Remarque}}
\newtheorem{exam}[thm]{\l@ng{Example}{Exemple}}
\newtheorem{rmk-}[thm-]{\l@ng{Remark}{Remarque}}
\newtheorem{exam-}[thm-]{\l@ng{Example}{Exemple}}
%\newtheorem*{ppty}{Property}





\renewcommand{\proofname}{\l@ng{Proof}{D\'emonstration}}

\newenvironment{dummy}{}{}

\newcounter{casify}
\newenvironment{casify}%
  {\begin{list}{{\it Case \arabic{casify}}. }
    {\usecounter{casify}
      \setlength{\leftmargin}{0pt}
      \setlength{\rightmargin}{0pt}
      \setlength{\labelwidth}{0pt}
      \setlength{\labelsep}{0pt}
      \setlength{\itemindent}{\parindent}}}
  {\end{list}}

\let\enumdepth\@enumdepth
\newcounter{enumsave}
\newenvironment{enumgen}[1]
  {\setcounter{enumsave}{\enumdepth}
     \enumdepth=#1
     \begin{enumerate}
     \ifnum\enumdepth>1 \renewcommand{\theenumi}{} \fi
     \ifnum\enumdepth>2 \renewcommand{\theenumii}{} \fi
     \ifnum\enumdepth>3 \renewcommand{\theenumiii}{} \fi
     \ifnum\enumdepth>4 \renewcommand{\theenumiv}{} \fi}
  {\end{enumerate} \enumdepth=\theenumsave}

\newcommand{\textx}[1]{\text{\begin{tabular}{c}#1\end{tabular}}}
\DeclareMathOperator{\im}{im}
\DeclareMathOperator{\Hom}{Hom}
\DeclareMathOperator{\Ext}{Ext}
\DeclareMathOperator{\Tor}{Tor}
\DeclareMathOperator{\Ind}{Ind}
\DeclareMathOperator{\Res}{Res}
\DeclareMathOperator{\Ad}{Ad}
\DeclareMathOperator{\ad}{ad}
\DeclareMathOperator{\gr}{gr}
\DeclareMathOperator{\sgn}{sgn}
\DeclareMathOperator{\real}{Re}
\DeclareMathOperator{\imag}{Im}
\DeclareMathOperator{\vcurl}{curl}
\DeclareMathOperator{\vgrad}{grad}
\DeclareMathOperator{\vdiv}{div}
\DeclareMathOperator{\cmod}{mod}
\DeclareMathOperator{\cHom}{\mathcal{H}\!\mathit{om}}
\DeclareMathOperator{\cExt}{\mathcal{E}\!\mathit{xt}}
\edef\det{\det\nolimits}

\newcommand{\wedgepwr}[1]{\mathchoice%
  {{\textstyle\bigwedge\nolimits^{#1}}}%
  {{\bigwedge^{#1}}}{{\bigwedge^{#1}}}{{\bigwedge^{#1}}}}
\newcommand{\inv}{^{-1}}
\newcommand{\trans}{^{\mathrm t}}
\newcommand{\Langdual}{{^{\scriptscriptstyle L\!}}}
\newcommand{\precheck}{{^{\scriptscriptstyle\vee\!}}}
\newcommand{\postcheck}{^{\scriptscriptstyle\vee}}

\newcommand{\quspace}{\vspace{0.25in}}
\newcommand{\haspace}{\vspace{0.50in}}
\newcommand{\whspace}{\vspace{1.00in}}
\newcommand{\sqspace}{\vspace{1.50in}}
\newcommand{\spacer}{\vspace{0pt plus \baselineskip}}

\catcode`\@=12
