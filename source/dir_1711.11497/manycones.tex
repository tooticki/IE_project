\section{Many Hyperbolic Perturbations of $e_d$} \label{sec:many-ed}
In this section we prove that even though $e_d(x_1,\ldots,x_n)$ is not in the
interior of the set of Hyperbolic polynomials, there is a large subspace
$\Pi_d$ of homogeneous polynomials of degree $d$ in $n$ variables such
that all sufficiently small perturbations of $e_d$ in this subspace remain
hyperbolic. 

The subspace will be spanned by certain homogeneous polynomials corresponding to matchings. For any matching $M$ containing $d$ edges on $\{1,\ldots,n\}$, define the polynomial
$$q_M(x_1,\ldots,x_n):=\prod_{i<j\in M}(x_i-x_j).$$
We say that a $d-$matching on $[n]$ {\em fully crosses} a $d-$subset $S$ of $[n]$ if every edge of $M$ has exactly one endpoint in $S$.
\renewcommand{\S}{\mathcal{S}}
\begin{lemma}[Many Uniquely Crossing Matchings] \label{lem:manymatchings} There is a set $\M_d$ of $d-$matchings on $[n]$ of size at least 
$$|\M_d|=:N\ge \frac{\binom{n}{d}}{4\cdot 2^d}$$
and a set of $d-$subsets $\S_d$ such that for every $S\in\S_d$ there is a {\em unique} matching $M\in\M_d$ which fully crosses it, and for every $M\in\M_d$ there is at least one $S\in\S_d$ which it fully crosses.

Moreover, for every indicator vector $\one_S$, $S\in \S_d$ there is a unique $M\in\M_d$ such that $q_M(\one_S)\neq 0$, so the dimension of the span of 
$$\Pi_d:=\{q_M:M\in\M_d\}$$ is exactly $N$.
\end{lemma}
\begin{proof} Let $\M$ denote the set of all $d-$matchings on $[n]$ and let $\S$ be the set of all $d-$subsets of $[n]$. Let 
$$E:=\binom{n-d}{d}\cdot d!$$
be the number of matchings fully crossing a fixed set $S\in\S$. Let $\M_d$ be a random subset of $\M$ in which each matching is included independently with probability $\alpha/E$ for some $\alpha\in (0,1)$ to be determined later, and let $X_M$ be the indicator random variable that $M\in\M_d$. For any set $S$, define the random variable
\renewcommand{\deg}{\mathrm{deg}}
\newcommand{\E}{\mathbb{E}}
\renewcommand{\P}{\mathbb{P}}
$$\deg(S): = \sum_{M\textrm{ fully crossing }S} X_M,$$
and observe that
$$\E \deg(S) = \alpha.$$
Call a set $S$ {\em good} if $\deg(S)=1$ and let $G$ be the number of good $S$. Observe that
$$\E G = \binom{n}{d}\cdot\P[\deg(\{1,\ldots,d\}=1] = \binom{n}{d}\cdot E\cdot (1-(\alpha/E))^{E-1}\cdot\frac{\alpha}{E}\ge \binom{n}{d}\alpha(1-\alpha\cdot(1-1/E)).$$ 
Setting $\alpha=1/2$ we therefore have
$$\E G \ge \binom{n}{d}/4,$$
so with nonzero probability there are at least $\binom{n}{d}/4$
good subsets.

Let $\S_d$ be the set of good subsets. Finally, remove from $\M_d$ all the matchings that do not fully cross a good subset. Since there are at least $(1/4)\binom{n}{d}$ good subsets, every good subset fully crosses at least one matching in $\M_d$, and at most $2^d$ subsets cross any given matching, the number of matchings whichremain in $\M_d$ is at least $$(1/4)\binom{n}{d}/2^d,$$
as desired.

The moreover part is seen by observing that $q_M(\one_S)\neq 0$ if and only if $M$ fully crosses $S$.
\end{proof}
\begin{remark} In fact, the dimension of the span of all of the polynomials $q_M$ is exactly $\binom{n}{d}-\binom{n}{d-1}$, by properties of the Johnson Scheme, but in order to obtain the additional property above we have restricted to $\M_d$.\end{remark}

Henceforth we fix $\{q_M:M\in\M_d\}$ to be a basis of $\Pi_d$ and define the
$\ell_1$ norm of a polynomial $$q = \sum_{M\in\M_d} s_M q_M\in \Pi_d$$ as
$\|q\|_1:=\|s\|_1$. We will use the notation
$$m_d(x_1,\ldots,x_n)=\max_{|S|=d}\prod_{i\in S}|x_i|,$$ to denote the product
of the $d$ largest entries of a vector in absolute value, and occasionally we
will write $e_d(p)$ and $m_d(p)$ for a real-rooted polynomial $p$, which means applying
$e_d$ or $m_d$ to its roots. The operator $D$ refers to differentiation with respect to $t$.
 

The main theorem of this section is:
 \begin{theorem}[Effective Relative Nuij Theorem for $e_d$]\label{thm:perturb} If $q\in\Pi_d$ satisfies $$\|q\|_1\le
	 \frac{\binom{n}{d}}{2^n\cdot n^{(d+1)(n-d)}}=:R$$
	 then $e_d+q$ is hyperbolic with respect
 to $\one$.\end{theorem}
Recalling the definition of hyperbolicity, our task is to show that all of the
restrictions $$t\mapsto e_d(t\one+x),\quad x\in\R^n$$ remain real-rooted after perturbation by $q$.
Many of these restrictions lie on the boundary of the set of (univariate) real-rooted polynomials, 
 or arbitrarily close to it, so it is not possible to simply discretize the set of $x$ by a net
 and choose $q$ to be uniformly small on this net; one must instead carry out a more delicate restriction-specific analysis
 which shows that for small $R$, the perturbation $q(t\one +x)$ is less than the distance of $e_d(t\one + x)$
 to the boundary of the set of real-rooted polynomials, for each $x\in\R^n$.
Since we are comparing vanishingly small quantities, it is not a priori 
  clear that such an approach will yield an effective bound on $R$ depending only
  on $n$ and $d$; Lemma \ref{lem:edmd} shows that this is indeed possible.
\begin{proof}[Proof of Theorem \ref{thm:perturb}]
 Fix any nonzero vector $x\in\R^n$ and perturbation $q\in\Pi_d$ with $\|q\|_1\le R$ and consider the perturbed restriction:
 $$ r(t):= e_d(t\one+x)+q(t\one+x).$$
Let $p(t):=e_d(t\one+x)$ and observe that since $q$ is translation invariant, we have $q(t\one+x)=q(x)$, so in fact
$$r(t)=p(t)+q(x).$$
Let $\gamma\ge 0$ be the largest constant such that $p(t)+\delta$ is real-rooted for all $\delta\in[-\gamma,\gamma]$ (note that $\gamma$ could be zero if $p$ has a repeated root). It is sufficient to show that $|q(x)|\le \gamma$. Observe that 
\begin{equation}\label{eqn:gap}\gamma = \min_{t:p'(t)=0}|p(t)|,\end{equation}
since the boundary of the set of real-rooted polynomials consists of polynomials with repeated roots, and at any double root $t_0$ of the $p+\gamma$ we have $p'(t_0)=(p+\gamma)'(t_0)=(p+\gamma)(t_0)=0$. Let $t_0$ be the minimizer in \eqref{eqn:gap} and replace $x$ by $x-t_0\one$, noting that this translates $r(t)$ to $r(t-t_0)$ and does not change $\gamma$ or $q(x)$, so that we now have:
$$p'(0)=d\cdot e_{d-1}(x)=0$$
and
$$\gamma = |p(0)|=|e_d(x)|.$$

On the other hand, observe that:
\begin{align*}
|q(x)| &\le\sum_{M\in\M_d} |s_M|\prod_{ij\in M}|x_i-x_j|
\\&\le \|q\|_1\cdot\max_{M\in\M_d}\prod_{ij\in M}(|x_i|+|x_j|)
\\&\le \|q\|_1\cdot 2^d\cdot m_d(x).
\end{align*}
Thus we have $\gamma\ge |q(x)|$ as long as $m_d(x)=0$ or
$$ \|q\|_1\le \frac{|e_d(x)|}{2^dm_d(x)},$$
which is implied by
	$$ \|q\|_1\le \frac{\binom{n}{d}}{2^d(2n^{d+1})^{(n-d)}}$$
by Lemma \ref{lem:edmd}, as advertised.
\end{proof}

The following lemma may be seen as a quantitative version of the fact that if a real-rooted polynomial
has two consecutive zero coefficients $e_d=e_{d-1}=0$ then it must have a root of multiplicity $d+1$ at zero.
\begin{lemma}\label{lem:edmd} If $x\in\R^n$ satisfies $e_{d-1}(x)=0$ then
	$$ |e_d(x)|\ge \frac{\binom{n}{d}}{(2n^{d+1})^{(n-d)}}|m_d(x)|.$$
\end{lemma}
\begin{proof}

Let $p(t):=\prod_{i=1}^n(t-x_i)$ and let $q_k(t):=D^{n-k}p$, noting that $q_k$
is real-rooted of degree exactly $k$. Assume for the moment that all of the
$x_i$ are distinct and that $q_k(0)\neq 0$ for all $k=n,\ldots,d$ (note
that this is equivalent to assuming that the last $d+1$ coefficients of
$p$ are nonzero). Note that these conditions imply that all of the polynomials $q_k$ have distinct
roots, since differentiation cannot increase the multiplicity of a root.\\

If $n=d$ then the claim is trivially true since 
	\begin{equation}\label{eqn:base} e_d(x)=e_d(q_d)=m_d(q_d)=m_d(x).\end{equation}
Observe that $e_d$ behaves predictably under differentiation:
	\begin{equation}\label{eqn:edd} e_d(q_d)=e_d(D^{n-d}p)=\frac{(n-d)!}{n\cdot\ldots\cdot (d+1)}e_d(p)=\binom{n}{d}^{-1}e_d(q_n).\end{equation}
We will show by induction that:
	$$ m_d(q_{d})\ge \frac{1}{2n^{d+1}}m_d(q_{d+1})\ge\ldots \ge \frac{1}{(2n^{d+1})^{k-d}}m_d(q_k)\ge\ldots\ge \frac{1}{(2n^{d+1})^{n-d}}m_d(q_{n}),$$
	which combined with \eqref{eqn:base} and \eqref{eqn:edd} yields the desired conclusion.\\


\noindent {\bf Case $\mathbf{k=d+1}.$}  Let $z_-$ and $z_+$ be the smallest
	(in magnitude) negative and positive roots of $q_d=Dq_{d+1}$, respectively. 
	Let $w\neq 0$ be the unique root of $q_{d+1}$ between $z_-$ and $z_+$; assume without loss of generality that $w>0$ (otherwise 
	consider the polynomial $p(-x)$). Let $x_-< z_-$ and $x_+< z_+$ be the smallest
	in magnitude negative and positive roots of $q_{d+1}$ other than $w$, so that $x_- < 0<  w< x_+$. There
	are two subcases, depending on whether or not $w$ is close to zero --- if it is, then it prevents any root from
	shrinking too much under differentiation, and if it is not, the hypothesis $e_{d-1}(p)=0$ shows that $|z_-|$ and $|z_+|$ are
	comparable, which also yields the conclusion.
	\begin{itemize}
		\item Subcase $|w|\le |x_-|/2n$. By Lemma \ref{lem:aspect}, we have
			$$|z_-|\ge |x_-| - (|x_-|+|w|)(1-1/n) \ge |x_-|(1-(1+1/2n)(1-1/n))\ge |x_-|/2n.$$
			For every root of $q_{d+1}$ other than $x_-$ there is another root of $q_{d+1}$ between it and zero,
			so Lemma \ref{lem:aspect} implies that for every such root the neighboring (towards zero) root of $Dq_{d+1}$
			is smaller by at most $1/n$. Thus, we conclude that
			$$m_d(q_d)=m_d(Dq_{d+1})\ge \frac{m_d(q_{d+1})}{2n^{d}}.$$
		\item Subcase $|w|>|x_-|/2n$. In this case we may assume that $m_d(q_{d+1})$ is witnessed by the $d$ roots of $q_{d+1}$
			excluding $x_-$, call this set $W$, losing a factor of at most $1/2n$. Observe that every root in $W\setminus \{w\}$ 
			is separated from zero by another root of $q_{d+1}$, so such roots shrink by at most $1/n$ under differentiation
			by Lemma \ref{lem:aspect}. Noting that $x_-\notin W$, we have by interlacing that:
			$$ \prod_{q_d(z)=0, z\neq z_-}|z| \ge \frac{1}{n^{d-1}}\prod_{x\in W\setminus \{w\}} |x|,$$
			and our task is reduced to showing $|z_-|$ is not small compared to $|w|$

			The hypothesis $e_{d-1}(q_{d+1})=0$ implies that
		$q_d'(0)=0$; applying Lemma \ref{lem:aspect}, we find that the magnitudes of the innermost roots of $q_d$ must be comparable:
		$$|z_-|\lor |z_+|\le d\cdot (|z_-|\land|z_+|).$$ We now have
			$$|z_-|>|z_+|/n>|w|/n,$$
			so we conclude that
			$$m_d(q_d)\ge \frac{m_d(q_{d+1})}{2n^d},$$
			as desired.
	\end{itemize}

\noindent {\bf Case $\mathbf{k\ge d+2}$.} We proceed by induction.
Assume $m_d(q_k)$ is witnessed by a set of $d$ roots $L\cup R$, where $L$ contains negative roots
and $R$ contains positive ones. If there is a negative root not in $L$ and a positive root not in $R$
then as before every root in $L\cup R$ is separated from zero by another root of $q_k$, and by Lemma 7 we have
\begin{equation}\label{eqn:safe} m_d(Dq_k)\ge \frac{1}{n^d}m_d(q_k),\end{equation}
so we are are done. So assume all of the negative roots are contained in $L$; since $|L\cup R|=d$ this implies
that there are at least two positive roots not in $R$; let $z_*$ be the largest positive root not contained in $R$.
Let $z_-$ and $z_+$ be the negative and positive roots of $q_k$ of least magnitude. 
There are two cases:
\begin{itemize}
\item $|z_+|>|z_-|/2n$. This means that we can delete $z_-$ from $L$ and add $z_*$ to $R$, and reduce to the previous situation,
incurring a loss of at most $1/2n$, which means by \eqref{eqn:safe}:
$$ m_d(q_{k-1})\ge \frac{1}{2n}\frac{1}{n^d}m_d(q_k).$$
\item $|z_+|\le |z_-|/2n$. By  Lemma 7, the smallest in magnitude negative root of $Dq_k$ has magnitude at least
$$ (1-(1-1/n)(1+1/2n))|z_-|\ge |z_-|/2n,$$
and all the positive roots decrease by at most $1/n$ upon differentiating by Lemma 7, whence we have
$$m_d(Dq_k)\ge \frac{1}{2n^d}m_d(q_k).$$
\end{itemize}

To finish the proof, the requirements that $q_k(0)\neq 0$ for all $k$ and that
all coordinates of $x$ are distinct may be removed by a density argument, since
the set of $x$ for which this is true is dense in the set of $x\in\R^n$
satisfying $e_{d-1}(x)=0$.

\end{proof}
\begin{remark} We suspect that the dependence on $n$ and $d$ in the above lemma can be improved,
	and it is even plausible that it holds with a polynomial rather than exponential dependence of $R$ on $n$. 
	Since we do not know how to do this at the moment, we have chosen to present the simplest proof we know,
	without trying to optimize the parameters.\end{remark}
	

The following lemma is a quantitative version of the fact that the roots of the derivative of a polynomial
interlace its roots.
\begin{lemma}[Quantitative Interlacing] \label{lem:aspect} If $p$ is real rooted of degree $n$ then
		every root of $p'$ between two distinct consecutive roots of
		$p$ divides the line segment between them in at most the ratio
		$1:n$. \end{lemma}
  \begin{proof} Begin by recalling that 
		if $p$ has distinct roots $z_1< \ldots < z_n$ then the
		roots $z_1' < \ldots < z_{n-1}'$ of $p'$ satisfy
		for $j=1,\ldots,n-1$:
		$$ \sum_{i\le j}\frac1{z_j'-z_i} = \sum_{i>j}\frac1{z_i-z_j'}.$$
		Note that the solution $z_j'$ is monotone increasing in the
		$z_i$ on the LHS and monotone decreasing in the $z_i$ on the
		RHS. Thus, $z_j'$ is at least the solution to:
		$$ \frac1{z_j'-z_j}=\frac{n}{z_{j+1}-z_j'},$$
		which means that it is at least $z_j+\frac{z_{j+1}-z_j}{n}$. A
		similar argument shows that it it at most
		$z_j+(1-1/n)(z_{j+1}-z_j)$. Adding the common roots of $p$ and
		$p'$ back in, we conclude that these inequalities are satisfied
		by all of the $z_j'$.\end{proof}
