\section{Proof of The Main Theorem}
We restate the theorem for convenience.
\begin{theorem} [Restatement of Main Theorem]\label{thm:main}
There exists an absolute constant $\kappa > 0$ such that for all sufficiently large $n, d$, 
there exists an n-variate degree $d$ hyperbolic polynomial $p$ whose hyperbolicity cone $K_p$ does not admit an $\eta-$approximate spectrahedral representation of dimension $B \leq \left(n/ d\right)^{\kappa d}$, for $\eta=1/n^{4nd}$. 
\end{theorem}


\begin{proof} Set $\epsilon$ smaller than $R$ and $R_2$.
Theorem \ref{thm:perturb}, Lemma \ref{lem:restrictfar} and Lemma \ref{lem:restrictiontocone}
  together imply a family of hyperbolic polynomials $\mathcal{P}$ with the following properties:
\begin{enumerate}
	\item $|\mathcal{P}| = 2^{|\mathcal{M}_d|}$ where $|\mathcal{M}_d | \geq (n/d)^{\kappa d}$ for some absolute constant $\kappa > 0$.        
    
    \item For all $p \neq p' \in \mathcal{P}$, $\hdist(K_p,K_p') \geq \gamma$ for $\gamma > \frac{1}{n^{3nd}}$.
    
    \item For every $p \in \mathcal{P}$, the positive orthant $\R_+^n$ is contained in $K_p$.
    \end{enumerate}
The last observation follows from the fact that positive orthant is contained in $K_{e_d}$, and the perturbations of $e_d$ in $\mathcal{P}$ are small enough to keep all the coefficients non-negative.

Suppose each of the hyperbolicity cones $K_p$ admitted a $\gamma/3-$approximate spectrahedral representation in dimension $B$.  By Lemma \ref{lem:normalization}, for each cone in $K_p$, there exists a normalized $\gamma/3$-approximate spectrahedral representation $C_p$ in dimension $B_p \leq B$
	Further, by Lemma \ref{lem:conetomatrices}, for every pair of polynomials $p, p' \in \mathcal{P}$, their corresponding $\gamma/3$-approximate spectrahedral representations satisfy $\mdist(C_p, C_{p'}) \geq n^{-3/2} \gamma/3 \coloneqq \eta$.  

Notice that in every normalized spectrahedral representation $C_p$ in dimension $B$, every matrix $C \in C_p$ satisfies $C \succeq 0$ and $C \preceq \Id_B$.  This implies that $\|C\|_{F} \leq \sqrt{B}$.  By a simple volume argument, for every $\eta > 0$, the number of normalized  spectrahedral representations in $\R^{B\times B} $ whose pairwise distances are $\geq \eta$ is at most $\left( \sqrt{B} /\eta \right)^{nB^2}$.  Since every cone $K_p$ for $p \in \mathcal{P}$ admits a normalized spectrahedral representation of dimension at most $B$, we get that
\[ |\mathcal{P}| \leq t \left(\sqrt{B} /\eta \right)^{nB^2} \ ,\]
which implies that
\[ B^2 \log B \geq  |\mathcal{M}_d |/\log(n^{3/2}/ \gamma) \geq \frac{1}{n \log n}  \cdot (n/ d)^{\kappa d}  \ . \]
which implies the lower bound of  $B \geq (n/d)^{\kappa' d}$ for some constant $\kappa' > 0$.
\end{proof}  

\subsection*{Acknowledgments} We thank Jim Renegar for pointing out a mistake in a previous version of the proof of Lemma \ref{lem:edmd}. We thank
the Simons Institute for the Theory of Computing and MSRI, where this work was partially carried out, for their hospitality.
