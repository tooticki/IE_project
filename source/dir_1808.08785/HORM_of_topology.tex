\documentclass[reqno]{amsart}
%\usepackage[english]{babel}
\usepackage{amssymb,amsmath,hyperref}
\usepackage{amsrefs}
\usepackage[foot]{amsaddr}
\usepackage{bbold,stackrel}
 
%%%%%%%%%%%%%%%%%%%%%%%%
% Dag Normann LaTeX definitions


%\newcommand{\R}{{\Bbb R}}
%\newcommand{\N}{{\Bbb N}}
%\newcommand{\B}{{\Bbb B}}
%\newcommand{\Q}{{\Bbb Q}}
%\newcommand{\Z}{{\Bbb Z}}
\newcommand{\lb}{[\![}
\newcommand{\rb}{]\!]}
\newcommand{\qcb}{\mathsf{QCB}}
\newcommand{\QCB}{\mathsf{QCB}}
\newcommand{\dminus}{\mbox{$\;^\cdot\!\!\!-$}}

%%%%%%%%%%%%%%%%%%%%%%%%%



\newtheorem{thm}{Theorem}
\newtheorem{lem}[thm]{Lemma}
\newtheorem{cor}[thm]{Corollary}
\newtheorem{defi}[thm]{Definition}
\newtheorem{rem}[thm]{Remark}
\newtheorem{nota}[thm]{Notation}
\newtheorem{exa}[thm]{Example}
\newtheorem{rul}[thm]{Rule}
\newtheorem{ax}[thm]{Axiom}
\newtheorem{set}[thm]{Axiom set}
\newtheorem{sch}[thm]{Axiom schema}
\newtheorem{princ}[thm]{Principle}
\newtheorem{algo}[thm]{Algorithm}
\newtheorem{tempie}[thm]{Template}
\newtheorem{ack}[thm]{Acknowledgement}
\newtheorem{specialcase}[thm]{Special Case}
\newtheorem{theme}[thm]{Theme}
\newtheorem{conj}[thm]{Conjecture}
\newtheorem*{tempo*}{Template}
\newtheorem{theorem}[thm]{Theorem}
%\theoremstyle{plain}\newtheorem{lemma}[thm]{Lemma}
\newtheorem{cth}{Classical Theorem}
\newtheorem{lemma}[thm]{Lemma}
\newtheorem{definition}[thm]{Definition}
\newtheorem{corollary}[thm]{Corollary}
\newtheorem{remark}[thm]{Remark}
\newtheorem{convention}[thm]{Convention}
\newtheorem{proposition}[thm]{Proposition}
\newtheorem{dagcom}{Dag-Comment}

\newtheorem{fact}{Fact}
\newtheorem{facts}[fact]{Facts}

\newcommand\be{\begin{equation}}
\newcommand\ee{\end{equation}} 

\usepackage{amsmath,amsfonts} 

\usepackage[applemac]{inputenc}


\def\bdefi{\begin{defi}\rm}
\def\edefi{\end{defi}}
\def\bnota{\begin{nota}\rm}
\def\enota{\end{nota}}
\def\FIVE{\Pi_{1}^{1}\text{-\textup{\textsf{CA}}}_{0}}
\def\FIVEK{\Pi_{k}^{1}\text{-\textup{\textsf{CA}}}_{0}}
\def\FIVEo{\Pi_{1}^{1}\text{-\textup{\textsf{CA}}}_{0}^{\omega}}
\def\FIVEFIVE{\Delta_{2}^{1}\text{-\textsf{\textup{CA}}}_{0}}
\def\SIX{\Pi_{2}^{1}\text{-\textsf{\textup{CA}}}_{0}}
\def\SIK{\Pi_{k}^{1}\text{-\textsf{\textup{CA}}}_{0}}
\def\SIXk{\Pi_{k}^{1}\text{-\textsf{\textup{CA}}}_{0}}
\def\SIXK{\Pi_{k}^{1}\text{-\textsf{\textup{CA}}}_{0}^{\omega}}
\def\meta{\textup{\textsf{meta}}}
\def\ATR{\textup{\textsf{ATR}}}
\def\LOC{\textup{\textsf{LOC}}}
\def\MON{\textup{\textsf{MON}}}
\def\PIT{\textup{\textsf{PIT}}}
\def\PITo{\textup{\textsf{PITo}}}
\def\PR{\textup{\textsf{PR}}}
\def\WPR{\textup{\textsf{WPR}}}
\def\ULK{\textup{\textsf{ULK}}}
\def\UB{\textup{\textsf{UB}}}
\def\Z{\textup{\textsf{Z}}}
\def\BIN{\textup{\textsf{BIN}}}
\def\LUS{\textup{\textsf{LUS}}}
\def\EGO{\textup{\textsf{EGO}}}
\def\NFP{\textup{\textsf{NFP}}}
\def\ZFC{\textup{\textsf{ZFC}}}
\def\cl{\textup{\textsf{cl}}}
\def\RM{\textup{\textsf{rm}}}
\def\URY{\textup{\textsf{URY}}}
\def\TIET{\textup{\textsf{TIET}}}
\def\ZF{\textup{\textsf{ZF}}}
\def\bs{\textup{\textsf{bs}}}
\def\IST{\textup{\textsf{IST}}}
\def\MU{\textup{\textsf{MU}}}
\def\MUO{\textup{\textsf{MUO}}}
\def\IRT{\textup{\textsf{IRT}}}
\def\BET{\textup{\textsf{BET}}}
\def\ALP{\textup{\textsf{ALP}}}
%\def\T{\mathcal{T}}
\def\TT{\mathcal{TT}}
 \def\r{\mathbb{r}}
\def\STP{\textup{\textsf{STP}}}
\def\DNS{\textup{\textsf{STC}}}
\def\PA{\textup{PA}}
\def\FAN{\textup{\textsf{FAN}}}
\def\DNR{\textup{\textsf{DNR}}}
\def\LUB{\textup{\textsf{LUB}}}
\def\RWWKL{\textup{\textsf{RWWKL}}}
\def\RWKL{\textup{\textsf{RWKL}}}
\def\H{\textup{\textsf{H}}}
\def\ef{\textup{\textsf{ef}}}
\def\ns{\textup{\textsf{ns}}}
\def\u{\textup{\textsf{u}}}
\def\c{\textup{\textsf{c}}}
\def\RCA{\textup{\textsf{RCA}}}
%\def\bennot{\textup{E-HA}^{\omega*}_{st}}
\def\({\textup{(}}
\def\){\textup{)}}
\def\WO{\textup{\textsf{WO}}}
\def\RCAo{\textup{\textsf{RCA}}_{0}^{\omega}}
\def\ACAo{\textup{\textsf{ACA}}_{0}^{\omega}}
\def\WKLo{\textup{\textsf{WKL}}_{0}^{\omega}}
\def\WKL{\textup{\textsf{WKL}}}
\def\PUC{\textup{\textsf{PUC}}}
\def\WWKL{\textup{\textsf{WWKL}}}
\def\bye{\end{document}}
%\def\rec{\textup{rec}}
%\def\sP{^{*}\mathcal  P}
\def\P{\textup{\textsf{P}}}
%\def\Pf{{\mathcal{P}_{\textup{fin}}}}
\def\N{{\mathbb  N}}
\def\Q{{\mathbb  Q}}
\def\R{{\mathbb  R}}
\def\L{\textsf{\textup{L}}}
\def\A{{\mathbb  A}}
\def\subsetapprox{\stackrel[\approx]{}{\subset}}
\def\C{{\mathbb  C}}
\def\PC{\textup{\textsf{PC}}}
%\def\CC{{\mathfrak  C}}
\def\NN{{\mathfrak  N}}
\def\B{{\mathbb  B}}
\def\I{{\textsf{\textup{I}}}}
\def\D{{\mathbb  D}}
\def\E{{\mathcal  E}}
\def\FAN{\textup{\textsf{FAN}}}
\def\PUC{\textup{\textsf{PUC}}}
\def\WFAN{\textup{\textsf{WFAN}}}
\def\UFAN{\textup{\textsf{UFAN}}}
\def\MUC{\textup{\textsf{MUC}}}
\def\ind{\textup{{ind }}}
\def\Ind{\textup{{Ind }}}
\def\MPC{\textup{\textsf{MPC}}}
\def\st{\textup{st}}
\def\di{\rightarrow}
\def\asa{\leftrightarrow}
\def\ACA{\textup{\textsf{ACA}}}
\def\PUNI{\textup{\textsf{PUNI}}}
\def\paai{\Pi_{1}^{0}\textup{-\textsf{TRANS}}}
\def\Paai{\Pi_{1}^{1}\textup{-\textsf{TRANS}}}
\def\QFAC{\textup{\textsf{QF-AC}}}
\def\ml{\textup{\textsf{ml}}}
\def\MLR{\textup{\textsf{MLR}}}
\def\HBU{\textup{\textsf{HBU}}}
\def\HBT{\textup{\textsf{HBT}}}
\def\IVT{\textup{\textsf{IVT}}}
\def\blambda{\pmb{\lambda}}
\def\sg{\textup{\textsf{sg}}}
\def\UBT{\textup{\textsf{UBT}}}
\def\ROL{\textup{\textsf{ROL}}}
\def\RIE{\textup{\textsf{RIE}}}
\def\WEI{\textup{\textsf{WEI}}}
\def\FEJ{\textup{\textsf{FEJ}}}
\def\DIN{\textup{\textsf{DIN}}}
\def\LIN{\textup{\textsf{LIN}}}
\def\LIL{\textup{\textsf{LIL}}}
\def\WHBU{\textup{\textsf{WHBU}}}
\def\UCT{\textup{\textsf{UCT}}}
\def\HBC{\textup{\textsf{HBC}}}
\def\BDN{\textup{\textsf{BD-N}}}
\def\RPT{\textup{\textsf{RPT}}}
\def\CCS{\textup{\textsf{CCS}}}
\def\csm{\textup{\textsf{csm}}}
\def\compact{\textup{\textsf{compact}}}
\def\POS{\textup{\textsf{POS}}}
\def\KB{\textup{\textsf{KB}}}
\def\MAX{\textup{\textsf{MAX}}}
\def\FIP{\textup{\textsf{FIP}}}
\def\CAC{\textup{\textsf{CAC}}}
\def\ORD{\textup{\textsf{ORD}}}
\def\DRT{\Delta\textup{\textsf{-RT}}}
\def\CCAC{\textup{\textsf{CCAC}}}
\def\SCT{\textup{\textsf{SCT}}}
\def\IPP{\textup{\textsf{IPP}}}
\def\COH{\textup{\textsf{COH}}}
\def\SRT{\textup{\textsf{SRT}}}
\def\CAC{\textup{\textsf{CAC}}}
\def\tnmt{\textup{\textsf{tnmt}}}
\def\EM{\textup{\textsf{EM}}}
\def\field{\textup{\textsf{field}}}
\def\CWO{\textup{\textsf{CWO}}}
\def\DIV{\textup{\textsf{DIV}}}
\def\KER{\textup{\textsf{KER}}}
\def\KOE{\textup{\textsf{KOE}}}
\def\MCT{\textup{\textsf{MCT}}}
\def\UMCT{\textup{\textsf{UMCT}}}
\def\UATR{\textup{\textsf{UATR}}}
\def\ST{\mathbb{ST}}
\def\mSEP{\textup{\textsf{-SEP}}}
\def\mUSEP{\textup{\textsf{-USEP}}}
\def\STC{\textup{\textsf{STC}}}
\def\ao{\textup{\textsf{ao}}}
\def\SJ{\mathbb{S}}
\def\PST{\textup{\textsf{PST}}}
\def\eps{\varepsilon}
\def\con{\textup{\textsf{con}}}
\def\X{\textup{\textsf{X}}}
\def\HYP{\textup{\textsf{HYP}}}
\def\DT{\textup{\textsf{DT}}}
\def\LO{\textup{\textsf{LO}}}
\def\CC{\textup{\textsf{CC}}}
\def\ADS{\textup{\textsf{ADS}}}
\def\WADS{\textup{\textsf{WADS}}}
\def\CADS{\textup{\textsf{CADS}}}
\def\SU{\textup{\textsf{SU}}}
\def\RF{\textup{\textsf{RF}}}
\def\RT{\textup{\textsf{RT}}}
\def\RTT{\textup{\textsf{RT22}}}
\def\SRTT{\textup{\textsf{SRT22}}}
\def\CRTT{\textup{\textsf{CRT22}}}
\def\WT{\textup{\textsf{WT}}}
\def\EPA{\textup{\textsf{E-PA}}}
\def\EPRA{\textup{\textsf{E-PRA}}}
\def\TRM{\textup{\textsf{TRM}}}
\def\FF{\textup{\textsf{FF}}}
\def\TOF{\textup{\textsf{TOF}}}
\def\ECF{\textup{\textsf{ECF}}}
\def\WWF{\textup{\textsf{WWF}}}
\def\NUC{\textup{\textsf{NUC}}}
%\def\VCF{\textup{\textsf{VCF}}}
\def\LMP{\textup{\textsf{LMP}}}
%\def\RKL{\textup{\textsf{RKL}}}
\def\TJ{\textup{\textsf{TJ}}}
%\def\SJ{\textup{\textsf{SJ}}}
%\def\SHJ{\textup{\textsf{SHJ}}}
%\def\META{\textup{\textsf{META}}}
\def\SCF{\textup{\textsf{SCF}}}
\def\ESC{\textup{\textsf{ESC}}}
\def\DSC{\textup{\textsf{DSC}}}
\def\SC{\textup{\textsf{SOC}}}
\def\SOC{\textup{\textsf{SOC}}}
\def\SOT{\textup{\textsf{SOT}}}
\def\TOT{\textup{\textsf{TOT}}}
\def\WCF{\textup{\textsf{WCF}}}
\def\HAC{\textup{\textsf{HAC}}}
\def\INT{\textup{\textsf{int}}}
\newcommand{\barn}{\bar \N}
\newcommand{\Ct}{\mathsf{Ct}}
\newcommand{\Tp}{\mathsf{Tp}}
\newcommand{\T}{\mathcal{T}}
\newcommand{\Ps}{{\Bbb P}}
\newcommand{\F}{{\bf F}}
%\newcommand{\I}{{\bf I}}
\newcommand{\rinf}{\rightarrow \infty}
\newcommand{\true}{{\bf tt}\hspace*{1mm}}
\newcommand{\false}{{\bf ff}\hspace*{1mm}}
\newcommand{\RP}{{Real-$PCF$}\hspace*{1mm}}
%\newcommand{\dminus}{\mbox{$\;^\cdot\!\!\!-$}}
\newcommand{\un}{\underline}
\newcommand{\rec}{\mathsf{Rec}}
\newcommand{\PCF}{\mathsf{PCF}}
\newcommand{\LCF}{\mathsf{LCF}}
\newcommand{\Cr}{\mathsf{Cr}}
\newcommand{\m}{{\bf m}}



\usepackage{graphicx}
\usepackage{tikz}
\usetikzlibrary{matrix, shapes.misc}

\setcounter{tocdepth}{3}
\numberwithin{equation}{section}
\numberwithin{thm}{section}

\usepackage{comment}

\begin{document}
\title[Reverse Mathematics of topology]{Reverse Mathematics of topology \\ {\tiny dimension, paracompactness, and splittings}}
%\author{Dag Normann}
%\address{Department of Mathematics, The University 
%of Oslo, P.O. Box 1053, Blindern N-0316 Oslo, Norway}
%\email{dnormann@math.uio.no}
\author{Sam Sanders}
\address{Centre for Advanced Studies, LMU Munich \& Department of Mathematics, TU Darmstadt, Germany}
\email{sasander@me.com}

\begin{abstract}
Reverse Mathematics (RM hereafter) is a program in the foundations of mathematics founded by Friedman and developed extensively by Simpson and others.  
The aim of RM is to find the minimal axioms needed to prove a theorem of ordinary, i.e.\ non-set-theoretic, mathematics.  As suggested by the title, 
this paper deals with the study of the topological notions of \emph{dimension} and \emph{paracompactness}, inside Kohlenbach's \emph{higher-order} RM. 
As to \emph{splittings}, there are some examples in RM of theorems $A, B, C$ such that $A\asa (B\wedge C)$, i.e.\ $A$ can be \emph{split} into two independent (fairly natural) parts $B$ and $C$, and the aforementioned topological notions give rise to a number of splittings involving \emph{highly natural} $A, B, C$.  
Nonetheless, the higher-order picture is markedly different from the second-one: in terms of comprehension axioms, the proof in higher-order RM of e.g.\ the paracompactness of the unit interval requires \emph{full second-order arithmetic}, while the second-order/countable version of paracompactness of the unit interval is provable in the base theory $\RCA_{0}$. 
We obtain similarly `exceptional' results for the \emph{Urysohn identity}, the \emph{Lindel\"of lemma}, and \emph{partitions of unity}.   
\end{abstract}
%
%\setcounter{page}{0}
%\tableofcontents
%\thispagestyle{empty}

\newpage
\maketitle
\thispagestyle{empty}

%\vspace{-0.1cm}
%\begin{quote}
%\emph{The whole is greater than the sum of its parts} (Aristotle)
%\end{quote}

\section{Introduction}\label{intro}
Reverse Mathematics (RM hereafter) is a program in the foundations of mathematics initiated around 1975 by Friedman (\cites{fried,fried2}) and developed extensively by Simpson (\cite{simpson2}) and others.  
We refer to \cite{stillebron} for a basic introduction to RM and to \cite{simpson2, simpson1} for an (updated) overview of RM.  We will assume basic familiarity with RM, the associated `Big Five' systems and the `RM zoo' (\cite{damirzoo}).  
We do introduce Kohlenbach's \emph{higher-order} RM in some detail Section \ref{HORM}.    

\smallskip

Topology studies those properties of space that are invariant under continuous deformations.
The modern subject was started by Poincar\'e's \emph{Analysis Situs} at the end of 19th century, 
and rapid breakthroughs were established by Brouwer in a two-year period starting 1910, as discussed in \cite{godsgeschenk}*{p.\ 168}. 
We generally base ourselves on the standard monograph by Munkres (\cite{munkies}).

\smallskip

Now, the RM of topology has been studied inside the framework of second-order arithmetic in e.g.\ \cite{mummymf, mummyphd, mummy}.
This approach makes heavy use of \emph{coding} to represent uncountable objects via countable approximations.  Hunter develops the higher-order RM of topology in \cite{hunterphd}, and points out some 
potential problems with the aforementioned coding practice.  Hunter's observations constitute our starting point and motivation: 
working in higher-order RM, we study the RM of notions like \emph{dimension} and \emph{paracompactness} motivated as follows: the former is among the most basic/fundamental notions of topology, while the latter has already been studied in second-order RM, e.g.\ in the context of metrisation theorems (\cite{simpson2, mummymf}). 

\smallskip

As it turns out, the picture we obtain in higher-order RM is \emph{completely different} from the well-known picture in second-order RM.  
For instance, in terms of comprehension axioms, the proof in higher-order RM of the paracompactness of the unit interval requires \emph{full second-order arithmetic} by Theorem \ref{paramaeremki}, 
while the second-order/countable version of paracompactness of the unit interval is provable in the base theory $\RCA_{0}$ of second-order RM by \cite{simpson2}*{II.7.2}. 
Furthermore, the \emph{Urysohn identity} connects various notions of dimension, and a proof of this identity {for $[0,1]$} similarly requires (comprehension axioms as strong as) full second-order arithmetic.  
We also study the \emph{Lindel\"of lemma} and \emph{partitions of unity}.

\smallskip

The aforementioned major difference between second-order and higher-order RM begs the question as to how robust the results in this paper are.  
For instance, do our theorems depend on the exact definition of cover?  What happens if we adopt a more general definition?  We show in Sections \ref{introke} and \ref{kerkend}
that our results indeed boast a lot of robustness, and in particular that they do not depend on the definition of cover, even in the absence of the axiom of (countable) choice. 
The latter feature is important in view of the topological `disasters' (See e.g.\ \cite{kermend}) that manifest themselves in the absence of the axiom of (countable) choice.  

\smallskip

We also obtain a number of highly natural \emph{spittings}, where the latter is defined as follows.  
As discussed in e.g.\ \cite{dsliceke}*{\S6.4}, there are (some) theorems $A, B, C$ in the RM zoo such that $A\asa (B\wedge C)$, i.e.\ $A$ can be \emph{split} into two independent (fairly natural) parts $B$ and $C$ (over $\RCA_{0}$).  
It is fair to say that there are only few \emph{natural} examples of splittings in second-order RM, though such claims are invariably subjective in nature. 
A large number of splittings in higher-order RM may be found in \cite{samsplit}.  
%As to the possibility of $A\asa (B\vee C)$, there is \cite{yukebox}*{Theorem~4.5} which states that a certain theorem about dynamical systems is equivalent to the \emph{disjunction} of weak K\"onig's lemma and induction for $\Sigma_{2}^{0}$-formulas; neither disjunct of course implies the other (over $\RCA_{0}$).
%Similar results are in \cite{boulanger} for model theory, but these are more logical in nature.     
%As it happens, Corollary~\ref{eessje} gives rise to numerous such `disjunctive' results. 
%
%\smallskip
%
%
%It is fair to say that there are only few \emph{natural} examples of splittings and disjunctions in RM, though such claims are invariably subjective in nature. 
%Nonetheless, the aim of this paper is to establish a \emph{plethora} of splittings and disjunctions in {higher-order} RM.  In particular, we obtain splittings and disjunctions involving (higher-order) $\WWKL_{0}$, the Big Five, and $\Z_{2}$, among others.  We similarly treat the covering theorems \emph{Cousin's lemma} and \emph{Lindel\"of's lemma} studied in \cite{dagsamIII}.  % where the former gives rise to disjunctions of \emph{arbitrary length}.  
%Our main results are in Section \ref{main}, while a summary may be found in Section \ref{konkelfoes}; our base theories are conservative over $\WKL_{0}$ (or are strictly weaker). 

\smallskip

Finally, like in \cite{dagsamIII, dagsamV}, statements of the form `a proof of this theorem requires full second-order arithmetic' should be interpreted in reference to the usual scale of comprehension axioms that is part of the \emph{G\"odel hierarchy} (See Appendix \ref{kurtzenhier} for the latter).  
The previous statement thus (merely) expresses that there is no proof of this theorem using comprehension axioms restricted to a sub-class, like e.g.\ $\Pi_{k}^{1}$-formulas (with only first and second-order parameters).  An intuitive visual clarification may be found in 
Figure \ref{xxy}, where the statement \emph{the unit interval is paracompact} is shown to be independent of the medium range of the G\"odel hierarchy.  Similarly, when we say `provable without the axiom of choice', we ignore the use of the very weak instances of the latter included in the base theory of higher-order RM.

\smallskip

In conclusion, it goes without saying that our results highlight a \emph{major} difference between second- and higher-order arithmetic, and the associated development of RM.  We leave it the reader to draw conclusions from this observation.    


\section{Preliminaries}\label{preli}
\subsection{Higher-order Reverse Mathematics}\label{HORM}
We sketch Kohlenbach's \emph{higher-order Reverse Mathematics} as introduced in \cite{kohlenbach2}.  In contrast to `classical' RM, higher-order RM makes use of the much richer language of \emph{higher-order arithmetic}.  

\smallskip

As suggested by its name, {higher-order arithmetic} extends second-order arithmetic.  Indeed, while the latter is restricted to numbers and sets of numbers, higher-order arithmetic also has sets of sets of numbers, sets of sets of sets of numbers, et cetera.  
To formalise this idea, we introduce the collection of \emph{all finite types} $\mathbf{T}$, defined by the two clauses:
\begin{center}
(i) $0\in \mathbf{T}$   and   (ii)  If $\sigma, \tau\in \mathbf{T}$ then $( \sigma \di \tau) \in \mathbf{T}$,
\end{center}
where $0$ is the type of natural numbers, and $\sigma\di \tau$ is the type of mappings from objects of type $\sigma$ to objects of type $\tau$.
In this way, $1\equiv 0\di 0$ is the type of functions from numbers to numbers, and where  $n+1\equiv n\di 0$.  Viewing sets as given by characteristic functions, we note that $\Z_{2}$ only includes objects of type $0$ and $1$.    

\smallskip

The language $\L_{\omega}$ includes variables $x^{\rho}, y^{\rho}, z^{\rho},\dots$ of any finite type $\rho\in \mathbf{T}$.  Types may be omitted when they can be inferred from context.  
The constants of $\L_{\omega}$ includes the type $0$ objects $0, 1$ and $ <_{0}, +_{0}, \times_{0},=_{0}$  which are intended to have their usual meaning as operations on $\N$.
Equality at higher types is defined in terms of `$=_{0}$' as follows: for any objects $x^{\tau}, y^{\tau}$, we have
\be\label{aparth}
[x=_{\tau}y] \equiv (\forall z_{1}^{\tau_{1}}\dots z_{k}^{\tau_{k}})[xz_{1}\dots z_{k}=_{0}yz_{1}\dots z_{k}],
\ee
if the type $\tau$ is composed as $\tau\equiv(\tau_{1}\di \dots\di \tau_{k}\di 0)$.  
Furthermore, $\L_{\omega}$ also includes the \emph{recursor constant} $\mathbf{R}_{\sigma}$ for any $\sigma\in \mathbf{T}$, which allows for iteration on type $\sigma$-objects as in the special case \eqref{special}.  
Formulas and terms are defined as usual.  
\bdefi The base theory $\RCAo$ consists of the following axioms:
\begin{enumerate}
\item  Basic axioms expressing that $0, 1, <_{0}, +_{0}, \times_{0}$ form an ordered semi-ring with equality $=_{0}$.
\item Basic axioms defining the well-known $\Pi$ and $\Sigma$ combinators (aka $K$ and $S$ in \cite{avi2}), which allow for the definition of \emph{$\lambda$-abstraction}. 
\item The defining axiom of the recursor constant $\mathbf{R}_{0}$: For $m^{0}$ and $f^{1}$: 
\be\label{special}
\mathbf{R}_{0}(f, m, 0):= m \textup{ and } \mathbf{R}_{0}(f, m, n+1):= f( \mathbf{R}_{0}(f, m, n)).
\ee
\item The \emph{axiom of extensionality}: for all $\rho, \tau\in \mathbf{T}$, we have:
\be\label{EXT}\tag{$\textsf{\textup{E}}_{\rho, \tau}$}  
(\forall  x^{\rho},y^{\rho}, \varphi^{\rho\di \tau}) \big[x=_{\rho} y \di \varphi(x)=_{\tau}\varphi(y)   \big].
\ee 
\item The induction axiom for quantifier-free\footnote{To be absolutely clear, variables (of any finite type) are allowed in quantifier-free formulas of the language $\L_{\omega}$: only quantifiers are banned.} formulas of $\L_{\omega}$.
\item $\QFAC^{1,0}$: The quantifier-free axiom of choice as in Definition \ref{QFAC}.
\end{enumerate}
\edefi
\bdefi\label{QFAC} The axiom $\QFAC$ consists of the following for all $\sigma, \tau \in \textbf{T}$:
\be\tag{$\QFAC^{\sigma,\tau}$}
(\forall x^{\sigma})(\exists y^{\tau})A(x, y)\di (\exists Y^{\sigma\di \tau})(\forall x^{\sigma})A(x, Y(x)),
\ee
for any quantifier-free formula $A$ in the language of $\L_{\omega}$.
\edefi
As discussed in \cite{kohlenbach2}*{\S2}, $\RCAo$ and $\RCA_{0}$ prove the same sentences `up to language' as the latter is set-based and the former function-based.  Recursion as in \eqref{special} is called \emph{primitive recursion}; the class of functionals obtained from $\mathbf{R}_{\rho}$ for all $\rho \in \mathbf{T}$ is called \emph{G\"odel's system $T$} of all (higher-order) primitive recursive functionals.  

\smallskip

We use the usual notations for natural, rational, and real numbers, and the associated functions, as introduced in \cite{kohlenbach2}*{p.\ 288-289}.  
\begin{defi}[Real numbers and related notions in $\RCAo$]\label{keepintireal}\rm~
\begin{enumerate}
\item Natural numbers correspond to type zero objects, and we use `$n^{0}$' and `$n\in \N$' interchangeably.  Rational numbers are defined as signed quotients of natural numbers, and `$q\in \Q$' and `$<_{\Q}$' have their usual meaning.    
\item Real numbers are coded by fast-converging Cauchy sequences $q_{(\cdot)}:\N\di \Q$, i.e.\  such that $(\forall n^{0}, i^{0})(|q_{n}-q_{n+i})|<_{\Q} \frac{1}{2^{n}})$.  
We use Kohlenbach's `hat function' from \cite{kohlenbach2}*{p.\ 289} to guarantee that every $f^{1}$ defines a real number.  
\item We write `$x\in \R$' to express that $x^{1}:=(q^{1}_{(\cdot)})$ represents a real as in the previous item and write $[x](k):=q_{k}$ for the $k$-th approximation of $x$.    
\item Two reals $x, y$ represented by $q_{(\cdot)}$ and $r_{(\cdot)}$ are \emph{equal}, denoted $x=_{\R}y$, if $(\forall n^{0})(|q_{n}-r_{n}|\leq \frac{1}{2^{n-1}})$. Inequality `$<_{\R}$' is defined similarly.  
We sometimes omit the subscript `$\R$' if it is clear from context.           
\item Functions $F:\R\di \R$ are represented by $\Phi^{1\di 1}$ mapping equal reals to equal reals, i.e. $(\forall x , y\in \R)(x=_{\R}y\di \Phi(x)=_{\R}\Phi(y))$.
\item The relation `$x\leq_{\tau}y$' is defined as in \eqref{aparth} but with `$\leq_{0}$' instead of `$=_{0}$'.  Binary sequences are denoted `$f^{1}, g^{1}\leq_{1}1$', but also `$f,g\in C$' or `$f, g\in 2^{\N}$'.  
%\item Sets of type $\rho$ objects $X^{\rho\di 0}, Y^{\rho\di 0}, \dots$ are given by their characteristic functions $f^{\rho\di 0}_{X}$, i.e.\ $(\forall x^{\rho})[x\in X\asa f_{X}(x)=_{0}1]$, where $f_{X}^{\rho\di 0}\leq_{\rho\di 0}1$.  
\end{enumerate}
\end{defi}
Finally, we mention the $\ECF$-interpretation, of which the technical definition may be found in \cite{troelstra1}*{p.\ 138, 2.6}.
Intuitively speaking, the $\ECF$-interpretation $[A]_{\ECF}$ of a formula $A\in \L_{\omega}$ is just $A$ with all variables 
of type two and higher replaced by countable representations of continuous functionals. 
The $\ECF$-interpretation connects $\RCAo$ and $\RCA_{0}$ (See \cite{kohlenbach2}*{Prop.\ 3.1}) in that if $\RCAo$ proves $A$, then $\RCA_{0}$ proves $[A]_{\ECF}$, again `up to language', as $\RCA_{0}$ is 
formulated using sets, and $[A]_{\ECF}$ is formulated using types, namely only using type zero and one objects.  

\subsection{Some axioms of higher-order arithmetic}\label{saxioms}
We introduce some functionals which constitute the counterparts of $\Z_{2}$, and some of the Big Five systems, in higher-order RM.
We use the formulation of these functionals as in \cite{kohlenbach2}.  

\smallskip
\noindent
First of all, $\ACA_{0}$ is readily derived from the following `Turing jump' functional:
\be\label{muk}\tag{$\exists^{2}$}
(\exists \varphi^{2}\leq_{2}1)(\forall f^{1})\big[(\exists n)(f(n)=0) \asa \varphi(f)=0    \big]. 
\ee
and $\ACA_{0}^{\omega}\equiv\RCAo+(\exists^{2})$ proves the same $\Pi_{2}^{1}$-sentences as $\ACA_{0}$ by \cite{yamayamaharehare}*{Theorem~2.2}. %  The (unique) functional $\exists^{2}$ in $(\exists^{2})$ is also called the \emph{Turing jump functional}.  
This functional is \emph{discontinuous} at $f=_{1}11\dots$, and $(\exists^{2})$ is equivalent to the existence of $F:\R\di\R$ such that $F(x)=1$ if $x>_{\R}0$, and $0$ otherwise (\cite{kohlenbach2}*{\S3}).  

%\smallskip
%\noindent
%Secondly, $\FIVE$ is readily derived from the following `Suslin functional':
%\be\tag{$S^{2}$}
%(\exists S^{2}\leq_{2}1)(\forall f^{1})\big[  (\exists g^{1})(\forall x^{0})(f(\overline{g}n)=0)\asa S(f)=0  \big], 
%\ee
%and $\FIVE^{\omega}\equiv \RCAo+(S^{2})$ proves the same $\Pi_{3}^{1}$-sentences as $\FIVE$ by \cite{yamayamaharehare}*{Theorem 2.2}.   %The (unique) functional $S^{2}$ in $(S^{2})$ is also called \emph{the Suslin functional} (\cite{kohlenbach2}).
%By definition, the Suslin functional $S^{2}$ can decide whether a $\Sigma_{1}^{1}$-formula (as in the left-hand side of $(S^{2})$) is true or false.   
%Note that we allow formulas with (type one) \emph{function} parameters, but \textbf{not} with (higher type) \emph{functional} parameters.

\smallskip
\noindent
Secondly, full second-order arithmetic $\Z_{2}$ is readily derived from the sentence:
\be\tag{$\exists^{3}$}
(\exists E^{3}\leq_{3}1)(\forall Y^{2})\big[  (\exists f^{1})Y(f)=0\asa E(Y)=0  \big], 
\ee
and we define $\Z_{2}^{\omega}\equiv \RCAo+(\exists^{3})$.   The (unique) functional from $(\exists^{3})$ is also called `$\exists^{3}$', and we will use a similar convention for other functionals.  

\smallskip
\noindent
%Fourth, weak K\"onig's lemma\footnote{Note that we take `$\WKL$' to be the $\L_{2}$-sentence \emph{every infinite binary tree has a path} as in \cite{simpson2}, while the Big Five system $\WKL_{0}$ is $\RCA_{0}+\WKL$, and $\WKL_{0}^{\omega}$ is $\RCAo+\WKL$.} ($\WKL$ hereafter) easily follows from both the `intuitionistic' and `classical' \emph{fan functional}, 
%which are defined as follows:  
%\be\tag{$\MUC$}
%(\exists \Omega^{3})(\forall Y^{2})(\forall f, g\in C)(\overline{f}\Omega(Y)=\overline{g}\Omega(Y)\di Y(f)=Y(g)),
%\ee 
%\be\tag{$\textsf{\textup{FF}}$}\label{FF}
%(\exists \Phi^{3})(\forall Y^{2}\in \textsf{\textup{cont}})(\forall f, g\in C)(\overline{f}\Phi(Y)=\overline{g}\Phi(Y)\di Y(f)=Y(g)),
%\ee
%where `$Y^{2}\in \textsf{cont}$' means that $Y$ is continuous on Baire space $\N^{\N}$.  
%Clearly, $\exists^{2}$ and $\exists^{3}$ are a kind of comprehension axiom.    
Thirdly, the \emph{comprehension for Cantor space} functional, introduced in \cite{dagsamV}, yields a conservative extension of $\WKL_{0}$ by \cite{kohlenbach2}*{Prop.\ 3.15}:
\be\tag{$\kappa_{0}^{3}$}
(\exists \kappa_{0}^{3}\leq_{3}1)(\forall Y^{2})\big[ \kappa_{0}(Y)=0\asa (\exists f\in C)(Y(f)>0)  \big].
\ee
\noindent
Fourth, recall that the Heine-Borel theorem (aka \emph{Cousin's lemma}) states the existence of a finite sub-cover for an open cover of a compact space. 
Now, a functional $\Psi:\R\di \R^{+}$ gives rise to the \emph{canonical} cover $\cup_{x\in I} I_{x}^{\Psi}$ for $I\equiv [0,1]$, where $I_{x}^{\Psi}$ is the open interval $(x-\Psi(x), x+\Psi(x))$.  
Hence, the uncountable cover $\cup_{x\in I} I_{x}^{\Psi}$ has a finite sub-cover by the Heine-Borel theorem; in symbols:
\be\tag{$\HBU$}
(\forall \Psi:\R\di \R^{+})(\exists \langle y_{1}, \dots, y_{k}\rangle){(\forall x\in I)}(\exists i\leq k)(x\in I_{y_{i}}^{\Psi}).
\ee
There is also the highly similar \emph{Lindel\"of lemma} stating the existence of a \emph{countable} sub-cover of possibly non-compact spaces.  We restrict ourselves to $\R$ as follows. 
\be\tag{$\LIN$}
(\forall \Psi:\R\di \R^{+})(\exists \Phi^{0\di 1})(\forall x\in \R)(\exists n^{0})(x\in I^{\Psi}_{\Phi(n)}),
\ee
By the results in \cite{dagsamIII, dagsamV}, $\Z_{2}^{\omega}$ proves $\HBU$, but $\SIXK$ cannot (for $k\geq 1$). The same holds for $\LIN$, if we add $\QFAC^{0,1}$, while the latter implies $\HBU\asa [\WKL+\LIN]$.
The importance/naturalness of $\HBU$ and $\LIN$ is discussed in Section \ref{introke}. 
%Hence, the Heine-Borel theorem for uncountable covers as in $\HBU$ falls \emph{far} outside of the Big Five of RM, as noted at the end of Section \ref{RM}.  
%By \cite{dagsamIII}*{\S3}, many basic properties of the \emph{gauge integral} are equivalent to $\HBU$.  
%By Remark \ref{kloti}, we may drop the requirement that $\Psi$ in $\HBU$ needs to be extensional on the reals, i.e.\ $\Psi$ does not have to satisfy \eqref{RE} from Definition \ref{keepintireal}.

\smallskip

Furthermore, since Cantor space (denoted $C$ or $2^{\N}$) is homeomorphic to a closed subset of $[0,1]$, the former inherits the same property.  
In particular, for any $G^{2}$, the corresponding `canonical cover' of $2^{\N}$ is $\cup_{f\in 2^{\N}}[\overline{f}G(f)]$ where $[\sigma^{0^{*}}]$ is the set of all binary extensions of $\sigma$.  By compactness, there is a finite sequence $\langle f_0 , \ldots , f_n\rangle$ such that the set of $\cup_{i\leq n}[\bar f_{i} F(f_i)]$ still covers $2^{\N}$.  By \cite{dagsamIII}*{Theorem 3.3}, $\HBU$ is equivalent to the same compactness property for $C$, as follows:
\be\tag{$\HBU_{\c}$}
(\forall G^{2})(\exists \langle f_{1}, \dots, f_{k} \rangle ){(\forall f^{1}\leq_{1}1)}(\exists i\leq k)(f\in [\overline{f_{i}}G(f_{i})]).
\ee
We now introduce the specification $\SCF(\Theta)$ for a functional $\Theta^{2\di 1^{*}}$ which computes such a finite sub-cover.  
We refer to such a functional $\Theta$ as a \emph{realiser} for the compactness of Cantor space, and simplify its type to `$3$' to improve readability.
%\bdefi\label{dodier}
%The formula $\SCF(\Theta)$ is as follows for $\Theta^{2\di 1^{*}}$:
\be\label{normaal}\tag{$\SCF(\Theta)$}
(\forall G^{2})(\forall f^{1}\leq_{1}1)(\exists g\in \Theta(G))(f\in [\overline{g}G(g)]).
\ee
%where `$f\in [\overline{g}G(g)]$' is the quantifier-free formula $\overline{f}G(g)=_{0^{*}}\overline{g}G(g)$.
Clearly, there is no unique $\Theta$ as in \ref{normaal} (just add more binary sequences to $\Theta(G)$); nonetheless, 
we have in the past referred to any $\Theta$ satisfying $\SCF(\Theta)$ as `the' \emph{special fan functional} $\Theta$, and we will continue this abuse of language.  

%Finally, we need a `trivially uniform' version of $\ATR_{0}$:
%\be\tag{$\UATR$}
%(\exists \Phi^{1\di 1})(\forall X^{1}, f^{1})\big[\WO(X)\di H_{f}(X, \Phi(X,f)) \big], 
%\ee
%where $\WO(X)$ expresses that $X$ is a countable well-ordering and $H_{\theta}(X, Y)$ expresses that $Y$ is the result from iterating $\theta$ along $X$ (See \cite{simpson2}*{V} for details), and where $H_{f}(X, Y)$ is just $H_{\theta}(X, Y)$ with $\theta(n, Z)$ defined as $(\exists m^{0})(f(n,m, \overline{Z}m)=0)$.


\section{Reverse Mathematics of Topology}
We study the RM of theorems of topology pertaining to the following notions: (topological) dimension and the Urysohn identity (Section \ref{diemensie}) and paracompactness (Section \ref{diemensie2}). 
We introduce a suitable notion of cover (Section \ref{introke}) and show (Section \ref{kerkend}) that our aforementioned results are independent of the definition of cover, without making use of the axiom of choice. 
We discuss similar results for the Lindel\"of lemma and partitions of unity (Section \ref{kerkend}).    
\subsection{Introduction: topology in higher-order arithmetic}\label{introke}
We discuss how higher-order arithmetic can accommodate the central topological notion of cover.  
In particular, we introduce a generalisation of the notion of cover used in \cite{dagsamIII, dagsamV} and shall show in Section \ref{kerkend} that the new notion yields covering lemmas
equivalent to the original, \emph{without} a need for the axiom of countable choice. 

\smallskip

First of all, early covering lemmas, like the \emph{Cousin and Lindel\"of lemmas}, did not make use of the (general) notion of cover. 
Indeed, Cousin and Lindel\"of talk about (uncountable) covers of $\R^{2}$ and $\R^{n}$ as follows (resp.\ in 1895 and 1903): 
\begin{quote}
we suppose that to each point of $S$ corresponds a circle of non-zero finite radius and with this point as centre (\cite{cousin1}*{p.\ 22} )
\end{quote}
\begin{quote}
for every point $\textsf{\textup{P}}$, let us construct a sphere $\textsf{\textup{S}}_{\textsf{\textup{P}}}$ with $\textsf{P}$ as the centre\\ and a variable radius $\rho_{\textsf{\textup{P}}}$ (\cite{blindeloef}*{p.\ 698}))
\end{quote}
%In particular, any $\Psi:[0,1]\di \R^{+}$ gives rise to a cover \emph{in the sense of the previous quote by Cousin} of the unit interval as follows: $\cup_{x\in [0,1]}(x-\Psi(x), x+\Psi(x))$ covers $[0,1]$.  
To stay close to the original formulation by Cousin and Lindel\"of, we introduced in \cite{dagsamIII, dagsamV} the notion of `canonical' open covers $\cup_{x\in I}I_{x}^{\Psi}$ of $I\equiv [0,1]$ generated by $\Psi:I\di \R^{+}$ and where $I_{x}^{\Psi}\equiv (x-\Psi(x), x+\Psi(x))$.
Unfortunately, such covers always involve points that are covered by arbitrarily many intervals; this property makes such covers unsuitable for e.g.\ the study of topological dimension, in which the (minimal) number of intervals covering a point is central.  

\smallskip

Secondly, the previous observation shows that we have to generalise our notion of canonical cover, and we shall do this by considering $\psi:I\di \R$. i.e.\ we allow empty $I_{x}^{\psi}$. 
In this way, we say that `$\cup_{x\in I}I_{x}^{\psi}$ covers $[0,1]$' if $(\forall x\in I)(\exists y\in I)(x\in I_{y}^{\psi})$.  
This notion of cover gives rise to the following version of the Heine-Borel theorem. 
\be\tag{$\HBT$}
(\forall \psi:I\di \R)\big[ I\subset \cup_{x\in I}I_{x}^{\psi}\di   (\exists y_{1}, \dots, y_{k} \in I)(I\subset \cup_{i\leq k}I_{y_{i}}^{\psi}) \big].
\ee
We establish in Section \ref{kerkend} that our `new' notion of cover is quite robust by showing that (i) $\HBU\asa \HBT$ over $\RCAo+\QFAC^{1,1}$, i.e.\ the new notion of cover is not a real departure from the old one, and (ii) the previous equivalence can also be proved without the axiom of choice.  
Item (ii) should be viewed in the light of the topological `disasters' (See e.g.\ \cite{kermend}) that apparently happen in the absence of the axiom of (countable) choice.  We also show that any notion of cover definable in $\Z_{2}^{\omega}$ inherits the aforementioned `nice' properties. 
Thus, we may conclude that our results boast a lot of robustness, and in particular that they do not depend on the definition of cover, even in the absence of the axiom of (countable) choice. 
 
\smallskip

Finally, we discuss the mathematical naturalness of $\HBU$ and $(\exists^{2})$.
\begin{rem}\rm
Dirichlet already discusses the  characteristic function of the rationals, which is essentially $\exists^{2}$, around 1829 in \cite{didi1}, while Riemann defines a function with countably many discontinuities via a series in his \emph{Habilitationsschrift} (\cite{kleine}*{p.~115}).  
Furthermore, the \emph{Cousin lemma} from \cite{cousin1}*{p.\ 22}, which is essentially $\HBU$, dates back\footnote{The collected works of Pincherle contain a footnote by the editors (See \cite{tepelpinch}*{p.\ 67}) which states that the associated \emph{Teorema} (published in 1882) corresponds to the Heine-Borel theorem.  Moreover, Weierstrass proves the Heine-Borel theorem (without explicitly formulating it) in 1880 in \cite{amaimennewekker}*{p.\ 204}.   A detailed motivation for these claims may be found in \cite{medvet}*{p. 96-97}.} about 135 years.  %while the \emph{Lindel\"of lemma} (\cite{blindeloef}*{p.\ 698}) dates back about 115 years.  
As shown in \cite{dagsamIII}, $(\exists^{2})$ and $\HBU$ are essential for the development of the \emph{gauge integral} (\cite{bartle1337}).  This integral was introduced by Denjoy (\cite{ohjoy}), in a different, more complicated form, around the same time as the Lebesgue integral; the reformulation of Denjoy's integral by Henstock and Kurzweil in Riemann-esque terms (See \cite{bartle1337}*{p.\ 15}), provides a \emph{direct} and elegant formalisation of the \emph{Feynman path integral} (\cites{burkdegardener,mullingitover,secondmulling}) and financial mathematics (\cites{mulkerror, secondmulling}).    
\end{rem}

\subsection{The notion of dimension}\label{diemensie}
The notion of dimension of basic spaces like $[0,1]$ or $\R^{n}$ is intuitively clear to most mathematicians, but finding a formal definition of dimension \emph{that does not depend on the topology} is a non-trivial problem.  

\smallskip

We introduce three notions of dimension: the  topological dimension $\dim X$ and the small and large inductive dimensions $\textup{ind } X$ and $\textup{Ind } X$.  We study the RM properties of the \emph{Uryoshn identity} (\cite{enc2}*{p.\ 272}) which expresses that these dimension are equal for a large class of spaces, including separable metric spaces.

\smallskip

First of all, the \emph{covering dimension}, later generalised to the \emph{topological dimension}, goes back to Lebesgue.  
Indeed, Munkres writes the following:
%As suggested by its name
%Lebesgue started it, then Brouwer. (See \cite{godsgeschenk}*{p.\ 191})
\begin{quote}
We shall define, for an arbitrary topological space $X$, a notion of topological dimension. It is the ``covering dimension" originally defined by Lebesgue.  (\cite{munkies}*{p.\ 305})
\end{quote}
The following definition of topological dimension may be found in Munkres' seminal monograph \cite{munkies}*{p.\ 161}, and in \cite{enc2}*{p.\ 274}, \cite{engeltjemijn}*{Ex.\ 1.7.E and Prop.\ 3.2.2}.  % for a characterisation involving this definition. .
\bdefi[Order]
A collection $\mathcal{A}$ of subsets of the space $X$ is said to have order $m + 1$ if
some point of $A$ lies in $m +1$ elements of $\mathcal{A}$, and no point of $X$ lies in more than $m +1$
elements of $A$.
\edefi
\bdefi[Refinement]
Given a collection $\mathcal{A}$ of subsets of $X$, a collection $\mathcal{B}$ is said to refine $\mathcal{A}$, or to be
a refinement of $\mathcal{A}$ if for each element $B \in \mathcal{B}$ there is an element $A\in \mathcal{A}$ such that $A\subset B$.
\edefi
\bdefi[Topological dimension]
A space $X$ is said to be \textbf{finite-dimensional} if there is $m\in\N$ such
that for every open covering $\mathcal{A}$ of $X$, there is an open covering $\mathcal{B}$ of $X$ that refines $\mathcal{A}$
and has order at most $m + 1$. The topological dimension of $X$ is the
smallest value of $m$ for which this statement holds; we denote it by $\dim X$.
\edefi
In the context of $\RCAo$, we say that `$\phi:I\di \R$ is a \emph{refinement} of $\psi:I\di \R$' if $(\forall x\in I)(\exists y\in I)(I_{x}^{\phi}\subseteq I_{y}^{\psi})$.  
With this definition in place, statements like `the topological dimension of $[0,1]$ is at most $1$', denoted `$\dim([0,1])\leq1$', makes perfect sense in $\RCAo$.  
Such a statement turns out to be quite hard to prove, as full second-order arithmetic is needed to prove $\HBT$ by Theorem \ref{ziedenauw}.  
\begin{thm}\label{rathergen}
The system $\ACAo+\QFAC^{1,1}+[\dim([0,1])\leq1]$ proves $\HBT$. 
\end{thm}
\begin{proof}
Let $\psi:I\di \R$ be such that $\cup_{x\in I}I_{x}^{\psi}$ covers $[0, 1]$, and let $\phi:I\di \R$ be the associated refinement of order at most $1$.  
Since the innermost formula is $\Sigma_{1}^{0}$ (with parameters), we may apply $\QFAC^{1,1}$ to $(\forall x\in I)(\exists y\in I)(x\in I_{y}^{\phi})$ to obtain $\Xi^{1\di 1}$ such that $\Xi(x)$ provides such $y$.
Define $\zeta^{0\di 1}$ as follows: $\zeta(0):=\Xi(0)+ \phi(\Xi(0))$ and $\zeta(n+1):= \Xi(\zeta(n))+\phi(\Xi(\zeta(n)))$.   Now consider the following formula:
\be\label{tuigs}
(\exists x\in I )(\forall n\in \N)(\zeta(n)<_{\R}x).
\ee
If \eqref{tuigs} is false, take $x=1$ and note that if $\zeta(n_{0})\geq_{\R}1$, the finite sequence $I_{\Xi(0)}^{\phi}, I_{\Xi(\zeta(0))}^{\phi}, I_{\Xi(\zeta(1))}^{\phi}, \dots, I_{\Xi(\zeta(n_{0}+1))}^{\phi}$ yield a finite sub-cover of $\cup_{x\in I}I_{x}^{\phi}$.  In this case, we apply $\QFAC^{1,1}$ (using also $(\exists^{2})$) to $(\forall x\in I)(\exists y\in I)(I_{x}^{\phi}\subseteq I_{y}^{\psi})$ 
to go from a finite sub-cover of $\cup_{x\in I}I_{x}^{\phi}$ to a finite sub-cover of $\cup_{x\in I}I_{x}^{\psi}$, and $\HBT$ follows.  

\medskip

If \eqref{tuigs} is true, let $x_{0}\in I$ be the least $x\in I$ such that $\varphi(x)\equiv(\forall n\in \N)(\zeta(n)<_{\R}x)$.  Since $\varphi(x)$ is $\Pi_{1}^{0}$, we can use $\exists^{2}$ and the usual interval-halving technique to find $x_{0}$; alternatively, use the monotone convergence theorem (\cite{simpson2}*{III.2.2}), provable in $\ACA_{0}$.  
However, $I_{\Xi(x_{0})}^{\phi}$ covers $x_{0}$, and thus for $n_{1}$ large enough, $\zeta(n)$ for $n\geq n_{1}$ will all be in the former interval, by the leastness of $x_{0}$.  But then there are points of order $3$ in the (by definition non-empty) intersection of $I_{\Xi(\zeta(n_{1}))}^{\phi}$ and $I_{\Xi(\zeta(n_{1}+1))}^{\phi}$, as this intersection is also inside $I_{\Xi(x_{0})}^{\phi}$.  This observation contradicts the assumption $\dim([0,1])\leq1$, and hence \eqref{tuigs} must be false, and we are done.  
%The sequence $\zeta$ has a convergent sub-sequence $\$
\end{proof}
The previous theorem has a number of corollaries.  First of all, we obtain an equivalence over a weak base theory; we believe the components of the left-hand side to be independent\footnote{Firstly, $\WKL$ cannot imply the other component, as $\RCAo+(\exists^{2})+\QFAC$ does not imply $\HBU$ by the results in \cite{dagsam}*{\S6}.  Secondly, $\dim(I)= 1$ seems to be consistent with recursive mathematics by the proof of \cite{beeson1}*{Theorem 6.1, p.\ 69}, i.e.\ the former cannot imply $\WKL$.}, i.e.\ that a proper `splitting' of $\HBT$ is achieved.

%as $\RCAo+[\dim(I)\leq 1]$ is a conservative\footnote{To see this, note that the $\ECF$-translation of `$\dim(I)\leq1$' is readily proved as in the final case in the proof of the corollary, as this translation essentially replaces type two functionals by RM-codes for continuous functionals.} extension of $\RCA_{0}$.
% It is a natural question whether $\QFAC^{0,1}$ suffices for this corollary. 
\begin{cor}\label{dorkeeee}
$\RCAo+\QFAC^{1,1}$ proves that $\big(\WKL+ [\dim(I)=1]\big)\asa \HBT$.
\end{cor}
\begin{proof}
For the forward direction, in case $(\exists^{2})$, the proof of the theorem goes through.  In case $\neg(\exists^{2})$, all $F:\R\di \R$ are continuous, while all $F^{2}$ are continuous on Baire space, and hence uniformly continuous (and thus bounded) on Cantor space by $\WKL$ (See \cite{kohlenbach2}*{Prop.\ 3.7 and 3.12} and \cite{kohlenbach4}*{Prop.~4.10}).  Now consider the following statement, which (only) holds since $\psi:I\di \R$ is continuous:
\be\label{piew}\textstyle
(\forall f\in C)(\exists q\in I\cap \Q)(\exists n\in \N)\underline{(\r(f)\in I_{q}^{\psi}\wedge \psi(q)\geq \frac{1}{2^{n}})}, 
\ee
where $\r(f)$ is $\sum_{n=0}^{\infty}\frac{f(n)}{2^{n}}$ for binary $f$, and where the underlined formula is $\Sigma^{0}_{1}$.  
Applying $\QFAC^{1,0}$ to \eqref{piew}, there is $\Xi^{2}$ such that $n\leq \Xi(f)$ in \eqref{piew}.  Since $\Xi$ is bounded on $C$, there is $N_{0}\in \N$ such that
\be\label{pieq2}\textstyle
(\forall f\in C)(\exists q\in I\cap \Q){(\r(f)\in I_{q}^{\psi}\wedge \psi(q)\geq \frac{1}{2^{N_{0}}})}, 
\ee
which immediately implies that $\cup_{x\in I}I_{x}^{\psi}$ has a finite sub-cover (generated by rationals), and the latter may be found by applying $\QFAC^{1,0}$ to \eqref{pieq2} and iterating the choice function at most $2^{N_{0}+1}$ times. 
Since $\frac{i}{2^{n}}$ has an obvious binary representation, we do not need to convert arbitrary $x\in I$ to binary.  
We obtain $\HBT$ in each case, and $(\exists^{2})\vee \neg(\exists^{2})$ finishes the proof. 

\smallskip

For the reverse direction, note that $\HBT\di \HBU\di \WKL$.  To prove $\dim(I)=1$, the finite sub-cover provided by $\HBT$ is readily converted to a refinement of order $1$ using $\exists^{2}$, as the latter functional can decide equality between real numbers.  
Now, in case $\neg(\exists^{2})$, obtain \eqref{pieq2} in the same way as above, and let $\Xi$ be a choice function that provides $\Xi(f)=q$.  
Define $\zeta$ as follows: $\zeta(0):=\Xi(00\dots)+\frac{1}{2^{N_{0}}} $ and $\zeta(n+1):= \Xi(\zeta(n))+\frac{1}{2^{N_{0}}}$.  
For $n> 2^{N_{0}+1}$, this function readily yields a finite open cover of $I$ that is also a refinement of the cover generated by $\psi$. 
Since all points are rationals, we can refine this cover to have order $1$, and $(\exists^{2})\vee \neg(\exists^{2})$ finishes the proof. 
\end{proof}
For future reference, we note that the proof actually establishes $\RCAo+\neg(\exists^{2})\vdash \big[[\WKL+\dim(I)=1]\asa \HBT\big]$, i.e.\ the axiom of choice is not used. 

\smallskip

It is a natural question (posed before by Hirschfeldt; see \cite{montahue}*{\S6.1}) whether the axiom of choice is really necessary in the previous (and below) theorems.  
We answer this question in the negative in Section \ref{kerkend}.

\smallskip

%\noindent A reversal can be proved in much the same way as in \cite{dagsamV}*{\S5}. 
%
%\smallskip
%
Next, in order to prove the next corollary concerning Urysohn's identity, we introduce the notion of inductive definition as in \cite{engeltjemijn}*{\S1.1.1}.
\bdefi[Inductive dimension]\label{roolin} We inductively define the emph{small inductive dimension}$ \ind X$ for a topological space $X$ as follows. 
\begin{enumerate}
\item[(d1)] For the empty set $\emptyset$, we define ${\textup{ind }} \emptyset={\textup{Ind }}\emptyset=-1$;
\item[(d2)] $\ind X\leq n$, where $n = 0,1,\dots,$ if for every point $x \in X$ and each neighbourhood $V \subset X$ of the point $x$ there exists an open set $U \subset X$ such that $x\in U\subset V$ and $\ind (\partial U)<n-1$;
\item[(d3)]  $\ind X = n$ if $\ind X \leq n$ and $\ind X > n- 1$, i.e., the inequality $\ind X <n- 1$ does not hold;
\item[(d4)] $\ind X= \infty$ if $\ind X>n$ for $n=-1,0,1,...$.  % and the same for $\Ind X$.
\end{enumerate}
The \emph{large inductive dimension} $\Ind X$ is obtained by replacing (d2) by:
\begin{enumerate}
\item[(d$2^{*}$)] $\Ind X<n$, where $n=0,1,...,$ if for every closed set $A\subset X$ and
each open set $V \subset X$ which contains the set $A$ there exists an open set $U \subset X$ such that $A\subset U \subset V$ and $\Ind (\partial U)< n-1$.
\end{enumerate}
If $X$ is Euclidean space, $V$ is generally chosen to be a ball centred at $x$.
\edefi 
In light of Definition \ref{roolin}, the (small and large) inductive dimension of real numbers, or the unit interval, makes sense in $\RCAo$, and is respectively $0$ and $1$.  
Moreover, the Urysohn identity is the statement that $\dim X=\textup{ind} X=\textup{Ind} X$, and holds for a large class of spaces $X$; this identity constitutes one of the main problems in \emph{dimension theory}, according to \cite{enc2}*{p.\ 274}, while it is called the \emph{the fundamental theorem of dimension theory} in \cite{engeltjemijn}.
\begin{cor}\label{hoerke}
The system $\RCAo+\QFAC^{1,1}$ proves that $\HBT$ is equivalent to: the conjunction of  $\WKL$ and Urysohn's identity for the unit interval.
\end{cor}
\begin{proof}
Immediate from Corollary \ref{dorkeeee}.
\end{proof}
%Note that we obtain a splitting similar to the one in Corollary \ref{dorkeeee}.


\subsection{Paracompactness}\label{diemensie2}
The notion of \emph{paracompactness} was introduced in 1944 by Dieudonn\'e in \cite{nogeengodsgeschenk} and plays an important role in the characterisation of metrisable spaces via e.g.\ \emph{Smirnov's metrisation theorem} (\cite{munkies}*{p.\ 261}).  
The fact that every metric space is paracompact is \emph{Stone's theorem} (See \cites{goodgoing,stoner2} and \cite{munkies}*{p.\ 252}). 

\smallskip

Our interest in paracompactness stems in part from its occurrence in classical RM (See e.g.\ \cites{simpson2, mummymf, mummyphd}), as detailed in Remark \ref{diemummy}.  
The aim of this section is to show that there is a \emph{huge} difference in logical and computational hardness between the `second-order/countable' version of paracompactness, and the `actual' definition.  
Indeed, the fact that the unit interval is paracompact implies $\HBT$; moreover, the latter can be `split' into the former plus $\WKL$ by Corollary \ref{XYW}.  

\smallskip

Munkres states the following definition of paracompactness in \cite{munkies}*{p.\ 253}.  
\bdefi[Locally finite]
A collection $\mathcal{A}$ of subsets of a space $X$ is \emph{locally finite} if any $x\in X$ has a neighbourhood that intersects only finitely many $A\in \mathcal{A}$.
\edefi
\bdefi[Paracompact]
A space $X$ is \emph{paracompact} if every open covering $\mathcal{A}$ of $X$ has a locally
finite open refinement $\mathcal{B}$ that covers $X$.
\edefi
With these definitions, the statement that the unit interval is paracompact, makes sense in $\RCAo$.  
By \emph{Stone's theorem}, a metric space is paracompact, but this fact is not provable in $\ZF$ alone (See \cite{goodgoing}).  
Similarly, Stone's theorem \emph{for the unit interval} is not provable in any system $\SIXK$ by the following theorem.
\begin{thm}\label{paramaeremki}
The system $\ACAo+\QFAC^{1,1}+ \textup{`$[0,1]$ is paracompact'}$ proves $\HBT$. 
\end{thm}
\begin{proof}
We use the proof of Theorem \ref{rathergen} with minor modification.
Let $\psi:I\di \R$ be such that $\cup_{x\in I}I_{x}^{\psi}$ covers $[0, 1]$, and let $\phi:I\di \R$ be a locally finite refinement. % of order at most $1$.  
Assume \eqref{tuigs}, where let $x_{0}\in I,\zeta^{0\di 1}, \Xi^{1,1}$ are as in the aforementioned proof.    
Clearly, any neighbourhood of $x_{0}$ will contain all intervals $I^{\phi}_{\Xi(\zeta(n))}$ for $n$ large enough.  
This observation contradicts the assumption that $[0,1]$ is paracompact, and hence \eqref{tuigs} must be false, implying $\HBT$ as in the proof of Theorem \ref{rathergen}.  
\end{proof}
The following corollary is proved in the same way as Corollary \ref{dorkeeee}; the left-hand side constitutes a proper `splitting' of $\HBT$, as the $\ECF$-translation of `$[0,1]$ is paracompact' is essentially the statement that $[0,1]$ is \emph{countably} paracompact, and the latter is provable in $\RCA_{0}$ by \cite{simpson2}*{II.7.2}.
\begin{cor}\label{XYW}
 $\RCAo+\QFAC^{1,1}$ proves $[\WKL + \textup{`$[0,1]$ is paracompact'}]\asa \HBT$. 
\end{cor}
%\begin{proof}
%X
%\end{proof}
Another interpretation of the previous corollary is as follows: by the results in \cite{wienszoon}, the notion of compactness is equivalent to `paracompact plus pseudo-compact' for a large class of spaces, and pseudo-compactness essentially expresses that continuous functions are bounded on the space at hand, i.e.\ the pseudo-compactness of $[0,1]$ is equivalent to $\WKL$ by \cite{simpson2}*{IV.2.3} and \cite{kohlenbach4}*{Prop.\ 4.10}. 

\smallskip

%The following corollary is proved in the same way as Theorem \ref{paramaeremki}.
%\begin{cor}\label{paramaeremkies}
%The system $\RCAo+\textup{`$\R$ is paracompact'}$ proves $\LIN$. 
%\end{cor}
The following remark highlights the difference between `actual' and `second-order/countable' paracompactness.  It also suggests formulating Corollary \ref{thetam}. 
\begin{rem}[Paracompactness in second-order RM]\label{diemummy}\rm
Simpson proves in \cite{simpson2}*{II.7.2} that over $\RCA_{0}$, complete separable metric spaces are \emph{countably} paracompact\footnote{The notion of `countably paracompact' is well-known from Dowker's theorem (See e.g.\ \cite{ooskelly}*{p.\ 172}), but Simpson and Mummert do not use the qualifier `countable' in \cites{simpson2,mummymf}.}, 
and Mummert in \cite{mummymf}*{Lemma 4.11} defines a realiser for paracompactness as in \cite{simpson2}*{II.7.2} inside $\ACA_{0}$.  This realiser plays a crucial role in the proof of Mummert's metrisation theorem, called `MFMT', inside $\SIX$ (See \cite{mummymf}*{\S4}).  
Note that $\SIX$ occurs elsewhere in the RM of topology (\cite{mummy, mummyphd}).  By Theorem~\ref{paramaeremki}, the (higher-order) statement \emph{the unit interval is paracompact} is equivalent to $\HBT$, and hence not provable in $\cup_{k}\SIXK$, i.e.\ there is a 
\emph{huge} difference in strength between `second-order/countable' and `actual' paracompactness.  In fact, the logical hardness of the aforementioned statement dwarfs $\SIX$ from the RM of topology.  
\end{rem}
Let us call $\Omega^{\mathbb{1}\di\mathbb{1}}$ a `realiser for the paracompactness of $[0,1]$' if $\Omega(\psi)(1):I\di \R$ yields a locally finite open refinement of the cover associated to $\psi:I\di \R$, and if 
\be\label{popol}
(\forall x\in I)(I_{x}^{\Omega(\psi)(1)}\subseteq I_{\Omega(\psi)(2)(x)}^{\psi}),
\ee
i.e.\ the refining cover is `effectively' included in the original one, just like in \cite{simpson2, mummymf}.
\begin{cor}\label{thetam}
A realiser $\Omega^{\mathbb{1}\di \mathbb{1}}$ for the paracompactness of $[0,1]$, together with Feferman's $\mu$, computes $\Theta$ such that $\SCF(\Theta)$ via a term of G\"odel's $T$.
\end{cor}
\begin{proof}
Immediate from the proof of Theorems \ref{rathergen} and \ref{paramaeremki}.  Note that $\Xi$ is the identity function in case we consider covers generated by $\Psi:I\di \R^{+}$ as in $\HBU$. 
Furthermore, a realiser for $\HBU$ computes a realiser for $\HBU_{\c}$, i.e.\ the special fan functional, via a term in G\"odel's $T$, as discussed in \cite{dagsamIII}*{\S3.1}
\end{proof}
As it turns out, the condition \eqref{popol} for a realiser for paracompactness has already been considered, namely as follows.  
\begin{quote}
all proofs of Stone's Theorem (known to the authors) actually prove a stronger conclusion which implies $\mathsf{AC}$. It
is based on an idea from [\dots]. Let us call a refinement $\mathcal{V}$ of $\mathcal{U}$ \emph{effective} if there is
a function $a : \mathcal{V} \di \mathcal{U}$ such that $V \subset a(V)$ for all $V \in  \mathcal{V}$. (\cite{goodgoing}*{p.\ 1217})
\end{quote}
As it turns out, the notion of `effectively paracompact' is intimately connected to the Lindel\"of lemma, as discussed in Section \ref{unite}. 
%Hence, the functional $\Omega$ turns out to be fairly natural (relative to set theory). 
%In conclusion, 
%Finally, we discuss the notion of \emph{partitions of unity} which was formally introduced by Dieudonne (C. R. 205 (1937) 593-595), while some special cases they were used by Whitney (TAMS, 36 (1934) 63-89). T}


%\subsection{The Lindel\"of property}



\subsection{Covers in higher-order arithmetic}\label{kerkend}
In Section \ref{introke}, we introduced a generalisation of the notion of cover used in \cite{dagsamIII, dagsamV}.  
In this section, we show that the new notion yields covering lemmas equivalent to the original, even in the absence of the axiom of choice. 
We also show that any notion of cover definable in second-order arithmetic inherits these `nice' properties.  We treat the Heine-Borel theorem, the Lindel\"of lemma, as well as theorems pertaining to partitions of unity. 
\subsubsection{The Heine-Borel theorem}
We prove $\HBT\asa \HBU$ with and without the axiom of choice in the base theory. 
In this way, our new notion of cover does not really change the Heine-Borel theorem.
\begin{thm}\label{ziedenauw}
The system $\RCA_{0}^{\omega}+\QFAC^{1,1}$ proves $\HBU\asa \HBT$.
\end{thm}
\begin{proof}
The reverse direction is immediate.  For the forward direction, in case $\neg(\exists^{2})$, we obtain $\HBU\di \WKL$ and proceed as in the proof of Corollary \ref{dorkeeee}. 
In case of $(\exists^{2})$, let $\psi$ be as in $\HBT$ and consider $(\forall x\in I)(\exists y\in I)(x\in I_{y}^{\psi})$.  Since the innermost formula is $\Sigma_{1}^{0}$, we may 
apply $\QFAC^{1,1}$ to obtain $\Xi$ such that $(\forall x\in I)(x\in I_{\Xi(x)}^{\psi})$.  Since $\exists^{2}$ provides a functional that converts real numbers in $I$ to a unique binary representation, we may assume that $\Xi$ is extensional on the reals.   
Now define $\Psi:I\di \R^{+}$ by $\Psi(x):=\min\big(|x-(\Xi(x)-\psi(\Xi(x)))|, |x-(\Xi(x)+\psi(\Xi(x)))|\big)$, and note that $I^{\Psi}_{x}\subseteq I_{\Xi(x)}^{\psi}$.  
Applying $\HBU$, we obtain a finite sub-cover of $\cup_{x\in I}I_{x}^{\Psi}$, say generated by $y_{1}, \dots, y_{k}\in I $, and $\cup_{i\leq k}I_{\Xi(y_{i})}^{\psi}$ is then a finite sub-cover of $\cup_{x\in I}I_{x}^{\psi}$.
\end{proof}
Recall that $\HBU$ is provable in $\Z_{2}^{\omega}$ by \cite{dagsamV}*{\S4}, i.e.\ without the axiom of choice.  
While the use of $\QFAC^{1,1}$ in $\HBU\di \HBT$ \emph{seems} essential, it is in fact not, by the following theorem.  
Note that $\textsf{IND}$ is the induction axiom for all formulas in the language of $\RCAo$; the base theory is not stronger than Peano arithmetic. 
\begin{thm}\label{fugu}
The system $\RCAo+\textsf{\textup{IND}}+(\kappa_{0}^{3})$ proves $\HBU\asa \HBT$
\end{thm}
\begin{proof}
The reverse direction is immediate.  For the forward direction, in case $\neg(\exists^{2})$, we obtain $\HBU\di \WKL$ and proceed as in the proof of Corollary \ref{dorkeeee}. 
In case of $(\exists^{2})$, let $\psi$ be as in $\HBT$ and note that $(\forall x\in I)(\exists y\in I)(x\in I_{y}^{\psi})$ implies:
\be\label{centrifuge}\textstyle
(\forall x\in I)(\exists n\in \N)\underline{(\exists y\in I)((x-\frac{1}{2^{n}}, x+\frac{1}{2^{n}})\subseteq I_{y}^{\psi})}, 
\ee
where the underlined formula is decidable thanks to $(\exists^{3})\equiv [(\exists^{2}) + (\kappa_{0}^{3})]$.
Hence, applying $\QFAC^{1,0}$ to \eqref{centrifuge}, we obtain $\Psi:I\di \R^{+}$ such that $\cup_{x\in I}I_{x}^{\Psi}$ is a canonical cover of $I$.  
%Since the innermost formula is $\Sigma_{1}^{0}$, we may 
%apply $\QFAC^{1,1}$ to obtain $\Xi$ such that $(\forall x\in I)(x\in I_{\Xi(x)}^{\psi})$.  Since $\exists^{2}$ provides a functional that converts real numbers in $I$ to a unique binary representation, we may assume that $\Xi$ is extensional on the reals.   
%Now define $\Psi:I\di \R^{+}$ by $\Psi(x):=\min\big(|x-(\Xi(x)-\psi(\Xi(x)))|, |x-(\Xi(x)+\psi(\Xi(x)))|\big)$, and note that $I^{\Psi}_{x}\subseteq I_{\Xi(x)}^{\psi}$.  
Applying $\HBU$, we obtain a finite sub-cover of $\cup_{x\in I}I_{x}^{\Psi}$, say generated by $x_{1}, \dots, x_{k}\in I $.  
By definition, we have $(\forall x\in I)(\exists y\in I)(I^{\Psi}_{x}\subseteq I_{y}^{\psi})$, and 
\be\label{conplete}
(\forall w^{1^{*}})(\exists v^{1^{*}})(\forall i<|w|)(I^{\Psi}_{w(i)}\subseteq I_{v(i)}^{\psi})
\ee
follows from $\textsf{IND}$ by induction on $|w|$.  Applying \eqref{conplete} for $w=\langle x_{1}, \dots, x_{k}\rangle$, we obtain a finite sub-cover for $\cup_{x\in I}I_{x}^{\psi}$. The law of excluded middle finishes the proof. 
% and $\cup_{i\leq k}I_{\Xi(y_{i})}^{\psi}$ is then a finite sub-cover of $\cup_{x\in I}I_{x}^{\psi}$.
\end{proof}
As to open questions, we do not know if the base theory proves $\HBT$ outright or not.  
Similarly, we do not know if $\RCAo+(\kappa_{0}^{3})$ proves $\WKL$ or not.  

\smallskip

%WRONG
%We do know that $(\kappa_{0}^{3})$ can be omitted, at the price of only obtaining $\HBU$.  
%\begin{cor}
%The system $\RCAo+\textsf{\textup{IND}}+ \textup{`$[0,1]$ is paracompact'}$ proves $\HBU$.
%\end{cor}
%\begin{proof}
%For the reverse direction, follow the proof of Theorem \ref{paramaeremki} noting that we do not need $\Xi$ from the proof of Theorem \ref{rathergen}.  
%Replace the final use of $\QFAC^{1,1}$ in the aforementioned proof by induction as in the proof of the theorem.   
%\end{proof}
In conclusion, we mention two important observations that stem from the above.

\smallskip

First of all, it is easy to see that the first two proofs go through for the Heine-Borel theorem for $[0,1]$ based on {any `reasonable' notion} of cover.
Indeed, as long as the formulas `$x\in U_{y}$' and `$[a,b]\subseteq U_{x}$' for the new notion of cover $\cup_{x\in I}U_{x}$ of $I$ are decidable in $\Z_{2}^{\omega}$, the above proofs go through (assuming $(\kappa_{0}^{3})$).  
Since $\Z_{2}^{\omega}$ can decide if $Y:\R\di \{0,1\}$ represents an open subset of $\R$ (using the textbook definition of open set), this notion of `reasonable' seems quite reasonable. 

\smallskip

Secondly, emulating the proof of Theorem \ref{fugu}, we observe that the above results go through in the base theory with $(\kappa_{0}^{3})+\textsf{IND}$ instead of $\QFAC^{1,1}$.  
These include Theorem~\ref{rathergen}, Corollary \ref{dorkeeee}, Corollary \ref{hoerke}, Theorem~\ref{paramaeremki}, and Corollary~\ref{XYW}. 
Thus, these results do not require the axiom of choice.  


\subsubsection{The Lindel\"of lemma}
We show that the Lindel\"of lemma does not depend on the definition of cover, similar to the case of the Heine-Borel theorem.  
On one hand, since $[\LIN+\WKL]\asa \HBU$, one expects such results.  
On the other hand, as shown in \cite{dagsamV}*{\S5}, the strength of the Lindel\"of lemma is highly dependent on the exact\footnote{The countable sub-cover in the Lindel\"of lemma can be given by a sequence of reals generating the intervals (strong version), or just a sequence of intervals (weak version). 
The strong version implies $\QFAC^{0,1}$ and hence is unprovable in $\ZF$, while the weak version is provable in $\Z_{2}^{\omega}$.  
} formulation, but this dependence is not problematic for our context.  
  
\smallskip  
  
We introduce the notion of cover used in \cite{dagsamV}*{\S5}, as follows.
%We only sketch our results for the Lindel\"of lemma, bearing in mind the previous.
We consider $\psi:I\di \R^{2}$ and covers $\cup_{x\in I}J_{x}^{\psi}$ in which the interval $J_{x}^{\psi}:=(\psi(x)(1), \psi(x)(2))$ is potentially empty but $(\forall x\in I)(\exists y\in I)(x\in J_{y}^{\psi})$. 
This notion of cover yields a `strong' version of the Lindel\"of lemma, as follows.  % similar to $\HBT$. 
\be\tag{$\LIL$}
(\forall \psi:\R\di \R^{2})\big[ \R\subseteq \cup_{x\in \R}J_{x}^{\psi}\di   (\exists f:\N\di \R)(\R\subset \cup_{n\in \N}J_{f(n)}^{\psi}) \big].
\ee
Similar to the proof of \cite{dagsamIII}*{Theorem 3.13}, one proves that $\HBT\asa [\WKL+\LIL]$ over $\RCAo+\QFAC^{1,1}$.
We first prove that the Lindel\"of lemma $\LIL$ is equivalent to $\LIN$ from \cite{dagsamIII}*{\S3}. % $\LIN$ from \cite{dagsamIII}, without countable choice in the base theory. 
We believe that $\LIN$ does not imply countable choice $\QFAC^{0,1}$. 
%In this way, our new notion of cover does not really change the Lindel\"of lemma, although we do not know how to prove the equivalence without the axiom of countable choice. 
\begin{thm}\label{brokken}
The system $\RCA_{0}^{\omega}+\QFAC^{1,1}$ proves $\LIN\asa \LIL$.
\end{thm}
\begin{proof}
Similar to the proof of Theorem \ref{ziedenauw}: the reverse direction is immediate, while in case of $\neg(\exists^{2})$ each principle is provable in $\RCAo$ using the sub-cover consisting of all rationals.
In case of $(\exists^{2})$, let $\psi$ be as in $\LIL$ and consider $(\forall x\in \R)(\exists y\in \R)(x\in J_{y}^{\psi})$.  Since the innermost formula is $\Sigma_{1}^{0}$, we may 
apply $\QFAC^{1,1}$ to obtain $\Xi$ such that $(\forall x\in \R)(x\in J_{\Xi(x)}^{\psi})$.  Since $\exists^{2}$ provides a functional that converts real numbers to a binary representation, we may assume that $\Xi$ is extensional on the reals.   
Now define $\Psi:I\di \R^{+}$ by $\Psi(x):=\min\big(|x-\psi(\Xi(x))(1)|, |x-\psi(\Xi(x))(2)|\big)$, and note that $I^{\Psi}_{x}\subseteq J_{\Xi(x)}^{\psi}$.  
Applying $\LIN$, we obtain a countable sub-cover of $\cup_{x\in I}I_{x}^{\Psi}$, say generated by $\Phi^{0\di1} $, and $\cup_{i\in \N}I_{\Xi(\Phi(i))}^{\psi}$ is a countable sub-cover of $\cup_{x\in I}I_{x}^{\psi}$.
\end{proof}
For completeness, we also mention the following corollary.  % similar to Theorem \ref{rathergen}.
\begin{cor}
$\RCAo+\QFAC^{1,1}$ proves $ \LIL\asa \dim(\R)\leq 1\asa \textup{$\R$ is paracompact}$.  
\end{cor}
\begin{proof}
We only prove the equivalence between $\LIL$ and the paracompactness of $\R$. 
By Theorem \ref{brokken}, it suffices to prove $\LIN$.  
In case $\neg(\exists^{2})$, the latter is provable outright, as all $\R\di \R$-functions are continuous, and then the rationals provide a countable sub-cover for any open cover as in $\LIN$.
Similarly, paracompactness reduces to countable paracompactness, and the latter is provable in $\RCAo$ by \cite{simpson2}*{II.7.2}.
In case of $(\exists^{2})$, the paracompactness of $\R$ (and hence $I$ with minor modification) implies $\HBT$ by Theorem \ref{paramaeremki}, and the aforementioned result $\HBT\asa [\WKL+\LIL]$ over $\RCAo+\QFAC^{1,1}$ finishes the forward direction.  The reverse direction is straightforward as $\exists^{2}$ decides inequalities between reals, and hence can easily refine the countable sub-cover provided by $\LIL$.  
%We follow the proof of Theorem \ref{rathergen}.  Note that $\Xi$ as in the proof of the latter is given as  
\end{proof}
%The following corollary follows in the same way. 
%%By Corollary \ref{brokken}, $\LIL$ is equivalent to the original Lindel\"of lemma from \cite{dagsamIII}, over $\RCAo+\QFAC^{1,1}$.  
%\begin{cor}
%The system $\RCAo+\QFAC^{1,1}$ proves $[\dim(\R)\leq 1]\asa \LIL$. 
%%The system $\RCAo$ proves $\big[\QFAC^{0,1}+[\dim(\R)\leq 1]\big]\asa \LIL$. 
%\end{cor}
%\begin{proof}
%The forward direction is trivial if $\neg(\exists^{2})$ as the latter implies that all $\psi:\R\di \R$ are continuous, and hence $\cup_{q\in \Q}I_{q}^{\psi}$ is a countable sub-cover, while $\QFAC^{0,1}$ follows from $\QFAC^{0,0}$.
%In case of $(\exists^{2})$, the corollary follows from the proof of Theorem \ref{rathergen} with minor/obvious modification to accommodate $\QFAC^{0,1}$.  The law of excluded middle finishes this proof.  
%
%\smallskip
%
%For the reverse implication, $\LIL\di \QFAC^{0,1}$ follows from \cite{dagsamV}*{Theorem 5.3}.  
%To prove $\dim(I)\leq 1$, the countable sub-cover provided by $\LIL$ is readily converted to a refinement of order $1$ using $\exists^{2}$, as the latter functional can decide equality between real numbers.  
%Now, in case $\neg(\exists^{2})$, obtain \eqref{pieq2} in the same way as above, and let $\Xi$ be a choice function that provides $\Xi(f)=q$.  
%Define $\zeta$ as follows: $\zeta(0):=\Xi(00\dots)+\frac{1}{2^{N_{0}}} $ and $\zeta(n+1):= \Xi(\zeta(n))+\frac{1}{2^{N_{0}}}$.  
%For $n> 2^{N_{0}+1}$, this function readily yields a finite open cover of $I$ that is also a refinement of the cover generated by $\psi$. 
%Since all points are rationals, we can refine this cover to have order $1$, and $(\exists^{2})\vee \neg(\exists^{2})$ finishes the proof. 
% 
%\end{proof}
As it turns out, we can avoid the use of $\QFAC^{1,1}$ as follows
\begin{thm}\label{kokken}
The system $\RCA_{0}^{\omega}+(\kappa_{0}^{3})+\textsf{\textup{IND}}$ proves $[\LIN+\QFAC^{0,1}]\asa \LIL$.
\end{thm}
\begin{proof}
For the forward implication, in case $\neg(\exists^{2})$, the rationals provides a countable sub-cover, as all functions on the reals are continuous by \cite{kohlenbach2}*{Prop.\ 3.7}.  In case of $(\exists^{2})$, fix $\psi:\R\di \R^{2}$ as in $\LIL$ and formulate a version of \eqref{centrifuge} as follows:
\be\label{centrifuge22}\textstyle
(\forall x\in \R)(\exists n\in \N)\underline{(\exists y\in \R)\big((x-\frac{1}{2^{n}}, x+\frac{1}{2^{n}})\subseteq J_{y}^{\psi}\big)}, 
\ee
The underlined formula is again decidable thanks to $\exists^{3}$, and $\QFAC^{1,0}$ yields a functional $\Psi:\R\di \R^{+}$ such that the canonical cover also $\cup_{x\in \R}I_{x}^{\Psi}$ covers $\R$. 
Applying $\LIN$ to \eqref{centrifuge22}, we obtain a functional $\Phi^{0\di 1}$ and the following version of \eqref{conplete}:
\be\label{conplete2}
(\forall n\in \N)(\exists v^{1^{*}})(\forall i\leq n)(I^{\Psi}_{\Phi(i)}\subseteq I_{v(i)}^{\psi}).
\ee
Applying $\QFAC^{0,1}$ to \eqref{conplete2}, we obtain $\LIL$, and this direction is done.  

\smallskip

For the reverse implication, note that $\LIL\di \QFAC^{0,1}$ follows from \cite{dagsamV}*{Theorem~5.3}, \emph{because} the base theory $\RCAo+(\kappa_{0}^{3})$ allows us to generalise the class of covers, as discussed in \cite{dagsamV}*{Remark 5.9}.  With that, we are done. 
\end{proof}
We believe that the previous 
splitting\footnote{In $\LIN$, any $x\in \R$ is covered by $I_{x}^{\Psi}$, while in $\LIL$ any $x\in \R$ is covered by $J_{y}^{\psi}$ \emph{for some $y\in \R$}.  
In the former case, we `know' which interval covers the point, while in the latter case, we only know \emph{that it exists}.    
We believe this (seemingly minor) difference determines whether one can obtain $\QFAC^{0,1}$ (like in the case of $\LIL$) or not (in the case of $\LIN$, we conjecture).  
Indeed, applying $\QFAC^{1,0}$ to the conclusion of $\LIL$, we obtain a functional that provides for any $x\in \R$, an interval $J_{y}^{\psi}$ covering $x$, i.e.\ $\LIL$ clearly exhibits `axiom of choice' behaviour, while $\LIN$ does not.} 
is proper.  The following corollary to the theorem is proved in the same way.  
\begin{cor}
The system $\RCAo+(\kappa_{0}^{3})+\textsf{\textup{IND}}$ proves 
\be
\big[[\dim(\R)\leq 1]+\QFAC^{0,1}\big]\asa \big[[\textup{$\R$ is paracompact}]+\QFAC^{0,1}\big] \asa \LIL. 
\ee
\end{cor}
In conclusion, it is easy to see that the proofs of this section go through for the Lindel\"of lemma for $\R$ based on {any `reasonable' notion} of cover.
Indeed, as long as the formulas `$x\in U_{y}$' and `$[a,b]\subseteq U_{x}$' for the new notion of cover $\cup_{x\in \R}U_{x}$ of $\R$ are decidable in $\Z_{2}^{\omega}$, the above proofs go through (assuming $(\kappa_{0}^{3})$).  
Since $\Z_{2}^{\omega}$ can decide if $Y:\R\di \{0,1\}$ represents an open subset of $\R$ (using the textbook definition of open set), this notion of `reasonable' again seems quite reasonable. 

\subsubsection{Partitions of unity}\label{unite}
The notion of \emph{partition of unity} was introduced in 1937 by Dieudonn\'e in \cite{nogeengodsgeschenkje} and this notion is equivalent to paracompactness in a rather general setting by \cite{bengelkoning}*{Theorem 5.1.9}.
We study partitions of unity in this section motivated as follows: on one hand, Simpson proves the existence of {partitions of unity} for complete separable spaces in the proof of \cite{simpson2}*{II.7.2}, i.e.\ this notion has been studied in RM.  
On the other hand, despite the equivalence, partitions of unity have nicer RM properties than paracompactness: we obtain versions of Theorems~\ref{fugu} and \ref{kokken} where the base theory is conservative over $\RCAo$, while there is also a nice connection to the Lindel\"of lemma.   

\smallskip

The definition of partition of unity is as follows in Munkres \cite{munkies}*{p.\ 258}
\bdefi
Let $\{U_{\alpha}\}_{\alpha\in J}$ be an indexed open covering of $X$. An indexed family of
continuous functions $\phi_{\alpha}: X \di [0, 1]$ is said to be a partition of unity on $X$, dominated\footnote{Munkres uses `dominated by' in \cite{munkies} instead of Engelking's `subordinate to' in \cite{bengelkoning}.} by $\{U_{\alpha}\}$, if:
\begin{enumerate}
\item $\textsf{support}(\phi_{\alpha})\subset U_{\alpha}$ for each $\alpha\in J$.
\item The indexed family $\{\textsf{support}(\phi_{\alpha})\}_{\alpha\in J}$ is locally finite.
\item $\sum_{\alpha\in J}\phi_{\alpha}(x)=1$ for each $x\in X$.
\end{enumerate}
where $\textsf{support}(f)$ is the closure of the open set $\{x\in X:f(x)\ne 0\}$.
\edefi\noindent
Note that the second item implies that the sum in the third one makes sense.  

\smallskip

With these definitions in place, $\PUNI(I)$ is the statement that for any cover generated by $\psi:I\di \R$,  there is a partition of unity of $I$ dominated by $\cup_{x\in I}I_{x}^{\psi}$.
%Let $\PUNI^{-}(I)$ be the same for $\Psi:I\di\R^{+}$.  
\begin{thm}\label{kokken3}
The system $\RCA_{0}^{\omega}+(\kappa_{0}^{3})$ proves $[\WKL+\PUNI(I)]\asa \HBT$.
\end{thm}
\begin{proof}
%By Theorem \ref{fugu}, we only need to prove $\HBU$ in the forward direction.  
In case of $\neg(\exists^{2})$, the equivalence is easy: all $\R\di \R$-functions are continuous and $\PUNI(I)$ is provable as in the proof of \cite{simpson2}*{II.7.2}, while $\HBT$ follows from $\WKL$ as in the proof of Corollary \ref{dorkeeee}. 
In case of $(\exists^{2})$, the reverse implication is also straightforward: the finite sub-cover provided by $\HBT$ is readily refined, and the existence of a partition 
of unity for a finite cover follows from \cite{simpson2}*{II.7.1}. 

\smallskip

Finally, for the forward direction assuming $(\exists^{2})$, let $\psi:I\di \R$ be as in $\HBT$ and obtain $\phi:I^{2}\di \R$ as in $\PUNI(I)$, i.e.\ for $U_{x}:=\textsf{support}(\phi(x, \cdot))$, 
the open cover $\cup_{x\in I }U_{x}$ of $I$ is locally finite and satisfies $U_{x}\subset I_{x}^{\psi}$.  Now consider:
\be\label{centrifuge2555}\textstyle
(\forall x\in I)(\exists n\in \N)\underline{(\exists y\in I)\big((x-\frac{1}{2^{n}}, x+\frac{1}{2^{n}})\subseteq U_{y}\big)}, 
\ee
Applying $\QFAC^{1,0}$ to \eqref{centrifuge2555}, since $\exists^{3}$ is given, we obtain $\Psi:I\di \R^{+}$ such that $I_{x}^{\Psi}\subset U_{x}$ for all $x\in I$. 
Now repeat the proof of Theorem \ref{paramaeremki} for $\Psi$ in place of $\phi$, which  yields $y_{1},\dots y_{k}\in I$ such  $\cup_{i\leq k }I_{x}^{\Psi}$ is a finite sub-cover of $I$.  
Note that in the previous `repeated proof', we do not need the choice function $\Xi$ (from the proof of Theorem \ref{rathergen}), as $I_{x}^{\Psi}$ covers $x$ for any $x\in I$.  
Since $I_{x}^{\Psi}\subset U_{x}$, $\cup_{i\leq k }U_{y_{i}}$ is a finite sub-cover of $\cup_{x\in I}U_{x}$, and since $U_{x}\subset I_{x}^{\psi}$, $\cup_{i\leq k }I_{y_{i}}^{\psi}$ is a finite sub-cover as required by $\HBT$, and we are done. 
\end{proof}
\begin{cor}\label{kokken322}
The system $\RCA_{0}^{\omega}+(\kappa_{0}^{3})+\PUNI(I)$ proves $\HBU\asa \HBT$.
\end{cor}
Note that previous base theory in the corollary (and hence the theorem) is conservative over $\RCAo$ by \cite{kohlenbach2}*{Prop.\ 3.12} and the proof of \cite{simpson2}*{II.7.2}.
%
%
%\begin{thm}\label{kokken34}
%The system $\RCA_{0}^{\omega}$ proves $[\WKL+\PUNI^{-}(I)]\asa \HBU$.
%\end{thm}
%\begin{proof}
%X
%\end{proof}
%We also have
%\begin{thm}\label{kokken32}
%The system $\RCA_{0}^{\omega}+\QFAC^{1,1}$ proves $[\WKL+\PUNI(I)]\asa \HBT$.
%\end{thm}
%Finally, 

\smallskip

Finally, we obtain a theorem that brings together a number of different strands from this paper, including \emph{effective paracompactness}, first discussed at the end of Section \ref{diemensie2}.  
In the context of $\RCAo$, we say that `$\phi:\R\di \R$ is an \emph{effective} refinement of $\psi:\R\di \R$' if $(\exists \xi:\R\di\R)(\forall x\in \R)(I_{x}^{\phi}\subseteq I_{\xi(x)}^{\psi})$.  % for some $\xi:\R\di \R$.  
Effective paracompactness expresses the existence of an effective refinement for any open cover.  Moreover, $\PUNI(\R)$ is the statement that for any cover generated by $\psi:\R\di \R^{2}$,  there is a partition of unity $\phi:\R^{2}\di \R^{2}$ dominated by $\cup_{x\in I}J_{x}^{\psi}$.
 
\begin{thm}\label{kokken327}
The system $\RCA_{0}^{\omega}+(\kappa_{0}^{3})$ proves the following 
\[
\PUNI(\R)\asa \LIL\asa \textup{$\R$ is effectively paracompact}.
\]
\end{thm}
\begin{proof}
We first prove the first equivalence.
In case of $\neg(\exists^{2})$, the equivalence is easy: all $\R\di \R$-functions are continuous and $\PUNI(\R)$ is provable as in the proof of \cite{simpson2}*{II.7.2}, while $\LIL$ follows by taking the countable sub-cover given by the rationals. 
In case of $(\exists^{2})$, the reverse implication is also straightforward: the countable sub-cover provided by $\LIL$ is readily refined, and the existence of a partition 
of unity for a countable cover follows from \cite{simpson2}*{II.7.1}. 

\smallskip

Finally, for the forward direction assuming $(\exists^{2})$, let $\psi:\R\di \R^{2}$ be as in $\LIL$ and obtain $\phi:\R^{2}\di \R^{2}$ as in $\PUNI(\R)$, i.e.\ for $U_{x}:=\textsf{support}(\phi(x, \cdot))$, 
the open cover $\cup_{x\in \R }U_{x}$ of $\R$ is locally finite and satisfies $U_{x}\subset I_{x}^{\psi}$.  Now consider:
\be\label{centrifuge25}\textstyle
(\forall x\in \R)(\exists n\in \N)\underline{(\exists y\in \R)\big((x-\frac{1}{2^{n}}, x+\frac{1}{2^{n}})\subseteq U_{y}\big)}, 
\ee
Applying $\QFAC^{1,0}$ to \eqref{centrifuge25}, since $\exists^{3}$ is given, we obtain $\Psi:\R\di \R^{+}$ such that $I_{x}^{\Psi}\subset U_{x}$ for all $x\in \R$. 
Now repeat the proof of Theorem \ref{paramaeremki} for $\Psi$ in place of $\phi$ and $\R$ instead of $I$.
Then instead of \eqref{tuigs}, we obtain
\be\label{tuigs1337}
(\forall x\in \R )(\exists n\in \N)(\zeta(n)\geq _{\R}|x|).
\ee
% which yields $y_{1},\dots y_{k}\in I$ such  $\cup_{i\leq k }I_{x}^{\Psi}$ is a finite sub-cover of $I$.  
Note that in the aforementioned `repeated proof', we do not need the choice function $\Xi$ (from the proof of Theorem \ref{rathergen}), as $I_{x}^{\Psi}$ covers $x$ for any $x\in I$.  
Applying $\QFAC^{1,0}$ to \eqref{tuigs1337}, we obtain $\Phi^{0\di 1}$ such that $\cup_{n\in \N}I_{\Phi(n)}^{\Psi}$ is a countable sub-cover of the canonical cover generated by $\Psi$.
Since $I_{x}^{\Psi}\subset U_{x}$, $\cup_{n\in \N}I_{\Phi(n)}^{\Psi}$ is a countable sub-cover of $\cup_{x\in \R}U_{x}$, and since $U_{x}\subset I_{x}^{\psi}$, $\cup_{n\in \N}I_{\Phi(n)}^{\psi}$ is a countable sub-cover as required by $\LIL$.  For the second equivalence, we note that the previous proof only makes use of the fact that $\phi:\R^{2}\di \R^{2}$ is effectively paracompact (via the identity function).  
Thus, the second equivalence follows in the same way, and we are done. 
\end{proof}
%How to prove


%How to prove $\LIL\di \dim(\R)\leq 1$?
%
%\subsection{Other notions of compactness}
%\subsubsection{Metacompactness}
%A space is compact if and only if it is \emph{pseudocompact} and \emph{metacompact} (\cite{wienszoon}).
%The pseudo-version for $[0,1]$ states that continuous functions are bounded on $[0,1]$, a weak property from the point of view of RM.  
%Hence, the meta-version for $[0,1]$ must `harbour all the logical strength'.   
%\bdefi
%A space $X$ is \emph{metacompact} if for any open cover $\mathcal{A}$ of $X$, there is an open refinement $\mathcal{B}$ with any $x\in X$ contained in only finitely many $B\in \mathcal{B}$.
%\edefi
%There are spaces that are paracompact, but not metacompact (See \cite{nogeengodsgeschenk}).  The notion of \emph{effectively metacompact} is introduced in \cite{goodgoing}.
%
%\subsubsection{Subparacompactness}
%See \url{https://math.stackexchange.com/questions/135618/subparacompact-spaces}
%
%\subsection{Choquet spaces}
%
%Being a Choquet space seems connected to the Cantor intersection theorem, which in turn is connected to $\HBU$.  Mummert uses Choquet spaces in \cite{mummymf}. 


\section{Conclusion}\label{konkelfoes}
We have studied the higher-order RM of topology, the notions of \emph{dimension} and \emph{paracompactness} in particular. 
Basic theorems regarding the latter turn out to be equivalent to the \emph{Heine-Borel theorem} for uncountable covers, i.e.\ the former are extremely hard to prove (in terms of comprehension axioms).    
A number of nice splittings was obtained, and we have shown that these results do not depend on the exact definition of cover, even in the absence of the axiom of choice.  
Finally, we obtained similar results for the \emph{Lindel\"of lemma}.  We refer to Figure \ref{xxy} for a visual depiction of our results vis-\`a-vis the \emph{G\"odel hierarchy}.

\smallskip

Regarding future work, the following two topics come to mind.  Firstly, there are a number of notions weaker than paracompactness, and it is an interesting question if there are \emph{natural} such notions that yield equivalences with $\HBT$ or weaker theorems.  
Secondly, in light of Remark \ref{diemummy}, it seems interesting to study metrisation theorems in higher-order RM.  We expect that such theorems go far beyond $\SIX$, which features in the second-order RM of topology.  
\begin{ack}\rm
Our research was supported by the John Templeton Foundation, the Alexander von Humboldt Foundation, and LMU Munich (via the Center for Advanced Studies of LMU).  % and the University of Oslo.
We express our gratitude towards these institutions. 
We thank Dag Normann for his valuable advice.
\end{ack}


%\begin{ack}\rm
%Our research was supported by the John Templeton Foundation, the Alexander von Humboldt Foundation, LMU Munich (via the Excellence Initiative and the Center for Advanced Studies of LMU), and the University of Oslo.
%We express our gratitude towards these institutions. 
%We thank Dag Normann for his valuable advice, especially regarding the properties of $(Z^{3})$.  
%We thank Denis Hirschfeldt for his valuable suggestions regarding $\T_{0}$.  
%\end{ack}

\appendix
\section{The G\"odel Hierarchy}\label{kurtzenhier}
The \emph{G\"odel hierarchy} is a collection of logical systems ordered via consistency strength, or essentially equivalent: ordered via inclusion\footnote{Simpson states in \cite{sigohi}*{p.\ 112} that inclusion and consistency strength yield the same hierarchy as depicted in \cite{sigohi}*{Table 1}, i.e.\ one gets the `same' G\"odel hierarchy. \label{fooker}}.  This hierarchy is claimed to capture most systems that are `natural' or have `foundational import', as follows. 
\begin{quote}
\emph{It is striking that a great many foundational theories are linearly ordered by $<$. Of course it is possible to construct pairs of artificial theories which are incomparable under $<$. However, this is not the case for the ``natural'' or non-artificial theories which are usually regarded as significant in the foundations of mathematics.} (\cite{sigohi})
\end{quote}
Arguably, the G\"odel hierarchy is a central object of study in mathematical logic, as also stated by Simpson in \cite{sigohi}*{p.\ 112}.  
However, the above results imply that e.g.\ $\HBT$ and basic topological theorems dealing with dimension and paracompactness, do not fit the G\"odel hierarchy.  
The same holds for basic properties of the gauge integral, including many covering lemmas (See \cite{dagsamIII}), as well as for so-called uniform theorems (See \cite{dagsamV}) in which the objects claimed to exist depend on few of the parameters of the theorem.
In particular, the aforementioned theorems yield a branch that is \emph{completely} independent of the medium range of the G\"odel hierarchy (with the latter based on inclusion$^{\ref{fooker}}$), as depicted in the following figure (where we assume the ordering based on inclusion): 
\begin{figure}[h]
\[
\begin{array}{lll}
&\textup{\textbf{strong}} \hspace{1.5cm}& 
\left\{\begin{array}{l}
\vdots\\
\textup{supercompact cardinal}\\
\vdots\\
\textup{measurable cardinal}\\
\vdots\\
\ZFC \\
\textsf{\textup{ZC}} \\
\textup{simple type theory}
\end{array}\right.
\\
&& \\
  &&~\quad{ {\Z_{2}^{\omega}}} \\%(\equiv \RCA_{0}^{\omega}+(\exists^{3})}})\\ %+\QFAC^{1,1}\\
&&\\
&\textup{\textbf{medium}} & 
\left\{\begin{array}{l}
 \mathcal{Z}_{2}^{\omega}( \equiv \cup_{k}\SIXK)+ \QFAC^{0,1}\\
\vdots\\
\textup{$\Pi_{2}^{1}\textsf{-CA}_{0}^{ {\omega}}$}\\
\textup{$\FIVE^{ {\omega}}$ }\\
\textup{$\ATR_{0}^{ {\omega}}$}  \\
\textup{$\ACA_{0}^{ {\omega}}$} \\
\end{array}\right.
%\begin{array}{c}
%\textup{Kohlenbach's}\\
%\textup{ {higher-order RM}}\\
%\end{array}
\\
&
\\
{ {\left\{\begin{array}{l}
\textup{covering lemmas like $\HBT$}\\
\textup{basic theorems about para-}\\
\textup{compactness and dimension}
\end{array}\right\}}}
&\begin{array}{c}\\\textup{\textbf{weak}}\\ \end{array}& 
\left\{\begin{array}{l}
\WKL_{0}^{ {\omega}} \\
\textup{$\RCA_{0}^{ {\omega}}$} \\
\textup{$\textsf{PRA}$} \\
\textup{$\textsf{EFA}$ } \\
\textup{bounded arithmetic} \\
\end{array}\right.
\\
\end{array}
\]
\caption{The G\"odel hierarchy with a side-branch for the medium range}\label{xxy}
\begin{picture}(250,0)
\put(155,200){ {\vector(-3,-2){125}}}
\put(160,117){ {\vector(-3,-2){53}}}
\multiput(100,70)(5,0){14}{\line(1,0){3}}
\put(167,70){ {\vector(1,0){8}}}
\put(125,100){ {\vector(3,2){50}}}
\put(150,100){{\line(-5,3){20}}}
\end{picture}
\end{figure}\\
Some remarks on the technical details concerning Figure \ref{xxy} are as follows. 
\begin{enumerate}
\item Note that we use a \emph{non-essential} modification of the G\"odel hierarchy, namely involving systems of higher-order arithmetic, like e.g.\ $\ACA_{0}^{\omega}$ instead of $\ACA_{0}$; these systems are (at least) $\Pi_{2}^{1}$-conservative over the associated second-order system (See e.g.\ \cite{yamayamaharehare}*{Theorem 2.2}).  
%\item In the spirit of RM, we show in \cite{dagsamV} that the Cousin lemma and (a version of) the Lindel\"of lemma are provable \emph{without} the use of $\QFAC^{0,1}$, as also discussed in Remark \ref{linpinpon}.     
\item The system $\Z_{2}^{\omega}$ is placed \emph{between} the medium and strong range, as the combination of the recursor $\textsf{R}_{2}$ from G\"odel's $T$ and $\exists^{3}$ yields a system stronger than $\Z_{2}^{\omega}$.  The system $\SIXK$ does not change in the same way.     
\item While $\HBT$ clearly implies $\WKL$, the paracompactness of the unit interval does not (by the $\ECF$-translation); this is symbolised by the dashed line.  % in Figure \ref{xxy}.  
\item While $\HBT$ and similar statements are \emph{hard} to prove (in terms of comprehension axioms), these theorems (must) have weak first-order strength in light of their derivability in intuitionistic topology (See e.g.\ \cite{waaldijkphd, troelstraphd}). 
\end{enumerate}
Finally, in light of the equivalences involving the gauge integral, uniform theorems, and the Cousin lemma (and hence $\HBT$) from \cite{dagsamIII, dagsamV}, we observe a serious challenge to the `Big Five' classification from RM,  the linear nature of the G\"odel hierarchy,
%NEW&&
 as well as Feferman's claim that the mathematics necessary for the development of physics can be formalised in relatively weak logical systems (See \cite{dagsamIII}*{p.\ 24}).
%Note that in Figure \ref{xxy}, we used a non-essential modification of the G\"odel hierarchy, namely involving systems of higher-order arithmetic, like e.g.\ $\ACA_{0}^{\omega}$ instead of $\ACA_{0}$; these systems are (at least) $\Pi_{2}^{1}$-conservative over the associated second-order system (See e.g.\ \cite{yamayamaharehare}*{Theorem 2.2}).\\
%





\begin{bibdiv}
\begin{biblist}
\bibselect{allkeida}
\end{biblist}
\end{bibdiv}

\bye
\appendix

\section{Other stuff}
\subsection{$\WWKL$ stuff}
Let $\textsf{WRIE}$ be the statement that 
\begin{thm}[$\textsf{WRIE}$]
For $g^{2}, N^{0},\eps>0$, there is $\delta>0$ such that for all $f:\R\di \R$
\[
[\MPC(g, f)\wedge (\forall x\in I)(|f(x)|\leq N)] \di \RIE(\eps, \delta, f)
\]

\end{thm}

\begin{thm}
The system $\RCAo$ proves that $\WWKL\asa (\exists^{2})\vee \textsf{WRIE}$
\end{thm}
\begin{proof}
Coding is a bitch, so it may not be true (or hard to prove).
\end{proof}

\subsection{Polya's theorem}
See Bartle's book ERA (elementary real analysis), 1964, p.\ 194.   This theorems generalises Dini's theorem and should easily yield $\HBU$ or so.  

\subsection{A theorem on searchable sets}
Escardo has a paper on searchable sets.  Here, `searchable' means that one can algorithmically find an element $x$ such that $P(x)$, where $P(x)$ is continuous.  

\smallskip

\emph{Searchable sets are uniformly searchable} should imply $\HBU$.  Here, `uniformly searchable' means that we can find an element (or upper bound or so) from the modulus of continuity of $P$ alone.  




\subsection{Urysohn lemma}
We study the \emph{Urysohn lemma} in higher-order RM.   Besides the mere classification of this lemma, 
our aim is to show that the exact representation of open sets (both in RM and set theory) has a huge influence on its strength.    

\medskip

First of all, the {Urysohn lemma} states that \emph{any two disjoint closed sets in a normal space can be separated by a continuous function} (\cite{uryne}*{p.\ 290}).  
On one hand, using codes for open sets as in RM, the Urysohn lemma becomes effective and is provable in $\RCA_{0}$ by \cite{simpson2}*{II.7.3}.  On the other hand, the Urysohn lemma is not provable in $\ZF$ by \cite{good}*{Cor.\ 2.2}.  
We will study two versions of the Urysohn lemma, based on two \emph{slightly} different definitions of open sets.  Our definitions are analogous to the pre-1900 notion of open cover respectively due to Cousin-Lindel\"of (\cite{cousin1, blindeloef}) and Borel-Schoenflies (\cite{schoen2,opborrelen}).   

\medskip

Thus, let us review some definitions of open set. 
The first item in Definition \ref{coiffu} is the usual RM-definition involving countably many open balls.  
The third item in Definition \ref{coiffu} is our most general definition where an open set is the uncountable collection $\cup_{x\in \R}J_{x}^{\Psi}$ for $\Psi:\R\di \R^{2}$, where $J_{x}^{\Psi}$ is the (possibly empty) open interval $ (\Psi(x)(1), \Psi(x)(2))$.  
%Our general definition is item $\bs$ below, where an open set is the collection $\cup_{x\in \R}J_{x}^{\Psi}$.   
The second item is an intermediate definition involving an uncountable collection as in the third item, but with a `realiser for membership' $\Phi$.  The latter also brings (symbolic) set membership for $V$ down to $\Sigma_{1}^{0}$.  
%Similarly, $I_{x}^{\Psi}$ is , and empty otherwise.  
%The latter definition makes sense in $\ACAo$, but the formula $I_{x}^{\Psi}\ne \emptyset$ is also meaningful in $\RCAo$.  
\bdefi[Representations of open sets]\label{coiffu}~
\begin{enumerate}
\item[(\textsf{rm})] An open set $U\subset \R$ is represented (aka coded) in RM (\cite{simpson2}*{II.5.6}) by sequences $a_{n}, r_{n}:\N\di \Q$, with the intuitive understanding that $U$ is $\cup_{n\in \N}B(a_{n}, |r_{n}|)$, and `$x\in U$' is short for $(\exists n\in \N)(x\in B(a_{n},| r_{n}|))$.
\item[(\textsf{cl})] An open set $V\subset \R$ is represented by $\Psi:\R\di \Q^{2}, \Phi:\R\di \R$ where $(\forall x, y\in \R)(x\in J_{y}^{\Psi}\di x\in J_{\Phi(x)}^{\Psi} )$, i.e.\ $V$ is $\cup_{x\in \R}J_{x}^{\Psi}$ and $\Phi$ tells us which interval points in $V$ belong to; we write `$x\in V$' for $ x\in J_{\Phi(x)}^{\Psi}$
%$a_{n}(x), r_{n}(x):(\N\times \R)\di \Q$ such that 
%\[
%(\forall x, y\in \R, n\in \N)\big[x\in B(a_{n}(y),| r_{n}(y)|)\di (\exists m\in \N)(x\in B(a_{m}(x),| r_{m}(x)|) )\big].
%\]
%%with the intuitive understanding that $V$ is $\cup_{x\in \R}\cup_{n\in \N}B(a_{n}(x),| r_{n}(x)|)$. 
%We can then write `$x\in V$' if $(\exists n\in \N)(x\in B(a_{n}(x), |r_{n}(x)|))$. 
%$\cup_{x\in \R}\cup_{n}$
% if $(\forall x, y\in \R)(x\in I_{y}^{\Psi}\di I_{x}^{\Psi}\ne \emptyset$; we write `$x\in V$' for $ I_{x}^{\Psi}\ne\emptyset$ and `$V=\cup_{x\in \R}I_{x}^{\Psi}$'.  
%\item[(\textsf{cs})] The functional $\Psi:\R\di \R^{2}$ represents an open set $V\subset\R$ 
% if $(\forall x, y\in \R)(x\in I_{y}^{\Psi}\di I_{x}^{\Psi}\ne \emptyset$; we write `$x\in V$' for $ I_{x}^{\Psi}\ne\emptyset$ and `$V=\cup_{x\in \R}I_{x}^{\Psi}$'.  
\item[($\bs$)] The functional $\Psi:\R\di \Q^{2}$ represents the open set $W\subset\R$ 
given by $\cup_{x\in \R}J_{x}^{\Psi}$, i.e.\ we write `$x\in W$' for $(\exists y\in \R)(x\in J_{y}^{\Psi})$.
\end{enumerate}
\edefi
For the first and second item, $\ACAo$ provides a functional which outputs an interval $(a,b)$ such that $x\in (a,b)\subset U, V$, whenever possible.
Moreover, elementhood for those sets is c.e., i.e.\ $x\in U$ and $x\in V$ are $\Sigma_{1}^{0}$-formulas, while it is $\Sigma_{1}^{1}$ with type two parameters for $W$.  % in the $\bs$-case.  
For brevity, a set is \emph{open}$_{\textsf{xy}}$ if it satisfies the associated item in the previous definition.  
Sets are called \emph{closed}$_{\textsf{xy}}$, if the complement is open$_{\textsf{xy}}$.  

\medskip

As noted above, the second item corresponds to the special case of open covers used by Cousin and Lindel\"of (\cite{cousin1, blindeloef}), while the third item corresponds to the more general case of open covers by Borel and Schoenflies (\cite{opborrelen, blindeloef})
\begin{thm}[$\URY^{\cl}$]
For closed$_{\cl}$ and disjoint sets $A, B\subset \R$, there exists continuous $h:\R\di [0,1]$ with $x\in A\di h(x)=0$ and $x\in B\di h(x)=1$ for $x\in \R$.
\end{thm}
Define $\URY^{\bs}$ as the previous theorem with $\cl$ replaced by $\bs$. 
\begin{thm}[$\LIN^{\cl}$]
For every open$_{\cl}$ set $V\subset \R$, % given by $\Psi:\R\di (\N\di \Q)^{2}$, 
there is $\phi^{0\di 1}$ such that $V$ is $\cup_{n\in \N}J_{\Phi(\phi(n))}^{\Psi}$
%$\cup_{m\in \N}\cup_{n\in \N}B(a_{n}(\Phi(m)),| r_{n}(\Phi(m))|)$ %$V=\cup_{n\in \N}I_{\Phi(n)}^{\Psi}$.  % $(\forall x\in \R)(x\in V \asa (\exists n\in \N)( x\in I_{\Phi(n)}))$
\end{thm}
\begin{thm}
The system $\RCAo(+\WKL?)$ proves $\URY^{\cl}\asa \LIN^{\cl}$, while $\Z_{2}^{\omega}$ proves $\LIN^{\cl}$.
\end{thm}
\begin{proof}
X
\end{proof}

\medskip

Use ``full'' definition of open set (as in LINDbs), and apply LIND to get countable sub-cover, and then the RM-Urysohn applies.  

\medskip

Perhaps Urysohnbs implies LINDbs or so?  It does! In fact, one needs\dots countable choice to prove it \cite{good}. 

\medskip

Porbably: LIND $\asa$ URY $\asa$ TIET and the same for bs versions!

\subsection{Osgood-Arzela theorem}
One of Bressoud's radical books.  Uniform version should be HBU or so.  


%\subsection{Open induction}

\subsection{Lebesgue criterion}
Lebesgue's criterion for Riemann integrability: uniform version should be WHBU.  


\subsection{Schoenflies proof of HBU}
See Thm VI, p.\ 52 in his 1899 book: this thm is application of HBU; uniform????



\subsection{Weaker principles like $\WWKL$}

Fourth, we now list equivalences similar to \eqref{corkukkk}; to this end, consider \cite{simpson2}*{X.1.9.3} which states that a cover $\cup_{n\in \N}(a_{n}, b_{n})$ of the unit interval is such that $\sum_{n=0}^{\infty}|a_{n}-b_{n}|\geq 1$.  Now let $\WHBU$ be the statement that for any $\Psi:\R\di \R^{+}$ the canonical cover $\cup_{x\in I}I_{x}^{\Psi}$ has a countable sub-cover $\cup_{n}(c_{n}, d_{n})$ such that $\sum_{n=0}^{\infty}|c_{n}-d_{n}|\geq 1$.  OR:
\[\textstyle
(\forall \Psi:\R\di \R^{+}, k\in \N)(\exists \langle y_{1}, \dots, y_{n}\rangle)\big( \sum_{i=1}^{n}  |I_{y_{i}}^{\Psi}|\geq 1-\frac{1}{2^{k}}   \big).
\]
We have the following equivalence in which neither disjunct implies the other:
\be\label{ohjee}
\WWKL\asa [(\exists^{2})\vee \WHBU]\asa [\WKL\vee \WHBU],  
\ee
which follows in the same was as in the theorem, as $\cup_{q\in [0,1]\cap \Q}I_{q}^{\Psi}$ is a countable sub-cover of the canonical cover if $\Psi$ is continuous.  One could also define a version of $\WKL_{\u}$ for any theorem from the RM zoo (\cite{damirzoo}); for instance the contraposition of $\ADS$ states that if 
a linear order $X$ has no ascending and no descending sequence, then $X$ is finite.  Then $\ADS_{\u}$ essentially states that the upper bound on $X$ only depends on a realiser $G$ for the fact that $X$ has no ascending and no descending sequence; 
such a realiser is as follows: $(\forall x: \N\di X)(\exists n\ \leq G(x)) (x(n)\leq x(n)\vee x(n+1)\geq x(n+1)) $.  The following statement is then provable in $\RCAo+\WKL$:
\be\label{ohjee2}
\ADS\asa \big[(\exists^{2}) \vee \ADS_{\u} \big]\asa \big[\CAC \vee \ADS_{\u} \big]\asa \big[\textsf{RT}_{2}^{2} \vee \ADS_{\u} \big],
\ee
where the principles satisfy $\RT_{2}^{2}\di \CAC\di \ADS$ (\cite{dsliceke}).  It is remarkable that the final disjunctions in \eqref{corkukkk}, \eqref{ohjee}, and \eqref{ohjee2} only involve one higher-order theorem. 
Finally, the proof of Corollary \ref{puilo} yields that $\RCAo+\WKL_{0}+\neg\ADS$ proves $\HBU_{\c}$.  



\subsection{Monotone and dominated Lebesgue convergence theorem}
It should totally work! Same as Dini, but weaker!



\subsection{$\WWKL$ version of Pincherle's theorem}
\be\tag{$\WHBU_{\c}$}\textstyle
(\forall G^{2}, k^{0})(\exists \langle f_{1}, \dots, f_{k} \rangle )\big[ \mu(\cup_{i\leq k} [f_{i}G(f_{i})])\geq 1-\frac{1}{2^{k}}  \big].
\ee
\bdefi[$\PIT_{\u}^{w}$]
\[
(\forall G:C\di \N, E \subset_{o} C)(\exists N\in \N )\big[ (\forall F:C\di \N)\big[  \LOC(F, G, E)\di (\forall g \in E)(F(g)\leq N)\big].
\]
\edefi
Here, `$E\subset_{o}C$' denotes that $E^{1}$ is an open subset op Cantor space (in the sense of RM; \cite{simpson2}*{II.5.6}) such that $\mu(E)=1$, i.e.\ with Lebesgue measure one. 
Also, $\LOC(F, G, E)$ means that $F$ is totally bounded on $E$, as witnessed by $G$.  
\begin{thm}\label{krooi2} 
The system $\ACA_{0}^{\omega}+\QFAC^{1,1}$ proves $\PIT_{\u}^{w}\di \WHBU_{\c}$. 
%\be\label{worski}
%\HBU_{\c}\asa \HBU\asa (\exists \Theta)\SCF(\Theta)\asa \PIT_{\u}\asa (\exists M)\PR(M).
%\ee
\end{thm}
\begin{proof}
The formula `$f\in E$' is $\Sigma_{1}^{0}$ by definition, but since $(\exists^{2})$ is available, we may treat it as quantifier-free.  
Fix $G^{2}, E_{0}\subset_{o}C$ and let $N_{0}\in \N$ be the bound provided by $\PIT_{\u}^{w}$.  
We claim:
\be\label{contrje337}
(\forall f\in E)(\exists g\in E)(G(g)\leq N_{0}\wedge f\in[ \overline{g}G(g)]).
\ee
Indeed, suppose $\neg\eqref{contrje337}$ and let $f_{0}\in E$ be such that $(\forall g\in E)( f_{0}\in [\overline{g}G(g)]\di G(g)> N_{0})$.
Now use $(\exists^{2})$ to define $F_{0}^{2}$ as follows: $F_{0}(h):=N_{0}+1$ if $h=_{1}f_{0}$, and zero otherwise.   
By assumption, we have $\LOC(F_{0}, G)$, but clearly $F(f_{0})>N_{0}$ and $\PIT_{\u}$ yields a contradiction.  
Hence, $\PIT_{\u}^{w}$ implies \eqref{contrje337}, and the latter provides a finite sub-cover for the canonical cover $\cup_{f\leq 1}[\overline{f}G(f)]$.  
Indeed, apply $\QFAC^{1,1}$ to \eqref{contrje337} to obtain a functional $\Phi^{1\di 1}$ providing $g$ in terms of $f$.  
The finite sub-cover (of length $2^{N_{0}}$) then consists of all $\Phi(\sigma*00\dots)$ for all binary $\sigma$ of length $N_{0}$.  
%Then there is a g ? C such that F(g) \UTF{2264} a and f ? Cg\CID{3}(F(g), since otherwise, the functional G that equals a + 1 on f and 0 elsewhere will satisfy the bounding condition induced by (F,F) without being bounded by a.
%Then, for each binary sequence s of length a, there will be a function gs such that Cs ? Cg\CID{3}s(F(gs)), and this gives us a finite sub-covering.
% 
%
\end{proof}

\be\tag{$\WHBU_{\c}$}\textstyle
(\forall G^{2}, k^{0})(\exists \langle f_{1}, \dots, f_{k} \rangle )\big[ \mu(\cup_{i\leq k} [f_{i}G(f_{i})])\geq 1-\frac{1}{2^{k}}  \big].
\ee
\bdefi[$\PIT_{\u}^{w}$]
\[
(\forall G:C\di \N, E \subset_{o} C)(\exists N\in \N )\big[ (\forall F:C\di \N)\big[  \LOC(F, G, E)\di (\forall g \in E)(F(g)\leq N)\big].
\]
\edefi
Here, `$E\subset_{o}C$' denotes that $E^{1}$ is an open subset op Cantor space (in the sense of RM; \cite{simpson2}*{II.5.6}) such that $\mu(E)=1$, i.e.\ with Lebesgue measure one. 
Also, $\LOC(F, G, E)$ means that $F$ is totally bounded on $E$, as witnessed by $G$.  
\begin{thm}\label{krooi2} 
The system $\ACA_{0}^{\omega}+\QFAC^{1,1}$ proves $\PIT_{\u}^{w}\di \WHBU_{\c}$. 
%\be\label{worski}
%\HBU_{\c}\asa \HBU\asa (\exists \Theta)\SCF(\Theta)\asa \PIT_{\u}\asa (\exists M)\PR(M).
%\ee
\end{thm}
\begin{proof}
The formula `$f\in E$' is $\Sigma_{1}^{0}$ by definition, but since $(\exists^{2})$ is available, we may treat it as quantifier-free.  
Fix $G^{2}, E_{0}\subset_{o}C$ and let $N_{0}\in \N$ be the bound provided by $\PIT_{\u}^{w}$.  
We claim:
\be\label{contrje337}
(\forall f\in E)(\exists g\in E)(G(g)\leq N_{0}\wedge f\in[ \overline{g}G(g)]).
\ee
Indeed, suppose $\neg\eqref{contrje337}$ and let $f_{0}\in E$ be such that $(\forall g\in E)( f_{0}\in [\overline{g}G(g)]\di G(g)> N_{0})$.
Now use $(\exists^{2})$ to define $F_{0}^{2}$ as follows: $F_{0}(h):=N_{0}+1$ if $h=_{1}f_{0}$, and zero otherwise.   
By assumption, we have $\LOC(F_{0}, G)$, but clearly $F(f_{0})>N_{0}$ and $\PIT_{\u}$ yields a contradiction.  
Hence, $\PIT_{\u}^{w}$ implies \eqref{contrje337}, and the latter provides a finite sub-cover for the canonical cover $\cup_{f\leq 1}[\overline{f}G(f)]$.  
Indeed, apply $\QFAC^{1,1}$ to \eqref{contrje337} to obtain a functional $\Phi^{1\di 1}$ providing $g$ in terms of $f$.  
The finite sub-cover (of length $2^{N_{0}}$) then consists of all $\Phi(\sigma*00\dots)$ for all binary $\sigma$ of length $N_{0}$.  
%Then there is a g ? C such that F(g) \UTF{2264} a and f ? Cg\CID{3}(F(g), since otherwise, the functional G that equals a + 1 on f and 0 elsewhere will satisfy the bounding condition induced by (F,F) without being bounded by a.
%Then, for each binary sequence s of length a, there will be a function gs such that Cs ? Cg\CID{3}s(F(gs)), and this gives us a finite sub-covering.
% 
%
\end{proof}
\subsection{Baire category theorem (BCT)}
BCT should yield something.  Nemoto-san proved that BCT implies there is a continuous nowhere differentiable function. 



\bye



