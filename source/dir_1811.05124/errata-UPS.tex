\citet*{ji2012ups} opted to model random signals, where each coordinate has a small probability of hosting a non-zero signal.
This small probability is calibrated such that the expected number of signals is $p^{1-\beta}$, as we do in \eqref{eq:sparsity-parametrized}. Signal sizes are modeled and parametrized in the same fashion as in our paper.

A gap can be identified in the proof of Theorem 1 of \citet*{ji2012ups}.
In particular, in Appendix A.1 of the supplementary material of \citep{ji2012ups}, Ji and Jin derived a lower bound for the probability that a single coordinate being mis-classified (sum of probabilities of Type I error and Type II error).
That is, for any coordinate $j\in \{1,\ldots,p\}$,
$$
\P\left[\mu(j) = 0, \widehat\mu(j) \neq 0\right] + \P\left[\mu(j) \neq 0, \widehat\mu(j) = 0\right] \ge J.
$$
where $J$ is an expression that involves both sparsity and signal size parameters.
Alternatively, in the notations of our paper (if the support set $S$ is modeled as random),
$$
\P\left[j\in S, j\not\in\widehat{S}\right] + \P\left[j\not\in S, j\in\widehat{S}\right] \ge J.
$$
The authors of \citep{ji2012ups} then argued (in line following equation (A.9) in their Appendix A.1) that there are in total $p$ coordinates, and the hamming distance, i.e., expected number of disagreements between $\beta$ and $\widehat\beta$, (or in our notations, the size of symmetric difference between $S$ and $\widehat S$, $\left|S\setminus\widehat S\right| + \left|\widehat S\setminus S\right|$) is lower bounded by $pJ$.

Unfortunately, the mis-classification events on each coordinate are \emph{not} disjoint, especially under correlated errors, and one cannot add up the probability lower bounds as was argued by Ji and Jin.
This final argument does not hold.

We provide a careful analysis for the dependent case in our paper, which shows that one can indeed achieve asymptoticaly exact support recovery for signal sizes below the said strong classification boundary (Theorem \ref{thm:sufficient} and \ref{thm:necessary}) for dependent errors. This is corroborated by simulation results.
