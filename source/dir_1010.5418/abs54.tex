
\begin{abstract}

We consider trap models in $\Z^d$. These are stochastic processes in a random environment as follows.
The  environment is given by a family $\tau=(\tau_x\,,\, x\in\Z^d)$ of positive iid random variables 
in the basin of attraction of an $\a$-stable law, $0<\a<1$. 
Given $\tau$, our process is a continuous time Markov pure jump process, 
whose jump chain is a in principle generic random walk in $\Z^d$, $d\geq1$, independent of $\tau$, and
$\tau$ represents the holding time averages of the continuous time process. We may think of the sites of $\Z^d$ as traps, and
of $\tau_x$ as the depth of trap $x$. We are interested in the {\em trap process}, namely the process that associates 
to time $t$ the depth of the currently visited trap. Our first result is the convergence of the law of that process under suitable scaling. 
The limit process is given by the jumps of a certain $\a$-stable subordinator at the inverse of another $\a$-stable subordinator, correlated with
the first subordinator.
For that result, the requirements on the underlying random walk are $a)$ the validity of a law of large numbers for its range, and 
$b)$ the slow variation at infinity of the tail of the distribution of its time of return to the origin: 
they include all transient random walks as well as all random walks in $d\geq2$, and also many one dimensional random walks,
but not the simple symmetric case. 
We then derive {\em aging results} for our process, namely scaling limits for some two-time correlation functions of the process; 
a strong form of those results requires an assumption of transience, stronger than $a,\,b$ above. 
%a slight strengthening of assumption $b$ above. 
The scaling limit result mentioned above is an averaged result with respect to the environment. 
Under an additional condition on the size of the intersection of the ranges of two independent copies of the underlying 
random walk, roughly saying that it is small compared with size of the range, we derive a stronger scaling limit result, 
roughly stating that it holds in probability with respect to the environment. With that additional condition, we also strengthen 
the aging results, from the averaged version mentioned above, to convergence in probability with respect to the environment.


\end{abstract}

\noindent Keywords and Phrases: trap models, random walks, scaling limit, aging, subordinators, random environment

\smallskip

\noindent AMS 2010 Subject Classifications: 60K35, 60K37

