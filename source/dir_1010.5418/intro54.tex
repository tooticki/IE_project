\section{Introduction} 
\setcounter{equation}{0} 


We begin with a more precise definition of random walks among random traps. These are constructed 
through the following two-step procedure. 
We first choose a probability measure on $\Z^d\setminus\{0\}$, say $\pi$, and let $\tau=(\tau_x\,,\, x\in\Z^d)$ be a collection of 
positive real numbers attached to the points of $\Z^d$. 
The {\it random walk in the trap environment $\tau$} 
is then the continuous time Markov process with values in $\Z^d$ that starts at the origin and 
 has generator 
\begin{equation}\label{eq:df0}
 {\cal L}^\tau f(x)=\frac 1{\tau_x}\sum_y (f(y)-f(x))\pi(y-x)\,,
\end{equation} 
so that 
when sitting at point $x\in\Z^d$, 
the process waits for an exponentially distributed time of mean $\tau_x$ and then jumps to point $x+y$ where $y$ 
is sampled from the distribution $\pi$. This procedure is then iterated with independent hopping times and jumps. 
$\tau$ will be taken at random so that the $(\tau_x\,,\, x\in\Z^d)$ is an independent identically distributed 
family of random variables whose common law belongs to the domain of attraction of a 
stable law of index $\alpha\in(0,1)$. The model thus defined is therefore an example of a random walk in a random 
environment. 
We denote by $\P$ the probability thus defined. More precisely, $\P$ is a probability measure on the product space 
$\Omega\times{\cal D}([0,\infty),\Z^d)$ where $\Omega=(0,\infty)^{\Z^d}$ is the space of {\it trap environments} 
and ${\cal D}([0,\infty),\Z^d)$ is the space of c\`adl\`ag trajectories from $[0,\infty)$ to $\Z^d$. The first marginal of 
$\P$ is of the form $Q^{\Z^d}$, where $Q$ belongs to the domain of attraction of a 
stable law of index $\alpha\in(0,1)$, and the conditional law of the second marginal given $\tau$ is the law of 
the random walk in the trap environment $\tau$. We use the notation $(\X_t\,,\, t\geq 0)$ for the canonical 
projections defined on ${\cal D}([0,\infty),\Z^d)$ that give the position of the random walker 
at times $(t\geq 0)$. 




\begin {rmk} 
It is not difficult to see that if we choose $Q$ with compact support in $(0,+\infty)$ then the behaviour of the random
walk with traps is very similar to the random walk without traps. For instance, if $\pi$ is symmetric with finite support, one 
finds that the scaling limit of $\X$ under diffusive scaling is a Brownian motion. Fluctuations of the environment 
only affect the value of the effective diffusivity. In order to observe stronger slowing down effects, in particular aging, one has to 
choose heavy tailed $\tau$'s as we do here. 
\end{rmk}

This process is an example of a {\it trap model} in the spirit of J-P.~Bouchaud.  
 One important aspect of it is the lack of dependence of $\pi$ on $\tau$. (A class of models where there is such a dependence,
known as {\it asymmetric trap models}, have also been considered in the physics and mathematics literature. See below. Unless explicitely mentioned,
we do not discuss these models here.)
Such processes were initially introduced in the context of statistical mechanics as 
toy models for spin glasses and in order to illustrate the phenomenon of {\it aging}, see \cite{kn:Bou}, 
\cite{kn:BD} or \cite{kn:BCKM} for instance. 
In usual models of spin glasses, the Hamiltonian is a random Gaussian field of large variance. 
 At low temperature, it is natural to guess that the main contributions to the dynamics come from states of 
 low energy. 
As the statistical properties of extremes of 
 log Gaussian fields, the Gibbs factors in this context, are described by random variables 
with polynomial tail, the choice of a law in the basin of attraction of a stable law for $\tau$, 
 which plays a similar role in the simplified model,
is also natural. Note that the parameter $\alpha$ can then be interpreted as the
 %inverse of the 
temperature, see \cite{kn:BBM}  and Subsection 3.2 in~\cite{kn:FL}. 
%{\sc Is that true?}

The aging property refers to the following phenomenon: as time increases, 
the process visits a larger and larger part of 
its state space and therefore increases its probability to find a location $x$ where $\tau_x$ is large. 
Since the time the process stays at location $x$ before jumping off is of order $\tau_x$, some slow down 
effect might take place. One way to measure how much the process is slow is to compute quantities of 
the form
 
\begin{equation}\label{eq:af}
\Pi(s,t)=\P(\X_r=\X_t\,,\, r\in[t,t+s])\,,
\end{equation}  
which are generally called {\it aging functions} in this context.
The Markov property implies that 
\begin{equation}\label{eq:af1}
\Pi(s,t)=\esp(e^{-s/{\tau_{\X_t}}})\,,
\end{equation}  
and thus we observe that a non trivial limit for $\Pi(s,t)$ as $s,t\to\infty$, with $s$ and $t$ related in a given way,
implies that, at large time $t$, 
$\tau_{\X_t}$ should be of order $s$, so that the (order of the) 'age' of the process can actually be approximately 
read from its position at large times. Thus, in order to describe aging, we are led to considering the asymptotics of 
the {\it age} (or  {\it trap}) {\it process} 
$\E=(\E_t=\tau_{\X_t}\,,\,t\geq 0)$. 

The first computations of J-P.~Bouchaud and D.~Dean in \cite{kn:Bou} and \cite{kn:BD} consisted in describing the asymptotics 
of trapped random walks on a large complete graph and in some appropriate scaling. 
Since then, 
the subject has developped into a rich mathematical theory.  
 Mathematical papers treating the model in the complete graph include~\cite{kn:BF} and~\cite{kn:FM}.
%{\sc Quote Fontes-Mathieu on the complete graph.} 
Although one motivation is 
certainly to understand the physicists' claims and prove aging for as realistic as possible 
models of spin glasses, see \cite{kn:ABG1},  \cite{kn:ABG2} and more recently \cite{kn:BBC}, 
it also turns out that trapping and aging effects 
also play a role in models without any connection to spin glass theory such as 
random walks with random conductances or random walks on Galton-Watson trees, see \cite{kn:BC},  \cite{kn:AFGH}.
%{\sc Add reference for trees.} 
The main strategy used in these papers has a strong potential theoretic flavour: for a given 
realisation of the trap environment $\tau$, one tries to identify, among the different points 
 $x$ with large $\tau_x$ which will be hit by the random walk. 
We refer to \cite{kn:AC2} for a presentation of this point of 
view in an abstract setting. One advantage of this approach is that it does not seem to require the state space 
to have many symmetries. It provides strong forms of aging properties that are valid for a given 
realisation of the traps. On the other hand this machinery is often quite heavy to use. 

As far as trap models on $\Z^d$ are concerned, excluding the asymmetric case (where $\pi$ depends on $\tau$
in a specific way, as mentioned above; see~\cite{kn:BC},~\cite{kn:C2},~\cite{kn:AC},~\cite{kn:M}), 
only the case of the simple symmetric random walk was investigated so far.
%(with the exception of the one dimensional model of \cite{kn:AC}).  
It corresponds to $\pi$ being the uniform law on the nearest neighbors of the origin. 
Then the paths of the process $\X$, i.e. the sequence of the different points visited by $\X$, is a symmetric 
nearest neighbor  random walk on $\Z^d$. 
The speed at which the process $\X$ moves i.e. the different hopping times at the successive locations 
are given by the environment $\tau$. 
The one-dimensional case happens to be special:
due to the strong recurrence properties of the simple symmetric random walk on $\Z$, the process localizes.  
This localization effect, aging properties and scaling limits are precisely 
described in \cite{kn:FIN}. The scaling limit is a singular diffusion now known under the name of {\it FIN}. 
In higher dimension $d\geq 2$, the scaling limit is known to be the so called 
{\it Fractional Kinetics} process: $d$-dimensional Brownian motion time changed by the inverse of an 
independent stable subordinator as proved in \cite{kn:C}, \cite{kn:ACM} and \cite{kn:AC1}, the strategy 
being similar to \cite{kn:AC2}. Besides a number of estimates on the Green kernel of simple symmetric random 
walk in $\Z^d$, the proof in the $d=2$ case involves rather sophisticated renormalization technics.
The same result also follows in $d\geq5$ as a particular case of results in~\cite{kn:M}, where a different approach is developed.

What do we do here? 
A first motivation of this paper is to derive aging properties for a more general class of random walks 
than the nearest neighbors case, in the form of an appropriate scaling limit of the age process, as suggested by our discussion above; 
see Theorem~\ref{teo:conv1} in Section~\ref{sec:mod} below. 
In doing so we hope to clarify which properties of the random walk are 
truly relevant for aging. Observe in particular that the usual recurrence versus transience dichotomy does not apply here 
 (as we can already conclude from the results of \cite{kn:AC1} for the simple symmetric case). 
As an outcome, we obtain a new proof of the scaling limit that applies to any genuinely $d$ dimensional 
random walk for $d\geq 2$. This proof is more conceptual than the approach previously used by other authors. 
Indeed we need to know very little about specific estimates for the transition probabilities or Green kernels. 
We also completely avoid the renormalization step, even in the $d=2$ case. It should also be mentioned that 
 Theorem~\ref{teo:conv1} is an annealed result with respect to the environment, as 
opposed to the analogue quenched result of~\cite{kn:AC1}. 
In particular it applies in situations where $\X$ does not have a quenched scaling limit,
see Remark~\ref{rmk:nec_cond} in Section~\ref{sec:str} below.  
It can however be strengthened  with little more effort, say, half way towards a quenched 
result, under a natural additional condition on the intersections of the ranges of independent copies of our process
(see Theorem~\ref{teo:str} in Section~\ref{sec:str} below).

Observe that in our general setting it does not make sense to look for scaling limits of the process $\X$ 
itself. Indeed the underlying random walk (with increments distributed according to $\pi$) may not have a 
non trivial scaling limit. 
 We choose then to focus on the age process $\E$, which is a natural object in the aging context, as we discuss next.
%this choice gives an equivalent representation for $\X$ in case 
The expression of the aging function $\Pi(s,t)$, suggests that we should look at the limiting law 
of $\E_t/s$ to derive  an aging property. We actually provide a more complete answer by describing the 
scaling limit of the full process $\E$ thus establishing a fuller aging picture. 
This scaling limit is expressed as the value of the jump of 
some subordinator computed at the inverse of another subordinator. 
Interestingly, this scaling limit is universal, even if the scale on which the process $\E$ lives depends 
on the random walk (and in particular is linear if and only if the random walk is transient). 

The topology under which we are able to establish Theorem~\ref{teo:conv1} is
quite weak, though, due to the nature of the age process. Obtaining a scaling limit result for aging functions
like~(\ref{eq:af}) requires more work, done in Section~\ref{sec:age} 
(for~(\ref{eq:af}) and two other examples) under the stronger assumption of transience,
and in Subsection~\ref{ssec:stra} with the additional condition of Section~\ref{sec:str}; see
Theorems~\ref{teo:age} and~\ref{teo:ages}. Integrated forms of those results follow from Theorem~\ref{teo:conv1},
under the original assumptions, as discussed separately at the end of those sections.

The remainder of this paper is organized as follows. In Section~\ref{sec:mod} we have a detailed presentation of the model,
assumptions and one result (annealed scaling limit of $\E$), with some more discussion. In Section~\ref{sec:rw}, we discuss some preliminary results
on random walks that are used subsequently. In Section~\ref{sec:conv}, we prove the annealed result just mentioned, and in
Sections~\ref{sec:age} and~\ref{sec:str} we state and prove our further scaling limit results for some aging functions, and in a stronger than
annealed sense, as discussed above.
