\documentclass[reqno]{amsart}
%\usepackage[english]{babel}
\usepackage{amssymb,amsmath,hyperref}
\usepackage{amsrefs}
\usepackage[foot]{amsaddr}
\usepackage{bbold,stackrel}
 
%%%%%%%%%%%%%%%%%%%%%%%%
% Dag Normann LaTeX definitions


%\newcommand{\R}{{\Bbb R}}
%\newcommand{\N}{{\Bbb N}}
%\newcommand{\B}{{\Bbb B}}
%\newcommand{\Q}{{\Bbb Q}}
%\newcommand{\Z}{{\Bbb Z}}
\newcommand{\lb}{[\![}
\newcommand{\rb}{]\!]}
\newcommand{\qcb}{\mathsf{QCB}}
\newcommand{\QCB}{\mathsf{QCB}}
\newcommand{\dminus}{\mbox{$\;^\cdot\!\!\!-$}}

%%%%%%%%%%%%%%%%%%%%%%%%%



\newtheorem{thm}{Theorem}
\newtheorem{lem}[thm]{Lemma}
\newtheorem{cor}[thm]{Corollary}
\newtheorem{defi}[thm]{Definition}
\newtheorem{rem}[thm]{Remark}
\newtheorem{nota}[thm]{Notation}
\newtheorem{exa}[thm]{Example}
\newtheorem{rul}[thm]{Rule}
\newtheorem{ax}[thm]{Axiom}
\newtheorem{set}[thm]{Axiom set}
\newtheorem{sch}[thm]{Axiom schema}
\newtheorem{princ}[thm]{Principle}
\newtheorem{algo}[thm]{Algorithm}
\newtheorem{tempie}[thm]{Template}
\newtheorem{ack}[thm]{Acknowledgement}
\newtheorem{specialcase}[thm]{Special Case}
\newtheorem{theme}[thm]{Theme}
\newtheorem{conj}[thm]{Conjecture}
\newtheorem*{tempo*}{Template}
\newtheorem{theorem}[thm]{Theorem}
%\theoremstyle{plain}\newtheorem{lemma}[thm]{Lemma}
\newtheorem{cth}{Classical Theorem}
\newtheorem{lemma}[thm]{Lemma}
\newtheorem{definition}[thm]{Definition}
\newtheorem{corollary}[thm]{Corollary}
\newtheorem{remark}[thm]{Remark}
\newtheorem{convention}[thm]{Convention}
\newtheorem{proposition}[thm]{Proposition}
\newtheorem{dagcom}{Dag-Comment}
\newtheorem{discussion}[thm]{Discussion}

\newtheorem{fact}{Fact}
\newtheorem{facts}[fact]{Facts}

\newcommand\be{\begin{equation}}
\newcommand\ee{\end{equation}} 

\usepackage{amsmath,amsfonts} 

\usepackage[applemac]{inputenc}


\def\bdefi{\begin{defi}\rm}
\def\edefi{\end{defi}}
\def\bnota{\begin{nota}\rm}
\def\enota{\end{nota}}
\def\FIVE{\Pi_{1}^{1}\text{-\textup{\textsf{CA}}}_{0}}
\def\FIVEK{\Pi_{k}^{1}\text{-\textup{\textsf{CA}}}_{0}}
\def\FIVEo{\Pi_{1}^{1}\text{-\textup{\textsf{CA}}}_{0}^{\omega}}
\def\FIVEFIVE{\Delta_{2}^{1}\text{-\textsf{\textup{CA}}}_{0}}
\def\SIX{\Pi_{2}^{1}\text{-\textsf{\textup{CA}}}_{0}}
\def\SIK{\Pi_{k}^{1}\text{-\textsf{\textup{CA}}}_{0}}
\def\SIXk{\Pi_{k}^{1}\text{-\textsf{\textup{CA}}}_{0}}
\def\SIXK{\Pi_{k}^{1}\text{-\textsf{\textup{CA}}}_{0}^{\omega}}
\def\meta{\textup{\textsf{meta}}}
\def\ATR{\textup{\textsf{ATR}}}
\def\LOC{\textup{\textsf{LOC}}}
\def\MON{\textup{\textsf{MON}}}
\def\PIT{\textup{\textsf{PIT}}}
\def\PITo{\textup{\textsf{PITo}}}
\def\PR{\textup{\textsf{PR}}}
\def\WPR{\textup{\textsf{WPR}}}
\def\ULK{\textup{\textsf{ULK}}}
\def\UB{\textup{\textsf{UB}}}
\def\Z{\textup{\textsf{Z}}}
\def\BIN{\textup{\textsf{BIN}}}
\def\NFP{\textup{\textsf{NFP}}}
\def\ZFC{\textup{\textsf{ZFC}}}
\def\ZF{\textup{\textsf{ZF}}}
\def\IST{\textup{\textsf{IST}}}
\def\MU{\textup{\textsf{MU}}}
\def\MUO{\textup{\textsf{MUO}}}
\def\BET{\textup{\textsf{BET}}}
\def\ALP{\textup{\textsf{ALP}}}
%\def\T{\mathcal{T}}
\def\TT{\mathcal{TT}}
 \def\r{\mathbb{r}}
\def\STP{\textup{\textsf{STP}}}
\def\DNS{\textup{\textsf{STC}}}
\def\PA{\textup{PA}}
\def\FAN{\textup{\textsf{FAN}}}
\def\DNR{\textup{\textsf{DNR}}}
\def\LUB{\textup{\textsf{LUB}}}
\def\RWWKL{\textup{\textsf{RWWKL}}}
\def\RWKL{\textup{\textsf{RWKL}}}
\def\H{\textup{\textsf{H}}}
\def\ef{\textup{\textsf{ef}}}
\def\ns{\textup{\textsf{ns}}}
\def\u{\textup{\textsf{u}}}
\def\c{\textup{\textsf{c}}}
\def\w{\textup{\textsf{w}}}
\def\bs{\textup{\textsf{bs}}}
\def\RCA{\textup{\textsf{RCA}}}
%\def\bennot{\textup{E-HA}^{\omega*}_{st}}
\def\({\textup{(}}
\def\){\textup{)}}
\def\WO{\textup{\textsf{WO}}}
\def\RCAo{\textup{\textsf{RCA}}_{0}^{\omega}}
\def\ACAo{\textup{\textsf{ACA}}_{0}^{\omega}}
\def\WKL{\textup{\textsf{WKL}}}
\def\PUC{\textup{\textsf{PUC}}}
\def\WWKL{\textup{\textsf{WWKL}}}
\def\bye{\end{document}}
%\def\rec{\textup{rec}}
%\def\sP{^{*}\mathcal  P}
\def\P{\textup{\textsf{P}}}
%\def\Pf{{\mathcal{P}_{\textup{fin}}}}
\def\N{{\mathbb  N}}
\def\Q{{\mathbb  Q}}
\def\R{{\mathbb  R}}
\def\L{\textsf{\textup{L}}}
\def\A{{\mathbb  A}}
\def\subsetapprox{\stackrel[\approx]{}{\subset}}
\def\C{{\mathbb  C}}
\def\PC{\textup{\textsf{PC}}}
%\def\CC{{\mathfrak  C}}
\def\NN{{\mathfrak  N}}
\def\B{{\mathbb  B}}
\def\I{{\textsf{\textup{I}}}}
\def\D{{\mathbb  D}}
\def\E{{\mathcal  E}}
\def\FAN{\textup{\textsf{FAN}}}
\def\PUC{\textup{\textsf{PUC}}}
\def\WFAN{\textup{\textsf{WFAN}}}
\def\UFAN{\textup{\textsf{UFAN}}}
\def\MUC{\textup{\textsf{MUC}}}
\def\MPC{\textup{\textsf{MPC}}}
\def\st{\textup{st}}
\def\di{\rightarrow}
\def\asa{\leftrightarrow}
\def\ACA{\textup{\textsf{ACA}}}
\def\paai{\Pi_{1}^{0}\textup{-\textsf{TRANS}}}
\def\Paai{\Pi_{1}^{1}\textup{-\textsf{TRANS}}}
\def\QFAC{\textup{\textsf{QF-AC}}}
\def\ml{\textup{\textsf{ml}}}
\def\MLR{\textup{\textsf{MLR}}}
\def\HBU{\textup{\textsf{HBU}}}
\def\inv{\textup{\textsf{inv}}}
\def\IVT{\textup{\textsf{IVT}}}
\def\MPR{\textup{\textsf{MPR}}}
\def\PRE{\textup{\textsf{PRE}}}
\def\oc{\textup{\textsf{opencov}}}
\def\WEI{\textup{\textsf{WEI}}}
\def\FEJ{\textup{\textsf{FEJ}}}
\def\DIN{\textup{\textsf{DIN}}}
\def\LIN{\textup{\textsf{LIND}}}
\def\LIND{\textup{\textsf{LIND}}}
\def\WHBU{\textup{\textsf{WHBU}}}
\def\UCT{\textup{\textsf{UCT}}}
\def\HBC{\textup{\textsf{HBC}}}
\def\BDN{\textup{\textsf{BD-N}}}
\def\RPT{\textup{\textsf{RPT}}}
\def\CCS{\textup{\textsf{CCS}}}
\def\csm{\textup{\textsf{csm}}}
\def\compact{\textup{\textsf{compact}}}
\def\POS{\textup{\textsf{POS}}}
\def\KB{\textup{\textsf{KB}}}
\def\MAX{\textup{\textsf{MAX}}}
\def\FIP{\textup{\textsf{FIP}}}
\def\CAC{\textup{\textsf{CAC}}}
\def\ORD{\textup{\textsf{ORD}}}
\def\DRT{\Delta\textup{\textsf{-RT}}}
\def\CCAC{\textup{\textsf{CCAC}}}
\def\SCT{\textup{\textsf{SCT}}}
\def\IPP{\textup{\textsf{IPP}}}
\def\COH{\textup{\textsf{COH}}}
\def\SRT{\textup{\textsf{SRT}}}
\def\CAC{\textup{\textsf{CAC}}}
\def\tnmt{\textup{\textsf{tnmt}}}
\def\EM{\textup{\textsf{EM}}}
\def\field{\textup{\textsf{field}}}
\def\CWO{\textup{\textsf{CWO}}}
\def\DIV{\textup{\textsf{DIV}}}
\def\KER{\textup{\textsf{KER}}}
\def\KOE{\textup{\textsf{KOE}}}
\def\MCT{\textup{\textsf{MCT}}}
\def\UMCT{\textup{\textsf{UMCT}}}
\def\UATR{\textup{\textsf{UATR}}}
\def\ST{\mathbb{ST}}
\def\mSEP{\textup{\textsf{-SEP}}}
\def\mUSEP{\textup{\textsf{-USEP}}}
\def\STC{\textup{\textsf{STC}}}
\def\ao{\textup{\textsf{ao}}}
\def\SJ{\mathbb{S}}
\def\PST{\textup{\textsf{PST}}}
\def\eps{\varepsilon}
\def\con{\textup{\textsf{con}}}
\def\X{\textup{\textsf{X}}}
\def\HYP{\textup{\textsf{HYP}}}
\def\DT{\textup{\textsf{DT}}}
\def\LO{\textup{\textsf{LO}}}
\def\CC{\textup{\textsf{CC}}}
\def\ADS{\textup{\textsf{ADS}}}
\def\WADS{\textup{\textsf{WADS}}}
\def\CADS{\textup{\textsf{CADS}}}
\def\SU{\textup{\textsf{SU}}}
\def\RF{\textup{\textsf{RF}}}
\def\RT{\textup{\textsf{RT}}}
\def\RTT{\textup{\textsf{RT22}}}
\def\SRTT{\textup{\textsf{SRT22}}}
\def\CRTT{\textup{\textsf{CRT22}}}
\def\WT{\textup{\textsf{WT}}}
\def\EPA{\textup{\textsf{E-PA}}}
\def\EPRA{\textup{\textsf{E-PRA}}}
\def\TRM{\textup{\textsf{TRM}}}
\def\FF{\textup{\textsf{FF}}}
\def\TOF{\textup{\textsf{TOF}}}
\def\ECF{\textup{\textsf{ECF}}}
\def\WWF{\textup{\textsf{WWF}}}
\def\NUC{\textup{\textsf{NUC}}}
%\def\VCF{\textup{\textsf{VCF}}}
\def\LMP{\textup{\textsf{LMP}}}
%\def\RKL{\textup{\textsf{RKL}}}
\def\TJ{\textup{\textsf{TJ}}}
%\def\SJ{\textup{\textsf{SJ}}}
%\def\SHJ{\textup{\textsf{SHJ}}}
%\def\META{\textup{\textsf{META}}}
\def\SCF{\textup{\textsf{SCF}}}
\def\ESC{\textup{\textsf{ESC}}}
\def\DSC{\textup{\textsf{DSC}}}
\def\SC{\textup{\textsf{SOC}}}
\def\SOC{\textup{\textsf{SOC}}}
\def\SOT{\textup{\textsf{SOT}}}
\def\TOT{\textup{\textsf{TOT}}}
\def\WCF{\textup{\textsf{WCF}}}
\def\HAC{\textup{\textsf{HAC}}}
\def\INT{\textup{\textsf{int}}}
\newcommand{\barn}{\bar \N}
\newcommand{\Ct}{\mathsf{Ct}}
\newcommand{\Tp}{\mathsf{Tp}}
\newcommand{\T}{\mathcal{T}}
\newcommand{\Ps}{{\Bbb P}}
\newcommand{\F}{{\bf F}}
%\newcommand{\I}{{\bf I}}
\newcommand{\rinf}{\rightarrow \infty}
\newcommand{\true}{{\bf tt}\hspace*{1mm}}
\newcommand{\false}{{\bf ff}\hspace*{1mm}}
\newcommand{\RP}{{Real-$PCF$}\hspace*{1mm}}
%\newcommand{\dminus}{\mbox{$\;^\cdot\!\!\!-$}}
\newcommand{\un}{\underline}
\newcommand{\rec}{\mathsf{Rec}}
\newcommand{\PCF}{\mathsf{PCF}}
\newcommand{\LCF}{\mathsf{LCF}}
\newcommand{\Cr}{\mathsf{Cr}}
\newcommand{\m}{{\bf m}}



\usepackage{graphicx}
\usepackage{tikz}
\usetikzlibrary{matrix, shapes.misc}

\setcounter{tocdepth}{3}
\numberwithin{equation}{section}
\numberwithin{thm}{section}

\usepackage{comment}

\begin{document}
\title{Uniformity in mathematics}
\author{Dag Normann}
\address{Department of Mathematics, The University 
of Oslo, Norway}
\email{dnormann@math.uio.no}
\author{Sam Sanders}
\address{Department of Mathematics, TU Darmstadt, Germany \& School of Mathematics, University of Leeds, UK}
\email{sasander@me.com}

\begin{abstract}
The 19th century saw a systematic development of real analysis in which many theorems were proved using \emph{compactness}.  
In the work of Dini, Pincherle, Bolzano, Young, Riesz, Hardy, and Lebesgue, one finds such proofs which (sometimes with minor modification) additionally are \emph{highly uniform} in the sense that the objects proved to exist only depend on few of the parameters of the theorem.  
More recently, similarly uniform results have been obtained as part of the redevelopment of analysis based on techniques from \emph{gauge integration}.  
Our aim is to study such `highly uniform' theorems in Reverse Mathematics and computability theory.  
Our prototypical example is \emph{Pincherle's theorem}, published in 1882, which states that a locally bounded function is bounded on certain domains.  
We show that both the `original' and `uniform' versions of Pincherle's theorem have noteworthy properties.  
In particular, the upper bound from Pincherle's theorem turns out to be \emph{extremely hard} to compute in terms of (some of) the data, while the uniform version of Pincherle's theorem requires \emph{full} second-order arithmetic for a proof.
We obtain similar results for \emph{Heine's uniform continuity theorem} and \emph{Fej\'er's theorem}.  
Our study of the role of the axiom of countable choice in the aforementioned results leads to the observation that the status of the \emph{Lindel\"of lemma} is highly dependent on its formulation (provable in second-order arithmetic vs unprovable in $\ZF$).    
\end{abstract}

%\setcounter{page}{0}
%\tableofcontents
%\thispagestyle{empty}
%\newpage
	
\maketitle
\thispagestyle{empty}

\vspace{-0.8cm}
%\begin{quote}
%\emph{The whole is greater than the sum of its parts} (Aristotle)
%\end{quote}
%Mention \cite{kermend}, esp. that CCMC is not provable in $\ZF^{-}$
\section{Introduction}
\subsection{Aim and motivation}\label{intro}
The motivation for this paper stems from the (historical and modern) connection between \emph{compactness} and \emph{uniformity}, as follows.  

\smallskip
 
As to \emph{compactness}, the importance of this notion cannot be overstated, as it provides a direct connection between local and global properties and guarantees that limits are well-behaved.  
Historically, the 19th century saw the first systematic development of real analysis -spearheaded by Bolzano, Weierstrass, and others- in which many now fundamental theorems were proved using {compactness}.    

\smallskip

As to \emph{uniformity}, some of these proofs (due to Dini, Pincherle, Bolzano, Young, Hardy, Riesz, and Lebesgue) deserve attention as they are \emph{highly uniform} (sometimes after minor modification) in the sense that the objects claimed to exist by the theorem only depend on few of the parameters of the theorem.  
More recently, similarly uniform results have been obtained as part of the development of analysis based on techniques from the \emph{gauge integral}, a generalisation of Lebesgue's integral.  

\smallskip

Our aim is to study such `highly uniform' theorems in Reverse Mathematics (RM hereafter) and computability theory; we discuss the latter fields in Section \ref{prelim}.  % as follows.
Our starting point, and illustrative example, is \emph{Pincherle's theorem}.  
\begin{thm}[Pincherle]\label{gem}
Let $E$ be a closed, bounded subset of $\mathbb{R}^{n}$ and let $f : E \di \R$ be
locally bounded. Then $f$ is bounded on $E$.
\end{thm}
%To this end, we shall study Theorem~\ref{gem}, which we shall refer to as \emph{Pincherle's theorem}, from the point of view of Reverse Mathematics (RM for short) and computability theory.  
As to its history, Theorem \ref{gem} was (essentially) established by \emph{Salvatore Pincherle} in 1882 in \cite{tepelpinch}*{p.\ 67} in a more verbose formulation. 
Indeed, Pincherle did not use the notion of \emph{local boundedness}, and a function is nowadays called \emph{locally bounded on $E$} if every $x\in E$ has a neighbourhood $U\subset E$ on which the function is bounded.  

\smallskip

Note that Pincherle assumed the existence of $L, r:E\di \R^{+}$ such that for any $x\in E$ the function is bounded by $L(x)$ on the ball $B(x, r(x))\subset E$ (\cite{tepelpinch}*{p.\ 66-67}).
We refer to these functions $L, r:E\di \R^{+}$ as \emph{realisers} for local boundedness.  We do not restrict the notion of realiser to any of its established technical definitions.
% in keeping with the usual computability-theoretic nomenclature. 

\smallskip

As to its conceptual nature, Pincherle's theorem may be found as \cite{gormon}*{Theorem~4} in a \emph{Monthly} paper aiming to provide conceptually easy proofs of well-known theorems.  % i.e.\ the following theorem should qualify as `basic' mathematics. %using tagged partitions.  
Furthermore, Pincherle's theorem is the \emph{sample theorem} in \cite{thom2}, a recent monograph dealing with \emph{elementary real analysis}.  
Thus, Pincherle's theorem qualifies as `basic' mathematics in any reasonable sense of the word, and is also definitely within the scope of RM as it essentially predates set theory (\cite{simpson2}*{I.1}).  % (See Section \ref{prelim} for RM).  

\smallskip

Despite the aforementioned `basic nature' of Pincherle's theorem, its proofs in \cites{gormon, thom2, tepelpinch,bartle2} actually provide `highly uniform' information: as shown in Section~\ref{pproof}, these proofs establish Pincherle's theorem \emph{and that the bound in the consequent only depends on the realisers $r, L:E\di \R^{+}$ for local boundedness}.  Note that in the case of \cite{tepelpinch} we need a minor modification of the proof, as discussed in Section \ref{forgopppp}.
\smallskip

As discussed in detail in Section \ref{sum}, one of our main aims is the study of the `highly uniform' version of Pincherle's theorem in which the bound in the consequent only depends on the realisers $r, L:E\di \R^{+}$.  
As it turns out, both the original and uniform versions of Pincherle's theorem have noteworthy properties from the point of view of RM and computability theory.  
In particular, we answer the following questions, where `computable' refers Kleene's S1-S9, as discussed in Section \ref{HCT}
\begin{enumerate}
\renewcommand{\theenumi}{\roman{enumi}}
\item How hard is it to compute the upper bound in Pincherle's theorem in terms of (some of) the data?
\item What is the computational strength of the ability to obtain the upper bounds from Pincherele's theorem?
\item How do the original and uniform versions of Pincherle's theorem compare to the \emph{Big Five} systems from RM and the G\"odel hierarchy?
\item How does Pincherle's theorem relate to basic theorems from RM, in particular those (about continuity) equivalent to \emph{weak K\"onig's lemma}?\label{Q3}
%\item Do the properties of Pincherle's theorem in RM and computability theory always match up?
\end{enumerate}
While Pincherle's theorem constitutes an illustrative example, it is by no means an isolated event: we analogously study \emph{Heine's theorem} on uniform continuity and sketch the (highly similar) approach for \emph{Fej\'er's theorem}.  
These results are a natural outgrowth of question \eqref{Q3}, and a number of theorems from the RM of weak K\"onig's lemma will be studied in a follow-up paper (See Remark \ref{flurki} for details).   
% in the same way.  

\smallskip

Finally, like in \cite{dagsamIII}, statements of the form `a proof of uniform Pincherle's theorem requires full second-order arithmetic' should be interpreted in reference to the usual scale of comprehension axioms that is part of the \emph{G\"odel hierarchy} (See Appendix \ref{kurtzenhier} for the latter).  
The previous statement thus (merely) expresses that there is no proof of uniform Pincherle's theorem using comprehension axioms restricted to a sub-class, like e.g.\ $\Pi_{k}^{1}$-formulas (with only first and second-order parameters).  An intuitive visual clarification may be found in 
Figure \ref{xxy}, where uniform Pincherle's theorem is shown to be independent of the medium range of the G\"odel hierarchy. 

\subsection{Pincherle's theorem and uniformity}\label{sum}
We formally introduce Pincherle's theorem and the aforementioned `highly uniform' version, and discuss the associated results, to be established in Sections \ref{PRS} and \ref{prm}.
% \marginpar{\footnotesize{We should be careful using "surprising and interesting" too often, this kind of "bragging" may irritate a reader. It must be up to the reader to decide what is surprising and what is interesting. This remark is valid for other parts of the paper as well.}}

\smallskip

First of all, to reduce technical details to a minimum, we mostly work with Cantor space, denoted $2^{\N}$ or $C$, rather than the unit interval; the former is homeomorphic to a closed subset of the latter anyway. 
The advantage is that we do not need to deal with the coding of real numbers using Cauchy sequences, which can get messy.  

\smallskip

Secondly, in keeping with Pincherle's use of $L, r:\R\di \R^{+}$, we say that $G:C\di \N$ is a \emph{realiser} for the \emph{local boundedness} of the functional $F:C\di \N$ if 
\[
\LOC(F, G)\equiv (\forall f , g\in C)\big[ g\in [\overline{f}G(f)] \di F(g)\leq G(f)    \big].
\]
Note that $\overline{f}n= \langle f(0), f(1), \dots, f(n-1)\rangle $ for $n\in \N$, while $g\in [\overline{f}n]$ means that $g(m)=f(m)$ for $m< n$.  
Hence, $\LOC(F, G)$ expresses that $G$ provides for every $f\in C$ a \emph{neighbourhood} $[\overline{f}G(f)]$ in $C$ in which $F$ is bounded by $G(f)$.  

\smallskip

  We make use of \emph{one} functional $G$ for \emph{both} the neighbourhood and upper bound, while Pincherle uses \emph{two} separate 
functions $L$ (for the upper bound) and $r$ (for the neighbourhood); as discussed in Remark \ref{nodiff}, this makes no difference.  % for any of the below results.  

\smallskip

Thirdly, the following are the \emph{original} and \emph{uniform} versions of Pincherle's theorem for Cantor space, respectively $\PIT_{o}$ and $\PIT_{\u}$.
As discussed in Section \ref{forgopppp}, Pincherle's proof from \cite{tepelpinch} (with minor modification only) yields $\PIT_{\u}$; the same holds for \cite{thom2, gormon, bartle2} without any changes to the proofs.  
%\bdefi[$\PIT_{o}$]
\be\tag{$\PIT_{o}$}
(\forall F, G:C\di \N)(\exists N\in \N)\big[  \LOC(F, G)\di (\forall g \in C)(F(g)\leq N)\big]
\ee
\vspace{-0.5cm}
\be\tag{$\PIT_{\u}$}
(\forall G:C\di \N)(\exists N\in \N)(\forall F:C\di \N)\big[  \LOC(F, G)\di (\forall g \in C)(F(g)\leq N)\big]
\ee
%\edefi
The difference in quantifier position is important: by Corollary \ref{ofmoreinterest}, $\PIT_{o}$ is essentially provable in the second Big Five of RM (i.e.\ $\WKL_{0}$ with higher types and a weak fragment of the axiom of choice), while $\PIT_{\u}$ requires \emph{full second-order arithmetic} 
for a proof.  Furthermore, by Theorem \ref{forgu}, \emph{local boundedness} is equivalent in weak systems to \emph{subcontinuity}, a kind of sequential continuity.  
Hence, Pincherle's theorem is a generalisation of the following theorem from the RM of $\WKL_{0}$:~\emph{a continuous function on Cantor space is bounded} (\cite{simpson2}*{IV.2.2}). 
%We discuss more results related to the RM of weak K\"onig's lemma in Section \ref{pichte2}.  
Finally, $\PIT_{\u}$ is equivalent to the Heine-Borel theorem \emph{for uncountable covers} by Corollary~\ref{eessje}, over the `base theory' of higher-order RM \emph{plus} a weak fragment of the axiom of choice. %  \marginpar{\footnotesize{modulo the comprehension axiom just for computable sets and a very weak fragment of the axiom of choice.}}

\smallskip

Fourth, it is a natural question how hard it is to compute an upper bound as in Pincherle's theorem from (some of) the data.  To this end, we consider the specification for a (non-unique) functional $M:(C\di \N)\di \N$ as follows.  
\be\tag{$\PR(M)$}
(\forall F, G:C\di \N)\big[  \LOC(F, G)\di (\forall g \in C)(F(g)\leq M(G))\big].
\ee
%Note that $M$ only takes $G$ as an input.  
Any $M$ satisfying $\PR(M)$ is called a \emph{realiser}\footnote{We use the term \emph {realiser} in a quite liberal way. In fact, Pincherle realisers are witnesses to the truth of \emph{uniform} Pincherle's theorem by selecting, to each $G$, an upper bound as in $\PIT_{\u}$.  However, the set of upper bounds, seen as a function of $G$, is highly complex: the PR that selects the \emph{least} bound is computationally equivalent to $\exists^3$ from Section \ref{HCT}, which is left as an exercise.} for Pincherle's theorem $\PIT_{\u}$, or a \emph{Pincherele realiser} (PR) for short.  
%A \emph{weak} Pincherle realiser additionally has the function $F$ from as input.  
By Section~\ref{PRR}, PRs are 
extremely hard to compute, as they are not computable in any type two functional.  The functional $\exists^{3}$ from Section~\ref{HCT} computes (certain) PRs, but yields full second-order arithmetic.    

\smallskip

In a nutshell, we answer the questions from Section \ref{intro} as follows. 
%\marginpar{\footnotesize{Rewrote (i)}}
\begin{enumerate}
\renewcommand{\theenumi}{\roman{enumi}}
%\item Both normal and weak Pincherle realisers cannot be computed (in the sense of Kleene's S1-S9) by any type two functional, but $\exists^{3}$ does compute them.  
\item Pincherle realisers cannot be computed (in the sense of Kleene's S1-S9) from any type two functional, but some may be computed from $\exists^3$.
\item Pincherle realisers compute realisers of $\Pi^1_1$-separation for subsets of $\N^\N$ and natural generalisations to sets of objects of type two.
\item Pincherle's theorem $\PIT_{\u}$ falls \emph{far} outside the Big Five of RM and requires far stronger systems than the latter, namely \emph{full second-order arithmetic}.
\item Pincherle's theorem(s) may be reformulated with `subcontinuous' instead of `locally bounded', making it a natural generalisation of \emph{a continuous function on Cantor space is bounded} from classical RM (\cite{simpson2}*{IV.2.2}).  
Furthermore, $\PIT_{\u}$ is equivalent to the Heine-Borel theorem for uncountable covers, over a weak system. See Corollary~\ref{eessje} for a precise statement.    
\end{enumerate}
These results are in line with in \cites{dagsamIII, dagsam, dagsamII}, where we answered similar questions for a number of covering theorems like the \emph{Cousin and Lindel\"of lemmas}, and the 
associated development of the gauge integral.  
A notable -and important- difference between the latter lemmas and Pincherle's (original) theorem is that the latter's behaviour in computability theory (See Corollary \ref{comb3} and Theorem \ref{horniii}) and RM (See Theorem \ref{ofinterest}) diverges \emph{completely}: by these results $\PIT_{o}$ is \emph{extremely easy to prove}, but the upper bound in $\PIT_{o}$ is \emph{extremely hard} to compute (in terms of $F, G$).  % (in any natural way). 
%We discuss this in more detail in Section \ref{prm1}. 
%We finish this section with a remark on Pincherle's theorem.   
%The latter aspect constitutes the core property of our \emph{Pinchere realisers}, which intuitively speaking compute the upper bound on $I$ in terms of the functionals $r, L$ as input \emph{only}.   
\begin{rem}[Variations of Pincherle's theorem]\label{kowlk}\rm
Pincherle describes the following theorem in a footnote on \cite{tepelpinch}*{p.\ 67}: 
\begin{quote}
\emph{Let $E$ be a closed, bounded subset of $\mathbb{R}^{n}$ and let $f : E \di \R$ be locally bounded away from $0$.  Then $f$  has a positive infimum on $E$}.
\end{quote}
He states that this theorem is proved in the same way as Theorem \ref{gem} and provides a generalisation of Heine's theorem as proved by Dini in \cite{dinipi}.  
We could formulate versions of the centred theorem, and they would be equivalent to the associated versions of Pincherle's theorem.    
Restricted to \emph{uniformly continuous} functions, the centred theorem is studied in \emph{constructive} RM (\cite{brich}*{Ch.\ 6}).  
Lest there be any doubt, we show in Remark \ref{kiekenkkk} that Pincherle works with \emph{arbitrary} functions.  % not just continuous ones.    
%\marginpar{\footnotesize{added "be" in the last line of the remark}}
\end{rem}
\subsection{Heine's theorem and uniformity}\label{heineke}
We formally introduce \emph{Heine's theorem} and the associated `highly uniform' version, and discuss the associated results, to be established in Sections \ref{PRS} and \ref{prm}. 
As in the previous section, we work over $2^{\N}$.  % to avoid coding real numbers.  

\smallskip

First of all, Heine's theorem is the statement that \emph{a continuous $f:X\di \R$ on a compact space $X$ is uniformly continuous}.   
Dini's proof (\cite{dinipi}*{\S41}) of Heine's theorem makes use of a \emph{modulus of continuity}, i.e.\ a functional computing $\delta$ from $\eps>0$ and $ x\in X$ in the usual $\eps$-$\delta$-definition of continuity.  
As discussed in \cite{ruskesnokken}, Bolzano's definition of continuity involves a modulus of continuity, while his (apparently faulty) proof of Heine's theorem may be found in \cite{nogrusser}*{p.\ 575}.    
The following formula expresses that $G$ is a modulus of (pointwise) continuity for $F$ on $C$:
\be\label{XYX}\tag{$\MPC(G, F)$}
(\forall f, g\in C)(\overline{f}G(f)=\overline{g}G(f)\di F(f)=F(g)).
\ee
Secondly, we introduce $\UCT_{\u}$, the \emph{uniform} Heine's theorem for $C$.  
By Section \ref{heikel}, the proofs by Dini, Bolzano, Young, Hardy, Riesz, Thomae, and Lebesgue (\cite{dinipi,lebes1,nogrusser,younger, manon,hardy, thomeke}) establish the uniform $\UCT_{\u}$ for $[0,1]$ (with minor modification for \cite{dinipi,nogrusser, thomeke }); the same for \cite{langebaard, thom2,knapgedaan ,gormon, bartle2, hobbelig, protput ,stillebron, botsko, lebes1} without changes.  
\bdefi[$\UCT_{\u}$]
\[
(\forall G^{2})(\exists m^{0})(\forall F^{2})\big[\MPC(G, F) \di (\forall  f,g \in C)(\overline{f}m=\overline{g}m\di F(f)=F(g))  ].
\]
\edefi
The difference in quantifier position has big consequences: Heine's theorem is essentially provable in the second Big Five system of RM by \cite{kohlenbach4}*{Prop.\ 4.10}, 
while $\UCT_{\u}$ requires \emph{full second-order arithmetic} for a proof.   Indeed, we prove in Section~\ref{myheinie} that $\UCT_{\u}$ is equivalent to the Heine-Borel theorem for \emph{uncountable} covers, and hence to $\PIT_{\u}$.  The previous equivalences require an `intermediate' version of Heine's theorem based on the codes used in RM, introduced next.   

\smallskip

Now, the logical framework for RM is \emph{second-order arithmetic}, i.e.\ only natural numbers and sets thereof are available.  
Thus, higher-order objects are represented in RM by (countable) \emph{codes}; the representation of continuous functions is given by \cite{simpson2}*{II.6.1}.  
The following formula, abbreviated `$\alpha\in K_{0}$', essentially expresses that $\alpha:\N\di \N$ is a code in the sense of RM; the sequence $\alpha$ is also called an `associate':
\[
(\forall f^{1})(\exists n^{0})(\alpha(\overline{f}n)>_{0}0) \wedge (\forall n^{0}, m^{0},f^{1}, )(m>n \wedge \alpha(\overline{f}n)>0\di  \alpha(\overline{f}n)=_{0} \alpha(\overline{f}m) ).
\]
The value $\alpha(f)$ for $\alpha\in K_{0}$ is defined as the unique $\alpha(\overline{f}n)-1$ for $n$ large enough.  It is standard (abuse of language) to treat $\alpha\in K_{0}$ as a type two functional $\lambda f.\alpha(f)$.  The set of associates $K_{0}$ is $\Pi_{1}^{1}$-complete, i.e.\ the quantifier `$(\forall \alpha \in K_{0})$' entails $\Pi_{2}^{1}$-complexity.       
Historically, associates were first introduced by Kleene in the early days of higher-order computability theory (\cite{longmann}*{\S2.3.1}).    
%Logically, the quantifier `$(\forall \alpha \in K_{0})$'     

\smallskip

In higher-order arithmetic, a functional $\Phi:\N^{\N}\di\N^{\N}$ has a \emph{continuous} modulus of continuity if and only if there is a code $\alpha\in K_{0}$ such that $\lambda f.\alpha(f)$ equals $\Phi$ on Baire space (\cite{kohlenbach4}*{Prop.~4.4}).  
The following principle is essentially uniform Heine's theorem for RM codes on $C$, which is also equivalent to $\UCT_{\u}$ by Corollary \ref{roofer}. % on Cantor space.  
\bdefi[$\UCT_{\u}'$]
\[
(\forall G^{2})(\exists m^{0})(\forall \alpha^{1}\in K_{0})\big[\MPC(G, \alpha) \di (\forall  f,g \in C)(\overline{f}m=\overline{g}m\di \alpha(f)=\alpha(g))  ], 
\]
\edefi
%\marginpar{\footnotesize{I do not think that $\MPC(G,\alpha)$ is defined, and we should at least comment on that. What we mean is of course obvious.}}
The computability-theoretic differences between the uniform and original versions of Heine's theorem are as follows: on one hand, assuming $\MPC(F, G)$, one computes\footnote{If $\MPC(G,F)$, one computes an associate for $F:C\di \N$ from $F$ and $\exists^2$, and one then computes an upper bound for $F$ on $C$, as the \emph{fan functional} has a computable code (\cite{noortje}*{p.\ 102}). } the upper bound from (original) Heine's theorem in terms of $F$ and $\exists^{2}$ from Section \ref{HCT}, i.e.\ the third Big Five system suffices.

\smallskip

On the other hand, given $G$, the class of $F$ such that $\MPC(G,F)$ is equicontinuous (and finite if $F$ is restricted to $C$), but computing a \emph{modulus} of equicontinuity from $G$ is as hard as computing a PR from $G$.
In this light, the (original) Heine theorem is simpler than the (original) Pincherle theorem in computability theory, while the uniform versions are equivalent both in RM and computability theory.  

\smallskip

Clearly, many theorems from the RM of $\WKL_{0}$ can be studied in the same way as Pincherle's and Heine's theorems; 
we provide one such example, namely Fej\'er's theorem, in Section \ref{myheinie}, while a systematic study is reserved for a follow-up paper.  
%in Section \ref{pichte2}, including \emph{weak K\"onig's lemma itself}, and \emph{Dini's theorem}.  

\smallskip

Finally, many results in this paper (and those in \cite{dagsamIII}) are obtained using fragments of the axiom of countable choice.   
It is a natural RM-question, posed previously by Hirschfeldt (See \cite{montahue}*{\S6.1}), whether such fragments of choice are necessary.  
We answer this question in Section \ref{rose}, and in the process reveal that the strength of the \emph{Lindel\"of lemma} is extremely dependent
on its formulation: one version is provable in $\Z_{2}^{\Omega}$ (and even provable in a conservative extension of $\WKL_{0}$); a slight variation of the first version is unprovable in $\ZF$. 

%\smallskip
%NEW NEW NEW
%\marginpar{\footnotesize{The problem with this new remark is that it is a different kind of uniformity we are talking about. In the important examples the proof leads to numerical information that is independent of some of the data, while I cannot see that this is another example of that kind of uniformity.}}
%Finally, we point out that uniformity does not have to lead to logical or computational strength: Cauchy's `infamous' theorem (\cite{cauchy1}*{p.\ 131-132}), formulated in modern terms, states that given a sequence of continuous functions $f_{n}:I\di \R$ which is uniformly convergent to $f:I\di \R$, the latter is also continuous on $I\equiv [0,1]$.  
%One readily defines the modulus of continuity for $f$ in terms of the moduli for continuity and uniform convergence (via a term of G\"odel's $T$), but independent of $f, f_{n}$.  
\section{Preliminaries}\label{prelim}
We sketch the program \emph{Reverse Mathematics} in Section \ref{RM}, as well as its generalisation to \emph{higher-order arithmetic} in Section \ref{KOH}.  
As our main results will be proved using techniques from \emph{computability theory}, we discuss the latter in Section~\ref{HCT}.
\subsection{Introducing Reverse Mathematics}\label{RM}
Reverse Mathematics (RM) is a program in the foundations of mathematics initiated around 1975 by Friedman (\cites{fried,fried2}) and developed extensively by Simpson (\cite{simpson2}) and others.  
We refer to \cite{stillebron} for a basic introduction to RM and to \cite{simpson2} for an overview of RM; we now sketch some of the aspects of RM essential to this paper.  
%In particular, we first describe the elegant `received view' of RM, and then formulate evidence against it.  

\smallskip
  
The aim of RM is to find the axioms necessary to prove a statement of \emph{ordinary}, i.e.\ \emph{non-set theoretical} mathematics.   
The classical base theory $\RCA_{0}$ of `computable mathematics' is always assumed.  
Thus, the aim of RM is:  
\begin{quote}
\emph{The aim of \emph{RM} is to find the minimal axioms $A$ such that $\RCA_{0}$ proves $ [A\di T]$ for statements $T$ of ordinary mathematics.}
\end{quote}
Surprisingly, once the minimal axioms $A$ have been found, we almost always also have $\RCA_{0}\vdash [A\asa T]$, i.e.\ not only can we derive the theorem $T$ from the axioms $A$ (the `usual' way of doing mathematics), we can also derive the axiom $A$ from the theorem $T$ (the `reverse' way of doing mathematics).  In light of these `reversals', the field was baptised `Reverse Mathematics'.  

\smallskip

Perhaps even more surprisingly, in the majority of cases, for a statement $T$ of ordinary mathematics, either $T$ is provable in $\RCA_{0}$, or the latter proves $T\asa A_{i}$, where $A_{i}$ is one of the logical systems $\WKL_{0}, \ACA_{0},$ $ \ATR_{0}$ or $\FIVE$ from \cite{simpson2}*{I}.  The latter four systema together with $\RCA_{0}$ form the `Big Five' and the aforementioned observation that most mathematical theorems fall into one of the Big Five categories, is called the \emph{Big Five phenomenon} (\cite{montahue}*{p.~432}).  

\smallskip

Furthermore, each of the Big Five has a natural formulation in terms of (Turing) computability (See \cite{simpson2}*{I}), and each of the Big Five also corresponds (sometimes loosely) to a foundational program in mathematics (\cite{simpson2}*{I.12}).  
The Big Five systems of RM also satisfy a linear order, as follows:
\be\label{linord}
\FIVE\di \ATR_{0}\di \ACA_{0}\di\WKL_{0}\di \RCA_{0}.
\ee
By contrast, there are many incomparable \emph{logical} statements in second-order arithmetic.  For instance, a regular plethora of such statements may be found in the \emph{Reverse Mathematics zoo} in \cite{damirzoo}.  The latter is intended as a collection of (somewhat natural) theorems outside of the Big Five classification of RM.  It is also worth noting that the Big Five only constitute a \emph{very tiny fragment} of $\Z_{2}$; on a related note, the RM of topology does give rise to theorems equivalent to $\SIX$ (\cite{mummy}), but that is the current upper bound of RM to the best of our knowledge.  % In particular, if $\SIXk$ is $
Moreover, the coding of topologies is not without problems, as discussed in \cite{hunterphd}. 
\subsection{Higher-order Reverse Mathematics}\label{KOH}
We sketch Kohlenbach's \emph{higher-order Reverse Mathematics} as introduced in \cite{kohlenbach2}.  In contrast to `classical' RM, higher-order RM makes use of the much richer language of \emph{higher-order arithmetic}.  

\smallskip

As suggested by its name, {higher-order arithmetic} extends second-order arithmetic.  Indeed, while the latter is restricted to numbers and sets of numbers, higher-order arithmetic also has sets of sets of numbers, sets of sets of sets of numbers, et cetera.  
To formalise this idea, we introduce the collection of \emph{all finite types} $\mathbf{T}$, defined by the two clauses:
\begin{center}
(i) $0\in \mathbf{T}$   and   (ii)  If $\sigma, \tau\in \mathbf{T}$ then $( \sigma \di \tau) \in \mathbf{T}$,
\end{center}
where $0$ is the type of natural numbers, and $\sigma\di \tau$ is the type of mappings from objects of type $\sigma$ to objects of type $\tau$.
In this way, $1\equiv 0\di 0$ is the type of functions from numbers to numbers, and where  $n+1\equiv n\di 0$.  Viewing sets as given by characteristic functions, we note that $\Z_{2}$ only includes objects of type $0$ and $1$.    

\smallskip

The language of $\L_{\omega}$ consists of variables $x^{\rho}, y^{\rho}, z^{\rho},\dots$ of any finite type $\rho\in \mathbf{T}$.  Types may be omitted when they can be inferred from context.  
The constants of $\L_{\omega}$ includes the type $0$ objects $0, 1$ and $ <_{0}, +_{0}, \times_{0},=_{0}$  which are intended to have their usual meaning as operations on $\N$.
Equality at higher types is defined in terms of `$=_{0}$' as follows: for any objects $x^{\tau}, y^{\tau}$, we have
\be\label{aparth}
[x=_{\tau}y] \equiv (\forall z_{1}^{\tau_{1}}\dots z_{k}^{\tau_{k}})[xz_{1}\dots z_{k}=_{0}yz_{1}\dots z_{k}],
\ee
if the type $\tau$ is composed as $\tau\equiv(\tau_{1}\di \dots\di \tau_{k}\di 0)$.  
Furthermore, $\L_{\omega}$ also includes the \emph{recursor constant} $\mathbf{R}_{\sigma}$ for any $\sigma\in \mathbf{T}$, which allows for iteration on type $\sigma$-objects as in the special case \eqref{special}.  
Formulas and terms are defined as usual.  
\bdefi The base theory $\RCAo$ consists of the following axioms:
\begin{enumerate}
\item  Basic axioms expressing that $0, 1, <_{0}, +_{0}, \times_{0}$ form an ordered semi-ring with equality $=_{0}$.
\item Basic axioms defining the well-known $\Pi$ and $\Sigma$ combinators (aka $K$ and $S$ in \cite{avi2}), which allow for the definition of \emph{$\lambda$-abstraction}. 
\item The defining axiom of the recursor constant $\mathbf{R}_{0}$: For $m^{0}$ and $f^{1}$: 
\be\label{special}
\mathbf{R}_{0}(f, m, 0):= m \textup{ and } \mathbf{R}_{0}(f, m, n+1):= f( \mathbf{R}_{0}(f, m, n)).
\ee
\item The \emph{axiom of extensionality}: for all $\rho, \tau\in \mathbf{T}$, we have:
\be\label{EXT}\tag{$\textsf{\textup{E}}_{\rho, \tau}$}  
(\forall  x^{\rho},y^{\rho}, \varphi^{\rho\di \tau}) \big[x=_{\rho} y \di \varphi(x)=_{\tau}\varphi(y)   \big].
\ee 
\item The induction axiom for quantifier-free\footnote{To be absolutely clear, variables (of any finite type) are allowed in quantifier-free formulas of the language $\L_{\omega}$: only quantifiers are banned.} formulas of $\L_{\omega}$.
\item $\QFAC^{1,0}$: The quantifier-free axiom of choice as in Definition \ref{QFAC}.
\end{enumerate}
\edefi
\bdefi\label{QFAC} The axiom $\QFAC$ consists of the following for all $\sigma, \tau \in \textbf{T}$:
\be\tag{$\QFAC^{\sigma,\tau}$}
(\forall x^{\sigma})(\exists y^{\tau})A(x, y)\di (\exists Y^{\sigma\di \tau})(\forall x^{\sigma})A(x, Y(x)),
\ee
for any quantifier-free formula $A$ in the language of $\L_{\omega}$.
\edefi
As discussed in \cite{kohlenbach2}*{\S2}, $\RCAo$ and $\RCA_{0}$ prove the same sentences `up to language' as the latter is set-based and the former function-based.  Recursion as in \eqref{special} is called \emph{primitive recursion}; the class of functionals obtained from $\mathbf{R}_{\rho}$ for all $\rho \in \mathbf{T}$ is called \emph{G\"odel's system $T$} of all (higher-order) primitive recursive functionals.  

\smallskip

We use the usual notations for natural, rational, and real numbers, and the associated functions, as introduced in \cite{kohlenbach2}*{p.\ 288-289}.  
\begin{defi}[Real numbers and related notions in $\RCAo$]\label{keepintireal}\rm~
\begin{enumerate}
\item Natural numbers correspond to type zero objects, and we use `$n^{0}$' and `$n\in \N$' interchangeably.  Rational numbers are defined as signed quotients of natural numbers, and `$q\in \Q$' and `$<_{\Q}$' have their usual meaning.    
\item Real numbers are coded by fast-converging Cauchy sequences $q_{(\cdot)}:\N\di \Q$, i.e.\  such that $(\forall n^{0}, i^{0})(|q_{n}-q_{n+i})|<_{\Q} \frac{1}{2^{n}})$.  
We use Kohlenbach's `hat function' from \cite{kohlenbach2}*{p.\ 289} to guarantee that every $f^{1}$ defines a real number.  
\item We write `$x\in \R$' to express that $x^{1}:=(q^{1}_{(\cdot)})$ represents a real as in the previous item and write $[x](k):=q_{k}$ for the $k$-th approximation of $x$.    
\item Two reals $x, y$ represented by $q_{(\cdot)}$ and $r_{(\cdot)}$ are \emph{equal}, denoted $x=_{\R}y$, if $(\forall n^{0})(|q_{n}-r_{n}|\leq \frac{1}{2^{n-1}})$. Inequality `$<_{\R}$' is defined similarly.         
\item Functions $F:\R\di \R$ mapping reals to reals are represented by $\Phi^{1\di 1}$ mapping equal reals to equal reals, i.e. 
\be\tag{$\textsf{\textup{RE}}$}\label{RE}
(\forall x , y\in \R)(x=_{\R}y\di \Phi(x)=_{\R}\Phi(y)).
\ee
\item The relation `$x\leq_{\tau}y$' is defined as in \eqref{aparth} but with `$\leq_{0}$' instead of `$=_{0}$'.  Binary sequences are denoted `$f^{1}, g^{1}\leq_{1}1$', but also `$f,g\in C$' or `$f, g\in 2^{\N}$'.  
\item Sets of type $\rho$ objects $X^{\rho\di 0}, Y^{\rho\di 0}, \dots$ are given by their characteristic functions $f^{\rho\di 0}_{X}$, i.e.\ $(\forall x^{\rho})[x\in X\asa f_{X}(x)=_{0}1]$, where $f_{X}^{\rho\di 0}\leq_{\rho\di 0}1$.  
\end{enumerate}
\end{defi}
We sometimes omit the subscript `$\R$' if it is clear from context.  
Finally, we introduce some notation to handle finite sequences nicely.  
\begin{nota}[Finite sequences]\label{skim}\rm
We assume a dedicated type for `finite sequences of objects of type $\rho$', namely $\rho^{*}$.  Since the usual coding of pairs of numbers goes through in $\RCAo$, we shall not always distinguish between $0$ and $0^{*}$. 
Similarly, we do not always distinguish between `$s^{\rho}$' and `$\langle s^{\rho}\rangle$', where the former is `the object $s$ of type $\rho$', and the latter is `the sequence of type $\rho^{*}$ with only element $s^{\rho}$'.  The empty sequence for the type $\rho^{*}$ is denoted by `$\langle \rangle_{\rho}$', usually with the typing omitted.  

\smallskip

Furthermore, we denote by `$|s|=n$' the length of the finite sequence $s^{\rho^{*}}=\langle s_{0}^{\rho},s_{1}^{\rho},\dots,s_{n-1}^{\rho}\rangle$, where $|\langle\rangle|=0$, i.e.\ the empty sequence has length zero.
For sequences $s^{\rho^{*}}, t^{\rho^{*}}$, we denote by `$s*t$' the concatenation of $s$ and $t$, i.e.\ $(s*t)(i)=s(i)$ for $i<|s|$ and $(s*t)(j)=t(|s|-j)$ for $|s|\leq j< |s|+|t|$. For a sequence $s^{\rho^{*}}$, we define $\overline{s}N:=\langle s(0), s(1), \dots,  s(N-1)\rangle $ for $N^{0}<|s|$.  
For a sequence $\alpha^{0\di \rho}$, we also write $\overline{\alpha}N=\langle \alpha(0), \alpha(1),\dots, \alpha(N-1)\rangle$ for \emph{any} $N^{0}$.  By way of shorthand, 
$(\forall q^{\rho}\in Q^{\rho^{*}})A(q)$ abbreviates $(\forall i^{0}<|Q|)A(Q(i))$, which is (equivalent to) quantifier-free if $A$ is.   
\end{nota}

\subsection{Higher-order computability theory}\label{HCT}
As noted above, some of our main results will be proved using techniques from computability theory.
Thus, we first make our notion of `computability' precise as follows.  
\begin{enumerate}
\item[(I)] We adopt $\ZFC$, i.e.\ Zermelo-Fraenkel set theory with the Axiom of Choice, as the official metatheory for all results, unless explicitly stated otherwise.
\item[(II)] We adopt Kleene's notion of \emph{higher-order computation} as given by his nine clauses S1-S9 (See \cites{longmann, Sacks.high}) as our official notion of `computable'.
\end{enumerate}
%Some of our results require familiarity with computability theory as in the second item.  
For the rest of this section, we introduce some existing axioms which will be used below.
These functionals constitute the counterparts of $\Z_{2}$, and some of the Big Five, in higher-order RM by Remark \ref{fookie}.
First of all, $\ACA_{0}$ is readily derived from:
\begin{align}\label{mu}\tag{$\mu^{2}$}
(\exists \mu^{2})(\forall f^{1})\big[ (\exists n)(f(n)=0) \di [f(\mu(f))=0&\wedge (\forall i<\mu(f))f(i)\ne 0 ]\\
& \wedge [ (\forall n)(f(n)\ne0)\di   \mu(f)=0]    \big], \notag
\end{align}
and $\ACA_{0}^{\omega}\equiv\RCAo+(\mu^{2})$ proves the same $\Pi_{2}^{1}$-sentences as $\ACA_{0}$ by \cite{yamayamaharehare}*{Theorem~2.2}.   The (unique) functional $\mu^{2}$ in $(\mu^{2})$ is also called \emph{Feferman's $\mu$} (\cite{avi2}), 
and is clearly \emph{discontinuous} at $f=_{1}11\dots$; in fact, $(\mu^{2})$ is equivalent to the existence of $F:\R\di\R$ such that $F(x)=1$ if $x>_{\R}0$, and $0$ otherwise (\cite{kohlenbach2}*{\S3}), and to 
\be\label{muk}\tag{$\exists^{2}$}
(\exists \varphi^{2}\leq_{2}1)(\forall f^{1})\big[(\exists n)(f(n)=0) \asa \varphi(f)=0    \big]. 
\ee
\noindent
Secondly, $\FIVE$ is readily derived from the following sentence:
\be\tag{$S^{2}$}
(\exists S^{2}\leq_{2}1)(\forall f^{1})\big[  (\exists g^{1})(\forall x^{0})(f(\overline{g}n)=0)\asa S(f)=0  \big], 
\ee
and $\FIVE^{\omega}\equiv \RCAo+(S^{2})$ proves the same $\Pi_{3}^{1}$-sentences as $\FIVE$ by \cite{yamayamaharehare}*{Theorem 2.2}.   The (unique) functional $S^{2}$ in $(S^{2})$ is also called \emph{the Suslin functional} (\cite{kohlenbach2}).
By definition, the Suslin functional $S^{2}$ can decide whether a $\Sigma_{1}^{1}$-formula (as in the left-hand side of $(S^{2})$) is true or false.   We similarly define the functional $S_{k}^{2}$ which decides the truth or falsity of $\Sigma_{k}^{1}$-formulas; we also define 
the system $\SIXK$ as $\RCAo+(S_{k}^{2})$, where  $(S_{k}^{2})$ expresses that $S_{k}^{2}$ exists.  Note that we allow formulas with \emph{function} parameters, but \textbf{not} with \emph{functional} parameters.
In fact, Gandy's \emph{Superjump} (\cite{supergandy}) constitutes a way of extending $\FIVE^{\omega}$ to parameters of type two.

\smallskip

\noindent
Thirdly, full second-order arithmetic $\Z_{2}$ is readily derived from $\cup_{k}\SIXK$, or from:
\be\tag{$\exists^{3}$}
(\exists E^{3}\leq_{3}1)(\forall Y^{2})\big[  (\exists f^{1})Y(f)=0\asa E(Y)=0  \big], 
\ee
and we therefore define $\Z_{2}^{\Omega}\equiv \RCAo+(\exists^{3})$ and $\Z_{2}^\omega\equiv \cup_{k}\SIXK$, which are conservative over $\Z_{2}$ by \cite{hunterphd}*{Cor.\ 2.6}, but see Remark \ref{fookie}.   The functional from $(\exists^{3})$ is also called `$\exists^{3}$', and we use the same convention for other functionals.  

\smallskip

Finally, recall that the Heine-Borel theorem (aka \emph{Cousin's lemma}) states the existence of a finite sub-cover for an open cover of a compact space. 
Now, a functional $\Psi:\R\di \R^{+}$ gives rise to the \emph{canonical} cover $\cup_{x\in I} I_{x}^{\Psi}$ for $I\equiv [0,1]$, where $I_{x}^{\Psi}$ is the open interval $(x-\Psi(x), x+\Psi(x))$.  
Hence, the uncountable cover $\cup_{x\in I} I_{x}^{\Psi}$ has a finite sub-cover by the Heine-Borel theorem; in symbols:
\be\tag{$\HBU$}
(\forall \Psi:\R\di \R^{+})(\exists \langle y_{1}, \dots, y_{k}\rangle)\underline{(\forall x\in I)}(\exists i\leq k)(x\in I_{y_{i}}^{\Psi}).
\ee
By Theorem \ref{mooi} below, $\Z_{2}^{\Omega}$ proves $\HBU$, but $\SIXK+\QFAC^{0,1}$ cannot (for any $k$).  %Hence, the Heine-Borel theorem for uncountable covers as in $\HBU$ falls \emph{far} outside of the Big Five of RM, as noted at the end of Section \ref{RM}.  
As studied in \cite{dagsamIII}*{\S3}, many basic properties of the \emph{gauge integral} are equivalent to $\HBU$.  
By Remark \ref{kloti}, we may drop the requirement that $\Psi$ in $\HBU$ needs to be extensional on the reals, i.e.\ $\Psi$ does not have to satisfy \eqref{RE} from Definition \ref{keepintireal}.

\smallskip

Furthermore, since Cantor space (denoted $C$ or $2^{\N}$) is homeomorphic to a closed subset of $[0,1]$, the former inherits the same property.  
In particular, for any $G^{2}$, the corresponding `canonical cover' of $2^{\N}$ is $\cup_{f\in 2^{\N}}[\overline{f}G(f)]$ where $[\sigma^{0^{*}}]$ is the set of all binary extensions of $\sigma$.  By compactness, there is a finite sequence $\langle f_0 , \ldots , f_n\rangle$ such that the set of $\cup_{i\leq n}[\bar f_{i} G(f_i)]$ still covers $2^{\N}$.  By \cite{dagsamIII}*{Theorem 3.3}, $\HBU$ is equivalent to the same compactness property for $C$, as follows:
\be\tag{$\HBU_{\c}$}
(\forall G^{2})(\exists \langle f_{1}, \dots, f_{k} \rangle )\underline{(\forall f^{1}\leq_{1}1)}(\exists i\leq k)(f\in [\overline{f_{i}}G(f_{i})]).
\ee
We now introduce the specification $\SCF(\Theta)$ for a (non-unique) functional $\Theta$ which computes a finite sequence as in $\HBU_{\c}$.  
We refer to such a functional $\Theta$ as a \emph{realiser} for the compactness of Cantor space, and simplify its type to `$3$'.  % to improve readability.
%\bdefi\label{dodier}
%The formula $\SCF(\Theta)$ is as follows for $\Theta^{2\di 1^{*}}$:
\be\tag{$\SCF(\Theta)$}
(\forall G^{2})(\forall f^{1}\leq_{1}1)(\exists g\in \Theta(G))(f\in [\overline{g}G(g)]).
\ee
%\edefi
Clearly, there is no unique such $\Theta$ (just add more binary sequences to $\Theta(G)$); nonetheless, 
we have in the past referred to any $\Theta$ satisfying $\SCF(\Theta)$ as `the' \emph{special fan functional} $\Theta$, and we will continue this abuse of language.  
%We shall however repeatedly point out the non-unique nature of the special fan functional $\Theta$ in the following.   
%While $\Theta$ may appear exotic at first, it provides the only method we can think of for computing gauge integrals \emph{in general}, as discussed in Remark~\ref{engauged}.   
%
%\smallskip
As to its provenance, $\Theta$ was introduced as part of the study of the \emph{Gandy-Hyland functional} in \cite{samGH}*{\S2} via a slightly different definition.  
These definitions are identical up to a term of G\"odel's $T$ of low complexity by \cite{dagsamII}*{Theorem 2.6}.  As shown in \cite{dagsamIII}*{\S3}, one readily obtains a realiser $\Theta$ from $\HBU$ if the latter is given; in fact, it is straightforward to establish $\HBU\asa (\exists \Theta)\SCF(\Theta)$ over $\ACA_{0}+\QFAC$. 

\smallskip

In conclusion, we have sketched the `received view' of RM in Section \ref{RM}, including the elegant `Big Five' picture and linear order \eqref{linord}.  
As noted in Section~\ref{heineke}, the framework of RM is \emph{second-order arithmetic}, i.e.\ higher-order objects are represented via codes.    
However, the higher-order language described in Section \ref{KOH} allows us to study e.g.\ the Heine-Borel theorem for \emph{uncountable} covers as in $\HBU$; the latter does not fit (at all) in the elegant `Big Five' picture as $\HBU$ can only be proved from \emph{full second-order arithmetic} (as given by $\exists^{3}$).    
Numerous natural higher-order theorems with similarly `deviant' behaviour are studied in \cite{dagsamIII}, and a number of new such results are obtained in this paper.    
We leave it to the reader to decide the implications of all this for the `Big Five picture' of RM.  
%Stillwell's book (p. 59) mentions HBU! and his proof of CONT $\di$ UCONT uses HBU / proves the uniform version!
%
\section{Pincherle's theorem in computability theory}\label{PRS}%\marginpar{\footnotesize{This section needs an almost complete rewriting, and it is pointless to mark any change. For comparison, I copied the original text of the section at the very end of the file.}}
We answer the first two questions from Section \ref{intro}.  % namely how hard is it to compute the upper bound in Pincherle's theorem in terms of (some of) the data?
In Section \ref{PRR}, we show that \emph{Pincherle realisers} (PR hereafter) from Section \ref{sum}, cannot be computed by any type two functional.  
We also show that any PR (uniformly) give rise to a non-Borel continuous functional.  The latter result follows from the \emph{extension theorem} (Theorem \ref{ET}). % which expresses that Pincherle realisers compute total extensions of partial functionals computable in Feferman's $\mu$. 
%there is an arithmetical definition of a functional of type 2, using a parameter for a PR,  that is not Borel continuous for any PR $ M$.
%In Section \ref{PRR2}, we obtain similar results for realisers of the \emph{original} Pincherle theorem $\PIT_{o}$.  
In Section \ref{PRR2}, we discuss similar questions for $\PIT_{o}$.
%The results about this concept are not just `more of the same': $\PIT_{o}$ follows from $(\exists^{2})$ by Theorem \ref{ofinterest}, i.e.\ the behaviour of $\PIT_{o}$ in RM diverges \emph{completely} from its computability-theoretic behaviour.  

  % which even surprised the authors.  
\subsection{Pincherle realisers}\label{PRR}
%Recall that a {Pincherle realiser} (PR for short) is a functional $M$ satisfying the specification $\PR(M)$ from Section \ref{sum}. 
In this section we show that any PR has both considerable computational strength and hardness, as captured by the following theorems.  
\begin{theorem}\label{poilo}
There is no PR that is computable in a functional of type two.
\end{theorem}
\begin{theorem}\label{thm8.3}
There is an arithmetical  functional $F$ of type $(1\times1)\rightarrow  0$ such that for any PR $M$ we have that
$G(f) = M(\lambda g.F(f,g))$ is not Borel continuous.
\end{theorem}
%\begin{enumerate}
%\item[(i)] No type two functional computes a PR (Theorem \ref{poilo}).
%\item[(ii)] Any PR give rise to a non-Borel continuous functional (Theorem \ref{thm8.3}).  
%\end{enumerate}
%As to the second item, we provide an arithmetical definition of a functional of type two, using a parameter for a PR,  that is not Borel continuous for any PR $ M$. 
%We first establish (i) as follows.  
\begin{theorem}[Extension Theorem]\label{ET} 
Let $M$ be a PR and let $e_0$ be a Kleene index for a partially computable functional $\Phi(F) = \{e_0\}(F,\mu)$. 
Then, uniformly in $M$, $\Phi$ has a total extension \(depending on $M$\) that is primitive recursive in $M,\mu$.
%OLDThen $\Phi$ has a total extension $\Psi$ that is primitive recursive in $M$ and $\mu$, uniformly in $M$.
\end{theorem}
Note that Theorem \ref{ET} is the `higher-order' version of a known extension theorem.  Indeed, by Corollary \ref{ofmoreinterest}, $\PIT_{o}$ is equivalent to $\WKL$, and the latter implies: 
\begin{center}
\emph{If a partially computable $f:\N\di \N$ is bounded by a total computable function, then $f$ has a total extension.}
\end{center}
By the low basis theorem, the extension may be chosen to be of low degree.
\newline
%In a sense, producing such objects of low degree is the computational power of one way of phrasing $\WKL$, if a binary tree is infinite, then it has a branch. 
Now, PRs are realisers for \emph{uniform} Pincherle's theorem (and for uniform $\WKL$ by Remark~\ref{flurki}), and Theorem \ref{ET} is the associated `higher-order' extension theorem,  where the concept of computability is relativised to Feferman's $\mu$ using S1-S9. 
By Corollary \ref{claptrap}, PRs also yield a higher-order version of the well-known separation theorem for $\Sigma_{1}^{0}$-sets that follows from $\WKL$ (See e.g.\ \cite{simpson2}*{I.11.7}).  The analogy with the low basis theorem will be that we can separate pairwise disjoint sets of type 2 functionals, semi-computable in $\mu$, with a set relative to  which not all semi-computable sets  are computable, so separation does not imply comprehension for sets semi-computable in $\mu$. It would be interesting to learn if some PRs can provide us with an analogue of sets of low degree. 
%The existence of a PR expresses the contra-positive formulation in a uniform way.  In this section we will prove that the existence of a PR  $M$ will generalise  this separation theorem to sets of functionals of type 2, when
%Theorem \ref{thm8.3} naturally follows the \emph{extension theorem}, and we now discuss the conceptual nature of the latter. 

\smallskip

We first prove Theorem \ref{poilo}.  The proof is similar to the proof of the fact that no special fan functional $\Theta$ is computable in any type two functional (See \cite{dagsam}*{\S3}).
\begin{proof}
Suppose that $M$ is a PR and that $M$ is computable in the functional $H$ of type two. Without loss of generality, we may assume that $\exists^2$ is computable in $H$, so the machinery of Gandy selection (\cite{longmann}*{p.\ 210}) is at our disposal. We define the (partial) functional $G:C \rightarrow \N \cup \{\bot\}$ by $G(f) = e+1$ where $e$ is the index of $f$ as a function computable in $H$ obtained by application of Gandy selection.  We put $G(f) =\bot$ if $f$ is not computable in $H$.

\smallskip

Now let $\hat G$ be any total extension of $G$. If we evaluate $M(\hat G) = a$ following the assumed algorithm for $M$ from $H$, we see that we actually can replace $\hat G$ with $G$ in the full computation tree (using that $G$ is partially  computable in $H$, so we will only call upon $\hat G(f)$ for $H$-computable $f$). Thus $M(\hat G)$ is independent of the choice of $\hat G$. On the other hand, we have that for any $N$, the set of $g$ where the bounding condition $\LOC(F,G)$ forces $F(g)$ to be bounded by $N$ is a small, clopen set, and if we let $\hat G(g) > N$ for all $g$ not computable in $H$, we obtain a contradiction. %This ends the proof.
\end{proof}
%\subsubsection{An extension theorem}
%It is well known that two disjoint $\Sigma^0_1$ subsets of $\N$ can be separated by a set of low degree, and that it is the fact that every computable infinite binary tree has a branch of low degree that is the source of this separation theorem. Another way of putting this is that whenever a partially computable function $f$ is bounded by a total computable $g$, then $f$ has a total extension $h$ of low degree. In a sense, producing such objects of low degree is the computational power of one way of phrasing $\WKL$, if a binary tree is infinite, then it has a branch. The existence of a PR expresses the contrapositive formulation in a uniform way.  In this section we will prove that the existence of a PR  $M$ will generalise  this separation theorem to sets of functionals of type 2, when the concept of computability is relativised to Feferman's $\mu$ using Kleene's S1 - S9. 
We now prove a number of theorems, culminating in a proof of Theorem \ref{ET}.  
We assume $M$ to be a PR for the rest of this section. 
\begin{theorem}\label{thm9.1}
For each Kleene-index $e_0$ and all numbers $a_0,n$ there are arithmetical, uniformly in $e_0,a_0,n$, functionals $F \mapsto F_{e_0,a_0,n}$ of type $2 \rightarrow 2$ such that if $\{e_0\}(F,\mu)\!\!\downarrow$, 
we can, independently of the choice of $M$, find the value  $a$ of the computation from $\lambda (a_0,n).M(F_{e_0,a_0,n})$ in an arithmetical manner.
\end{theorem}
\begin{proof}
We let $M$, $F$, $e_0$, $a_0$ and $n$ be fixed throughout.
%In a nutshell, we define a specific $\Pi_{1}^{0}$-set called $\textsf{PRE}$ based on the computation $\{e_0\}(F,\mu)\!\!\downarrow$.  We then use $\PRE$ to define $F_{e_0,a_0,n}$ as required by the theorem.
We first need some notation. 

\smallskip

Let $R$ be a preordering of a domain $D\subseteq \N$. For $x \in D$, we denote
\begin{itemize}
\item $[x]^R =  \{y \in D \mid (y,x) \in R\}$
\item $[x]_R = \{y \in D \mid (y,x) \in R \wedge \neg((x,y) \in R)\}$
\item $R^x$ is $R$ restricted to $[x]^R$
\item $R_x$ is $R$ restricted to  $[x]_R$
\end{itemize}

\smallskip

  Let $f \in C$ and define $D_f := \{x  \mid f(\langle x,x \rangle) = 1\}$ and $R_f := \{(x,y) \mid f(\langle x,y \rangle) = 1\}$, where $x,y \in \N$.
Let $\PRE$ be the set of $f \in C$ such that $R_f$ is a preordering of $D_f$. Then $\PRE$ is a $\Pi^0_1$-set, and for each $f \not \in {\rm \PRE}$, we can find an integer $k$ such that $[\bar fk] \cap {\rm \PRE} = \emptyset.$

\smallskip

Let $\Gamma_F$ be the monotone inductive definition of $D_F = \{\langle e,\vec a , b\rangle \mid \{e\}(F, \mu , \vec a) = b\}.$
Since each valid computation $\{e\}(F,\mu , \vec a) = b$ has an ordinal rank $\| \langle e,\vec a , b\rangle \|_F < \aleph_1$, let $R_F$ be the pre-well-ordering on $D_F$ induced by $\| \cdot \|_F$. 
Then $R_F$ is the least fixed point of an, uniformly in $F$, arithmetical and monotone inductive definition $\Delta_F$ such that $\Delta_F^{\alpha + 1}$ is always an end extension of $\Delta_F^\alpha$, where we write $\Delta_F^\alpha$ for $\Delta_F^\alpha(\emptyset)$.

Now, let $R$ be any preordering of the domain $D \subseteq \N$. We call $x \in D$ an \emph{$F$-point} if $R^x = \Delta_F(R_x)$
We let $D[F]$ be the maximal $R$-initial segment consisting of $F$-points, and we let $R[F]$ be $R$ restricted to $D[F]$.



\smallskip

{\em Claim 1}:  If $R[F]$ does not contain $R_F$ as an initial segment, then $R[F]$ is an initial segment of $R_F$.
\vspace{2mm}
\newline
{\em Proof of Claim 1}.  Let $\alpha$ be the least ordinal such that $\Delta^\alpha_F$ is not an initial segment of $R[F]$. Then $\bigcup_{\beta < \alpha}\Delta_F^\beta$ is an initial segment of $R[F]$. If this is all of $R[F]$, we are through, since then $R[F]$ is an initial segment of $R_F$. If not, there is some $x \in D[F]$ such that $\bigcup_{\beta < \alpha}\Delta^\beta_F$ is an initial segment of $R_x[F]$.
But since $x$ is an $F$-point and $\Delta_F$ is monotone we have that $R^x[F] = \Delta_F(R_x[F])$ and that $\Delta^\alpha_F$ is an initial segment of $R^x[F]$, contradicting the choice of $\alpha$. Claim 1 now follows.
\smallskip

\noindent For now, assume that $f \in {\rm \PRE}$.


\smallskip

{\em Claim 2}:  If $R_f[F]$ is not a fixed point of $\Delta_F$, there is $k\in \N$, $\mu$-computable from $F, f$,
%identifiable from $R_f[F]$ and $f$ in an arithmetical manner, 
such that whenever $g \in {\rm \PRE}$ such that $\Delta_F(R_f[F])$ is an initial segment of $R_g[F]$ we have that $g(k) \neq f(k)$.
\vspace{2mm}
\newline
{\em Proof of Claim 2}.
If there is a pair $(y,x) \in \Delta_F(R_f[F])$ such that $f(\langle y,x \rangle) = 0$, we can just let $k = \langle y,x\rangle$ for one such pair, chosen by numerical search. 
Now assume $f(\langle y,x \rangle) = 1$ when $(y,x) \in \Delta_F(R_f[F])$. Since we for all $x \in D_f[F]$ have that $(R_f)_x \subseteq (R_f)^x = \Delta_F((R_f)_x)$ and $\Delta_F$ is monotone, we must have that 
$R_f[F] \subseteq \Delta_F(R_f[F])$. Further, since $R_f[F]$ is not a fixed point of $\Delta_F$, we must have some $x$ such that $(x,x) \in \Delta_F(R_f[F]) \setminus R_f[F]$.
Since this $x$ is not an $F$-point, and since $$f(\langle x,y \rangle) = f(\langle y , x \rangle) = 1$$ whenever $(x,y) \in \Delta_F(R_f[F])$ and $(y,x) \in \Delta_F(R_f[F])$, there must be a $y$ such that $f(\langle x,y \rangle) = f(\langle y,x \rangle) = 1$, but $(x,y) \not \in \Delta_F(R_f[F])$ or $(y,x) \not \in \Delta_F(R_f[F])$. We can find such a pair $k = \langle x,y\rangle$ or $k = \langle y,x \rangle$ by effective search. Claim 2 now follows.

\smallskip

We now define $F_{e_0,a_0,n}(f)$, where $f \in C$ is not necessarily in $\PRE$ anymore.
\bdefi\label{godallemagtig}
We define $F_{e_0,a_0,n}(f)$ by cases, assuming for each case that the previous cases fail:
\begin{enumerate}
\item For $f \not \in {\rm \PRE}$, let $F_{e_0,a_0,n}(f) = k$ for the least $k$ such that $[\bar fk] \cap {\rm \PRE}  = \emptyset$
%
\item There is an $a \in \N$ such that 
\begin{itemize}
\item[(2.i)] $\langle e_0,a\rangle$ is in the domain of $R_f[F]$
\item[(2.ii)] For no $b \in \N$ with $b \neq a$ do we have that $(\langle e_0,b \rangle , \langle e_0,a\rangle) \in R_f[F]$.
\end{itemize}
We then let $F_{e_0,a_0,n}(f) = 0$ if $a \neq a_0$ and $n$ if $a = a_0$.
%
\item $R_f[F]$ is a fixed point of $\Delta_F$. Then let $F_{e_0,a_0,n}(f) = 0$.
%
\item $R_f[F]$ is not a fixed point of $\Delta_F$. Then let $F_{e_0,a_0,n}(f) = k+1$, where $k$ is the number identified in Claim 2.
\end{enumerate}
\edefi
\noindent
We now prove the theorem via establishing the following final claim. 

\smallskip

{\em Claim 3}: If $\{e_0\}(F,\mu)\!\!\downarrow$, the following algorithm provides the result:
\begin{center} 
$\{e_0\}(F,\mu)$ is the unique $a_0$ for which $\{M(F_{e_0,a_0,n}) \mid n \in \N\}$ is infinite.
\end{center}
This algorithm is uniformly arithmetical in $M$, by definition. 
\vspace{2mm}
\newline
{\em Proof of Claim 3}.
Assume that $\{e_0\}(F,\mu) = a$. Then $\langle e_0,a\rangle$ is in the well founded part of $R_F$.
If $a \neq a_0$, we see from the definition of $F_{e_0,a_0,n}$ that this functional is independent of $n$, so $M(F_{e_0,a_0,n})$ has a fixed value independent of $n$. 
 If $a = a_0$ we claim that $M(F_0,a_0,n) \geq n$, and the conclusion follows:
%, and let $\Theta(F_{e_0}) = \{f_1 , \ldots , f_n\}.$ 
Let $g \in C$ be such that $R_F = R_g[F]$, and let $f$ be arbitrary such that $g \in [\bar fF_{e_0,a_0,n}(f)]$.
If $f \not \in {\rm \PRE}$, we clearly do not have that $g \in [\bar fF_{e_0,a_0,n}(f)]$, so the first item from Definition \ref{godallemagtig} does not apply.
If $f \in {\rm \PRE}$, but $\langle e_0 , a_0 \rangle$ is not in the domain of $R_{f}[F]$, then by Claim 1, $R_{f}[F]$ is a proper initial segment of $R_F$, and using Claim 2 we have chosen $F_{e_0,a_0,n}(f) = k+1$ in such a way that $g(k) \neq f(k)$.
Then, by our assumption on $f$, we must have that $\langle e_0,a_0\rangle$ is in the domain of $R_{f}[F]$, and since this appearance will be in the well-founded part, there will be no competing values $b$ at the same or lower level.  Then we set the value of $F_{e_0,a_0,n}(f)$ to $n$.
%This ends the proof of Claim 3, and of Theorem \ref{thm9.1}.
\end{proof}
As an immediate consequence, we obtain a proof of Theorem \ref{ET}.
\begin{proof}
Let $F_{e_0,a_0,n}$ be as in the proof of Theorem \ref{thm9.1}. Define $\Psi(F) = a_0$ if $a_0$ is unique such that $M(F_{e_0,a_0,n}) \geq n$ for all $n$, and define $\Psi(F) = 0$ if there is no such unique $a_0$.
\end{proof}

%
%\subsection{Adjusting to $M_u$}\label{9.2}
%In this subsection, we will modify the proof of Theorem \ref{thm9.1} to a proof giving us
%
%\begin{theorem}\label{thm9.2}
%For each index $e_0$, each integer $a_0$ and each  integer $n$ there is an arithmetical functional $F \mapsto F_{e_0,a_0,n}$ of type $2 \rightarrow 2$, uniformly in $e_0,a_0,n$, such that if $\{e_0\}(F,\mu)\!\!\downarrow$, we can, independently of the choice of $M_u$, find the value  $a$ of the computation from $\{M_u(F_{e_0,a_0,n})\}_{(a_0,n) \in \N^2}$ in an arithmetical manner.\end{theorem}
%{\em Proof}
%\newline
%We define $F_{e_0,a_0,n}(f)$ as we defined $F_{e_0}(f)$ in the proof of Theorem \ref{thm9.1} with one exception: If we are in Case 2, and identify an $a$ as in that case, we let $F_{e_0,a_0,n}(f)$ be 0 if $a \neq a_0$ and $n$ if $a = a_0$.
%\newline
%If $\{e_0\}(F,\mu) = a$, and $a \neq a_0$, we see that $F_{e_0,a_0,n}$, and thus $M_u(F_{e_0,a_0,n})$, will be independent of $n$. On the other hand, if $\{e_0\}(F,\mu) = a_0$, we see that $M_u(F_{e_0,a_0,n}) \geq n$ (due to the way $M_u$ and $\Theta$ are  related and the proof of Theorem \ref{thm9.1}). Thus we may use the following ``algorithm" for ending the proof:
%\begin{itemize}
%\item[] If there is a unique $a_0$ such that $\{M_u(F_{e_0,a_0,n})\}_{n \in \N}$ is unbounded, then output $a_0$, otherwise output 0.
%\end{itemize}
Finally, we list some corollaries to the theorem. 
\begin{corollary}\label{cor9.3}
Let $\Phi^{3}$ be partial and Kleene-computable in $\mu$.
Then for any PR $M$ there is a total extension of $\Phi$ that is primitive recursive in $M$ and $\mu$.
\end{corollary}
This is almost a rephrasing of Theorem \ref{ET}, modulo some coding of mixed types.
\begin{corollary}\label{claptrap}
Let $X$ and $Y$ be disjoint sets of functionals of type 2, both semicomputable in $\mu$.
Then, for each PR $M$, there is a set $Z$ primitive recursive in $M$ and $\mu$, that separates $X$ and $Y$.
\end{corollary}
\begin{proof}
Since we use $\mu$ as a parameter, we have Gandy selection in a uniform way, so there will be a partial function computable relative to $\mu$ that takes the value 0 on $X$ and 1 on $Y$. Then apply Corollary \ref{cor9.3}.
\end{proof}
As a special case, we obtain the proof of Theorem \ref{thm8.3}, as follows.
\begin{proof}
Let $X = \{(e,f) \mid \{e\}(e,f,\mu) = 0\}$ and $Y = \{(e,f) \mid \{e\}(e,f,\mu) = 1\}$.
\newline $X$ and $Y$ are Borel-inseparable disjoint $\Pi^1_1$-sets, but can be separated using one parameterised application of $M$.
\end{proof}
As another application of Corollary \ref{cor9.3} we see that the partial enumeration of all hyperarithmetical functions, which is partially computable in $\mu$, can be extended to a total enumeration primitive recursive in $M$ and $\mu$ for all Pincherle realisers $M$.  We leave further applications to the imagination of the reader.

%Next, we show that if $M$ is a PR, then like the special fan functional $\Theta$, $M$ breaks out of the Borel-continuous functionals. 
%We will do so by constructing an arithmetical sequence $G_e$ of functions $G_e:C^2\rightarrow \N$ such that, independent of the choice of $M$, the function $(f,e) \mapsto M(\lambda h. G_e(f,h))$ is not Borel continuous. 

%\smallskip

%We first go through the construction of this sequence $G_e$ (Definition \ref{corkiiii}), prove the essential properties (Theorem \ref{thm8.3}), and then state the theorem (Corollary \ref{gahoe}) in due time. We follow the method from the proof of the corresponding theorem for the special fan functional $\Theta$ (\cite{dagsamIII}*{\S4}), with the necessary deviations.
%Throughout, we assume $e = \langle e_1,e_2,e_3\rangle$ and also introduce the following notations.
%\begin{nota}\rm
%We assume $f, h\in C$ and let $g$ be the partial function $\{e_1\}^f$. We write `$a \prec_g b$' if $g$ is total and $g(\langle a,b \rangle) = 0$. The domain of this relation is `$D_{g}$'. If $\prec_g$ is a total, irreflexible ordering of $D_g$ and 
 %$d  \in D_g$, define $h^d(b) := h(\langle d,b\rangle)$ and 
 %\[
%h_d(\langle a,b\rangle) := \left \{ \begin{array}{ccc} h(\langle a,b\rangle)&{\rm if}&a \prec_g d \\ 0 & {\rm if}& d \preceq_g a \end{array} \right . 
%\]
%These functions are viewed as characteristic functions of sets.   Under the assumptions above, we say that $h$ is an \emph{$f,g$-chain} if 
%\be\label{XXX}
%\text{$h^d$ is the Turing-jump of $\langle f,h_d\rangle$ for each $d \in D_g$.}
%\ee
%\end{nota}
%\bdefi[Definition of $G_{e}$]\label{corkiiii}
%For any $\phi\in \N^{\N}$, we let $[d]^\phi$ be the $d$-th primitive recursive function relative to $\phi$ via some standard enumeration. 
%\begin{enumerate}
%$\item  If $g$ is not total, if $\prec_g$ is not a total ordering of its domain or if $e_2 \not \in D_g$, we let $G_e(f,h) = 0$ for all  $h$. 
%\item Assume that  $g$ is total, that  $\prec_g$ is a total ordering of its domain $D_g$ and that $e_2 \in D_g$.   Let $E$ be the maximal initial segment of $D_g$ where \eqref{XXX} holds, possibly all of $D_g$.
 %We distinguish between three cases:
%\begin{enumerate}
%\item $e_2 \not \in E$, and there is a  $\prec_g$-minimal $d \prec_g e_2$ such that \eqref{XXX} fails. Let $c$ be minimal such that $h^d$ and the jump of $\langle f,h_d\rangle$ differ at $c$, and let $G_e(f,h) = \langle d,c\rangle + 1$.
%\item $e_2 \not \in E$, but there is no such $\prec_g$-minimal $d$. We let $G_e(f,h) = 0$.
%\item  $e_2 \in E$.  We let $G_e(f,h) = [e_3]^{h^{e_2}}(e) + 1$. (This case covers when $h$ is an $f,g$-chain.)
%\end{enumerate}
%\end{enumerate}
%\end{defi}
%\noindent
%We now show that $(f,e) \mapsto M(\lambda h. G_e(f,h))$ is not Borel continuous as follows.  
%\begin{theorem}\label{thm8.3} 
%Let $M$ be a PR and let $G_{e}$ be as above.
%For each $f \in C$ and $\phi \in C$ that is hyperarithmetical in $f$ there is a number $e$ such that 
%$M(\lambda h. G_e(f,h)) > \phi(e).$
%\end{theorem}
%
%\begin{proof}
%We use a relativised variant of a well-known theorem from the theory of hyperarithmetical sets. Relative to $f$, there will be a computable total ordering $\prec$ of a subset $D$ of $\N$ with the following properties.
%\begin{enumerate}
%\item[(i)] The ordering $\prec$ is not a well-ordering.
%\item[(ii)] There are no infinite descending sequences in $\prec$ hyperarithmetical  in $f$.
%\item[(iii)] There is a jump-chain modulo $f$ defined on $D$.
%\end{enumerate}
%Throughout, we let $e_1$ be an $f$-index  for $g$ such that $\prec_g $ is $ \prec$. Let $H$ be the %non-empty set of $f,g$-chains. If $h_1,h_2 \in H$ and $d$ is in the well-ordered initial segment of $%\prec$, then $h_1^d = h_2^d$ and $(h_1)_d = (h_2)_d$.
%As a general property, if $\phi$ is hyperarithmetical in $f$, there is some $e_2$ in the well founded part %of $\prec$ and some index $e_3$ such that $\phi = [e_3]^{h^{e_2}}.$ Putting $e: = \langle %e_1,e_2,e_3\rangle$, it suffices to prove $M(\lambda h.G_e(f,h)) > \phi(e)$.
%\smallskip
%In light of the previous, fix this $e$ and let $g$, $\prec_g$ etc.\ be as above. Let $G(h) = G_e(f,h)$. %Independent of $h$, we will be in case (2.a) or (2.c) of Definition \ref{corkiiii}. 
%If $G(h) = \langle d,c \rangle + 1$ due to case (2.a), $d$  will be in the well-ordered part of $\prec_g$, %since $e_2$ is there.  
%Then we have  defined $G(h)$ so large that the neighbourhood $C_{\bar h(G(h))}$ is disjoint from the %set $H$. 
%\smallskip
%On the other hand, if $G(h) = [e_3]^{h^{e_2}}(e) + 1$ due to case (2.c), then $G(h) = \phi(e)+1$. 
%Thus, for any neighbourhood in the covering given by $G$ that intersects $H$, the bound on that %neighbourhood, also  given by $G$, will be $\phi(e) + 1$, and consequently, we have that $M(G) = %M(\lambda h G_e(f,h)) > \phi(e)$ as claimed.
%\end{proof}
%\begin{corollary}\label{gahoe}
%Let $M$ be a PR, and let $G$ be the arithmetical functional constructed above.
%Then the map $f \mapsto \lambda e.M(\lambda h.G_e(f,h))$ is not Borel continuous.
%\end{corollary}

\smallskip

The previous results, as well as the equivalence in Corollary \ref{eessje}, suggest
%\marginpar{\footnotesize{corrected: suggests $\rightarrow$ suggest}} 
a strong similarity between the special fan functional 
%\marginpar{\footnotesize{Deleted one ocurence of the word "functional".}} 
$\Theta$ and PRs.  
In fact, Theorem \ref{poilo} can be seen as a consequence of the following theorem and the properties of $\Theta$ established in \cite{dagsam, dagsamII}.  
We establish (and make essential use of) the equivalences in Theorem \ref{nondezju} when discussing Heine's theorem below.  
\begin{thm}\label{nondezju}
Let $G:C \rightarrow \N$. The following are equivalent for each $n \in \N$
\begin{enumerate}
\item There is a {\rm PR} $M$ with $M(G) = n$
\item There is a special fan functional $\Theta$ such that $G(f) \leq n$ for each $f \in \Theta(G)$
\item There are $f_1 , \ldots , f_k \in C$ with $C\subset \cup_{i\leq k}[\bar f_iG(f_i)]$ and $n \geq G(f_i)$ for $i\leq k$.
\end{enumerate}
\end{thm}  
Despite these similarities, there are certain fundamental differences between the special fan functional and Pincherle realisers, leading to the following conjecture.  
%
% as we discuss now.  
%\begin{rem}[PR versus $\Theta$]\rm 
%First note that the distinction between the cases for $G_e(f,h)$ from Definition \ref{corkiiii} only depends on $f$, $e_1$ and $e_2$, not on the entire $e$. 
%
%\smallskip
%
%Secondly, for \emph{any} $\Theta$ as in $\SCF(\Theta)$ and $\Theta(\lambda h.G_e(f,h)) := \{h_1, \ldots , h_n\}$, where $f,e_1,e_2$ are as in the proof of Theorem \ref{thm8.3}, there must be an $h_i$ such that we use case (2.c) to define $G_e(f,h_i)$; note that we can \emph{uniformly} compute the function $[e_3]^{h^{e_2}}$ from this information. If we let $f$ be constant zero, this means that we uniformly in $\Theta$, and modulo $\exists^2$, can compute a sequence $\{f_e\}_{e \in \N}$ of total functions such that for all hyperarithmetical $\phi^1$ there is a fixed  $e$, essentially the index for $\phi$ as a hyperarithmetical function,  such that $\phi = f_e$.
%
%\smallskip
%
%Thirdly, as we have seen, any PR $M$ can  be used to diagonalise out of the hyperarithmetical functions relative to any $f$, but how  we perform the latter procedure \emph{depends on the chosen instance} of $M$.  
%Furthermore, we see no way to use $M$ to produce a total enumeration of functions containing all hyperarithmetical ones \emph{in fixed places}, but $\Theta$ does produce such an enumeration.    
%\end{rem}
%The previous remark supports the following conjecture.  
Even if the latter turns out to be incorrect, we still expect that there is no \emph{uniform} way to compute an instance of $\Theta$ from an instance of $M$, even modulo $\exists^2$.
\begin{conj}\label{corkes}
%There is a term $t$ from G\"odel's $T$ such that $(\forall \Theta)(\SCF(\Theta)\di\PR(t(\Theta)))$, provable in $\RCAo$.
There is $M_{0}^{3}$ satisfying $\PR(M_{0})$ such that no $\Theta^{3}$ as in $\SCF(\Theta))$ is computable \textup{(S1-S9)} in $M_{0}^{3}$.   
\end{conj}
We finish this section with a remark on the exact formulation of (realisers for) local boundedness; recall that we used \emph{one functional} $G$ in $\LOC(F, G)$. 
\begin{rem}\label{nodiff}\rm
In order to be faithful to the original formulation of Pincherle, the bounding condition has to be given by two functionals $G_1$ and $G_2$, as follows: 
\[
\LOC^*(F, G_1,G_2)\equiv (\forall f , g\in C)\big[ g\in [\overline{f}G_1(f)] \di F(g)\leq G_2(f)    \big].
\]
Let $M^*$ be a functional which on input $(G_{1}, G_{2})$ provides an upper bound on $C$ for $F$ satisfying $\LOC^{*}(F, G_{1}, G_{2})$.  
A PR $M$ can be reduced to such $M^{*}$, and vice versa, as follows: $M(G) = M^*(G,G)$ and  $M^*(G_1,G_2) = M(\max\{G_1,G_2\})$.  % they are essentially equivalent. 
\end{rem}


\subsection{Realisers for Pincherle's original theorem}\label{PRR2}
In this section, we study the computational properties of realisers for $\PIT_{o}$.  
As discussed in Section \ref{introwpr}, there are two natural examples of such realisers (in contrast to $\PIT_{\u}$, where there was only one natural choice).  
We show in Sections \ref{stream1} and \ref{stream2} that these two classes of realisers have extremely different computational properties.  
\subsubsection{Introduction}\label{introwpr}
 In the previous section we have established that Pincherle realisers, i.e.\ realisers for $\PIT_{\u}$, are \emph{hard to compute}, and Theorem \ref{mooier} shall establish
that $\PIT_{\u}$ is similarly \emph{hard to prove}.  This correspondence between computational and first-order `hardness' also\footnote{Indeed, the special fan functional $\Theta$ is a realiser for $\HBU_{\c}$, and $\Theta$ cannot be computed by any type two functional, while $\SIXK$ cannot prove $\HBU_{\c}$ by the results in \cite{dagsam}*{\S3}.} holds for Heine-Borel compactness by \cite{dagsamIII}*{\S3}.  Moreover, the linear order \eqref{linord}, and even the \emph{G\"odel hierarchy} (See Appendix \ref{kurtzenhier}), is based on the very idea that computational and first-order hardness line up.  

\smallskip

In this section, we show that $\PIT_{o}$ does not follow the aforementioned correspondence.
 Indeed, on one hand $\PIT_{o}$ is \emph{easy to prove}: it essentially follows from $\WKL$ by Corollary \ref{ofmoreinterest}.  On the other hand, the two natural notions of `realiser for $\PIT_o$' will be shown to be hard to compute. 
These two kinds of realisers arise from the two possible kinds of realisers for $\ATR_{0}$: based on \eqref{WPR1} and \eqref{WPR2} respectively.  The latter formulas are classically equivalent, but yield very different realisers.  
% Before we introduce these realisers, 
% we will discuss the distinction in the light of some previous results.
% %
 \begin{rem}\rm
 In \cite{dagsam,dagsamII} we proved that a special fan functional $\Theta$ (with Feferman's $\mu$) computes a realiser for $\ATR_0$ as follows: given a total ordering `$\prec$' and an arithmetical operator `$\Gamma$', we can compute a pair $(x,y)$ such that either $x$ codes a $\Gamma$-chain over $\prec$, or $y$ codes a $\prec$-descending sequence. This is a realiser for:
\be\label{WPR1}
 \neg \WO(\prec) \vee (\exists X\subset \N)(\textup{$X$ is a $\Gamma$-chain over $\prec$}).
 \ee
The situation is different for PRs: \emph{if} $\prec$ is a well-ordering, \emph{then} we can compute the unique $\Gamma$-chain $X$, and by the Extension Theorem \ref{ET}, there is for any $\PR$ $M$, 
a total functional $\Delta(\prec,\Gamma)$ that is primitive recursive in $M, \mu$, and that gives us a $\Gamma$-chain (over $\prec$) \emph{assuming} $\prec$ is a well-ordering.  The difference is that for PRs, no information is provided when $\prec$ is \emph{not} a well-ordering. This yields a realiser for: 
 \be\label{WPR2}
 \WO(\prec)\di (\exists X\subset \N)(\textup{$X$ is a $\Gamma$-chain over $\prec$}).
 \ee
 %We refer to the realiser of \eqref{WPR1} (resp.\ \eqref{WPR2}) as the \emph{strong} (resp.\ \emph{weak}) realiser. 
 %\begin{center} $\WO(\prec) \rightarrow $ there is a $\Gamma$-chain over $\prec$\end{center}
 Below, we will consider two similar kinds of  realisers for $\PIT_o$, and we will see that the difference in complexity is considerable.
As an aside, it is an open problem if it is possible to compute realisers for $\ATR_0$ of the strong kind \eqref{WPR1} from a PR, a problem intimately connected to the problem if $\HBU$ is computationally derivable from $\PIT_o$ via realisers (and relative to $\mu$).
\end{rem}
\subsubsection{Weak Pincherle realisers}\label{stream1}
We introduce a notion of realiser for $\PIT_{o}$ based on \eqref{WPR1}.
 %\marginpar{\footnotesize{ The phrase "no type two functional suffices" is an understatement, and I think that we should leave it out. I suggest to replace it with something like "We will later discuss why this is to be expected."}}
To this end, note that \eqref{DOG} is the latter with all quantifiers brought to the front. It is \emph{extremely hard} to compute the underlined objects in \eqref{DOG} in terms of $F, G$, by Corollary \ref{comb3}.  We will later discuss why this is to be expected.
\begin{align}
(\forall F, G:C\di \N)\underline{(\exists N\in \N, f, g\in C})&(\forall h\in C)\label{DOG} \\
&\big( \big( g\in [\overline{f}G(f)] \di F(g)\leq G(f)\big)    \di (F(h)\leq N)\big).\notag
\end{align}
By contrast, bringing $\WKL$ in the same form as \eqref{DOG}, one readily\footnote{One readily brings $\WKL$ in the following equivalent form:
\be\label{forgukkkk}
(\forall G^{2}, T\leq 1)(\exists m^{0}, \alpha\in C)\big[\overline{\alpha}G(\alpha)\not\in T\di (\forall \beta^{0^{*}})(|\beta|=m\di  \overline{\beta}\not\in T )     \big].
\ee
The formula in square brackets is quantifier-free.  Then $\QFAC^{2,1}$ yields a witnessing functional. 
} obtains a witnessing functional. 
In conclusion, the behaviour of $\PIT_{o}$ in RM seems to diverge \emph{completely} from its computability-theoretic behaviour.  

\smallskip

We now introduce the following specification for a non-unique functional computing $N, f, g$ as in \eqref{DOG}. %$M^{(2\times 2)\di ((1\times 1)\times 0 )}$. 
%We write $F^{i}(f)$ for $F(f)(i)$.  
Note that the number $i$ can be obtained by checking if $f, g$ witness $\neg\LOC(F, G)$. 
\bdefi[$\WPR(M_{o})$]\label{weakling}
For any $F, G:C\di \N$, $M_{o}(F, G)=(i^{0},N^{0}, f^{1}, g^{1})$ is such that either $i = 0$ and $N$ is an upper bound for $F$ on $C$, or $i = 1$ and $g \in [\bar fG(f)]\wedge F(g) > G(f)$, i.e.\ the functions $f, g$ witness $\neg \LOC(F,G)$. 
\edefi
Any $M_{o}$ satisfying $\WPR(M_{o})$ is called a \emph{\textbf{weak} Pincherle realiser} (WPR for short).
We emphasise the modifier `weak': \eqref{DOG} and Definition \ref{weakling} may seem to be the most natural choice, esp.\ following the idea of realisers and the aforementioned results on $\HBU_{\c}$, but the computational strength of any WPR, as established below, immediately disqualifies it as a `true' realiser.   

\smallskip

For our next results, we need the following functional, similar to $\kappa^{3}$ from \cite{dagsam}:
%We will see that $\exists^3$, or an equivalent $\kappa^3$ defined just on $C$, will be computable in any $M_o$ satisfying $\WPR$.
\be\tag{$\kappa_{0}^{3}$}
(\exists \kappa_{0}^{3}\leq_{3}1)(\forall Y^{2})\big[\kappa_{0}(Y)=0\asa (\exists f\in C)Y(f)=0  \big].
\ee
Note that $\RCAo+\WKL+(\kappa_{0}^{3})+\QFAC^{0,1}$ is conservative\footnote{To be absolutely clear, we take `$\WKL$' to be the $\L_{2}$-sentence \emph{every infinite binary tree has a path} as in \cite{simpson2}, while the Big Five system $\WKL_{0}$ is $\RCA_{0}+\WKL$, and $\WKL_{0}^{\omega}$ is $\RCAo+\WKL$.} \emph{up to language}\footnote{The fundamental objects in the language of $\RCAo$ are functions, with sets being definable from these, while it is exactly the opposite for $\RCA_{0}$. This however makes no difference.} over $\WKL_{0}$ by \cite{kohlenbach2}*{Prop.\ 3.15}, while $\RCAo$ proves that $[(\exists^{2})+(\kappa_{0}^{3})]\asa (\exists^{3})$ by \cite{dagsam}*{Rem.\ 6.13}.  
\begin{thm}\label{napjeir}
The system $\ACAo$ proves $ (\exists M)\WPR(M)\di (\kappa^{3}_{0})$.
\end{thm}
\begin{proof}
%So \exists^3(F) = 0 if F is total 0 and 1 otherwise.
%
For each $F^{2}$, define $\sigma_{n}$ as the sequence $1\dots 1$ of length $n$, and define $H^{2}$ as:
\[
H(f):=
\begin{cases}
0 & \textup{if $ f=_{1}1$}\\
(n+1)\cdot F(g) & \textup{if $f=_{1}\sigma_{n}*0*g$}
\end{cases}.
\]
Note that $\mu^{2}, F$ suffices to define the functional $H$;
%H(f) = 0 if f is constant 1
%
%H(f) = (n+1) \times F(g) if f = (1^n)0g
%
The latter is constant $0$ on $C$ if $F$ is constant $0$ on $C$, and unbounded otherwise.
%Let $G_{0}^{2}$ be the constant $1$ functional of type two. Then $\LOC(H,G)$ if and only if $H$ is constant $0$.
Let $M$ be such that $\WPR(M)$, i.e.\ if $M(H,G) = (i,N,f,g)$ we have that either $N$ is an upper bound for $H$ on Cantor space or that $H(g) > G(f)$. 
Hence, by evaluating $H(g)$ we can decide if $H$, and thus $F$, is constant $0$.
\end{proof}
\begin{cor}\label{comb1}
The system $\ACAo+\QFAC^{2,1}$ proves $ (\exists M)\WPR(M)\asa (\exists^{3})$.  
\end{cor}
\begin{proof}
The reverse implication follows from obtaining $\PIT_{o}$ via Theorem \ref{ofinterest}, and then applying $\QFAC^{2,1}$ to obtain $(\exists M)\WPR(M)$.  
The forward implication follows from $[(\exists^{2})+(\kappa_{0}^{3})]\asa (\exists^{3})$ and the theorem.  % the latter fact follows from \cite{dagsam}*{Rem.\ 6.13}.  
\end{proof}
\begin{cor}\label{comb3}
A WPR combined with $\mu^{2}$ computes $\exists^{3}$ via a term of G\"odel's $T$.
\end{cor}
%
 The converse of this corollary', even when we replace G\"odel's $T$ with Kleene's S1-S9, is  not provable in $\ZFC$, essentially due to the fact that $\QFAC^{2,1}$ has no realiser provably computable in $\exists^3$.

\subsubsection{Another realiser for Pincherle's theorem}\label{stream2}
We introduce another realiser for $\PIT_{o}$, based on \eqref{WPR2}, after some discussion why WPRs are not satisfactory.

\smallskip

First of all, the concept of WPRs turned out to be too strong, because defining a realiser for the prenex normal form of $\PIT_o$ (almost) induces the ability to decide the relation $\LOC(F,G)$, and the definition will use unbounded quantifiers over $C$ with type two parameters. Moreover it (almost) induces the ability to select an element of an arbitrary non-empty subset of $C$.  It does not reflect what we aim for with realisers: given that $\LOC(F,G)$, how hard is it to find an upper bound for $G$?

\smallskip

Secondly, if we chose not to rewrite $\PIT_{o}$ to its prenex normal form, then it is natural to consider a functional $M_o^*$ as a `realiser for $\PIT_o$' if $M_o^*(F,G)$ is an upper bound for $G$ \emph{whenever} $\LOC(F,G)$, but containing no information about $F$ or $G$ in the case of $\neg\LOC(F,G)$, similar to \eqref{WPR2}. Clearly, every realiser for $\PIT_u$ is a realiser for $\PIT_o$ in this sense, but since $\PIT_u$ is not logically derivable from $\PIT_o$,  we cannot expect to be able to compute any PR from these simpler forms. 

\smallskip

However, even though by Corollary \ref{ofmoreinterest}, $\PIT_o$ is provable in a weak logical system, using a modest version of the axiom of choice, it is impossible to compute any of these modest realisers from any type two functional:
\begin{theorem}
There is no functional $M_o^*$ at type level 3 computable in any  type 2 functional  such that
\be\tag{$\MPR(M_{o}^{*})$}
\forall F^2,G^2 (\LOC(F,G) \rightarrow \forall f^1(G(f) \leq M_o^*(F,G))),
\ee
\end{theorem}
%
\begin{proof} The proof follows the pattern of our proofs  of similar results.  Let $H$ with $\mu\leq_{Kleene} H$ be any type 2 functional, and assume that $M_o^*$ is computable in $H$.  Let $F^*$ be partially $H$-computable and injective on the set of $H$-computable functions, taking only values $>1$ and let $G^*$ be the constant 0. Then $\LOC(F,G^*)$ for any total extension $F$ of $F^*$. 

\smallskip

The computation of $M_o^*(F,G^*) = N$ from $H$ will then only make oracle calls $F(f) = F^*(f)$ or $G(f) = 0$ for a countable set of $f$'s enumerable by an $H$-comptutable function. If we let $G(f) = N+1$ if $f$ is neither in this enumerated set nor in any neighbourhood induced by $F^*(f)$ where $F^*(f) \leq N$, and 0 elsewhere, and we let $F(f) = F^*(f)$ when defined, and $N+1$ elsewhere, we still have that $M_0^*(F,G) = N$, $\LOC(F,G)$, but not that $N$ is an upper bound for $G$. This is the desired contradiction.
\end{proof}
Note that `$\MPR$' stands for `modest PR' in the theorem.  
Despite this suggestive name, the combination of Theorem \ref{horniii} and Theorem \ref{mooier} yields a model that satisfies $\SIXK$, but falsifies $\PIT_{u}$ and is lacking any and all realisers for Pincherle's theorem. 
A functional of type two is \emph{normal} if it computes the functional $\exists^{2}$.  
%The following associated type structure is needed for the proof of Theorem \ref{horniii}.  
\bdefi\label{nikeh}
For normal $H^{2}$, the type structure ${\mathcal M}^H = \{{\mathcal M}^H_k\}_{k \in \N}$ is defined as ${\mathcal M}^H_0 = \N$ and ${\mathcal M}^H_{k+1}$ consists of all $\phi:{\mathcal M}^H_k \rightarrow \N$ computable in $H$ via Kleene's S1-S9.
The set ${\mathcal M}^H_1$ is the \emph{1-section} of $H$; the restriction of $H$ to $\mathcal{M}_{1}^{H}$ is in ${\mathcal M}^H_2$.
\edefi
%By Theorem \ref{ofinterest} below, $\ACAo+\QFAC^{0,1}$ proves $\PIT_{o}$. %, but realisers for the latter are nonetheless \emph{extremely hard} to obtain by the following theorem and corollary.  
\begin{theorem}\label{horniii}
For any normal $H^{2}$, the type structure ${\mathcal M}^H$ is a model for $\QFAC^{0,1}$, $ \neg \PIT_{\u}$, $\PIT_{o}$, $(\forall M_{o})\neg\WPR(M_{o})$, $(\forall M_u) \neg \PR(M_u)$, and $(\forall M_{o}^{*})\neg\MPR(M_{o}^{*})$. 
%There is even no modest realiser for $\PIT_o$ in  ${\mathcal M}^H$.
\end{theorem}
\begin{proof}
Fix $N\in \N$ and let $H^*(f) = e+1$ where $e$ is some $H$-index for $f$ found using Gandy selection. The first claim follows readily from Gandy selection. The second claim is proved as for \cite{dagsamIII}*{Theorem 3.4} by noting that $H^*$ restricted to the finite set $\{f_1 , \ldots , f_k\}$ of functions $f$ with $H^*(f) \leq N$ does not induce a sufficiently large sub-cover of $C$ to guarantee that all $F$ satisfying the bounding condition induced by $H^*$ is bounded by $N$. 
%\marginpar{\footnotesize{I have corrected the notation used in this proof, but I do not see the point in including the claim in this form. There is an alternatiove definition of WPR for which the statement makes sense, but this will involve partial functionals.}} 

\smallskip

In order to prove the third claim, let $G\in {\mathcal M}^H_2$ be arbitrary, and let $F \in {\mathcal M}^H_2$ satisfy the bounding condition induced by $G$. Assume that $F$ is unbounded. Then, employing Gandy selection we can, computably in $H$, find a sequence $\{f_i\}_{i \in \N}$ such that $F(f_i) > i$ for all $i$. Using $\exists^2$ we can then find a convergent subsequence and compute its limit $f$. Then $F$ will be bounded by $G(f)$ on the set $[\bar fG(f)]$, contradicting the choice of the sequence $f_i$. 

\smallskip

In order to prove the fourth claim, assume that $M_o \in {\mathcal M}^H_3$ is a WPR in ${\mathcal M}^H$. Let $F$ be the constant zero, and let $M_o( H^*,F) = (i,N,f_0,g_0)$. We now use that $M_o$ is computable in $H$, and that thus the computation tree of $M_o(F,H^*)$ in itself is computable in $H$. Let $f_1 , \ldots , f_k$ be as in the argument for the second claim. There will be some $f$ computable in $H$ that is not in any of the neigbourhoods $[\bar f_iH^*(f_i)]$ and such that $F(f)$ is not called upon in the computation of $M_o(F,H^*)$. We may now define $F_N$ so that $F_N(f) = 0$ if $F(f) = 0$ is used in the computation of $M_o(G, H^*)$ or if $f \in [\bar f_i(H^*(f_i)]$ for $i = 1 , \ldots , k$, and we let $F_N(f) = N+1$ otherwise. Then the computation of $M_o(F_N,H^*)$ yields the same value as the computation of $M_o(F,H^*)$ and $F_N$ still satisfies the bounding condition induced by $H^*$, but the output does not give an upper bound for $F_N$.
Since we never used $f_0,g_0$ in $M_o(H^*,F)$ in this argument, the fifth and sixth claims follow.
\end{proof}



%We have established in the previous section that Pincherle realisers, i.e.\ realisers for $\PIT_{\u}$, are \emph{hard to compute}, and Theorem \ref{mooier} shall establish
%that $\PIT_{\u}$ is similarly \emph{hard to prove}.  This correspondence between computational and first-order `hardness' also\footnote{Indeed, the special fan functional $\Theta$ is a realiser for $\HBU_{\c}$, and $\Theta$ cannot be computed by any type two functional, while $\SIXK$ cannot prove $\HBU_{\c}$ by the results in \cite{dagsam}*{\S3}.} holds for Heine-Borel compactness by \cite{dagsamIII}*{\S3}.  Moreover, the linear order \eqref{linord}, and even the \emph{G\"odel hierarchy} (\cite{sigohi}), is based on the very idea that computational and first-order hardness line up.  
%
%\smallskip
%
%In this section, we show that $\PIT_{o}$ does not follow the aforementioned correspondence.
% Indeed, on one hand $\PIT_{o}$ is \emph{easy to prove}: it essentially follows from $\WKL$ by Corollary \ref{ofmoreinterest}.  
% %\marginpar{\footnotesize{ The phrase "no type two functional suffices" is an understatement, and I think that we should leave it out. I suggest to replace it with something like "We will later discuss why this is to be expected."}}
%On the other hand, \eqref{DOG} is just $\PIT_{o}$ with all quantifiers brought to the front, but it is \emph{extremely hard} to compute the underlined objects in \eqref{DOG} in terms of $F, G$, by Corollary \ref{comb3}.  We will later discuss why this is to be expected.
%\begin{align}
%(\forall F, G:C\di \N)\underline{(\exists N\in \N, f, g\in C})&(\forall h\in C)\label{DOG} \\
%&\big( \big( g\in [\overline{f}G(f)] \di F(g)\leq G(f)\big)    \di (F(h)\leq N)\big).\notag
%\end{align}
%By contrast, bringing $\WKL$ in the same form as \eqref{DOG}, one readily\footnote{One readily brings $\WKL$ in the following equivalent form:
%\be\label{forgukkkk}
%(\forall G^{2}, T\leq 1)(\exists m^{0}, \alpha\in C)\big[\overline{\alpha}G(\alpha)\not\in T\di (\forall \beta^{0^{*}})(|\beta|=m\di  \overline{\beta}\not\in T )     \big].
%\ee
%The formula in square brackets is quantifier-free.  Then $\QFAC^{2,1}$ yields a witnessing functional. 
%} obtains a witnessing functional. 
%In conclusion, the behaviour of $\PIT_{o}$ in RM seems to diverge \emph{completely} from its computability-theoretic behaviour.  
%
%\smallskip
%
%We now introduce the following specification for a non-unique functional computing $N, f, g$ as in \eqref{DOG}. %$M^{(2\times 2)\di ((1\times 1)\times 0 )}$. 
%%We write $F^{i}(f)$ for $F(f)(i)$.  
%Note that the number $i$ can be obtained by checking if $f, g$ witness $\neg\LOC(F, G)$. 
%\bdefi[$\WPR(M_{o})$]\label{weakling}
%For any $F, G:C\di \N$, $M_{o}(F, G)=(i^{0},N^{0}, f^{1}, g^{1})$ is such that either $i = 0$ and $N$ is an upper bound for $F$ on $C$, or $i = 1$ and $g \in C_{\bar f(G(f))}\wedge F(g) > G(f)$, i.e.\ the functions $f, g$ witness $\neg \LOC(F,G)$. 
%\edefi
%Any $M_{o}$ satisfying $\WPR(M_{o})$ is called a \emph{\textbf{weak} Pincherle realiser} (WPR for short).
%We emphasise the modifier `weak': \eqref{DOG} and Definition \ref{weakling} may seem to be the most natural choice, esp.\ following the idea of realisers and the aforementioned results on $\HBU_{\c}$, but the computational strength of any WPR, as established below, immediately disqualifies it as a `true' realiser.   
%
%\smallskip
%
%For our next results, we need the following functional, similar to $\kappa^{3}$ from \cite{dagsam}:
%%We will see that $\exists^3$, or an equivalent $\kappa^3$ defined just on $C$, will be computable in any $M_o$ satisfying $\WPR$.
%\be\tag{$\kappa_{0}^{3}$}
%(\exists \kappa_{0}^{3}\leq_{3}1)(\forall Y^{2})\big[\kappa_{0}(Y)=0\asa (\exists f\in C)Y(f)=0  \big].
%\ee
%Note that $\RCAo+(\kappa_{0}^{3})+\QFAC^{0,1}$ is conservative \emph{up to language}\footnote{The fundamental objects in the language of $\RCAo$ are functions, with sets being definable from these, while it is exactly the opposite for $\RCA_{0}$. This however makes no difference.} over $\WKL_{0}$ by \cite{kohlenbach2}*{Prop.\ 3.15}, while $\RCAo$ proves that $[(\exists^{2})+(\kappa_{0}^{3})]\asa (\exists^{3})$ by \cite{dagsam}*{Rem.\ 6.13}.  
%\begin{thm}\label{napjeir}
%The system $\ACAo$ proves $ (\exists M)\WPR(M)\di (\kappa^{3}_{0})$.
%\end{thm}
%\begin{proof}
%%So \exists^3(F) = 0 if F is total 0 and 1 otherwise.
%%
%For each $F^{2}$, define $\sigma_{n}$ as the sequence $1\dots 1$ of length $n$, and define $H^{2}$ as:
%\[
%H(f):=
%\begin{cases}
%0 & \textup{if $ f=_{1}1$}\\
%(n+1)\cdot F(g) & \textup{if $f=_{1}\sigma_{n}*0*g$}
%\end{cases}.
%\]
%Note that $\mu^{2}, F$ suffices to define the functional $H$;
%%H(f) = 0 if f is constant 1
%%
%%H(f) = (n+1) \times F(g) if f = (1^n)0g
%%
%The latter is constant $0$ on $C$ if $F$ is constant $0$ on $C$, and unbounded otherwise.
%%Let $G_{0}^{2}$ be the constant $1$ functional of type two. Then $\LOC(H,G)$ if and only if $H$ is constant $0$.
%Let $M$ be such that $\WPR(M)$, i.e.\ if $M(H,G) = (i,N,f,g)$ we have that either $N$ is an upper bound for $H$ on Cantor space or that $H(g) > G(f)$. 
%Hence, by evaluating $H(g)$ we can decide if $H$, and thus $F$, is constant $0$.
%\end{proof}
%\begin{cor}\label{comb1}
%The system $\ACAo+\QFAC^{2,1}$ proves $ (\exists M)\WPR(M)\asa (\exists^{3})$.  
%\end{cor}
%\begin{proof}
%The reverse implication follows from obtaining $\PIT_{o}$ via Theorem \ref{ofinterest}, and then applying $\QFAC^{2,1}$ to obtain $(\exists M)\WPR(M)$.  
%The forward implication follows from $[(\exists^{2})+(\kappa_{0}^{3})]\asa (\exists^{3})$ and the theorem.  % the latter fact follows from \cite{dagsam}*{Rem.\ 6.13}.  
%\end{proof}
%\begin{cor}\label{comb3}
%A WPR combined with $\mu^{2}$ computes $\exists^{3}$ via a term of G\"odel's $T$.
%\end{cor}
%A functional of type two is \emph{normal} if it computes $\exists^{2}$.  
%The following associated type structure is needed below.  
%\bdefi\label{nikeh}
%For normal $H^{2}$, the type structure ${\mathcal M}^H = \{{\mathcal M}^H_k\}_{k \in \N}$ is defined as ${\mathcal M}^H_0 = \N$ and ${\mathcal M}^H_{k+1}$ consists of all $\phi:{\mathcal M}^H_k \rightarrow \N$ computable in $H$ via Kleene's S1-S9.
%The set ${\mathcal M}^H_1$ is the \emph{1-section} of $H$; the restriction of $H$ to $\mathcal{M}_{1}^{H}$ is in ${\mathcal M}^H_2$.
%\edefi
%%By Theorem \ref{ofinterest} below, $\ACAo+\QFAC^{0,1}$ proves $\PIT_{o}$. %, but realisers for the latter are nonetheless \emph{extremely hard} to obtain by the following theorem and corollary.  
%\begin{theorem}\label{horniii}
%For any normal functional $H^{2}$, the type structure ${\mathcal M}^H$ is a model for $\QFAC^{0,1}, \neg \PIT_{\u}$, $\PIT_{o}$, $(\forall M_{o})\neg\WPR(M_{o})$,  and $(\forall M_u) \neg \PR(M_u)$.
%\end{theorem}
%\begin{proof}
%Fix $N\in \N$ and let $H^*(f) = e+1$ where $e$ is some $H$-index for $f$ found using Gandy selection. The first claim follows readily from Gandy selection. The second claim is proved as for \cite{dagsamIII}*{Theorem 3.4} by noting that $H^*$ restricted to the finite set $\{f_1 , \ldots , f_k\}$ of functions $f$ with $H^*(f) \leq N$ does not induce a sufficiently large sub-cover of $C$ to guarantee that all $F$ satisfying the bounding condition induced by $H^*$ is bounded by $N$. 
%%\marginpar{\footnotesize{I have corrected the notation used in this proof, but I do not see the point in including the claim in this form. There is an alternatiove definition of WPR for which the statement makes sense, but this will involve partial functionals.}} 
%
%\smallskip
%
%In order to prove the third claim, let $G\in {\mathcal M}^H_2$ be arbitrary, and let $F \in {\mathcal M}^H_2$ satisfy the bounding condition induced by $G$. Assume that $F$ is unbounded. Then, employing Gandy selection we can, computably in $H$, find a sequence $\{f_i\}_{i \in \N}$ such that $F(f_i) > i$ for all $i$. Using $\exists^2$ we can then find a convergent subsequence and compute its limit $f$. Then $F$ will be bounded by $G(f)$ on the set $C_{\bar f(G(f)}$, contradicting the choice of the sequence $f_i$. 
%
%\smallskip
%
%In order to prove the fourth claim, assume that $M_o \in {\mathcal M}^H_3$ is a WPR in ${\mathcal M}^H$. Let $F$ be the constant zero, and let $M_o( H^*,F) = (i,N,f_0,g_0)$. We now use that $M_o$ is computable in $H$, and that thus the computation tree of $M_o(F,H^*)$ in itself is computable in $H$. Let $f_1 , \ldots , f_k$ be as in the argument for the second claim. There will be some $f$ computable in $H$ that is not in any of the neigbourhoods $C_{\bar f_i(H^*(f_i))}$ and such that $F(f)$ is not called upon in the computation of $M_o(F,H^*)$. We may now define $F_N$ so that $F_N(f) = 0$ if $F(f) = 0$ is used in the computation of $M_o(G, H^*)$ or if $f \in C_{\bar f_i(H^*(f_i))}$ for $i = 1 , \ldots , k$, and we let $F_N(f) = N+1$ otherwise. Then the computation of $M_o(F_N,H^*)$ yields the same value as the computation of $M_o(F,H^*)$ and $F_N$ still satisfies the bounding condition induced by $H^*$, but the output does not give an upper bound for $F_N$.
%Since we never used $f_0,g_0$ in $M_o(H^*,F)$ in this argument, the fifth claim follows.
%\end{proof}
%A possible explanation for the computational strength of WPRs is as follows: to obtain $\Theta$ from $\HBU_{\c}$ (resp.\ $\HBU$), it suffices to note that we may replace the underlined quantifier in its definition by a quantifier over the \emph{finite} binary sequences (resp.\ the rationals).  
%Hence, the modified formula becomes $\Pi_{1}^{0}$, and hence decidable using $\exists^{2}$, and $\QFAC$ yields $\Theta$.  
%However, the variable `$h$' in \eqref{DOG} and $\PIT_{o}$ occurs as `call by name', i.e.\ such a replacement is not possible; we really seem to need $\exists^{3}$ to decide the innermost formula in \eqref{DOG} and apply $\QFAC$ to obtain a WPR.    
%
%\smallskip
%
%Another possible explanation is that the high computational complexity of WPRs is caused by the fact that all objects involved are assumed \emph{total}.  Using functionals as \emph{realisers}, this is natural, since a realiser of a disjunction will contain information about which disjunct is true, and then an object that realises the truth of that object. It is the classical view of implications-as-disjunctions, combined with the natural definition of realisers for disjunctions, that leads to our definition of WPRs. However, if we view implications more in the tradition of constructive mathematics, then a realiser for an implication will map data satisfying the antecedent to data realising, or verifying,  the conclusion. This requires \emph{partial} functionals and most natural computational frameworks do allow for \emph{partial} objects; we now briefly sketch what such a modification would look like.  
%Intuitively speaking, we just let a \emph{verifier} be undefined if there is no upper bound, i.e.\ $\LOC(F, G)$ is false.  
%\begin{definition} 
%\rm{Let $M^p_o$ be a partial functional of type $(2,2) \rightarrow 0$; the formula $\WPR^*(M^p_o)$ expresses that $M^p_o(F,G)$ is defined whenever $\LOC(F,G)$ and then $M^p_o(F,G)$ is an upper bound for $F$.}
%\end{definition}
%The following theorem is then proved via trivial adjustments of previous proofs.
%\begin{thm}~
%\begin{enumerate}
%\item If $\PR(M_u)$ and we define $M^p_o(F,G) = M_u(G)$, then $\WPR^*(M^p_o)$.
%\item There is no $M^p_o$ satisfying $\WPR^*(M_{o}^{p})$ that is partially computable in any type 2 functional.
%\item In the models ${\mathcal M}^H$, there is no $M^p_u$ partially computable in any functional in the model that satisfies $\WPR^*(M_{u}^{p})$.
%\end{enumerate}
%\end{thm}
%In future research, we intend to explore this further. 
% 
%


\section{Pincherle's and Heine's theorem in Reverse Mathematics}\label{prm}
We classify the various forms of Pincherle's and Heine's theorems from Sections~\ref{sum} and \ref{heineke} within the framework of higher-order RM. 
Sections~\ref{finne} is devoted to a detailed study of the proof techniques used in this section, the role of the axiom of choice and the law of excluded middle in particular.  
%On one hand, our proofs often exploit the law of excluded middle in a curious way, and this technique leads to previously thought rare results in RM involving disjunctions and splittings; this technique may also be viewed as a general/universal/classical version of \emph{Ishihara's tricks} introduced in \cite{discihara}.  On the other hand, we often make use of the axiom of countable choice, and the exact role  
\subsection{Pincherle's theorem and second-order arithmetic}\label{prm1}
We formulate a supremum principle which allows us to easily obtain $\HBU$ and $\PIT_{\u}$ from $(\exists^{3})$; this constitutes a significant improvement over the results in \cite{dagsamIII}*{\S3}.   
We show that $\PIT_{\u}$ is not provable in any $\SIXK$, but that $\ACAo+\QFAC^{0,1}$ proves $\PIT_{o}$.
% the third
%Big Five system $\ACA_{0}^{\omega}\equiv \RCAo+\QFAC+(\exists^{2})$, which is $\Pi_{2}^{1}$-conservative over $\ACA_{0}$. 

\smallskip

First of all, a formula $\varphi(x^{1})$ is called \emph{extensional on $\R$} if we have
\[
(\forall x, y\in \R)(x=_{\R}y\di \varphi(x)\asa \varphi(y)).  
\]
Note that the same condition is used in RM for defining open sets as in \cite{simpson2}*{II.5.7}.  
%DAG: Lebesgue proved $\HBU$ using $\LUB$ (or something like it) in 1904 (see the maa.org article on HB proofs). 
\begin{princ}[$\LUB$]
For second-order $\varphi$ \(with any parameters\), if $\varphi(x^{1})$ is extensional on $\R$ and $\varphi(0)\wedge \neg\varphi(1)$, there is a least $y\in [0,1]$ such that $(\forall z \in (y, 1])\neg\varphi(y)$.  
\end{princ}
Secondly, we have the following theorem, which should be compared\footnote{Keremedis proves in \cite{kermend} that the statement \emph{a countably compact metric space is compact} is not provable in $\ZF$ minus the axiom of foundation.  This theorem does follow when the axiom of countable choice is added.} to \cite{kermend}*{\S2}.    
The reversal of the final implication is proved in Corollary \ref{eessje}, using $\QFAC^{0,1}$.
\begin{thm}\label{mooi}
The system $\RCAo$ proves $(\exists^{3})\di \LUB\di \HBU\di \HBU_{\c}\di  \PIT_{\u}$.
\end{thm}
\begin{proof}
For the first implication, note that $(\exists^{3})$ can decide the truth of any formula $\varphi(x)$ as in $\LUB$.  
Hence, the usual interval-halving technique yields the least upper bound as required by $\LUB$.  
%\marginpar{\footnotesize{This argument reminds me of a proof for HBU by Lebesgue. I found the proof on a web-page for old proofs of Heine-Borel, and shouls be able to find it again if needed.}}
For the second implication, fix $\Psi:\R\di \R^{+}$ and consider 
\[
\varphi(x)\equiv x\in [0,1] \wedge (\exists w^{1^{*}})(\forall y^{1}\in [0,x])(\exists z\in w)(y\in I_{z}^{\Psi}), 
\]
which is clearly extensional on $\R$.  Note that $\varphi(0)$ holds with $w=\langle 0\rangle$, and if $\varphi(1)$, then $\HBU$ for $\Psi$ follows.  In case $\neg\varphi(1)$, we use $\LUB$ to find 
the least $y_{0}\in [0,1)$ such that $(\forall z>_{\R} y_{0})\neg\varphi(y)$.  However, by definition $[0, y_{0}-\Psi(y_{0})/2]$ has a finite sub-cover (of the canonical cover provided by $\Psi$), and hence clearly so does $[0, y_{0}+\Psi(y_{0})/2]$, a contradiction. 
For the final implication, to obtain $\PIT_{\u}$, let $F_{0}, G_{0}$ be such that $\LOC(F_{0}, G_{0})$ and let $w_{0}^{1^{*}}$ be the finite sequence from $\HBU_{\c}$ for $G=G_{0}$.  
Then $F_{0}$ is clearly bounded by $\max_{i<|w|}G_{0}(w(i))$ on Cantor space, and the same holds for \emph{any} $F$ such that $\LOC(F, G_{0})$, as is readily apparent. 

\smallskip

Finally, $\HBU\di \HBU_{\c}$ is readily proved \emph{given} $(\exists^{2})$, since the latter provides a functional which converts real numbers into their binary representation(s).  
Moreover, in case $\neg(\exists^{2})$ all functions on Baire space are continuous by \cite{kohlenbach2}*{Prop.\ 3.7}.  
Hence, $\HBU_{\c}$ just follows from $\WKL_{0}$ (which is immediate from $\HBU$): the latter lemma suffices to prove that a continuous function is uniformly continuous on Cantor space by \cite{kohlenbach4}*{Prop.~4.10}, and hence bounded.  
%But $\cup_{q\in \Q}I_{q}^{\Psi}$ is a countable sub-cover of the canonical sub-cover for \emph{continuous} $\Psi:\R\di \R^{+}$, and hence $\LIND$ follows.  
The law of excluded middle $(\exists^{2})\vee \neg (\exists^{2})$ finishes this part, as we proved $\HBU\di \HBU_{\c}$ for each disjunct.   
\end{proof}
The first part of the proof is similar to Lebesgue's proof of the Heine-Borel theorem from \cite{lebes1}.  We also note that Bolzano used a theorem similar to $\LUB$ (See \cite{nogrusser}*{p.\ 269}).
We now establish that $\PIT_{\u}$ is extremely hard to prove.  
\begin{thm}\label{mooier}
The system ${\Z}_{2}^{\omega}$ proves $\PIT_{\u}$, while no system $\FIVEK^{\omega}+\QFAC^{0,1}$ \($k\geq1$\) proves it.  
\end{thm}
\begin{proof}
The first part follows from Theorem \ref{mooi}.  % and the fact that $\ACAo$ proves $\HBU\asa \HBU_{\c}$; indeed, $\mu^{2}$ readily defines a functional which coverts between reals and their binary representation. 
For the second part, we construct a countable model $\mathcal{M}$ for $\FIVEK^{\omega} + \QFAC^{0,1}$ assuming that $\textsf{V = L}$, where $\textsf{L}$ is G\"odel's universe of constructible sets. This is not a problem, since the model $\mathcal M$ we construct also is a model in the full set-theoretical universe $\textsf{V}$. However, this means that when we write $S^2_k$ in this proof, we really mean the relativised version  $(S^2_k)^{\textsf{L}}$. The advantage is that due to the $\Delta^1_2$-well-ordering of $\N^\N$ in \textsf{L}, if a set $A \subseteq \N^\N$ is closed under computability relative to all $S^2_k$, all $\Pi^1_k$-sets are absolute for $(A,{\textsf{L}})$ and hence  $(S^2_k)^A$ is a sub-functional of  $(S^2_k)^{\textsf{L}}$ for each $k$. We now drop the superscript `\textsf{L}' for the rest of the proof.
Put $S^2_\omega(k,f) := S^2_k(f)$ and note that $S^2_\omega$ is a normal functional in which all $S^2_k$ are computable.
Let ${\mathcal M} = {\mathcal M}^{S^2_\omega}$ be as in Definition \ref{nikeh}. 
%\marginpar{\footnotesize{You need to label Definition 3.8}}
This model is as requested by Theorem \ref{horniii}, i.e.\ $\neg\PIT_{\u}$ holds.  
\end{proof}
The model $\mathcal{M}$ can be used to show that many classical theorems based on uncountable data cannot be proved in any system $\SIXK+ \QFAC^{0,1}$,
%The model $\mathcal M$ from the above proof can likely be used to show that $\FIVEK^{\omega} + \QFAC^{0,1}$ is insufficient for many classical theorems based on uncountable data.
%The model $\mathcal{M}$ can be used to show that $\FIVEK^{\omega} + \QFAC^{0,1}$ does not prove many classical theorems based on uncountable data, 
e.g.\ the Vitali covering lemma and the uniform Heine theorem from Section~\ref{heineke}. 
%\marginpar{\footnotesize{We should give a reference to where we discuss the uniform Heine Theorem.}}

\smallskip

Finally, we show that $\PIT_{o}$ is much easier to prove than $\PIT_{\u}$.
By contrast, weak Pincherle realisers, i.e.\ \emph{realisers} for $\PIT_{o}$, are extremely hard to compute as established in Section \ref{PRR2}.  
%$$ where the latter states the existence of $\PIT_{o}$ cannot be derived i$ $(\exists^{2})$. 
As a result, the behaviour of $\PIT_{o}$ in RM diverges \emph{completely} from its computability-theoretic behaviour.  
\begin{thm}\label{ofinterest}
The system $\ACAo+\QFAC^{0,1}$ proves $\PIT_{o}$.  
\end{thm}
\begin{proof}
Recall that $\ACA_{0}$ is equivalent to various convergence theorems by \cite{simpson2}*{III.2}, i.e.\ $\ACAo$ proves that a sequence in Cantor space has a convergent subsequence.  
Now let $F, G$ be such that $\LOC(F, G)$ and suppose $F$ is unbounded, i.e.\ $(\forall n^{0})(\exists \alpha\leq 1)(F(\alpha)>n)$.  Applying $\QFAC^{0,1}$, we get a sequence $\alpha_{n}$ in Cantor space such that $(\forall n^{0})(F(\alpha_{n})>n)$.
By the previous, the sequence $\alpha_{n}$ has a convergent subsequence, say with limit $\beta\leq_{1}1$.  By assumption, $F$ is bounded by $G(\beta)$ in $[\overline{\beta}G(\beta)]$, which contradicts the fact that $F(\alpha_{n})$ becomes 
arbitrarily large close enough to $\beta$.   
\end{proof}
We show that $\PIT_{o}\asa \WKL$ in Corollary \ref{ofmoreinterest}.  On one hand, for conceptual reasons\footnote{The $\ECF$-translation is discussed in the context of $\RCAo$ in \cite{kohlenbach2}*{\S2}.  Applying $\ECF$ to $\PIT_{o}$, we obtain a sentence equivalent to $\WKL_{0}$, and hence $\PIT_{o}$ has the first-order strength of $\WKL$.}, $\PIT_{o}$ cannot be stronger than $\WKL$ in terms of first-order strength.  On the other hand, reflection upon the previous proof suggests that any proof of $\PIT_{o}$ has to involve $\ACAo$.  Thus, the aforementioned equivalence is surprising.    

\subsection{Pincherle's theorem and uncountable Heine-Borel}
We establish that Pincherle's theorem $\PIT_{\u}$ and Heine-Borel $\HBU$ are equivalent; note that the base theory in the following theorem is $\Pi_{2}^{1}$-conservative over $\ACA_{0}$ by \cite{yamayamaharehare}*{Theorem 2.2}.        
\begin{thm}\label{krooi} 
The system $\ACA_{0}^{\omega}+\QFAC$ proves 
\be\label{worski}
\HBU_{\c}\asa \HBU\asa (\exists \Theta)\SCF(\Theta)\asa \PIT_{\u}\asa (\exists M)\PR(M).
\ee
\end{thm}
\begin{proof}
The first two equivalences in \eqref{worski} are in \cite{dagsamIII}*{Theorem 3.3}, while $\HBU_{\c}\di \PIT_{\u}$ may be found in Theorem \ref{mooi}. 
By \cite{dagsamIII}*{\S2.3}, $\Theta$ as in $\SCF(\Theta)$ computes a finite sub-cover on input an open cover of Cantor space (given by a type two functional); hence $(\exists \Theta)\SCF(\Theta)\di (\exists M)\PR(M)$ follows in the same was as for $\HBU_{\c}\di \PIT_{\u}$ in the proof of Theorem \ref{mooi}.  
Finally, the implication $(\exists M)\PR(M)\di \PIT_{\u}$ is trivial, and we now prove the remaining implication $\PIT_{\u}\di \HBU_{\c}$ in $\ACAo+\QFAC$.  %\marginpar{\footnotesize{Do we only use $\ACAo$ here, and no choice?  I think so, but like you to double-check.}}
To this end, fix $G^{2}$ and let $N_{0}$ be the bound provided by $\PIT_{\u}$.  
We claim:
\be\label{contrje}
(\forall f\leq 1)(\exists g\leq 1)(G(g)\leq N_{0}\wedge f\in[ \overline{g}G(g)]).
\ee
Indeed, suppose $\neg\eqref{contrje}$ and let $f_{0}$ be such that $(\forall g\leq 1)( f_{0}\in [\overline{g}G(g)]\di G(g)> N_{0})$.
Now use $(\exists^{2})$ to define $F_{0}^{2}$ as follows: $F_{0}(h):=N_{0}+1$ if $h=_{1}f_{0}$, and zero otherwise.   
By assumption, we have $\LOC(F_{0}, G)$, but clearly $F(f_{0})>N_{0}$ and $\PIT_{\u}$ yields a contradiction.  
Hence, $\PIT_{\u}$ implies \eqref{contrje}, and the latter provides a finite sub-cover for the canonical cover $\cup_{f\leq 1}[\overline{f}G(f)]$.  
Indeed, apply $\QFAC^{1,1}$ to \eqref{contrje} to obtain a functional $\Phi^{1\di 1}$ providing $g$ in terms of $f$.  
The finite sub-cover (of length $2^{N_{0}}$) then consists of all $\Phi(\sigma*00\dots)$ for all binary $\sigma$ of length $N_{0}$.  
\end{proof}
By the previous proof, a Pincherle realiser $M$ provides an upper bound, namely $2^{M(G)}$, for the size of the finite sub-cover of the canonical cover of $G$, but the contents 
of that cover is not provided (explicitly) in terms of $M$. 
%\marginpar{\footnotesize{I think it is better to say that $M$ gives us an upper bound for the length of a finite subcover, namely $2^{M(G)}$, of the cover provided by $G$.}} 
This observed difference between the special fan functional $\Theta$ and Pincherle realisers also supports the conjecture that $\Theta$ is not computable in any PR as in Conjecture \ref{corkes}.

\smallskip

The previous theorem is of historical interest: Hildebrandt discusses the history of the Heine-Borel theorem in \cite{wildehilde} and qualifies Pincherle's theorem as follows.
\begin{quote}
Another result carrying within it the germs of the Borel Theorem is due to S. Pincherle [\dots] (\cite{wildehilde}*{p.\ 424})
\end{quote}
The previous theorem provides evidence for Hildebrandt's claim, while the following two corollaries provide a better result, for $\PIT_{\u}$ and $\PIT_{o}$ respectively.  
%The equivalence between $\PIT_{\u}$ and $\HBU_{\c}$ can be obtained
%over a weaker base theory, as follows.  
\begin{cor}\label{eessje}
The system $\RCAo+\QFAC^{0,1}$ proves $\PIT_{\u}\asa \HBU_{\c}\asa \HBU$.
\end{cor}
%\marginpar{\footnotesize{Why is this corollary not just a subclaim of Theorem \ref{krooi}, where we did not seem to need even $\QFAC^{0,1}$ in the proof of this particular equivalence.}}
\begin{proof}
The reverse implications are immediate (over $\RCAo$) from Theorem~\ref{mooi}.  For the first forward implication, $\PIT_{o}$ readily implies $\WKL$ as follows: If a tree $T\leq_{1}1$ has no path, i.e.\ $(\forall f\leq 1)(\exists n)(\overline{f}n\not \in T)$, then using quantifier-free induction and $\QFAC^{1,0}$, there 
is $H^{2}$ such that $(\forall f\leq 1)(\overline{f}H(f)\not \in T)$ and $H(f)$ is the least such number.  Clearly $H^{2}$ is continuous on Cantor space and has itself as a modulus of continuity.   
Hence, $H^{2}$ is also locally bounded, with itself as a realiser for this fact.  By $\PIT_{o}$, $H$ is bounded on Cantor space, which yields that $T\leq 1$ is finite.   

\smallskip

Secondly, if we have $(\exists^{2})$, then the (final part of the) proof of Theorem \ref{krooi} goes through by applying $\QFAC^{0,1}$ to 
\be\label{contrje2}
(\forall \sigma^{0^{*}}\leq 1)(\exists g\leq 1)(|\sigma|=N_{0} \wedge G(g)\leq N_{0}\wedge \sigma\in[ \overline{g}G(g)]).
\ee
rather than using \eqref{contrje}.  
On the other hand, if we have $\neg(\exists^{2})$, then \cite{kohlenbach2}*{Prop.~3.7} yields that all $G^{2}$ are continuous on Baire space.  
Since $\WKL$ is given, \cite{kohlenbach4}*{4.10} implies that all $G^{2}$ are uniformly continuous on Cantor space, and hence have an upper bound there.  The latter immediately 
provides a finite sub-cover for the canonical cover of $G^{2}$, and $\HBU_{\c}$ follows.  Since we are working with classical logic, we may conclude $\HBU_{\c}$ by invoking the law of excluded middle $(\exists^{2})\vee \neg(\exists^{2})$.

\smallskip

Finally, $\HBU_{c}\di \HBU$ was proved over $\RCAo$ in \cite{dagsamIII}*{Theorem 3.3}. 
\end{proof}
The previous proof suggests the RM of $\HBU$ is rather robust: given a theorem $\T$ such that $[\T+(\exists^{2})]\di \HBU\di \T\di \WKL$, we `automatically' obtain $\HBU\asa \T$ over the same base theory, using the previous `excluded middle trick'. 

\smallskip

Note that $\QFAC^{0,1}$ is interesting in its own right as it is exactly what is needed to prove the \emph{pointwise} equivalence
%\marginpar{\footnotesize{equivalent $\rightarrow$ equivalence?}} 
between epsilon-delta and sequential continuity for Polish spaces, i.e.\ $\textsf{ZF}$ alone does not suffice (See \cite{kohlenbach2}*{Rem.\ 3.13}).  Furthermore, the previous proof provides a method for `upgrading' a result 
$\RCAo+\WKL+\X\not\vdash \HBU_{\c}$ to $\RCAo+(\exists^{2})+\X\not\vdash \HBU_{\c}$, for any classical axiom $\X$.   Note the lower types in $\WKL$ compared to $(\exists^{2})$.   
The following corollary follows by the same method.  % the latter is of great conceptual importance, as discussed in Section~\ref{disjunkie}.
%NEWNEWNEW
\begin{cor}\label{ofmoreinterest}
The system $\RCAo+\QFAC^{0,1}$ proves $\WKL\asa\PIT_{o}$.  
\end{cor}
\begin{proof}
The reverse direction is immediate by the first part of the proof of the previous corollary.  
For the forward direction, working in $\RCAo+\QFAC^{0,1}+\WKL$, first assume $(\exists^{2})$ and note that Theorem \ref{ofinterest} yields $\PIT_{o}$ in this case. 
Secondly, again working in $\RCAo+\QFAC^{0,1}+\WKL$, assume $\neg(\exists^{2})$ and note that all functions on Baire space are continuous by \cite{kohlenbach2}*{Prop.\ 3.7}.  
Hence, $\HBU_{\c}$ just follows from $\WKL$ as the latter suffices to prove that a continuous function is uniformly continuous (and hence bounded) on Cantor space (\cite{kohlenbach4}*{Prop.~4.10}).  
By Theorem \ref{mooi}, we obtain $\PIT_{\u}$, and hence $\PIT_{o}$.
%But $\cup_{q\in \Q}I_{q}^{\Psi}$ is a countable sub-cover of the canonical sub-cover for \emph{continuous} $\Psi:\R\di \R^{+}$, and hence $\LIND$ follows.  
The law of excluded middle $(\exists^{2})\vee \neg (\exists^{2})$ now yields the forward direction, and we are done.    
%But $\cup_{q\in \Q}I_{q}^{\Psi}$ is a countable sub-cover of the canonical sub-cover for \emph{continuous} $\Psi:\R\di \R^{+}$, and hence $\LIND$ follows.  The law of excluded middle $(\mu^{2})\vee \neg (\mu^{2})$ now finishes this part. 
\end{proof}
By the \emph{low basis theorem} (\cite{simpson2}*{VIII.2.16}), a binary tree $T$ has a path $\alpha$ which is \emph{low} relative to the tree, i.e.\ the path $\alpha$ is such that the Turing jump of $\alpha$ is computable from the Turing jump; the previous are statements in \emph{classical} recursion theory.  In the case of $\PIT_{o}$, a similar result is out of the question: the functions $f, g$ in \eqref{DOG} are extremely hard to compute from the inputs $F, G$ by Theorem \ref{comb3}.   
%%%
%Next, we show that the axiom of choice is not needed in the previous two corollaries.  
%\begin{cor}
%$\RCAo+(\kappa_{0}^{3})$ proves $\PIT_{\u}\asa \HBU_{\c}\asa \HBU$ and $\WKL\asa \PIT_{o}$.
%\end{cor}
%\begin{proof}
%Working in $\RCAo+(\kappa_{0}^{3})$, if $(\exists^{2})$ holds, then we have $(\exists^{3})$, and all theorems are outright provable by Theorem \ref{}.  
%In case $\neg(\exists^{2})$, 
%\end{proof}

\smallskip

Finally, one further improvement of Theorem \ref{krooi} is possible, using the \emph{fan functional} as in \eqref{FF}, where `$Y^{2}\in \textsf{cont}$' means that $Y$ is continuous on $\N^{\N}$.
\be\tag{$\textsf{\textup{FF}}$}\label{FF}
(\exists \Phi^{3})(\forall Y^{2}\in \textsf{\textup{cont}})(\forall f, g\in C)(\overline{f}\Phi(Y)=\overline{g}\Phi(Y)\di Y(f)=Y(g)).
\ee
Note that the previous two corollaries only dealt with third-order objects, while the following corollary \emph{connects} third and fourth-order objects.
\begin{cor}
The system $\RCAo+\textsf{\textup{FF}}+\QFAC$ proves $\HBU_{\c}\asa (\exists \Theta)\SCF(\Theta)$.
\end{cor}
\begin{proof}
We only need to prove the forward implication. 
Working in $\RCAo+\textsf{\textup{FF}}$, assume $(\exists^{2})$ and note that the forward implication follows from Theorem \ref{krooi}.    
In case of $\neg(\exists^{2})$, 
%Finally, $\HBU\di \HBU_{\c}$ is readily proved \emph{given} $(\exists^{2})$, since the latter provides a functional which (uniformly) converts real numbers into their binary representation(s).  
%Moreover, in case $\neg(\exists^{2})$ 
all functions on Baire space are continuous by \cite{kohlenbach2}*{Prop.\ 3.7}.  
Hence, $\Phi(Y)$ from $\textsf{FF}$ provides a modulus of uniform continuity for \emph{any} $Y^{2}$.
The special fan functional $\Theta$ is then defined as outputting the finite sequence of length $2^{\Phi(Y)}$ consisting of all sequences $\sigma*00$ for binary $\sigma$ of length $\Phi(Y)$.
\end{proof}
The base theory in the previous corollary is a (classical) conservative extension of $\WKL_{0}$ by \cite{kohlenbach2}*{Prop.\ 3.15}, which is a substantial improvement over the base theory $\ACAo$ from \cite{dagsamIII}*{Theorem 3.3}.  
One proves $ \PIT_{\u}\asa (\exists M)\PR(M)$ over the same base theory, and the same hold for Theorem \ref{napjeir}.  We can also improve Corollary \ref{comb1}.  % as follows.
\begin{cor}\label{comb1339}
The system $\RCAo+\FF+\QFAC^{2,1}$ proves $ (\exists M)\WPR(M)\asa (\kappa_{0}^{3})$.  
\end{cor}
\begin{proof}
Use $(\exists^{2})\vee \neg(\exists^{2})$ and Corollary \ref{comb1}.  
\end{proof}
Hence, WPRs amount to little more than the known functional, namely $\kappa_{0}$ which was essentially introduced in \cite{dagsam}.  
Finally, we use the above `excluded middle trick' in the context of the axiom of extensionality on $\R$.  
%Can we prove $[FF+\HBU_{\c}]\asa (\exists \Theta)\SCF(\Theta)$,  i.e. does Theta yield the (classical) fan functional?
%KNEWKNEW
 \begin{rem}[Real extensionality]\label{kloti}\rm
 The trick from the previous proofs involving $(\exists^{2})\vee\neg(\exists^{2})$ has another interesting application, namely that $\HBU$ does not really change if we drop the extensionality condition \eqref{RE} from Definition \ref{keepintireal} for $\Psi^{1\di 1}$.  In particular, $\RCAo$ proves $\HBU\asa \HBU^{+}$, where the latter is $\HBU$ generalised to \emph{any} functional $\Psi^{1\di 1}$ such that $\Psi(f)$ is a positive real, i.e.\ $\Psi^{1\di 1}$ need not satisfy \eqref{RE}.  To prove $\HBU\di \HBU^{+}$, note that $(\exists^{2})$ yields a functional $\xi$ which converts $x\in [0,1]$ to a unique binary representation $\xi(x)$, choosing $\sigma*00\dots$ if $x$ has two binary representations; then $\lambda x.\Psi(\r(\xi(x)))$ with $\r(\alpha):=\sum_{n=0}^{\infty}\frac{\alpha(n)}{2^{n}}$ satisfies \eqref{RE} restricted to $[0,1]$, even if $\Psi^{1\di 1}$ does not, and we have $\HBU\di \HBU^{+}$ assuming $(\exists^{2})$.  In case of $\neg(\exists^{2})$, all functionals on Baire space are continuous by \cite{kohlenbach2}*{Prop~3.7}, and $\HBU\di \WKL$ yields that all functions on Cantor space are uniformly continuous (and hence bounded).  
Now, consider $\Psi$ as in $\HBU^{+}$ and note that for $\lambda\alpha.\Psi(\r(\alpha))$ there is $n_{0}\in \N$ such that $(\forall \alpha \in C)(\Psi(\r(\alpha))> \frac{1}{2^{n_{0}}})$.  Hence, the canonical cover of $\Psi$ has a finite sub-cover consisting of $\r(\sigma_{i}*00\dots)$ where $\sigma_{i}$ is the $i$-th binary sequence of length $n_{0}+1$.  
 i.e.\ $\HBU\di \HBU^{+}$ follows in this case. % and the law of excluded middle finishes this proof.  
 \end{rem}
\subsection{Subcontinuity and Pincherle's theorem}\label{pitche}
We sketch an equivalent version of Pincherle's theorem based on an existing notion of continuity, called \emph{subcontinuity}.  
As it happens, subcontinuity is actually used in (applied) mathematics in various contexts: see e.g.\ \cite{gordon3}*{\S4.7}, \cite{migda}*{\S14.2}, \cite{mizera}*{p.\ 318}, and \cite{lola}*{\S4}.  

\smallskip

First of all, in a rather general setting, local boundedness is equivalent to the notion of \emph{subcontinuity}, introduced by Fuller in \cite{voller}.  
The equivalence between subcontinuity and local boundedness (for first-countable Haussdorf spaces $X$ and functions $f:X\di \R$) may be found in \cite{roykes}*{p.\ 252}.  For the purposes of this paper, we restrict ourselves to $I\equiv [0,1]$, which simplifies the definition.  
\bdefi[Subcontinuity]
A function $f:\R\di \R$ is \emph{subcontinuous on $I$} if for any sequence $x_{n}$ in $I$ convergent to $ x\in I$, $f(x_{n})$ has a convergent subsequence.  
\edefi
Secondly, the equivalence between subcontinuity and local boundedness (without realisers) can then be proved as in Theorem \ref{forgu}.  The weak base theory in the latter constitutes a surprise: subcontinuity 
has a typical `sequential compactness' flavour, while local boundedness has a typical `open-cover compactness' flavour.  The former and the latter are classified in the RM of resp.\ $\ACA_{0}$ and $\WKL$ ($\HBU$).      
 % (compared to e.g.\ $(\exists^{3})$ needed to prove $\HBU$).  
\begin{thm}\label{forgu}
The system $\RCAo+\QFAC^{0,1}$ proves that a function $f:\R\di \R$ is locally bounded on $I$ if and only if it is subcontinuous on $I$. 
\end{thm}
\begin{proof}
We establish the equivalence in $\RCAo+\QFAC^{0,1}$ in two steps: first we prove it assuming $(\exists^{2})$ and then prove it again assuming $\neg(\exists^{2})$.  
The law of excluded middle as in $(\exists^{2})\vee \neg(\exists^{2})$ then yields the theorem.  

\smallskip

Hence, assume $(\exists^{2})$ and suppose $f:\R\di \R$ is subcontinuous on $I$ but not locally bounded.  The latter assumption implies that there is $x_{0}\in I$ such that 
\be\label{contrake}\textstyle
{(\forall n^{0})(\exists x\in I)(|x-x_{0}|<_{\R}\frac{1}{n+1}\wedge |f(x)|>_{\R}n)}.  
\ee
Both conjuncts in \eqref{contrake} are $\Sigma_{1}^{0}$-formula, i.e.\ we may apply $\QFAC^{0,1}$ to \eqref{contrake} to obtain $\Phi^{0\di 1}$ such that for $y_{n}:=\Phi(n)$ and $x_{0}$ as in \eqref{contrake}, we have 
\be\label{missyoumuch}\textstyle
(\forall n\in \N)(|y_{n}-x_{0}|<_{\R}\frac{1}{n+1}\wedge |f(y_{n})|>_{\R}n), 
\ee
Clearly $y_{n}$ converges to $x_{0}$, and hence for some function $g:\N\di \N$, the subsequence $f(y_{g(n)})$ converges to some $y\in \R$ by the subcontinuity of $f$.  
However, $f(y_{g(n)})$ also grows arbitrarily large by \eqref{missyoumuch}, a contradiction, and the reverse implication follows.  

\smallskip

Next, again assume $(\exists^{2})$; for the forward implication, suppose $f$ is locally bounded and let $y_{n}$ be a sequence in $I$ convergent to $x_{0}\in I$.  
Then there is $k\in \N$ such that for all $y\in B(x_{0}, \frac{1}{k})$, $|f(y)|\leq k$.  However, for $n$ large enough, $y_{n}$ lies in $B(x, \frac{1}{k})$, implying that $|f(y_{n})|\leq k$ for $n$ large enough.  
In other words, the sequence $f(y_{n})$ eventually lies in the interval $[-k, k]$, and hence has a convergent subsequence by $(\exists^{2})$ and \cite{simpson2}*{I.9.3}.
Thus, $f$ is subcontinuous, and we are done with the case $(\exists^{2})$.

\smallskip

Finally, in case that $\neg(\exists^{2})$, any function $f:\R\di \R$ is everywhere sequentially continuous and everywhere $\eps$-$\delta$-continuous by \cite{kohlenbach2}*{Prop.\ 3.12}.  
Hence, any $f:\R\di \R$ is also subcontinuous on $I$ and locally bounded on $I$, and the equivalence from the theorem is then trivially true.  
\end{proof}
%``''Mention Dag's result about realisers for $\PIT_{o}$

\subsection{Heine's theorem, Fej\'er's theorem, and uncountable Heine-Borel}\label{myheinie}
We prove that the uniform versions of Heine's theorem from Section \ref{heineke} are equivalent to $\HBU_{\c}$.
We prove similar results for Heine's theorem \emph{for the unit interval} and the related \emph{Fej\'er's theorem}.
The latter states that for continuous $f:\R\di \R$, the Ces\`aro mean of the partial sums of the Fourier series uniformly converges to $f$. 

\smallskip
 
We first obtain the following intermediate result. 
\begin{thm}\label{BIGC}
The system $\RCAo+\QFAC^{0,1}$ proves $\UCT_{\u}'\di  \HBU_{\c}$.  
\end{thm}
\begin{proof}
%The reverse implication follows as in the proof of Corollary \ref{horkuolo}.
%For the forward implication, 
Fix $G^{2}$ and let $m=N_{0}-1$ be the number provided by $\UCT_{\u}'$.  Suppose \eqref{contrje} is false, i.e.\ for some $f_{0}\leq 1$, we have $(\forall g\leq 1)( f_{0}\in[ \overline{g}G(g)] \di G(g)>N_{0})$, implying $G(f_{0})\geq N_{0}+1$.   
Define $\alpha_{0}:\N\di \N$ as : 
\[
\alpha_{0}(\sigma):=
\begin{cases}
2 & \textup{if } |\sigma|\geq N_{0}\wedge \overline{\sigma}N_{0}=\overline{f_{0}}N_{0}\\
0 & \textup{if } |\sigma|< N_{0}\wedge \sigma=\overline{f_{0}}|\sigma| \\
1 & \textup{otherwise}
\end{cases}
\]
Clearly, we have $\alpha_{0}\in K_{0}$ (in fact $\alpha(\sigma)>0$ if $|\sigma|\geq N_{0}$), and \textbf{if} $\alpha_{0}$ and $G$ satisfy the antecedent of $\UCT_{\u}'$, we obtain a contradiction as $\alpha_{0}(\overline{f_{0}}(N_{0}-1)*0)\ne \alpha_{0}(\overline{f_{0}}(N_{0}-1)*1)$ by definition.    
Hence, \eqref{contrje} must hold and the proof of Corollary \ref{eessje} now readily yields $\HBU_{\c}$.  What remains to be shown is that $\MPC(G, \alpha_{0})$, for which we distinguish the following cases.  
First of all, $\MPC(G, \alpha_{0})$ holds for $f=f_{0}$ since $G(f_{0})\geq N_{0}+1$ and hence $\alpha(f_{0})=2=\alpha(g)$ for any $g$ such that $\overline{f_{0}}G(f_{0})=\overline{g}G(f_{0})$.
Secondly, if $f(0)\ne f_{0}(0)$, we have $\alpha({f})=\alpha({g})=1$ for any $g$ such that $f(0)=g(0)$, i.e.\ $\MPC(G, \alpha_{0})$ also holds in this case.  
Thirdly, if $\overline{f}n=\overline{f_{0}}n$ for some $n\leq N_{0}$, then also $G(f)>n$; indeed $G(f)\leq n$ would imply $f_{0}\in [\overline{f}G(f)]$, and hence $G(f)\geq N_{0}+1$ by the assumption on $f_{0}$, yielding the contradiction $N_{0}\geq N_{0}+1$.  Hence, if $\overline{f}N_{0}=\overline{f_{0}}N_{0}$ but $f\ne f_{0}$ then $G(f)\geq N_{0}$, implying that $\alpha(f)=2=\alpha(g)$ for any $g\leq 1$ such that $\overline{f}G(f)=\overline{g}G(f)$.  
Similarly, if $\overline{f}n=\overline{f_{0}}n$ but $\overline{f}(n+1)\ne\overline{f_{0}}(n+1)$ for $n<N_{0}$, then $G(f)> n$.  By the latter, $\alpha(f)=1=\alpha(g)$ for any $g\leq 1$ such that $\overline{f}G(f)=\overline{g}G(f)$.  Hence $\MPC(G, \alpha_{0})$ follows, and we are done. 
%
%\smallskip
%
%
%Hence, \eqref{XYX} also holds if $f$ shares an initial segment with $f_{0}$, and we have $\MPC(G, \alpha_{0})$.  
%Finally, \eqref{XYX} holds for $f $ such that $\overline{f_{0}}N_{0}=\overline{f}N_{0}$. 
\end{proof}
\begin{cor}\label{roofer}
The system $\RCAo+\QFAC^{0,1}$ proves $\UCT_{\u}'\asa \UCT_{\u}\asa  \HBU_{\c}$.  
\end{cor}
\begin{proof}
The implication $\HBU_{\c}\di \UCT_{\u}$ follows by applying $\HBU_{\c}$ to the canonical cover associated to $G^{2}$ from $\UCT_{\u}'$ and taking the maximum of $G$ evaluated at the points in the finite sub-cover. 
We now prove $\UCT_{\u}\di \UCT_{\u}'$, and the corollary then follows from the theorem.  
Let $G^{2}$ and $m^{0}$ be as in $\UCT_{\u}'$.  Now for $\alpha \in K_{0}$ such that $\MPC(G, \alpha)$, apply $\QFAC^{1,0}$ to $(\forall f^{1})(\exists n^{0})(\alpha(\overline{f}(n+1))>0\wedge \alpha(\overline{f}n)=0)$ to obtain $H^{2}$ which computes such $n^{0}$ in terms of $f^{1}$. 
Define $F(f):= \alpha(\overline{f}(H(f)+1))-1$ and note that $(\forall f\leq 1)(\forall m\geq G(f))(\alpha(\overline{f}m)=F(f)+1)$ by assumption.  Hence, $F$ is continuous with modulus of continuity $G$, implying that $m^{0}$ is a modulus of \emph{uniform} continuity for $F$ by $\UCT_{\u}'$.  
But this implies $(\forall f^{1})(\exists n^{0}\leq m)(\alpha(\overline{f}n)>0)$, i.e.\ $m$ is also a modulus of uniform continuity for $\alpha$, and we are done. 
\end{proof}
An alternative proof of `$\UCT_{\u} \di \HBU_{\c}$'  is as follows.  This argument also shows that the same type three functionals may serve as realisers for $\PIT_{\u}$ and $\UCT_{\u}$.

\begin{proof}
This proof is based on that of $\PIT_{\u} \di \HBU_{\c}$: For fixed $G$, let $N$ be as in $\UCT_{\u}$. Then for $g \leq 1$ there is $f \leq 1$ such that $G(f) \leq N$ and $\bar g(G(f)) = \bar f(G(f))$.
Because, if this is not the case, there is a binary sequence $s$ of length $N$ such that for all $f$ extending $s$ we have that $G(f) > N$. Then we can define $F(f) = 0$ if $f$ does not extend $s$ and $F(f) = f(N)$ if $f$ extends $s$. Then $F$ has a modulus of continuity given by $G$, but not a modulus of uniform continuity given by $N$. 
\end{proof}
The previous results establish the equivalence between the uniform version of Heine's theorem and the Heine-Borel theorem for uncountable covers, \emph{in the case of Cantor space}.  
One similarly proves the equivalence between the Heine-Borel theorem $\HBU$ and uniform Heine's theorem \emph{for the unit interval}, as follows.  
\begin{princ}[$\UCT_{\u}^{\R}$]
For any $\eps>_{\R}0$ and $g:(I\times \R)\di \R^{+}$, there is $\delta>_{\R}0$ such that for any $f:I\di \R$ with modulus of continuity $g$, we have
\[
(\forall x, y\in I)(|x-y|<_{\R}\delta)\di |f(x)-f(y)|<_{\R}\eps ).
\]
\end{princ}
%\begin{dagcom}\rm From this point, and through the rest of Section 4, I do not really see why I should be a co-author of what is written there. This seems to be pure reverse mathematics, of which I may check the correctness but have no mathematical influence. I do not think that questions of relative computability for a bunch of other cases, how interesting they might be from a RM point of view, are worth investigating until some more basic problems, like the comparison between $\Theta$ and $M_u$ , are settled. This said, I accept that  the results are of interest in a RM-setting.  We discuss this further by e-mail. \bf Endcomment \end{dagcom}
We shall prove that $\UCT_{\u}^{\R}$ is equivalent to the uniform version of Fej\'er's theorem.  
We follow the approach in \cite{kohabil}*{p.\ 65} and we define $I_{\pi}\equiv[-\pi, \pi]$.
\bdefi\label{popolop}
Define $\sigma_{n}(f, x) :=\frac{1}{n} \sum_{k=0}^{n-1}S(k, f, x)$, where
$S(n,f,x):= \frac{a_{0}}{2} +  \sum_{k=1}^{n} ( a_{k} \cdot \cos(kx)+b_{k} \cdot \sin(kx) )$ and
$ a_{k} := \frac{1}{\pi}\int_{-\pi}^{\pi}   f (t)\cos(kt)dt, b_{k} :=\frac1\pi \int_{-\pi}^{\pi}   f (t) \sin(kt)dt$.
\edefi
Note that Fej\'er's theorem already deals with \emph{uniform} convergence, i.e.\ the notion of 
convergence in $\FEJ_{\u}$ below is `super-uniform' in that it only depends on the modulus of continuity for the function.  
%\begin{princ}[$\FEJ_{\u}$]
%For any $k\in\N$ and $g:(I_{\pi}\times \R)\di \R^{+}$, there is $N\in \N$ such that for any $f:I_{\pi}\di \R$ with modulus of continuity $g$ and $\|f\|_{\infty}\leq 1$, we have
%\[\textstyle
%(\forall n\geq N, x\in I_{\pi})(|\sigma_{n}(f, x)-f(x)|<\frac1k).
%%(\forall x, y\in I)(|x-y|<_{\R}\delta)\di |f(x)-f(y)|<_{\R}\eps ).
%\]
%\end{princ}
\begin{princ}[$\FEJ_{\u}$]
For any $k\in\N$ and $g:(I_{\pi}\times \R)\di \R^{+}$, there is $N\in \N$ such that for any $f:I_{\pi}\di \R$ with modulus of continuity $g$ and $f(0)=0$, we have
\be\label{trop}\textstyle
(\forall n\geq N, x\in I_{\pi})(|\sigma_{n}(f, x)-f(x)|<\frac1k) \wedge (\forall y\in I_{\pi}, n\in \N)(|\sigma_{n}(f, x)|\leq nN).
%(\forall x, y\in I)(|x-y|<_{\R}\delta)\di |f(x)-f(y)|<_{\R}\eps ).
\ee
\end{princ}
Note that functions like $\sin x$ and $e^{x}$ can be defined in $\RCAo$ by \cite{simpson2}*{II.6.5}, 
%Let $\WWKL$ be $\WKL$ restricted to trees of positive measure (See \cite{simpson2}*{X.1}) and 
while $\WKL$ is needed to make sure $\sigma_{n}$ makes sense by \cite{simpson2}*{IV.2.7}.
\begin{thm}\label{fejerpejerpotloodgat}
The system $\RCAo+\WKL$ proves $\UCT_{\u}^{\R}\asa \FEJ_{\u}$.
\end{thm}
\begin{proof} 
For the forward implication, the modulus of uniform convergence $\Psi$ for Fej\'er's theorem from \cite{kohabil}*{p.\ 65} is
\[
\Psi(f, k):=48(k+1)\cdot \|f\|_{\infty} \cdot(\omega_{f}(2(k+1))+1)^{2},
\]
for a modulus of uniform continuity $\omega_{f}:\N\di \N$ for $f$.  Note that we can replace $\|f\|_{\infty}$ by $16\omega_{f}(1)$ if $f(0)=0$.  
Due to the high level of uniformity of $\Psi$, the first conjunct of \eqref{trop} immediately follows from $\UCT_{\u}^{\R}$.  For the the second conjunct of \eqref{trop}, fix $g$ and apply $\UCT_{\u}^{\R}$ for $\eps=1$ to obtain $\delta_{1}$ as in the latter.  
Now note that any $f$ such that $f(0)=0$ and $g$ is a modulus of continuity for $f$, we have $(\forall x\in I_{\pi})(|f(x)|\leq N)$ where $N=\lceil \frac{2\pi}{\delta_{1}}\rceil$.  Intuitively, this $N$ is a `uniform' bound for $f$ that only depends on a modulus 
of continuity for the latter.  By definition, this also yields a uniform bound for $\sigma_{n}(f, x)$ (in terms of $n$ and $N$ only). 

\smallskip

For the reverse implication, note that $\UCT_{\u}^{\R}$ does not change if we additionally require $f(0)=0$, since we can consider $f_{0}(x):=f(x)-f(0)$, which has the same modulus of continuity as $f$.  
%note that functions like $\sin x$ and $e^{x}$ can be defined in $\RCAo$ by \cite{simpson2}*{II.6.5} and that
%this system proves their basic properties.  As a result, we may assume $\sigma_{n}(f, x)$ (for any fixed $n$ and continuous $f:I\di \R$) is uniformly continuous on $I_{\pi}$.  
%Hence, $\FEJ_{\u}$ immediately implies that 
Now fix $g$ as in $\UCT_{\u}^{\R}$, fix $x, y\in I_{\pi},\eps>0$ and consider 
\be\label{argo}
|f(x)-f(y)|\leq |f(x)-\sigma_{n}(f,x)|+|\sigma_{n}(f, x)-\sigma_{n}(f, y)|+|f(y)-\sigma_{n}(f, y)|
\ee
for $f$ with $g$ as modulus of continuity and $f(0)=0$.  
%We may assume $\|f\|_{\infty}\leq 1$ as $|f|$ has an upper bound $N\in \N $ on $I_{\pi}$ by \cite{simpson2}*{IV.2.3}, and we can consider $f/N$ instead.  
The first and third part of the sum in \eqref{argo} are both below $\eps/3$ for $n$ large enough.  Such number, with the required independence properties, is provided by $\FEJ_{\u}$.  
Moreover, $\sigma_{n}(f, x)$ is uniformly continuous on $I_{\pi}$ with a modulus which depends on $n$ but not $f$ due to the second conjunct of \eqref{trop}.  
%\emph{by definition} depends on a mod\ulus of uniform continuity for sine and cosine, but not $f$.  
Hence, \eqref{argo} implies that $f$ is uniformly continuous in the sense required by $\UCT_{\u}^{\R}$.
\end{proof}
%As discussed in \cite{opborrelen}*{Note, p.\ 50-51}, Borel actually proves the (countable) Heine-Borel theorem to justify his use of the following lemma: \emph{If $1>_{\R}\sum_{n=0}^{\infty}|a_{n}-b_{n}|$, then $\cup_{n\in \N}(a_{n}, b_{n})$ does not cover $[0,1]$}.  
%However, the latter is equivalent to $\WWKL$ by \cite{simpson2}*{X.1.9}, which provides some historical motivation and context.  
%
%\smallskip  
%
Using a proof similar to that of Corollary \ref{roofer}, we obtain.  
\begin{cor}
 $\RCAo+\WKL+\QFAC^{0,1}$ proves $\UCT_{\u}^{\R}\asa \FEJ_{\u}\asa \HBU$.
\end{cor}
By the previous, realisers for $\FEJ_{\u}$ and  $\UCT_{\u}$ are equi-computable modulo $\exists^2$. When coded as functionals of pure type 3, these realisers are all essentially PRs, modulo a computable scaling.
Similarly, many theorems from analysis yield analogous uniform versions, and there are at least two sources: on one hand, as noted above, the redevelopment of analysis based on techniques from the gauge integral (as in e.g.\ \cite{bartle3}) yields uniform theorems.  % as also discussed in Remark \ref{vraio}.  
On the other hand, as hinted at in the proof of Theorem \ref{fejerpejerpotloodgat}, Kohlenbach's \emph{proof mining} program is known to produce highly uniform results (See e.g.\ \cite{kohlenbach3}*{Theorem~15.1}), which yield uniform versions, like $\FEJ_{\u}$ for Fej\'er's theorem.
We finish this section with some conceptual remarks. 
\begin{rem}[Atsuji spaces]\rm
A metric space $X$ is called \emph{Atsuji} if for any metric space $Y$, any continuous function $f:X\di Y$ is uniformly continuous.
The RM study of {Atsuji spaces} may be found in \cite{withgusto}*{\S4}, and one of the results is that the Heine-Borel theorem for countable covers of $[0,1]$ is equivalent to the latter being Atsuji.
Theorem \ref{BIGC} may be viewed as a generalisation (or refinement) establishing that $\HBU$ is equivalent to $[0,1]$ being `uniformly' Atsuji, i.e.\ as in $\UCT_{\u}$.
\end{rem}
\begin{rem}[Other uniform theorems]\label{flurki}\rm
It is possible to formulate uniform versions (akin to $\PIT_{\u}, \UCT_{\u}$, and $\FEJ_{\u}$) of many theorems.  
For reasons of space, we delegate the study of such theorems to a future publication.  
We point the reader to \cite{gauwdief}*{Example 2} and \cite{taokejes} for `real-world' examples using $\HBU$ by two Fields medallists. 
We also provide the example of \emph{uniform} weak K\"onig's lemma $\WKL_{\u}$:
\[
(\forall G^{2})(\exists m^{0})(\forall T\leq_{1}1)\big[(\forall \alpha \in C)(\overline{\alpha}G(\alpha)\not\in T)\di (\forall \beta \in C)( \overline{\beta}m\not\in T )     \big],
\]
Note that $\WKL_{\u}$ expresses that a binary tree $T$ is finite if it has no paths, \emph{and} the upper bound $m$ only depends on a realiser $G$ of `$T$ has no paths'.  
It is fairly easy to show that $\WKL_{\u}$ is equivalent to $\HBU$ by adapting the proof of Theorem \ref{eessje}.
\end{rem}
\begin{comment}
\subsection{Uniform theorems and weak K\"onig's lemma}\label{pichte2}
In light of the above, Pincherle's theorem can be viewed as a generalisation of a theorem from the RM of $\WKL$, while Heine's theorem belongs to the latter class by \cite{simpson2}*{IV.2}.
It is then a natural question whether we can modify other theorems from the RM of $\WKL$ to theorems equivalent to $\HBU$.  
We shall study some examples in this section, including the fan theorem, \emph{countable} Heine-Borel, and Dini's theorem.  
\subsubsection{Fan theorem}
Our first example makes use of the \emph{fan theorem}, i.e.\ the contraposition of $\WKL$, which is the statement that a tree without paths is also finite.  %i.e.\ the \emph{fan theorem}.  
We reserve the variable `$T^{1}$' for trees, while `$T\leq 1$' denotes that $T$ is a binary tree.  
The following sentence, which we refer to as $\WKL_{\u}$, additionally states that the upper bound in the consequent of the fan theorem
only depends on the choice of realiser for the fact that $T$ has no paths.  
\[
(\forall G^{2})(\exists m^{0})(\forall T\leq_{1}1)\big[(\forall \alpha\leq_{1}1)(\overline{\alpha}G(\alpha)\not\in T)\di (\forall \beta^{0^{*}})(|\beta|= m \di \beta\not\in T )     \big],
\]
The following corollary now follows in the same way as above.  
\begin{cor}\label{horkuolo}
The system $\RCAo+\QFAC^{0,1}$ proves $\WKL_{\u}\asa \HBU_{\c}$.  
\end{cor}
%\marginpar{\footnotesize{The functional realiser for $\WKL_{\u}$ seems to be the same as for Pincherle's theorem, uniform version. What is interesting, is that we need $\QFAC^{0,1}$ to prove $\HBU_{\u}$ from most of its (equivalent) corollaries, but not the other way round. To what extent are these corollaries equivalent without a reference to AC? Recall that even $\HBU_{\c}$ is provable in ZF without choice.}}
 \begin{proof}
For the forward implication, fix $G^{2}$ and let $m=N_{0}-1$ be the number provided by $\WKL_{\u}$.  Suppose \eqref{contrje} is false, i.e.\ for some $f_{0}\leq 1$, we have $(\forall g\leq 1)( f_{0}\in[ \overline{g}G(g)] \di G(g)>N_{0})$, implying $G(f_{0})\geq N_{0}+1$.   
Now define the binary tree $T_{0}$ as follows:  $\sigma\in T_{0}\asa[ \sigma \subset \overline{f_{0}}G(f_{0})\wedge |\sigma|\leq N_{0}]$.      
By definition, we have $\overline{f_{0}}G(f_{0})\not\in T_{0}$ and $\overline{f}G(f)\not \in T_{0}$ if $f(0)\ne f_{0}(0)$.  
In case we have $n\leq N_{0}$ and $\overline{f}n=\overline{f_{0}}n$, then $G(f)>n$; indeed, $G(f)\leq n$ implies $f_{0}\in [\overline{f}G(f)]$, implying $G(f)\geq N_{0}+1$ by assumption, which yields the contradiction $N_{0}\geq N_{0}+1$.  

\smallskip

Hence, if $f$ has an initial segment $\overline{f_{0}}n$ for $n\leq N_{0}$, then $G(f)>n$, and we distinguish two cases: Firstly, if $n<N_{0}$, $\overline{f_{0}}n=\overline{f}n$, and $f(n+1)\ne f_{0}(n+1)$, then $G(f)\geq n+1$ and $\overline{f}(n+1)\not\in T_{0}$.  
Secondly, if $n\geq N_{0}$ and $\overline{f_{0}}n=\overline{f}n$, and $f\ne f_{0}$, then $G(f)\geq N_{0}+1$ and $\overline{f}(N_{0}+1)\not \in T_{0}$. 
As a result, we have established $(\forall \alpha\leq_{1}1)(\overline{\alpha}G(\alpha)\not\in T_{0})$, i.e.\ the antecedent of $\WKL_{\u}$, but $T_{0}$ has a sequence of length at least $N_{0}$, again by definition.  
This contradicts the conclusion of $\WKL_{\u}$, and \eqref{contrje} must hold.  The proof of Corollary \ref{eessje} now readily yields $\HBU_{\c}$.  

\smallskip

For the reverse implication, fix $G^{2}$ and use $\HBU_{\c}$ to obtain a finite sequence $\langle f_{0}, \dots , f_{k}\rangle$ in $C$ which yields a finite sub-cover of the canonical cover $\cup_{f\in C}[\overline{f}G(f)]$.   
The number $m^{0}$ required for $\WKL_{\u}$ is then given by $\max_{i\leq k}G(f_{i})$.
\end{proof}
Similar results can be obtained in the same way: one establishes a version of \eqref{contrje} and derives $\HBU_{\c}$ or $\HBU$ using $\QFAC$.  

\subsubsection{Heine-Borel for countable covers}
Our second example makes use of a `uniform' version of the Heine-Borel theorem for \emph{countable} covers, as follows. 
\bdefi[$\HBC_{\u}$] For any $G:\R\di \N$, there is $m\in \N$ such that for any sequence $(a_{n}^{1}, b_{n}^{1})$ of open intervals, we have
\begin{align}
%(\forall G:\R\di \N)(\exists m^{0})( \forall a_{(\cdot)}^{1}, b_{(\cdot)}^{1})\big[ 
\textstyle(\forall x\in I)\textstyle(\exists i,k\leq G(x))&\textstyle\big(x\in [a_{i}+\frac{1}{k+1}, b_{i}-\frac{1}{k+1}]\big)\textstyle\label{harsh}\\
&\textstyle\di (\forall x\in I)(\exists i, k\leq m)(x\in  [a_{i}+\frac{1}{k+1}, b_{i}-\frac{1}{k+1}])     \big].\notag 
%&\textstyle\di (\forall x\in I)(\exists j,n\leq m)(x\in (a_{j}+\frac{1}{n+1}, b_{j}-\frac{1}{n+1}))     \big].\notag 
\end{align}
\edefi
We say that $G:\R\di \N$ is a \emph{realiser} for $I\subset \cup_{i\in \N}(a_{i}, b_{i})$ if $G$ satisfies the antecedent of \eqref{harsh}.  
Hence, $\HBC_{\u}$ states that given a realiser for $I\subset \cup_{i\in \N}(a_{i}, b_{i})$, the latter has a finite sub-cover, with upper bound \emph{independent} of the cover itself.  
Note that $\HBC_{\u}$ restricted to \emph{continuous} $G$ is equivalent to $\WKL$, i.e.\ the aforementioned independence property is not really that exotic.    
%\marginpar{\footnotesize{Is there a proof for $\HBC_{\u}$ in $\ZFC$? If so, is it equivalent to $\HBU$ in a reasonably weak theory? If the answer to the first question is  'yes'  while the answer to the second one is 'no', this example is interesting.}}
%the antecedent of $\HBC_{\u}$.    
\begin{thm}\label{mooiman}
The system $\RCAo+\QFAC^{0,1}$ proves $\HBC_{\u}\di \HBU$.  
\end{thm}
\begin{proof}
The proof is similar to that of Corollary \ref{eessje}.  
Fix $G:\R\di \N$ and let $N_{0}\geq 2$ be the number provided by $\HBC_{\u}$.  
Now consider:
\be\label{cored}\textstyle
(\forall x\in I)(\exists y\in I)\big(  x\in (y-\frac{1}{G(y)+1}, y+\frac{1}{G(y)+1})\wedge  G(y)\leq N_{0}\big).
\ee
Suppose \eqref{cored} is false, i.e.\ there is $x_{0}\in I $ such that 
\be\label{lel}\textstyle
(\forall y\in I)\big(  x_{0}  \in (y-\frac{1}{G(y)+1}, y+\frac{1}{G(y)+1})\di   G(y)> N_{0}\big), 
\ee
and note in particular that $G(x_{0})\geq N_{0}+1$.  
%Define $c_{0}, d_{0}$ as the midpoint between $x_{0}$ and $x_{0 }\mp \frac{1}{G(x_{0})+1}$.  
If $G(z)$ is $0$ for some $z\in I$, then the canonical cover $\cup_{y\in I}(y-\frac{1}{G(y)+1}, y+\frac{1}{G(y)+1})$ has a trivial two point sub-cover.
If $(\forall z\in I)(G(z)\geq 1)$, define a countable cover $(a_{n}, b_{n})$ of $I$ as follows:  
$(a_{0}, b_{0})$ is $({-1}, x_{0}+\frac{1}{G(x_{0})+1})$, $(a_{1}, b_{1})$ is $(x_{0}-\frac{1}{G(x_{0})+1}, 2)$, $(a_{N_{0}+1}, b_{N_{0}+1})$ is $(x_{0}-\frac{1}{G(x_{0})+1}, x_{0}+\frac{1}{G(x_{0})+1})$, 
%$(a_{0}, b_{0})$ is $({-1}, x_{0}+\frac{1}{G(x_{0})+1})$, $(a_{1}, b_{1})$ is $(x_{0}-\frac{1}{G(x_{0})+1}, 2)$, $(a_{N_{0}+1}, b_{N_{0}+1})$ is $(x_{0}-\frac{2}{G(x_{0})+1}, x_{0}+\frac{2}{G(x_{0})+1})$, 
and all other $(a_{n}, b_{n})$ are $(a_{0}, b_{0})$.  
By definition, we have $(\forall x\in I)(\exists i, k\leq G(x))(x\in [a_{i}+\frac{1}{k+1}, b_{i}-\frac{1}{k+1}])$, but the consequent of $\HBU_{\u}$ is false for $x=x_{0}$ as this real is in 
$[a_{N_{0}+1}+\frac{1}{G(x_{0})+1}, b_{N_{0}+1} -\frac{1}{G(x_{0})+1}]$, but not in any interval $[a_{i}+\frac{1}{k+1}, b_{i} -\frac{1}{k+1}]$ with index $i<N_{0}+1$ and $k\leq N_{0}$.
Hence, \eqref{cored} follows, and immediately implies:
\be\label{tang}\textstyle
(\forall q\in I\cap \Q)(\exists y\in I)\big(  q\in (y-\frac{1}{G(y)+1}, y+\frac{1}{G(y)+1})\wedge  G(y)\leq N_{0}\big).
\ee
The first conjunct in \eqref{tang} is a $\Sigma_{1}^{0}$-formula and using Kohlenbach's `hat' function (\cite{kohlenbach2}*{p.\ 289}), we may view any sequence of naturals as a real number.  
Hence, \eqref{tang} is easily modified into a formula as in the antecedent of $\QFAC^{0,1}$, and let $\Phi^{0\di 1}$ be the resulting functional which produces $y\in I$ from $q\in I\cap \Q$ as in \eqref{tang}.  
The required finite sub-cover of  $\cup_{y\in I}(y-\frac{1}{G(y)+1}, y+\frac{1}{G(y)+1})$ now consists of at most $2 N_{0}+1$ intervals, the first of which is $(-x_{1},x_{1})$ for $x_{1}=\frac{1}{G(\Phi(0))+1}$, the second of which is $(x_{1}-\frac{1}{G(x_{1})+1}, x_{1}+\frac{1}{G(x_{1})+1})$, and so on.  
\end{proof}
There are a number of variations of $\HBC_{\u}$ with similar properties, including:
\[
\textstyle(\forall x\in I)\textstyle(\exists i\leq G(x))\textstyle\big(x\in (a_{i}/2, b_{i}/2)\big)\di (\forall x\in I)(\exists i\leq m)\big(x\in  (a_{i}, b_{i})\big)     \big].
\]
Since the proofs are all similar, we do not go into details.

\subsubsection{Dini's theorem}
Our third example is \emph{Dini's theorem} for the unit interval, which states that if a sequence of continuous $f_{n}:[0,1]\di \R$ converges to the continuous $f:[0,1]\di \R$ in a monotone fashion, then the convergence is uniform. 
Note that Dini's theorem is part of the RM of $\WKL_{0}$, and constructively equivalent to the fan theorem (\cite{diniberg}).  Dini's theorem may be found \cite{dinipi}*{\S97} as \emph{Teorema V}.   

\smallskip

First of all, we consider the the conditions of Dini's theorem with moduli:  
\begin{align} 
 &(\forall n,k\in \N)(\forall x, y\in I)(|x-y|<g_{n}(x, k)\di |f_{n}(x)-f_{n}(y)|<1/k)\tag{{$\textsf{C}_{n}$}}\label{kilo}\\
 &(\forall k\in \N)(\forall x, y\in I)(|x-y|<g(x, k)\di |f(x)-f(y)|<1/k)\tag{{$\textsf{C}$}}\label{kilo2}\\
  &(\forall k\in \N, x\in I)(\forall n\geq h(x, k))( |f_{n}(x)-f(x)|<1/k)\tag{{$\textsf{Conv}$}}\\
 &(\forall n\in \N)(\forall x\in I)(f_{n}(x)\leq f_{n+1}(x)\leq f(x))\tag{{$\textsf{Mon}$}}
 \end{align}
The formulas \eqref{kilo} and \eqref{kilo2} express that $g_{n}$ and $g$ are moduli of continuity for $f_{n}$ and $f$.  
The formulas $(\textsf{Mon})$ and $(\textsf{Conv})$ express that $f_{n}$ converges to $f$ in a monotone fashion with modulus of convergence $h$.
To improve readability, we will not include any typing for the variables $f, f_{n}, g, g_{n}, h$ as above. 
Let $\DIN(f,g, h , f_{n}, g_{n})$ be the conjunction of the above four formulas.  The uniform version of Dini's theorem is:
\begin{thm}[$\DIN_{u}$] For all $k\in \N, g, g_{n},h$, there is $N\in \N$ such that 
\[
(\forall f, f_{n})\big[ \DIN(f,g, h , f_{n}, g_{n})\di (\forall n\geq N, x\in I)(|f(x)-f_{n}(x)|<1/k) \big].
\]
\end{thm}
Note that the number $N\in \N$ from the definition of uniform convergence of $f_{n}$ to $f$ \emph{does not depend} on the latter two variables.  
\begin{thm}
The system $\RCAo+\QFAC^{0,1}$ proves $\DIN_{\u}\asa \HBU$.
\end{thm}
\begin{proof}
For the reverse implication Dini's theorem is proved in \cite{bartle2}*{p.\ 238} and \cite{josting}*{p.\ 157} using $\HBU$; it is immediate that the uniform version $\DIN_{\u}$ is obtained. 
For the forward implication, the proof is similar to that of Theorem \ref{mooiman}.  In particular, one obtains a variation of \eqref{contrje} or \eqref{cored} using $\DIN_{\u}$, and $\HBU$ follows.  
\end{proof}
\end{comment}

\section{A finer analysis: the role of the axiom of choice}\label{finne}
%In the previous section, we studied Pincherle's and Heine's theorem within the framework of higher-order RM. 
%This section is devoted to a detailed study of the proof techniques we used, broken down into two parts.  
%
%\smallskip
%
%First of all, our previous proofs often exploit the law of excluded middle in a curious way.  In Section \ref{disjunkie}, we show that this technique leads to previously thought rare results in RM involving disjunctions and splittings; this technique may also be viewed as a universal/classical version of \emph{Ishihara's tricks}, first introduced in \cite{discihara}.  
Our above proofs often make use of the axiom of countable choice, and its exact role is studied in this section.
%This study turns out to be intimately connected to the \emph{Lindel\"of lemma}.  
\begin{comment}
\subsection{Splittings and disjunctions in Reverse Mathematics}\label{disjunkie}
We discuss how the above proofs give rise to \emph{numerous} instances of two rather rare RM-phenomena, namely 
\emph{splittings} and \emph{disjunctions}, introduced next. 
%\subsubsection{Splittings and disjunctions involving the Heine-Borel theorem}

\smallskip

First of all, as discussed in \cite{dsliceke}*{\S6.4}, there are theorems $A, B, C$ in the RM zoo such that $A\asa (B\wedge C)$, i.e.\ $A$ can be \emph{split} into two independent (fairly natural) parts $B$ and $C$ (over $\RCA_{0}$).  
As to the possibility of $A\asa (B\vee C)$, there is \cite{yukebox}*{Theorem~4.5} which states that a certain theorem about dynamical systems is equivalent to the \emph{disjunction} of $\WKL$ and induction for $\Sigma_{2}^{0}$-formulas; neither disjunct of course implies the other.
Similar results are in \cite{boulanger} for model theory.   
As it happens, Corollary~\ref{eessje} gives rise to numerous such `disjunctive' results. %let $\ACA$ be \emph{arithmetical comprehension} as in \cite{simpson2}*{I.3.2}.   
%Finally, the previous proof establishes results akin to those in \cite{yukebox}, namely that $\RCAo$ proves
\begin{thm}The system $\RCAo$ proves that 
\be\label{corkukkk}
\WKL\asa[(\exists^{2})\vee \HBU_{\c}] \asa [(\exists^{2})\vee \PIT_{\u}]\asa [\X\vee \HBU]. %\asa [(\exists^{2})\vee \PIT_{o}].  
\ee
for any $\X$ such that $\ACA_{0}\di \X\di \WKL_{0}$, where the second implication is strict.  
\end{thm}
\begin{proof}
We prove the first equivalence and note that the other ones follow in the same way by \cite{kohlenbach2}*{Prop.\ 3.7 and 3.12} and the above.  
The reverse implication follows from $\HBU_{\c}\di\WKL$ and $(\exists^{2})\di \ACA_{0}\di \WKL_{0}$.  % is well-known.  
For the forward implication, note that all
functionals on $\N^{\N}$ are continuous given $\neg(\exists^{2})$, and hence uniformly continuous on $C$ by $\WKL$, as in the proof of Corollary~\ref{eessje}.  Hence, all functionals on $C$ have an upper bound, which immediately implies $\HBU_{\c}$.  
The law of excluded middle $(\exists^{2})\vee \neg(\exists^{2})$ finishes the proof.
\end{proof}
\begin{cor}\label{kapaf}
The system $\RCAo+\WKL+(\kappa_{0}^{3})$ proves $\HBU_{\c}$
\end{cor}
\begin{proof}
Immediate from the first equivalence, Theorem \ref{mooi}, and  $[(\exists^{2})+(\kappa_{0}^{3})]\asa (\exists^{3})$; the latter fact follows from \cite{dagsam}*{Rem.\ 6.13}.  
\end{proof}
Secondly, while \eqref{corkukkk} may come across as \emph{spielerei}, $\WKL\asa [\ACA_{0}\vee \HBU]$ is actually of great conceptual importance, as follows.  
\begin{tempie}\label{kcuf}\rm
To prove a theorem $\T$ in $\RCAo+\WKL$, proceed as follows:  
\begin{enumerate}
 \renewcommand{\theenumi}{\alph{enumi}}
\item Prove $\T$ in $\ACA_{0}$ (or even using $\exists^{2}$), which is \emph{much}\footnote{For instance, the functional $\exists^{2}$ uniformly converts between binary-represented reals and reals-as-Cauchy-sequences.  In this way, one need not worry about representations and the associated extensionality like in Definition \ref{keepintireal}.(5).  By the proof of \cite{kohlenbach4}*{Prop.\ 4.7}, $\exists^{2}$ also uniformly converts a continuous function into an RM-code, i.e.\ we may `recycle' proofs in second-order arithmetic.} easier than in $\WKL_{0}$.\label{itema}
\item Prove $\T$ in $\RCAo+\HBU$ using the existing `uniform' proof from the literature based on Cousin's lemma (See e.g.\ \cite{bartle2,bartle3, gormon, thom2, josting, stillebron}).\label{itemb}
\item Conclude from \eqref{itema} and \eqref{itemb} that $\T$ can be proved in $\RCAo+\WKL$.
\end{enumerate}
\end{tempie}
Hence, even though the goal of RM is to find the \emph{minimal} axioms needed to prove a theorem, one can nonetheless achieve this goal by (only) using non-minimal axioms.  
We leave it to the reader to ponder how much time and effort could have been (and will be) saved using the previous three steps (for $\WKL$ or other axioms).    


\smallskip

Thirdly, we now show that Template \ref{kcuf} constitutes a general/universal/classical version of \emph{Ishihara's tricks}.  
The latter proof technique was first introduced in \cite{discihara} and allows one (in very specific cases) to \emph{constructively} (in the sense of Bishop \cite{bish1}) prove 
a disjunction which at first glance would seem to require the law of excluded middle.   
Ishihara's tricks can also be used (in certain specific cases) to obtain a constructive proof from (i)
a proof in classical mathematics (using \textsf{LPO}, i.e.\ the law of excluded middle for $\Sigma_{1}^{0}$-formulas), and (ii) a proof in intuitionistic mathematics (using Brouwer's continuity principle); an example is \cite{basspeeksel}*{Theorem 4.3}.       

\smallskip

Now, the disjunction in Ishihara's tricks -intuitively speaking- distinguishes between two cases: either we are working in classical mathematics (involving $\textsf{LPO}$ and hence discontinuous functions exist), or we are working in intuitionistic mathematics (where all functions are continuous).  But this case distinction is \emph{exactly} how we obtained $\WKL\di[ (\exists^{2})\vee \HBU]$, and the constructive reading of $\textsf{LPO}$ is also essentially $(\exists^{2})$.  
In light of this similarity, Template \ref{kcuf} may be viewed as a general/universal/classical version of Ishihara's tricks since (i) because of our use of classical logic, we may \emph{always} apply our `law of excluded middle trick', and (ii) there is a \emph{huge} body of \emph{classical} proofs based on $\HBU$ thanks to the redevelopment of calculus in light of the gauge integral (See e.g.\ \cite{bartle2, bartle3, gormon, botsko, thom2}). 

\smallskip

Fourth, in \cite{nonnonrienabelanger}*{Theorem 2.1}, an equivalence between $\neg\WKL_{0}\vee \ACA_{0}$ and the following theorem $\T_{0}$ from model theory is established: \emph{there is a complete theory with a non-principal type and only finitely many models up to isomorphism}.  
This equivalence, and the associated contraposition, is interesting as follows.   
\begin{cor}\label{puilo}
The system $\RCAo+\WKL_{0}+\neg\ACA_{0}$ proves $\HBU$.  
The system $\RCAo$ proves $\T_{0}\asa [\neg\WKL_{0}\vee \ACA_{0}] \asa [\neg \HBU\vee \ACA_{0}]\asa [\neg \PIT_{\u}\vee \ACA_{0}]$.  
\end{cor}
\begin{proof}
In $\T_{0}\asa [\neg\WKL_{0}\vee \ACA_{0}]$, use \eqref{corkukkk} to replace $\WKL$ by $\ACA_{0}\vee \HBU$, i.e.\ 
\[
\T_{0}\asa\big[ [\neg\ACA_{0}\wedge \neg \HBU]\vee \ACA_{0}\big]\asa \big[ \underline{[\neg\ACA_{0}\vee  \ACA_{0}]}\wedge[ \ACA_{0}\vee \neg \HBU]\big].
\]
Omitting the underlined formula, the second part follows, and we proceed in the same way for $\PIT_{\u}$.  
For the first part, the contraposition of $(\exists^{2})\di \ACA_{0}$ yields $\HBU$ by the proof of the theorem.  
\end{proof}
By the previous, the negation of $\WKL_{0}$ or $\ACA_{0}$ implies axioms of Brouwer's intuitionistic mathematics, i.e.\ strange (as in `non-classical') behaviour is almost guaranteed.  The equivalence involving $\neg\WKL_{0}\vee  \ACA_{0}$ remains surprising.

\smallskip

Finally, we study splittings and disjunctions for the \emph{Lindel\"of lemma} from \cite{dagsamIII}.  
We stress that our formulation of $\HBU$ and $\LIN$ is faithful to the original theorems from 1895 and 1903 by Cousin (\cite{cousin1}) and Lindel\"of (\cite{blindeloef}).  % as discussed in Remark \ref{zeideblind}.
\bdefi[$\LIN$] 
For every $\Psi:\R\di \R^{+}$, there is a sequence of open intervals $\cup_{n\in \N}(a_{n}, b_{n})$ covering $\R$ such that $(\forall n \in\N)(\exists x \in \R)[(a_{n}, b_{n}) = I_{x}^{\Psi} ]$.  % and $\cup_{n}(a_{n},b_{n})$ covers $\R$.
\edefi
As to disjunctions, for $\textsf{Y}$ provable in $\ACA_{0}$ but not provable in $\RCA_{0}$, we have $\RCAo\vdash (\textsf{Y}\vee \LIN)$ while no disjunct is provable in the base theory.
If $\textsf{Y}$ is additionally not provable in $\WKL_{0}$, then $\RCAo+\WKL\vdash (\textsf{Y}\vee\HBU)$ while no disjunct is provable in the base theory.  Also, $\LIN\asa [\HBU \vee \neg\WKL]$ over $\RCAo+\QFAC^{0,1}$.   
 % i.e.\ the statement that any uncountable cover of $\R$ has a countable sub-cover.  
%Finally, the original and uniform versions of Pincherle's theorem yield a splitting.  
%As to splittings, we have:  % from \cite{dagsamIII}*{\S4}.  
%\begin{proof}
%Since \textsf{ZF} does not prove that $\R$ is a Lindel\"of space (See \cite{jechheeftpech}), but $\Z_{2}^{\Omega}$ proves $\HBU$ by Theorem \ref{mooi}, $\textsf{ZF}$ does not prove $\HBU\asa [\LIN+\WKL]$.
%The latter equivalence was proved in $\RCAo+\QFAC^{0,1}$ in \cite{dagsamIII}*{\S4}, i.e.\ as required.  
%  % i.e.\ $\QFAC^{0,1}$ is essential.  
%\end{proof}
\begin{thm}\label{hankel}
$\RCAo+\QFAC^{0,1}$ proves $\PIT_{\u}\asa [\LIN +\PIT_{o}]$. % but $\ZF$ does not
\end{thm}
\begin{proof}
The following was established in \cite{dagsamIII}*{\S4}: $\RCAo+\QFAC^{0,1}$ proves $\HBU\asa [\LIN+\WKL]$.  
The theorem now follows from Corollaries \ref{eessje} and \ref{ofmoreinterest}.  
\end{proof}
Similarly, one obtains $\UCT_{\u}\asa [\LIN +\UCT]$, where the latter is just Heine's theorem, and there are many analogous splittings. 

\smallskip

In conclusion, the higher-order framework yields plenty of equivalences for disjunctions and splittings, ixg $\HBU$ and $\LIN$ in particular.  
The above results, \eqref{corkukkk} in particular, suggests that mathematical naturalness does not inherit to (components of) disjunctions, which is in accordance with our intuitions.  
\end{comment}
%\subsection{The use of the axiom of choice}\label{prochoice}
%We study the role of the axiom of choice in our results.  
We first discuss some required preliminaries in Section~\ref{introp}
We study the tight connection between $\QFAC^{0,1}$ and the \emph{Lindel\"of lemma} in Section \ref{rose}.  
We show that the logical status of the latter is highly dependent on its formulation (provable in a weak fragment of $\Z_{2}^{\Omega}$ versus unprovable in $\ZF$).    

\subsection{Historical and mathematical context}\label{introp}
To appreciate the study of countable choice and the Lindel\"of lemma, some mathematical/historical facts are needed.  

\smallskip

First of all, many of the results proved above or in \cite{dagsamIII} make use of the axiom of choice, esp.\ $\QFAC^{0,1}$ in the base theory.  
Whether the axiom of choice is really necessary is then a natural RM-question (posed first by Hirshfeldt; see \cite{montahue}*{\S6.1}).  
Moreover, $\QFAC^{0,1}$ also figures in the grander scheme of things: e.g.\ the local equivalence of `epsilon-delta' and sequential continuity is not provable in \textsf{ZF} set theory, 
while $\QFAC^{0,1}$ yields the equivalence in a general context (\cite{kohlenbach2}*{Rem.\ 3.13}).   
Finally, countable choice for subsets of $\R$ is equivalent to the fact that $\R$ is a Lindel\"of space over $\ZF$ (\cite{heerlijk}).  Thus, the role of $\QFAC^{0,1}$ is connected to the status of the \emph{Lindel\"of property}, i.e.\ that every open cover has a countable sub-cover.
%As it happens, we studied a number of covering lemmas in \cite{dagsamIII}, including the \emph{Lindel\"of lemma}, which constitutes the first instance of the Lindel\"of property in the literature.  
%Hence, we  $\QFAC^{0,1}$ entails the study of the Lindel\"of property.  % and we will obtain the following results.
%Now, one expects the Lindel\"of lemma to follow from $\HBU$ (applied to the intervals $[-N, N]$ for $N\in \N$).  However, $\HBU$ is provable in $\Z_{2}^{\Omega}$ by Theorem~\ref{mooi}, while even $\textsf{ZF}$  does not prove that $\R$ is a Lindel\"of space (\cite{jechheeftpech}).  

\smallskip

Secondly, the previous points give rise to a clear challenge: find a version of the Lindel\"of lemma equivalent to $\QFAC^{0,1}$, over $\RCAo$.  
An immediate difficulty is that the aforementioned results from \cite{heerlijk} are part of set theory, while the framework of RM is much more minimalist by design; for instance, what is a (general) open cover in $\RCAo$?
Fortunately, the pre-1900 work by Borel and Schoenflies on open-cover compactness provides us with a suitable starting point.  %as follows.   

\smallskip

Thirdly, we consider Lindel\"of's \emph{original} lemma from \cite{blindeloef}*{p.\ 698}.
%regarding $\LIND$, which is the version of the Lindel\"of lemma from \cite{dagsamIII}, as follows.
%We stress that our formulation of $\HBU$ and $\LIN$ is faithful to the original from 1895 and 1903 by Cousin (\cite{cousin1}) and Lindel\"of (\cite{blindeloef}).  
%\bdefi[$\LIN$] 
%For every $\Psi:\R\di \R^{+}$, there is a sequence of open intervals $\cup_{n\in \N}(a_{n}, b_{n})$ covering $\R$ such that $(\forall n \in\N)(\exists x \in \R)[(a_{n}, b_{n}) = I_{x}^{\Psi} ]$.  % and $\cup_{n}(a_{n},b_{n})$ covers $\R$.
%\edefi
% as discussed in Remark \ref{zeideblind}.
%We note that the covers in $\LIND$ and $\HBU$ are `special' in that for $x\in \R$, one \emph{knows} the open set covering $x$, namely $I_{x}^{\Psi}=(x-\Psi(x), x+\Psi(x))$ from the canonical cover.  % covering $x$. 
%Our choice for such covers is motivated by the fact that both Cousin and Lindel\"of (\cite{cousin1, blindeloef}) did \emph{not} study (general) open covers, but instead worked with collections of open balls $\cup_{x\in P}B(x, \rho(x))$ for $P\subset \R^{n}$ and $\rho:\R\di \R^{+}$.   %which is the formulation we adopted via our `canonical covers'.  
%Indeed, Lindel\"of formulates his lemma in \cite{blindeloef}*{p.\ 698} as: 
\begin{quote}
Let $P$ be any set in $\R^{n}$ and construct for every point of $P$ a sphere $S_{P}$ with $x$ as center and radius $\rho_{P}$, where the latter can vary from point to point; there exists a countable infinity $P'$ of such spheres such that every point in $P$ is interior to at least one sphere in $P'$.  
%Soit \textsf{\textup{(P)}} un ensemble quelconque situ\'e dans l'espace $\R^{n}$ et, de chaque point $\textsf{\textup{P}}$ comme centre, construisons une sph\`ere $\textsf{\textup{S}}_{\textsf{\textup{P}}}$ d'un rayon $\rho_{\textsf{\textup{P}}}$ qui peut varier de l'un point
%\`a l'autre; il existe {une infinit\'e} 
%{d\'enombrable de ces sph\`eres} de telle sorte que tout point de l'ensemble donn\'e
%soit int\'erieur \`a au moins l'une d'elles.   % (underlining ours)
\end{quote}
A similar formulation was used by Cousin in \cite{cousin1}.  
%Hence, Cousin and Lindel\"of did \emph{not} study (general) open covers, but instead worked with collections of open balls $\cup_{x\in P}B(x, \rho(x))$ for $P\subset \R^{n}$ and $\rho:\R\di \R^{+}$.   %which is the formulation we adopted via our `canonical covers'.  
However, these covers are `special' in that for $x\in \R^{n}$, one \emph{knows} the open set covering $x$, namely $B(x, \rho(x))$, similar to our notion of canonical cover.  % covering $x$. 
%Our choice for such covers is motivated by the fact that both Cousin and Lindel\"of (\cite{cousin1, blindeloef}) did \emph{not} study (general) open covers, but instead worked with collections of open balls $\cup_{x\in P}B(x, \rho(x))$ for $P\subset \R^{n}$ and $\rho:\R\di \R^{+}$.   %which is the formulation we adopted via our `canonical covers'.  
%Indeed, Lindel\"of formulates his lemma in \cite{blindeloef}*{p.\ 698} as: 
By contrast, a (general) open cover of $\R$ is such that for every $x\in \R$, there \emph{exists} a set in the cover containing $x$.  
This is the modern definition, and one finds its roots with Borel (\cite{opborrelen}) as early as 1895 (and in 1899 by Schoenflies), the same year Cousin published \emph{Cousin's lemma} (aka $\HBU$) in \cite{cousin1}.  

\smallskip

Motivated by the above, we shall study the Borel-Schoenflies formulation of the Lindel\"of lemma (and $\HBU$) in Section~\ref{rose}.  This version turns out to be equivalent to $\QFAC^{0,1}$ on the reals, and also provides further nice results.  
  %We finish this remark by stating the obvious: that $\Phi$ as in $\LIND_{2}$ can be defined \emph{because} of the aforementioned special nature of the covers in $\LIND$.  
%In Section \ref{linneke}, we show that the status of the Lindel\"of lemma depends crucially on its formulation.  In particular, we prove $\LIND$ inside $\Z_{2}^{\Omega}$, where the latter stems from \cite{dagsamIII} and expresses the Lindel\"of property of $\R$ in \emph{Lindel\"of's original 1903 formulation} from \cite{blindeloef}.   
%In Section \ref{rose}, we formulate versions of the Heine-Borel theorem and Lindel\"of lemma based on the pre-1900 work of Borel and Schoenflies on open-cover compactness.  
%The Borel-Schoenflies framework yields a full classification in line with the aforementioned equivalence between countable choice and the Lindel\"of property for $\R$.  Our base theory is however $\RCAo$, i.e.\ much weaker than $\ZF$.   

\begin{comment}
\subsubsection{On the Lindel\"of lemma}\label{linneke}
We show that the logical status of the Lindel\"of lemma depends crucially on its formulation: Lindel\"of's original 1903 lemma for $\R$ is provable in $\Z_{2}^{\Omega}$ by Theorem \ref{koel}, while $\ZF$ does not prove that $\R$ is a Lindel\"of space, i.e.\ every open cover has a countable sub-cover (\cite{jechheeftpech}).  

\smallskip

We consider two versions of the Lindel\"of lemma from \cite{dagsamIII}.  
We stress that our formulation of $\HBU$ and $\LIN$ is faithful to the original theorems from 1895 and 1903 by Cousin (\cite{cousin1}) and Lindel\"of (\cite{blindeloef}), as discussed in Remark \ref{zeideblind}.
\bdefi[$\LIN$] 
For every $\Psi:\R\di \R^{+}$, there is a sequence of open intervals $\cup_{n\in \N}(a_{n}, b_{n})$ covering $\R$ such that $(\forall n \in\N)(\exists x \in \R)[(a_{n}, b_{n}) = I_{x}^{\Psi} ]$.  % and $\cup_{n}(a_{n},b_{n})$ covers $\R$.
\edefi
\bdefi[$\LIN_{2}$] 
$(\forall \Psi:\R\di \R^{+})(\exists \Phi^{0\di 1})(\forall x\in \R)(\exists n^{0})(x\in I^{\Psi}_{\Phi(n)})$.
%For every $\Psi^{2}$, there is a sequence of open intervals $(a_{n}, b_{n})$ such that $(\forall n^{0})(\exists x^{1}\in I)[(a_{n}, b_{n}) = I_{x}^{\Psi} ]$ and $[0,1]\subset \cup_{n}(a_{n},b_{n})$.
\edefi
%By the results in \cite{dagsamIII}*{\S4}, $\RCAo+\QFAC^{0,1}$ proves $[\LIND+\WKL]\asa \HBU$.  
\begin{thm}\label{koel}
The system $\Z_{2}^{\Omega}$ proves $\LIND$ and $\LIND_{2}$.  % and $\RCAo$ proves $[\LIND+\WKL]\asa \HBU$.  
\end{thm}
\begin{proof}
Using $\mu^{2}$, we can `shrink' every interval $I_{x}^{\Psi}$ enough to make sure it has rational endpoints, i.e.\ we may restrict ourselves to $\Psi:\R\di \Q^{+}$.
For the first result, by Theorem \ref{mooi}, we may use $\HBU$.    
Now, $\HBU$ generalises to $[-N, N]$ for any $N\in \N$, i.e.\ for $\Psi:\R\di \Q^{+}$, we have
\be\label{whynotsaidthegruffalo}
(\forall N \in \N)(\exists  x_{0}, \dots, x_{k}\in [-N, N])(\forall y\in [-N, N])(\exists i\leq k)(y\in I_{y_{i}}^{\Psi}  ).
\ee
Since $I_{x}^{\Psi}$ has rational endpoints, we may rewrite the formula \eqref{whynotsaidthegruffalo} as follows: 
\begin{align}
(\forall N \in \N)(\exists  a_{0},b_{0},  &\dots, a_{k}, b_{k}\in \Q )(\exists x_{1}, \dots, x_{k}\in [-N, N])\big[(\forall y\in [-N, N])(\exists i\leq k)\label{murki}\\
&\big(y\in  (x-a_{i}, x+b_{i})\big)
\wedge (\forall i\leq k)(\exists z\in [-N, N]) ( (a_{i}, b_{i})= I_{z}^{\Psi}  )\big].\notag
\end{align}
The formula in square brackets in \eqref{murki} may be treated as quantifier-free, thanks to $(\exists^{3})$.  
Applying $\QFAC^{0,0}$, included in $\RCAo$, \eqref{murki} yields $\LIND$.  To obtain $\LIND_{2}$, let $\cup_{n\in \N}(a_{n}, b_{n})$ be the cover provided by $\LIND$, and define $\Phi(n):=\frac{a_{n}+b_{n}}{2}$.
\end{proof}
Note that the variable $z^{1}$ occurs in \eqref{murki} as `call by name', which explains why $\SIXK$ does not suffice for the proof.  
We now explain why the previous theorem is consistent with the fact that $\ZF$ cannot prove that $\R$ is a Lindel\"of space. 
\begin{rem}\label{zeideblind}\rm
We start with the observation that the covers in $\LIND$ (and $\HBU$) are `special' in that for $x\in \R$, one \emph{knows} the open set covering $x$, namely $I_{x}^{\Psi}=(x-\Psi(x), x+\Psi(x))$ from the canonical cover.  % covering $x$. 
Our choice for such covers is motivated by the fact that both Cousin and Lindel\"of (\cite{cousin1, blindeloef}) did \emph{not} study (general) open covers, but instead worked with collections of open balls $\cup_{x\in P}B(x, \rho(x))$ for $P\subset \R^{n}$ and $\rho:\R\di \R^{+}$.   %which is the formulation we adopted via our `canonical covers'.  
Indeed, Lindel\"of formulates his lemma in \cite{blindeloef}*{p.\ 698} as: 
\begin{quote}
Let $P$ be any set in $\R^{n}$ and construct for every point of $P$ a sphere $S_{P}$ with $x$ as center and radius $\rho_{P}$, where the latter can vary from point to point; there exists a countable infinity $P'$ of such spheres such that every point in $P$ is interior to at least one sphere in $P'$.  
%Soit \textsf{\textup{(P)}} un ensemble quelconque situ\'e dans l'espace $\R^{n}$ et, de chaque point $\textsf{\textup{P}}$ comme centre, construisons une sph\`ere $\textsf{\textup{S}}_{\textsf{\textup{P}}}$ d'un rayon $\rho_{\textsf{\textup{P}}}$ qui peut varier de l'un point
%\`a l'autre; il existe {une infinit\'e} 
%{d\'enombrable de ces sph\`eres} de telle sorte que tout point de l'ensemble donn\'e
%soit int\'erieur \`a au moins l'une d'elles.   % (underlining ours)
\end{quote}
By contrast, a (general) open cover of $\R$ is such that for every $x\in \R$, there \emph{exists} a set in the cover containing $x$.  
This is the modern definition, which was nonetheless already used by Borel (\cite{opborrelen}) as early as 1895 (and in 1899 by Schoenflies), the same year Cousin published the proof of his version of $\HBU$ (aka \emph{Cousin's lemma}) in \cite{cousin1}.  For this reason, we shall study the 
Borel-Schoenflies versions of $\HBU$ and $\LIND$ in Section~\ref{rose}.  We finish this remark by stating the obvious: that $\Phi$ as in $\LIND_{2}$ can be defined \emph{because} of the aforementioned special nature of the covers in $\LIND$.  
%Note that Lindel\"of \emph{only} states the existence of a \emph{sequences of balls} in the underlined text, \textbf{not} a sequence of reals \emph{with} the associated balls.  
%Indeed, applying $\QFAC^{0,1}$ to $\LIND$, one \emph{could} obtain $\Phi^{0\di1}$ such that $(\forall n \in\N)[(a_{n}, b_{n}) = I_{\Phi(n)}^{\Psi} ]$, but such a $\Phi$ is nowhere to be found in Lindel\"of's formulation.
%Hence, the strength of the Lindel\"of lemma seems to crucially depend on how we formulate the notion of `countable sub-cover'.  
%The aforementioned $\Phi$ identifies a \emph{sequence} of reals which define the countable cover $\cup_{n\in \N}(a_{n}, b_{n})$, while $\LIND$ only states the \emph{existence} of the latter.  This slight difference in formulation apparently makes a huge `logical' difference, namely provable in $\Z_{2}^{\Omega}$ versus not provable in \textsf{ZF}.  
%$Theorem \ref{corekl} makes this difference more explicit.   
\end{rem}
%\begin{cor}
%The system $\RCAo$ proves $[\LIND+\WKL]\asa \HBU$.  
%\end{cor}
%\begin{proof}
%???
%\end{proof}
%Note that $\LIN$ was called $\textsf{LIND}_{2}$ in \cite{dagsamIII}, where we also studied other variations.  % in \cite{dagsamIII}*{\S4}
%
%\smallskip
%
Finally, as a side-note, we mention some splittings and disjunctions involving $\LIND$. 
As to disjunctions, for $\textsf{Y}$ provable in $\ACA_{0}$ but not provable in $\RCA_{0}$, we have $\RCAo\vdash (\textsf{Y}\vee \LIN)$ while no disjunct is provable in the base theory.
If $\textsf{Y}$ is additionally not provable in $\WKL_{0}$, then $\RCAo+\WKL\vdash (\textsf{Y}\vee\HBU)$ while no disjunct is provable in the base theory.  Also, $\LIN\asa [\HBU \vee \neg\WKL]$ over $\RCAo+\QFAC^{0,1}$.   
 % i.e.\ the statement that any uncountable cover of $\R$ has a countable sub-cover.  
%Finally, the original and uniform versions of Pincherle's theorem yield a splitting.  
%As to splittings, we have:  % from \cite{dagsamIII}*{\S4}.  
%\begin{proof}
%Since \textsf{ZF} does not prove that $\R$ is a Lindel\"of space (See \cite{jechheeftpech}), but $\Z_{2}^{\Omega}$ proves $\HBU$ by Theorem \ref{mooi}, $\textsf{ZF}$ does not prove $\HBU\asa [\LIN+\WKL]$.
%The latter equivalence was proved in $\RCAo+\QFAC^{0,1}$ in \cite{dagsamIII}*{\S4}, i.e.\ as required.  
%  % i.e.\ $\QFAC^{0,1}$ is essential.  
%\end{proof}
\begin{thm}\label{hankel}
$\RCAo+\QFAC^{0,1}$ proves $\PIT_{\u}\asa [\LIN +\PIT_{o}]$. % but $\ZF$ does not
\end{thm}
\begin{proof}
The following was established in \cite{dagsamIII}*{\S4}: $\RCAo+\QFAC^{0,1}$ proves $\HBU\asa [\LIN+\WKL]$.  
The theorem now follows from Corollaries \ref{eessje} and \ref{ofmoreinterest}.  
\end{proof}
Similarly, one obtains $\UCT_{\u}\asa [\LIN +\UCT]$, where the latter is just Heine's theorem, and there are many analogous splittings.  
%Finally, it is a natural question whether $\QFAC^{0,1}$ is really needed in Theorem \ref{hankel} in light of Hirschfeldt's question (See \cite{montahue}*{\S6.1}) concerning equivalences in RM which require a stronger base theory.  The answer here is tricky, as we discuss in the next section.  

%In fact, since $\RCAo+\QFAC^{0,1}$ is conservative over $\RCAo$ anyway, the former is perhaps a better base theory for higher-order RM. 
\end{comment}
\subsection{A rose by many other names}\label{rose}
We formulate versions of the Heine-Borel theorem and Lindel\"of lemma based on the 1895 and 1899 work of Borel and Schoenflies on open-cover compactness (\cite{opborrelen, schoen2}).
These versions provide a nice classification involving $\QFAC^{0,1}$ and show that the logical status of the Lindel\"of lemma is highly dependent on its formulation (provable in second-order arithmetic versus unprovable in $\ZF$).    
%This work was part of early set theory, and will be seen to yield nice RM-results, but over $\RCAo$ rather than the `usual' base theory $\ZF$. 
%We stress that Borel (\cite{opborrelen}) in 1895 and Schoenflies in 1899 (\cite{schoen2}) were already talking about open covers of $[0,1]$ in the form of infinite collections (countable and uncountable) of intervals such that each point in $[0,1]$ is in at least one of the latter.  The subtle-but-crucial difference is that in the Cousin-Lindel\"of formulation as in $\HBU$ and $\LIND$, the ball covering $x$ is `given' (by $I_{x}^{\Psi}$ in our notation), while it only `exists' in the Borel-Schoenflies formulation.     
%As a side-note, Hausdorff in his authoritative `foundations of set theory' assumes that 
%
%\smallskip
%
%We now introduce versions of $\HBU$ and $\LIND$ for the Borel-Schoenfield ($\bs$) framework.  
We note that Schoenfield in \cite{schoen2}*{Theorem V, p.~51} first reduces an uncountable cover to a countable sub-cover, and then to a finite sub-cover.  

\smallskip

For our purposes it suffices that open covers are `enumerated' by $2^{\N}$ and have rational endpoints.  
As discussed in Remark \ref{subm}, this restriction is insignificant in our context.
As to notation, $J_{g}^{\Psi}$ is the open set $ (\Psi(g)(1), \Psi(g)(2))$ for $\Psi:C\di \Q^{2}$, 
while we say that \emph{$\Psi:C\di \R^{2}$ provides an open cover of $\R$} if $(\forall x\in \R)(\exists g\in C)(x\in J_{g}^{\Psi})$.  %Open covers of e.g.\ $[0,1]$ then have an obvious definition.  
We first study the following version of the Lindel\"of lemma for the real line.  
%\[
%(\forall f\in C)(\Psi(f)(1)<_{\R}\Psi(f)(2))\wedge (\forall x\in \R)(\exists g\in C)(x\in J_{g}^{\Psi}).
%\]     
%The previous should be generalised to include e.g.\ $\cup_{f\in \N^{\N}}(-G(f), G(f))$ for $G^{2}$ unbounded on $\N^{\N}$.  For instance as follows:
\bdefi[$\LIN^{\bs}$] For every open cover of $\R$ provided by $\Psi:C\di \Q^{2}$, there exists $\Phi:\N\di C$ such that $(\forall x\in \R)(\exists n\in \N)(x\in J_{\Phi(n)}^{\Psi})$.
\edefi
To gauge the strength of $\LIND^{\bs}$, we first prove that $\QFAC^{0,1}$ in Corollary \ref{ofmoreinterest} may be replaced by the latter.  
While this theorem also follows from Theorem \ref{corekl}, the following proof is highly illustrative.  
%$\LIN^{\bs}$. 
\begin{thm}
The system $\RCAo+\LIN^{\bs}$ proves $\WKL\asa\PIT_{o}$.  
\end{thm}
\begin{proof}
The reverse implication is immediate from (the proof of) Corollary \ref{eessje}.  
The proof of the forward implication in Corollary \ref{ofmoreinterest} makes use of $\QFAC^{0,1}$ \emph{once}, namely to conclude from $(\forall n^{0})(\exists \alpha\leq 1)(F(\alpha)>n)$ the existence of a 
sequence $\alpha_{n}$ in Cantor space such that $(\forall n^{0})(F(\alpha_{n})>n)$ in the proof of Theorem \ref{ofinterest}.  This application of $\QFAC^{0,1}$ can be replaced by $\LIN^{\bs}$ as follows: since $F$ is unbounded on Cantor space, 
%it yields an unbounded functional $G$ on the unit interval, using $\exists^{2}$ to convert reals to their binary representation. 
%Note that in case $(\forall n^{0})(\exists \sigma^{0}\leq 1)(F(\sigma*00)>n)$ or $(\forall n^{0})(\exists \sigma^{0}\leq 1)(F(\sigma*11\dots)>n)$, $\QFAC^{0,0}$ yields the required sequence $\alpha_{n}$, i.e.\ the non-unique nature of binary representation does not pose a problem.  However, since $G$ is unbounded on $I$, 
$\Psi(x):=(-F(x), F(x))$ yields an open cover of $\R$, and the countable sub-cover $\Phi$ provided by $\LIN^{\bs}$ is such that $(\forall m\in \N)(\exists n\in \N)( F(\Phi(n))>m )$.  Applying $\QFAC^{0,0}$ now yields the sequence $\alpha_{n}$.  % and we are done.    
\end{proof}  
The previous proof goes through, but becomes a lot messier, if we assume $\Psi$ from $\LIND^{\bs}$ has $[0,1]$ or $\R$ as a domain, rather than Cantor space.   
This is the reason we have chosen the latter domain. 
As expected, we also have the following theorem.   
%Schoenflies' 1899 and Borel 1895 formulated things this ways!  Schoenflies also proves uncount $\di$ count $\di$ finite!
%Formulate $\LIN^{+}$ less verbose!  Then $\LIN^{+}+\WKL$ implies $\PIT_{o}$ (even assuming $\Psi$ has to be RE), and 
%
%\be\tag{$\LIN_{\N^{\N}}$}
%(\forall \Psi^{2})(\exists \Phi^{0\di 1})(\forall f\in \N^{\N})(\exists n^{0})(f\in [\overline{\Phi(n)}\Psi(\Phi(n))] ).
%\ee
%The previous results for $\LIN$ have a certain elegance, but a fuller (and surprising) classification is provided by the following.  
%It should be noted that \textsf{ZF} does not prove that $\R$ is a Lindel\"of space (\cite{jechheeftpech}) or $\QFAC^{0,1}$ restricted to subsets of $\R$, i.e.\ these results are not `garden variety RM'. 
%\begin{thm}
%The system $\RCAo+\LIN_{\N^{\N}}$ proves $\QFAC^{0,1}$.
%\end{thm}
%\begin{proof}
%%In case of $\neg(\exists^{2})$, all functions on the reals are continuous by \cite{kohlenbach2}*{Prop.~3.12} and the antecedent of \eqref{rivool} then implies $(\forall n\in \N)(\exists q\in \Q)F(q, n)=0$; by definition $\QFAC^{0,0}$ is included in $\RCAo$, and finishes this case.  
%%In case of $(\exists^{2})$, we 
%%
%Fix $F^{2}$ such that $(\forall n^{0})(\exists x^{1})F(x, n)=0$, and define $G^{2}$ as:
%\[
%G(\langle n\rangle *f):=
%\begin{cases}
%n+1 & \textup{ if } (\forall i\leq n)F(\pi(f,n)(i),n )=0 \\
%0 & \textup{ otherwise }\\
%\end{cases}, 
%\] 
%%where $\textsf{den}(q)$ is zero if $q\in \Q$ is zero, and its denominator otherwise. 
%where $\pi^{(1\times 0)\di 1^{*}}$ is the inverse of a functional which codes $n$ sequences into one.  % i.e.\ $\pi(x, n)$ is a sequence $w^{1^{*}}$ of length $n$ such that for $i\leq n$ and $j$, $w(i)(j)=$
%Then $\cup_{f\in \N^{\N}}[\overline{f}G(f)] $ covers $\N^{\N}$ and $\LIN_{\N^{\N}}$ provides $\Phi^{0\di 1}$ such that the sub-cover $\cup_{m\in \N}[\overline{\Phi(n)}G(\Phi(n))]$ also covers $\N^{\N}$.  
%
%\end{proof}
%
%
\begin{thm}\label{corekl}
$\RCAo+\LIN^{\bs}$ proves $\QFAC^{0,1}_{\R}$, i.e.\ for all $F:\R\di \N$, we have
\be\label{rivool}
(\forall n\in \N)(\exists x\in \R)(F(x, n)=0)\di (\exists Y^{0\di 1})(\forall n\in \N)(F(Y(n), n)=0).
\ee
\end{thm}
\begin{proof}
In case of $\neg(\exists^{2})$, all functions on the reals are continuous by \cite{kohlenbach2}*{Prop.~3.12}, and the antecedent of \eqref{rivool} then implies $(\forall n\in \N)(\exists q\in \Q)F(q, n)=0$; 
by definition, $\QFAC^{0,0}$ is included in $\RCAo$ and finishes this case.  
In case of $(\exists^{2})$, we fix $F:\R\di \N$ such that $(\forall n\in \N)(\exists x\in \R)(F(x, n)=0)$.  Now use $(\exists^{2})$ to define $\textsf{inv}(x)$ as $0$ if $x=_{\R}0$ and $1/x$ otherwise; note that:
\be\label{foiklo}
(\forall n\in \N)(\exists x\in [0,1])\big(F(x, n)\times F(\textsf{inv}(x), n)\times F(-x, n)\times F(-\textsf{inv}{(x)}, n)=0 \big).  
\ee
Thus, we may assume that  $(\forall n\in \N)(\exists x\in [0,1])(F(x, n)=0)$. Using $\exists^{2}$, define $G: C\di \N$ as follows for $f\in C$ and $w_{n}=\langle 1\dots 1\rangle$ with length $n$:
\be\label{Defg}
G( w_{n}*f):=
\begin{cases}
n+2 & \textup{ if } (\forall i\leq n)F(\mathbb{b}(\pi(f,n)(i)),i )=0 \wedge f(0)=0 \\
1 & \textup{ if there is no such $n$ }\\
\end{cases}, 
\ee
%\[
%G(x):=
%\begin{cases}
%m+2 & \textup{ if } F(x,m )=0 \wedge m\leq |x|<m+1 \\
%0 & \textup{ otherwise }\\
%\end{cases}, 
%\] 
%where $\textsf{den}(q)$ is zero if $q\in \Q$ is zero, and its denominator otherwise. 
where $\mathbb{b}(x)=\sum_{i=0}^{\infty}\frac{x(i)}{2^{i}}$ and $\pi^{(1\times 0)\di 1^{*}}$ is the inverse of a function which codes $n$ sequences into one.  % i.e.\ $\pi(x, n)$ is a sequence $w^{1^{*}}$ of length $n$ such that for $i\leq n$ and $j$, $w(i)(j)=$
Since $\exists^{2}$ can compute a binary representation of any real in the unit interval, we have $(\forall n\in \N)(\exists x\in C)F(\mathbb{b}(x), n)=0$, and $\Psi(x):=(-G(x), G(x))$ yields an open cover of $\R$.  Then $\LIN^{\bs}$ provides $\Phi^{0\di 1}$ such that the countable sub-cover $\cup_{n\in \N}(-G(\Phi(n)), G(\Phi(n)))$ still covers $\R$.  
Hence, $(\forall m^{0})(\exists n^{0})(G(\Phi(n))>m+1)$, and applying $\QFAC^{0,0}$, there is $g^{1}$ such that  $(\forall m^{0})(G(\Phi(g(m)))>m+1)$.  In the latter, the first case of $G$ from \eqref{Defg} must always hold, 
and we have  that $(\forall m^{0})(F(\mathbb{b}(\pi(\Phi(g(m)))(m)),m )=0$, as required. 
%Should be similar to the previous proof, but the details are hard.  Works if we may drop real extensionality for $\Psi$.  
\end{proof}
%\LIND_{2}^{\bs}$ is equivalent to quantifier-free countable choice on the reals over $\Z_{2}^{\Omega}$ by Corollary \ref{corekl} , i.e.\ 
\begin{cor}
The system $\ZF$ cannot prove $\LIND^{\bs}$.
\end{cor}
\begin{proof}
By the proof of \cite{kohlenbach4}*{Prop.\ 4.1}, $\QFAC^{0,1}_{\R}$ suffices to prove that for any $f:\R\di \R$ and $x\in \R$, $f$ is `epsilon-delta' continuous at $x$ if and only if $f$ is sequentially continuous at $x$.  
However, this equivalence is independent of $\ZF$ (\cite{heerlijk}). 
\end{proof}
In hindsight, the previous theorem is not \emph{that} surprising: applying $\QFAC^{1, 0}$ to the conclusion of $\LIND^{\bs}$, we obtain a functional which provides for each $x\in \R$ 
an interval $J_{g}^{\Psi}$ covering $x$, while we only assume $(\forall x\in \R)(\exists g\in C)(x\in J_{g}^{\Psi})$, i.e.\ a typical application of the axiom of choice.    
%On the other hand, that is not the entire story: any cover $\cup_{x\in \R}I^{\Psi}_{x}$ can be replaced by 
%$\cup_{x\in [0,1]}\big(I_{x}^{\Psi}\cup I_{-x}^{\Psi}\cup I_{\inv(x)}^{\Psi}\cup I_{\inv(-x)}^{\Psi}\big)$ in light of \eqref{foiklo}, and $\LIND_{2}'$, a version of the Lindel\"of lemma involving covers enumerated by $[0,1]$, also implies $\QFAC^{0,1}_{\R}$ by Corollary \ref{kiolpio}.
%% in which every $x\in \R$ is covered by $I_{x}^{\Psi}$ for $\Psi:\R\di \R^{+}$.
%\be\tag{$\LIN_{2}'$}
%(\forall \Psi:I\di \R^{+})\big[ \R\subset \cup_{x\in [0,1]}I_{x}^{\Psi} \di  (\exists \Phi^{0\di 1})( \R\subset \cup_{n\in \N}I_{\Phi(n)}^{\Psi} ).
%\ee 
%By the previous, $\LIN_{2}'$ is essentially $\LIN_{2}$ from \cite{dagsamIII}*{\S3}.
%\begin{cor}\label{kiolpio}
%$\RCAo+\LIN_{2}'$ proves $\QFAC^{0,1}_{\R}$.
%\end{cor}
%\begin{proof}
%We use the proof of the theorem with slight modification.  The case $\neg(\exists^{2})$ goes through in the same way.  
%In case $(\exists^{2})$, we define $G$ as in \eqref{Defg} and note that it is unbounded on Cantor space; as a result, $\R$ is covered by $\cup_{x\in [0,1]}I_{x}^{G\circ \zeta}$ where $\zeta^{1\di 1}$ is a functional (definable in $\exists^{2}$) that converts real numbers in $[0,1]$ into binary representation, choosing the one with trailing zeros in case of multiple options.
%Now use $\LIND_{2}'$ to obtain a countable sub-cover of $\cup_{x\in \R}I_{x}^{G\circ \zeta}$, and proceed as in the proof of the theorem to obtain $\QFAC^{0,1}_{\R}$.
%\end{proof}  
Indeed, the functional $\Phi$ from $\LIND^{\bs}$ is essential to the proof of the theorem, and it is a natural question what the status is of 
the following \emph{weaker} version which only states the \emph{existence} of a countable sub-cover, but does not provide a sequence of reals which \emph{generates} the sub-cover.  
\bdefi[$\LIN^{\bs}_{\w}$]
 For every open cover of $\R$ provided by $\Psi:C\di \Q^{2}$, there is a sequence $\cup_{n\in \N}(a_{n}, b_{n})$ covering $\R$ such that $(\forall n \in\N)(\exists x \in \R)[(a_{n}, b_{n}) = J_{x}^{\Psi} ]$
\edefi
We also study the associated version of the Heine-Borel theorem. 
\bdefi[$\HBU^{\bs}$] 
For every open cover of $[0,1]$ provided by $\Psi:C\di \Q^{2}$, there exists a finite sub-cover, i.e.\
$(\exists  y_{1}, \dots, y_{k}\in C)(\forall x\in \R)(\exists i\leq k)(x\in J_{y_{i}}^{\Psi})$.
\edefi
%\bdefi[$\HBU^{\bs}_{\w}$] 
%For every open cover of $[0,1]$ provided by $\Psi:C\di \Q^{2}$, there are $a_{0}, b_{0}, \dots, a_{k}, b_{k}$ in $\Q$ such that $[0,1]\subset \cup_{i\leq k}(a_{i}, b_{i})\wedge (\forall i\leq k)(\exists g\in C)\big((a_{i}, b_{i})=J_{g}^{\Psi}\big)$.
%\edefi
In contrast to its sibling, $\LIND_{\w}^{\bs}$ is provable in $\ZF$, as follows.  
%We have the following theorem, of which the proof is based on the previous. 
\begin{thm}
The system $\Z_{2}^{\Omega}$ proves $\HBU^{\bs}$ and $\LIND_{\w}^{\bs}$, while $\Z_{2}^{\Omega}+\QFAC^{0,1}_{\R}$ proves $\LIND^{\bs}$.
\end{thm}
\begin{proof}
To prove $\HBU^{\bs}$ from $(\exists^{3})$, use the same proof as for $\HBU$ in Theorem~\ref{mooi}.  
Note that the point $y_{0}$ in the proof of the latter is such that we only need to know that is \emph{has} a covering interval, namely $y_{0}\in J_{g_{0}}^{\Psi}$ for some $g_{0}\in C$; note that this interval need not be centred at $y_{0}$.
To obtain $\LIND^{\bs}_{\w}$ from $\HBU^{\bs}$, note that the latter readily generalises to $[-N,  N]$, implying
%proceed in the same way as in Theorem \ref{koel}.  In particular, we use the following version of \eqref{murki}:
\begin{align}
(\forall N \in \N)(\exists  a_{0},b_{0},  \dots, a_{k}, b_{k}\in \Q )\big[(\forall y\in& [-N, N])(\exists i\leq k)\big(y\in  (a_{i}, b_{i})\big)\label{murki}\\ 
&\wedge (\forall i\leq k)(\exists f\in C) ( (a_{i}, b_{i})= J_{f}^{\Psi}  )\big].\notag
\end{align}
where $\Psi:C\di \Q^{2}$ provides an open cover of $\R$; the formula in square brackets in \eqref{murki} is treated as quantifier-free by $(\exists^{3})$. 
Applying $\QFAC^{0,0}$, \eqref{murki} yields $\LIND^{\bs}_{\w}$.  Apply $\QFAC^{0,1}$ and $(\exists^{2})$ to the final formula in $\LIND^{\bs}_{\w}$ to obtain $\LIND^{\bs}$.  
%let $\cup_{n\in \N}(a_{n}, b_{n})$ be the cover provided by $\LIND$, and define $\Phi(n):=\frac{a_{n}+b_{n}}{2}$.
\end{proof}
As it turns out, $\LIND^{\bs}_{\w}$ and $\LIND^{\bs}$ are even \emph{finitistically reducible}\footnote{Recall that $\RCAo+(\kappa_{0}^{3})+\QFAC^{0,1}$ is conservative over $\WKL_{0}$, as noted just before Theorem~\ref{napjeir}.  
According to Simpson in \cite{simpson2}*{IX.3.18}, the versions of the Lindel\"of lemma as in $\LIND^{\bs}$ and $\LIND_{\w}^{\bs}$ are thus \emph{reducible to finitistic mathematics in the sense of Hilbert}.} as follows.  % similar to Corollary \ref{kapaf}.  
\begin{cor}\label{pallap}
 $\RCAo+(\kappa_{0}^{3})$ proves $\LIND_{\w}^{\bs}$.  Adding $\QFAC^{0,1}$ yields $\LIND^{\bs}$.
\end{cor}
\begin{proof}
In case of $(\exists^{2})$, the theorem applies, using $[(\exists^{2})+(\kappa^{3}_{0})]\asa (\exists^{3})$.  % as in the proof of Corollary \ref{kapaf}.  
In case of $\neg(\exists^{2})$, all functions on Baire space are continuous, and the countable sub-cover is provided by listing $J_{\sigma*00\dots}^{\Psi}$ for all finite binary $\sigma$.  
\end{proof}
Before we continue, we discuss why our restriction to $\Psi:C\di \Q^{2}$ is insignificant.  
\begin{remark}\label{subm}\rm
By way of a practical argument, while we \emph{could} have formulated $\LIN^{\bs}$ using $\Psi:\R\di \R^{2}$, we already obtain $\QFAC^{0,1}_{\R}$ with the above version, i.e.\ $\Psi:C\di \Q^{2}$ `is enough', and this choice makes the above proofs easier.
On a more conceptual level, $\exists^{2}$ computes a functional converting reals in the unit interval into a binary representation, which combines nicely with our `excluded middle trick' in the proof of Theorem \ref{corekl}.  Moreover, $\RCAo+(\kappa_{0}^{3})$ seems to be the weakest system that still proves $\LIND^{\bs}_{\w}$, and this system also readily generalises $\LIND_{\w}^{\bs}$ from $\Psi:C\di \Q^{2}$ to $\Psi:C\di \R^{2}$. 
\end{remark}
As noted above, $\ZF$ proves the equivalence between the fact that $\R$ is a Lindel\"of space and the axiom of countable choice for subsets of $\R$ (\cite{heerlijk}). 
The base theory in the following theorem is significantly weaker than $\ZF$. 
\begin{cor}
The system $\RCAo$ proves $\LIND^{\bs}\asa [\QFAC^{0,1}_{\R}+ \LIND^{\bs}_{\w}]$, while $\Z_{2}^{\Omega}$ proves $\LIND^{\bs}\asa \QFAC^{0,1}_{\R}$.  %  \asa [\PIT_{\u}+\QFAC^{0,1}_{\R}]\asa[\LIN_{\bs}+ \PIT_{o}]$.  % and  $\PIT_{\u}\asa [\LIN +\PIT_{o}]$
\end{cor}
The following theorem provides a nice classification of the above theorems.
%We now consider a version of $\HBU$ based on the Borel-Schoenfield formulation. 
\begin{cor}
The system $\RCAo$ proves $[\HBU^{\bs}+\QFAC^{0,1}_{\R}]\asa [\LIN^{\bs} +\WKL]$. %  \asa [\PIT_{\u}+\QFAC^{0,1}_{\R}]\asa[\LIN_{\bs}+ \PIT_{o}]$.  % and  $\PIT_{\u}\asa [\LIN +\PIT_{o}]$
\end{cor}
\begin{proof}
The reverse implication follows from the theorem and the equivalence between $\WKL$ and the Heine-Borel theorem for countable covers (See \cite{simpson2}*{IV.1}). 
For the forward implication, $\neg(\exists^{2})$ implies the continuity of all functionals on Baire space, and a countable sub-cover as in $\LIN^{\bs}$ is in this case provided by 
the sequence of all finite binary sequences.  In the case of $(\exists^{2})$, note that $\HBU^{\bs}$ implies:
\be\label{whynotsaidthegruffalo}
(\forall N \in \N)(\exists  x_{0}, \dots, x_{k}\in [-N, N])(\forall y\in [-N, N]\cap \Q)(\exists i\leq k)(y\in I_{y_{i}}^{\Psi}  ).
\ee 
Now use $(\exists^{2})$ and $\QFAC^{0,1}_{\R}$ to obtain the theorem in this case.  The law of excluded middle as in $(\exists^{2})\vee \neg(\exists^{2})$ finishes the proof.  
%The corollary immediately follows from Theorem \ref{proud}.  
\end{proof}
By \cite{simpson2}*{p.\ 54, Note 1}, $\WKL_{0}\asa \Pi_{1}^{0}\textsf{-AC}_{0}$ over $\RCA_{0}$, yielding the elegant equation:
\[
[\HBU^{\bs}+\QFAC^{0,1}_{\R}]\asa [\LIN^{\bs} + \Pi_{1}^{0}\textsf{\textup{-AC}}_{0}].
\]
%Finally, we consider $\Sigma_{1}^{1}\textsf{\textup{-AC}}_{0}$, which is $\ACA_{0}$ plus the axiom of choice restricted to $\Sigma_{1}^{1}$-formulas in the $\L_{2}$-language (\cite{simpson2}*{VII.6.1}). 
%This system is not a `Big Five' system, and nor does it boast many equivalent theorems, to the best of our knowledge.  The following natural higher-order version of $\Sigma_{1}^{1}\textsf{\textup{-CA}}_{0}^{\omega}$ turns out to have plenty of equivalences, by the above and Theorem \ref{dede}. 
%\bdefi[$\Sigma_{1}^{1}\textsf{\textup{-AC}}^{\omega}$]
%For any $Z^{2}$ and $\varphi(n, X)\equiv (\exists Y^{1})(\forall n^{0})(Z(X, Y, n)=0)$: 
%\be\label{fling}
%(\forall n^{0})(\exists X^{1})\varphi(n, X)\di (\exists Y^{0\di 1})(\forall n^{0})\varphi(n, Y(n)), 
%\ee
%where $X, Y$ are subsets of $\N$ represented by binary sequences.  
%\edefi
%%The system $\Sigma_{1}^{1}\textsf{\textup{-CA}}_{0}^{\omega}$ is defined as $\RCAo+(\exists)$
%\begin{thm}\label{dede}
%The system $\WKL_{0}^{\omega}$ proves $\QFAC^{0,1}_{\R}\asa \Sigma_{1}^{1}\textsf{\textup{-AC}}^{\omega}$.
%\end{thm}
%\begin{proof}
%The reverse direction is immediate, since $\RCA_{0}$ proves that every real has a binary expansion (\cite{polahirst}).  
%For the forward direction, in case $(\exists^{2})$, we may remove the `$(\forall n^{0})$' from $\varphi$ in $\Sigma_{1}^{1}\textsf{\textup{-AC}}^{\omega}$, and this modified version clearly follows from $\QFAC^{0,1}_{\R}$.  
%In case $\neg(\exists^{2})$, all functions on Baire space are continuous by \cite{kohlenbach2}*{Prop.\ 3.7}, and hence uniformly continuous on Cantor space by \cite{kohlenbach4}*{Prop.\ 4.10}, since we assume $\WKL$.  Using the uniformity continuity of $Z^{2}$, the antecedent of \eqref{fling} becomes 
%\[
%(\forall n^{0})(\exists x^{0}, y^{0} \in 2^{<\N})(\forall n^{0})(Z(x*00\dots, y*00\dots, n)=0), 
%\]
%and applying $\Pi_{1}^{0}$-\textsf{AC} (which is equivalent to $\WKL$ by \cite{simpson2}*{p.\ 54, Note 1}), the consequent of \eqref{fling} follows. 
%\end{proof}

%This corollary provides a finitistically reducible base theory for the theorem.
%\begin{cor}
%The system $\RCAo+\WKL+(\kappa^{3})$ proves $\LIND_{\w}^{\bs}$.
%\end{cor}
%\begin{proof}
%In case of $\neg(\exists^{2})$, all functions on Baire space are continuous, 
%%The corrollary also  if we can show that the formula in square brackets in \eqref{murki} is equivalent to quantifier-free given $(\kappa^{3})$.  
%%To this end, note that $\kappa^{3}$ readily removes the `$(\exists f\in C)$' in the final formula of \eqref{murki}.  Since the intervals $(a_{0}, b_{0}), \dots (a_{k}, b_{k})$ in \eqref{murki} have finite overlap, we 
%%can replace `$a_{i}<y < b_{i}$' by a quantifier-free formula involving a rational approximation of $y$.  Hence, the formula $(\forall y \in [-N, N])(\exists i\leq k)(y\in (a_{i}, b_{i}))$ becomes quantifier-free modulo $\kappa^{3}$, as reals in $[-N,N]$ have a binary representation.  
%\end{proof}
In conclusion, we have formulated two versions of the Lindel\"of lemma based on the Borel-Schoenflies framework; one version is provable in (a weak fragment of) $\Z_{2}^{\Omega}$, while the other one is not provable in $\ZF$.  
The latter is due to the `hidden presence of the axiom of choice' in $\LIND^{\bs}$: an open cover in the sense of the latter only tells us that $x\in \R$ is in some interval, but not which one. 
The sequence $\Phi$ however provides such an interval for $x\in \R$ by applying $\QFAC^{1,0}$ to $(\forall x\in \R)(\exists n\in \N)(x\in J_{\Phi(n)}^{\Psi})$.
In a nutshell, the Lindel\"of lemma only becomes unprovable in $\ZF$ \emph{if} we build some choice into it, something of course set theory is wont to do.  
\appendix



\section{Uniform proofs in the literature}\label{pproof}
We discuss numerous proofs of Heine's and Pincherle's theorem from the literature and show that 
these proofs actually establish the uniform versions, sometimes after minor modifications (only).  
Our motivation is to convince the reader that mathematicians like Dini, Pincherle, Lebesgue, Young, Riesz, and Bolzano were using strong axioms (like the centred theorem below) in their proofs, and the latter establish (sometimes after minor modification) highly uniform theorems.  % before or at the dawn of set theory.  
%The lack of an established formal language (for logic, computability, etc) meant these uniformity properties were not articulated; the success of Weierstrass' more constructive approach meant that these uniform proofs hitherto went 
%undiscovered.  

 
\smallskip

Some of the aforementioned proofs are only discussed briefly due to their similarity to the above proofs.      
We first discuss Heine's theorem in Section \ref{heikel}, as Dini's proof of the latter (\cite{dinipi}) predates the proof of Pincherle's theorem from \cite{tepelpinch}; the latter theorem is discussed in Section \ref{forgopppp}.  
A comparison between the proofs by Dini and Pincherle suggests that Pincherle based his proof on Dini's.   
Both proofs make use of the following version of the Bolzano-Weierstrass theorem.  
\begin{quote}
If a function has a definite property infinitely often within a finite domain, then there is a point such that in any neighbourhood of this point there are infinitely many points with the property.
\end{quote}
Note that Weierstrass has indeed formulated this theorem in \cite{weihimself}*{p.\ 77}, while Pincherle mentions it in \cite{pinkersgebruiken}*{p.\ 237} (with an attribution to Weierstrass); Dini states a special case of the centred theorem in \cite{dinipi}*{\S36}.  

\smallskip

%%CHANGE
Finally, we stress the speculative nature of historical claims (say compared to mathematical ones).  
We have taken great care to accurately interpret all the mentioned proofs, but more certainty than the level of interpretation we cannot claim.  
%For instance, Dini's proof of Heine's theorem (\cite{dinipi, dinipi2}) contains a vague definition at a critical point, as discussed just before the proof of Theorem \ref{heineforeal}.  
%We believe our interpretation of Dini's proof is \emph{compatible} with Dini's vague definition, but more certainty than that seems unattainable.  

\subsection{Proofs of Heine's theorem}\label{heikel}
First of all, the proofs of Heine's  theorem in \cite{messias}*{\S4.20}, \cite{bartle2}*{p.~148}, \cite{botsko}*{Theorem 3}, \cite{gormon}*{Theorem 7}, \cite{hardy}*{V},  \cite{hobbelig}*{p.\ 239}, \cite{knapgedaan}*{p.\ 111}, \cite{langebaard}*{p.\ 35}, \cite{leaderofthepack}*{p.\ 14}, \cite{lebes1}*{p.\ 105}, \cite{mensiesson}*{p.~185}, \cite{munkies}*{p.\ 178}, \cite{protput}*{p.\ 82}, \cite{rudin}*{p.\ 91}, \cite{stillebron}*{p.\ 62}, \cite{thom2}*{Example~3, p.~474}, and \cite{younger}*{p.\ 218} are basic compactness arguments, i.e.\ they amount to little more than $\HBU_{\c}\di \UCT_{\u}$ from Corollary~\ref{roofer}. 

\smallskip

Secondly, the proof of Heine's theorem by Dini in \cite{dinipi}*{\S41} (Italian) and \cite{dinipi2}*{\S41} (German) is essentially as in Theorem \ref{heineforeal}, with one difference: Dini does not use the function from \eqref{hopla}, but introduces a modulus of continuity as follows:
\begin{quote}
the number $\eps$ should be interpreted as the supremum of all values of $\eps$ that, in reference to the point $x$, are compatible with those properties any $\eps$ should have. (See \S41 in \cite{dinipi, dinipi2})
%
% soll unter s die obere Grenze der Werthe von s ver-
%standen werden, welche in Bezug auf den Punkt x mit den
%Eigenschaften, die alle � haben m\UTF{00FC}ssen, vereinbar sind (eine
%solche Grenze ist offenbar vorhanden. � 15). s(6, x) be-
\end{quote}
Thus, Dini's modulus modulus of continuity $\eps(x, \sigma)$ is the supremum of all $\eps'>0$ such that $(\forall x, y\in I)(|x-y|<\eps' \di |f(x)-f(y)|<\sigma)$. 
Our modulus $\eps_{0}(x, \sigma)$ from \eqref{hopla} is always below $\eps(x, \sigma)$, but does not depend on the function $f$ and hence yields \emph{uniform} Heine's theorem.  
% takes the supremum of certain intervals $I_{y}^{\eps(y,\sigma)}$ that cover $x$, which would
%seem to be in line with Dini's previous quote.  

\begin{thm}\label{heineforeal}
Any continuous $f:[a,b]\di \R$ is uniformly continuous on $[a,b]$.  
\end{thm}
\begin{proof}
For simplicity, we work over $I\equiv [0,1]$.  
Using Dini's notations, let $\eps:(I\times \R)\di \R^{+}$ be a modulus of (pointwise) continuity for $f:I\di \R$, i.e.\
\[
(\forall \sigma >_{\R}0)(\forall x, y\in I)(|x-y|<_{\R}\eps(x, \sigma)\di |f(x)-f(y)|<_{\R}\sigma ). 
\]
Without loss of generality, we may assume that $\eps(x, \sigma)<2$ for all $x\in I$.  
There are many moduli of continuity, and we need a `nice' modulus, or similar object.    
To this end, define $I_{x}^{\eps(x, \sigma)}$ as the interval $(x-\eps(x, \sigma), x+\eps(x, \sigma))$ and define
\be\label{hopla}
\eps_{0}(x, \sigma):= \sup \big\{  {\eps(y, \sigma)} : y\in I \wedge x\in  I_{y}^{\frac{1}{2}\eps(y, \sigma)}  \big\}.
%\eps_{0}(x, \sigma):= \sup \{ | I_{y}^{\eps(y, \sigma)}| : y\in I \wedge I_{x}^{\eps(x, \sigma)}\subseteq I_{y}^{\eps(y, \sigma)}  \}.\\
%\eps_{0}(x, \sigma):= \sup \{ I_{y}^{\eps(y, \sigma)} : y\in I \wedge I_{x}^{\eps(x, \sigma)}\subseteq I_{y}^{\eps(y, \sigma)}  \}.\\
%\eps_{0}(x, \sigma):= \sup \{ (a,b) : (\forall y\in I)( I_{x}^{\eps(x, \sigma)}\subseteq I_{y}^{\eps(y, \sigma)}\di (a, b)\subseteq I_{y}^{\eps(y, \sigma)})  \}.
\ee
Note that if $|x-z| < \eps_{0}(x,\sigma)/2$, then $|f(x) - f(z)| < 2\sigma$, i.e.\ $\eps_{0}$ is essentially a modulus of continuity for $f$ too.  
Now fix $\sigma>_{\R}0$ and let $\lambda_{0}$ be $\inf_{z\in I}\eps_{0}(z, \sigma/2)$.  Then there is a point $x'\in I$ such that for any neighbourhood $U$ of $x'$, no matter how small, we have $\inf_{z\in U} \eps_{0}(z, \sigma/2) = \lambda_{0}$.  
Now consider $U_{0}=I_{x'}^{\frac{1}{2}\eps(x', \sigma/2)}$ and note that $\inf_{z\in U_{0}} \eps_{0}(z, \sigma/2) = \lambda_{0}$ by definition.
However, for $z\in U_{0}$, \eqref{hopla} (for $\sigma/2$) implies that $\eps_{0}(z, \sigma/2)$ is at least ${\eps(x', \sigma/2)}$, i.e.\ $\eps_{0}(z, \sigma/2)\geq \eps(x', \sigma/2)$. 
Taking the infimum, $\lambda_{0}=\inf_{z\in U_{0}} |\eps_{0}(z, \sigma/2)| \geq \eps(x', \sigma/2)$.  Define $\eps_{1}:=\frac{1}{2}\eps(x', \sigma/2)$ and note
\[
(\forall x, y\in I)(|x-y|<_{\R}\eps_{1})\di |f(x)-f(y)|<_{\R}\sigma ),  
\]
and the uniform continuity of $f$ follows. 
%in \cite{dinipi} (where Pincherle considers $L(x)$) and applies essentially 
%the same steps as in Theorem \ref{gemtoo}, including the crucial step involving the centred theorem in the first paragraph of the proof of Theorem \ref{gemtoo}. 
%It is highly likely that Pincherle's proof was inspired by Dini's proof.  
\end{proof}
%\marginpar{\footnotesize{I have some problems with this proof. If this is meant to be a proof Dini essentially gave of Heine's theorem, I'd like to see the German version of Dini's proof, if possible.}}
L\"uroth's proof of Heine's theorem \cite{grosselul} proceeds in the same way: a \emph{nice} modulus of continuity is defined, for which it is argued that 
the infimum cannot be zero anywhere in the interval, establishing uniform continuity.  With inessential modification, L\"uroth's proof also yields \emph{uniform} Heine's theorem. 

\smallskip

Incidentally, Weierstrass' proof from \cite{amaimennewekker}*{p.\ 203-204} establishes the Heine-Borel theorem (without explicit formulation) and also starts with the introduction of a nice modulus (in casu: of uniform convergence).   
A detailed motivation for this observation is in \cite{medvet}*{p. 96-97}.  
The following corollary is now immediate.  % from the proof. 
\begin{cor}
For any $\eps>_{\R}0$ and $g:(I\times \R)\di \R^{+}$, there is $\delta>_{\R}0$ such that for any $f:I\di \R$ with modulus of continuity $g$, we have
\[
(\forall x, y\in I)(|x-y|<_{\R}\delta)\di |f(x)-f(y)|<_{\R}\eps ), 
\]
%i.e.\ there is a 
\end{cor}
Thirdly, as discussed in Remark \ref{kowlk}, Pincherle mentions a variation of Pincherle's theorem in \cite{tepelpinch}*{Footnote 1} and states it is a generalisation of Heine's theorem as proved by Dini in \cite{dinipi}*{\S41}.  
As discussed in Section \ref{forgopppp}, Pincherle's proof of Pincherle's theorem \emph{with minor modification} also establishes the uniform version, and the uniform version of the variation from Remark \ref{kowlk} immediately yields \emph{uniform} Heine's theorem when applied to a modulus of continuity.  
Hence, Pincherle's proof from \cite{tepelpinch} establishes uniform Heine's theorem  \emph{with minor modification}.  

\smallskip
 %and applying the variation of Pincherle's theorem from Remark \ref{kowlk}.  
%\smallskip

Fourth, Bolzano provides an incorrect proof of Heine's theorem in \cite{nogrusser}*{p.\ 575, \S6}.  However, Russnock claims in \cite{russje}*{p.\ 113} that Bolzano's basic strategy is solid and 
provides a correct proof, which he calls \emph{a Bolzanian proof of Heine's theorem}, in \cite{russje}*{Appendix}.  The latter proof can establish uniform Heine's theorem as it is is similar in spirit to the proof of Theorem \ref{heineforeal}: one starts from a modulus of continuity, then defines a certain sequence in terms of the latter, and the cluster point of this sequence is used to define a modulus of \emph{uniform} continuity.  

\smallskip

Fifth, Lebesgue provides (what he refers to as) a `pretty proof' of Heine's theorem in \cite{lebes1}*{p.\ 105, Footnote 1} as an application of the Heine-Borel theorem for \emph{uncountable covers}. The proof is in prose (only), and can be summarised as follows. 
\begin{quote}
For fixed $\eps>0$, every point $x\in [a,b]$ is covered by a ball in which the oscillation of $f(x)$ is at most $\eps$.  
By the Heine-Borel theorem, finitely many of those balls cover $[a,b]$.  The length of the smallest ball is then as required for the uniform continuity of $f$.
\end{quote}
Now Lebesgue's proof arguably also establishes $\HBU\di \UCT_{\u}^{\R}$ as follows: Lebesgue's notion of (uniform) continuity (See \cite{lebes1}*{p.\ 22}) seems to involve a \emph{modulus} of (uniform) continuity.  
Of course, given a modulus of continuity $g$ for $f$ on $[a, b]$, the ball $(x-g(x, \eps), x+g(x, \eps))$ is such that the oscillation of $f(x)$ is at most $\eps$.  
Hence, applying $\HBU$ to the cover $\cup_{x\in I}I^{g}_{x}$ immediately implies $\UCT_{\u}^{\R}$.  The proofs by Riesz, Hardy, and Young in \cites{manon, younger, hardy} amount to the same proof.  % of $\HBU\di \UCT_{\u}^{\R}$.    
%ZIn this way, Lebesgue's proof may be interpreted 
%as the `obvious' proof that $\HBU\di \UCT_{\u}^{\R}$.  

\smallskip

Sixth, Thomae's proof (\cite{thomeke}*{p.\ 5}) of Heine's theorem is not correct, but actually suggests using \eqref{hopla}.  Indeed, for the associated canonical cover, build a sequence in which the first interval covers zero, and the next one the right end-point of the previous one, as in Thomae's proof.  The latter now yields \emph{uniform} Heine's theorem.    

\smallskip

Finally, neither Weierstrass' proof in \cite{weihimself}, or Heine's proof in \cite{keine}, or Dirichlet's proof in \cite{didi2} establish the uniform version of Heine's theorem, as far as we can see.  

%\smallskip
\subsection{Proofs of Pincherle's theorem}\label{forgopppp}
First of all, the proofs of Pincherle's theorem in \cite{bartle2}*{p.\ 149}, \cite{gormon}*{p.\ 111}, and \cite{thom2}*{p.\ 185} are basic compactness arguments, amounting to little more than the proof of $\HBU_{\c}\di \PIT_{\u}$ in Theorem \ref{mooi}.  % but for bounded intervals rather than Cantor space.  

\smallskip

Secondly, the proof of Pincherle's theorem by Pincherle himself is essentially as follows (See \cite{tepelpinch}*{p.\ 67 for the Italian original}).    
\begin{thm}[Pincherle]\label{gemtoo}
Let $E$ be a closed, bounded subset of $\mathbb{R}^{n}$ and let $f : E \di \R$ be
locally bounded with realisers $L, r:\R\di \R^{+}$. Then $f$ is bounded on $E$.
\end{thm}
\begin{proof}
We start with a note regarding references: Pincherle motivates the crucial step in the proof in \cite{tepelpinch}*{p.\ 67} as follows: \emph{per le proposizioni generali sulle grandezze variabili}, which translates to \emph{due to general propositions on variable magnitudes}.    
Pincherle does not provide references, but it is clear from his proof that he meant the version of the Bolzano-Weierstrass theorem from the beginning of this section. 

\smallskip

Now suppose $f:E\di \R$ is locally bounded with realisers $L', r:\R\di \R^{+}$, i.e.\ for every $x\in E$ and $y\in E\cap B(x, r(x)) $, we have $|f(y)|\leq L'(x)$. 
Let $L(x)$ be the lim sup of $|f(y)|$ for $y \in E\cap B(x, r(x))$.  By assumption $L:E\di\R^+$ is always finite (and well-defined) for inputs from $E$.
Now let $L\in \R^{+}\cup\{+\infty\}$ be the lim sup of $L(x)$ for $x\in E$; we show that $L$ is a finite number.  

\smallskip

In fact, there is, due to the first paragraph, %\footnote{The `definite property' is that there is a neighbourhood $U\subset C$ such that the lim sup of $L(x)$ for $x\in U$ is $L$.}, 
a point $x'\in E$
such that for any neighbourhood $U$ of $x'$, however small, the lim sup of $L(y)$ for $y\in U$ is $L$.  By locally boundeness, the lim sup of $|f(y)|$ for $y\in B(x', r(x'))$ is a finite number, namely less than $L':=L(x')$.    
By the previous, the lim sup of $|L(y)|$ for $y\in B(x', r(x')/2)$ is $L$.  But since $B(x', r(x')/2)\subset B(x', r(x')$, we have $L\leq L'$, and $L$ is indeed finite.  
\end{proof}
A minor modification of the previous proof now yields the uniform version.  
%Note that most of the proof, in particular the crucial step stating the existence of $x'$, make use of $L$ rather than $f$.  
%Thus, repeating the previous proof for any $g:E\di \R$ with the same realisers $L, r:\R\di\R^{+}$ for local boundedness, we observe that the bound $L\in \R$ only depends on the realisers, yielding the following. 
\begin{cor}
Let $E$ be a closed, bounded subset of $\mathbb{R}^{n}$ and let $f : E \di \R$ be
locally bounded with realisers $L, r:\R\di \R^{+}$. Then $|f|$ has an upper bound on $E$ that only depends on the latter.
\end{cor}
\begin{proof}
It suffices to define a suitable $L(x)$ in terms of $L'(x)$ (rather than in terms of $f(x)$).  This can be done in the same way as $\eps_{0}(x, \sigma)$ in \eqref{hopla} is defined in terms of $\eps(x, \sigma)$.  
For instance, define $L:E\di \R^{+}$ as follows:
\be\label{dorkioplk}
L(x):=\inf_{z\in E}\{ L'(z): I_{x}^{r}\subseteq I_{z}^{r}   \},
\ee
where $L', r:E\di \R^{+} $ are realisers for the local boundedness of $f$. 
\end{proof}
\begin{rem}[A function by any other name]\label{kiekenkkk}\rm
We show that Pincherle intended to formulate his theorem for \emph{any} function, not just continuous ones.  
First of all, Pincherle includes the following expression in his theorem: 
\begin{quote}
\emph{Funzione di $x$ nel senso pi\`u generale della par\'ola} (\cite{tepelpinch}*{p.\ 67}),
\end{quote}
which translates to `function of $x$ in the most general sense'.  However, discontinuous functions had already enjoyed a long history by 1882: they were discussed by Dirichlet in 1829 (\cite{didi1}); Riemann studied such functions in his 1854 \emph{Habilitationsschrift} (\cite{kleine}*{p.\ 115}), 
and the 1870 dissertation of Hankel, a student of Riemann, has `discontinuous functions' in its title (\cite{hankelijkheid}).  
We also mention \emph{Thomae's function}, similar to Dirichlet's function and introduced in \cite{thomeke}*{p.\ 14} around 1875.     

\smallskip

Secondly, Pincherle refers to a number of theorems due to Dini and Weierstrass as \emph{special cases} of his theorem in \cite{tepelpinch}*{p.\ 66-68}.
He also mentions that Dini's theorem is about continuous functions, i.e.\ it seems unlikely he just implicitly assumed his theorem to be about continuous functions.  
Finally, the proof on \cite{tepelpinch}*{p.\ 67} does not require the function to be continuous (nor does it mention the latter word).  
Since Pincherle explicitly mentions establishing \emph{una proposizione generale}, it seems unlikely he overlooked the fact that his \emph{Teorema} was about \emph{arbitrary} functions.  
\end{rem}



In conclusion, Dini \emph{almost} establishes $\UCT_{\u}^{\R}$ in \cites{dinipi, dinipi2}, while Pincherle later probably adapted Dini's proof to obtain Pincherle's theorem in \cite{tepelpinch}.  
Pincherle's proof is uniform \emph{if} we define $L(x)$ as in \eqref{dorkioplk} rather than in terms of $f$ itself, i.e.\ similar to \eqref{hopla}.  Moreover, the proof in \cite{russje}*{Appendix} 
seems to establish $\UCT_{\u}^{\R}$, and is claimed by the historian Rusnock to be a \emph{Bolzanonian proof of Heine's theorem}.  
Finally, Lebesgue, Riesz, and Young prove $\HBU\di\UCT_{\u}^{\R}$ in \cite{lebes1,manon, younger}.  

\smallskip

In a nutshell, we observe that the version of the Bolzano-Weierstrass theorem 
from the beginning of this section, as well as the Heine-Borel theorem for uncountable covers, was (or could be) used to prove \emph{uniform} versions of Heine's and Pincherle's theorems.  
Weierstrass' more `constructive' approach as in \cites{weihimself, didi2} later became the norm however, until  the redevelopment of analysis as in e.g.\ \cite{bartle2} based on techniques from gauge integration.  
With that, both history and this paper have come full circle, which constitutes a nice ending for this paper.  



\section{The G\"odel Hierarchy}\label{kurtzenhier}
The \emph{G\"odel hierarchy} is a collection of logical systems ordered via consistency strength, or essentially equivalent: ordered via inclusion\footnote{Simpson states in \cite{sigohi}*{p.\ 112} that inclusion and consistency strength yield the same hierarchy as depicted in \cite{sigohi}*{Table 1}, i.e.\ one gets the `same' G\"odel hierarchy. \label{fooker}}.  This hierarchy is claimed to capture most systems that are natural or have foundational import, as follows. 
\begin{quote}
\emph{It is striking that a great many foundational theories are linearly ordered by $<$. Of course it is possible to construct pairs of artificial theories which are incomparable under $<$. However, this is not the case for the ``natural'' or non-artificial theories which are usually regarded as significant in the foundations of mathematics.} (\cite{sigohi})
\end{quote}
Burgess makes essentially the same claims in \cite{dontfixwhatistoobroken}*{\S1.5}.
However, the above results, as well as those in \cite{dagsamIII}, imply that e.g.\ $\HBU$, basic properties of the gauge integral, and uniform theorems, do not fit the G\"odel hierarchy.  In particular, these theorems yield a branch that is \emph{completely} independent of the medium range of the G\"odel hierarchy (with the latter based on inclusion$^{\ref{fooker}}$), as depicted in the following figure: 
%(where we assume the ordering based on inclusion): 
\begin{figure}[h]
\[
\begin{array}{lll}
&\textup{\textbf{strong}} \hspace{1.5cm}& 
\left\{\begin{array}{l}
\vdots\\
\textup{supercompact cardinal}\\
\vdots\\
\textup{measurable cardinal}\\
\vdots\\
\ZFC \\
\textsf{\textup{ZC}} \\
\textup{simple type theory}
\end{array}\right.
\\
&& \\
  &&~\quad{ {\Z_{2}^{\Omega}\equiv \RCA_{0}^{\omega}+(\exists^{3})}}\\
&&\\
&\textup{\textbf{medium}} & 
\left\{\begin{array}{l}
 {\Z}_{2}^{\omega} \equiv \cup_{k}\SIXK\\
\vdots\\
\textup{$\Pi_{2}^{1}\textsf{-CA}_{0}^{ {\omega}}$}\\
\textup{$\FIVE^{ {\omega}}$ }\\
\textup{$\ATR_{0}^{ {\omega}}$}  \\
\textup{$\ACA_{0}^{ {\omega}}$} \\
\end{array}\right.
%\begin{array}{c}
%\textup{Kohlenbach's}\\
%\textup{ {higher-order RM}}\\
%\end{array}
\\
&
\\
{ {\left\{\begin{array}{l}
\textup{covering lemmas like $\HBU$}\\
\textup{basic prop.\ of gauge integral}\\
\textup{uniform theorems like ${\PIT}_{\u}$}
\end{array}\right\}}}
&\begin{array}{c}\\\textup{\textbf{weak}}\\ \end{array}& 
\left\{\begin{array}{l}
\WKL_{0}^{ {\omega}} \\
\textup{$\RCA_{0}^{ {\omega}}$} \\
\textup{$\textsf{PRA}$} \\
\textup{$\textsf{EFA}$ } \\
\textup{bounded arithmetic} \\
\end{array}\right.
\\
\end{array}
\]
\caption{The G\"odel hierarchy with a side-branch for the medium range}\label{xxy}
\begin{picture}(250,0)
\put(155,200){ {\vector(-3,-2){125}}}
\put(160,117){ {\vector(-3,-2){53}}}
\put(100,70){ {\vector(1,0){80}}}
\put(125,100){ {\vector(3,2){50}}}
\put(150,100){{\line(-5,3){20}}}
\end{picture}
\end{figure}\\
Arguably, the G\"odel hierarchy is a central object of study in mathematical logic, as e.g.\ argued by Simpson in \cite{sigohi}*{p.\ 112} or Burgess in \cite{dontfixwhatistoobroken}*{p.\ 40}.  
Precursors to the G\"odel hierarchy may be found in the work of Wang (\cite{wangjoke}) and Bernays (See \cite{theotherguy}, and the English translation in \cite{puben}).
Friedman (\cite{friedber}) has studied the linear nature of the G\"odel hierarchy, including many more systems than present in Figure \ref{xxy}.

\smallskip

Some remarks on the technical details concerning Figure \ref{xxy} are as follows. 
\begin{remark}\label{fookie}\rm
First of all, we use a \emph{non-essential} modification of the G\"odel hierarchy, namely involving systems of higher-order arithmetic, like e.g.\ $\RCAo, \ACA_{0}^{\omega},$ $\FIVE^{\omega}$, and $ \Z_{2}^{\omega}$ instead of $\RCA_{0}, \ACA_{0}, \FIVE$, and $ \Z_{2}$; these higher-order systems are (at least) $\Pi_{2}^{1}$-conservative over the associated second-order system, by respectively \cite{kohlenbach2}*{\S2}, \cite{yamayamaharehare}*{Theorem 2.2}, and \cite{hunterphd}*{Cor.\ 2.6}.  
%\item In the spirit of RM, we show in \cite{dagsamV} that the Cousin lemma and (a version of) the Lindel\"of lemma are provable \emph{without} the use of $\QFAC^{0,1}$, as also discussed in Remark \ref{linpinpon}.

\smallskip
     
Secondly, $\Z_{2}^{\Omega}$ is placed \emph{between} the medium and strong range, as the combination of the recursor $\textsf{R}_{2}$ from G\"odel's $T$ and $\exists^{3}$ yields a system stronger than $\Z_{2}^{\Omega}$.  Note that $\SIXK$ and $\Z_{2}^{\omega}$ do not change in this way.    

\smallskip

Thirdly, in light of the extreme (logical and computational) differences between second-order and higher-order theorems (like e.g.\ $\HBU$ and its counterpart for countable covers), it is a natural questions how robust higher-order theorems actually are.  
As shown in \cite{sahotop}, the properties of the Cousin and Lindel\"of lemmas do not depend on the exact definition of cover, even in the absence of the axiom of choice.    
 
\end{remark}
The previous remark also establishes that the systems with superscript `$\omega$' deserve to be called the \emph{higher-order counterparts} of the corresponding second-order systems, 
while $\Z_{2}^{\Omega}$ does not seem to fall into the same category.  

\smallskip

Finally, in light of the equivalences involving the gauge integral and the Cousin lemma in \cite{dagsamIII}*{\S3}, the latter seriously challenges the `Big Five' classification from RM,  the linear nature of the G\"odel hierarchy,
%NEW&&
 as well as Feferman's claim that the mathematics necessary for the development of physics can be formalised in relatively weak logical systems (See \cite{dagsamIII}*{p.\ 24}).
%Note that in Figure \ref{xxy}, we used a non-essential modification of the G\"odel hierarchy, namely involving systems of higher-order arithmetic, like e.g.\ $\ACA_{0}^{\omega}$ instead of $\ACA_{0}$; these systems are (at least) $\Pi_{2}^{1}$-conservative over the associated second-order system (See e.g.\ \cite{yamayamaharehare}*{Theorem 2.2}).\\
%



\begin{ack}\rm
Our research was supported by the John Templeton Foundation, the Alexander von Humboldt Foundation, LMU Munich (via the Excellence Initiative and the Center for Advanced Studies of LMU), and the University of Oslo.
We express our gratitude towards these institutions. 
We thank Fernando Ferreira, Paul Rusnock, and Anil Nerode for their valuable advice.  
%We also thank the anonymous referee for the helpful suggestions.  
Opinions expressed in this paper do not reflect those of the John Templeton Foundation.    
\end{ack}



\begin{bibdiv}
\begin{biblist}
\bibselect{allkeida}
\end{biblist}
\end{bibdiv}
%\newpage
%We start from $\PIT_{o}$, i.e.\ 
%\[
%(\forall F, G:C\di \N) \big(\underline{(\forall f, g\in C) \big( g\in [\overline{f}G(f)] \di F(g)\leq G(f)\big)}    \di (\exists N\in \N)(\forall h\in C)(F(h)\leq N)\big).
%\]
%The underlined formula is purely universal, as it has no quantifiers except `$(\forall f, g\in C)$'.  Hence, we may apply $\textsf{IP}_{\forall}^{\omega}$ to obtain:
%\[
%(\forall F, G:C\di \N) (\exists N\in \N) \big({(\forall f, g\in C) \big( g\in [\overline{f}G(f)] \di F(g)\leq G(f)\big)}    \di(\forall h\in C)(F(h)\leq N)\big).
%\]
%Using Markov's principle $\textsf{M}^{\omega}$, we obtain (this step I am not sure of): 
%\[
%(\forall F, G:C\di \N) (\exists N\in \N)(\exists f, g\in C) \big({ \big( g\in [\overline{f}G(f)] \di F(g)\leq G(f)\big)}    \di(\forall h\in C)(F(h)\leq N)\big).
%\]
%Now we may bring outside the quantifier `$(\forall h\in C)$', and we are done
%\[
%(\forall F, G:C\di \N){(\exists N\in \N, f, g\in C})(\forall h\in C)\label{DOG} \big( \big( g\in [\overline{f}G(f)] \di F(g)\leq G(f)\big)    \di (F(h)\leq N)\big).
%\]

\bye
