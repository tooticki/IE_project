\documentclass[a4paper]{amsart}
\setlength{\marginparwidth}{1.2in}
%-------Packages---------
\usepackage{amssymb,amsfonts}
\usepackage[all,arc]{xy}
\usepackage{enumerate}
\usepackage{mathrsfs}
\usepackage{tikz}
\usetikzlibrary{arrows,snakes,backgrounds}
%\tikzset{>=stealth}
\usepackage{color}   %May be necessary if you want to color links
\usepackage{marginnote}
\usepackage{tikz-cd}
\usetikzlibrary{positioning}
 %\usepackage{relsize}
\usepackage{enumitem}
%\usepackage{nameref}
\usepackage{hyperref}
\usepackage{MnSymbol} 
%\usepackage[backref=true]{biblatex}
\hypersetup{
    colorlinks=true, %set true if you want colored links
    linktoc=all,     
    citecolor=black,
    filecolor=black,
    linkcolor=blue,
    urlcolor=black
}
\usepackage{cleveref}% Has to be loaded after hyperref
%\usepackage{natbib}
\usetikzlibrary{arrows}
%\hypersetup{
%    colorlinks=true, %set true if you want colored links
   % linktoc=all,     %set to all if you want both sections and subsections linked
    %linkcolor=blue,  %choose some color if you want links to stand out
%}
%\usepackage{titlesec}



\listfiles
%--------Theorem Environments--------
%theoremstyle{plain} --- default
\newtheorem{thm}{Theorem}[section]
\newtheorem{cor}[thm]{Corollary}
\newtheorem{prop}[thm]{Proposition}
\newtheorem{lem}[thm]{Lemma}
\newtheorem{conj}[thm]{Conjecture}
\newtheorem{quest}[thm]{Question}
\newtheorem{claim}[thm]{Claim}
\newtheorem{problem}[thm]{Problem}
\newtheorem*{openproblem*}{Problem}
\newtheorem*{quest*}{Question}
\newtheorem*{problem*}{Problem}

\theoremstyle{definition}
\newtheorem{defn}[thm]{Definition}
\newtheorem{defns}[thm]{Definitions}
\newtheorem{con}[thm]{Construction}
\newtheorem{exmp}[thm]{Example}
\newtheorem{exmps}[thm]{Examples}
\newtheorem{notn}[thm]{Notation}
\newtheorem{notns}[thm]{Notations}
\newtheorem{addm}[thm]{Addendum}
\newtheorem{exer}[thm]{Exercise}
\newtheorem*{cond}{Condition}

\theoremstyle{remark}
\newtheorem{rem}[thm]{Remark}
\newtheorem{rems}[thm]{Remarks}
\newtheorem{warn}[thm]{Warning}
\newtheorem{sch}[thm]{Scholium}
\newtheorem{obs}[thm]{Observation}


\newcommand{\bA}{\mathbb{A}}
\newcommand{\bB}{\mathbb{B}}
\newcommand{\bC}{\mathbb{C}}
\newcommand{\bD}{\mathbb{D}}
\newcommand{\bE}{\mathbb{E}}
\newcommand{\bF}{\mathbb{F}}
\newcommand{\bG}{\mathbb{G}}
\newcommand{\bH}{\mathbb{H}}
\newcommand{\bI}{\mathbb{I}}
\newcommand{\bJ}{\mathbb{J}}
\newcommand{\bK}{\mathbb{K}}
\newcommand{\bL}{\mathbb{L}}
\newcommand{\bM}{\mathbb{M}}
\newcommand{\bN}{\mathbb{N}}
\newcommand{\bO}{\mathbb{O}}
\newcommand{\bP}{\mathbb{P}}
\newcommand{\bQ}{\mathbb{Q}}
\newcommand{\bR}{\mathbb{R}}
\newcommand{\bS}{\mathbb{S}}
\newcommand{\bT}{\mathbb{T}}
\newcommand{\bU}{\mathbb{U}}
\newcommand{\bV}{\mathbb{V}}
\newcommand{\bW}{\mathbb{W}}
\newcommand{\bX}{\mathbb{X}}
\newcommand{\bY}{\mathbb{Y}}
\newcommand{\bZ}{\mathbb{Z}}

\newcommand{\MT}[2]{\bold{MT #1}(#2)}
\newcommand{\MTtheta}{\bold{MT \theta}}
\newcommand{\Q}[1]{\Omega^\infty_{#1} \Sigma^\infty}
\newcommand\lra{\longrightarrow}
\newcommand\lla{\longleftarrow}
\newcommand\trf{\mathrm{trf}}
\newcommand\Diff{\mathrm{Diff}}
\newcommand\Homeo{\mathrm{Homeo}}
\newcommand\BDiff{\mathrm{BDiff}}
\newcommand\dDiff{\mathrm{Diff}^{\delta}}
\newcommand\drDiff{\mathrm{Diff}^{r,\delta}}
\newcommand\dDiffo{\mathrm{Diff}^{\delta}_{0}}
\newcommand\BdDiff{\mathrm{BDiff}^{\delta}}
\newcommand\BdrDiff{\mathrm{BDiff}^{r,\delta}}
\newcommand\BdDiffo{\mathrm{BDiff}^{\delta}_{0}}
\newcommand\Emb{\mathrm{Emb}}
\newcommand\End{\mathrm{End}}
\newcommand\Bun{\mathrm{Bun}}
\newcommand\Th{\mathrm{Th}}
\newcommand\colim{\operatorname*{colim}}
\newcommand\hocolim{\operatorname*{hocolim}}
\newcommand\Coker{\operatorname*{Coker}}
\newcommand\Ker{\operatorname*{Ker}}
\newcommand\Cotor{\operatorname*{Cotor}}
\newcommand\Rk{\operatorname*{Rank}}
\newcommand{\cotensor}{\Box}
\newcommand\nmcg[1]{\mathcal{N}_{#1}}
\newcommand\ns[1]{S_{#1}}
\newcommand{\hcoker}{/\!\!/}
\newcommand{\hker}{\backslash\!\!\backslash}
\newcommand{\vect}[1]{\boldsymbol{#1}}
\newcommand{\X}{\mathbf{X}}
\newcommand{\K}{\mathcal{K}}
\newcommand{\Conn}{\mathbf{Conn}}
\newcommand{\Fib}{\mathrm{Fib}}
\newcommand{\SnSn}{\#^g S^n \times S^n}
\newcommand{\tE}{\text{E}}
\newcommand{\tH}{\text{\textnormal{Homeo}}}
\newcommand{\BH}{\mathrm{B}\text{\textnormal{Homeo}}}
\newcommand{\BwH}{\mathrm{B}\underline{\text{\textnormal{Homeo}}}}
\newcommand{\wH}{\underline{\text{\textnormal{Homeo}}}}
\newcommand{\tdH}{\text{Homeo}^{\delta}}
\newcommand{\tdwH}{\widetilde{\text{Homeo}}^{\delta}}
\newcommand{\BdH}{\mathrm{B}\text{\textnormal{Homeo}}^{\delta}}
\newcommand{\BdwH}{\mathrm{B}\underline{\text{\textnormal{Homeo}}}^{\delta}}
\newcommand{\dwH}{\underline{\text{\textnormal{Homeo}}}^{\delta}}
\newcommand{\tW}{\text{W}}
\newcommand{\W}{\text{W}_{g,1}}
\newcommand{\Su}[2]{\Sigma_{#1,#2}}
\newcommand{\WW}{\text{\textnormal{W}}_{g,1}}
\newcommand{\ttW}{\text{\textnormal{W}}}
\newcommand{\bCP}{\bC\mathrm{P}}
\newcommand{\bp}{^\bullet}
\newcommand{\ob}{\mathrm{ob}}
\newcommand{\map}{\mathrm{map}}
\newcommand{\gen}{\mathrm{span}}
\newcommand{\ind}{\mathrm{ind}}

%\setcounter{secnumdepth}{4}
%
%\titleformat{\paragraph}
%{\normalfont\normalsize\bfseries}{\theparagraph}{1em}{}
%\titlespacing*{\paragraph}
%{0pt}{3.25ex plus 1ex minus .2ex}{1.5ex plus .2ex}
%\makeatletter
%\renewcommand\paragraph{\@startsection{paragraph}{4}{\z@}%
%            {-2.5ex\@plus -1ex \@minus -.25ex}%
%            {1.25ex \@plus .25ex}%
%            {\normalfont\normalsize\bfseries}}
%\makeatother
%\setcounter{secnumdepth}{4} % how many sectioning levels to assign numbers to
%\setcounter{tocdepth}{4}    % how many sectioning levels to show in ToC

\addtocontents{toc}{\protect\setcounter{tocdepth}{1}}
%\setlength\emergencystretch\textwidth
\makeatletter
\let\c@equation\c@thm
\makeatother
\numberwithin{equation}{section}
\bibliographystyle{plain}
\title{A local to global argument on low dimensional manifolds}

\author{Sam Nariman}
\email{sam@math.northwestern.edu}
\address{Department of Mathematics\\
  Northwestern University\\
2033 Sheridan Road\\
Evanston, IL  60208}
%\author{Sam Nariman}
%\email{nariman@uni-muenster.de}
%\address{Mathematical Institute\\
%Universit{\"a}t M{\"u}nster\\
%Einsteinstr. 62\\
%D-48149 M{\"u}nster}

\begin{document}

\begin{abstract}
For   an orientable manifold $M$ whose dimension is less than $4$, we use the contractibility of certain complexes associated to its submanifolds to cut $M$ into simpler pieces in order to do  local to global arguments. One of the deep theorems of Thurston in foliation theory is a homology h-principle theorem  that says the natural map 
\[
\BdH(M)\to \BH(M),
\]
induces a homology isomorphism where $\tdH(M)$ denotes the group of homeomorphisms of $M$ made discrete. In low dimensions, we give a different proof of this theorem without using foliation theory. Secondly, we show that the same method applies to give  a different proof of the h-principle theorem in  smoothing theory i.e. the map 
\[ \BDiff(M)\to \BH(M)\]
is a weak equivalence. 

Finally,  we give a different proof that the diffeomorphism groups of Haken $3$-manifolds with boundary are homotopically discrete.%, in particular 
%we give a simpler proof of Gabai's theorem (Smale's conjecture for hyperbolic three manifolds, see \cite{gabai2001smale}) that the identity component of the diffeomorphism group of a hyperbolic three manifold is contractible.
%In most h-principle theorems, the local data is an statement about  open disks that one wishes to glue them together to prove the corresponding statement for a closed compact manifold $M$. But there are situations that one has a local statement for a closed disk relative to the boundary namely the Alexander trick for contractibility of the group of homeomorphisms of a disk relative to the boundary or the Mather's trick (\cite{MR0288777}) for acyclicity of the same group made discrete. The purpose of this short note is illustrate this idea by giving a short proof of the homology h-principle theorem of Thurston 
\end{abstract}
\maketitle
\section{Introduction}
Often, in h-principle type theorems (e.g. Smale-Hirsch theory), it is easy to check that the statement holds for the open disks (local data) and then one wishes to glue them together to prove that the statement holds for closed compact manifolds (global statement). But there are cases where one has a local statement for a closed disk relative to the boundary. To use such local data to great effect, instead of covering the manifold by open balls, we use certain ``resolutions" associated to submanifolds (see \Cref{sec2}) to cut the manifold into disks.

The first example of this sort is from  smoothing theory. Let $\Diff(D^n,\partial D^n)$  denote the group of compactly supported $C^{\infty}$-diffeomorphisms of the interior of the disk $D^n$ with $C^{\infty}$-topology and let $\tH(D^n,\partial D^n)$ denote the group of compactly supported homeomorphisms of the interior of the disk $D^n$ with the $C^0$-topology. By the Alexander trick, we know that the group $\tH(D^n,\partial D^n)$  is  contractible for all $n$. On the other hand, it is a well-known theorem of Smale that $\Diff(D^2,\partial D^2)$ is contractible and by the  theorem of Hatcher (\cite{hatcher1983proof}) so is $\Diff(D^3,\partial D^3)$. Therefore the natural map between classifying spaces
\[
\BDiff(D^n,\partial D^n)\to \BH(D^n,\partial D^n),
\]
is a weak equivalence for $n=2$ and $n=3$. The h-principle theorem in  smoothing theory for low dimensional manifolds says 
\begin{thm}[Earle-Eells, Hamstrom, Cerf]\label{sm} For a smooth manifold $M$, the map
\begin{equation}\label{BL}
\eta: \BDiff(M)\to \BH(M),
\end{equation}
is a weak equivalence provided $\text{dim}(M)=2$ (see \cite{hamstrom1974homotopy}) or $\text{dim}(M)=3$ (see \cite{cerf1961topologie}). Similar version holds for manifolds with boundary.
\end{thm}
 The second example is from foliation theory. Let $\tdH(D^n,\partial D^n)$ denote  the same group as $\tH(D^n,\partial D^n)$ but with the discrete topology. By an infinite repetition trick due to   Mather (\cite{MR0288777}), it is known that $\BdH(D^n,\partial D^n)$ is acyclic. Therefore, the natural map
 \[
 \BdH(D^n,\partial D^n)\to \BH(D^n,\partial D^n)
 \]
 induced by the identity homomorphism is in particular a homology isomorphism. Thurston generalized Mather's work on foliation theory in \cite{thurston1974foliations} and as a corollary he obtained  the following surprising result.
   \begin{thm}[Thurston]\label{Main}
  For a  smooth manifold $M$, the map 
  \[
  \iota: \BdH(M)\to \BH(M),
  \]
  induces an isomorphism on homology.
  \end{thm}
The first  proof of this theorem in the literature was given by McDuff following Segal's program in  foliation theory (see \cite{mcduff1980homology}). Thurston in fact proved a more general homology h-principle theorem for foliations such that \Cref{Main} is just its consequence for $C^0$-foliations. 

In foliation theory, Haefliger defined a topological groupoid $\Gamma_q^r$ whose space of objects are points in $\bR^q$ with the usual topology and the space of morphisms between two points is given by germs of $C^r$-diffeomorphisms sending $x$ to $y$ (see \cite[Section 1]{haefliger1971homotopy} for more details). The homotopy type of the classifying space of this groupoid, $\mathrm{B}\Gamma_q^r$, plays an important role in the classification of $C^r$-foliations (see \cite{MR0370619} and \cite{MR0425985}). One of Thurston's deep theorem in  foliation theory relates the homotopy type of $\mathrm{B}\Gamma_q^r$ to the group homology of $C^r$-diffeomorphism groups made discrete. For $r=0$, he first uses Mather's theorem (\cite{MR0288777}) to show that $\mathrm{B}\Gamma_q^0$ is weakly equivalent to the classifying space of rank $q$ microbundles, $\mathrm{B} \text{Top}(q)$, and as a consequence he deduces that the map $\iota$ in \Cref{Main} is in fact acyclic. 
 % The purpose of this paper is to illustrate a different way of local to global argument by first giving a short proof of the above homology h-principle theorem of Thurston in all dimensions except dimension $4$.

In the theory of foliations and smoothing theory respectively, people have studied the homotopy fiber of the maps $\iota$ and $\eta$. In fact there are general h-prinicple theorems in all dimensions that identify the homotopy fibers as holonomic sections of certain section spaces associated to the manifold. Our goal in this paper is to show that in low dimensions one can directly study the maps instead of their homotopy fibers. To do so, we provide the strategy in detail for \Cref{Main} in low dimensions that does not use any foliation theory and the method is general enough that can unify the proof of \Cref{sm} for both dimensions $2$ and $3$. 

The reason that we restrict ourselves to low dimensions is that for surfaces and $3$-manifolds, there is a procedure to split up  the manifold into disks. For the surfaces, this procedure is given by cutting along handles. But for $3$-manifolds, it is more subtle. To do so, we use the prime decomposition theorem and Haken's hierarchy to cut the manifold into disks. 

Finally using this technique, we also give a different proof of the contractibility of the identity component of diffeomorphisms in low dimensions. %First we revisit the case of hyperbolic surfaces and Haken three manifolds:
\begin{thm}[Earle-Schatz, Hatcher] The identity components of diffeomorphism groups of surfaces with boundary (see \cite{MR0277000}) and Haken manifolds with boundary (see \cite{MR0420620}) are contractible.
\end{thm}
Instead of working with the diffeomorphisms groups, we work with their classifying spaces. Considering the delooping of these topological groups has the advantage that one can apply homological techniques to the classifying spaces to extract homotopical information about diffeomorphism groups. 
%For a hyperbolic three manifold $M$, it was conjectured in \cite{kirby1995problems} that the isometry group $\text{Isom}(M)$ is homotopy equivalent to the group $\Diff(M)$. Using the Mostow rigidity (\cite{MR0236383}) it was known that $\text{Isom}(M)$ is a discrete group and there is a surjective map from $\pi_0(\Diff(M))$ to $\text{Isom}(M)$. To show that this map is injective, Gabai in \cite{gabai1997geometric} developed an ``insulator" machine to solve a very nontrivial ``homotopy implies isotopy" problem. After proving the $\pi_0$-part of the Smale conjecture for hyperbolic three manifolds, Gabai improved his ``insulator" technology in \cite{gabai2001smale} to prove $\text{Isom}(M)\simeq \Diff(M)$. We give a different proof of the higher homotopy part of Smale's conjecture which is equivalent to show that $\Diff_0(M)$ is contractible without using Gabai's ``insulator" techniques.
%\begin{thm}[Gabai]
%For a hyperbolic closed three manifold $M$, the identity component of the diffeomorphism group, $\Diff_0(M)$ is contractible.
%\end{thm}
%\vspace{.2cm}
\subsection{Outline} The paper is organized as follows: in \Cref{sec2}, we describe semi-simplicial resolutions for the classifying spaces of homeomorphisms using embedded submanifolds. We will treat the case of three manifolds separately because we first have to cut three manifolds into their prime pieces. In \Cref{sec3}, given the techniques of the previous section, we prove a theorem of Cerf that $\Diff_0(M)\hookrightarrow \Homeo_0(M)$  is a weak homotopy equivalence where $M$ is a three manifold. In \Cref{sec4}, we give a short proof of the contractibility of the identity component of the diffeomorphism groups for certain low dimensional manifolds.
%\noindent\textbf{Outline of the main idea:} We sketch the idea for \Cref{Main} which was the starting point for this paper. Let $M$ be a manifold with a handle decomposition and let $\tH_0(M)$ denote the group of the identity component of the topological group $\tH(M)$. Note that the group of connected components $\pi_0(\tH(M))$ is a discrete group and sits in a short exact sequence
%\[
%1\to \tH_0(M)\to \tH(M)\to \pi_0(\tH(M))\to 1. 
%\]
%An easy spectral sequence argument reduces \Cref{Main} to proving that  the map $$\BdH_0(M)\to \BH_0(M),$$ induces a homology isomorphism. To prove this reduced version, we fix a handle decomposition of $M$ and for each handle $\phi$, we construct a  semisimplicial space $A_{\bullet}(M,\phi)$ on which the topological group $\tH_0(M)$ acts (see \cite[Section 2]{randal2009resolutions} for definitions of (augmented) semisimplicial objects and their fat realizations). Similarly we construct a semisimplicial set $A^{\delta}_{\bullet}(M,\phi)$ from  the underlying semisimplicial set of the semisimplicial space $A_{\bullet}(M,\phi)$ on which the group $\tdH_0(M)$ acts. 
%These semisimplicial spaces are constructed so that their fat realizations are weakly contractible. Therefore we obtain semisimplicial resolutions \footnote{For a topological group $G$ acting on a topological space $X$, the homotopy quotient is denoted by $X\hcoker G$ and is given by $X\times_G \mathrm{E}G$ where $\mathrm{E}G$ is a contractible space on which $G$ acts freely.}
%\[
%|A^{\delta}_{\bullet}(M,\phi)\hcoker \tdH_0(M)|\xrightarrow{\simeq}\BdH_0(M),
%\]
%\[
%|A_{\bullet}(M,\phi)\hcoker \tdH_0(M)|\xrightarrow{\simeq}\BH_0(M).
%\] We then construct a zig-zag of maps from the space $A^{\delta}_{\bullet}(M,\phi)\hcoker \tdH_0(M)$ to the space $A_{\bullet}(M,\phi)\hcoker \tH_0(M)$ that induces a commutative diagram 
% \[
% \begin{tikzpicture}[node distance=5.6cm, auto]
% % \node (A) {$|A^{\delta}_{\bullet}(M,\phi)|$};
%  \node (B)  {$H_*(|A^{\delta}_{\bullet}(M,\phi)\hcoker \tdH_0(M)|;\bZ)$};
%  \node (C) [below of= B, node distance=1.6cm ] {$H_*(\BdH_0(M);\bZ)$};  
% % \node (D) [right of= A] {$|A_{\bullet}(M,\phi)|$};
%    \node (E) [right of= B] {$ H_*(|A_{\bullet}(M,\phi)\hcoker \tH_0(M)|;\bZ)$};
%  \node (F) [right of= C ] {$H_*(\BH_0(M);\bZ).$};  
%%   \draw [->] (A) to node {$$}(D);
%  \draw [->] (B) to node {$f_*$}(E);
%  \draw [->] (C) to node {$\iota_*$}(F);
% % \draw [->] (A) to node {$$}(B);
%  \draw [->] (B) to node {$\cong$} (C);
%% \draw [->] (D) to node {$$} (E);
%  \draw [->] (E) to node {$\cong$} (F);
%
%\end{tikzpicture}
%\]
% Therefore, it is enough to prove that $f_*$ is an  isomorphism.  As we shall see in \Cref{high}, proving that $f_*$ induces a homology isomorphism is equivalent to the statement of \Cref{Main} for a manifold with fewer handles than $M$. Then, by induction we can reduce \Cref{Main} to the case of a disk relative to its boundary that
%\[
%\BdH(D^n,\partial D^n)\to \BH(D^n,\partial D^n),
%\]
%induces a homology isomorphism. We restricted ourselves to low dimensions, because we still do not know how to make a certain surgery argument in \Cref{eq:7} work in  dimensions higher than $3$. 
\subsection*{Acknowledgment} I am indebted to Allen Hatcher for his careful  reading and many helpful comments on different versions of this paper. I first aimed to prove Thurston's theorem without using foliation theory in all dimensions but Hatcher pointed out a flaw in my argument for manifolds of dimension larger than $4$ which made me focus only on low dimensional manifolds.  I would also thank Sander Kupers for his comments on  transversality issues for topological manifolds and answering my questions about topological embeddings. Finally, I am also grateful to S\o ren Galatius for his comments on the first draft of this paper and to Kathryn Mann for her comment on \Cref{primedecomp}.  
\section{Resolving classifying spaces by embedded submanifolds}\label{sec2}
Let us first sketch the idea for \Cref{Main}. Let $M$ be a smooth manifold and let $\tH_0(M)$ denote the group of the identity component of the topological group $\tH(M)$. Note that the group of connected components $\pi_0(\tH(M))$ is a discrete group and sits in a short exact sequence
\[
1\to \tH_0(M)\to \tH(M)\to \pi_0(\tH(M))\to 1. 
\]
An easy spectral sequence argument reduces \Cref{Main} to proving that  the map $$\BdH_0(M)\to \BH_0(M),$$ induces a homology isomorphism. To prove this version, we want to inductively reduce \Cref{Main} to the case of a simpler manifold. Such simpler manifolds are obtained from $M$  by cutting along its submanifolds. Let $\phi$ be an embedding of a manifold into $M$. To cut along this embedding, we construct a  semisimplicial space $A_{\bullet}(M,\phi)$ on which the topological group $\tH_0(M)$ acts (see \cite{ebert2017semi} or \cite[Section 2]{randal2009resolutions} for definitions of (augmented) semisimplicial objects and their fat realizations). Similarly we construct a semisimplicial set $A^{\delta}_{\bullet}(M,\phi)$ from  the underlying semisimplicial set of the semisimplicial space $A_{\bullet}(M,\phi)$ on which the group $\tdH_0(M)$ acts. 
These semisimplicial spaces are constructed so that their fat realizations are weakly contractible. Therefore we obtain semisimplicial resolutions \footnote{For a topological group $G$ acting on a topological space $X$, the homotopy quotient is denoted by $X\hcoker G$ and is given by $X\times_G \mathrm{E}G$ where $\mathrm{E}G$ is a contractible space on which $G$ acts freely.}
\[
|A^{\delta}_{\bullet}(M,\phi)\hcoker \tdH_0(M)|\xrightarrow{\simeq}\BdH_0(M),
\]
\[
|A_{\bullet}(M,\phi)\hcoker \tdH_0(M)|\xrightarrow{\simeq}\BH_0(M).
\] We then construct a zig-zag of maps from the space $A^{\delta}_{\bullet}(M,\phi)\hcoker \tdH_0(M)$ to the space $A_{\bullet}(M,\phi)\hcoker \tH_0(M)$ that induces a homotopy commutative diagram 
 \[
 \begin{tikzpicture}[node distance=5.6cm, auto]
 % \node (A) {$|A^{\delta}_{\bullet}(M,\phi)|$};
  \node (B)  {$H_*(|A^{\delta}_{\bullet}(M,\phi)\hcoker \tdH_0(M)|;\bZ)$};
  \node (C) [below of= B, node distance=1.6cm ] {$H_*(\BdH_0(M);\bZ)$};  
 % \node (D) [right of= A] {$|A_{\bullet}(M,\phi)|$};
    \node (E) [right of= B] {$ H_*(|A_{\bullet}(M,\phi)\hcoker \tH_0(M)|;\bZ)$};
  \node (F) [right of= C ] {$H_*(\BH_0(M);\bZ).$};  
%   \draw [->] (A) to node {$$}(D);
  \draw [->] (B) to node {$f_*$}(E);
  \draw [->] (C) to node {$\iota_*$}(F);
 % \draw [->] (A) to node {$$}(B);
  \draw [->] (B) to node {$\cong$} (C);
% \draw [->] (D) to node {$$} (E);
  \draw [->] (E) to node {$\cong$} (F);

\end{tikzpicture}
\]
 Therefore, it is enough to prove that $f_*$ is an  isomorphism.  As we shall see in \Cref{high}, proving that $f_*$ induces a homology isomorphism is equivalent to the statement of \Cref{Main} for a manifold that is obtained from $M$ by cutting it along $\phi$. Then, by induction we can reduce \Cref{Main} to the case of a disk relative to its boundary that
\[
\BdH(D^n,\partial D^n)\to \BH(D^n,\partial D^n),
\]
induces a homology isomorphism. We restricted ourselves to low dimensions, because we still do not know how to make a certain surgery argument in \Cref{eq:7} work in  dimensions higher than $3$. 


 We want to cut up $M$ into disks in a ``contractible space of choices" (e.g. see \Cref{claim} and \Cref{eq:7}). As we shall explain at the end of \Cref{sec2}, the easiest case is when $M$ is homeomorphic to a circle (see also \cite[Theorem 4]{MR2871163}). For $M$ being a surface, we define certain space of handles to cut the surface along them. Finally if $M$ is a three manifold, we first reduce to the case of irreducible three manifolds and we cut it along incompressible surfaces in a ``contractible space of choices". For this reason, we consider the case of three manifolds separately.
\subsection{The case of surfaces} \label{disk}The first step is to reduce the statement of \Cref{Main}  to the case of the surfaces with boundary to be able to remove $1$-handles. Hence we first want to remove  disks ($0$-handles) from a closed surface $M$. To consider different choices of $0$-handles, we define a semisimplicial spaces. But we give the definitions in all dimensions and restrict to the case of surfaces whenever it is necessary. 
\begin{defn}[$0$-handle resolutions] \label{def1}We give both topological and discrete versions:
\begin{itemize}[leftmargin=*]\item {\bf Topological versions:}
Let $[p]$ denote the set $\{0,1,...,p\}$ of $p+1$ ordered elements. Let $$ A_p(M)=\text{\textnormal{Emb}}(\coprod_{[p]} D^n, M)$$ denote the subspace of the embedding space (equipped with compact-open topology) consisting of  orientation preserving  embeddings of $p$ disjoint union of $n$-disks that admit an external collar into the manifold $M$.  The collection $A_{\bullet}(M)$ is a semisimplicial space where the face maps are given by forgetting disks. 

We also define an auxiliary semisimplicial space $\overline{A}_{\bullet}(M)$ whose space of $0$-simplices is the same as $A_0(M)$ but its space of $p$-simplices is the subspace of $A_0(M)^{p+1}$ consisting of $(p+1)$-tuples $(\phi_0,\phi_1,\dots,\phi_p)$ where the centers of the embedded disks $\phi_i$ are pairwise disjoint.

Note that the group $\tH_0(M)$ acts on $A_p(M)$ \footnote{In fact the action is transitive in all dimensions thanks to the annulus theorem and the hypothesis on having external collar for embedded disks is imposed so that the annulus theorem holds.} transitively.  The $0$-handle resolution of $\BH_0(M)$ is defined to be the augmented semisimplicial space
\[
X_{\bullet}(M):= A_{\bullet}(M)\hcoker \tH_0(M)\to \BH_0(M).
\]
\item {\bf Discrete version:} We say two embeddings $g_1$ and $g_2$ in $ \text{\textnormal{Emb}}(\coprod_{[p]} D^n, M)$ have the same germ if there exists an open neighborhood $U\subset D^n$ around the origin so that $g_1|_{\coprod_{[p]} U}= g_2|_{\coprod_{[p]} U}$. Let
\[
A^{\delta}_{\bullet}(M):= \text{\textnormal{Emb}}^{\text{g},\delta}(\coprod_{[\bullet]} D^n, M),
\]
denote the set of germs of embeddings of disjoint union of $p+1$ disks compatible with the orientation of $M$. We define an auxiliary semisimplicial set $\overline{A}^{\delta}_{\bullet}(M)$ which is given by the underlying set of the semisimplicial space $\overline{A}_{\bullet}(M)$.

Also the $0$-handle resolution for $\BdH_0(M)$ is the augmented semisimplicial space
\[
X_{\bullet}^{\delta}(M):= A_{\bullet}^{\delta}(M)\hcoker \tdH_0(M)\to \BdH_0(M),
\]
\end{itemize}
\end{defn}
Note that there are natural maps $$A_{\bullet}(M)\xrightarrow{\simeq} \overline{A}_{\bullet}(M)\leftarrow \overline{A}^{\delta}_{\bullet}(M)\to A^{\delta}_{\bullet}(M) ,$$ where the first map is the inclusion (it is easy to see that it induces a weak homotopy equivalence levelwise), the second map is the identity map from the underlying set of a topological space to itself and the last map is induced by taking germs of embdeddings of disks at their centers.  

\subsubsection{The homotopy type of $X_p(M)$ and $X_p^{\delta}(M)$} To determine the homotopy type of $X_p(M)$, fix an element $e_p\in  A_p(M)$. Let $M\backslash e_p$ denote the manifold obtained from $M$ by removing the interior of the image of $e_p$. The action of $\tH_0(M)$ on $e_p$ gives rise to a map 
\begin{equation}
\tH_0(M)\to A_p(M).
\end{equation}
  Let $\tH(M,e_p)$ denote those homeomorphisms that are the identity on the image of $e_p$. The fiber over $e_p$ is the topological group $\underline{\tH}_{0}(M\backslash e_p,\partial(M\backslash e_p))$ whose identity component is $\tH_0(M,e_p)$. But given that the embedding $e_p$ has a collar, it is easy to show that the inclusion 
\begin{equation}\label{e_p}\tH_{0}(M\backslash e_p,\partial(M\backslash e_p))\hookrightarrow \tH_0(M,e_p),\end{equation}
is a weak homotopy equivalence. 
\begin{lem}\label{quasi-fib}
There is a quasi-fibration
\begin{equation}\label{eq1}
\underline{\tH_{0}}(M\backslash e_p,\partial(M\backslash e_p))\to \tH_0(M)\to A_p(M).
\end{equation}
\end{lem}
\begin{proof} We assume $p=0$ and the argument for the general case is the same. Let us recall the parametrized isotopy extension theorem in the topological setting (\cite[page 19]{burghelea1974homotopy}). We consider the simplicial set $\text{Emb}^{\text{lf}}_{\bullet}(D^n,M)$ whose $k$-simplices is given by locally flat embeddings $$\Delta^k\times D^n\hookrightarrow \Delta^k\times M,$$ that lives over projection to $\Delta^k$. Let $\text{Sing}_{\bullet}(\tH_0(M))$ be the singular set of the homeomorphism group. Fixing an element in  $e_0\in\text{Emb}^{\text{lf}}_{\bullet}(D^n,M)$, we have an evaluation map
\[
\text{Sing}_{\bullet}(\tH_0(M))\to \text{Emb}^{\text{lf}}_{\bullet}(D^n,M),
\]
which is a Kan fibration. It is a well known result of Quillen (\cite{MR0238322}) that the realization of the Kan fibration is a Serre fibration. Also for codimension zero similar to codimension higher than two (\cite[Appendix]{lashof1976}), the natural map
\[
|\text{Emb}^{\text{lf}}_{\bullet}(D^n,M)|\to \text{Emb}(D^n,M)=A_0(M),
\]
is a weak homotopy equivalence. And by the theorem of Milnor (\cite{MR0084138}), the natural map 
\[
|\text{Sing}_{\bullet}(\tH_0(M))|\to \tH_0(M),
\]
is a weak homotopy equivalence. Hence the evaluation map $\tH_0(M)\to A_0(M)$ is a quasi-fibration with the fiber $\underline{\tH_{0}}(M\backslash e_0,\partial(M\backslash e_0))$.
\end{proof}
Recall that for a group $G$ acting on a topological space $X$, we have a natural map $\mathrm{B}\text{Stab}(\sigma)\to X\hcoker G$ where $\text{Stab}(\sigma)$ is the stabilizer of an element $\sigma\in X$. Note that for $e_p\in A_p(M)$, the group $\wH_{0}(M\backslash e_p,\partial(M\backslash e_p))$ is the stabilizer of $e_p$ for the action of $\tH_0(M)$ on $A_p(M)$. Therefore, we have a map 
\begin{equation}\label{eq2}
h_p: \BwH_{0}(M\backslash e_p,\partial(M\backslash e_p))\xrightarrow{}A_p(M)\hcoker \tH_0(M)=X_p(M).
\end{equation}
\begin{prop}\label{prop1}
The map $h_p$ is a weak homotopy equivalence.
\end{prop}
\begin{proof}
For brevity, let us denote $\wH_{0}(M\backslash e_p,\partial(M\backslash e_p))$ by $H_p$. Note that we have the following homotopy commutative diagram
 \begin{equation}
 \begin{gathered}
 \begin{tikzpicture}[node distance=3.2cm, auto]
 % \node (A) {$H_p$};
  \node (B) {$H_p\hcoker H_p$};
  \node (C) [below of= B, node distance=1.5cm] {$\mathrm{B}H_p$};  
 % \node (D) [right of= A] {$H_p$};
    \node (E) [right of= B] {$\tH_0(M)\hcoker \tH_0(M) $};
  \node (F) [right of= C] {$A_p(M)\hcoker \tH_0(M).$};  
%   \draw [->] (A) to node {$$}(D);
  \draw [->] (B) to node {$$}(E);
  \draw [->] (C) to node {$h_p$}(F);
%  \draw [->] (A) to node {$$}(B);
  \draw [->] (B) to node {$$} (C);
% \draw [->] (D) to node {$$} (E);
  \draw [->] (E) to node {$g$} (F);
\end{tikzpicture}
\end{gathered}
\end{equation}
The left vertical map is a fibration with $H_p$ as the fiber. If we show that $g$ is a quasi-fibration with $H_p$ as the fiber., then the comparison of the long exact sequence of the (quasi)fibrations induced by vertical maps implies that $h_p$ is a weak homotopy equivalence.

Recall that for a group $G$ acting on a topological space $X$, the two sided bar construction $B_{\bullet}(X,G,*)=X\times G^{\bullet}$ is a simplicial space with the usual face maps and degeneracies. If $G$ is a well-pointed topological group, the realization of the two-sided bar construction $B_{\bullet}(X,G,*)$ is a model for the homotopy quotient. Since $\tH_0(M)$ is a well-pointed group (\cite{edwards1971deformations}), the map $g$ is induced by taking the geometric realization of the simplicial map
\[
g_{\bullet}: B_{\bullet}(\tH_0(M),\tH_0(M),*)\to B_{\bullet}(A_p(M),\tH_0(M),*).
\]

To prove that $g$ is a quasi-fibration whose homotopy fiber is $H_p$, it is enough to prove that the following diagram is homotopy cartesian
{\Small\[
 \begin{tikzpicture}[node distance=7.2cm, auto]
 % \node (A) {$H_p$};
  \node (B) {$\tH_0(M)=B_{0}(\tH_0(M),\tH_0(M),*)$};
  \node (C) [below of= B, node distance=1.5cm] {$A_p(M)=B_{0}(A_p(M),\tH_0(M),*)$};  
 % \node (D) [right of= A] {$H_p$};
    \node (E) [right of= B] {$|B_{\bullet}(\tH_0(M),\tH_0(M),*)|$};
  \node (F) [right of= C] {$|B_{\bullet}(A_p(M),\tH_0(M),*)|.$};  
%   \draw [->] (A) to node {$$}(D);
  \draw [->] (B) to node {$$}(E);
  \draw [->] (C) to node {$$}(F);
%  \draw [->] (A) to node {$$}(B);
  \draw [->] (B) to node {$$} (C);
% \draw [->] (D) to node {$$} (E);
  \draw [->] (E) to node {$g$} (F);
\end{tikzpicture}
\]}
Using Segal's gluing lemma (\cite[Proposition 1.6]{MR0353298}), it is enough to show that the diagram
\[
 \begin{tikzpicture}[node distance=6.2cm, auto]
 % \node (A) {$H_p$};
  \node (B) {$B_{k}(\tH_0(M),\tH_0(M),*)$};
  \node (C) [below of= B, node distance=1.5cm] {$B_{k}(A_p(M),\tH_0(M),*)$};  
 % \node (D) [right of= A] {$H_p$};
    \node (E) [right of= B] {$B_{k-1}(\tH_0(M),\tH_0(M),*)$};
  \node (F) [right of= C] {$B_{k-1}(A_p(M),\tH_0(M),*),$};  
%   \draw [->] (A) to node {$$}(D);
  \draw [->] (B) to node {$d_i$}(E);
  \draw [->] (C) to node {$d_i$}(F);
%  \draw [->] (A) to node {$$}(B);
  \draw [->] (B) to node {$g_k$} (C);
% \draw [->] (D) to node {$$} (E);
  \draw [->] (E) to node {$g_{k-1}$} (F);
\end{tikzpicture}
\]
is a homotopy cartesian for all face maps $d_i, 0\leq i\leq k$. Since the multiplication in $\tH_0(M)$ has an inverse (see the discussion after \cite[Proposition 1.6]{MR0353298}), it is enough to show that the diagram

\begin{equation}\label{facemaps}
\begin{gathered}
 \begin{tikzpicture}[node distance=4.2cm, auto]
 % \node (A) {$H_p$};
  \node (B) {$\tH_0(M)\times\tH_0(M)$};
  \node (C) [below of= B, node distance=1.5cm] {$A_p(M)\times\tH_0(M)$};  
 % \node (D) [right of= A] {$H_p$};
    \node (E) [right of= B] {$\tH_0(M)$};
  \node (F) [right of= C] {$A_p(M),$};  
%   \draw [->] (A) to node {$$}(D);
  \draw [->] (B) to node {$d_1$}(E);
  \draw [->] (C) to node {$d_1$}(F);
%  \draw [->] (A) to node {$$}(B);
  \draw [->] (B) to node {$g_1$} (C);
% \draw [->] (D) to node {$$} (E);
  \draw [->] (E) to node {$g_0$} (F);
\end{tikzpicture}
\end{gathered}
\end{equation}

is a homotopy cartesian. For this, note that $g_0$ and $g_1$ are quasi-fibrations and for example the fiber of $g_1$ over $(e_p,f)$ is $\tH_0(M,e_p)$ (see \Cref{e_p}) and the fiber of $g_0$ over $f(e_p)$ is $\tH_0(M,f(e_p))$. Given that the action of $\tH_0(M)$ on $A_p(M)$ is transitive, these two groups $\tH_0(M,e_p)$ and $\tH_0(M,f(e_p))$ are homeomorphic. Therefore, the diagram \ref{facemaps} is homotopy cartesian.
\end{proof}
\begin{rem}
The author does not know if the quasi-fibration \ref{eq1} is a locally trivial bundle similar to the smooth category. If this were true, the space $A_p(M)$ would become homeomorphic to $\tH_0(M)/H_p$ and therefore the proof of \Cref{prop1} would become much easier. 
\end{rem}

It is easier to determine the homotopy type of $X^{\delta}_p(M)$.  To do so, let $M(e_p)$ denote the manifold $M$ with  $(p+1)$ punctures at the centers of the germs of embedding of disks $e_p$ in $M$. We may consider $e_p$ as an element of $A_p^{\delta}(M)$, and denote the stabilizer of the element $e_p$ under the action of $\tdH_0(M)$ on $A_p^{\delta}(M)$ by $\underline{\tH_{0,c}}^{\delta}(M(e_p))$.  Let   $\underline{\tH_{0,c}}(M(e_p))$ be the same group but consider it as a subgroup of $\tH_0(M)$ with the subspace topology. It is useful (in particular in the diagram \ref{e}) to note that the group of connected components of $\underline{\tH_{0,c}}(M(e_p))$ is the same as the group of components of $\underline{\tH_{0}}(M\backslash e_p,\partial(M\backslash e_p))$. Hence, we have a short exact sequence
\begin{equation}\label{connectedcomponent}1\to \tdH_0((M\backslash e_p)\to \underline{\tH_{0,c}}^{\delta}(M(e_p))\to \pi_0(\underline{\tH_{0}}(M\backslash e_p,\partial(M\backslash e_p)))\to 1.
\end{equation}
Recall that  Shapiro's lemma for discrete groups $H<G$ says that the natural map $\mathrm{B}H\to (G/H)\hcoker G$ is a weak homotopy equivalence. Therefore, the map 
\begin{equation}\label{eq3}
 \BdwH_{0,c}(M(e_p))\xrightarrow{\simeq}X^{\delta}_p(M),
\end{equation}
is also a weak homotopy equivalence.
%\begin{rem}Note that the inclusion 
%\[
%\underline{\tH}_0(M\backslash e_p,\partial (M\backslash e_p))\hookrightarrow \underline{\tH}_{0,c}(M(e_p)),
%\]
%is a weak homotopy equivalence, therefore the above two topological groups have the same group of components.
%\end{rem}
\subsubsection{A lemma in homotopy theory}
Here the goal is to show that $ |\overline{A}^{\delta}_{\bullet}(M)|$ and $|A_{\bullet}(M)|$ are weakly contractible. Proving the fat realization of the discrete version $ |\overline{A}^{\delta}_{\bullet}(M)|$ is contractible is easier. Using a  lemma  in homotopy theory, we show that the contractibility of $ |\overline{A}^{\delta}_{\bullet}(M)|$ implies the weak contractibility of  $|A_{\bullet}(M)|$. This technique is originally due to Segal (\cite[Appendix]{segal1978classifying}) and it is reformulated by Weiss in  \cite[Lemma 2.2]{weiss2005does}. In particular, in the setting of semi-simplicial spaces, we use an application of this technique (\cite[Proposition 2.8]{galatius2014homological}) due to Galatius and Randal-Williams.
\begin{prop}\label{claim}
The realizations $|A^{\delta}_{\bullet}(M)|$ and $ |\overline{A}^{\delta}_{\bullet}(M)|$ are weakly contractible.
\end{prop}
\begin{proof}
We give a proof for weak contractibility of $|A^{\delta}_{\bullet}(M)|$, the case of $ |\overline{A}^{\delta}_{\bullet}(M)|$ is similar. Let $ f: S^k\to |A^{\delta}_{\bullet}(M)|$ be an element in the $k$-th homotopy group of $|A^{\delta}_{\bullet}(M)|$. Since $|A^{\delta}_{\bullet}(M)|$ is a CW-complex and $S^k$ is compact, the map $f$ hits finitely many simplices of $|A^{\delta}_{\bullet}(M)|$. Hence, there exists a point   ${\bf p}$ and an embedded disk $e(D^n)$ around it such that as an element of $A^{\delta}_{0}(M)$ is not hit by the map $f$. Thus, we have $f(S^k)\subset |A^{\delta}_{\bullet}(M\backslash e(D^n))|$. Adding the germ of $e$ at ${\bf p}$ to the list of germs of embeddings of disks in $M\backslash e(D^n)$ gives a semisimplicial null-homotopy for the inclusion $A^{\delta}_{\bullet}(M\backslash e(D^n))\hookrightarrow A^{\delta}_{\bullet}(M)$. Therefore, the element $f(S^k)$ can be coned off inside $|A^{\delta}_{\bullet}(M)|$.
\end{proof}
\begin{rem}
Note that because $|A^{\delta}_{\bullet}(M)|$ and $ |\overline{A}^{\delta}_{\bullet}(M)|$ have CW structures, they are in fact contractible.
\end{rem}
Since the space $A^{\delta}_{\bullet}(M)$ is discrete and $A_{\bullet}(M)$ is compactly generated weak Hausdorff space, by \cite[Lemma 2.1]{randal2009resolutions}, the maps
\begin{equation}\label{ee}
\begin{gathered}
|X_{\bullet}^{\delta}(M)|\xrightarrow{} \BdH_0(M),\\
|X_{\bullet}(M)|\xrightarrow{} \BH_0(M),
\end{gathered}
\end{equation}
are locally trivial fiber bundles with fibers $|A^{\delta}_{\bullet}(M)|$ and $|A_{\bullet}(M)|$ respectively. Therefore, by \Cref{claim} the first map $|X_{\bullet}^{\delta}(M)|\xrightarrow{\simeq} \BdH_0(M)$ is a weak homotopy equivalence. To prove that the second map is also a weak homotopy equivalence, we need to show that $|A_{\bullet}(M)|$ is weakly contractible. To do so, we use the bisimplicial technique due to Quillen \cite[Proof of Theorem A]{MR0338129}. First note that since the map
\[
A_{\bullet}(M)\xrightarrow{\simeq} \overline{A}_{\bullet}(M)
\]
is a weak equivalence, it induces a weak homotopy equivalence between the fat realizations. Hence, to show that $| A_{\bullet}(M)|$ is weakly contractible, it is enough to show that in the zig-zag
\begin{equation}\label{eq:3}
A_{\bullet}(M)\xrightarrow{\simeq} \overline{A}_{\bullet}(M)\xleftarrow{\beta} \overline{A}^{\delta}_{\bullet}(M)
\end{equation}
the second map $\beta$ induces a weak homotopy equivalence between fat realizations.  Note that $\beta$ is equivariant with respect to the map $\tdH_0(M)\to \tH_0(M)$ and the first map is equivariant with respect to the action of $\tH_0(M)$ on its both sides.
\begin{defn}
Let $A_{\bullet,\bullet}(M)$ be the bisemisimplicial space such that $A_{p,q}(M)$ is the subspace of  $\overline{A}^{\delta}_{p}(M)\times  \overline{A}_{q}(M)$ consisting of those $(p+q+2)$-tuples $$(a_0,\dots,a_p,c_0,\dots,c_q),$$ where the centers of the disks $a_i$ and the disks $c_j$ are pairwise disjoint.
\end{defn}
The bisemisimplicial space $A_{\bullet,\bullet}(M)$ is augmented in two different directions
\[
\epsilon_p:A_{p,\bullet}(M)\to \overline{A}^{\delta}_{p}(M),
\]
\[
\delta_q: A_{\bullet,q}(M)\to \overline{A}_{q}(M).
\]
Similar to \cite[Lemma 5.8]{galatius2014homological} ,one can show  that the following diagram is homotopy commutative
 \begin{equation}\label{commdiagram}\begin{gathered}
% \begin{tikzpicture}[node distance=2cm, auto]
%  \node (A) {$|A_{\bullet}^{\delta}(M)|$};
%  \node (B) [above of=A, node distance=1.7cm]{$|A_{\bullet,\bullet}(M)|$};
%  \node (C) [right of=A, node distance=4cm]{$|A^g_{\bullet}(M)|.$};
%      \node (D)[above of=C,node distance=1.7cm]  {$|A_{\bullet}(M)|$};
%  \draw [<-] (A) to node {$\epsilon$} (B);
%  \draw [->] (B) to node {$\delta$} (D);
%  \draw [-> ] (A) to node {$\beta$} (C);
%    \draw [->] (D) to node {$\simeq$} (C);
% \end{tikzpicture}
  \begin{tikzpicture}[node distance=2.5cm, auto]
  \node (A) {$|\overline{A}^{\delta}_{\bullet}(M)|$};
  \node (B) [right of=A, below of=A, node distance=1.7cm]{$|A_{\bullet,\bullet}(M)|.$};
  \node (C) [right of=A, node distance=5cm]{$|\overline{A}_{\bullet}(M)|$};
  \draw [->] (A) to node {$$} (C);
  \draw [->] (B) to node {$\delta$} (C);
  \draw [<-] (A) to node {$\epsilon$} (B);
 \end{tikzpicture}
\end{gathered}
\end{equation}
\begin{prop}\label{claim1}
The fat realization $|A_{\bullet}(M)|$ is weakly contractible.
\end{prop}
\begin{proof}
Since $|A_{\bullet}(M)|\xrightarrow{\simeq}|\overline{A}_{\bullet}(M)|$, we instead show that $|\overline{A}_{\bullet}(M)|$ is weakly contractible. Because the diagram \ref{commdiagram} is homotopy commutative and $|\overline{A}^{\delta}_{\bullet}(M)|$ is weakly contractible, if we show that the map $\delta$ is a weak homotopy equivalence, we then deduce that $|\overline{A}_{\bullet}(M)|$ is also weakly contractible.

Let $\mathcal{Q}$ be in $\overline{A}_{q}(M)$. By the definition of the bisemisimplicial space $A_{\bullet,\bullet}(M)$, the fiber of the map $\delta_q$ over $\mathcal{Q}$ is
\[
\delta_q^{-1}(\mathcal{Q})=\overline{A}_{\bullet}^{\delta}(M\backslash \text{centers of}(\mathcal{Q})).
\]
Note that by \Cref{claim}, we know that $|\delta_q^{-1}(\mathcal{Q})|$ is contractible. Using \cite[Proposition 2.8]{galatius2014homological}, we deduce that the map \begin{equation}\label{eq:1}|\delta_q|:|A_{\bullet,q}(M)|\to \overline{A}_{q}(M),\end{equation} is a microfibration with a contractible fiber, hence it is a fibration (see \cite[Lemma 2.2]{weiss2005does} or \cite[Proposition 2.6]{galatius2014homological}). Therefore $|\delta_q|$ induces a weak equivalence. By realizing in $q$-direction of both sides of the map $|\delta_q|$ in \ref{eq:1}, we deduce that $\delta$ is also a weak homotopy equivalence.
\end{proof}
\subsubsection{Reducing \Cref{Main} to the case of manifolds with boundary} Recall that the maps in \ref{ee} are weak homotopy equivalence so the semi-simplicial spaces $X_{\bullet}(M)$ and $X^{\delta}_{\bullet}(M)$ are resolutions for $\BH_0(M)$ and $\BdH_0(M)$ respectively. Therefore, to compare  $\BH_0(M)$ and $\BdH_0(M)$, we need to compare their resolutions. But there is no direct map between them. We, however, show that there is a map on the level of homology induced by  the zig-zag of maps
\begin{equation}\label{eq:4}
X_{\bullet}^{\delta}(M)\leftarrow \overline{A}^{\delta}_{\bullet}(M)\hcoker \tdH_0(M) \to X_{\bullet}(M),
\end{equation}
which in turn is induced by the zig-zag $A^{\delta}_{\bullet}(M)\leftarrow \overline{A}^{\delta}_{\bullet}(M)\to A_{\bullet}(M)$. 

Given any $p$-simplex $\sigma$ in $\overline{A}^{\delta}_{p}(M)$, we have a map
\[
\mathrm{BStab}(\sigma)\to \overline{A}^{\delta}_{\bullet}(M)\hcoker \tdH_0(M),
\]
where $\mathrm{Stab}(\sigma)$ is the stabilizer of $\sigma$ under the action $\tdH_0(M)$. Recall that we fixed an element $e_p\in A_p(M)$ in the quasi-fibration \ref{eq1}, we can consider the same element $e_p\in \overline{A}^{\delta}_{p}(M)$ and therefore the stabilizer of $e_p$ is the group $\dwH_{0}(M\backslash e_p,\partial(M\backslash e_p))$. Thus we have a map 
\[
\BdwH_0(M\backslash e_p,\partial(M\backslash e_p))\to  \overline{A}^{\delta}_{\bullet}(M)\hcoker \tdH_0(M).
\]

Given the weak equivalences \ref{eq1}, \ref{eq3} and the zig-zag \ref{eq:4}, we have a homotopy commutative diagram 
{\small
\begin{equation}\label{e}
\begin{gathered}
\begin{tikzcd}
X_p^{\delta}(M)& \overline{A}^{\delta}_{p}(M)\hcoker \tH_0(M)\arrow{l}\arrow[""]{r}&X_{p}(M) \\ \BdwH_{0,c}(M(e_p))\arrow["\simeq"]{u}&\BdwH_0(M\backslash e_p,\partial(M\backslash e_p))\arrow[""]{l}\arrow[]{u}\arrow[""]{r}& \BwH_0(M\backslash e_p,\partial(M\backslash e_p))\arrow["\simeq"]{u}.
\end{tikzcd}
 \end{gathered}
 \end{equation}}
The short exact sequence \ref{connectedcomponent} and \cite[Corollary 2.3]{nariman2014homologicalstability} \footnote{This corollary that says certain pushing collar maps between diffeomorphism groups induce homology isomorphisms also works for homeomorphisms} implies that the bottom left  horizontal map induces a homology isomorphism. Therefore, we have a zig-zag
\begin{equation}\label{eq}
\begin{gathered}
 X_{p}^{\delta}(M)\xleftarrow{H_*-\text{iso}} \BdwH_0(M\backslash e_p,\partial(M\backslash e_p))\to X_{p}(M),
 \end{gathered}
 \end{equation}
 which induces a map $\alpha_*: H_*(X_{\bullet}^{\delta}(M))\to H_*(X_{\bullet}(M))$.  Since the map $\alpha_*$ commutes with the face maps, we have an induced map between the spectral sequences given by the skeletal filtration of the realizations
%As we shall prove below in \Cref{lem1} that $h$ induces a homology isomorphism, therefore the zig-zag \ref{eq:4} induces a map 
%\[
%\alpha_* : H_*(X_{\bullet}^{\delta}(M))\to H_*(X_{\bullet}(M)),
%\]
%that is compatible with the face maps.
%\begin{lem}\label{lem1}
%The map $h$ in the zig-zag \ref{eq:4} induces a weak equivalence.
%\end{lem}
%\begin{proof}
%Note that we have a map between fibrations
%\[
%\begin{tikzcd}
%A_{\bullet}(M)\arrow["\simeq"]{r}\arrow[rightarrow]{d}&A^g_{\bullet}(M)\arrow[rightarrow]{d}\\X_{\bullet}(M)\arrow["h"]{r}\arrow{d}& A^g_{\bullet}(M)\hcoker \tH_0(M)\arrow{d}\\ \BH_0(M)\arrow["id"]{r}& \BH_0(M).
%\end{tikzcd}
%\]
%Hence, by the five lemma for the long exact sequence of homotopy groups of the two fibrations, we deduce that $h$ is a weak equivalence.%the comparison of the Serre spectral sequences implies that $h$ induces a homology isomorphism.
%\end{proof}

%is equivariant with respect to the homomorphism $\tdH_0(M)\to \tH_0(M)$. Hence, we obtain a semisimplicial map
%\[
%\alpha:X_{\bullet}^{\delta}(M)\to X_{\bullet}(M).
%\]
%The map $\alpha$ induces a comparison map between the following two spectral sequences
% \[
% \begin{tikzpicture}[node distance=3.9cm, auto]
%  \node (A) {$H_q(X_{p}^{\delta}(M))$};
%  \node (B) [below of=A, node distance=1.6cm] {$H_{p+q}(|X_{\bullet}^{\delta}(M)|)$};
%  \node (C) [below of= B, node distance=1.6cm ] {$H_{p+q}(\BdH_0(M))$};  
%  \node (D) [right of= A] {$H_q(X_{p}(M))$};
%    \node (E) [right of= B] {$ H_{p+q}(|X_{\bullet}(M)|)$};
%  \node (F) [right of= C ] {$H_{p+q}(\BH_0(M)).$};  
%   \draw [->] (A) to node {$$}(D);
%  \draw [->] (B) to node {$$}(E);
%  \draw [->] (C) to node {$\iota_*$}(F);
%  \draw [double, shorten <=1pt,>=angle 90,thick, ->] (A) to node {$$}(B);
%  \draw [->] (B) to node {$\cong$} (C);
% \draw [shorten <=1pt,>=angle 90,thick,double,->, ] (D) to node {$$} (E);
%  \draw [->] (E) to node {$\cong$} (F);
%
%\end{tikzpicture}
%\]
\begin{equation}\label{eq:5}
\begin{gathered}
\begin{tikzcd}
H_q(X_{p}^{\delta}(M))\arrow["\alpha_*"]{r}\arrow[Rightarrow]{d}&H_q(X_{p}(M))\arrow[Rightarrow]{d}\\H_{p+q}(|X_{\bullet}^{\delta}(M)|)\arrow["\tilde{\iota}_*"]{r}\arrow["\cong"]{d}& H_{p+q}(|X_{\bullet}(M)|)\arrow["\cong"]{d}\\ H_{p+q}(\BdH_0(M))\arrow["\iota_*"]{r}& H_{p+q}(\BH_0(M)).
\end{tikzcd}
\end{gathered}
\end{equation}
To reduce Thurston's theorem to the case of manifolds with boundary, we need the following lemma.
\begin{prop}\label{lem2}
Given Thurston's theorem \ref{Main} for manifolds with boundary, the map $\alpha_*$ is an isomorphism
\end{prop}
\begin{proof}
%Given the weak equivalences \ref{eq1}, \ref{eq3} and the zig-zag \ref{eq:4}, we have a homotopy commutative diagram 
%\[
%\begin{tikzcd}
%X_{\bullet}^{\delta}(M)\arrow{r}& A^g_{\bullet}(M)\hcoker \tH_0(M)&X_{\bullet}(M)\arrow["h"]{l} \\ \BdwH_{0,c}(M(e_p))\arrow["g"]{r}\arrow["\simeq"]{u}& \BwH_{0,c}(M(e_p))\arrow["H_*\text{-iso}"]{u}& \BwH_0(M\backslash e_p,\partial(M\backslash e_p))\arrow["\simeq"]{l}\arrow["\simeq"]{u}.
%\end{tikzcd}
%\]
%The bottom right map is weakly equivalent because the inclusion of topological groups
%\[
%\underline{\tH}_0(M\backslash e_p,\partial (M\backslash e_p))\hookrightarrow \underline{\tH}_{0,c}(M(e_p)),
%\]
%is a weak equivalence. Thus, the middle vertical map induces a homology isomorphism because $h$ does. Hence, it is enough to show that $g$ induces a homology isomorphism given the hypothesis of the lemma. Consider the homotopy commutative diagram 
%\[
% \begin{tikzpicture}[node distance=2.9cm, auto]
%  \node (A) {$\BdH_{0,c}(M(e_p))$};
%  \node (B) [above of=A, node distance=1.7cm]{$ \BdH_0(M\backslash e_p,\partial(M\backslash e_p))$};
%  \node (C) [right of=A, node distance=4.9cm]{$\BH_{0,c}(M(e_p)).$};
%      \node (D)[above of=C,node distance=1.7cm]  {$ \BH_0(M\backslash e_p,\partial(M\backslash e_p))$};
%  \draw [<-] (A) to node {$H_*\text{-iso}$} (B);
%  \draw [->] (B) to node {$$} (D);
%  \draw [-> ] (A) to node {$g'$} (C);
%    \draw [->] (D) to node {$\simeq$} (C);
% \end{tikzpicture}
% \]
Given the diagram \ref{e}, proving $\alpha_*$ is an isomorphism is equivalent to proving the map
\[
\BdwH_0(M\backslash e_p,\partial(M\backslash e_p))\to \BwH_0(M\backslash e_p,\partial(M\backslash e_p))
\]
induces a homology isomorphism. On the other hand, by the hypothesis, we know that the map 
\[
 \BdH_0(M\backslash e_p,\partial(M\backslash e_p))\to  \BH_0(M\backslash e_p,\partial(M\backslash e_p)),
\]
induces a homology isomorphism. Recall that the identity component of the topological group $\underline{\tH}_0(M\backslash e_p,\partial(M\backslash e_p))$ is weakly homotopy equivalent to the group $\tH_0(M\backslash e_p,\partial(M\backslash e_p))$. Now using the comparison of Serre spectral sequences for the fibrations
 \[
 \begin{tikzcd}
 \BdH_0(M\backslash e_p,\partial(M\backslash e_p))\arrow["'"]{r}\arrow[rightarrow]{d}& \BH_0(M\backslash e_p,\partial(M\backslash e_p))\arrow[rightarrow]{d}\\\BdwH_{0}(M\backslash e_p,\partial(M\backslash e_p))\arrow[""]{r}\arrow{d}& \BwH_{0}(M\backslash e_p,\partial(M\backslash e_p))\arrow{d}\\ \mathrm{B}\pi_0(\underline{\tH}_{0}(M\backslash e_p,\partial(M\backslash e_p)))\arrow["\cong"]{r}& \mathrm{B}\pi_0(\underline{\tH}_{0}(M\backslash e_p,\partial(M\backslash e_p)),
\end{tikzcd}
 \]
 we readily conclude that the middle horizontal map induces a homology isomorphism. 
\end{proof} 
 Hence, if we prove Thurston's theorem for manifolds without $0$-handles or more generally for manifolds with boundary, the comparison map on the $E^1$-page
\[
E^1_{p,q}(X_{p}^{\delta}(M))\xrightarrow{\alpha_*} E^1_{p,q}(X_{p}(M)),
\]
is an isomorphism, so is on the $E^{\infty}$-page. Hence, we deduce that $\tilde{\iota}$ in the diagram \ref{eq:5} induces a homology isomorphism which implies Thurston's theorem for the closed manifold $M$.
%\begin{rem}
%One can use acylicity of the group $\tdH_c(\bR)$ and the $0$-handle resolution in dimension one to prove that $\BdH(\bS^1)$ and $\BH(\bS^1)$ has the same homology (see also \cite{jekel2011euler} for a rather different argument). 
%\end{rem}
\subsubsection{Higher dimensional handles}\label{high}
Now we want to cut along the core of the higher dimensional handles. To do so, we use the same notation for the handlebody decomposition as in \cite{crowley2014surgery}. To recall their notation, let $W$ be a  manifold with boundary. To attach a handle of index $q$, let $\tilde{\phi^q}: S^{q-1}\times D^{n-q} \hookrightarrow  \partial W$ be an  embedding that admits an external collar similar to the $0$-handle case. Let $W\plus (\phi^q)$ denote the manifold  $W\cup_{\tilde{\phi^q}} D^q\times D^{n-q}$. And let $\phi^q$ denote the embedding $D^q\times D^{n-q}\hookrightarrow W+(\phi^q)$.

\begin{defn}\label{germ}
We say two handles $\phi_1, \phi_2: D^q\times D^{n-q}\hookrightarrow M$ have the same germ around the core if there exists $\epsilon>0$ such that 
\[
\phi_1|_{D^q\times D^{n-q}_{\epsilon}}=\phi_2|_{D^q\times D^{n-q}_{\epsilon}}
\]
where $D^{n-q}_{\epsilon}$ consists of all $x\in D^{n-q}$ with the norm $|x|\leq \epsilon$. We denote the class of the germ of $\phi_i$ by $[\phi_i]$. 
\end{defn}
Given what we proved in the previous section, we can assume that $M$ is a manifold with boundary whose boundary components are in fact homeomorphic to spheres. %We write the boundary $\partial M$ as a disjoint sum $\partial_0 M\coprod \partial_1 M$ where $\partial_0 M\cong S^{n-1}$. We think of $M$ as a bordism between $\partial_0 M$ and $\partial_1 M$ and by the handle body decomposition theorem, we can obtain the bordism $M$ from the trivial bordism $\partial_0 M\times [0,1]$ by attaching handles to $\partial_0 M\times \{ 1\}$. Hence, we write 
%\begin{equation}\label{decomp}
%M\cong \partial_0 M\times [0,1]+(\phi_1^{q_1})+(\phi_2^{q_2})+\cdots+(\phi_r^{q_r}).
%\end{equation}
%We can interchange the role of $\partial_0 M$ and $\partial_1 M$ and obtain $M$ by attaching handles to $\partial_1 M$. Therefore, $M$ is homeomorphic to 
%\[
%M\cong \partial_1 M\times [0,1]+(\psi_1^{n-q_1})+(\psi_2^{n-q_2})+\cdots+(\psi_r^{n-q_r}).
%\]
%This is called {\it dual handle decomposition} (see \cite[Section 1.4]{crowley2014surgery}) and its effect is the number of $q$ handles become the number of $n-q$ handles in the handle decomposition \ref{decomp}.

To reduce the problem to a manifold with fewer number of handles, we use the same idea as $0$-handle resolutions. We  shall define  semisimplicial spaces encoding the space of choices of removing a handle. 
\begin{defn}[$q$-handle resolutions]\label{def2}
There are versions with different topologies as \Cref{def1}:
\begin{itemize}[leftmargin=*]\item {\bf Discrete versions:} Let $\phi^q: D^q\times D^{n-q}\hookrightarrow M$ be a $q$-handle with an external collar such that $\phi^q(D^q\times D^{n-q})\cap \partial M=\phi^q(S^{q-1}\times D^{n-q})$. We first define a semisimplicial set  $\mathcal{H}^{\delta}_{\bullet}(M,\phi^q)$ associated to $\phi^q$ as follows:
\begin{itemize}
\item Let $e_{q+1}$ be the $(q+1)$-st standard basis element. The set of $0$-simplices $\mathcal{H}^{\delta}_{0}(M,\phi^q)$, consists of pairs $(t,[\phi])$ where $[\phi]$ is a germ of a $q$-handle $D^q\times D^{n-q}\hookrightarrow M$ so that for a small $\epsilon$ we have 
%\item Let $e_{q+1}$ be the $(q+1)$-th standard basis element. The space of $0$-simplices $\mathcal{H}_0(M,\phi^q)$, consist of pairs $(t,\phi)$ where $\phi$ is an embedded  $q$-handle $D^q\times D^{n-q}\hookrightarrow M$ so that
\begin{equation}\label{eq15}
\phi|_{S^{q-1}\times D^{n-q}_{\epsilon}}=\phi^q|_{S^{q-1}\times(D^{n-q}_{\epsilon} +  t. e_{q+1})},
\end{equation}
and $\phi(D^q\times \{0\})$ is isotopic to $\phi^q(D^q\times \{t.e_{q+1}\})$ relative to the boundary. %The space $\mathcal{H}^g_0(M,\phi^q)$ is topologized as the subspace of real numbers times the space of germs of embeddings of $q$-handles into $M$.
\item The set of  $p$-simplices   $\mathcal{H}^{\delta}_{p}(M,\phi^q)$, consists of $(p+1)$-tuples $$((t_0,[\phi_0]), (t_1,[\phi_1]),\dots, (t_p, [\phi_p])),$$ in $\mathcal{H}^{\delta}_{0}(M,\phi^q)^{p+1}$ so that $t_0<t_1<\cdots<t_p$ and  the embedded cores  $\phi_i(D^q\times \{0\})$ are disjoint.
% It is topologized as the subspace of $\mathcal{H}^g_0(M,\phi^q)^{p+1}$.
%\item Let $\mathcal{H}_{\bullet}(M,\phi^q)$ be the semisimplicial space whose $0$-simplices consist of pairs $(t,\phi)$ where $(t,[\phi])\in \mathcal{H}^g_0(M,\phi^q)$. We topologize $\mathcal{H}_0(M,\phi^q) $ as the subspace of real numbers times the space of embeddings of $q$-handles into $M$. The space of $p$-simplices $\mathcal{H}_p(M,\phi^q)$ consist of $(p+1)$-tuples $$((t_0,\phi_0), (t_1,\phi_1),\dots, (t_p, \phi_p)),$$ in $\mathcal{H}_0(M,\phi^q)^{p+1}$ so that $t_0<t_1<\cdots<t_p$ and  the embedded handles  $\phi_i(D^q\times D^{n-q})$ are disjoint. It is topologized as the subspace of $\mathcal{H}_0(M,\phi^q)^{p+1}$.
\item Let $\overline{\mathcal{H}}^{\delta}_{\bullet}(M,\phi^q)$ be the semisimplicial set whose  $0$-simplices consist of pairs $(t,\phi)$ where $(t,[\phi])\in \mathcal{H}^{\delta}_{0}(M,\phi^q)$. Note that the difference here is we consider actual embeddings not just their germs around the core. And let $p$-simplices be the subset of $\mathcal{H}^{\delta}_{0}(M,\phi^q)^{p+1}$ consisting of those $(p+1)$-tuples
$$((t_0,\phi_0), (t_1,\phi_1),\dots, (t_p, \phi_p)),$$
where the cores of $\phi_i$'s are pairwise disjoint.
\end{itemize}
 \begin{rem}
Note that by definition, for every pair $(t,[\phi])\in \mathcal{H}_{0}^{\delta}(M,\phi^q)$, the real number $t$ is uniquely determined by $\phi$. We denote this $t$-coordinate by $t_{\phi}$.
\end{rem}
 The group $\tdH_0(M,\partial M)$ acts on $\mathcal{H}^{\delta}_{\bullet}(M,\phi^q)$ and $\overline{\mathcal{H}}^{\delta}_{\bullet}(M,\phi^q)$. The $q$-handle resolution associated to $\phi^q$ in this case is
\[
X^{\delta}_{\bullet}(M,\phi^q):= \mathcal{H}^{\delta}_{\bullet}(M,\phi^q)\hcoker \tdH_0(M,\partial M)\xrightarrow{g_{\phi^q}} \BdH_0(M,\partial M).
\]

\item {\bf Topological versions:} Let $\mathcal{H}_{\bullet}(M,\phi^q)$ be the semisimplicial space whose $0$-simplices as a set consists of pairs $(t,\phi)$ where $(t,[\phi])\in\mathcal{H}^{\delta}_{0}(M,\phi^q)$. We topologize $\mathcal{H}_0(M,\phi^q) $ as the subspace of real numbers times the space of embeddings of a $q$-handle into $M$ equipped with the compact-open topology. The space of $p$-simplices $\mathcal{H}_p(M,\phi^q)$ is a subspace of  $\mathcal{H}_0(M,\phi^q)^{p+1}$ consisting of $(p+1)$-tuples $$((t_0,\phi_0), (t_1,\phi_1),\dots, (t_p, \phi_p)),$$ so that $t_0<t_1<\cdots<t_p$ and  the embedded handles  $\phi_i(D^q\times D^{n-q})$ are disjoint. It is topologized with the subspace topology.

Note that $\tH_0(M,\partial M)$ acts on $\mathcal{H}_{\bullet}(M,\phi^q)$ and we define the $q$-handle resolution associated to $\phi^q$ in this case to be the augmented semisimplicial space:
\[
X_{\bullet}(M,\phi^q):= \mathcal{H}_{\bullet}(M,\phi^q)\hcoker \tH_0(M,\partial M)\xrightarrow{f_{\phi^q}} \BH_0(M,\partial M).
\]


%and $\overline{\mathcal{H}}^{\delta}_{\bullet}(M,\phi^q)$ are defined as underlying semisimplicial sets (i.e. with the discrete topology) of $\mathcal{H}^g_{\bullet}(M,\phi^q)$ and $\overline{\mathcal{H}}_{\bullet}(M,\phi^q)$ respectively. The $q$-handle resolution associated to $\phi^q$ in this case is
%\[
%X^{\delta}_{\bullet}(M,\phi^q):= \mathcal{H}^{\delta}_{\bullet}(M,\phi^q)\hcoker \tdH_0(M,\partial M)\xrightarrow{g_{\phi^q}} \BdH_0(M,\partial M).
%\]
\end{itemize}
\end{defn}

We want to prove that $f_{\phi^q}$ and $g_{\phi^q}$ induce weak homotopy equivalences. Similar to semi-simplicial resolutions \ref{ee} and \Cref{claim1}, it is enough to show that $|\mathcal{H}_{\bullet}^{\delta}(M,\phi^1)|$ is contractible. 
\begin{thm}\label{eq:7}
Let $M$ be a surface with boundary and let $\phi^1$ be a $1$-handle, then the fat realizations $|\mathcal{H}_{\bullet}^{\delta}(M,\phi^1)|$ and $|\overline{\mathcal{H}}^{\delta}_{\bullet}(M,\phi^1)|$ are weakly contractible.
\end{thm}
\begin{rem}\label{rem1}
In fact for a manifold $M$ whose dimension is larger than $4$, one can show that $|\mathcal{H}_{\bullet}^{\delta}(M,\phi^q)|$ is contractible if the handle index $q\leq \text{dim}(M)/2$. If one shows that $|\mathcal{H}_{\bullet}^{\delta}(M,\phi^q)|$ is contractible for handle indices larger than the middle dimension, one could deduce \Cref{Main} in all dimensions without resorting to foliation theory.
\end{rem}
\begin{proof}
We give a proof that $|\mathcal{H}_{\bullet}^{\delta}(M,\phi^1)|$ is  contractible and the proof for contractibility of $|\overline{\mathcal{H}}^{\delta}_{\bullet}(M,\phi^1)|$ is the same. To show that a continuous map $f:S^k\to |\mathcal{H}_{\bullet}^{\delta}(M,\phi^1)|$ is nullhomotopic, we fix a triangulation $K$ of $S^k$ and without loss of generality, we assume that $f$ is a PL-map from $K$ to  $|\mathcal{H}_{\bullet}^{\delta}(M,\phi^1)|$.
%Then we shall prove that $f$ can be homotoped to a PL-map so that the image of each vertex of $K$ is given by a germ of a {\it smoothly} embedded $q$-handle.
%
%Let $v\in K$ be a vertex so that $f(v)=(t,[\phi])$ is given by a germ of a $q$-handle $\phi$ so that $\phi$ is not smooth. We want to find a vertex $[\phi']\in \mathcal{H}_{0}^{\delta}(M,\phi^q)$ connected to $[\phi]$ so that it is smooth and its core is in a small enough neighborhood of the germ of the embedded core $\phi(D^q\times \{0\})$. Let $\epsilon$ be a small positive number so that  $\phi(D^q\times D^{n-q}_{\epsilon})$ is disjoint from all the cores of the vertices in $f(\text{Link}(v))$. Since the core $\phi(D^q\times \{0\})$ is flatly embedded, its complement in a tubular neighborhood, $\phi(D^q\times \text{int}(D^{n-q}_{\epsilon})\backslash D^q\times \{0\})$, is diffeomorphic to $(D^q\backslash \{0\})\times \text{int}(D^{n-q}_{\epsilon})$. Therefore, we can find a germ of a smoothly embedded $q$-handle $[\phi']\in \mathcal{H}_{0}^{\delta}(M,\phi^q)$ so that $$\phi'(D^q\times D^{n-q})\subset \phi(D^q\times \text{int}(D^{n-q}_{\epsilon})\backslash D^q\times \{0\}).$$  We can choose  $\epsilon$  small enough so that the vertex $[\phi']$ is connected to all vertices in $f(\text{Star}(v))$. Therefore, by the linear homotopy, we can homotope $f$ to a map $f'$ that sends $v$ to $[\phi']$ and is the same as $f$ on all the other vertices. By iterating this process, we can homotope $f$ to a map $g$ that on all vertices of $K$ gives a germ of a smoothly embedded $q$-handle and also the $t$-coordinates of the vertices of $g(K)$ are all different.

 Let $f:K\to  |\mathcal{H}_{\bullet}^{\delta}(M,\phi^1)|$ be a PL-map from a triangulation $K$ of the sphere $S^k$. To show that $f$ is nullhomotopic, we show that there exists $[\phi]\in  \mathcal{H}_{\bullet}^{\delta}(M,\phi^1)$ so that one can homotope the image $f(K)$ into $\text{Star}([\phi])$. Note that for every $v\in K$, the core of the germ of the embedded $1$-handle $f(v)\in  \mathcal{H}_{\bullet}^{\delta}(M,\phi^1)$ has a normal (micro)bundle since every core comes equipped with the germ of its cocore. 
 %Given that $K$ has finitely many vertices, let us enumerate the image of these vertices by $[\psi_1], [\psi_2],\dots, [\psi_m]$ where $[\psi_i]$'s are in $ \mathcal{H}_{0}^{\delta}(M,\phi^1)$. By Quinn's transversality theorem \cite{MR929089}, we can choose $[\phi]\in \mathcal{H}_{0}^{\delta}(M,\phi^q)$ so that $\phi(D^1\times\{0\})$ is transverse to the core $\psi_1(D^1\times \{0\})$. But transversality with this codimension implies that the germ of the core $\phi(D^1\times \{0\})$ is disjoint from the germ of the core of $[\psi_1]$. Since disjointness (transversality in this codimension) is preserved under small isotopy, we can make $\phi(D^1\times \{0\})$ disjoint from the core of $[\psi_2]$ while it remains disjoint from the core $[\psi_1]$. If we continue this procedure, we can assume that the core of $[\phi]$ is disjoint from the cores of $[\psi_i]$ for all $i$. Hence $f(K)\subset \text{Star}([\phi])$.
 
 First we  show that we can homotope $f$ so that the cores of $f(v)$ for all $v\in K$ are pairwise transverse  to each other. This part of the argument works for higher dimensional manifolds. But in dimension $2$, the transversality argument is easier and in higher dimensions, one has to use the transversality in the sense of \cite[Essay 3, section 1]{MR0645390}. 
 %Then similar to the previous case, we show that $f$ can homotoped to a map $g$ such that $g(K)\subset \text{Star}([\phi])$ for some $[\phi]\in \mathcal{H}_{0}^{\delta}(M,\phi^2)$.

To do the first step, we need to consider the parallel copies of the handles. To explain what we mean by parallel copies, let $f(v)=\phi_0\in f(K)$ be a vertex and $\{\phi_1,\phi_2,\dots, \phi_n\}$ be all the vertices in $f(K)$ that are connected to $\phi_0$. Recall that by definition of $\mathcal{H}_{0}^{\delta}(M,\phi^1)$, there exists a small positive $\epsilon$ such that 
\[
\phi_0|_{S^{0}\times D^{1}_{\epsilon}}=\phi^1|_{S^{0}\times(D^{1}_{\epsilon} +  t_{\phi_0}. e_{2}))}.
\]
A nearby parallel copy $\phi_0':D^1\times D^{1}\hookrightarrow M$  can be described by $\phi_0$ restricted to $D^1\times (D^{1}_{\epsilon/3}+\epsilon/2 \cdot e_{2})$.  Note that $[\phi_0']$ is a vertex in $\mathcal{H}_{0}^{\delta}(M,\phi^1)$ and since the cores of $\phi_0'$ and $\phi_0$ are disjoint, the vertices  $[\phi_0']$ and $[\phi_0]$ are connected in $\mathcal{H}_{0}^{\delta}(M,\phi^1)$.

Let us enumerate   the vertices of $f(K)$ by $[\psi_1], [\psi_2],\dots, [\psi_m]$. First we choose a parallel copy of $[\psi_2]$ and perturb it by a small isotopy to obtain $[\psi'_2]$ so that its core  becomes transverse to the core of $[\psi_1]$. If the isotopy is small enough the core of $[\psi'_2]$ is disjoint from the core of $\psi_2$ and the core of all vertices in $f(K)$ that $[\psi_2]$ was disjoint from. Therefore, there is a homotopy replacing $[\psi_2]$ with $[\psi'_2]$ and fixing the image of other vertices. Thus we may assume that $[\psi_1]$ and $[\psi_2]$ have transverse cores.  Hence the intersection of their cores is a set of points. Now we move on to $[\psi_3]$. Similarly by choosing a nearby copy of $[\psi_3]$ and a small perturbation $[\psi'_3]$ of this nearby copy, we obtain a handle whose core is disjoint from the points in the intersection of the previous two handles $[\psi_1]$ and $[\psi_2]$.  Hence we can choose a small neighborhood $U$ of the intersection of the cores $[\psi_1]$ and $[\psi_2]$ such that the core of $[\psi'_3]$ is also disjoint from $U$. Now by Quinn's transversality, we can find a small isotopy whose support is away from $U$ and we obtain a handle $[\psi''_3]$ whose core is transverse to the manifold $(\psi_1(D^1\times \{0\})\cup \psi_2(D^1\times \{0\}))\backslash U$. If we choose the isotopy small enough the core of $[\psi''_3]$ is disjoint from the core of $[\psi_3]$ and all the cores of the vertices of $f(K)$ that the core of $[\psi_3]$ was disjoint from. Hence by a homotopy of the map $f$, we can replace $[\psi_3]$ with $[\psi''_3]$. Thus we may assume that the core of $[\psi_3]$ is transverse to the core of $[\psi_1]$ and $[\psi_2]$. By continuing this process we can change $f$ up to homotopy to make the core of $[\psi_i]$ transverse to the core of $[\psi_j]$ for $j<i$. Therefore, we may assume that the core of the vertices of $f(K)$ are pairwise transverse to each other.

Similarly we can find a vertex $[\phi]\in \mathcal{H}_{0}^{\delta}(M,\phi^1)$ whose core is transverse to the core of all vertices of $f(K)$. For any $[\phi_0]\in \mathcal{H}_{0}^{\delta}(M,\phi^1)$ whose core is transverse to $\phi(D^1\times \{0\})$, we can order the intersection points. Let $p$ and $q$ be two consecutive points in the intersection of the cores $[\phi]$ and $[\phi_0]$. Let $D$ and $D_0$ be the arcs connecting $p$ and $q$ in the core of $[\phi]$ and $[\phi_0]$ respectively. Since the cores ate isotopic, by \cite[Proposition 1.7]{farb2011primer} there is a Whitney disk $N$ (or bigon in the context of surgery of arcs on surfaces) that bounds $D\cup D_0$. Let $[\phi_0]$  be the vertex among the vertices of $f(K)$ whose core  gives rise to an innermost bigon. Similar to the previous case, we can homotope $f$ by replacing $[\phi_0]$ with a handle obtained by applying the Whitney trick to the disk $N$. By continuing this procedure, we reduce the number of intersections between the vertices of $f(K)$ and the core of $[\phi]$, so we can homotope $f$ so that its image lies in $\text{Star}([\phi])$.
\end{proof}

% The circles in the intersection of $\phi(D^2\times\{ 0\})$ and the cores $f(K)$ are all nullhomotopic in the core of $[\phi]$. Thus they bound a $2$-disk in $\phi(D^2\times\{ 0\})$. Choose a metric on the core $\phi(D^2\times\{ 0\})$  and among the circles in the intersection of the core of $[\phi]$ and the core of vertices of $f(K)$, let $[\phi_0]$ be the one  which bounds a $2$-disk $D$ with the minimal area. The circle $\partial D$ also bounds a $2$-disk $D_0$ in the core of $[\phi_0]$. Note that the embedded sphere $D\cup D_0$ in $M$ is nullhomotopic because the cores of $[\phi]$ and $[\phi_0]$ are isotopically disjoint. By \cite[Proposition 3.10]{hatcher2000notes}, the embedded sphere $D\cup D_0$ bounds a contractible $3$ manifold $N$ and by Perelman's theorem (\cite{perelman2002entropy},\cite{perelman2003ricci}), we know that it is homeomorphic to the a $3$-disk. 

%Let $\{ [\phi_1],[\phi_2],\dots,[\phi_n]\}$ be the vertices in $f(K)$ that are connected to $[\phi_0]$.  Note that since $D$ is minimal, the cores of $[\phi_i]$'s are disjoint from $N$. Therefore by an isotopy we can move the core of $[\phi_0]$ first by pushing $D_0$ toward $D$ along the $3$-disk $N$ and a little further to obtain a handle $[\phi'_0]$ whose core is disjoint from the core of $[\phi_i]$'s for $0\leq i\leq n$. Thus by a homotopy we can replace $[\phi_0]$ by $[\phi'_0]$ and by doing so we reduced the number of circles in the intersection of the core $[\phi]$ and the cores of vertices of $f(K)$. By iterating this procedure, we can homotope $f$ so that its image lies in $\text{Star}([\phi])$.
%\vspace{.2cm}
%\noindent{\bf Case 1:} Suppose $q< \text{dim}(M)/2$. Let $f:K\to  |\mathcal{H}_{\bullet}^{\delta}(M,\phi^q)|$ be a PL-map from a triangulation $K$ of the sphere $S^k$. To show that $f$ is nullhomotopic, we show that there exists $[\phi]\in  \mathcal{H}_{\bullet}^{\delta}(M,\phi^q)$ so that one can homotope the image $f(K)$ into $\text{Star}([\phi])$. Note that for every $v\in K$, the core of the germ of the embedded $q$-handle $f(v)\in  \mathcal{H}_{\bullet}^{\delta}(M,\phi^q)$ has a normal microbundle since every core comes equipped with the germ of its cocore. Since $K$ has finitely many vertices,  one can inductively use Quinn's transversality theorem \cite{MR929089} to conclude that there exists $[\phi]\in  \mathcal{H}_{\bullet}^{\delta}(M,\phi^q)$ where the core of $[\phi]$ is transverse to the the cores of $f(v)$ for all $v\in K$. But transversality with this codimension implies that the germ of the core $\phi(D^q\times \{0\})$ is disjoint from the germ of the cores of the vertices in $f(K)$. Hence $f(K)\subset \text{Star}([\phi])$.

%\vspace{.2cm}
%\noindent{\bf Case 2:} Suppose $2<q=\text{dim}(M)/2$. Let $f:K\to  |\mathcal{H}_{\bullet}^{\delta}(M,\phi^q)|$ be as before.  Similar to case $1$, we can choose $[\phi]$ and arrange $f$ so that the core of the vertices in $f(K)$ are transverse (in the sense of \cite[Essay 3, section 1]{MR0645390}) to the core of the fixed vertex $\phi$ without any transversality assumption between $f(v)$ and $f(w)$ for different vertices $v$ and $w$. Now we want to homotope $f$ to a map $g$ so that $g(K)\subset \text{Star}(\phi^q)$. To do so, we want to do surgery on the vertices in $f(K)$ so that it becomes disjoint from $\phi^q$. 
%
%Let $f(v)=\phi_0\in f(K)$ be a vertex and $\{\phi_1,\phi_2,\dots, \phi_n\}$ be all the vertices in $f(K)$ that are connected to $\phi_0$. First we choose $\phi_0'\in \mathcal{H}_{0}^{\delta}(M,\phi^q)$ that is connected to $\{ \phi_0,\phi_1,\dots,\phi_n\}$ in $\mathcal{H}_{0}^{\delta}(M,\phi^q)$ in a following way. Recall that by definition of $\mathcal{H}_{0}^{\delta}(M,\phi^q)$, there exists a small positive $\epsilon$ such that 
%\[
%\phi_0(S^{q-1}\times D^{n-q}_{\epsilon})=\phi^q(S^{q-1}\times(D^{n-q}_{\epsilon} +  t_{\phi_0}. e_{q+1})).
%\]
%Let $\phi_0':D^q\times D^{n-q}\hookrightarrow M$  be given by $\phi_0$ restricted to $D^q\times (D^{n-q}_{\epsilon/3}+\epsilon/2 \cdot e_{q+1})$.  Note that $[\phi_0']$ is a vertex in $\mathcal{H}_{0}^{\delta}(M,\phi^q)$ and since the cores of $\phi_0'$ and $\phi_0$ are disjoint, the vertices  $[\phi_0']$ and $[\phi_0]$ are connected in $\mathcal{H}_{0}^{\delta}(M,\phi^q)$.

%Even though we have no transversality assumption on the cores of $\phi_i$ for $1\leq i\leq n$, fixing the handle $\phi^q$, we claim that we can find an isotopy $h_t$ such that the handle $h_1(\phi_0)$ is disjoint from $\phi^q$ and $\phi_i$ for $1\leq i\leq n$.   To do so, let $p$ and $q$ be two intersection points of $\phi'_0$ and $\phi$ (note that since they can be moved off of each other as they are isotopic relative to the boundary, the intersection points must come in pairs of opposite signs), also assume that $P_0$ is a path between $p$ and $q$ in $\phi'_0(D^q\times\{0\})$ and $P_1$ is a path between $p$ and $q$ in $\phi(D^q\times\{0\})$ so that no other intersection points of $\phi$ and $\phi_i$'s lie on $P_0$ and $P_1$. Since $\phi'_0(D^q\times\{0\})$ is isotopic to $\phi^q(D^q\times \{t_{\phi_0'}.e_{q+1}\})$, and the Whitney circle $P_0\cup P_1$ is contractible, we obtain an embedded Whitney disk bounding $P_0\cup P_1$. Whitney trick provides  an isotopy $h_t$ that is supported in a neighborhood of the Whitney disk so that $h_1(\phi'_0(D^q\times\{0\}))$ intersects $\phi(D^q\times\{0\})$ in $\phi'_0(D^q\times\{0\})\cap \phi(D^q\times\{0\})\backslash \{p,q\}$. 
%
%If we choose the Whitney disk so that it is disjoint from $\phi_i(D^q\times\{0\})$ for $1\leq i\leq n$, the isotopy $h_t$ gives rise to a homotopy from $f:K\to |\mathcal{H}_{\bullet}^{\delta}(M,\phi^q)|$ to a map $f'$ so that $f'(v)=h_1(\phi'_0(D^q\times\{0\}))$ and is equal to $f$ on the other vertices. But since the codimension of each core $\phi_i(D^q\times\{0\})$ is at least $3$, by the transversality theorem, we can perturb the chosen Whitney disk so that it becomes disjoint from $\phi_i(D^q\times\{0\})$ for all $1\leq i\leq n$. By iterating this process, we can homotope $f$ to a map $g$ that does intersect $\phi(D^q\times\{0\})$, hence it is nullhomotopic. 
%
%\vspace{.2cm}
%
%\noindent{\bf Case 3:} Suppose $1=q=\text{dim}(M)/2$.  Similar to case $2$ i.e. we want to do surgery on the vertices of $f(K)$ in order to homotope $f$ to a map $g$ so that $g(K)\subset \text{Star}(\phi)$. As the previous case, we may assume that the core of the vertices of $f(K)$ are transverse and we choose a $1$-handle $[\phi]\in \mathcal{H}_{0}^{\delta}(M,\phi^1)$ so that its core is transverse to the core of vertices of $f(K)$.

%\begin{lem}[Normal form] There exits a diffeomorphism $h$ of the manifold $M$ so that  $h(\phi_i(D^q\times\{ 0\}))=\tilde{\phi^q}(D^q\times \{t.e_{q+1}\})$  for $0\leq i\leq n$.
%\end{lem}
%{\it Proof of the claim:} It is clear for $n=0$, because by definition $\phi_0(D^q\times\{ 0\})$ is isotopic to the embedding $\tilde{\phi^q}(D^q\times \{t.e_{q+1}\})$ hence is disjoint from $\tilde{\phi^q}(D^q\times \{0\})$. Therefore by isotopy extension theorem, we can find an isotopy of the manifold $M$ so that it sends $\phi_0(D^q\times\{ 0\})$ to $\tilde{\phi^q}(D^q\times \{t.e_{q+1}\})$.
%
%Now suppose for $i\leq k$, we constructed an isotopy $H^k_t$ so that  $H^k_1(\phi_i(D^q\times\{ 0\}))\cap \tilde{\phi^q}(D^q\times \{0\})$ is empty for $0\leq i\leq k$.
%\begin{rem}
%It is not clear how to make this argument work in dimension $4$, because one needs to choose the Whitney disk bounding $P_0\cup P_1$ that is disjoint from the cores of the handles $\phi_i$ for $1\leq i\leq n$. This might be possible in our special set up because the algebraic intersections are zero.
%\end{rem}
\begin{rem}
For a three manifold $M$, it is easy to show that the complex of  $\mathcal{H}_{\bullet}^{\delta}(M,\phi^1)$ has contractible realization. To prove a similar result for $2$-handles, the first part of the argument that makes the core of the handles pairwise transverse still works. But the second half which is a surgery argument to reduce the intersections does not work. The issue is the Whitney disk may not exist in this case. If we have two transverse isotopic cores of $2$-handles, they intersect in circles. Each circle bounds a $2$-disk in the core of the handles. Hence, each circle in the intersection gives an embedded $2$-sphere in $M$ which may not bound a ball. So we may not be able  to reduce the number of circles by doing surgery following the above argument.
\end{rem} 
\begin{rem}\label{codim}
Note that in general if $q<\text{dim}(M)/2$, the realization of $\mathcal{H}_{\bullet}^{\delta}(M,\phi^q)$ is contractible by the same argument as \Cref{claim}. Because transversality in this codimension implies disjointness. 
\end{rem}
\begin{lem}
Let $M$ be a surface with boundary and let $\phi^1$ be a $1$-handle, the fat realization $|\mathcal{H}_{\bullet}(M,\phi^1)|$ is weakly contractible.
\end{lem}
\begin{proof}
Similar to \Cref{claim1}.
\end{proof}

Now since the fibers of  the maps $|f_{\phi^1}|: |X_{\bullet}(M,\phi^1)|\xrightarrow{} \BH_0(M,\partial M)$ and $|g_{\phi^1}|:|X^{\delta}_{\bullet}(M,\phi^1)|\xrightarrow{} \BdH_0(M,\partial M)$ are weakly contractible, by  \cite[Lemma 2.1]{randal2009resolutions} they induce weak equivalences. 

\subsubsection{The homotopy type of $X_{\bullet}(M,\phi^q)$ and $X^{\delta}_{\bullet}(M,\phi^q)$}Given that the boundary condition \ref{eq15} is fixed by the action of $\tH_0(M,\partial M)$, unlike the case of the disk resolutions \ref{disk}, the action of $\tH_0(M,\partial M)$ on $\mathcal{H}_{0}(M,\phi^q)$ and the action of $\tdH_0(M,\partial M)$ on $\mathcal{H}^g_{0}(M,\phi^q)$ are not transitive. But there is a bijection between the set of the orbits of these actions.

\begin{itemize}[leftmargin=*]
\item \textit{Action of $\tH_0(M,\partial M)$ on $\mathcal{H}_{p}(M,\phi^q)$:}\label{action1} To determine the weak homotopy type of $X_{p}(M,\phi^q)$, we shall first describe the orbits of the action $\tH_0(M,\partial M)$ on $\mathcal{H}_{p}(M,\phi^q)$. To do so, let us first introduce few notations. Let $\sigma=(\phi_0, \phi_1, \dots, \phi_p)$ be a $p$-simplex in $\mathcal{H}_{p}(M,\phi^q)$ and let $M\backslash \sigma$ be the manifold  obtained from $M$ by removing the handles $ \phi_i(D^q\times \text{int}(D^{n-q}))$. Let also 
\[
\tH_{0}(M\backslash \sigma, \partial(M\backslash \sigma)),
\]
denote the identity component of the compactly supported homeomorphisms of $\text{int}(M)\backslash \cup_i\phi_i(D^q\times \text{int}(D^{n-q}))$. Note that the submanifold $M\backslash \sigma\hookrightarrow M$ might have different connected components. Similar to \Cref{quasi-fib}, the sequence
\[
\text{Stab}(\sigma)\to \tH_0(M,\partial M)\to \text{orbit}(\sigma),
\]
is a quasi-fibration where the topological group $\text{Stab}(\sigma)$ is naturally identified with  the group $\underline{\tH}_{0}(M\backslash \sigma, \partial(M\backslash \sigma))$ whose identity component has the same homotopy type as  $\tH_{0}(M\backslash \sigma, \partial(M\backslash \sigma))$ (similar to the inclusion \ref{e_p}). Therefore, choosing once and for all, a $p$ simplex in each orbit, we obtain the map
\[
 \coprod_{\sigma} (\BwH_{0}(M\backslash \sigma, \partial(M\backslash \sigma)))\xrightarrow{\simeq} X_{p}(M,\phi^q),
\]
which is a weak homotopy equivalence (see \Cref{prop1}).
%\begin{rem}
%Note that the handle body decomposition \ref{decomp} induces a handle body decomposition on $M\backslash \sigma$ with fewer handles of index $q$. 
%\end{rem}
%Let $\sigma=(\phi_0, \phi_1, \dots, \phi_p)$ be a $p$-simplex in $\mathcal{H}_{p}(M,\phi^q)$. Note that   when we remove the handles $\phi_i$ from $M$, we obtain disjoint union of $p$ embedded disks denoted by $D_{\sigma,1},D_{\sigma,2},\dots, D_{\sigma,p}$ with a manifold $M_{\sigma}$ which is homeomorphic to $M\backslash \phi_0$. 
\item \textit{Action of $\tdH_0(M,\partial M)$ on $\mathcal{H}^{\delta}_{p}(M,\phi^q)$:} \label{action2}To determine the weak homotopy type of $X^{\delta}_{p}(M,\phi^q)$, we shall first describe the orbits of the action $\tdH_0(M,\partial M)$ on $\mathcal{H}^{\delta}_{p}(M,\phi^q)$. Let $[\sigma]=([\phi_0], [\phi_1], \dots, [\phi_p])$ be a $p$-simplex in $\mathcal{H}^{\delta}_{p}(M,\phi^q)$ and let us denote $\text{int}(M)\backslash \cup_i\phi_i(D^q\times\{0\})$ by $M( [\sigma])$. Let $\underline{\tdH}_{0,c}(M([\sigma])$ denote the stabilizer of $\sigma$ as an element of $\mathcal{H}^{\delta}_{p}(M,\phi^q)$ acted on by $\tdH_0(M,\partial M)$. Similar to the short exact sequence \ref{connectedcomponent}, we have 
\[
1\to \tdH_{0,c}(M([\sigma])\to \underline{\tdH}_{0,c}(M([\sigma])\to \pi_0(\underline{\tH}_{0}(M\backslash \sigma, \partial(M\backslash \sigma)))\to 1.
\]

%Note that $\tH_0(M,\partial M)$ acts on $\mathcal{H}^{g}_{p}(M,\phi^q)$ and if we consider $\sigma$ as an element of $\mathcal{H}^{g}_{p}(M,\phi^q)$, the stabilizer of $\sigma$ is the group $\underline{\tH}_{0,c}(M([\sigma])$ whose identity component is $\tH_{0,c}(M([\sigma])$. Therefore the stabilizer of $\sigma$ as an element of $\mathcal{H}^{\delta}_{p}(M,\phi^q)$ acted on by $\tdH_0(M,\partial M)$  is the group $\underline{\tdH}_{0,c}(M([\sigma])$. 

Given that there is a bijection between the set of the orbits of these two actions, we shall consider the germs of the representatives of the first action, as a representative set of the orbits of the action of $\tdH_0(M,\partial M)$ on $\mathcal{H}^{\delta}_{0}(M,\phi^q)$. Hence, by Shapiro's lemma, we obtain a map
\[
 \coprod_{[\sigma]} (\BdwH_{0,c}(M([\sigma]))\xrightarrow{\simeq} X^{\delta}_{p}(M,\phi^q),
\]
which is a weak equivalence. 
\end{itemize}
\begin{rem}Recall that the inclusion $M\backslash \sigma\hookrightarrow M([\sigma])$ induces a natural map
\[
\BwH_0(M\backslash \sigma, \partial(M\backslash \sigma))\xrightarrow{\simeq} \BwH_{0,c}(M([\sigma])).
\]
which is a weak equivalence (similar to the map \ref{e_p}). The same map between discrete homeomorphisms
\[
\BdwH_0(M\backslash \sigma, \partial(M\backslash \sigma))\xrightarrow{H_*\text{ iso}} \BdwH_{0,c}(M([\sigma])),
\]
induces a homology isomorphism by the same argument as \cite[Corollary 2.3]{nariman2014homologicalstability}.
\end{rem}
\begin{proof}[Proof of \Cref{Main} for $\text{dim}(M)\leq 2$] \label{proof} In  \Cref{lem2}, we reduced the theorem to manifolds with boundary.  Let us fix a handle decomposition of $M$ 
\[
M= \partial_0 M\times [0,1]+(\phi_1^{q_1})+(\phi_2^{q_2})+\cdots+(\phi_r^{q_r}).
\]
We want to reduce the statement of the theorem for $M$ to the case of a disk (Mather's theorem \cite{MR0288777}) by cutting handles from $M$. First suppose $\text{dim}(M)=1$. Since we already reduced the theorem to the case of manifolds with boundary and in this dimension, a manifold with boundary is homeomorphic to the union of disks for which the theorem holds by Mather's theorem (\cite{MR0288777}). Now we assume $M$ is a surface with boundary.

\begin{claim}\label{reduction}
If Thurston's theorem holds for $M\backslash \sigma$ for all $\sigma\in\mathcal{H}_{p}(M,\phi^1)$ and for all $p$, it also holds for $M$.
\end{claim}
\noindent\textit{Proof of the claim:}
Similar to \Cref{lem2}, we have a zig-zag of maps from $X^{\delta}_{p}(M,\phi^1)$ to $X_{p}(M,\phi^1)$ which corresponds to the zig-zag
\[
\BdwH_{0,c}(M([\sigma]))\xleftarrow{H_*-\text{iso}} \BdwH_{0}(M\backslash \sigma, \partial(M\backslash \sigma))\to \BwH_0(M\backslash \sigma, \partial(M\backslash \sigma)).
\]
Given the hypothesis of the lemma, the above zig-zag induces a homology isomorphism between $\BdwH_{0,c}(M([\sigma]))$ and $\BwH_0(M\backslash \sigma, \partial(M\backslash \sigma))$. Hence the induced map between $H_*(X^{\delta}_{p}(M,\phi^1))$ and $H_*(X_{p}(M,\phi^1))$ is an isomorphism. Therefore, by the comparison of the spectral sequences 
\[
\begin{tikzcd}
H_q(X_{p}^{\delta}(M,\phi^1))\arrow["\cong"]{r}\arrow[Rightarrow]{d}&H_q(X_{p}(M,\phi^1))\arrow[Rightarrow]{d}\\H_{p+q}(|X_{\bullet}^{\delta}(M,\phi^1)|)\arrow{r}\arrow["\cong"]{d}& H_{p+q}(|X_{\bullet}(M,\phi^1)|)\arrow["\cong"]{d}\\ H_{p+q}(\BdH_0(M,\partial M))\arrow["\iota_*"]{r}& H_{p+q}(\BH_0(M,\partial M)),
\end{tikzcd}
\]
we conclude that $\iota_*$ is an isomorphism.
$\blacksquare$

For a $1$-handle $[\phi^1]$ and a $p$-simplex $\sigma\in \mathcal{H}_{p}(M,\phi^1)$, the manifold $M\backslash \sigma$ is homeomorphic to the union of $p$ disks and $M\backslash \phi^1(D^1\times \text{int}(D^1))$. Thus using the claim, the theorem holds for $M$ if it holds for $M\backslash \phi^1(D^1\times \text{int}(D^1))$. Therefore, by removing $1$-handles inductively from the surface $M$, we can reduce the theorem for $M$ to the case of a disk which is given by Mather's theorem (\cite{MR0288777}).
\end{proof}

%{\bf Case 3:} Suppose $\text{dim}(M)=3$. Note that for a $2$-handle $[\phi^2]$ and a $p$-simplex $\sigma\in \mathcal{H}_{p}(M,\phi^2)$, the manifold $M\backslash \sigma$ is homeomorphic to disjoint union of $p$ disks and $M\backslash \phi^1(D^2\times \text{int}(D^1))$. If $[\phi^1]$ is a $1$-handle and $\sigma$ is a $p$-simplex in $\mathcal{H}_{p}(M,\phi^1)$, the handle decomposition  of the manifold $M\backslash \sigma$ has fewer $1$-handles. Thus by the claim, we can reduce the theorem for $M$ to the trivial bordism $S\times D^1$ where $S$ is a surface. Note that $S\times D^1$ is a handlebody (a $3$-ball with $1$-handles attached). Therefore there exits finitely many $2$-handles in $S\times D^1$ such that if we cut $S\times D^1$ along those $2$ handles we obtain a $3$-ball. Hence again by the claim the case of $S\times D^1$ can be reduced to the case a disk.
\begin{rem}
In fact using \Cref{rem1} and a similar argument as above, one can show that Thurston's theorem for a manifold $M$ whose dimension is larger than $4$ is equivalent to Thurston's theorem for a trivial bordism $N\times D^1$ where $N$ is a manifold whose dimension is $\text{dim}(M)-1$.
\end{rem}
%We reduced the theorem to manifolds with boundary.  Now recall, we fixed a handle decomposition of $M$ in \ref{decomp}
%\[
%M= \partial_0 M\times [0,1]+(\phi_1^{q_1})+(\phi_2^{q_2})+\cdots+(\phi_r^{q_r}).
%\]
%By \Cref{reduction}, we know that Thurston's theorem holds for the bordism $M$ if it does for the bordisms with handles of index less than or equal to $\text{dim}(M)/2$.  Note if the above handle decomposition has handles of index only larger than $\text{dim}(M)/2$, we can turn the cobordism upside down to obtain a handle decomposition 
%\[
%M= \partial_1 M\times [0,1]+(\psi_1^{n-q_1})+(\psi_2^{n-q_2})+\cdots+(\psi_r^{n-q_r}).
%\]
%where $(\psi_i^{n-q_i})$ is the handle dual to $(\phi_i^{q_i})$. Hence, using \Cref{reduction}, we can reduce Thurston's theorem to the trivial bordism $N\times [0,1]$ where $N$ is a closed $(\text{dim}(M)-1)$-dimensional manifold. Using the handle resolution for $N$, we can reduce Thurston's theorem for trivial bordisms $N\times [0,1]$ to Thurston's theorem for manifolds with corners of the form $W\times [0,1]^2$ where $W$ is a closed manifold of dimension $\text{dim}(M)-2$. Hence, inductively Thurston's theorem for $M$ is reduced to the case of $[0,1]^{\text{dim}(M)}$ which is Mather's theorem (\cite{MR0288777}).


%Since the map $\mathcal{H}^{\delta}_{\bullet}(M,\phi^q)\to \mathcal{H}_{\bullet}(M,\phi^q)$ is equivariant with respect to the homomorphism $\tdH_0(M,\partial M)\to \tH(M,\partial M)$, we have a semisimplicial map
%\[
%\iota_{\phi^q}: X^{\delta}_{\bullet}(M,\phi^q)\to X_{\bullet}(M,\phi^q),
%\]
%which induces a map between the spectral sequences given by the skeletal filtration of the realizations
%%\[
%% \begin{tikzpicture}[node distance=4.7cm, auto]
%%  \node (A) {$H_q(X^{\delta}_{p}(M,\phi^q))$};
%%  \node (B) [below of=A, node distance=1.6cm] {$H_{p+q}(|X^{\delta}_{\bullet}(M,\phi^q)|)$};
%%  \node (C) [below of= B, node distance=1.6cm ] {$H_{p+q}(\BdH_0(M,\partial M))$};  
%%  \node (D) [right of= A] {$H_q(X_{p}(M,\phi^q))$};
%%    \node (E) [right of= B] {$ H_{p+q}(|X_{\bullet}(M,\phi^q)|)$};
%%  \node (F) [right of= C ] {$H_{p+q}(\BH_0(M,\partial M)).$};  
%%   \draw [->] (A) to node {${\iota_{\phi^q}}_*$}(D);
%%  \draw [->] (B) to node {$$}(E);
%%  \draw [->] (C) to node {$\iota_*$}(F);
%%  \draw [double, shorten <=1pt,>=angle 90,thick, ->] (A) to node {$$}(B);
%%  \draw [->] (B) to node {$\cong$} (C);
%% \draw [shorten <=1pt,>=angle 90,thick,double,->, ] (D) to node {$$} (E);
%%  \draw [->] (E) to node {$\cong$} (F);
%%\end{tikzpicture}
%%\]
%\[
%\begin{tikzcd}
%H_q(X_{p}^{\delta}(M,\phi^q))\arrow{r}\arrow[Rightarrow]{d}&H_q(X_{p}(M,\phi^q))\arrow[Rightarrow]{d}\\H_{p+q}(|X_{\bullet}^{\delta}(M,\phi^q)|)\arrow{r}\arrow["\cong"]{d}& H_{p+q}(|X_{\bullet}(M,\phi^q)|)\arrow["\cong"]{d}\\ H_{p+q}(\BdH_0(M,\partial M))\arrow["\iota_*"]{r}& H_{p+q}(\BH_0(M,\partial M)).
%\end{tikzcd}
%\]
%Hence, the map $\iota$ induces a homology isomorphism if and only if $\iota_{\phi^q}$ does. We want to show that $\iota_{\phi^q}$ induces a homology isomorphism if \Cref{Main} holds for the manifold $M\backslash \phi^q$ obtained from $M$ by removing the handle $(\phi^q)$. 
%
%Note that $\tH_0(M,\partial M)$ does not act transitively on  $\mathcal{H}_{p}(M,\phi^q)$ because $\tH_0(M,\partial M)$ fixes the boundary. To describe the orbits, let $\sigma=([\phi_0], [\phi_1], \dots, [\phi_p])$ be a $p$-simplex $\mathcal{H}_{p}(M,\phi^q)$. Note that   when we cut out the cores of the handles $[\phi_i]$ from $M$, we obtain disjoint union of $p$ disks with a manifold $M_{\sigma}$ which homeomorphic to $M\backslash \phi_0$. The isotopy extension theorem for homeomorphisms implies that 
%\[
%\text{Stab}(\sigma)\to \tH_0(M,\partial M)\to \text{orbit}(\sigma),
%\]
%is a fibration, therefore the continuous version of Shapiro's lemma imply
%\[
%X_{\bullet}(M,\phi^q)\simeq \coprod _{\sigma} \mathrm{B}\text{Stab}(\sigma)\simeq \coprod_{\sigma} \BH_0(M_{\sigma},\partial M_{\sigma})\times (\BH_0(D^n,\partial D^n))^{p}.
%\]
%Similarly we have 
%\[
%X_{\bullet}^{\delta}(M,\phi^q)\simeq\coprod_{\sigma} \BdH_0(M_{\sigma},\partial M_{\sigma})\times (\BdH_0(D^n,\partial D^n))^{p},
%\]
%and the map $X_{\bullet}^{\delta}(M,\phi^q)\to X_{\bullet}(M,\phi^q)$ corresponds to the map induced by the identity homomorphism
%{\small\[
%\BdH_0(M_{\sigma},\partial M_{\sigma})\times (\BdH_0(D^n,\partial D^n))^{p}\to \BH_0(M_{\sigma},\partial M_{\sigma})\times (\BH_0(D^n,\partial D^n))^{p}.
%\]}
%Hence, using the Kunneth formula and Mather's theorem (\cite{MR0288777}), one concludes that $\iota$ induces a homology isomorphism if and only if $\iota_{\phi^q}$ does.
\subsection{The case of three manifolds} To do exactly similar argument as the case of surfaces, we need to find contractible semi-simplicial spaces that cut the manifold into union of disks. Doing an inductive process to cut a three manifold into disks, however, is harder than the case of surfaces. 

For certain types of three manifolds,  namely for Haken $3$-manifolds, this process of cutting into disks is well known. Recall that $M$ is Haken if it is irreducible and contains a properly embedded two sided incompressible surface. Being an irreducible $3$-manifold means that every embedded $2$-sphere bounds a ball. The existence of this ball allows us to do a similar surgery argument as we did for isotopic arcs in a surface. Recall that a compact connected surface $S$, not $S^2$, in $M$ is an incompressible surface, if it is properly embedded $S\cap \partial  M=\partial S$, and the normal bundle of $S$ is trivial and the inclusion $S\hookrightarrow M$ is $\pi_1$ injective. Given the Haken manifold theory, there is a finite sequence of incompressible surfaces that as we cut a Haken manifold $M$ along those surfaces, we obtain disjoint union of balls. 

The idea is to  induct on the number of prime factors in a prime decomposition of $M$ to reduce Thurston's theorem to the case of Haken manifolds and then use the hierarchy of Haken manifolds to reduce it to the case of disks. 

Let $M\cong kP\# N$ be the connected sum of $k$ copies of a prime manifold $P$ and a manifold $N$ where $N$ has no prime factor homeomorphic to $P$. We will define semi-simplicial spaces with contractible realizations that encode different ways of cutting $M$ into the union of $k$ copies of $P\backslash \text{int}(D^3)$  and $N\backslash \cup_{i=1}^{k}\text{int}(D^3)$ and copies of $S^2\times [0,1]$. By the same argument as the previous section, a spectral sequence argument shows that Thurston's theorem holds for $M$ if it does for $P\backslash \text{int}(D^3)$  and $N\backslash \cup_{i=1}^{k}\text{int}(D^3)$ and $S^2\times [0,1]$. We then show that Thurston's theorem for Haken manifolds implies that the theorem holds for $S^2\times [0,1]$ and for prime manifolds with any number of disks removed. Therefore, we deduce that it holds for $P\backslash \text{int}(D^3)$ and by induction for $N\backslash \cup_{i=1}^{k}\text{int}(D^3)$. 
%And finally we show that one can prove the irreducible case using the similar surgery argument as for the case of surfaces. 
\subsubsection{Cutting along  separating spheres} Let $\phi: S\hookrightarrow M$ be an embedding of a surface $S$ in $M$ with a trivial normal bundle. To cut along this embedding, similar to \Cref{def2}, we define different semisimplicial spaces. 
\begin{defn}\label{surfaces}Fix an embedding of the two-sided collar $\psi:S\times [-1,1]\hookrightarrow M$. The germ of embeddings of $S$ into $M$ is defined similar to \Cref{germ} and we define the core of $\psi$ to be its restriction to $S\times \{0\}$. We consider the following semi-simplicial spaces associated to $\psi$.

\begin{itemize}[leftmargin=*]\item {\bf Discrete version:} Let $\mathcal{K}^{\delta}_{\bullet}(M,\psi)$ be a semisimplicial set defined as follows:
\begin{itemize}
\item If $S$ has no boundary,  the set of $0$-simplices $\mathcal{K}^{\delta}_0(M,\psi)$, consists of germs of embeddings $[\phi]$ so that the core of $\phi$ is isotopic to a parallel copy of the core of $\psi$. And the set of  $p$-simplices   $\mathcal{K}^{\delta}_p(M,\psi)$, consists of $(p+1)$-tuples $([\phi_0], [\phi_1],\dots, [\phi_p])$ of germs of embeddings so that their cores are disjoint. 
\item If $S$ has a boundary, the set of $0$-simplices $\mathcal{K}^{\delta}_0(M,\psi)$, consists of pairs $(t,[\phi])$ where $[\phi]$ is a germ of an embedding $S\times [-1,1]\hookrightarrow M$ so that for a small $\epsilon$ we have 
%\item Let $e_{q+1}$ be the $(q+1)$-th standard basis element. The space of $0$-simplices $\mathcal{H}_0(M,\phi^q)$, consist of pairs $(t,\phi)$ where $\phi$ is an embedded  $q$-handle $D^q\times D^{n-q}\hookrightarrow M$ so that
\[
\phi(\partial S\times (-\epsilon,\epsilon))=\psi(\partial S\times(t-\epsilon,t+\epsilon)),
\]
and $\phi(S\times \{0\})$ is isotopic to $\psi(S\times \{t\})$ relative to the boundary. 
\item The set of  $p$-simplices   $\mathcal{K}^{\delta}_p(M,\psi)$, consists of $(p+1)$-tuples $$((t_0,[\phi_0]), (t_1,[\phi_1]),\dots, (t_p, [\phi_p])),$$ in $\mathcal{K}^{\delta}_0(M,\psi)^{p+1}$ so that $t_0<t_1<\cdots<t_p$ and  the embedded surfaces  $\phi_i(S\times \{0\})$ are disjoint. The face maps are given by forgetting the embeddings.

Note that for every $(t,\phi)\in \mathcal{K}^{\delta}_{0}(M,\psi)$, the coordinate $t$ is uniquely determined by $\phi$. We might just write $\phi$ for a vertex and refer to its $t$-coordinate by $t_{\phi}$.  
\item The face maps are given by omitting the coordinates.

%\item Let $\mathcal{K}_{\bullet}(M,\psi)$ be the semisimplicial space whose $0$-simplices consist of pairs $(t,\phi)$ where $(t,[\phi])\in \mathcal{K}^g_0(M,\psi)$. We topologize $\mathcal{K}_0(M,\psi) $ as the subspace of real numbers times the space of embeddings of $S$ into $M$. The space of $p$-simplices $\mathcal{K}_p(M,\psi)$ consists of $(p+1)$-tuples $$((t_0,\phi_0), (t_1,\phi_1),\dots, (t_p, \phi_p)),$$ in $\mathcal{K}_0(M,\psi)^{p+1}$ so that $t_0<t_1<\cdots<t_p$ and  the collared embedings  $\phi_i(S\times [-1,1])$ are disjoint. It is topologized as the subspace of $\mathcal{K}_0(M,\psi)^{p+1}$.
\end{itemize}
\item {\bf Topological versions:} For a surface $S$ with boundary, let $\mathcal{K}_{\bullet}(M,\psi)$  be the semisimplicial space whose $0$-simplices as a set consists of pairs $(t,\phi)$ where $(t,[\phi])\in\mathcal{K}^{\delta}_{0}(M,\psi)$. We topologize $\mathcal{K}_0(M,\psi) $ as the subspace of real numbers times the space of embeddings the collared surface $S\times [-1,1]$ into $M$ equipped with the compact-open topology. The space of $p$-simplices $\mathcal{K}_p(M,\psi)$ is a subspace of  $\mathcal{K}_0(M,\psi)^{p+1}$ consisting of $(p+1)$-tuples $$((t_0,\phi_0), (t_1,\phi_1),\dots, (t_p, \phi_p)),$$ so that $t_0<t_1<\cdots<t_p$ and  the embedded collared surfaces  $\phi_i(S\times [-1,1])$ are disjoint. It is topologized with the subspace topology.  The case of the closed surface $S$ is defined similarly without the $t$-coordinate. The face maps are given by omitting the coordinates.


\end{itemize}
\end{defn}
\begin{defn}
Let  $\psi_i:S\times [-1,1]\hookrightarrow M$ for $1\leq i\leq k$ be a fixed set of disjoint proper embeddings. We define the semi-simplicial set $\mathcal{K}^{\delta}_{\bullet}(M;\psi_1,\psi_2,\dots,\psi_k)$ whose $p$-simplices 
\[\mathcal{K}^{\delta}_{p}(M;\psi_1,\psi_2,\dots,\psi_k)\subset \mathcal{K}^{\delta}_{p}(M,\psi_1)\times \mathcal{K}^{\delta}_{p}(M,\psi_2)\times \dots\times \mathcal{K}^{\delta}_{p}(M,\psi_k)
\]
consist of those $k$-tuples whose cores are pairwise disjoint. We define the topological version $\mathcal{K}_{\bullet}(M;\psi_1,\psi_2,\dots,\psi_k)$ similarly.
\end{defn}
We assume that $M$ is orientable, the non-orientable case is similar. Now for $1\leq i\leq k$, let 
\begin{equation}\label{spheres}\phi_i:S^2\times [-1,1]\hookrightarrow M\end{equation}
 be embeddings whose cores cut $M$ into $k+1$ connected components that are homeomorphic to the disjoint union of $k$ copies of $P\backslash \text{int}(D^3)$ and $N\backslash \cup_{i=1}^{k}\text{int}(D^3)$. Since in the prime decomposition of $N$ there is no factor homeomorphic to $P$, for a $p$-simplex $\sigma_p\in \mathcal{K}_p(M;\phi_1,\phi_2,\dots,\phi_k)$, the manifold $M\backslash \sigma_p$ is homeomorphic to the disjoint union of $k$ copies of $P\backslash \text{int}(D^3)$, $N\backslash  \cup_{i=1}^{k}\text{int}(D^3)$ and $pk$ copies of $S^2\times [0,1]$.

To show that the realization the semi-simplicial set $\mathcal{K}^{\delta}_{\bullet}(M;\phi_1,\phi_2,\dots,\phi_k)$ is contractible, we need to assume that all prime manifolds in the prime decomposition of $M$ are irreducible. So let us first reduce to the case that this assumption holds. To secure this assumption, we need to cut out solid tori from $M$ by defining certain semi-simplicial spaces.
\begin{defn} Let $\phi: S^1\times D^2\hookrightarrow M$ be a $\pi_1$-injective embedding. Let $ \mathcal{T}_0(M;\phi)$  be the space of embeddings of a solid torus whose core is isotopic to the core of $\phi$ and $\mathcal{T}_p(M;\phi)\subset \mathcal{T}_0(M;\phi)^{p+1}$ is a subspace consisting of $p+1$ tuples of disjoint embeddings. We define the discrete version $\mathcal{T}^{\delta}_{\bullet}(M;\phi)$ similar to \Cref{surfaces}, by taking germs of embeddings with the discrete topology.
\end{defn} 
\begin{lem}\label{torus}
The fat realizations $|\mathcal{T}_{\bullet}(M;\phi)|$ and $|\mathcal{T}^{\delta}_{\bullet}(M;\phi)|$ are weakly contractible.
\end{lem}
\begin{proof}
It is enough to show that $|\mathcal{T}^{\delta}_{\bullet}(M;\phi)|$ is contractible (see \Cref{claim1}). Similar to \Cref{eq:7}, to show that a continuous map $f:S^k\to |\mathcal{T}^{\delta}_{\bullet}(M;\phi)|$ is nullhomotopic, we fix a triangulation $K$ of $S^k$ and without loss of generality, we assume that $f$ is a PL-map from $K$ to  $|\mathcal{T}^{\delta}_{\bullet}(M;\phi)|$. Again by the similar argument as \Cref{eq:7}, we can assume that the core of vertices in the image of $f$ are pairwise transverse. But note that in this case the codimension of the core of a solid torus is $2$ so transversality in this codimension implies disjointness. Therefore, by applying transversality we can find a vertex $v$ in $\mathcal{T}^{\delta}_{0}(M;\phi)$ whose core is disjoint from the core of vertices in the image of the map $f$ which implies that $f(K)\subset \text{Star}(v)$. Hence, the map $f$ is null-homotopic.
\end{proof}
\begin{prop}
If \Cref{Main} holds for those three manifolds that are homeomorphic to a connected sum of irreducible manifolds, then it also holds for any three manifold.
\end{prop}
\begin{proof}
It is well-known (see \cite[Proposition 1.4]{hatcher2000notes}) that the only orientable prime $3$-manifold that is not irreducible is $S^1\times S^2$. For $1\leq i\leq n$, let $\theta_i:S^1\times D^2\hookrightarrow S^1\times S^2$ be  $\pi_1$-injective embeddings of solid tori. If the embeddings $\theta_i$'s are disjoint, it is easy to see that $S^1\times S^2\backslash \cup_{i=1}^{n} \theta_i(S^1\times \text{int}(D^2))$ is irreducible. 

Suppose in the prime decomposition of $M$ there are $k$ copies of $S^1\times S^2$. We inductively reduce to the case with fewer copies of $S^1\times S^2$'s. To do so, we want to cut out solid tori from these summands. Let $\phi: S^1\times D^2\hookrightarrow M$ be a $\pi_1$-injective embedding whose image is in one of the copies of $S^1\times S^2$. Note that for all $\sigma\in \mathcal{T}_{\bullet}(M;\phi)$, the manifold $M\backslash \sigma$ obtained from $M$ by removing the interior of the solid tori in $\sigma$, has fewer non-irreducible summand in its prime decomposition. Therefore, the argument in \Cref{reduction} implies if we have Thurston's theorem for those $3$-manifolds with irreducible summands, we have the theorem for all $3$-manifolds.
\end{proof}
Now that we can assume the prime factors in $M$ are all irreducible, we prove the contractibility of the semi-simplicial spaces of separating spheres $\phi_i$ in \ref{spheres}.
\begin{thm}\label{primedecomp}
If $M$ is a connected sum of irreducible $3$-manifolds, the fat realizations $|\mathcal{K}^{\delta}_{\bullet}(M;\phi_1,\phi_2,\dots,\phi_k)|$ and $|\mathcal{K}_{\bullet}(M;\phi_1,\phi_2,\dots,\phi_k)|$ are weakly contractible.
\end{thm}
\begin{proof}
We give the proof for the case where $k=1$ and for the general $k$, the argument is the same. Recall from \Cref{claim} that  the contractibility of $|\mathcal{K}^{\delta}_{\bullet}(M;\phi_1)|$ implies the weak contractibility of $|\mathcal{K}_{\bullet}(M;\phi_1)|$.  So it is enough to prove the former.

 We can represent an element of the $k$-th homotopy group of $|\mathcal{K}^{\delta}_{\bullet}(M;\phi_1)|$ by a PL map $f: K\to |\mathcal{K}^{\delta}_{\bullet}(M;\phi_1)|$ where  $K$ is a triangulation of $S^k$. By the similar argument as \Cref{eq:7}, we can assume that the core of vertices in the image of $f$ are pairwise transverse. Let $v_1\in \mathcal{K}^{\delta}_{0}(M;\phi_1)$ be a vertex whose core is transverse to the core of vertices in $f(K)$. To show that $f$ is null-homotopic, we homotope $f$ to a map $g$ so that $g(K)\subset \text{Star}(v_1)$. To do this we need to consider all separating spheres in the prime decomposition at once. Let  $\{w_1, w_2, \dots, w_m\}$ be a set of separating spheres  in a prime decomposition of $M$ where $w_1$ is the core of $v_1$. We also assume that all $w_i$'s are transverse to the cores of the vertices in $f(K)$.  
\begin{claim}
Let $w$ be an embedded sphere in $M$ that is isotopic to $w_1$ and is transverse to all $w_i$'s. Let $C$ be an innermost circle in the intersection of $w$ and $w_i$'s, i.e. it bounds a disk $D$ in $w$ so that the interior of $D$ does not intersect $w_i$ for any $i$. If $C$ is in the intersection of $w_j$ and $w$, it bounds a disk $D'$ in $w_j$ so that the embedded sphere $D\cup D'$ bounds a ball in $M$.
\end{claim}
We call the ball  whose boundary is $D\cup D'$, the Whitney ball. Because we can push $D$ along the ball  to remove the intersection $C$.

\noindent\textit{Proof of the claim:} Since $\text{int}(D)$ does not intersect any of the spheres $w_i$'s, it lies entirely in one of the irreducible components, say $P_j\backslash \text{int}(D^3)$ whose boundary is the sphere $w_j$. Because $P_j$ is irreducible either of two disks in $w_j$ that bounds $C$ union $D$ is an embedded sphere in $P_j$, hence bounds a ball but one of these balls lies entirely in $P_j\backslash \text{int}(D^3)$. Therefore, the circle $C$ bounds a disk $D'$ in $w_j$ so that the sphere $D'\cup D$ bounds a ball in $M$. $\blacksquare$

By doing surgery similar to \Cref{eq:7}, we want to homotope the map $f$ to reduce the number of circles in the intersection of the core of vertices of $f(K)$ with the spheres $\{w_1, w_2, \dots, w_m\}$. By the claim for the core of any vertex in $f(K)$ that intersect the union of $w_i$'s, there exists a Whitney ball.  Let $\theta_0=f(s_0)\in f(K)$ be a vertex whose Whitney ball is innermost, i.e. if $\{\theta_1,\theta_2,\dots,\theta_n\}$ are the vertices that are connected to $\theta_0$ in $f(K)$, they do not intersect the Whitney ball of $\theta_0$. Therefore, by pushing the core of $\theta_0$ along the the Whitney ball, we could obtain a vertex $\theta_0'$ whose core is still disjoint from the core of vertices $\{\theta_1,\theta_2,\dots,\theta_n\}$. By considering a  near parallel copy of $\theta_0'$, we can assume that the core of $\theta_0'$ is also disjoint from the core of $\theta_0$. Therefore, we can homotope the map $f$ to a map $g$ so that it takes the same value on all vertices in $K$ but $s_0$ and $g(s_0)=\theta_0'$. By repeating this process, we reduce the number of circles in the intersection of the cores of $f(K)$ with the spheres $\{w_1, w_2, \dots, w_m\}$ until we homotope $f$ into the star of the vertex $v_1$. 
\end{proof}
The contractibility of these semi-simplicial spaces, similar to the case of surfaces, reduces Thurston's theorem to the case of $P\backslash \text{int}(D^3)$  and $N\backslash \cup_{i=1}^{k}\text{int}(D^3)$ and copies of $S^2\times [0,1]$.
\subsubsection{Reducing Thurston's theorem to the case of Haken manifolds} Let us first consider the case $S^2\times [0,1]$. 
\begin{prop}
\Cref{Main} holds for $M=S^2\times [0,1]$ if it holds for Haken $3$-manifolds.
\end{prop}
\begin{proof}
Choose a $1$-handle $\phi:D^1\times D^2\hookrightarrow S^2\times [0,1]$ so that $\phi(\{0\}\times D^2)\subset S^2\times \{0\}$ and $\phi(\{1\}\times D^2)\subset S^2\times \{1\}$. Note that the codimension of this $1$-handle is less that the half of the dimension of the ambient manifold. Therefore, similar to \Cref{torus}, transversality implies that $|\mathcal{H}^{\delta}_{\bullet}(S^2\times [0,1], \phi)|$ is contractible. Hence, as in \Cref{reduction}, Thurston's theorem holds for $S^2\times [0,1]$ if it holds for $S^2\times [0,1]\backslash \sigma$ for all $p$-simplices $\sigma\in \mathcal{H}^{\delta}_{p}(S^2\times [0,1], \phi)$ and all $p$. But for a $p$-simplex $\sigma$, the manifold $S^2\times [0,1]\backslash \sigma$ is a handle-body, so it is Haken. 
\end{proof}
As the general strategy is to cut along submanifolds, we always get manifolds with boundary. Furthermore, an irreducible $3$-manifold with boundary is Haken. But note that the sphere boundaries in $P\backslash \text{int}(D^3)$  and $N\backslash \cup_{i=1}^{k}\text{int}(D^3)$ destroys the irreducibility. So to reduce Thurston's theorem for $P\backslash \text{int}(D^3)$  and $N\backslash \cup_{i=1}^{k}\text{int}(D^3)$ to the case for Haken manifolds, we first cut along certain $1$-handles to reduce the number of sphere boundaries. To do so, we need to show that $P\backslash \text{int}(D^3)$  and $N\backslash \cup_{i=1}^{k}\text{int}(D^3)$  are not simply connected.
\begin{lem}\label{poincare}
If a $3$-manifold $M$ with boundary is simply connected, it is obtained from $S^3$ by removing the interior of a union of disjoint  balls in $S^3$. 
\end{lem}
\begin{proof}
It is enough to show that the boundary $\partial M$ is homeomorphic to union of $S^2$'s. Because if we fill in the sphere boundaries by balls, we obtain a simply connected closed $3$-manifold which has to be homeomorphic to $S^3$ by Perelman's theorem (\cite{perelman2002entropy, perelman2003ricci}). Since $M$ is simply connected, we have $H_1(M)=0$, so by the Poincar\' e-Lefschetz duality, we also have $H_2(M,\partial M)\cong H^1(M)=0$. The homology long exact sequence for the pair $(M, \partial M)$ implies that $H_2(M,\partial M)\to H_1(\partial M)\to H_1(M)$ is exact. Therefore, $H_1(\partial M)=0$ which implies that $\partial M$ is homeomorphic to a union of $S^2$'s.
\end{proof}
Let $Q$ be the manifold obtained from $P$ by removing the interior of $m$ disjoint balls in $P$. To prove Thurston's theorem for $Q$, we want to cut $1$-handles from $Q$ to make it irreducible. Not that since $P$ is not simply connected and is not homeomorphic to sphere, so by \Cref{poincare}, the manifold $Q$ is not simply connected either. Let $\partial_iQ$ be the $i$-th boundary component. We choose  an arc $\gamma_i$ with the two ends on $\partial_iQ$ so that the arc $\gamma_i$ with a path between its two ends on the boundary is non-trivial in the fundamental group of $P$. 

Let $\phi_i: D^1\times D^2\hookrightarrow Q$ be a $1$-handle whose core is $\gamma_i$. Let us denote the manifold obtained from $Q$ by removing the interior of the handle $\phi_i$ by $Q\backslash \cup_{i=1}^m \phi_i$. Given that $P$ is irreducible, it is easy to see that   $Q\backslash \cup_{i=1}^m \phi_i$ is also irreducible. Because every embedded sphere in $Q\backslash \cup_{i=1}^m \phi_i$ bounds a ball in $P$. If this ball contains any of the boundary components with the $1$-handle attached to it, then the core union the path between the two ends of the core on the boundary would be trivial in the fundamental group of $P$, which is a contradiction.
\begin{prop}\label{irr}
Thurston's theorem \ref{Main} holds for $Q$, if it does for Haken manifolds.
\end{prop}
\begin{proof}
 By the above discussion, if we remove at least one handle in $\mathcal{H}^{\delta}_{0}(Q,\phi_i)$ from  $Q$ for each $i$, we obtain a Haken manifold. Because it is an irreducible manifold whose boundary components have positive genus. We inductively reduce the number of sphere boundary components by cutting along $1$-handles whose cores are isotopic to a parallel copy of the core of $\phi_i$'s. 
 
 Similar to \Cref{reduction}, we want to  show that  \Cref{Main} holds for $Q$ if it holds for $Q\backslash \sigma$ for all $\sigma\in \mathcal{H}^{\delta}_{p}(Q,\phi_i)$ and all $p$. To do this, it is enough to show  the semi-simplicial sets $\mathcal{H}^{\delta}_{\bullet}(Q,\phi_i)$ has contractible realization. Since the codimension of the cores is larger than half of the dimension of $Q$, transversality implies that $|\mathcal{H}^{\delta}_{\bullet}(Q,\phi_i)|$  is contractible (see \Cref{torus}).
\end{proof}
Similarly, we can reduce \Cref{Main} for $N\backslash \cup_{i=1}^k \text{int}(D^3)$  the case where all boundary components have positive genus. Now we can apply prime decomposition for $3$-manifolds with boundary (\cite[Section 3]{MR2098385}). Note that all the prime factors are irreducible again, so we can apply \Cref{primedecomp} to inductively reduce to the case with fewer prime factors. Hence, \Cref{Main} for $N\backslash \cup_{i=1}^k \text{int}(D^3)$ is also deduced from \Cref{irr}.
\subsubsection{\Cref{Main} for Haken $3$-manifolds}  By the theory of Haken manifolds (\cite{MR0160196}), we know that they have a hierarchy, where they can be split up into 3-balls along incompressible surfaces. Let $\psi:S\times [-1,1]\hookrightarrow M$ be a proper embedding of an incompressible surface with its trivial normal bundle. Given the case of Haken manifolds which are lower compared to $M$ in the Haken hierarchy, we inductively prove \Cref{Main} for $M$ by considering the semi-simplicial set $\mathcal{K}^{\delta}_{\bullet}(M,\psi)$ (see \Cref{surfaces} to recall its definition).

 Note that for any $\sigma \in\mathcal{K}^{\delta}_{p}(M,\psi)$, the manifold $M\backslash \sigma$ is homeomorphic to the disjoint union of $M\backslash \psi(S\times\{0\})$ with $p$ copies of $S\times [-1,1]$. By  induction on the Haken hierarchy, we can assume that \Cref{Main} holds for  $M\backslash \psi(S\times\{0\})$. To apply \Cref{reduction}, we need also to know \Cref{Main} for  $S\times [-1,1]$. But $S\times [-1,1]$ is a handlebody (a $3$-ball with $1$-handles attached) so there are finitely many properly embedded $2$-disks (that are in fact incompressible surface in $S\times [-1,1]$) such that if we cut along those disks, we obtain a $3$-ball. Hence, it is a special case of Haken manifolds. Therefore, to finish the proof of \Cref{Main} for three manifolds, it is left to prove the following proposition.
 \begin{prop}\label{cutincomp}
 Let $M$ be a Haken manifold with boundary. The fat realization $|\mathcal{K}^{\delta}_{\bullet}(M,\psi)|$ is contractible.
 \end{prop}  
 \begin{proof}
Let us represent an element of the homotopy group $f:S^k\to |\mathcal{K}^{\delta}_{\bullet}(M,\psi)|$ by a PL map with respect to some triangulation $K$ on $S^k$. Similar to \Cref{eq:7} we can homotope $f$ so that the core of the vertices of $f(K)$ are pairwise transverse. Also by the same argument, we can choose $\phi\in \mathcal{K}^{\delta}_{0}(M,\psi)$ so that the collared embedding $\phi(S)$ is transverse to the core of vertices of $f(K)$ and its $t$-coordinate $t_{\phi}$ is different from that of vertices of $f(K)$. We want to homotope $f$ to a PL map $g:K\to |\mathcal{K}^{\delta}_{\bullet}(M,\psi)|$ so that $g(K)\subset \text{Star}(\phi)$, hence $f$ becomes nullhomotopic. 

Since the intersections of $\phi(S)$ with the core of vertices of $f(K)$ are transverse and also they do not intersect on the boundary $\partial M$, all intersections are circles. We want to do surgery on the image of $f$ to remove these circles. We first do surgery on the circles that are nullhomotopic in $M$.

\noindent{\bf Case 1:} Since $\phi(S)$ is incompressible, any nullhomotopic circle in the intersection of  $\phi(S)$ and the core of the vertices of $f(K)$ is in fact nullhomotopic in $\phi(S)$. Therefore such circles bound  a disk $D$ in $\phi(S)$. Choose a metric on the surface $\phi(S)$ and among the nullhomotopic circles in the intersection, let $C$ be the one whose interior has  the minimal area. Suppose $C$ is in the intersection of $\phi(S)$ and $\phi_0(S)$ where $\phi_0=f(v)\in f(K)$ is a vertex in the image of $f$ and $\{\phi_1,\phi_2,\dots, \phi_n\}$ is the set of all the vertices in $f(K)$ that are connected to $\phi_0$.

Again by the incompressibility the circle $C$ bounds a disk $D_0$ in $\phi_0(S)$. Since $M$ is irreducible, the sphere $D\cup D_0$ bounds a ball $B$ in $M$. Note that by the choice of the circle $C$, the ball $B$ does not intersect $\phi_i(S)$ for $1\leq i\leq n$. By pushing $D$ across $B$ to $D_0$ and considering a nearby parallel copy, we obtain $\phi'_0\in \mathcal{K}^{\delta}_{0}(M,\psi)$ whose core is disjoint from $\phi_0(S)$ and the core of all vertices that are connected to $\phi_0$ in $f(K)$. Therefore, we obtain a homotopy $F:K\times [0,1]\to |\mathcal{K}^{\delta}_{\bullet}(M,\psi)|$ where $F(-,0)=f$, $F(v,1)=\phi'$  and $F(-,1)$ is the same as $f$ on vertices other than $v$. But also note that $\phi_0'(S)$ has fewer circle component in its intersection with $\phi(S)$. Hence, by repeating this process, we can eliminate all nullhomotopic intersections. 
%To do surgery to remove the intersection $C$, we first need to move $\phi_0(S)$ slightly so that it becomes disjoint from itself. Because $\phi_0(S)$ has a trivial normal bundle we can move the embedding $\phi_0$ slightly in the normal direction of the surface $\phi_0(S)$ to obtain a vertex $\phi'_0\in \mathcal{K}^{\delta}_{0}(M,\psi)$ so that $\phi'_0$ is connected to all vertices in $\{\phi_0,\phi_1,\dots,\phi_n\}$. 
%
%Now let $C'$ be the intersection of $\phi(S)$ and $\phi'_0(S)$ that is inside of $D$. Similarly by the incompressibility, $C'$ bounds disks $D'$ and $D'_0$ in $\phi(S)$ and $\phi'_0(S)$ respectively. Let $B'$ be the ball in $M$ whose boundary is $D'\cup D'_0$.  To eliminate $C'$ from $\phi(S)\cap \phi'_0(S)$, we can isotope $D'$ across $B'$ to $D'_0$ and slightly further. By the isotopy extension theorem, we extend this isotopy to an ambient isotopy $h_t$ supported near $B$.  We can choose the isotopy so that $\phi_i(S)$ does not intersect the support of $h_t$ for $1\leq i\leq n$. Therefore, we obtain a homotopy $F:K\times [0,1]\to |\mathcal{K}^{\delta}_{\bullet}(M,\psi)|$ where $F(-,0)=f$, $F(v,1)=h_1(\phi')$  and $F(-,1)$ is the same as $f$ on vertices other than $v$. Now the number of circles in the intersections of $\phi(S)$ and the vertices of $F(-,1)$ has been reduced by one. By repeating this process, we can eliminate all nullhomotopic intersections. 

\noindent{\bf Case 2:} Now suppose none of the circles in the intersection of the cores of the vertices of $f(K)$ and $\phi(S)$ is nullhomotopic. We use Hatcher's idea (see \cite[Page 342]{MR0420620}) to deal with this case. Let $p:\tilde{M}\to M$ be the covering corresponding to the subgroup $\pi_1(\phi(S),a)\subset \pi_1(M,a)$ where $a$ is a base point in $\phi(S)$. Let $S_0$ be the homeomorphic lift of $\phi(S)$ passing through the base point $\tilde{a}$ of $\tilde{M}$. Let $\{ S_i\}$ be the components of $p^{-1}(\phi(S))$. Each $S_i$ separates $\tilde{M}$ into two components. Let $M_i$ be the closure of the component that does not contain the boundary of $S_0$. Let $M_{i_0}$ be minimal with respect to inclusion among $M_i$'s that intersect the lifts of the vertices of $f(K)$. 

Suppose that for a vertex $v\in K$, we denote the lift of the surface $f(v)(S)$  that intersects $M_{i_0}$ by $\widetilde{f(v)}(S)$, if it exists. Let $C_v$ be a component of $\widetilde{f(v)}(S)\cap M_{i_0}$. Laudenbach (see \cite[Corollary II.4.2]{MR0356056} and also \cite[page 8]{hatcher1999spaces}) showed that  there is a unique trivial h-cobordism $W_v$ in $M_{i_0}$ whose  one end is $C_v$ and the other end lies on a lift of $\phi(S)$. Furthermore, $p$ is a homeomorphism restricted to $W_v$.

Among the trivial cobordisms, for a vertex $v\in K$, let $W_v$ be  minimal with respect to inclusion.  Let $\{v_1,v_2,\dots,v_n\}$ be the vertices connected to $v$ in $K$. Since $f(v)(S)$ and $f(v_i)(S)$ are disjoint for all $i$ and $W_v$ is minimal, one can see that $W_v$ does not intersect $W_{v_i}$. Now the trivial cobordism $p(W_v)$ plays the role of the ball $B$ in the previous case. 

By pushing $p(C_v)$ across the trivial bordism to its other boundary and considering a nearby parallel copy, we obtain $\phi'_0\in \mathcal{K}^{\delta}_{0}(M,\psi)$ whose core is disjoint from $\phi_0(S)$ and the core of all vertices that are connected to $\phi_0$ in $f(K)$.  Therefore, we get a homotopy $F:K\times [0,1]\to |\mathcal{K}^{\delta}_{\bullet}(M,\psi)|$ where $F(-,0)=f$,  $F(v,1)=\phi'_0$  and $F(-,1)$ is the same as $f$ on vertices other than $v$. Now the number of circles in the intersections of $\phi(S)$ and the vertices of $F(-,1)$ has been reduced by one. By repeating this process, we can eliminate all the remaining intersections. 
 \end{proof}

% To imitate the previous argument, there is an isotopy that pushes $p(C_v)$ across the trivial bordism to its other boundary and slightly beyond. We extend this isotopy to an ambiant isotopy $h_t$. Note that by the choice of $v$, the surface $h_1(f(v)(S))$ is disjoint from $f(v_i)(S)$ for all $i$. Similar to the case $1$, we can move $h_1(f(v)(S))$ slightly in the normal direction to obtain $\phi'\in \mathcal{K}^{\delta}_{\bullet}(M,\psi)$ so that $\phi'$ is connected to $f(v)$ and $f(v_i)$ for all $i$. Therefore, we get a homotopy $F:K\times [0,1]\to |\mathcal{K}^{\delta}_{\bullet}(M,\psi)|$ where $F(-,0)=f$,  $F(v,1)=\phi'$  and $F(-,1)$ is the same as $f$ on vertices other than $v$. Now the number of circles in the intersections of $\phi(S)$ and the vertices of $F(-,1)$ has been reduced by one. By repeating this process, we can eliminate all the remaining intersections. 
\section{Smoothing theory in low dimensions} \label{sec3} Let $M$ be a smooth manifold of  dimension $2$ or $3$. 
To use a similar technique to prove \Cref{sm} which says 
\begin{equation}
\BDiff(M)\to \BH(M),
\end{equation}
is a weak equivalence, first one needs to show that \begin{equation}\label{assumption}\pi_0(\Diff(M))=\pi_0(\tH(M)).\end{equation} This is not hard for surfaces (\cite{boldsen2009different}) but for $3$-manifolds, it follows from a  theorem of Cerf (\cite{cerf1961topologie}). Assuming that $\Diff(M)$ and $\tH(M)$ have the same group of connected components, the statement is reduced to showing that the map
\[
\eta: \BDiff_0(M)\to \BH_0(M),
\]
induces a weak homotopy equivalence. But note that both spaces $\BDiff_0(M)$ and $ \BH_0(M)$ are simply connected, therefore it is enough to show that $\eta$ induces a homology isomorphism. 

Similar to the previous section, by using certain semi-simplicial spaces, we want to cut the manifold into pieces until we get to the disks. And the case for disks is a corollary of Smale's theorem (\cite{MR0112149}) for $2$-disks and Hatcher's theorem (\cite{hatcher1983proof}) for $3$-disks. But  the only difference to the previous section is instead of having  discrete and topologized versions of semisimplicial spaces, we would have the smooth version and the topological version with a slight modification. For those semi-simplicial spaces that involve the boundary of the manifold, we have to control the behavior near the boundary.  
\begin{cond}Let $M$ be a smooth manifold with boundary and let $c:\partial M\times [0,1)\hookrightarrow M$ be a collar neighborhood. For all semi-simplicial spaces (topological versions) that involved the boundary, namely  $\mathcal{H}_{\bullet}(M,\phi^q)$, we impose an extra condition of being smooth near the boundary:  for example a vertex $[\phi]\in \mathcal{H}_{0}(M,\phi^q)$ which is given by an embedding $\phi: D^q\times D^{n-q}\hookrightarrow M$ that restricts to a smooth embedding in a neighborhood of the boundary $c(\partial M\times [0,\epsilon) )$ for some $\epsilon$. Note that still  $\tH_0(M,\partial M)$ acts on $\mathcal{H}_{\bullet}(M,\phi^q)$ since $\tH_0(M,\partial M)$ consists of  homeomorphisms whose supports are away from the boundary. 
\end{cond}
\begin{defn}
We can define the smooth version of the topological version of all semi-simplicial spaces considered in the previous section and we denote them by superscript $sm$. For example, let $\mathcal{H}_{\bullet}^{sm}(M,\phi^q)\subset \mathcal{H}_{\bullet}(M,\phi^q)$ be a sub-semisimplicial space consisting of smooth handles with an induced $C^0$-topology rather than $C^{\infty}$-topology. 
\end{defn}
\begin{lem}
 The maps from the smooth version of the semi-simplicial spaces to the corresponding topological versions are equivariant with respect to the map $\Diff_0(M)\to \tH_0(M)$ and induce  bijections between the set of orbits of the corresponding actions.
\end{lem}
\begin{proof} We give the proof for the map 
\[
\mathcal{H}_{\bullet}^{sm}(M,\phi^q)\to \mathcal{H}_{\bullet}(M,\phi^q),
\]
and the other cases are similar. Note that in every orbit of the action of $\tH_0(M)$ on $\mathcal{H}_{\bullet}(M,\phi^q)$, there is a smooth handle. Hence, the induced map between orbits is surjective. To show that it is also injective, we want to show that for two smooth handles $\phi,\phi'\in \Emb_{\partial}^{sm}(D^q\times D^{n-q},M)\subset\Emb_{\partial}(D^q\times D^{n-q},M)$, if $\phi$ and $\phi'$ are in the same orbit of the action of $\tH_0(M,\partial M)$ on $\Emb_{\partial}(D^q\times D^{n-q},M)$, then they are in the same orbit of the action of $\Diff_0(M,\partial M)$ on $\Emb_{\partial}^{sm}(D^q\times D^{n-q},M)$. Now by the isotopy extension theorems for homeomorphisms and diffeomorphisms, we have maps between quasi-fibrations
\[
\begin{tikzcd}
\Diff(M\backslash\phi^q,\partial (M\backslash\phi^q))\arrow{r}\arrow{d}& \Diff(M,\partial M) \arrow{r}\arrow{d}&\Emb_{\partial}^{sm}(D^q\times D^{n-q},M)\arrow{d} \\\tH(M\backslash\phi^q,\partial (M\backslash\phi^q))\arrow[""]{r}& \tH(M,\partial M)\arrow{r}& \Emb_{\partial}(D^q\times D^{n-q},M).
\end{tikzcd}
\]
By our assumption \ref{assumption}, since the first two vertical maps induce bijection on $\pi_0$ so does the third vertical map. So, $\phi$ and $\phi'$ are in the same path component of $ \Emb_{\partial}^{sm}(D^q\times D^{n-q},M)$. Therefore, by the isotopy extension theorem again there exists an element in $\Diff_0(M,\partial M)$ that sends $\phi$ to $\phi'$.
\end{proof}
Similar to \Cref{eq:7}, one can prove that the smooth version of semi-simplicial spaces are weakly contractible.  Hence, a similar argument as the previous section reduces  \Cref{sm} to the fact that the map
\begin{equation}\label{disk}
\BDiff(D^n,\partial D^n)\to \BH(D^n,\partial D^n),
\end{equation}
is a weak equivalence in these dimensions.

%Recall in dimension $2$,  it is a theorem of Earle and Eells \cite{earle1969fibre} that for a surface $M$ with genus larger than $1$, the group $\Diff_0(M)$ is contractible. Therefore, by this technique the contractibility of $\tH_0(M)$ which is a theorem of Hamstrom \cite{hamstrom1974homotopy} can be reduced to the contractibility of $\Diff_0(M)$ which has a more concrete proof.
\begin{rem} In dimension $3$, Hatcher (\cite{hatcher1983proof}) proved that the map 
\[
\text{SO}(4)\xrightarrow{\simeq}\Diff(S^3),
\]
is a weak equivalence. It is standard to see that this version of Hatcher's theorem is equivalent to the weak equivalence \ref{disk} for $n=3$.  Cerf in (\cite{cerf1961topologie}) also used a different method to prove that the weak equivalence 
\[
\Diff(M)\xrightarrow{\simeq} \tH(M),
\]
can be reduced to  Hatcher's theorem.
\end{rem}
\section{Contractibility of the identity component of the diffeomorphism group for certain low dimensional manifolds}\label{sec4} Note that for a manifold $M$, the (weak) contractibility of $\Diff_0(M)$ is equivalent to the acyclicity of the classifying space $\BDiff_0(M)$. Similar to previous sections, we obtain a semisimplicial resolution for $\BDiff_0(M)$ by cutting the manifold into simpler pieces. To show  that $\BDiff_0(M)$ is acyclic, we then study the spectral sequence associated to the semisimplicial resolutions.
\subsection{Contractibility of $\Diff_0(\Sigma,\partial\Sigma)$ for a surface $\Sigma$ with  boundary} We sketch a new proof of the contractibility of the identity component of the diffeomorphisms of a surface with  boundary by first showing that $\BDiff_0(\Sigma,\partial \Sigma)$ is acyclic. Therefore, by the Whitehead theorem it should be contractible.  By the argument of the previous section, we deduce that $\tH_0(\Sigma,\partial \Sigma)$ is also weakly contractible.

For a closed surface $\Sigma$ with a negative Euler number, the contractibility of $\Diff_0(\Sigma)$ was first proved by Earle and Eells (\cite{earle1969fibre}) using Teichm\"{u}ller theory and it was later extended to the surfaces with boundary by Earle and Schatz (\cite{MR0277000}). Therefore, by the techniques of the previous sections, the contractibility of $\tH_0(M)$ which is a theorem of Hamstrom \cite{hamstrom1974homotopy} can be reduced to the contractibility of $\Diff_0(M)$ which has a more concrete proof. 

Gramain (\cite{MR0326773}) gave a topological proof of contractibility of $\Diff_0(\Sigma)$ for  $\Sigma$ with a negative Euler number and hence found a new proof of the contractibility of the Teichm\"{u}ller space.  As was explained in Hatcher's exposition (\cite[Appendix B]{hatcher2011short}), the case of the closed surface can be easily reduced to the case of a surface with boundary. 

Gramain's proof reduces to the case of a disk by proving that certain space of embeddings of arcs into a surface is contractible. But the advantage of working with semi-simplicial sets is that proving the contractibility of their realizations is often easier and more combinatorial. Having the contractibility of such semi-simplicial sets, it was a homotopy theory lemma (\Cref{claim1}) that implies that the realization of the corresponding semi-simplicial {\it spaces} is weakly contractible.  

As the input to our proof, we also use the contractibility of $\Diff(D^2,\partial D^2)$ and for a non-separating arc $\sigma$ between two points on the boundary, we use a $\pi_0$-statement that the the map between the mapping class groups
\begin{equation}\label{map}
\pi_0(\Diff(\Sigma\backslash \sigma, \partial \Sigma\backslash \sigma)) \to \pi_0(\Diff(\Sigma,\partial\Sigma)),
\end{equation}
is injective where $\Sigma\backslash \sigma$ is a surface obtained from $\Sigma$ by cutting along $\sigma$. 
 
\begin{thm}\label{surface}
Let $\Sigma$ be a surface with a boundary, $\BDiff_0(\Sigma,\partial \Sigma)$ is acyclic. 
\end{thm}
\begin{proof}
Similar to \cref{proof}, we use induction on handles to reduce to the case of the disk. For a $1$-handle $\phi$ and a $p$-simplex $\sigma_p\in \mathcal{H}_{p}(\Sigma,\phi)$, by the induction hypothesis, $\BDiff_0(\Sigma\backslash \sigma_p,\partial (\Sigma\backslash \sigma_p))$ is acyclic. Note that the surface $\Sigma\backslash \sigma_p$ is a union of of $p$ disjoint disks and a surface that is diffeomorphic to $\Sigma\backslash \phi$.

 By the isotopy extension theorem, we have a fibration
\[
\Diff(\Sigma\backslash \sigma_p,\partial (\Sigma\backslash \sigma_p))\to \Diff(\Sigma,\partial \Sigma)\to \text{Emb}_{\partial}(\sigma_p,\Sigma). 
\]
Given  the injectivity of the map \ref{map}, we deduce that 
\[
\Diff_0(\Sigma\backslash \sigma_p,\partial (\Sigma\backslash \sigma_p))\to \Diff_0(\Sigma,\partial \Sigma)\to \text{orb}(\sigma_p),
\]
is also a fibration. Hence, there is a weak equivalence  $$\coprod_{\sigma_p}\BDiff_0(\Sigma\backslash \sigma_p,\partial (\Sigma\backslash \sigma_p))\xrightarrow{\simeq}X_p(\Sigma, \phi),$$ where the disjoint union is over a representative set of orbits.
Given the induction hypothesis that $\BDiff_0(\Sigma\backslash \sigma_p,\partial (\Sigma\backslash \sigma_p))$ is acyclic, the spectral sequence
\[
E^1_{p,q}=H_q(X_p(\Sigma,\phi))\Rightarrow H_{p+q}(\BDiff_0(\Sigma,\partial\Sigma);\bZ),
\]
is concentrated in the first row $q=0$. We have $$H_0(X_p(\Sigma,\phi))= \bZ[\text{the set of the orbits of the $p$-simplices}].$$
Note that the set of orbits of the action of $\Diff_0(\Sigma,\partial \Sigma)$ on $\mathcal{H}_{p}(\Sigma,\phi)$ is in bijection with $(p+1)$-tuples $(t_0,t_1,\dots,t_p)$ in $\phi(\{0\}\times \text{int}(D^1))\subset \partial\Sigma$. Therefore, we can denote the set of the orbits by the semi-simplicial set $\text{Conf}(\bullet)$ where $\text{Conf}(p)$ is the set of $p+1$ points in $\bR$. Let us denote the first differential of the spectral sequence by $\delta$ which is given by the alternating sum $\sum (-1)^i {d_i}_*$ of the maps induced by the face maps, $d_i$, of the semi-simplicial set $\text{Conf}(\bullet)$. Hence, it is enough to prove the following claim:

\begin{claim}The chain complex $(\bZ[\text{\textnormal{Conf}}(\bullet)],\delta)$ is acyclic.
\end{claim}
\noindent{\it Proof of the claim: }To prove the claim, let us recall a theorem attributed to Moore in \cite[Theorem 4.1]{mostow1973notes}. For a topological space $X$, let $S_p(X)$ denote the the group of singular $p$-chains of $X$ with coefficients in $\bZ$.  For a semisimplicial space $X_{\bullet}$, let $\delta: S_*(X_{\bullet})\to S_*(X_{\bullet-1})$ denote the map given by the alternating sum of  maps induced by the face maps. Let $d: S_*(X_{\bullet})\to S_{*-1}(X_{\bullet})$ be the singular boundary maps. Recall the total differential is $D=d+(-1)^p\delta$ for elements in $S_p(X_{\bullet})$. Hence, we obtain a total chain complex $(S_*(X_{\bullet}),D)$. Moore (\cite[Theorem 4.1]{mostow1973notes}) proved that there is a natural chain equivalence
\[
f: (S_*(X_{\bullet}),D)\to (S_*(|X_{\bullet}|),d).
\]
Let us apply this theorem for $X_{\bullet}=\text{Conf}(\bullet)$. To compute the homology of the double complex $(S_*(\text{Conf}(\bullet)),D)$, we first filter it in the simplicial direction. The first page of the associated spectral sequence is $E^1_{p,q}=H_q(\text{Conf}(p);\bZ)$. But $\text{Conf}(\bullet)$ is a discrete space, therefore it is concentrated in the first row $q=0$. Note that the chain complex $E^1_{p,0}$ is the same as $(\bZ[\text{Conf}(\bullet)],\delta)$. Since this spectral sequence collapses, by the Moore theorem, we have
\[
E^2_{p,0}=E^{\infty}_{p,0}=H_p(|\text{Conf}(\bullet)|;\bZ).
\]
But similar to \Cref{claim}, one can show that $|\text{Conf}(\bullet)|$ is weakly contractible, therefore the chain complex $E^1_{*,0}=(\bZ[\text{Conf}(\bullet)],\delta)$ must be acyclic. 
\end{proof}
\subsection{Contractibility of $\Diff_0(M,\partial M)$ for a Haken manifold $M$ with  boundary} Hatcher computed the homotopy type of the space of PL homemorphisms of Haken manifolds in \cite{MR0420620}. Given his proof of Smale's conjecture, his computation of PL homeomorphisms carries over  to diffeomorphisms of the Haken manifolds (\cite{hatcher1999spaces}).  Here, we simplify his  proof of the contractibility of $\Diff_0(M,\partial M)$ for a Haken manifold $M$ with  boundary using the same idea as the proof of \Cref{surface}. Hatcher improved Laudenbach's surgery techniques (\cite[Chapter 2.5]{MR0356056}) to a parametrized surgery on the space of incompressible surfaces. In a way, we simplify Hatcher's proof by avoiding his parametrized surgery argument. 

%Here instead of cutting the manifold $M$ along the handles, we cut $M$ along sequence of incompressible surfaces. Recall that a compact connected surface $S$, not $S^2$, in $M$ is an incompressible surface, if $S\cap \partial  M=\partial S$, the normal bundle of $S$ is trivial and the inclusion $S\hookrightarrow M$ is $\pi_1$ injective. Given the Haken manifold theory, there is a finite sequence of incompressible surfaces that as we cut $M$ along those surfaces, we obtain disjoint union of balls. Therefore similar to the handle resolution (\Cref{def2}), we define a resolution associated to the incompressible surface $S$. Given that the normal of $S$ is trivial, we fix an embedding of the two-sided collar $\psi:S\times [-1,1]\hookrightarrow M$. We define the germ of embeddings of $S$ into $M$ similar to \Cref{germ} and we define the core of $\psi$ be its restriction to $S\times \{0\}$.
Let $\psi:S\times [-1,1]\hookrightarrow M$ be an embedding of a two-sided collar of an incompressible surface. Recall in \Cref{surfaces}, we defined $\mathcal{K}_{\bullet}(M,\psi)$ whose realization is weakly contractible as the corollary of \Cref{cutincomp}. Hence, we can define  an augmented semi-simplicial space $X_{\bullet}(M,\psi)\to \BDiff_0(M,\partial M)$ as follows
\[
X_{\bullet}(M,\psi):= \mathcal{K}_{\bullet}(M,\psi)\hcoker \Diff_0(M,\partial M).
\]
Given that $|\mathcal{K}_{\bullet}(M,\psi)|$ is weakly contractible,  $X_{\bullet}(M,\psi)$ is a semisimplicial resolution for $\BDiff_0(M,\partial M)$ i.e. the induced map
\[
|X_{\bullet}(M,\psi)|\to\BDiff_0(M,\partial M),
\]
is a weak equivalence.
%\begin{defn}Similar to \Cref{def2}, we define different semisimplicial spaces associated to the embedding $\psi$.
%\begin{itemize}\item {\bf Discrete version:} For $\psi:S\times [-1,1]\hookrightarrow M$, let $\mathcal{K}^{\delta}_{\bullet}(M,\psi)$ be a semisimplicial set defined as follows:
%\begin{itemize}
%\item The set of $0$-simplices $\mathcal{K}^{\delta}_0(M,\psi)$, consists of pairs $(t,[\phi])$ where $[\phi]$ is a germ of an embedding $S\times [-1,1]\hookrightarrow M$ so that for a small $\epsilon$ we have 
%%\item Let $e_{q+1}$ be the $(q+1)$-th standard basis element. The space of $0$-simplices $\mathcal{H}_0(M,\phi^q)$, consist of pairs $(t,\phi)$ where $\phi$ is an embedded  $q$-handle $D^q\times D^{n-q}\hookrightarrow M$ so that
%\[
%\phi(\partial S\times (-\epsilon,\epsilon))=\psi(\partial S\times(t-\epsilon,t+\epsilon)),
%\]
%and $\phi(S\times \{0\})$ is isotopic to $\psi(S\times \{t\})$ relative to the boundary. 
%\item The set of  $p$-simplices   $\mathcal{K}^{\delta}_p(M,\psi)$, consists of $(p+1)$-tuples $$((t_0,[\phi_0]), (t_1,[\phi_1]),\dots, (t_p, [\phi_p])),$$ in $\mathcal{K}^{\delta}_0(M,\psi)^{p+1}$ so that $t_0<t_1<\cdots<t_p$ and  the embedded surfaces  $\phi_i(S\times \{0\})$ are disjoint. The face maps are given by forgetting the embeddings.
%
%Note that for every $(t,\phi)\in \mathcal{K}^{\delta}_{0}(M,\psi)$, the coordinate $t$ is uniquely determined by $\phi$. We might just write $\phi$ for a vertex and refer to its $t$-coordinate by $t_{\phi}$. 
%%\item Let $\mathcal{K}_{\bullet}(M,\psi)$ be the semisimplicial space whose $0$-simplices consist of pairs $(t,\phi)$ where $(t,[\phi])\in \mathcal{K}^g_0(M,\psi)$. We topologize $\mathcal{K}_0(M,\psi) $ as the subspace of real numbers times the space of embeddings of $S$ into $M$. The space of $p$-simplices $\mathcal{K}_p(M,\psi)$ consists of $(p+1)$-tuples $$((t_0,\phi_0), (t_1,\phi_1),\dots, (t_p, \phi_p)),$$ in $\mathcal{K}_0(M,\psi)^{p+1}$ so that $t_0<t_1<\cdots<t_p$ and  the collared embedings  $\phi_i(S\times [-1,1])$ are disjoint. It is topologized as the subspace of $\mathcal{K}_0(M,\psi)^{p+1}$.
%\end{itemize}
%\item {\bf Topological versions:} Let $\mathcal{K}_{\bullet}(M,\psi)$  be the semisimplicial space whose $0$-simplices as a set consist of pairs $(t,\phi)$ where $(t,[\phi])\in\mathcal{K}^{\delta}_{0}(M,\psi)$. We topologize $\mathcal{K}_0(M,\psi) $ as the subspace of real numbers times the space of embeddings the collared surface $S\times [-1,1]$ into $M$ equipped with the compact-open topology. The space of $p$-simplices $\mathcal{K}_p(M,\psi)$ is a subspace of  $\mathcal{K}_0(M,\psi)^{p+1}$ consisting of $(p+1)$-tuples $$((t_0,\phi_0), (t_1,\phi_1),\dots, (t_p, \phi_p)),$$ so that $t_0<t_1<\cdots<t_p$ and  the embedded collared surfaces  $\phi_i(S\times [-1,1])$ are disjoint. It is topologized with the subspace topology.
%
%
%\end{itemize}
%\end{defn}

%Note that $\Diff_0(M,\partial M)$ acts on $\mathcal{K}_{\bullet}(M,\psi)$, so we can define an augmented semisimplicial space $X_{\bullet}(M,\psi)\to \BDiff_0(M,\partial M)$ as follows
%\[
%X_{\bullet}(M,\psi):= \mathcal{K}_{\bullet}(M,\psi)\hcoker \Diff_0(M,\partial M).
%\]
%Recall to prove that $X_{\bullet}(M,\psi)$ is a semisimplicial resolution for $\BDiff_0(M,\partial M)$ i.e. the induced map
%\[
%|X_{\bullet}(M,\psi)|\to\BDiff_0(M,\partial M),
%\]
%is a weak equivalence, it is enough to prove that $|\mathcal{K}_{\bullet}(M,\psi)|$ is weakly contractible and to prove the latter, similar to \Cref{claim1}, it is enough to prove 
%\begin{lem}\label{incompressible}
%The geometric realization $|\mathcal{K}^{\delta}_{\bullet}(M,\psi)|$ is weakly contractible.
%\end{lem}
%We shall use Laudenbach's surgery technique on incompressible surfaces to prove this lemma. But before proving the lemma, let us see how one can proceed to show that $\Diff_0(M,\partial M)$ is contractible. Similar to \Cref{surface}, we need to show
\begin{thm}\label{haken}
If $M$ is a Haken manifold with boundary, the classifying space $\BDiff_0(M,\partial M)$ is acyclic. 
\end{thm}
\begin{proof}
Note that for every $p$-simplex $\sigma_p\in \mathcal{K}_{p}(M,\psi)$, the space $M\backslash \sigma_p$ is diffeomorphic to the disjoint union of $M\backslash \psi(S)$ with $p$ copies of $S\times [-1,1]$. By the induction on the Haken hierarchy, we can assume that $\BDiff_0(M\backslash \psi(S),\partial (M\backslash \psi(S)))$ is acyclic. Recall that $S\times [-1,1]$ is a handlebody  so there are finitely many embedded $2$-disks such that if we cut along those disks, we obtain a $3$-ball. Thus contractibility of  $\Diff_0(S\times [-1,1],\partial(S\times [-1,1]))$ is in fact a special case of \Cref{haken}. Therefore, we can assume that for all $\sigma_p$, the space $\BDiff_0(M\backslash \sigma_p,\partial(M\backslash \sigma_p))$ is acyclic. 

To identify the weak homotopy type of $X_p(M,\psi)$, we need to determine the homotopy type of $\text{Stab}(\sigma_p)$ for each $\sigma_p\in  \mathcal{K}_{p}(M,\psi)$. Recall by the isotopy extension theorem, we have a fibration
\[
\Diff(M\backslash \sigma_p,\partial (M\backslash \sigma_p))\to \Diff(M,\partial M)\to \text{Emb}_{\partial}(\sigma_p,M). 
\]
By \cite[Chapter 2, Section 7.2]{MR0356056}, the fundamental group of $ \text{Emb}_{\partial}(\sigma_p,M)$ is trivial, therefore we have an injection
\[
\pi_0(\Diff(M\backslash \sigma_p,\partial (M\backslash \sigma_p)))\hookrightarrow \pi_0( \Diff(M,\partial M)).
\]
Thus we have a fibration 
\[
\Diff_0(M\backslash \sigma_p,\partial (M\backslash \sigma_p))\to \Diff_0(M,\partial M)\to \text{orb}(\sigma_p),
\]
which implies that there is a weak equivalence  $$\coprod_{\sigma_p}\BDiff_0(M\backslash \sigma_p,\partial (M\backslash \sigma_p))\xrightarrow{\simeq}X_p(M, \psi),$$ where the disjoint union is over a representative set of orbits. Therefore, similar to the proof of \Cref{surface}, the spectral sequence
\[
E^1_{p,q}=H_q(X_p(M,\psi))\Rightarrow H_{p+q}(\BDiff_0(M,\partial M);\bZ),
\]
implies that $\BDiff_0(M,\partial M)$ is acyclic.
\end{proof}
%\begin{proof}[Sketch of the proof of \Cref{incompressible}] Note that $|\mathcal{K}^{\delta}_{\bullet}(M,\psi)|$ is a simplicial complex. Suppose $f:S^k\to |\mathcal{K}^{\delta}_{\bullet}(M,\psi)|$ is a PL map with respect to some triangulation $K$ on $S^k$. Similar to \Cref{eq:7} we can homotope $f$ so that the core of the vertices of $f(K)$ are pairwise transverse. By Thom's transversality, we can choose $\phi\in \mathcal{K}^{\delta}_{0}(M,\psi)$ so that the collared embedding $\phi(S)$ is transverse to the core of vertices of $f(K)$ and its $t$-coordinate $t_{\phi}$ is different from that of vertices of $f(K)$. We want to homotope $f$ to a PL map $g:K\to |\mathcal{K}^{\delta}_{\bullet}(M,\psi)|$ so that $g(K)\subset \text{Star}(\phi)$, hence $f$ becomes nullhomotopic. 
%
%Since the intersections of $\phi(S)$ with the core of vertices of $f(K)$ are transverse and also they do not intersect on the boundary $\partial M$, all intersections are circles. We want to do surgery on the image of $f$ to remove these circles. We first do surgery on the circles that are nullhomotopic in $M$.
%
%\noindent{\bf Case 1:} Since $\phi(S)$ is incompressible, any nullhomotopic circle in the intersection of  $\phi(S)$ and the core of the vertices of $f(K)$ is in fact nullhomotopic in $\phi(S)$. Therefore such circles bound  a disk $D$ in $\phi(S)$. Choose a metric on the surface $\phi(S)$ and among the nullhomotopic circles in the intersection, let $C$ be the one whose interior has  the minimal area. Suppose $C$ is in the intersection of $\phi(S)$ and $\phi_0(S)$ where $\phi_0=f(v)\in f(K)$ is a vertex in the image of $f$ and $\{\phi_1,\phi_2,\dots, \phi_n\}$ is the set of all the vertices in $f(K)$ that are connected to $\phi_0$.
%
%Again by the incompressibility the circle $C$ bounds a disk $D_0$ in $\phi_0(S)$. Since $M$ is irreducible, the sphere $D\cup D_0$ bounds a ball $B$ in $M$. Note that by the choice of the circle $C$, the ball $B$ does not intersect $\phi_i(S)$ for $1\leq i\leq n$. To do surgery to remove the intersection $C$, we first need to move $\phi_0(S)$ slightly so that it becomes disjoint from itself. Because $\phi_0(S)$ has a trivial normal bundle we can move the embedding $\phi_0$ slightly in the normal direction of the surface $\phi_0(S)$ to obtain a vertex $\phi'_0\in \mathcal{K}^{\delta}_{0}(M,\psi)$ so that $\phi'_0$ is connected to all vertices in $\{\phi_0,\phi_1,\dots,\phi_n\}$. 
%
%Now let $C'$ be the intersection of $\phi(S)$ and $\phi'_0(S)$ that is inside of $D$. Similarly by the incompressibility, $C'$ bounds disks $D'$ and $D'_0$ in $\phi(S)$ and $\phi'_0(S)$ respectively. Let $B'$ be the ball in $M$ whose boundary is $D'\cup D'_0$.  To eliminate $C'$ from $\phi(S)\cap \phi'_0(S)$, we can isotope $D'$ across $B'$ to $D'_0$ and slightly further. By the isotopy extension theorem, we extend this isotopy to an ambient isotopy $h_t$ supported near $B$.  We can choose the isotopy so that $\phi_i(S)$ does not intersect the support of $h_t$ for $1\leq i\leq n$. Therefore, we obtain a homotopy $F:K\times [0,1]\to |\mathcal{K}^{\delta}_{\bullet}(M,\psi)|$ where $F(-,0)=f$, $F(v,1)=h_1(\phi')$  and $F(-,1)$ is the same as $f$ on vertices other than $v$. Now the number of circles in the intersections of $\phi(S)$ and the vertices of $F(-,1)$ has been reduced by one. By repeating this process, we can eliminate all nullhomotopic intersections. 
%
%\noindent{\bf Case 2:} Now suppose none of the circles in the intersection of the cores of the vertices of $f(K)$ and $\phi(S)$ is nullhomotopic. We use Hatcher's idea (see \cite[Page 342]{MR0420620}) to deal with this case. Let $p:\tilde{M}\to M$ be the covering corresponding to the subgroup $\pi_1(\phi(S),a)\subset \pi_1(M,a)$ where $a$ is a base point in $\phi(S)$. Let $S_0$ be the homeomorphic lift of $\phi(S)$ passing through the base point $\tilde{a}$ of $\tilde{M}$. Let $\{ S_i\}$ be the components of $p^{-1}(\phi(S))$. Each $S_i$ separates $\tilde{M}$ into two components. Let $M_i$ be the closure of the component that does not contain the boundary of $S_0$. Let $M_{i_0}$ be minimal with respect to inclusion among $M_i$'s that intersect the lifts of the vertices of $f(K)$. 
%
%Suppose that for a vertex $v\in K$, we denote the lift of the surface $f(v)(S)$  that intersects $M_{i_0}$ by $\widetilde{f(v)}(S)$, if it exists. Let $C_v$ be a component of $\widetilde{f(v)}(S)\cap M_{i_0}$. Laudenbach (\cite[Corollary II.4.2]{MR0356056}) showed that  there is a unique trivial h-cobordism $W_v$ in $M_{i_0}$ with one end $C_v$ and the other end on a lift of $\phi(S)$. Furthermore, $p$ is a homeomorphism restricted to $W_v$.
%
%Among the trivial cobordisms, for a vertex $v\in K$, let $W_v$ be  minimal with respect to inclusion.  Let $\{v_1,v_2,\dots,v_n\}$ be the vertices connected to $v$ in $K$. Since $f(v)(S)$ and $f(v_i)(S)$ are disjoint for all $i$ and $W_v$ is minimal, one can see that $W_v$ does not intersect $W_{v_i}$. Now the trivial cobordism $p(W_v)$ plays the role of the ball $B$ in the previous case.  To imitate the previous argument, there is an isotopy that pushes $p(C_v)$ across the trivial bordism to its other boundary and slightly beyond. We extend this isotopy to an ambiant isotopy $h_t$. Note that by the choice of $v$, the surface $h_1(f(v)(S))$ is disjoint from $f(v_i)(S)$ for all $i$. Similar to the case $1$, we can move $h_1(f(v)(S))$ slightly in the normal direction to obtain $\phi'\in \mathcal{K}^{\delta}_{\bullet}(M,\psi)$ so that $\phi'$ is connected to $f(v)$ and $f(v_i)$ for all $i$. Therefore, we get a homotopy $F:K\times [0,1]\to |\mathcal{K}^{\delta}_{\bullet}(M,\psi)|$ where $F(-,0)=f$,  $F(v,1)=\phi'$  and $F(-,1)$ is the same as $f$ on vertices other than $v$. Now the number of circles in the intersections of $\phi(S)$ and the vertices of $F(-,1)$ has been reduced by one. By repeating this process, we can eliminate all the remaining intersections. 
%\end{proof}
\begin{rem} We end with a question about hyperbolic three manifolds. Let $M$ be closed hyperbolic $3$-manifold. Gabai in \cite{gabai2001smale} used his high powered ``insulator" machinary (see \cite{gabai1997geometric}) and minimal surface theory to prove that $\Diff_0(M)$ is contractible by reducing to the case of Haken manifolds with boundary. We wondered if the same techniques could prove Gabai's theorem without using high powered tools in geometry.  To find a semisimplicial resolution for $\BDiff_0(M)$ let  $\gamma$ be a closed geodesic in $M$. Fix a parametrized tubular neighborhood of $\gamma$ by embedding $\phi: D^2\times S^1\hookrightarrow M$ so that $\phi(\{(0,0)\}\times S^1)=\gamma$. 
\begin{defn}Let $B_{\bullet}(M)$ be a semisimplicial space whose space of $0$ simplices is given by the space of oriented closed curves that are isotopic to $\gamma$. We define $B_p(M)$ as a subspace of $B_0(M)^{p+1}$ to be the space of $(p+1)$-tuples $\sigma_p=(\gamma_0,\gamma_1,\dots, \gamma_p)$ so that there exists a diffeomorphism $f_{\sigma_p}\in \Diff_0(M)$ where $f_{\sigma_p}(\gamma_i)=\phi(\{(t_i,0)\}\times S^1)$ for a $t_i$ such that $ t_0<t_1<\dots<t_p$. The $i$-th face maps is given by forgetting the $i$-th curve.
\end{defn}
\begin{quest}
Is $|B_{\bullet}(M)|$ weakly contractible?
\end{quest}
Note that similar to \Cref{claim1}, it is enough to show that realization of the semi-simplicial set $B_{\bullet}(M)^{\delta}$ is contractible. If the answer to this question is affirmative, one could give a simpler proof of Gabai's theorem as follows: Consider the semisimplicial resolution
\[
B_{\bullet}(M)\hcoker \Diff_0(M)\to \BDiff_0(M).
\]
Since the action of $\Diff_0(M)$ on $B_{\bullet}(M)$ is transitive, for a $p$-simplex $\sigma_p$ in $B_p(M)$, we have $B_p(M)\hcoker \Diff_0(M)\simeq \mathrm{B}\text{Stab}(\sigma_p)$. Given that the complement of $\sigma_p$ in $M$ is a Haken manifold, the identity component of $\text{Stab}(\sigma_p)$ is contractible, therefore $ \mathrm{B}\text{Stab}(\sigma_p)\simeq  \mathrm{B}\pi_0(\text{Stab}(\sigma_p))$. On the other hand, using JSJ decomposition and some hyperbolic geometry, it is not hard to show that $\pi_0(\text{Stab}(\sigma_p))$ is isomorphic to the pure braid group $\text{PBr}_{p+1}$. Hence, one might have a spectral sequence
\[
E^1_{p,q}=H_q(\mathrm{B}\text{PBr}_{p+1})\Rightarrow H_{p+q}(\BDiff_0(M);\bZ),
\]
but recall that a model for $\mathrm{B}\text{PBr}_{p+1}$ is an ordered configuration space $\text{Emb}([p], D^2)$. Thus the above spectral sequence converges to the realization of the semi simplicial space $\text{Emb}([\bullet], D^2)$. Now from \Cref{claim1} we know that the realization of $\text{Emb}([\bullet], D^2)$ is weakly contractible, therefore the above spectral sequence converges to zero in positive degrees.  
\end{rem}
%\subsection{Contractibility of $\Diff_0(M)$ for a hyperbolic three manifold $M$} Let $M$ be a closed hyperbolic $3$-manifold. Gabai in \cite{gabai2001smale} used his high powered ``insulator" machinary (see \cite{gabai1997geometric}) to prove that $\Diff_0(M)$ is contractible by reducing to the case of Haken manifolds with boundary. In this section, we use similar techniques as previous sections to give a simpler  proof that $\Diff_0(M)$ is contractible. The only geometric inputs in our proof are the JSJ torus decomposition (see \cite[Theorem 1.9]{hatcher2000notes}) and a lemma in \cite[Lemma 6.1]{gabai2001smale} which implies that if $T$ is a tubular neighborhood of a closed geodesic in $M$ and $f\in \Diff_0(M)$ fixes $T$ pointwise, then $f|_{M\backslash T}$ as an element of $\Diff(M\backslash T,\partial(M\backslash T))$ is also isotopic to the identity.
%
%Similar to previous sections we want to find a semisimplicial resolution for $\BDiff_0(M)$. Let  $\gamma$ be a closed geodesic in $M$. Fix a parametrized tubular neighborhood of $\gamma$ by embedding $\phi: D^2\times S^1\hookrightarrow M$ so that $\phi(\{(0,0)\}\times S^1)=\gamma$. 
%
%%\begin{defn}Let $B_{\bullet}(M)$ be a semisimplicial space whose space of $0$ simplices is given by the space of oriented closed curves that are isotopic to $\gamma$. We define $B_p(M)$ as a subspace of $B_0(M)^{p+1}$ to be the space of $(p+1)$-tuples $\sigma_p=(\gamma_0,\gamma_1,\dots, \gamma_p)$ so that there exists a diffeomorphism $f_{\sigma_p}\in \Diff_0(M)$ where $f_{\sigma_p}(\gamma_i)=\phi(\{(t_i,0)\}\times S^1)$ for a $t_i$ such that $ t_0<t_1<\dots<t_p$. The $i$-th face maps is given by forgetting the $i$-th curve.
%%\end{defn}
%\begin{defn}\label{torus} Let $T_{\bullet}(M)$ be a semisimplicial space whose space of $0$ simplices is given by the space of embedded solid tori whose cores are isotopic to $\gamma$. We define $T_p(M)$ as a subspace of $T_0(M)^{p+1}$ consisting of $(p+1)$-tuples $\sigma_p=(\tau_0,\tau_1,\dots, \tau_p)$  of disjoint embedded solid tori so that there exists  a diffeomorphism $f_{\sigma_p}\in \Diff_0(M)$ such that $\phi(D^2\times S^1)$ contains $f_{\sigma_p}(\tau_i)$ for all $i$. The $i$-th face map is given by forgetting the $i$-th solid torus.
%
%
%%=\phi(\{D^2_{\epsilon}+(t_i,0)\}\times S^1)$ for a $t_i$ such that $ t_0<t_1<\dots<t_p$. Recall from \Cref{germ} that $D^2_{\epsilon}$ is a disk of radius $\epsilon$. The $i$-th face map is given by forgetting the $i$-th solid torus.
%\end{defn}
%\begin{lem}\label{claim2}
%The geometric realization $|T_{\bullet}(M)|$ is weakly contractible.
%\end{lem}
%\begin{proof}
%We first define an auxiliary semisimplicial set $T^{\delta}_{\bullet}(M)$ where $T^{\delta}_{0}(M)$ is  the same as $T_0(M)$ as a set (with the discrete topology) and $T^{\delta}_{p}(M)$ is the set of $(p+1)$-tuples $\sigma_p=(\tau_0,\tau_1,\dots, \tau_p)$  of  embedded solid tori with disjoint cores so that there exists a    diffeomorphism $f_{\sigma_p}\in \Diff_0(M)$ such that $\phi(D^2\times S^1)$ contains $f_{\sigma_p}(\text{core}(\tau_i))$ for all $i$. Similar to the first case in the proof of \Cref{eq:7}, one can see $|T_{\bullet}^{\delta}(M)|$ is contractible, hence \Cref{claim1} implies that $|T_{\bullet}(M)|$ is weakly contractible. 
%%$f_{\sigma_p}(\text{core}(\tau_i))=\phi(\{(t_i,0)\}\times S^1)$ for a $t_i$ such that $ t_0<t_1<\dots<t_p$. Similar to the first case in the proof of \Cref{eq:7}, one can see $|T_{\bullet}^{\delta}(M)|$ is contractible, hence \Cref{claim1} implies that $|T_{\bullet}(M)|$ is weakly contractible. 
%\end{proof}
%Note that since the solid tori $\tau_i$ might be linked together, the action of $\Diff_0(M)$  on the space of $p$ simplices, $T_p(M)$ is not transitive. Nonetheless by taking the homotopy quotient of this action, we obtain an augmented semisimplicial space 
%\begin{equation}\label{s1}
%T_{\bullet}(M)\hcoker \Diff_0(M)\to \BDiff_0(M).
%\end{equation}
%Therefore \Cref{claim2} implies that after realizing the above augmentation map we have a weak equivalence
%\[
%|T_{\bullet}(M)\hcoker \Diff_0(M)|\xrightarrow{\simeq} \BDiff_0(M).
%\]
%\begin{thm}[Gabai \cite{gabai2001smale}]
%For a hyperbolic closed three manifold $M$, the identity component $\Diff_0(M)$ is contractible.
%\end{thm}
%\begin{proof}
%Recall that it is enough to show that $\BDiff_0(M)$ is acyclic. Hence we want to show that the spectral sequence associated to skeleta filtration of $|T_{\bullet}(M)\hcoker \Diff_0(M)|$ whose first page is 
%\begin{equation}\label{1}
%E^1_{p,q}=H_q(T_{p}(M)\hcoker \Diff_0(M))\Rightarrow H_{p+q}(\BDiff_0(M))
%\end{equation}
%converges to zero in positive degrees. To determine the weak homotopy type of $T_{\bullet}(M)\hcoker \Diff_0(M)$, fix an element $\sigma_p\in T_p(M)$ and let $\text{Stab}(\sigma_p)$ be the stabilizer group of $\sigma_p$ topologized as the subgroup of $\Diff_0(M)$. Using the fibration
%\[
%\text{Stab}(\sigma_p)\to \Diff_0(M)\to \text{orbit}(\sigma_p),
%\]
%similar to \Cref{action1} we obtain \begin{equation}\label{rr}T_{p}(M)\hcoker \Diff_0(M)\simeq\coprod_{[\sigma_p]} \mathrm{B}\text{Stab}(\sigma_p),\end{equation} where the disjoint union is over the orbits of the action of $\Diff_0(M)$ on $T_p(M)$. In fact the set of orbits can be described by how the embedded solid tori in $\sigma_p$ are linked but we do not need it.
%
%Note that the identity component of $\text{Stab}(\sigma_p)$ is the group $\Diff_0(M\backslash \sigma_p,\partial(M\backslash \sigma_p))$. Recall $M\backslash \sigma_p$ is the manifold obtained by removing the interior of the solid tori in $\sigma_p$ from $M$. Note that  $M\backslash \sigma_p$ is a manifold whose boundary is diffeomorphic to disjoint union of tori, hence it is Haken. Thus by Hatcher's theorem \ref{haken}, we know $\Diff_0(M\backslash \sigma_p,\partial(M\backslash \sigma_p))$ is contractible. Given that the identity component of $\text{Stab}(\sigma_p)$ is contractible, to identify the homotopy type of $ \mathrm{B}\text{Stab}(\sigma_p)$, we have to determine $\pi_0(\text{Stab}(\sigma_p))$. Let us denote the mapping class group of $M\backslash \sigma_p$ that fixes the boundary by $\text{Mod}_{\partial}(M\backslash \sigma_p)$ and the mapping class group of $M$ by $\text{Mod}(M)$. The inclusion $M\backslash \sigma_p\hookrightarrow M$ induces a map
%\[
%\text{Mod}_{\partial}(M\backslash \sigma_p)\to \text{Mod}(M),
%\]
%that extends mapping classes by the identity over $\sigma_p$.
%Note that elements of $\text{Stab}(\sigma_p)$ as a subgroup of $\Diff(M\backslash \sigma_p,\partial(M\backslash \sigma_p))$ become isotopic to the identity when we extend them by the identity over $\sigma_p$. Thus one can see
%\begin{equation}\label{rrr}
%\pi_0(\text{Stab}(\sigma_p))\cong \ker(\text{Mod}_{\partial}(M\backslash \sigma_p)\to \text{Mod}(M)).
%\end{equation}
%As we shall see in the proof of the claim below, the above kernel is independent of the manifold $M$ which implies that the weak homotopy type of $T_{p}(M)\hcoker \Diff_0(M)$ is independent of $M$.
%
%{\bf Claim:} We have the weak equivalence \begin{equation}\label{emb}
%T_{p}(M)\hcoker \Diff_0(M)\simeq \text{Emb}(\coprod_{[p]} D^2\times S^1, \text{int}(D^2)\times S^1),\end{equation}where the right hand side is the ordered configuration space of $p+1$ disjoint solid tori inside a bigger solid torus whose cores are isotopic to the core of the bigger solid torus.
%
%%We have $\pi_0(\text{Stab}(\sigma_p))\cong \text{PR}\beta_{p+1}$ where $\text{PR}\beta_{p+1}$ is the pure ribbon braid group on $p+1$ strands which is isomorphic to $ \pi_0(A_p(D^2))$ (see \Cref{def1}).
%
%\noindent\textit{Proof of the claim:} Note that the right hand side of the weak equivalence \ref{emb} is not connected and the set of connected components depends on how the embedded tori are linked together. Recall that for all $\sigma_p\in T_p(M)$ there exists a diffeomorphism $f_{\sigma_p}$ that sends the tori in $\sigma_p$ into the solid torus $T:=\phi(D^2\times S^1)$. Hence in every orbit of the action of $\Diff_0(M)$ on $T_p(M)$, we can choose a $p$-simplex whose tori are embedded into $T$. Let $\text{Emb}_{[\sigma_p]}(\coprod_{[p]} D^2\times S^1, \text{int}(D^2)\times S^1)$ be the connected component that contains a representative of the orbit $[\sigma_p]$. We want to show that 
%\[
%\text{Emb}_{[\sigma_p]}(\coprod_{[p]} D^2\times S^1, \text{int}(D^2)\times S^1)\simeq \mathrm{B}\text{Stab}(\sigma_p).
%\]
%We can assume that  $\sigma_p=(\tau_0,\tau_1,\dots,\tau_p)$ is chosen so that the solid torus $T$ contains all $\tau_i$. Now the isotopy extension theorem implies that we have a fibration
%\[
%\Diff(T\backslash \sigma_p,\partial(T\backslash \sigma_p))\to \Diff(T,\partial T)\to \text{Emb}_{[\sigma_p]}(\coprod_{[p]} D^2\times S^1, \text{int}(D^2)\times S^1).
%\]
% But by Hatcher's theorem \cite[Appendix]{hatcher1983proof}, the group $\Diff(T,\partial T)$ is contractible, therefore we deduce
% \[
% \text{Emb}_{[\sigma_p]}(\coprod_{[p]} D^2\times S^1, \text{int}(D^2)\times S^1)\simeq \BDiff(T\backslash \sigma_p,\partial(T\backslash \sigma_p)).
% \]
% Given that $T\backslash \sigma_p$ is Haken, by \Cref{haken} we know that the identity component of the diffeomorphism group of $T\backslash \sigma_p$ is contractible, therefore we have 
% \[
% \text{Emb}_{[\sigma_p]}(\coprod_{[p]} D^2\times S^1, \text{int}(D^2)\times S^1)\simeq \mathrm{B}\text{Mod}_{\partial}(T\backslash \sigma_p).
% \]
% Now given the weak equivalence \ref{rr} and the fact that the identity component of $\text{Stab}(\sigma_p)$ is contractible, to prove the claim it suffices to show that 
% \[
% \pi_0(\text{Stab}(\sigma_p))\cong \text{Mod}_{\partial}(T\backslash \sigma_p).
% \]
% To do so we use the JSJ torus decomposition for $M\backslash \sigma_p$. Note that  the torus $\phi(S^1\times S^1)$ separates $M\backslash \sigma_p$ into atoroidal (in this case hyperbolic) $M\backslash T$ and a Seifert fiber space $T\backslash \sigma_p$. By the uniqueness of the torus decomposition up to isotopy (\cite[Theorem 1.9]{hatcher2000notes}), we can write the mapping class group $\text{Mod}_{\partial}(M\backslash \sigma_p,)$ as the product of $\text{Mod}_{\partial}(T\backslash \sigma_p)$ and a subgroup $G$ of $\pi_0(\Diff(M\backslash T))$  that is isotopic to the identity on the torus boundary.
%
%It is clear that $\text{Mod}_{\partial}(T\backslash \sigma_p)$ is in the kernel of the map $\text{Mod}_{\partial}(M\backslash \sigma_p)\to \text{Mod}(M)$. Given the isomorphism \ref{rrr}, to prove the claim it is enough to show that the map
%%By \cite[Theorem 3.9.3]{hong2012diffeomorphisms}, the diffeomorphism group of the $S^1$-bundle $\phi(D^2\times S^1)\backslash \sigma_p$ is homotopy equivalent to fiber preserving diffeomorphisms. Therefore this group is homotopy equivalent to the diffeomorphism group of the base of the $S^1$-bundle. Thus $\text{Mod}_{\partial}(\phi(D^2\times S^1)\backslash \sigma_p)\cong \text{PR}\beta_{p+1}$. It is clear that this group is in the kernel of the map $\text{Mod}_{\partial}(M\backslash \sigma_p)\to \text{Mod}(M)$.  Hence to show that the kernel is exactly $\text{PR}\beta_{p+1}$, we have to show that the map
%\[\iota:G\to \text{Mod}(M),\]
%is injective. Let $f\in \Diff(M)$ be a representative of an element in the kernel of $\iota$. Thus, we can assume that $f$ is in $\Diff_0(M)$ and fixes $T$ pointwise. The lemma in \cite[Lemma 6.1]{gabai2001smale} implies that $f|_{M\backslash T}$ is in fact in $\Diff_0(M\backslash \phi(D^2\times S^1),\partial(M\backslash T))$. Therefore $\iota$ is injective which implies that $\pi_0(\text{Stab}(\sigma_p))\cong\text{Mod}_{\partial}(T\backslash \sigma_p)$ .
%$\blacksquare$
%
%\vspace{.3cm}
%
%
%Note that the definition \ref{torus} makes sense not just for hyperbolic manifolds but also for $T$. For $T$ we take $\gamma$ to be its core and we can define the semisimplicial space $T_{\bullet}(T)$ similarly. Thus the space $\text{Emb}(\coprod_{[p]} D^2\times S^1, \text{int}(D^2)\times S^1)$ in the claim is in fact the same as $T_p(T)$.  Given the claim, the spectral sequence \ref{1} can be written as
%\begin{equation}\label{2}
%E^1_{p,q}=H_q(T_p(T))\Rightarrow H_{p+q}(\BDiff_0(M)).
%\end{equation}
%%where $\mathrm{B}\text{PR}\beta_{p+1}$ is the classifying space of the group $\text{PR}\beta_{p+1}$. A model for the classifying space $\mathrm{B}\text{PR}\beta_{p+1}$ is $A_p(D^2)$ (see \Cref{def1} for its definition). Therefore the above spectral sequence can be written as
%%\begin{equation}\label{2}
%%E^1_{p,q}=H_q(A_p(D^2))\Rightarrow H_{p+q}(\BDiff_0(M)).
%%\end{equation}
%But the first page of the above spectral sequence is the same as the spectral sequence associated to the skeleta filtration of $|T_{\bullet}(T)|$. Given that $|T_{\bullet}(T)|$ is weakly contractible (similar to  \Cref{claim2}), the spectral sequence \ref{2} converges to zero in positive degrees. Therefore $\BDiff_0(M)$ is acyclic. 
%\end{proof}
\bibliographystyle{alpha}
\bibliography{reference}
\end{document}
