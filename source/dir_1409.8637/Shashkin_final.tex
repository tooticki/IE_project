
\documentclass[a4paper,12pt]{article}

% \usepackage[T2A]{fontenc}
%%\usepackage[cp1251]{inputenc}  % (\usepackage[cp1251]{inputenc} )
\usepackage{color}
\usepackage[pdftex]{graphicx}
\usepackage{amsmath,amssymb,amsthm}
\usepackage{wrapfig}
\usepackage{ifpdf}


%%%\language 1
%%%%\documentstyle[11pt,Russian,epsfig]{article}

%%%%\documentclass[12pt]{article}
%%%\usepackage{Russian}
%%%\usepackage{graphics}
\textwidth 6.5in

\textheight 8.4in

\hoffset -2.0cm

\voffset -1.5cm

\sloppy

\frenchspacing
\righthyphenmin=2

\newtheorem{theorem}{Theorem}[section]
\newtheorem{cor}{Corollary}[section]
\newtheorem{lemma}{Lemma}[section]
\newtheorem{prop}{Proposition}[section]
\newtheorem{remark}{Remark}

\newcommand{\fix}{\mathop{\rm Fix}\nolimits}
\newcommand{\orb}{\mathop{\rm Orb}\nolimits}
\newcommand{\dist}{\mathop{\rm dist}\nolimits}
\newcommand{\rk}{\mathop{\rm rank}\nolimits}
\newcommand{\gl}{\mathop{\rm GL}\nolimits}
\newcommand{\const}{\mathop{\rm const}\nolimits}
\newcommand{\md}{\mathop{\rm (mod}\nolimits}
\newcommand{\conv}{\mathop{\rm conv}\nolimits}
\newcommand{\cat}{\mathop{\rm cat}\nolimits}
\newcommand{\cov}{\mathop{\rm cov}\nolimits}
\newcommand{\sgn}{\mathop{\rm sign}\nolimits}
\newcommand{\lbl}{\mathop{\rm labels}\nolimits}
\newcommand{\nl}{\mathop{\rm NL}\nolimits}
\newcommand{\ns}{\mathop{\rm SN}\nolimits}
\newcommand{\anl}{\mathop{\rm ANL}\nolimits}

\DeclareMathOperator{\dg2}{deg_2}


\title {Generalizations of Tucker--Fan--Shashkin lemmas}

\author {Oleg R. Musin\thanks{This research is partially supported by the NSF grant DMS-1400876 and the RFBR grant 15-01-99563.}}


\begin{document}

	\ifpdf \DeclareGraphicsExtensions{.pdf, .jpg, .tif, .mps} \else
	\DeclareGraphicsExtensions{.eps, .jpg, .mps} \fi	
	
\date{}
\maketitle



\begin{abstract} Tucker and Ky Fan's lemma are combinatorial analogs of the Borsuk--Ulam theorem (BUT). In 1996, Yu. A. Shashkin proved a version of Fan's lemma, which is a combinatorial analog of the odd mapping theorem (OMT).  We consider generalizations of these lemmas for BUT--manifolds, i.e. for manifolds that satisfy BUT. Proofs rely on a generalization of the OMT and on a lemma about the doubling of manifolds with boundaries that are BUT--manifolds.
\end{abstract}

\medskip

\noindent {\bf Keywords:} Tucker lemma, Ky Fan lemma, Shashkin lemma, Borsuk--Ulam theorem, degree of mapping. 

\section{Tucker's, Fan's and Shashkin's lemmas}

Throughout this paper the symbol ${\mathbb R}^d$ denotes the Euclidean space of dimension $d$.  We denote by  ${\mathbb B}^d$ the $d$-dimensional unit ball and by  ${\mathbb S}^d$ the $d$-dimensional unit sphere. If we consider ${\mathbb S}^d$ as the set of unit vectors $x$ in ${\mathbb R}^{d+1}$, then points $x$ and $-x$ are called {\it antipodal} and the symmetry given by the mapping
 $x \to -x$ is called the {\it antipodality} on  ${\mathbb S}^d$.



\subsection{Tucker and Fan's lemma}



Let $T$ be a triangulation of the $d$-dimensional ball ${\mathbb B}^d$. We call $T$ {\it antipodally symmetric on the boundary}  if the set of simplices of $T$ contained in the boundary of  ${\mathbb B}^d = {\mathbb S}^{d-1}$ is an antipodally symmetric triangulation of  ${\mathbb S}^{d-1}$; that is if $s\subset {\mathbb S}^{d-1}$ is a simplex of $T$, then $-s$ is also a simplex of $T$.

\medskip

\noindent {\bf Tucker's lemma \cite{Tucker}} {\it Let $T$ be a triangulation of  ${\mathbb B}^d$ that is antipodally symmetric on the boundary. Let $$L:V(T)\to \{+1,-1,+2,-2,\ldots, +d,-d\}$$ be a  labelling of the vertices of $T$ that is antipodal (i. e. $L(-v)=-L(v)$)  for every vertex $v$ on the boundary. Then there exists an edge in $T$ that is {\bf complementary}, i.e. its two vertices are labelled by opposite numbers.}

\medskip

\begin{figure}
\begin{center}


  \includegraphics[clip,scale=0.7]{sptuck2}
\end{center}
\caption{Illustration of Tucker's lemma}
\end{figure}


\medskip

There is also a version of Tucker's lemma for spheres:

\medskip

\noindent  {\bf Spherical Tucker's lemma.}  {\it Let $T$ be a centrally symmetric triangulation of the sphere ${\Bbb S}^d$. Let $$L:V(T)\to \{+1,-1,+2,-2,\ldots, +d,-d\}$$ be an antipodal labelling. Then there exists a complementary edge.}

\medskip

Tucker's lemma was extended by Ky Fan \cite{KyFan}:

\medskip

\noindent {\bf Ky Fan's lemma.}  {\it Let $T$ be a centrally symmetric triangulation of the sphere ${\Bbb S}^d$. Suppose that each vertex $v$ of $T$ is assigned a label $L(v)$ from $\{\pm1,\pm2,\ldots,\pm n\}$ in such a way that $L(-v)=-L(v)$. Suppose this labelling does not have complementary edges. Then there are an odd number of $d$-simplices of $T$ whose labels are of the form $\{k_0,-k_1,k_2,\ldots,(-1)^dk_d\}$, where $1\le k_0<k_1<\ldots<k_d\le n$. In particular, $n\ge d+1$.}




\subsection{Shashkin's lemma}
In the 1990's, Yu. A. Shashkin published several works related to discrete versions of classic fixed point theorems  \cite{Shashkin, Shashkin1, Shashkin2, ShashkinT, Shashkin99}. In  \cite{ShashkinT} he proved the following theorem:

\medskip

\noindent {\bf Shashkin's lemma.}  {\it Let $T$ be a triangulation of a planar polygon that is antipodally symmetric on the boundary. Let $$L:V(T)\to \{+1,-1,+2,-2, +3,-3\}$$ be a labelling of the vertices of $T$ that satisfies $L(-v)=-L(v)$ for every vertex $v$ on the boundary. Suppose that this labelling does not have complementary edges. Then for any numbers $a,b,c$, where $|a|=1,\; |b|=2, \; |c|=3$, the total number of triangles in $T$  with labels $(a,b,c)$ and $(-a,-b,-c)$ is odd.}


\medskip

%---------------Fig 2------------------------------------------------------------
\begin{center}
\begin{picture}(320,160)(-140,-80)
% Fig.
\put(-90,-70){Figure 2: Illustration of Shashkin's lemma.}

\multiput(-70,-40)(60,0){4}%
{\line(0,1){120}}

\multiput(-70,-40)(0,40){4}%
{\line(1,0){180}}

\put(-70,40){\line(3,2){60}}
\put(-70,0){\line(3,2){120}}
\put(-70,-40){\line(3,2){180}}

\put(50,-40){\line(3,2){60}}
\put(-10,-40){\line(3,2){120}}



\put(-84,-36){2}
\put(-84,4){1}
\put(-84,44){3}
\put(-84,84){-1}



\put(-22,-36){3}
\put(-22,4){3}
\put(-22,44){1}
\put(-22,84){2}

\put(37,-36){-2}
\put(37,4){-1}
\put(37,44){-2}
\put(37,84){-3}

\put(114,-36){1}
\put(114,4){-3}
\put(114,44){-1}
\put(114,84){-2}



\end{picture}
\end{center}

\medskip

\noindent{\bf Remark.} In other words, Shashkin proved that if $(a,b,c)=(1,2,3), \, (1,-2,3), \, (1,2,-3)$ and $(1,-2,-3)$, then the number of triangles with labels $(a,b,c)$ or $(-a,-b,-c)$ is odd. Denote this number by $\ns(a,b,c)$. Then in Fig. 2 we have
$$ \ns(1,2,3)=3, \; \ns(1,-2,3)=1,\; \ns(1,2,-3)=3, \; \ns(1,-2,-3)=3.$$

Note that this result was published only in Russian and only for two--dimensional case. Moreover, Shashkin attributes this theorem to Ky Fan \cite{KyFan}.

Actually, Shashkin's lemma can be derived from Ky Fan's lemma for $n=d+1$.  However. Shashkin's proof is different and relies on the odd mapping theorem (OMT).  In fact, this lemma is a discrete version of the OMT. That is why we distinguish this result as {\it Shashkin's lemma}.

\medskip

The following is a spherical version of Shashkin's lemma.

\medskip

\noindent {\bf Spherical Shashkin's lemma.}   {\it Let $T$ be a centrally symmetric  triangulation of \, ${\Bbb S}^d$. Let
$$
L:V(T)\to \Pi_{d+1}:=\{+1,-1,+2,-2,\ldots, +(d+1),-(d+1)\}
$$
be an antipodal  labelling of $T$.  Suppose that this labelling does not have complementary edges. Then for any set of  labels $\Lambda:=\{\ell_1,\ell_2,\ldots,\ell_{d+1}\}\subset\Pi_{d+1}$ with $|\ell_i|=i$ for all $i$, the number of $d$--simplices in $T$ that are labelled by $\Lambda$  is odd. }

\subsection{Main results}

In \cite{Mus} we invented BUT (Borsuk--Ulam Type) -- manifolds.
Theorems 3.1--3.4 in this paper extend Tucker's and Shashkin's lemmas for BUT--manifolds. Namely, Theorem 3.1 and Theorem 3.2 are extensions of the spherical Tucker and Shahskin lemmas,  where ${\Bbb S}^d$ is substituted by a BUT--manifold.  Theorems 3.3 and 3.4 are extensions  of the original Tucker and Shashkin lemmas, where   ${\Bbb B}^d$ is substituted by a manifold $M$ with boundary $\partial M$ that is a BUT--manifold.

 Our proof of Theorem 3.2  is relies on a generalization of the odd mapping theorem for BUT--manifolds:

\medskip

\noindent {\bf Theorem 2.1.} {\it Let $(M_1,A_1)$ and $(M_2,A_2)$ be  BUT--manifolds. Then any odd  continuous mapping $f:M_1\to M_2$ has odd degree.}

\medskip


Theorems 3.3 and 3.4 follow from Theorems 3.1 and 3.2 by using Lemma 3.1, which is about the doubling of manifolds with boundaries that are BUT--manifolds.

In Section 4 we extend  for BUT--manifolds Shaskin's proof of two Tucker's theorems about covering families from \cite{Tucker}. Actually, these theorems are corollaries of Theorem 3.2. 









\section{The odd mapping theorem}

We say that a mapping $f:{\Bbb S}^d\to {\Bbb S}^d$ is {\it odd} or {\em antipodal} if $f(-x)=-f(x)$ for all $x\in {\Bbb S}^d$. If $f$ is a continuous mapping, then  $\deg{f}$ (the degree of $f$) is well defined.

Let $f:M_1\to M_2$ be a continuous map between two closed manifolds $M_1$ and $M_2$ of the same dimension. The degree  is a number that represents the amount of times that the domain manifold wraps around the range manifold under the mapping. Then $\dg2(f)$ (the degree modulo 2) is 1 if this number is odd and 0 otherwise. It is well known that $\dg2(f)$ of a continuous map $f$  is a homotopy invariant (see \cite{Milnor}).


The classical {\bf odd mapping theorem} states that

\medskip

\noindent {\it Every continuous odd mapping $f:{\Bbb S}^d\to {\Bbb S}^d$  has odd degree.}

\medskip

Shashkin \cite{ShashkinT} (see also \cite[Proposition 2.4.1]{Mat}) gives a proof of this theorem for simplicial mappings $f:{\Bbb S}^d\to {\Bbb S}^d$. Conner and Floyd \cite{CF60} considered Theorem 2.1 for a wide class of spaces. Here we extend the odd mapping theorem for BUT--manifolds. In our paper \cite{Mus}, we extended the Borsuk--Ulam theorem for manifolds.\\
Let  $M$ be a connected compact PL (piece-wise linear) $d$-dimensional manifold without boundary with a free simplicial involution $A:M\to M$, i. e. $A^2(x)=A(A(x))=x$ and $A(x)\ne x$.
We say that a pair $(M,A)$ is a {\it BUT (Borsuk-Ulam Type) manifold} if for any continuous  $g:M \to {\Bbb R}^d$ there is a point $x\in M$ such that $g(A(x))=g(x)$. Equivalently, if a continuous  map $f:M \to {\Bbb R}^d$  is { antipodal}, i.e. $f(A(x))=-f(x)$,  then the set of zeros $Z_f:=f^{-1}(0)$ is not empty.

In \cite{Mus}, we found several equivalent necessary and sufficient conditions for manifolds to be BUT. In particular,

\medskip

\noindent {\it $M$ is a $d$--dimensional BUT--manifold if and only if $M$ admits an antipodal continuous transversal to zeros mapping $h:M \to {\Bbb R}^d$ with $|Z_h|=2\pmod{4}$.}


\medskip

\noindent A continuous mapping $h:M \to {\Bbb R}^d$ is called {\it transversal to zero} if there is an open set $U$ in ${\Bbb R}^d$ such that $U$ contains $0$, $U$ is homeomorphic to the open $d$-ball and $h^{-1}(U)$ consists of a finite number open sets in $M$ that are homeomorphic to open $d$-balls.

The class of BUT--manifolds is sufficiently large. It is clear that $({\Bbb S}^d,A)$ with $A(x)=-x$ is a BUT-manifold.   Suppose that $M$ can be represented as a connected sum $N\# N$, where $N$ is a closed PL manifold. Then $M$ admits a free involution. Indeed, $M$ can be  ``centrally symmetrically" embedded to ${\Bbb R}^k$, for some $k$, and the antipodal symmetry $x\to -x$ in ${\Bbb R}^k$ implies a free involution $A:M\to M$ \cite[Corollary 1]{Mus}. For instance,   orientable  two-dimensional  manifolds $M^2_g$ with even genus $g$ and non-orientable manifolds $P^2_m$ with even $m$, where $m$ is the number of  M\"obius bands,  are BUT-manifolds.

Let $M_i,\; i=1,2$, be a manifold with a free involution $A_i$. We say that a mapping $f:M_1\to M_2$ is {\em antipodal} (or {\em odd}, or {\em equivariant}) if $f(A_1(x))=A_2(f(x))$ for all $x\in M_1$.

\begin{theorem} Let $(M_1,A_1)$ and $(M_2,A_2)$ be $d$-dimensional BUT--manifolds. Then any odd  continuous mapping $f:M_1\to M_2$ has odd degree.
\begin{proof} Since $(M_2,A_2)$ is BUT, there is a continuous antipodal transversal to zeros mapping $g:M_2 \to {\Bbb R}^d$ with $|Z_g|=4m_2+2$ \cite[Theorem 2]{Mus}.
	
Let $h:=g\circ f$. Then $h:M_1 \to {\Bbb R}^d$ is continuous and antipodal. 	Since the degree of a mapping is a homotopy invariant, without loss of generality we may assume that $h$ is a transversal to zero mapping (see  \cite[Lemma 3]{Mus}). Therefore $|Z_h|=4m_1+2$.
On the other hand,
$$
|Z_h|=\sum\limits_{x\in Z_g}{|f^{-1}(x)|}.
$$
Then
$$
2m_1+1=(2m_2+1)\dg2{f}\pmod{2}.
$$
 Thus, the degree of ${f}$ is odd.
\end{proof}

\end{theorem}

\section{Tucker's and Shashkin's lemmas for BUT--manifolds}


In our papers \cite{Mus,MusSpT,MusVo} are considered extensions of Tucker's lemma. Here we consider generalizations of Tucker's and Shashkin's lemmas for manifolds with and without boundaries.


Let $T$ be an {\it antipodally symmetric} (or {\it antipodal}) triangulation of a BUT--manifold $(M,A)$. This means that $A:T\to T$ sends simplices to simplices. Denote by $\Pi_n$ the set of labels $\{+1,-1,+2,-2,\ldots, +n,-n\}$   and let $L:V(T)\to \Pi_n$ be a labeling of $T$. We say that this labelling is {\it antipodal} if $L(A(v))=-L(v)$. An edge $uv$ in $T$ is called {\it complementary} if $L(u)=-L(v)$.

\begin{theorem}{\bf (\cite[Theorem 4.1]{MusSpT})} \label{TBUT} Let $(M,A)$ be a $d$-dimensional BUT--manifold. Let $T$ be an antipodal triangulation of   $M$.   Then for any antipodal  labelling $L:V(T)\to \Pi_d$  there exists a complementary edge.
\end{theorem}

Any antipodal labelling $L:V(T)\to \Pi_n$ of an antipodally symmetric triangulation $T$ of $M$  defines a simplicial map $f_L:T\to {\Bbb R}^n$. Let $\{e_1,-e_1,e_2,-e_2,\ldots,e_n,-e_n\}$ be the standard orthonormal basis in ${\Bbb R}^n$.
For $v\in V(T)$, set $f_L(v):=e_i$ if $L(v)=i$ and  $f_L(v):=-e_i$ if $L(v)=-i$. Since $f_L$ is defined on $V(T)$,  it defines a simplicial  mapping $f_L:T\to  {\Bbb R}^n$ (See details in \cite[Sec. 2.3]{Mat}.)

The following theorem is a version of Shashkin's lemma for manifolds without boundary.

\begin{theorem} \label{SBUT} Let $(M,A)$ be a $d$-dimensional BUT--manifold. Let $T$ be an antipodally symmetric triangulation of   $M$.   Let $L:V(T)\to \Pi_{d+1}$  be an antipodal  labelling of $T$.  Suppose that this labelling does not have complementary edges. Then for any set of  labels $\Lambda:=\{\ell_1,\ell_2,\ldots,\ell_{d+1}\}\subset\Pi_{d+1}$ with $|\ell_i|=i$ for all $i$, the number of $d$--simplices in $T$ that are labelled by $\Lambda$  is odd.
\end{theorem}
\begin{proof} Since $L$ has no complimentary edges, $f_L:T\to  {\Bbb R}^{d+1}$ is an antipodal mapping of $M$ to the boundary of the crosspolytope $C^{d+1}$ that is the  convex hull
$
\conv{\{e_1,-e_1,\ldots,e_{d+1},-e_{d+1}\}}.
$
Note that $\partial C^{d+1}$ is a simplicial sphere ${\Bbb S}^d$, which is a BUT-manifold. Therefore, 
Theorem 2.2 implies that the number of preimages of the simplex in  $\partial C^{d+1}$ with indexes from $\Lambda$ is odd. It completes the proof.
\end{proof}


\noindent{\bf Remark.} Theorem 3.1 can be proved using the same arguments. Indeed, suppose that $L:V(T)\to \Pi_d$  has no complementary edges. Then $f_L$ sends $M$ to $\partial C^d$. Since $\dim{\partial C^d}=d-1$, $\deg{f_L}=0$. This contradicts Theorem 2.1.

\medskip

Now we extend Tucker's and Shashkin's lemmas for the case when $M$ is a manifold with boundary that is a BUT--manifold.  But first, prove that there exists a ``double'' of $M$ that is a BUT-manifold.




\begin{lemma} \label{Lemma1}  Let $M$ be a compact PL manifold with boundary $\partial M$. Suppose  $(\partial M,A)$ is a BUT--manifold. Then there is a BUT--manifold $(\tilde M,\tilde A)$ and a submanifold $N$ in $\tilde M$  such that  $N\simeq M$, $\tilde A|_{\partial N}\simeq A$, $ (N\setminus\partial N) \cap \tilde A(N\setminus\partial N)=\emptyset$ and
$$\tilde M\simeq (N\setminus\partial N)\cup \partial N \cup \tilde A(N\setminus\partial N).$$
\end{lemma}

\begin{proof} {\bf 1.} First we prove the following statement:\\
{\it
 Let $X$ be a finite simplicial complex. Let $Y$ be a subcomplex of $X$ with a free involution $A:Y\to Y$. Then there is a simplicial embedding $F$ of $X$ into 
${\Bbb R}^q_+:=\{(x_1,\ldots,x_q)\in{\Bbb R}^q: x_1\ge 0\}$, where $q$ is sufficiently large,  such that $Y$ is centrally symmetrically embedded in ${\Bbb R}^q$, i.e. $F(A(y))=-F(y)$ for all $y\in Y$, and $X\setminus Y$ is mapped into the interior of  ${\Bbb R}^q_+$.}
	
Indeed, let $v_1,v_{-1},\ldots,v_m,v_{-m}$ denote vertices of $Y$ such that $A(v_k)=v_{-k}$. Let $\{v_{m+1},\ldots, v_n\}$ be the set of vertices of $X\setminus Y$.

Denote by $C^n$ the $n$--dimensional crosspolytope that is the boundary of convex hull
$$
\conv{\{e_1,-e_1,\ldots,e_n,-e_n\}}
$$
of the vectors of the standard orthonormal basis and their negatives.

Now define an embedding $F:X\to C^n$. Let $F(v_k):=e_k$, $F(v_{-k}):=e_{-k}$, where $1\le k\le m$, $F(v_k):=e_k$, and $k=m+1,\ldots,n$.
Since $F$ is defined for all of the vertices of $X$,  it uniquely defines a simplicial (piecewise linear) mapping $F:X\to C^n\subset {\Bbb R}^n$.
Then
$$F(Y)\subset C^m\subset{\Bbb R}^m=\{(x_1,\ldots,x_n)\in {\Bbb R}^n: x_i=0, \; i=m+1,\ldots,n \},$$
$F(A(y))=-F(y)$ for all $y\in Y$  and
$$
F(X\setminus Y)\subset {\Bbb R}^{n-1}_+:=\{(x_1,\ldots,x_n)\in {\Bbb R}^n: x_{m+1}+\ldots+x_n>0\},
$$
as required.

\medskip
	
\noindent {\bf 2.}	Let $X=M$ and $Y=\partial M$. Then it follows from {\bf 1} that there is an embedding  $F:M\to{\Bbb R}^q_+$ with $F(\partial M)\subset {\Bbb R}^q$ and $F(A(y))=-F(y)$ for all $y\in \partial M$, where $q=n-1$.
Let
$$
\tilde M:=F(M)\cup(-F(M))\subset{\Bbb R}^{q+1}={\Bbb R}^q_+\cup (-{\Bbb R}^q_+) \; \mbox{ and } \; \tilde A(x):=-x \mbox{ for all } x\in\tilde M.
$$
	 It is clear that $\tilde M\simeq (N\setminus\partial N)\cup \partial N \cup \tilde A(N\setminus\partial N),$ where $N:=F(M).$
	
\medskip

\noindent {\bf 3.}	Let us prove that $(\tilde M,\tilde A)$ is BUT. Indeed, since $(\partial M,A)$ is BUT, there is a continuous antipodal transversal to zeros mapping $g:\partial M\simeq\partial N \to {\Bbb R}^{d-1}$ with $|Z_g|=4m+2$, where $d:=\dim{M}.$ We extend this mapping to  $h:\tilde M \to {\Bbb R}^{d}$  with $h|_{\partial N}=g$ and $|Z_h|=|Z_g|=4m+2$.


Let $v=(x_1,\ldots,x_{n})\in{\Bbb R}^{n}$ be a vertex of $\tilde M$. If $v\in\partial N$, then put
$$
h(v):=(g(v),0)\in {\Bbb R}^{d}.
$$
For $v\in\tilde M\setminus\partial N$ define
$$
h(v):=(0,\ldots,0,x_{m+1}+\ldots+x_n)\in{\Bbb R}^{d}.
$$
Then $h:\tilde M\to{\Bbb R}^{d}$ is an antipodal transversal to zeros mapping and $h^{-1}(0)=g^{-1}(0)$.
\end{proof}

\begin{theorem}
	Let $M$ be a $d$--dimensional compact PL manifold with boundary $\partial M$. Suppose  $(\partial M,A)$ is a BUT--manifold.
Let $T$ be a triangulation of  $M$ that antipodally symmetric on $\partial M$. Let $L:V(T)\to\Pi_d$ be a   labelling of  $T$ that is antipodal on the boundary. Then there is a complementary edge in $T$.
\end{theorem}	

\begin{theorem}
	Let $M$ be a $d$--dimensional  compact PL manifold with boundary $\partial M$. Suppose  $(\partial M,A)$ is a BUT--manifold.
Let $T$ be a triangulation of  $M$ that antipodally symmetric on $\partial M$.  Let $L:V(T)\to\Pi_{d+1}$ be a   labelling of  $T$ that is antipodal on the boundary and has no complementary edges. Then for any set of  labels $\Lambda:=\{\ell_1,\ell_2,\ldots,\ell_{d+1}\}\subset\Pi_{d+1}$ with $|\ell_i|=i$ for all $i$, the number of $d$--simplices in $T$ that are labelled by $\Lambda$ or $(-\Lambda)$  is odd.
\end{theorem}
\begin{proof} By Lemma \ref{Lemma1} there is a BUT--manifold $(\tilde M,\tilde A)$ that is the double of $M$.  We can extend  $T$ and $L$ from $M$ to an antipodal triangulation $\tilde T:=T\cup\tilde A(T)$ of $\tilde M$ and an antipodal labelling $\tilde L:V(\tilde T)\to \Pi_n$, where $n=d$ in Theorem 3.3 and $n=d+1$ in Theorem 3.4, such that $\tilde L|_T=L$.
	
Thus, for the case $n=d$ Theorem 3.3 follows from Theorem 3.1 and for $n=d+1$ Theorem 3.2 yields Theorem 3.4.
\end{proof}

\section{Shashkin's proof of Tucker's theorems}

In this section we consider two Tucker's theorems about covering families. 
Note that Tucker \cite{Tucker} obtained these theorem only for \, ${\Bbb S}^2$. Bacon \cite{Bacon} proved that statements in Theorems \ref{t41} and \ref{t42} are equivalent to the Borsuk--Ulam theorem for normal topological spaces $X$ with free continuous involutions $A:X\to X$.  (See also Theorem 2.1 in our paper \cite{MusVo}.)  Actually, these theorems  can be proved from properties of Schwarz's genus \cite{Sv} or Yang's cohomological index  \cite{Kar,MusVo}. 


For the two--dimensional case  in the book \cite{Shashkin99}  Shashkin derives Tucker's theorems from his lemma. Here we extend his proof for BUT--manifolds of all dimensions.


\begin{theorem} \label{t41} Let $(M,A)$ be a $d$-dimensional BUT--manifold. Consider a family of closed sets $\{B_i, B_{-i}\},\, i=1,\ldots, d+1$, where $B_{-i}:=A(B_i)$, is such that $B_i\cap B_{-i}=\emptyset$ for all $i$.  If this family  covers $M$, then for any set of  indices $\{k_1,k_2,\ldots,k_{d+1}\}\subset\Pi_{d+1}$ with $|k_i|=i$ for all $i$, the intersection of all $B_{k_i}$  is not empty.
\end{theorem}
\begin{proof} Note that any PL manifold admits a metric. For a triangulation $T$ of $M$, the norm of $T$, denoted by $|T|$, is the diameter of the largest simplex in $T$.

Let $T_1,T_2,\ldots$ be a sequence of antipodal triangulations of $M$ such that $|T_i|\to0$.
Now for all $i$ define an antipodal labelling $L_i:V(T_i)\to \Pi_{d+1}$. For every  $v\in V(T_i)\subset M$ set
$$
L_i(v):=\ell, \mbox{ where } v\in B_\ell \mbox{ and } |\ell|=\min{\{|k|: v\in B_k\}}.
$$
Then $L_i$ satisfies the assumptions in Theorem \ref{SBUT} and $T_i$ contains a simplex $s_i$ with labels $\{k_1,k_2,\ldots,k_{d+1}\}\subset\Pi_{d+1}$.

 Since $M$ is compact and $|s_i|\to0$, the sequence $\{s_i\}$ contains a converging subsequence $P$  with limit $w\in M$. Then for $s_i\in P$ we have $V(s_i)\to w$.

By assumption, all $B_k$ are closed sets.  Therefore $w\in B_{k_j}$ for all $j=1,\ldots, d+1$, and thus $w\in \cap_j{B_{k_j}}$.
\end{proof}

\begin{theorem} \label{t42} Let $(M,A)$ be a $d$-dimensional BUT--manifold. Suppose that $M$ is covered by a family $\mathcal F$ of $d+2$ closed subsets $C_1,\ldots,C_{d+2}$. Suppose that all $C_i$ have no antipodal pairs $(x,A(x))$, in other words, $C_i\cap A(C_i)=\emptyset$. Let $0<k<d+2$. Then  any $k$ subsets from $\mathcal F$ intersect and there is a point $x$ in this intersection such that $A(x)$ belongs to the intersection of the remaining $(d+2-k)$ subsets in $\mathcal F$.
\end{theorem}
\begin{proof} Without loss of generality, we can assume that $k\ge(d+2)/2$ and that the $k$ subsets from $\mathcal F$ are $C_1,\ldots,C_k$. Therefore, we have to prove that there is $x\in M$ such that
	$$
	x\in \bigcap\limits_{i=1}^k{C_i} \; \mbox{ and } \; A(x)\in \bigcap\limits_{i=k+1}^{d+2}{C_i}
	$$
	
Set $C_{-i}:=A(C_i)$. Let $m:=\lceil{d/2}\rceil,$
$$
B_1:=C_1\cap(C_{-2}\cup\ldots\cup C_{-(m+1)}\cup C_{-(d+2)}),
$$
$$
B_2:=C_2\cap(C_{-3}\cup\ldots\cup C_{-(m+2)}\cup C_{-(d+2)}),
$$
$$
\vdots
$$
$$
B_d:=C_d\cap(C_{-(d+1)}\cup C_{-1}\cup\ldots\cup C_{-(m-1)}\cup  C_{-(d+2)}),
$$
$$
B_{d+1}:=C_{d+1}\cap(C_{-1}\cup\ldots\cup C_{-m}\cup C_{-(d+2)}).
$$

If $B_{-i}:=A(B_i)$, then
$$
\bigcup\limits_{i=1}^{d+1}{B_i\cup B_{-i}}=\bigcup\limits_{i=1}^{d+2}{C_i\cap(C_1\cup\ldots\cup C_{d+2})}=\bigcup\limits_{i=1}^{d+2}{C_i\cap M}=\bigcup\limits_{i=1}^{d+2}{C_i}=M.
$$
On the other hand, $B_i\subset C_i$ and $B_{-i}\subset C_{-i}$, hence $B_i\cap B_{-i}=\emptyset$. Therefore, the family of subsets $\{B_i\}$ satisfies the assumptions of Theorem \ref{t41}. It follows that
$$
Q:=B_1\cap\ldots\cap B_k\cap B_{-(k+1)}\cap\ldots\cap B_{-(d+1)}\ne\emptyset.
$$
Let $x\in Q$. Then $$x\in C_1\cap\ldots\cap C_k \, \mbox{ and } \, A(x)\in C_{-(k+1)}\cap\ldots\cap C_{-(d+1)}.$$
Since $k\ge m+1$ and $x\in B_1=C_1\cap(C_{-2}\cup\ldots\cup C_{-(m+1)}\cup C_{-(d+2)})$, we have $x\in C_{-(d+2)}$, i.e. $A(x)\in C_{d+2}$.
\end{proof}

\begin{cor} Let $(M,A)$ be a $d$-dimensional BUT--manifold. Then $M$ cannot
be covered by $d+1$ closed sets, none containing a pair $(x,A(x))$ of antipodal points.
\end{cor}

Note that the case $M={\Bbb S}^d$ was first considered by Lusternik and Schnirelmann in 1930.

\begin{proof} Suppose the converse, so $M$ can be covered by closed subsets $C_1,\ldots,C_{d+1}$.
	
	Let $C_{d+2}:=C_1$. Then this covering satisfies the assumptions of Theorem \ref{t42}. So there is $x$ such that
	$$
	x\in \bigcap\limits_{i=1}^{d+1}{C_i} \; \mbox{ and } \; A(x)\in C_{d+2}, \; \mbox{ i.e. } \; (x,A(x)) \in C_1,
	$$
a contradiction. 	
\end{proof}

\medskip


\medskip

\noindent{\bf Acknowledgment.} I  wish to thank Fr\'ed\'eric Meunier  for helpful discussions and  comments.


\begin{thebibliography}{99}
	

\bibitem{Bacon}
P. Bacon, Equivalent formulations of the Borsuk-Ulam theorem, {\it Canad. J. Math.,}
{\bf 18} (1966), 492--502.

\bibitem{CF60}
P. E. Conner and E. E. Floyd, Fixed points free involutions and equivariant maps, {\it Bull. Amer. Mat. Soc.,} {\bf 60} (1960), 416-441.


\bibitem{KyFan}
K. Fan, A generalization of Tucker's combinatorial lemma with topological applications. {\it Ann. of Math.,} {\bf 56} (1952), 431-437.



\bibitem{Kar}
R. N. Karasev, Theorems of Borsuk-Ulam type for flats and common transversals of families of convex compact sets, {\it Sb. Math.,} {\bf 200} (2009), 1453--1471.

\bibitem{Mat}
J. Matou\v{s}ek, Using the Borsuk-Ulam theorem, Springer-Verlag, Berlin, 2003.


\bibitem{Milnor}
J. W. Milnor, Topology from the differentiable viewpoint, The University Press of Virginia, Charlottesville, Virginia, 1969.

\bibitem{Mus}
O. R. Musin, Borsuk-Ulam type theorems for manifolds,  {\it Proc. Amer. Math. Soc.}  {\bf 140} (2012), 2551-2560.

\bibitem{MusSpT}
O. R. Musin, Extensions of Sperner and Tucker's lemma for manifolds, {\it J. of Combin. Theory Ser. A,} {\bf 132} (2015), 172--187.




\bibitem{MusVo}
O. R. Musin and A.\,Yu. Volovikov, Borsuk--Ulam Type spaces, {\it Mosc. Math. J.,}  {\bf 15:4} (2015), 749--766.


\bibitem{Shashkin}
Yu. A. Shashkin, Fixed Points, American Mathematical Society, Providence, RI, 1991.


\bibitem{Shashkin1}
Yu. A. Shashkin, Local degrees of simplicial mappings, {\it Publ. Math. Debrecen}, {\bf 45} (1994), 407--413.


\bibitem{Shashkin2}
Yu. A. Shashkin, Remark on local degrees of simplicial mappings, {\it Publ. Math. Debrecen}, {\bf 49} (1996), 301--304.

\bibitem{ShashkinT}
Yu. A. Shashkin, Variants of Tucker's combinatorial lemma, {\it Proc. IMM UrB of RAS}, {\bf 4} (1996),  127--132 (in Russian).

\bibitem{Shashkin99}
Yu. A. Shashkin, Combinatorial lemmas and simplicial mappings, Ural State University press, Yekaterinburg, 1999 (in Russian).


\bibitem{Sv}
A. S. \v Svarc, The genus of a fiber space, {\it Trudy Moskov. Mat. Obsc.,} {\bf 10} (1961), 217-272 and {\bf 11} (1962), 99-126; (Russian), English translation in {\it Amer. Math. Soc. Translat.,} II. Ser., {\bf 55} (1966), 49--140.


\bibitem{Tucker}
A. W. Tucker,  Some topological properties of the disk and sphere. In: Proc. of the First Canadian Math. Congress, Montreal, 285-309, 1945.




 \end{thebibliography}

 \medskip

\noindent O. R. Musin\\ 
 University of Texas Rio Grande Valley, School of Mathematical and Statistical Sciences, One West University Boulevard, Brownsville, TX, 78520 \\
{\it E-mail address:} oleg.musin@utrgv.edu


\end{document}
