% 0-heat.tex 15 Oct 2014 01:00
%
%  "Classical Morse theory revisited -- I
%         Backward $\lambda$-Lemma and
%         homotopy type"
%
%  By Joa Weber
%
\documentclass{article}
%\documentclass{book}
%
%%%%%%%%%%%%%%%%%%%%%%%%%%%%%%
%%%%%%%%%  Packages %%%%%%%%%%%%%
%%%%%%%%%%%%%%%%%%%%%%%%%%%%%%
%
% Select what to do with todonotes:
%\usepackage[disable]{todonotes}
                                     % notes not showed
\usepackage[draft]{todonotes}   
                                     % notes showed
%
%use Times fonts if available on your TeX system
%\usepackage{mathptmx}      
%
% automatically loaded by ams document classes:
\usepackage{amsthm,amsmath}
%
% amssymb automatically loads amsfonts
\usepackage{amscd,amssymb,latexsym,graphicx}
%
% commutative diagrams: amscd xy
%\usepackage{amscd}
\usepackage[all]{xy}
%
\usepackage{cite}
\usepackage{mathtools}
%\usepackage{euscript,eufrak}
% 
% shows labels (great for complex documents):
%\usepackage{showkeys}
%\usepackage{pdfpages}
%
% required for package hyperref:
\usepackage{hycolor}
\usepackage{xcolor}
% always the last package (except for package geometry):
\usepackage[
                       colorlinks=true,
                       linkcolor=black, %red,
                       citecolor=black, %green,
                       urlcolor=blue,
%
%                     linkbordercolor=red,   %red {1 0 0}
%                     citebordercolor=green,   %green {0 1 0}
%                     urlbordercolor =cyan,    %cyan {0 1 1}
                     ]{hyperref}
% \href{http://www.namsu.de}{\LaTeX{} Kurs 2009}
% \href{mailto:test@example.net}{Text des Links}
%
%
% auf Joa's Macbook Air:
% xfig -specialtext -latexfonts -startlatexFont default
% fig2pdf fig2pdf-times
% (fig2eps fig2eps-times)
%
% \includegraphics%[width=0.75\textwidth]
%                            {fig-bead-potential}
%                             no .pdf or .eps ending       !!!
%  this way latex (looking for .eps) 
%  and also pdflatex (looking for .pdf) works   !!!
%
%%%%%%%%%%%%%%%%%%%%%%%%%%%%%%
%%%%%%%%%%% XFIG %%%%%%%%%%%%%%%
%%%%%%%%%%%%%%%%%%%%%%%%%%%%%%
%\usepackage{color}
%
% start xfig using the command
% xfig -specialtext -latexfonts -startlatexFont default xfig-test.fig
%
% use textsize *10point*
%

%% Method I
% 1) draw figure using xfig
% 2) save as 'file.fig'
% 3) export as combined PS/Latex (both parts)
% 4) run 'fig2eps --add=amsmath,amssymb,amstext --nogv $1 file.fig'
%    resulting in 'file.eps'
% 5)\includegraphics{file.eps}
%
%% Method Ii (xdvi does NOT show infigure formulae)
% 1) draw figure using xfig
% 2) save as 'file.fig'
% 3) export as combined PS/Latex (both parts)
% 4) \input{file.pstex_t}
%
%%%%%%%%%%%%%%%%%%%%%%%%
%%%% CESAR %%%%%%%%%%%%%%%
%%%%%%%%%%%%%%%%%%%%%%%%
%\usepackage[brazil]{babel} 
    %Babel (Brazil): para traduzir os nomes para o português
%\usepackage[utf8]{inputenc} 
    % Input Encodification: para reconhecer 
    % o texto de entrada corretamente
%\usepackage[top=3cm, bottom=2cm, left=3cm, right=2cm]{geometry}
    %Geometry: para arrumar as margens (?)
%\usepackage[pdftex]{graphicx, color} 
    %Graphicx e Color: para adicionar imagens e mudar cores.
%\usepackage{indentfirst}
    %Identar o primeiro parágrafo.
%\usepackage{enumerate}
    %Mais opções de listas.
%\usepackage{caption}
%\usepackage{picture}
%\usepackage{graphpap}
%\usepackage{tikz}
%\usepackage{cite}
%\usetikzlibrary{positioning,shapes,fit,arrows}
%\definecolor{myblue}{RGB}{56,94,141}
%\linespread{1.3}
% espaçamento 1,5 linha

%%%%%%%%%%%%%%%%%%%%%%%%%%%%

% Definições de Teoremas, Definições, Provas, etc
%\newtheorem{teo}{Teorema}[section]
%\newtheorem{lema}[teo]{Lema}
%\newtheorem{coro}[teo]{Corolário}
%\newtheorem{prop}[teo]{Proposição}
%\renewcommand{\qedsymbol}{\textsc{q.e.d.}}
%\renewcommand{\mod}{\mbox{mod }}
%\renewcommand{\sin}{\mbox{ sen}}
%\theoremstyle{definition}
%\newtheorem{defi}{Definição}
%\newtheorem{exem}{Exemplo}	
%\newtheorem*{exer}{Exercícios}	
%\newtheorem*{obs}{Observações}
%
% Novos comandos
%\newcommand{\HRule}{\rule{\linewidth}{0.5mm}}

%%%%%%%%%%%%%%%%%%%%%%%%
%%%% END CESAR %%%%%%%%%%%
%%%%%%%%%%%%%%%%%%%%%%%%



%%%%%%%%%%%%%%%%%%%%%%%%%%%%%%%%%%
%%%%%%%%%%%%%%%%  header %%%%%%%%%%%%
%%%%%%%%%%%%%%%%%%%%%%%%%%%%%%%%%%
\title{Classical Morse theory revisited -- I \\
         Backward $\lambda$-Lemma and
         homotopy type
        }
\author{Joa Weber\footnote{
        Financial support:
        FAPESP grant 2013/20912-4,
        FAEPEX grant 1135/2013, and 
        CNPq, Conselho Nacional de Desenvolvimento Cient\'{\i}fico
        e Tecnol\'ogico - Brasil.
%\newline
  Instituto de Matem\'{a}tica, Estat\'{\i}stica
  e Computa\c{c}\~{a}o Scient\'{\i}fica,
  Universidade Estadual de Campinas,
  Rua S\'{e}rgio Buarque de Holanda~651,
  SP~13083-859 ,
  Campinas, Brasil.
% MSC 37Dxx 58E05
%
        \hfill
        joa@ime.unicamp.br
        }
        \\
        IMECC UNICAMP }
%        joa@math.sunysb.edu
%               Instituto de Matem\'{a}tica, Estat\'{\i}stica
%               e Computa\c{c}\~{a}o Scient\'{\i}fica 
%               Universidade Estadual de Campinas 
%               Rua S\'{e}rgio Buarque de Holanda~651,
%               Cidade Universit\'{a}ria "Zeferino Vaz",
%               CEP~13083-859, Campinas-SP, Brasil.
           %     \\
%                 joa@math.sunysb.edu.
%              \\
%              Tel.: +55-19-3521xxxx\\
%              Fax: +55-19-3521xxxx\\
%            }
\date{\today}
\sloppy
%
%%%%%%%%%%%%%%%% Environments %%%%%%%%%%
%joa
\newtheorem{theoremABC}{Theorem}
\renewcommand{\thetheoremABC}{\Alph{theoremABC}}
\newtheorem{theoremA}{Theorem}
\renewcommand{\thetheoremA}{{}}

%%

\newtheorem{theorem}{Theorem}[section]
\newtheorem{corollary}[theorem]{Corollary}
\newtheorem{cor}[theorem]{Corollary}
\newtheorem{prop}[theorem]{Proposition}
\newtheorem{lemma}[theorem]{Lemma}
\newtheorem{proposition}[theorem]{Proposition}
\newtheorem{conjecture}[theorem]{Conjecture}
\newtheorem{question}[theorem]{Question}
%\newtheorem{definition}[theorem]{Definition}
%\newtheorem{remark}[theorem]{Remark}
%\newtheorem{exercise}[theorem]{Exercise}
%\newtheorem{example}[theorem]{Example}
%   
\theoremstyle{definition}
%\newtheorem{definition}[Definition]{section}
\newtheorem{definition}[theorem]{Definition}
\newtheorem{hypothesis}[theorem]{Hypothesis}
\newtheorem{remark}[theorem]{Remark}
\newtheorem{exercise}[theorem]{Exercise}
\newtheorem{example}[theorem]{Example}
%
\theoremstyle{remark}
%\newtheorem{remark}{Remark}[section]
%\newtheorem{exercise}[remark]{Exercise}
%\newtheorem{example}[remark]{Example}
\newtheorem*{notation}{Notation}
%
% \numberwithin{equation}{section}
%
%%%%%%% macros

%\input amssym.def
%\input amssym.tex
%\input mssymb12.tex
%
%
\newcommand{\NI}{{\noindent}}
\newcommand{\MS}{{\medskip}}
\newcommand{\BS}{{\bigskip}}
\newcommand{\pbar}{{\overline\partial}}
%
%%%%not accepted by package xy-pic:
%\newcommand{\1}{{{\mathchoice {\rm 1\mskip-4mu l} {\rm 1\mskip-4mu l}
%{\rm 1\mskip-4.5mu l} {\rm 1\mskip-5mu l}}}}
%%%% use for xy-pic:
\renewcommand{\1}{{{\mathchoice {\rm 1\mskip-4mu l} {\rm 1\mskip-4mu l}
{\rm 1\mskip-4.5mu l} {\rm 1\mskip-5mu l}}}}
\renewcommand{\graph}{{\rm graph }}  % graph
%%%%%%%%%%%%%%%%%%%%%%%%%%%%%%%%%%
%
\newcommand{\dslash}{/\mskip-6mu/}
%
%
\newcommand{\A}{{\mathbb{A}}}
\newcommand{\B}{{\mathbb{B}}}
\newcommand{\C}{{\mathbb{C}}}
\newcommand{\D}{{\mathbb{D}}}
%\newcommand{\E}{{\mathbb{E}}}
\newcommand{\F}{{\mathbb{F}}}
\newcommand{\HH}{{\mathbb{H}}}
\newcommand{\LL}{{\mathbb{L}}}
\newcommand{\M}{{\mathbb{M}}}
\newcommand{\N}{{\mathbb{N}}}
\renewcommand{\P}{{\mathbb{P}}}
\newcommand{\Q}{{\mathbb{Q}}}
\newcommand{\R}{{\mathbb{R}}}
\renewcommand{\SS}{{\mathbb{S}}}
\newcommand{\T}{{\mathbb{T}}}
\newcommand{\Z}{{\mathbb{Z}}}
%
\newcommand{\Aa}{{\mathcal{A}}}   % connections
\newcommand{\Bb}{{\mathcal{B}}}
\newcommand{\Cc}{{\mathcal{C}}}   % configuration space
\newcommand{\Dd}{{\mathcal{D}}}
\newcommand{\Ee}{{\mathcal{E}}}
\newcommand{\Ff}{{\mathcal{F}}}
\newcommand{\Gg}{{\mathcal{G}}}   % gauge transformations
\newcommand{\Hh}{{\mathcal{H}}}
\newcommand{\Ii}{{\mathcal{I}}}
\newcommand{\Jj}{{\mathcal{J}}}
\newcommand{\Kk}{{\mathcal{K}}}
\newcommand{\Ll}{{\mathcal{L}}}   % Lagrangian planes
\newcommand{\Mm}{{\mathcal{M}}}   % moduli space
\newcommand{\Nn}{{\mathcal{N}}}
\newcommand{\Oo}{{\mathcal{O}}}
\newcommand{\Pp}{{\mathcal{P}}}
\newcommand{\Qq}{{\mathcal{Q}}}
\newcommand{\Rr}{{\mathcal{R}}}
\newcommand{\Ss}{{\mathcal{S}}}
\newcommand{\Tt}{{\mathcal{T}}}
\newcommand{\Uu}{{\mathcal{U}}}
\newcommand{\Vv}{{\mathcal{V}}}
\newcommand{\Ww}{{\mathcal{W}}}
\newcommand{\Xx}{{\mathcal{X}}}
\newcommand{\Yy}{{\mathcal{Y}}}
\newcommand{\Zz}{{\mathcal{Z}}}
%
%\newcommand{\ker}{{\rm ker }}     % kernel
\newcommand{ \coker}{{\rm coker\, }}  % cokernel
%\newcommand{\coker}{\mathrm{coker}}  % cokernel
\newcommand{\im}{{\rm im\, }}      % image
\newcommand{\range}{{\rm range\, }}  % range
\newcommand{\SPAN}{{\rm span\, }}    % span
\newcommand{\dom}{{\rm dom\, }}      % domain
%\newcommand{\det}{{\rm det }}     % determinant
\newcommand{\DIV}{{\rm div\, }}      % divergence
\newcommand{\trace}{{\rm trace\, }}  % trace
\newcommand{\tr}{{\rm tr\, }}        % trace
\newcommand{\sign}{{\rm sign\, }}    % sign
\newcommand{\id}{{\rm id}}         % identity
\newcommand{\Id}{{\rm Id}}
\newcommand{\rank}{{\rm rank\, }}    % rank
\newcommand{\codim}{{\rm codim\, }}   % codimension
\newcommand{\diag}{{\rm diag}}     % diagonal matrix
\newcommand{\cl}{{\rm cl\, }}         % closure
\newcommand{\dist}{{\rm dist}}     % distance
\newcommand{\diam}{{\rm diam\, }}   % diameter
\newcommand{\INT}{{\rm int\, }}       % interior
\newcommand{\supp}{{\rm supp\, }}     % support
\newcommand{\modulo}{{\rm mod\,\,}}% modulo
\newcommand{\INDEX}{{\rm index}}   % (Fredholm)index
\newcommand{\IND}{{\rm ind}}       % (Morse)index
\newcommand{\ind}{{\rm Ind}} 
\newcommand{\NULL}{{\rm null}}       % Nullity
%\newcommand{\deg}{{\rm deg}}      % degree
\newcommand{\grad}{{\rm grad }}    % gradient
\newcommand{\IM}{{\rm Im }}        % imaginary part
\newcommand{\RE}{{\rm Re}}         % real part
\renewcommand{\Re}{{\rm Re}}       % real part
\renewcommand{\Im}{{\rm Im}}       % imaginary part
%\newcommand{\Lie}{{\rm Lie}}          % Lie algebra of
\newcommand{\Aut}{{\rm Aut}}          % Automorphisms
\newcommand{\Out}{{\rm Out}}          % Outer automorphisms
\newcommand{\Diff}{{\rm Diff}}        % Diffeomorphisms
\newcommand{\Vect}{{\rm Vect}}        % Vector fields
\newcommand{\Vol}{{\rm Vol}}          % Volume
\newcommand{\Symp}{{\rm Symp}}        % Symplectomorphisms
\newcommand{\Ham}{{\rm Ham}}          % Hamiltonian Symplectomorphisms
\newcommand{\Per}{{\rm Per}}          % periodic orbits
\newcommand{\Rat}{{\rm Rat}}          % rational maps
\newcommand{\Flux}{{\rm Flux}}        % Fluxhomomorphism
\newcommand{\Map}{{\rm Map}}          % Maps
\newcommand{\Met}{{\rm Met}}          % Metrics
\newcommand{\Or}{{\rm Or}}            % Orientations
\newcommand{\Res}{{\rm Res}}          % Residue
\newcommand{\Fix}{{\rm Fix}}          % Fixed points
\newcommand{\Crit}{{\rm Crit}}        % Critical points
\newcommand{\Hom}{{\rm Hom}}          % Homomorphisms
\newcommand{\End}{{\rm End}}          % Endomorphisms
%\newcommand{\Form}{{\Omega}}          % Differential forms
\newcommand{\Tor}{{\rm Tor}}          % Torsion
\newcommand{\Jac}{{\rm Jac}}          % Jacobian torus
\newcommand{\MOR}{{\rm Mor}}          % Morphisms
\newcommand{\OB}{{\rm Ob}}            % Objects
\newcommand{\Spin}{{\rm Spin}}        % Spin structures
\newcommand{\Spinc}{{{\rm Spin}^c}}   % Spin-c structures
%
\newcommand{\VERT}{{\rm Vert}}        % vertical subspace
\newcommand{\Hor}{{\rm Hor}}          % horizontal subspace
%
\newcommand{\cHZ}{{\rm c_{HZ}}}     % Hofer-Zehnder capacity
\newcommand{\CZ}{{\rm CZ}}            % Conley-Zehnder
%
%% JOA %%%%%%%%%%%%%%%%%%%%%%%%%%%%%%%
\newcommand{\HI}{{\rm HI}}             % homol. Conley index
\newcommand{\SH}{{\rm SH}}            % Sympl. homology
\newcommand{\Ho}{{\rm H}}             % Homology
\newcommand{\Co}{{\rm C}}              % Chain complex
\newcommand{\HM}{{\rm HM}}          % Morse homology
\newcommand{\CM}{{\rm CM}}          % Morse chain cplx
\newcommand{\HF}{{\rm HF}}           % Floer homology
\newcommand{\CF}{{\rm CF}}            % Floer chain complex
\newcommand{\RP}{\R{\rm P}}       % real projective space
\newcommand{\CP}{\C{\rm P}}       % complex projective space
\newcommand{\Hess}{\mathrm{Hess}}
\newcommand{\cat}{\mathrm{cat}\,}

%%%%%%%%%%%%%%%%%%%%%%%%%%%%%%%%%%%%%%
%%
%\newcommand{\E}{{\bf e}}
\renewcommand{\d}{{\rm d}}
%\newcommand{\D}{{\rm D}}
\renewcommand{\L}{{\rm L}}
\newcommand{\W}{{\rm W}}
\newcommand{\w}{{\rm w}}
\newcommand{\z}{{\bf z}}
%
\newcommand{\vc}{{\rm vc}}
\newcommand{\ad}{{\rm ad}}
\newcommand{\Ad}{{\rm Ad}}
\newcommand{\point}{{\rm pt}}
\newcommand{\ev}{{\rm ev}}
\newcommand{\e}{{\rm e}}
\newcommand{\ex}{{\rm ex}}
\newcommand{\odd}{{\rm odd}}
\newcommand{\can}{{\rm can}}
\newcommand{\eff}{{\rm eff}}
\newcommand{\sym}{{\rm sym}}
\newcommand{\symp}{{\rm symp}}
\newcommand{\norm}{{\rm norm}}
\newcommand{\torsion}{{\rm torsion}}
%\newcommand{\NORM}{{\rm norm}}
\newcommand{\MAX}{{\rm max}}
\newcommand{\FLAT}{{\rm flat}}
\newcommand{\dvol}{{\rm dvol}}
\newcommand{\loc}{{\rm loc}}
\newcommand{\const}{{\rm const}}
\newcommand{\hor}{{\rm hor}}          
%
\newcommand{\al}{{\alpha}}
\newcommand{\be}{{\beta}}
\newcommand{\ga}{{\gamma}}
\newcommand{\de}{{\delta}}
%\renewcommand{\phi}{{\varphi}}
%\renewcommand{\psi}{{\varpsi}}
\newcommand{\eps}{{\varepsilon}}
\newcommand{\ups}{{\upsilon}}
%\renewcommand{\i}{{\iota}}
\newcommand{\la}{{\lambda}}
\newcommand{\si}{{\sigma}}
\newcommand{\om}{{\omega}}
%
\newcommand{\Ga}{{\Gamma}}
\newcommand{\De}{{\Delta}}
\newcommand{\La}{{\Lambda}}
\newcommand{\Si}{{\Sigma}}
\newcommand{\Om}{{\Omega}}
%
\newcommand{\G}{{\rm G}}
\newcommand{\HG}{{\rm H}}
\newcommand{\EG}{{\rm EG}}
\newcommand{\BG}{{\rm BG}}
\newcommand{\ET}{{\rm ET}}
\newcommand{\BT}{{\rm BT}}
\newcommand{\GL}{{\rm GL}}
\renewcommand{\O}{{\rm O}}
\newcommand{\SO}{{\rm SO}}
\newcommand{\U}{{\rm U}}
\newcommand{\SU}{{\rm SU}}
\newcommand{\Sl}{{\rm SL}}
\newcommand{\SL}{{\rm SL}}
\newcommand{\ASL}{{\rm ASL}}
\newcommand{\PSL}{{\rm PSL}}
%\newcommand{\Sp}{{\rm Sp}}
\newcommand{\gl}{{\mathfrak g \mathfrak l}}
\renewcommand{\o}{{\mathfrak o}}
\newcommand{\so}{{\mathfrak s \mathfrak o}}
\renewcommand{\u}{{\mathfrak u}}
\newcommand{\su}{{\mathfrak s  \mathfrak u}}
\newcommand{\ssl}{{\mathfrak s \mathfrak l}}
\newcommand{\ssp}{{\mathfrak s \mathfrak p}}
\newcommand{\g}{{\mathfrak g}}     % Lie algebra of G
\newcommand{\h}{{\mathfrak h}}     % Lie algebra of H
\newcommand{\kk}{{\mathfrak k}}    % Lie algebra of H
\renewcommand{\tt}{{\mathfrak t}}  % Lie algebra of T
%% JOA
\newcommand{\mpo} {\mathfrak{m}}    % Morse poly
\newcommand{\ppo} {\mathfrak{p}}      % Poincare poly 
\newcommand{\qpo} {\mathfrak{q}}      % feedback
\newcommand{\sS} {\mathfrak{s}}    % stable conorm
%%
\newcommand{\Cinf}{C^{\infty}}
\newcommand{\CS}{{\mathcal{CS}}}
\newcommand{\CSD}{{\mathcal{CSD}}}
\newcommand{\YM}{{\mathcal{YM}}}
\newcommand{\SW}{{\rm SW}}
\newcommand{\Gr}{{\rm Gr}}
\newcommand{\reg}{{\rm reg}}
\newcommand{\sing}{{\rm sing}}
\newcommand{\Areg}{{\mathcal{A}}_{\rm reg}}
\newcommand{\Sreg}{{\mathcal{S}}_{\rm reg}}
\newcommand{\Jreg}{{\mathcal{J}}_{\rm reg}}
\newcommand{\JregK}{{\mathcal{J}}_{\rm reg,K}}
\newcommand{\Hreg}{{\mathcal{H}}_{\rm reg}}
\newcommand{\HJreg}{{\mathcal{HJ}}_{\rm reg}}
\newcommand{\HKreg}{{\mathcal{HK}}_{\rm reg}}
\newcommand{\XJreg}{{\mathcal{XJ}}_{\rm reg}}
\newcommand{\SP}{{\rm SP}}
\newcommand{\PD}{{\rm PD}}
\newcommand{\DR}{{\rm DR}}
%
\newcommand{\inner}[2]{\langle #1, #2\rangle}   
\newcommand{\INNER}[2]{\left\langle #1, #2\right\rangle}  
\newcommand{\Inner}[2]{#1\cdot#2}
\newcommand{\winner}[2]{\langle #1{\wedge}#2\rangle}   
%
\def\NABLA#1{{\mathop{\nabla\kern-.5ex\lower1ex\hbox{$#1$}}}}
\def\Nabla#1{\nabla\kern-.5ex{}_{#1}}
\def\Tabla#1{\Tilde\nabla\kern-.5ex{}_{#1}}
%
\def\abs#1{\mathopen|#1\mathclose|}   
\def\Abs#1{\left|#1\right|}            
\def\norm#1{\mathopen\|#1\mathclose\|}
\def\Norm#1{\left\|#1\right\|}
\def\NORM#1{{{|\mskip-2.5mu|\mskip-2.5mu|#1|\mskip-2.5mu|\mskip-2.5mu|}}}
%
\renewcommand{\Tilde}{\widetilde}
\renewcommand{\Hat}{\widehat}
\newcommand{\half}{{\scriptstyle\frac{1}{2}}}
\newcommand{\p}{{\partial}}
\newcommand{\notsub}{\not\subset}
\newcommand{\iI}{{I}}                   % unit interval [0,1]
\newcommand{\bI}{{\partial I}}     % boundary of same
\newcommand{\multidots}{\makebox[2cm]{\dotfill}}
%
\newcommand{\IMP}{\Longrightarrow}
\newcommand{\IFF}{\Longleftrightarrow}
\newcommand{\INTO}{\hookrightarrow}
\newcommand{\TO}{\longrightarrow}
%
%\newcommand{\proof}[1]{\noindent{\bf Proof#1:\  }}
%\newcommand{\jdef}[1]{{\bf #1}}
%\newcommand{\QED}{\hfill$\Box$}
%
%\def\thebibliography#1{\section*{References}
%       \addcontentsline{toc}{section}{References}
%       %\@mkboth{REFERENCES}{REFERENCES}
%       \begingroup\list{\arabic{enumi}.}
%       {\settowidth\labelwidth{[#1]}
%       \leftmargin\labelwidth
%       \itemsep 0pt
%       \parsep \itemsep
%       \advance\leftmargin\labeLsep
%       \Small
%       \usecounter{enumi}}}

%%%%%%%%% Hyphenation %%%%%%%%%%%
\hyphenation{
  diffeo-mor-phism
  Eliash-berg
  homo-topy homo-topies
  iso-mor-phism
  homeo-mor-phism
  Lip-schitz
  para-bolische
  quasi-klassi-scher
  }
%%%%%%%%% End Hyphenation %%%%%%%%

%\includeonly{I}
%\includeonly{II}

%%%%%%%%%%%%%%%%%%%%%%%%%%%%%%%%%%
%%%%%%%%%%%%%%%%%%%%%%%%%%%%%%%%%%
%%%%%%% main paper %%%%%%%%%%%%%%%%%%
%%%%%%%%%%%%%%%%%%%%%%%%%%%%%%%%%%
%%%%%%%%%%%%%%%%%%%%%%%%%%%%%%%%%%
\begin{document}
\maketitle
%%%%%%%%%%%% Abstract %%%%%%%%%%%%%%
\begin{abstract}
We introduce two tools, dynamical thickening and flow selectors,
to overcome the infamous discontinuity of the
gradient flow endpoint map near non-degenerate
critical points.
More precisely, we interpret the stable fibrations
of certain Conley pairs $(N,L)$, established
in~\cite{weber:2014c}, as a
\emph{dynamical thickening of the stable manifold}.
As a first application and to illustrate efficiency
of the concept we reprove a
fundamental theorem of classical Morse theory,
Milnor's homotopical cell attachment
theorem~\cite{milnor:1963a}.
Dynamical thickening leads to a
conceptually simple and short proof.
\end{abstract}%
%\tableofcontents

%\include{II}


%%%%%%%%%%%%%%%%%%%%%%%%%%%%%%%
%%%%%%%%%%%%%%%% Main Text  %%%%%%
%%%%%%%%%%%%%%%%%%%%%%%%%%%%%%%

Consider a connected smooth manifold
$M$ of finite dimension $n$. Suppose $f:M\to\R$
is a smooth function and $x$ is a non-degenerate
critical point of $f$ of Morse index $k$,
that is $df_x=0$ and
in local coordinates the Hessian matrix
$(\p^2 f/\p x^i\p x^j)_{i,j}$ at $x$ has precisely
$k$ negative eigenvalues, counting
multiplicities, and zero is not an eigenvalue.
Set $c:=f(x)$ and assume for simplicity
that the level set $\{f=c\}$ carries no
critical point other than $x$.

Morse theory studies how the topology
of sublevel sets $M^a=\{f\le a\}$ changes when
$a$ runs through a critical value $c$.
A fundamental tool is the concept of a flow,
also called a $1$-parameter group of
diffeomorphisms of $M$.
A common choice is the downward gradient
flow $\{\varphi_s\}_{s\in\R}$, namely the one
generated by the initial value problems
$\frac{d}{ds}\varphi_s=-(\nabla f)\circ\varphi_s$
with $\varphi_0=\id_M$. Existence is guaranteed,
for instance, if the vector field is of compact support.
Here $\nabla f$ denotes the gradient vector field
of $f$ on $M$. It is uniquely determined by
the identity $df(\cdot)=g(\nabla f,\cdot)$ after
fixing an auxiliary Riemannian metric $g$ on $M$.
Key properties of the downward gradient flow
are that $f$ decays along flow
lines $s\mapsto \varphi_s p$, for $p\in M$, and that
$\nabla f$ is orthogonal to level sets.
Consequently sublevel sets are forward
flow invariant. As $df_x=0$ $\Leftrightarrow$
$(\nabla f)_x=0$, any critical point $x$ is a
fixed point of the flow and non-degeneracy
translates into hyperbolicity.

By non-degeneracy of $x$ its unstable manifold
$W^u$ and descending disk $W^u_\eps$,
$$
     W^u=\{p\in M\mid\lim_{s\to-\infty}
     \varphi_sp=x\},\quad
    W^u_\eps=W^u\cap\{f\ge c-\eps\},
$$
are embedded open, respectively closed, disks in $M$ of
dimension $k=\IND(x)$; an embedding
$W^u_\eps\hookrightarrow M$ as a closed $k$-disk
exists only for every \emph{sufficiently small} $\eps>0$
(use the Morse-Lemma).
The boundary $S^u_\eps:=\p W^u_\eps$ is
called a descending sphere.
%
Consider instead the limit $s\to+\infty$
to get the stable manifold $W^s$
and ascending disk $W^s_\eps=W^s\cap\{f\le c+\eps\}$.
They have analogous properties
except that they are of codimension $k$.

In~\cite{weber:2014c}, see~\cite[Thm.~5.1]{Weber:2015c}
for details in the present finite dimensional case,
we implemented the structure of a disk bundle
on the compact neighborhood
$$
     N=N_x^{\eps,\tau}
     :=\left\{p\in M\mid\text{$f(p)\le c+\eps$,
     $f(\varphi_\tau p)\ge c-\eps$}\right\}
     _{\text{connected component of $x$}}
$$
of $x$ whenever $\eps>0$ is small and
$\tau>0$ is large.
The fibers are codimension-$k$ disks
with boundaries in the upper level set $\{f=c+\eps\}$
and parametrized by their unique point of
intersection, say $q^T$, with the unstable manifold.
The fiber over $x$ is $W^s_\eps$.
Each point of a fiber $N(q^T)$
reaches the lower level set
$\{f=c-\eps\}$ in time $T$ under the
downward gradient flow. Note that
$\{f=c-\eps\}$ intersects $W^u$
in the descending $(k-1)$-sphere
$S^u_\eps=\p W^u_\eps$.
%
Choose a tubular neighborhood $\Dd$ of
$S^u_\eps$ in $\{f=c-\eps\}$ to get a family
of codimension-$k$ disks $\Dd_q$, one for
each $q\in S^u_\eps$.
%
\begin{figure}%[b]
  \centering
  \includegraphics{fig-N}
  \caption{Dynamical thickening $(N,\theta)$
                 of the local stable manifold $(W^s_\eps,\varphi|)$
           }
  \label{fig:fig-N}
\end{figure}
%
By~\cite{weber:2014c,Weber:2015c}
we get a Lipschitz continuous ($C^{0,1}$) disk bundle
\begin{equation*}\label{eq:N_a-foliated}
     N
     =W^s_\eps
     \mathop{\dot{\cup}}_{T\ge\tau,q\in S^u_\eps}
      N(q^{T}),
     \quad
      N(q^{T})={\varphi_{T}}^{-1}(\Dd_q)
     \cap\{f\le c+\eps\},
\end{equation*}
over $\varphi_{-\tau} W^u_\eps$
which is $C^{1,1}$ away from the ascending disk
$W^s_\eps$. It is a key fact that the fibers
are diffeomorphic to $W^s_\eps$ via $C^1$
maps $\Gg^T_q:W^s_\eps\to N(q^T)$
which converge in $C^1$ to the identity on
$W^s_\eps$, as $T\to\infty$.
Furthermore, the fibration is forward flow invariant in the
sense that $\varphi_s$ maps a fiber $ N(q^T)$
into $ N(\varphi_sq^T)$.
Figure~\ref{fig:fig-N} illustrates the fibration and
the qualitative behavior of the forward
flow which is transverse to all fibers except the one
over $x$ which is invariant.
Conjugation by the diffeomorphism $\Gg^T_q$
provides on each fiber $N(q^T)$
a copy $\theta_s$ of the forward flow
$\varphi_s$ on $W^s_\eps$.
%
Now we reprove the cell attachment theorem.

\begin{theoremA}
[Milnor {\cite[I Thm.~3.2]{milnor:1963a}}]
Let $f:M\to\R$ be a smooth function, and let $x$
be a non-degenerate critical point with Morse
index $k$. Setting $f(x)=c$, suppose that
$f^{-1}[c-\eps,c+\eps]$ is compact and
contains no critical point of $f$ other than~$x$,
for some $\eps>0$. Then, for all sufficiently
small $\eps$, the set $M^{c+\eps}$ has the
homotopy type of $M^{c-\eps}$ with a
$k$-cell attached.
\end{theoremA}

\begin{proof}
Fix a Riemannian metric on $M$. 
Without loss of generality assume that $-\nabla f$
is of compact support,\footnote{
  Otherwise, substitute for $-\rho\nabla f$
  where $\rho:M\to\R$ is a smooth compactly
  supported cut-off function with $\rho\equiv 1$
  on the compact set $K:=f^{-1}[c-\eps,c+\eps]$.
  }
so it generates a flow
$\{\varphi_s\}_{s\in\R}$ on $M$.
Pick constants $\eps>0$ small and $\tau>0$ large
in order to meet the assumptions in~\cite{Weber:2015c} of
Theorem~5.4 (existence of the invariant fibration $N=N_x^{\eps,\tau}$)
and Definition~5.6 (induced fiberwise semi-flow $\theta$).
Figure~\ref{fig:fig-selector} illustrates the proof: First deform
$N\subset M^{c+\eps}$ along
$\theta$ towards the flow selector $\Ss^+$ and $W^u_\eps$,
then deform along $\varphi$.

\vspace{.1cm}
\textbf{0.~Definition of flow selector} (hypersurface
transverse to two flows): View
$$
     \Ss^+:=\{\varphi_{-\mathfrak{s}\circ\mathfrak{t}^-(p)}p\mid
     p\in\Ss^-\}\subset N
$$
as graph of a function $\mathfrak{s}\circ\mathfrak{t}^-$
over an open subset $\Ss^-\subset f^{-1}(c-\eps)$
where the coordinate lines are backward flow lines of $\varphi$
starting at $\Ss^-$ with coordinate the backward time.
By the flow box theorem this makes sense, as there
is no singularity of $\nabla f$ on $\Ss^-$. By the graph property
\emph{$\varphi$ will be transverse to $\Ss^+$}.

By~\cite[Thm.~1.2]{Weber:2015c} there is a $C^0$ time label function
$\mathfrak{t}:N\to[\tau,\infty]$, of class $C^1$ as a function
$N_\times:=N\setminus W^s\to[\tau,\infty)$, which assigns to
each point $p$ the time it takes to reach the lower level set
$f^{-1}(c-\eps)$ under the gradient flow $\varphi$. The hypersurface
$N^+:=\{p\in N\cap f^{-1}(c+\eps)\mid\mathfrak{t}(p)<\tau\}$
is called the \textbf{entrance set} of $N$ and
$N^+_\times:=N^+\setminus W^s$ its \textbf{regularization};
see Figure~\ref{fig:fig-N}. As each point of $N^+_\times$ hits
$f^{-1}(c-\eps)$ under $\varphi$ precisely once and transversely, the
corresponding subset $\Ss^-_\times\subset f^{-1}(c-\eps)$ is
diffeomorphic to $N^+_\times$. The \textbf{time label function}
$\mathfrak{t}^-:\Ss^-_\times\to(\tau,\infty)$ is defined by transfering
the time labels of $N^+_\times$. It is of class $C^1$.
Add the descending disk $S^u_\eps$ to define
$$
     \Ss^-:=\Ss^-_\times\mathop{\dot{\cup}} S^u_\eps
     =\{p\in f^{-1}(c-\eps)\mid N^+\cap\varphi_\R p\not=\emptyset\}
     \mathop{\dot{\cup}} S^u_\eps,\qquad
     \Ss^-_\times\stackrel{\varphi}{\cong} N^+_\times,
$$
as an open subset of $f^{-1}(c-\eps)$; see Figure~\ref{fig:fig-selector}.
Set $\mathfrak{t}^-=\infty$ on $S^u_\eps$.
%
\begin{figure}%[b]
  \centering
  \includegraphics{fig-selector}
  \caption{Flow selector $\Ss^+_\times=\Ss^+\setminus W^u$
          with transverse flows $\theta$ and $\varphi$
          }
  \label{fig:fig-selector}
\end{figure}
The function
$$
     \mathfrak{s}:(\tau,\infty)\to(\tau,2\tau),\quad
     \mathfrak{t}\mapsto 2\tau-\tau^2/\mathfrak{t},
$$
is smooth and extends continuously to  $[\tau,\infty]$ such that
$\mathfrak{s}(\tau)=\tau$ with $\mathfrak{s}^\prime(\tau)=1$
and $\mathfrak{s}(\infty)=2\tau$ with $\mathfrak{s}^\prime(\infty)=0$;
see Figure~\ref{fig:fig-selector} for the corresponding graph $\Ss^+$.
Observe that critical points of $\mathfrak{s}$
correspond precisely to tangencies of $\theta$ to the hypersurface
$\Ss^+_\times:=\Ss^+\setminus W^u$. But $\mathfrak{s}$ admits
no critical points on $(\tau,\infty)$, so \emph{$\theta$ is transverse to
$\Ss^+_\times$}. This proves that $\Ss^+_\times$ is a flow selector
with respect to $\varphi$~and~$\theta$.

\vspace{.1cm}
\textbf{I.~Strong deformation retraction} $r:M^{c+\eps}\to
M^{c-\eps}\cup W^u_\eps\cup X$ via $\theta$:
Let $\Ss$ be the region under the graph of $\Ss^+$,
that is the region bounded by $\Ss^-$ and $\Ss^+$
and the hypersurfaces indicated by dashed arrows
in Figure~\ref{fig:fig-selector}. The arrows are dashed to
indicate that they do not belong to $\Ss$, but to the closure $\bar\Ss$.
Consider the compact set
$X:=\left(f^{-1}[c-\eps,c+\eps]\setminus N\right)\cup\bar\Ss$ whose
boundary is given by $f^{-1}(c-\eps)$ and
$\left(f^{-1}(c+\eps)\setminus N^+\right)\cup\Ss^+$.
Deforming $N\setminus\bar\Ss$ along the flow lines of $\theta$
until the flow line hits either the flow selector $\Ss^+$ or the
descending disk $W^u_\eps$, while not moving the other points
of $M^{c+\eps}$ at all, defines the required strong deformation
retraction $r$. Continuity of $r$ holds since $\theta$ is transverse to
$\Ss^+_\times$ and $\Ss^+\setminus\Ss^+_\times=\Ss^+\cap W^u_\eps$
is reached under $\theta$ in infinite time just as is
$W^u_\eps\setminus \Ss$.

\vspace{.1cm}
\textbf{II.~Homotopy equivalence} $M^{c-\eps}\cup W^u_\eps\cup X
\sim M^{c-\eps}\cup W^u_\eps$ via $\varphi$:
Given the pair of closed sets $A:=M^{c-\eps}\subset \left(X\cup A\right)$,
consider the entrance time function $\Tt_A:X\cup A\to [0,\infty)$
which assigns to each point $p\in X\cup A$ the time it takes to reach
$A$ under $\varphi$. To see that $\Tt_A$ is well defined note that
$A$ and $X\cup A$ are both forward flow invariant under $\varphi$.
Indeed $\p A$ is a level set along which $-\nabla f$ is downward,
hence inward, pointing. The (topological) boundary of $X\cup A$ is
$\left(f^{-1}(c+\eps)\setminus N^+\right)\cup\Ss^+$
and $-\nabla f$ points inward along both pieces.

Given that $\varphi$ is transverse to $\p X$,
the function $\Tt_A$ is lower and upper
semi-continuous, hence continuous, because the
subset $A$ of $X\cup A$ is closed and
forward flow invariant, respectively;
cf.~\cite[Pf. of Thm.~B]{weber:2014c}.
%
Since $X$ is compact without critical points
$\Tt_A$ is bounded.  The map 
$h:[0,1]\times Z\to Z$ given by
\begin{equation*}
\begin{split}
     h(\lambda,p)=
     \begin{cases}
        p&
        \text{, $p\in A=M^{c-\eps}$,}
        \\
        \varphi_{\lambda\Tt_A(p)}p&
        \text{, $p\in X$,}
        \\
        \varphi_{\lambda4\tau^2/\mathfrak{t}(p)}p&
        \text{, $p\in\overline{W^u_\eps\setminus\Ss}=\varphi_{-2\tau} W^u_\eps$.}
     \end{cases}
\end{split}
\end{equation*}
is continuous as it is defined by three continuous parts which agree on overlaps:
$\Tt_A=0$ on $A\cap X$ and $\Tt_A=4\tau^2/\mathfrak{t}=2\tau$
on $\varphi_{-2\tau} S^u_\eps$.
The inclusion $\iota:A\cup W^u_\eps=:B\hookrightarrow Z:=X\cup A\cup
W^u_\eps$ and $h_1:=h(1,\cdot):Z\to B$ are reciprocal homotopy
inverses. Indeed $\iota\circ h_1=h_1\sim h_0=\id_Z$
and $h_1\circ\iota=h_1|_B\sim h_0|_B=\id_{B}$.
\end{proof}

% \footnote{
  Part two of $h_1$ unfortunately
  eliminates an outer piece of $W^u_\eps$ which we recover by
  $\varphi_{4\tau^2/\mathfrak{t}(\cdot)}(\cdot):\varphi_{-2\tau} W^u_\eps\to W^u_\eps$.
  So $h_1$ does not restrict to the identity on $W^u_\eps$,
  hence $h$ is not a deformation retraction
  of $X\cup A\cup W^u_\eps$ onto $A\cup W^u_\eps$.
%  }

%%%%%%%%%%%%%%%%%%%%%%%%%%%%%%%
%%%%%%%%%% Proof of theorem %%%%%%%%
%%%%%%%%%%%%%%%%%%%%%%%%%%%%%%%
%\subsubsection*{Proof}
%
%%% APPENDIX %%%%%%
%\appendix
%
% ACKNOWLEDGEMENTS %
% * various anonymous referees

%%%%%%%%%%%%%%%%%%%%%%%%%%%%%%%
\subsubsection*{Perspectives}
In the history of Morse theory
discontinuity of the flow trajectory end point
map $\varphi_\infty$ obstructed to carry out,
in a simple fashion, various constructions suggested
by geometry, for instance, to extend continuously
the inclusion map of an unstable manifold towards the closure.
It will be a future research project
to investigate the role of dynamical thickening
and flow selectors in such cases.

By~\cite{weber:2014c} dynamical thickening
can be defined in infinite dimensional contexts.


%%%%%%%%%%%%%%%%%%%%%%%%%%%%%%%
\subsubsection*{Added in proof}
Flow selector added
to correct the discontinuity in previous version.
Flow selectors arose in cooperation with 
Pietro Majer (2015) in two flavors - via Conley blocks
and via carving. Here we use a version of the
Conley block technique.


%%%%%%%%%%%%%%%%%%%%%%%%%%%%%%%%
%%%%%%%%% Acknowledgements %%%%%%%%
%%%%%%%%%%%%%%%%%%%%%%%%%%%%%%%%
\vspace{.1cm}
\noindent
{\small\bf Acknowledgements.} {\small
The author is grateful to %would like to thank
Stephan Weis for asking the right question just in time
and Kai Cieliebak for useful remarks concerning the flow selector.
}


%%%%%%%%%%%%%%%%%%%%%%%%%%%%%%%%%%%
%%%%%%%%%%%%%%%%%%%%%%%%%%%%%%%%%%%
%%%%%%%%%%%%%%%% References %%%%%%%%%%
%%%%%%%%%%%%%%%%%%%%%%%%%%%%%%%%%%%
%%%%%%%%%%%%%%%%%%%%%%%%%%%%%%%%%%%


%%%%%%%%%%%%%%%%%%%%%%%%%
%%%%%%%%% BibDesk %%%%%%%%%%
%%%%%%%%%%%%%%%%%%%%%%%%%
\bibliography{$HOME/Dropbox/0-Libraries+app-data/Bibdesk-BibFiles/library_math}{}
%$
%\bibliographysty le{plain}
         %   erzeugt:     [1] Joa Weber
\bibliographystyle{abbrv}
         %  erzeugt:      [1] J. Weber and 
%\bibliographystyle{alpha}
         %  article:    [Web05]  J. Weber
         %  book:      [Web05]  Joa Weber
         % more authors: [HZ87]

%%%%%%%%%%%%%%%%%%%%%%%%%
%%%%%%%%% standard %%%%%%%%%
%%%%%%%%%%%%%%%%%%%%%%%%%
%\begin{thebibliography}{00000}
%\small
%\end{thebibliography}


%\newpage
%%%%%%%%%%%%%%%%%%%%%%%%%%%%%%%%%%%%
%%%%%%%%%%%%% SYMBOLS %%%%%%%%%%%%%%
%%%%%%%%%%%%%%%%%%%%%%%%%%%%%%%%%%%%
%\subsection*{Symbols}
%$\N:=\{1,2,3,\dots\}$
%%%%%%%%%%%%%%%%%%%%%%%%%%%%%%%%%%%%
%%%%%%%%%% NOTIZEN %%%%%%%%%%%%%%%%%%%
%%%%%%%%%%%%% TODO's %%%%%%%%%%%%%%%%
%%%%%%%%%%%%%%%%%%%%%%%%%%%%%%%%%%%%
%\newpage
%\include{NOTIZEN}
%
%\listoftodos
%%%%%%%%%%%% To Do's %%%%%%%%%%%%%%%
%\vspace{1cm}\noindent
%{\Large\bf GENERAL TO DOs}
%\begin{itemize}
%\item 
%  MTC 
%\end{itemize}
%%%%%%%%%%%%%%%%%%%%%%%%%%
\end{document}




