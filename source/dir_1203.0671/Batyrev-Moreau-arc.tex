% user docs for compositio.cls
\def\Filedate{2011/01/24} % date this file last revised
\def\Fileversion{1.15}    % version of compositio.cls documented

\documentclass{compositio}
% If you have the AMSLaTeX distribution installed on your system,
% please delete the "[noams]" option above.

%
%
% Package for Dynkin diagrams - uses picture environment
%
% Use the environment "Dynkin"
% then it is like array (but without the arguments).
% The entries are "Dbloc" and you can put (hard) circles, lines in different directions,
% and text (at the end please for tabbing). 
% The idea is that the circle is in the middle (inspired by the game Labyrinth)
% "Dskip" is the equivalent of "//" 
% "Dspace" is the empty block. 
%

%%% Beginning of routines %%%

% The environments Tiles and Dynkin
\newenvironment{Tiles}[1]{\setlength{\unitlength}{#1pt}\begin{array}{l}}{\end{array}}
\newenvironment{Dynkin}{\setlength{\unitlength}{1.5pt}\begin{array}{l}}{\end{array}}

% The command \Dbloc{args}
\newcommand{\Dbloc}[1]{\begin{picture}(20,20)#1\end{picture}}

% Possible arguments inside \Dbloc
\newcommand{\Dcirc}{\put(10,10){\circle{4}}}
\newcommand{\Dbullet}{\put(10,10){\circle*{4}}}
\newcommand{\Deast}{\put(12,10){\line(1,0){8}}}
\newcommand{\Dwest}{\put(8,10){\line(-1,0){8}}}
\newcommand{\Dnorth}{\put(10,12){\line(0,1){8}}}
\newcommand{\Dsouth}{\put(10,8){\line(0,-1){8}}}
\newcommand{\Dnortheast}{\put(20,20){\line(-1,-1){8.6}}}
\newcommand{\Dnorthwest}{\put(0,20){\line(1,-1){8.6}}}
\newcommand{\Dsoutheast}{\put(20,0){\line(-1,1){8.6}}}
\newcommand{\Dsouthwest}{\put(0,0){\line(1,1){8.6}}}
\newcommand{\Ddoubleeast}{\put(10,12){\line(1,0){10}}\put(10,8){\line(1,0){10}}}
\newcommand{\Ddoublewest}{\put(10,12){\line(-1,0){10}}\put(10,8){\line(-1,0){10}}}
\newcommand{\Ddots}{\put(5,10){\circle*{0.9}}\put(10,10){\circle*{0.9}}\put(15,10){\circle*{0.9}}}

\newcommand{\Dtext}[2]{\makebox(20,20)[#1]{\scriptsize $#2$}}
\newcommand{\Ddoubletext}[2]{\makebox(40,20)[#1]{\scriptsize $#2$}}
\newcommand{\Dlongtext}[3]{\makebox(#3,20)[#1]{\scriptsize $#2$}}

% New arguments for tile drawing
\newcommand{\Ttop}{\put(0,20){\line(1,0){20}}}
\newcommand{\Tbottom}{\put(0,0){\line(1,0){20}}}
\newcommand{\Tleft}{\put(0,0){\line(0,1){20}}}
\newcommand{\Tright}{\put(20,0){\line(0,1){20}}}
\newcommand{\Tantidiag}{\put(0,0){\line(1,1){20}}}
\newcommand{\Tdiag}{\put(20,0){\line(-1,1){20}}}
\newcommand{\Ttopdots}{\put(4,20){\circle*{0.1}}\put(8,20){\circle*{0.1}}\put(12,20){\circle*{0.1}}\put(16,20){\circle*{0.1}}}
\newcommand{\Tbottomdots}{\put(4,0){\circle*{0.1}}\put(8,0){\circle*{0.1}}\put(12,0){\circle*{0.1}}\put(16,0){\circle*{0.1}}}
\newcommand{\Tleftdots}{\put(0,4){\circle*{0.1}}\put(0,8){\circle*{0.1}}\put(0,12){\circle*{0.1}}\put(0,16){\circle*{0.1}}}
\newcommand{\Trightdots}{\put(20,4){\circle*{0.1}}\put(20,8){\circle*{0.1}}\put(20,12){\circle*{0.1}}\put(20,16){\circle*{0.1}}}
\newcommand{\Tdiagdots}{\put(4,16){\circle*{0.1}}\put(8,12){\circle*{0.1}}\put(12,8){\circle*{0.1}}\put(16,4){\circle*{0.1}}}
\newcommand{\Tantidiagdots}{\put(4,4){\circle*{0.1}}\put(8,8){\circle*{0.1}}\put(12,12){\circle*{0.1}}\put(16,16){\circle*{0.1}}}
\newcommand{\Ttopleftdot}{\put(0,20){\circle*{0.1}}}
\newcommand{\Ttoprightdot}{\put(20,20){\circle*{0.1}}}
\newcommand{\Tbottomleftdot}{\put(0,0){\circle*{0.1}}}
\newcommand{\Tbottomrightdot}{\put(20,0){\circle*{0.1}}}

% Other commands outside \Dbloc

\newcommand{\Dskip}{\\ [-4.5pt]}
\newcommand{\Dspace}{\Dbloc{}}
\newcommand{\Dleftarrow}{\hskip-5pt{\makebox(20,20)[l]{\Large$<$}}\hskip-25pt}
\newcommand{\Drightarrow}{\hskip-5pt{\makebox(20,20)[l]{\Large$>$}}\hskip-25pt}

%%% End of routines %%%








\usepackage{amsmath,amsfonts,amssymb,amsthm,amscd}
\usepackage{dsfont, mathrsfs,enumerate}
\usepackage{xspace}

\usepackage{eucal}

\usepackage[english]{babel}

\usepackage{xy, xypic}
\usepackage{color}
\usepackage{graphicx}
\usepackage{psfrag}
%\usepackage{pst-all}


\usepackage{xcolor}
\definecolor{rouge}{rgb}{0.9,0.1,0}
\definecolor{bleu}{rgb}{0.1,0,0.9}
\definecolor{violet}{rgb}{0.7,0,0.8}
\usepackage[colorlinks=true,linkcolor=bleu,urlcolor=violet,citecolor=rouge]{hyperref}
\usepackage{breakurl}
\usepackage{xspace}
%\usepackage{hyperref}



\newtheorem{thm}{Theorem}[section]
\newtheorem{lemma}[thm]{Lemma}
\newtheorem{claim}[thm]{Claim}
\newtheorem{prop}[thm]{Proposition}
\newtheorem{cor}[thm]{Corollary}
\theoremstyle{definition}
\newtheorem{defi}[thm]{Definition}
\newtheorem{rem}[thm]{Remark}
\newtheorem{question}[thm]{Question}
\newtheorem{ex}[thm]{Example}
\newtheorem{conj}[thm]{Conjecture}


\def\le{\leqslant}
\def\ge{\geqslant}


\def\F{\mathcal{F}}     % color



\def\G{\mathbb G}
\def\L{\mathbb L}
\def\Z{\mathbb Z}
\def\Q{\mathbb Q}
\def\N{\mathbb N}
\def\R{\mathbb R}
\def\C{\mathbb C}
\def\id{{\it id}}
\def\A{\mathbb A}
\def\P{\mathbb P}
\def\O{\mathcal  O}      % \C [[ t ]]
\def\K{\mathcal K}      % \C ((t))

\def\D{\mathcal{D}}    % G-stable divisors
\def\M{\mathcal{M}_{\C}}   % localization of K...
\def\E{\mathcal{E}}    % stringy volume
\def\V{{\rm Var}_{\mathbb C}}    % category of cplx. alg. var.

%\DeclareMathOperator{\Mor}{Mor}
%\DeclareMathOperator{\Hom}{Hom}
\DeclareMathOperator{\Spec}{Spec}
%\DeclareMathOperator{\PGL}{PGL}
%\DeclareMathOperator{\Desc}{Desc}
\DeclareMathOperator{\Pic}{Pic}
\DeclareMathOperator{\Div}{Div}
%\DeclareMathOperator{\NE}{NE}

\def\ord{\mathrm{ord}}   % order function
\def\card{\mathrm{card}}

\def\root{\mathcal{R}}


%
\begin{document}

\title{The arc space of horospherical varieties and
motivic integration}
%

\author[Victor Batyrev]{Victor Batyrev}
\email{victor.batyrev@uni-tuebingen.de}
\address{Victor Batyrev, Mathematisches Institut, Universit\"at T\"ubingen,
 72076 T\"ubingen, Germany}

\author[Anne Moreau]{Anne Moreau}
\email{anne.moreau@math.univ-poitiers.fr}
\address{Anne Moreau, Laboratoire de Math{\'e}matiques et Applications,
 Universit{\'e} de Poitiers, France}



%
%\dedication{A dedication can be included here.}
\classification{14L30,14M27}
%least one subject code is required. Please refer to
%\url{http://www.ams.org/msc/} for a list of codes.}
\keywords{Horospherical variety, arc space, motivic integration, stringy invariant}

\begin{abstract}

For an arbitrary connected reductive group $G$ we consider the
motivic integral over the arc space of an arbitrary  $\Q$-Gorenstein
horospherical $G$-variety $X_\Sigma$ associated with
a colored fan $\Sigma$ and prove a formula for
the stringy $E$-function of $X_\Sigma$
which generalizes the one for toric varieties.
We remark that  in contrast to toric varieties the stringy $E$-function of
a Gorenstein horospherical variety $X_\Sigma$ may be not a polynomial
if some cones in $\Sigma$ have nonempty sets  of colors.
Using the stringy $E$-function, we can formulate and prove
a new smoothness criterion for locally factorial horospherical
varieties.
We expect that this smoothness criterion holds for
arbitrary spherical varieties.

\end{abstract}

\maketitle



\section*{Introduction}


Throughout the paper, we consider
algebraic varieties and algebraic groups
over the ground field $\C$.
%

\smallskip

Let $G$ be a connected reductive group and $H \subseteq G$ a closed subgroup.
The homogeneous space $G/H$ is called {\em horospherical} if $H$ contains
a maximal unipotent subgroup $U \subseteq G$.
In this case, the normalizer
$N_G(H)$ is a parabolic subgroup $P \subseteq G$ and $P/H$ is an algebraic
torus $T$.
The horospherical homogeneous space $G/H$ can be described
as a principal torus bundle with the fiber $T$ over the projective
homogeneous space  $G/P$.
The dimension $r$ of the torus $T$ is called
the {\it rank} of the horospherical homogeneous space $G/H$.
Let $M$ be the
lattice of characters of the torus $T$, and $N= {\rm Hom}(M, \Z)$ the
dual lattice.
According to the
Luna-Vust theory \cite{LV83}, any
$G$-equivariant embedding $G/H \hookrightarrow X$ of a
horospherical homogeneous
space $G/H$ can be described combinatorially by  a colored fan $\Sigma$ in
the $r$-dimensional vector space $N_\R:= N \otimes_\Z \R$.
In the case $H=U$,
$G$-equivariant embeddings of $G/U$ have been considered independently  by
Pauer \cite{Pau81,Pau83}.
Equivariant embeddings of horospherical homogeneous
spaces are generalizations of the well-known toric varieties which are
torus embeddings $T \hookrightarrow X$ ($G=T$, $H= \{e\}$).
%%

Our paper is motivated by some known formulas for
stringy invariants of toric varieties.
Let $X$ be a $\Q$-Gorenstein toric variety defined by a fan
$\Sigma \subset N_\R$ and denote
by $|\Sigma| \subset N_\R$ its support.
%
Then there is a  piecewise linear function
$\omega_X\, : \, | \Sigma |  \to \R$ such that its restriction to every cone
$\sigma \in \Sigma$ is linear and $\omega_X$ has value $-1$ on all
primitive lattice generators of $1$-dimensional faces of $\sigma$.
It was shown in \cite{Ba98} that the
stringy $E$-function of the toric variety $X$ can be computed by the
formula
%
\begin{eqnarray} \label{Est-intro}
%
	E_{\rm st} (X ; u,v ) := \big( uv -1 \big)^r \sum_{n \in |\Sigma| \cap N} (uv)^{\omega_X(n)} .
%
\end{eqnarray}
%
%
If $X$ is smooth and projective, then the stringy $E$-function
%$E_{\rm st} (X ; u,v )$
of $X$ coincides with the usual $E$-function,
$$
	E(X ; u,v ) = \sum_{i = 1}^{ r} b_{2i} (X) (uv)^{i} \, ,
$$
where $b_{2i}(X)$ is the $2i$-th Betti number of $X$.
Using the decomposition of $X$ into torus orbits,
we can compute $E(X ; u,v)$ by the formula,
$$
	E (X ; u,v) = \sum\limits_{\sigma \in \Sigma} (uv - 1)^{r -\dim \sigma}
		 = \big(uv -1 \big)^r \sum\limits_{\sigma \in \Sigma}
		 	\displaystyle{\frac{(-1)^{\dim \sigma}}{(1 -uv)^{\dim \sigma}}} \, .
$$
Hence,
$$
	\sum_{n \in N} (uv)^{\omega_X(n)}
		= \sum\limits_{\sigma \in \Sigma}
		 	\displaystyle{\frac{(-1)^{\dim \sigma}}{(1 -uv)^{\dim \sigma}}}
		= (-1)^r P( R_\Sigma, uv)
		= (-1)^r \displaystyle{\frac{ \sum_{i = 1}^{ r} b_{2i} (X) (uv)^{i}}{(1 -uv)^r}} \, ,
$$
%
where $P (R_\Sigma, t) = \sum_{i \ge 0} \dim R_\Sigma^{i}\, t^{i}$
is the Poincar\'e series of the graded Stanley-Reisner ring
$R_\Sigma = \bigoplus_{i \ge 0} R_\Sigma^{i}$
associated with the fan $\Sigma$.
%

\smallskip

Recall the definition of the Stanley-Reisner ring
$R_\Sigma$.
Let $e_1,\ldots,e_s$ be the primitive integral generators
of all $1$-dimensional cones in $\Sigma$.
We consider the polynomial
ring $\C[z_1,\ldots,z_s]$ whose
the variables $z_1,\ldots,z_s$ are in
bijection to lattice vectors $e_1,\ldots,e_s$.
%
Then the {\em Stanley-Reisner ring} $R_\Sigma$ is
the quotient of $\C[z_1,\ldots,z_s]$
by the ideal generated by those square free monomials
$z_{i_1} \ldots z_{i_k}$
such that the lattice vectors
$e_{i_1} \ldots e_{i_k}$ do not generate any
$k$-dimensional cone in $\Sigma$.
%
The cohomology ring $H^\ast (X, \C)$ of
the smooth projective toric variety $X$ associated with $\Sigma$
is isomorphic to the quotient of $R_\Sigma$
modulo the ideal generated by a regular sequence $f_1, \ldots, f_r$
in $R_\Sigma^{1}$ (see e.g.\,\cite[Theorem 10.8]{D}).



\smallskip

In this paper, we prove a similar to (\ref{Est-intro}) formula
for any $\Q$-Gorenstein horospherical variety $X$ defined
by a colored fan $\Sigma$:
%
\begin{eqnarray} \label{Est-2}
%
	E_{\rm st} (X ; u,v ) := E(G/H ; u,v) \sum_{n \in |\Sigma| \cap N} (uv)^{\omega_X(n)} \, ,
%
\end{eqnarray}
%
where $\omega_X : |\Sigma| \to \R$ is a certain $\Sigma$-piecewise linear function
(cf.\,Theorem \ref{t:main}).
%
Let $X$ be a complete and locally factorial horospherical variety
defined by a colored cone $\Sigma$.
Let $e_1,\ldots,e_s$ be the primitive integral generators
of all $1$-dimensional cones in $\Sigma$.
Consider the positive integers $a_i := - \omega_X(e_i)$
for $i \in \{�1,\ldots, s\}$
and define the \emph{weighted Stanley-Reisner ring}
$R_\Sigma^w$
corresponding to the colored fan $\Sigma$
by putting $\deg z_i = a_i$
in the standard Stanley-Reisner ring $R_\Sigma$
(here we consider $\Sigma$ as an uncolored fan).
%
In Proposition \ref{p:SR}, we prove that
$$
	\sum_{n \in N} (uv)^{\omega_X(n)}
	= (-1)^r P(R_\Sigma^w, uv)
	= (-1)^r  \sum\limits_{\sigma \in \Sigma}
		\displaystyle{ \frac{(-1)^{\dim \sigma}}{ \prod_{e_i \in \sigma} \big(1 - (uv)^{a_i} \big)} } \, ,
$$
where $P(R_\Sigma^w, t)$ is the Poincar\'e series associated with
the weighted Stanley-Reisner ring $R_\Sigma^w$.
%
So we get
%
$$
	E_{\rm st} (X ; u,v ) = (-1)^r  E(G/H ; u,v) P(R_\Sigma^w, uv)  \, .
$$
%

In contrast to toric varieties, the stringy $E$-function of a locally factorial horospherical
variety $X$ needs not be a polynomial.
%
If $X$ is smooth, then $E_{\rm st}(X ; u,v) = E(X ; u,v)$
is polynomial and in particular the \emph{stringy
Euler number},  $e_{\rm st}(X) : = E_{\rm st} (X ; 1,1)$,
is equal to the usual Euler number $e(X) : = E(X ; 1,1)$.
%
If $X$ is a locally factorial horospherical variety
whose closed orbits are projective, then we show that
$e_{\rm st}(X) \ge e(X)$ and that the equality
 holds if and only if $X$ is smooth (cf.\,Theorem \ref{t:smo}).
%
%
We conjecture  that the equality
$$
	e_{\rm st}(X) =  e(X)
$$
%
can be used as a smoothness criterion for arbitrary
locally factorial spherical varieties (cf. Conjecture \ref{c:smo}).

\medskip

The key idea behind the formula (\ref{Est-2})
for toric varieties is the isomorphism
$$
	T({\mathcal K}) / T({\mathcal O}) \simeq N,
$$
where ${\mathcal O} := \C[[t]]$, ${\mathcal K} := \C((t))$ and
$T({\mathcal O})$ (resp.\,$T({\mathcal K})$) denotes the
set of ${\mathcal O}$-valued (resp.\,${\mathcal K}$-valued) points
in $T$.
%
We remark that the \emph{stringy motivic integral} over the arc space $X({\mathcal O})$
of a toric variety $X$ is equal to its restriction to the arc space
$T({\mathcal K})$.
The latter contains countably many
orbits of the maximal compact subgroup
$T({\mathcal O}) \subset T({\mathcal K})$ that are
parametrized by the elements $n$ of the lattice $N$.
%
The stringy motivic integral over a $T({\mathcal O})$-orbit corresponding
to an element $n \in N$ is equal to $(\L-1)^r \L^{\omega_X(n)}$
where $(\L-1)^r$ is the stringy motivic volume of the torus $T$
and $\L$ is the class of the affine line
in the Grothendieck ring K$_0({\rm Var}_\C)$ of algebraic varieties.
%
Our approach in the proof of the formula (\ref{Est-2})
is to use a more general bijection
$$
	G({\mathcal O}) \setminus (G/H)({\mathcal K}) \simeq N
 $$
which holds for any horospherical homogeneous space $G/H$,
see \cite{LV83} and \cite{GN10}.

\medskip

The paper is organized as follows.

\medskip

Section \ref{S:Mot}
contains a review of known facts about the spaces of arcs
of algebraic
varieties and their relation to motivic integrals
and stringy $E$-functions.
%
In Section \ref{S:Hor}, we collect basic results on horospherical
embeddings.
%
In Section \ref{S:Arc}, we prove that there is a bijection
between the quotient by $G(\O)$ of the intersection $X(\O) \cap  (G/H)(\K)$
and  the set of lattice points $|\Sigma| \cap  N$
for any horospherical $G/H$-embedding (cf.\,Theorem \ref{t:lattX}).
%
Section\,\ref{S:Est} is devoted to the formula
which expresses the stringy motivic volume
of any $\Q$-Gorenstein horospherical variety
as a sum over lattice points $n \in N \cap |\Sigma|$ (cf.\,Theorem \ref{t:main}).
%
We use this formula to obtain a smoothness criterion for locally factorial horospherical embeddings
in Section \ref{S:Smo}  (Theorem \ref{t:smo}).
%
Section \ref{S:App} contains some applications,
examples, open questions and a conjecture related to our results.



\medskip

\noindent
{\bf Acknowledgments}:
We would like to thank
M. Brion, B. Pasquier and D. Timashev for useful discussions and A. Szenes for his comments.
%
We are also indebted to the referee for his numerous and judicious comments
and his careful attention to our paper.

%
Our work was partially supported
by the DFG-project  "Geometrie und Kombinatorik von Toruswirkungen auf
algebraischen Variet\"{a}ten"
and by the ANR-project
10-BLAN-0110.







%%%%%%%%%%%%%%%%%%%%%%%%%%%%%%
%%%%%%%%%%%%%%%%%%%%%%%%%%%%%%
%%%%
%%%%
\section{Arc spaces, motivic integration and stringy motivic volumes}      \label{S:Mot}
%%%%
%%%%
%%%%%%%%%%%%%%%%%%%%%%%%%%%%%%
%%%%%%%%%%%%%%%%%%%%%%%%%%%%%%


Interesting invariants of a singular algebraic variety
$X$ can be obtained via
the nonarchimedean motivic integration over its space
of arcs  $\mathcal{J}_{\infty}(X)$.

\medskip

Here we recall the basic definitions on the arc space of an algebraic
variety and
refer the reader to \cite{DL99}, \cite{M01} or \cite{EM09} for
more details concerning  this topic.
%
Let $X$ be an algebraic variety  over $\C$.
For any $m \ge 0$, we denote by
$\mathcal{J}_m(X)$ the {\em $m\textrm{-}{\rm th}$ jet scheme} of $X$ over $\C$
whose $\C$-valued points are all morphisms of  schemes
$\Spec \C [ t ]/ (t^{m+1}) \to X.$
One has  $\mathcal{J}_0(X) = X$ and   $\mathcal{J}_1(X) = {\rm T}X$ is
the total space
of the tangent bundle over  $X$.
%
For $m \ge n$, the natural ring homomorphism
$\C[t]/ (t^{m+1})  \to \C[t]/(t ^{n+1})$
induces truncation morphisms
$$�\pi_{m,n} \, : \,  \mathcal{J}_m(X) \longrightarrow \mathcal{J}_n(X) . $$
The truncation morphisms
form a projective system
whose projective limit is an infinite dimensional
scheme $\mathcal{J}_\infty(X)$ over $\C$.
The scheme $\mathcal{J}_\infty(X)$ is called the {\em arc space} of $X$,
and the $\C$-valued points of $\mathcal{J}_\infty(X)$ are all morphisms
$\Spec \C [ [ t ]  ]  \to X .$
For each $m$, there is a natural  morphism
$$�\pi_m \, : \mathcal{J}_\infty(X) \longrightarrow \mathcal{J}_m(X) $$
induced by the ring homomorphism
$\C [ [ t ]  ]  \to \C[ [ t ] ]/ (t^{m+1}) \simeq
\C[ t ] /(t^{m+1}).$

\medskip

%
The motivic integration over the
arc space of a smooth variety is due to Kontsevich \cite{Ko95}.
One of its generalizations for singular varieties
was suggested by Denef and Loeser in \cite{DL99}.
Another generalization motivated by stringy invariants
was proposed in \cite{Ba98}; see also \cite{Cr} and~\cite{V}.
%

Let $\V$ be the category of complex algebraic varieties
and denote by K$_{0}(\V)$ the Grothendieck ring of $\V$.
For  an element $X$ in $\V$
we denote by $[X]$ its  class in K$_{0}(\V)$.
The symbol $\L$ stands for the class of the affine line $\A^1$
and we denote by $1$ the class of $\Spec \C$.
For example,
$$
	[�\mathbb{P}^n ] = \L^n + \L^{n-1} + \cdots + \L + 1  .
$$
The map $X \mapsto [X]$ naturally
extends to the category of constructible algebraic sets.
%
There is a natural function, $\dim : {\rm K}_0(\V) \to \Z \cup \{�\infty \}$,
which can be extended to the localization $\M := {\rm K}_{0}(\V)[\L^{-1}]$
of K$_{0}(\V)$ with respect to $\L$ simply
by setting $\dim(\L^{-1}) := -1$.
For any $m \in \Z$, set $F^m \M : = \{ \tau \in \M \ | \ \dim \tau \le m\}$.
Then $\{ F^m \M\}_{m \in \Z}$ is a decreasing filtration
of $\M$ and we denote by $\hat{\M}$ the separated completion of $\M$
with respect
to this filtration.

\medskip

Let  $X$ be a $d$-dimensional smooth variety.

\begin{defi}  \label{d:cyl}

A subset $C$ in $\mathcal{J}_\infty(X)$ is called {\it a cylinder} if
there are $m \in \N$
and a constructible subset $B_m \subseteq \mathcal{J}_m(X)$
such that $C = \pi_m^{-1}( B_m)$.
Such a set $B_m$ is called a {\em $m$-base} of $C$.

If $C \subseteq \mathcal{J}_\infty(X)$ is a cylinder
with $m$-base $B_m \subseteq \mathcal{J}_m(X)$,
we define its {\em motivic measure} $\mu_X(C)$ by
%
$$
	\mu _{X}(C) : = [ B_m ] \L^{- m d}
		= [\pi_m(C) ] \L^{- md} \ \in \, {\rm K}_0(\V).
$$

\end{defi}

This definition does not depends on $m$:
Indeed, because $X$ is smooth, the map
$$
	\pi_{n,m} \, : \,  \pi_{n}(C) \to \pi_m(C)
$$
is a locally trivial $\A^{(n-m)d}$-bundle for any $n \ge m$.
%
The collection
of cylinders forms an algebra of sets which means that $\mathcal{J}_\infty(X)$
is a cylinder and if $C,C'$ are cylinders, then also are
$\mathcal{J}_\infty(X) \smallsetminus  C$
and $C \cap C'$.
On the set on cylinders, the measure $\mu_X$ is additive on
finite disjoint unions.
Furthermore, for cylinders $C \subseteq C'$, one has
$\dim \mu_X(C) \le \dim \mu_X(C')$.

\begin{defi}

A subset $C \subset \mathcal{J}_\infty(X)$ is called {\em measurable}
if for all $n \in \N$ there is a cylinder $C_n$ and cylinders
$D_{n,i}$ for $i \in \N$ such that
$$
	C \bigtriangleup C_n \subseteq \bigcup_{i \in \N} D_{n,i}
$$
and $\dim \mu_X(D_{n,i}) \le - n$ for all $i$.
Here $C \bigtriangleup C_n = (C\smallsetminus C_n) \cup (C_n \smallsetminus C)$
denotes the symmetric difference of two sets.

If $C$ is measurable, we define its {\em motivic measure} $�\mu_X(C)$  by
$$
	\mu_X(C) :=	\lim_{n \to \infty} \mu_X(C_n) .
$$
This limit converges in $\hat{\M}$
and is independent of the $C_n$'s, cf.~\cite[Theorem 6.18]{Ba98}.

\end{defi}


\begin{prop} [{\cite[Prop.\,6.19 and 6.22]{Ba98}}]    \label{p:mot}

{\rm (i)} The measurable sets form an algebra of sets
and the motivic measure $\mu_X$ is additive on finite disjoint unions.
If $(C_i)_{i \in \N}$ is a disjoint sequence of measurable sets such that
$\lim_{i \to \infty} \mu_X(C_i) = 0$,
then $C: = \bigcup_{i\in\N} C_i$ is measurable and
$$\mu_X(C) = \sum_{i \in \N} \mu_X(C_i) .
$$

\smallskip

{\rm (ii)} Let $Y \subseteq X$ be a locally closed subvariety.
Then $\mathcal{J}_\infty(Y)$ is a measurable subset of $\mathcal{J}_\infty(X)$
and if $\dim Y < \dim X$ then $\mu_X(\mathcal{J}_\infty(Y))=0$.

\end{prop}


\begin{defi}

A function $F :  \mathcal{J}_\infty(X)  \to \Z \cup \{+\infty \}$
is called {\em measurable}
 if $F^{-1}(s)$ is measurable for all $s \in \Z \cup \{+\infty \}$.

\end{defi}
%
Let $A \subseteq \mathcal{J}_\infty(X)$ be a measurable set and
$F : \mathcal{J}_\infty(X) \to \Z \cup \{+\infty\}$ a measurable function
such that $\mu_X(F^{-1}(+\infty)) = 0$.
Then we set
$$	
	\int_{A} \L ^{-F} {\rm d}\mu_X :=
	\sum\limits_{s \in \Z} \mu_X( A \cap F^{-1}(s)) \L^{-s}
$$
%\qA{Why do we define the integral like that,
%$\L^F$, and not simply $F$?}
in $\hat {\M}$ whenever the right hand side converges in $\hat{\M}$.
In this case, we say that $\L^{-F}$ is {\em integrable on} $A$.
%
To any subvariety $Y$ of $X$, one associates the {\em order function}
$$�
	\ord_Y \, : \,  \mathcal{J}_\infty(X) \to \N \cup \{\infty \}
$$
sending an arc $\nu \in \mathcal{J}_\infty(X)$ to the order of
vanishing of $\nu$ along $Y$.
An important example of an integrable function is the  function
$\L^{-\ord_Y}$ where $Y$ is
a  smooth hypersurface in $X$.


\medskip


We consider now the case where $X$ is a singular
normal irreducible variety.
Let $K_X$ be a canonical divisor of $X$.
%
Assume that $X$ is $\Q$-Gorenstein,
that is $m K_X$ is Cartier for some $m \in \N$.
Let $f :  X' \to X$ be a resolution of singularities of $X$
such that the exceptional locus of $f$ is a divisor 	
whose irreducible components $D_1,\ldots,D_l$ are smooth divisors
with only normal crossings,
and set
$$�
	K_{X'/X} := K_{X'} - f^*K_X = \sum\limits_{i=1}^l \nu_i D_i,
$$
where the rational numbers $\nu_i$ $(1 \leq i \leq l)$ are called
the discrepancies of $f$.
The rational numbers $\nu_i$  $(1 \leq i \leq l)$ can be computed  as follows.
Since $m K_X$ is Cartier,
we can consider $f^* (m K_X)$ as a pullback of the Cartier divisor
and write
$$
	m K_{X'} - f^* (m K_X) = \sum\limits_{i=1}^l n_i D_i
$$
with $n_i \in \Z$ for all $i$.
Then $K_{X'/X}$ can be viewed as an abbreviation of
the $\Q$-divisor $\sum\limits_{i=1}^l \nu_i D_i$
where $\nu_i := \frac{n_i}{m} $ for all $i$.
Assume further that $X$ has at worst log-terminal singularities,
that is $\nu_i > -1$ for all $i$ (cf.\,\cite{KMM}).
Set $I :=\{1,\ldots, l \}$ and for any subset $J \subseteq I$,
$$
	D_J : =
		\left\lbrace
		\begin{array}{ll}
			\bigcap_{j\in J} D_J & \textrm{if } J\not=\varnothing \\
			Y  & \textrm{if } J=\varnothing
		 \end{array}
		\right.
	\quad \textrm{ and  }\quad
	D_J^0 : = D_J \smallsetminus \bigcup \limits_{j \in I \smallsetminus J} D_j.
$$
%

\begin{defi}    \label{d:Est}

We define the {\em stringy motivic volume} $\E_{\rm st}(X)$
of $X$ by
$$
	 \E_{\rm st}(X) := \sum\limits_{J \subseteq \{1, \ldots , l \}}
[D_J^0] \prod\limits_{j \in J}
		\, \displaystyle{\frac{  \L - 1 }{ \L^{\nu_j+1} - 1}}
\, \in \, \hat{\M} (\L^{\frac{1}{m}}) .
$$
%
(In \cite{V}, the element $\E_{\rm st}(X)$ is also called the
{\em stringy $\E$-invariant} of $X$.)



\end{defi}

\medskip

The inequality $\nu_i > -1$ for any $i$ implies that
the function $\ord_{K_{X'/X}} : = \sum_{i=1}^l \nu_i \, \ord_{D_i}$
is integrable on $\mathcal{J}_\infty(X')$, see \cite[Theorem 6.28]{Ba98}.
%
So we can express $\E_{\rm st}(X)$ as a motivic integral:

\begin{prop}      \label{p:Est}

$$
	\E_{\rm st}(X) =
		\int_{\mathcal{J}_\infty(X')} \L^{-\ord_{K_{X'/X}}} \, {\rm d}\mu_{X'}
		\, \in \, \hat{\M} (\L^{\frac{1}{m}}) .
$$

\end{prop}

The crucial point is that the above expressions of $\E_{\rm st}(X)$
do not depend on the chosen resolution,
see \cite[Theorem 3.4]{Ba98}.
This relevant fact essentially comes from the transformation rule
for motivic integrals, see \cite{DL99}.


\medskip


Recall that the \emph{$E$-polynomial} of an arbitrary
$d$-dimensional complex
algebraic variety $Z$ is defined by
$$
	E(Z ; u,v) := \sum_{ p,q=0 }^d \sum_{i=0}^{2d}
		(-1)^{i} h^{p,q} ({\rm H}^{i}_{c}(Z ; {\C})) u^{p}v^{q} \, ,
$$
where $h^{p,q} ({\rm H}^{i}_{c}(Z ; {\C}))$ ($0 \le i \le 2d$)
is the dimension
of the $(p,q)$-type Hodge component in the $i$-th cohomology group
${\rm H}^{i}_{c}(Z ; {\C})$ with compact support.
%
The polynomial $E$ has properties similar to the ones of the usual Euler characteristic.
In particular, the map $Z \mapsto E(Z ; u,v)$ factors through the ring K$_{0}(\V)$.
The map $Z \mapsto E(Z ; u,v)$ extends to $\M$ by setting $E(\L^{-1} ; u,v) := (uv)^{-1}$.
So, we get a map from $\M$ to $\Z[u,v, (uv)^{-1}]$
which uniquely extends to  $\hat{\M}$. %(see \cite[Prop.\,3.3]{Cr}).
This extension will be again denoted by $E$.


\smallskip

\begin{defi}    \label{d:Est2}

The {\em stringy $E$-function} of $X$ is given by (cf.\,\cite{Ba98}):
$$
	E_{\rm st}(X ; u , v)
 		:=  \sum\limits_{J \subseteq \{1, \ldots , l \}} E(D_J^0;u,v) \prod\limits_{j \in J}
		\displaystyle{\frac{  uv-1 }{ (uv)^{\nu_j+1} -1 }}  \, .
$$

\end{defi}
%

\noindent
Note that  $E_{\rm st}(X;u,v) = E(\E_{\rm st}(X) ; u,v )$.

\medskip


\begin{rem}

Whenever $X$ is smooth, then $\E_{\rm st}(X)=\mu_X(\mathcal{J}_\infty(X))=[X]$
and $E_{\rm st}(X ; u,v ) = E(X ; u,v )$.

\end{rem}



\bigskip

%%%%%%%%%%%%%%%%%%%%%%%%%%%%%%
%%%%%%%%%%%%%%%%%%%%%%%%%%%%%%
%%%%
%%%%
\section{Horospherical varieties}          \label{S:Hor}
%%%%
%%%%
%%%%%%%%%%%%%%%%%%%%%%%%%%%%%%
%%%%%%%%%%%%%%%%%%%%%%%%%%%%%%

In this section, we use our notations from the introduction: $G$ is a connected reductive group over $\C$,
$H \subset G$ is a closed horospherical subgroup, $G/H$ is the corresponding
horospherical homogeneous space,   $U$ is a maximal unipotent subgroup in
$G$ such that $U \subseteq H$, $B := N_G(U)$ is the corresponding Borel subgroup of $G$,
$P:=N_G(H)$ is a parabolic subgroup, $T:=P/H$ a $r$-dimensional algebraic torus, $M$ is
the group of characters of $T$, and $N:= {\rm Hom}(M, \Z)$.
%
%\begin{defi}
%
%Recall %from the Introduction
%that a closed subgroup $H \subset G$ is called {\em horospherical} if
%$H$ contains a maximal unipotent subgroup of $G$.
%If so, the corresponding homogeneous space .
%
%\end{defi}
%
%Choose a Borel subgroup $B$ of $G$ with  (horospherical) closed subgroup of $G$ containing $U$.
%Then, adopt the related notations as in the Introduction.



\smallskip


Let $S$ be the set of simple roots of $(G,B)$
with respect to a maximal torus of $B$.
There is a bijective map $I \mapsto P_I$
sending a subset $I$ of $S$ to the parabolic subgroup $P_I$ of $G$ containing $B$
such that $P_I = B W_I B$, where $W_I \subseteq W$ is the subgroup of the
Weyl group $W=W_S$ generated by the reflections $s_\alpha$ ($\alpha \in I)$. In particular,
 one has $P_\varnothing = B$ and $P_S=G$.
From now on, we denote by $I$ the subset of $S$ corresponding to $P:=N_G(H)$.
%
Let $U_0 \subset G/P$ be the open dense $B$-orbit.
Then $U_0$ is isomorphic to an affine space and the Picard
group of $G/P$ is free generated by
the classes $[\varGamma_\alpha]$
of irreducible
components $\{\varGamma_\alpha \ | \  \alpha \in S\smallsetminus I \}$
in the complement $(G/P) \smallsetminus U_0$. The space of global sections
$H^0( G/P, {\mathcal O}( \varGamma_\alpha))$ is an irreducible
representation of the universal cover of the semisimple group $G':= [G,G]$ corresponding to the fundamental weight $\varpi_\alpha$ associated with $\alpha \in S\smallsetminus I$.
Let $\phi \, : \, G/H \to G/P$ be the canonical surjective morphism whose fibers are isomorphic to
the torus $T$.
Then the divisors $\varDelta_\alpha := \phi^{-1}(\varGamma_\alpha)$,
for $\alpha \in S\smallsetminus I$,
are exactly the irreducible components in the complement to the
open dense $B$-orbit $\widetilde{U}_0 \simeq U_0 \times T$ in $G/H$.
The lattice $M$ can be identified with the group
$\C[\widetilde{U}_0]^* / \C^*$
of invertible
regular functions over $\widetilde{U}_0$ modulo nonzero constant functions.

\begin{defi}
A normal $G$-variety $X$ is said to be {\em horospherical}
if $G$ has an open orbit in $X$ isomorphic to the horospherical homogeneous space $G/H$.
In that case, $X$ is also called a {\em $G/H$-embedding}.
\end{defi}


Horospherical varieties are special examples of
spherical varieties.
%This fact results from the Bruhat decomposition.
%
According to the Luna-Vust theory \cite{LV83}, any
%$G$-equivariant
$G/H$-embedding $X$
can be described by  a colored fan $\Sigma$ in
the $r$-dimensional vector space $N_\R:= N \otimes_\Z \R$.
Our basic reference for spherical varieties is \cite{K91}.
For recent accounts about horospherical varieties,
see also \cite[Chap.\,1]{Pa07} or \cite[Chap.\,5]{Ti10}.
%


Let $X$ be a horospherical $G/H$-embedding.
Each irreducible divisor $D$ in $X$ defines a valuation $v_D : \C(X)^* \to \Z$
on the function field $\C(X)$ which vanishes on $\C^*$.
The restriction of $v_D$ to the lattice $M \simeq \C[\widetilde{U}_0]^* / \C^*$
yields an element $\varrho_D$ of the dual lattice $N$.


Let ${\mathcal X}(P)$ be the character group of the parabolic subgroup
$P=P_I$. This group
can be identified with the set of all characters $\chi \in {\mathcal X}(B)$ of the Borel
subgroup $B$
such that $\langle \chi, \check{\alpha} \rangle =0$
for all $\alpha \in I$ where $\check{\alpha} \in {\rm Hom}({\mathcal X}(B), \Z)$ denotes the coroot corresponding to $\alpha$.
Since every character of $P$ induces a line bundle
over $G/P$,  we get a homomorphism ${\mathcal X}(P) \to
{\rm Pic}(G/P)$.
Its composition with the monomorphism of character
groups $M \to {\mathcal X}(P)$,
induced by the epimorphism $P \to T = P/H$,
gives a homomorphism
$\delta \, : \, M  \to {\rm Pic}(G/P)$.
%
Let $\delta^* \, : \, {\rm Pic}(G/P)^* \to  N$ be the dual map.
Then, the lattice points
$\{\varrho_{\varDelta_\alpha} \ | \ \alpha \in S \smallsetminus I \} \subset N$
corresponding to the divisors $\varDelta_\alpha \subset X$,
$\alpha \in S \smallsetminus I$,
are exactly the $\delta^*$-images of the dual basis to $\{[\varGamma_\alpha] \ | \  \alpha \in S \smallsetminus I \}$
in $ {\rm Pic}(G/P)^*$.
%
For simplicity, we set $\varrho_\alpha := \varrho_{\varDelta_\alpha}$
for any $\alpha \in S \smallsetminus I$. We note that $\varrho_\alpha$ is equal to the restriction
to the sublattice $M \subseteq {\mathcal X}(B)$ of the corresponding
coroot $\check{\alpha}$.
%

Let  $\D_X = \{D_1, \ldots, D_t \}$ be the set of $G$-stable irreducible divisors of $X$.
For any divisor $D_i$, we denote by $\varrho_i$ the lattice point $\varrho_{D_i} \in N$.
Thus, we get a map
$$
	\varrho  \, : \,  \{\varDelta_\alpha \ | \ \alpha \in�S \smallsetminus I \}
		\cup \D_X \to N
$$
which sends $\varDelta_\alpha$ ($\alpha \in S \smallsetminus I$)  to $\varrho_\alpha$
and $D_i \in \D_X$ ($1 \le i \le t$) to $\varrho_i$.
The restriction of $\varrho$ to $\D_X$ is injective,
but in general the restriction of $\varrho$ to $\{�\varDelta_\alpha \ | \ \alpha \in�S \smallsetminus I \}$ is not injective.


\medskip

Let $Z$ be a $G$-orbit in $X$.
Denote by $X_Z$ the union of all $G$-orbits in $X$ which contain $Z$ in their closure.
Then $X_Z$ is open in $X$. Moreover, $X_Z$ is  a $G/H$-embedding having $Z$ as a unique
closed $G$-orbit. Such a  $G/H$-embedding is called {\em simple}.
%
It is well-known that any simple embedding is quasi-projective.
This fact follows from a result of Sumihiro \cite[Lemma 8]{Su74} which
states that
any normal $G$-variety is covered by $G$-invariant
quasi-projective open subsets  (if $X$ is a simple embedding
of $G/H$ with closed $G$-orbit $Y$, then any $G$-stable open neighborhood
of $Y$ in $X$ is the whole $X$).
%
The {\em colored cone corresponding to }$Z$ is the pair $(\sigma_{Z},\F_{Z})$ where
$\F_{Z}$ is the set $\{ \alpha \in S \smallsetminus I \ | \
\overline{\varDelta_\alpha} \supset Z\}$
and $\sigma_{Z}$ is the convex cone in  $N_\R$ generated by
$\{�\varrho_\alpha \ | \ \alpha \in \F_{Z} \}$ and
$\{ \varrho_i \ | \ D_i \supset Z \}$.
%
The {\em colored fan $\Sigma$ of $X$} is
the collection of the colored cones $(\sigma_{Z} , \F_{Z})$
where $Z$ runs through the set of $G$-orbits of $X$.
%
We call $\F : = \bigcup \, \F_{Z}$ the set of {\em colors of $X$}.
%


The set of colored cones in the colored fan $\Sigma$
is a partially ordered set:
We write $(\sigma',\F') \leq (\sigma,\F)$
and call $(\sigma',\F')$ a {\em face} of $(\sigma,\F)$
if $\sigma'$ is a face of $\sigma$
and $\F'   = \{ \alpha \in \F \ | \ \varrho_\alpha \in \sigma' \}$.
%
On the other hand, we have a partial order on the set of orbits,
$\big( Z \leq  Z' \iff Z \subseteq \overline{Z'} \big)$,
and the map $Z \mapsto (\sigma_{Z},\F_{Z})$ is an order-reversing bijection
between the set of orbits of $X$ and
the set of colored cones,~\cite{K91}.
%
Denote by $Z_{\sigma,\F}$ the $G$-orbit of $X$ corresponding
to $(\sigma,\F)$.
The open orbit $G/H$ corresponds to the cone $(0,\varnothing)$.
%

A arbitrary  pair $(\sigma,\F)$ consisting of a convex rational polyhedral
cone $\sigma \subset N_\R$
and a subset $\F \subset S \smallsetminus I$
is said to be a {\em strictly convex colored cone}
if $\sigma$ is strictly convex (i.e. $-\sigma \cap \sigma$ =0)
and if $\varrho_\alpha$ is a nonzero element in $\sigma$ for any $\alpha \in \F$.
A {\em colored fan} $\Sigma \subset N_\R$ is a collection of strictly convex
colored cones such that all faces of any colored cone $(\sigma, \F) \in
\Sigma$ belong to $\Sigma$
and the intersection of two colored cones is a common face of both cones, \cite[Section 3]{K91}.
%
The following result
was proved by Luna-Vust
in a more general context, \cite[Proposition 8.10]{LV83}
(see also \cite[Theorem 3.3]{K91}):

\begin{thm}   \label{t:bij}

The correspondence $X \to \Sigma$
is a bijection between $G$-equivariant isomorphism classes of $G/H$-embeddings
$X$ and colored fans $\Sigma$ in $N_\R$.

\end{thm}

We denote by $X_\Sigma$ the $G$-equivariant $G/H$-embedding
corresponding to a colored fan $\Sigma \subset N_{\R}$.
For simplicity, we denote $X_\Sigma$ by  $X_{\sigma,\F}$
whenever $\Sigma$ has only one maximal colored cone $(\sigma,\F)$.
The latter happens if and only if $X$ has a unique closed $G$-orbit,
i.e., $X$ is simple.


A horospherical $G/H$-embedding $X$
whose fan $\Sigma$ has no colors is said to be {\em toroidal}.
%
There is a simple method to construct a toroidal horospherical
variety associated with the (uncolored) fan $\Sigma$.
One considers the toric $T$-embedding
$Y_\Sigma$ with fan $\Sigma$.
Using the canonical epimorphism $P \to T$
we can consider $Y_\Sigma$ as a $P$-variety.
Then $X_\Sigma$ is isomorphic to the quotient
space $(G \times Y_\Sigma )/P$
where the action of $P$
on $G \times Y_\Sigma$
is given by
$p(g,y) := (g p^{-1}, py)$
for any $p \in P$, $g \in G$ and $y \in Y_\Sigma$.
%
One has a natural surjective morphism
$\phi : X_\Sigma \to G/P$
whose fibers are isomorphic to the toric variety $Y_\Sigma$
and $X \simeq X_\Sigma$.
Over the open dense $B$-orbit $U_0$ in $G/P$ the fibration
$\phi : \phi^{-1}(U_0) \to U_0$ is trivial.
%
Every toroidal horospherical variety is obtained as $(G_\Sigma \times Y)/P$
for a unique toric variety $Y_\Sigma$.
Moreover, $X_\Sigma$ is simple if and only if $Y_\Sigma$ is affine.

\smallskip

Each horospherical variety is dominated by a toroidal variety
in the following sense \cite[$\negmedspace$\S3.3]{Br91}:

\begin{prop}    \label{p:dom}

For any horospherical $G$-variety $X$,
there is a toroidal $G$-variety $\tilde{X}$ and a proper birational $G$-equivariant morphism
$$f \, :  \, \tilde{X}  \to X  .$$

\end{prop}
%
%
To obtain this toroidal variety $\tilde{X}$,
we just need to remove all colors from all colored cones in the fan of $X$.
It is worth mentioning that $\tilde{X} = (G \times Y )/P$, where $Y$ denotes the closure of $T$ in $X$.

In general, the toroidal variety $\tilde{X}$ is not smooth, but its singularities are locally isomorphic
to toric singularities.
In the sequel, it will useful to use a resolution of singularities
$f'  \, :  \, X'  \to X$, where $f'$
if a proper birational $G$-equivariant morphism
and where $X'$ is  a smooth toroidal $G$-equivariant embedding
with (uncolored) fan $\Sigma'$
obtained from $\Sigma$ by removing colors in all colored cones of $\Sigma$
and subdividing them into subcones generated by parts of $\Z$-bases of the lattice  $N$.
Note that the fans $\Sigma'$ and $\Sigma$ share the same support $|\Sigma|$.

\medskip

Next proposition describes the stabilizer of $G$-orbits
$Z_{\sigma,\F}$ in the horospherical case:

\begin{prop}    \label{p:orb}
%
Let $X$ be a horospherical $G/H$-embedding where $P:=N_G(H) = P_I$ is
the parabolic subgroup corresponding to a subset $I \subseteq S$.
Consider a colored cone  $(\sigma,\F) \in \Sigma$  $(\F \subseteq S \setminus I)$.
Define the sublattice $M_\sigma  : = M \cap \sigma^\perp$ consisted of
all elements in $M$ that are orthogonal to $\sigma \subset N_\R$.
Then every element $m \in M_\sigma$ defines a character $\chi_m$ of the
parabolic subgroup $P_{I \cup \F}$, and
the closed $G$-orbit $Z_{\sigma,\F} \subseteq X_{\sigma,\F}$
is isomorphic to $G/ H_{\sigma,\F}$
where
\[ H_{\sigma,\F} := \{ g \in P_{I \cup \F} \; | \;
\chi_m(g) =1 \;\; \forall m \in M_\sigma \}. \]
%
In particular, one has:
$$ \dim Z_{\sigma,\F} = {\rm rk} \, M_\sigma + \dim G/P_{I \cup \F}.$$
\end{prop}



\begin{proof}
First of all we recall that the nonzero elements
$\varrho_\alpha$ $(\alpha \in {\mathcal F})$
are the restrictions of the coroots $\check{\alpha}$ to the sublattice
$M \subseteq {\mathcal X}(B)$.
Since  $\varrho_\alpha \in \sigma$ for all $\alpha
\in {\mathcal F}$, the restriction of the coroot $ \check{\alpha}$ to $M_\sigma$
is zero for all $\alpha \in {\mathcal F}$.
The inclusions $M_\sigma \subseteq M \subseteq {\mathcal X}(P_I)$ imply
that the restriction of the coroot $\check{\alpha}$ to $M_\sigma$
is zero for all $\alpha \in I$, too.
Hence, we can consider the elements of $M_\sigma$ as characters of $B$ that
extend to the parabolic subgroup $P_{I \cup \F}$.


Without loss of generality, we can assume that $X = X_{\sigma,\F}$ is the simple
horospherical $G/H$-embedding corresponding to a colored
cone $(\sigma,\F)$.
%
Consider the  proper birational $G$-equivariant morphism
$f : X_{\sigma,\varnothing} \to X_{\sigma,\F}$
where $X_{\sigma,\varnothing}$ is the simple  toroidal variety
associated with the uncolored cone $({\sigma,\varnothing})$,
i.e.,  $X_{\sigma,\varnothing}$ is exactly the variety
$\widetilde{X_{\sigma,\F}}$ in the notations of Proposition \ref{p:dom}.

Then the toroidal simple horospherical variety $X_{\sigma,\varnothing}$
is a fibration
over $G/P$ with the affine toric fiber $Y_\sigma$. We remark
that  $f$ induces a bijection
between the set of $G$-orbits in $X_{\sigma,\varnothing}$ and
 the set of $G$-orbits in $X_{\sigma,\F}$.
It immediately follows
from the theory of toric varieties that the closed $T$-orbit
$Z_\sigma$ in $Y_\sigma$
is isomorphic to $T/T_{\sigma}$ where the subtorus $T_\sigma
\subseteq T$ is the kernel
of characters of $T$ in the sublattice $M_\sigma = M \cap \sigma^\perp$
of $M$.
Moreover, $Z_{\sigma, \varnothing}:= f^{-1}(Z_{\sigma, \F})$ is
 the closed $G$-orbit in $X_{\sigma,\varnothing}$ which is
isomorphic to $G \times_P (T/T_\sigma) $.  
%
% $(G \times_P (T/T_\sigma)$
%%%%%%% I prefer to use this above notation as before.
%
This implies that
 the closed $G$-orbit
 $Z_{\sigma, \varnothing}$ is
isomorphic to $G/H_{\sigma,\varnothing}$ where
\[ H_{\sigma,\varnothing} := \{ g \in P=P_{I} \; | \;
\chi_m(g) =1 \;\; \forall m \in M_\sigma \}. \]


Let $z_0 \in Z_{\sigma, \varnothing}$ be a point with the stabilizer
$H_{\sigma,\varnothing}$. Then the stabilizer of $f(z_0) \in Z_{\sigma,\F}$
is a subgroup $H_{\sigma,\F} \subseteq G$ containing $H_{\sigma,\varnothing}$ so that we 
have the isomorphism $Z_{\sigma,\F} \cong
G/H_{\sigma,\F}$. 
We remark that all fibers of the proper birational $G$-equivariant morphism
$f : X_{\sigma,\varnothing} \to X_{\sigma,\F}$ are  connected
and proper.  In particular, $f$ induces a proper $G$-equivariant surjective
morphism of the $G$-orbits  $Z_{\sigma, \varnothing} \to  Z_{\sigma,\F}$
whose fibers are connected proper algebraic varieties isomorphic to
$H_{\sigma,\F}/H_{\sigma,\varnothing}$. 
Since the horospherical subgroup $H_{\sigma,\F}$ contains the horospherical subgroup $H_{\sigma,\varnothing}$, 
the normalizer $N_G( H_{\sigma,\F}) = : P_1$  contains the normalizer $N_G(H_{\sigma,\varnothing})=P$. 
%
%$H_{\sigma,\varnothing}$ is contained in the horospherical subgroup $H_{\sigma,\F}$, the normalizer $N_G(H_{\sigma,\varnothing})=P$ is contained
%in the normalizer $N_G( H_{\sigma,\F}) =P_1$ 
%(we give a proof of this fact at the end). 
%
Indeed, we have that $H_{\sigma,\varnothing} \supseteq [P,P]$ since $P/H_{\sigma,\varnothing}$ is commutative. 
It follows that $P_1 = B[P_1,P_1] = B H_{\sigma,\F} = P H_{\sigma,\F} \supseteq P$. 
%
Let $H'$ be the
intersection  $H_{\sigma,\F} \cap P$. The inclusions
\[H_{\sigma,\varnothing}  \subseteq H' \subseteq H_{\sigma,\F} \]
enable to decompose the proper morphism $f \, : \, G/H_{\sigma,\varnothing} \to
G/H_{\sigma,\F}$ into the composition of two proper morphisms with
connected fibers:
\[ f_1 \, : \, G/H_{\sigma,\varnothing} \to
G/H', \;\;    \;\; f_2 \, : \,G/H' \to  G/H_{\sigma,\F}. \]
The inclusions
$$[P,P] \subseteq H_{\sigma,\varnothing}  \subseteq H' \subset P$$
imply that the fibers of $f_1$ are isomorphic to a diagonalisable
subgroup $H'/H_{\sigma,\varnothing}$ in the torus $P/H_{\sigma,\varnothing}$.
But $H'/H_{\sigma,\varnothing}$ is connected and proper only if it consists
of one point, i.e., we get $H' :=H_{\sigma,\F} \cap P =
H_{\sigma,\varnothing}$. 
Let $M_1 \subset {\mathcal X}(P_1)$
be the sublattice of all characters of $P_1$  that vanish on
$H_{\sigma,\F}$. Since $P/[P,P]$ is a torus with the group 
of characters  ${\mathcal X}(P)$, it follows from the properties 
of diagonalisable groups that there exists one-to-one correspondence 
between the sublattices in  the group of characters ${\mathcal X}(P)$ 
and the closed subgroups in $P$ containing $[P,P]$. Therefore, 
the equality $H_{\sigma,\F} \cap P =
H_{\sigma,\varnothing}$ and the injectivity of the restriction map ${\mathcal X}(P_1) \to
{\mathcal X}(P)$  imply that   $M_1$ is also the
sublattice of all characters of $P$  that vanish on
$H_{\sigma,\varnothing}$, i.e., we get the equality $M_1 =M_\sigma$.
%
%The injectivity of the restriction map ${\mathcal X}(P_1) \to
%{\mathcal X}(P)$ 
%and the equality of quotient groups
%$$P/H_{\sigma,\varnothing} = P/P\cap H_{\sigma,\F} 
%= P H_{\sigma,\F}/ H_{\sigma,\F} = P_1 / H_{\sigma,\F}, $$
%where the final equality holds since $P_1 = PH_{\sigma,\F}$, 
%implies that $M_1 =M_\sigma$. 

It remains to show that $P_1 = {P_{I \cup \F}}$. Since $P_1$ contains $P=P_I$, 
we get $P_1=P_J$ for some subset $J \subseteq S$ containing 
$I$. 
%
Let $\alpha \in S \smallsetminus I$. 
By the definition  of the set of colors $\F$,
the simple root $\alpha$ belongs to $\F$
if and only if the closure of the $B$-invariant divisor
$\Delta_\alpha:= \phi^{-1}(\varGamma_\alpha) \subset G/H$ 
in  $X_{\sigma, \F}$
contains the closed orbit $Z_{\sigma, \F} \subseteq X_{\sigma, \F}$.
On the other hand, the horospherical homogeneous $G$-space
$Z_{\sigma, \F}$ is a torus fibration over $G/P_J$,
and the intersection
of a closed $B$-invariant divisor
$\overline{\Delta_\alpha} \subset X_{\sigma, \F}$
with the closed $G$-orbit $Z_{\sigma, \F}$ is either a closed
$B$-invariant divisor  in  $Z_{\sigma, \F}$ (which projects to
a $B$-invariant divisor  in $G/P_J$), or the whole $G$-orbit
$Z_{\sigma, \F}$. 
%
%
The latter implies that $\overline{\Delta_\alpha}$ contains 
$Z_{\sigma, \F}$ (i.e., $\alpha \in \F)$ 
if and only if $\alpha \in J$. So we obtain  
$J = I  \cup \F$. 

%in $G/P_1$ is the set $\{ \varGamma_\alpha \, | \, \alpha \in S \setminus ( I
%\cup \F) \}$, hence $P_1 =  P_{I \cup \F}$. 
%because we have obtained a bijection
%between 
%
%corresponds the set of all roots $\alpha \in S \setminus ( I
%\cup \F)$. 



%Finally, we have to explain why an inclusion of two horospherical
%subgroups $H_1 \subseteq H_2$ implies the inclusion
%for the corresponding parabolic subgroups $P_1:= N_G(H_1) \subseteq P_2:=
%N_G(H_2)$.  Let $U \subset H_1$ be a maximal unipotent
%subgroup contained in $H_1$, and let $B:= N_G(U)$ be its normalizer. Then
%$B \subseteq P_1$, $B \subseteq P_2$ and $[P_1,P_1]
%\subseteq H_1$, $[P_2,P_2] \subseteq H_2$ (e.g. see the proof
%of Thm. 3.2 in \cite{KK11}). So we obtain $B[P_1,P_1] \subseteq P_2$.
%Since the subgroup $B[P_1,P_1]$ contains $P_1$ we get $P_1 \subseteq P_2$.

\end{proof}


\bigskip

%%%%%%%%%%%%%%%%%%%%%%%%%%%%%%
%%%%%%%%%%%%%%%%%%%%%%%%%%%%%%
%%%%
%%%%
\section{Arcs spaces of horospherical varieties}     \label{S:Arc}
%%%%
%%%%
%%%%%%%%%%%%%%%%%%%%%%%%%%%%%%
%%%%%%%%%%%%%%%%%%%%%%%%%%%%%%


Let $\K := \C ( ( t ) )$ be the field of formal Laurent series,
and let $\O := \C [ [ t ] ]$ be the ring of formal power series.
If $X$ is a scheme of finite type over $\C$,
denote by $X(\K)$ and $X(\O)$ the sets of $\K$-valued points
and $\O$-valued points of $X$ respectively.
Remark that the set $X(\O)$ coincides
with the set of $\C$-points of the scheme $J_\infty(X)$.
If $X$ is a normal variety admitting an action of an algebraic group $A$,
then $X(\K)$ and $X(\O)$ both admit a canonical action of the group $A(\O)$
induced from the $A$-action on $X$.


\smallskip


The following result can be viewed as a generalization
in a slightly different context of \cite[\S8.2]{GN10}
(see also \cite{LV83} or \cite{Do09}):

\begin{thm}   \label{t:lattX}

Let $X$ be a horospherical $G/H$-embedding
defined by a colored fan $\Sigma$.
We consider the two sets $X(\O)$ and $(G/H)(\K)$
as subsets of $X(\K)$.
Then there is a surjective map
$$ \mathcal{V} \, : \, X(\O) \cap
(G/H)(\K)  \longrightarrow  |\Sigma| \cap  N
$$
whose fiber over any $n \in |\Sigma| \cap N$ is precisely one $G(\O)$-orbit.
In particular, we obtain
a one-to-one correspondence between the lattice points in $|\Sigma| \cap N$
and the $G(\O)$-orbits in $X(\O) \cap (G/H)(\K)$.

\end{thm}

In the special case where $X$ is a toric $T$-embedding,
Theorem \ref{t:lattX} is due to Ishii, \cite[Theorem 4.1]{Ish04}.
In more detail, by \cite[Theorem 4.1]{Ish04} (and its proof),
we have:

\begin{lemma}   \label{l:lattX}

%$\clubsuit\clubsuit\clubsuit$
Let $Y:=Y_\Sigma$ be a toric $T$-embedding
defined by a fan $\Sigma$.
%
For any $\K$-rational point $\lambda \in T(\K)$, we denote by  $\lambda^*$
the corresponding ring homomorphism
 $\lambda^* : \C[M] \to \K$ and
define
 the element $n_\lambda$ of the dual lattice
$N = {\rm Hom}(M,\Z)$ as the composition of
$\lambda^*|_M : M \to \K^*$ and the standard valuation map
$\ord :  \K^* \to \Z$.
%
Then the map
$$\nu\; : T(\K)  \rightarrow N,
\, \lambda \mapsto n_\lambda
$$
induces a canonical isomorphism $T(\K)/T(\O) \cong N$
and one obtains a surjective map
$$\nu\; : \; Y(\O) \cap  T(\K)  \rightarrow  |\Sigma| \cap  N,
\, \lambda \mapsto n_\lambda
$$
whose fiber over any $n \in |\Sigma| \cap N$ is precisely one $T(\O)$-orbit.
%


\end{lemma}

The above lemma will be used in the proof of Theorem \ref{t:lattX}:


\begin{proof}[Proof of Theorem \ref{t:lattX}]
%
%$\clubsuit\clubsuit\clubsuit$
Consider the canonial surjective morphism $\phi \, : \, G/H \to G/P$ whose
fibers are isomorphic to the algebraic torus $T:= P/H$. 
We consider $p_0:=[P]$
as a distinguished $\C$-point of $G/P$ such that  the fiber $\phi^{-1}(p_0) =T$
is the closed subvariety $P/H \subseteq G/H$.


Since $G/P$ is a projective variety, the valuative criterion of properness
implies that the natural map $(G/P)(\O) \to (G/P)(\K)$ from
$\O$-points of $G/P$
to $\K$-points of $G/P$ is bijective. 
It follows from  the local triviality of the map $G \to G/P$ 
that $(G/P)(\O) = G(\O)/P(\O)$. 
Thus, the group $G(\O)$ 
transitively acts on $G(\O)/P(\O) = (G/P)(\O) = (G/P)(\K)$. 


Let $\lambda \in
(G/H)(\K)$ be a $\K$-point of $G/H$. Then 
$\phi(\lambda) \in (G/P)(\K) = G(\O)/P(\O)$. 
So there exists an element $\gamma \in
G(\O)$ such that $\gamma(\phi(\lambda)) = p_0 \in (G/P)(\C) \subset
(G/P)(\K)$. 
Since the morphism $\phi\, : \, G/H \to G/P$ commutes with the left $G$-action, 
the equality $\gamma(\phi(\lambda)) = p_0 = [P]$ 
implies that $\gamma(\lambda) \in T(\K) = (P/H)(\K) \subset (G/H)(\K)$. 

Now we   set $n_\lambda := \nu({\gamma(\lambda)})$
where $\nu$ is
the map $T(\K) \to N = {\rm Hom}(M,\Z) \cong T(\K)/T(\O)$
defined by Lemma \ref{l:lattX}.
 It is easy to see that  the lattice point $n_\lambda$ does no depend
on the choice of the element $\gamma \in G(\O)$.
%
Indeed,
if $\gamma' \in G(\O)$ is another element such that
$\gamma'(\phi(\lambda)) = p_0$ then the equality 
$\gamma'(\phi(\lambda)) = \gamma(\phi(\lambda)) =p_0$ implies 
that the element $\delta:= \gamma' \gamma^{-1}$ 
belongs to  $P(\O)$ and its 
image under the homomorphism  $P \to T = P/H$
is contained in $T(\O)$. So, we obtain that  the $\K$-points  
$\gamma'(\lambda),
 \gamma(\lambda) \in T(\K)$ define the same element  $n_\lambda \in
N=T(K)/T(\O)$.
%
Finally,
we get a map $\mathcal{V} : (G/H)(\K) \to N, \,
\lambda \mapsto n_\lambda$
which is constant on $G(\O)$-orbits.


Denote by $\widetilde{X}$ the toroidal embedding
of $G/H$ corresponding the decolorization $\widetilde{\Sigma}$
of $\Sigma$.
Let $f  :  \widetilde{X}  \to X $
be the proper birational $G$-equivariant morphism
as in Proposition \ref{p:dom}.
The valuative criterion of properness for $f$ implies
the equality
\[ \widetilde{X}(\O) \cap (G/H)(\K) = {X}(\O) \cap (G/H)(\K). \]
Since $|\Sigma| = | \widetilde{\Sigma}|$, it remains to prove the statement
only for the toroidal horospherical variety  $\widetilde{X}$.



Let $Y_{\widetilde{\Sigma}}$ be  the closure
of the torus $T = P/H \subset G/H$ in
$\widetilde{X}$. Recall that the toroidal  horospherical variety
$\widetilde{X}$ is a
homogeneous fiber bundle $G\times_P Y_{\widetilde{\Sigma}}$ over $G/P$
with fiber isomorphic to the toric variety $Y_{\widetilde{\Sigma}}$
(see the discussion after the Theorem \ref{t:bij} for that point).
This allows to consider  the set
 $Y_{\widetilde{\Sigma}}(\O) \cap T(\K)$ as a subset
of $\widetilde{X}(\O) \cap (G/H)(\K)$.
The restriction of  $\mathcal{V}$ to  $Y_{\widetilde{\Sigma}}(\O) \cap T(\K)$
is exactly the map $\nu\, : \,Y_{\widetilde{\Sigma}}(\O) \cap T(\K) \to
|\widetilde{\Sigma}| \cap N$ from Lemma \ref{l:lattX}.
So the image of $\mathcal{V}$ contains $|\widetilde{\Sigma}|$.


In the toric fibration  
$\phi\, : \, \widetilde{X} \to G/P$  the fiber 
$\phi^{-1}(p_0) \subset  \widetilde{X}$ is exactly the toric variety 
$Y_{\widetilde{\Sigma}}$ and the intersection $Y_{\widetilde{\Sigma}} \cap G/H$ 
is exactly  the torus $T = P/H$. 
Since the group $G(\O)$ acts transitively on
$(G/P)(\O) = (G/P)(\K)$, 
 for any $\lambda \in  \widetilde{X}(\O) \cap (G/H)(\K)$
there exists an element $\gamma \in G(\O)$ such that
$\gamma(\phi(\lambda))  = p_0$. This implies that 
$\gamma(\lambda) \in Y_{\widetilde{\Sigma}}(\O) \cap T(\K)$ and
$\mathcal{V}(\lambda) = \mathcal{V}(\gamma(\lambda))$. Therefore the images
of $\nu$ and  $\mathcal{V}$ are the same.

It  remains only to show that the fibers of $\mathcal{V}$
are precisely the $G(\O)$-orbits.
The latter follows from the $G(\O)$-action on $X(\O)$ and from
the canonical isomorphism
$G(\O) \setminus (G/H)(\K) \simeq N$
induced by $\mathcal{V}$,
see e.g.\,\cite[$\negmedspace$\S8.2]{GN10} (or \cite{LV83}),
because the subset $X(\O) \cap (G/H)(\K) \subset
(G/H)(\K)$ is $G(\O)$-invariant.
\end{proof}


We assume until the end of the section that $X$ is a smooth toroidal $G/H$-embedding such that
every closed orbit in $X$ is projective. This means that
$X$ corresponds to an uncolored fan $\Sigma$ such that every maximal cone of $\Sigma$ is generated
by a $\Z$-basis of $N$.
Then  $X$ is a fibration over $G/P$ with fiber
isomorphic to the smooth toric $T$-embedding $Y:=Y_\Sigma$
and the surjective map $\phi : X \to G/P$ induces,
for $m \in \N$,
surjective morphisms $\phi_m : \mathcal{J}_m(X) \to \mathcal{J}_m(G/P)$.
For any $m \in \N$, denote by
$\pi_m  :  \mathcal{J}_\infty(X) \to \mathcal{J}_m(X)$ and
$\pi'_m :  \mathcal{J}_\infty(Y) \to \mathcal{J}_m(Y)$
the canonical projection maps.
%
For any $n \in |\Sigma| \cap N$,
denote by $\mathcal{C}_{X,n}$ (resp.~$\mathcal{C}_{Y,n}$) the
$G(\O)$-orbit (resp.~$T(\O)$-orbit) of $X(\O) \cap (G/H)(\K)$ (resp.~$Y(\O) \cap T(\K)$)
corresponding to $n$ (see Theorem \ref{t:lattX} and Lemma \ref{l:lattX}).
%
As a consequence of the above proof of Theorem \ref{t:lattX}, we get:


\begin{cor}   \label{c:lattX}

%
Let $n \in |\Sigma| \cap N$ and $m \in \N$.
Then the restriction to $\pi_m(\mathcal{C}_{X,n})$
of $\phi_m$ is surjective onto $\mathcal{J}_m(G/P)$
and its fiber is isomorphic to $\pi'_m(\mathcal{C}_{Y,n})$.

\end{cor}

%
We aim to calculate the motivic measure (with respect to
$\mu_X$; cf. Definition \ref{d:cyl}) of the $G(\O)$-orbits in
$X(\O) \cap (G/H)(\K)$,
the other orbits having zero measure.


Let $n \in |\Sigma| \cap N$
and let $\sigma$ be a $r$-dimensional cone of $\Sigma$
such that $n \in \sigma$.
Fix a basis  $\{ u_1 ,\ldots, u_r \}$ of the semi-group $\sigma^\vee \cap M$.




\begin{lemma}   \label{l:meas}

%
Let $q \ge \max (\{  \langle n , u_j \rangle \ | \ j=1 ,\ldots, r \} )$.
In the notations of Corollary \ref{c:lattX},
the set $\mathcal{C}_{Y,n}$ is a cylinder with $q$-basis
$\pi_q' (\mathcal{C}_{Y,n}) \simeq (\A \smallsetminus 0 )^{r} \times \A^{q r - \sum\limits_{j=1}^r \langle n , u_j \rangle }.$

\end{lemma}


\begin{proof}

By our choice of $q$, for any $\nu \in \pi'_q(\mathcal{C}_{Y,n})$,
the truncated arc $\pi'_q(\nu)$
can be viewed as a $r$-tuple
$(\nu^{(1)},\ldots,\nu^{(r)})$
where
%
$$
	\nu^{(j)} = \nu^{(j)}_{\langle n, u_j  \rangle} t^{\langle n, u_j \rangle}
	+ \nu^{ (j) }_{\langle n, u_j  \rangle + 1} t^{\langle n, u_j  \rangle +1}
	+ \cdots + \nu^{(j)}_{q} t^{q} \,  ;  \quad j \in 1,\ldots,r  \, ,
$$
%
for $ \nu_{\langle n, u_j  \rangle}^{(j)} \in \C^*$ and
$(\nu^{(j)}_{\langle n, u_j  \rangle + 1}, \ldots, \nu^{(j)}_{q}) \in \C^{q- \langle n , u_j  \rangle}$.
Indeed, the orbit $\mathcal{C}_{Y,n}$ is the set of all arcs $\nu \in Y_\sigma(\O) \cap T(\K)$ such that $n_\nu = n$ (see Lemma \ref{l:lattX}).
So, the space of the truncated arcs $\pi'_q(\nu)$ is isomorphic to
%
$$
	(\A \smallsetminus 0 )^{r} \times \A^{\sum\limits_{j=1}^r (q- \langle n , u_j  \rangle) }
	=  (\A \smallsetminus 0 )^{r} \times \A^{qr - \sum\limits_{j=1}^r \langle n , u_j  \rangle } \, .
$$
%
Moreover, if $\nu \in Y(\O)$ lies in $\pi'^{-1}_{q}(\pi'_q (\mathcal{C}_{Y,n}))$
then $\nu \in \mathcal{C}_{Y,n}$.
Hence $\mathcal{C}_{Y,n}=\pi'^{-1}_{q}(\pi'_q (\mathcal{C}_{Y,n}))$
and $\mathcal{C}_{Y,n}$ is a cylinder whose $q$-basis
is the constructible set $\pi'_q(\mathcal{C}_{Y,n})$.


\end{proof}


\begin{thm}     \label{t:meas}

%
We have $\mu_{X} (\mathcal{C}_{X,n}) =  [ G/H ] \, \L^{ - \sum\limits_{j=1}^r \langle n , u_j \rangle }.$

\end{thm}

\begin{proof}

By Corollary \ref{c:lattX} and Definition \ref{d:cyl},
the motivic measure of the cylinder
$\mathcal{C}_{X,n}  = \pi_q^{-1}(\pi_q (\mathcal{C}_{X,n} ))$
of  $X(\O)$, for $q \gg 0$, is expressed by the formula:
%
$$\mu_{X}( \mathcal{C}_{X,n} ) = [ \pi_q( \mathcal{C}_{X,n} )] \L^{-q d}
 = [ \mathcal{J}_q(G/P)  ]  \, (\L-1)^r \, \L^{ q r - \sum\limits_{j=1}^r \langle n , u_j \rangle} \L^{-q d}  \, .$$
Since $\mathcal{J}_q(G/P)$ is a locally trivial
$\A^{q (d-r)}$-bundle over $G/P$
and $[ G/P ] (\L-1)^r  = [G/H ]$, we get
$\mu_{X} (\mathcal{C}_{X,n}) =  [ G/H ] \, \L^{ - \sum\limits_{j=1}^r \langle n , u_j \rangle }$.

\end{proof}




%%%%%%%%%%%%%%%%%%%%%%%%%%%%%%%
%%%%%%%%%%%%%%%%%%%%%%%%%%%%%%%
%%%%
%%%%
 \section{The stringy motivic volume of horospherical varieties}     \label{S:Est}
%%%%
%%%%
%%%%%%%%%%%%%%%%%%%%%%%%%%%%%%%
%%%%%%%%%%%%%%%%%%%%%%%%%%%%%%%



The aim of this section is to prove a formula for $\E_{\rm st}(X)$
for any $\Q$-Gorenstein
horospherical embedding $G/H \hookrightarrow X$,
see Theorem \ref{t:main}.

\smallskip


%
For our purpose, we need to explain the canonical class of a horospherical variety.
Let $G/H \hookrightarrow X$ be a $\Q$-Gorenstein $d$-dimensional
horospherical embedding.
For $\alpha \in S$, denote by $\varpi_\alpha$ the corresponding fundamental weight of $S$.
Let $\rho_S$ (resp.~$\rho_I$) be the half sum of positive roots of $S$ (resp.~$I$).
Note that $\rho_S =\sum_{\alpha \in S} \varpi_\alpha$.
%
For any $\alpha \in S\smallsetminus I$, we define the integers $a_\alpha$ by the equality:
$$
	2(\rho_S - \rho_I ) =\sum\limits_{\alpha \in S \smallsetminus I} a_\alpha \varpi_\alpha .
$$
%
We refer to  \cite[$\negmedspace$\S4.1]{Br93} or \cite[Theorem 4.2]{Br97} for the following result:

%
\begin{prop}              \label{p:KX}

Let $X$ be a $G/H$-embedding.
Then
$$
	K_X =  \sum_{\alpha \in S\smallsetminus I}  - a_\alpha \overline{\varDelta_\alpha} +
	\sum_{ j = 1}^t -D_j \, ,
$$
where $D_1, \ldots, D_t$ are the irreducible divisors in the complement of $X$
to the dense open $G$-orbit, and $\overline{\varDelta_\alpha}$
$(\alpha \in S\smallsetminus I)$ is the closure of $\varDelta_\alpha$ in $X$.


\end{prop}
%
Let $\Sigma \subset N_\R$ be the colored fan corresponding to $X$.
The $\Q$-Gorenstein property  is equivalent to  the existence of a continuous
function
$$�\omega_X \, : \, | \Sigma | \to \R $$
satisfying the following conditions (cf.~\cite[Proposition 4.1]{Br93}):

(P1) \; $\omega_X(e_\tau) = -1$ for a primitive integral generator $e_\tau$
of an uncolored ray $\tau$ of $\Sigma$;

(P2) \; $\omega_X(\varrho_\alpha) = - a_\alpha$ for a colored cone $(\sigma,\F)$ of $\Sigma$
and $\alpha \in \F$;

(P3) \; $\omega_X$ is linear on each cone $\sigma \in \Sigma$.



\medskip

%Consider a resolution of singularities of $X$.
Let $f' \, :  \, X'  \to X$ be a proper birational $G$-equivariant morphism
where $X'$ is  a smooth toroidal $G$-equivariant embedding
with (uncolored) fan $\Sigma'$
obtained from $\Sigma$ by removing colors and subdividing
(see the discussion after the proposition \ref{p:dom}).
%Note that the fans $\Sigma'$ and $\Sigma$ share the same support $|\Sigma|$.
Denote by
%
$$
	K_{X'/X} : = K_{X'}  - {f'}^* K_X
$$
%
the discrepancy divisor of $f'$.

\medskip


Let $\tau_1 ',\ldots,\tau_q '$ be the rays
of $\Sigma'$ which are not rays of $\Sigma$ (this set may be empty),
$e_{\tau_1'},\ldots, e_{\tau_q '}$ the respective primitive integral generators,
and $D'_1,\ldots, D'_q$ the respective irreducible $G$-stable
divisors of $X'$.
%
Let also $\tau_{1},\ldots,\tau_{t}$
be the uncolored rays of $\Sigma$
and $(\tau_{t+1}, \F_{t+1}),\ldots, (\tau_s ,\F_s)$ the colored ones.
Denote by $D_1,\ldots ,D_s$
the irreducible $G$-stable divisors of $X'$
corresponding to the rays $\tau_1,\ldots ,\tau_s$ of $\Sigma'$.
Thus,
%
$$
	\{ D'_1,\ldots, D'_m \} \cup \{ D_1, \ldots, D_s \}
$$
%
is the set of irreducible $G$-stable divisors of $X'$.
Let $e_{\tau_1}, \ldots, e_{\tau_s}$ be primitive integral generators of the rays
$\tau_1,\ldots,\tau_s$ of $\Sigma'$ respectively.


\begin{prop}  \label{p:div}

Assume that $X$ is $\Q$-Gorenstein.
Then
$$
	K_{X'/X} =   \sum\limits_{i=1}^{q} (- 1 - \omega_X(e_{\tau_i'}) ) D'_i
		+ \sum\limits_{j = t+1}^s (- 1- \omega_X( e_{\tau_j}) ) D_j  .
$$
Moreover, $K_{X'/X}$ is a smooth simple normal crossings Cartier divisor
and $X$ has at worst log-terminal singularities.

\end{prop}


\begin{proof}


Since $X'$ is smooth,
there is a continuous function, $\omega_{X'} : |\Sigma'| \to \R$,
satisfying the following conditions:

(P1$^\prime$) \; $\omega_{X'}(e_{\tau_i'}) = \omega_{X'}(e_{\tau_j}) = -1$
for all $i=1,\ldots, q$ and $j=1,\ldots,s$;

(P2$^\prime$) \; $\omega_{X'}$ is linear on each cone of $\Sigma'$.

\smallskip

\noindent
Define a function $\psi :  |\Sigma'| \to \R$
by setting $\psi  (n) : = \omega_{X'} (n) - \omega_X (n)$ for any $n \in N_\R$.
Then $\psi$ is a continuous map
which  is linear on each cone of $\Sigma'$ (use properties (P3) and (P2$^\prime$)).
By Proposition~\ref{p:KX},
$$
	K_{X'} =  \sum_{\alpha \in S\smallsetminus I}  - a_\alpha \overline{\varDelta_\alpha} +
	\sum_{ i = 1}^q -D_i' + \sum_{ j = 1}^s - D_j  \quad \textrm{ and }   \quad
    	K_{X} =  \sum_{\alpha \in S\smallsetminus I}  - a_\alpha \overline{\varDelta_\alpha} +
	\sum_{ j = 1}^t - D_j \, . $$
So, by the conditions (P1), (P2) and (P1$^\prime$), we get
$$
	K_{X'/X} = \sum\limits_{i=1}^{q} (- 1 - \omega_X(e_{\tau_i'}) ) D'_i
	+ \sum\limits_{j = t+1}^s (- 1 - \omega_X(e_{\tau_j}) ) D_j  \, .
$$
Since $X$ is $\Q$-Gorenstein, $X$ has at worst log-terminal singularities,
see \cite[Theorem 4.1]{Br93}.
%This fact can be also traced back from the above formula.
At last, $X'$ being smooth and toroidal, $K_{X'/X}$ is a smooth
simple normal crossings divisor.


\end{proof}


We are now in the position to state the main result of this section:


%%
%
\begin{thm}  \label{t:main}

Let $G/H \hookrightarrow X$ be a $\Q$-Gorenstein $d$-dimensional
horospherical embedding with colored fan $\Sigma \subset N_\R$,
and $\omega_X$ as above.
Then
$$
	\E_{\rm st}(X ) = [G/H ]
			\sum \limits_{ n \in | \Sigma| \cap N } \L^{\omega_X(n)} \, .
$$

\end{thm}
%
%%

\medskip


The remaining of the section is devoted to the proof of Theorem \ref{t:main}:
Theorem \ref{t:main} will be a straightforward consequence of Lemma \ref{l:main1}
and Lemma \ref{l:main2}.
%
Keep the above notations and
denote by $\mathcal{C}_{X',n}$ the
$G(\O)$-orbit in $X'(\O) \cap (G/H)(\K)$ corresponding to $n \in |\Sigma| \cap N$
(cf.\,Theorem\,\ref{t:lattX}).

\begin{lemma}   \label{l:main1}

We have:
%
\begin{eqnarray*}
	\E_{\rm st}(X ) =
		\sum\limits_{n \in |\Sigma| \cap N}\  \int\limits_{ \ \mathcal{C}_{X',n}}
			\L^{- {\rm ord}_{K_{X'/X}}} \, {\rm d} \mu_{X'}.
\end{eqnarray*}
%

\end{lemma}


\begin{proof}

Since the $G(\O)$-orbits in $X'(\O)$
which are not contained in $(G/H)(\K)$
have zero motivic measure, we get by Definition \ref{d:Est}:
%
$$
	\E_{\rm st}(X) =  \int\limits_{X'(\O)} \L^{- {\rm ord}_{K_{X'/X}}} \, {\rm d} \mu_{X'}
		= \int\limits_{X'(\O) \cap (G/H)(\K)} \L^{- {\rm ord}_{K_{X'/X}}} \, {\rm d} \mu_{X'} \,  .
$$
%
In addition, by Theorem \ref{t:lattX}, $X'(\O) \cap (G/H)(\K)$ is a countable disjoint union of $G(\O)$-orbits
and each of these $G(\O)$-orbits corresponds to a point
$n \in |\Sigma| \cap N$:
%
$$
	X'(\O) \cap (G/H)(\K) =
		\bigsqcup \limits_{n \in |\Sigma| \cap N } \mathcal{C}_{X',n} .
$$
%
All $\mathcal{C}_{X',n}$ are cylinders %(cf.\,Lemma\,\ref{l:meas})
whose union is a measurable set.
The lemma is then a consequence of Proposition \ref{p:mot}(i).


\end{proof}

		
\begin{lemma}   \label{l:main2}
For any lattice point $n \in |\Sigma| \cap N $,
we have
$$ \int\limits_{ \mathcal{C}_{X',n} } \L^{- {\rm ord}_{K_{X'/X}}} \, {\rm d} \mu_{X'} �
= [ G/H ]  \, \L^{\omega_X(n)}. $$

\end{lemma}

\begin{proof}

Let $(\sigma,\F)$ be a colored cone in $\Sigma$
such that $\sigma$ contains $n$.
We remark that the statement
of the lemma is local.
So, it is enough to prove it in the case where $X$ is
the simple horospherical variety corresponding to $(\sigma,\F)$.
%and $X'$ is defined by
%deleting  colors and by a smooth subdivision of $\sigma$.
Furthermore,
we can assume that $\sigma$ has the maximal dimension $r$ (i.e., the unique closed
$G$-orbit in $X$ is projective).
Otherwise we can embed $\sigma$ as a face into
some  $r$-dimensional cone $\hat{\sigma}$ such that the restriction of the
linear function $\omega_{\hat{X}}$ to $\sigma$ coincides with $\omega_X$ and the
smooth subdivision of $\sigma$ extends to a smooth subdivision of $\hat{\sigma}$.
Here, $\hat{X}$ is the simple horospherical $G/H$-embedding
corresponding to the $r$-dimensional colored cone
$(\widehat{\sigma},\F)$.
%
Thus, it is enough to consider the case where every maximal cone of $\Sigma'$
is generated by a $\Z$-basis of $N$.

For the sake of the simplicity, we set, in the notations of Proposition \ref{p:div}:
$c'_{i} : = -1 - \omega_X(e_{\tau_i'}) $, for $i \in \{1, \ldots, q\}$,
and $c_{j} : = - 1 - \omega_X (e_{\tau_j})$, for $j \in \{t +1,\ldots, s\}$.
Thus,
$$
	K_{X'/X} =  \sum\limits_{i=1}^{q} c'_{i} D'_i
			+ \sum\limits_{j = t+1}^s c_{j}  D_j \, .
$$
%
Let $n \in |�\Sigma| \cap N$.
By the definition of motivic integrals,
$$
	\int\limits_{ \mathcal{C}_{X',n} } \L^{- {\rm ord}_{K_{X'/X}}} \, {\rm d} \mu_{X'} �
		= \sum\limits_{\nu \in \mathbb{Q} } \mu_{X'} ( \{ \lambda \in \mathcal{C}_{X',n} \ | \
			{\rm ord}_{K_{X'/X}} (\lambda) = \nu \} )  \, \L^{ - \nu} \,  .
$$
%
Let $\sigma$ be a $r$-dimensional cone of $\Sigma'$ containing $n$
and generated by a basis $\{e_1,\ldots,e_{r}\}$ of $N$.
% such that $e_1,\ldots,e_{s}$ generate $\sigma$ for some $s \in \{1, \ldots,r\}$.

Its dual basis, $\{u_1,\ldots,u_r \}$,
is a basis of the semi-group $\sigma^\vee \cap M$.
Possibly renumbering the vectors $e_1,\ldots,e_{r}$,
we can assume that there exist $l \in \{ 1,\ldots, q\}$
and $k \in \{ 1 ,\ldots, s \}$ such that, in the notations of Proposition \ref{p:div},
$\{e_{1}, \ldots, e_{l}\}$ is a part of $\{e_{\tau'_1}, \ldots, e_{\tau'_q}\}$,
$\{e_{l+1}, \ldots, e_{l+k}\}$ is a part of $\{e_{\tau_1}, \ldots, e_{\tau_t}\}$
and $\{e_{l+k+1}, \ldots, e_{r}\}$ is a part of $\{e_{\tau_{t+1}}, \ldots, e_{\tau_s}\}$.

%
It follows from the description of $ \mathcal{C}_{X',n}$ (see the proof of Lemma\,\ref{l:meas})
that, for any $\lambda \in  \mathcal{C}_{X',n}$,
$$
	{\rm ord}_{K_{X'/X}} (\lambda) =  \sum\limits_{i=1}^{l} c'_{i} \langle n , u_i \rangle
						 + \sum\limits_{j=l+k+1}^{r} c_{j} \langle n , u_j \rangle .
$$
%
As a result, we get:
%
$$
	\int\limits_{ \mathcal{C}_{X',n} } \L^{- {\rm ord}_{K_{X'/X}}} \, {\rm d} \mu_{X'} �
		= \mu_{X'} (\mathcal{C}_{X',n} )
			\, \L^{ -   \sum\limits_{i=1}^{l} c'_{i} \langle n , u_i \rangle
	 		- \sum\limits_{j=l+k+1}^{r} c_{j} \langle n , u_j \rangle}.
$$
%
In addition, by Theorem \ref{t:meas},
$$�
	\mu_{X'} (\mathcal{C}_{X',n} )
		= [ G/H ] \, \L^{ - \sum\limits_{ j=1 }^{r} \langle n , u_j \rangle} \, .
$$
%
So, it only remains to show that
$\omega_ X ( n )= - \sum\limits_{ j=1 }^{r} \langle n , u_j \rangle
   -   \sum\limits_{i=1}^{l} c'_{i} \langle n , u_i \rangle
   -   \sum\limits_{j=l+k+1}^{r} c_{j} \langle n , u_j \rangle $.
%
By the properties (P1), (P2) and (P3) of $\omega_X$, one has:
%
$$�
	\omega_X ( n )  = \omega_X ( \sum\limits_{ j=1 }^{r} \langle n , u_j  \rangle e_{j})
		= \sum\limits_{i=1}^{l} \langle n , u_i \rangle  \omega_X(e_{i})
		- \sum\limits_{j=l+1}^{l+k}  \langle n , u_j \rangle
  		+ \sum\limits_{j=l+k+1}^{r}   \langle n , u_j \rangle \omega_X(e_{j})
$$
%
$$�
	\qquad \qquad \qquad = - \sum\limits_{ j=1 }^{r} \langle n , u_j \rangle
   		-   \sum\limits_{i=1}^{l} c'_{i} \langle n , u_i \rangle
   		-   \sum\limits_{j=l+k+1}^{r} c_{j} \langle n , u_j \rangle \, .
$$
%
Then, the expected expression for $\omega_X(n)$ follows.
%Indeed, $n \in \sigma$ forces $\langle n , u_j \rangle =0$ for any $j\in\{ s+1,\ldots,r \}$,
%so $\sum\limits_{ j=1 }^{s} \langle n , u_j \rangle  =\sum\limits_{ j=1 }^{r} \langle n , u_j \rangle$.


\end{proof}

As noticed before, Lemma \ref{l:main1} together with Lemma \ref{l:main2}
complete the proof of Theorem \ref{t:main}.

\medskip



\begin{ex}   \label{ex:Q}

Consider the case where $G=SL_3(\C)$,
$B$ is the Borel subgroup of $G$
consisted of upper triangular matrices of $G$, $S=\{\beta_1,\beta_2\}$
and $H=U$.
Then $G/H$ is a quasi-affine homogeneous horospherical variety
whose affine closure is the $5$-dimensional affine quadric
$$ Q= \{ (x_1,x_2,x_3,y_1,y_2,y_3) \in \A^6 \ | \ x_1 y_1 + x_2 y_2 + x_3 y_3= 0 \} \, ; $$
$Q$ is the affine cone over the Grassmannian $G(2,4)$.
%
Denote by $\check{\beta_1}$ and $\check{\beta_2}$ the coroots of $\beta_1$ and $\beta_2$ respectively.
The representation of $SL_3(\C)$ on $\A^6$ is
the sum of two fundamental $3$-dimensional
irreducible representations with the dominant weights  $\varpi_{\beta_1}$,
$\varpi_{\beta_2}$ and $Q$ has for maximal colored cone
$(\sigma, \{\beta_1, \beta_2\})$
where $\sigma$ is the cone of $N_\R$ generated by ${\check{\beta}_1}|_M$
and ${\check{\beta}_2}|_M$.
%
The quadric $Q$ admits four $G$-orbits:
$0$, two copies of $\A^3 \smallsetminus 0$, and the dense orbit $G/U$.
We have $[G/U] =  (\L^2-1)(\L^3 -1)$.
Using this decomposition into $G$-orbits of $Q$,
one gets $[Q] = \L^2(\L^3 + \L -1)$.
%\qA{Fullfill it!!!}.
%
On the other hand, by Theorem \ref{t:main},
$$ \E_{\rm st}(Q) = [ G/U ]\,  \left( \sum\limits_{k\ge 0} \L^{-2k} \right)^{\!\!2}
	=  \displaystyle{ \frac{ (\L^2-1)(\L^3 -1) }{ (1 - \L^{-2} )^2}}
	= \displaystyle{ \frac{\L^4  (\L^2+\L+1)  }{\L+1}} \, .
$$
	
\smallskip

Let us show how this result can be obtained
using resolutions of singularities of $Q$.
We consider two different resolutions:
the blowing-up of the point $0 \in Q$
and a decolorization of $Q$.


\smallskip

1) Let $p \, : \, \hat{Q}  \to Q$ be the blowing-up of $0 \in Q$
and $D$ the exceptional divisor.
We have $K_{\hat{Q}} - p^* K_Q = 3 D$ and
$$
	[ \hat{Q} \smallsetminus D ] = [Q] - 1 =  \L^2(\L^3 + \L -1) - 1  .
$$
On the other hand, $D \simeq G(2,4)$ and $[ D ]$ can be readily computed using the Betti numbers.
Then by Definition \ref{d:Est}, we get:
$$
	\E_{\rm st}(Q)
 = [ \hat{Q} \smallsetminus  D ] +  [ D ] \left( \displaystyle{\frac{\L-1}{\L^{4}-1}} \right)
 = \displaystyle{\frac{ \L^4 (\L^2 + \L +1) }{\L +1} } \, .
$$


\smallskip

2) Let $Q'$ be the smooth toroidal variety
corresponding to the uncolored fan obtained from $\Sigma$
and $f' : Q' \to Q$ the corresponding proper birational
$G$-morphism.
Note that $Q'$ is the homogeneous vector bundle on $G/B$
associated with the representation of $B$ on $\A^2$ with weights
the fundamental weights $\varpi_{\beta_1}$, $\varpi_{\beta_2}$.
%Here, we go directly to a smooth variety by removing colors.
The exceptional locus of $f'$ has two irreducible components,
$D_1$ and $D_2$,
and $K_{Q'/Q} = D_1 + D_2$.
%
The set $Q' \smallsetminus (D_1 \cup D_2)$
 is isomorphic to the open orbit $G/U$
and $D_1 \smallsetminus (D_1 \cap D_2)$ is a locally trivial fibration over
$\A^3 \smallsetminus 0$ with fiber $\P^1$.
Moreover, $D_1\cap D_2$ is the unique closed $G$-orbit
which is here isomorphic to $G/B$.
%
Hence, by Definition \ref{d:Est},
\begin{eqnarray*}
\E_{\rm st} (Q) = [Q' \smallsetminus (D_1 \cup D_2) ]
        + 2\, \displaystyle{\frac{ [ D_1 \smallsetminus (D_1 \cap D_2) ] }{\L+1}}
        + \displaystyle{\frac{ [ D_1 \cap D_2 ] }{(\L+1)^2}}
 =   \displaystyle{\frac{ \L^4 (\L^2+\L+1)}{\L+1}}  \, .
\end{eqnarray*}



\end{ex}






\bigskip

%%%%%%%%%%%%%%%%%%%%%%%%%%%%%%
%%%%%%%%%%%%%%%%%%%%%%%%%%%%%%
%%%%
%%%%
\section{Smoothness criterion}           \label{S:Smo}
%%%%
%%%%
%%%%%%%%%%%%%%%%%%%%%%%%%%%%%%
%%%%%%%%%%%%%%%%%%%%%%%%%%%%%%


We obtain in this section (Theorem \ref{t:smo}) a smoothness criterion for
locally factorial horospherical embeddings
in term of their stringy Euler numbers
(cf.\,Definition \ref{d:Eul}).
Since the smoothness condition is a local condition,
we can restrict our study to the case of simple horospherical embeddings.

\smallskip

Recall that a normal variety is called {\em locally factorial}
if any Weil divisor is a Cartier divisor.
The following criterion for the locally factorial condition
can be readily extracted from \cite[Proposition 3.1]{Br89} and \cite[Proposition 4.2]{Br93}:

\begin{thm}   \label{t:lf}

Let $X$ be a simple horospherical $G/H$-embedding with maximal cone $(\sigma,\F)$.
Then, $X$ is locally factorial if and only if
the following two conditions are satisfied:

{\rm (L1)}  the restriction to $\{\varDelta_\alpha \ | \ \alpha \in \F\}$ of the map $\varrho$ is injective;

{\rm (L2)} \! $\sigma$ is generated by part of a basis of $N$ which contains
all $\varrho_\alpha$ for $\alpha \in \F$.

\end{thm}

Recall that the {\em usual Euler number} $e(V)$ of any complex algebraic variety $V$ is defined by
$$
	e(V) : = E (V ; 1,1) .
$$

\begin{defi}     \label{d:Eul}

Let $X$ be a $d$-dimensional normal $\Q$-Gorenstein variety.
Adopt the notations of Definition \ref{d:Est}
and define the {\em stringy Euler number} $e_{\rm st}(X)$ of $X$ by
$$ �
	e_{\rm st}(X) :=  \sum\limits_{J \subseteq \{1,\ldots, l \}} e(D_J^0) \, \prod\limits_{j \in J}
		\, \displaystyle{\frac{ 1 }{ \nu_j +1 }}.
$$

\end{defi}


The stringy $E$-function of $X$ was defined in Definition \ref{d:Est2}.
Note that $e_{\rm st}(X)$ is nothing but $ E_{\rm st} (X ; 1 ,1)$.
We refer to \cite{Ba98} of \cite{Ba99} for more details about the stringy Euler numbers.


%%
%
\begin{thm}   \label{t:smo}

Let $X$ be a simple locally factorial horospherical $G/H$-embedding.
Assume that the maximal cone associated with $X$ has dimension $r$.
Then one has $e_{\rm st}(X) \ge e (X)$, and
the equality holds if and only if $X$ is smooth.

\end{thm}
%
%%

Our assumption that the maximal cone associated with $X$ has dimension $r$
means that the closed orbit of $X$ is projective.
%
The proof of Theorem\,\ref{t:smo}
will be achieved at the end of the section.

\begin{ex}

The affine quadric $Q$ introduced in Example\,\ref{ex:Q}
yields an example of horospherical variety which is locally factorial
but not smooth,
$$
	e_{\rm st} (Q) = \frac{3}{2}
		> e(Q) = 1.
$$

\end{ex}


\begin{ex}    \label{ex:Grass}

Here we give an example of a singular horospherical variety $X$
for which the stringy $E$-function is polynomial.

Consider the case where $G=SL_4(\C)$, $B$ is the set of upper triangular matrices of $G$
and set $S = \{\beta_1,\beta_2,\beta_3\}$.
The representation of $G$ on
$\C^4 \oplus \wedge^2 \C^4 $
is the sum of two fundamental representations with
the dominant weights $\varpi_{\beta_1}$ and $\varpi_{\beta_2}$.
The stabilizer of $(e_1 , e_1 \wedge e_2) \in  \C^4 \oplus \wedge^2 \C^4$
in $G$ is the horospherical subgroup $H = P_{ \{ \beta_3 \} }  \cap (\ker \varpi_{\beta_1} \cap \ker \varpi_{\beta_2})$
where $(e_1,e_2,e_3,e_4)$ is the canonical basis of $\C^4$.
%
We have $\dim G/H =7$ and ${\rm rk} \, G/H = 2$.
%
Let $X \subset \wedge^2 \C^5 \simeq \C^4 \oplus \wedge^2 \C^4$ be the closure of the $G$-orbit
of $(e_1,e_1 \wedge e_2)$ in $\C^4 \oplus \wedge^2 \C^4$.
Then $X$ is  the affine cone over the Grassmannian
$G(2,5)$ and contains three more $G$-orbits:
$(\wedge^2 \C^4 \smallsetminus 0)$,
$(\C^4 \smallsetminus 0)$ and $0$.
%
From this, we get:
$[X] = \L^7 + \L^5 - \L^2$.
%
The maximal colored cone corresponding to $X$
is  $(\sigma,\{ \beta_1, \beta_2 \})$
where $\sigma$ is the cone of $N_{\R}$ generated by ${\check{\beta}_1}|_M$ and ${\check{\beta}_2}|_M$.
%
We have $a_{\beta_1} = 2$ and $a_{\beta_2} = 3$.
Hence, by Theorem\,\ref{t:main},
%
$$
	\E_{\rm st}(X) =
	\, \displaystyle{\frac{ (\L-1)^2  \, (\L+1) \, (\L^2+1) \, (\L^2 + \L +1) }
				{(1 - \L^{-2})(1 - \L^{-3})}}  =  \L^5 (\L^2 +1) \, .
$$
%
We have,
$e_{\rm st}(X) = 2 > e(X) = 1.$

\end{ex}


For $S' \subseteq S$, denote by $\Gamma_{S'}$ the Dynkin diagram corresponding to $S'$;
the vertices of $\Gamma_{S'}$ are the elements of $S'$.
In \cite[\S3.5]{Pau83}, Pauer gives a smoothness criterion
for any $G/H$-embedding in the case where $H=U$;
for the general case, see  \cite[Theorem 2.6]{Pa07} or \cite[Theorem 28.10]{Ti10}.
Recall here the criterion:

\begin{prop}   \label{p:PP}

Let $X$ be a simple locally factorial
horospherical $G/H$-embedding with maximal colored cone $(\sigma,\F)$
and let $I \subseteq S$ be such that $N_G(H) = P_I$.
Then, $X$ is smooth if and only if any connected component $\Gamma$
of $\Gamma_{I \cup \F}$ verifies one of the following conditions:

\smallskip

{\rm (C1)} \; $\Gamma$ is a Dynkin diagram of type ${\bf A}_\ell$, $\ell \ge 1$,
	and $\Gamma$ contains exactly one vertex in $\F$ which is extremal:
%

$
\begin{Dynkin}
	\Dbloc{\Dbullet\Deast}
	\Dbloc{\Dcirc\Dwest\Deast}
	\Dbloc{\Dcirc\Dwest\Deast}
	\Dbloc{\Ddots}
	\Dbloc{\Dcirc\Dwest\Deast}
	\Dbloc{\Dcirc\Dwest\Deast}
	\Dbloc{\Dcirc\Dwest}
\end{Dynkin}
$
	


{\rm (C2)} \; $\Gamma$ is a Dynkin diagram of type ${\bf C}_{\ell}$, $\ell \ge 3$,
	and $\Gamma$ contains exactly one vertex in $\F$ which is the simple extremal one:
%

$
\begin{Dynkin}
	\Dbloc{\Dbullet\Deast}
	\Dbloc{\Dcirc\Dwest\Deast}
	\Dbloc{\Dcirc\Dwest\Deast}
	\Dbloc{\Ddots}
	\Dbloc{\Dcirc\Dwest\Deast}
	\Dbloc{\Dcirc\Dwest\Ddoubleeast}
	\Dleftarrow
	\Dbloc{\Dcirc\Ddoublewest}
\end{Dynkin}
$


{\rm (C3)} \; $\Gamma$ is any Dynkin diagram whose vertices are all in $I$.

\end{prop}


\begin{ex}


1) The standard representation $(\C^{\ell +1},\varpi_1)$ of $G=SL_{\ell +1}(\C)$
is a smooth affine horospherical variety corresponding to the situation (C1).
Namely, the dense orbit $\C^{\ell+1} \smallsetminus 0$ of $\C^{\ell +1}$
is isomorphic to $G/H$ where $H$ is the kernel in the standard maximal parabolic $P$
whose Levi part contains the $\alpha_j$-root subgroups, for $j=2,\ldots,\ell$,
of the restriction to $P$ of $\varpi_1$.

2) The standard representation $(\C^{2\ell},\varpi_1)$ of $G=Sp_{2\ell}(\C)$
is a smooth affine horospherical variety corresponding to the situation (C2).
We have the same description of the dense orbit as in 1):
The dense orbit $\C^{2\ell} \smallsetminus 0$ of $\C^{2\ell}$
is isomorphic to $G/H$ where $H$ is the kernel in the standard maximal parabolic $P$
whose Levi part contains the $\alpha_j$-root subgroups, for $j=2,\ldots,\ell$,
of the restriction to $P$ of $\varpi_1$.

3) The case where $\F$ is empty (situation (C3)) corresponds to locally factorial toroidal embeddings
which are known to be smooth.




\end{ex}


We state several technical lemmas useful for the proof of Theorem\,\ref{t:smo}.
Our main reference for basics on Lie algebras and root systems is \cite{OV}.
Assume that $\Gamma_S$ is connected.
Let $I$ be a subset of $S$ and
%of cardinality $\ell'$ such that $\Gamma_{S'}$ is connected.
let us introduce standard related notations.
	
%
\textbullet \; We denote by $\root$ the root system of $G$,
by $\root^+$ the set of positive root of $\root$,
by $\root_{I}$
the root subsystem of $\root$ generated by $I$
and by $\root_{I}^+$ the set $\root_{I} \cap \root^+$.

\textbullet \;
For any $\gamma \in \root$, we denote by $\check{\gamma}$
its coroot, and set $\check{S}:=\{�\check{\beta} \, ; \, \beta \in S\}$.


%
\textbullet \; If $\Gamma_{I}$ is connected,
we denote by $W_{I}$ the Weyl group associated with
$\root_{I}$, that is the subgroup of $GL(V)$ where $V:=\Z \root_{I} \otimes_{\Z} \R$
	generated by the reflections,
	$$s_{\alpha}\, : \,  V  \rightarrow V , \
			x  \mapsto  x - \langle x,\check{\alpha} \rangle \, \alpha,
		\qquad \alpha \in I.$$
			
%
\textbullet \; The exponents of $S$ (or $\check{S}$) will be denoted by $m_1, \ldots , m_{\ell}$.
We can assume that $m_1 \le \cdots \le m_\ell$.
The integers $m_1 +1 ,\ldots, m_\ell +1$ are the degrees of the basic $W_S$-invariant polynomials and we have
	$$| W_{S} |= \prod_{i=1}^\ell (m_i+1). $$
In addition, $\sum_{i=1}^{\ell} m_i = |\root^+|$.


%	

\textbullet \; For $\gamma \in \root^+$, the {\em height}
of $\gamma$ is ${\rm ht} ( \gamma) := \sum_{\beta \in S}  \langle \check{\varpi}_{\beta}, {\gamma} \rangle$
where for $\beta \in S$, $\check{\varpi}_{\beta}$
is the fundamental weight of $\check{S}$ corresponding to $\check{\beta}$.
We denote by $\theta_{S}$ the highest root of $S$
and by $\theta_{\check{S}}$ the highest root of $\check{S}$.
%The partition of $|\mathcal{R}^+|$ given by the exponents is
%dual to the one given by the number of positive roots of each height.
%In particular,
One has $m_\ell = {\rm ht}(\theta_{S}) =  {\rm ht}(\theta_{\check{S}})$.
	
	
%
\textbullet \;  We denote by $\rho_{I} := \frac{1}{2} \sum_{\gamma \in \root_{I}^+}  \gamma$
the half sum of positive roots of $I$.
We have $\rho_{S}  =  \sum_{\beta \in S} \varpi_\beta$
and $\langle \rho_I ,\check{\beta} \rangle =1$
for any $\beta \in I$.
	
%
\textbullet \; Set $J := S \smallsetminus I$.
	The integers $a_\alpha$, for $\alpha \in J$, are defined by:
	$$
		a_\alpha :=	2 \, \langle \rho_S -\rho_I , \check{\alpha} \rangle
			  =	2 -  2 \langle \rho_I , \check{\alpha} \rangle
			  =     2 - \sum_{\gamma \in \root_I^+}  \langle \gamma , \check{\alpha} \rangle \, .
	$$
%We have $ 2( \rho_S - \rho_I ) = \sum\limits_{\alpha \in S \smallsetminus I} a_\alpha \varpi_\alpha$.


A dominant weight $\mu$ %, that is an element of  $\bigoplus_{\beta \in S} \N \varpi_\beta$,
is called {\em minuscule} if $\langle \mu, \theta_{\check{S}} \rangle =1$.
If $\mu$ is minuscule then there is $\beta \in S$ such
that $\mu = \varpi_{\beta}$, cf. \cite[Chapter VI, \S2, exercise 24]{Bo}.


\begin{lemma}    \label{l:root}

Let $\alpha \in J = S \smallsetminus I$.
Then, $a_\alpha \in \{2,\ldots, m_\ell + 1\}$.
Furthermore, the equality $a_\alpha = m_\ell +1$ holds
if and only if $J=\{\alpha\}$
and $\varpi_\alpha$ is minuscule,
that is if $\alpha$ is one of the simple roots
as described below:

%\smallskip

\begin{tabular}{llll}
&&&\\
${\bf A}_\ell$, $\ell \ge 1$  :&
$
\begin{Dynkin}
	\Dbloc{\Dbullet\Deast\Dtext{t}{\beta_1}}
	\Dbloc{\Dbullet\Dwest\Deast\Dtext{t}{\beta_2}}
	\Dbloc{\Dbullet\Dwest\Deast\Dtext{t}{\beta_3}}
	\Dbloc{\Ddots}
	\Dbloc{\Dbullet\Dwest\Deast\Dtext{t}{\beta_{\ell-2}}}
	\Dbloc{\Dbullet\Dwest\Deast\Dtext{t}{\beta_{\ell-1}}}
	\Dbloc{\Dbullet\Dwest\Dtext{t}{\beta_\ell}}
\end{Dynkin}
$
&  $\alpha \in \{ \beta_1,\ldots,\beta_\ell\}$;\\ % and $a_{\alpha}=\ell +1$;\\
%&&&\\
%&&&\\
${\bf B}_\ell$, $\ell \ge 2$ :&
$
\begin{Dynkin}
	\Dbloc{\Dcirc\Deast\Dtext{t}{\beta_1}}
	\Dbloc{\Dcirc\Dwest\Deast\Dtext{t}{\beta_2}}
	\Dbloc{\Dcirc\Dwest\Deast\Dtext{t}{\beta_3}}
	\Dbloc{\Ddots}
	\Dbloc{\Dcirc\Dwest\Deast\Dtext{t}{\beta_{\ell-2}}}
	\Dbloc{\Dcirc\Dwest\Ddoubleeast\Dtext{t}{\beta_{\ell-1}}}
	\Drightarrow
	\Dbloc{\Dbullet\Ddoublewest\Dtext{t}{\beta_\ell}}
\end{Dynkin}
$
& $\alpha=\beta_\ell$; \\ %and $a_{\alpha}=2 \ell$; \\
%&&&\\
%&&&\\
${\bf C}_\ell$,  $\ell \ge 3$ :&
$
\begin{Dynkin}
	\Dbloc{\Dbullet\Deast\Dtext{t}{\beta_1}}
	\Dbloc{\Dcirc\Dwest\Deast\Dtext{t}{\beta_2}}
	\Dbloc{\Dcirc\Dwest\Deast\Dtext{t}{\beta_3}}
	\Dbloc{\Ddots}
	\Dbloc{\Dcirc\Dwest\Deast\Dtext{t}{\beta_{\ell-2}}}
	\Dbloc{\Dcirc\Dwest\Ddoubleeast\Dtext{t}{\beta_{\ell-1}}}
	\Dleftarrow
	\Dbloc{\Dcirc\Ddoublewest\Dtext{t}{\beta_\ell}}
\end{Dynkin}
$  & $\alpha=\beta_1$; \\  %and $a_{\alpha}=2 \ell$; \\
%&&&\\
%&&&\\
%&&& \\
${\bf D}_\ell$, $\ell \ge 4$ : &
$
\begin{Dynkin}
	\Dspace\Dspace\Dspace\Dspace\Dspace
		\Dbloc{\Dbullet\Dsouthwest\Dtext{t}{\beta_{\ell}}}
	\Dskip
	\Dbloc{\Dbullet\Deast\Dtext{t}{\beta_1}}
	\Dbloc{\Dcirc\Dwest\Deast\Dtext{t}{\beta_2}}
	\Dbloc{\Dcirc\Dwest\Deast\Dtext{t}{\beta_3}}
	\Dbloc{\Ddots}
	\Dbloc{\Dcirc\Dwest\Dnortheast\Dsoutheast\Dtext{t}{\beta_{\ell-2}}}
	\Dskip
	\Dspace\Dspace\Dspace\Dspace\Dspace
		\Dbloc{\Dbullet\Dnorthwest\Dtext{t}{\beta_{\ell-1}}}
	\end{Dynkin}
$
& $\alpha \in \{ \beta_1,\beta_{\ell-1},\beta_\ell \}$;\\ %and $a_{\alpha}=2 \ell -2$;   \\
%&&& \; \\
%&&&\\
%&&&\\
${\bf E}_6$  :&$
\begin{Dynkin}
	\Dbloc{\Dbullet\Deast\Dtext{t}{\beta_1}}
	\Dbloc{\Dcirc\Dwest\Deast\Dtext{t}{\beta_3}}
	\Dbloc{\Dcirc\Dwest\Deast\Dsouth\Dtext{t}{\beta_4}}
	\Dbloc{\Dcirc\Dwest\Deast\Dtext{t}{\beta_5}}
	\Dbloc{\Dbullet\Dwest\Dtext{t}{\beta_6}}
	\Dskip
	\Dspace\Dspace\Dbloc{\Dcirc\Dnorth\Dtext{r}{\beta_2}}
\end{Dynkin}
$
&
$\alpha \in \{\beta_1,\beta_6\}$; \\ %and $a_{\alpha}=12$; \\
%&&&\\
%&&&\\
%&&&\\
${\bf E}_7$ :&
$
\begin{Dynkin}
	\Dbloc{\Dcirc\Deast\Dtext{t}{\beta_1}}
	\Dbloc{\Dcirc\Dwest\Deast\Dtext{t}{\beta_3}}
	\Dbloc{\Dcirc\Dwest\Deast\Dsouth\Dtext{t}{\beta_4}}
	\Dbloc{\Dcirc\Dwest\Deast\Dtext{t}{\beta_5}}
	\Dbloc{\Dcirc\Dwest\Deast\Dtext{t}{\beta_6}}
	\Dbloc{\Dbullet\Dwest\Dtext{t}{\beta_7}}
	\Dskip
	\Dspace\Dspace\Dbloc{\Dcirc\Dnorth\Dtext{r}{\beta_2}}
\end{Dynkin}$
	&  $\alpha=\beta_7$. \\ %and $a_{\alpha}=18$. \\
%&&&\\
%&&&\\
\end{tabular}




\end{lemma}

\begin{proof}

Let $\alpha \in J$.
To begin with, since the coefficients of the Cartan matrix of $S$
are nonpositive outside the diagonal,  one has
$a_\alpha \ge 2$.
Moreover,
$a_\alpha \le 2 -
 2 \langle \rho_{S \smallsetminus \{ \alpha \}} , \check{\alpha} \rangle$.
Hence, we may assume that $J = \{ \alpha \}$,
i.e., $I = S \smallsetminus \{�\alpha\}$.
Consider now the two cases
depending on whether $\varpi_\alpha$ is minuscule or not.



\textasteriskcentered \; Assume that $\varpi_\alpha$ is not minuscule, i.e., $ \langle \varpi_\alpha, \theta_{\check{S}} \rangle >1$.
Then we have
%
\begin{eqnarray*}
m_\ell + 1  =   {\rm ht}(\theta_{\check{S}}) +1
= \sum_{\beta \in S} \langle \varpi_\beta, \theta_{\check{S}} \rangle +1
& = & \langle \varpi_\alpha, \theta_{\check{S}} \rangle
+  \sum_{\beta \in I } \langle \varpi_\beta, \theta_{\check{S}} \rangle +1\\
& > & 2 + \sum_{\beta \in I } \langle \varpi_\beta, \theta_{\check{S}} \rangle
 =  2 +  \langle \rho_I , \theta_{\check{S}} - \langle \varpi_\alpha , \theta_{\check{S}} \rangle \check{\alpha} \rangle .
\end{eqnarray*}
%
Since $\varpi_\alpha$ is not minuscule, $\langle \varpi_\alpha , \theta_{\check{S}} \rangle \ge 2$.
So,
$\langle \rho_I , - \langle \varpi_\alpha , \theta_{\check{S}} \rangle \check{\alpha} \rangle
\ge -2 \langle \rho_I , \check{\alpha} \rangle$ because $- \langle \rho_I , \check{\alpha} \rangle \ge 0$.
On the other hand,  one has $\langle \rho_I , \theta_{\check{S}} \rangle \ge 0$.
Otherwise there would be $\beta \in I$ such that $\langle \beta , \theta_{\check{S}} \rangle < 0$
which is impossible since $\theta_{\check{S}}$ is the highest root.
%
In conclusion, we get
$
m_\ell + 1 > 2 - 2  \langle \rho_I , \check{\alpha} \rangle = a_\alpha$
as desired.



\textasteriskcentered \; Assume that $\varpi_\alpha$ is minuscule, i.e., $ \langle \varpi_\alpha, \theta_{\check{S}} \rangle = 1$.
Then,
$a_\alpha = 2 \langle \rho_S - \rho_I , \check{\alpha} \rangle
 =   2 \langle \rho_S - \rho_I , \check{\alpha} +  \sum_{\beta \in I } \langle \varpi_\beta, \theta_{\check{S}} \rangle \check{\beta} \rangle
 = 2 \langle \rho_S - \rho_I , \theta_{\check{S}} \rangle$.
 Hence, we have
%
\begin{eqnarray*}
a_\alpha
 =  2 \langle \rho_S - \rho_I , \theta_{\check{S}} \rangle = {\rm ht}(\theta_{\check{S}}) + \langle \rho_S  - \rho_I ,\theta_{\check{S}} \rangle - \langle   \rho_I , \theta_{\check{S}} \rangle
=  m_\ell + 1 + \frac{1}{2} (
\sum_{\gamma \in \root^+ \smallsetminus  \root_I \atop\check{\gamma} \not= \theta_{\check{S}} } \langle \gamma   , \theta_{\check{S}} \rangle
- \sum_{\delta \in \root_{I}^+ } \langle   \delta, \theta_{\check{S}} \rangle )
\end{eqnarray*}
since $\langle \gamma   , \theta_{\check{S}} \rangle = 2$ whenever $\check{\gamma} = \theta_{\check{S}}$.
%
Then, our goal is to show that
$$
 \sum_{\gamma \in \root^+ \smallsetminus  \root_I \atop \check{\gamma} \not= \theta_{\check{S}} } \langle \gamma  , \theta_{\check{S}} \rangle
= \sum_{\delta \in  \root_I^+ } \langle   \delta, \theta_{\check{S}} \rangle .
$$
%
For any $\gamma \in \root^+$, %such that $\check{\gamma} \not= \theta_{\check{S}}$,
we have $\langle \gamma, \theta_{\check{S}} \rangle \ge 0$ since $\theta_{\check{S}}$
is the highest root.
%(cf. e.g. \cite[Theorem 18.9.2(iv)]{TY}).
Set $\root' := \{�\gamma \in \root_{S}^+ \smallsetminus  \root_I \; | \;  \gamma \not= \theta_{\check{S}}
\textrm{ and } \langle \gamma, \theta_{\check{S}} \rangle > 0 \}$
and  $\root'' := \{�\delta \in \root_I^+ \; | \;  \langle \delta, \theta_{\check{S}} \rangle > 0 \}$.
Then we have to show the equality:
%
\begin{eqnarray}  \label{eq:root}
 \sum_{\gamma \in \root' } \langle \gamma  , \theta_{\check{S}} \rangle
= \sum_{\delta \in  \root'' } \langle   \delta, \theta_{\check{S}} \rangle .
\end{eqnarray}


Let $\gamma \in \root'$.
Since $\langle \gamma, \theta_{\check{S}} \rangle > 0$,
$\check{\delta} = \theta_{\check{S}} - \check{\gamma} $ is a root
of $\check{S}$
and $\theta_{\check{S}} - \check{\delta}$ is a root too.
In particular, $\langle \delta, \theta_{\check{S}} \rangle > 0$.
Next, show that $\delta \in \root_I^+$.

Since $\gamma \not \in \root_I$, $\check{\gamma} \not \in \root_{\check{I}}$.
Moreover, since $\varpi_\alpha$
is minuscule, $\langle \varpi_\alpha, \check{\gamma} \rangle = \langle \varpi_\alpha, \theta_{\check{S}} \rangle =1$.
So, $\check{\delta} = \theta_{\check{S}} - \check{\gamma} \in \root_{\check{I}}^+$
and then $\delta \in \root_I^+$.
Conversely, if $\delta \in \root''$, then $\check{\gamma} = \theta_{\check{S}} -\check{\delta}$
is a root and so $\langle \gamma, \theta_{\check{S}} \rangle > 0$.
Moreover, $\gamma$ is clearly an element of $\root_{S}^+ \smallsetminus  \root_I^+$
which is different from $\theta_{\check{S}}$,
that is $\gamma \in \root'$.
Therefore, the map from $\root'$ to $\root''$
sending $\gamma$ to $\delta$, where $\check{\delta} = \theta_{\check{S}} -\check{\gamma}$,
gives a bijection between the sets $\root'$ and $\root''$.
%
So, in order to prove the equality (\ref{eq:root}),
it remains to show that for any $\gamma \in \root'$,
we have $\langle \gamma, \theta_{\check{S}} \rangle = \langle \delta, \theta_{\check{S}} \rangle$
where $\check{\delta} = \theta_{\check{S}} -\check{\gamma}$.

Let $\gamma \in \root'$ and set $p := \langle \gamma, \theta_{\check{S}} \rangle > 0$.
%Since $\theta_{\check{S}}$ is the highest root of $\check{S}$,
Then the $\check{\gamma}$-string through
$\theta_{\check{S}}$ is $\{ \theta_{\check{S}}, \ldots, \theta_{\check{S}} - p\check{\gamma} \}$.
%So, $\theta_{\check{S}} - p\check{\gamma} = \check{\delta} - (p-1)\check{\gamma}$ is a root and $\theta_{\check{S}} - (p+1)\check{\gamma}
%=\check{\delta} - p \check{\gamma}$
%is not.
Since there is no minuscule weight in type {\bf G}$_{2}$,
we have $p \in \{�1,2\}$.
If $p =1$, then $\theta_{\check{S}}$ and $\theta_{\check{S}} - \check{\gamma} = \check{\delta}$
are roots but not $\theta_{\check{S}} - 2\check{\gamma} = \check{\delta} - \check{\gamma} = - (\theta_{\check{S}} - 2\check{\delta})$.
So, the $\check{\delta}$-string through
$\theta_{\check{S}}$ is $\{�\theta_{\check{S}}, \theta_{\check{S}} - \check{\delta} \}$
and $\langle \delta, \theta_{\check{S}} \rangle = 1$.
If $p=2$, then $\theta_{\check{S}}$, $\theta_{\check{S}} - \check{\gamma} = \check{\delta}$
and $\theta_{\check{S}} - 2\check{\gamma} = \check{\delta} - \check{\gamma} = - (\theta_{\check{S}} - 2\check{\delta})$
are roots.
So $\langle \delta, \theta_{\check{S}} \rangle \ge 2$ and then $\langle \delta, \theta_{\check{S}} \rangle = 2$.
Hence, in both cases, we have obtained
that $\langle \delta, \theta_{\check{S}} \rangle = p = \langle \gamma, \theta_{\check{S}} \rangle$
and the equality (\ref{eq:root}) is proven.

In conclusion, if $\varpi_\alpha$ is minuscule,
we have showed that $a_\alpha = m_\ell +1$.




%

\end{proof}





\begin{lemma}   \label{l:ht}

Let $S'$ be a subset of $S$
such that $\Gamma_{S'}$ is connected
and denote by $m'_1  \le \cdots \le m'_{l}$ the exponents of $S'$.
Then we have $m'_j \le m_j$ for any $j \in \{1,\ldots,l\}$.
In particular, ${\rm ht}( \theta_{S'} )  \le m_l$.

\end{lemma}

\begin{proof}

By a classical result, \cite{Ko59}, the partition of $|\mathcal{R}^+|$ formed by the exponents is
the dual to that formed by the number of positive roots of each height.
This easily implies the statement.


\end{proof}



Let $k$ be the cardinality of $I$,
and $m_1', \ldots , m_k ' $ the union
of all the exponents of subsets $S'$ such
that $\Gamma_{S'}$ is a connected component of $\Gamma_I$.
Order them so that
$m_1' \le \cdots \le m_k '.$
Number the roots $\alpha_{k+1},\ldots,\alpha_\ell$ of $J$
so that
$a_{\alpha_{k+1}} \le \cdots \le a_{\alpha_\ell}$
and set for simplicity $a_j : = a_{\alpha_j}$ for any $j \in \{k+1,\ldots,\ell\}$.
%

\begin{lemma}  \label{l:eq}

{\rm (i)} For all $i \in \{ 1,\ldots, k\}$, one has $m_i ' \le m_i$
and, for all $j \in \{ k+1,\ldots, \ell \}$, one has $a_{j} \le m_j +1$.
In particular,
$$
	| W_I | \, a_{k+1} \cdots  a_\ell
		\le  | W_{S}|.
$$


{\rm (ii)} Equality
holds in the above inequality if and only if $I$ and $J$
are in one of the configurations {\rm (C1)}, {\rm (C2)}
or {\rm (C3)} as described in Proposition \ref{p:PP}
with $\F=J$.


\end{lemma}


\begin{proof}

(i) By Lemma \ref{l:ht},
for all $i \in \{ 1,\ldots, k\}$, we have $m_i ' \le m_i$.
%
Turn to the second statement.
Set for $j \in \{k+1, \ldots, \ell \}$,
$I_{j} := I \cup \{ \alpha_{k+1} , \ldots, \alpha_j\}$.
%
Let $j \in \{k+1, \ldots, \ell \}$
and $S_j$ the connected component of $I_j$ containing $\alpha_j$.
We have
$a_j = 2 -\langle \rho_{I} ,\check{\alpha} \rangle
= 2 -\langle \rho_{I \cap S_j} ,\check{\alpha} \rangle
= 2\langle \rho_{S_j} - \rho_{I \cap S_j} ,\check{\alpha} \rangle$.
So, by Lemma \ref{l:root},
$a_j \le {\rm ht}(\theta_{S_j} ) + 1$.
Hence, by Lemma \ref{l:ht}, $a_j \le m_j +1$ since $I_j$ has cardinality $j$.
%
All this shows:
$$�| W_I |   \, \prod_{j=k+1}^{\ell} a_{j}  = \prod_{i=1}^{k}( m_i '  +1)  \prod_{j=k+1}^{\ell} a_{j}
\le \prod_{i=1}^{\ell}( m_i  +1) = |W_S |.
$$

\medskip

(ii) By the proof of (i),
if equality holds in the above inequality then $|W_I | = \prod_{i=1}^{k}( m_i   +1) $
and for all $j \in \{ k+1,\ldots, \ell \}$, $a_{j} = m_j +1.$
In particular, $a_\ell = m_\ell +1$.
Therefore, we are in one of the situations
of the Lemma \ref{l:root}
and we consider the six cases as described in it.


%
%
\textbullet \; Type ${\bf A}_\ell$, $\ell \ge 1$: The $\ell-1$ smallest degrees of the basic invariants are
$2,3\ldots, \ell$.
If $\alpha_\ell$ is not an extremal vertex,
then $| W_{S \smallsetminus \{\alpha_\ell\}} | < \ell \, !$ as we easily verify.
So $\alpha_\ell$ must be extremal and $I$ and $J$ are in the configuration (C1).

%
%
\textbullet \; Type ${\bf B}_\ell$,  $\ell \ge 2$:
The $\ell-1$ smallest degrees of the basic invariants are
$2,4,\ldots, 2 (\ell-1)$.
So their product is strictly greater than $|W_{S \smallsetminus \{\beta_\ell\}} | = \ell ! $
and the equality does not hold.

%
%
\textbullet \; Type ${\bf C}_\ell$, $\ell \ge 3$: $I$ and $J$ are in the configuration (C2).

%
%
\textbullet \; Type ${\bf D}_\ell$, $\ell \ge 4$:
The degrees of the basic invariants of ${\bf D}_{\ell}$, for $\ell \ge 4$, are
$2,4,\ldots, 2 \ell -2, \ell$.
So, the $\ell-1$ smallest are $2,4,\ldots, 2 \ell - 4, \ell$ ($\ell \ge 4$)
and their product is $2^{\ell - 2} \ell$.
But for any $i \in \{1,\ldots,\ell\}$,
$| W_{S \smallsetminus \{\beta_i\}} | \le | W_{S \smallsetminus \{\beta_1\}} | =2^{\ell -2} (\ell-1) ! <  2^{\ell - 2} \ell$;
so the equality does not hold.

%
%
\textbullet \; Type ${\bf E}_6$:
The 5-th smallest exponents of ${\bf E}_{6}$ are
$1,4,5,7,8$ and those of $S \smallsetminus \{ \beta_1\}$
(or of $S \smallsetminus \{ \beta_6\}$)
are $1,3,4,5,7$; so, the equality does not hold.

%
\textbullet \; Type ${\bf E}_7$:
The 6-th smallest exponents of ${\bf E}_{7}$ are
$1,5,7,9,11,13$ and those of $S \smallsetminus \{ \beta_7 \}$
are $1,4,5,7,8,11$; so, the equality does not hold.

\smallskip

One has proven one implication.
The converse implication is an easy computation,
left to the reader.

\end{proof}





\begin{prop}  \label{p:Eul}

Assume that $X$ is a simple locally factorial $G/H$-embedding with maximal colored cone $(\sigma,\F)$
of dimension $r$.
Let $I$ be the subset of $S$ such that $N_G(H) = P_I$.
Then,

$$ e_{\rm st} (X) = \displaystyle{\frac{ | W_{S} | }{  | W_{I} | \,
		\prod_{ \alpha \in \F} �a_\alpha }} \quad
		\textrm{ and } \quad
	e(X) = \displaystyle{\frac{ | W_{S} | }{ | W_{I \cup \F} |}}  \, .
$$



\end{prop}


\begin{proof}

First of all, observe that the Euler number of $G/B$
is the number of fixed points of a maximal torus, i.e.,
the order of the Weyl group $W_S$.
More generally, for any $S' \subset S$, the Euler number of
$G/P_{S'}$ is $|W_S| / |W_{S'}|$.
Thus, we have to show:
$$ e_{\rm st} (X) = \displaystyle{\frac{ e(G/P_I )}{
		\prod_{ \alpha \in \F} �a_\alpha }}   \quad
\textrm{ and } \quad
e(X) = e(G/ P_{I \cup \F} )
\, .
$$
%
Now, we observe that the usual Euler number of a horospherical homogeneous space
is nonzero if and only if it has rank zero.
As a consequence, one has $e(X) = e(G /P_{I \cup \F})$,
according to the description of $G$-orbits in $X$ (see Proposition \ref{p:orb}).




Turn to the formula for $e_{\rm st}(X)$.
Let $e_1,\ldots,e_r$ be the primitive generators of $\sigma$.
Since $X$ is locally factorial, $e_1,\ldots,e_r$ is
a $\Z$-basis of $\sigma \cap N$ (cf. Theorem \ref{t:lf}).
Then
%
$$
%
	\sum\limits_{e_i \in \sigma \cap N} \L^{\omega_X (e_i)}
	 =  \prod_{i=1}^r \displaystyle{\frac{1}{1 - \L^{\omega_X (e_i)}}}
	= \displaystyle{\frac{1}{ (\L -1)^r}}
		\prod_{i=1}^r \displaystyle{\frac{\L^{ - \omega_X(e_i)} }{\L^{-\omega_X (e_i) -1 } + \cdots +1}} .
%
$$
%
Then, by Theorem\,\ref{t:main}, one has:
%
\begin{eqnarray*}
%
	\E_{\rm st}(X ) \ = \ [G/H] \sum\limits_{e_i \in \sigma \cap N} \L^{\omega_X (e_i)}
		& = & [ G/P] \, [T] \ \displaystyle{\frac{1}{ (\L -1)^r}}
		\prod_{i=1}^r \displaystyle{\frac{\L^{ - \omega_X(e_i)} }{\L^{-\omega_X (e_i) -1 } + \cdots +1}} \\
		& = &  [ G/P ]
		\prod_{i=1}^r \displaystyle{\frac{\L^{ - \omega_X(e_i)} }{\L^{-\omega_X (e_i) -1 } + \cdots +1}} \, .
%
\end{eqnarray*}
%
From this, we get
$$
	e_{\rm st} (X) = e(G/P) \prod_{i=1}^r \displaystyle{\frac{1}{\big( -\omega_X(e_i) \big)}}
	= \displaystyle{\frac{e(G/P_I)}{\prod_{\alpha \in \F} a_\alpha}} \, .
$$
%
The last equality holds because the set of elements $\varrho_\alpha$
($\alpha \in \F$)
is a subset of the basis $\{e_1,\ldots,e_r\}$ (cf. Theorem \ref{t:lf}).


\end{proof}


We are in a position to prove Theorem\,\ref{t:smo}.


\begin{proof}[Proof of Theorem\,\ref{t:smo}]


We can assume without loss of generality that $S$ is connected
and $I \cup \F = S$.
By Lemma \ref{l:eq} and Proposition \ref{p:Eul},
we have $e_{\rm st}(X) \ge e(X) $.
This proves one part of the theorem.
%
Moreover, the equality holds if and only if $(I,\F)$ is
in one of the configurations (C1), (C2)
or (C3) as described in Proposition \ref{p:PP},
that is to say if and only if $X$ is smooth by Proposition \ref{p:PP}.


\end{proof}



\begin{rem}

As a matter of fact,
we gave another proof for the first implication
of Pauer's criterion (Proposition \ref{p:PP}).
%
Indeed, whenever $(I,\F)$ is not in one of the configurations (C1), (C2)  or (C3)
of Proposition \ref{p:PP}, we have shown that $e_{\rm st}(X) > e(X)$,
and so $X$ is not smooth.


\end{rem}


%%%%%%%%%%%%%%%%%%%%%%%%%%%%%%
%%%%%%%%%%%%%%%%%%%%%%%%%%%%%%
%%%%
%%%%
\section{Some applications and open questions}                   \label{S:App}
%%%%
%%%%
%%%%%%%%%%%%%%%%%%%%%%%%%%%%%%
%%%%%%%%%%%%%%%%%%%%%%%%%%%%%%





Let $X$ be a complete locally factorial horospherical $G/H$-embedding
with colored fan $\Sigma$.
Let $e_1,\ldots, e_s$ be the primitive integral
generators of all $1$-dimensional %oK?
cones in $\Sigma$
and set $a_i : = -\omega_X (e_i)$ for all $i \in \{�1,\ldots, s\}$.

%
Consider the polynomial
ring $\C[z_1,\ldots,z_s]$ whose
variables $z_1,\ldots,z_s$ are in
bijection with the lattice vectors $e_1,\ldots,e_s$.
%
Recall that the Stanley-Reisner ring $R_\Sigma$ is
the quotient of $\C[z_1,\ldots,z_s]$
by the ideal generated by all square free monomials
$z_{i_1} \ldots z_{i_k}$
such that the lattice vectors
$e_{i_1} \ldots e_{i_k}$ do not generate any
$k$-dimensional cone in $\Sigma$.
%
Recall also that the \emph{weighted Stanley-Reisner ring}
$R_\Sigma^w$ is  defined
by putting $\deg z_i = a_i$
in the standard Stanley-Reisner ring $R_\Sigma$.
%


\begin{prop}       \label{p:SR}

Let $X$ be a complete locally factorial horospherical $G/H$-embedding
with colored fan $\Sigma$.
Then, one has:
%
\begin{eqnarray}   \label{eq:P}
%
	&& \sum\limits_{n\in N} (uv)^{\omega_X(n)}
	= P(R_\Sigma^w , (uv)^{-1})
	=  \sum\limits_{\sigma \in \Sigma}
		\displaystyle{ \frac{(-1)^{\dim \sigma}}{ \prod_{e_i \in \sigma} \big(1 - (uv)^{a_i} \big)} }  \, ;
			\\ \label{eq:ai}
%
	&& E_{\rm st} (X ; u,v ) = E(G/H ; u,v) (-1)^r P(R_\Sigma^w ,uv) \, ,
%
\end{eqnarray}
%
where $P(R_\Sigma^w, t)$
denotes the Poincar\'e series of the weighted Stanley-Reisner ring $R_\Sigma^w$.


\end{prop}


\begin{proof}


The ring $R_\Sigma$ has a monomial basis over $\C$
whose elements are in one-to-one correspondence
with $N$.
Namely,
any monomial $z_{i_1}^{k_1} \ldots z_{i_t}^{k_t}$
in $R_\Sigma$ corresponds to the
lattice point
$k_{1} e_{i_1} + \cdots + k_t e_{i_t}$
and the weighted degree of  $z_{i_1}^{k_1} \ldots z_{i_t}^{k_t}$
is $- k_{1} \omega_X(e_{i_1}) - \cdots  - k_t \omega_X(e_{i_t})$.
Thus, the $k$-homogeneous component of the weighted Stanley-Reisner
ring $R_{\Sigma}^{w}$ consists of all
monomials $z_{i_1}^{k_1} \ldots z_{i_t}^{k_t}$
corresponding
to lattice points $n \in N$ such that
$\omega_X(n) = -k$.
%
This implies the first equality in (\ref{eq:P}).
%
For any cone $\sigma \in \Sigma$, we denote by
$\sigma^\circ$ the relative interior of $\sigma$.
Since $X$ is locally factorial, one has by Theorem \ref{t:lf}:
%
\begin{eqnarray}     \label{eq:P2}
%
	\sum\limits_{n \in N} t^{\omega_X (n)}
	& =&  \sum\limits_{\sigma \in \Sigma} \sum_{n \in \sigma^\circ} t^{\omega_X (n)}
	= \sum\limits_{\sigma\in \Sigma} \prod_{e_i \in \sigma} \displaystyle{ \frac{t^{-a_i}}{1 - t^{-a_i}}}
	= \sum\limits_{\sigma\in \Sigma} \prod_{e_i \in \sigma} \displaystyle{ \frac{(-1)^{\dim \sigma}}{1 - t^{a_i}}} \, .
%
\end{eqnarray}
%
This implies the second equality in (\ref{eq:P}).
%

\smallskip

Let us prove the equality (\ref{eq:ai}).
By Theorem \ref{t:main} and (\ref{eq:P}), we have:
$$
	 E_{\rm st} (X ; u ,v ) =  E(G/H ; u,v) P(R_\Sigma^w, (u v)^{-1}) .
$$
%
By the Poincar\'e duality \cite[Theorem 3.7]{Ba98},
we have
%
\begin{eqnarray*}
%
	&& (uv)^{\dim X} E_{\rm st} (X ; u^{-1} ,v^{-1} )
		 = E_{\rm st} (X ;u,v), \\
	&& (uv)^{\dim G/P} E (G/P ; u^{-1} ,v^{-1} )
		= E (G/P ;u,v) \, .
%
\end{eqnarray*}
The above equalities imply:
\begin{eqnarray*}
	 E_{\rm st} (X ; u ,v )  & = & (uv)^{\dim X} E_{\rm st} (X ; u^{-1} ,v^{-1} ) \\
	 & = &  (uv)^{\dim X} E(G/H ; u^{-1},v^{-1})  P(R_\Sigma^w, u v) \\
	& = &(uv)^{\dim G/P} E(G/P ; u^{-1},v^{-1}) (uv)^{r} ((uv)^{-1} -1)^r P(R_\Sigma^w, u v) \\
	 & = &  E(G/P ; u ,v) (uv-1)^r (-1)^r P(R_\Sigma^w, u v) \\
	& = & E(G/H ; u,v) (-1)^r P(R_\Sigma^w, u v)\, .
\end{eqnarray*}

	
\end{proof}

\begin{ex}

1) Consider the locally factorial
completion $\overline{Q}$ of the affine 5-dimensional quadric $Q$ in Example \ref{ex:Q};
$\overline{Q}$ is a singular projective quadric.
The colored fan $\overline{\Sigma}$ of $\overline{Q}$ is represented in Figure \ref{fig:1}
and  the positive integer $a_i = -\omega_{\overline{Q}}(e_i)$ ($i=1,2,3$)
is written down near to  the integral point $e_i$.
The circles stand for the colors $\varrho_\alpha$,
$\alpha \in \F$.


\begin{figure}[htb]
{\setlength{\unitlength}{0.1in}

\begin{center}
\begin{picture}(8,8)(0,0)

\put(4,4){\line(1,0){4}}
\put(4,4){\line(0,1){4}}
\put(4,4){\line(-1,-1){4}}

\put(6,4){\circle{0.8}}
\put(4,6){\circle{0.8}}

\put(6,4){\circle*{0.4}}
\put(4,6){\circle*{0.4}}

\put(2,2){\circle*{0.4}}

\put(1,1.9){\tiny$1$}
\put(6.1,2.8){\tiny$2$}
\put(2.8,5.9){\tiny$2$}



%\qbezier[15](0,0)(4,0)(8,0)
\qbezier[15](0,2)(4,2)(8,2)
\qbezier[15](0,4)(4,4)(8,4)
\qbezier[15](0,6)(4,6)(8,6)
%\qbezier[15](0,8)(4,8)(8,8)


%\qbezier[15](0,0)(0,4)(0,8)
\qbezier[15](2,0)(2,4)(2,8)
\qbezier[15](4,0)(4,4)(4,8)
\qbezier[15](6,0)(6,4)(6,8)
%\qbezier[15](8,0)(8,4)(8,8)

\qbezier[10](5,4)(4.5,4.5)(4,5)
\qbezier[20](6,4)(5,5)(4,6)
\qbezier[30](7,4)(5.5,5.5)(4,7)
\qbezier[40](8,4)(6,6)(4,8)
\qbezier[30](8,5)(6.5,6.5)(5,8)
\qbezier[20](8,6)(7,7)(6,8)
\qbezier[10](8,7)(7.5,7.5)(7,8)

\qbezier[40](4,4)(4,2)(4,0)
\qbezier[40](4.5,4)(4.5,2)(4.5,0)
\qbezier[40](5,4)(5,2)(5,0)
\qbezier[40](5.5,4)(5.5,2)(5.5,0)
\qbezier[40](6,4)(6,2)(6,0)
\qbezier[40](6.5,4)(6.5,2)(6.5,0)
\qbezier[40](7,4)(7,2)(7,0)
\qbezier[40](7.5,4)(7.5,2)(7.5,0)
\qbezier[40](8,4)(8,2)(8,0)

\qbezier[35](3.5,3.5)(3.5,1.75)(3.5,0)
\qbezier[30](3,3)(3,1.5)(3,0)
\qbezier[25](2.5,2.5)(2.5,1.25)(2.5,0)
\qbezier[15](2,2)(2,1)(2,0)
\qbezier[10](1.5,1.5)(1.5,.75)(1.5,0)
\qbezier[5](1,1)(1,0.5)(1,0)
\qbezier[3](0.5,0.5)(0.5,0.25)(0.5,0)


\qbezier[40](4,4)(2,4)(0,4)
\qbezier[40](4,4.5)(2,4.5)(0,4.5)
\qbezier[40](4,5)(2,5)(0,5)
\qbezier[40](4,5.5)(2,5.5)(0,5.5)
\qbezier[40](4,6)(2,6)(0,6)
\qbezier[40](4,6.5)(2,6.5)(0,6.5)
\qbezier[40](4,7)(2,7)(0,7)
\qbezier[40](4,7.5)(2,7.5)(0,7.5)
\qbezier[40](4,8)(2,8)(0,8)

\qbezier[35](3.5,3.5)(1.75,3.5)(0,3.5)
\qbezier[30](3,3)(1.5,3)(0,3)
\qbezier[25](2.5,2.5)(1.25,2.5)(0,2.5)
\qbezier[15](2,2)(1,2)(0,2)
\qbezier[10](1.5,1.5)(.75,1.5)(0,1.5)
\qbezier[5](1,1)(0.5,1)(0,1)
\qbezier[3](0.5,0.5)(0.25,0.5)(0,0.5)


\end{picture}
\end{center}}


\caption{The colored fan $\overline{\Sigma}$ of $\overline{Q}$} \label{fig:1}
\end{figure}

\noindent
The Stanley-Reisner ring is $R_{\overline{\Sigma}} \simeq \C[z_1,z_2,z_3]/ {(z_1 z_2 z_3)}$
and we have
%
$$
	P (R_{\overline{\Sigma}}^w , t) =  \displaystyle{\frac{1 - t^5 }{ (1-t) (1-t^2)^2}}.
$$
%
Hence, by Proposition \ref{p:SR}, we get
%
$$
	E_{\rm st}(\overline{Q} ; u,v )
	= \displaystyle{\frac{(1 + uv + (uv)^2)(1+ uv +(uv)^2+(uv)^3) }{ (1+ uv)}} \, .
$$


\smallskip

2) Consider the locally factorial
completion $\overline{X}$ of the affine 7-dimensional cone $X$ over the Grassmannian
$G(2,5)$ from Example \ref{ex:Grass}; $\overline{X}$ is the projective cone over
the Grassmannian $G(2,5)$.
The colored fan $\overline{\Sigma}$ of $\overline{X}$ is represented in Figure \ref{fig:2}.

\begin{figure}[htb]
{\setlength{\unitlength}{0.1in}

\begin{center}
\begin{picture}(8,8)(0,0)

\put(4,4){\line(1,0){4}}
\put(4,4){\line(0,1){4}}
\put(4,4){\line(-1,-1){4}}

\put(6,4){\circle{0.8}}
\put(4,6){\circle{0.8}}

\put(6,4){\circle*{0.4}}
\put(4,6){\circle*{0.4}}

\put(2,2){\circle*{0.4}}

\put(1,1.9){\tiny$1$}
\put(6.1,2.8){\tiny$2$}
\put(2.8,5.9){\tiny$3$}



%\qbezier[15](0,0)(4,0)(8,0)
\qbezier[15](0,2)(4,2)(8,2)
\qbezier[15](0,4)(4,4)(8,4)
\qbezier[15](0,6)(4,6)(8,6)
%\qbezier[15](0,8)(4,8)(8,8)


%\qbezier[15](0,0)(0,4)(0,8)
\qbezier[15](2,0)(2,4)(2,8)
\qbezier[15](4,0)(4,4)(4,8)
\qbezier[15](6,0)(6,4)(6,8)
%\qbezier[15](8,0)(8,4)(8,8)

\qbezier[10](5,4)(4.5,4.5)(4,5)
\qbezier[20](6,4)(5,5)(4,6)
\qbezier[30](7,4)(5.5,5.5)(4,7)
\qbezier[40](8,4)(6,6)(4,8)
\qbezier[30](8,5)(6.5,6.5)(5,8)
\qbezier[20](8,6)(7,7)(6,8)
\qbezier[10](8,7)(7.5,7.5)(7,8)

\qbezier[40](4,4)(4,2)(4,0)
\qbezier[40](4.5,4)(4.5,2)(4.5,0)
\qbezier[40](5,4)(5,2)(5,0)
\qbezier[40](5.5,4)(5.5,2)(5.5,0)
\qbezier[40](6,4)(6,2)(6,0)
\qbezier[40](6.5,4)(6.5,2)(6.5,0)
\qbezier[40](7,4)(7,2)(7,0)
\qbezier[40](7.5,4)(7.5,2)(7.5,0)
\qbezier[40](8,4)(8,2)(8,0)

\qbezier[35](3.5,3.5)(3.5,1.75)(3.5,0)
\qbezier[30](3,3)(3,1.5)(3,0)
\qbezier[25](2.5,2.5)(2.5,1.25)(2.5,0)
\qbezier[15](2,2)(2,1)(2,0)
\qbezier[10](1.5,1.5)(1.5,.75)(1.5,0)
\qbezier[5](1,1)(1,0.5)(1,0)
\qbezier[3](0.5,0.5)(0.5,0.25)(0.5,0)


\qbezier[40](4,4)(2,4)(0,4)
\qbezier[40](4,4.5)(2,4.5)(0,4.5)
\qbezier[40](4,5)(2,5)(0,5)
\qbezier[40](4,5.5)(2,5.5)(0,5.5)
\qbezier[40](4,6)(2,6)(0,6)
\qbezier[40](4,6.5)(2,6.5)(0,6.5)
\qbezier[40](4,7)(2,7)(0,7)
\qbezier[40](4,7.5)(2,7.5)(0,7.5)
\qbezier[40](4,8)(2,8)(0,8)

\qbezier[35](3.5,3.5)(1.75,3.5)(0,3.5)
\qbezier[30](3,3)(1.5,3)(0,3)
\qbezier[25](2.5,2.5)(1.25,2.5)(0,2.5)
\qbezier[15](2,2)(1,2)(0,2)
\qbezier[10](1.5,1.5)(.75,1.5)(0,1.5)
\qbezier[5](1,1)(0.5,1)(0,1)
\qbezier[3](0.5,0.5)(0.25,0.5)(0,0.5)


\end{picture}
\end{center}}


\caption{The colored fan $\overline{\Sigma}$ of $\overline{X}$} \label{fig:2}
\end{figure}

\noindent
We have,
%
$$
	P (R_{\overline{\Sigma}}^w , t) =  \displaystyle{\frac{1 - t^6 }{ (1-t)(1-t^2)(1-t^3)}} \, ,
$$
%
and
%
$$
	E_{\rm st}(\overline{X} ; u,v ) = (1 + (uv)^2)(1 + uv + (uv)^2 + (uv)^3 + (uv)^4 + (uv)^5) .
$$

\end{ex}


\smallskip

It would be interesting to compute the cohomology ring $H^\ast (X_\Sigma , \C)$
of an arbitrary smooth projective horospherical variety $X_\Sigma$
defined by a colored fan $\Sigma$.
%
If $X_\Sigma$ is a toroidal horospherical variety, then $X_\Sigma$ is a toric bundle
over $G/P$, and  a general result of Sankaran
and Uma \cite[Theorem 1.2]{SU03} implies the following description
of the cohomology ring of $X_\Sigma$:


\begin{prop}    \label{p:coh}

Let $X_\Sigma$ be a smooth projective toroidal horospherical variety
defined by a (uncolored) fan $\Sigma$.
Then the cohomology ring $H^\ast (X_\Sigma , \C)$ is isomorphic
to the quotient of  $H^\ast (G/P ,\C) \otimes_\C R_\Sigma$
by the ideal generated by the regular sequences $f_1,\ldots,f_r$
where $f_i $ is given by
$$
	f_i :=  \delta(m_i) \otimes 1 + 1\otimes  \sum_{j=1}^s \langle m_i , e_j \rangle
		\in \big( H^2 (X, \C) \otimes R_\Sigma^0 \big) \oplus
		\big( H^0 (X, \C) \otimes R_\Sigma^1 \big) \, ,
$$
for some integral basis $\{ m_1, \ldots, m_r \}$ of the lattice $M$.


\end{prop}

%
Together with Proposition \ref{p:coh}, our formula (\ref{eq:ai}) in Proposition \ref{p:SR} motivates the following
question:

\begin{question}        \label{q:coh}

Does there exist an analogous description of the cohomology
ring of an arbitrary smooth projective horospherical variety
defined by a colored fan $\Sigma$
which involves the weighted Stanley-Reisner ring $R_\Sigma^w$?

\end{question}


Another interesting question is motivated by Theorem \ref{t:main}:

\begin{question}     \label{q:st}

How to compute  $E_{\rm st}(X ; u,v)$
for an arbitrary $\Q$-Gorenstein spherical $G/H$-embedding?

\end{question}

\begin{rem}
We hope that there is  a formula for $E_{\rm st}(X ; u,v)$ similar to the one
in the horospherical case, e.g.,  which involves
the summation of $(uv)^{\omega_X(n)}$
over all lattice points
in the valuation cone $\mathcal{V}(G/H)$
of the spherical homogeneous space $G/H$.
\end{rem}


A smoothness criterion for arbitrary spherical varieties
was obtained by M. Brion in \cite{Br91}.
Unfortunately, this criterion is difficult to apply in practice.
%
We expect that the smoothness criterion
for locally factorial horospherical varieties (see Theorem \ref{t:smo})
can be extended to arbitrary locally factorial spherical varieties:


%
\begin{conj}       \label{c:smo}

Let $X$ be a locally factorial spherical $G/H$-embedding
whose closed orbits are projective.
Then one has $e_{\rm st}(X) \ge e (X)$, and
the equality holds if and only if $X$ is smooth.

\end{conj}



\begin{thebibliography}{............}



\bibitem[Ba98]{Ba98} V. Batyrev,
{\em Stringy Hodge numbers of varieties with Gorenstein canonical singularities},
in Integrable systems and algebraic geometry (Kobe/Kyoto, 1997)
World Scientific, River Edge, NJ (1998), 1-32.


\bibitem[Ba99]{Ba99} V. Batyrev,
{\em Non-Archimedean integrals and stringy Euler numbers of log-terminal pairs},
J. Eur. Math. Soc. (JEMS) {\bf 1} (1999), no. 1, 5-33.

\bibitem[Bo68]{Bo} N. Bourbaki,
{\em Lie groups and Lie algebras},
Chapters 4--6, Translated from the 1968 French original by Andrew Pressley, Springer-Verlag, Berlin, 2002.


\bibitem[Br89]{Br89} M. Brion,
{\em Groupe de Picard et nombres caract\'eristiques des vari\'et\'es sph\'eriques},
Duke Math. J. {\bf 58} (1989), no. 2, 397-424.


\bibitem[Br91]{Br91} M. Brion,
{\em Sur la g{\'e}om{\'e}trie des vari{\'e}t{\'e}s sph{\'e}riques},
Comment. Math. Helvetici {\bf 66} (1991), 237-262.


\bibitem[Br93]{Br93} M. Brion,
{\em Spherical varieties and Mori theory},
Duke Math. J. {\bf 72} (1993), no. 2, 369-404.


\bibitem[Br97]{Br97} M. Brion,
{\em Curves and divisors in spherical varieties. Algebraic groups and Lie groups},
Austral. Math. Soc. Lect. Ser., {\bf 9},
Cambridge Univ. Press, Cambridge (1997), 21-34.




\bibitem[Cr04]{Cr} A. Craw,
{\em An introduction to motivic integration},
Amer. Math. Soc., Providence,  (2004), 203-225.


\bibitem[D78]{D} V.I. Danilov,
{\em The geometry of toric varieties},
Uspekhi Mat. Nauk {\bf 33} (1978), no. 2(200), 85-134.


\bibitem[DL99]{DL99} J. Denef, F. Loeser,
{\em Germs of arcs on singular varieties
and motivic integration}, Invent. Math. \textbf{135} (1999), 201-232.


\bibitem[D09]{Do09} R. Docampo,
{\em Arcs on Determinantal Varieties}, PhD thesis, University of Illinois
at Chicago, 2009.


\bibitem[EM05]{EM09} L. Ein and M. Musta\c{t}\u{a},
{\em Jet schemes and singularities},
Proc. Sympos. Pure Math., {\bf 80}, Part 2, Amer. Math. Soc., Providence, (2009).



\bibitem[GN10]{GN10} D. Gaitsgory, D. Nadler,
{\em Spherical varieties and Langlands duality},
Mosc. Math. J. {\bf 10}  (2010),  no. 1, 65-137.


\bibitem[I04]{Ish04} S. Ishii,
{\em The arc space of a toric variety},
J. Algebra \textbf{278} (2004), no. 2, 666-683.

%\bibitem[KK11]{KK11} K. Kaveh, A.G. Khovanskii,
%{\em Newton polytopes for horospherical spaces.}
%Mosc. Math. J. 11 (2011), no. 2, 265-283, 407.

\bibitem[KMM87]{KMM} Y. Kawamata, K. Matsuda and K. Matsuki,
{\em Introduction to the Minimal Model Program},
Adv. Studies in Pure Math. {\bf 10} (1987), 283-360.


\bibitem[Kn91]{K91} F. Knop,
{\em The Luna--Vust theory of spherical embeddings},
In: Proceedings of the Hyderabad Conference on Algebraic Groups
(Hyderabad, 1989) (Madras), Manoj Prakashan (1991), 225-249.


\bibitem[Kon95]{Ko95} M. Kontsevich,
{\em Motivic integration}, Lecture at Orsay (1995),
\texttt{http://www.mabli.org/old/jet-preprints/Kontsevich-MotIntNotes.pdf}.


\bibitem[Kos59]{Ko59} B. Kostant,
{\em The principal three-dimensional subgroup and the Betti numbers of a complex simple Lie group},
Amer. J. Math. {\bf 81} (1959), 973-1032.


\bibitem[LV83]{LV83} D. Luna, T. Vust,
{\em Plongements d'espace homog{\`e}nes}, Comment. Math. Helv.,
{\bf 58} (1983), 186-245.


\bibitem[M01]{M01} M. Musta\c{t}\u{a},
{\em Jet schemes of locally complete intersection canonical singularities},
Invent. Math. {\bf 145} (2001), no. 3, 397-424,
with an appendix by D. Eisenbud and E. Frenkel.


\bibitem[OV90]{OV} A.L. Onishchik and E.B. Vinberg,
{\em Lie groups and algebraic groups},
Translated from the Russian and with a preface by D. A. Leites,
Springer Series in Soviet Mathematics, Springer-Verlag, Berlin, 1990.


\bibitem[Pas07]{Pa07} B. Pasquier,
{\em Vari{\'e}t{\'e}s horosph{\'e}riques de Fano},
thesis available at \texttt{http://tel.archives-ouvertes.fr/tel-00111912}.


\bibitem[Pas08]{Pa08} B. Pasquier,
{\em Vari\'{e}t\'{e}s horosph\'{e}riques de Fano},
Bull. Soc. Math. {\bf 136} (2008), no. 2, 195-225.


\bibitem[Pau81]{Pau81} F. Pauer,
{\em Normale Einbettungen von $G/U$}, Math. Ann. {\bf 257} (1981), 371-396.


\bibitem[Pau83]{Pau83} F. Pauer,
{\em Glatte Einbettungen von $G/U$}, Math. Ann. {\bf 262} (1983),
no. 3, 421-429.

\bibitem[SU03]{SU03} P. Sankaran, V. Uma,
{\em Cohomology of toric bundles}, Comment. Math. Helv. {\bf 78} (2003), 540-554.


\bibitem[S74]{Su74} H. Sumihiro,
{\em Equivariant completion}. J. Math. Kyoto Univ. {\bf 14} (1974), 1-28.


\bibitem[T11]{Ti10} D. Timashev,
{\em Homogeneous spaces and equivariant embeddings},
Invariant Theory and Algebraic Transformation Groups, {\bf 8}. Springer, Heidelberg, 2011.


\bibitem[V06]{V} W. Veys,
{\em Arc spaces, motivic integration and stringy invariants},
Advanced Studies in Pure Mathematics {\bf 43},
Proceedings of "Singularity Theory and its applications, Sapporo (Japan), 16-25 september 2003" (2006), 529-572.


\end{thebibliography}




\end{document}

