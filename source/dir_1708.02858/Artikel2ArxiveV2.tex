\documentclass[a4paper,12pt,bibliography=totocnumbered,titlepage=false,abstracton,bookmarksnumbered=true]{scrartcl}
\usepackage[utf8]{inputenc}
\usepackage{amsmath}
\usepackage{amssymb}
\usepackage{amsfonts}
\usepackage{amsthm}
\usepackage{mathtools}
\usepackage{stmaryrd}
\usepackage[pdftex]{graphicx,color}
%\usepackage[labelformat = empty]{caption}
\usepackage[font=small, labelfont=it]{caption}
\usepackage{mathrsfs}
\usepackage[arrow,matrix,curve]{xy}
\usepackage{bbm}
\usepackage{bigints}
\usepackage{epigraph}
\usepackage[]{hyperref}
%\usepackage[a-1b]{pdfx}

\DeclareMathOperator*{\consumt}{\#}
\DeclareMathOperator*{\consumd}{\text{\Large$\#$}}
\DeclareMathOperator{\im}{im}
\DeclareMathOperator{\rk}{rk}
\renewcommand{\figurename}{Fig.}
\newcommand{\dt}{\,\text{dt}}
\newcommand{\hi}[1]{^{( #1 )}}
\newcommand{\veps}{\varepsilon}
\newcommand{\omegac}{\omega_{\scriptscriptstyle\mathbb{C}}}
\newcommand{\sbeta}{{\scriptscriptstyle\beta}}
\newcommand{\overr}{\overrightarrow}
\newcommand{\crita}{crit\mspace{-4mu}\left(\mathcal{A}^H\right)}
\newcommand{\ZN}{\left(\begin{smallmatrix} Z\\ N\end{smallmatrix}\right)}
\newtheorem{defn}{Definition}
\newtheorem{theo}[defn]{Theorem}
\newtheorem*{theo*}{Theorem}
\newtheorem{subtheo}{Theorem}
\renewcommand{\thesubtheo}{\thedefn\alph{subtheo}}
\newtheorem{lemme}[defn]{Lemma}
\newtheorem{prop}[defn]{Proposition}
\newtheorem{cor}[defn]{Corollary}
\newtheorem*{cor*}{Corollary}
\newtheorem{subcor}{Corollary}
\theoremstyle{definition}
\newtheorem*{ex}{Examples}
\newtheorem{dis}[defn]{Discussion}
\newtheorem{remarks}[defn]{Remarks}
\newtheorem*{rem}{Remarks}
\newtheorem*{proper}{Properties}
\newtheorem*{note}{Notation}
\renewcommand{\thesubcor}{\thedefn\alph{subcor}}
\setlength{\parindent}{0pt}
\renewcommand{\proofname}{\textnormal{\textbf{Proof: }}}
\setlength{\epigraphrule}{0pt}
\setlength{\epigraphwidth}{0.6\textwidth}

\title{Cieliebak's Invariance Theorem and contact structures via connected sums}
\author{Alexander Fauck\footnote{Parts of this paper are based on the authors PhD-thesis, during which he was supported by the Studienstiftung des deutschen Volkes, the graduate school of the SFB 647 ``Raum, Zeit, Materie'', and the Berlin Mathematical School.}}

%\date{}
\begin{document}

\maketitle

\begin{abstract}
 We present a strong version of Abouzaid's No-Escape Lemma, which allows varying contact forms on the boundary and which can be used instead of the Maximum Principle. Moreover, we give a clarified proof of Cieliebak's Invariance Theorem for Symplectic homology under subcritical handle attachment. Finally, we introduce the notion of asymptotically finitely generated contact structures, which states essentially that the Symplectic homology in a certain degree of any filling of such contact manifolds is uniformly generated by only finitely many Reeb orbits. This property is then used to show that a large class of manifolds carries infinitely many exactly fillable contact structures. 
\end{abstract}

\tableofcontents

\section{Introduction}
This paper deals with symplectic manifolds $(V,\omega)$, which are $2n$-dimensional manifolds together with a 2-form $\omega$ such that $\omega^n$ is a volume form. Moreover, we also consider the odd dimensional counterparts called contact manifolds $(\Sigma,\xi)$, where $\xi$ is a totally non-integrable hyperplane distribution on $\Sigma$. Both concepts are closely related, namely if $V$ is compact with boundary, then its a natural condition to impose that $\omega$ induces on $\partial V=\Sigma$ a contact structure. Depending on the focus, one calls $\Sigma$ a contact  boundary of $(V,\omega)$ or $(V,\omega)$ a symplectic filling of $(\Sigma,\xi)$.\\
The study of the relations between $(V,\omega)$ and $(\Sigma,\xi)$ is a vast and fruitful field in Symplectic and Contact geometry. It deals with questions as: What topological or symplectic invariants of $(V,\omega)$ are determined by $(\Sigma,\xi)$? Does $(\Sigma,\xi)$ posses a symplectic filling and if so, how many? In as far does $(V,\omega)$ provide contact invariants for $(\Sigma,\xi)$? Can fillings be used to distinguish different contact structures on $\Sigma$? Many of these questions are still open in general though interesting partial results are known.\\
In this paper, we focus solely on exact symplectic manifolds $(V,\lambda)$, where $\lambda$ is a primitive for $\omega$, i.e. $d\lambda=\omega$. Contact manifolds $(\Sigma,\xi)$ which posses exact symplectic fillings $(V,\lambda)$ such that $\partial V=\Sigma$ and $\xi=\ker \lambda|_{T\Sigma}$, are called exactly fillable. For an exact symplectic manifold $(V,\lambda)$ with contact boundary $(\Sigma,\xi)$ one can construct its Symplectic (co)homology $SH_\ast(V)$ resp. $SH^\ast(V)$, which are symplectic invariants of $(V,\lambda)$. This (co)homology is closely related to the contact structure on $\partial V=\Sigma$, as its generators are either critical points of a Morse function on $V$ or closed Reeb orbits on $\Sigma$ (see \ref{secsetup}, \ref{secsymhom} and \ref{secSymCoHom} for details). This connection enabled the author in \cite{Fauck1} to use Symplectic homology\footnote{More precisely Rabinowitz-Floer homology, which can be thought of as a relative version of $SH$.} to prove for the first time rigorously the result by I. Ustilovsky, \cite{Usti}, that the standard spheres $S^{4m+1}$ carry infinitely many different exactly fillable cotact structures not distiguished by algebraic topology. Later, this fact was also shown by Kwon and van Koert, \cite{KwKo}, and Gutt, \cite{Gutt2}, using a variant of $SH$, namely $S^1$-equivariant Symplectic homology. These so called exotic contact structures on $S^{4m+1}$ are all explicitely given as the famous Brieskorn manifolds (see sectio \ref{secBries}).\\
One way to construct new exactly fillable contact manifolds from given ones is by attaching a symplectic handle to the filling $(V,\lambda)$ along the boundary $\Sigma=\partial V$ (see section \ref{secsur}). The effect of this procedure on $(\Sigma,\xi)$ is that of contact surgery which includes in particular contact connected sums. Particular interesting is the connected sum of any exactly fillable contact manifold with a standard sphere carrying an exotic contact structure. In general, it is unknown if this gives always new contact structures on the original manifold. However, we will show in this paper that it holds true on a big subclass of exactly fillable contact manifolds.\\
Here is a brief sketch of the definition of this subclass: A contact form $\alpha$ for $\xi$ is a 1-form such that $\xi=\ker\alpha$. The Reeb vector field $R$ of $\alpha$ is the unique vector field on $\Sigma$ such that $\alpha(R)=1$ and $d\alpha(R,\cdot)=0$ (see section \ref{secsetup}). To closed Reeb orbits $\gamma$ we can associate a Morse-Bott index $\mu$ which is determined by the Conley-Zehnder index of $\gamma$ and a Morse index (see \ref{secConZeh}). We call a contact structure \emph{asymptotically finitely generated in degree $k$ with bound $b_k(\xi)$} (abreviated a.f.g.) if there exists $f_l:\Sigma\rightarrow \mathbb{R}$ with $f_{l+1}\leq f_l\leq 0$ and $(\mathfrak{a}_l)\subset\mathbb{R}$ with $\mathfrak{a}_{l+1}\geq\mathfrak{a}_l$ and $\lim_{l\rightarrow\infty}\mathfrak{a}_l=\infty$ such that for each contact form $\alpha_l=e^{f_l}{\cdot}\alpha$ with Reeb vector field $R_l$ holds that all contractible Reeb orbits of length less than $\mathfrak{a}_l$ are transversely non-degenerate and that of these orbits at most $b_k(\xi)$ have index $k$.\\
We show in the appendix that a contact structure which admits a totally periodic Reeb flow is asymptotically finitely generated in almost all degrees $k$. This includes all Brieskorn manifolds, the unit cotangent bundle of a sphere and the odd dimensional spheres with their usual contact structures. Moreover, we show in Proposition \ref{propcofinalHamilt} that a.f.g. in degree $k$ is invariant under subcritical contact surgery. Starting from the standard sphere $S^{2n-1}$ with the unit ball as exact filling, we hence find that the boundaries of all subcritical Weinstein domains are a.f.g. in almost all degrees. Moreover any connected sums of finitely many contact manifolds which are a.f.g in degree $k$ is again a.f.g in degree $k$. The notion of a.f.g. in degree $k$ allows us to state the following general Theorem.
\begin{theo}\label{theoinfinitecontactstr} 
 Suppose that $(\Sigma,\xi)$ has an exact filling $(V,\lambda)$ such that for the inclusion $i:\Sigma\rightarrow V$ holds $i_\ast: \pi_1(\Sigma)\rightarrow\pi_1(V)$ is injective and the integral 
  $I_{c_1}:\pi_2(V)\rightarrow\mathbb{Z}$ of the first Chern class $c_1(TV)$ vanishes on spheres. Moreover, assume that $(\Sigma,\xi)$ is asymptotically finitely generated in a degree $k\geq \frac{1}{2}\dim V+2$.\\ Then $\Sigma$ carries infinitely many non-contactomorphic exactly fillable contact structures.
\end{theo}
\begin{rem}~
\begin{itemize}
 \item The conditions on $\pi_1(\Sigma)$ and $c_1(TV)$ ensure that the Conley-Zehnder index is well-defined on $\Sigma$. Moreover, both conditions are invariant under attaching a symplectic $k$-handle, at least if $k\neq 2$ (see \cite{FauckThesis}, Lemma 66 and 67).
 \item Note that Theorem \ref{theoinfinitecontactstr} does not require index positivity or dynamical convexity -- assumtions that are usually assumed for results of this form (cf. Espina, \cite{Esp}, or Kwon and van Kort, \cite{KwKo}).
\end{itemize}
\end{rem}
The crucial ingredient for the proof of Theorem \ref{theoinfinitecontactstr} is again the Symplectic homology of fillings and how it behaves under symplectic handle attachement. The founding, and to this date, only paper where this behaviour is explored is \cite{Cie} by K. Cieliebak. There, the following fundamenal result is given:
\begin{theo}[Invariance of $SH$ under subcritical handle attachement]\label{theoinvsur}~\\
 Let $W$ and $V$ be compact $2n$-dimensional symplectic manifolds with positive contact boundaries and assume that the Conley-Zehnder index is well-defined on $W$. If $V$ is obtained from $W$ by attaching to $\partial W$ a subcritical symplectic handle $\mathcal{H},\,k<n$, then it holds that \[SH_\ast(V)\cong SH_\ast(W)\qquad\text{ and }\qquad SH^\ast(V)\cong SH^\ast(W)\qquad \forall\, \ast\in\mathbb{Z}.\]
\end{theo}
This theorem is widely used in the literature. Its applications apart from distinguishing exotic contact structures include the vanishing of Symplectic homology of subcritical Stein manifolds and the proof of certain cases of the Cord conjecture (see \cite{Cie}). Unfortunately, Cieliebaks original proof of this theorem has two flaws:
\begin{enumerate}
 \item[a)] His version of the Maximum Principle is not strong enough for his proof.
 \item[b)] The construction of the special Hamiltonians on the handle is too vague (see Discussion \ref{dis1}).
\end{enumerate}
In this paper, we also deal with both of these issues. First, we present a generalization of Abouzaid's No-Escape Lemma (see \ref{secNoEsc}), which allows for varying contact forms on the boundary $\Sigma$. The advantage  of this is not only that it allows to prove Cieliebaks Invariance Theorem, but it also makes the definition of Symplectic homology  more flexible, allowing a more general class of Hamiltonians. In particular, we can define for an a.f.g. contact structure (with varying contact forms $\alpha_l$) a sequence of Hamiltonians $H_l$ (each adapted to $\alpha_l$ respectively) which defines directly the Symplectic homology.\\
Secondly, we carefully construct the special Hamiltonians, needed in symplectic handle attachement, and we study their dynamics (see section \ref{secsur}). This then leads to the proof of Theorem \ref{theoinvsur} together with Viterbo's transfer maps presented in \ref{sectransfer}.\\
Finally, we turn to Brieskorn manifolds, where we show that taking iteratively connected sums of Ustilovsky's spheres produces new contact structures on $S^{4m+1}$ that are also all non-contactomorphic. This is interesting in its own, since this produces examples which are not distinguished by the Mean Euler characteristic.

\section{Symplectic homology and cohomology}
\subsection{Setup}\label{secsetup}
Let $(V,\omega)$ be an $2n$-dimensional compact symplectic manifold with boundary $\partial V= \Sigma$ such that $\omega =d\lambda$ is exact. The 1-fom $\lambda$ defines the Liouville vector field $Y$ by $\omega(Y,\cdot)=\lambda$. The boundary $\Sigma$ is called a positive/negative contact boundary if $Y$ points out of / into $V$ along $\Sigma$. If $\Sigma$ is a positive contact boundary, then $(V,\lambda)$ is called a Liouville domain.\\
Note that any hypersurface $\Sigma$ in $V$ transverse to $Y$ is a contact manifold, as the 1-form $\alpha:=\lambda|_{T\Sigma}$ is a contact form since $\alpha{\wedge}(d\alpha)^{n-1}\neq0$ pointwise. We write $\xi:=\ker \alpha$ for the contact structure and $R$ for the Reeb vector field defined by $d\alpha(R,\cdot)= 0$ and $\alpha(R)=1$. Note that $\alpha$ defines via $\alpha\wedge(d\alpha)^{n-1}$ an orientation of $\Sigma$. The spectrum $spec(\Sigma,\alpha)$ of a contact form $\alpha$ on $\Sigma$ is then defined by
\[spec(\Sigma,\alpha)=\{\eta\in\mathbb{R}\,|\,\exists\, \text{closed orbit of $R$ with period $\eta$}\}.\]
We say that $\alpha$ is \textit{transversely non-degenerate} if it satisfies the Morse-Bott assumption:
\begin{equation}\label{CondMB}\begin{aligned}
 &\textit{The set $\mathcal{N}^\eta\subset\Sigma$ formed by the $\eta$-periodic Reeb orbits is a submanifold}\\
 &\textit{for all $\eta\in\mathbb{R}$ and $T_p\,\mathcal{N}^\eta = \ker \left(D_p\phi^\eta - \mathbbm{1}\right)$ holds for all $p\in\mathcal{N}^\eta$.}
 \end{aligned}\tag{MB}
\end{equation}
A closed Reeb orbit $x$ is called transversely non-degenerate if (\ref{CondMB}) holds locally.\\
A symplectization of a contact manifold $\Sigma$ with contact form $\alpha$ is a manifold $N=I\times\Sigma$, where $I\subset\mathbb{R}$ is an interval, together with the symplectic form $\omega:=d(e^r\alpha)$, $r\in I$. For $I=\mathbb{R}, I=[0,\infty)$ or $I=(-\infty,0]$, we call $(N,\omega)$ the whole/positive/negative symplectization of $\Sigma$. If $\beta$ is a different contact form on $\Sigma$ defining the same contact structure, i.e.\ $\ker \beta = \xi=\ker\alpha$,  and the same orientation then we find a function $f:\Sigma\rightarrow\mathbb{R}$ such that $\beta_p=e^{f(p)}{\cdot} \alpha_p \;\forall p\in\Sigma$. To such a $\beta$, we associate the following hypersurface in the whole symplectization of $\Sigma$:
\[\Sigma_\beta :=\big\{(f(p),p)\,\big|\,p\in\Sigma\big\}.\]
Note that $(e^r{\cdot}\alpha)|_{T\Sigma_\beta}=\beta$ and that $\Sigma$ and $\Sigma_\beta$ are naturally diffeomorphic.\\
The flow $\varphi_Y$ of the Liouville vector field $Y$ on a Liouville domain $(V,\lambda)$ with contact boundary $(\Sigma=\partial V,\alpha=\lambda|_{T\Sigma})$ allows us to identify a collar neighborhood of $\Sigma$ with the negative symplectization $\big((-\infty,0]{\times}\Sigma, d(e^r\alpha)\big)$. This holds true as $Y$ points out of $V$ along $\Sigma$ and $V$ is compact, so that $\varphi_Y^t$ is well-defined for all $t\leq 0$. Moreover, we have
\begin{align*}
 \mathcal{L}_Y \omega &=\iota_Y d\omega + d(\iota_Y\omega)=0+d\lambda=\omega\\
 \mathcal{L}_Y \lambda &=\iota_Y d\lambda + d(\iota_Y\lambda)=\iota_Y\omega+d\omega(Y,Y)=\lambda,
\end{align*}
so that $\varphi_Y$ expands $\omega|_{T\Sigma}$ and $\alpha=\lambda|_{T\Sigma}$ exponentially over time. This identification allows us to define the completion $(\widehat{V},\widehat{\lambda})$ of $(V,\lambda)$ by
\[\widehat{V}:=V\cup_{\varphi_Y}\big((-\delta,\infty]{\times}\Sigma\big)\qquad\widehat{\lambda}:=\begin{cases}\lambda &\text{on $V$}\\e^r\alpha & \text{on $\mathbb{R}{\times}\Sigma$.}\end{cases}\]
For the moment write $V=V_\alpha$. As mentioned above, a different contact form $\beta$ for the same contact structure $\xi=\ker \alpha$ defines a contact manifold $\Sigma_\beta$ in $\mathbb{R}{\times}\Sigma$. As $(\mathbb{R}{\times}\Sigma,d(e^r\alpha))$ embedds symplectically into $(\widehat{V},\widehat{\omega})$, we can think of $\Sigma_\beta$ as being an embedded contact hypersurface in $\widehat{V}$. Note that $\Sigma_\beta$ bounds a compact region $V_\beta\subset\widehat{V}$ and that $(V_\beta,\widehat{\lambda}|_{V_\beta})$ is a Liouville domain with contact boundary $(\Sigma_\beta,\beta)$. We remark that the completions $\widehat{V}_\beta$ and $\widehat{V}_\alpha=\widehat{V}$ are naturally identified.\\
More generally, given two Liouville domains $(V_1,\lambda_1)$ and $(V_2,\lambda_2)$, a diffeomorphism  $\varphi : \widehat{V}_1\rightarrow \widehat{V}_2$ is called a Liouville isomorphism if $\varphi^\ast\widehat{\lambda}_2=\widehat{\lambda}_1+dg$ for a compactly supported function $g$. It is shown in \cite{Sei}, page 3, that for any Liouville isomorphism $\varphi$ there exist $R\in\mathbb{R}$, $f\in C^\infty(\partial V_1)$ and a contactomorphism $\psi: \partial V_1\rightarrow \partial V_2$ satisfying $\psi^\ast \lambda_2|_{\partial V_2}=e^f{\cdot} \lambda_1|_{\partial V_1}$, such that on $[R,\infty){\times} \partial V_1\subset \widehat{V}_1$ the map $\varphi$ has the form
\[\varphi(r,p)=(r-f(p),\psi(p)).\]
It follows that any Liouville isomorphism preserves the contact structure of the boundary, as $\ker e^f{\cdot}\lambda_1|_{\partial V_1}=\ker \lambda_1|_{\partial V_1}$. The corresponding contact forms on the other hand may change arbitrarily. For example, if we consider $(V_\alpha,\widehat{\lambda}|_{V_\alpha})$ and $(V_\beta,\widehat{\lambda}|_{V_\beta})$ as above in $(\widehat{V},\widehat{\lambda})$ and if we identify $\widehat{V}_\alpha=\widehat{V}=\widehat{V}_\beta$, then $\varphi=Id: \widehat{V}_\alpha\rightarrow\widehat{V}_\beta$ provides a Liouville isomorphism. As we may think of $(V_\alpha,\widehat{\lambda}|_{V_\alpha})$ and $(V_\beta,\widehat{\lambda}|_{V_\beta})$ as essentially the same filling for $(\Sigma,\xi)$, depending only on the contact form but not on the contact structure, we are hence led to the following definition:
\begin{defn}\label{filling}
 Let $(\Sigma,\xi)$ be a contact manifold. If there exists a Liouville domain $(V,\lambda)$ such that $\partial V=\Sigma$ and $\xi=\ker \lambda|_\Sigma$, then we call the equivalence class of $(V,\lambda)$ under Liouville isomorphisms an \textbf{\textit{exact (contact) filling}} of $(\Sigma,\xi)$.
\end{defn}
It will turn out that Symplectic homology is in fact invariant under Liouville isomorphisms (see section \ref{secsymhom}) thus providing an invariant for contact structures (with a filling).\bigskip\\
A Hamiltonian on $\widehat{V}$ is a smooth $S^1$-family of functions $H_t: \widehat{V}\rightarrow \mathbb{R}$ with Hamiltonian vector field $X_H^t$ defined by $\omega(\cdot,X^t_H)=dH_t$ for each $t\in S^1$. The Hamiltonian action of a loop $x: S^1\rightarrow \widehat{V}$ with respect to $H$ is defined by
\[\mathcal{A}^H(x)=\int^1_0 x^\ast\lambda - \int^1_0 H_t(x(t)) dt.\]
The critical points of the functional $\mathcal{A}^H$ are exactly the closed 1-periodic orbits of $X^t_H$. We denote the set of these solutions by $\mathcal{P}(H)$. Let $J_t$ denote an $S^1$-family of $\omega$-compatible almost complex structures. As usual, $\omega$-compatible means that $\omega(\cdot,J_t\cdot)$ defines a Riemannian metric on $V$ for every $t$. The $L^2$-gradient of $\mathcal{A}^H$ with respect to this metric is then given by
\[\nabla\mathcal{A}^H(x)=-J(\partial_t x-X_H^t).\]
An $\mathcal{A}^H$-gradient trajectory $u:\mathbb{R}\times S^1\rightarrow \widehat{V}$ is hence a solution of the following partial differential equation:
\begin{equation}\label{eqast}
 \partial_s u-\nabla\mathcal{A}^H=\partial_s u + J(\partial_t u-X_H^t)=0\qquad\Leftrightarrow \qquad\, \big(Du-X_H^t{\otimes} dt\big)^{0,1}=0.\quad
\end{equation}
For the second equation, recall that the differential $Du$ of $u$ can be viewed as a 1-form on $\mathbb{R}\times S^1$ with values in $TV$ and that the antiholomorphic part of such differential forms $\beta$ is given by $\beta^{0,1}:=\frac{1}{2}(\beta+J\beta j)$, where $j$ is the standard almost complex structure on $\mathbb{R}\times S^1$, defined by $j\partial_s=\partial_t$. In the course of this article, we will also be interested in homotopies $H_s$ of Hamiltonians. In this case, we call $\mathcal{A}^{H_s}$-gradient trajectories solutions of (\ref{eqast}) with $X_H^t$ and $J$ depending on $s$.
\subsection{The No-Escape Lemma}\label{secNoEsc}
For the construction of symplectic (co)homology we look at solutions $u$ of (\ref{eqast}) satisfying $\displaystyle\lim_{s\rightarrow\pm\infty}u(s,t)=x_{\pm}(t)\in\mathcal{P}(H)$. In general, these solutions might not stay in a compact subset of $\widehat{V}$, even for $x_\pm$ fixed. So it could be that the moduli space of these solutions is neither compact nor has a suitable compactification. However, the No-Escape Lemma (see below) shows that for certain pairs $(H,J)$ all such $u$ stay in a compact set.\\
In order to state the lemma in full generality, let $(W,d\lambda)$ be an exact symplectic manifold with compact negative contact boundary and such that the flow $\varphi_Y^t$ of the Liouville vector field $Y$ exists for all $t\geq 0$. Then $\varphi_Y$ provides a symplectic embedding of $\big([0,\infty){\times}\partial W, d(e^r\alpha)\big)$ into $W$. For example, consider $\big([-\delta,\infty){\times}\Sigma, d(e^r\alpha)\big)$ inside $(\widehat{V},\widehat{\omega})$.\\
Let $f_s:\partial W\rightarrow\mathbb{R}$ be a smooth family of functions, such that for some $s_0\geq 0$ holds $f_s\equiv f_{\pm s_0}$ for $|s|\geq s_0$. They define on $[0,\infty)\times\partial W$ an $s$-dependent coordinate change by $r_s:=r-f_s$ and a compact $s$-depending family of contact hypersurfaces $\Sigma_s$ by 
\[\Sigma_s:=\{r_s\equiv R_0\}=\big\{(R_0+f_s(p),p)\,\big|\,p\in\partial W\big\} \quad \text{ for a constant }\quad R_0\geq - \min_{s,p} f_s(p).\]
Let $J_s$ be an $s$-dependent family of almost complex structures. We assume that $J_s$ is of contact type along $\Sigma_s$, meaning that $J_s^\ast\lambda = d(e^{r_s})$ holds for fixed $s$ at all points $p\in\Sigma_s$. Let $H_s: W\rightarrow \mathbb{R}$ be a homotopy of Hamiltonians such that
\begin{align}
 &\bullet& H_s(r,p)=h_s\big(e^{r-f_s(p)}\big)&=h_s(e^{r_s})&&\text{ near }\Sigma_s,\notag\\
 &\bullet&\label{eqconditiononH} \partial_s\Big(H_s-h_s(e^{R_0})+e^{R_0}\cdot h_s'(e^{R_0})\Big)&\leq 0&&\text{ everywhere on $W$.}
\end{align}
Finally, let $S\subset \mathbb{R}\times S^1$ be a compact Riemann surface with smooth boundary.
\begin{lemme}[\textbf{No-Escape Lemma}]\label{maxprinc}~\\
Let $W, S, J, r_s, R_0$ and $H_s$ be as above. Assume that for a solution $u: S\rightarrow W$ of (\ref{eqast}) holds that $u(s,t)\in\Sigma_s$ for $(s,t)\in\partial S$ and $e^{r_s}\circ u(s,t)\geq e^{R_0}$ for all $(s,t)$.\\
Then it holds for all $(s,t)$ that $ u(s,t)\in \bigcup_{s\in\mathbb{R}} \Sigma_s =\big\{p\in W\big|\exists\, s\in\mathbb{R}: p\in\Sigma_s\big\}$.
\end{lemme}
\begin{rem}~
\begin{itemize}
 \item If $h_s$ and $f_s$ are independent of $s$, then condition (\ref{eqconditiononH}) is empty, i.e.\ the No-Escape Lemma holds for all $(H,J)$ that are cylindrical along a fixed $\Sigma$.
 \item If $H$ is linear in $e^{r_s}$ along $\Sigma_s$, i.e.\ $H_s=\mathfrak{a_s}e^{r_s}+\mathfrak{b}_s$, then (\ref{eqconditiononH}) reads as $\partial_s\big(H_s-\mathfrak{b}_s)\leq 0$.
 \item If $W=[0,\infty)\times\partial W$ and $H_s$ is linear everywhere, then (\ref{eqconditiononH}) reads as $\partial_s (\mathfrak{a}_s e^{r-f_s})\leq 0$, which is equivalent to $\partial_s (\log \mathfrak{a}_s-f_s)\leq 0$.
 \item If $W=[0,\infty)\times\partial W$ and $f_s=0$, then (\ref{eqconditiononH}) can be replaced by $\partial_s h'\leq 0$, as
 \[(\partial_s h_s)(e^r\circ u)-(\partial_s h_s)(e^{R_0})+e^{R_0}\cdot (\partial_s h_s')(e^{R_0})=\int_{e^{R_0}}^{e^r\circ u}\partial_s h'(t)\,dt+e^{R_0}\cdot (\partial_s h_s')(e^{R_0}).\]
 \item The No-Escape Lemma and Sard's theorem imply the following corollary.
\end{itemize}
\end{rem}
\begin{cor}\label{Cornoesc}
 Let $V_0\subset \widehat{V}$ be a relatively compact open set with contact boundary $\partial V_0$, let $H: \widehat{V}\rightarrow \mathbb{R}$ be a Hamiltonian satisfying (\ref{eqconditiononH}) on $W:=\widehat{V}\setminus V_0$ and let $J$ be an almost complex structure which is cylindrical along a collar neighborhood of $\partial V_0$. Then any solution $u:\mathbb{R}\times S^1\rightarrow \widehat{V}$ to (\ref{eqast}) with asymptotes in $V_0$ stays inside $V_0$ for all time.
\end{cor}
\begin{proof}[\textbf{Proof of the No-Escape Lemma:}]~\\
 At first, we calculate $\lambda$ applied to the Hamiltonian vector field on $\Sigma_s$:
 \begin{equation*}
  \lambda(X_{H_s})=d\lambda(Y,X_{H_s})=dH_s(Y)=\partial_r H_s(r,y)=h_s'(e^{r-f_s(p)})\cdot e^{r-f_s(p)}=h'_s(e^{R_0})\cdot e^{R_0},\tag{$\ast$}
 \end{equation*}
 where the last equality holds only on $\Sigma_s$, as there $r-f_s(p)=r_s=R_0$. We define the energy $E_S(u)$ of $u$ over $S$ as $E_S(u):=\int_S ||\partial_s u||^2 ds\wedge dt$. Clearly, $E_S(u)$ is non-negative. Using a trick of M. Abouzaid, we will show that $E_S(u)\leq 0$ and hence $E_S(u)=0$, so that $\partial_s u \equiv 0$. As $u|_{\partial S}\subset \bigcup_{s\in\mathbb{R}} \Sigma_s$ and $S\subset \mathbb{R}\times S^1$, this implies that $u(s,t)\in \bigcup_{s\in\mathbb{R}} \Sigma_s$ for all $(s,t)\in S$. To prove $E_S(u)\leq 0$, we calculate:
 \begin{align*}
  &\phantom{\,=\,}\,E_S(u) =\int_S||\partial_s u||^2ds{\wedge} dt =\int_S d\lambda(\partial_s u, J\partial_s u)ds{\wedge} dt\\
  &=\int_S d\lambda(\partial_s u,\partial_t u)-d\lambda(\partial_s u, X_{H_s})ds{\wedge} dt \qquad =\int_S u^\ast d\lambda - dH_s(\partial_s u)ds{\wedge} dt\\
  &=\int_S u^\ast d\lambda -\partial_s \big(H_s(u)\big)ds{\wedge} dt + (\partial_s H_s)(u) ds{\wedge} dt\\
  &=\int_S u^\ast d\lambda- d\big(H_s(u) dt\big) +(\partial_s H_s)(u) ds{\wedge} dt =\int_{\partial S} u^\ast\lambda- H_s(u) dt +\int_S(\partial_s H_s)(u) ds{\wedge} dt\\
  &=\int_{\partial S} u^\ast\lambda-\Big(\lambda(X_{H_s})(u)-\lambda(X_{H_s})(u)-H_s(u)\Big)dt +\int_S(\partial_s H_s)(u) ds{\wedge} dt\\
  &\overset{(\ast)}{=}\int_{\partial S}\lambda\big(Du-X_{H_s}{\otimes} dt\big)+\int_{\partial S}\Big(h'_s\big(e^{R_0}(u)\big) e^{R_0}(u)-h_s(e^{R_0})\Big)dt+\int_S(\partial_s H_s)(u) ds{\wedge} dt\\
  &=\int_{\partial S}-\lambda J\big(Du - X_{H_s}{\otimes} dt\big)j + \int_S\partial s \Big(h'_s\big(e^{R_0}(u)\big) e^{R_0}(u)-h_s(e^{R_0})\Big)+(\partial_s H_s)(u) ds{\wedge} dt\\
  &\overset{(\ref{eqconditiononH})}{\leq}\int_{\partial S}-de^{r_s}(Du-X_{H_s}{\otimes} dt)j=\int_{\partial S}-de^{r_s}(Du)j.
 \end{align*}
Here, we used that orbits of $X_{H_s}$ stay inside level sets of $e^{r_s}$, so that $de^{r_s}(X_{H_s})=0$.\\ To calculate the last integral, let $n$ be the outward normal direction along $\partial S\subset S$. Then $(n,jn)$ is an oriented frame and hence $\partial S$ is oriented by $jn$. So along  $\partial S$ holds
\[-de^{r_s}(Du)j(jn)=-d(e^{r_s}\circ u)(-n)\leq 0,\]
as in the inward direction $-n$, $e^{r_s}\circ u$ can only increase since $e^{r_s}\circ u$ attains its minimum $e^{R_0}$ along $\partial S$. So $E_S(u)\leq 0$ and hence $E_S(u)=0$.
\end{proof}

A 1-periodic orbit $x\in\mathcal{P}(H)$ is called \textit{non-degenerate} if the flow $\varphi^t_{X_H}$ of $X_H$ satisfies $\det \big(D\varphi^1_{X_H}(x(0))-\mathbbm{1}\big)\neq 0$. It is called \textit{transversely non-degenerate} if near $x$ holds that $\mathcal{N}:=\big\{y(0)\,\big|\,y\in\mathcal{P}(H)\big\}$ is a submanifold of $V$ such that $\ker\big(D\varphi^1_{X_H}(y(0))-\mathbbm{1}\big)=T_{y(0)}\mathcal{N}$ for all $y\in\mathcal{P}(H)$ near $x$. Note that $\mathcal{N}$ is always closed and consists of finitely many points if all orbits are non-degenerate.\\
In view of the No-Escape Lemma (Lemma \ref{maxprinc}) we make the following definitions:
\begin{itemize}
 \item A Hamiltonian $H$ is \emph{admissible}, writing $H\in Ad(V)$, if all 1-periodic orbits of $X_H$ are (transversely) non-degenerate and if $H$ is \emph{(weakly) linear at infinity}, that is if there exist $\mathfrak{a},\mathfrak{b},R\in\mathbb{R}$ and $f\in C^\infty(\Sigma)$ such that $\mathfrak{a}\not\in spec(\Sigma,e^{f(p)}{\cdot} \alpha)$ and $H$ is on $[R,\infty){\times}\Sigma\subset \widehat{V}$ of the form
 \[ H(r,p)=\mathfrak{a}\cdot e^{r-f(p)}+\mathfrak{b}.\]
 \item A homotopy $H_s$ between admissible Hamiltonians $H_\pm$ is admissible if there exist $S,R\geq0$ such that $H_s=H_\pm$ for $\pm s\geq S$ and $H_s$ has on $[R,\infty)\times\Sigma$ the form
 \[H_s=\mathfrak{a}_s\cdot e^{r-f_s(p)}+\mathfrak{b}_s \qquad \text{ with }\qquad \partial_s\big(\mathfrak{a}_s\cdot e^{r-f_s(p)}\big)\leq 0.\]
 \item A possibly $s$-dependent almost complex structure $J$ is admissible for a Hamiltonian/homotopy $H$, if for some $\displaystyle R_0\geq \min \{R-f_s(p)\,|\,p\in\Sigma,s\in\mathbb{R}\}$ holds that  $J_s$ is of contact type near $\Sigma_s:=\big\{r{-}f_s(p)=R_0\big\} \subset\mathbb{R}{\times}\Sigma$, meaning that
 \[\lambda\circ J_s=d\big(e^{r-f_s}\big) \qquad\text{ holds for $s$ fixed and all }(r,p)\in(-\veps,\veps){\times}\Sigma_s.\]
\end{itemize}

\subsection{Symplectic homology}\label{secsymhom}
For an admissible Hamiltonian $H$ with all 1-periodic orbits non-degenerate, we define the Floer homology $FH_\ast(H)$ as follows: The chain group $FC_\ast(H)$ is the $\mathbb{Z}_2$-vector space generated by $\mathcal{P}(H)$. Note that due to $\mathfrak{a}\not\in spec(\Sigma,e^{f(p)}\cdot\alpha)$ and the non-degeneracy of the 1-periodic orbits, we find that $\mathcal{P}(H)$ is in fact a finite set and hence $FC_\ast(H)$ has the finite dimension $|\mathcal{P}(H)|$. For $x_\pm\in\mathcal{P}(H)$ let $\widehat{\mathcal{M}}(x_-,x_+)$ denote the space of solutions $u$ of (\ref{eqast}) with $\displaystyle\lim_{s\rightarrow\pm\infty} u = x_\pm$. There is an $\mathbb{R}$-action on this space given by time shift. The quotient under this action is called the moduli space of $\mathcal{A}^H$-gradient trajectories between $x_-$ and $x_+$ and denoted by $\mathcal{M}(x_-,x_+):=\widehat{\mathcal{M}}(x_-,x_+)/\mathbb{R}$.\\
For a generic $J$, the space $\mathcal{M}(x_-,x_+)$ is a manifold. Its zero-dimensional component $\mathcal{M}^0(x_-,x_+)$ is compact and hence a finite set. Let $\#_2\mathcal{M}^0(x_-,x_+)$ denote its cardinality modulo 2. We define the operator $\partial: FC_\ast(H)\rightarrow FC_\ast(H)$ as the linear extension of
\[\partial x:=\sum_{y\in\mathcal{P}(H)}\#_2\mathcal{M}^0(y,x)\cdot y.\]
A standard argument in Floer theory, involving the compactification of $\mathcal{M}^1(y,x)$, shows that $\partial^2=0$, so that $\partial$ is a boundary operator. We set as usual $FH_\ast(H):=\ker \partial\big/\text{im }\partial$.\\
To an admissible homotopy $H_s$ between admissible Hamiltonians $H_\pm$ we consider for $x_\pm\in\mathcal{P}(H_\pm)$ the moduli space of $s$-dependent $\mathcal{A}^{H_s}$-gradient trajectories $\mathcal{M}_s(x_-,x_+)$. Note that we have no time shift, as equation (\ref{eqast}) now depends on $s$. On chain level, we define a map $\sigma_\sharp(H_-,H_+):FC_\ast(H_+)\rightarrow FC_\ast(H_-)$ as the linear extension of 
\[\sigma_\sharp(H_-,H_+)x_+=\sum_{x_-\in\mathcal{P}(H_-)}\#_2\mathcal{M}^0_s(x_-,x_+)\cdot x_-.\]
By considering the compactification of $\mathcal{M}^1_s(x_-,x_+)$, we obtain from Floer theory that $\partial\circ \sigma_\sharp =\sigma_\sharp\circ \partial$, so that $\sigma_\sharp(H_-,H_+)$ is a chain map, which descends to a map  $\sigma_\ast(H_-,H_+):FH_\ast(H_+)\rightarrow FH_\ast(H_-)$, called continuation map. Considering homotopies of homotopies, one can show that $\sigma_\ast(H_-,H_+)$ is independent of the chosen homotopy. For three admissible Hamiltonians $H_1, H_2$ and $H_3$, the continuation maps obey the composition rule
\[\sigma_\ast(H_1,H_3)=\sigma_\ast(H_1,H_2)\circ\sigma_\ast(H_2,H_3).\]
We introduce a partial ordering $\prec$ on $Ad(V)$ by saying $H_+\prec H_-$ if and only if for some $R$ holds on $[R,\infty)\times\Sigma$ that $H_+{-}H_-$ is constant or $H_+\leq H_-$. Observe that admissibility of a homotopy $H_s$ between $H_-$ and $H_+$ implies that $H_+\prec H_-$. It follows from the above that the groups $FH_\ast(H)$ together with the maps $\sigma_\ast(H_-,H_+)$ for $H_+\prec H_-$ define a direct system over the directed set $(Ad(V),\prec)$. The Symplectic homology groups $SH_\ast(V)$ are then defined to be the direct limit of this system:
\[SH_\ast(V):=\varinjlim_{H\in Ad(V)} FH_\ast(H).\]
As we want to use Symplectic homology to distinguish contact structures, we will sometimes also write $SH_\ast(V,\partial V)$ for $SH_\ast(V)$ to emphasize the connection with the contact boundary $(\partial V,\lambda|_{T\partial V})$.\\
A \emph{cofinal sequence} $(H_n)\subset Ad(V)$ is a sequence of Hamiltonians such that $H_n\prec H_{n+1}$ and for any $H\in Ad(V)$ there exists $n\in\mathbb{N}$ such that $H\prec H_n$. Recall that a direct limit can be computed from any cofinal sequence, i.e.\ that $\displaystyle SH_\ast(V)=\varinjlim_{n\rightarrow \infty}FH_\ast(H_n)$.\\
More general, a set $\mathcal{F}\subset Ad(V)$ is cofinal if for any $H\in Ad(V)$ there exists $F\in \mathcal{F}$ such that $H\prec F$. For $\mathcal{F}$ cofinal holds again $\displaystyle SH_\ast(V)=\varinjlim_{F\in\mathcal{F}} FH_\ast(F)$.\\
When calculating $SH_\ast(V)$, it is usefull to consider autonomous (i.e.\ time-independent) Hamiltonians $H$. However, the 1-periodic orbits of such $H$ come in families, unless they are constant. Typically, this situation arises if on a symplectization $\mathbb{R}\times\Sigma$ the Hamiltonian is of the form $H(r,p)=h(e^r)$, where the 1-periodic orbits on level $e^r\equiv c$ correspond to closed Reeb orbits of period $h'(c)$. These orbits are transversely non-degenerate if $(\Sigma,\alpha)$ satisfies (\ref{CondMB}). In this situation, we define $FH_\ast(H)$ as follows: Choose a Morse-function $f$ on the manifold $\mathcal{N}$ of the closed 1-priodic orbits of $H$. Then
\begin{itemize}
 \item either perturb $H$ time-depending using $f$ to a time-dependent Hamiltonian $\widetilde{H}$ having exactly one non-degenerate 1-periodic orbit for each critical point of $f$ (see \cite{Oan} or \cite{CiFlHoWy}) or
 \item let $\mathcal{P}(H,f)$ consist of the critical points of $f$, let $FC_\ast(H)$ be the $\mathbb{Z}_2$-vector space generated by $\mathcal{P}(H,f)$ and let $\mathcal{M}(x_-,x_+)$ consist of unparametrized flow lines with cascades between $x_\pm\in\mathcal{P}(H,f)$. Here, a flow line with cascades is a tuple $(u_1,u_2,...,u_m)$ whose components are solutions of (\ref{eqast}) and satisfy
 \begin{itemize}
  \item $\displaystyle \lim_{s\rightarrow-\infty} u_1(s,0)$ lies in the unstable manifold of $x_-$ and $\displaystyle\lim_{s\rightarrow+\infty} u_m(s,0)$ lies in the stable manifold of $x_+$, both with respect to the gradient flow of $f$ on $\mathcal{N}$,
  \item for $i=1,...,m{-}1$, the limit orbits $\displaystyle \lim_{s\rightarrow+\infty}u_i$ and $\displaystyle \lim_{s\rightarrow-\infty}u_{i+1}$ lie in the same component of $\mathcal{N}$ and are connected by a positive gradient flow line of $f$ with finite (possibly zero) length.
 \end{itemize}
For $x\in\mathcal{P}(H,f)$, we define $\partial x$ by $\displaystyle \qquad\partial x:=\sum_{y\in\mathcal{P}(H,f)}\#_2\mathcal{M}^0(y,x)\cdot y.$\medskip\\
One can show that this $\partial$ is well-defined and satisfies $\partial ^2=0$. The resulting homology is still denoted by $FH_\ast(H)$ and forms a directed system in exactly the same way as in the non-autonomous case. Its direct limit is again the Symplectic homology $SH_\ast(V)$.\\
In the case where $\mathcal{N}$ consists of isolated circles, this approach was carried out in detail by Bourgeois and Oancea  in \cite{BourOan1}. Though not stated explicitly, their methods are general enough to work also if $\mathcal{N}$ is of higher dimension (see also \cite{Fra}, App. A and \cite{FauckThesis} for Morse-Bott constructions using flow lines with cascades).
\end{itemize}

\subsection{The Conley-Zehnder index}\label{secConZeh}
We can $\mathbb{Z}$-grade Symplectic homology by the Conley-Zehnder index $\mu_{CZ}$. We restrict ourselves to contractible 1-periodic orbits of $X_H$, which is no restriction if the manifold $V$ is simply connected. Otherwise, note that any solution $u$ of (\ref{eqast}) with $\lim_{s\rightarrow\pm\infty}=x_\pm$ provides a homotopy between $x_-$ and $x_+$, so that the contractible 1-periodic orbits of $X_H$ form a subcomplex of $FC_\ast(H)$.\\
Moreover, we assume for the first Chern class $c_1(TV)$ that $\int_{S^2} s^\ast c_1(TV)=0$ for every continuous map $s:S^2\rightarrow V$. If the map $i_\ast:\pi_1(\partial V)\rightarrow\pi_1(V)$ induced by the inclusion is injective, then the grading is even independent from $V$.\\
To compute $\mu_{CZ}(v)$ for a closed contractible 1-periodic Hamiltonian orbit $v$ choose a map $u$ from the unit disc $D\subset\mathbb{C}$ to $V$ such that $u(e^{2\pi it})=v(t)$. Then choose a symplectic trivialization $\Phi:D{\times}\mathbb{R}^{2n}\rightarrow u^\ast TV$ of the pullback bundle $(u^\ast TV, u^\ast \omega)$. Such trivializations exist and are homotopically unique as $D$ is contractible. The linearization of the Hamiltonian flow $\varphi^t_{X_H}$ along $v$ with respect to $\Phi$ defines a path $\Psi$ in the group $Sp(2n)$ starting at $\mathbbm{1}$ via
\[\Psi(t):=\Phi(v(t))^{-1}\circ D\varphi^t_{X_H}(v(0))\circ\Phi(v(0)).\]
The Conley-Zehnder index of this path is $\mu_{CZ}(v)$. Its definition is independent of $u$ due to the assumption on $c_1(TV)$. In \cite{RoSa} and \cite{Sal}, Robbin and Salamon defined $\mu_{CZ}$ for paths in $Sp(2n)$ as a Maslov type index as follows. Every smooth path $\Phi:[a,b]\rightarrow Sp(2n)$ can be uniquely expressed as a solution of an ODE of the form
\[\frac{d}{dt}\Phi(t)=J_0 S(t)\Phi(t),\qquad \Phi(a)\in Sp(2n),\]
where $t\mapsto S(t)=S(t)^T$ is a smooth path of symmetric matrices.
A time $t$ is called a crossing if $\det(\Phi(t){-}\mathbbm{1})=0$. The index $\mu_{CZ}$ is the sum over all crossings $t$ of the signatures of $S(t)$ restricted to $\ker (\Phi(t){-}\mathbbm{1})$. If the end points $a$ and $b$ are crossings, then only half the signature is added. The resulting index has in particular the following properties (see \cite{RoSa} and \cite{Gutt1}):
\begin{description}
 \item[(CZ0)] If $sign(S)=0$ everywhere, then $\mu_{CZ}(\Phi)=0$.\label{CZ0}
 \item[(CZ1)] If $\Phi:[0,T]\rightarrow Sp(2n),\,\Phi(t)=e^{it}$, then $\displaystyle\mu_{CZ}(\Phi)=\left\lfloor\frac{T}{2\pi}\right\rfloor + \left\lceil\frac{T}{2\pi}\right\rceil.$
 \item[(product)] For $\Phi{\oplus}\Phi': [a,b]\rightarrow Sp(2n){\oplus} Sp(2n')\subset Sp\big(2(n{+}n')\big)$ holds \\$\mu_{CZ}(\Phi{\oplus}\Phi')=\mu_{CZ}(\Phi)+\mu_{CZ}(\Phi')$.
 \item[(naturality)] $\mu_{CZ}(\Psi\Phi\Psi^{-1})=\mu_{CZ}(\Phi)$, if $\Psi: [a,b]\rightarrow Sp(2n)$ is a contractible loop.
 \item[(zero)] If $\dim \ker(\Phi(t){-}\mathbbm{1})=k$ is constant on $[a,b]$, then $\mu_{CZ}(\Phi)=0$.
 \item[(homotopy)] $\mu_{CZ}(\Phi_0)=\mu_{CZ}(\Phi_1)$, if $\Phi_s, \, s\in[0,1]$, is a homotopy with fixed endpoints.
 \item[(catenation)] $\mu_{CZ}(\Phi|_{[a,b]})=\mu_{CZ}(\Phi|_{[a,c]})+\mu_{CZ}(\Phi|_{[c,b]})$ for any $a<c<b$. 
\end{description}
In the autonomous case, a transversely non-degenerate orbit $v\in\mathcal{P}(H,f)$ is not graded by the Conley-Zehnder index alone, but by the Morse-Bott index (see \cite{BourOan1} or \cite{FraCie})
\begin{equation}\label{eqmu}
 \mu(v)=\mu_{CZ}(v)+\mu_{Morse}(v)-{\textstyle\frac{1}{2}}\dim_v\mathcal{N}+{\textstyle\frac{1}{2}} sign \big(h''(e^r)\big),
\end{equation}
where $\mu_{Morse}(v)$ is the Morse index of $v$, $\dim_v\mathcal{N}$ is the dimension of the connected component of $\mathcal{N}$ that contains $v$ and $sign \big(h''(e^r)\big)$ is the sign of $h''$ on the level $e^r$, where $v$ lives. If $h$ is convex, then this sign is ${+}1$, a situation encountered in most cases.\\
Note that in the autonomous case $\mu_{CZ}(v)$ is constant on connected components of $\mathcal{N}$. Indeed, if $v_0, v_1$ are two 1-periodic orbits of $X_H$ in the same connected component of $\mathcal{N}$ and if $u_0: D\rightarrow\mathbb{C}$ is such that $u_0(e^{2\pi it})=v_0(t)$, then we may construct $u_1:D\rightarrow\mathbb{C}$ such that $u_1(e^{2\pi it})=v_1(t),\; u_0(z)=u_1(\frac{1}{2}z)\;\forall\, z {\in} D$ and $u_1(\frac{1}{2}(s{+}1)e^{2\pi it})=v_s(t)$ for $s\in[0,1]$ defines a path of 1-periodic $X_H$-orbits between $v_0$ and $v_1$. Choosing a symplectic trivialization $\Phi_0$ of $(u_0^\ast TV, u_0^\ast\omega)$, we may extend  this to a trivialization $\Phi_1$ of $(u_1^\ast TV, u_1^\ast\omega)$. Then, we find that $\Psi_1=\Phi_1^{-1}(v_1)\circ D\varphi^t_{X_H}(v_1)\circ \Phi_1(v_1)$ is homotopic with fixed endpoints to the catenation $\Lambda_1\ast\Psi_0\ast\Lambda_0$, where 
\begin{align*}
 \Psi_0(t)&=\Phi_0^{-1}(v_0(t))\circ D\varphi^t_{X_H}(v_0(0))\circ\Phi_0(v_0(0)),\\
 \Lambda_0(s)&=\Phi^{-1}_{1-s}(v_{1-s}(0))\circ D\varphi^0_{X_H}(v_{1-s}(0))\circ\Phi_{1-s}(v_{1-s}(0)),\\
 \Lambda_1(s)&=\Phi^{-1}_s(v_s(1))\circ D\varphi^1_{X_H}(v_s(0))\circ\Phi_s(v_s(0)).
\end{align*}
Here, $\Lambda_0=\mathbbm{1}$ is constantly the identity while $\dim\ker(\Lambda_1(s){-}\mathbbm{1})=\dim_v\mathcal{N}$. Using the zero, homotopy and catenation property of $\mu_{CZ}$, we conclude $\mu_{CZ}(\Psi_1)=\mu_{CZ}(\Lambda_1)+\mu_{CZ}(\Psi_0)+\mu_{CZ}(\Lambda_0)=\mu_{CZ}(\Psi_0)$.\pagebreak\\
We remark that the Conley-Zehnder index is also defined for Reeb orbits (just use the Reeb flow instead of the Hamiltonia flow). Moreover, if $v$ is a closed $X_H$-orbit that is a reparametrization of a closed Reeb orbit, then these two indices coincide.\\
Finally let $v$ be a transversely non-degenerate Reeb orbit and let $k{\cdot} v$ denote its $k$-fold iteration. Then we have for $\mu_{CZ}(k{\cdot} v)$ by \cite{SaZeh}, Lemma 13.4, or \cite{FauckThesis}, Lemma 55, the following formula
\begin{equation}\label{eqIteration}
 \text{\textbf{(iterations formula)}}\qquad \mu_{CZ}(k{\cdot} v) = k \Delta (v)+ R(k,v).
\end{equation}
Here, $\Delta(v)$ is the \textit{mean index} of $v$ and $R(k,v)$ is an error term that is bounded by $|R(k,v)|\leq 2\dim(\xi)=2(n{-}1)$.

\subsection{Asymptotically finitely generated contact structures}\label{secAsmFini}
In the literature, the following type of autonomous Hamiltonians $H$ on $\widehat{V}$ is often used: $H$ is a $C^2$-small Morse function $g$ inside $V$ that becomes cylindrical near $\partial V=\Sigma$ and is on $[-\veps,\infty){\times}\Sigma$ of the form $H(r,p)=h(e^r)$ for $h:\mathbb{R}\rightarrow\mathbb{R}$ with $h''\geq 0$ and $h(e^r)=\mathfrak{a} e^r+\mathfrak{b}$ on $[0,\infty){\times}\Sigma$ with $\mathfrak{a}\not\in spec(\Sigma,\alpha)$. As for $H(r,p)=h(e^r)$ holds that $X_H(r,p)=h'(e^r){\cdot} R(p)$, we find that Hamiltonians of this form have two types of 1-periodic orbits:
\begin{itemize}
 \item constant orbits inside $V$ corresponding to critical points,
 \item non-constant orbits near $\Sigma$ corresponding to Reeb orbits of length $h'(e^r)$.
\end{itemize}
 If we assume that $\alpha$ satisfies (\ref{CondMB}) then we know that the non-constant orbits of $X_H$ form a manifold $\mathcal{N}$ which agrees with the manifold formed by the closed Reeb orbits. If we choose a Morse function $f$ on $\mathcal{N}$, then we saw that $FC(H,f)$ is generated by critical points of $f$ and $g$.\\
 Let $crit_k(g)$ denote the critical points of $g$ with Morse-Bott index $k$. We find that $FC_k(H,f)$ is generated by at most $\# crit_k(f)+\#crit_k(g)$ many critical points for any Hamiltonian $H$ of the above form and hence that
 \[\rk FH_k(H,g)\leq \# crit_k(f)+\# crit_k(g)\quad\Rightarrow\quad \rk SH_k(V)\leq \# crit_k(f)+\# crit_k(g).\]
 Unfortunately, we cannot always assume that there are globally only finitely many closed Reeb orbits with Morse-Bott index $k$. In particular, performing surgery may create infinitely many new orbits. However, we may assume that we have only finitely many up to a certain length which motivates the following definition.
 \begin{defn}~\\
  Let $(\Sigma,\alpha)$ be a compact contact manifold. We say that $\xi=\ker\alpha$ is an \textbf{asymptotically finitely generated contact structure in degree $k$ with bound $b_k(\xi)$}, if there exists
  \begin{itemize}
   \item a non-increasing sequence of smooth functions $f_l:\Sigma\rightarrow\mathbb{R}$ with \\$f_{l+1}(p)\leq f_l(p)\leq 0\quad\forall p\in\Sigma$,
   \item an increasing sequence $\mathfrak{a}_l\in\mathbb{R}$ with $\mathfrak{a}_{l+1}\geq \mathfrak{a}_l$ and $\lim_{l\rightarrow\infty}\mathfrak{a}_l=\infty$,
  \end{itemize}
such that for each contact form $\alpha_l:=e^{f_l}{\cdot}\alpha$ with Reeb vector field $R_l$ holds that \linebreak $\mathfrak{a}_l\not\in spec(\Sigma,\alpha_l)$, all contractible closed $R_l$-orbits of length at most $\mathfrak{a}_l$ are transversely non-degenerate and of these orbits at most $b_k(\xi)$ have Morse-Bott index $k$.
 \end{defn}
\begin{rem}~
 \begin{itemize}
  \item We do not require that the $\alpha_l$ are distinct. In particular $\alpha=\alpha_l\; \forall\, l$ is possible, if for $\alpha$ itself all closed Reeb orbits are transversely non-degenerate and only finitely many have Morse-Bott index $k$. In this situation, $b_k(\xi)$ can be chosen to equal the number of closed Reeb orbits of $\alpha$ having Morse-Bott index $k$.
  \item It is shown in the appendix that all contact structures which admit a contact form $\alpha$ with totally periodic Reeb flow are asymptotically finitely generated for $|k|>3(n{-}1)$. In particular all Brieskorn manifolds are asymptotically finitely generated for all $k$ with $|k|>3(n{-}1)$. Moreover, it is shown that for such contact structures one can chose contact forms $\alpha_l$ as in the definition such that the Reeb flow for no $\alpha_l$ is totally periodic.
  \item Recently, definitions similar to ours have been introduced in the literature. For example in \cite{KwKo} the notion of convenient dynamics and in \cite{Lazarev} the notion of asymptotically dynamically convex. Though all three definitions work with sequences of contact forms with ``nice'' closed Reeb orbits below a certain length here are the main differences to our definition: We do not require any form of index positivity or negativity, but we require that there are only finitely many closed orbits with index $k$.
 \end{itemize}
\end{rem}
\begin{prop}\label{PropAsympfiniteGene}
 Let $(V,\lambda)$ be a Liouville domain with compact contact boundary $(\Sigma,\alpha)$. If $\xi=\ker\alpha$ is asymptotically finitely generated in degree $k$ with bound $b_k(\xi)$, then 
 \[\rk SH_k(V)\leq b_k(\xi) + \# crit_{n-k}^{Morse}(g)\footnote{In fact $\# crit_{n-k}(g)$ can be replaced by $\rk H_{n-k}(V,\partial V)$ with the help of the canonical map $H_{n-k}(V,\partial V)\rightarrow SH_k(V)$.},\]
 where $g$ is any Morse function on $V$ and $crit_{n-k}^{Morse}(g)$ denotes the set of critical points of $g$ having Morse index $n{-}k$.
\end{prop}
\begin{proof}
 Let $\alpha_l=e^{f_l}{\cdot} \alpha$ and $(\mathfrak{a}_l)\subset\mathbb{R}$ be the sequences that show that $\xi$ is asymptotically finitely generated. Let $\Sigma_l=\{(f_l(p),p)\,|\,p\in\Sigma\}$ be the contact hypersurfaces in $\widehat{V}$ associated to $\alpha_l$ and let $V_l$ be the associated Liouville domains (see section \ref{secsetup}). Now let $H_l$ be a Hamiltonian on $\widehat{V}$ as above that is a $C^2$-small Morse function $g$ inside $V_l$ and is on $[-\veps,\infty){\times}\Sigma_l$ of the form $H_l(r,p)=h_l(e^r)$ with $h''_l\geq 0$ and $h_l(e^r)=\mathfrak{a}_le^r+\mathfrak{b}_l$ on $[0,\infty){\times}\Sigma_l$.\\
 Note that $H_l$ is expressed in different coordinates on the cylindrical end of $\widehat{V}$ for each $l$. If written in the fixed coordinates $[0,\infty){\times}\Sigma$, it takes for $r$ sufficiently large the form
 \[H_l(r,p)=\mathfrak{a}_le^{r-f_l(p)}+\mathfrak{b}_l.\]
 Apparently each $H_l$ is an admissible Hamiltonian. As $(\mathfrak{a}_l)$ and $({-}f_l)$ are non-decreasing with $\lim \mathfrak{a}_l=\infty$, it holds that $(H_l)$ is cofinal in $Ad(V)$. Hence we have that
 \[\lim_{l\rightarrow\infty} FH_\ast(H_l)=SH_\ast(V).\]
 Note that the Conley-Zehnder index of a critical point $x$ of a Morse function is related to the Morse index of $x$ by $\mu_{CZ}(x)=n-\mu_{Morse}(x)$. As there are for each $l$ at most $b_k(\xi)$ closed Reeb orbits of $\alpha_l$ with Morse-Bott index $k$, we find that
 \[\rk FH_k(H_l)\leq \rk FC_k(H_l)\leq b_k(\xi)+\#crit_{n-k}^{Morse}(g)\]
 and applying the direct limit yields $\quad\rk SH_k(V) \leq b_k(\xi)+\#crit_{n-k}^{Morse}(g)$.
\end{proof}

\subsection{Action filtration}\label{sectrunc}
The action functional $\mathcal{A}^H$ provides filtrations of $SH_\ast(V)$ as follows: For a fixed admissible Hamiltonian $H$ and $b\in\mathbb{R}$ consider the subchain groups
\[FC_\ast^{<b}(H)\subset FC_\ast(H),\]
which are generated by whose $x\in\mathcal{P}(H)$ with $\mathcal{A}^H(x)<b$. For $a<b$, we set 
\[FC^{[a,b)}_\ast(H):=\raisebox{.2em}{$FC^{<b}_\ast(H)$}\left/\raisebox{-.2em}{$FC^{<a}_\ast(H)$}\right..\]
We call $FC^{[a,b)}_\ast(H)$ truncated chain groups in the action window $[a,b)$. By setting $a=-\infty$, they include the cases $FC_\ast^{[-\infty,b)}(H)=FC_\ast^{<b}(H)$.
Analogously one defines
\begin{align*}
 &FC_\ast^{\leq b}(H),\; FC^{>b}_\ast(H):=\raisebox{.2em}{$FC_\ast(H)$}\left/\raisebox{-.2em}{$FC^{\leq b}_\ast(H)$},\right.\; FC^{\geq b}_\ast(H),\\
 &FC^{(a,b]}_\ast(H),\;FC^{(a,b)}_\ast(H)\text{ and }FC^{[a,b]}_\ast(H).
\end{align*}
Note that $FC^{[a,b)}_\ast(H)=FC^{(a,b)}_\ast(H)$ if $a\not\in\mathcal{A}^H(\mathcal{P}(H))$. In the following, we restrict ourselves for simplicity to $FC^{(a,b)}_\ast(H)$. However, most of the subsequent results hold for all versions of action windows.\\
Lemma \ref{monolem} below shows that the boundary operator $\partial$ reduces the action. It induces therefore a boundary operator $\partial=\partial^{(a,b)}$ on $FC_\ast^{(a,b)}(H)$ and for this we define
\[FH_\ast^{(a,b)}(H):=\frac{\ker \partial^{(a,b)}}{\text{im } \partial^{(a,b)}}.\]
\begin{lemme}\label{monolem}
 If $H$ is a Hamiltonian or a (everywhere) monotone decreasing homotopy and $u$ a solution of (\ref{eqast}) with $\displaystyle\lim_{s\rightarrow\pm\infty}u=x_\pm\in\mathcal{P}(H)$, then $\mathcal{A}^H(x_+)\geq\mathcal{A}^H(x_-)$.
\end{lemme}
\begin{proof}
\[\mathcal{A}^H(x_+)-\mathcal{A}^H(x_-)=\int^\infty_{-\infty}\mspace{-10mu}\partial_s\mathcal{A}^H(u(s))ds=\int^\infty_{-\infty} \mspace{-10mu}||\nabla\mathcal{A}^H||^2ds-\int^\infty_{-\infty}\int^1_0 \mspace{-10mu}\partial_s  H(u(s))dt\,ds\geq 0.\]
 Note that the second term is zero, if $H$ does not depend on $s$, i.e.\ if $H$ is a Hamiltonian. This shows that the monotone decreasing condition is only needed for homotopies.
\end{proof}
Let $H_-,H_+$ be two admissible Hamiltonians such that $H_->H_+$ everywhere. Then we may choose a monotone decreasing admissible homotopy $H_s$ between them and it follows from Lemma \ref{monolem} that the associated continuation map $\sigma_\sharp(H_-,H_+)$ also decreases action. We obtain hence a well-defined map
\[\sigma_\ast(H_-,H_+): FH_\ast^{(a,b)}(H_+)\rightarrow FH_\ast^{(a,b)}(H_-).\]
The truncated Symplectic homology in the action window $(a,b)$ is then defined as the direct limit under these maps:
\[SH_\ast^{(a,b)}(V):=\varinjlim FH_\ast^{(a,b)}(H).\]
\underline{\textbf{\textit{Attention}}}: Without further restrictions, we have for all $a>-\infty$ and any $b$:
\[SH^{(a,b)}_\ast(V)=0\qquad\text{ and }\qquad SH^{(-\infty,b)}_\ast(V)=SH_\ast(V).\]
To see this, take any cofinal sequence of Hamiltonians $(H_n)$ and take an increasing sequence $(\beta_n)\subset\mathbb{R}$ such that $\displaystyle\beta_n>\max_{x\in\mathcal{P}(H_n)}\mathcal{A}^{H_n}(x)$. Define $K_n:=H_n+\beta_n-a$ and $L_n:=H_n+\beta_n-b$, which yield also cofinal sequences satisfying
\[\max_{x\in\mathcal{P}(K_n)}\mathcal{A}^{K_n}(x)=\max_{x\in\mathcal{P}(H_n)}\mathcal{A}^{H_n}(x)-\beta_n+a<a\quad\text{ and }\quad \max_{x\in\mathcal{P}(L_n)}\mathcal{A}^{L_n}(x)<b.\]
It follows that $FC^{(a,b)}_\ast(K_n)=FH_\ast^{(a,b)}(K_n)=0$ for all $n$ and hence $SH^{(a,b)}_\ast(V)=0$, while $FC_\ast^{(-\infty,b)}(L_n)=FC_\ast(L_n)$ for all $n$ and hence $SH^{(-\infty,b)}_\ast(V)=SH_\ast(V)$.\\ 
To obtain a meaningful action filtered version of $SH$, we have to restrict the set of admissible Hamiltonians. In this article, it will be usefull to require that all Hamiltonians $H$ are smaller then $0$ inside a fixed Liouville subdomain $W\subset \widehat{V}$ bounded by a contact hypersurface $\partial W$\footnote{Other possibilities are $H|_{\partial V}<0$ which leads to the $V$-shaped homology/Rabinowitz-Floer homology or $H|_{V\setminus W}<0$ which leads to the Symplectic homology of a cobordism. See \cite{CieOan} for more details.}. In particular, one can take $W=V$. We write $SH^{(a,b)}(W{\subset}V)$ for the direct limit of these Hamiltonians\footnote{These Hamiltonians coincide with whose defining $SH_\ast(W)$ in the sense of \cite{CieOan}. However, $SH_\ast(W)\neq SH_\ast(W{\subset}V)$ in general, but $SH^{\geq 0}_\ast(W{\subset}V)=SH_\ast(W)$ as shown in Cor. \ref{transfer}.}, as this filtration of $SH_\ast(V)$ gives informations about the embedded subdomain $W$. Note that different choices of $W\subset V$ give different filtrations of $SH_\ast(V)$! To ease notation, we write $SH^{\geq0}_\ast(W{\subset}V)$ instead of $SH^{[0,\infty)}_\ast(W{\subset}V)$.\\
For the definition of $FH^{(a,b)}_\ast(H)$ it suffices that only the 1-periodic orbits $x$ of $X_H$ with $\mathcal{A}^H(x)\in(a,b)$ are non-degenerate, as the others are discarded. Therefore, we call a Hamiltonian $H$ admissible for $SH^{(a,b)}_\ast(W{\subset}V)$, writing $H\in Ad^{(a,b)}(W{\subset}V)$, if it satisfies
\begin{itemize}
 \item $H<0$ on $W$,
 \item $H(r,p)=\mathfrak{a}\cdot e^{r-f(p)}+\mathfrak{b}$ on $[R,\infty)\times\Sigma$ for some $R$ and $f:\Sigma\rightarrow\mathbb{R}$,
 \item all $x\in\mathcal{P}(H)$ with $\mathcal{A}^H(x)\in(a,b)$ are (transversely) non-degenerate.
\end{itemize}
The partial ordering on $Ad^{(a,b)}(W{\subset}V)$ is given by $H\prec K$ if $H\leq K$ everywhere.\\
Note that we may choose for the computation of $SH^{(a,b)}_\ast(W{\subset}V)$ cofinal sequences $(H_n)$ which are also admissible for the whole Symplectic homology, but we do not have to. Then considering $H\in Ad(W{\subset}V):=Ad^{(-\infty,\infty)}(W{\subset}V)$ no orbits get discarded, so that all orbits are non-degenerate. Thus $Ad(W{\subset}V)\subset Ad(V)$ and in fact it is a cofinal subset, so that
\[SH_\ast^{(-\infty,\infty)}(W{\subset}V)=SH_\ast(W{\subset}V)=SH_\ast(V).\]
When taking a Hamiltonian $H\in Ad(W{\subset}V)$, we find that the projection
\[\pi:\;FC_\ast(H)\rightarrow FC_\ast^{> b}(H)=\raisebox{.2em}{$FC_\ast(H)$}\left/\raisebox{-.2em}{$FC^{\leq b}_\ast(H)$}\right.\]
and the short exact sequence
\[0\rightarrow FC_\ast^{(a,b)}(H)\rightarrow FC_\ast^{(a,c)}(H)\rightarrow FC_\ast^{(b,c)}(H)\rightarrow 0\]
induce in homology a map $\pi:\,FH_\ast(H)\rightarrow FH_\ast^{\geq b}(H)$ and a long exact sequence
\[\dots\rightarrow FH_\ast^{(a,b)}(H)\rightarrow FH_\ast^{(a,c)}(H)\rightarrow FH_\ast^{(b,c)}(H)\rightarrow\dots\]
Applying the direct limit then yields a map
\[SH_\ast(V)=SH_\ast(W{\subset}V)\overset{\pi}\longrightarrow SH_\ast^{> b}(W{\subset}V)\]
and (as $\displaystyle\varinjlim$ is an exact functor) a long exact sequence
\[\dots\rightarrow SH^{(a,b)}_\ast(W{\subset}V)\rightarrow SH^{(a,c)}_\ast(W{\subset}V)\rightarrow SH^{(b,c)}_\ast(W{\subset}V)\rightarrow\dots\]

\subsection{Symplectic cohomology}\label{secSymCoHom}
By dualizing the constructions from \ref{secsymhom}, we obtain the Symplectic cohomology. Explicitly, we define for an admissible Hamiltonian $H$ the cochain groups $FC^\ast(H)$ as the dual of $FC_\ast(H)$. As $FC_\ast(H)$ is $\mathbb{Z}_2$-generated by the finite set $\mathcal{P}(H)$, we can view $FC^\ast(H)$ also as the $\mathbb{Z}_2$-vector space generated by $\mathcal{P}(H)$. The coboundary operator $\delta$, which is the dual of $\partial$, is then given as the linear extension of
\[\delta x:=\sum_{y\in\mathcal{P}(H)} \#_2 \mathcal{M}^0(x,y)\cdot y.\]
Note that the operator $\delta$ reverses the direction of maps. The analogue construction of chain maps $\sigma^\sharp(H_-,H_+)$ associated to an admissible homotopy $H_s$ between Hamiltonians $H_-$ and $H_+$ yields hence continuation maps in the opposite direction:
\[\sigma^\ast(H_-,H_+):FH^\ast(H_-)\rightarrow FH^\ast(H_+),\]
where $H_->H_+$ on $[R,\infty)\times\Sigma$ for $R$ sufficiently large. It obeys the composition rule
\[ \sigma^\ast(H_1,H_3)=\sigma^\ast(H_2,H_3)\circ\sigma^\ast(H_1,H_2).\]
By taking the same partial ordering on $Ad(V)$ as for homology, we obtain hence an inverse system. The Symplectic cohomology $SH^\ast(V)$ is then defined to be the inverse limit of this system
\[SH^\ast(V):=\varprojlim FH^\ast(H).\]
Again, it can be calculated using cofinal sequences $(H_n)$ of admissible Hamiltonians. Note that $\delta$ decreases action, that for the truncated version of Symplectic cohomology we now have to consider 
\[FC^\ast_{>a}(H)\subset FC^\ast(H)\]
generated by those 1-periodic orbits with action greater then $a$. Then, we define
\[ FC_{(a,b]}^\ast(H):=\raisebox{.2em}{$FC^\ast_{>a}(H)$}\left/\raisebox{-.2em}{$FC_{>b}^\ast(H)$}\right.\]
and all other truncated groups accordingly. As $\delta$ increases action, it is well-defined on the truncated chain groups and yields analogously $FH^\ast_{>a}(H)$ and $FH^\ast_{(a,b)}(H)$ as cohomology groups. When considering only (globally) monotone decreasing homotopies between $H\in Ad^{(a,b)}(W{\subset}V)$, the continuation maps are also well-defined on truncated groups and we obtain as inverse limits
\[SH^\ast_{>a}(W{\subset}V)=\varprojlim FH^\ast_{>a}(H)\quad\text{ and }\quad SH^\ast_{(a,b)}(W{\subset}V)=\varprojlim FH^\ast_{(a,b)}(H).\]
\textbf{\textit{Note:}} In cohomology, the long exact sequence
\[\dots\rightarrow FH^\ast_{(b,c)}(H)\rightarrow FH^\ast_{(a,c)}(H)\rightarrow FH^\ast_{(a,b)}(H)\rightarrow\dots\]
induces in general \textbf{not} a long exact sequence in Symplectic cohomology as the inverse limit is not an exact functor. However $\varprojlim$ is left exact (see \cite{bourbaki2} or \cite{eilenberg}) and the inclusion $FC^\ast_{\geq a}(H)\rightarrow FC^\ast(H)$ still induces a map
\[SH^\ast_{\geq a}(W{\subset}V)\rightarrow SH^\ast(W{\subset}V)=SH^\ast(V).\]

\subsection{The transfer morphisms}\label{sectransfer}
Following Viterbo, \cite{Vit}, we construct in this section for a Liouville subdomain $W\subset V$ the so called transfer maps \[\pi_\ast(W,V):SH_\ast(V)\rightarrow SH_\ast(W) \quad\text{ and }\quad \pi^\ast(W,V):SH^\ast(W)\rightarrow SH^\ast(V).\]
As shown above, we have maps $SH_\ast(V)\rightarrow  SH_\ast^{\geq0}(W{\subset}V)$ and $SH^\ast_{\geq0}(W{\subset}V)\rightarrow SH^\ast(V)$. By showing the identities $SH_\ast^{\geq0}(W{\subset}V)= SH_\ast(W)$ and $SH^\ast_{\geq0}(W{\subset}V)= SH^\ast(W)$ we will see that these give us the transfer maps. This is done in Prop.\ \ref{proptrans} and Cor.\ \ref{transfer} by giving an explicit cofinal sequence $(H_n)\subset Ad(W{\subset}V)$.\\
The following proposition is based on ideas by Viterbo, \cite{Vit}, however the proof follows McLean, \cite{McLeanDis}. We include it here for completeness and as the use of the No-Escape Lemma provides simplifications.
\begin{prop}[McLean,\cite{McLeanDis}]\label{proptrans}~\\
 There exists an increasing cofinal sequence $(H_n)\subset Ad^{\geq0}(W{\subset}V)$ and a sequence of monotone decreasing admissible homotopies $(H_{n,n+1})$ between them such that
 \begin{enumerate}
  \item $H_n|_W, \; H_{n,n+1}|_W$ are increasing sequences of admissible Hamiltonians / decreasing homotopies on $(W,\omega)$,
  \item all 1-periodic orbits of $X_{H_n}$ in $W$ have positive action and all 1-periodic orbits of $X_{H_n}$ in $\widehat{V}\setminus W$ have negative action,
  \item all $\mathcal{A}^H$-gradient trajectories of $H_n$ or $H_{n,n+1}$ connecting 1-periodic orbits in $W$ are entirely contained in $W$ for all admissible $J$ that are of contact type near $\partial W$.
 \end{enumerate}
\end{prop}
\begin{proof}
 It will be convenient to use $z=e^r$ rather than $r$ for the radial coordinate in the completions $(\widehat{W},\widehat{\omega})$ and $(\widehat{V},\widehat{\omega})$. Note that we can embed $\widehat{W}$ into $\widehat{V}$ using the flow of the Liouville vector field $Y$. The cylindrical end $[1,\infty)\times\partial W$ is then a subset of $\widehat{V}$. The radial coordinate will be denoted $z_W$ on $\partial W\times(0,\infty)$ and  $z_V$ on $\partial V\times(0,\infty)$. Note that we can find a constant $P$ such that $\{z_W{\leq} 1\}\subset \{z_V{\leq} P\}$ and that this implies $\{z_W{\leq} C\}\subset\{z_V{\leq} C{\cdot} P\}$ for any $C>0$. Let $\alpha_W:=\lambda|_{T\partial W}$, $\alpha_V:=\lambda|_{T\partial V}$ and assume that $(\partial W,\alpha_W)$ and $(\partial V,\alpha_V)$ satisfy (\ref{CondMB}).\\
 For the construction of $H_n$ choose an increasing sequence $(\mathfrak{a}_n)\subset\mathbb{R}^+$ with $\mathfrak{a}_n\rightarrow\infty$ and 
 \[(\mathfrak{a}_n)\not\in \Big(spec(\partial W,\alpha_W)\cup 4P{\cdot} spec(\partial V,\alpha_V)\Big)\qquad\text{ for all $n$}.\]
 \[\text{Let }\qquad\mu_n:=dist\big(\mathfrak{a}_n,spec(\partial W,\alpha_W)\big)=\min_{a\in spec(\partial W,\alpha_W)}|\mathfrak{a}_n-a|>0\]
 and let $(\veps_n)$ be a decreasing sequence with $\veps_n\rightarrow 0$ and $\veps_1$ sufficiently small.\bigskip\\ 
 Finally, choose an increasing sequence $Z_n$ such that: $\quad Z_n>\frac{\mathfrak{a}_n}{\mu_n}$ and $Z_n>2$.\hfill $(\ast)$\bigskip\\
To ease notation, we write only $Z,\mathfrak{a},\mu,\veps$, whenever there is no danger of confusion.\\
 \begin{figure}[ht]
\centering
 \resizebox{15cm}{!}{\input{Hamilton.pdf_t}}
 \caption{\label{fig7}The Hamiltonian $H_n$ and the areas of the five obit types}
\end{figure}\\
 Next, we describe the Hamiltonian $H_n$ (see figure \ref{fig7} for a schematic illustration):\\
 Inside $W\setminus\big([1-\veps,1){\times}\partial W\big)$ let $H_n$ be a $C^2$-small Morse function with $-2\veps<H_n< \veps$. On $[1-\veps,Z]{\times}\partial W$ let it be of the form $H_n(z_W,p)=g(z_W)$ with $g(1)=-\veps,\;0\leq g'(z_W)\leq \mathfrak{a}$ and $g'(z_W)\equiv \mathfrak{a}$ for $1\leq z_W \leq Z{-}\veps$. On $[Z,2Z]{\times} \partial W$ let $H_n\equiv B$ be constant with $B=B_n\approx \mathfrak{a}_n{\cdot}(Z_n{-}1)$.\\
 On $[1,\infty){\times}\partial V$ keep $H_n$ constant until we reach the hypersurface defined by $z_V=2ZP{-}\veps$ (recall $\{z_W{\leq} 2Z\}\subset\{z_V{\leq} 2ZP\}$). Then let $H_n$ be of the form $H_n(z_V,p)=f(z_V)$ for $z_V\geq 2ZP{-}\veps$ with $0\leq f'(z_V)\leq \frac{1}{4P}\mathfrak{a}$ and $f'(z_V)\equiv\frac{1}{4P}\mathfrak{a}$ for $z_V\geq 2ZP$, i.e.\ $f(z_V)$ is $C^0$-close to the linear function $\frac{\mathfrak{a}}{4P}{\cdot} z_V + B - \frac{\mathfrak{a}Z}{2}$.\bigskip\\
 Note that for $H(z,p)=h(z)$ the action of an $X_H$-orbit on a fixed $z$-level is $h'(z)\cdot z-h(z)$. Hence we distinguish five types of 1-periodic orbits of $X_H$:
 \begin{itemize}
  \item[\textbf{\textit{I}}]: critical points inside $W$ of action $\geq\veps$ (as $H_n\leq-\veps$ and $C^2$-small inside $W$)
  \item[\textbf{\textit{II}}]: non-constant orbits near $z_W=1$ of action $\approx g'(z)>0$
  \item[\textbf{\textit{III}}]: non-constant orbits on $z_W=c$ for $c$ near $Z$ of action\\ $\Big.\approx g'(c)\cdot c-B<(\mathfrak{a}-\mu)\cdot Z-B\approx -\mu\cdot Z+\mathfrak{a}\overset{(\ast)}{<}0$ 
  \item[\textbf{\textit{IV}}]: critical points in $Z<z_W , z_V<2ZP-\veps$ of action $-B<0$
  \item[\textbf{\textit{V}}]: non-constant orbits on $z_V=c$ for $c$ near $2ZP$ of action\\ $\approx f'(c){\cdot} c-B\leq\frac{1}{4P}\mathfrak{a}{\cdot} 2ZP-B\approx\frac{1}{2}\mathfrak{a}Z-\mathfrak{a}(Z{-}1)=\frac{1}{2}\mathfrak{a}(2{-}Z)\overset{(\ast)}{<}0.$
 \end{itemize}
Hence, $(H_n)$ satisfies the second claim of the proposition. To see the first claim, note that $H_n|_W<H_{n+1}|_W$ (as $-2\veps_{n+1}>-2\veps_n$ and $\mathfrak{a}_{n+1}>\mathfrak{a}_n$) and that the linear extensions of $H_n|_W$ to $\widehat{W}$ form a cofinal sequence of admissible Hamiltonians on $W$.
\begin{figure}[ht]
\centering
 \resizebox{10cm}{!}{\input{Homotopy1.pdf_t}}
 \caption{\label{fighomo}Two Hamiltonians $H_n$ and $H_{n+1}$}
\end{figure}\\
In fact $H_n<H_{n+1}$ holds even globally. As $\frac{1}{4P}\mathfrak{a}_n<\frac{1}{4P}\mathfrak{a}_{n+1}$, we have $H_n<H_{n+1}$ for $z_V$ large enough. Yet, the most crucial area is around $z_V=2Z_{n+1}P$, where $H_{n+1}$ is still constant while $H_n$ has already regained its slope (see figure \ref{fighomo}). However, there we have the estimate
\begin{align*}
 H_n(2Z_{n+1}P)=h_n(2Z_{n+1}P)&=\textstyle\frac{1}{4P}\mathfrak{a}_n{\cdot} 2Z_{n+1}P+B_n-\frac{1}{2}\mathfrak{a}_nZ_n\\
 &\approx \textstyle\frac{1}{2}\mathfrak{a}_nZ_{n+1}+\mathfrak{a}_n(Z_n{-}1)-\frac{1}{2}\mathfrak{a}_nZ_n\\
 &<\textstyle\frac{1}{2}\mathfrak{a}_{n+1}Z_{n+1}+\mathfrak{a}_{n+1}(\frac{Z_{n+1}}{2}-1)\\
 &=\mathfrak{a}_{n+1}(Z_{n+1}{-}1)\qquad\qquad\qquad=B_{n+1}\approx H_{n+1}(2Z_{n+1}P).
\end{align*}
Thus $H_n<H_{n+1}$ everywhere and we can find an admissible decreasing homotopy $H_{n,n+1}$ connecting $H_{n+1}$ and $H_n$ such that:
\begin{align*}
 H_{n,n+1}(z_W,p)&=\mathfrak{a}_s^W{\cdot} z_W + \mathfrak{b}^W_s & &\text{near }z_W=1,& \partial_s \big(H_{n,n+1}{-}\mathfrak{b}^W_s\big)&\leq 0 &&\text{for }z_W\geq 1\\
 H_{n,n+1}(z_V,p)&=\mathfrak{a}_s^V{\cdot} z_V + \mathfrak{b}^V_s & &\text{and}& \partial_s \big(H_{n,n+1}{-}\mathfrak{b}^V_s\big)&\leq 0 &&\text{for $z_V\geq 2Z_{n+1}P$}\mspace{-1mu}.
\end{align*}
If we choose an admissible $J$ which is of contact type near $z_W=1$, then it follows from the No-Escape Lemma (Cor. \ref{Cornoesc}) that all $\mathcal{A}^H$-gradient trajectories for $H_n$ or $H_{n,n+1}$ connecting 1-periodic orbits inside $W$ stay inside $W$ for all time. Hence the third claim is satisfied.
\end{proof}
\begin{cor}\label{transfer} $\quad\displaystyle SH^{\geq 0}_\ast(W{\subset}V)\simeq SH_\ast(W)\qquad\text{ and }\qquad SH^\ast_{\geq0}(W{\subset}V)\simeq SH^\ast(W).$
\end{cor}
\begin{proof}
 We only prove the corollary for homology, cohomology being completely analog. Take the sequence of Hamiltonians $(H_n)$ constructed in Proposition \ref{proptrans}. Clearly it is cofinal and $(H_n)\subset Ad^{\geq0}(W{\subset}V)$, as 1-periodic orbits with positive action are either isolated critical points inside $W$ (as $H$ is Morse and $C^2$-small there) or isolated Reeb-orbits near $z_W=1$ -- in both cases non-degenerate. Hence we have
 \[ SH^{\geq0}_\ast(W{\subset}V)=\lim_{\longrightarrow} FH^{\geq0}_\ast(H_n).\]
 Let $\tilde{H}_n\in Ad(W)$ be the linear extension of $H_n|_W$ to $\widehat{W}$ with slope $\mathfrak{a}_n$. Then we have obviously $FC^{\geq0}_\ast(H_n)=FC_\ast(\tilde{H}_n)$. As any $\mathcal{A}^H$-gradient trajectory connecting 1-periodic orbits in $W$ stays in $W$, the two boundary operators $\partial_{H_n}^{\geq 0}$ and $\partial_{\tilde{H}_n}$ coincide and we have $FH^{\geq0}_\ast(H_n)=FH_\ast(\tilde{H}_n)$. As the $\mathcal{A}^H$-gradient trajectories for the homotopies $H_{n,n+1}$ stay inside $W$, the continuation maps $\sigma(H_{n+1},H_n): FH^{\geq0}_\ast(H_n)\rightarrow FH^{\geq0}_\ast(H_{n+1})$ coincide with the continuation maps $\sigma(\tilde{H}_{n+1},\tilde{H}_n): FH_\ast(\tilde{H}_n)\rightarrow FH_\ast(\tilde{H}_{n+1})$. Hence we have 
 \[SH^{\geq0}_\ast(W{\subset}V)= \varinjlim FH^{\geq0}_\ast(H_n) = \varinjlim FH_\ast(\tilde{H}_n)=SH_\ast(W).\qedhere\]
\end{proof}
In the literature, there is a second description of the transfer maps which goes as follows. Let $(H_n)$ be the Hamiltonians described in Proposition \ref{proptrans} and let $(K_n)$ be the following sequence of Hamiltonians: Inside $V$ we require that $K_n$ is a $C^2$-small Morse function such that $K_n|_W\leq H_n|_W$ and on $[1-\veps,\infty)\times\partial V$ let it be of the form $K_n(z_V,p)=f(z_V)$ with $0\leq f'(z_V)\leq\frac{\mathfrak{a}_n}{4P}$, where $\mathfrak{a}_n$ and $P$ are as in Prop. \ref{proptrans}. In particular, $K_n$ is on $[1{-}\veps,\infty)\times\partial V$ arbitrarily close to the linear function $z_V{\cdot}\frac{\mathfrak{a}_n}{4P}-\frac{\mathfrak{a}_n}{4P}$ (see figure \ref{figHam2nd}).\\
\begin{figure}[ht]
\centering
 \resizebox{15cm}{!}{\input{Hamilton2ndDef.pdf_t}}
 \caption{\label{figHam2nd}The two Hamiltonians $H_n$ and $K_n$}
\end{figure}\\
We find that $K_n\leq H_n$ everywhere, as in particular
\[K(2ZP)\approx 2ZP{\cdot}\frac{\mathfrak{a}}{4P}-\frac{\mathfrak{a}}{4P}=\mathfrak{a}\Big(Z-\Big(\frac{Z}{2}{+}\frac{1}{4P}\Big)\Big)\overset{(Z>2)}{<}\mathfrak{a}(Z{-}1)\approx B=H(2ZP).\]
Hence, we can find an everywhere increasing homotopy between $K_n$ and $H_n$, which defines a continuation map $\sigma_\ast(K_n,H_n):FH_\ast(K_n)\rightarrow FH_\ast(H_n)$ which respects action filtration. The second version of the transfer map $\tilde{\pi}_\ast(W,V)$ is the limit of these continuation maps:
\[\begin{xy}\xymatrix{\tilde{\pi}_\ast(W,V)\;:\; SH_\ast(V)\overset{(\ast)}{=}\varinjlim FH^{\geq 0}_\ast(K_n)\ar[rr]^{\qquad\sigma_\ast(K_n,H_n)} && \varinjlim FH^{\geq 0}_\ast(H_n)=SH_\ast(W)}\end{xy}.\]
Here, the identity $(\ast)$ is due to the fact that all 1-periodic orbits of $K_n$ have positive action. The advantage of this transfer map is precisely that it respects action filtration. However, the advantage of the first definition of $\pi_\ast(W,V)$ is that it fits into the following exact triangle, where the third group has a geometric meaning:
\[\begin{xy}\xymatrix{SH_\ast(V)=SH_\ast(W{\subset} V)\ar[rr]^{\pi_\ast(W,V)} && SH_\ast^{\geq 0}(W{\subset} V)=SH_\ast(W) \ar[dl]\\ & \ar[ul]^{[-1]} SH_\ast^{<0}(W{\subset} V)}\end{xy}.\]
\begin{prop}
 $\pi_\ast(W,V)$ and $\tilde{\pi}_\ast(W,V)$ are the same map: $SH_\ast(V)\rightarrow SH_\ast(W)$. The same holds true for the corresponding maps on cohomology.
\end{prop}
\begin{proof}
 We can slightly modify the construction of the $H_n$, such that its 1-periodic orbits are all transversely non-degenerate and $H_n\in Ad(V)$. We just have to require on $Z\leq z_W,z_V\leq 2ZP-\veps$ that $H_n- B$ is a $C^2$-small Morse function instead of $H_n\equiv B$ being constant there. Then we have that $FH_\ast(H_n)$ is well-defined.\medskip\\
 \emph{\underline{Claim} :} $FH_\ast(H_n)\cong FH_\ast(K_n)$.\\
 \emph{Proof:} Note that $K_n$ and $H_n$ differ on $[R,\infty)\times\partial V$ only by a constant for $R$ large. Hence we can find a homotopy $H_s$ between them such that $H_s$ and $H_{-s}$ are both admissible. This provides continuation maps $\sigma_\ast(K_n,H_n): FH_\ast(K_n)\rightarrow FH_\ast(H_n)$ and \linebreak $\sigma_\ast(H_n,K_n): FH_\ast(H_n)\rightarrow FH_\ast(K_n)$. They satisfy $\sigma_\ast(H_n,K_n)\circ\sigma_\ast(K_n,H_n)=id_{FH(H_n)}$ and $\sigma_\ast(K_n,H_n)\circ\sigma_\ast(H_n,K_n)=id_{FH(K_n)}$, which implies that they are isomorphisms.\medskip\\
 As the homotopy from $K_n$ to $H_n$ can be chosen everywhere increasing, we find that $\sigma_\ast(K_n,H_n)$ respects action filtration and we have the following commutative diagram
 \[\begin{xy}\xymatrix{FH_\ast(H_n)\ar[rr]^{\pi^{H_n}} && FH_\ast^{\geq0}(H_n)\\
  FH_\ast(K_n) \ar[u]^\cong_{\sigma_\ast(K_n,H_n)} \ar[rr]^{\pi^{K_n}}_\cong && FH^{\geq 0}(K_n). \ar[u]_{\sigma_\ast(K_n,H_n)}}\end{xy}\]
 Applying the direct limit yields again a commutative diagram, where isomorphisms are taken to isomorphisms:
 \[\begin{xy}\xymatrix{&\varinjlim FH_\ast(H_n)\ar[rr]^{\pi_\ast(W,V)} && \varinjlim FH_\ast^{\geq0}(H_n)\ar@{=}[r] & SH_\ast(W)\\
  SH_\ast(V)\ar@{=}[r] &\varinjlim FH_\ast(K_n) \ar[u]^\cong \ar[rr]_{\cong} &&\varinjlim FH^{\geq 0}(K_n). \ar[u]_{\tilde{\pi}_\ast(W,V)}}\end{xy}\]
  Hence, we find that $\pi_\ast(W,V)$ and $\tilde{\pi}_\ast(W,V)$ coincide on homology. The same line of arguments works also for cohomology, since even though $\varprojlim$ is not an exact functor it still takes isomorphisms to isomorphisms, as it is left exact.
\end{proof}

\section{Contact surgery and handle attaching}\label{secsur}
In this section, we first describe the general construction for contact surgery, which is done by attaching a symplectic handle $\mathcal{H}$ to the symplectization of a contact manifold. Then, we describe explicitly symplectic handles as subsets of $\mathbb{R}^{2n}$ given by the intersection of two sublevel sets $\{\psi {<}{-}1\}\cap\{\phi{>}{-}1\}$ for functions $\phi$ and $\psi$ on $\mathbb{R}^{2n}$. Subsequently, we describe how to extend an admissible Hamiltonian over the handle to a new admissible Hamiltonian with only few new 1-periodic Hamiltonian orbit. The proof of the Invariance Theorem (Theorem \ref{theoinvsur}) and the proof that being asymptotically finitely generated is invariant under subcritical surgery (Prop.\ \ref{propcofinalHamilt}) conclude this section.

\subsection{Surgery along isotropic spheres}
Let us briefly recall the contact surgery construction due to Weinstein, \cite{Wein}. Consider an isotropic sphere $S^{k-1}$ in a $(2n-1)$-dimensional contact manifold $(N,\xi)$. The 2-form $\omega=d\alpha$ for a contact form $\alpha$ (with $\xi=\ker\alpha$) defines a natural conformal symplectic structure on $\xi$. Denote the $\omega$-orthogonal on $\xi$ by $\perp_\omega$. Since $S$ is isotropic, it holds that $TS\subset TS^{\perp_\omega}$. So, the normal bundle of $S$ in $N$ is given by
\[ TN/ TS= TN/ \xi \,\oplus\, \xi/(TS)^{\perp_\omega} \oplus (TS)^{\perp_\omega}/ TS.\]
The Reeb field $R$ trivializes $TN/\xi$. The bundle $\xi/(TS)^{\perp_\omega}$ is canonically isomorphic to $T^\ast S$ via $v\mapsto \iota_v\omega$. The conformal symplectic normal bundle $CSN(S):=(TS)^{\perp_\omega}/TS$ carries a natural conformal symplectic structure induced by $\omega$.\\
Since $S$ is a sphere, the embedding $S^{k-1}\subset\mathbb{R}^k$ provides a natural trivialization of the bundle $\mathbb{R}R\oplus T^\ast S$. This trivialization together with a conformally symplectic trivialization of $CNS(S)$ specifies a standard framing for $S$ in $N$. Note that we have to assume that $CNS(S)$ is trivializable. This holds certainly true for $S=S^0 =\{N,S\}$ (two points) or $S=S^{n-1}$. In the latter case we have $(TS)^{\perp_\omega}=TS$ and hence $CNS(S)=(0)$. Therefore, taking connected sums and surgery along Legendrian spheres is always possible.\\ 
Following Weinstein, we define an isotropic setup as a quintuple $(P,\omega,Y,\Sigma,S)$, where $(P,\omega)$ is a symplectic manifold, $Y$ a Liouville vector field for $\omega$, $\Sigma$ a hypersurface transverse to $Y$ (so $\Sigma$ is contact) and $S$ an isotropic submanifold of $\Sigma$. In \cite{Wein}, Weinstein proves the following variant of his famous neighborhood theorem for isotropic manifolds:
\begin{prop}[\textbf{Weinstein}] \label{X}
 Let $(P_0,\omega_0,Y_0,\Sigma_0,S_0)$ and $(P_1,\omega_1,Y_1,\Sigma_1,S_1)$ be two iso-tropic setups. Given a diffeomorphism from $S_0$ to $S_1$ covered by an isomorphism of their symplectic subnormal bundles, there exist neighborhoods $U_j$ of $S_j$ in $P_j$ and an isomorphism of isotropic setups
 \[\phi : (U_0,\omega_0,Y_0,\Sigma_0\cap U,S_0) \rightarrow (U_1,\omega_1,Y_1,\Sigma_1\cap U_1,S_1)\]
 which restricts to the given mappings on $S_0$.
\end{prop}
We may now define contact surgery along an isotropic sphere as follows. Let $\mathcal{H}\approx D^k{\times} D^{2n-k}$ be a symplectic handle (see \ref{handle}) and let $S^{k-1}$ be an isotropic sphere in $(N,\xi)$. Then, Prop.\ \ref{X} allows us to glue the (lower) boundary $S^k{\times} D^{2n-k}$ of $\mathcal{H}$ to the symplectization $N\times[0,1]$ along the boundary part $U_1\cap N{\times}[0,1]$ of a tubular neighborhood $U_1$ of $S{\times}\{1\}$ (see Figure \ref{fig2}). We obtain an exact symplectic manifold $P:=N{\times}[0,1]\cup_{S}\mathcal{H}$ with a Liouville vector field $Y$ which is on $N\times[0,1]$ simply $\partial_t$, where $t$ denotes the coordinate on $[0,1]$. Note that $Y$ points inwards along $\partial^-P :=N{\times}\{0\}$ and outwards along the other boundary component $\partial^+P$. Both manifolds are hence contact and $\partial^+P$ is obtained from $N$ by surgery along $S$. Moreover, $P$ is an exact symplectic cobordism between $\partial^-P$ and $\partial^+P$.
\begin{figure}[ht]
\centering 
 \resizebox{7cm}{!}{\input{Handleattachment.pdf_t}}
 \caption{\label{fig2} $N\times[0,1]$ with handle attached}
\end{figure}

\subsection{Symplectic handles}\label{handle}
We consider $\mathbb{R}^{2n}$ with symplectic coordinates $(q,p)=(q_1,p_1,...\,,q_n,p_n)$ and the following Weinstein structure (cf. \cite{Wein}):
\begin{align*}
\lambda &:= \sum_{j=1}^k\left(2q_jdp_j + p_jdq_j\right)+\sum_{j=k+1}^n\frac{1}{2}\left(q_jdp_j -p_jdq_j\right),& d\lambda &= \omega &:= \sum_{j=1}^n dq_j\wedge dp_j,\\
 Y &:= \sum_{j=1}^k \left(2q_j\frac{\partial}{\partial q_j} - p_j\frac{\partial}{\partial p_j}\right) + \sum_{j=k+1}^n \frac{1}{2}\left(q_j\frac{\partial}{\partial q_j} + p_j\frac{\partial}{\partial p_j}\right),\\
  \phi&:=\sum_{j=1}^k\left( q_j^2-\frac{1}{2}p_j^2\right) + \sum_{j=k+1}^n \frac{1}{2}\left(q_j^2+p_j^2\right).
\end{align*}
Note that $Y$ is the Liouville vector field for $\lambda$, as $\iota_Y\omega = \lambda$. For convenience, we introduce furthermore three functions $x,y,z:\mathbb{R}^2\rightarrow\mathbb{R}$ with Hamiltonian vector fields $X_x, X_y, X_z$:
\begin{align*}
 x&=x(q,p):= \sum_{j=1}^k q_j^2, & y&=y(q,p):= \sum_{j=1}^k \frac{1}{2} p_j^2, & z&=z(q,p):= \sum_{j=k+1}^n \frac{1}{2}\left( q_j^2+p_j^2\right),\\
 X_x &= \sum_{j=1}^k 2q_j\frac{\partial}{\partial p_j},&
 X_y &= \sum_{j=1}^k -p_j\frac{\partial}{\partial q_j},&
 X_z &= \sum_{j=k+1}^n \left(q_j\frac{\partial}{\partial p_j} - p_j\frac{\partial}{\partial q_j}\right).
\end{align*}
This allows us to write $\phi=x-y+z$ and $X_\phi=X_x-X_y+X_z$.\\
We consider the level surface $\Sigma^- :=\{\phi=-1\}$ and note that $Y$ is transverse to $\Sigma^-$. It follows that $(\Sigma^-,\lambda|_{\Sigma^-})$ is a contact manifold. The set $S:=\{x{=}z{=}0,\;y{=}{+}1\}$ is an isotropic sphere in $\Sigma^-$ and the quintuple $(\mathbb{R}^{2n},\omega,Y,\Sigma^-,S)$ is the isotropic setup where we glue a symplectic handle $\mathcal{H}$ to a contact manifold. In order to specify $\mathcal{H}$, we choose a different Weinstein function $\psi$ on $\mathbb{R}^{2n}$, satisfying the following assumptions:
\begin{itemize}
 \item[$(\psi 1)$] $X_\psi=C_x{\cdot} X_x-C_y{\cdot} X_y+C_z{\cdot} X_z,$ where $C_x,C_y,C_z\in C^\infty(\mathbb{R}^{2n})$ with $C_x,C_y,C_z> 0$,
 \item[$(\psi 2)$] $\psi=\phi$ on $\{\phi\leq -1\}$ except for a small neighborhood of $S$,
 \item[$(\psi 3)$] The closure $\overline{\{\psi{<} {-}1\}\cap\{\phi{>} {-}1\}}$ is diffeomorphic to $\overline{D^k{\times} D^{2n-k}}$.
\end{itemize}
The handle is then defined as $\mathcal{H}:=\overline{\{\psi{<} {-}1\}\cap\{\phi{>} {-}1\}}\;\,$ (see Fig. \ref{fig3}).
\begin{figure}[ht]
\centering
 \resizebox{10cm}{!}{\input{surgery.pdf_t}}
 \\\caption{\label{fig3} The handle $\mathcal{H}$}
\end{figure}
\begin{remarks} ~\label{dislyapunov}
 \begin{enumerate}
  \item If $\psi(0)\neq -1$, it follows from $(\psi 1)$ that the level set $\Sigma^+:=\{\psi{=}{-}1\}$ is also a contact hypersurface, as $Y{\cdot} \psi > 0$ away from 0. Due to $(\psi 2)$, we have $\Sigma^-=\Sigma^+$ away from a neighborhood of $S$. Condition $(\psi 3)$ on the other hand assures that $\Sigma^+$ is obtained from $\Sigma^-$ by surgery along $S$.
  \item Condition $(\psi1)$ is automatically satisfied if $\psi=\psi(x,y,z)$ is given as a function on $x,y,z$ such that $\partial_x \psi,\partial_z \psi >0$ and $\partial_y \psi<0$ because its Hamiltonian vector field is then given by
  \[ X_\psi= \left(\frac{\partial \psi}{\partial x}{\cdot} X_x + \frac{\partial \psi}{\partial y}{\cdot} X_y + \frac{\partial \psi}{\partial z}{\cdot} X_z\right).\]
  \item The handle stays unchanged if we take $\phi'=\mathfrak{a}{\cdot} \phi + \mathfrak{b}$ and $\psi'=\mathfrak{a}{\cdot}\psi+\mathfrak{b}$ for $\mathfrak{a},\mathfrak{b}\in\mathbb{R}$, $\mathfrak{a}> 0$, provided that we set $\bigg.\quad \mathcal{H}=\overline{\big.\{\psi'{<} {-}\mathfrak{a}{+}\mathfrak{b}\}\cap\{\phi'{>} {-}\mathfrak{a}{+}\mathfrak{b}\}}$.
  \item  Consider the Lyapunov function $L(q,p):=\sum_{j=1}^k q_jp_j$. Note that $(\psi1)$ implies $X_{\psi}{\cdot} L>0$ away from the critical points of $L$, which shows that all periodic orbits of $X_{\psi}$ are contained in the set $\{x{=}y{=}0\}$. The same holds true for $\psi'=\mathfrak{a}{\cdot}\psi+\mathfrak{b}$.
 \end{enumerate}
\end{remarks}
It is not difficult to find a Weinstein function $\psi:\mathbb{R}^{2n}\rightarrow\mathbb{R}$ which satisfies $(\psi 1)$--$(\psi 3)$. Fix two constants $\veps,\delta>0$ and choose a smooth monotone function $g:\mathbb{R}\rightarrow (-\infty,1]$ such that
\begin{align}
 g(t) &\phantom{:}= \begin{cases}\frac{1}{1+2\veps}\cdot t& \text{ for }\quad t\leq 1\\ 1&\text{ for }\quad t\geq 1+ 3\veps\end{cases}\quad\text{ and }\quad 0\leq g'(t) \leq {\textstyle \frac{1}{1+2\veps}}.\notag\\
 \text{Then set }\hspace{0.5cm}\psi_\delta&:=x-y+z-(1{+}\veps)+(1{+}\veps)\cdot g\big(y+{\textstyle \frac{1}{\delta}}(x{+}z)\big).\label{eqXX}\hspace{2.5cm}\\
 \mathcal{H}_\delta&:=\overline{\big.\{\psi_\delta{<}{-}1\}\cap\{\phi{>}{-}1\}}.\notag
\end{align}
\begin{remarks}\label{remarksonpsi}~
 \begin{itemize}
  \item Decreasing $\veps$ or $\delta$ makes the handle thinner, i.e. $\Sigma^-\cap \mathcal{H}_\delta$ becomes smaller. However, we will fix $\veps$ and only decrease $\delta$.
  \item For reference, let us fix $\delta_0$, $\psi_{\delta_0}$, the associated handle $\mathcal{H}_{\delta_0}$  and the hypersurface $\Sigma^+:=\big\{\psi_{\delta_0}=-1\big\}$. A different choice of $\delta$  and consequently a different function $\psi_\delta$ defines a different handle $\mathcal{H}_\delta$ (see Fig.\ \ref{fig2handles}). However, the symplectic geometric result is the same, meaning that if we attach the handles to a symplectic manifold $W$, then the completions $\widehat{W{\cup} \mathcal{H}_{\delta_0}}$ and $\widehat{W{\cup} \mathcal{H}_\delta}$ agree.\\
  Indeed, if $\delta\leq \delta_0$ then $\psi_\delta\geq\psi_{\delta_0}$ everywhere and as $\psi_{\delta_0}$ and $\psi_\delta$ both increase along flow lines of $Y$, we find that each flow line of $Y$ not inside $\{x{=}z{=}0\}$ first hits $\Sigma^\delta:=\{\psi_\delta=-1\}$ and then $\Sigma^+:=\{\psi_{\delta_0}=-1\}$. Hence $\Sigma^\delta$ can be identified with a hypersurface in the symplectization of $\Sigma^+$ given as a graph of a function $f:\Sigma^+\rightarrow(-\infty,0]$, where $f(p)$ is the unique time such that $\varphi^{-f(p)}(p)\in\Sigma^{\delta}$ for the flow $\varphi^t$ of $Y$.
  \begin{figure}[!htb]
\centering
 \resizebox{10cm}{!}{\input{twohandles.pdf_t}}
 \\\caption{\label{fig2handles} Two handles}
\end{figure}
  \item Note that $\big(\{x{=}y{=}0\},Y|_{\{x{=}y{=}0\}}\big)$ is a Liouville subspace. This allows us to identify $\{x{=}y{=}0\}$ with the symplectization $\mathbb{R}\times\big(\Sigma^\delta{\cap}\{x{=}y{=}0\}\big)$. Moreover for $x{=}y{=}0$ and $z\leq \delta$ holds
  \[\psi_\delta(0,0,z) = \left(1+\frac{1+\veps}{\delta(1{+}2\veps)}\right) z -(1{+}\veps).\]
  As $\varphi^t$ satisfies $z(\varphi^t(q,p))=e^t{\cdot} z(q,p)$ for any $(q,p)\in\mathbb{R}^{2n}, \;t\in\mathbb{R}$, we can for $z\leq \delta$ express $\psi_\delta$ in symplectization coordinates $(r,p)$ on $\{x{=}y{=}0\}$ in the form
  \[\psi_\delta(r,p)=\mathfrak{a}_\delta{\cdot} e^r-(1{+}\veps).\]
  As $r=0$ corresponds to $\Sigma^\delta$ in $\{x{=}y{=}0\}$ and as the $z$-value on $\Sigma^\delta\cap\{x{=}y{=}0\}$ is smaller then $\delta$ and $\psi_\delta(\Sigma^\delta)=-1$, we have 
  \[-1=\mathfrak{a}_\delta{\cdot} e^0-(1{+}\veps)\qquad\Leftrightarrow\qquad \mathfrak{a}_\delta=\veps.\]
  This shows that $\psi_\delta$ has for any $\delta$ on $\Sigma^\delta\cap\{x{=}y{=}0\}$ the slope $\veps$ in radial direction.
 \end{itemize}
\end{remarks}

\begin{dis}\label{dishandlesdifferentcontact}
 Let $\alpha_0=\lambda|_{\Sigma^-}$ be the contact form on $\Sigma^-$ and let $f:\Sigma^-\rightarrow\mathbb{R}$ be a smooth function. Then $\alpha_f=e^f{\cdot}\alpha_0$ is a contact form on $\Sigma^-$ defining the same contact structure. As discussed in \ref{secsetup}, $\alpha_f$ is the contact form on the hypersurface $\Sigma^-_f=\big\{(f(p),p)\,|\,p\in\Sigma^-\big\}$ in the symplectization of $(\Sigma^-,\alpha_0)$. Note that $(\Sigma_f^-,\alpha_f)$ is easily identified with a contact hypersurface in $\mathbb{R}^{2n}$ (also denoted by $\Sigma_f$) via the Liouville flows of $\partial_r$ on $\mathbb{R}{\times}\Sigma^-$ and of $Y$ on $\mathbb{R}^{2n}$.\\
 If $f$ is constant on the isotropic sphere $S$, the we can define (as above) a handle $\mathcal{H}_f$ attached to $\Sigma^-_f$ instead of $\Sigma^-$. If $\mathcal{H}_f$ is sufficiently tin, we can understand it as a set inside $\mathcal{H}$ as follows: Denote by $\Sigma_f^+$ the other boundary of $\mathcal{H}_f$. Using again the flows of $\partial_r$ and $Y$, we can identify $\Sigma_f^+$ with the hypersurface $\big\{(f(p),p)\,\big|\,p\in\Sigma^+\big\}$ in the symplectization of $\Sigma^+$ and consequently with a hypersurface in $\mathbb{R}^{2n}$ which agrees with $\Sigma_f^-$ outside a compact set. The compact region bounded by $\Sigma_f^-$ and $\Sigma_f^+$ in $\mathbb{R}^{2n}$ is then identified with $\mathcal{H}_f$ (see Fig.\ \ref{fig2handles2contact}).
 \begin{figure}[!htb]
\centering
 \resizebox{9cm}{!}{\input{twohandlestwocontac.pdf_t}}
 \\\caption{\label{fig2handles2contact} A handle for a different contact form}
\end{figure}
\end{dis}

\subsection{Linear extensions over the handle}\label{ExPsi}
For the proof of the Invariance Theorem, we need Hamiltonians $H\in Ad(W{\subset}V)$, i.e\ $H$ is globally admissable with $H|_W<0$ and arbitrarily large away from $W$. Moreover, $H$ has to be linear on $[R,\infty)\times\partial(W{\cup}\mathcal{H})$ in $\widehat{W{\cup}\mathcal{H}}$ for $R$ large. As we saw in the previous secion, it is not difficult to extend a Hamiltonian $H$ on $W$ that is linear near $\partial W$ to $W{\cup}\mathcal{H}$. However, it is the extension of $H$ to the completion $\widehat{W{\cup}\mathcal{H}}$ which causes some problems, in particular if we require that $X_H$ has outside $W$ very few 1-periodic orbits with negative action, possibly only one. Unfortunately, it is here where the fundamental article on subcritical handle attachment, \cite{Cie}, is quite vague. In particular, it does not address the following difficulties:
\begin{dis} \label{dis1}
Any Hamiltonian $H$ on $\widehat{W{\cup} \mathcal{H}}$ has at least one critical point outside $W$ and we may assume that it lies at the center of $\mathcal{H}$. If this should be the only 1-periodic orbit with negative action outside $W$, we need that $H$ is on the symplectic cocore of $\mathcal{H}$ (the set $\{x=y=0\}$) of the form
\[H=\mathfrak{a}^+{\cdot} e^r+\mathfrak{b}^+\]
in cylindrical coordinates adapted to a hypersurface $\Sigma^+=\{r{=}0\}$ above the handle. Note that $\mathfrak{b}^+$ is the value of $H$ at the center of $\mathcal{H}$ and hence $\mathfrak{b}^+>0$, as $H>0$ outside of $W$. In fact, $\mathfrak{b}^+$ has to be thought of as arbitrarily large, as $H$ is part of an increasing cofinal sequence of Hamiltonians. As $H$ is constructed by extending an admissible Hamiltonian of $W$, we need near $\partial W$ that $H$ is of the form
\[H=\mathfrak{a}^-{\cdot} e^r-\mathfrak{a}^-{\cdot} e^{r^-}-\veps.\]
Here we use the same cylindrical coordinates as above and $\partial W=\{r{=}r^-\}$. The absolute term $-\mathfrak{a}^-{\cdot} e^{r^-}-\veps$ with $\mathfrak{a}^-,\veps>0$ and $\veps$ small is due to $H<0$ on $W$.\\
Finally, we may assume that $H(r,p)=h(e^r)$ is cylindrical away from the handle and its symplectization, as otherwise we would loose all information on 1-periodic orbits of $H$. Then, we may consider away from $\mathcal{H}$ the action function
\[g(r)=\partial_r H-H=h'(e^r)\cdot e^r-h(e^r).\]
As $H$ is linear at infinity, we can find an $r^+\geq 0$ such that for all $r\geq r^+$ holds that $H(r,p)=\mathfrak{a}^+{\cdot} e^r+\mathfrak{b}^+$. Hence, we find that $g(r^+)=-\mathfrak{b}^+$, while $g(r^-)=\mathfrak{a}^-{\cdot} e^{r^-}+\veps$. This implies that the slope $h'$  of $H$ has to vary away from $\mathcal{H}$, as $g$ is constant if $h'$ is constant. In fact, we even obtain with the mean value theorem that for some $r\in(r^-,r^+)$ holds:
\begin{align*}
  h'(e^{r^+})\cdot e^{r^+}\leq h(e^{r^+})- h(e^{r^-}) = (e^{r^+}-e^{r^-})\cdot h'(e^r)\quad\Rightarrow\quad h'(e^r)>h(e^{r^+}).
\end{align*}
In other words, independently from the choices of $\mathfrak{a}^-{\cdot} e^{r^-},\mathfrak{a} ^+$ and $\mathfrak{b}^+$, we always have to decrease the slope $h'$ somewhere away from $\mathcal{H}$. More generally, for fixed $\Sigma^+$, i.e. fixed cylindrical coordinates, there exists a constant $0<\mu<\infty$ such that for all $\mathfrak{a}^+$ holds
\[dist.(\mathfrak{a}^+, spec(\Sigma^+))\leq \mu.\]
 Let $r_0$ be the largest value, such that on $\{r{=}r_0\}$ there exists a 1-periodic orbit of $H$ away from $\mathcal{H}$, i.e. $h'(e^{r_0})\in spec(\Sigma^+)$. Then $|\mathfrak{a}^+-h'(e^r)|\leq \mu$ for all $r\geq r_0$ and we have the following estimate
\begin{align*}
 H(r_0)=h(e^{r_0})&\geq \mathfrak{a}^+{\cdot} e^{r^+}+\mathfrak{b}^+-(\mathfrak{a}^+{+}\mu)\cdot(e^{r^+}{-}e^{r_0})=\mathfrak{b}^++(\mathfrak{a}^+{+}\mu) e^{r_0}-\mu{\cdot} e^{r^+}\\
 \Longrightarrow\qquad g(r_0) &=h'(e^{r_0}){\cdot} e^{r_0}-h(e^{r_0})\\
  &\leq (\mathfrak{a}^+{+}\mu) e^{r_0}-\big(\mathfrak{b}^++(\mathfrak{a}^+{+}\mu) e^{r_0}-\mu{\cdot} e^{r^+}\big)= \mu{\cdot} e^{r^+}-\mathfrak{b}^+.
\end{align*}
This implies that if $r^+$ is not large enough, then $g(r_0)<0$ and all 1-periodic orbits on the level $\{r=r_0\}$ have negative action. So in order to avoid orbits with negative action, $r^+$ has to become arbitrarily large, as $\mu$ is fixed and $\mathfrak{b}^+$ is growing to $+\infty$. As $g(r^-)>0$, the same estimate shows that $r^+$ also grows to $+\infty$, if $|h'(e^r){-}\mathfrak{a}^+|< \mu$ for all $r$, i.e.\ if no 1-periodic orbits outside $W$ and away from $\mathcal{H}$ exist.\\
This means for a cofinal sequence of Hamiltonians $(H_n)$ that the area where $H_n$ is not linear increases in $n$ and fills in the limit the whole positive symplectzation of $\partial(W{\cup}\mathcal{H})$. Up to now, the author does not see how this can be accomplished without creating new 1-periodic orbits near $\mathcal{H}$.\bigskip\\
The solution to this dilemma is to vary the slope of $H$ on $\{x{=}y{=}0\}$, the symplectic cocore of $\mathcal{H}$, first keeping it fixed coming from the center of $\mathcal{H}$ and increasing the slope sharply near $\Sigma^+$. Using the Lyapunov function $f$, we can then show that this construction creates 1-periodic $X_H$-orbits only on $\{x{=}y{=}0\}$. These can be explicitly described and one can show that they do not contribute to the Symplectic homology.
\end{dis}
For the construction of such an $H$, we need the following two technical lemma:
\begin{lemme}\label{lemHamilt}
 Consider $\mathbb{R}^{2n}$ with the standard symplectic structure, the Liouville vector field $Y$  and the functions $x,y,z$ with Hamiltonian vector fields $X_x,X_y,X_z$ as given in \ref{handle}. Let $\Sigma\subset\mathbb{R}^{2n}$ be a smooth hypersurface transverse to $Y$ (i.e.\ $\Sigma$ contact) such that its Reeb vector field $R$ is of the form 
 \[R=c_x{\cdot} X_x-c_y{\cdot} X_y + c_z{\cdot} X_z,\qquad\qquad c_x,c_y,c_z \in C^\infty(\Sigma),\qquad c_x,c_y,c_z>0.\]
 Consider the function $\tilde{h}_\Sigma(y,r)=\mathfrak{a}{\cdot} e^r+\mathfrak{b}$ on $\mathbb{R}\times\Sigma$ and let $h_\Sigma:=\tilde{h}_\Sigma\circ \Phi^{-1}$ be its pushforward onto $\mathbb{R}^{2n}$ by the symplectic embedding $\Phi:\mathbb{R}\times\Sigma\rightarrow\mathbb{R}^{2n}, (r,p)\mapsto \varphi^r(p)$ provided by the flow $\varphi^t$ of $Y$. Then, the Hamiltonian vector field $X_h$ of $h_\Sigma$ is of the form
 \begin{align*}
  &X_h=C_x{\cdot} X_x - C_y{\cdot} X_y + C_z{\cdot} X_z,\qquad\qquad C_x,C_y,C_z\in C^{\infty}(\mathbb{R}^{2n}),\qquad C_x,C_y,C_z>0.
 \end{align*}
\end{lemme}
\begin{rem}
The assumptions on $\Sigma$ are satisfied, if $\Sigma=\psi^{-1}(c)$ for a function $\psi$ on $x,y,z$ with $\partial_x \psi\big|_\Sigma,\partial_z \psi\big|_\Sigma>0$ and $\partial_y \psi\big|_\Sigma<0$ and $0\not\in\Sigma$.
\end{rem}
\begin{proof}
 As $X_{\tilde{h}}=\mathfrak{a}{\cdot} R$ on $\mathbb{R}{\times}\Sigma$, it follows that on $\mathbb{R}^{2n}$ holds $X_h|_{\varphi^t(\Sigma)}=\mathfrak{a} e^t{\cdot} R_t$, where $R_t$ is the Reeb vector field on $\varphi^t(\Sigma)$. By assumption, the Reeb vector field $R$ on $\Sigma$ satisfies
 \begin{align*}
  R &= c_x X_x-c_y X_y+c_z X_z = \sum_{j=1}^k\left( c_x\,2q_j\frac{\partial}{\partial p_j}+c_y\,p_j\frac{\partial}{\partial q_j}\right) + c_z\sum_{j=k+1}^n \left(q_j\frac{\partial}{\partial p_j}-p_j\frac{\partial}{\partial q_j}\right).
 \end{align*}
 Since $Y=\sum_{j=1}^k\big(2q_j\partial_{q_j}-p_j\partial_{p_j}\big)+\frac{1}{2}\sum_{j=k+1}^n\big(q_j\partial_{q_j}+p_j\partial_{p_j}\big)$, its flow $\varphi^t$ is given by
 \[\varphi^t(q,p)=\Big(\underbrace{...\,,e^{2t}\cdot q_j,e^{-t}\cdot p_j,...}_{j=1,...\,,k}\text{\huge,}\underbrace{...\,,e^{t/2}\cdot q_j,e^{t/2}\cdot p_j,...}_{j=k+1,...\,,n}\Big).\]
 As $\mathcal{L}_Y\lambda=\lambda$ and $\mathcal{L}_Y\omega=\omega$, we find for $R$ and any $\xi\in T_{\varphi^t(p)}\varphi^t(\Sigma)$ that
 \begin{align*}
  \lambda_{\varphi^t(p)}\big(D\varphi^t_p R\big) &=\big({\varphi^t}^\ast \lambda\big)_p(R)&&=e^t\cdot \lambda_p(R)&&=e^t,\\
  \omega_{\varphi^t(p)}\big(D\varphi^t_p R\,,\,\xi\big) &=\big({\varphi^t}^\ast\omega\big)_p\big(R,(D\varphi^t_p)^{-1}(\xi)\big)&&=e^t\cdot\omega_p\big(R,(D\varphi^t_p)^{-1}(\xi)\big)&&=0,
 \end{align*}
as $R$ is the Reeb vector field and $(D\varphi^t_p)^{-1}(\xi)\in T\Sigma$. This shows that $e^{-t}\cdot D\varphi^t R$ is the Reeb vector field $R_t$ of $\varphi^t(\Sigma)$. Hence we find that $X_h$ is of the announced form as
\[X_h|_{\varphi^t(\Sigma)}=\mathfrak{a} e^t{\cdot} R_t = \mathfrak{a}\cdot D\varphi^t(R) = \mathfrak{a} e^{-t}c_xX_x - \mathfrak{a} e^{2t}c_yX_y+\mathfrak{a} e^{t/2}c_zX_z.\qedhere\]
\end{proof}
\begin{lemme} \label{interpolation}
Let $(\Sigma,\alpha)$ be a compact contact manifold with contact form $\alpha$ and symplectization $\big(\mathbb{R}{\times}\Sigma,\omega{=}d(e^r\alpha)\big)$ and let $||\cdot||$ denote a norm with respect to a metric given by an $\omega$-compatible almost complex structure. Let $\veps,\delta,c>0$ be constants.\\
Then there exists a smooth monotone increasing function $g:\mathbb{R}\rightarrow[0,1]$ such that 
 \[g(e^r)=0\quad\text{ for }\quad r\leq -\veps\qquad\text{ and }\qquad g(e^r)=1\quad\text{ for }\quad r\geq 0\tag{$\ast$}\]
 and for all $\phi,\psi\in C^1(\Sigma\times\mathbb{R})$ with $\phi|_{\Sigma\times\{0\}}=\psi|_{\Sigma\times\{0\}}$ and $|\partial_r\phi(r,p)-\partial_r\psi(r,p)|<c$ for all $(r,p)\in[-\veps,0]{\times}\Sigma$, holds that the Hamiltonian vector fields $X_\phi, X_\psi$ satisfy
 \[\sup_{(r,p)\in\Sigma\times\mathbb{R}}\left|\left|X_{\phi+(\psi-\phi)g}(r,p)-\Big(X_\phi(r,p)+\big(X_\psi(r,p)-X_\phi(r,p)\big)\cdot g(e^r)\Big)\right|\right|\leq \delta.\tag{$\ast\ast$}\]
 In other words, we can interpolate between $\phi$ and $\psi$ along $[-\veps,0]\times\Sigma$, such that the Hamiltonian vector field $X_{\phi+(\psi-\phi)g}$ of the interpolation is arbitrary close to the interpolation of the Hamiltonian vector fields $X_\phi$ and $X_\psi$.
\end{lemme}
\begin{proof}
 As the Hamiltonian vector field of $e^r$ is the Reeb vector field $R$, we calculate
 \[X_{\phi+(\psi-\phi)g}(r,p)=X_\phi(r,p)+\big(X_\psi-X_\phi\big)(r,p)\cdot g(e^r)+\big(\psi-\phi\big)(r,p)\cdot g'(e^r)\cdot R(p).\]
Therefore, $(\ast\ast)$ translates to
\[ \left|\left|\big(\psi-\phi\big)(r,p)\cdot g'(e^r)\cdot R(p)\right| \right| \leq \delta \qquad\forall (r,p)\in\Sigma{\times}[-\veps,0].\]
Using $\phi|_{\Sigma\times\{0\}}=\psi|_{\Sigma\times\{0\}}$, we can estimate the left hand side as follows:
\begin{align*}
 \left|\left|\big(\psi-\phi\big)(r,p)\cdot g'(e^r)\cdot R(p)\right| \right|&=\left|\left|-\int_r^0\partial_s\big(\psi-\phi\big)(p,s)\,ds\cdot g'(e^r)\cdot R(p)\right| \right|\\
 &\leq c\cdot (-r) \cdot g'(e^r)\cdot ||R||_\infty.
\end{align*}
 If we write $z=e^r$, we hence find that $(\ast\ast)$ is satisfied, if $0\leq g'(z) \leq \frac{-\delta}{c||R||\log z}$ for all $z\in[e^{-\veps},1]$. As $\int_{e^{-\veps}}^1 \frac{-\delta}{c||R||\log z}\,dz=\infty$, we can choose a smooth function $\tilde{g}$ satisfying
 \[0\leq \tilde{g}(z)\leq \frac{-\delta}{c||R||_\infty \log z},\quad\; \tilde{g}\equiv 0 \;\text{ for } \;z\leq e^{-\veps}\,\;\text{ or }\;\,z\geq 1 \quad\; \text{ and }\quad\; \int_{e^{-\veps}}^1 \tilde{g}(z)\,dz = 1.\]
Setting $\displaystyle g(e^r)=g(z):=\int_{e^{-\veps}}^z\tilde{g}(s)\,ds$ then gives the desired function.
\end{proof}
Now, we construct $H$ in two steps, first extending an admissible Hamiltonian $H|_W$ on $W$ via $\psi_\delta$ to $W{\cup} \mathcal{H}_\delta$, then constructing a linear extension to $\widehat{W{\cup} \mathcal{H}_\delta}$. For simplicity, we assume that $H|_W$ is of the form $H|_W=1{\cdot} e^r-2$ near $\Sigma^-=\{r\}{\times}\partial W$. Then $\Sigma^-=\{H|_W=-1\}$ and $H|_W$ has slope 1 on $\Sigma^-$. For the general case $H|_W=\mathfrak{a}{\cdot} e^r+\mathfrak{b}$, we take $H$ as constructed below and extend $H|_W$ with $\mathfrak{a}{\cdot} H +\mathfrak{b}+2\mathfrak{a}$. Explicitly, the two steps are as follows:
\begin{itemize}
 \item \underline{Step 1}: Recall that the isotropic sphere $S\subset\Sigma^-=\{\phi=-1\}$ is given by
\[ S:=\{x{=}z{=}0,\; y=1\}.\]
For a small neighborhood $U$ of $S$ identify the isotropic setup $\big(U,\omega,Y,\Sigma^-\cap U,S\big)$ with an isotropic setup $\big(U,\omega,Y,(\{r\}{\times}\partial W)\cap U,S\big)$ in $W$ (by abuse of notation, we use the same letters for identified objections on $W$ resp. $\mathbb{R}^{2n}$). Consider the following linear Hamiltonian \[\tilde{h}^-_\Sigma:\mathbb{R}\times\Sigma^-\rightarrow\mathbb{R}, \tilde{h}_\Sigma^-(,p)=1{\cdot} e^r-2\]
and its pushforward $h_\Sigma^-$ onto $\mathbb{R}^{2n}$ defined by $h^-_\Sigma=\tilde{h}_\Sigma^-\circ \Phi^{-1}$, where $\Phi(r,p)=\varphi^r(p)$ is the symplectic embedding provided by the flow $\varphi^t$ of $Y$. Note that $H|_W$ coincides with $h_\Sigma^-$ under the identification of isotropic setups.\\
As the Reeb vector field $R_{\Sigma^-}$ of $(\Sigma^-,\lambda|_{T\Sigma^-})$ coincides with the Hamiltonian vector field $X_\phi$ on $S$, we find $X_{h_\Sigma^-}=R_{\Sigma^-}=X_\phi$ and hence $dh_\Sigma^-=d\phi$ on $S$. As also $h_\Sigma^-(\Sigma^-)=\phi(\Sigma^-)=-1$, we find that $h_\Sigma^-$ and $\phi$ coincide up to first order on $S$. Therefore, given any neighborhood $U^\delta$ of $S$, there exists a function $\hat{\phi}\in C^\infty(\mathbb{R}^{2n})$ and a neighborhood $\hat{U}^\delta\subset U^\delta$, such that $\hat{\phi}\equiv h_\Sigma^-$ on $\mathbb{R}^{2n}\setminus U^\delta,\; \hat{\phi}\equiv\phi$ on $\hat{U}^\delta$ and $\hat{\phi}$ is arbitrarily $C^1$-close to $h_{\Sigma}^-$. As $X_\phi=X_x-X_y+X_z$ and $X_{h_\Sigma^-}=C_x^- X_x-C_y^- X_y+C_z^- X_z$ with $C_x^-,C_y^-,C_z^->0$ by Lemma \ref{lemHamilt}, we can additionally arrange that
\[X_{\hat{\phi}} = \hat{C}_x{\cdot} X_x - \hat{C}_y{\cdot} X_y + \hat{C}_z{\cdot} X_z\quad\text{ with }\quad \hat{C}_x,\hat{C}_y,\hat{C}_z>0.\]
As the $X_x$- and $X_z$-part of $X_\phi$ and $X_{h^-_\Sigma}$ are both 0 on $\{x=z=0\}$, we can make $U^\delta$ and $\hat{U}^\delta$ arbitrarily thin in the $x$- and $z$-direction, while keeping a fixed size in the $y$-direction. This allows us to choose in (\ref{eqXX}) for the definition of $\psi_\delta$ a fixed $\veps$ for all handles.\\
Fix such an $\veps$ sufficiently small and choose $\delta$ depending on $U^\delta$ so small such that the lower boundary $\mathcal{H}_\delta{\cap}\Sigma^-=\Sigma^-\setminus\Sigma^\delta$ lies in $\hat{U}^\delta$. Then set
\begin{align*}
 \hat{H}& : W\cup \mathcal{H}_\delta\rightarrow \mathbb{R},\qquad\qquad \hat{H}=
 \begin{cases}\psi_\delta & \text{ on }\big(\hat{U}^\delta\cap\{\phi{\leq}{-}1\}\big)\cup \mathcal{H}_\delta\\ 
 \hat{\phi} & \text{ on }\big(U^\delta\cap\{\phi{\leq}{-}1\}\big)\setminus \hat{U}^\delta\\
 H|_W & \text{ on }W\setminus U^\delta
\end{cases}.
\end{align*}
Since $\psi_\delta=\hat{\phi}$ outside a small neighborhood of $\mathcal{H}_\delta$, $\phi=\hat{\phi}$ on $\hat{U}^\delta$ and $\hat{\phi}=h_\Sigma^-=K$ outside $U^\delta$, we find that $\hat{H}$ is smooth on its domain. Moreover, as $\psi_\delta,\hat{\phi}$ and $h_\Sigma^-$ satisfy $(\psi1)$ on $U\cup \mathcal{H}_\delta$, so does $\hat{H}$, i.e.\ there exist smooth $\hat{C}_x,\hat{C}_y,\hat{C}_z>0$ such that
\[X_{\hat{H}}=\hat{C}_x{\cdot} X_x-\hat{C}_y{\cdot} X_y+\hat{C}_z{\cdot} X_z.\]
See Figure \ref{fig6} for the areas where $\hat{H}$ is defined.\pagebreak\\
\begin{figure}[ht]
\centering
\begin{minipage}[ht]{7cm}
 \resizebox{7cm}{!}{\input{extension_psi1.pdf_t}}
 \caption*{The handle $\mathcal{H}_\delta$}
 \end{minipage}
 \begin{minipage}[ht]{7cm}
  \resizebox{7cm}{!}{\input{extension_psi2.pdf_t}}
 \caption*{The sets $U^\delta\cap\{\phi\leq -1\}$ and $\hat{U}^\delta\cap\{\phi\leq -1\}$}
 \end{minipage}
\begin{minipage}[ht]{7cm}
 \resizebox{7cm}{!}{\input{extension_psi3.pdf_t}}
 \caption*{The area, where $\hat{H}$ is defined}
\end{minipage}
\begin{minipage}[ht]{7cm}
 \resizebox{7cm}{!}{\input{extension_psi4.pdf_t}}
 \caption*{The completion of the handle $[0,\infty)\times\Sigma^\delta$}
\end{minipage}
 \caption{\label{fig6}Areas, where $\hat{\psi}$ is defined}
\centering
\end{figure}
\item \underline{Step 2} Consider on $\mathbb{R}\times\Sigma^\delta$ the linear function $h^\delta$ given by
\[\tilde{h}^\delta(r,p)=1{\cdot} e^r-2\]
and its pushforward $h^\delta=\tilde{h}^\delta$ to $\mathbb{R}^{2n}$ by the flow of $Y$.\\
On $\widehat{W{\cup} \mathcal{H}_\delta}\setminus\big(W{\cup} \mathcal{H}_\delta\big)$ we define $H$ by $H=h^\delta$.\\
On $W{\cup} \mathcal{H}_\delta$ we define $H$ as an interpolation between $\widehat{H}$ and $h^\delta$, i.e.\
\begin{align*}H:=\hat{H}+\big(h^\delta-\hat{H}\big)\cdot g(h^\delta),
\end{align*}
where $g$ is a function given by Lemma \ref{interpolation} such that for $\tau>0$ small holds
\[g\big(e^r{-}2\big)= 0\quad \text{ for } \quad r\leq -\tau\qquad\text{ and }\qquad g\big(e^r{-}2\big) = 1\quad \text{ for } \quad r\geq 0\]
and $X_H$ is arbitrarily close to the interpolation of $X_{\hat{H}}$ and $X_{h^\delta}$.\\
Note that on $\mathbb{R}\times\big(\partial W{\setminus} U^\delta\big)$ holds that $H|_W=h_\Sigma^-=e^r{-}2=h^\delta$ so that in this area we interpolate between the same Hamiltonians. On the other hand on $\mathbb{R}\times\Big(\partial\big(W{\cup} \mathcal{H}_\delta\big)\setminus\big(\partial W{\setminus} U^\delta\big)\Big)$ we have by Lemma \ref{lemHamilt} and the construction of $\widehat{H}$ that
\begin{align*}
 X_{\hat{H}} &= \hat{C}_x{\cdot} X_x-\hat{C}_y{\cdot} X_y + \hat{C}_z{\cdot} X_z, & \hat{C}_x,\hat{C}_y,\hat{C}_z&>0\\
 X_{h^\delta} &= C_x^\delta{\cdot} X_x-C_y^\delta{\cdot} X_y + C_z^\delta{\cdot} X_z, & C_x^\delta,C_y^\delta,C_z^\delta&>0\\
 X_{g(h^\delta)}&=g'(h^\delta)\big(C_x^\delta{\cdot} X_x-C_y^\delta{\cdot} X_y + C_z^\delta{\cdot} X_z\big).
\end{align*}
Hence $X_H$ is also of this form and satisfies $(\psi1)$, as it is close to the interpolation of $X_{\hat{H}}$ and $X_{h^\delta}$. It follows from $(\psi1)$ with the help of the Lyapunov function $f$ that 1-periodic orbits of $X_H$ are either 1-periodic orbits of $X_H|_W$ inside $W$ or the constant orbit at the center of $\mathcal{H}_\delta$, non-constant orbits on the handle near $\Sigma^\delta\cap\mathcal{H}_\delta$ or Hamiltonian orbits that pass over the handle, go into $W$ and come back. Note that the last type of orbits is not 1-periodic if the handle is chosen sufficiently thin, so that the only new orbits after attaching the handle are the constant one at the center of $\mathcal{H}_\delta$ and those on the outer boundary of $\mathcal{H}_\delta$.
\end{itemize}

\subsection{New closed orbits and their Conley-Zehnder indices}\label{sec5.4}
We saw at the end of the last paragraph that all 1-periodic orbits of $X_H$ that are not orbits of $X_H|_W$ lie on the handle. Moreover, we saw that there exist functions $C_x, C_y, C_z>0$ such that near the handle $X_H=C_x{\cdot} X_x-C_y{\cdot} X_y+C_z{\cdot} X_z$. Hence all new 1-periodic orbits are contained in $\{x{=}y{=}0\}$. There we have
\[\widehat{H}(0,0,z)=\psi_\delta(0,0,z)=\Big(1+\frac{1+\veps}{\delta(1{+}2\veps)}\Big)\cdot z-(1{+}\veps)=\veps{\cdot} e^r-(1{+}\veps)\]
in symplectization coordinates adapted to $\Sigma^\delta$ (see Remarks \ref{remarksonpsi}). As $h_\delta=e^r{-}2$, we find that on $\{x{=}y{=}0\}$ the function $C_z$ is an incereasing function in $z=\frac{1}{2}\sum_{j=k+1}^n (q_j^2+p_j^2)$ interpolating between the constants $\big(1+\frac{1+\veps}{\delta(1+2\veps)}\big)$ and $\big(1+\frac{1+\veps}{\delta(1+2\veps)}\big)\big/\veps$.\\
For the proof of the Invariance Theorem, we need to determine the 1-periodic Hamiltonian orbits of $\mathfrak{a}{\cdot} H +\mathfrak{b}+2\mathfrak{a}$ for $\mathfrak{a},\mathfrak{b}\in\mathbb{R}$. The resulting Hamiltonian vector field is simply $\mathfrak{a}{\cdot} X_H$. Let $\varphi_H^t$ denote its flow. Then we calculate
\[\frac{d}{dt}z(\varphi^t_H)= dz(\mathfrak{a}{\cdot} X_H)=\mathfrak{a} C_z\cdot dz(X_z)=0.\]
It follows that $z$ is constant along flow lines of $\varphi_H^t$ and as $C_z$ is a function of $z$ on $\{x{=}y{=}0\}$ it follows that $C_z$ is also constant along flow lines of $\varphi_H^t$ on $\{x{=}y{=}0\}$. Introducing the complex coordinates $z_j=q_j+i\cdot p_j$, we find that 
\[ X_z=\Big(\underbrace{0,\dots,0}_{j=1,...,k}\, \textbf{,}\underbrace{\dots,iz_j,\dots}_{j=k+1,...,n}\Big).\]
As $X_x=X_y=0$ on $\{x{=}y{=}0\}$, the flow $\varphi^t_H$ on his set is given by
\begin{equation}\label{eqA}
  \varphi^t_H(0,\dots,0,z_{k+1},\dots,z_n)=\Big(0,\dots,0,e^{i\mathfrak{a} C_z t}\cdot z_{k+1},\dots, e^{i\mathfrak{a} C_z t}\cdot z_n\Big).
 \end{equation}
A non-constant orbit of $\mathfrak{a}{\cdot} X_H$ is hence 1-periodic if and only if $\mathfrak{a} C_z(z)\in 2\pi\mathbb{Z}$. As $z=\frac{1}{2}\sum_{j=k+1}^n(q_j^2{+}p_j^2)=\frac{1}{2}\sum_{j=k+1}^n ||z_j||^2$, the 1-periodic orbits form families that are diffeomorphic to the standard sphere $S^{2(n-k)-1}$, except for the constant orbit at $z=0$.\bigskip\\
Next, we calculate the Conley-Zehnder indices $\mu_{CZ}$ of these orbits. Let $\gamma$ denote a 1-periodic orbit of $\mathfrak{a}\cdot X_H$. In order to calculate $\mu_{CZ}(\gamma)$, we identify $T_{\gamma(t)}\mathbb{R}^{2n}$ with $\mathbb{R}^{2n}$ in the obvious way. This yields a path $\Phi_\gamma$ in $Sp(2n)$ given by $\Phi_\gamma(t)=D\varphi^t(0)$. Differentiating $\Phi_\gamma$ yields:
\begin{align*}
 \frac{d}{dt}\Phi_\gamma(t)&=\frac{d}{dt}D\varphi^t_H(0)=D\left(\frac{d}{dt}\varphi^t_H(0)\right)=D\; \mathfrak{a}{\cdot} X_H\big(\varphi^t_H(0)\big)\\
 &=\mathfrak{a}\cdot D\big(C_x X_x- C_y X_y+ C_z X_z\big)\big(\varphi^t_H(0)\big)\tag{cf.\ (\ref{eqXX})}\\
&=\mathfrak{a}\cdot diag\bigg(\underbrace{\dots,\Big(\begin{smallmatrix} 0 & C_y\\2C_x& 0\end{smallmatrix}\Big),\dots}_{j=1,...\,,k}\text{\Huge,}\underbrace{\dots, i C_z,\dots}_{j=k+1,...,n}\bigg)\circ\Phi_\gamma(t).
\end{align*}
Note that no derivatives of $C_x$ or $C_y$ are involved, as $X_x{=}X_y{=}0$ on $\{x{=}y{=}0\}$. It follows that $\Phi_\gamma$ is of block form $\Phi_\gamma = diag\big(\Phi^1_\gamma,...\,,\Phi_\gamma^n\big)$, where the $\Phi^j_\gamma$ are paths of $2{\times}2$-matrices which are solutions of ordinary differential equations with $\Phi^j_\gamma(0)=\mathbbm{1}$ and
\begin{align*}
 {\textstyle\frac{d}{dt}}\Phi^j_\gamma(t)&=\mathfrak{a}\begin{pmatrix}0 & C_y\\ 2C_x & 0\end{pmatrix}\Phi^j_\gamma(t) = \begin{pmatrix} 0 & -1\\ 1& 0\end{pmatrix}\mathfrak{a}\begin{pmatrix}  2C_x & 0\\0& -C_y\end{pmatrix}\Phi^j_\gamma(t)& j&=1,...\,,k\\
 {\textstyle\frac{d}{dt}}\Phi^j_\gamma(t)&=i\mathfrak{a} C_z\cdot\Phi^j_\gamma(t) & j&=k{+}1,...\,, n.
\end{align*}
As the matrix $\mathfrak{a}\left(\begin{smallmatrix}2C_x & 0\\ 0 & - C_y\end{smallmatrix}\right)$ has for all $t$ signature zero, it follows with (CZ0) that $\mu_{CZ}(\Phi^j_\gamma)=0$ for $1{\leq} j{\leq} k$. For $k{+}1{\leq} j{\leq} n$, we find $\Phi^j_\gamma(t)=e^{i\mathfrak{a} C_z\cdot t}$ and by (CZ1) that
\[\mu_{CZ}\big(\Phi^j_\gamma\big)= \left\lfloor\frac{\mathfrak{a} C_z}{2\pi}\right\rfloor+\left\lceil\frac{\mathfrak{a} C_z}{2\pi}\right\rceil,\qquad j=k{+}1,...\,n.\]
Using its direct sum property, we get therefore for the whole Conley-Zehnder index of $\gamma$
\[\mu_{CZ}(\gamma)=\mu_{CZ}(\Phi_\gamma) =\sum_{j=1}^n \mu_{CZ}\big(\Phi^j_\gamma\big)
=(n{-}k)\cdot \left(\left\lfloor\frac{\mathfrak{a} C_z}{2\pi}\right\rfloor+\left\lceil\frac{\mathfrak{a} C_z}{2\pi}\right\rceil\right).\]
As $C_z\geq 1$, we find for $\mathfrak{a}\rightarrow\infty$ that $\mu_{CZ}(\gamma)\rightarrow \infty$. Choosing a Morse function with 2 critical values on $S^{2(n-k)-1}$, we find that the Morse-Bott index for non-constant $\gamma$ is given by
\begin{align*}
 \mu(\gamma)&=(n{-}k)\left(\left\lfloor\frac{\mathfrak{a} C_z}{2\pi}\right\rfloor+\left\lceil\frac{\mathfrak{a} C_z}{2\pi}\right\rceil\right)-\frac{1}{2}\big(2(n{-}k)-1\big)+\frac{1}{2}+\left\lbrace\begin{smallmatrix}0\\2(n-k)-1\end{smallmatrix}\right.\\
 &=(n{-}k)\left(\left\lfloor\frac{\mathfrak{a} C_z}{2\pi}\right\rfloor+\left\lceil\frac{\mathfrak{a} C_z}{2\pi}\right\rceil\right)-(n{-}k)+1+\left\lbrace\begin{smallmatrix}0\\2(n-k)-1\end{smallmatrix}\right..
\end{align*}
Since for 1-periodic $\gamma$ holds $\mathfrak{a} C_z(z) \in 2\pi\mathbb{Z}$, their Morse-Bott indices are of the form
\begin{equation}\label{indexonhandle}
 \mu(\gamma)=(n{-}k)(2l{-}1)+\left\lbrace\begin{smallmatrix}0\\1\end{smallmatrix}\right.\qquad\text{ for } l\in\mathbb{N}.
\end{equation}

\subsection{Handle attachement and Symplectic homology}
In this section, we like to discuss some applications of our construction of Hamiltonians on subcritical $k$-handles $\mathcal{H}$. The first is the proof of the Invariance Theorem namely that $SH_\ast(W{\cup}\mathcal{H})=SH_\ast(W)$, if we attache $\mathcal{H}$ to a $(2n)$-dimensional Liouville domain, $k<n$. Secondly, we prove that subcritical contact surgery on an a.f.g. contact structure produces again an a.f.g. contact structure. Finally, we prove Theorem \ref{theoinfinitecontactstr}.
\begin{proof}[\textnormal{\textbf{Proof of Theorem \ref{theoinvsur} (Invariance Theorem)}}]~\\
\textit{(If $V=W{\cup}\mathcal{H}$, then $SH_\ast(W)=SH_\ast(V)$ and $SH^\ast(W)=SH^\ast(V)$.)}\bigskip\\
We will show that the asserted isomorphisms are given  by the transfer maps
\[\pi_\ast(W,V): SH_\ast(V)\rightarrow SH_\ast(W)\quad\text{ and }\quad \pi^\ast(W,V): SH^\ast(W)\rightarrow SH^\ast(V),\]
for exactly embedded subdomains $W\subset W{\cup}\mathcal{H}= V$. They were given in \ref{sectransfer}, Cor.\ \ref{transfer}, as the truncation maps $SH_\ast(V)\rightarrow SH_\ast^{\geq0}(W{\subset}V)$ and $SH^\ast_{\geq0}(W{\subset}V)\rightarrow SH^\ast(V)$ composed with the isomorphisms $SH_\ast^{\geq0}(W{\subset}V)\cong SH_\ast(W)$ and $SH^\ast_{\geq0}(W{\subset}V)\cong SH^\ast(W)$. Therefore, we only have to prove that the truncation maps are isomorphisms. The idea is to construct a cofinal sequence of Hamiltonians $(H_l)\subset Ad(V)\cap Ad(W{\subset}V)$ for which we can directly show
\begin{equation}\label{proofinvtheo}
 \begin{aligned}
  SH^{\geq0}_\ast(W{\subset}V) = \varinjlim_{l\rightarrow\infty} FH^{>0}_\ast(H_l)&\overset{(1)}{\simeq}\varinjlim_{l\rightarrow\infty} FH_\ast(H_l)=SH_\ast(V)\\
 SH_{\geq0}^\ast(W{\subset}V) = \varprojlim_{l\rightarrow\infty} FH_{>0}^\ast(H_l) &\overset{(2)}{\simeq}\varprojlim_{l\rightarrow\infty} FH^\ast(H_l) = SH^\ast(V).
 \end{aligned}
\end{equation}
We assume that all closed Reeb orbits of the contact form $\alpha=\lambda|_{\partial W}$ on $\partial W$ are transversely non-degenerate and that the attaching area does not intersect with any closed Reeb orbit. This is generically satisfied (see \cite{FraCie}, App.\ B). To start the construction fix an increasing sequences of real numbers $(\mathfrak{a}_l)$, such that $\mathfrak{a}_l\not\in spec(\partial W,\alpha)$ for all $l$ and $\mathfrak{a}_l\rightarrow\infty$. Moreover, fix a positive sequence $\veps(l)$ with $\veps(l)\rightarrow 0$. Then choose an increasing sequence of non-degenerate Hamiltonians $H_l$ on $W$ that is on $\partial W\times (-\veps(l),0]$ of the form
\[H_l|_{\partial W\times(-\veps(l),0]} = \mathfrak{a}_l{\cdot} e^r-\big(1{+}\veps(l)\big)\]
and extend $H_l$ over the handle by a function $\psi$ with $\mathfrak{a}=\mathfrak{a}_l$ and $\mathfrak{b}=-(1{+}\veps(l))$ as described in \ref{handle} and \ref{ExPsi}. For each $l$ choose the handle $\mathcal{H}_l$ so thin such that each trajectory of $X_{H_l}$ which leaves and reenters the handle has length greater than 1. This is possible as the attaching sphere is strictly isotropic, $\dim S=k<n=\frac{1}{2}\dim V$, and can hence be chosen to have no Reeb chords.\\
Note that $\widehat{W{\cup}\mathcal{H}_l}$ is the same symplectic manifold for all choices of $\mathcal{H}_l$, as choosing different handles is to be understood as choosing different parametrizations on the cylindrical part of the completion. Moreover, if $\mathcal{H}_l$ is thinner then $\mathcal{H}_{l-1}$, then $\partial(W{\cup}\mathcal{H}_l)$ lies inside $W{\cup}\mathcal{H}_{l-1}$ (see Remarks \ref{remarksonpsi}). This guarantees that $H_l\geq H_{l-1}$ everywhere, since inside $W$ this holds by assumption, on the handle this holds by construction and on the cylindrical part of the symplectization the slope of $H_l$ in coordinates adapted to $\partial(W{\cup}\mathcal{H}_{l-1})$ is at least $\mathfrak{a}_l>\mathfrak{a}_{l-1}$, as $\partial(W{\cup}\mathcal{H}_l)$ lies inside $W{\cup}\mathcal{H}_{l-1}$. Thus we obtain a cofinal admissible sequence $(H_l)$, whose 1-periodic orbits having positive action are all contained in $W$. Recall from \ref{sectrunc} that we have the long exact sequences
\begin{align*}
 \dots\rightarrow FH^{\geq 0}_{j+1}(H_l)\rightarrow FH^{<0}_j(H_l)\rightarrow FH_j(H_l)\rightarrow FH_j^{\geq 0}(H_l)\rightarrow\dots\\
 \dots\rightarrow FH_{< 0}^{j-1}(H_l)\rightarrow FH_{\geq0}^j(H_l)\rightarrow FH^j(H_l)\rightarrow FH^j_{< 0}(H_l)\rightarrow\dots
\end{align*}
and note that $FH^{\geq0}_j(H_l)$ is generated by all 1-periodic orbits of $H_l$ inside $W$, while $FH^{< 0}_j(H_l)$ is generated by all other orbits. The orbits of negative action all lie on the handle and are explicitly given in (\ref{eqA}). They are all transversely non-degenerate and their Morse-Bott indices are given by $(n{-}k)\big(\frac{\alpha C_z}{\pi}{-}1\big)+\left\lbrace\begin{smallmatrix}1\\ 2(n-k)\end{smallmatrix}\right.$. It follows that the possible values of $\mu(\gamma)$ increase to $\infty$ as the slope $\mathfrak{a}=\mathfrak{a}_l$ tends to $\infty$ (see \ref{sec5.4}). Therefore, $FH^{<0}_j(H_l)$ becomes eventually zero as $l$ increases and so does $FH^{<0}_{j+1}(H_l)$. This implies that $FH_j(H_l)\rightarrow FH^{\geq 0}_j(H_l)$ is an isomorphism for $l$ large enough. As the direct limit is an exact functor, these maps converge to an isomorphism in the limit, proving (\ref{proofinvtheo}(1)).\\
In the cohomology case, the line of arguments is the same. Even though taking inverse limits is not exact, it still takes the isomorphism $FH^j_{\geq 0}(H_n)\rightarrow FH^j(H_n)$ to an isomorphism in the limit, as it is left exact (see \cite{eilenberg}, Thm.\ 5.4 or \cite{bourbaki2}, $\S6$, no.3, prop.\ 4). This proves (\ref{proofinvtheo}(2)), which implies the theorem.
\end{proof}
In \cite{CieOan}, Prop. 9.14, Cieliebak and Oancea show that $SH_\ast(W{\cup}\mathcal{H})\rightarrow SH_\ast(W)$ being an isomorphism implies the following corollary.
\begin{cor}
 If $V=W{\cup}\mathcal{H}$ is a Liouville domain that is obtained from $W$ by attaching a subcritical handle, then the Rabinowitz-Floer homologies $RFH(V,\partial V)$ and $RFH(W,\partial W)$ of $V$ and $W$ respectively are isomorphic.
\end{cor}
Now, we show that the property of being asymptotically finitely generated is preserved under subcritical contact surgery. Depending on the degree, the bound $b_k(\xi)$ stays unchanged under this procedure or is increased by 1. Here, we will perform contact surgery by attaching a symplectic handle $\mathcal{H}$ to the negative symplectization $W$ of a contact manifold $(\Sigma,\alpha)$. The result of the surgery is then the boundary $\partial\big(W{\cup}\mathcal{H}\big)$ together with its induced contact structure.
\begin{prop}\label{propcofinalHamilt}
 Let $(\Sigma,\alpha)$ be a $(2n{-}1)$-dimensional compact contact manifold such that $\xi=\ker \alpha$ is an asymptotically finitely generated contact structure in degree $j$ with bound $b_j(\xi)$. Let $f_l:\Sigma\rightarrow\mathbb{R}$ and $\mathfrak{a}_l$ be the sequences that show that $\xi$ is asymptotically finitely generated. Let $W=\big([-\infty,0]{\times}\Sigma,e^r{\cdot}\alpha\big)$ denote the negative symplectization of $(\Sigma,\alpha)$.\\
 Assume that $W{\cup}\mathcal{H}$ is obtained from $W$ by attaching a subcritical $k$-handle $\mathcal{H}$, $k<n$, such that along the attaching sphere $S\subset\Sigma$ holds that $f_l|_S$ is constant and if $S$ intersects a Reeb orbit of $\alpha_l=e^{f_l}{\cdot}\alpha$ twice, then the length of this orbit between the two intersections is longer than $\mathfrak{a}_l$.\\
 Then the contact structure on $\partial\big(W{\cup}\mathcal{H}\big)$ is also symptotically finitely generated in degree $j$ with bound $b_j(\xi)$, unless $j=(n{-}k)N+\left\lbrace\begin{smallmatrix}0\\1\end{smallmatrix}\right.$ for $N>0$ odd, then the bound is $b_j(\xi)+1$. Moreover, the contact forms $(\beta_l)$ that prove this can be chosen to agree with $(\alpha_l)$ outside an arbitrary small neighborhood of $\mathcal{H}$.
\end{prop}
\begin{rem}
 The conditions on the attaching sphere $S$ are automatically satisfied if $\alpha_l$ does not depend on $l$ near $S$ and if $S$ does not lie on any closed Reeb orbit.
\end{rem}
\begin{proof}
 Let $\widehat{W}$\footnote{As for Liouville domains, one can define the completion for any exact symplectic manifold $V$ with positive contact boundary $\Sigma$ by glueing the positive symplectization of $\Sigma$ to $V$  along the boundary $\partial V=\Sigma$.} denote the completion of $W$, that is here exactly the whole symplectization of $(\Sigma,\alpha)$. As in \ref{secsetup}, consider the subset $W_l$ of $\widehat{W}$ defined by $W_l:=\big\{(r,p)\,\big| \linebreak r{\leq} f(p), p{\in}\Sigma\big\}$. We note that $f_l(p)\leq f_{l-1}(p)\leq 0\;\forall\,p,l$ implies $W_l\subset W_{l-1}\subset W$ and that $\partial W_l=\Sigma_l=\big\{(f(p),p)\,\big|\,p\in\Sigma\big\}$.\\
 For every $l$ choose the handle $\mathcal{H}_l$ attached to $W_l$ so thin, such that each Reeb trajectory that leaves the attaching area and reenters it later is longer then $\mathfrak{a}_l$. This is possible as the attaching sphere $S$ does not intersect twice with Reeb orbits shorter then $\mathfrak{a}_l$. Moreover, choose $\mathcal{H}_l$ so thin that $W_l{\cup}\mathcal{H}_l$ lies inside $W_{l-1}{\cup}\mathcal{H}_{l-1}$ (consult Dis.\ \ref{dishandlesdifferentcontact} on how to interpret $W_l{\cup}\mathcal{H}_l$ as a subset of $\widehat{W_{l-1}{\cup}\mathcal{H}_{l-1}}$). Note that $\widehat{W_l{\cup}\mathcal{H}_l}$ and $\widehat{W{\cup}\mathcal{H}}$ are easily identified since $\partial\big(W_l{\cup}\mathcal{H}_l\big)$ can be understood as a contact hypersurface inside the negative symplectization of $\partial\big(W{\cup}\mathcal{H}\big)$ (see Rem. \ref{remarksonpsi} and Dis. \ref{dishandlesdifferentcontact}).\\
 If we construct the handle $\mathcal{H}_l$ with the help of a function $\psi$ as in \ref{handle} and \ref{ExPsi}, then there are no closed Reeb orbits on the handle except on the cocore and closed Reeb orbits that go over the handle into $\partial W_l$ are longer then $\mathfrak{a}_l$ as $\mathcal{H}_l$ is sufficiently thin.\\
 Let $\beta_l$ denote the contact form on $\partial\big(W_l{\cup}\mathcal{H}_l\big)$. We find that $\beta_l$ agrees with $\alpha_l$ away from $\mathcal{H}_l$. Moreover, the closed Reeb orbits of $\beta_l$ shorter than $\mathfrak{a}_l$ are the same as whose of $\alpha_l$ plus the ones on the cocore of the handle. Of the latter there exists by (\ref{indexonhandle}) at most one, if the degree $j$ equals $(n{-}k)N+\left\lbrace\begin{smallmatrix}0\\1\end{smallmatrix}\right.$ for $N>0$ odd. This gives the bound $b_j(\xi)$ or $b_j(\xi)+1$.\\
 Finally, as $W_l{\cup}\mathcal{H}_l\subset W_{l-1}{\cup}\mathcal{H}_{l-1}\subset W{\cup}\mathcal{H}$, we find that if we express $\beta_l$ in coordinates of the symplectization of $\partial\big(W{\cup}\mathcal{H}\big)$ in the form $\beta_l=e^{g_l}{\cdot}\beta$ with $g_l:\partial\big(W{\cup}\mathcal{H}\big)\rightarrow\mathbb{R}$ then we have
 \[g_l(p)\leq g_{l-1}(p)\leq 0\qquad\forall\; p\in \big(W{\cup}\mathcal{H}\big).\]
 This shows that the contact structure on $\big(W{\cup}\mathcal{H}\big)$ is a.f.g..
\end{proof}

Theorem \ref{theoinvsur} and Proposition \ref{propcofinalHamilt} finally enable us to prove Theorem \ref{theoinfinitecontactstr}.
\begin{proof}[\textnormal{\textbf{Proof of Theorem \ref{theoinfinitecontactstr}}}]~\\
\textit{(If $(\Sigma^{2n-1},\xi)$ is exactly fillable and a.f.g.\ in degree $k\geq n{+}2$, then $\Sigma$ carries infinitely many different exactly fillable contact structures.)}
\begin{itemize}
 \item \underline{Step 1} : It was shown in the authors thesis, \cite{FauckThesis} Thm.\ 109, that for any $n\geq 3$ and $k\geq n{+}2$ there is a Brieskorn manifold $\Sigma_a$ diffeomorphic to $S^{2n-1}$ with an exact contact filling $V_a$ such that 
 \[SH_k(V_a,\Sigma_a)=\big(\mathbb{Z}_2\big)^2.\tag{$\ast$}\]
 Actually, the result was shown for Rabinowitz-Floer homology RFH, but the same arguments or the long exact sequence relating SH and RFH (see \cite{FraCieOan}, (3)) can be used to deduce the same result for Symplectic homology. Explicitly, $\Sigma_a$ can be chosen as
  \[\Sigma_a=\Big\{z{\in}\mathbb{C}^{n+1}\Big|z_0^2{+}z_1^2{+}z_2^2{+}z^{a_3}_3{+}...{+}z^{a_n}_n=0,\;||z||^2{=}1\Big\}, \quad \min_{3\leq j\leq n} a_j > k{-}n{+}3,\]
  where the $a_j$ are sufficiently large positive odd integers satisfying $gcd(a_i,a_j)=1$ for $i\neq j$. This Brieskorn manifold carries a standard contact structure $\xi_a$ with contact form $\alpha_a$ and an exact filling $V_a$ (see \ref{secGenBrie} for a general introduction in Brieskorn manifolds). We claim that $(\Sigma_a,\xi_a)$ has a contact form $\alpha_a'$ which has no totally periodic Reeb flow and is a.f.g. with bound
  \[b_k(\xi_a)=2\qquad \text{ for }\quad n<k<n{-}3+\min_{3\leq j\leq n} a_j.\tag{$\ast\ast$}\]
  A similar, but coarser estimate on $b_k(\xi_a)$ can be obtained via the perturbation described in appendix \ref{appendixA}. To show $(\ast\ast)$, use the perturbation (\ref{UstiPer}) described in \ref{secConSum} with $\left(\begin{smallmatrix} w_0\\ w_1\end{smallmatrix}\right)=\frac{1}{\sqrt{2}}\left(\begin{smallmatrix} z_0+iz_1\\ z_0-iz_1\end{smallmatrix}\right), w_j=z_j$ for $j\geq 2$ and $K(w)=||w||^2+\veps(|w_0|^2-|w_1|^2)$. Define a new contact form $\alpha_a':=K^{-1}{\cdot}\alpha_a$. The Reeb vector field of $\alpha_a'$ is given by 
  \[R_a'=\Big(2i(1{+}\veps)w_0, 2i(1{-}\veps)w_1, 2iw_2, \frac{4i}{a_3}w_3,\dots,\frac{4i}{a_n}w_n\Big).\]
  Note that for $L<\pi{\cdot}\min a_j$, all $L$-periodic $R_a'$-orbits are circles living either in the first or second coordinate. Using the general formula (\ref{muCZperturbed}) for the Conley-Zehnder index on a Brieskorn manifold, one can then show that for $n<k<n{-}3 + \min a_j$ there are exactly two closed $R_a'$-orbits having Morse-Bott index $k$.
  \item \underline{Step 2} : Now let $(\Sigma,\xi)$ be any asymptotically finitely generated contact manifold in degree $k$ with bound $b_k(\xi)$ and with an exact filling $(V,\lambda)$ with well-defined Conley-Zehnder index (conditions on $\pi_1(\Sigma)$ and $c_1(TV)$). Let us denote by $\Sigma\# \Sigma_a$ the connected sum and by $\xi\#\xi_a$ the resulting contact structure. Taking the boundary connected sum of $(V,\Sigma)$ with $N$ copies of $(V_a,\Sigma_a)$ yields by the Invariance Theorem and $(\ast)$:
  \begin{align*}
   && SH_k\Big(V \#\consumd_{j=1}^{N} V_a,\; \Sigma \#\consumd_{j=1}^{N} \Sigma_a\Big)&= SH_k(V,\Sigma)\oplus \bigoplus_{j=1}^{N} SH_k(V_a,\Sigma_a)\\
   &\Rightarrow& \rk SH_k\Big(V \#\consumd_{j=1}^{N} V_a,\; \Sigma \#\consumd_{j=1}^{N} \Sigma_a\Big)&\geq 0+N\cdot 2=2N.\tag{$+$}
  \end{align*}
  Note that $\Sigma_a$ is diffeomorphic to $S^{2n-1}$ so that $\Sigma\#\big(\consumt_{j=1}^{N}\Sigma_a\big)\cong\Sigma$ and that $\big(\Sigma,\xi \#(\consumt_{j=1}^{N} \xi_a)\big)$ is exactly fillable by $V \#(\consumt_{j=1}^{N} V_a)$. Moreover, Prop.\ \ref{propcofinalHamilt} with $(\ast\ast)$ tells us that $\big(\Sigma, \xi \#(\consumt_{j=1}^{N} \xi_a)\big)$ is again a.f.g. in degree $k$ with bound
  \[b_k\Big(\xi \#\consumd_{j=1}^{N} \xi_a\big)\Big)\leq b_k(\xi)+N\cdot\underbrace{b_k(\xi_a)}_{=2}+\underbrace{N}_{\text{orbits on the $N$ handles}}=b_k(\xi)+3N.\]
  This estimate and Prop.\ \ref{PropAsympfiniteGene} imply for any filling $(W,\lambda_W)$ of $\big(\Sigma,\xi\#(\consumt_{j=1}^{N} \xi_a)\big)$ that
  \[\rk SH_k(W)\leq b_k\Big(\xi\#\consumd_{j=1}^{N} \xi_a\Big)\leq b_k(\xi)+3N.\tag{$++$}\]
   It follows from $(+)$ and $(++)$ that for $N,M\in\mathbb{N}$ with $b_k(\xi)+3M<2N$ holds that $\xi \#(\consumt_{j=1}^{M} \xi_a)$ and $\xi \#(\consumt_{j=1}^{N} \xi_a)$ cannot be contactomorphic on $\Sigma$, as the latter has a filling which cannot be a filling of the first. If we set in particular $N_l:=2^l(b_k(\xi)+1)$, then we find
   \[b_k(\xi)+3N_l<(3{\cdot}2^l+1)(b_k(\xi)+1)\leq 2{\cdot}2^{l+1}(b_k(\xi)+1)=2N_{l+1}\leq 2N_m.\]
   Thus $\xi \#(\consumt_{j=1}^{N_l} \xi_a)$ and $\xi \#(\consumt_{j=1}^{N_m} \xi_a)$ are not contactomorphic if $l\neq m$, which provides infinitely many different exactly fillable contact structures on $\Sigma$.
   \item \underline{Bonus} : All contact structures $\xi \#(\consumt_{j=1}^N \xi_a)$ for different values of $N\in\mathbb{N}{\cup}\{0\}$ are different. To show this, let us abbreviate $\xi \#(\consumt_{j=1}^N \xi_a)$ by $\xi\#N{\cdot}\xi_a$. We argue by contradiction. Assume that there exists $M<N$ with $\xi\#M{\cdot}\xi_a$ contactomorphic to $\xi\#N{\cdot}\xi_a$, then we have
   \[\xi\#(N{+}K)\xi_a=\xi\# N{\cdot}\xi_a\#K{\cdot}\xi_a \cong \xi\# M{\cdot}\xi_a\#K{\cdot}\xi_a=\xi\#(M{+}K)\xi_a\qquad \forall \;K{\in}\mathbb{N}.\]
   In particular for $K=N{-}M$, we have
   \[\xi\#(M{+}2K){\cdot}\xi_a=\xi\#(N{+}K){\cdot}\xi_a\cong\xi\#N{\cdot}\xi_a\cong\xi\#M{\cdot}\xi_a.\]
   Repeatig this argument, we find that $\xi\#(M{+}lK){\cdot}\xi_a\cong\xi\#\xi_a$ for all $l\in\mathbb{N}$. However, for $l$ sufficiently large, we know by Step 2 that these two contact structures are different, thus providing a contradiction.\qedhere 
\end{itemize}
\end{proof}


\section{Brieskorn manifolds}\label{secBries}
\subsection{General results}\label{secGenBrie}
In this subsection, we recall the construction of Brieskorn manifolds and their contact structures and fillings. Moreover, we give their Symplectic homology and present Ustilovsky's infinitely many different contact structures on $S^{4m+1}$. This is a shortened version of a similar section in \cite{Fauck1}.\bigskip\\
Let $a = (a_0,a_1, ...\,, a_n)$ be a vector of natural numbers with $a_i\geq 2$ and define a complex polynomial $f\in C^\infty(\mathbb{C}^{n+1})$ by
\[f(z) = z_0^{a_0}+ z_1^{a_1} + ... + z_n^{a_n}.\]
Its level sets $V_a(t):=f^{-1}(t)$ are smooth complex hypersurfaces except for $V_a(0)$, which has a single singularity at zero. The link of this singularity $\Sigma_a:=V_a(0)\cap S^{2n+1}$ is the \textit{Brieskorn manifold} $\Sigma_a$. In \cite{LuMe}, Lutz and Meckert proved that the following 1-form $\lambda_a$ on $\mathbb{C}^{n+1}$ restricts to a contact form $\alpha_a:=\lambda_a|_\Sigma$ on $\Sigma_a$ with Reeb vector field $R_a$:
  \[\lambda_a =\frac{i}{8} \sum^n_{k=0} a_k (z_kd\bar{z}_k - \bar{z}_kdz_k)\qquad\text{ and }\qquad R_a = 4i\left( \frac{z_0}{a_0},\,\dots\,,\frac{z_n}{a_n}\right).\]
In order to define a Symplectic homology associated with $\Sigma_a$, we need a Liouville domain $(V,\lambda)$ with contact boundary $(\Sigma_a,\alpha_a)$. Unfortunately, we cannot take $V_a(0)\cap B_1(0)$, due to the singularity at 0. We overcome this obstacle by constructing an interpolation between $V_a(0)$ and $V_a(\veps)$ for $\veps>0$ small. To do this, choose a smooth monotone decreasing cut-off function $\beta\in C^\infty (\mathbb{R})$ with $\beta(x)= 1$ for $x\leq 1/4$ and  $\beta(x)=0$ for $x\geq 3/4$. Then define
\[ V_a:=V_\veps\cap B_1(0), \quad\text{ where }\quad V_\veps:=\Big\{z\in \mathbb{C}^{n+1}\,\Big|\, z_0^{a_0}+z_1^{a_1}+...+z_n^{a_n} = \veps\cdot\beta\big(||z||^2\big)\Big\}.\]
For $\veps$ small enough, $(V_a,\lambda_a)$ is a Liouville domain with boundary $(\Sigma_a,\alpha_a)$ and vanishing first Chern class $c_1(TV)$ (see \cite{Fauck1}, Prop. 2.1.3 or \cite{FauckThesis}, Prop. 99).\\
Let $\xi_a:=\ker \alpha_a$ denote the contact structure on $\Sigma_a$ defined by $\alpha_a$ and consider on $\mathbb{C}^{n+1}$ the symplectic form $\omega_a = d\lambda_a$ with Liouville vector field $Y_\lambda$, explicitly given by:
\[\omega_a=\frac{i}{4}\sum^n_{k=0} a_kdz_k\wedge d\bar{z}_k\qquad\text{ and }\qquad Y_\lambda(z)=\frac{1}{2}z.\]
The complex gradient of $f$ together with $Y_\lambda$ and $R_a$ trivialize the symplectic complement $\xi_a^\bot$ of $\xi_a$. A symplectic orthonormalization process with respect to $\omega_a$ yields the following symplectic standard basis for $\xi_a^\bot$:
\begin{align*}
 X_1 &= \sqrt{\frac{2}{\sum a_k|z_k|^{2(a_k-1)}}}\cdot \Big( \bar{z}_0^{a_0-1},\dots,\bar{z}_n^{a_n-1}\Big),&Y_1&=i\cdot X_1,\\
 X_2 &= \frac{1}{2}\cdot\left(z_0-\frac{\sum a_kz_k^{a_k}}{\sum a_k|z_k|^{2(a_k-1)}}\cdot \bar{z}_0^{a_0-1} , \dots,z_n-\frac{\sum a_kz_k^{a_k}}{\sum a_k|z_k|^{2(a_k-1)}}\cdot \bar{z}_n^{a_n-1}\right),&Y_2&=R_a.
\end{align*}
The flow $\varphi^t_a$ of $R_a$ is given by $\qquad\displaystyle\varphi^t_a(z)=\left(e^{4it/a_0}\cdot z_0,\,\dots\,,e^{4it/a_n}\cdot z_n\right)$.\bigskip\\
We can obviously view $R_a$ and $\varphi_a^t$ as being defined on $\mathbb{C}^{n+1}$ instead of $\Sigma_a$. This allows us to calculate Conley-Zehnder indices directly on $T\mathbb{C}^{n+1}$ instead of $T\Sigma_a$. The action of $D\varphi_a^t$ on $T\mathbb{C}^{n+1}$ in terms of the standard trivialization is given by the path of diagonal matrices:
\[\Phi^t:=D\varphi^t_a = diag\left( e^{4it/a_0},\dots,e^{4it/a_n}\right)\in Sp(2n+2).\]
Using the above trivialization of $\xi_a^\bot$ by $X_1,Y_1,X_2,Y_2$, we find by some calculation that the action of $D\varphi_a^t$ on $\xi_a^\bot$ is given by
\begin{align*}
 D\varphi^t_a\bigl( X_1(z)\bigr) &= e^{4it}\cdot X_1\bigl( \varphi^t_a(z)\bigr), & D\varphi^t_a \bigl( Y_1(z)\bigr) &= e^{4it}\cdot Y_1\bigl( \varphi^t_a(z)\bigr),\\
 D\varphi^t_a\bigl( X_2(z)\bigr) &= X_2\bigl(\varphi^t_a(z)\bigr), & D\varphi^t_a\bigl( Y_2(z)\bigr) &= Y_2\bigl(\varphi^t_a(z)\bigr).
\end{align*}
The action of $D\varphi_a^t$ on $\xi_a^\bot$ in this trivialization is hence the path of diagonal matrices:
\[\Phi^t_2:=diag (e^{4it}, 1)\in Sp(4).\]
Any trivialization of $\xi_a$ over a disc $u\subset\Sigma_a$, with $\partial u=v$ being a Reeb trajectory, provides us with a linearization $\Phi^t_1\in Sp(2n-2)$ of $\varphi^t_a$ on $\xi_a$. Using this trivialization of $\xi_a$ together with the above trivialization of $\xi^\bot_a$ gives a trivialization of $T\mathbb{C}^{n+1}$, which is homotopic to the standard one, as $T\mathbb{C}^{n+1}$ is trivial. Therefore, we obtain that $\Phi^t = \Psi^t(\Phi^t_1{\oplus} \Phi^t_2)(\Psi^0)^{-1}$ for some contractible loop $\Psi\in Sp(2n{+}2)$. Using $\mu_{CZ}(\Psi\Phi\Psi^{-1})=\mu_{CZ}(\Phi)$, the additivity of $\mu_{CZ}$ with respect to direct sums and $\mu_{CZ}\big(e^{it}|_{[0, T]}\big)=\big\lfloor\frac{T}{2\pi}\big\rfloor{+}\big\lceil\frac{T}{2\pi}\big\rceil$ (see \ref{secConZeh}), we find that for any Reeb trajectory $v$ of length $L\pi/2$ holds:
\begin{equation}\label{eqConZeh}
 \mu_{CZ}(v)=\mu_{CZ}(\Phi_1) = \mu_{CZ}(\Phi)-\mu_{CZ}(\Phi_2) =\underbrace{\sum_{k=0}^n\left(\left\lfloor\frac{L}{a_k}\right\rfloor {+} \left\lceil\frac{L}{a_k}\right\rceil\right)}_{\Phi - comp.} - \underbrace{\left(\left\lfloor L\right\rfloor {+} \left\lceil L\right\rceil\right)}_{\Phi_2-comp.}.
\end{equation}
Let us leave now the general picture and turn to the Brieskorn manifolds, where
\[a=(2,...,2,\,p)\in\mathbb{N}^{n+1},\; p \text{ odd and } n=2m{+}1 \text{ odd.}\]
For fixed $p$, we write $\Sigma_p, \xi_p, \alpha_p,\lambda_p, V_p$ instead of $\Sigma_a, \xi_a, \alpha_a, \lambda_a, V_a$. In \cite{Brie}, Brieskorn showed that $\Sigma_p$ is a $4m{+}1$-dimensional homotopy sphere. Moreover, he showed that $\Sigma_p$ is diffeomorphic to $S^{4m+1}$ if and only if $p\equiv \pm 1\mod 8$ and diffeomorphic to the Kervaire sphere otherwise.\bigskip\\
On $\Sigma_p$, the Reeb flow $\varphi_p^t$ is of the form $\quad \displaystyle \varphi^t_p(z)=\left(e^{2it}{\cdot} z_0,\,...\,,e^{2it}{\cdot} z_{n-1}\,,\,e^{4it/p}{\cdot} z_n\right)$.\bigskip\\
If we set $t=L\pi/2$, we see that $\varphi^t_p$ has closed periodic orbits exactly for $t\in\pi\mathbb{Z}\Leftrightarrow\linebreak L\in2\mathbb{Z}$. Note that we do not have closed orbits of period $t=k\frac{p}{2}\pi\Leftrightarrow L=kp$, since any point $z$ in $\Sigma_p$ has at least two non-zero components, so that $(0,...,0, e^{4it/p}{\cdot} z_n)\not\in\Sigma_p$. The closed orbits associated to a period $t=L\pi/2$ form manifolds $\mathcal{N}^L$ of two types:
\begin{itemize}
 \item if $p\nmid L$, then $\mathcal{N}^L=\big\{z{\in}\Sigma_p\,\big|\,z_n{=}0\big\}=\big\{\,z{\in}\mathbb{C}^{n+1}\,\big|\,z_n{=}0,\,\sum z_k^2{=}0,\,||z||^2{=}1\big\}$, which is diffeomorphic to the unit tangent bundle $S^\ast S^{n-1}$ of $S^{n-1}$.
 \item if $p\;{\mid}\, L$, then $\mathcal{N}^L=\Sigma_p$, which is for $p\equiv \pm 1\mod 8$ diffeomorphic to $S^{2n-1}$.
\end{itemize}
On $S^\ast S^{n-1}$ there exists a Morse function with exactly 4 critical points of Morse indices $0, n{-}2,n{-}1, 2n{-}3$. On $S^{2n-1}$ there exists a Morse function having exactly 2 critical points of Morse index 0 and $2n{-}1$. Let us denote these critical points by
\begin{align*}
 &L\gamma_0,\; L\gamma_{n-2},\; L\gamma_{n-1},\; L\gamma_{2n-3}\; &&\text{ on $\mathcal{N}^L\cong S^\ast S^{n-1} $ if $ p\nmid L$}\\
 &L\gamma_0,\; L\gamma_{2n-1}&&\text{ on $\mathcal{N}^L\cong S^{2n-1}\phantom{S} $ if $ p\mid L$.} 
\end{align*}
With (\ref{eqConZeh}), we find that the Morse-Bott index $\mu(L\gamma_c)$ is given by
\begin{equation}\label{eqMorBot}\begin{aligned}
 \mu(L\gamma_c)&=\mu_{CZ}(L\gamma_c)+\mu_{Morse}(L\gamma_c)-\dim\mathcal{N}^L+{\textstyle \frac{1}{2}}\\
 &=n{\cdot} L+\big\lfloor L/p\big\rfloor+\big\lceil L/p\big\rceil-2L+c-{\textstyle\frac{1}{2}}\dim \mathcal{N}^L+{\textstyle\frac{1}{2}}\\
 &=(n{-}2) L + 2\big\lceil L/p\big\rceil+c-(n{-}1)\\
 &=(n{-}2)(L{-}1)+2\big\lceil L/p\big\rceil+c-1.\end{aligned}
\end{equation}
Here, we used $a=(2,...,2,p), L\in2\mathbb{Z}$ and for $p\nmid L$ that $\lceil L/p\rceil=\lfloor L/p\rfloor{+}1$ and $\dim\mathcal{N}^L=2n{-}3$, while for $p\mid L$ holds $\lceil L/p\rceil=\lfloor L/p\rfloor$ and $\dim\mathcal{N}^L=2n{-}1$. Thus
\[\mu(L\gamma_c)=
\begin{cases}
 (L{-}1)(n{-}2)+2\big\lceil \textstyle\frac{L}{p}\big\rceil-1 & \text{for }c=0\\
 \phantom{(-1} L\phantom{)}(n{-}2)+2\big\lceil \textstyle\frac{L+1}{p}\big\rceil-1 & \text{for } c=n{-}2 \text{ and }p\nmid L\\
 \phantom{(-1} L\phantom{)}(n{-}2)+2\big\lceil \textstyle\frac{L+1}{p}\big\rceil & \text{for } c=n{-}1 \text{ and }p\nmid L\\
 (L{+}1)(n{-}2)+2\big\lceil \textstyle\frac{L+2}{p}\big\rceil & \text{for } c=2n{-}3 \text{ and }p\nmid L, p\nmid L{+}1\\
 (L{+}1)(n{-}2)+2\big\lceil \textstyle\frac{L+2}{p}\big\rceil-2 & \text{for } c=2n{-}3 \text{ and }p\nmid L, p\mid L{+}1\\
 (L{+}1)(n{-}2)+2\big\lceil \textstyle\frac{L+2}{p}\big\rceil & \text{for } c=2n{-}1 \text{ and }p\mid L
\end{cases}\]
\begin{defn}~\\
 Let $f_p:\mathbb{Z}\rightarrow\mathbb{Z}$ be the strictly increasing function $f_p(l)=(l{-}1)(n{-}2)+2\big\lceil l/p\big\rceil.$
\end{defn}
Recall that $b_k(\xi)$ for a fixed contact form can be chosen as the number of critical points of Morse functions on the manifolds $\mathcal{N}^\eta$ with Morse-Bott index $k$ (see \ref{secAsmFini}). From the above calculations, we immediately see:
\begin{prop}\label{propgeneratorssigmap}
 $b_k(\xi_p) = 0$, except
 \begin{align*}
 &b_k(\xi_p) = 1&&\text{ if }\quad k=f_p(l)+\left\lbrace\begin{smallmatrix}-1\\\phantom{-}0\end{smallmatrix}\right.&&\text{ for } l\in\phantom{2}\mathbb{N} \text{ and }p\nmid l{-}1,\\
 &b_k(\xi_p) = 1&&\text{ if }\quad k=f_p(l)+\left\lbrace\begin{smallmatrix}-2\\-1\end{smallmatrix}\right.&&\text{ for } l\in2\mathbb{N} \text{ and } p\mid l{-}1.
\end{align*}
\end{prop}
\begin{theo}\label{theoSHofSigmap}
 For $n\geq 5$ and $k\geq n$ holds that $SH_k(V_p,\Sigma_p) = 0$, except
 \begin{align*}
 &\rk SH_k(V_p,\Sigma_p) = 1&&\text{ if }\quad k=f_p(l)+\left\lbrace\begin{smallmatrix}-1\\\phantom{-}0\end{smallmatrix}\right.&&\text{ for } l\in\phantom{2}\mathbb{N} \text{ and }p\nmid l{-}1,\\
 &\rk SH_k(V_p,\Sigma_p) = \;?&&\text{ if }\quad k=f_p(l)+\left\lbrace\begin{smallmatrix}-2\\-1\end{smallmatrix}\right.&&\text{ for } l\in2\mathbb{N} \text{ and } p\mid l{-}1.
\end{align*}
The ranks at the place of ? are not known, but are either 0 or 1.
\end{theo}
\begin{proof}
 In \cite{Fauck1}, it is shown that the Rabinowitz-Floer homology groups $RFH_k(V_p,\Sigma_p)$ are $\mathbb{Z}_2$, ? or $0$ depending on $k$ in exactly the same way as in this theorem (but for all $k$, not only $k\geq n$). Note that in \cite{Fauck1} a different index convention is used, so that all indices there are shifted by $(n{-}2)+1$ in comparison to our indices here.\\
 The Rabinowitz-Floer homology is, apart from the constant orbits, also generated by the $L\gamma_c$. It was shown via an action estimate that the boundary operator vanishes for almost all indices $k$ (the only exceptions being whose where the ? appears). The same argument can be used to show that this holds also true for Symplectic homology. Alternatively, one can use the following long exact sequence from \cite{FraCieOan}, (3), relating singular, symplectic and Rabinowitz-Floer homology
 \begin{equation}\label{longexseqRFHSH}
  H^{-k+n}(V_p,\Sigma_p)\rightarrow SH_k(V_p,\Sigma_p)\rightarrow RFH_k^{\geq 0}(V_p,\Sigma_p)\rightarrow H^{-k+n+1}(V_p,\Sigma_p)
 \end{equation}
 Note that on $\Sigma_p$ the Morse-Bott index $\mu$ is positive only if the period of the closed Reeb orbit is non-negative. Hence we have for $k>0$ that $RFH^{\geq 0}_k(V_p,\Sigma_p)=RFH_k(V_p,\Sigma_p)$. As furthermore both singular homology groups vanish for $k\geq n$, we find that (\ref{longexseqRFHSH}) reduces to an isomorphism between $SH_k(V_p,\Sigma_p)$ and $RFH_k(V_p,\Sigma_p)$ for $k\geq n$. Hence, the theorem is obtained from the similar result for Rabinowitz-Floer homology.
\end{proof}
\begin{theo}[Ustilovsky, \cite{Usti}]~\\
 The contact structures $\xi_p$ on $S^{4m+1}$ are pairwise non-contactomorphic.
\end{theo}
\begin{proof}
In \cite{Fauck1} and \cite{FauckThesis}, it is also shown that $RFH_k(V_p,\Sigma_p)$ is independent of the filling. This result was then used to show that the contact structures $\xi_p$ on $S^{4m+1}$ are all different for $m>1$. Here is a slightly different proof for this crucial fact: For $p<q$ consider the index
\begin{align*}
 k=(p{+}1)(n{-}2)+2&=(p{+}1)(n{-}2)+2\left\lceil{\textstyle\frac{p+2}{p}}\right\rceil{-}2 = f_p(p{+}2)-2\\
 &=(p{+}1)(n{-}2)+2\left\lceil{\textstyle\frac{p+2}{q}}\right\rceil \phantom{-2}= f_q(p{+}2).
\end{align*}
As $k\geq n$, we know by Theorem \ref{theoSHofSigmap} for the filling $V_q$ of $(\Sigma_q,\alpha_q)$ that $SH_k(V_q,\Sigma_p)=\mathbb{Z}_2$. On the other hand, there is by Proposition \ref{propgeneratorssigmap} on $(\Sigma_p,\alpha_p)$ no closed Reeb orbit having index $k$. Hence it follows for every filling $V$ of $\Sigma_p$ that $SH_k(V,\Sigma_p)=0$, i.e. $V_q$ is not a filling of $(\Sigma_p, \xi_p)$. Therefore $(\Sigma_p,\xi_p)$ and $(\Sigma_q,\xi_q)$ cannot be contactomorphic.
\end{proof}

\subsection{Connected sums of Brieskorn spheres}\label{secConSum}
We saw that the Reeb flow $\varphi_p^t$ of $\alpha_p$ on $\Sigma_p$ is totally periodic. Unfortunately, the contact surgery construction needs at least one point that does not lie on a closed Reeb orbit. For this purpose, we will perturb $\alpha_p$ to a new contact form $\alpha_p'$ defining the same contact structure $\xi_p$. We will use a perturbation due to Uebele, \cite{Ueb}, which is similar to the one used by Ustilovsky in \cite{Usti}. Note that this is a different perturbation from the one described in the appedix. The advantage of this perturbation here is that the resulting bounds $b_k(\xi_p')$ on the number of generators for symplectic homology are smaller. First, we make the following change of coordinates
\begin{equation}\label{UstiPer}
\begin{pmatrix}w_0\\w_1\end{pmatrix}=\frac{1}{\sqrt{2}}\begin{pmatrix}1 & i\\1& -i\end{pmatrix}\begin{pmatrix}z_0\\z_1\end{pmatrix},\quad w_2=z_2,\quad\dots\quad,\quad w_n=z_n.
\end{equation}
In these coordinates $\displaystyle\quad\Sigma_p=\Big\{w\in\mathbb{C}^{n+1}\;\Big|\; 2w_0w_1+w_2^2+...+w_{n-1}^2+w_n^p=0,\; ||w||^2=1\Big\}.$\bigskip\\
Next we introduce the new contact form $\alpha_p':=K^{-1}{\cdot} \alpha_p$, where
\[K(w):=||w||^2+\veps\big(|w_0|^2-|w_1|^2\big)\]
and $\veps>0$ is a sufficiently small irrational number. As $\alpha_p'$ is obtained from $\alpha_p$ by multiplication with a positive function, they define the same contact structure. In fact, $\alpha_p'$ should be though of as the restriction of $\lambda_p$  to the hypersurface $\Sigma_p'\subset \mathbb{R}{\times}\Sigma_p$ inside $\widehat{V_p}$ defined by
\[\Sigma_p':=\Big\{\big({-}\log K(y),y\big)\;\Big|\;y\in\Sigma_p\Big\}.\]
Ustilovsky shows in \cite{Usti}, Lemma 4.1, that the Reeb vector field of $\alpha_p'$ is
\[R_p'=\Big(\;2i(1{+}\veps)w_0, 2i(1{-}\veps)w_1,2iw_2,\,...,\,2iw_{n-1},\frac{4i}{p}w_n\Big).\]
Hence, the closed Reeb orbits form 4 different types of manifolds parametrized by $L\in\mathbb{N}$:
\begin{itemize}
 \item $\widetilde{\mathcal{N}}^L=\Big\{ w\in \Sigma_p\,\Big|\,w_0{=}w_1{=}w_n{=}0\Big\}\cong S^\ast S^{n-3}$ of $\frac{L\pi}{2}$-periodic orbits for $p\nmid L$
 \item $\widetilde{\mathcal{N}}^L=\Big\{ w\in \Sigma_p\,\Big|\,w_0{=}w_1{=}0\Big\}\cong S^\ast S^{2n-5}$ of $\frac{L\pi}{2}$-periodic orbits for $p\mid L$
 \item $\widetilde{\mathcal{N}}^L_+=\Big\{ (w_0,0,...,0)\in\Sigma_p\Big\}\cong S^1$ of $\frac{L\pi}{2(1+\veps)}$-periodic orbits
 \item $\widetilde{\mathcal{N}}^L_-=\Big\{(0,w_1,0,...,0)\in\Sigma_p\Big\}\cong S^1$ of $\frac{L\pi}{2(1-\veps)}$-periodic orbits.
\end{itemize}
Note that the flow of $R_p'$ is no longer totally periodic. In particular points $w\in \Sigma_p$ with $w_0, w_1, w_2$ all non-zero do not lie on any closed Reeb orbit.\\
For the computation of the Conley-Zehnder indices of the closed orbits, we can still use the symplectic form $\omega_p$ instead of $d(K^{-1}\lambda_p)$, as $\omega_p|_{\xi_p}=K{\cdot} d(K^{-1}\lambda_p)|_{\xi_p}$. Hence, the same arguments as for (\ref{eqConZeh}) yield that for any Reeb trajectory $v$ of $\alpha'_p$ of length $L\pi/2$ holds:
\begin{equation}\label{muCZperturbed}
\begin{aligned}
 \mu_{CZ}(v)=&\phantom{\,+\,}\left(\left\lfloor{\textstyle\frac{L(1+\veps)}{2}}\right\rfloor\mspace{-5mu}+\mspace{-5mu}\left\lceil{\textstyle\frac{L(1+\veps)}{2}}\right\rceil\right)+\left(\left\lfloor{\textstyle\frac{L(1-\veps)}{2}}\right\rfloor\mspace{-5mu}+\mspace{-5mu}\left\lceil{\textstyle\frac{L(1-\veps)}{2}}\right\rceil\right)+{\textstyle\sum_{k=2}^{n-1}}\Big(\left\lfloor{\textstyle\frac{L}{2}}\right\rfloor\mspace{-5mu}+\mspace{-5mu}\left\lceil{\textstyle\frac{L}{2}}\right\rceil\Big)\\
 &+\left\lfloor L/p\right\rfloor\mspace{-5mu}+\mspace{-5mu}\left\lceil L/p\right\rceil-\left(\left\lfloor L\right\rfloor\mspace{-5mu}+\mspace{-5mu}\left\lceil L\right\rceil\right).\end{aligned}
\end{equation}
We can choose Morse-functions on $\widetilde{\mathcal{N}}^L, \widetilde{\mathcal{N}}^L_\pm$ having critical points $\gamma_c, \gamma_c^\pm$ with indices $0, n{-}4, n{-}3, 2n{-}7$ for $p\nmid L$ or $0, 2n{-}5$ for $p\mid L$ on $\widetilde{N}^L$ or $0,1$ on $\widetilde{\mathcal{N}}^L_\pm$. For $\veps$ fixed and $L$ not too large such that $L\veps\leq 1$, we can easily estimate the Gau{\ss} brackets in (\ref{muCZperturbed}) to get for the Morse-Bott index:
\begin{align*}
\shortintertext{$\bullet$ on $\widetilde{\mathcal{N}}^L_+$ for period $\frac{L}{1+\veps}{\cdot} \frac{\pi}{2}, L\in2\mathbb{N}$ and $c\in\{0,1\}$}
  \mu(L\gamma_c^+)&=\textstyle\left(\left\lfloor\frac{L(1+\veps)}{2(1+\veps)}\right\rfloor\mspace{-5mu}+\mspace{-5mu}\left\lceil\frac{L(1+\veps)}{2(1+\veps)}\right\rceil\right)+\left(\left\lfloor\frac{L(1-\veps)}{2(1+\veps)}\right\rfloor\mspace{-5mu}+\mspace{-5mu}\left\lceil\frac{L(1-\veps)}{2(1+\veps)}\right\rceil\right)+{ \sum_{k=2}^{n-1}}\left(\left\lfloor\frac{L}{2(1+\veps)}\right\rfloor\mspace{-5mu}+\mspace{-5mu}\left\lceil\frac{L}{2(1+\veps)}\right\rceil\right)\\
 &\phantom{\,+\,}\textstyle +\left\lfloor\frac{L}{p(1+\veps)}\right\rfloor\mspace{-5mu}+\mspace{-5mu}\left\lceil\frac{L}{p(1+\veps)}\right\rceil-\left(\left\lfloor \frac{L}{(1+\veps)}\right\rfloor\mspace{-5mu}+\mspace{-5mu}\left\lceil \frac{L}{(1+\veps)}\right\rceil\right)-\frac{1}{2}\dim\widetilde{\mathcal{N}}^L_++\frac{1}{2}+c\\
 &= \textstyle 2 \frac{L}{2}+\left(\frac{L}{2}{-}1{+}\frac{L}{2}\right)+(n{-}2)\left(\frac{L}{2}{-}1{+}\frac{L}{2}\right)+2\left\lceil\frac{L}{p}\right\rceil -1 -(L{-}1{+}L)+c\\
 &=(n{-}2)(L{-}1)+\textstyle 2\left\lceil\frac{L}{p}\right\rceil + \left\lbrace\begin{smallmatrix}-1\\\phantom{-}0\end{smallmatrix}\right.\allowdisplaybreaks
 \shortintertext{$\bullet$ on $\widetilde{\mathcal{N}}^L$ for period $L{\cdot}\frac{\pi}{2},\; L\in2\mathbb{N},\;p\nmid L$ and $c\in\{0,n{-}4,n{-}3, 2n{-}7\}$}
 \mu(L\gamma_c) &=\textstyle\left(\left\lfloor\frac{L(1+\veps)}{2}\right\rfloor\mspace{-5mu}+\mspace{-5mu}\left\lceil\frac{L(1+\veps)}{2}\right\rceil\right)+\left(\left\lfloor\frac{L(1-\veps)}{2}\right\rfloor\mspace{-5mu}+\mspace{-5mu}\left\lceil\frac{L(1-\veps)}{2}\right\rceil\right)+{\displaystyle \sum_{k=2}^{n-1}}\left(\left\lfloor\frac{L}{2}\right\rfloor\mspace{-5mu}+\mspace{-5mu}\left\lceil\frac{L}{2}\right\rceil\right)\\
 &\phantom{\,+\,}\textstyle +\left\lfloor\frac{L}{p}\right\rfloor\mspace{-5mu}+\mspace{-5mu}\left\lceil\frac{L}{p}\right\rceil-\left(\left\lfloor L\right\rfloor\mspace{-5mu}+\mspace{-5mu}\left\lceil L\right\rceil\right)-\frac{1}{2}\dim\widetilde{\mathcal{N}}^L+\frac{1}{2}+c\\
 &= \textstyle \left(\frac{L}{2}{+}\frac{L}{2}{+}1\right)+\left(\frac{L}{2}{-}1{+}\frac{L}{2}\right)+(n{-}2)2\frac{L}{2}+2\left\lceil\frac{L}{p}\right\rceil -1 -2L-(n{-}4)+c\\
 &=\begin{cases}
    (n{-}2)(L{-}1)+\textstyle 2\left\lceil\frac{L}{p}\right\rceil + 1 & c=0\\
    \bigg.(n{-}2)\phantom{(}L\phantom{-1)}+\textstyle 2\left\lceil\frac{L+1}{p}\right\rceil + \left\lbrace\begin{smallmatrix}-1\\\phantom{-}0\end{smallmatrix}\right. & c=n{-}4, n{-}3, \text{ as }p\nmid L\\
    (n{-}2)(L{+}1)+\textstyle 2\left\lceil\frac{L+2}{p}\right\rceil -2 & c=2n{-}7 \text{ if }p\nmid L{+}1\\
    \bigg.(n{-}2)(L{+}1)+\textstyle 2\left\lceil\frac{L+2}{p}\right\rceil -4 & c=2n{-}7 \text{ if }p\mid L{+}1
   \end{cases}
   \shortintertext{$\bullet$ on $\widetilde{\mathcal{N}}^L$ for period $L{\cdot}\frac{\pi}{2},\; L\in2\mathbb{N},\;p\mid L$ and $c\in\{0, 2n{-}5\}$}
 \mu(L\gamma_c) &= \textstyle \left(\frac{L}{2}{+}\frac{L}{2}{+}1\right)+\left(\frac{L}{2}{-}1{+}\frac{L}{2}\right)+(n{-}2)2\frac{L}{2}+2\left\lceil\frac{L}{p}\right\rceil -2L-(n{-}3)+c\\
 &=\begin{cases}
    \bigg.(n{-}2)(L{-}1)+\textstyle 2\left\lceil\frac{L}{p}\right\rceil + 1 & c=0\\
    (n{-}2)(L{+}1)+\textstyle 2\left\lceil\frac{L+2}{p}\right\rceil -2 & c=2n{-}5 \text{ as }p\mid L
   \end{cases}
   \shortintertext{$\bullet$ on $\widetilde{\mathcal{N}}^{L}_-$ for period $\frac{L}{(1-\veps)}{\cdot}\frac{\pi}{2}, L\in2\mathbb{N}$ and $c\in\{0,1\}$}
 \mu\Big(L\gamma_c^-\Big)&=\textstyle\left(\left\lfloor\frac{L(1+\veps)}{2(1-\veps)}\right\rfloor\mspace{-5mu}+\mspace{-5mu}\left\lceil\frac{L(1+\veps)}{2(1-\veps)}\right\rceil\right)+\left(\left\lfloor\frac{L(1-\veps)}{2(1-\veps)}\right\rfloor\mspace{-5mu}+\mspace{-5mu}\left\lceil\frac{L(1-\veps)}{2(1-\veps)}\right\rceil\right)+{\displaystyle \sum_{k=2}^{n-1}}\textstyle \left(\left\lfloor\frac{L}{2(1-\veps)}\right\rfloor\mspace{-5mu}+\mspace{-5mu}\left\lceil\frac{L}{2(1-\veps)}\right\rceil\right)\\
 &\phantom{\,+\,}\textstyle+\left\lfloor\frac{L}{p(1-\veps)}\right\rfloor\mspace{-5mu}+\mspace{-5mu}\left\lceil\frac{L}{p(1-\veps)}\right\rceil -\left(\left\lfloor \frac{L}{(1-\veps)}\right\rfloor\mspace{-5mu}+\mspace{-5mu}\left\lceil \frac{L}{(1-\veps)}\right\rceil\right) -\frac{1}{2}\dim\widetilde{\mathcal{N}}^L_-+\frac{1}{2}+c\\
 &= \textstyle \left(\frac{L}{2}{+}\frac{L}{2}{+}1\right)+2\frac{L}{2}+(n{-}2)\left(\frac{L}{2}{+}\frac{L}{2}{+}1\right)+\textstyle 2\left\lceil\frac{L}{p(1-\veps)}\right\rceil -1-(L{+}L{+}1)+c\\
 &=\begin{cases}(n{-}2)(L{+}1)+2\left\lceil\frac{L+2}{p}\right\rceil + \left\lbrace\begin{smallmatrix}-1\\\phantom{\,-}0\end{smallmatrix}\right. &\text{if } p\nmid L{+}1\\
 \bigg.(n{-}2)(L{+}1)+2\left\lceil\frac{L+2}{p}\right\rceil-2 + \left\lbrace\begin{smallmatrix}-1\\\phantom{\,-}0\end{smallmatrix}\right. &\text{if } p\mid L{+}1\end{cases}.
\end{align*}
From the above calculations, we can directly read off the number $b_k(\xi_p')$ of closed Reeb orbits of $\alpha_p'$ having Morse-Bott index as:
\begin{align*}
 &b_k(\xi_p') = 2 && \text{if } k =f_p(l)+ \left\lbrace\begin{smallmatrix}-1\\\phantom{-}0\end{smallmatrix}\right. &&\text{for } p\nmid l{-}1 \text{ and } 2\mid l\\
 &b_k(\xi_p') = 1 && \text{if } k =f_p(l)+ \left\lbrace\begin{smallmatrix}-2\\\phantom{-}1\end{smallmatrix}\right. &&\text{for } p\nmid l{-}1 \text{ and } 2\mid l\\
 &b_k(\xi_p') = 1 && \text{if } k =f_p(l)+ \left\lbrace\begin{smallmatrix}-1\\\phantom{-}0\end{smallmatrix}\right. &&\text{for } p\nmid l{-}1 \text{ and } 2\nmid l\\
 &b_k(\xi_p') = 1 && \text{if } k =f_p(l)+ \left\lbrace\begin{smallmatrix} -4\\-3\\-2\\-1\end{smallmatrix}\right. &&\text{for } p\mid l{-}1 \text{ and } 2\mid l\\
 &b_k(\xi_p') = 0 && \text{otherwise}.
\end{align*}
We see that non-zero entries for $b_k(\xi_p')$ come in blocks indexed by $f_p(l)$, which we may visualize as follows. For $l$ even and $p\nmid l, l{+}1, l{+}2$ we have:
\[
 \begin{array}{c|cccc||cc||cccc||cc|c}
  \scriptstyle \text{index } k  & \scriptstyle -2 &\scriptstyle -1 & \scriptstyle f_p(l) & \scriptstyle  +1  & \scriptstyle -1 & \scriptstyle f_p(l+1)  & \scriptstyle -2 & \scriptstyle -1 & \scriptstyle f_p(l+2) & \scriptstyle +1 & \scriptstyle -1 & \scriptstyle f_p(l+3) & \scriptstyle ... \\\hline
  \Big.b_k(\xi_p') & 1 & 2 & 2 & 1 & 1 & 1 & 1 & 2 & 2 & 1 & 1 & 1 &... 
 \end{array}
 \]
 Note that between these blocks there may be several 0 entries (if $n{-}2>3$) or the blocks may be directly adjacent (if $n{-}2=3$). For $l$ even and $p\mid l{+}1$, the situation looks slightly different:
\[
 \begin{array}{c|cccc||cc||cccccc||cc}
  \scriptstyle \text{index } k  & \scriptstyle -2 &\scriptstyle -1 & \scriptstyle f_p(l) & \scriptstyle  +1  & \scriptstyle -1 & \scriptstyle f_p(l+1)  & \scriptstyle -4 & \scriptstyle -3 & \scriptstyle -2 & \scriptstyle -1 & \scriptstyle f_p(l+2) & \scriptstyle +1 & \scriptstyle -1 & \scriptstyle f_p(l+3)  \\\hline
  \Big.b_k(\xi_p') & 1 & 2 & 2 & 1 & 1 & 1 & 1 & 1 & 1 & 1 & 1 & 1 & 1 & 1  
 \end{array}
 \]
Note that the blocks do not overlap, as $f_p(l{+}1)=l(n{-}2)+2\lceil (l{+}1)/p\rceil<(l{+}1)(n{-}2)+2\lceil(l{+}2)/p\rceil-4$, as $p\mid l{+}1$ and $n{-}2\geq 3$. Finally, for $l$ even and $p\mid l$, there is a third situation:
\begin{equation}\label{indextable}
 \begin{array}{c|cccc||c||cccc||cc|c}
  \scriptstyle \text{index } k  & \scriptstyle -2 &\scriptstyle -1 & \scriptstyle f_p(l) & \scriptstyle  +1  &  & \scriptstyle -2 & \scriptstyle -1 & \scriptstyle f_p(l+2) & \scriptstyle +1 & \scriptstyle -1 & \scriptstyle f_p(l+3) & \scriptstyle ... \\\hline
  \Big.b_k(\xi_p') & 1 & 2 & 2 & 1 & 0 & 1 & 2 & 2 & 1 & 1 & 1 &... 
 \end{array}
 \end{equation}
 Here, the $0$-block in between contains at least four $0$ entries as
 \begin{equation}\begin{aligned}\label{indextableindex}
  f_p(l{+}2)-2&=(l{+}1)(n{-}2)+2\big\lceil(l{+}2)/p\big\rceil-2\\
  &\geq\phantom{+1))}l(n{-}2)+3+2\big\lceil(l{+}1)/p\big\rceil-2 &&= f_p(l{+}1)+1\\
  &\geq(l{-}1)(n{-}2)+6+2\big\lceil l/p\big\rceil\phantom{-2} &&= f_p(l)+6\end{aligned}
 \end{equation}
We remark that of course $SH_k(V_p,\Sigma_p)=SH_k(V_p,\Sigma_p')$. However, a comparison with Prop. \ref{propgeneratorssigmap} shows that after the perturbation of the contact form we find that $b_k(\xi_p')$ is bigger then $b_k(\xi_p')$. This means in particular that the Floer-boundary operator $\partial$ after perturbation is non-zero on more elements.\\
Now, let us for $J\in\mathbb{N}$ denote by $ \consumt_{j=1}^J \Sigma_p$ the $J$-fold connected sum of the Brieskorn manifold $\Sigma_p$. As $\Sigma_p$ is homeomorphic to $S^{2n-1}$, it follows that $ \consumt_{j=1}^J \Sigma_p$ is also homeomorphic to $S^{2n-1}$ and for $p\equiv \pm 1\mod 8$ both homeomorphisms are even diffeomorphisms. Let us denote by $\consumt_{j=1}^J \xi_p$ the corresponding contact structure on $S^{2n-1}$.
\begin{theo}\label{differentcontacttheo}
 Let $n\geq 5$ and $I,J,p,q\in\mathbb{N}$ be arbitrary with $p,q\geq 3$ odd. If $(J,p)\neq (I,q)$, then it holds that $\consumt_{j=1}^J \xi_p$ and $\consumt_{i=1}^I \xi_q$ are non-contactomorphic contact structures on $S^{2n-1}$.
\end{theo}
\begin{proof}
 From the Invariance Theorem (Theorem \ref{theoinvsur}), we obtain that
 \begin{align*}
   SH_k\Big(\consumd_{j=1}^J V_p,\consumd_{j=1}^J \Sigma_p\Big)= \bigoplus_{j=1}^J SH_k(V_p,\Sigma_p).\tag{$\ast$}
 \end{align*}
 Moreover, $\xi_p$ is asymptotically finitely generated in every degree $k$ with bound $b_k(\xi_p')$ for a contact form $\alpha_p'$ having a not totally periodic Reeb flow. With Prop.\ \ref{propcofinalHamilt}, it follows that $\consumt_{j=1}^J\xi_p$ is also a.f.g. in every degree with bounds $b_k\big(\consumt_{j=1}^J\xi_p\big)=J{\cdot}b_k(\xi_p')$ or $b_k\big(\consumt_{j=1}^J\xi_p\big)=J{\cdot}b_k(\xi_p')+(J{-}1)$ if $k$ is of the form
 \[(2l{-}1)(n{-}1)+\left\lbrace\begin{smallmatrix}0\\1\end{smallmatrix}\right.,\qquad l\in\mathbb{N}.\tag{$\ast\ast$}\]
 If $k$ is not of the form $(\ast\ast)$ form, we have therefore by Prop.\ \ref{PropAsympfiniteGene} for any filling $V$ of $(S^{2n-1}, \consumt_{j=1}^J\xi_p)$ the estimate
 \[\rk SH_k\Big(V,\consumd_{j=1}^J\Sigma_p\Big)\leq J\cdot b_k(\xi_p').\tag{$\ast\ast\ast$}\]
 Now assume without loss of generality that $p<q$. We distinguish the three cases $q>2p{+}1,\quad q<2p{+}1\quad$ and $\quad q=2p{+}1$.\bigskip\\
 \underline{If $q>2p{+}1$}, then (since $q$ is odd) $q\geq2p{+}3$. Here, we look at the index
 \begin{align*}
  k=(2p{+}1)(n{-}2)+2&=(2p{+}1)(n{-}2)+2\left\lceil\frac{2p{+}2}{q}\right\rceil\phantom{-4} = f_q(2p{+}2)\\
  &= (2p{+}1)(n{-}2)+2\left\lceil\frac{2p{+}2}{p}\right\rceil{-}4=f_p(2p{+}2)-4.
 \end{align*}
 On one the hand, we know by the additivity of $SH$ (see ($\ast$)) and Theorem \ref{theoSHofSigmap} that
 \[\rk SH_k\Big(\consumd_{i=1}^I V_q,\consumd_{i=1}^I \Sigma_q\Big)=I\cdot \rk SH_k(V_q,\Sigma_q) = I.\]
 On the other hand, we find that $k$ is odd (as $n$ is odd). So if $k$ were of the form $(\ast\ast)$, then we would have for some $l\in\mathbb{N}$ that
 \begin{align*}
  &&(2p{+}1)(n{-}2)+2&=(2l{-}1)(n{-}1)+1 \\
  &\Leftrightarrow& 2p(n{-}2)+n&=(2l{-}2)(n{-}1)+n &&\Leftrightarrow& \frac{p(n{-}2)}{n{-}1}&=l{-}1.
 \end{align*}
This is impossible as $p$ and $n$ are odd. Hence, we
 know for any filling $V$ of $\consumt_{j=1}^J \Sigma_p'$ by (\ref{indextable}) and $(\ast\ast\ast)$ that
 \[\rk SH_k\Big(V,\consumd_{j=1}^J\Sigma_p\Big)\leq J\cdot b_k(\xi_p') =J\cdot 0=0\neq I =\rk SH_k\Big(\consumd_{i=1}^I V_q,\consumd_{i=1}^I \Sigma_q\Big).\]
 This implies that $ \consumt_{j=1}^J \xi_p$ and $ \consumt_{i=1}^I \xi_q$ cannot be contactomorphic.\bigskip\\
 \underline{If $q<2p{+}1$}, then $p<q\leq 2p{-}1$ and we look a the indices
 \begin{align*}
  && k=2p(n{-}2)+4&=2p(n{-}2)+2\left\lceil\frac{2p{+}1}{q}\right\rceil \phantom{-2}= f_q(2p{+}1)\\
  &&&=2p(n{-}2)+2\left\lceil\frac{2p{+}1}{p}\right\rceil {-}2=f_p(2p{+}1)-2\\
  &\text{and}& k=2p(n{-}2)+3&=f_q(2p{+}1)-1=f_p(2p{+}1)-3
 \end{align*}
In both cases, we know by the additivity of $SH$ and Theorem \ref{theoSHofSigmap} that
 \[\rk SH_k\Big(\consumd_{i=1}^I V_q,\consumd_{i=1}^I \Sigma_q\Big)=I\cdot \rk SH_k(V_q,\Sigma_q) = I.\]
 Now, the first value for $k$ is even, while the second one is odd. So if $k$ were of the form $(\ast\ast)$, then we would have for some $l\in\mathbb{N}$ that
 \begin{align*}
  &&2p(n{-}2)+4&=(2l{-}1)(n{-}1) \\
  &\Leftrightarrow& 2p(n{-}2)+4 &= 2l(n{-}1)-(n{-}1) &&\Leftrightarrow& p(n{-}2)+\frac{n{+}3}{2}&=l(n{-}1)\\
  &\text{or}&2p(n{-}2)+3&=(2l{-}1)(n{-}1)+1 \\
  &\Leftrightarrow& 2p(n{-}2)+2&=2l(n{-}1)-(n{-}1) &&\Leftrightarrow& p(n{-}2)+\frac{n{+}1}{2} &=l(n{-}1).
 \end{align*}
 The first case is impossible for $n\equiv 1\mod 4$, while the second is impossible for $n\equiv 3\mod 4$. So for $k$ appropriately chosen, we find again $\rk SH_k\left(V,\consumt_{j=1}^J\Sigma_p\right)=0$ for any filling $V$ of $\consumt_{j=1}^J \Sigma_p'$ by (\ref{indextable}) and ($\ast\ast\ast$). Hence $\consumt_{j=1}^J \xi_p$ and $\consumt_{i=1}^I \xi_q$ cannot be contactomorphic.\bigskip\\
 \underline{If $q=2p{+}1$}, we look at the index
 \begin{align*}
  k=(4p{+}1)(n{-}2)+4&=(2q{-}1)(n{-}2)+2\left\lceil\frac{2q}{q}\right\rceil\phantom{+-6}=f_q(2q)\\
  &=(4p{+}1)(n{-}2)+2\left\lceil\frac{4p{+}2}{p}\right\rceil{-}6=f_p(4p{+}2)-6.
 \end{align*}
 We find again that $\rk SH_k\Big(\consumt_{i=1}^I V_q, \consumt_{i=1}^I \Sigma_q\Big) = I.$ Moreover, $k$ is odd, so if $k$ were of the form $(\ast\ast)$ for some $l\in\mathbb{N}$, then we would have
 \begin{align*}
  && (4p{+}1)(n{-}2)+4 &= (2l{-}1)(n{-}1)+1 \\
  &\Leftrightarrow& 4p(n{-}2)+n+2 &=(2l{-}2)(n{-}1)+n &&\Leftrightarrow& \frac{2p(n{-}2)+1}{n{-}1} &= l{-}1,
 \end{align*}
 which is impossible, as the numerator is odd, while the denominator is even. Thus, for any filling $V$ of $\consumt_{j=1}^J \Sigma_p'$ follows $\rk SH_k\left(V,\consumt_{j=1}^J\Sigma_p\right)=0$ 
 and hence that $\consumt_{j=1}^J \xi_p$ and $\consumt_{i=1}^I \xi_q$ are not contactomorphic.
\end{proof}

\appendix
\section{Pertubing totally periodic Reeb flows}\label{appendixA}
The purpose of this appendix is to show that contact manifolds which admit a contact form with totally periodic Reeb flow are asymptotically finitely generated in every degree, in particular via contact forms what are not totally periodic. We need such contact forms for the estimate of the symplectic homology of conntected sums with these manifolds (see Prop.\ \ref{propcofinalHamilt}).
\subsection{Setup}
Let $(\Sigma,\alpha)$ be a compact $(2n{-}1)$-dimensional contact manifold with contact structure $\xi=\ker \alpha$ and Reeb vector field $R$ such that the flow $\varphi^t$ of $R$ is totally periodic, i.e. there exists a time $T_0>0$ with $\varphi^{T_0}=Id_{\Sigma}$. As $R(p)\neq 0$ for all $p\in\Sigma$, it follows that the length of a closed Reeb orbit is bounded from below. Furthermore, any length $T\leq T_0$ of a closed Reeb orbit must satisfy $T_0/T \in\mathbb{N}$. These two facts imply that there are only finitely many $0<T_1<...<T_N<T_0$ such that $T_j$ is the length of a closed Reeb orbit.\\
The set of diffeomorphisms $\{\varphi^{T_j}\,|\,0\leq j\leq N\}$ clearly carries the structure of a finite abelian group which acts on $\Sigma$. Let $Fix(\varphi^{T_j})=\{p{\in}\Sigma\,|\,\varphi^{T_j}(p){=}p\}$ denote the fixed point set of $\varphi^{T_j}$. For every finite group action on a manifold $\Sigma$ one can always find an invariant Riemannian metric $g$ on $\Sigma$. Using the exponential map of such a $g$, we can find an $\varphi^{T_j}$-invariant neighborhood $U$ around any $p\in Fix(\varphi^{T_j})$ such that the action of $\varphi^{T_j}$ on $U$ is conjugated to the action of $D_p\varphi^{T_j}$ on $T_p\Sigma$ restricted to a sufficiently small neighborhood $B$ of the origin. We find that $U\cap Fix(\varphi^{T_j})$ is conjugated to $B\cap Fix(D_p\varphi^{T_j})=B\cap \ker (D_p\varphi^{T_j}{-}\mathbbm{1})$. This shows that $Fix(\varphi^{T_j})$ is a submanifold of $\Sigma$ and that $T_pFix(\varphi^{T_j})=\ker (D_p\varphi^{T_j}{-}\mathbbm{1})$.\\
As the $Fix(\varphi^{T_j})=\mathcal{N}^{T_j}=\mathcal{N}^{T_j+k\cdot T_0}, k\in\mathbb{N}$, are exactly the sets formed by the periodic orbits of $R$, we find that every contact manifold $(\Sigma,\alpha)$ with totally periodic Reeb flow satisfies the Morse-Bott assumption (\ref{CondMB}). Moreover, $\xi$ is an asymptotically finitely generated contact structure in almost all degrees. Indeed, if the mean index $\Delta(\Sigma)$ of the principal orbit $\Sigma=Fix(\varphi^{T_0})=\mathcal{N}^{T_0}$ is non-zero, then we find by the iterations formula (\ref{eqIteration}) that the Conley-Zehnder index of the orbits $\mathcal{N}^{T_j+k\cdot T_0}$ growths approximately linear in $k$ with slope $\Delta(\Sigma)$. Hence there are only finitely many closed Reeb orbits having any given Morse-Bott index $k$. If however $\Delta(\Sigma)=0$, then the iterations formla implies that the Morse-Bott index of any closed Reeb orbit stays in the inervall $[-3(n{-}1,3(n{-}1)]$, so that no closed orbit has a Morse-Bott index $k$ with $|k|>3(n{-}1)$.
\subsection{An explicit perturbation}
The following construction was first described in \cite{Bour}, section 2. We present it here for readability and to add some details missing in the original argument.\\
In order to pertub $\alpha$, we first construct positive Morse functions $\bar{f}_1,..., \bar{f}_N,\bar{f}_0$ on the orbit spaces $Fix(\varphi^{T_j})\big/S^1$ which we denote by $Q_j$. Note that $Q_j$ is in general a symplectic orbifold as the Reeb flow does not act freely on $Fix(\varphi^{T_j})$ if there exists an $i<j$ such that $T_i|T_j$. We construct the $\bar{f}_j$ by the following inductive procedure:
\begin{enumerate}
 \item $Q_1$ is in fact a smooth manifold. Thus pick any positive Morse function $\bar{f}_1$ on $Q_1$.
 \item For $Q_j$, $j>1$, the singular set of $Q_j$ is exactly $S_j=\bigcup_{T_i|T_j} Q_i$. At first extend for every $i<j$ with $T_i|T_j$ the functions $\bar{f}_i$ to a function $\tilde{f}_j$ on a small tubular neighborhood of $S_j$ via the quadratic function $v\mapsto ||v||^2_g$ on the normal bundles to $Q_i$ in $Q_j$ (here, $g$ is a fixed invariant metric on $\Sigma$). Then extend $\tilde{f}_j$ to a positive Morse function $\bar{f}_j$ on $Q_j$. If $Q_j$ has no singular set, pick any positive Morse function.
 \item Repeate this procedure for all $j$, in particular up to $Q_0=Fix(\varphi^{T_0})\big/S^1=\Sigma\big/S^1$.
\end{enumerate}
Now lift $\bar{f}_0$ to $\Sigma$ to obtain a function $f$ which is invariant under the Reeb flow $\varphi^t$. As $R\in\ker df$ for any $\varphi^t$-invariant function $f$ and as $\xi=\ker \alpha$ and $\alpha\wedge(d\alpha)^{n-1}$ is non-degenerate, we can associate to any $\varphi^t$-invariant function a unique Hamiltonian vector field $X_f$ by the two conditions
\[X_f(p)\in\xi(p)\quad \forall p\in\Sigma\qquad\text{ and }\qquad d\alpha(\cdot, X_f)=df.\]
Now set $\alpha_\lambda:=(1{+}\lambda f){\cdot} \alpha$ as the perturbed contact form. Then, we find that the Reeb vector field $R_\lambda$ of $\alpha_\lambda$ is
\begin{align*}
R_\lambda&:=\frac{1}{1{+}\lambda f}\cdot R-\frac{\lambda}{(1{+}\lambda f)^2}\cdot X_{f}, \qquad\text{ since}\\
 \alpha_\lambda(R_\lambda) &= (1{+}\lambda f)\alpha\Big(\frac{1}{1{+}\lambda f}R-\frac{\lambda}{(1{+}\lambda f)^2}X_{f}\Big)=\frac{1{+}\lambda f}{1{+}\lambda f}\alpha(R)=1\allowdisplaybreaks\\
 d\alpha_\lambda (R_\lambda,\cdot) &=\big((1{+}\lambda f)d\alpha+\lambda df{\wedge}\alpha\big)\Big({\textstyle\frac{1}{1+\lambda f}}R-{\textstyle \frac{\lambda}{(1+\lambda f)^2}} X_{f} , \cdot\Big)\\
 &= d\alpha(R,\cdot)-{\textstyle \frac{\lambda}{1+\lambda f}} d\alpha(X_{f},\cdot)+{\textstyle \frac{\lambda}{1+\lambda f}}(df{\wedge}\alpha)(R,\cdot)-{\textstyle\frac{\lambda^2}{(1+\lambda f)^2}}(df{\wedge}\alpha)(X_{f},\cdot)\\
 &= 0 + {\textstyle \frac{\lambda}{1+\lambda f}}df-{\textstyle\frac{\lambda}{1+\lambda f}}df-0 =0,
\end{align*}
as $R\in\ker df$ and $X_{f}\in\xi=\ker \alpha$ and $df(X_{f})=d\alpha(X_{f},X_{f})=0$.\\ Finally, we calculate for the Lie bracket of $\frac{1}{1+\lambda f}R$ and $\frac{-\lambda}{(1+\lambda f)^2}X_{f}$ that
\begin{align*}
 \alpha\Big(\big[{\textstyle\frac{1}{1+\lambda f}}R, {\textstyle\frac{-\lambda}{(1+\lambda f)^2}} X_{f}\big]\Big) &=\mathcal{L}_{\frac{R}{1+\lambda f}} \underbrace{\alpha\big({\textstyle\frac{-\lambda}{(1+\lambda f)^2}} X_{f}\big)}_{=0}-\iota_{\frac{-\lambda X_{f}}{(1+\lambda f)^2}}\mathcal{L}_{\frac{R}{1+\lambda f}} \alpha\\
 &=-\iota_{\frac{-\lambda X_{f}}{(1+\lambda f)^2}}\Big(\underbrace{\iota_{\frac{R}{1+\lambda f}} d\alpha}_{=0}+d\Big(\underbrace{\alpha\big({\textstyle\frac{1}{1+\lambda f}}R\big)}_{=\frac{1}{1+\lambda f}}\Big)\Big)\\
 &={\textstyle \frac{-\lambda}{(1+\lambda f)^2}} df\Big({\textstyle \frac{-\lambda}{(1+\lambda f)^2}} X_{f}\Big)=0\allowdisplaybreaks\\
 d\alpha\Big(\big[{\textstyle\frac{1}{1+\lambda f}}R, {\textstyle\frac{-\lambda}{(1+\lambda f)^2}} X_{f}\big],\cdot\Big) &=\mathcal{L}_{\frac{R}{1+\lambda f}} d\alpha\big({\textstyle\frac{-\lambda}{(1+\lambda f)^2}} X_{f},\cdot\big)-\iota_{\frac{-\lambda X_{f}}{(1+\lambda f)^2}}\mathcal{L}_{\frac{R}{1+\lambda f}} d\alpha\\
 &=\mathcal{L}_{\frac{R}{1+\lambda f}} {\textstyle\frac{\lambda}{(1+\lambda f)^2}} df-\iota_{\frac{-\lambda X_{f}}{(1+\lambda f)^2}}\Big(\iota_{\frac{R}{1+\lambda f}}\underbrace{dd\alpha}_{=0}+d\big(\underbrace{d\alpha({\textstyle\frac{1}{1+\lambda}}R,\cdot)}_{=0}\big)\Big)\\
 &=\iota_{\frac{R}{1+\lambda f}}d\big({\textstyle\frac{\lambda}{(1+\lambda f)^2}}df\big)+d\Big({\textstyle\frac{\lambda}{(1+\lambda f)^2}} \underbrace{df\big({\textstyle\frac{1}{1+\lambda f}}R\big)}_{=0}\Big)\\
 &=\iota_{\frac{R}{1+\lambda f}}\Big({\textstyle\frac{\lambda}{(1+\lambda f)^2}}\underbrace{dd f}_{=0}-{\textstyle\frac{-2\lambda^2}{(1+\lambda f)^3}}\underbrace{df\wedge df}_{=0}\Big)=0
 \end{align*}
 \[\Rightarrow\;\;\big(\alpha\wedge(d\alpha)^{n-1}\big)\Big(\big[{\textstyle\frac{1}{1+\lambda f}}R, {\textstyle\frac{-\lambda}{(1+\lambda f)^2}} X_{f}\big],\cdot\Big) =0.\]
As $\alpha{\wedge}(d\alpha)^{n-1}$ is a volume form, we conclude that $\big[{\textstyle\frac{1}{1+\lambda f}}R, {\textstyle\frac{-\lambda}{(1+\lambda f)^2}} X_{f}\big]=0$ and hence that the flows of $\frac{1}{1+\lambda f}R$ and $\frac{-\lambda}{(1+\lambda f)^2}X_f$ commute. This implies that the flow of $R_\lambda$ is the composition of these two flows. Closed orbits of $R_\lambda$ are therefore compositions of closed orbits of $\frac{1}{1+\lambda f}R$ and $\frac{-\lambda}{(1+\lambda f)^2}X_f$, where both orbits have the same starting point and the same period.\\
In particular, we have closed orbits of $R_\lambda$ through any critical point $p$ of $f$. These correspond exactly to critical points of $\bar{f}_0$. As $\bar{f}_0$ is Morse, we find that the Hessian of $f$ in the normal direction to the closed $R$-orbit through $p$ is positive definite and hence that these closed orbits of $R_\lambda$ are non-degenerate.\\
On the other hand we can choose for any given $T>0$ the constant $\lambda$ so small that the only $T$-periodic orbits of $\frac{-\lambda}{(1+\lambda f)^2}X_f$ are the constant ones (see Prop.\ \ref{smallHamilt} below). This implies that for any given $T$ and $\lambda$ sufficiently small the only closed orbits of $R_\lambda$ with period less than $T$ are those over critical points of $\bar{f}_0$. As $\bar{f}_0$ is Morse on a compact set, we find that there are only finitely many of them. In \cite{Bour}, Lem.\ 2.4, the Conley-Zehnder index of these orbits through $p\in Fix(\varphi^{k\cdot T_j})$ was calculated as 
\[\quad\mu_{CZ}\big(Fix(\varphi^{k\cdot T_j})\Big)-{\textstyle \frac{1}{2}}\dim\big(Fix(\varphi^{k\cdot T_j})\big/ S^1\big)+\mu_{Morse} \big(\bar{f}_{k\cdot T_j}([p])\big).\]
Here, $\bar{f}_{k\cdot T_j}=\bar{f}_i$ for $k T_j\equiv T_i \mspace{-7mu}\mod T_0$. If the mean index $\Delta(\Sigma)$ of the principal orbit is zero, then this implies that all Conley-Zehnder indices of closed orbits of $R_\lambda$ shorter then $T$ stay in the interval $[-3(n{-}1),3(n{-}1)]$. If $\Delta(\Sigma)$ is non-zero, then only finitely many of these orbits have degree $k$  for any given integer $k$.\\
Hence, we can choose sequences $T_l\rightarrow\infty$ and $\lambda_l\rightarrow 0$ such that $(\alpha_{\lambda_l})$ shows that $\xi=\ker \alpha$ is asymptotically finitely generated in degree $k$, for any $k$ if $\Delta(\Sigma)\neq 0$ or for $|k|>3(n{-}1)$ if $\Delta(\Sigma)=0$. Moreover, the Reeb flow for any $\alpha_{\lambda_l}$ is not totally periodic.\bigskip\\
To finish the argument, we only need (as mentioned) the following contact version of the well-known fact that for any $C^2$-small Hamiltonian the only 1-periodic orbits of $X_H$ are the constant ones at critical points of $H$.
\begin{prop}\label{smallHamilt}
 Let $(\Sigma,\alpha)$ be a compact contact manifold with Reeb vector field $R$ and let $H:\Sigma\rightarrow\mathbb{R}$ be a smooth function which is invariant under the flow of $R$. Then there exists a constant $T>0$ such that each closed orbit of the Hamiltonian vector field $X_H$ (as defined above) with period less than $T$ is a constant orbit at a critical point of $H$.
\end{prop}
\begin{proof}
 Assume that there exists a sequence of closed $X_H$ orbits $(\gamma_l)$ with periods $T_l, \; \lim_{l\rightarrow\infty} T_l =0$. As $\Sigma$ is compact, we may assume by the Arzela-Ascoli Theorem that $(\gamma_l$) converges uniformly to a closed $X_H$-orbit $\gamma$ with period $T=0$. Hence, $\gamma$ is a constant orbit at a critical point $p$ of $H$, as $d\alpha(\cdot, X_H)=dH$.\\
 Around $p$ we can find by the contact version of Darboux's Theorem (see \cite{Geiges}, Thm. 2.5.1) a neighborhood $U$ and coordinates $x_1,...,x_{n-1},y_1,...,y_{n-1},z$ such that $p=(0,...,0)$ and
 \[\alpha|_U = dz+\sum_{j=1}^{n-1} x_jdy_j.\]
 Note that $\partial_z$ is the Reeb vector field in these coordinates. Without loss of generality we may assume that $U=(-\veps,\veps)^{2n-1}$ for some small $\veps>0$. The quotient $Q$ of $U$ under the flow of the Reeb vector field $\partial_z$ is then easily identified with $(-\veps,\veps)^{2n-2}$ with the coordinates $x_j,y_j, j=1,...,n{-}1$. On $Q$ we have the symplectic form $\bar{\omega}=\sum_j dx_j{\wedge} dy_j$. If $\pi:U\rightarrow Q$ denotes the quotient map, then it is easy to see that $\pi^\ast \bar{\omega}=d\alpha$. Moreover, as $H$ is invariant under the flow of $\partial_z$, we find that it descends to a function $\bar{H}$ on $Q$ with $\bar{H}\circ \pi= H$. Consequently, we find that the Hamiltonian vector fields satisfy $d\pi(X_H)=X_{\bar{H}}$. Thus, any closed $X_H$-orbit $\gamma$ in $U$ yields a closed $X_{\bar{H}}$-orbit $\bar{\gamma}=\pi\circ \gamma$ in $Q$ of the same period.\\
 Now recall that on $\mathbb{R}^{2n-2}$ with the symplectic form $\sum_j dx_j{\wedge} dy_j$ it is a well-known fact that there exists for a $C^2$-bounded autonomous Hamiltonian $\bar{H}$ a constant $T$ such that any closed $X_{\bar{H}}$-orbit of period less then $T$ is constant (see \cite{Laud}, Lemma 2.2). Thus almost all of the $\gamma_l$ are constant. Moreover, since $\Sigma$ is compact, we can cover the critical set of $H$ by a finite number of Darboux charts and $H$ is $C^2$-bounded in all of them. Thus we can find a global constant $T$ such that every closed $X_H$-orbit of period less than $T$ is constant.
\end{proof}

\bibliographystyle{plain}
\bibliography{../References/References.bib}

\end{document} 