% !TeX spellcheck = en_US

\documentclass[a4paper]{amsart}
\usepackage{amsmath,amssymb,enumerate}

\title[Addendum to: ``The Bolzano-Weierstrass Theorem ...'']{Addendum to:\\ ``The Bolzano-Weierstrass Theorem\\ is the Jump of Weak K\H{o}nig's Lemma''}

\author[V.\ Brattka]{Vasco Brattka}
\author[A.\ Cettolo]{Andrea Cettolo}
\author[G.\ Gherardi]{Guido Gherardi}
\author[A.\ Marcone]{Alberto Marcone}
\author[M.\ Schr\"oder]{Matthias Schr\"oder}

\address{Department of Mathematics \& Applied Mathematics\\
University of Cape Town, South Africa and Faculty of Computer Science, Universit\"at der Bundeswehr M\"unchen, Germany}
\address{Dipartimento di Scienze Matematiche, Informatiche e Fisiche\\ Universit\`a di Udine\\ Italy}
\address{Dipartimento di Filosofia e Comunicazione\\ Universit\`a di Bologna\\ Italy}
\address{Dipartimento di Scienze Matematiche, Informatiche e Fisiche\\ Universit\`a di Udine\\ Italy}
\address{Fachbereich Mathematik, Universit\"{a}t Darmstadt, Germany}

\email{Vasco.Brattka@cca-net.de}
\email{Andrea.Cettolo@spes.uniud.it}
\email{Guido.Gherardi@unibo.it}
\email{Alberto.Marcone@uniud.it}
\email{Matthias.Schroeder@cca-net.de}

% Abbreviations

% Calligraphical font

\def\AA{{\mathcal A}}
\def\BB{{\mathcal B}}
\def\CC{{\mathcal C}}
\def\DD{{\mathcal D}}
\def\EE{{\mathcal E}}
\def\FF{{\mathcal F}}
\def\GG{{\mathcal G}}
\def\HH{{\mathcal H}}
\def\II{{\mathcal I}}
\def\JJ{{\mathcal J}}
\def\KK{{\mathcal K}}
\def\LL{{\mathcal L}}
\def\MM{{\mathcal M}}
\def\NN{{\mathcal N}}
\def\OO{{\mathcal O}}
\def\PP{{\mathcal P}}
\def\QQ{{\mathcal Q}}
\def\RR{{\mathcal R}}
\def\SS{{\mathcal S}}
\def\TT{{\mathcal T}}
\def\UU{{\mathcal U}}
\def\VV{{\mathcal V}}
\def\WW{{\mathcal W}}
\def\XX{{\mathcal X}}
\def\YY{{\mathcal Y}}
\def\ZZ{{\mathcal Z}}

% Bold font

\def\bA{{\bf A}}
\def\bB{{\bf B}}
\def\bC{{\bf C}}
\def\bD{{\bf D}}
\def\bE{{\bf E}}
\def\bF{{\bf F}}
\def\bG{{\bf G}}
\def\bH{{\bf H}}
\def\bI{{\bf I}}
\def\bJ{{\bf J}}
\def\bK{{\bf K}}
\def\bL{{\bf L}}
\def\bM{{\bf M}}
\def\bN{{\bf N}}
\def\bO{{\bf O}}
\def\bP{{\bf P}}
\def\bQ{{\bf Q}}
\def\bR{{\bf R}}
\def\bS{{\bf S}}
\def\bT{{\bf T}}
\def\bU{{\bf U}}
\def\bV{{\bf V}}
\def\bW{{\bf W}}
\def\bX{{\bf X}}
\def\bY{{\bf Y}}
\def\bZ{{\bf Z}}

% Blackboard font

\def\IB{{\mathbb{B}}}
\def\IC{{\mathbb{C}}}
\def\IF{{\mathbb{F}}}
\def\IN{{\mathbb{N}}}
\def\IQ{{\mathbb{Q}}}
\def\IR{{\mathbb{R}}}
\def\IS{{\mathbb{S}}}
\def\IT{{\mathbb{T}}}
\def\IZ{{\mathbb{Z}}}

\def\IIB{{\mathbb{\bf B}}}
\def\IIC{{\mathbb{\bf C}}}
\def\IIN{{\mathbb{\bf N}}}
\def\IIQ{{\mathbb{\bf Q}}}
\def\IIR{{\mathbb{\bf R}}}
\def\IIZ{{\mathbb{\bf Z}}}

% Definitions

\def\Low{\mathfrak{L}}
\def\Bar{\overline}

\def\AND{\;\mbox{and}\;}
\def\OR{\;\mbox{or}\;}

\def\To{\longrightarrow}
\def\TO{\Longrightarrow}
\def\In{\subseteq}
\def\sm{\setminus}
\def\Inneq{\In_{\!\!\!\!/}}
\def\dmin{-^{\!\!\!\!\cdot}\;}
\def\splus{\oplus}
\def\SEQ{\triangle}
\def\DIV{\uparrow}
\def\INV{\leftrightarrow}
\def\SET{\Diamond}

\def\kto{\equiv\!\equiv\!>}
\def\kin{\subset\!\subset}
\def\pto{\leadsto}
\def\into{\hookrightarrow}
\def\onto{\to\!\!\!\!\!\to}
\def\prefix{\sqsubseteq}
\def\rel{\leftrightarrow}
\def\mto{\rightrightarrows}

\def\Cf{C\!f}
\def\id{{\rm id}}
\def\pr{{\rm pr}}
\def\inj{{\rm in}}
\def\cf{{\rm cf}}
\def\dom{{\rm dom}}
\def\range{{\rm range}}
\def\graph{{\rm graph}}
\def\ker{{\rm kern}}
\def\epi{{\rm epi}}
\def\hypo{{\rm hypo}}
\def\Lim{{\rm Lim}}
\def\diam{{\rm diam}}
\def\dist{{\rm dist}}
\def\supp{{\rm supp}}
\def\union{{\rm union}}
\def\fiber{{\rm fiber}}
\def\ev{{\rm ev}}
\def\mod{{\rm mod}}
\def\conv{{\rm conv}}
\def\seq{{\rm seq}}
\def\span{{\rm span}}
\def\sat{{\rm sat}}
\def\card{{\rm card}}


\def\Add{{\rm Add}}
\def\Mul{{\rm Mul}}
\def\SMul{{\rm SMul}}
\def\Neg{{\rm Neg}}
\def\Inv{{\rm Inv}}
\def\Ord{{\rm Ord}}
\def\Sqrt{{\rm Sqrt}}
\def\Re{{\rm Re}}
\def\Im{{\rm Im}}
\def\Sup{{\rm Sup}}
\def\Inf{{\rm Inf}}
\def\LSC{{\mathcal LSC}}
\def\USC{{\mathcal USC}}
\def\Li{{\rm Li}}
\def\Ls{{\rm Ls}}

\def\Cantor{{\{0,1\}^\IN}}
\def\Baire{{\IN^\IN}}

\def\QED{$\hspace*{\fill}\Box$}
\def\rand#1{\marginpar{\rule[-#1 mm]{1mm}{#1mm}}}

\def\ll#1{\ell_{#1}}
\def\BL{\BB}

\def\IF{{\rm IF}}
\def\Tr{{\rm Tr}}
\def\Sierp{{\rm Sierpi{\'n}ski}}
\def\psisierp{{\psi^{\mbox{\scriptsize\Sierp}}}}

\newcommand{\SO}[1]{{{\bf\Sigma}^0_{#1}}}
\newcommand{\SI}[1]{{{\bf\Sigma}^1_{#1}}}
\newcommand{\PO}[1]{{{\bf\Pi}^0_{#1}}}
\newcommand{\PI}[1]{{{\bf\Pi}^1_{#1}}}
\newcommand{\DO}[1]{{{\bf\Delta}^0_{#1}}}
\newcommand{\DI}[1]{{{\bf\Delta}^1_{#1}}}
\newcommand{\sO}[1]{{\Sigma^0_{#1}}}
\newcommand{\sI}[1]{{\Sigma^1_{#1}}}
\newcommand{\pO}[1]{{\Pi^0_{#1}}}
\newcommand{\pI}[1]{{\Pi^1_{#1}}}
\newcommand{\dO}[1]{{{\Delta}^0_{#1}}}
\newcommand{\dI}[1]{{{\Delta}^1_{#1}}}
\newcommand{\sP}[1]{{\Sigma^\P_{#1}}}
\newcommand{\pP}[1]{{\Pi^\P_{#1}}}
\newcommand{\dP}[1]{{{\Delta}^\P_{#1}}}
\newcommand{\sE}[1]{{\Sigma^{-1}_{#1}}}
\newcommand{\pE}[1]{{\Pi^{-1}_{#1}}}
\newcommand{\dE}[1]{{\Delta^{-1}_{#1}}}

\def\LPO{\text{\rm\sffamily LPO}}
\def\LLPO{\text{\rm\sffamily LLPO}}
\def\WKL{\text{\rm\sffamily WKL}}
\def\RCA{\text{\rm\sffamily RCA}}
\def\ACA{\text{\rm\sffamily ACA}}
\def\SEP{\text{\rm\sffamily SEP}}
\def\BCT{\text{\rm\sffamily BCT}}
\def\IVT{\text{\rm\sffamily IVT}}
\def\IMT{\text{\rm\sffamily IMT}}
\def\OMT{\text{\rm\sffamily OMT}}
\def\CGT{\text{\rm\sffamily CGT}}
\def\UBT{\text{\rm\sffamily UBT}}
\def\BWT{\text{\rm\sffamily BWT}}
\def\HBT{\text{\rm\sffamily HBT}}
\def\BFT{\mbox{\rm\sffamily BFT}}
\def\FPT{\text{\rm\sffamily FPT}}
\def\WAT{\text{\rm\sffamily WAT}}
\def\LIN{\text{\rm\sffamily LIN}}
\def\B{\text{\rm\sffamily B}}
\def\BF{\text{\rm\sffamily B$_{\rm\mathsf F}$}}
\def\BI{\text{\rm\sffamily B$_{\rm\mathsf I}$}}
\def\C{\mbox{\rm\sffamily C}}
\def\UC{\mbox{\rm\sffamily UC}}
\def\CF{\text{\rm\sffamily C$_{\rm\mathsf F}$}}
\def\CN{\text{\rm\sffamily C$_{\IN}$}}
\def\CI{\text{\rm\sffamily C$_{\rm\mathsf I}$}}
\def\CK{\text{\rm\sffamily C$_{\rm\mathsf K}$}}
\def\CA{\text{\rm\sffamily C$_{\rm\mathsf A}$}}
\def\LPO{\mbox{\rm\sffamily LPO}}
\def\LLPO{\mbox{\rm\sffamily LLPO}}
\def\WPO{\text{\rm\sffamily WPO}}
\def\WLPO{\text{\rm\sffamily WLPO}}
\def\MLPO{\mbox{\rm\sffamily MLPO}}
\def\MP{\text{\rm\sffamily MP}}
\def\BD{\text{\rm\sffamily BD}}
\def\MCT{\text{\rm\sffamily MCT}}
\def\UBWT{\text{\rm\sffamily UBWT}}
\def\WBWT{\text{\rm\sffamily WBWT}}
\def\K{\text{\rm\sffamily K}}
\def\AP{\text{\rm\sffamily AP}}
\def\L{\text{\rm\sffamily L}}
\def\CL{\text{\rm\sffamily CL}}
\def\KL{\text{\rm\sffamily KL}}
\def\KC{\text{\rm\sffamily KC}}
\def\UCL{\text{\rm\sffamily UCL}}
\def\U{\text{\rm\sffamily U}}
\def\A{\text{\rm\sffamily A}}
\def\CA{\text{\rm\sffamily CA}}


\def\leqm{\mathop{\leq_{\mathrm{m}}}}
\def\equivm{\mathop{\equiv_{\mathrm{m}}}}
\def\leqT{\mathop{\leq_{\mathrm{T}}}}
\def\nleqT{\mathop{\not\leq_{\mathrm{T}}}}
\def\equivT{\mathop{\equiv_{\mathrm{T}}}}
\def\lT{\mathop{<_{\mathrm{T}}}}
\def\leqtt{\mathop{\leq_{\mathrm{tt}}}}
\def\equiPT{\mathop{\equiv_{\P\mathrm{T}}}}
\def\leqW{\mathop{\leq_{\mathrm{W}}}}
\def\equivW{\mathop{\equiv_{\mathrm{W}}}}
\def\nequivW{\mathop{\not\equiv_{\mathrm{W}}}}
\def\leqSW{\mathop{\leq_{\mathrm{sW}}}}
\def\leqSSW{\mathop{\leq_{\mathrm{ssW}}}}
\def\equivSW{\mathop{\equiv_{\mathrm{sW}}}}
\def\equivSSW{\mathop{\equiv_{\mathrm{ssW}}}}
\def\leqPW{\mathop{\leq_{\widehat{\mathrm{W}}}}}
\def\equivPW{\mathop{\equiv_{\widehat{\mathrm{W}}}}}
\def\nleqW{\mathop{\not\leq_{\mathrm{W}}}}
\def\nleqSW{\mathop{\not\leq_{\mathrm{sW}}}}
\def\nleqSSW{\mathop{\not\leq_{\mathrm{ssW}}}}
\def\lW{\mathop{<_{\mathrm{W}}}}
\def\lSW{\mathop{<_{\mathrm{sW}}}}
\def\nW{\mathop{|_{\mathrm{W}}}}
\def\nSW{\mathop{|_{\mathrm{sW}}}}
\def\nSSW{\mathop{|_{\mathrm{ssW}}}}

\def\bigtimes{\mathop{\mathsf{X}}}

\def\botW{\mathbf{0}}
\def\midW{\mathbf{1}}
\def\topW{\emptyset}

\def\stars{*_{\rm s}\;\!}
%\def\ups{{^{\rm s}}'}
\def\ups{{'}^{_{\rm\footnotesize s}}}




\newcommand{\fa}{\forall}
\newcommand{\ex}{\exists}
\newcommand{\set}[2]{\{\,{#1}\mid{#2}\,\}}
\newcommand{\toto}{\rightrightarrows}
\newcommand{\eps}{\emptyset}
\newcommand{\bbN}{\mathbb{N}}
\newcommand{\bbZ}{\mathbb{Z}}
\newcommand{\bbQ}{\mathbb{Q}}
\newcommand{\bbR}{\mathbb{R}}
\newcommand{\I}{\mathbf{I}}
\renewcommand{\a}{\alpha}
\newcommand{\mcM }{\mathcal{M}}
\newcommand{\sbsq}{{\subseteq}}
\newcommand{\sqsbsq}{\sqsubseteq}
%\newcommand{\dom}{\operatorname{dom}}
\newcommand{\ran}{\operatorname{ran}}
\newcommand{\Can}{\ensuremath{{2^\bbN}}}
\newcommand{\Bai}{\ensuremath{{\bbN^\bbN}}}
\newcommand{\Seqd}{\ensuremath{2^{<\bbN}}}
\newcommand{\Seq}{\ensuremath{\bbN^{<\bbN}}}
\newcommand{\omu}{{\omega+1}}
\newcommand{\omom}{{\omega+\omega}}
\newcommand{\fS}{\mathbf{\Sigma}^0}
\newcommand{\mcA}{\mathcal{A}}
\newcommand{\Card}{\operatorname{Card}}

%\newcommand{\LPO}{\text{\rm\sffamily LPO}}
\newcommand{\AS}{\text{\rm\sffamily AS}}
%\newcommand{\LLPO}{\text{\rm\sffamily LLPO}}
%\newcommand{\BWT}{\text{\rm\sffamily BWT}}
\newcommand{\Sep}{\text{\rm\sffamily SEP}_2}
%\newcommand{\C}{\text{\rm\sffamily C}}
%\newcommand{\B}{\text{\rm\sffamily B}}
%\newcommand{\CN}{\text{\rm\sffamily C$_{\bbN}$}}
\newcommand{\de}{\delta}
\newcommand{\D}{\Delta}
%\newcommand{\leqW}{\mathop{\leq_{\mathrm{W}}}}
%\newcommand{\equivW}{\mathop{\equiv_{\mathrm{W}}}}
%\newcommand{\nleqW}{\mathop{\not\leq_{\mathrm{W}}}}
\newcommand{\incW}{\mathop{|_{\mathrm{W}}}}
\newcommand{\W}{\mathrm{W}}

%\thanks{Ccc}
\date{\today}

\newtheorem{theorem}{Theorem}
\newtheorem{proposition}[theorem]{Proposition}
\newtheorem{lemma}[theorem]{Lemma}
\newtheorem{fact}[theorem]{Fact}
\newtheorem{conjecture}[theorem]{Conjecture}
\newtheorem{corollary}[theorem]{Corollary}
\newtheorem{claim}{Claim}[theorem]
\newtheorem{question}{Question}[theorem]

\theoremstyle{definition}
\newtheorem{definition}[theorem]{Definition}
\newtheorem{remark}[theorem]{Remark}
\newtheorem{notation}[theorem]{Notation}
\newtheorem{example}[theorem]{Example}

\begin{document}

\maketitle

%\tableofcontents
%\pagebreak

\begin{abstract}
The purpose of this addendum is to close a gap in the proof of \cite[Theorem~11.2]{BGM12},
which characterizes the computational content of the Bolzano-Weierstra\ss{} Theorem for arbitrary
computable metric spaces.
\end{abstract}



%\section{Introduction}

In \cite[Theorem~11.2]{BGM12} it is stated that $\BWT_X\equivSW \K_X'$ holds
for all computable metric spaces $X$.
Here $\BWT_X$ denotes the Bolzano-Weierstra\ss{} Theorem, $\K_X'$ denotes the jump of compact
choice and $\equivSW$ stands for strong Weihrauch equivalence.
We refer the reader to \cite{BGM12} for the definition of all notions that are not defined here.

While the reduction $\BWT_X\leqSW \K_X'$ was proved correctly in \cite{BGM12},
the proof provided for $\K_X'\leqSW\BWT_X$ contains a gap and is
only correct for the special case of compact $X$ as it stands.
This fact was pointed out by one of us (M.\ Schr\"oder)
and is due to the fact that in general the closure of $\L_X^{-1}(K)$ is not compact.
We close this gap in this addendum.

We start with a lemma that shows that compact sets given in $\KK_-'(X)$ are
effectively totally bounded in a particular sense. By $\OO(X)$ we denote the
set of open subsets of $X$, represented as complements of elements of
$\AA_-(X)$, i.e., $p$ is a name of an open set $U$ if and only if it is a
$\psi_-$--name of the closed set $X\setminus U$. We call an open ball
$B(a,r)$ {\em rational}, if $a$ is a point of the dense subset of $X$ (that
is used to define the computable metric space $X$) and $r \geq 0$ is a
rational number.

\begin{lemma}
\label{lem1}
Let $X$ be a computable metric space. Consider the multivalued function
$F_X:\In\KK_-'(X)\mto\OO(X)^\IN$ with $\dom(F_X)=\{K\in\KK_-'(X):K\neq\emptyset\}$
and such that, for each $K\neq\emptyset$, we have $(U_n)_n\in F_X(K)$ if and only if the following conditions hold for each $n\in\IN$:
\begin{enumerate}
\item $U_n$ is a union of finitely many rational open balls of radius $\leq 2^{-n}$,
\item $K\subseteq U_n $.
\end{enumerate}
Then $F_X$ is computable.
\begin{proof}
Let $X$ be a computable metric space and let $K\subseteq X$ be a nonempty compact set. Let $\langle p_i\rangle_i$ be a $\kappa'_-$--name of $K$.
This means that $p:=\lim_{i\to\infty}p_i$ is a $\kappa_-$--name for $K$ and, in particular, for each $n\in\IN$:			 	
\begin{itemize}
\item  $p_i(n)$ is a name for a finite set of rational open balls for each $i\in\IN$,
\item there exists $k\in\IN$ such that the finite set of rational balls given by $p_k(n)$ covers $K$ and ${p_k(n)=p_i(n)}$ for all $i\geq k$.
\end{itemize}
We also have that $\{p(n):n\in\IN\}$ is a set of names of all finite covers of $K$ by rational open balls.
We want to build a sequence of open sets $(U_n)_n$ such that (1) and (2) hold.
We describe how  to construct a name of a generic open set $U_n$ for $n\in\IN$.
We start at stage $0$ with $U_n=\emptyset$.
At each stage $s=\langle m,i\rangle$ that the computation reaches, we focus on the balls $B(a_0,r_0),\dots,B(a_l,r_l)$ given by $p_i(m)$
and we check whether $r_0,\dots,r_l\leq 2^{-n}$. If this is not true, then we go to stage $s+1$.
Otherwise, if the condition is met, we add these balls to the name of $U_n$  and we check whether $p_i(m)=p_{i+1}(m)$.
If this is the case we add again $B(a_0,r_0),\dots,B(a_l,r_l)$ to the name of $U_n$.
We repeat this operation as long as we find the same open balls given by $p_j(m)$ for $j>i$.
If we find $p_i(m)\neq p_j(m)$ for some $j>i$, then the computation goes to stage $s+1$.

We claim that, for each $n$, there exists a stage in which the computation goes on indefinitely.
Consider, in fact, $\{B(a_0,r_0),\dots,B(a_l,r_l)\}$, a finite rational cover of $K$ with $r_0,\dots,r_l\leq 2^{-n}$,
which exists by a simple argument using the compactness of $K$. Since $\langle p_i\rangle_i$ is a $\kappa_-'$-name of $K$,
there exists a minimum $\langle m,i\rangle$ such that:
		\begin{itemize}
			\item $p_i(m)$ is a name for the cover $\{B(a_0,r_0),\dots,B(a_l,r_l)\}$,
			\item $p_i(m)=p_j(m)$ for each $j>i$.
		\end{itemize}
If the algorithm reaches stage $s=\langle m,i\rangle$, then it is clear that the computation goes on indefinitely within this stage.
If the algorithm never reaches stage $s$, then necessarily it already stopped at a previous stage. In both cases our claim is true.

Finally, since we built the name of $U_n$ by adding only balls of radius $\leq 2^{-n}$ and since the computation stabilizes at a finite stage, it is clear that conditions (1) and (2) are met.
	\end{proof}
\end{lemma}

We note that even though the open sets $U_n$ constructed in the previous proof are finite unions of rational open balls,
the algorithm does not provide a corresponding rational cover in a finitary way. It rather provides an infinite list of rational open balls
that is guaranteed to contain only finitely many distinct rational balls.
This is a weak form of effective total boundedness and the best one can hope for, given that the input is represented
by the jump of $\kappa_-$.

The following lemma shows that sequences that we choose in $\range(F_X)$ in a particular way
give rise to totally bounded sets.

\begin{lemma}
\label{lem2}
Let $X$ be a metric space and let $U_n\In X$ be a finite union of balls of radius $\leq 2^{-n}$ for each $n\in\IN$.
Let $(x_n)_n$ be a sequence in $X$ with $x_n\in\bigcap_{i=0}^n U_i$. Then $\overline{\{x_n:n\in\IN\}}$ is totally bounded.
\end{lemma}
\begin{proof}
We obtain  $\{x_n:n\in\IN\}\In\bigcap_{i=0}^\infty\left(U_i\cup\bigcup_{n=0}^{i-1}B(x_n,2^{-i})\right)$ and the
set on the right-hand side is clearly totally bounded. Hence the set on the left-hand side is totally bounded and so is its closure.
\end{proof}

We mention that it is well known that a subset of a metric space is totally
bounded if and only if any sequence in it has a Cauchy subsequence
\cite[Exercise~4.3.A~(a)]{Eng89}.

Now we use the previous two lemmas to complete the proof of \cite[Theorem
11.2]{BGM12}. Within the proof we use the canonical completion $\hat{X}$ of a
computable metric space. It is known that this completion is a computable
metric space again and that the canonical embedding $X\into\hat{X}$ is a
computable isometry that preserves the dense sequence
\cite[Lemma~8.1.6]{Wei00}. We will identify $X$ with a subset of $\hat{X}$
via this embedding.

\begin{theorem}[{\cite[Theorem 11.2]{BGM12}}]  $\BWT_X\equivSW\K'_X$ for all computable metric spaces $X$.
\end{theorem}
\begin{proof} The reduction $\BWT_X\leqSW\K'_X$ has been proved in \cite{BGM12},
so we focus on the reduction $\K'_X\leqSW\BWT_X$.
Let $(X,d,\alpha)$ be a computable metric space and let $K\subseteq X$ be a nonempty compact set given by a $\kappa'_-$--name $\langle p_i\rangle_i$.
We want to compute a point of $K$ using $\BWT_X$. The idea is to define a sequence $(x_n)_n$ in $X$,
working within the completion $\hat{X}$ of $X$ and using the open sets built in Lemma~\ref{lem1}, such that $\overline{\{x_n:n\in\IN\}}$ is compact in $X$.

It is clear that $K$ is a compact set of $\hat{X}$ and that $\langle p_i\rangle_i$ can be considered as a $\kappa_-'$--name for $K$ in $\hat{X}$.
We consider the map
\[\L_{\hat{X}}:\hat{X}^\IN\to\AA'_-(\hat{X}),(x_n)_n\mapsto\{x\in \hat{X}:\mbox{$x$ is a cluster point of $(x_n)_n$}\}.\]
By \cite[Corollary 9.5]{BGM12} $\L^{-1}_{\hat{X}}$ is computable and hence $\L_{\hat{X}}^{-1}(K)$ yields a sequence
$(z_m)_m$ in $\hat{X}$ whose cluster points are exactly the elements of $K$.

Let $F_{\hat{X}}$ be the multivalued function defined in Lemma~\ref{lem1}. We
can compute a sequence $(U_n)_n\in F_{\hat{X}}(K)$. Since
$\overline{\{z_m:m\in\IN\}}$ is not compact (and hence not in $\dom(\BWT_X)$)
in general, we refine it recursively to a sequence $(y_n)_n$ using $(U_n)_n$
in the following way: for each $n\in\IN$, $y_n:=z_{m_n}$ for the first $m_n$
that we find with $z_{m_n}\in U_0\cap\dots\cap U_n$ and such that $m_i<m_n$
for all $i<n$. Note that we can always find such a $y_n$, since
$U_0\cap\dots\cap U_n$ covers $K$ which is the set of cluster points of
$(z_m)_m$. Clearly every cluster point of $(y_n)_n$ is also a cluster point
of $(z_m)_m$, hence it belongs to $K$.

Recall now that $(y_n)_n$ is a sequence of points in $\hat{X}$ and that we want a sequence $(x_n)_n$ in $X$ in order to apply $\mathsf{BWT}_X$.
We compute $(x_n)_n$ as follows: for each $n\in\IN$, $x_n$ is the first element that we find in the dense subset $\range(\alpha)$ such that $d(x_n,y_n)<2^{-n}$
and $x_n\in U_0\cap\dots\cap U_n$, where $d$ also denotes the extension of the metric to $\hat{X}$.
By density of $X$ in $\hat{X}$ such an $x_n$ always exists and it is clear that the cluster points of $(x_n)_n$ and those of $(y_n)_n$ are the same in $\hat{X}$.

Now $A:=\overline{\{x_n:n\in\IN\}}$ is totally bounded in $X$ by Lemma~\ref{lem2} and hence every sequence in $A$ has a Cauchy subsequence,
which has a limit in $\hat{X}$, since $\hat{X}$ is complete. By construction of $(x_n)_n$ the limit of such a subsequence is in $K$ and hence in $X$.
Thus every sequence in $A$ has a subsequence that converges in $X$ and hence $A$ is compact in $X$.
%
%It remains to prove that $\overline{\{x_n:n\in\IN\}}$ is compact. It suffices to show that each sequence  $(w_n)_n$ in $\{x_n:n\in\IN\}$ has a convergent subsequence in $X$.
%Without loss of generality, we can assume that $(w_n)_n$ has no constant subsequences. We reason in the following way: since  $w_i\in\{x_n:n\in\IN\}\subseteq U_0$
%and $U_0$ is a union of finitely many balls of radius $\leq 1$, one of these balls, let us say $B(a_0,r_0)$, contains $w_i$ for infinitely many $i\in\IN$. Let $i_0$ be the least such index.
%Now we can repeat the argument using the subsequence contained in $B(a_0,r_0)$, which, by abuse of notation, we continue to call $(w_n)_n$.
%This sequence, since it has no constant subsequences, is eventually contained in $U_1$,
%which means that there exists a ball $B(a_1,r_1)\subseteq U_1$ with $r_1<\frac{1}{2}$ that contains $w_i$ for infinitely many $i\in\IN$. Let $i_1$ be the least such index with $i_1>i_0$.
%By iterating this process we can build a subsequence $(w_{i_k})_k$ of $(w_n)_n$ with the property that,
%for each $k\in\IN$ and for each $l\geq k$, $w_{i_l}\in B(a_k,r_k)\subseteq B(a_k,2^{-k})$, hence $(w_{i_k})_k$ is a Cauchy sequence.
%In $\hat{X}$ such a sequence has a limit point which, due to the definition of $(x_n)_n$, belongs necessarily to $K$.
%Therefore $(w_{i_k})_k$ converges in $X$. This proves that $\overline{\{x_n:n\in\IN\}}$ is compact.

Finally, we can obtain an element of $K$ by applying $\mathsf{BWT}_{X}$ to $(x_n)_n$.
\end{proof}






\bibliographystyle{plain}
\bibliography{C:/Users/Vasco/Dropbox/Bibliography/lit}
%\bibliography{C:/Users/vbrattka/Dropbox/Bibliography/lit}









\end{document}
