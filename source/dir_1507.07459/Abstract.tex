\chapter*{\centering \begin{normalsize}Abstract\end{normalsize}}

\thispagestyle{plain}
\addcontentsline{toc}{chapter}{\numberline{}Abstract}

\begin{quotation}
\noindent Set packing is a fundamental problem that generalises some well-known combinatorial optimization problems and knows a lot of applications. It is equivalent to hypergraph matching and it is strongly related to the maximum independent set problem.

In this thesis we study the $k$-set packing problem where given a universe $\mathcal{U}$ and a collection $\mathcal{C}$ of subsets over $\mathcal{U}$, each of cardinality $k$, one needs to find the maximum collection of mutually disjoint subsets. Local search techniques have proved to be successful in the search for approximation algorithms, both for the unweighted and the weighted version of the problem where every subset in $\mathcal{C}$ is associated with a weight and the objective is to maximise the sum of the weights. We make a survey of these approaches and give some background and intuition behind them. In particular, we simplify the algebraic proof of the main lemma for the currently best weighted approximation algorithm of Berman (\cite{Berman}) into a proof that reveals more intuition on what is really happening behind the math.

The main result is a new bound of $\frac{k}{3} + 1 + \varepsilon$ on the integrality gap for a polynomially sized LP relaxation for $k$-set packing by Chan and Lau (\cite{LapChiLau}) and the natural SDP relaxation \textbf{[NOTE: see page iii]}. We provide detailed proofs of lemmas needed to prove this new bound and treat some background on related topics like semidefinite programming and the Lov\'{a}sz Theta function.

Finally we have an extended discussion in which we suggest some possibilities for future research. We discuss how the current results from the weighted approximation algorithms and the LP and SDP relaxations might be improved, the strong relation between set packing and the independent set problem and the difference between the weighted and the unweighted version of the problem.

%Finally we discuss some directions in which the current results might be improved, the issues one encounters when mimicking the unweighted approximation algorithms in the weighted problem and why the $\frac{k+1}{3}$-approximations require much more work than the $\frac{k+2}{3}$-approximations.

%The main result is a new bound of $\frac{k}{3} + 1 + \varepsilon$ on the integrality gap for an LP relaxation for $k$-set packing by Chan and Lau (\cite{LapChiLau}), known as the intersecting family LP. We provide detailed proofs of lemmas needed to prove this new bound and show that the LP can be written in a size polynomial in the input. Then we treat some background on semidefinite programming and show that there exists a polynomially sized SDP for $k$-set packing with the same improved bound on the integrality gap.
\end{quotation} 