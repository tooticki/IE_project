%=x_{i+\frac12},\\x_{i+1}& \mbox{if }  \xi \!\in (x_{i+\frac12},x_{i+1} ].
%\end{array}\right. \]

\section{Extended Pierce lemma and applications}\label{Pierce}

The aim of this section is to provide a non-asymptotic ``universal" upper-bound for the optimal  (extended) $L^p$-mean 
dual quantization error in the spirit of~\cite{pierce}: it achieves nevertheless the optimal rate of convergence when the size $n$
goes to infinity.
Like for  Voronoi quantization this upper-bound deeply relies on a
random quantization argument and will be a key in the proof of the sharp rate (step~2 of the proof of Theorem~\ref{thm:DQRate}).
%In fact, it can be established for a (slightly) more general family of
%error functionals than the ones considered so far for dual and regular
%quantization.

For every integer $n\ge 1$, we define the set of ``non-decreasing" $n$-tuples of
$\R^n$ by
\[ {\cal I}_n := \{(x_1,\ldots,x_n)\!\in \R^n,\; -\infty<x_1\le
x_2\le\cdots\le x_n<+\infty\}.
\] 
Let $(x_1,\ldots,x_n)\!\in {\cal I}_n$ (so that $\Gamma=\{x_1,\ldots, x_n\}$ has at most $n$ elements) and  let $\xi\!\in \R$. When $d=1$, it is clear that  the minimization problem~(\ref{eq:Fp}) always
 has a unique solution when $\xi\!\in [x_1,x_n]$ so that, for every $\omega_0\!\in  \Omega_0=[0,1]$, one has 
 %(with an obvious abuse of notation induced by considering the $n$-tuple rather than the grid) 
% for every $\xi\!\in \R$
%\noindent $(b)$ {\em Dual quantization.} $\Omega_0=[0,1]$ and
%\begin{eqnarray*}
\begin{eqnarray*}
{\cal \bar J}^*_{(x_1,\ldots,x_n)}(\omega_0,\xi)&=&
%&=& 
\sum_{i=1}^{n-1}\Big(x_i\mbox{\bf 1}_{\{\omega_0 \le \frac{x_{i+1}-\xi}{x_{i+1}-x_i}\}} 
+x_{i+1}\mbox{\bf 1}_{\{\omega_0\ge  \frac{x_{i+1}-\xi}{x_{i+1}-x_i}\}}\Big) \mbox{\bf 1}_{[x_i,x_{i+1})}(\xi)\\
&& +x_1\mbox{\bf 1}_{(-\infty,x_1)}(\xi)+x_n  \mbox{\bf 1}_{[
x_n,+\infty)}(\xi).
\end{eqnarray*}
%\end{eqnarray*}

\smallskip
It follows from~(\ref{def:Fbarp}) that
%from $(i)$  that a splitting functional at level $n$ satisfies for every $p>0$,  $\omega\!\in \Omega_0$, $(x_1,\ldots,x_n)\!\in {\cal I}_n$, $\xi \!\in \R$, 
%\begin{equation}
% d(\xi,\{x_1,\ldots,x_n\})^p  \le  \displaystyle \left|\xi-
% \Phi_n(\omega,x_1,\ldots,x_n,\xi)\right|^p\le A_{p,n} (x_1,\ldots,x_n,\xi)^p
%\label{bornes}
%\end{equation}
%where
\begin{eqnarray}
\nonumber\bar F_n^p(\xi, x_1,\ldots,x_n) &=&   \E_{\Prob_0}\big|\xi -{\cal
\bar J}^*_{(x_1,\ldots,x_n)}(\omega_0,\xi)\big|^p\\
\label{eq:Fbard=1}
&=& \sum_{i=1}^{n-1}\left(\frac{(x_{i+1}-\xi)^p(\xi-x_i)}{x_{i+1}-x_i}+\frac{(x_{i+1}-\xi)(\xi-x_i)^p}{x_{i+1}-x_i}\right ) \mbox{\bf 1}_{[x_i,x_{i+1})}(\xi)\\
&& + (x_1-\xi )^p\mbox{\bf 1}_{(-\infty,x_1)}(\xi)+ (\xi-x_n)^p \mbox{\bf 1}_{[
x_n,+\infty)}(\xi)\nonumber
\end{eqnarray}
(the subscript $_n$ is temporarily added to the functional $\bar F^p$, $\bar F_p$, etc,  to emphasize that they are defined on $ {\cal I}_n\times \R$).
%Let  $X$ be random variable defined on $(\Omega_0,{\cal A},\Prob)$. 
%\[
%X\!\in L^p(\Prob) \Longrightarrow X-\Phi_n(.,x_1,\ldots,x_n,X),\; A(x_1,\ldots,x_n,X)\!\in L^p(\Prob).
%\]
%Furthermore, it follows from~(\ref{bornes}) that
%\[
%\inf_{(x_1,\ldots,x_n)\in {\cal I}_n}\|A_{p,n}(x_1,\ldots,x_n,X)\|_{L^p}\ge\inf_{(x_1,\ldots,x_n)\in {\cal I}_n}\|X-\Phi_n(.,x_1,\ldots,x_n,X)\|_{L^p}\ge e_{n,p}(X).
%\]
The   functionals $\bar F_n^p$ share three important properties   extensively used  in what follows:

\begin{itemize}
\item {\em Additivity}: Let $(x_1, \ldots,x_{i_0},\ldots,x_n)\!\in {\cal I}_n$. Then for every $\xi\!\in \R$
\[
\bar F^p_n (\xi, x_1,\ldots,x_n)= \bar F^p_{i_0} (\xi,
x_1,\ldots,x_{i_0})\mbox{\bf 1}_{(-\infty,x_{i_0})}(\xi)+ \bar F^p_{n- i_0+1}
(\xi, x_{i_0},\ldots,x_{n})\mbox{\bf 1}_{ [x_{i_0}, +\infty)}(\xi).
\]
\item
 {\em  Consistency and monotony}:
 % if, for every $\omega\!\in \Omega_0$, 
 % for every  
 Let $(x_1,\ldots,x_{n})\!\in {\cal I}_{n}$ and $\widetilde x_i\!\in [x_i,x_{i+1}]$ for an $i\!\in \{1,\ldots,n-1\}$. For every  $\xi \!\in \R$, 
\[
\bar F^{p}_{n+1}(\xi, x_1,\ldots,x_{i-1},x_i,\widetilde x_i,x_{i+1},\ldots,x_n)\le \bar F^{p}_{n}(\xi, x_1,\ldots,x_{i-1},x_i,x_{i+1},\ldots,x_n).
\]
When $\xi\!\in [x_1,x_n]$, $\bar F_n^p(\xi;x_1,\dots,x_n)$ coincides with  $
F^p(\xi,\{x_1,\dots,x_n\})$ and this inequality is  a consequence of the
definition of $ F_p$ as the value function of the minimization
problem~(\ref{eq:Fp}). Outside, the above inequality holds as an equality since
it amounts to the nearest distance of $\xi$ to $[x_1,x_n]$. 
%and if $i=n$ ($i=1$ respectively) and $\widetilde x_n\ge x_n$ ($\widetilde x_0\le x_1 $ respectively) then
%\[
%F^{p}_{n+1}(x_1,\ldots,x_n, \widetilde x_{n+1},\xi)\le F^{p}_{n}(x_1, \ldots,x_n,\xi).
%\]
%and 
%\[
%F^{p}_{n+1}(x_0, x_1, \ldots,x_n, \widetilde x_{n+1},\xi)\le  F^{p}_{n}(x_1, \ldots,x_n,\xi).
%\]
%
As a  consequence, 
%one checks that
%\textcolor{red}{Shouldn't this be $\bar F$ in the ff?}
\begin{equation}\label{monotonie1}
n\longmapsto \bar d_{n,p}(X)=  \inf_{(x_1,\ldots,x_n)\in {\cal I}_n}\|\bar F_{p,n}(X,
x_1,\ldots,x_n)\|_{L^p} \;\mbox{ is non-increasing,}
\end{equation}
More generally, for every fixed $x^0\!\in \R$, both
\begin{equation}\label{monotonie2}
\hskip -0.5cm  n\longmapsto\hskip -0.65cm   \inf_{(x^0,x_2,\ldots,x_n)\in {\cal I}_n}\|\bar F_{p,n}(X,x^0, x_2,\ldots,x_n)\|_{L^p} \,\mbox{ and }\,n\longmapsto \hskip -0.65cm  \inf_{(x_1,x_2,\ldots,x_{n-1},x^0)\in {\cal I}_n}\|\bar F_{p, n}(X, x_1,\ldots,x_{n-1},x^0)\|_{L^p} 
\end{equation}
are non-increasing.
%idem for every fixed fixed $x^0_n\!\in \R$.
%\[
%n\mapsto  \inf_{(x_1,x_2,\ldots,x^0_n)\in {\cal I}_n}\|F^p_{n}(x_1,\ldots,x^0_n,X)\|_{L^p} \;\mbox{ is non-increasing.}
%\]

\item {\em Scaling}: $\forall\,\omega\!\in \Omega_0$, $\forall\, (x_1,\ldots,x_n)\!\in {\cal I}_n$, $\forall\, \xi \!\in \R$, $\forall\, \alpha\!\in \R_+$,  $\forall\, \beta\!\in \R$,
\begin{eqnarray*}\label{eq:scaling}
(i)& \bar F^p_n(\alpha\, \xi+\beta, \alpha\, x_1 +\beta ,\ldots,\alpha\, x_n +\beta)&= \alpha \,\bar F_n^p(\xi,x_1,\ldots,x_n),\\
(ii) &\bar F^p_n(\xi, x_1   ,\ldots,x_n)&=\bar
F^p_n(-\xi,-x_n,\ldots,-x_1).
\end{eqnarray*}
\end{itemize}

%\medskip
%The main result of this section shows the existence of a universal non-asymptotic upper bound for the local dual quantization error. It appears as an extension of the so-called Pierce Lemma established in~\cite{Foundations} (see also~\cite{meanRegular}) as crucial step towards Zador's Theorem for regular Voronoi quantization.

\begin{thm}\label{Pierce0}Let $p,\,\eta>0$. There exists a real
constant $C_{p,\eta}>0$
%  and an integer $n_{p,\eta}\ge 1$ 
such that for every random variable  $X:(\Omega,{\cal A},\Prob)\to \R$,
%and any consistent sequence of splitting functionals $(\Phi_n)_{n\ge 1}$
 %defined on a probability space $(\Omega_0,{\cal A}_0,\Prob_0)$
\[
\forall\, n\ge 1,\;\quad
%n_{p,\eta},\;\quad 
\inf_{(x_1,\ldots,x_n)\in {\cal I}_n}\|\bar
F_{p,n}(X, x_1,\ldots,x_n)\|_{L^p} \le C_{p,\eta}\|X\|_{L^{p+\eta}}n^{-1}.
\] 
%(where $X$ and $\Phi_n$ have been canonically extended to $\Omega_0\times\Omega$).
\end{thm}


%\medskip
The proof below relies  on a random quantization argument involving an $n$-sample of the Pareto$(\delta)$-distribution on $[1,+\infty)$. Though significantly more demanding, it plays the same  crucial role in establishing the sharp rate result as the so-called Pierce Lemma established in~\cite{meanRegular} (see also~\cite{Foundations}) for Voronoi quantization    to prove the original Zador Theorem.  

In the proof, we will make use of the $\Gamma$ and $B$ functions defined by
$\Gamma(a)=\int_0^{+\infty}u^{a-1}e^{-u}du$, $a>0$, and
$B(a,b) =\int_0^1u^{a-1}(1-u)^{b-1}du$, $a,b>0$,  respectively, and satisfying $B(a,b)= \frac{\Gamma(a)\Gamma(b)}{\Gamma(a+b)}$.


\begin{proof} 
%First note that if $n=1$, then $\bar F_1^p(\xi,x_1) = |\xi-x_1|^p$ so that for any r.v. $X$, $\bar d_{1,p}(X)= \sigma_{p,|.|}(X)\le  \sigma_{p+\eta,|.|}(X)$. When $n=2$, $\bar d_{2,p}(X)\le \bar d_{,p}(X)\le 2 \sigma_{p+\eta,|.|}(X)\frac 12$.  
%So in what follows we may assume that the size $n$ satisfies $n\ge 3$.
%
%\medskip
{\sc Step~1.} 
%First, we note that owing to~(\ref{bornes}), for every $n\ge 1$, 
%\[
%\inf_{(x_1,\ldots,x_n) \in {\cal I}_n}\|X-\Phi_n(.,x_1,\ldots,x_n,X)\|_{L^p}\le \inf_{(x_1,\ldots,x_n) \in {\cal I}_n}\|A_{p,n}x_1,\ldots,x_n,X)\|_{L^p}
%\]
%so that we may deal from now on with the functional $A$ to establish our upper-bound.
%
%{\sc Step~2.} 
We first assume that $X$ is $[1,+\infty)$-valued and $n\ge 2$. Let $(Y_n)_{n\ge1}$ be a sequence of i.i.d. Pareto$(\delta)$-distributed random variables (with probability density $f_{_Y}(y) =\delta y^{-\delta-1}\mbox{\bf 1}_{\{y\ge 1\}}$) defined on a probability space $(\Omega',{\cal A}', \Prob')$. 
%
%By considering $\widetilde \Omega=\Omega\times\Omega'$,  $\widetilde{{\cal A}} = {\cal A}\otimes {\cal A}'$, 
%$\widetilde \Prob= \Prob\otimes \Prob'$, one may assume without loss of generality that $X$ and the sequence 
%$(Y_n)_{n\ge 1}$ are independent (and defined on the same probability space $(\Omega,{\cal A},\Prob)$). 
%For convenience we will denote by $\|\,.\,\|_{L^p}$ the $L^p$-norm on all the probability spaces coming out in the 
%proof like $(\Omega_0\times\Omega, {\cal A}_0\otimes {\cal A}, \Prob_0\otimes \Prob)$, etc.
%

\smallskip Let $\delta=\delta(\eta,p)\!\in (0,\frac{\eta}{\lceil p \rceil})$ be chosen so that $\ell=\ell(p,\eta)=\frac{p}{\delta}$  is an integer and $\ell\ge 2$. 
 For every $n\ge \ell(p,\eta)$, set $\widetilde n= n-\ell+2\!\in \N$, $\widetilde n\le n$. It follows from the monotony property~(\ref{monotonie2}) that
\begin{eqnarray*}
\inf _{(1, x_2,\ldots,x_n)\in {\cal I}_n}\| \bar F_{p,n} (X, 1,x_2,\ldots,x_n) \|_{L^p}& \le & \hskip -0.25 cm \inf_{(1,x_2,\ldots,x_{\widetilde n})\in {\cal I}_{\widetilde n}} \hskip -0.25 cm\|\bar F_{p,\widetilde n }(X,1,x_2,\ldots,x_{\widetilde n})\|_{L^p}\\
&\le& \hskip -0.25 cm\|\bar F_{p,\widetilde n}(X,Y^{(n)}_0,Y^{(n)}_1,\ldots,Y^{(n)}_{\widetilde n-1})\|_{L^p(\Omega\times \Omega', \Prob\otimes \Prob')}
\end{eqnarray*}

\noindent where, for every $n \ge 1$,  $Y^{(n)}=(Y^{(n)}_1,\ldots,Y^{(n)}_n)$ denotes the standard order statistics of the first $n$  terms of the sequence $(Y_k)_{k\ge1}$ and $Y^{(n)}_0=1$.  On the other hand, we recall (see $e.g.$~\cite{CAVC}) that  the joint distribution of $(Y^{(n)}_i,Y^{(n)}_{i+1})$, $1\le i\le n-1$,  is given by 
\[
\Prob'_{(Y^{(n)}_i,Y^{(n)}_{i+1})}(du,dv)=\delta^2  \frac{n!}{(i-1)!(n-i-1)!}(1-u^{-\delta})^{i-1}v^{-\delta(n-i-1)}(uv)^{-\delta-1}\,du\,dv.
\]
%\textcolor{red}{Isn't there a $\delta^2$ missing?}
%Then, using that $X$ and $(Y_k)_{k\ge 1}$ are independent, we get
%\begin{eqnarray*}
%\E \,A_{p,\widetilde n}(1,Y^{(n)}_1,\ldots,Y^{(n)}_{\widetilde n},X)^p&\le& \sum_{i=0}^{n-\ell}\E\Big(\big( Y^{(n)}_{i+1}-Y^{(n)}_i\big)^p\!\mbox{\bf 1}_{\{X\in[Y^{(n)}_i,Y^{(n)}_{i+1})\}}\Big)\\
%&&+\E\Big(\big (X-Y^{(n)}_{n-\ell+1}  \big)^p\!\mbox{\bf 1}_{\{X\ge Y^{(n)}_{n-\ell+1} \} } \Big).

%\end{eqnarray*}

\medskip
\noindent {\sc Step~2.} Assume that $n\ge 3$.  Since $X$ and $(Y_1,\ldots,Y_0)$ are  independent and $X\ge 1$
\[
\|\bar F_{p,\widetilde n}(X,Y^{(n)}_0,Y^{(n)}_1,\ldots,Y^{(n)}_{\widetilde n-1})\|^p_{L^p(\Omega\times \Omega', \Prob\otimes \Prob')}= \int_{[1,+\infty)}\|\bar F_{p,\widetilde n}(\xi, Y^{(n)}_0,Y^{(n)}_1,\ldots,Y^{(n)}_{\widetilde n-1})\|^p_{L^p( \Omega', \Prob')}\Prob_{_X}(d\xi).
\]
%since $X\ge 1$. 
%we will consider first $[1,+\infty)$-valued r.v.s, we can evaluate the function $\bar F_n^p(1,x_2,\ldots,x_n,\xi)$ only for $\xi\!\in [1,+\infty)$.  
Relying on the expression~(\ref{eq:Fbard=1}) of the functional $\bar F^p_n$, we set for every $i=0,\ldots,n-\ell$ and $\xi\ge 1$
 \[
 (a)_i := \E \left(\frac{(Y^{(n)}_{i+1}-\xi)^p(\xi-Y^{(n)}_i)}{Y^{(n)}_{i+1}-Y^{(n)}_i}\mbox{\bf 1}_{\{Y^{(n)}_i< \xi\le Y^{(n)}_{i+1}\}}\right),\,
% \]
% \[
 (b)_i := \E \left(\frac{(Y^{(n)}_{i+1}-\xi)(\xi-Y^{(n)}_i)^p}{Y^{(n)}_{i+1}-Y^{(n)}_i}\mbox{\bf 1}_{\{Y^{(n)}_i< \xi\le Y^{(n)}_{i+1}\}}\right)
 \]
 and $\displaystyle (c)_{\widetilde n-1}:= \E\Big(\big ( \xi-Y^{(n)}_{n-\ell+1}\big)^p\mbox{\bf 1}_{\{\xi\ge Y^{(n)}_{n-\ell+1}\}} \Big)$.
 
 We will first inspect the sum $\sum_{i=0}^{n-\ell} (\Box)_i$, $\Box=a,b$   successively.
 %on the one hand and the terms $(\Box)_0$, $x=a,b$, separetely.
 
 Let $i\!\in \{1, \ldots, \tilde n-1\}$. It follows from the above expression of the distribution of $(Y^{(n)}_i,Y^{(n)}_{i+1})$ that
\[
 (a)_i = \delta^2 \int\!\!\int_{1\le u\le \xi\le v} \frac{(v-\xi)^p(\xi-u)}{v-u}(1-u^{-\delta})^{i-1}v^{-\delta(n-i-1)}(uv)^{-\delta-1} \,du\,dv \frac{n!}{(i-1)!(n-i-1)!}.
\]
The change of variable $v= \xi(w+1)$   yields
\[
(a)_i =  n(n-1)\left(\begin{smallmatrix} n-2\\i-1\end{smallmatrix}\right)\delta^2 \int_1^{\xi} \hskip -0.15cm du\, (\xi-u)(1-u^{-\delta})^{i-1} u^{-\delta-1} \xi^{p-\delta(n-i)} \hskip -0.15cm  \int_{0}^{+\infty} \hskip -0.15cm  dw\,  \frac{w^p}{\xi(w+1)-u}(w+1)^{-\delta(n-i)-1}.
\]
Noting that $\frac{\xi-u}{\xi(w+1)-u}\le \frac{1}{w+1}$ then leads to
\[
(a)_i \le   n(n-1)\left(\begin{smallmatrix} n-2\\i-1\end{smallmatrix}\right)\delta^2n(n-1)\xi^{p-\delta (n-i)}\int_1^{\xi} \!(1-u^{-\delta})^{i-1} u^{-\delta-1}du \,\times \int _0^{+\infty} \! w^p(1+w)^{-\delta(n-i)-2}dw .
\]
The change of variable $w=\frac{1}{y}-1$ shows that $\displaystyle  \int _0^{+\infty} w^p(1+w)^{-\delta(n-i)-2}dw  = B(\delta(n-i)-p+1, p+1)$ whereas $\displaystyle \int_1^{\xi}  (1-u^{-\delta})^{i-1} u^{-\delta-1} du = \frac{(1-\xi^{-\delta})^i}{\delta i}$ so that
\[
(a)_i \le  \delta n \left(\begin{smallmatrix} n-1\\i\end{smallmatrix}\right) (1-\xi^{-\delta})^i \xi^{p-\delta (n-i)} \frac{\Gamma(p+1)\Gamma(\delta(n-i)-p+1)}{\Gamma(\delta (n-i)+2)}
\]
where we used the standard  identity 
%$B(a,b) = \frac{\Gamma(a)\Gamma(b)}{\Gamma(a+b)}$ and 
$ \left(\begin{smallmatrix} n-1\\i\end{smallmatrix}\right)= \frac{n-1}{i}\left(\begin{smallmatrix} n-2\\i-1\end{smallmatrix}\right)$. 

When $i=0$, noting that the density of $Y^{(n)}_1= \min_{1\le i\le n}Y_i$ is
$\delta n y^{-\delta n-1}\mbox{\bf 1}_{\{y\ge 1\}}$, we get
 \begin{eqnarray*}
(a)_0&= &\E \left(\frac{(Y^{(n)}_1-\xi)^p(\xi-1)}{Y^{(n)}_1-1}\mbox{\bf 1}_{\{1\le \xi\le Y^{(n)}_1\}}\right) \\
%\mbox{\bf 1}_{\{Y^(n)}_1\ge \xi\}} \\
&=& \delta n \int_{\xi }^{+\infty} (\xi-1)\frac{(v-\xi )^p}{v-1} v^{-\delta n-1} dv\\
&=& \delta n \xi^{p-\delta n}\int_0^{+\infty}\frac{(\xi-1)}{\xi(w+1)-1}  w^p (w+1)^{-\delta n-1} dw \quad \mbox{where we set  $v=\xi (w+1)$}  \\
&\le &   \delta n \xi^{p-\delta n} B(\delta n -p+1,p+1)
  \end{eqnarray*}
where we used in the last line that $\frac{\xi-1}{\xi(w+1)-1} \le \frac{1}{w+1}$.   As a consequence 
 \begin{eqnarray*}
\sum_{i=0}^{n-\ell} (a)_i &\le &  \delta \,n\, \Gamma(p+1) \sum_{i=0}^{n-\ell} \left(\begin{smallmatrix} n-1\\ i \end{smallmatrix}\right)\xi^{p-\delta(n-i)} (1-\xi^{-\delta})^i\frac{\Gamma(\delta(n-i)-p+1}{\Gamma(\delta(n-i)+2}\\
&\le  &\delta\, n\, \Gamma(p+1)\xi^p (1-\xi^{-\delta})^n\sum_{j=\ell}^n \left(\begin{smallmatrix} n-1\\ j -1\end{smallmatrix}\right)(\xi^{\delta}-1)^{-j} \frac{\Gamma(\delta j-p+1)}{\Gamma(\delta j +2)}.
\end{eqnarray*}
Now using that for every $a>0$, $\frac{\Gamma(x+a)}{\Gamma(x)}\sim x^a$ as $x\to \infty$, we derive the existence of a real constants $\tilde \kappa^{(0)}_{p,\delta}, \,\kappa^{(0)}_{p,\delta}>0$ such that
\[
\forall\, j\ge 0,\quad  \frac{\Gamma(\delta j-p+1)}{\Gamma(\delta j +2)} \le \tilde \kappa^{(0)}_{p,\delta}\, j^{-(p+1)}\le  \kappa^{(0)}_{p,\delta} \, \frac{j^{\lceil p\rceil -p}}{j(j+1)\cdots(j+\lceil p\rceil )}.
\]
In turn, using that
\[
\left(\begin{smallmatrix} n+\lceil p\rceil \\ j+\lceil p\rceil  \end{smallmatrix}\right) = \frac{(n+\lceil p\rceil )\cdots n}{(j+\lceil p\rceil ) \cdots j} \left(\begin{smallmatrix} n-1\\ j-1 \end{smallmatrix}\right),
\]
we finally obtain
 \begin{eqnarray*}
\sum_{i=0}^{n-\ell} (a)_i& \le& \kappa^{(0)}_{p,\delta}\,  n \Gamma(p+1)\xi^p
\delta  (1-\xi^{-\delta})^n \frac{1}{(n+\lceil p\rceil )\cdots
(n+1)n}\sum_{j=\ell}^{n} \left(\begin{smallmatrix} n+\lceil p\rceil \\ j+\lceil p\rceil  \end{smallmatrix}\right)(\xi^{\delta}-1)^{-j}j^{\lceil p\rceil -p}\\
&\le&  \kappa^{(0)}_{p,\delta} \, \Gamma(p+1)\xi^p \delta  (1-\xi^{-\delta})^n \frac{n^{\lceil p\rceil -p}}{(n+\lceil p\rceil )\cdots (n+1)}(\xi^{\delta}-1)^{\lceil p\rceil }\Big(1+(\xi^{\delta}-1)^{-1}\Big)^{n+\lceil p\rceil }.  
\end{eqnarray*}
Now 
\[
 (1-\xi^{-\delta})^n \xi^p(\xi^{\delta}-1)^{\lceil p\rceil} \Big(1+(\xi^{\delta}-1)^{-1}\Big)^{n+\lceil p\rceil }= \xi^{p + \delta\lceil p\rceil}
\]
%\textcolor{red}{I'm getting here 
%\[
% (1-\xi^{-\delta})^n \xi^p(\xi^{\delta}-1)^{\lceil p \rceil}
% \Big(1+(\xi^{\delta}-1)^{-1}\Big)^{n+\lceil p\rceil }= \xi^{p + \delta\lceil p\rceil}
%\]
%which is also works out\ldots??
%}
%
so that, using that $\xi\ge 1$ and $ \delta <\frac{\eta}{\lceil p\rceil}$,  we get $\xi^{p + \delta\lceil p\rceil}
\le  \xi^{p+\eta}$ which in turn implies   
\[
\sum_{i=0}^{n-\ell} (a)_i  \le  \kappa^{(0)}_{p,\delta}  \, \delta\,   \Gamma(p+1) \xi^{p+\eta} \frac{1}{n^p}.
\]

%
%\smallskip
Let us pass now to the second sum involving $(b)_i$.  First note that, on the event $\displaystyle\Big \{Y^{(n)}_i \le \xi\le \frac{Y^{(n)}_i+Y^{(n)}_{i+1}}{2}\Big\}$ (which is clearly included in $\big\{Y^{(n)}_i \le \xi\le Y^{(n)}_{i+1}\big\}$), one has $(\xi-Y^{(n)}_i)^p(Y^{(n)}_{i+1}-\xi)\le (\xi-Y^{(n)}_i)(Y^{(n)}_{i+1}-\xi)^p$
 so that, owing to what precedes, we can focus on $\displaystyle \sum_{i=0}^{n-\ell} (\widetilde b)_i$ where 
 \[
  (\widetilde b)_i:=\E \left((\xi-Y^{(n)}_i)^p\mbox{\bf 1}_{\big\{  \frac{Y^{(n)}_i+Y^{(n)}_{i+1}}{2}\le \xi\le Y^{(n)}_{i+1}\big\}}\right)\ge \E \left(\frac{(Y^{(n)}_{i+1}-\xi)(\xi-Y^{(n)}_i)^p}{Y^{(n)}_{i+1}-Y^{(n)}_i}\mbox{\bf 1}_{\big\{  \frac{Y^{(n)}_i+Y^{(n)}_{i+1}}{2}\le \xi\le Y^{(n)}_{i+1}\big\}}\right).
 \]
 
This time we will analyze  successively the sum over $i=1,\ldots,n-\ell$ and the case $i=0$.
% It is clear that
\begin{eqnarray*}
\sum_{i=1}^{n-\ell} (\widetilde b)_i&=& \delta^2 n(n-1)
\int\!\!\int_{\{1\le u\le \xi\le v\le 2\xi-u\}}\hskip -0.5 cm du\,dv\,
(uv)^{-\delta-1}(\xi-u)^p \sum_{i=1}^{n-\ell} \left(\begin{smallmatrix} n-2\\ i-1
\end{smallmatrix}\right)v^{-\delta(n-2-(i-1))} (1-u^{-\delta})^{i-1}
\\
&\le& \delta^2 n(n-1) \int\!\!\int_{\{1\le u\le \xi\le v\le
2\xi-u\}}du\,dv(uv)^{-\delta-1}(\xi-u)^p(1-u^{-\delta} +v^{-\delta})^{n-2}\\
&\le &  \delta^2 n(n-1) \int_1^{\xi} du u^{-\delta-1}(\xi-u) ^p  \int_{ 
\xi}^{2\xi-u} dv\, v^{-\delta-1}e^{-(n-2)(u^{-\delta}-v^{-\delta})}\\
&=&  \delta^2 n(n-1) \int_1^{\xi} du u^{-\delta-1}(\xi-u) ^p
e^{-(n-2)u^{-\delta}} \int_{  \xi}^{2\xi-u} dv\, v^{-\delta-1}e^{(n-2)v^{-\delta}}
\end{eqnarray*}
where we used in the  in the second line that $n-\ell-1\le n-2$ since $\ell\ge 1$. Setting $v= y^{-\frac{1}{\delta}}$ yields
%\textcolor{blue}{
\begin{eqnarray*}
 \int_{  \xi}^{2\xi-u}  v^{-\delta-1}e^{(n-2)v^{-\delta}}\, dv&=& \frac{1}{\delta} \int_{(2\xi-u)^{-\delta}}^{\xi^{-\delta}}e^{(n-2)y}dy\\
 &\le& \frac{1}{\delta} \big(\xi^{-\delta} -(2\xi-u)^{-\delta}\big)e^{(n-2)\xi^{-\delta}}\\
 &\le& (\xi-u) \xi^{-\delta-1}e^{(n-2)\xi^{-\delta}}
\end{eqnarray*}
where we used in the last line  the fundamental formula of Calculus. Consequently, 
%using that $v^{-\delta-1}\le \xi^{-\delta-1}$, we get
\begin{eqnarray*}
\sum_{i=1}^{n-\ell} (\widetilde b)_i&\le&  n(n-1) \delta^2 \xi^{-\delta-1}\int_1^{\xi} u^{-\delta-1} (\xi-u)^{p+1} e^{-(n-2)(u^{-\delta}-\xi^{-\delta})}du\\
&=& n(n-1) \xi^{-\delta-1} \delta \int_0^{(n-2)(1-\xi^{-\delta})}  \Big(\xi-\big(\frac{x}{n-2}+\xi^{-\delta}\big)^{-\frac {1}{\delta}}\Big)^{p+1} e^{-x}\frac{dx}{n-2}
\end{eqnarray*}
where we   put $u=\big(\frac{x}{n-2}+\xi^{-\delta}\big)^{-\frac{1}{\delta}}$.
Now, applying again fundamental formula of Calculus to the function
$z^{-\frac{1}{\delta}}$ yields,
\[
\xi-\Big(\frac{x}{n-2}+\xi^{-\delta}\Big)^{-\frac {1}{\delta}} = (\xi^{-\delta})^{-\frac{1}{\delta}}-\Big(\frac{x}{n-2}+\xi^{-\delta}\Big)^{-\frac {1}{\delta}}\le \frac{x}{\delta(n-2)}\xi^{\delta+1}
\]
%}
%\textcolor{red}{here I could not follow. Also don't see where $v^{-\delta-1}\le
%\xi^{-\delta-1}$ is used} 
\begin{eqnarray*}
\mbox{so that  }\quad \sum_{i=1}^{n-\ell} (\widetilde b)_i & \le&   \frac{n(n-1)}{(n-2)^{p+2}} \delta^{-p} \xi^{(p+1)(\delta+1)-(\delta+1)} \int_0^{(n-2)(1-\xi^{-\delta})} x^{p+1} e^{-x}dx \\
&\le & \kappa^{(1)}_{p,\delta} \Gamma(p+2) n^{-p} \xi^{p(\delta+1)}
\end{eqnarray*}
for some constant $\kappa^{(1)}_{p,\delta} > 0$.

When $i=0$, keeping in mind that $Y^{(n)}_1 = \min_{1\le i\le n}Y_i$, 
\begin{eqnarray*}
(\widetilde b)_0& \le& (\xi-1)^p \Prob(\xi \le Y^{(n)}_1\le 2\xi-1)= (\xi-1)^p \big(\xi^{-n\delta}-(2\xi-1)^{-n\delta}\big) \\
&\le& n\delta (\xi-1)^{p+1} \xi^{-n\delta-1}=  n\delta \xi^{p(1+\delta)} g(1/\xi)
\end{eqnarray*}
where $g(u) = (1-u)^{p+1}u^{(n+p)\delta}$, $u\!\in (0,1)$. One checks that $g$ attains its maximum over $(0,1]$ at $u^*= \frac{(n+p)\delta}{(n+p)\delta +p+1}$
so that
\[
\sup_{u\in (0,1]}g(u) = g(u^*)= \left(\frac{p+1}{(n+p)\delta +p+1}\right)^{p+1}(u^*)^{(n+p)\delta} \le  \left(\frac{1}{1+\frac{n+p}{p+1}\delta }\right)^{p+1}.
\]
Finally, there exists a real constant $\kappa^{(2)}_{p,\delta}>0$ such that 
\[
(\widetilde b)_0 \le \xi^{p(\delta+1)} \frac{\delta n}{(1+\frac{n+p}{p+1}\delta)^{p+1}}\le \kappa^{(2)}_{p,\delta}  \xi^{p(\delta+1)}  n^{-p}.
\]

As concerns the $(c)_{n-\ell+1}$ term, we proceed as follows. 
\begin{eqnarray*}
\E\Big(\big ( \xi-Y^{(n)}_{n-\ell+1}\big)^p\mbox{\bf 1}_{\{\xi \ge Y^{(n)}_{n-\ell+1}\}} \Big)&\le&  \xi ^p \Prob(\xi \ge Y^{(n)}_{n-\ell+1})\\
&\le& \xi^{p(1+\delta)}  \E (Y^{(n)}_{n-\ell+1})^{-p\delta}.
\end{eqnarray*}
Note that 
\begin{eqnarray*}
\E (Y^{(n)}_{n-\ell+1})^{-p\delta} &=&\frac{\Gamma(n+1)}{\Gamma(n-\ell+1)\Gamma(\ell)}\int_0^1 (1-v)^{n-\ell}v^{\ell+p-1}dv =\frac{\Gamma(n+1)}{\Gamma(\ell)} \frac{\Gamma(\ell+p)  }{\Gamma(n+p+1)}\\
&\sim& \frac{\Gamma(\ell+p)}{\Gamma(\ell)}n^{-p} = O(n^{-p}).
\end{eqnarray*}
%since $\frac{\eta}{\delta}>r$. 
Finally, for every $ \xi \ge 1$, 
\[
(c)_{n-\ell+1}\le \kappa_{p,\delta}^{(3)}\xi^{p(1+\delta)}n^{-p}.
\]
Consequently, there exists a real constant  $\kappa_{p,\eta} =
\max_{j=0,\ldots,3} \kappa^{(j)}_{p, \delta}>0$ such that for
every $n\ge n_{p,\eta}= \ell(\eta,p)\vee 3$,
\[
 \forall\, \xi \ge 1,\qquad n^p \,\E\,
 \bar F^p_{n}(\xi, Y^{(n)}_0,\ldots,Y^{(n)}_{\widetilde n+1})\le
 \kappa_{p,\eta} \,\xi^{p+\eta}
\]
since $p\,\delta \le \eta$. Hence for every r.v.  $X$, we derive by integrating in $\xi\!\in [1,+\infty)$ with respect to $\Prob_{_X}(d\xi)$:
\[
n^p \inf_{(1,x_2,\ldots,x_n)\in {\cal I}_n}\E\, \bar F^p_{n}(X,1,x_2,\ldots,x_n)\le  n^p \E\, \bar F^p_{n}(X,Y^{(n)}_0,\ldots,Y^{(n)}_{\widetilde n+1})\le \kappa_{p,\eta}\,\E \,X^{p+\eta}.
\]
%This completes the proof of Step~2 since $p\delta\le \eta$.



 %\medskip 
 \noindent {\sc Step~3.} If $X$ is a non-negative random variable, applying the second step to $X+1$ and using the scaling property $(i)$ satisfied by $F_{p,n}$ yields for $n\ge  n_{p,\eta}$ (as defined in Step~2), 
\begin{eqnarray*} 
 \inf_{ (0,x_2,\ldots,x_n)\in {\cal I}_n}\|\bar F_{p,n}(X,0,x_2,\ldots,x_n)\|_{L^p} &=& \inf_{ (1,x_2,\ldots,x_n)\in {\cal I}_n}\|\bar F_{p,n}(X+1,1,\ldots,x_n)\|_{L^p} \\
 &\le&  \kappa^{1/p}_{p,\eta}\frac{\|1+X\|_{L^{p+\eta}}^{1+\frac{\eta}{p}}}{n}\\
 &\le&  C^{(0)}_{p,\eta}\frac{(1+ \|X\|_{L^{p+\eta}}^{1+\frac{\eta}{p}})}{n}\; \mbox{ with } C^{(0)}_{p,\eta}= (2^{1+\eta}\kappa_{p,\eta})^{\frac 1p}.
 \end{eqnarray*}
 We may assume that $\|X\|_{L^{p+\eta}}\!\in(0,\infty)$. Then, applying the above bound to the non-negative random variable $\widetilde X= \frac{X}{\|X\|_{L^{p+\eta}}}$ taking again advantage of the scaling property $(i)$, we obtain
 \begin{eqnarray*}
  \inf_{ (0,x_2,\ldots,x_n)\in {\cal I}_n}\|\bar
  F_{p,n}(X,0,x_2,\ldots,x_n)\|_{L^p}&=&  \|X\|_{L^{p+\eta}} \inf_{
  (0,x_2,\ldots,x_n)\in {\cal I}_n}\|\bar F_{p,n}(\widetilde
  X,0,x_2,\ldots,x_n)\|_{L^p}\\  &\le & \|X\|_{L^{p+\eta}}
  C^{(0)}_{p,\eta}\frac{1+1}{n} = 2C^{(0)}_{p,\eta}
  \,\|X\|_{L^{p+\eta}} \frac{1}{n}.
 \end{eqnarray*}
%
%Furthermore, since $X\ge0$, it is obvious from the definition of $\Phi_n$ that 
%\[
% \inf_{ (x_1,\ldots,x_n)\in {\cal I}_n}\|X-\Phi_n(.,x_1,\ldots,x_n,X)\|_{L^p} =   \inf_{ (x_1,\ldots,x_n)\in {\cal I}_n\cap \R_+^n}\|X-\Phi_n(.,x_1,\ldots,x_n,X)\|_{L^p}
% \]
 
  
%\medskip 
 \noindent {\sc Step~4.} Let $X$ be a real-valued random variable  and let for every integer $n\ge 1$, $x_1,\ldots,x_n\!\in (-\infty,0)$,  $x_{n+1}=0$  and $x_{n+2},\ldots,x_{2n+1}\!\in (0,+\infty)$. It follows from the additivity property that  that 
\begin{eqnarray*}  
 \nonumber \bar F^p_{2n+1}(X, x_1,\ldots,x_{2n+1}) &=&\bar F^{p}_{n+1}(X_+, x_{n+1},\ldots,x_{2n+1})\mbox{\bf 1}_{\{X\ge 0\}}  \\
 &&+  \bar F^{p}_{n+1}(-X_-, x_{1},\ldots,x_{n+1})\mbox{\bf 1}_{\{X<0\}} \\
\label{IneqA} &=& \bar F^{p}_{n+1}(X_+, x_1,\ldots,x_{n+1})\mbox{\bf 1}_{\{X
\ge 0\}}   +  \bar F^{p}_{n+1}(X_-, -x_{n+1},\ldots,-x_{1})\mbox{\bf 1}_{\{X<0\}}.
 \end{eqnarray*} 
% so that, using again the scaling property
% \[
%| X-\Phi_{2n+1}(.,x_1,\ldots,x_{2n+1},X) |^p\le  \Big|X_+-\Phi_{2n+1}(.,x_1,\ldots,x_{2n+1},X_+)  \Big|^p+  \Big|X_- -\Phi_{2n+1}(.,-x_{2n+1},\ldots,-x_1 X_-)  \Big|^p 
%\]
%As a consequence of Inequality~(\ref{bornes}), we get
%\begin{eqnarray*}  
%\E   \Big|X_+-\Phi_{2n+1}(.,x_1,\ldots,x_{2n+1},X_+)  \Big|^p \!\!&\!\!\le\!\!&\!\! \sum_{i=1}^{2n}(x_{i+1}-x_i)^p \Prob(X_+\in[x_i,x_{i+1}))+\E\Big(\!\big(X_+-x_{2n+1}\big)^p\mbox{\bf 1}_{\{X_+\ge x_{2n+1}\}}\Big)\\
%\!\!&\!\!=\!\!&\!\!  \underbrace{ \sum_{i=n+1}^{2n+1}(x_{i+1}-x_i)^p \Prob(X_+\in[x_i,x_{i+1}))+\E\Big(\!\big(X_+-x_{2n+1}\big)^p\mbox{\bf 1}_{\{X_+\ge x_{2n+1}\}}\Big)}_{=:A_{X_+}(x_{n+1},\ldots,x_{2n+1})}
% \end{eqnarray*} 
% and likewise 
% \[
% \E \Big|X_- -\Phi_{2n+1}(.,x_1,\ldots,x_{2n+1}, X_-)  \Big|^p \le \underbrace{\sum_{j=1}^n (\tilde x_{i+1}-\tilde x_i)^p \Prob(X_-\in[\tilde x_i,\tilde x_{i+1}))+\E\Big(\!\big(X_--\tilde x_{n+1}\big)^p\mbox{\bf 1}_{\{X_-\ge \tilde  x_{n+1}\}}\Big)}_{=:A_{X_-}(\tilde x_{1},\ldots,\tilde x_{n+1})}
% \]
% where $\tilde x_j= -x_{n+2-j}$, $j=1,\ldots,n+1$. 
Consequently, using that $X_+\times X_- \equiv 0$ and that $x_{n+1}=0$, we get 
%using that 
%$ (u+v)^{\frac 1p}\le u^{\frac 1p}+v^{\frac 1p}$, $u$, $v\ge 0 $,
\begin{eqnarray*}  
  \!\!\!\!\inf_{\substack{(x_1,\ldots,x_{2n+1})\in {\cal
  I}_{2n+1}\\ x _{n+1}=0}}\|\bar F_{p,2n+1}(X,
  x_1,\ldots,x_{2n+1})\|^p_{L^p} \!\!&\!\!\le\!\!&\!\!  \inf_{ (0,y_2,\ldots,y_{n+1})\in {\cal I}_{n+1}} \|\bar F_{p,n+1}(X_+,0,y_2,\ldots,y_{n+1}) \|^p_{L^p} \\
  &&+ \inf_{ (0,y_2,\ldots,y_{n+1})\in {\cal I}_{n+1}}\|\bar F_{p,n}(X_-,0,y_2,\ldots,y_{n+1}) \|^p_{L^p}  .
  \end{eqnarray*} 
   Hence,  it follows from Step~2 that, for every  $n\ge n_{p,\eta}-1$,
\begin{eqnarray*}  
 \inf_{ (x_1,\ldots,x_{2n+1})\in {\cal I}_{2n+1}}\|\bar
 F_{p,2n+1}(X,x_1,\ldots,x_{2n+1})\|^p_{L^p}   &\le & \Big(
 \|X_-\|^p_{L^{p+\eta}} +\|X_+\|^p_{L^{p+\eta}}
 \Big)\Big(\frac{2C^{(0)}_{p,\eta} }{n+1}\Big)^p.
 %,\\
 % &=& C'_{p,\eta}  \|X\|_{L^{p+\eta}}  \frac{1}{n+1}
 \end{eqnarray*} 
 Now using that $(a+b)\le2^{1-\frac 1q} (a^q+b^q)^{\frac 1q}$, $a,b\ge 0$, with $q=1+\frac{\eta}{p}\ge 1$, we derive that
 \[
  \|X_-\|^p_{L^{p+\eta}} +\|X_+\|^p_{L^{p+\eta}} \le 2^{\frac{\eta}{p+\eta}} \Big(\|X_-\|^{p+\eta}_{L^{p+\eta}} +\|X_+\|^{p+\eta}_{L^{p+\eta}} \Big)^{\frac{p}{p+\eta}}= 2^{\frac{\eta}{p+\eta}}  \|X\|_{L^{p+\eta}}^p
 \]
%where we used that $ \|X\|^{p+\eta}_{L^{p+\eta}} =  \|X_-\|_{L^{p+\eta}} +\|X_+\|_{L^{p+\eta}}  $.
 since $X_-\times X_+\equiv 0$. 
Now, the monotonicity property~(\ref{monotonie1})  implies that, for every $n\ge 2\,n_{p,\eta}$,
 \[
 \bar d_{n,p}(X)=  \inf_{ (x_1,\ldots,x_{n})\in {\cal I}_{n}}\|\bar F_{p,n}(X,x_1,\ldots,x_{n})\|_{L^p}\le 2^{\frac{\eta}{p(p+\eta)}}2C^{(0)}_{p,\eta}\frac{\|X\|_{L^{p+\eta}} }{n}.
 \]
Still calling upon~(\ref{monotonie1}), we note that, for every $n\!\in\{1,\ldots,2n_{p,\eta}\}$, $\bar d_{n,p}(X)\le \bar d_{1,p}(X) = \inf_{x \in \R}\|X-x_1\|_{L^p}\le \|X\|_{L^p}$ so that
\[
\bar d_{n,p}(X)\le 2n_{p,\eta} \frac{ \|X\|_{L^{p+\eta}}}{n}
\]
which completes the proof by setting $C_{p,\eta} = \max\big(2n_{p,\eta},2^{1+\frac{\eta}{p(p+\eta)}}C^{(0)}_{p,\eta}\big)$.
% If $p\!\in (0,1)$, one obtains using directly~(\ref{IneqA}) that 
%  \[
%  \inf_{ (x_1,\ldots,x_{2n+1})\in {\cal I}_{2n+1}}\|F_{p,2n+1}(x_1,\ldots,x_{2n+1},X)\|^p_{L^p} \le (C'_{p,\eta})^p  \Big( \|X_-\|^p_{p+\eta} +\|X_+\|^p_{p+\eta} \Big)\frac{1}{(n+1)^p}.
%   \]
% Now  
% $$ 
% \|X_-\|^p_{p+\eta} +\|X_+\|^p_{L^{p+\eta}}\le  (\|X_-\|^{p+\eta}_{L^{p+\eta}} +\|X_+\|^{p+\eta}_{L^p{p+\eta}} )^{\frac{p}{p+\eta}}= \| X\|_{L^{p+\eta}}^p
% $$ 
% so that the  conclusion remains the same.
%
%\medskip 
% \noindent \textcolor{blue}{{\sc Step~5.} \textcolor{red}{ I do not know how to proceed}If $X$ is compactly supported, we may assume up to a translation that $\conv({\rm supp}(\Prob_{_X}))= [0,L]$, $L\ge 0$  without changing $d_{n,p}(X)$. Then, for every $x\!\in {\cal I}_n$,  $F_n^p(\xi,x_1,\ldots,x_n)\le \bar F^p_n(\xi,x_1,\ldots,x_n)<+\infty$ if $\xi \!\in [0,x_n]$  (and $=+\infty $ otherwise). consequently
%  \[
% \|F_p(X,x_1,\ldots,x_n)\|_{L^p}
% \]
% }
% This completes the proof. 
 \end{proof}
 
 
 
 
 
 
 
  %\section{Application to the rate of convergence}
 \subsection{A $d$-dimensional non-asymptotic upper-bound for the dual
 quantization error}\label{sec:ddimBound}
 
% \begin{lemma} \label{productdq} (Product grids) Let $d\ge1$ and let
% $p\!\in[1,+\infty)$. Then there exists a real constant $C_{d,p,|.|}>0$ such that for every product dual quantizer $\gamma=\prod_{\ell=1}^d \gamma^{(\ell)}$, $\gamma^{(\ell)}= (x^{\ell}_1,\ldots,x^{\ell}_{n_{\ell}})\!\in {\cal I}_{n_{\ell}}$, $\ell=1,\ldots,d$,
% \[
% \forall\, \xi\!\in \R^d,\quad 
% \bar F_{n,p}(\xi, \gamma) \le C_{d,p,|.|} \sum_{\ell=1}^d \bar F_{n,p}(\xi^{\ell}, \gamma^{(\ell)})
% \]
% Furthermore, if $|\,.\,|=|\,.\,|_{\ell^p}$  is the canonical $\ell^p$-norm on $\R^d$ and $\xi \!\in {\rm Conv} \gamma$ (hypercube), then the inequality stands as an equality with $C_{d,p,|.|}=1$.
% \end{lemma}
% 
% 
% \begin{proof} All norms being equivalent on $\R^d$ there exists a positive real
% constant $K_{d,r,|.|}$ such that $|\,.\,|\le \kappa_{d,r,|.|}|\,.\,|_{\ell^p}$ where $|\,\xi \,|_{\ell^p}= \Big(\sum_{1\le \ell\le d} |\xi^{\ell}|^p\Big)^{\frac 1p}$ denotes the canonical $\ell^p$-norm on $\R^d$. So we may assume without loss of generality that the norm $|\,.\,|$ is the $\ell^p$-norm.  Moreover, note that we may describe the  value set of the product quantizer as follows
% \[
%\Gamma_{\gamma}= \Big\{x_{\underline i}=(x^{\ell}_{i_{\ell}})_{1\le \ell\le d},\; \underline{i}=(i_1,\ldots,i_d)\!\in I\Big\}, \;I:= \Prod_{\ell=1}^d \{1,\ldots,n_{\ell}\}.
% \]
% 
% 
% Let $\xi =(\xi^1,\ldots,\xi^d)\!\in {\rm Conv}(\gamma)$. Then, $\bar F_{n,p}(\xi)= F_{n,p}(\xi)$ and using the dual form~(\ref{dualF}) for  the function $F_{n,p}$ we get
% \begin{eqnarray*}
% \bar F_{n,p}(\xi,\gamma)&\le &  \max_{u\in \R^d} \min_{\underline{i}\in I}|\xi-x_{\underline i}|_{\ell^p}^p+(u|\xi-x_{\underline i})\\
% &=& \max_{u\in \R^d}\sum_{\ell=1}^d |\xi^{\ell}-x^{\ell}_{i_{\ell}}|_{\ell^p}^p +u^{\ell}(\xi^{\ell}-x^{\ell}_{i_{\ell}})\\
% &=& \sum_{\ell=1}^d \max_{u\in \R}|\xi^{\ell}-x^{\ell}_{i_{\ell}}|_{\ell^p}^p +u(\xi^{\ell}-x^{\ell}_{i_{\ell}})\\
%&=& \sum_{\ell=1}^d \bar F_{n_{\ell},p}(\xi^{\ell}, \gamma^{(\ell)}).
% \end{eqnarray*}
% 
% 
% If $\xi \! \notin {\rm Conv}(\Gamma_{\gamma})$, then
% \begin{eqnarray*}
% \bar F_{n,p}(\xi, \gamma) &=& \min_{\underline i\in I} |\xi-x_{\underline i}|^p\\
% &=&  \sum_{\ell=1}^d \min_{1\le i_{\ell}\le n_{\ell}}|\xi-x_{i_{\ell}}|^p\\
% &\le& \sum_{\ell=1}^d  \bar F_{n_{\ell},p} (\xi^{\ell}, \gamma^{(\ell)}).%\cqfd
% \end{eqnarray*}
% \end{proof}
% 
%% \subsection{The one-dimensional setting}
Now, combining Theorem~\ref{Pierce0} and Proposition~\ref{prop:rappels}$(b)$, we are in position to show Proposition~\ref{PdtQErrop} (the
$d$-dimensional version of the extended Pierce Lemma)  which provides a non-asymptotic upper-bound at the exact rate for dual quantization error moduli.

% \begin{prop}[$d$-dimensional extended Pierce Lemma] \label{PdtQErrop} Let
% $p,\,\eta>0$. There exists an integer $n_{d,p,\eta}\ge 1$ and a real constant
% $C_{d, p,\eta}$ such that, for every $n\ge n_{p,\eta}$ and every random
% variable $X\!\in L_{\R^d}^{p+\eta}(\Omega_0,{\cal A}, \Prob)$, 
% \[ 
% \bar d_{n,p}(X)\le C_{d,p,\eta}\sigma_{p+\eta,\|.\|}(X)\,n^{-1/d}. 
% \] 
% where $\sigma_{p+\eta,\|.\|}(X)= \inf_{a\in \R^d} \|X-a\|_{L^{p+\eta}}$. 
% 
% If $\supp(\ProbX)$ is
% compact then the same inequality holds true for $d_{n,p}(X)$.
%% [[Not so clear at a first glance except if $\Gamma^{ext}$ is itself a product
%% quantizer: If furthermore, its the convex hull of its support is spanned by a
%% finite grid of extrema points $\Gamma^{ext}$ then the ineqality is satisfied
%% by $d^{ext}_{n,p}(X)$ as well]].
% \end{prop}
 
\medskip
\noindent {\it Proof of Proposition~\ref{PdtQErrop}.} $(a)$ First note that $ \bar d_{n,p}(X)= \bar d_{n,p}(X-a)$, $a\!\in \R^d$ (invariance by translation)    so we may assume that $X$  is $L^{p+\eta}$-centered $i.e.$ $\sigma_{p+\eta,\|.\|}(X)= \|X\|_{L^{p+\eta}}$. When $d=1$, Theorem~\ref{Pierce0} solves the problem.

%\smallskip
%\noindent $\rhd$ $d=1$. In this one dimensional setting one may consider only
%ordered $n$-tuples $\gamma=(x_1,\ldots,x_n)$. One derives  from
%Theorem~\ref{Pierce0} and the example $(b)$  that follows that, for every $n\ge
%n_{p,\eta}$,
% \[ 
% \bar d_{n,p}(X)= \inf_{\gamma\in {\cal I}_n}\| X-\Phi_n^{dq}(U,
% \gamma)\|_{-p}\le C_{p,\eta}\|X\|_{L^{p+\eta}}n^{-1} 
% \] 
% where $U\sim U([0,1])$ is independent of $X$.
% 
% \medskip
% 
%\noindent $\rhd$  
Let $d\ge 2$. Let  $X=(X^1,\ldots,X^d)$ ($X^i$ components of $X$).   It follows form Proposition~\ref{prop:rappels}
 that, if $\Gamma=\prod_{1\le i \le d} \Gamma_i$, with $\Gamma_i\subset \R$, $|\Gamma_i|=n_i$ with $n_1\cdots n_d\le n$. Then for every $\xi=(\xi^1,\ldots,\xi^d)\!\in \R^d$
 \[
  \bar F^p_{\norm{.}}(\xi;\Gamma) \le C_{p, \norm{.}}\bar F^{p}_{\ell^p}(\xi;\Gamma) = \sum_{j=1}^d\bar F^p(\xi^j, \Gamma_j) 
 \]
where $C_{p,\norm{.}}\!=\! \sup_{|\xi|_{\ell^p}=1}\|\xi\|^p$. Integrating with respect to the distribution of $X$ yields  $\displaystyle \bar d^p (X,\Gamma)\!\le C_{p, \norm{.}} \sum_{j=1}^d \bar d^p(X^j, \Gamma_j)$ 
which in turn easily implies  
\[
\bar d_n^p(X)\le C_{p, \norm{.}} \sum_{j=1}^d \bar d_{n_j}^p(X^j). 
\]
Now set $n_j = \lfloor n^{\frac 1d}\rfloor$, $j=1,\ldots,d$. It follows from Theorem~\ref{Pierce0} that
% if $n\ge n_{d,p,\eta}^d= n_{p,\eta}^d\ge 2^d$, 
%
%Let $\gamma$ be an optimal quantizer of size $n_1\cdots n_d\le n$.
% Then if , one has if
% $\min_{\ell}n_{\ell}\ge n_{p,\eta}$ (from the one dimensional case) using
% Proposition~\ref{prop:product}
 \begin{eqnarray*}
 \bar  d_{n}^p(X) & \le &  C^p_{p, \norm{.}}C_{p,\eta}  \sum_{j=1}^d \|X^{j}\|^p_{L^{p+\eta}}\lfloor n^{\frac 1d}\rfloor^{-p} \\
 & \le &  C_{p, \norm{.}} C_{p,\eta} \sup_{k\ge 2}\Big(\frac{k^{\frac 1d}}{k^{\frac 1d}-1}\Big)^p n^{-\frac pd} \sum_{j=1}^d \|X^{j}\|^p_{L^{p+\eta}}\\
& \le &  C_{p, \norm{.}} C_{p,\eta} 2^p n^{-\frac pd}d^{\frac{\eta}{p+\eta}}\E |X|^{p+\eta}_{\ell^{p+\eta}}\\
&\le&  d^{\frac{\eta}{p+\eta}} C_{p, \norm{.}} C_{p,\eta} \,2^p \widetilde C_{\norm{.}, p+\eta}  \|X\|_{L^{p+\eta}}^{p+\eta}n^{-\frac pd}
 \end{eqnarray*}
 where $\widetilde C_{\norm{.}, r}=\sup_{\|x\|=1}|x |^r_{\ell^r}$, $r>0$. 
 
\smallskip
\noindent $(b)$ Let $C$ be  the smallest hypercube withe edges
parallel to the coordinate axis containing $\conv({\rm Supp}(\Prob_{_X}))$. Up
to a translatation, which  leaves $d_{n,p}(X)$ invariant, we may assume that $C=[0,L]^d$ where $0\le L\le {\rm diam}_{\norm{.}}({\rm Supp}(\Prob_{_X}))$. The conclusion follows by integrating Inequality~(\ref{eq:prodFp}) with respect to $\Prob_{_X}(d\xi)$ with $m=\lfloor n^{\frac 1d} \rfloor$ and following the lines of the proof of claim~$(a)$. $\qquad \Box$ 
% \end{proof}
 

 
