\section{Introduction}
Starting with~\cite{cdo} and continued in~\cite{dualStat}, we introduced a new notion of vector quantization called {\em
dual quantization} (or  {\em Delaunay quantization}  in a Euclidean framework). We developed in~\cite{dualAppl} some first applications towards the design of numerical schemes for multi-dimensional optimal stopping and stochastic control problems arising in Finance (see also~\cite{BCC}).
In general, the principle of dual quantization consists of mapping an $\R^d$-valued random vector 
(r.v.) onto a non-empty finite subset (or {\em grid}) $\Gamma\subset \R^d$ using an appropriate random
splitting operator ${\cal J}_{\Gamma} : \Omega_0\times \R^d\to \Gamma$ (defined
on an exogenous probability space $(\Omega_0,\mathcal{S}_0, \Prob_0)$)
which satisfies the intrinsic {\em stationarity property}
\begin{equation}\label{1a}
\forall\, \xi \!\in \conv(\Gamma),\qquad \E_{\Prob_0}({\cal J}_{\Gamma}
(\xi))=\int_{\Omega_0} {\cal J}_{\Gamma} (\omega_0,\xi)
\,\Prob_0(d\omega_0)=\xi,
\end{equation}
where $\conv(\Gamma)$ denotes the convex hull of $\Gamma$ in $\R^d$. Every r.v.  $X:(\Omega,\mathcal{S}, \Prob)\to \conv(\Gamma) $ defined on a probability space   can be  canonically
extended to $(\Omega_0\times\Omega,\mathcal{S}_0\otimes\mathcal{S},
\Prob_0\otimes\Prob)$ in order to define  {\em dual quantization} induced by $\Gamma$ as
\[
\widehat X^{\Gamma, dual}(\omega_0, \omega) =   {\cal J}_{\Gamma} (\omega_0,X(\omega)).
\]
As a specific feature inherited from~\eqref{1a}, it always satisfies the {\em dual} or {\em reverse stationary property}
 \[ 
 \E_{\Prob\otimes\Prob_0}({\cal J}_{\Gamma} (X)\,|\,
X)=X. 
\] 
This 
%means that the resulting approximation $\widehat X^{dual}$ of $X$ always satisfies a {\em reverse} stationarity 
can be compared to the more  classical Voronoi framework   where the $\Gamma$-quantization of $X$ is defined from a  Borel nearest neighbour projection ${\rm Proj}_{\Gamma}$ by
\[
\widehat X^{\Gamma, vor}( \omega) = {\rm Proj}_{\Gamma}(X(\omega)).
\]
The stationary property then reads: $\displaystyle \E(X\,|\, \widehat X^{\Gamma, vor})=\widehat X^{\Gamma, vor}$, except that it holds 
only for grids which are critical points (typically local minima) of the so-called distortion function (see $e.g.$~\cite{Foundations})
%for exactly optimal
%grids for the quadratic mean  (Voronoi )  quantization error (see below) and exclusively  
in a Euclidean framework.

 
%\smallskip
To each  quantization is corresponds  a functional approximation operator:    Voronoi quantization is related to the {\em stepwise constant functional  approximation operator}~$f \!\circ {\rm Proj}_{\Gamma}$ whereas dual quantization leads  to an operator defined for every $\xi\!\in\conv(\Gamma)$ by
\begin{equation}\label{eq:JGAmma}
\mathbb{J}_{\Gamma}(f)(\xi)= \E_{\Prob_0}\big(f(J_{\Gamma}(\omega_0,\xi))\big)= \sum_{x\in \Gamma} f(x)\lambda_x(\xi),
\end{equation} 
where $\lambda_x(\xi)=\Prob_0(J_{\Gamma}(.,\xi)=x)$, $x\!\in \Gamma$, are barycentric ``pseudo-coordinates" of $\xi$ in $\Gamma$ satisfying $\lambda_x(\xi)\!\in [0,1]$, $\sum_{x\in \Gamma} \lambda_x(\xi) =1$ and $\sum_{x\in \Gamma} \lambda_x(\xi) x=\xi$. The operator $\mathbb{J}_{\Gamma}$  is an {\em interpolation} operator which turns out,  under appropriate conditions, to be  more regular (continuous and stepwise affine, see~\cite{dualAppl}) than the ``Voronoi" one.
 It is shown  in~\cite{dualStat,cdo,dualAppl} how we can take advantage of this intrinsic stationary property to  produce more accurate error bounds for the resulting  {\em cubature formula}
 \begin{equation}\label{eq:cubature}
\E_{\Prob}(f(\widetilde X^{\Gamma,dual}))=  \E_{\Prob}(\mathbb{J}_{\Gamma}(f)(X))=   \E_{\Prob\otimes\Prob_0} \big(f(J_{\Gamma}(\omega_0,\xi))\big)= \sum_{x\in \Gamma}w_x^{dual}f(x) 
 \end{equation}
where $w_x^{dual} = \E_{\Prob}(\lambda_x(X))= \Prob\otimes\Prob_0(J_{\Gamma}(\omega_0,X)=x)$, $x\!\in \Gamma$, regardless of any optimality property  $\Gamma$ with respect to   $\Prob_{X}$.   Typically, if  $f\! \in {\rm Lip}(\R^d,\R)$ (Lipschitz continuous function)  with  coefficient~$[f]_{\rm Lip}$, 
\begin{eqnarray*}
  \Big|\E_{\Prob}f(X)-\E_{\Prob\otimes\Prob_0}f(\widetilde X^{\Gamma,dual})\Big|&\le& [f]_{\rm Lip}\big\|X-\widehat X^{\Gamma,dual}\big\|_{L^1(\Prob\otimes\Prob_0)}\\&=&  [f]_{\rm Lip}\E_{\Prob \otimes \Prob_0}\big(\|X-J_{\Gamma}(\omega_0,X)\| \big)   \\
  &=& [f]_{\rm Lip}\E_{\Prob \otimes \Prob_0}\big(\E_{\Prob \otimes \Prob_0}(\|X-J_{\Gamma}(\omega_0,X)\|  \,|\,X)  \big)   
\end{eqnarray*}
whereas, if $f$ has Lipschitz continuous differential (the norm on $\R^d$ is denoted $\|\,.\,\|$), a second order Taylor expansion yields
\begin{eqnarray}
\nonumber    \Big|\E_{\Prob}f(X)-\E_{\Prob\otimes\Prob_0}f(\widetilde X^{\Gamma,dual})\Big|&\le& \Big\|f(X) -\E_{\Prob \otimes \Prob_0}\big(f(J_{\Gamma}(\omega_0,X))\,|\,X\big) \Big\|_{L^1(\Prob\otimes\Prob_0)}    \\
\nonumber     &\le& [ Df]_{\rm Lip}\E_{\Prob \otimes \Prob_0}\big(\|X-J_{\Gamma}(\omega_0,X)\|^2  \big)  \\
  &\le& [ Df]_{\rm Lip}\E_{\Prob \otimes \Prob_0}\big(\E_{\Prob \otimes \Prob_0}(\|X-J_{\Gamma}(\omega_0,X)\|^2  \,|\,X)  \big)   \label{eq:fstatiodual}
\end{eqnarray}
where $\displaystyle \E_{\Prob\otimes\Prob_0}(\|X-J_{\Gamma}(\omega_0,X)\|^p\,|\,X)= \sum_{x\in\Gamma}\lambda_x(X)\|X-x\|^p= \mathbb{J}_{\Gamma}(\|.\|^2)(X),\; p=1,2$.

More generally, if one aims at approximating $\E\big (f(X)\,|\, g(Y) \big)$ by its dually quantized counterpart $\E_{\Prob\otimes\Prob_0\otimes\Prob_1}\big (f(J_{\Gamma_X}(\omega_0,X))\,|\, J_{\Gamma_Y}(\omega_1,Y) \big)$  (with obvious notations),  it is also possible under natural additional assumptions  to get  error bounds based on both related dual quantization error moduli, see e.g. the proof (Step~2) of ~Proposition~2.1 in~\cite{dualAppl}.


%\smallskip
This suggests  to investigate the properties and the asymptotic    behaviour  of the $(\Gamma, L^p)$-mean dual quantization error, $p\!\in (0,\infty)$, defined by
$$
\Big\|X-\widehat X^{\Gamma,dual}\Big\|^p_{L^p(\Prob\otimes\Prob_0)}= \Big\|X-J_{\Gamma}(\omega_0,X)\Big\|^p_{L^p(\Prob\otimes\Prob_0)}= \E_{\Prob\otimes\Prob_0} \Big( \E_{\Prob\otimes\Prob_0}\big(\|X-J_{\Gamma}(\omega_0,X)\|^p\,|\, X\big) \Big)
$$ 
so as to make it as small as possible. This program can be summed up in four phases:

\smallskip
-- The first step is to minimize the above conditional expectation, $i.e.$  $\E(\|\xi-J_{\Gamma}(\omega_0,\xi)\|^p)$ for every $\xi \!\in\conv(\Gamma)$, for a fixed grid $\Gamma$ $i.e.$ to determine the {\em best}  splitting random operator $J_{\Gamma}$. In a  regular quantization, this phase corresponds to showing that the nearest neighbour projection on $\Gamma$ is   the best projection on $\Gamma$.


\smallskip
-- The second step is ``optional" . It aims at finding grids which minimize the
mean dual quantization error $\Big\|X-J_{\Gamma}(\omega_0,X)\Big\|_{L^p(\Prob\otimes\Prob_0)}$ {\em among all grids $\Gamma$ whose convex hull contains the support of the distribution of $X$} or equivalently such that $\Prob(X\!\in \conv(\Gamma))=1$.

\smallskip
-- The third step is to extend  dual quantization to r.v.s $X$ with unbounded support while  the performances of the resulting  cubature formula (see~\eqref{eq:fstatiodual}), having in mind that  the stationarity can no longer holds.

\smallskip
The first two steps have  been already solved in~\cite{dualStat}. We discuss  in-depth the third one in Section~\ref{sec:motiv}). The aim of this paper is to solve the fourth and last  step:   elucidate  is the rate of  decay  to $0$ of the   {\em optimal} $L^p$-mean dual quantization error modulus, $i.e.$ minimized over all grids $\Gamma$ of size at most $N$ -- as $N$ grows to infinity.

This is the  to establish  in a dual
quantization framework the counterpart of   Zador's Theorem  which
rules the convergence rate of  optimal ``regular" (Voronoi) quantization and is
recalled below. 
To be more precise, we will  establish such a  theorem, for $L^{\infty}$-bounded r.v.s but also, 
once mean dual quantization error will have been extended in an appropriate way
following~\cite{dualStat}, to general r.v.s.


%, especially for calibration purpose. opens new promising perspective for
% model calibration since in such an procedure, the underlying structure model
% is modified at each optimization step.
\smallskip
Let us now introduce in more formal way   the (local and mean) dual quantization error moduli,  following~\cite{dualStat}.  For a grid $\Gamma\subset\R^d$, we define  the {\em $L^p$-mean dual quantization  error} of  $X$ induced by the grid $\Gamma $ by
\begin{equation}\label{def:dpGamma}
d_{p}(X; \Gamma)= \|F_p(X;\Gamma)\|_{L^p(\Prob)}
\end{equation}
where $F_p$ denotes the {\em  local dual quantization error} function  defined by
\begin{eqnarray}
\label{eq:Fp} F_p(\xi;\Gamma)&=&\inf\left \{  \Big(\sum_{x\in \Gamma} \lambda_x \|\xi-x\|^p
\Big)^{\frac 1p},\; \lambda_x\!\in [0,1],\, \sum_{x\in \Gamma}
\lambda_x\, x =\xi,\,
\sum_{x\in \Gamma} \lambda_x=1\right \} 
%\\
%&=& +\infty  \mbox{ otherwise}.
\end{eqnarray}
Note that $F_p(\xi;\Gamma)<+\infty$ if and only if $\xi\!\in \conv(\Gamma)$ so that  $d_{p}(X; \Gamma)<+\infty $ if and only if $ X\!\in \conv(\Gamma)$ $\Prob$-$a.s.$. and that $d_{p}(X; \Gamma)= \Big\|X-\widehat X^{\Gamma,dual}\Big\|^p_{L^p(\Prob\otimes\Prob_0)}$. Hence, this notion only makes sense for compactly supported r.v.s. In particular if the support of $\Prob_{_X}$ is compact and contains $d+1$ affinely independent points, $d_{n,p}(X,\Gamma)=+\infty$ as long as $n\le d$.This new quantization modulus leads to an optimal dual quantization problem at {\em level} $N$, 
\begin{equation}\label{def:dnpsharp}
d_{n,p}(X)\!=\!\inf\Big\{d_{n,p}(X, \Gamma),\, \Gamma\!\subset \R^d, \; |\Gamma|\le
n\Big\}
\!=\!\inf\Big\{\|F_p(X;\Gamma)\|_p,\, \Gamma\!\subset \R^d, \; |\Gamma|\le
n\Big\}.
\end{equation}
%Then we define so that 
%\begin{equation}\label{def:dnpbarp}
%d_{n,p}(X)\end{equation}

One important application of quantization in general is the use of quantization grids as numerical cubature formula (see~\eqref{eq:cubature}). 
The main feature here is the stationarity which allows to derive a second order formula for the integration error. 
Since, by construction, dual quantization can  achieve stationarity only on a compact set, we show in section 2.2 that the extension of dual quantization to non-compactly supported random variables as defined in~\cite{dualStat} preserves this second order rate on the whole support of the r.v.

We therefore define the splitting operator $\mathcal{J} _{\Gamma}$ outside $\conv(\Gamma)$ by setting

%\textcolor{red}{To try to preserve the performances of dual quantization to non-compactly supported r.v. $X$ (see~), we  introduced in~\cite{dualStat}  an extension of the  splitting  operator $\mathcal{J} _{\Gamma}$ outside $\conv(\Gamma)$ by setting
\[ 
\forall\,\xi \in \R^d\setminus
\conv(\Gamma),\quad \mathcal{J}_{\Gamma}(\omega_0,\xi)= {\rm
Proj}_{\conv{(\Gamma)}\cap \partial\Gamma}(\xi) 
\] 
where $ {\rm Proj}_{\conv{(\Gamma)}\cap \partial\Gamma}$ is a Borel nearest neighbour projection on $\conv{(\Gamma)}\cap \partial\Gamma$. This choice is not unique: an alternative extension could be  to set   $\mathcal{J}_{\Gamma}(\omega_0,\xi)={\rm Proj}_{\conv(\Gamma)}(\xi)$. 
But the above  choice is   tractable in terms of simulation and we will prove that it does not deteriorate  the resulting mean error when $|\Gamma|\to +\infty$. Though the stationary property is lost as expected, we point out  in Section~\ref{sec:motiv} that this operator remains as performing as $\mathcal{J}_{\Gamma}$ is for bounded r.v.s when implementing cubature formulas for unbounded r.v.s. 

% in term although it
%preserves the continuity of the functional approximator
%$\mathbb{J}_{\Gamma}(f)$ if $f$ is continuous).
%
%\medskip {\sc Canonical extension for unbounded random vectors} As already emphasized, if
%a random vector $X$ is not $\Prob$-essentially  bounded, the above approach cannot be developed. We need to extend the definition of  $\mathcal{J}^*_{\Gamma}$ and more generally of splitting operators $\mathcal{J}_{\Gamma}$ {\em outside the convex hull} of $\Gamma$. One way to proceed
%(see~\cite{dualStat}) is to consider again a (deterministic Borel) nearest neighbour
%projection ${\rm Proj}_{\Gamma}$ by setting
%An alternative extension could be  to set   $\mathcal{J}_{\Gamma}(\omega_0,\xi)={\rm Proj}_{\conv(\Gamma)}(\xi)$ 
%(but this choice is   more demanding in computational terms although it
%preserves the continuity of the functional approximator
%$\mathbb{J}_{\Gamma}(f)$ if $f$ is continuous).
%This led us to define the extended $L^p$-local dual quantization error function $\bar F_p$ by~(\ref{def:Fbarp}). 
%Of course, we loose the intrinsic stationary property, however we
%were able to show in most situations (see below) the existence of an optimal
%grid solution to the resulting minimization problem which defines the $L^p$-mean dual quantization error 
%\[ 
%\bar d_{n,p}(X) =\inf \left\{
%\norm{\bar F_p(X;\Gamma)}_{L^p(\Prob)},\; \Gamma\subset  \R^d, \, |\Gamma|\le n \right\}.
%\]

Then, we to derive the {\em extended  local dual quantization error} function by
\begin{equation}\label{def:Fbarp}
\bar F_p(\xi;\Gamma):= F_p(\xi;\Gamma)\,\mbox{\bf 1}_{\conv(\Gamma)}(\xi)+{\rm
dist}(X,\Gamma)\,\mbox{\bf 1}_{\conv(\Gamma)^c}(\xi),
\end{equation}
and  the {\em  extended $L^p$-mean dual quantization error} of $X$
induced by  $\Gamma$ by  
\begin{equation}\label{def:dbarpGamma}
\bar d_{p}(X; \Gamma)= \|\bar F_p(X;\Gamma)\|_{L^p(\Prob)}.
\end{equation}
Finally, we define   the {\em extended $L^p$-mean dual quantization error}  at level $n$ given by
\begin{equation}\label{def:dbarnp}
\bar d_{n,p}(X)=\inf \Big\{\bar d_p(X,\Gamma),\; \Gamma\!\subset \R^d, \;
|\Gamma|\le n\Big\}.
\end{equation}

%\textcolor{red}{This choice and its consequences are  discussed and justified in in Section~\ref{sec:motiv}: it remains tractable and preserves the performance of dual quantization for numerical integration (cubature formula)  (though stationarity is necessarily lost in this extension).}


Finally, we briefly recall a few facts about  the (regular) {\em Voronoi optimal quantization problem} at level $n$ associated to the nearest neighbour projection ${\rm Proj}_{\Gamma}$: it reads
\begin{equation}\label{def:enp}
e_{n,p}(X)=\inf \left\{ \| {\rm dist}(X,\Gamma)\|_{L^p(\Prob)} ,\; \Gamma\subset \R^d,\, |\Gamma|\le n\right\}
\end{equation}
(where ${\rm dist}(x,A)= \inf_{a\in A}\|x-a\|$). It is well-known that $e_{n,p}(X)\downarrow 0$ as soon as $n\to +\infty$ and
$X\!\in L^p(\Prob)$. Moreover, the rate of convergence to $0$ of
$e_{n,p}(X)$ is ruled by  Zador's  Theorem (see~\cite{Foundations}).

\begin{thm}[Zador]
Let $X\!\in L_{\R^d}^{p'}(\Prob)$, $p'\!>\!p$. Let $\ProbX= h.\lambda_d +\nu$, $\nu\perp\lambda_d$ be the distribution of $X$ where $\lambda_d$ denotes the Lebesgue measure on $(\R^d,{\cal B}or(\R^d))$.
Then
\[
\limn n^{\frac 1d} e_{n,p}(X) =
\VQpn\, \|h\|_{\frac{d}{p+d}}^{\frac 1p}
\] 
where $ \|h\|_{\frac{d}{p+d}}=\displaystyle \Big(\int_{\R^d} h(\xi)^{\frac{d}{p+d}}d\xi\Big)^{1+\frac pd}$ and $\displaystyle \VQpn= \inf_{n} n^{\frac 1d}\,e_{n,p}(U([0,1]^d))\!\in
(0,\infty)$.
\end{thm}

The above  rate depends on $d$ and  is known as the {\em curse of dimensionality}.
 Its statement  and proof goes back to Zador (PhD, 1954) for the uniform distributions on hypercubes, its 
extension to   absolutely  continuous distributions is due to 
Bucklew and Wise in~\cite{bucklew}. A first general rigor proof 
 (according to mathematical  standards) was provided 
in~\cite{Foundations} in 2000 (see also \cite{GRAY} for a survey of the history of quantization).

\smallskip
It should be noted that $d_{n,p}(X)$ and $\bar d_{n,p}(X)$ do not coincide even for bounded r.v.s. We will extensively use (see~\cite{dualStat}) that
\[
 d_{n,p}(X)\ge  \bar d_{n,p}(X)\ge e_{n,p}(X).
\]
%where $e_{n,p}(X)$ is the ``regular" Voronoi $L^p$-mean quantization error at level $n$ defined by
%\[
%e_{n,p}(X) = \inf \left\{ \|X-{\rm Proj}_{\Gamma}(X)\|_{L^p(\Prob)},\; |\Gamma|\le n\right\}.
%\]

%\medskip 
This paper is entirely devoted to establishing the sharp asymptotics of the
optimal dual quantization error moduli $d_{n,p}(X)$ and $\bar d_{n,p}(X)$ as $n$ goes to infinity. 
The main result is stated in Theorem~\ref{thm:DQRate} (Zador's like theorem)
(see Section~\ref{mains} below). 
Proposition~\ref{PdtQErrop} (a Pierce like Lemma) is a companion result
which provides a non-asymptotic upper bound for the exact rate simply involving  
moments of the r.v. $X$ (higher than $p$).  Our   proof   has the same structure  as that of the original Zador  Theorem (see~$e.g.$~\cite{Foundations} where it has been rigorously completed for   the first time), except that the  splitting operator $\mathbb{J}_{\Gamma}$ is much more demanding to handle than the plain nearest neighbour projection: it requires more sophisticate arguments borrowed from convex analysis (including dual primal/methods) and geometry,  both in  a probabilistic framework.  In one dimension the exact rate $O(n^{-1})$ for $d_{n,p}(X)$ and $\bar d_{n,p}(X)$ follows from a random quantization argument detailed in Section~\ref{Pierce} (extended Pierce Lemma for $d_{n,p}(X)$).
% (in fact, we even state a slightly more general result
%than requested for our purpose). 
This rate can be transferred in a $d$-dimensional framework  to  $O(n^{-\frac
1d})$  using  a product (dual) quantization
argument (see~Proposition~\ref{prop:asymptQ} below and Section~\ref{sec:localProperties}). Finally, the sharp upper bound
is obtained in Section~\ref{sec:rate} by successive approximation procedures of
the density of $X$,  whereas the lower bound relies on a new ``firewall" Lemma.

\smallskip
\noindent {\sc Notations}:  $\bullet$ $\conv(A)$ stands for the convex hull of $A\subset \R^d$,
$\abs{A}$ for its cardinality, ${\rm diam}_{\norm{.}}(A)=\sup_{x,y\in A}\norm{x-y}$ for its diameter and ${\rm aff.dim}(A)$ for the dimension of the affine subspace of $\R^d$ spanned by $A$. 

%\smallskip  
\noindent  $\bullet$ We denote 
%$ \Big(\begin{array}{c} n\\ i \end{array}\Big):=\frac{n!}{i!(n-i)!}, \; n,\, i\!\in \{0,\ldots,n\}$, 
$ \big(\begin{smallmatrix} n\\ i \end{smallmatrix}\big):=\frac{n!}{i!(n-i)!}, \; n,\, i\!\in \{0,\ldots,n\}$, $n\!\in \mathbb N$.
 
%\smallskip  
\noindent $\bullet$ $\lfloor x \rfloor$ and $\lceil x\rceil$ will denote the lower and the upper
integral part of the real number $x$ respectively; set likewise
$x_{\pm}=\max(\pm x,0)$.  For two
sequences of real numbers $(a_n)$ and $(b_n)$,  $a_n\sim b_n$ if $a_n =u_n b_n$ with $\lim_n u_n =1$.

%\smallskip 
\noindent $\bullet$  For every $x=(x^1,\ldots,x^d)\!\in \R^d$, $|x|_{\ell^r}=(|x^1|^r
+\cdots|x^d|^r)^{1/r}$ denotes the $\ell^r$-norm or pseudo-norm, $0<r<+\infty$  and $|x|_{\ell^{\infty}}=\max_{1\le i\le d}|x_i|$ denotes the $\ell^{\infty}$-norm. A general norm on $\R^d$ will be denoted $\norm{\cdot}$.
 
%\smallskip  
\noindent $\bullet$ ${\rm supp}(\mu)$ denotes the support of a distribution $\mu$ on $(\R^d,{\cal B}or(\R^d))$. 

\section{Main results and motivation for extended dual quantization}
\subsection{Main results}\label{mains} 
The theorem below establishes  for any $p>0$ and any norm on $\R^d$
the counterpart of Zador's Theorem in the framework of dual quantization for both
$d_{n,p}$ and $\bar d_{n,p}$   error moduli.

\begin{thm}\label{thm:DQRate} $(a)$ 
Let $X\!\in L_{\R^d}^{\infty}(\Prob)$. Assume the distribution $\ProbX$ of
$X$ reads $\ProbX= h.\lambda_d+\nu$, $\nu\perp\lambda_d$. Then
\[
\limn n^{\frac 1d}\, d_{n,p}(X) 
= \limn n^{\frac 1d}\, \bar d_{n,p}(X)
= \Qpn\,\|h\|_{\frac{d}{p+d}}^{\frac 1p}
\] 
where $\displaystyle \Qpn= \inf_{n\ge 1} n^{\frac 1d}\, d_{n,p}(U([0,1]^d))\!\in
(0,\infty)$.

\noindent $(b)$  Let $X\!\in L_{\R^d}^{p'}(\Prob)$, $p'>p$. Assume the
distribution $\ProbX$ of $X$ reads $\ProbX= h.\lambda_d+\nu$,
$\nu\perp\lambda_d$. Then
\[
\limn n^{\frac 1d}\, \bar d_{n,p}(X) 
%= \lim_n n^{\frac 1d}\, \bar d_{n,p}(X)
= \Qpn\,\|h\|_{\frac{d}{p+d}}^{\frac 1p}.
\] 
\noindent $(c)$ If $d=1$, then 
\[
d_{n,p}(U([0,1])) = \biggl(\frac{2}{(p+1)(p+2)} \biggr)^{\frac 1p}
\frac{1}{n-1},
\]  
which implies $Q^{dq}_{|\,.\,|,p,1}=\left(\frac{2^{p+1}}{p+2}\right)^{\frac
1p}Q^{vq}_{|\,.\,|,p,1}$.
\end{thm}

Moreover, we will also establish in Section~\ref{sec:rate} an upper bound for the
dual quantization coefficient $\Qpn$ when $\norm{\cdot} =
\abs{\cdot}_{\ell^r}$.
% For the asymptotic behaviour of the dual
% quantization coefficient $\Qpn$ as $d\to\infty$ we establish the following upper bound when $\norm{\cdot} =
% \abs{\cdot}_{\ell^r}$.
\begin{prop}[Product quantization]\label{prop:asymptQ}
Let $r,p\in[1,\infty)$ with $r\leq p$.
% and let $|(x_1,\ldots,x_d)|_{\ell^r}= \big(|x_1|^r+\cdots+|x_d|^r\big)^{1/r}$  denote the $\ell^r$-norm on $\R^d$. 
Then it holds for every $d\in\N$
\[
	Q^{\text{dq}}_{\abs{\cdot}_{\ell^r}, p, d} \leq d^{\frac{1}{r}}\cdot
	Q^{\text{dq}}_{\abs{\cdot}, p, 1}
\]
where $|\,.\,|$ denotes standard absolute value on $\R$.\end{prop}

Since this upper bound achieves the same asymptotic rate as in the case of
regular quantization (cf. Corollary~9.4 in~\cite{Foundations}), this suggests the rate
 $O(d^\frac{1}{r})$ to be also the true one for $\Qpn$ as $d\to\infty$.

%\smallskip
As a step towards the above sharp rate theorem, we  also establish a counterpart
of the so-called Pierce Lemma (as stated in an operating form $e.g.$ 
in~\cite{meanRegular}). 
In practice, it turns out to be quite  useful for applications since it provides 
non-asymptotic error bounds which  only depend on the moments of the r.v. $X$
and the size of the optimal grid as emphasized in~\cite{dualAppl} (see section
\ref{sec:ddimBound} for the proof).

 \begin{prop}[$d$-dimensional extended Pierce Lemma] \label{PdtQErrop} $(a)$ Let
 $p,\,\eta>0$. There exists 
 %an integer $n_{d,p,\eta}\ge 1$ and 
 a real constant $C_{d, p,\eta}>0$ such that, for every $n\ge 1$
%n_{d,p,\eta}$ 
and every r.v.  $X\!\in L_{\R^d}^{p+\eta}(\Omega,{\cal A}, \Prob)$, 
 \[ 
 \bar d_{n,p}(X)\le C_{d,p,\eta}\sigma_{p+\eta,\|.\|}(X)\,n^{-1/d}
 \] 
 where $\sigma_{p+\eta,\|.\|}(X)= \inf_{a\in \R^d} \|X-a\|_{L^{p+\eta}}$ denotes the $L^{p+\eta}$-pseudo-standard deviation of $X$. 
 
 \smallskip
\noindent $(b)$  If $\supp(\ProbX)$ is compact then there exists 
 %an integer $n_{d,p,\eta}\ge 1$ and 
 a real constant $C'_{d, p,\eta}>0$ such that, for every $n\ge 1$
 \[
 d_{n,p}(X)\le C'_{d,p,\eta}{\rm diam}_{\norm{.}}({\rm supp}(\Prob_{_X}))\,n^{-1/d}. 
 \]
\end{prop}

 % , this strongly  supports the fact  that the outside (Voronoi) quantization component in the extended dual quantization error %modulus acts as {\em penalization term}   
  
  
  
  
  
%\textcolor{red}{ If $\supp(\ProbX)$ is
% compact then the same inequality holds true for $d_{n,p}(X)$. Probably not correct}
% [[Not so clear at a first glance except if $\Gamma^{ext}$ is itself a product
% quantizer: If furthermore, its the convex hull of its support is spanned by a
% finite grid of extrema points $\Gamma^{ext}$ then the inequality is satisfied
% by $d^{ext}_{n,p}(X)$ as well]].
% \end{prop}
\subsection{How to use the extended $L^p$-dual quantization error modulus?} \label{sec:motiv}
%Why and how consider distributions with unbounded support? In usual applications we need to perform   numerical integration or conditional expectation computation with respect to distributions and/or random vectors having unbounded support. On the other hand, the goo properties of dual quantization --mostly generic stationarity -- are only shared by distributions with compact supports. 
%
%A naive idea could be to truncate {\em a priori} the distribution of interest  outside a large enough compact domain or to condition the underlying random vector to stay-y in this domain. This a priori truncation becomes a difficult problem in higher dimension especially for random distributions sharing no symmetry as the result of a highly ``non linear black box" (typically a stochastic control problem with a jump underlying diffusion dynamics).

We  briefly explain why the extended dual quantization error modulus,  already been  introduced in~\cite{dualStat} for non-compactly supported distributions, is the right  tool to {\em perform automatically an optimized truncation} of non-compactly supported distributions.  basically, it uses its  additional  ``outer Voronoi projection"   as a {\em penalization term}   which expands automatically the convex hull of the dually optimal   grid at its appropriate ``amplitude", making altogether   the distribution outside of its  convex hull  ``negligible"   and sharing  an optimal  rate of decay $n^{-\frac  1d}$ as its size $n$ goes to infinity. The specific choice of a Voronoi quantization among other possible  solutions  for this penalization is motivated by both  its   theoretical tractability and its simple implementability in stochastic grid optimization algorithms.
% the fact that the resulting   $L^p$-mean extended dual quantization error modulus $\bar d_{n,p}(X)$ has the same rate of decay with comparable constants as $L^p$- dual quantization %error of compactly supported random vector as well as regular Voronoi $L^p$-mean quantization error. 
%   the optimal dual quantization grid in such a way that makes the outside of the convex hull always becomes negligible with %respect to the inside. 
This feature if of the highest importance for numerical integration or conditional execration approximation. This is the main motivation  to introduce and deeply  investigate  the sharp asymptotics of  this   $L^p$-mean extended dual quantization error modulus  $\bar d_{n,p}(X)$. 

We saw in~\cite{dualStat} that {\em Euclidean  dual quantization}  of a compactly supported distribution produces {\em  stationary} (dual) quantizers, namely r.v.s $\widehat X^{dual}$ satisfying $\E(\widehat X^{dual}\,|\, X)=X$, so that (see Proposition~?? in~\cite{dualStat}), dual quantization based cubature formula induce on functions $f\!\in {\cal C}_{\rm Lip}^1(\R^d,\R)$ (Lipschitz functions with Lipschitz continuous gradient) an error at most equal to $[D f]_{\rm Lip}d_{2,n}(X)^2$. Taking into account the rate established in Theorem~\ref{thm:DQRate}$(a)$, this yields  a $O(n^{-\frac{2}{d}})$ error rate.

There is no way to extend dual quantization to (possibly) unbounded r.v.s so that it preserves the above stationarity property. However, with the choice we made (nearest neighbor projection on the  grid outside its convex hull), natural heuristic arguments strongly suggest that the above order $O(n^{-\frac{2}{d}})$ is still satisfied for functions in  $ {\cal C}_{\rm Lip}^1(\R^d,\R)$.

%\smallskip
We consider an  unbounded Borel  distribution $\mu=\Probb_{_X}$ of an $\R^d$-valued r.v. $X$. Let $\Gamma_{n}$ be an {\em  Euclidean $L^2$-optimal  extended} dual quantization grid of size $n$ for $\mu$ (see~\cite{dualStat} or  Theorem~\ref{thm:existence}) and $\widehat X^{dual}$
%= \widehat X^{\Gamma_{_N}}$
the resulting $\Gamma_n$-valued extended dual quantization of $X$. Let $C_{n}= \conv\big(\Gamma_{n}\big) $ denote the  convex hull of $\Gamma_n$. It is clear by construction of $\widehat X^{dual}$  that $\widehat X^{dual}\!=\! \widetilde X^{dual}
%\mbox{\bf 1}_{\{X\in C_n\}}\!
+\!\widetilde X^{vor}$
%\mbox{\bf 1}_{\{X\notin C_n\}}$ 
where, with obvious notations, 
\[
\mbox{\bf 1}_{\{X\in C_{n}\}} \E\big(\widetilde X^{dual}|X\big) = \mbox{\bf 1}_{\{X\!\in C_{n}\}} X\; \mbox{ (dual stationarity) and }\;  \widetilde X^{vor}= {\rm Proj}_{\Gamma_n \cap C_n}(X).
\]
%where ${\rm Proj}_{\Gamma_n \cap C_n}$ is the nearest neighbour projection on $\Gamma_n \cap C_n$. 
Hence, if $f\!\in {\cal C}_{\rm Lip}^1(\R^d,\R)$, $ \E\big((D f(X)|X-\widetilde X^{dual}) |X\!\in C_{n}\big)=0$ and 
%\begin{eqnarray*}
%\left|\E\Big(f\big(\widehat X^{dual}\big) \mbox{\bf 1}_{\{X\!\in C_{n}\}} \Big)-\E\Big(f\big( X\big) \mbox{\bf 1}_{\{X\!\in C_{n}\}} \Big)\right|&=&\left| \E\left(\Big(f\big(\widehat X^{dual}\big)-f\big( X\big) -(D f(X)|X-\widehat X^{dual}) \Big)\mbox{\bf 1}_{\{X\!\in C_{n}\}}\right)\right|\\
%&\le& [D f]_{\rm Lip}
%\end{eqnarray*}
\begin{eqnarray*}
\left|\E\Big(f\big(\widetilde X^{dual}\big) |X\!\in C_{n}  \Big)\!-\!\E\Big(f\big( X\big)|X\!\in C_{n}\Big)\right|&\!=\!&\left| \E\left(f\big(\widetilde X^{dual}\big)\!-\!f\big( X\big) \!-\!D f(X).(X\!-\!\widetilde X^{dual}) |X\!\in C_{n}\right)\right|\\
&\!\le\! & [D f]_{\rm Lip}d_{n,2}(\Gamma_{n},\widetilde X^{dual}| X\!\in C_{n})^2.
%\\
%&\le & [D f]_{\rm Lip}\bar d^2_{2,n}(\Gamma_{_N},)\Probb( X\!\in C_{n})
\end{eqnarray*}
Consequently, 
%if $\mu^{C_{n}}$ denotes the conditional distribution of $X$ given $\{X\!\in C_n\}$, one has 
\begin{eqnarray*}
\left|\E\Big(f\big(\widetilde  X^{dual}\big)\mbox{\bf 1}_{\{X\in C_{n}\}} \Big) -\E\Big(f\big( X\big)\mbox{\bf 1}_{\{X\in C_{n}\}} \Big)\right|&\le&  [D f]_{\rm Lip}d_{n,2}(\widetilde X^{dual},\Gamma_{n})^2/\Probb( X\!\in C_{n})\\ &\le&  [D f]_{\rm Lip}\bar d_{n,2}(X,\Gamma_{n})^2/\Probb( X\!\in C_{n}).
\end{eqnarray*}
On the other  hand, 
%assuming that $f$ is itself Lipschitz continuous, 
%since $\widehat X^{dual}= \widehat X^{vor}$ outside $C_{n}$, 
\begin{eqnarray*}
\left|\E\Big(f\big(\widetilde X^{vor}\big) \mbox{\bf 1}_{\{X\notin C_{n}\}} \Big)-\E\Big(f\big( X\big) \mbox{\bf 1}_{\{X\notin C_{n}\}} \Big)\right|&\le& [f]_{\rm Lip}  e_{n,2}\big(X,\Gamma_n\big)\Probb\big(X\notin C_{n}\big)^{\frac 12}\\
& \le& [f]_{\rm Lip}\bar d_{n,2}(X)\Probb\big(X\notin C_{n}\big)^{\frac 12}. 
\end{eqnarray*}
Relying on Theorem~\ref{thm:DQRate}$(b)$, we know that, if $\mu = h.\lambda_d \stackrel{\perp}{+} \nu$, then  $\bar d_{n,2}(X)\sim  Q^{dq}_{2,|.|_{eucl}}\,\|h\|_{\frac{d}{2+d}}^{\frac 1p} n^{-\frac 1d}$. 
%The  order  $n^{-\frac 2d}$  resulting from the stationarity  will be preserved as soon as 
The ``outside" contribution will be negligible compared to the    ``inside" one as soon as 
%\begin{equation}\label{eq:PCn}
%\Probb\big(X\notin C_{n}\big) = O\Big( \bar d_{n,2}(X,\Gamma_{n})^2 \Big)=  O\Big( n^{-\frac 2d}\Big).
%\end{equation}
\begin{equation}\label{eq:PCn}
\Probb\big(X\notin C_{n}\big) = o\Big( \bar d_{n,2}(X,\Gamma_{n})^2 \Big)=  o\Big( n^{-\frac 2d}\Big).
\end{equation}

 This condition turns out to be  not very demanding and can be checked, at least heuristically, as illustrated below: if $X\stackrel{d}{=} {\cal N}(0;I_d)$, one may conjecture, taking advantage of the spherical symmetries of the normal  distribution, that $C_{n}$ is approximately a sphere centered at $0$ with radius $\rho_{n}= \max_{a\in \Gamma_{n}}|a|$.
As 
 \[
 \Probb(|X|\ge \xi) \sim V_d\,\xi^{d-2}e^{-\frac{\xi^2}{2}}\;\mbox{ as }\; \xi\to +\infty\quad (\mbox{with }V_d=\lambda_{d-1}(S_d(0,1))).
 \]
Condition~\eqref{eq:PCn} is satisfied as soon as $\liminf_n \frac{\rho_{n}}{\sqrt{\log n}}>\frac{2}{\sqrt{d}}$ ($\ge $ if $d=1,2$). As an example, one must have in mind that, for optimal {\em Voronoi} quantization, this inequality is satisfied since (see~\cite{PASA}) $\lim_n \frac{\rho_{n}}{\sqrt{\log n}}=\sqrt{2(1+2/d)}>\frac{2}{\sqrt{d}}$. More precisely, we have
\[
\Probb\big(X\notin C_{n}\big)\sim \kappa_d(\log n)^{\frac d2-1}n^{-1-\frac 2d} \; \mbox{ so that }\; \bar d_{n,2}(X)\Probb\big(X\notin C_{n}\big)^{\frac 12} = O\big(n^{-\frac 2d-\frac 12}(\log n)^{\frac{d-2}{4}}\big).
\] 
 
% \smallskip 
Numerical experiments, not reproduced here, carried out with the above ${\cal N}(0;I_d)$ distribution confirm   that the radius of optimal dual quantizers always achieves this asymptotics which makes the above partially heuristic reasoning very likely. Moreover, we also tested the two rates of convergence of $\Prob(X\!\in C_n)$ and $\bar d_{n,2}(X)^2$, this time on the joint distribution of the $(W_1, \sup_{t\in [0,1]}W_t)$, $W$ standard Brownian motion which has less symmetries (see appendix \ref{app:num}). They also confirm that the above partially heuristic reasoning is very likely. 




