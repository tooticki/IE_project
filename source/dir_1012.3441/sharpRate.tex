\section{Proof of the sharp rate theorem}\label{sec:rate}

On the way to proving the sharp rate theorem, we have to establish few additional 
propositions.

\begin{prop}[Sub-linearity]\label{prop:subLin}
Let $\Probb = \sum_{i=1}^m s_i \Probb_i$ where $s_1,\ldots,s_m\!\in [0,1]$,  $\sum_{i=1}^m s_i = 1$ and let $n_1,\ldots,n_m\!\in \N$ such that $\sum_{i=1}^m n_i \leq n$.
Then
\[
d_{n}^p(\Probb) \leq \sum_{i=1}^m s_i\, d_{n_i}^p(\Probb_i).
\]
\end{prop}
\begin{proof}
For $\varepsilon > 0$ and every $i = 1, \ldots, m$, choose $\Gammai \subset
\R^d,\,\abs{\Gammai} \leq n_i$ such that 
\[
d^p(\Probb_i; \Gammai) \leq (1+\varepsilon)\, d_{n_i}^p(\Probb_i).
\]
% 
% Let $\alpha_i$ be an optimal dual $n_i$-quantizer for $\Probb_i,\, \Gamma =
% \bigcup_{i=1}^m \Gammai$ (see Proposition~?? in~\cite{DQ1}). 
Set $\Gamma =
\bigcup_{i=1}^m \Gammai$~; from Proposition~\ref{prop:PtsInsertion} we get
\begin{eqnarray*}
\begin{split}
  d_{n}^p(\Probb) & \leq d_{n}^p(\Probb; \Gamma) = \sum_{i=1}^m s_i \int F^p(\xi; \Gamma)\,\Probb_i(d\xi)\\
  & \leq \sum_{i=1}^m s_i \int F^p(\xi; \Gammai)\,\Probb_i(d\xi)  \leq (1+\varepsilon) \sum_{i=1}^m s_i\, d_{n_i}^p(\Probb_i).
\end{split}
\end{eqnarray*}
Letting $\varepsilon\to 0$ completes the proof.
\end{proof}

\begin{remark}
Proposition~\ref{prop:subLin} does not hold for $\bar d_n^p$ since $\bar F^p$ is not decreasing for the inclusion order on grids.
%
% and $\tilde d_n^p$
This induces substantial difficulties in the proof of the
sharp rate compared to the regular quantization setting.
\end{remark}


\begin{prop}[Scaling property]\label{prop:scaleProp}
Let $C= a+\rho [0,1]^d$ ($a\!\in \R^d$, $\rho>0$) be a $d$-dimensional hypercube, with edges parallel to the coordinate axis and edge-length $\rho > 0$. Then
\[
	d_{n,p}(\U(C)) = \rho \cdot d_{n,p}(\Unifd).
\]
\end{prop}
%\begin{proof}
\noindent {\it Proof.} Keeping in mind that $\lambda_d([0,\rho]^d)=\rho^d$, it
holds that
\begin{eqnarray*}
\begin{split}
\hskip 1,5 cm  \dqp(\U(C); \{a+\rho x_1, \ldots, a+ \rho x_n\}) & = \int_{[0, \rho]^d} \underset{
  \text{s.t. } \left[  \begin{smallmatrix} \rho x_1 & \ldots & \rho x_n\\
          1 & \ldots & 1\\
        \end{smallmatrix}  \right] \lambda =
      \left[\begin{smallmatrix}
         \xi\\ 1\\
        \end{smallmatrix} \right],\,
      \lambda \geq 0 }{\min_{\lambda\in \R^n }\sum_{i=1}^n \lambda_i
      \,
  \norm{\xi-\rho x_i}^p}
  \frac{\lambda_d(d\xi)}{\lambda_d\bigl([0,\rho]^d\bigr)}\\ & =  \int_{[0,
  1]^d} \underset{ \text{s.t. } \left[  \begin{smallmatrix} \rho x_1 & \ldots &
  \rho x_n\\ 1 & \ldots & 1\\
        \end{smallmatrix}  \right] \lambda =
      \left[\begin{smallmatrix}
         \rho u\\ 1\\
        \end{smallmatrix} \right],\,
      \lambda \geq 0 }{\min_{\lambda\in \R^n }\sum_{i=1}^n \lambda_i
      \,
  \norm{\rho u-\rho x_i}^p} \lambda_d(du)\\
  & = \rho^p \int_{[0, 1]^d} \LP{u} \lambda_d(du)\\
  & = \rho^p \cdot \dqp(\Unifd; \{x_1, \ldots, x_n\}).\hskip 3 cm \Box
\end{split} 
\end{eqnarray*}
%which completes the proof.
%\end{proof}

% \begin{lemma}\label{lem:radius}
% Let $K \subset  \opSuppP$ be a compact set  and let $\Gamma_n$ be   a sequence
% of quantizers such that $\bar d_{n,p}(\Probb,\Gamma_n)\to 0$ as $n\to \infty$.
% Then there exists $n_0\in\N$ such that for all $n\geq n_0$
% \[
% K \subset \conv(\Gamma_n).
% \]
% \end{lemma}
% \begin{proof}
% % {\it I have a proof for a depreciated version of this lemma for $\Unifd$, but
% % it should hold in the same way for general $\Probb$}.
% {\rm Step~1:} Let $x\!\in \mathring{K}$. There exists $(x_i)_{i=1,\ldots,d+1}$ a
% system of affinely independent points of $K$ (an affine basis) such that $x=
% \frac{1}{d+1} \sum_{i=1}^{d+1} x_i$.
% 
% For every $i\!\in \{1,\ldots,d+1\}$ there exists a sequence $(x^{n}_i)_{n\ge 1}$
% having values in $\conv(\Gamma_n)$ and converging to $x_i$.  Otherwise there
% would be $\varepsilon_0>0$ and a subsequence $(n')$ such that
% $B(x_i,\varepsilon_0)\subset \conv(\Gamma_{n'})^c$. Then $\bar
% d_{n',p}(X,\Gamma_{n'}) \ge
% \left(\frac{\varepsilon_0}{2}\right)^pP(B(x_i,\varepsilon_0/2))$ since $x_0\!\in
% {\rm supp}(P)$ which contradicts the assumption on the sequence $(\Gamma_n)_{n\ge
% 1})$.
% 
% Now, for large enough $n$, $(x^n_i)_{i=1,\ldots,d+1}$ is an affine basis since 
% $(x_i)_{i=1,\ldots,d+1}$ is.  So we may write
% 
% \[ x_i= \sum_{j=1}^{d+1} \mu^{n,i}_jx^n_j,\quad
% \sum_{i=1}^{d+1}\mu^{n,i}_j=1,\quad i=1,\ldots,d+1. \] Furthermore one easily
% checks that $\mu^{n,i}_j\to \delta_{il}$ (Kronecker symbol). As a consequence \[
% x=\sum_{j=1}^{d+1} \frac{1}{d+1} \left(\sum_{i=1}^{d+1} \mu^{n,i}_j  \right)
% x^n_j. \] Then, $\frac{1}{d+1}\sum_{i=1}^{d+1} \mu^{n,i}_j >0$ for every
% $i=1,\ldots,d+1$ for large enough $n$ since these quantities converge toward
% $\frac{1}{d+1}$.
% 
% Hence, there exists an integer $n_1$ such that for every $n\ge n_1$, $x\!\in
% \mathring{\overbrace{\conv(\Gamma_n)}}$.
% 
% \smallskip \noindent {\rm Step~2:} Let $\eta>0$. Then $K^{\eta}:= \{\xi\!\in K,
% \; d(\xi, K^c)\le  \eta\}\subset   \mathring{K}$  is a (nonempty) compact set
% (for small enough $\eta$). Let $(n')$ be a subsequence. For every $n_0$,
% $K^{\eta}\subset \bigcup_{n\ge n_0} \conv(\Gamma_{n'})$ so there exists $n_1'\ge
% n'_0$ such that $K^{\eta}\subset \conv(\Gamma_{n_1'})$. By induction, there
% exists a second subsequence $n''$ of $n'$ such that $K^{\eta}\subset
% \conv(\Gamma_{n''})$. This double extraction criterion implies that, in fact,
% $K^{\eta}\subset \conv(\Gamma_{n})$ for large enough $n$.
% 
% \smallskip \noindent {\rm Step~3:} On concludes by noting that if $K\subset 
% \opSuppP$ then, for small enough $\eta>0$, its closed $\eta$-neighbourhood
% $K_{\eta}$ is still included into  $\opSuppP$ and one concludes by noting that
% $K\subset (K_{\eta})^{\eta}$.
% \end{proof}

The following lemma shows that also for $\bar d_{n,p}$ the convex hull 
spanned by a sequence of ``semi-optimal'' quantizers asymptotically covers the
interior of $\supp(\ProbX)$. This fact is trivial for $d_{n,p}$ if $X$ has  a compact
support.  

\begin{lemma}\label{lem:radius} Let $K  = \conv\{a_1, \ldots, a_k\} \subset  \opSuppP$ be a set with $\mathring
K \neq \emptyset$ and let $\Gamman$ be a sequence of quantizers such that $\bar
d_{n,p}(\Probb,\Gamman)\to 0$ as $n\to +\infty$. Then there exists $n_0\in\N$ such that for all $n\geq n_0$
\[
K \subset \conv(\Gamman).
\]
\end{lemma}
%\begin{proof}

\noindent {\it Proof.} Set $a_0 = \frac{1}{k} \sum_{i=1}^k a_i$ and define for every $\rho > 0$
\[
\tilde K(\rho) = \conv\{\tilde a_1(\rho), \ldots, \tilde a_k(\rho)\}
\quad \text{with}\quad
\tilde a_i(\rho)= a_0 + (1+\rho) (a_i - a_0).
\]

Since $K\subset\opSuppP$ there exists  $\rho_0 > 0$ such that $\tilde K =
\tilde K(\rho_0) \subset \supp(\Probb)$. From now on, we denote $\tilde
a_i(\rho_0)$ by $\tilde a_i$.
Since moreover $\tilde a_i \!\in \supp(\Probb)$, there exists a sequence
$(a^{n}_i)_{n\ge 1}$ having values in $\conv(\Gamman)$ and converging to
$\tilde a_i$. Otherwise there would exist $\varepsilon_0>0$ and a subsequence
$(n')$ such that $B(\tilde a_i,\varepsilon_0)\subset (\conv(\Gamma_{\!n'}))^c$.
Then 
$$
\bar d_{n'}^p(X,\Gamma_{\!n'}) \ge \E\, {\rm dist}(X,\Gamma_{\!n'})^p\mbox{\bf 1}_{\{X\in B(\tilde a_i,\varepsilon_0/2)\}}
\left(\frac{\varepsilon_0}{2}\right)^p \Probb(B(\tilde a_i,\varepsilon_0/2))>0
$$
since $\tilde a_i\!\in {\rm supp}(\Probb)$. This contradicts the assumption on the sequence
$(\Gamman)_{n\ge 1}$.  

Since $K$ has a nonempty interior, it follows that $\adim
\{a_1, \ldots, a_k\} = \adim \{\tilde a_1, \ldots, \tilde a_k\} = d$. Consequently,
we may choose a subset $I^\ast \subset\{1, \ldots, k\}, \,\abs{I^\ast} = d+1$, so
that $\{\tilde a_j: j\in I^\ast\}$ is an affinely independent system in $\R^d$
and furthermore there exists  $n_0 \in \N$ such that the same holds for
$\{a^n_j: j\in I^\ast\}$, $n \geq n_0$.
%, $(a^n_i)_{i\in I^}$ is an affinely
%independent 
Hence, we may write for $n\geq n_0$
\begin{equation}\label{eq:lemmaoneproofLS}
 \tilde a_i= \sum_{j\in I^\ast} \mu^{n,i}_ja^n_j,\quad
 \sum_{j\in I^\ast}\mu^{n,i}_j=1,\quad i=1,\ldots,k. 
 \end{equation}
 This linear system has the unique asymptotic solution $\mu^{\infty,i}_j =
 \delta_{ij}$ (Kronecker symbol), which implies $\mu^{n,i}_j\to \delta_{ij}$
when  $n\to+\infty$.

Now let $\xi\in K\subset \tilde K$ and write
\[
	\xi = \sum_{i=1}^k \lambda_i a_i\;
	\text{ for some }\; \lambda_i \geq 0, \sum_{i=1}^k \lambda_i = 1.
\]
One easily checks that it also holds 
\[
	\xi = \sum_{i=1}^k \tilde \lambda_i \tilde a_i\, \text{ with  } \, \tilde
	\lambda_i = \frac{\rho_0}{k(1+\rho_0)} + \frac{\lambda_i}{ 1+\rho_0} \geq \frac{\rho_0}{k(1+\rho_0)} > 0 \quad  \text{and}\quad \sum_{i=1}^k
	\tilde \lambda_i = 1.
\]

Furthermore, we may choose  $n_1 \geq n_0$ such that, for every $n\geq n_1$,
\[
\mu^{n,i}_i > \frac{1}{2} \quad \text{ and } \quad  \forall j \neq i,\; \abs{\mu^{n,i}_j} \leq
\frac{\rho_0}{4k(1+\rho_0)}.
\]

Using (\ref{eq:lemmaoneproofLS}),  this leads to
%Writing $x$ for $n\geq n_1$ in the coordinates of $x^{n,i}$ then leads to
\[
 \xi = \sum_{j\in I^\ast} \Bigl(\sum_{i=1}^k \tilde \lambda_i  \mu^{n,i}_j
 \Bigr) a^n_j
\]
and 
\[
	\sum_{i=1}^k \tilde \lambda_i  \mu^{n,i}_j > \tilde \lambda_j  \mu^{n,j}_j -
	\sum_{i=1,i\neq j}^k \tilde \lambda_i  \abs{\mu^{n,i}_j} >
	\frac{\rho_0}{2k(1+\rho_0)} - \frac{\rho_0}{4k(1+\rho_0)} =
	\frac{\rho_0}{4k(1+\rho_0)} > 0, \;j \in I^\ast.
\]
Finally, one completes the proof by noting that 
%\[
$\displaystyle 	\sum_{j\in I^\ast} \sum_{i=1}^k \tilde \lambda_i  \mu^{n,i}_j = \sum_{i=1}^k \tilde
	\lambda_i \sum_{j\in I^\ast} \mu^{n,i}_j = 1.  \quad \Box$
%\]
%finally completes the proof.
%\end{proof}

%\noindent{\bf Remark.} The case where $K$ is not convex seems not so clear to me.

% \begin{lemma}\label{lem:dualOptCdtBds}
% Let $u, \xi, b, a \in\R^d, b\neq a$.
% \begin{enumerate}
%   \item If
% \[
% (b-a)^Tu  \leq \norm{\xi -b}^p - \norm{\xi -a}^p, 
% \]
% then it holds
% \[
% 	\norm{\xi-a}^p - a^Tu \leq \max\bigl\{ \norm{\xi -b}^p; \norm{\xi -a}^p
% 	\bigr\}.
% \] 
% \item If moreover
% \[
% (b-a)^Tu  = \norm{\xi -b}^p - \norm{\xi -a}^p, 
% \]
% then
% \[
% 	\norm{\xi-a}^p - a^Tu \geq \frac{1}{2} \max\bigl\{ \norm{\xi -b}^p; \norm{\xi
% 	-a}^p \bigr\}.
% \] 
% \end{enumerate}
% \end{lemma}
% \begin{proof}
% W.l.o.g. there is an ONS $(e_i)$ such that for an $\alpha > 0$
% \[
% a = - \alpha e_1, \qquad b = \alpha e_1, \qquad u = \sum_{i=1}^d \lambda_i e_i. 
% \]
% Consequently, the first assumptions
% \[
% 	(b-a)^Tu  \leq \norm{\xi -b}^p - \norm{\xi -a}^p =: R
% \] 
% yields
% \[
% \lambda_1 \leq \frac{R}{2\alpha}.
% \]
% Thus, we arrive at
% \begin{eqnarray*}
% \begin{split}
%   \norm{\xi-a}^p - a^Tu & = \norm{\xi-a}^p + \alpha \lambda_1 \leq
%   \norm{\xi-a}^p + \frac{R}{2}\\
%   & \leq \max\bigl\{ \norm{\xi -b}^p; \norm{\xi -a}^p
% 	\bigr\}.
% \end{split}
% \end{eqnarray*}
% 
% In the second case, we derive as above
% \[
% \lambda_1 = \frac{R}{2\alpha},
% \]
% so that
% \begin{eqnarray*}
% \begin{split}
%   \norm{\xi-a}^p - a^Tu & = 
%   \norm{\xi-a}^p + \frac{R}{2} = \frac{\norm{\xi -b}^p + \norm{\xi -a}^p}{2} \\
%   & \geq \frac{1}{2} \max\bigl\{ \norm{\xi -b}^p; \norm{\xi -a}^p
% 	\bigr\}
% \end{split}
% \end{eqnarray*}
% completes the proof.
% \end{proof}
% 
% \begin{lemma}[Firewall]\label{lem:firewall}
% Let $C$ be a closed hypercube in $\R^d$ with edge-length $l$ and $C_\varepsilon,
% \varepsilon > 0$ be a closed hypercube with same origin as $C$ but edge-length $l-
% 2\varepsilon$.
% 
% Let $\beta$ be an arbitrary quantizer with $C_\varepsilon \subset
% \conv\{\beta\}$ and set $\mathring \beta = \beta \cap \mathring C$.
% 
% For $0 < \varepsilon' < \eta/2 $ and $\eta = \dist(C^c, C_\varepsilon)$, let
% $\gamma = \gamma(\varepsilon')$ be a grid such that $\forall\xi C_\varepsilon: F(\xi;
% \gamma) \leq \varepsilon'$ and there exists $g,g'\in\gamma, g\neq g'$ with
% $\norm{\xi-g}^p,\norm{\xi-g'}^p \leq \varepsilon'$.
% 
% Then for every $\xi \in C_\varepsilon \setminus \mathcal{H}(\beta\cup \gamma)$
% \[
% F(\xi;\beta\cup\gamma) = F(\xi;\mathring \beta\cup\gamma).
% \]
% \end{lemma}
% 
%  \noindent \textcolor{red}{BEGIN Suggestion for the statement of the firewall
% lemma}  By the way: what do we call hypercube in this lemma? in my opinion it
% is clearly $C= a + [-l/2,l/2]^d$, $a\!\in \R^d$, $l\ge0$\dots
% \begin{lemma}[Firewall]\label{lem:firewall} Let $C$ be a closed hypercube in
% $\R^d$ with edge-length $l$. for every $\varepsilon \!\in (0,l/2)$, let  
% $C_\varepsilon$ be the closed hypercube with the same origin as $C$ but edge
% length $l- 2\varepsilon$.  \smallskip For every $ \eta\!\in (0,
% \delta_{C,\varepsilon})$ with $\delta_{C,\varepsilon}= {\rm
% dist}(C^c,C_{\varepsilon})$, there exists a grid  $\Gamma_{\eta}$   such that 
% \smallskip \noindent $(i)$ $\forall\,\xi \!\in  C_\varepsilon$, $F(\xi;\gamma)
% \leq \eta$,  \smallskip \noindent $(ii)$  $\forall\,\xi \!\in C_\varepsilon$, 
% there exists $g$, $g'\!\in\Gamma_{\eta}$,  $g\neq g'$ such that
% $\max(\norm{\xi-g}^p,\norm{\xi-g'}^p )\leq \eta$.   \medskip Furthermore, for
% any such grid and any grid  $\Gamma$ such that $\conv{\Gamma}\supset
% C_{\varepsilon}$, % %\smallskip %\noindent $(\alpha)$   $C_{\varepsilon}
% \subset %\conv(\Gamma_{\eta})$ [[??? In fact $C\subset \conv(\Gamma_{\eta})$ I
% do no remeber why/if  we need that explictely somewhere.]] % and set
% $\mathring \Gamma = \beta \cap \mathring C$. % %\smallskip %\noindent
% $(\beta)$   Then for every $\xi \in C_\varepsilon \setminus
% \mathcal{H}(\Gamma\cup \Gamma_{\eta})$ \[ \Fp(\xi;\Gamma\cup\Gamma_{\eta}) =
% \Fp(\xi; ( \Gamma \cap \mathring{C})\cup\Gamma_{\eta}).
% \]
% \end{lemma}   In my opinion a explicit example of grid $\Gamma_{\eta}$ should
% be provided. $E.g.$, if $C= a + [-l/2,l/2]^d$, it seems that the lattice grid
% $$ \Gamma_{\eta'} =\left\{a+\eta \mathbf{k},\, \mathbf{k}\!\in\Big
% \{-\Big\lceil \frac{l}{2\eta'}\Big\rceil, \ldots,-1,0,1,\ldots, \Big\lceil
% \frac{l}{2\eta'}\Big\rceil\Big\}^d\right\} $$ does the job for $\eta'$ small
% enough (depending on $p$, $d$ and the norm $\|\,.\,\|$).  \medskip WARNING! I
% did not include my handwritten suggestions already mentioned on formerly. And
% I did not change the notations in the proof below since we did not come to a
% final agreement about the notations.  \noindent  \textcolor{red}{END
% Suggestion for the statement of the firewall lemma}
% \begin{proof}
% Since $\xi\notin \mathcal{H}(\beta\cup \gamma)$, there exist $d+1$ affinely
% independent points $\{a_1, \ldots,
% a_{d+1}\} \subset \{x_1, \ldots, x_n \} = \beta \cup \gamma $ (a basis!), such
% that
% \[
% F(\xi; \beta \cup \gamma) = \sum_{i=1}^{d+1} \lambda_i \norm{\xi - a_i}^p
% \]
% with $\lambda = \left(\begin{array}{ccc}
%                 a_1 & \ldots & a_{d+1}\\
%                 1&  \ldots & 1
%                 \end{array} \right)^{-1} 
% \left(\begin{array}{c}
%                 \xi \\ 1
%                 \end{array} \right) \geq 0$, (Primal feasibility)
% 
% and 
% \[
% \left(\begin{array}{cc}
% x_1^T & 1\\
% \vdots & \vdots\\
% x_n^T & 1\\
% \end{array}\right)
% \left(\begin{array}{c}
% u_1\\
% u_2\\
% \end{array}\right) \leq
% \left(\begin{array}{c}
%                 \norm{\xi-x_1}^p\\
%                 \vdots\\
%                 \norm{\xi-x_{n}}^p
%                 \end{array} \right)
% \]
% for
% \[
% \left(\begin{array}{c}
% u_1\\
% u_2\\
% \end{array}\right)
% =
% \left(\begin{array}{cc}
% a_1^T & 1\\
% \vdots & \vdots\\
% a_{d+1}^T & 1\\
% \end{array}\right)^{-1}
% \left(\begin{array}{c}
%                 \norm{\xi-a_1}^p\\
%                 \vdots\\
%                 \norm{\xi-a_{d+1}}^p
%                 \end{array} \right),
% \] 
% i.e. $(u_1, u_2)$ is the unique solution to
% \[
% \left(\begin{array}{cc}
% a_1^T & 1\\
% \vdots & \vdots\\
% a_{d+1}^T & 1\\
% \end{array}\right)
% \left(\begin{array}{c}
% u_1\\
% u_2\\
% \end{array}\right)
% =
% \left(\begin{array}{c}
%                 \norm{\xi-a_1}^p\\
%                 \vdots\\
%                 \norm{\xi-a_{d+1}}^p
%                 \end{array} \right).
% \]
% (Dual feasibility)
% 
% But the latter optimality conditions are nothing else then $u_2 = \norm{\xi -
% a_1}^p - a_1^Tu_1$,
% \[
% \norm{\xi -a_1}^p - a_1^Tu_1 = \norm{\xi -a_i}^p - a_i^Tu_1, \qquad i = 2,
% \ldots, d+1
% \]
% or
% \begin{equation}\label{eq:dualOptCdt1}
%   (a_i - a_1)^T u_1 = \norm{\xi -a_i}^p - \norm{\xi -a_1}^p, \qquad i = 2,
% \ldots, d+1
% \end{equation}
% and 
% \[
% \norm{\xi -a_1}^p - a_1^Tu_1 \leq \norm{\xi -x_i}^p - x_i^Tu_1, \qquad i = 2,
% \ldots, n
% \]
% or equivently
% \begin{equation}\label{eq:dualOptCdt2}
%   (x_i - a_1)^T u_1 \leq \norm{\xi -x_i}^p - \norm{\xi -a_1}^p, \qquad i = 2,
% \ldots, n.
% \end{equation}
% 
% Since
% \[
% F(\xi; \beta\cup\gamma) = \sum_{i=1}^{d+1} \lambda_i \norm{\xi - a_i}^p \leq
% \varepsilon'
% \]
% and $0 \leq \lambda_i \leq 1$, there exists $a \in \{a_1, \ldots, a_{d+1}\}$
% such that $\norm{\xi - a}^p \leq \varepsilon'$.
% Moreover $a \in \mathring \beta \cup \gamma$, since $\dist(\xi, \beta \setminus
% \mathring \beta) \geq \eta > \varepsilon'$.
% 
% Furthermore, there is by assumption a $y\in \gamma$, such that
% \[
% y\neq a \quad\text{ and }\quad \norm{\xi - y}\leq \varepsilon'.
% \]
% 
% Since $u_1$ satisfies due to~(\ref{eq:dualOptCdt2})
% \[
% (y-a)^Tu_1 \leq \norm{\xi - y}^p - \norm{\xi - a}^p
% \]
% we derive from Lemma~\ref{lem:dualOptCdtBds}
% \begin{equation}\label{eq:dualOptCdtUpBnd}
% u_2 = \norm{\xi - a}^p - a^Tu_1 \leq \max \bigl\{ \norm{\xi-y}^p; \norm{\xi -
% a}^p \bigr\} \leq \varepsilon'.
% \end{equation}
% 
% Suppose now
% \[
% F(\xi; \beta\cup\gamma) < F(\xi; \mathring \beta \cup \gamma).
% \]
% 
% This implies the existence of a point $b \in \{ a_1, \ldots, a_{d+1} \cap
% (\beta \setminus \mathring \beta)$, for which it holds due to
%~(\ref{eq:dualOptCdt1})
% \[
% (b-a)^T u_1 = \norm{\xi - b}^p - \norm{\xi - a}^p.
% \] 
% 
% Again by Lemma~\ref{lem:dualOptCdtBds} we derive
% \begin{eqnarray*}
% \begin{split}
%   u_2 & = \norm{\xi-a}^p - a^Tu_1 \geq \frac{1}{2} \max \bigl\{ \norm{\xi-b}^p;
%   \norm{\xi - a}^p \bigr\}\\
%   & \geq \eta/2 > \varepsilon',
% \end{split}
% \end{eqnarray*}
% which contradicts~(\ref{eq:dualOptCdtUpBnd}).
% 
% Hence, it holds $F(\xi; \beta\cup\gamma) \geq F(\xi; \mathring \beta \cup
% \gamma)$, which actually yields
% \[
% F(\xi; \beta\cup\gamma) = F(\xi; \mathring \beta \cup \gamma).
% \]
% since $\mathring \beta \cup  \gamma \subset \beta \cup \gamma$.
% \end{proof}

As already said, Proposition~\ref{prop:subLin} does not hold anymore for $\bar
d_{n,p}$. As a consequence we have to establish a 
%non-asymptotic 
``firewall Lemma",
which will be a useful tool to overcome this problem  in the
non-compact setting.

\begin{lemma}[Firewall]\label{lem:firewall} Let $K \subset \R^d$ be compact and convex with $\mathring K \neq \emptyset$.
Moreover, let $\varepsilon > 0$ be small enough so that 
\[
	K_\varepsilon = \{ x\in K : \dist_{\ell^{\infty}}(x, K^c) \geq \varepsilon \} \neq \emptyset.
\]
Let $\Gae$ be a subset of the lattice $\alpha \Z^d$ with edge-length $\alpha > 0$
satisfying
\[
 K \setminus K_\varepsilon \subset \conv(\Gae)\;\mbox{ and }\; \forall\, x\!\in K\setminus K_{\varepsilon},\;{\rm
dist}_{\norm{\cdot}}(x,\Gamma_{\alpha,\varepsilon})\le C_{\norm{\cdot}}\alpha
\] 
%and, for every $x\!\in K\setminus K_{\varepsilon}$, ${\rm
%dist}(x,\Gamma_{\alpha,\varepsilon})\le C_{\norm{\cdot}}\alpha$ 
where $C_{\norm{\cdot}} > 0$ is  a real  constant   only  depending on the norm $\norm{\cdot}$.

Then, for every grid $\Gamma\subset \R^d$ containing $K$ and every $\eta \in (0,1)$, it holds
\[
\forall\, \xi \in
K_\varepsilon,\quad 	\Fp(\xi; \Gamma) \geq \frac{1}{(1+\eta)^{p+d+1}} \Fp(\xi; (\Gamma \cap \mathring K)
	\cup \Gae )   - (1+\eta)^{-d-1}\eta^{-p} 
	(d+1)\, C^p_{\norm{\cdot}}  \alpha^p.
\]
\end{lemma}

\begin{remark} The lattice $\Gamma_{\alpha,\varepsilon}$ and its size will be carefully defined and estimated  for the specified compact sets $K$ when calling upon the firewall lemma  in what follows.
%A natural choice   for $\Gae$ is
%\[
%\Gae = \alpha \Z^d\cap(K\setminus K_{\varepsilon})+\{0, \pm \alpha e^i\, i=1,\ldots,d\}
%\]
%where $(e^1,\ldots,e^d)$ denotes the canonical basis of $\R^d$.
\end{remark}

%\begin{proof}
{\sc Proof.}
Let $\Gamma = \{ x_1, \ldots, x_n\}$ and let $\xi\!\in K_\varepsilon$.
Then we may choose $I=I(\xi)\subset \{1, \ldots, n\}$, $|I|\le d+1$ such that
\[
\Fp(\xi; \Gamma) = \sum_{i\in I} \lambda_j \norm{\xi - x_i}^p, \quad
\sum_{i\in I} \lambda_i x_i= \xi,\, \lambda_i \geq 0,\, \sum_{i\in I} \lambda_i
= 1.
\]
If for every $x_i \!\in \Gamma \setminus \mathring K$ $\lambda_i=0$ then $F^p(\xi,\Gamma)= F^p(\Gamma\cap \mathring K)$ and our claim is trivial.
Therefore, let $J(\xi)=\{i\,:\, x_i \!\in \Gamma \setminus \mathring K, \,
\lambda_i>0\}\subset I(\xi)$ and choose one fixed  $i_0 \in J(\xi)$.
% such that 
%$x_{i_0} \!\in \Gamma \setminus \mathring K$ and $\lambda_{i_0} > 0$.
%(otherwise the assertion is trivial). 
%Note that there are at most $d$ such components in $I(\xi)$
 Let  $\theta= \theta(i_0)\! \in (0,1)$ such that 
 \[ \tilde x_{i_0} = \xi +
\theta (x_{i_0} - \xi) \in K \setminus K_\varepsilon \quad\mbox{and}\quad \frac{\theta^{p\wedge 1}}{\theta+\lambda_{i_0}(1-\theta)} \le 1+\eta
\]
(when $p\ge 1$ the   right constraint is
empty).
Setting
\[
	\tilde \lambda^0_i = \frac{\lambda_i \theta}{\theta + \lambda_{i_0}(1-\theta)},\,
	i \in I\setminus\{i_0\}, \quad \tilde \lambda^0_{i_0} =
	\frac{\lambda_{i_0}}{\theta + \lambda_{i_0}(1-\theta)}
\]
we arrive at
\[ 
\tilde \lambda^0_{i_0} \tilde x_{i_0} +\sum_{i\in I \setminus \{i_0\}} \tilde
\lambda^0_i x_i = \xi, \;\tilde \lambda^0_i \geq 0, \;\sum_{i\in I}
\tilde \lambda^0_i = 1.
\] 
Consequently 
\begin{equation*}
\begin{split}
\tilde \lambda^0_{i_0} \norm{\xi - \tilde x_{i_0}}^p + \sum_{j\in I \setminus
\{i_0\}} \tilde \lambda^0_i \norm{\xi - x_i}^p  &=
\frac{\lambda_{i_0}\theta^p}{\theta + \lambda_{i_0}(1-\theta)} \norm{\xi -
x_{i_0}}^p
%\\ 
%& \qquad 
+ \sum_{i\in I \setminus
		\{i_0\}}\frac{\lambda_i \theta}{\theta + \lambda_{i_0}(1-\theta)}\norm{\xi - x_i}^p \\
& \leq \frac{\theta^{p\wedge 1}}{\theta + \lambda_{i_0}(1-\theta)} \sum_{i\in I}\lambda_i
		\norm{\xi - x_i}^p\\
&		\le(1+\eta)\sum_{i\in I}\lambda_i
		\norm{\xi - x_i}^p.
%		& < \Fp(\xi; \Gamma)
\end{split}
\end{equation*}
%\noindent where we used that $\theta^p\le \theta$ since $p\ge 1$. 
Repeating the
procedure 
%(at most $d+1$ times) 
for every $i\!\in J(\xi)$ 
%$x_{i}\!\in\Gamma\setminus \mathring{K}$
finally yields by induction the existence of $\tilde{x}_i\!\in K\setminus
K_{\varepsilon}$ and  $\tilde \lambda_i$, $i\in I$ such that
 \[ \sum_{i\in I :
x_i\notin \mathring{K}} \tilde \lambda_{i} \tilde x_{i} +\sum_{i\in I : x_i\in
\mathring{K}} \tilde \lambda_i x_i = \xi, \;\tilde \lambda_i \geq 0,
\;\sum_{i\in I} \tilde  \lambda_i = 1 
\] 
and 
\begin{equation}\label{eq:FirewallProofIneq}
(1+\eta)^{|J(\xi)|} \Fp(\xi; \Gamma)\ge \sum_{i\in I :
x_i\notin \mathring{K}} \tilde \lambda_{i} \norm{\xi - \tilde x_{i}}^p
+\sum_{i\in I : x_i\in \mathring{K}} \tilde \lambda_i \norm{\xi -  x_{i}}^p. 
\end{equation}

Let us denote $\Gae = \{a_1, \ldots, a_m\}$ and let   $i_0\!\in J(\xi)$ so that $\tilde x_{i_0}$ is a
``modified" $x_{i_0}$ (originally lying in $\Gamma\setminus \mathring K$). By
construction $\tilde x_{i_0}\in K \setminus K_\varepsilon\subset\conv(\Gae)$ and there is
$J_{i_0}\subset \{1, \ldots, m\}$ such that 
\[ \Fp(\tilde x_{i_0}, \Gae) =
\sum_{j\in J_{i_0}} \mu^{i_0}_j \norm{\tilde x_{i_0} - a_j}^p, \; \sum_{j\in
J_{i_0}} \mu^{i_0}_j x_j = \tilde x_{i_0},\, \mu^{i_0}_j \geq 0,\, \sum_{j\in
J_{i_0}} \mu^{i_0}_j = 1 
\] 
and 
\[ \forall\, j\!\in J_{i_0},\quad \norm{\tilde
x_{i_0}-a_j}\le C_{\norm{\cdot}}\,\alpha. 
\] 
%for a real constant $C_{\norm{\cdot}}
%> 0$, which only depends  on the norm $\norm{\cdot}$.

%A possible explicit construction (when $\Gae$ is as in the above  remark) is the
%following: one may select $\underline k\!\in \Z^d$ such that
%$\alpha\underline{k}$ is the nearest neighbour of $\tilde x_{i_0}$ in $\Gae \cap
%K\setminus K_{\varepsilon}$. Then there exists
%$\varepsilon^{j_0}_1,\ldots,\varepsilon^{j_0}_d\!\in \{\pm 1\}$ such that 
%\[
%\tilde x_{i_0} \!\in \conv(\alpha \underline{k}, \alpha
%\underline{k}+\varepsilon^{i_0}_je^j). 
%\] 
%The resulting index set $J_{i_0}$
%clearly satisfies the above claim.

 \smallskip Using the elementary inequality
\[ 
\forall\,p>0, \; \forall \eta > 0, \;\forall\, u,v \geq 0,\quad (u+v)^p \leq (1+\eta)^p u^p +
\Bigl(1+\frac{1}{\eta} \Bigr)^p v^p, 
\] 
we derive that for every $j\!\in J_{i_0}$
\begin{equation*}
%\begin{split}
  \norm{\xi - a_j}^p  \leq \bigl( \norm{\xi - \tilde x_{i_0}} + \norm{\tilde
  x_{i_0} - a_j}  \bigr)^p  \leq (1+\eta)^p \norm{\xi - \tilde x_{i_0}}^p + \Bigl(1+\frac{1}{\eta}
  \Bigr)^p\, C^p_{\norm{\cdot}}\, \alpha^p.
%\end{split}
\end{equation*}

As a consequence, 
\[ 
\sum_{j\in J_{i_0}} \mu^{i_0}_j \norm{\xi - a_j}^p  \leq
(1+\eta)^p \norm{\xi - \tilde x_{i_0}}^p + \Bigl(1+\frac{1}{\eta}
  \Bigr)^p\, C^p_{\norm{\cdot}}\, \alpha^p
\] 
which in turn implies
 \[
\norm{\xi - \tilde x_{i_0}}^p\ge \frac{1}{(1+\eta)^p}\sum_{j\in J_{i_0}} \mu^{i_0}_j \norm{\xi - a_j}^p- \eta^{-p}\, C^p_{\norm{\cdot}}\, \alpha^p.
\] 
Plugging this inequality in~(\ref{eq:FirewallProofIneq}) yields and using that
$|J(\xi)|\le d+1$, we finally get
\begin{equation*}
\begin{split}
 (1+\eta)^{|J(\xi)|} \Fp(\xi; \Gamma) 
  \ge &  \sum_{i\in I : x_i\in \mathring{K}} \tilde \lambda_i  \norm{\xi -  x_{i}}^p  +  \frac{1}{(1+\eta)^p}\sum_{i\in I : x_i\notin \mathring{K}}\tilde \lambda_i\sum_{j\in J_{i}}\mu^i_j\norm{\xi-a_j}^p\\
		&- |J(\xi)|\eta^{-p} d\, C^p_{\norm{\cdot}}\, \alpha^p\\
   \geq  &\; \frac{1}{(1+\eta)^p} \Fp\bigl(\xi; (\Gamma \cap \mathring K\})
  \cup \Gae \bigr) -\eta^{-p}\,(d+1)\,   C^p_{\norm{\cdot}}\, \alpha^p.\hfill \Box
\end{split}
\end{equation*}
%One concludes by noting that $|J(\xi)|\le d+1$. 
%where $ \tilde C^p_{\norm{\cdot}}=d C^p_{\norm{\cdot}}>0$. 
%\end{proof}

Now we can establish
% like in a first step the optimal 
the sharp rate for the uniform distribution $U([0,1]^d)$.

\begin{prop}[Uniform distribution]\label{prop:rateU} For every $p\ge 1$,
\[
\Qpn:= \inf_{n\geq 0} n^{1/d}\,
d_{n,p}\bigl(\Unifd \bigr) = \limn n^{1/d}\, d_{n,p}\bigl(\Unifd \bigr).
\]
\end{prop}
%\begin{proof}
{\sc Proof.}
Let $n,m \in \N,\, m<n$ and set $k = k(n,m) = \left\lfloor
\bigl(\frac{n}{m}\bigr)^{1/d} \right\rfloor \ge 1$.

Covering the unit hypercube $[0,1]^d$ by $k^d$ translates $C_1, \ldots,
C_{k^d}$ of the hypercube $\bigl[0,\frac{1}{k}\bigr]^d$,
we arrive at $\Unifd = k^{-d} \sum_{i=1}^{k^d} \U(C_i)$. Hence, Proposition~\ref{prop:subLin} yields
\[
d_{n,p}^p\bigl(\Unifd \bigr) \leq k^{-d}\sum_{i=1}^{k^d} \dqp_m(\U(C_i)).
\]
%
Furthermore, Proposition~\ref{prop:scaleProp} implies
\[
d_{m,p}(\U(C_i)) = k^{-1}\, d_{m,p}\bigl(\Unifd \bigr),
\]
so that we may conclude for all $n,m\in\N,\, m<n$,
\[
	d_{n,p}\bigl(\Unifd \bigr) \leq k^{-1}\, d_{m,p}\bigl(\Unifd \bigr).
\]
Thus, we get 
\begin{eqnarray*}
\begin{split}
  n^{1/d}\,d_{n,p}\bigl(\Unifd \bigr) & \leq k^{-1}\, n^{1/d}\,d_{m,p}\bigl(\Unifd
  \bigr)\\
  	& \leq  \frac{k+1}{k} \, m^{1/d}\, d_{m,p}\bigl(\Unifd
  \bigr), 
\end{split}
\end{eqnarray*}
which yields for every fixed integer $m \geq 1$%\in\N$
\[
	\limsn n^{1/d}\, d_{n,p}\bigl(\Unifd \bigr) \leq m^{1/d}\, d_{m,p}\bigl(\Unifd
	\bigr),
\]
since $\limn k(n,m) = +\infty$. This finally implies
\[
\limn n^{1/d}\,d_{n,p}\bigl(\Unifd \bigr)  = \inf_{m\geq 0} m^{1/d}\,
d_{m,p}\bigl(\Unifd \bigr).\hskip 3 cm  \Box
\]
%\end{proof}

\begin{prop}\label{prop:rateUNN} For every $p\ge 1$, 
\[ 
\Qpn = \limn n^{1/d}\, d_{n,p}\bigl(\Unifd \bigr)  = \limn n^{1/d}\,
\bar d_{n,p}\bigl(\Unifd \bigr) 
\]
\end{prop}
\begin{proof}
Since for every compactly supported distribution $\Probb$ we have $\bar d_{n,p}(\Probb) \leq  d_{n,p}(\Probb)$
% \leq d_{n,p}(X) 
%\]
it remains to show
\[
	\Qpn \leq \limin n^{1/d}\, \bar d_{n,p}\bigl(\Unifd \bigr).
\]

%Let $(\Gamman)$ be a sequence of optimal quantizers for $\bar d_{n,p}(\Unifd)$
%and 

For $0 < \varepsilon < 1/2$ let $C_\varepsilon= (1/2,\ldots,1/2)+
\frac{1-\varepsilon}{2}[-1,1]^d$ be the centered hypercube in $[0,1]^d$ with
edge-length $1-\varepsilon$ and midpoint $(1/2,\ldots,1/2)$.
Moreover let $(\Gamman)$ be a sequence of quantizers such that, for every $n\ge 1$, 
\[
\bar d_{p}(\Unifd; \Gamman) \leq (1+\varepsilon) \bar d_{n,p}(\Unifd).
\]

Owing to Lemma~\ref{lem:radius}, as $C_{\varepsilon} \subset (1,1)^d$, there is an integer $n_\varepsilon \in \N$ such
that \[ \forall n \geq n_\varepsilon, \quad C_\varepsilon \subset
\conv(\Gamman) . 
\]

We therefore get for any $n\geq n_\varepsilon$
 \begin{eqnarray*}
 (1+\varepsilon)^d\bar d^p_{n}\bigl(\Unifd \bigr)& \geq &\bar d^p\bigl(\Unifd;\Gamman \bigr)\\
 &\ge & \int _{C_{\varepsilon}}\bar F^p(\xi,\Gamma_n)^pd\xi=\int _{C_{\varepsilon}} F^p(\xi,\Gamma_n)^pd\xi= \lambda_d(C_{\varepsilon}) d^p \bigl(\U(C_\varepsilon), \Gamma_n\bigr) \\
&\ge& (1 - \varepsilon)^{d} d^p_{n}\bigl(\U(C_\varepsilon)\bigr)  = (1 - \varepsilon)^{d+p} d^p_{n}\bigl(\Unifd \bigr) 
\end{eqnarray*}
where we used the scaling property (Proposition~\ref{prop:scaleProp}) in the last line.
%, we get
% arrive %for $n\geq n_\varepsilon$  at
%\[
% (1+\varepsilon)d_{n,p}\bigl(\Unifd \bigr) \geq 
% d_{n,p}\bigl(\Unifd_{|C_\varepsilon}\bigr) =( \lambda_d(C_\varepsilon))^{\frac1p}  d_{n,p}\bigl(\U(C_\varepsilon)\bigr) =
%(1 - \varepsilon)^{1+d/p} d_{n,p}\bigl(\Unifd \bigr).
% \]

Hence, we obtain for all $0 < \varepsilon < 1/2$
\[
\limin n^{1/d}\, \bar d_{n,p}\bigl(\Unifd \bigr) \geq \frac{(1 -
\varepsilon)^{1+d/p}}{(1+\varepsilon)^{d/p}}\, \Qpn,
\]
so that letting $\varepsilon \to 0$ completes the proof. 
\end{proof}

\begin{prop}\label{prop:rateCubewise}
Let $\Probb = \sum_{i=1}^m s_i\, \U(C_i),\, \sum_{i=1}^m s_i = 1$,  $s_i>0$, $i=1,\ldots,m$,
where $C_i=a_i+[0,l]^d$,  $i=1,\ldots,m$, are pairwise disjoint hypercubes in $\R^d$ with common edge-length
$l$. Set 
\[
h := \frac{d\Probb}{d\lambda_d}= \sum_{i=1}^m s_i
l^{-d}\ind{C_i}.
\]
Then
\[
 \limn n^{1/d}\, d_{n,p}(\Probb)=  \limn n^{1/d}\, \bar d_{n,p}(\Probb)= \Qpn\cdot \normdp{h}^{\frac 1p} .
\]
\end{prop}
\begin{proof}Since $d_{n,p}(\Probb)\ge \bar d_{n,p}(\Probb)$ it suffices to show that 
\[
%\begin{enumerate}
%  \item[(a)] $\;\displaystyle
 \limsn n^{1/d}\, d_{n,p}(\Probb) \leq \Qpn\cdot \normdp{h}^{\frac 1p} \quad 
 % $.
 % \item[(b)] $\;\displaystyle
 \mbox{ and } \quad \limin n^{1/d}\, \bar d_{n,p}(\Probb) \geq \Qpn\cdot \normdp{h}^{\frac 1p} .
%\end{enumerate}
\]
%$(a)$ 
For $n\in\N$, set
\[
t_i = \frac{s_i^{d/(d+p)}}{\sum_{j=1}^m s_j^{d/(d+p)}} \qquad\text{ and }\qquad
n_i = \lfloor t_i n\rfloor, \,1\leq i \leq m.
\]

Then, by Proposition~\ref{prop:subLin} and Proposition~\ref{prop:scaleProp}, we
get for every $n\geq \max_{1\leq i \leq m}(1/t_i)$
\[
d_{n}^p(\Probb) \leq \sum_{i=1}^m s_i\, d_{n}^p(\U(C_i)) = l^p \sum_{i=1}^m s_i\,
d^p_{n_i}(\Unifd).
\]

Proposition~\ref{prop:rateU} then yields
\[
n^{\frac pd}\, \dqp_{n_i}(\Unifd) = \biggl(\frac{n}{n_i}\biggr)^{\frac pd}\,
n_i^{\frac pd}\,\dqp_{n_i}(\Unifd) \longrightarrow  t_i^{-\frac pd} \Qpn \quad\text{ as } n\to+\infty.
\]
Noting that  $\normdp{h} = l^p \Bigl( \sum s_i^{d/(d+p)} 
\Bigr)^{(d+p)/d}$,
we get
\[
\limsn n^{\frac pd}\, \dqpn(\Probb) \leq \Qpn  l^p \sum_{i=1}^m s_i\, t_i^{-\frac pd} = \Qpn\cdot
\normdp{h}.
\]

\noindent $(b)$ Let $\varepsilon \in(0,l/2)$ and let $\Cie$ denote  the closed hypercube with
the same center as $C_i$ but with edge-length $l-\varepsilon$.
For $\alpha \in (0, \varepsilon/2)$, we set $\tilde \alpha = \frac{l}{\lceil l/\alpha\rceil }$ and we define the lattice 
$$
 \Gamma_{\alpha, \varepsilon,i} = \big(a_i+ \tilde \alpha\Z^d\big)\cap (C_i\setminus C_{i,\varepsilon}\big) \bigcup \{\mbox{vertices of }  C_i\}.
$$

 
It is clear that $\conv(\Gamma_{\alpha,\varepsilon,i}) = C_i \subset C_i\setminus C_{i,\varepsilon}$ since it contains the vertices of $C_i$.
Moreover, for every $\xi\!\in C_i\setminus C_{i,\varepsilon}$, ${\rm dist}_{\ell^{\infty}}(\xi,\Gamma_{\alpha,\varepsilon,i})\le \alpha$ so that there exists a real constant $C_{\norm{\cdot}}>0$ only depending on the norm $\|.\|$ such that ${\rm dist}_{\|.\|}(\xi,\Gamma_{\alpha,\varepsilon,i})\le C_{\norm{\cdot}} \alpha$. Consequently the lattice $\Gamma_{\alpha,\varepsilon,i}$ satisfies the assumption of the firewall lemma (Lemma~\ref{lem:firewall}).
 
 On the other hand, easy combinatorial arguments show that number of points $m_i$ of $\Gamma_{\alpha,\varepsilon,i}$ falling in $C_i$ satisfies $\lceil \frac{l}{\tilde \alpha}\rceil^d \le m_i \le \big(\lceil \frac{l}{\tilde \alpha}\rceil+1\big)^d +2^d$ whereas the number $m_{i,\varepsilon}$ of points falling in $C_{i,\varepsilon}$ satisfies  $\big(\lceil \frac{l-\varepsilon}{\tilde \alpha}\rceil-1\big)^d \le m_{i,\varepsilon} \le \big(\lceil \frac{l-\varepsilon}{\tilde \alpha}\rceil+1\big)^d$ so that
 \[
\Big\lceil\frac{l}{\tilde \alpha}\Big\rceil^d - \Big(\Big\lceil\frac{l-\varepsilon}{\tilde \alpha}\Big\rceil+1\Big)^d \le  |\Gamma_{\alpha,\varepsilon,i}| \le \Big(\Big\lceil\frac{l}{\tilde \alpha}\Big\rceil+1\Big)^d +2^d-\Big(\Big\lceil\frac{l-\varepsilon}{\tilde \alpha}\Big\rceil-1\Big)^d.
 \]
%the lattice with edge-length $\tilde \alpha = \frac{l}{\lceil l/\alpha \rceil}$
%covering $C_i \setminus C_{i,\varepsilon}$.
%
%We then get for every $i\in\{1, \ldots, m\}$
%\[
%	\abs{\gaei} = \Bigl(\frac{l}{\tilde{\alpha}} + 1 \Bigr)^d - \Bigl(
%	\frac{l}{\tilde{\alpha}}- 2 \Bigl\lceil \frac{\varepsilon/2}{\tilde{\alpha}}
%	\Bigr\rceil - 1 \Bigr)^d
%\]
%
We define for every $\varepsilon\in(0,l/2), \alpha\in(0,\varepsilon/2)$
\[
	g_{l,\varepsilon}(\alpha) = \alpha^d \abs{\Gamma_{\alpha,\varepsilon,i}}.
%	 =
%	\Bigl(\alpha \frac{l}{\tilde{\alpha}} + \alpha \Bigr)^d - \Bigl(
%	\alpha \frac{l}{\tilde{\alpha}}- 2 \alpha \Bigl\lceil
%	\frac{\varepsilon/2}{\tilde{\alpha}} \Bigr\rceil - \alpha \Bigr)^d. 
\]
Since $\frac{\alpha}{\tilde{\alpha}} \to 1$ and $2\alpha \Bigl\lceil
	\frac{\varepsilon/2}{\tilde{\alpha}} \Bigr\rceil \to \varepsilon$ as $\alpha
	\to 0$, we conclude from the above inequalities that
	\begin{equation}\label{eq:proofFirewallDefgl}
	\forall \varepsilon \in (0, l/2),\quad	\lim_{\alpha \to 0} g_{l,\varepsilon}(\alpha) = l^d - (l-\varepsilon)^d.	 
	\end{equation}	
Let $\eta\in(0,1)$ and denote by $\Gamman$ a sequence of
$n$-quantizers such that $\bar d^p(\Probb; \Gamman) \leq (1+\eta)
d^p_n(\Probb)$. It follows from Proposition~\ref{PdtQErrop} that
$\bar d^p(\Probb; \Gamman) \to 0$ for $n\to \infty$ so that 
Lemma~\ref{lem:radius} yields the existence of $n_\varepsilon\in\N$ such that for any $n\geq n_\varepsilon$
\[
\bigcup_{1\leq i \leq m} \Cie \subset \conv(\Gamman).
\]

We then derive from Lemma~\ref{lem:firewall} (firewall) %, that for every $\eta\in(0,1)$
\begin{equation*}
\begin{split}
  \dqbp(\U(C_i); \Gamman) & = l^{-d} \int_{C_i} \Fbp(\xi; \Gamman)\,
  \lambda_d(d\xi) \\
  & \geq l^{-d} \int_{\Cie} \Fbp(\xi; \Gamman)\,
  \lambda_d(d\xi) = l^{-d} \int_{\Cie} \Fp(\xi; \Gamman)\,
  \lambda_d(d\xi) \\
  & \geq \frac{l^{-d }\,(l-\varepsilon)^d}{(1+\eta)^{p+d+1}}\, \dqp\bigl(\U(\Cie);
  (\Gamma_n\cap\mathring C_i)\cup \Gamma_{\alpha,\varepsilon,i}\bigr) -
  l^{-d}\,(l-\varepsilon)^d \frac{(1+\eta)^{-d-1}}{\eta^{p}}\, (d+1) C_{\norm{\cdot}}\cdot\alpha^p.
\end{split}
\end{equation*}

At this stage, we set for every $\rho > 0$
\begin{equation}\label{eq:proofFirewallDefalpha}
\alpha_n=\alpha_n(\rho) = \Bigl( \frac{m}{\rho n} \Bigr)^{1/d}
\end{equation}
and denote
\[
	n_i = \abs{(\Gamman\cap\mathring C_i)\cup \gaei}.
\]
Proposition~\ref{prop:scaleProp} yields $d_{n_i,p}(\U(\Cie)) =
(l-\varepsilon) d_{n_i,p}(\Unifd) $, so that we get 
\begin{equation}\label{eq:proofRateCubewiseb}
  \begin{split}
      n^{\frac pd}d_{n}^p(\Probb) & \geq \frac{1}{1+\eta} \sum_{i=1}^m s_i\,
      n^{\frac pd}\,\dqbp(\U(C_i);\Gamman)\\
      & \geq \frac{l^{-d }\,(l-\varepsilon)^d}{(1+\eta)^{p+d+2}}\, \sum_{i=1}^m s_i\, n^{\frac pd}\, \dqp\bigl(\U(\Cie);
  (\Gamma_n\cap\mathring C_i)\cup \gaei\bigr) \\
  & \qquad -  l^{-d}\,(l-\varepsilon)^d
  \frac{(1+\eta)^{-d-2}}{\eta^p} \sum_{i=1}^m
  s_i\, (d+1)\, C_{\norm{\cdot}}\cdot\alpha^p \cdot
  n^{\frac pd} \\
  &\geq \frac{l^{-d }\,(l-\varepsilon)^{d+p}}{(1+\eta)^{p+d+2}}\, \sum_{i=1}^m s_i\,
      n^{\frac pd}\, \dqpnn{n_i}\bigl(\Unifd\bigr) 
  -  l^{-d}\,(l-\varepsilon)^d
  \frac{(1+\eta)^{-d-2}}{\eta^p}  (d+1)\,
       C_{\norm{\cdot}} 
  \Bigl(\frac{m}{\rho}\Bigr)^{\frac pd}.
  \end{split}
\end{equation}

Since
\[
\frac{n_i}{n} \leq \frac{\abs{\Gamman\cap\mathring C_i}}{n} +
\frac{g_{l,\varepsilon}(\alpha_n)}{n\alpha_n^d} = \frac{\abs{\Gamman\cap\mathring
C_i}}{n} + \frac{\rho}{m} g_{l,\varepsilon}(\alpha_n),
\]
we conclude from (\ref{eq:proofFirewallDefgl}) and (\ref{eq:proofFirewallDefalpha}) that
\[
	\limsn\sum_{i=1}^m \frac{n_i}{n} \leq 1 + \rho \bigl(l^d -
	(l-\varepsilon)^d\bigr).
\]
We may choose a subsequence (still denoted by $(n)$), such that
\[
n^{1/d}\, \bar
d_{n,p}(\Probb) \to \limin n^{1/d}\, d_{n,p}(\Probb)
 \qquad\text{ and }\qquad 
\frac{n_i}{n} \to v_i \in [0,1 + \rho (l^d -
	(l-\varepsilon)^d)].
\]

As a matter of fact, $v_i>0,$ for every $i=1,\ldots  m$: otherwise
Proposition~\ref{prop:rateU} would yield
\begin{equation*}
\begin{split}
n^{\frac pd}\, \dqbpn(\Probb)  \geq& \frac{l^{-d}\,(l-\varepsilon)^{d+p}}{(1+\eta)^{p+d+2}}
\sum_{i=1}^m s_i\, \Bigl(\frac{n_i}{n}\Bigr)^{-\frac pd} n_i^{\frac pd}\,
\dqpnn{n_i}\bigl(\Unifd\bigr)\\
& \;-\,  l^{-d}\,(l-\varepsilon)^d
  \frac{(1+\eta)^{p-d-2}}{\eta^p} (d+1) C_{\norm{\cdot}}\cdot
  \Bigl(\frac{m}{\rho}\Bigr)^{\frac pd}\\
  & \to +\infty \quad \mbox{ as } n\to +\infty
  \end{split}
\end{equation*}
which contradicts $(a)$. Consequently, we may normalize the $v_i$'s by setting
\[
	\widetilde v_i = \frac{v_i}{1+\rho (l^d - (l-\varepsilon)^d)}, \; i=1,\ldots,m,
\]
so that $\sum_{i=1}^m \widetilde v_i \leq 1$. We derive from 
Proposition~\ref{prop:rateU} that
%and Lemma 6.8
\begin{equation*}
\begin{split}
 % \bigl(1+\rho (l^d - (l-\varepsilon)^d)\bigr)^{\frac pd} \cdot \qquad \\
  \limin \sum_{i=1}^m s_i\, n^{\frac pd}\,
  \dqpnn{n_i}\bigl(\Unifd\bigr) 
  &\ge\sum_{i=1}^m s_i\,v_i^{-\frac pd}   n_i^{\frac pd}\,
  \dqpnn{n_i}\bigl(\Unifd\bigr) \\ 
  & = \Qpn (1+\rho (l^d - (l-\varepsilon)^d)^{-\frac pd}\sum_{i=1}^m s_i\, \widetilde v_i^{\,-\frac pd}\\
 &\geq \Qpn(1+\rho (l^d - (l-\varepsilon)^d)^{-\frac pd}\inf_{\sum_i y_i\le 1, y_i\ge 0} \sum_{i=1}^m s_i y_i^{-\frac pd}
\\
 &= \Qpn(1+\rho (l^d - (l-\varepsilon)^d)^{-\frac pd} \biggl(\sum_{i=1}^m s_i^{d/(d+p)} \biggr)^{(d+p)/d}.
  \end{split}
\end{equation*}
 

Hence, we derive from 
(\ref{eq:proofRateCubewiseb})
\begin{equation*}
\begin{split}
  \limin n^{\frac pd}\, \dqbpn(\Probb) & 
  \geq \frac{l^{-d }\,(l-\varepsilon)^{d+p}}{(1+\eta)^{p+d+2}\bigl(1+\rho (l^d -
  (l-\varepsilon)^d)\bigr)^{\frac pd}}\,\Qpn\,\biggl(\sum_{i=1}^m s_i^{d/(d+p)}
  \biggr)^{(d+p)/d}
  \\ 
 & \qquad -  l^{-d}\,(l-\varepsilon)^d
  \frac{(1+\eta)^{-d-2}}{\eta^p} (d+1)\,        C_{\norm{\cdot}}\cdot
  \Bigl(\frac{m}{\rho}\Bigr)^{\frac pd}. \\ 
\end{split}
\end{equation*}
Letting $\varepsilon\to 0$ implies
\begin{equation*}
\begin{split}
  \limin n^{\frac pd}\, \dqbpn(\Probb) & \geq
  \frac{l^{p}}{(1+\eta)^{p+d+2}}\, \Qpn \,\biggl(\sum_{i=1}^m s_i^{d/(d+p)}
  \biggr)^{(d+p)/d}
 \, - \,  \frac{(1+\eta)^{-d-2}}{\eta^p}(d+1)\,  
       C_{\norm{\cdot}} 
  \Bigl(\frac{m}{\rho}\Bigr)^{\frac pd}\\
  & =  \frac{1}{(1+\eta)^{p+d+2}}\,\Qpn \cdot\normdp{h} 
 \, - \, \frac{(1+\eta)^{-d-2}}{\eta^p}d\,
       C_{\norm{\cdot}} 
  \Bigl(\frac{m}{\rho}\Bigr)^{\frac pd}
\end{split}
\end{equation*}
and, finally, letting successively  $\rho$ go to $+\infty$ and $\eta$ go to $0$
completes the proof.
\end{proof}

\begin{prop}\label{prop:rateCompact}
Assume that $\Probb$
is  absolutely continuous w.r.t. $\lambda_d$ with  compact support. Then
\[
 \limn n^{\frac pd}\,d_{n,p}(\Probb) = \limin n^{\frac pd}\, \bar d_{n,p}(\Probb)= \Qpn\cdot \normdp{h}^{\frac 1p}
\]
%\begin{enumerate}
%  \item[(a)] $\displaystyle \qquad \limsn n^{\frac pd}\,d_{n,p}(\Probb) \leq \Qpn\cdot \normdp{h}^{\frac 1p}$.
%  \item[(b)] $\displaystyle \qquad \limin n^{\frac pd}\, \bar d_{n,p}(\Probb) \geq \Qpn\cdot \normdp{h} ^{\frac1p}.$
%\end{enumerate}
\end{prop}
%\begin{proof} 
 {\sc Proof.} Since $d_{n,p}(\Probb)\ge \bar d_{n,p}(\Probb)$ it suffices to show that 
\[ \limsn n^{\frac pd}\,d_{n,p}(\Probb) \leq \Qpn\cdot \normdp{h}^{\frac 1p}\; \mbox{ and }\;\displaystyle   \limin n^{\frac pd}\, \bar d_{n,p}(\Probb) \geq \Qpn\cdot \normdp{h} ^{\frac1p}.
\]
% \textcolor{red}{Notations are not homogeneous between the first and the second part of the proof}.
\noindent {\em Preliminary step.} Let $C=[-l/2,l/2]^d$ be a  closed hyper hypercube centered at the origin, parallel to the coordinate axis  with
edge-length $l$, such that $\supp(\Probb) \subset C$.
For $k\in\N$ consider the tessellation of $C$ into $k^d$ closed hypercubes with
common edge-length $l/k$.
To be precise, for every $\underline i=(i_1,\ldots,i_d)\!\in \Z^d$, we set
\[
C_{\underline i} =\Prod_{r=1}^d\Big [-\frac l2+\frac{i_rl}{k},-\frac l2+\frac{(i_r+1)l}{k}\Big ].
\]
Then, set \begin{equation} 
h = \frac{d\Probb}{d\lambda_d}\;\mbox{ and }\; 
  \Probb_k  = \sum_{\substack{\underline i\in \Z^d\\ 0\leq i_r < k}}
  \Probb(C_{\underline i}) \, \U(C_{\underline i}),\;h_k =
  \frac{d\Probb_k}{d\lambda_d} = \sum_{\substack{\underline i\in \Z^d\\ 0\leq i_r < k}}
  \frac{\Probb(C_{\underline i})}{\lambda_d(C_{\underline i})}
  \ind{C_{\underline i}}, \; k\ge 1.
\end{equation}

By differentiation of measures we obtain $h_k \to h$, $\lambda_d$-$a.s.$ as $k\to\infty$.
Which in turn implies, owing to  Scheff\'e's Lemma,
\[
\lim_{k\to + \infty} \norm{h_k-h}_1 = 0.
\]
Furthermore, 
\[
\lim_{k\to +\infty} \normdp{h_k} = \normdp{h}
\] 
since $\normdp{h_k - h} \leq \Bigl( \lambda_d(C) \Bigr)^{\frac pd} \norm{h_k
- h}_1$ by  Jensen's Inequality applied to the probability measure
$\frac{\lambda_{d\,|C}}{\lambda_d(C)}$. Moreover, by
Proposition~\ref{prop:rateCubewise} we have
\begin{equation}\label{eq:limit_hk}
 \limn n^{1/d}\, d_{n,p}(\Probb_k) = \Qpn\, \normdp{h_k}^{\frac1p}.
\end{equation}

Likewise, we define an inner approximation of $\Probb$: 
%Therefore, 
denote by
\[
C^k = \bigcup_{C_{\underline i} \subset \opSuppP} C_{\underline i}
\]
the union of the hypercubes $C_{\underline i}$ lying  in
the interior of $\supp(\Probb)$.
Setting
\begin{eqnarray*}
\begin{split}
  \mathring \Probb_k  = \sum_{\substack{C_{\underline i} \subset \opSuppP}}
  \Probb(C_{\underline i} ) \, \U(C_{\underline i} )\quad &\mbox{ and }\quad 
  \mathring h_k &= \frac{d\mathring \Probb_k}{d\lambda_d} = h_k \ind{C^k},
\end{split}
\end{eqnarray*}
we have as above that 
\[
\mathring h_k \to h, \quad \lambda_d\mbox{-}\text{a.s.}\quad\text{ as } k\to +\infty.
\]
Consequently we also have
\[
\lim_{k\to\infty} \norm{\mathring h_k-h}_1 = 0 \quad\text{ and }\quad
\lim_{k\to\infty} \normdp{\mathring h_k} = \normdp{h}.
\]

We   get likewise by Proposition~\ref{prop:rateCubewise} that, for every
$k\in\N$,
\begin{equation}\label{eq:limit_ohk}
 \limn n^{1/d}\, d_{n,p}(\mathring \Probb_k) = \Qpn\cdot \normdp{\mathring
 h_k}^{\frac 1p}.
\end{equation}

\noindent $(a)$ Let $0 < \varepsilon < 1$ and $n \geq 2^d/\varepsilon$.
If we divide each edge of the hypercube $C$ into
\[
 m = \bigl\lfloor (\varepsilon n)^{1/d} \bigr\rfloor - 1
\]
intervals of equal length $l/m$, the interval endpoints define $m+1$ grid
points on each edge. Denoting by $\Gammaone = \Gammaone(\varepsilon, n)$ the product quantizer made up
by this procedure, we clearly have
\[
	\abs{\Gammaone} = (m+1)^d = \bigl\lfloor (\varepsilon n)^{1/d} \bigr\rfloor^d =:
	n_1.
\]
For this product quantizer it follows from  Proposition~\ref{prop:rappels} 
that, for all $\xi\in C$, 
%\begin{eqnarray*}
%\begin{split}
 $$
  F^p(\xi; \Gammaone)  \leq C_{\norm{\cdot}} \sum_{i=1}^d
  \Big(\frac{l}{2m}\Big)^p  \leq C_{\norm{\cdot},p,d}\, \frac{l^p}{(\varepsilon n)^{\frac pd}}.
$$
%\end{split}
%\end{eqnarray*}

% Let $\Gammatwo = \Gammatwo(n_2, k)$ be an optimal quantizer for
% $d_{n_2}^p(\Probb_k)$ and $n_2 = \lfloor(1-\varepsilon)n\rfloor$.
%Setting $\delta = \delta(n,k,\varepsilon) = \alpha \cup \gamma$,

For $n_2 = \lfloor(1-\varepsilon)n\rfloor$ let $\Gammatwo$ be an $n_2$-quantizer
such that $d^p(\Probb_k; \Gammatwo) \leq (1+\varepsilon) d^p_{n_2}(\Probb_k)$.
We clearly
have $\abs{\Gammaone \cup \Gammatwo}\leq n$ and
\begin{eqnarray*}
\begin{split}
 \hskip -0.5cm  n^{\frac pd} \biggabs{\int  F^p(\xi; \Gammaone \cup \Gammatwo) 
  d\Probb_k(\xi) - \int F^p(\xi; \Gammaone \cup \Gammatwo) d\Probb(\xi) } & \leq n^{\frac pd} \int F^p(\xi; \Gammaone \cup \Gammatwo) \abs{ h_k(\xi) - h(\xi) } d\lambda_d\xi\\ & \leq  C_{\norm{\cdot},p,d}\, \frac{l^p}{\varepsilon^{\frac pd}} \norm{h_k -
  h}_1= c_{1,\varepsilon} \norm{h_k -  h}_1 
\end{split}
\end{eqnarray*}
for $k\in\N$ and $n \geq \max\Bigl\{\frac{2^d}{\epsilon}, \frac{1}{1-\varepsilon}
\Bigr\}$. This implies
\begin{eqnarray*}
\begin{split}
  n^{\frac pd} d_n^p (\Probb) & \leq  n^{\frac pd} \int F^p(\xi; \Gammaone \cup
  \Gammatwo) d\Probb(\xi)\\ 
  &  \leq n^{\frac pd} \int F^p(\xi; \Gammaone \cup \Gammatwo) d\Probb_k(\xi) +
  c_1 \norm{h_k - h}_1\\ 
  & \leq n^{\frac pd} \int F^p(\xi; \Gammatwo) d\Probb_k(\xi) + c_1 \norm{h_k -
  h}_1\\ 
  & \leq (1+\varepsilon)\, n^{\frac pd} d_{n_2}^p(\Probb_k) + c_{1,\varepsilon}
  \norm{h_k - h}_1,
\end{split}
\end{eqnarray*}
so that we can conclude from~(\ref{eq:limit_hk}) that
\[
\limsn n^{\frac pd} d_{n}^p (\Probb) \leq
\frac{1+\varepsilon}{(1-\varepsilon)^{\frac pd}} (\Qpn)^p \normdp{h_k} +
c_{1,\varepsilon} \norm{h_k - h}_1.
\]

Letting first $k$ go to infinity and then letting $\varepsilon$ go to zero yields
\[
\limsn n^{1/d} d_n^p(\Probb) \leq \Qpn \normdp{h_k}^{\frac 1p}.
\]

\noindent $(b)$ Assume now  that $\Gammathree$ is an $n_2$-quantizer such that $\bar
d^p(\Probb;\Gammathree) \leq (1+\varepsilon) \,\bar d^p_{n_2}(\Probb)$.
% 
% 
% $\Gammathree = \Gammathree(n_2)$ is an optimal dual
% quantizer for $P$ at level $n_2$ ($i.e.$ $\bar d_{n_2,p}(\Probb)$).
% %Setting $\tau = \tau(n,\varepsilon) = \beta \cup \gamma,\, \abs{\tau}\leq n$,
Again it holds $\abs{\Gammaone \cup \Gammathree} \leq n$ and
we derive as above
\begin{equation}\label{eq:boundOpenhk}
  n^{\frac pd} \biggabs{\int  F^p(\xi; \Gammaone \cup \Gammathree) 
  d\mathring\Probb_k(\xi) - \int F^p(\xi; \Gammaone \cup \Gammathree)
  d\Probb(\xi) } \leq c_{2,\varepsilon} \norm{\mathring h_k - h}_1.
\end{equation}

Moreover, Lemma~\ref{lem:radius} yields for every $k\in\N$ the existence of
$n_{k,\varepsilon}\in\N$ such that, for all $n\geq n_{k,\varepsilon}$,
\begin{eqnarray*}
\begin{split}
 (1+\varepsilon)\, \bar d_{n_2}^p(\Probb) & \geq \bar d^p(\Probb; \Gammathree)
 \geq \int_{\conv(\Gammathree)} F^p(\xi; \Gammathree) d\Probb(\xi)\\
  & \geq \int_{C^k}
  F^p(\xi; \Gammathree) d\Probb(\xi) \geq \int_{C^k}
  F^p(\xi; \Gammaone \cup \Gammathree) d\Probb(\xi).
\end{split}
\end{eqnarray*}
Thus, we derive from~(\ref{eq:boundOpenhk}) that, for every $n\geq
\max\Big(n_{k,\varepsilon}, \frac{2^d}{\varepsilon}, \frac{1}{1-\varepsilon} \Big)$,
\begin{eqnarray*}
\begin{split}
  (1+\varepsilon)\,n^{\frac pd}\, \bar d_{n_2}^p(\Probb) & \geq n^{\frac pd}
  \int_{C^k} F^p(\xi; \Gammaone \cup \Gammathree) d\Probb(\xi)  \\
  & \geq n^{\frac pd} \int_{C^k}  F^p(\xi; \Gammaone \cup \Gammathree)
  d\mathring \Probb_k(\xi) - c_{2,\varepsilon} \norm{\mathring h_k - h}_1\\
  & \geq n^{\frac pd} d_n^p(\mathring \Probb_k) - c_{2,\varepsilon} \norm{\mathring h_k -
  h}_1,
\end{split}
\end{eqnarray*}
which yields, once  combined with~(\ref{eq:limit_ohk}),
\[
\frac{1+\varepsilon}{(1-\varepsilon)^{\frac pd}} \limin n_2^{\frac pd} \, \bar
d_{n_2,p}^p(\Probb) \geq  \Qpn \normdp{\mathring h_k}- c_{2,\varepsilon}
\norm{\mathring h_k - h}_1.
\]
Letting first $k$ go to $\infty$ and then letting $\varepsilon$ go to $0$, we get
\[
\limin n^{\frac 1d} \, \Dqbpn(\Probb) \geq  \Qpn
\normdp{h}^{\frac 1p}.\hfill \Box
\]
%\end{proof}

\begin{prop}[Singular distribution]\label{prop:rateSingular}
Assume that $\Probb$ is singular with respect to $\lambda_d$ and has compact
support. Then
 \[
  \limsn n^{\frac pd}\, \bar d_{n,p}(\Probb) = 0.
 \]

\end{prop}

\begin{proof} 
%This proof closely follows the lines of  Step 4 in Graf and
%Luschgy's proof of Zador's Theorem (see~\cite{Foundations}). 
Let $A$  be a Borel set such that $\Probb(A)=1$
and $\lambda_d(A)=0$. Let $\varepsilon>0$; by the outside regularity of
$\lambda_d$, there exists an open set $O=O(\varepsilon)\supset  A$ such that
$\lambda_d(O)\le \varepsilon$ (and $\Probb(O)=1$). Let $C$ be an open hypercube with
edges parallel to the coordinate axis, edge-length $\ell$ and  containing the closure of
$A$.

Let $C_k= \prod_{i=1}^d [c_{k,i}, c_{k,i}+\ell_i)$, $k\!\in \N$, be a countable
partition of $O$ consisting of nonempty half-open hypercubes, still with edges
parallel to the coordinate axis (see,  $e.g.$ Lemma 1.4.2 in~\cite{COH}).


Let $m=m(\varepsilon)\!\in \N$ such that $\displaystyle \sum_{k\ge m+1}\Probb(C_k)\le
\varepsilon^{\frac pd} \ell^{-p}$.

Let $n\!\in \N$, $n\ge 2^{d+1}$ and let $n_1, \ldots, n_d\ge 2$ be integers such
that the product $n^d_1 + \cdots + n_m^d \le n/2$. One designs a grid $\Gamma$
as follows.

For every $k\!\in \{1,\ldots,m\}$, we consider the lattice of $C_k$ of size $n_i
^d$ defined by 
\[ 
\Prod_{i=1}^d \Bigl\{c_{k,i}+ \frac{r_i}{n_k-1}\ell_i,\,
r_i=0,\ldots, n_k-1,\, i=1,\ldots, d\Bigr\}. 
\]

Then, one defines likewise the lattice of $C$ of size $n_{m+1}^d \le n/2$ 
\[
\Prod_{i=1}^d \Bigl\{c_{k,i}+ \frac{r_i}{n_{m+1}-1}\ell_i,\, r_i=0,\ldots,
n_{m+1}-1,\, i=1,\ldots,d\Bigr\}. 
\] 
The grid $\Gamma$ is made up with all the points
of the $m+1$ above finite lattices.

Now let $\xi \!\in A$. It is clear from the definition of the function $F_p$
that 
\[ F_p(\xi; \Gamma) \le \left\{\begin{array}{ll}
C_{\norm{.}} \big(\ell_k/n_k\big)^p& \mbox{if } \; \xi \!\in \bigcup_{k=1}^mC_k \\
C_{\norm{.}} \big(\ell/n_{m+1}\big)^p& \mbox{if } \; \xi \!\in C \setminus 
\bigcup_{k=1}^mC_k        \end{array}\right. 
\] where $C_{\norm{.}} >0$ is a
real constant only depending on the norm. As a consequence
\begin{eqnarray*}
d_{n}^p(\Probb)&=& \sum_{k=1}^m \int_{C_k}F^p(\xi;\Gamma)d\Probb(\xi) +\int_{C \setminus  \bigcup_{k=1}^mC_k } F^p(\xi; \Gamma)d\Probb(\xi)\\
&\le& C_{\norm{.}}  \Big( \sum_{k=1}^m (\ell_k/n_k)^p\Probb(C_k) + (\ell/n_{m+1})^p\Probb (C \setminus  \bigcup_{k=1}^mC_k )\Big).
\end{eqnarray*}

Set for every $k \!\in \{1,\ldots,m\}$, $\displaystyle n_k =\left \lfloor \frac{\ell_k(n/2)^{\frac 1d}}{(\sum_{k'=1}^d\ell_{k'}^d)^{\frac 1d}}\right \rfloor$ 
and $\displaystyle n_{m+1} = \lfloor (n/2)^{\frac 1d} \rfloor$. Note that 
\[
\sum_{k'=1}^d\ell_{k'}^d= \sum_{k=1}^m \lambda_d(C_k) \le \lambda_d(O)\le \varepsilon.
\]
Elementary computations show that for large enough $n$,  all the integers $n_k$ are greater
than $1$ and that 
\begin{eqnarray*} 
\sum_{k=1}^m (\ell_k/n_k)^p\Probb(C_k) + (\ell/n_{m+1})^p\Probb(C \setminus  \bigcup_{k=1}^mC_k ) & \le & (\sum_{k'=1}^d\ell_{k'}^d)^{\frac pd}(n/2)^{-\frac pd} \Probb\big(\cup_{1\le k\le m} C_k\big) + \\
&& +(n/2)^{-\frac pd} \ell^p \Probb\big(C \setminus  \bigcup_{k=1}^mC_k \big)\\ 
\end{eqnarray*}
so that \[ \limsup_n n^{\frac pd} d_{n}^p(\Probb) \le C_{\norm{.}} (\varepsilon/2)^{\frac pd}\]
which in turn
implies, by letting $\varepsilon $ go to $0$, that $\displaystyle \limsup_n n^{\frac pd}
d_{n}^p(\Probb)=0$. 
% \textcolor{red}{It remains to insert that in the global proof. In GL's proof of
% Zador's theorem, a ``Lemma 6.5" is called upon (p. 81 in the LN). But I guess it
% is almost trivial reasoning in $\limsup$ and easy considerations on bit allocation. \\
% Next theoretical question is the empirical measure theorem...Probably true but
% not as/so easy }
\end{proof}


% \begin{thm}
% Let $X\in L^{p+\delta}(\Prob)$, absolutely continuous w.r.t. to $\lambda_d$ and
% $\Probb = \ProbX = h \lambda_d$. Then
% \[
% n^{\frac 1d} \, \Dqbpn(\Probb) \geq  \Qpn
% \normdp{h}^{\frac 1p}.
% \]
% \end{thm}
 {\sc Proof of Theorem~\ref{thm:DQRate}:}
 Claim~(a) follows directly from Propositions~\ref{prop:rateCompact} , 
~\ref{prop:rateSingular} and  Proposition~\ref{prop:subLin}: Assume $\Probb = \rho \Probb_a +(1-\rho)\Probb_s$ where $\Probb_a= \frac {h}{\rho}\lambda_d$ and $\Probb_s$ denote the absolutely continuous and singular part of $\Probb$ respectively. The following inequalities hold true 
\[
 \rho \bar d_{n,p} (\Probb_a)\le \bar d_{n,p}(\Probb)\le \rho \bar d_{n_1,p}(\Probb_a)+(1-\rho) \bar d_{n_2,p}(\Probb_s)
 \]
 for every triplet of integers $(n_1,n_2,n)$ with $n_1+n_2\le n$. Set $n_1= n_1(n)= \lfloor (1- \varepsilon n) \rfloor$, $n_2= n_2(n)= \lfloor  \varepsilon n \rfloor$. Then we derive that 
 \[
\hskip -1cm\rho  \Qpn\cdot \normdp{\frac{h}{\rho} }^{\frac 1p} \liminf_n n^{\frac p d}\bar d_{n,p} (\Probb_a)\le  \liminf_n n^{\frac p d}\bar d_{n,p} (\Probb)\le  \limsup_n n^{\frac p d}\bar d_{n,p} (\Probb)\le \rho (1-\varepsilon)^{-\frac p d}  \Qpn\cdot \normdp{\frac{h}{\rho}}^{\frac 1p}
 \]
 Letting $\varepsilon $ go to $0$ completes the proof.
 
%the fact that it holds $\bar d_{n,p}(X)\leq
% d_{n,p}(X)$ for every $n\in\N$.
 Furthermore, part $(c)$ 
 was derived in~\cite{dualStat}, Section~5.1. 
 %is an easy consequence of example~\ref{ex:uniform}.
 Hence, it remains to prove $(b)$ 
  \begin{proof} {\sc Step 1.} (Lower bound)
If $X$ is compactly supported, the assertion follows from Proposition
\ref{prop:rateCompact}. Otherwise, set for every $R\!\in(0,\infty)$, 
$$
C_{_R} = [-R,R]^d\; \mbox{ and }\; \Probb(\cdot | C_k) = \frac{h\ind{C_k}}{\Probb(C_k)} \lambda_d,\; k\in\N.
$$
Proposition~\ref{prop:rateCompact} yields again
\begin{equation}\label{eq:limitCk}
  \limn n^{\frac 1d} \, \bar d_{n,p}(\Probb(\cdot | C_k)) =  \Qpn \cdot
\normdp{h\ind{C_k}/\Probb(C_k)}^{\frac 1p},
\end{equation}
so that $\dqbpn(\Probb) \geq \Probb(\cdot | C_k) \dqbpn(\Probb(\cdot |
C_k))$ implies for all $k\in\N$
\[
\limin n^{\frac 1d} \, \bar d_{n,p}(\Probb) \geq  \Qpn \cdot
\normdp{h\ind{C_k}}^{\frac 1p}.
\]
Sending $k$ to infinity, we get at
\[
\limin n^{\frac 1d} \, \bar d_{n,p}(\Probb) \geq  \Qpn \cdot
\normdp{h}^{\frac 1p}.
\]

%\noindent {\sc Step 2.} \textcolor{red}{(Upper bound when $\supp(\Probb) = \R^d$).} Assume $\supp(\Probb) = \R^d$.
%For the upper bound let $0<\varepsilon< 1$ and set 
%\[
%n_1 = \bigl\lfloor(1-\varepsilon)n\bigr\rfloor, \quad n_2 = \lfloor\varepsilon n\rfloor \quad\text{ for
%} n\geq \max\biggl(\frac{1}{\varepsilon}, \frac{1}{1-\varepsilon} \biggr).
%\]

%Consider the decomposition
%\[
%\Probb = \Probb(C_k) \cdot \Probb(\cdot | C_k) + \Probb(C_k^c) \cdot \Probb(\cdot |
%C_k^c),
%\]
%Let $\beta_1=\beta(n_1)$ be optimal for the compact dual quantization error
%$d_{n_1,p}^p\bigl( \Probb(\cdot | C_k) \bigr)$ and let $\beta_2=\beta(n_2)$ be optimal for
%$\bar d_{n_2,p}^p\bigl(\Probb(\cdot | C_k^c)\bigr)$.

%  \smallskip \textcolor{red}{Let $\eta_0>0$ and $K(k,\eta_0) =
% C_{k+\eta_0}\setminus C_{k+\eta_0/2} \subset
% \mathring{\overbrace{\supp(\Probb)}}=\R^d$. Consequently, for $n$ large
% enough, $K_{k,\eta_0}\subset \conv(\beta_2)$ owing to Lemma~\ref{lem:radius}.
% Then one derives that \[ C_k \subset \conv(K_{k,\eta_0})\subset
% \conv(\beta_2). \] % On the other hand, by the very definition of $d_n$, it is
% clear that $C_k \subset \conv(\beta_1)$ if $n\ge d+1$. This implies that }

% %For large enough $n$, this yields %\[ %\textcolor{red}{C_k \subset
% \conv(\beta_1) \;\mbox{ and }\;  C_k \subset  \conv(\beta_2),} %\] %which
% implies \begin{eqnarray*} \begin{split} \bar d_{n,p}^p(\Probb) & \leq
% \int_{\conv(\beta_2)} F^p(\xi, \beta_1 \cup \beta_2)\, d\Probb(\xi) +
% \int_{\conv(\beta_2)^c} \min_{b\in\beta_1 \cup
%  \beta_2} \norm{\xi-b}^p\, d\Probb(\xi) \\
% & = \Probb(C_k)  \int_{C_k} F^p(\xi; \beta_1 \cup \beta_2)\,
% d\Probb(\xi|C_k) \\ 
% & \quad + \Probb(C_k^c)  \int_{C_k^c \cap \conv(\beta_2)} F^p(\xi;
% \beta_1 \cup \beta_2)\, d\Probb(\xi|C_k^c) \\
% & \textcolor{red}{\quad + \Probb(C_k) \underbrace{\int_{C_k \cap
% \conv(\beta_2)^c} \min_{b\in\beta_1 \cup
%  \beta_2} \norm{\xi-b}^p\, d\Probb(\xi|C_k)}_{=0}}\\ 
% & \quad + \Probb(C_k^c) \int_{C_k^c \cap \conv(\beta_2)^c} \min_{b\in\beta_1
%  \cup \beta_2} \norm{\xi-b}^p\, d\Probb(\xi|C_k^c)\\
% & \leq  \Probb(C_k)  \int_{C_k} F^p(\xi; \beta_1)\,
% d\Probb(\xi|C_k) \\
% & \quad + \Probb(C_k^c) \bigg[ \int_{\conv(\beta_2)} F^p(\xi;\beta_2)\,
% d\Probb(\xi|C_k^c) + \int_{\conv(\beta_2)^c} \min_{b\in\beta_2}
% \norm{\xi-b}^p\,
% d\Probb(\xi|C_k^c) \biggr]\\
% & = \Probb(C_k) \cdot d_{n_1,p}^p\bigl(\Probb(\cdot | C_k)\bigr) +
% \Probb(C_k^c) \cdot \bar d_{n_2,p}^p\bigl(\Probb(\cdot | C_k^c)\bigr).
% \end{split} \end{eqnarray*}

% Hence we get, using~(\ref{eq:limitCk}), \begin{eqnarray*} \begin{split} \limsn
% n^{\frac pd}\, \bar d_n^p(\Probb) &\leq (1-\varepsilon)^{-\frac pd}\,
% (\Qpn)^p\,
%\normdp{h\ind{C_k}}\\
% & \qquad + \varepsilon^{-\frac pd}\, \Probb(C_k^c) \limsn n^{\frac pd}\, \bar
% d_n^p\bigl(\Probb(\cdot|C_k^c) \bigr). \end{split} \end{eqnarray*}

% \textcolor{red}{At this stage, the $d$-dimensional extended Pierce
% Lemma~\ref{PdtQErrop} yields \[ \limsn \Probb(C_k^c)\, n^{\frac pd}\, \bar
% d_{n,p}^p\bigl(\Probb(\cdot|C_k^c) \bigr) \leq C_{d,p, \delta} \int_{C_k^c}
% \norm{\xi}^{p+\delta} \, d\Probb(\xi). %+ c_2 \Probb(C_k^c), \] Letting  $k$
% go to infinity and   $\varepsilon$ go  to $0$ successively, we arrive at \[
% \limsn n^{\frac 1d}\, \bar d_{n,p}(\Probb) \leq \Qpn \normdp{h}^{\frac 1p}. \]
% }

\noindent  {\sc Step 2} ({\em Upper bound}, ${\rm supp}(\Probb)= \R^d$). Let
$\rho\!\in(0,1)$. Set $K=C_{k+\rho}$ and $K_{\rho}= C_k$. Let
$\Gamma_{k,\alpha,\rho}$ be the lattice grid associated to $K\setminus K_{\rho}$
with edge $\alpha>0$ as defined  in the proof of Proposition~\ref{prop:rateCubewise}.
%remark that follows the ``firewall" Lemma~\ref{lem:firewall}. 
%\textcolor{red}{This remark seems to be missing now\ldots}
It is straightforward that there exists a real constant $C>0$ such that
\[
\forall\, k\in\N, \forall\, \rho\in(0,1), \forall\,\alpha\in(0,\rho):\quad
|\Gamma_{\alpha,\rho}|\le C d\rho k^{d-1}\alpha^{-d}  .
\] 
% $$ |\Gamma_{\alpha,\rho}|\le C d\rho
% k^{d-1}\alpha^{-d}\quad \mbox{ as } \quad \alpha\to 0. $$
Let $\varepsilon\!\in (0,1)$. For every $n\ge 1$, set $\alpha_n = \tilde
\alpha_0 n^{-\frac 1d}$ where $\tilde \alpha_0\in(0,1)$ is a real constant and
\[ 
n_0= |\Gamma_{k,\alpha_n,\rho}|,\quad n_1=\lfloor
(1-\varepsilon)(n-n_0)\rfloor,\quad n_2= \lfloor \varepsilon (n-n_0)\rfloor, 
\]
so that $\alpha_n\in(0.\rho)$, $n_0+n_1+n_2\le n$ and $n_i\ge1$ for large enough
$n$. 

\smallskip For every $\xi\!\in K_{\rho}=C_k$, for every grid $\Gamma\subset
\R^d$ containing $K_{\rho}$, we know by the ``firewall" 
Lemma~\ref{lem:firewall} that
\[ 
F^p(\xi;(\Gamma\cap
\mathring{K})\cup\Gamma_{\alpha,\rho})\le (1+\eta)^p F^p(\xi;\Gamma)
+(1+\eta)^p(1+1/\eta)^pC_{\norm{.}}\alpha^p. 
\]

Let $\Gammaone=\Gammaone(n_1,k)$ be an $n_1$ quantizer such that
$d^p_{n_1}(\Probb(.|C_k); \Gammaone) \leq (1+\eta) d^p_{n_1}(\Probb(.|C_k))$.
%Let $\Gammaone=\Gammaone(n_1)$ be an optimal dual quantizer for the error
%modulus $d_{n_1,p}$ and the distribution $\Probb(.|C_k)$ at level $n_1$. 
Set $\Gammaone'=((\Gammaone\cap \mathring C_{k+\rho})\cup
\Gamma_{k,\alpha_n,\rho})$. One has $\Gammaone'\subset C_{k+2\rho}$ for large
enough $n$ (so that $\alpha_n<\rho$).

Let moreover $\Gammatwo=\Gammatwo(n_2,k)$ be an $n_2$ quantizer such that
$\bar d^p_{n_2}(\Probb(.|C_k^c); \Gammatwo) \leq (1+\eta)
\bar d^p_{n_2}(\Probb(.|C_k^c))$. 
%$\Gammatwo=\Gammatwo(n_2)$ be  an optimal
%dual quantizer for $\bar d_{n_2,p}$ at level $n_2$ for $P(.|C_k^c)$.   
For $n\ge n_{\rho}$, we may assume that
$C_{k+2\rho}\subset \conv{\Gammatwo}$ owing to Lemma~\ref{lem:radius} since
$C_{k+2\rho} =\conv(C_{k+2\rho}\setminus C_{k+\frac 32\rho})$ and
$C_{k+2\rho}\setminus C_{k+\frac 32\rho}\subset \mathring{\overbrace{{\rm supp}
\Probb(.|C_k^c)}}$.
As a consequence $\Gammaone'\subset \conv(\Gammatwo)$ so that
$\conv(\Gammaone')\subset \conv (\Gammatwo)=\conv(\Gamma)$ where
$\Gamma=\Gammaone'\cup \Gammatwo$and \[ C_{k+\rho}\subset
\conv(\Gamma)=\conv(\Gammatwo). 
\] 
Now
\begin{eqnarray*}
\bar d_{n}^p(\Probb) & \le & \int_{C_k}\Big(F^p(\xi;\Gamma)\mbox{\bf
1}_{\{\xi \in \conv (\Gammatwo)\}}+ \underbrace{ d(\xi,\Gamma)^p\mbox{\bf
1}_{\{\xi \notin \conv (\Gammatwo)\}} }_{=0}\Big) d\Probb(\xi)\\ &&+
\int_{C^c_k}\left(F^p(\xi;\Gamma)\mbox{\bf 1}_{\{\xi \in \conv (\Gammatwo)\}}+
d(\xi,\Gamma)^p \mbox{\bf 1}_{\{\xi \notin \conv
(\Gammatwo)\}}\right)d\Probb(\xi).
\end{eqnarray*}
%
Using that, for every $\xi\!\in C_k$, 
\begin{eqnarray*}
F^p(\xi;\Gamma) &\le& F^p(\xi;\Gammaone')\\
&\le& (1+\eta)^p\Big(F^p(\xi;\Gammaone)+
(1+1/\eta)^p\,C_{\norm{.}}\,\alpha_n^p\Big)
\end{eqnarray*}
implies
\begin{eqnarray*}
\bar d_{n}^p(\Probb) & \le & \Probb(C_k) (1+\eta)^p\Big(
(1+\eta)\,d^p_{n_1}(\Probb(.|C_k)) + (1+1/\eta)^p\,C_{\norm{.}}\,\tilde
\alpha_0 \,n^{-\frac 1d} \Big)\\
 && + \Probb(C^c_k)\,(1+\eta)\, \bar d^p_{n_2}(\Probb(.|C^c_k)).
 \end{eqnarray*}
 %
 Consequently
 %
 \begin{eqnarray*}
n^{\frac pd} \bar d_{n}^p(\Probb) & \le & \Probb(C_k)
(1+\eta)^p\Big[(1+\eta)\, \Big(\frac{n}{n_1}\Big)^{\frac pd}\,n_1^{\frac
pd}\,d^p_{n_1}(\Probb(.|C_k)) 
+ (1+1/\eta)^pC_{\norm{.}}\tilde \alpha_0 \Big]\\ 
 && + (1+\eta)\,
\Big(\frac{n}{n_2}\Big)^{\frac pd}\Probb(C^c_k)\, n_2^{\frac pd}\,\bar
d_{n_2}^p(\Probb(.|C^c_k))
 \end{eqnarray*}
 which in turn implies, using Proposition~\ref{prop:rateCompact} for the modulus $d_{n,p}$ and the $d$-dimensional version of the extended Pierce
 Lemma (Proposition~\ref{PdtQErrop}) for $\bar d_{n,p}$,
 \begin{eqnarray*}
 \limsup_nn^{\frac pd} \bar d_{n}^p(\Probb)  &\le&  \Probb(C_k)
 (1+\eta)^p\left(\left(\frac{(1+\eta)^{-p/d}}{(1-\varepsilon)(1-Cd\rho
 k^{d-1}\tilde \alpha^{-d}_0)}\right)^{\frac pd}Q^{dq}_{\norm{.}}\norm{h\mbox{\bf 1}_{C_k}}_{L^{\frac{d}{d+p}}}\right.\\ &&\left. 
 +  (1+1/\eta)^pC_{\norm{.}}\tilde \alpha_0 \right)\\ 
 &&+ \Probb(C^c_k) \,(1+\eta)\,\,C_{p,d}\,\norm{X \mbox{\bf 1}_{\{X\in C^c_k\}}
 }^p_{L^{p+\delta}} \left(\frac{1}{\varepsilon(1-Cd\rho k^{d-1}\tilde \alpha^{-d}_0)}\right)^{\frac pd}.
 \end{eqnarray*}
 %
 One concludes by letting successively $\rho$, $\tilde \alpha_0$, $\eta$ go
 to $0$,   $k \to \infty$ and finally $\varepsilon$ to  $0$.
 
 
\medskip
\noindent {\sc Step 3.} (Upper bound: general case). Let $\rho\!\in (0,1)$. Set
$\Probb_{\rho}= \rho \Probb+(1-\rho)\Probb_0$ where $\Probb_0= {\cal N}(0;I_d)$
($d$-dimensional normal distribution). It is clear from the very definition of
$\bar d_{n,p}$ that $\bar d_{n,p}(\Probb)\le \frac{1}{\rho} \bar
d_{n,p}(\Probb_{\rho})$ since $\Probb\le \frac{1}{\rho} \Probb_{\rho}$. The
distribution $\Probb_{\rho}$ has $h_{\rho}= \rho h+(1-\rho)h_0$ as a density (with obvious
notations) and one concludes by noting that 
\[ 
\lim_{\rho\to
0}\|h_{\rho}\|_{d/(d+p)}= \|h\|_{d/(d+p)} 
\] 
owing to the Lebesgue dominated convergence Theorem.
\end{proof}

{\em Proof of Proposition~\ref{prop:asymptQ}:}
%\begin{proof}
Using H\"older's inequality one easily checks that for $0\leq r\leq p$ and
$x\in\R^d$ it holds
\[
	\abs{x}_{\ell^r} \leq d^{\frac{1}{r} - \frac{1}{p}}\, \abs{x}_{\ell^p}.
\]
 
Moreover, for $m\in\N$ set $n = m^d$ and let $\Gamma'$ be an optimal quantizer
for $d_{m,p}(\Unif)$ (or at least $(1+\varepsilon)$-optimal for $\varepsilon > 0$).
Denoting $\Gamma = \prod_{i=1}^d \Gamma'$, it then follows from Proposition~\ref{prop:rappels}$(b)$ that
\[
n^\frac{p}{d}\, d^p_n\big(\Unifd\big) \leq n^\frac{p}{d}\, d^p\big(\Unifd;\Gamma\big) = m^p
	\sum_{i=1}^d d^p\big(\Unif; \Gamma'\big)  = d\, m^p \,d^p_m\big(\Unif\big).
\]

Combining both results and reminding that $\Qpn$ holds as an infimum, we obtain
for $r\!\in [0,p]$,  
\[
\bigl(Q^{\text{dq}}_{\abs{\cdot}_{\ell^r}, p, d}\bigr)^p \leq d^{\frac{p}{r}-1}
\, n^{\frac{p}{d}}\, d^p_{n, \abs{\cdot}_{\ell^p}}(\Unifd) \leq d^{\frac{p}{r}}
\, m^p \,d^p_m\big(\Unif\big),
\]
which finally proves the assertion by sending $m\to+\infty$.$\qquad \Box$
%\end{proof}
