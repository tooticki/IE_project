\section{Dual quantization: background and basic properties}\label{sec:defs}

Throughout  the paper, except specific mention, $\R^d$ is equipped with a norm $\norm{\cdot}$. 

\subsection{More background}
%First we recall the
%definitions of the regular quantization error moduli for a r.v.  $X:
%(\Omega, \mathcal{S}, \Prob) \to (\R^d, \mathcal{B}^d)$.
%\begin{definition}
%Let $X\in L_{\R^d}^p(\Prob)$  and $\Gamma \subset \R^d$.
%\begin{enumerate}
%  \item We define the $L^p$-mean regular quantization error for a grid
%  $\Gamma$ as
%\[ 
%	e_p(X; \Gamma)  = (\E \min_{x\in\Gamma} \norm{X-x}^p)^{1/p} =\|d(X,\Gamma)\|_{L^p}.
%\]  
%	\item The optimal regular quantization error, which can be achieved by a grid
%	$\Gamma$ of size not exceeding $n$ is given by
%\[ 
%	e_{n,p}(X) = \inf \bigl\{ e_p(X; \Gamma): \Gamma \subset \R^d, \abs{\Gamma}\leq
%	n \bigr\}.
%\] 

%\end{enumerate}
%\end{definition}

%%A well known result characterizes the regular quantization error as the best
%%approximation to $X$ which can be achieved by a Borel transformation that
%%takes not more than $n$ values.
%%Morerover, this best approximation is achieved by a nearest neighbor
%%projection operator induced by a Voronoi partition of $\R^d$.
%%
%%
%%\begin{prop}\label{prop:regularQNN}
%%Let $X\in L^p_{\R^d}(\Prob)$. Then
%%\begin{equation*}
%%\begin{split}
%%e_{n,p}(X) & = \inf \bigl\{  \norm{X-f(X)}_p: f: \R^d \to \R \text{ borel
%%mb}, \abs{f(\R^d)}\leq n \bigr\}\\
%%& = \inf \bigl\{ \norm{X-\pi_\Gamma(X)}_p: \Gamma \subset \R^d
%%\abs{\Gamma}\leq n \bigr\},
%%\end{split}
%%\end{equation*}
%%where $\pi_\Gamma:\R^d \to \R$ denotes a nearest neighbor projection operator,
%%$i.e. $ a function 
%%\[
%%	\pi_\Gamma(\xi) = \sum_{x\in\Gamma} x \cdot \ind{C_x(\Gamma)}(\xi)
%%\]
%%with  $(C_x(\Gamma))_{x\in\Gamma}$ is a Borel partition of $\R^d$
%%satisfying
%%\[
%%	C_x(\Gamma) \subseteq \{ \xi \in \R^d: \norm{\xi - x} \leq \min_{y\in\Gamma}
%%	\norm{\xi-y}\}.
%%\]
%%\end{prop}

%Following~\cite{dualStat}, the dual quantization error can be introduced as follows.
%%Concerning the dual quantization error we define the following.

%\begin{definition}
%Let $X\in L^p(\Prob)$  and $\Gamma \subset \R^d$.
%\begin{enumerate}
%  \item We define the local $p$-dual quantization error induced by a grid $\Gamma$ as
%\[
%	F_p(\xi; \Gamma) = \inf  \Bigl\{ \Bigl( \sum_{x\in\Gamma} \lambda_x
%	\norm{\xi - x}^p\Bigr)^{\frac 1p} : \lambda_x \in [0,1] ,\,\sum_{x\in\Gamma}
%	\lambda_x x = \xi,\,
%	\sum_{x\in\Gamma} \lambda_x = 1  \Bigr\}.
%\]
%  \item The $L^p$-mean dual   quantization error for $X$ induced by a grid
%  $\Gamma$ is then given by
%\begin{eqnarray*} 
%	\Dqp(X; \Gamma) &=&  \| F_p(X; \Gamma)\|_{L^p}\\
%	& =& \left(\E \inf  \Bigl\{ 
%	\sum_{x\in\Gamma} \lambda_x \norm{X - x}^p : \lambda_x \in [0,1] \text{ and }  \sum_{x\in\Gamma} \lambda_x x = X,
%	\sum_{x\in\Gamma} \lambda_x = 1 \Bigr\} \right)^{1/p}.
%\end{eqnarray*}
%  \item The optimal dual quantization error, which can be achieved by a grid
%	$\Gamma$ of size not exceeding $n$ will be denoted by
%\[ 
%	\Dqpn(X) = \inf \bigl\{ \Dqp(X; \Gamma): \Gamma \subset \R^d, \abs{\Gamma}\leq
%	n \bigr\}.
%\]
%  \item The extended $L^p$-mean dual quantization error induces by a grid
%  $\Gamma$ is defined by
%  \[
%  \bar d_p(X,\Gamma)=\Big\|F_p(X;\Gamma)\,\mbox{\bf 1}_{\conv(\Gamma)}(X)+{\rm
%  dist}(X,\Gamma)\,\mbox{\bf 1}_{\conv(\Gamma)^c}(X)\Big\|_{L^p}.
%  \]
%  \item The optimal extended dual quantization error, which can be achieved by a grid
%	$\Gamma$ of size not exceeding $n$ will be denoted by
%\[ 
%	\Dqbpn(X) = \inf \bigl\{ \Dqbp(X; \Gamma): \Gamma \subset \R^d, \abs{\Gamma}\leq
%	n \bigr\}.
%\]
%\end{enumerate}
%\end{definition}
In the introduction, the definitions related to Voronoi (or regular)and  dual quantizations of a r.v.  $X$ defined on a probability space $(\Omega,{\cal S}, \Prob)$ have been recalled (see~(\ref{def:dnpsharp})-(\ref{def:dbarnp})). 
The aim of this section is to come back briefly to the origin and the
motivations which led us to introduce dual quantization in~\cite{dualStat}. On the way, we will also recall several basic results on dual quantization established in~\cite{dualStat}.
First, we will assume throughout the paper that the r.v.  of interest, $X$, is
truly  $d$-dimensional in the sense that
\[
{\rm aff.dim}({\rm supp}(\Prob_{_X})) =  d.
\] 
%\begin{remarks}
%\begin{enumerate}
 % \item 
 
%\smallskip 
Let us start by a few practical points.  First note that although all
these definitions are related to a r.v.  $X$, in fact it only depends on the distribution $\Probb = \ProbX$, so we will also often write $\Dqp(\Probb, \Gamma)$ for $\Dqp(X, \Gamma)$ and $\Dqpn(\Probb)$.
%for $\Dqpn(X)$ where $\Probb = \ProbX$.
% \item In some cases it will be useful to consider instead of a grid
% $\Gamma = \{x_1, \ldots, x_n\}$ the $n$-Tuple $\gamma = (x_1, \ldots, x_n)$ and
% define the local dual quantization error as
% \[
% \Fp_n(\xi; \gamma) = \Fp_n(\xi; x_1, \ldots, x_n) = \LP{\xi}.
% \]
% If all the components of $\gamma$ are pairwise distinct, both definitions
% clearly coincide. Otherwise, we may remove equal components of $\gamma$, since
% \[
% \Fp_{n+1}(\xi; x_1, \ldots, x_i, x_i,\ldots, x_n) = \Fp_{n}(\xi; x_1, \ldots,
% x_i,\ldots, x_n).
% \] 
% 
% We therefore may abuse notations and write $\Fp(\xi; \gamma)$ for
% $\Fp(\xi; \Gamma_\gamma)$ with $\Gamma_\gamma = \{ x_i: 1 \leq i \leq n\}$ as
% well as $\Fp_n(\xi; \Gamma)$ with either $n = \abs{\Gamma}$ or $\gamma_\Gamma$
% is extended with duplicate elements of $\Gamma$ when $n$ is supposed to be
% greater than $\abs{\Gamma}$.
%\item 
%Since  $F_p(\xi; \Gamma) = +\infty$ if and only if $\xi \notin
%\conv\{\Gamma\}$, the definition of the optimal quantization error only makes 
%only sense for probability distributions with bounded support.
 Furthermore, to alleviate notations, we will denote from now on $F^p$, $d^p$ and $\bar
 d^p$, \dots instead of  $(F_p)^p$, $(d_p)^p$ and $(\bar d_p)^p$,\dots
% \end{enumerate}
%\end{remarks}

%
% =============== be careful : notations! etc=============  To establish the
% already mentioned link between the above definition and stationary
% quantization rules, we have to precise the notion of intrinsic stationarity. 
% Nevertheless, we first want to clarify what we mean by a random operator $\J$.
% %: \R^d\to \R$. % for a given grid $\Gamma \subset \R^d$: The random operator
% $\J: \R^d\to \R$, defined on some probability space $(\tilde{\Omega},
% \tilde{\mathcal{S}}, \tilde{\Prob})$, is actually supposed to map
% $\tilde{\Omega} \times \R^d$ to $\R$ such that, for every $\xi \in \R^d$,
% $\J(\cdot, \xi): (\tilde{\Omega}, \tilde{\mathcal{S}}, \tilde{\Prob}) \to
% (\R^d, \mathcal{B}^d)$ is a r.v. and, for every $\tilde \omega \in \tilde
% \Omega$, $\J(\tilde{\omega}, \cdot): \R^d \to \R$ is a Borel-function.  To
% ease notations, we usually drop the first argument $\tilde{\omega}$ and assume
% that the random operator $\J$ is defined on the common abstract probability
% space $(\Omega, \mathcal{S}, \Prob)$, but being independent of any other r.v.
% which will occur in this paper.   \begin{definition} Let $\Gamma \subset \R^d,
% \, \abs{\Gamma} < \infty$. We call a random operator $\JG: \R^d \to \Gamma$
% {\it intrinsic stationary}, if %it holds for any grid $\Gamma \subset \R^d$
% and for every $\xi \in %\conv\{\Gamma\}$: \[ %\forall \xi \in \conv\{\Gamma\}:
% \qquad \E \bigl(\JG(\xi) \bigr) = \xi \qquad \forall \xi \in \conv\{\Gamma\}.
% \] \end{definition}  Note, that this property is equivalent to the condition
% \[ \E\bigl(\JG(X)|X\bigr) = X \] for %any grid $\Gamma \subset \R^d$ and any
% r.v. $X: (\Omega, \mathcal{S}, \Prob) \to (\R^d, \mathcal{B}^d)$ such that
% $\JG$ and $X$ are independent and $\supp(\Prob_Y) \subset \conv\{\Gamma\}$.  
% Moreover, such an intrinsic stationarity condition, as we will see later,
% makes no sense for $\xi\notin\conv\{\Gamma\}$.%\ldots intrinsic stationarity
% %impossible for $\xi\notin\conv\{\Gamma\}$\ldots   As a matter of fact, the
% optimal $L^p$-th mean approximation induced by such an intrinsic stationary
% random operator equals the above defined optimal dual quantization error. 
% ================================  \begin{prop}\label{prop:DQLinkStat} Let
% $X\in L^p(\Prob)$ with bounded support. Then \begin{equation*} \begin{split}
% e_{n,p}(X) \leq \dqpn(X) = \inf\bigl\{ \norm{X -  \JG(X)}_p: &\,
% \JG:\R^d\to\Gamma
%\text{ is intrinsic stationary},\\
% & \supp(\ProbX) \subset\conv\{\Gamma\},\, \abs{\Gamma} \leq n \bigr\}.
% \end{split} \end{equation*} \end{prop} \begin{proof} The first inequality
% follows from suppressing the constraint $\sum_{x\in\Gamma} \lambda_x\cdot x =
% \xi$ in the definition of $F_p^p(\xi; \Gamma)$. To prove
% \begin{equation}\label{eq:proofStatIneq1} \begin{split} \dqpn(X) \leq
% \inf\bigl\{ \E \norm{X -  \JG(X)}^p: & \, \JG:\R^d\to\Gamma \text{
%is intrinsic stationary},\\
% & \supp(\ProbX) \subset\conv\{\Gamma\},\, \abs{\Gamma} \leq n \bigr\},
% \end{split} \end{equation} fix a grid $\Gamma\subset\R^d$ and an intrinsic
% random operator $\JG:\R^d\to\Gamma$ such that $\supp(\ProbX)
% \subset\conv\{\Gamma\}$ and $ \abs{\Gamma} \leq n$.  Let $\xi\in\supp(\ProbX)$
% and set \[ \lambda_x = \lambda_x(\xi) = \Prob(\JG(\xi) = x), \qquad
% x\in\Gamma. \] Consequently, \[ \lambda_x \in [0,1], \quad \text{  } \quad
% \sum_{x\in\Gamma} \lambda_x = 1, \] and since $\JG$ is intrinsic stationary,
% it also holds \[ \sum_{x\in\Gamma} \lambda_x(\xi) \cdot x = \E(\JG(\xi)) =
% \xi. \]  Thus, \begin{equation*} \begin{split} F_p^p(\xi; \Gamma) & = \inf 
% \biggl\{  \sum_{x\in\Gamma} \mu_x \norm{\xi - x}^p : \mu_x \in   [0,1],
% \sum_{x\in\Gamma} \mu_x x = \xi,
%	\sum_{x\in\Gamma} \mu_x = 1  \biggr\}\\
% %\LP{\xi} & \leq \sum_{x\in\Gamma} \lambda_x \norm{\xi - x}^p = \E\norm{\xi -
% \JG(\xi)}^p, \end{split} \end{equation*} which yields by the independence of
% $X$ and $\JG$ \[ \E \Fp(X; \Gamma) \leq \E\norm{X -\JG(X)}^p. \]  Since
% $\dqp(X; \Gamma) = +\infty$, whenever
% $\supp(\ProbX)\not\subset\conv\{\Gamma\}$, we obtain~(\ref{eq:proofStatIneq1})
% by taking the infimum overall grids $\Gamma$ with  $\supp(\ProbX)
% \subset\conv\{\Gamma\}$ and $ \abs{\Gamma}
%\leq n$.\\
%  To prove the converse inequality, fix again a grid $\Gamma\subset \R^d$ such
% taht $\supp(\ProbX) \subset\conv\{\Gamma\}$ and $ \abs{\Gamma} \leq n$. 
% Denote $\Gamma = \{x_1, \ldots, x_k\}, k \leq n$ and let $\mathcal{I} =
% \bigl\{J \subset \{1, \ldots, k\}: \abs{J} = d\!+\!1 \text{ and } \rk A_J =
% d+1 %\adim\{x_j, j\in J\} = d \bigr\}$. W.l.o.g. we may assume that
% $\mathcal{I}\neq\emptyset$.  For $I\in\mathcal{I}$, we denote by $D_I$ the
% optimality regions for $\Gamma$ as introduced in section
% \ref{sec:preliminaries}. We then may choose a Borel partition
% $(C_I)_{I\in\mathcal{I}}$ of $\supp(\ProbX)$ satisfying \[ C_I \subset D_I,
% \quad I \in \mathcal{I}. \]  As a consequence, for each $\xi \in
% \supp(\ProbX)$ there is a unique $I \in\mathcal{I}$ such that the solution to
% the LP \[ \LPk{\xi} \] is given (after reordering of rows) by \[
% \lambda^I(\xi) = \left(\begin{smallmatrix} A_I^{-1} \left(\begin{smallmatrix}
% \xi\\ 1 \end{smallmatrix}  \right) \\ 0 \end{smallmatrix} \right), \] which in
% turn yields %with $A_I = %\left(\begin{smallmatrix} x^I_1 \ldots
% x^I_{d\!+\!1}\\ 1 \ldots 1 %\end{smallmatrix}\right)$ \[ %\Fp(\xi; \Gamma) =
% \sum_{I\in\mathcal{I}} \Biggl[ \sum_{j=1}^{d+1} %\lambda^I_j(\xi) \norm{\xi -
% x^I_j}^p \Biggr] \ind{C_I}(\xi). \Fp(\xi; \Gamma) = \sum_{I\in\mathcal{I}}
% \Biggl[ \sum_{j=1}^{k} \lambda^I_j(\xi) \norm{\xi - x_j}^p \Biggr]
% \ind{C_I}(\xi). \]  Suppose that $U$ is a uniform distributed r.v. independent
% of $X$, then we may define the random operator $\JG:\supp(\ProbX) \to \Gamma$
% as \[ %\JG(\xi) = \sum_{I\in\mathcal{I}} \Biggl[ \sum_{j=1}^{d+1} x^I_j
% %\ind{]\sum_{l=1}^{j-1} \lambda^I_l(\xi), \sum_{l=1}^{j} \lambda^I_l(\xi)
% %]}(U)\Biggr] \ind{C_I}(\xi), \JG(\xi) = \sum_{I\in\mathcal{I}} \Biggl[
% \sum_{j=1}^{k} x_j \cdot \ind{]\sum_{l=1}^{j-1} \lambda^I_l(\xi),
% \sum_{l=1}^{j} \lambda^I_l(\xi) ]}(U)\Biggr] \ind{C_I}(\xi), \] so that due to
% the identity \[ \Prob(\JG(\xi) = x_j) = \lambda_j(\xi) \]for $\lambda_j(\xi) =
% \sum_{I\in\mathcal{I}} \lambda^I_j(\xi) \ind{C_I}(\xi),\, {1\leq j \leq k}$ ,
% we finally arrive at \[ \E\norm{\xi - \JG(\xi)}^p = \sum_{j=1}^k
% \lambda_j(\xi)\, \norm{\xi - x_j}^p = \Fp(\xi; \Gamma). \]  Thus, \[ \E
% \norm{X - \JG(X)}^p = \E \Fp(X; \Gamma), \] $\JG$ is  intrinsic stationary by
% construction and taking again the infimum over all grids $\Gamma$ with 
% $\supp(\ProbX) \subset\conv\{\Gamma\}$ and $ \abs{\Gamma} \leq n$ yields the
% assertion.  \end{proof}   \subsection{Extension to distributions with
% unbounded support}   Since we have seen in the proof of Proposition
% \ref{prop:DQLinkStat} that intrinsic stationarity for a grid $\Gamma$ is
% equivalent to the existence %, for any $\xi \in \supp(\ProbX)$, of \[
% \lambda_x \in [0,1], \qquad \sum_{x\in\Gamma} \lambda_x = 1 \quad \text{ and }
% \quad \sum_{x\in\Gamma} \lambda_x \cdot x = \xi, \] which can be fulfilled by
% the very definition of convex combinations only for $\xi \in \conv\{\Gamma\}$,
% intrinsic stationarity cannot hold for a r.v. $X$ with unbounded support (and
% therefore we have $\dqpn(X) = +\infty, n \geq d+1$, whenever $\supp(\ProbX)$
% is unbounded).  Nevertheless, we may restrict the stationarity requirement in
% the definition of the dual quantization error for unbounded $X$ to its
% ``natural domain'' $\conv\{\Gamma\}$, which means that we drop the constraint
% $\supp(\ProbX) \subset \conv\{\Gamma\}$ from Propostion \ref{prop:DQLinkStat}.
%  \begin{definition} We define the extended $L^p$-dual quantization error as \[
% \Dqbpn(X) = \inf\bigl\{ \norm{X -  \JG(X)}_p: \, \JG:\R^d\to\Gamma \text{ is
% intrinsic stationary}, \Gamma \subset \R^d, \abs{\Gamma} \leq n \bigr\}. \]
% \end{definition}  Combining Propositions \ref{prop:regularQNN} and Proposition
% \ref{prop:DQLinkStat} we get \begin{prop} Let $X\in L^p(\Prob)$. Then \[
% \Dqbpn(X) =\inf\bigl\{  \|\bar F_p(X;\Gamma)\|_p \,: \Gamma \subset \R^d,
% \abs{\Gamma} \leq n \bigr\} \] for \[ \bar F_p^p(\xi;\Gamma) = \bar F_p^p(\xi;
% \Gamma) \ind{\conv\{\Gamma\}}(\xi) + \norm{\xi - \pi_\Gamma(\xi)}
% \ind{\conv\{\Gamma\}^c}(\xi). \] \end{prop}  Note, that we have for any $X\in
% L^p(\Prob)$ \begin{equation}\label{Dineq} \Dqbpn(X) \leq \Dqpn(X),
% \end{equation} where equality in general even does not hold anymore for $X$
% with bounded support. However, we will see later in Section~\ref{sec:rate},
% that both quantities coincide asymptotically in the bounded case.  % In case
% the r.v. $X$ has unbounded support, we define extensions of the dual %
% quantization error as follows. % % \begin{definition} % \begin{enumerate} %
% Let $X\in L^p(\Prob)$, $\Probb = \ProbX$ and $\Gamma = \{x_1, \ldots, x_n\}$ %
%   \item nearest neighbor extension with $\Probb_{|\Gamma} = \Probb(\cdot \cap
% %   \conv\{\Gamma\})$, $ \Probb_{|\Gamma^c} = \Probb(\cdot \cap %  
% \conv\{\Gamma\}^c) $ % \[ %  \dqbp(X; \Gamma) = \dqp(\Probb_{|\Gamma}; \Gamma)
% + % e^p(\Probb_{|\Gamma^c}; %  \Gamma) % \] % and % \[ %  \dqpn(X) = \inf
% \bigl\{ \dqp(X; \Gamma): \Gamma \subset \R^d, %  \abs{\Gamma}\leq n \bigr\} %
% \] %   \item smooth extension (Hilbert space case) with $\Pr_\Gamma$ the
% Hilbert %   space projection onto $\conv\{\Gamma\}$ % \[ %  \tilde \dqp(X;
% \Gamma) = \E % 
% \underset{\consLP{\Pr_\Gamma(X)}}{\min_{\lambda\in\R^n}\sum_{i=1}^n % 
% \lambda_i %  \, %  \norm{X-x_i}^p} % \] % and % \[ %  \tilde \dqpn(X) = \inf
% \bigl\{ \tilde \dqp(X; \Gamma): \Gamma \subset \R^d, %  \abs{\Gamma}\leq n
% \bigr\} % \] % \end{enumerate} % \end{definition} % % %
% \begin{prop}\label{prop:lowIneqNNDQ} % For all $\Gamma \subset \R^d$: % \[ % 
% \dqbp(X; \Gamma) \leq \tilde \dqp(X; \Gamma) \leq \dqp(X; \Gamma) % \] %
% \end{prop} % \begin{proof} % Follows directly from the definitions. %
% \end{proof}

\smallskip Let us come back to the terminology {\em dual quantization}: it refers to   a canonical example
of the intrinsic stationary splitting operator: the dual quantization operator.

\smallskip To be more precise,  
%assume $\R^d$ is equipped with a norm $\|\,.\,\|$ and 
let $p\!\in[1,+\infty)$ and let $\Gamma=\{x_1,\ldots,x_n\}\subset \R^d$ be a
grid of size $n\ge d+1$ such that  ${\rm aff.dim}(\Gamma) =  d$ $i.e.$ $\Gamma$ contains at least one $d+1$-tuple of  affinely independent points.


\smallskip The underlying idea is to ``split" $\xi\!\in \conv(\Gamma)$ across at most $d+1$ affinely
independent points in $\Gamma$
% (which convex hull contains $\xi$) 
proportionally to its barycentric coordinates of $\xi$. There are usually many possible choices of such a $\Gamma$-valued $(d+1)$-tuple of affinely independent points, so  we
introduced a minimal inertia based criterion to select the most appropriate
one  $\xi$, namely the function $F_p(\xi;\Gamma)$ defined for every
$\xi$ as the value of the minimization problem 
\begin{equation}
F_p(\xi;\Gamma)=\inf_{(\lambda_1,\ldots,\lambda_n)}\Big\{\Big(\sum_{i=1}^n
\lambda_i\|\xi-x_i\|^p\Big)^{\frac 1p}, \lambda_i\!\in[0,1] ,
\sum_i\lambda_i\Big[\!\begin{array}{c}x_i\\1\!\end{array}\Big]=\Big[\!\begin{array}{c}\xi\\1\end{array}\!\Big]\Big\}.
\end{equation}
Owing to the compactness of the constraint set ($\lambda_i\ge 0$,
$\sum_i\lambda_i =1$, $\sum_i \lambda_i x_i = \xi$), there exists at least one
solution $\lambda^*(\xi)$  to the above minimization problem. Moreover,  for any such solution,  one shows using convex
extremality arguments, that the set $I^*(\xi):=\big\{i\!\in\{1,\ldots,n\}\mbox{ s.t. } \lambda_i^*(\xi)>0\big\}$ defines an affinely independent subset $\{x_i,\; i\!\in I^*(\xi)\}$.

\smallskip
If, for every $\xi\!\in  conv(\Gamma)$, this solution is  unique, the {\em  dual quantization operator} is simply
defined on $\conv(\Gamma)$ by 
\begin{equation}\label{eq:Jstar}
\forall\, \xi\!\in \conv(\Gamma),\; \forall\,
\omega_0\!\in \Omega_0,\quad \mathcal{J}^*_{\Gamma}(\omega_0, \xi) = \sum_{i\in
I(\xi)^*}x_i \mbox{\bf 1}_{\{\sum_{j=1}^{i-1} \lambda^*_j(\xi)\le U(\omega_0)< 
\sum_{j=1}^{i} \lambda^*_j(\xi)\}},
\end{equation}
where $U$ denotes a  random variable  uniformly distributed over $[0,1]$ on an exogenous probability space  $(\Omega_0,\mathcal{S}_0,
\Prob_0)$. 
This operator ${\cal J}^*_{\Gamma}$ is then measurable
(see~\cite{dualStat}).

The above uniqueness assumption  is not so stringent, especially for applications.
Thus, in a purely Euclidean  quadratic framework:  $\|\,.\,\|= |\,.\,|_{\ell^2}$ (canonical Euclidean norm) and $p=2$  and if $\Gamma$ is said in  ``general position"~(\footnote{no $d+2$ points of $\Gamma$ lie on a sphere in $\R^d$.}), then $\displaystyle\Big\{\{\xi\, \mbox{ s.t. }\, I^*(\xi) =I\}, \, |I|\le d+1\Big\}$ makes up a Borel partition of
$\conv(\Gamma)$ (with possibly empty elements), known in $2$-dimension as the
{\em Delaunay triangulation} of $\Gamma$ (see \cite{rajan} for the connection
with  Delaunay triangulations). 

%\smallskip
In a more  general framework, we refer to~\cite{dualStat}
for a construction of dual quantization operators. Such operators are splitting operators since,  by construction, they satisfy the stationarity  property~(\ref{1a}).

\smallskip One must have in mind that the dual quantization operators $\mathcal{J}^*_{\Gamma}(\omega_0, \xi) $   play the  role of the nearest neighbour projections for regular Voronoi quantization. One checks that, by construction, 
\[
\forall\, \xi\!\in \conv(\Gamma),\quad \|\mathcal{J}^*_{\Gamma}(\xi)-\xi\|_{L^p(\Prob_0)}= \|F_p(\xi;\Gamma)\|_{L^p(\Prob_0)}
%\\
% &=& \E \inf \left\{\left(\sum_{i=1}^n \lambda_i\|X-x_i\|^p\right)^{\frac 1p}, \lambda_i\ge 0, \sum_i\lambda_i\left[\!\begin{array}{c}x_i\\1\end{array}\!\right]=\left[\!\begin{array}{c}X\\1\end{array}\!\right]\right\}.
\]
so that, as soon as  ${\rm supp}(\Prob_{_X})\subset \Gamma$ (or equivalently $\Prob(X\!\in \conv(\Gamma))=1$),
\[
d_{p}(X;\Gamma) = \| \mathcal{J}^*_{\Gamma}(X)-X\|_{L^p(\Prob_0\otimes \Prob)}= \|F_p(X;\Gamma)\|_{L^p(\Prob_0\otimes \Prob)}.
\] 
At this stage, it appears naturally that the the second step of the optimization process is to find (at least) one grid  which optimally
``fits" (the distribution of) $X$ for this criterion $i.e.$ which is the solution to the second
level  optimization problem 
\[ 
d_{n,p}(X)
=\inf\left\{\|\mathcal{J}^*_{\Gamma}(X)-X\|_{L^p(\Prob_0\otimes \Prob)}, \;
\mathcal{J}^*_{\Gamma}: \Omega_0\times \conv(\Gamma)\to \Gamma,
\conv(\Gamma)\supset {\rm supp}(\ProbX),\, |\Gamma|\le n\right\}. 
\] 
Note that
if $X\!\in L_{\R^d}^{\infty}(\Prob)$, $d_{n,p}(X)<+\infty$ if and only if $n\ge d+1$ (whereas it is
identically infinite if $X$ is not essentially bounded). The existence of an
optimal grid (or dual quantizer) has been established in~\cite{dualStat} (see below). 

The error modulus  $d_{n,p}(X)$ can also be characterized as  the {\em lowest $L^p$-mean
approximation error by a r.v.  having at most $n$ values and satisfying the
intrinsic stationarity property} as established in~\cite{dualStat} (Theorem~2, precisely recalled in Theorem~\ref{thm:DQLinkStat} below). 
%\[ 
%d_{n,p}(X) = \inf\left\{\|X-\widehat
%X\|_{L^p(\Prob_0\otimes \Prob)},\, |\widehat X(\Omega_0\times \Omega)|\le n,\,
%\E_{\Prob_0\otimes \Prob}(\widehat X\,|\, X)=X\right\}. 
%\]
It should be compared to the well-known property satisfied by the mean (regular) quantization error modulus $e_{n,p}(X)$, namely
\[ 
e_{n,p}(X) = \inf\Big\{\|X-\widehat
X\|_{L^p(\Prob)},\, |\widehat X( \Omega)|\le n\Big\}. 
\]
% \medskip {\sc Phase~I}: Let $\xi\!\in \conv(\Gamma)$. Find a $\Gamma$-valued
% $d+1$-simplex which convex hull contains $\xi$ with a minimal $p$-inertia. in
% mathematical terms this means finding a solution $\lambda^*(\xi)\!\in \R_+^n$ 
%  solution to the minimization problem \[
% F_p(\xi)=\inf\left\{\left(\sum_{i=1}^n \lambda_i\|\xi-x_i\|^p\right)^{\frac
% 1p}, \lambda_i\ge 0,
% \sum_i\lambda_i\left[\begin{array}{c}x_i\\1\end{array}\right]=\left[\begin{array}{c}\xi\\1\end{array}\right]\right\}
% \] One shows using convex extremality arguments that $\lambda^*(\xi)$ does
% exist and that the set $I^*(\xi):=\{i\!\in I \mbox{ s.t. }
% \lambda_i^*(\xi)>0\}$ defines an affinely independent subset $\{x_i,\; i\!\in
% I^*(\xi)\}$.

% In an Euclidean quadratic framework ($p=2$ and $\|\,.\,\|$ is an Euclidean
% norm), this solution can be shown to be unique.

%  \medskip \noindent {\sc Phase~II}: Let $D_I=\{\xi \,:\, I^*(\xi)\subset I\} $
% when $I$ runs over $\mathcal{I}_{\Gamma}$ of all subset of $\{1,\ldots,n\}$ 
% defining an affine basis $(x_i)_{i\in I}$. The $D_I$, $i\!\in \mathcal{I}$,
% are Borel sets (see~\cite{PW1}) which clearly make a covering of $
% \conv(\Gamma)$. This covering plays the role of the Voronoi diagram for the
% nearest neighbour projection.  In turn one may define (in a non-unique way) a
% partition $(C_I)_{I\in \mathcal{I}}$ (with possibly empty $C_I$) such that \[
% C_I\subset D_I, \; I\!\in \mathcal{I}. \] Furthermore, still in an Euclidean
% qudartic framework, if the grid $\Gamma$ is in general position, one shows
% (see~\cite{RAJ, PW1}) that  the family $\{\xi,\; I^*(\xi)=I\}$, $|I|\le d+1$
% makes up directly a partition of $\conv(\Gamma)$ known as the {\em Delaunay
% triangulation} of $\Gamma$.  \smallskip. The dual quantizar operator is
% defined as follows: set $(\Omega_0,\mathcal{S}_0, \Prob_0)=([0,1],
% \mathcal{B}([0,1]), \lambda_{[0,1]})$ and $U=Id_{[0,1]}$ the canonical random
% variable with uniform distribution over the unit interval and \[ \forall\,
% \xi\!\in \conv(\Gamma),\; \forall\, \omega_0\!\in \Omega_0,\quad
% \mathcal{J}^*_{\Gamma}(\omega_0, \xi) = \sum_{i\in
% \mathcal{I}}\left(\sum_{i\in I(\xi)^*}x_i \mbox{\bf 1}_{\{\sum_{j=1}^{i-1}
% \lambda^*_j(\xi)\le U(\omega_0)<  \sum_{j=1}^{i}
% \lambda^*_j(\xi)\}}\right)\mbox{\bf 1}_{\{\xi\in C_I\}}. \]

% On easily checks that this random operator is intrinsic stationary and that \[
% \|\mathcal{J}^*_{\Gamma}(X)-X\|_{L^p(\Prob_0\otimes \Prob)}=
% \|F_p(X)\|_{L^p(\Prob)} \] so that \[
% \|\mathcal{J}^*_{\Gamma}(X)-X\|_{L^p(\Prob_0\otimes \Prob)}^p= \E \inf
% \left\{\left(\sum_{i=1}^n \lambda_i\|X-x_i\|^p\right)^{\frac 1p}, \lambda_i\ge
% 0,
% \sum_i\lambda_i\left[\begin{array}{c}x_i\\1\end{array}\right]=\left[\begin{array}{c}X\\1\end{array}\right]\right\}.
% \] The quantity $\|\mathcal{J}^*_{\Gamma}(X)-X\|_{L^p(\Prob_0\otimes \Prob)}$
% is defined as the $L^p$-mean dual quantization error of $\Gamma$ induced by
% (the distribution of)$X$ (note that this quantity only depends on the
% distribution of $X$).




%  However, at this stage, it is natural to introduce a second optimization
% problem, namely finding the grid, if any, having the lowest possible
% $L^p$-mean dual quantization error $i.e.$ a grid solution to the minimization
% problem \[ d_{n,p}(X)
% =\inf\left\{\|\mathcal{J}^*_{\Gamma}(X)-X\|_{L^p(\Prob_0\otimes \Prob)}, \;
% \mathcal{J}^*_{\Gamma}: \Omega_0\times \conv(\Gamma)\to \Gamma,
% \conv(\Gamma)\supset {\rm supp}\Prob_{_Y},\, |\Gamma|\le n\right\}. \] We
% showed in~\cite{PW1} that, for every $n\ge d+1$, $d_{n,p}(X)<+\infty $ iff
% $X\!\in L^{\infty}(\Prob)$ and that there exists an optimal grid
% $\Gamma^{*,n}$ such that \[ d_{n,p}(X)
% =\|\mathcal{J}^*_{\Gamma^{*,n}}(X)-X\|_{L^p(\Prob_0\otimes \Prob)}. \]

A stochastic optimization procedure based on a
stochastic gradient approach has been devised in~\cite{dualStat} to  compute optimal dual quantization grids w.r.t. various
distributions  (so far, uniform over $[0,1]^2$, normal, $(W_1,\sup_{t\in[0,1]}  W_t)$, $W$ standard Brownian motion in a purely Euclidean framework). 

Let us conclude by two results established in~\cite{dualStat}. The first one is the   characterization of dual quantization operator in terms  in terms of best
$L^p$-approximation (see~\cite{dualStat}, Theorem~2).

\begin{thm}\label{thm:DQLinkStat}
Let 
%$X\!\in L^0(\Omega, \mathcal{S},\Prob)$ 
$X: \Omega, \mathcal{S},\Prob)\to \R^d$ be a r.v. such that ${\rm aff.dim}({\rm supp}(\Prob_{_X}))= d$ and let  $n\!\in\N$, $n\ge d+1$. Then
\begin{equation*}
\begin{split}
d_{n,p}(X) & = \inf\bigl\{ \E \norm{X -  \JG(X)}_{L^p}: \,
\JG: \Omega_0\times \R^d\to\Gamma, \text{ intrinsic stationary},\\
& \qquad\qquad\qquad\qquad\qquad\qquad\supp(\ProbX) \subset\conv(\Gamma),\,
\abs{\Gamma} \leq n \bigr\}\\
& = \inf\bigl\{ \E \norm{X -  \widehat X}_{L^p}: \widehat X :\PSpace\to \R^d, \\
& \qquad\qquad\qquad\qquad\qquad\qquad \abs{\widehat
X(\Omega_0\times\Omega)} \leq n,\, \E(\widehat X|X) = X \bigr\}\le +\infty.
\end{split}
\end{equation*}
This quantity is finite if and only if $X\in L^\infty(\Omega, \mathcal{S},\Prob)$.
\end{thm}

Finally, the following existence result 
for optimal dual quantizers  {\em at level $n\in\N$}  and the $L^p$-norm
with $p\in(1,\infty)$  is  established in \cite{dualStat}.
 Although we will not
use it in our proofs, this result is recalled for the reader's convenience.
%\pagebreak

\begin{thm}[Existence of optimal quantizers]\label{thm:existence}
Let $X \in L^p(\Prob)$ for some $p \in (1,\infty)$. %then,\\
\begin{enumerate}
  \item[(a)] If $\supp(\ProbX)$ is compact, then  there
  exists for every $n\in \N$ a grid $\Gamman^{\ast} \subset \R^d,\,
  \abs{\Gamman^{\ast}}\leq n$ such that $d_p(X;\Gamman^{\ast}) = d_{n,p}(X)$.
  \item[(b)] If $\ProbX$ is strongly continuous in the sense that it assigns 
  no mass  to  hyperplanes of $\R^d$, then  there exists for every $n\in \N$
  a grid $\Gamman^{\ast} \subset \R^d,\, \abs{\Gamman^{\ast}}\leq n$ such that
$\bar d_p(X;\Gamman^{\ast}) = \bar d_{n,p}(X)$.
\end{enumerate}
If furthermore $\abs{\supp(\ProbX)}\geq n$, then the above statements hold with
$\abs{\Gamman^{\ast}}= n$.
% (a) if $\supp(\ProbX)$ is compact, for every $n\in \N$ there exists a grid
% $\Gamma^{n,\ast} \subset \R^d,\, \abs{\Gamma^{n,\ast}}\leq n$ such that
% $d_p(X;\Gamma^{n,\ast}) = d_{n,p}(X)$.\\ 
% (b) if $\ProbX$ is strongly continuous in the sense that it assigns  mass zero
% to all Hyperplanes in $\R^d$, for every $n\in \N$ there exists a grid
% $\Gamma^{n,\ast} \subset \R^d,\, \abs{\Gamma^{n,\ast}}\leq n$ such that
% $\bar d_p(X;\Gamma^{n,\ast}) = \bar d_{n,p}(X)$.
\end{thm}



\subsection{Local properties of the dual quantization functional}\label{sec:localProperties}

We establish or recall  in this paragraph some first general properties of the local $L^p$-dual
quantization functional $F^p$, which will be needed for the final proof of
Theorem \ref{thm:DQRate}.

\begin{prop}\label{prop:PtsInsertion} Let $\Gamma_1$, $\Gamma_2 \subset \R^d$
be finite grids and let $\xi\!\in \R^d$. Then
\[
\Gamma_1 \subset \Gamma_2 \Longrightarrow F_p(\xi; \Gamma_2) \leq F_p(\xi;
\Gamma_1).
\]
\end{prop}
%\begin{proof} 
\noindent {\it Proof.} First note that the set $\{\lambda\!\in \R^n\,|\,  \left[  \begin{smallmatrix}
          x_1 & \ldots & x_m\\
          1 & \ldots & 1\\
        \end{smallmatrix}  \right] \lambda =
      \left[\begin{smallmatrix}
         \xi\\ 1\\
        \end{smallmatrix} \right]\}$ 
        is clearly a compact set on which the continuous function 
        $\lambda
        \mapsto \sum_{i=1}^n \lambda_i \|\xi-x_i\|^p$ attains a minimum.
Assume $\Gammaone = \{x_1, \ldots, x_m\}$ and $\Gammatwo = \{x_1, \ldots, x_m,
x_{m+1}, \ldots, x_n\}$. Then
\begin{eqnarray*}
\begin{split}
  F^p(\xi; \Gammatwo)  = \LP{\xi} 
  & \leq
  \underset{ \text{s.t. } \left[  \begin{smallmatrix}
          x_1 & \ldots & x_m\\
          1 & \ldots & 1\\
        \end{smallmatrix}  \right] \lambda =
      \left[\begin{smallmatrix}
         \xi\\ 1\\
        \end{smallmatrix} \right],\,
      \lambda \geq 0 }{\min_{\lambda \in \R^n, \lambda_{m+1}=\cdots= \lambda_n = 0}\sum_{i=1}^n \lambda_i
      \, \norm{\xi-x_i}^p}\\
  & = \underset{ \text{s.t. } \left[  \begin{smallmatrix}
          x_1 & \ldots & x_m\\
          1 & \ldots & 1\\
        \end{smallmatrix}  \right] \lambda =
      \left[\begin{smallmatrix}
         \xi\\ 1\\
        \end{smallmatrix} \right],\,
      \lambda \geq 0 }{\min_{\lambda\in \R^m }\sum_{i=1}^m \lambda_i
      \, \norm{\xi-x_i}^p}
      %\\&
   = F^p(\xi; \Gammaone).\qquad \Box
\end{split}
\end{eqnarray*}
%\end{proof}


We will also make use of the following three properties established
in~\cite{dualStat} (Propositions~11, 12, 13 respectively). In particular, the third claim yields  a first upper
bound for the asymptotics of the local $L^p$-dual quantization error 
%of a distribution with bounded support 
when the size of
the grid goes to infinity.

\begin{prop}\label{prop:rappels}
%label{prop:scalarBoundF}
%
$(a)$ {\em Scalar bound:} Let $\Gamma = \{x_1, \ldots, x_n\}\subset  \R$ with $x_1\leq \ldots\leq x_n$.
Then
\[
 \forall \xi \in [x_1, x_n],\quad \Fp(\xi; \Gamma) \leq \max_{1\leq i \leq n-1}
\Bigl(\frac{x_{i+1}-x_i}{2}\Bigr)^p.
\]
%\end{prop}
%\begin{proof}
%If $\xi\in \Gamma$, then $\Fp(\xi; \gamma) = 0$ and the assertion holds. 
%Suppose now $\xi \in (x_i, x_{i+1})$. 
%Then $\xi = \lambda
%x_i + (1- \lambda) x_{i+1}$ with  $\lambda = \frac{x_{i+1} - \xi}{x_{i+1} - x_i}$,
%so that
%\[
%	\Fp(\xi; \gamma) \leq \Bigl(\frac{x_{i+1} - \xi}{x_{i+1} -
%	x_i}\Bigr)\abs{\xi - x_i}^p + \Bigl(\frac{\xi - x_i}{x_{i+1} -
%	x_i}\Bigr)\abs{\xi - x_{i+1}}^p.
%\]
%The right hand side attains its maximum at $\xi = \frac{x_i + x_{i+1}}{2}$.
%This implies
%\[
%	\Fp(\xi; \gamma) \leq \Bigl(  \frac{1}{2} + \frac{1}{2} \Bigr) 
%\Bigabs{\frac{x_{i+1} - x_i}{2}}^p,
%\]
%which yields the assertion.
%\end{proof}
%
%\begin{prop}
$(b)$ {\em Local product Quantization:}
%\label{prop:product}
Let $\norm{\cdot} = |\cdot|_{\ell^p}$ 
%defined for every  $\xi = (\xi^1, \ldots,
%\xi^d)\!\in \R^d$ by  $|\xi|_{\ell^p}=\Big(\sum_{1\le i \le d}|\xi^i|^p\Big)^{1/p}$ 
and let $\Gamma = \Prod_{1\le j\le d} \Gamma_j$ for some $\Gamma_j
\subset \R$. Then 
\[ 
 \forall\, \xi\!\in \R^d, \quad F_{p, |.|_{\ell^p}}(\xi; \Gamma) = \Big(\sum_{j=1}^d F^p(\xi^j;
\Gamma_j)\Big)^{\frac 1p} 
\]
and the same holds true with $\bar F_{p,\ell^p}$ on $\R^d$.
%\end{prop}
%\begin{proof}
%Denoting $\alpha_j = \{a_1^j, \ldots, a_{n_j}^j\},\,
%\Gamma = \{x_1, \ldots, x_n\}$ % = \times_{j=1}^d \alpha_j$.
%and due to the fact that $\{x_1, \ldots, x_n\}$ is made up by the cartesian
%product of $\{a_1^j, \ldots, a_{n_j}^j\},\, j = 1, \ldots, d$ we have for any $u=(u^1,\ldots,u^d),\, \xi
%= (\xi^1, \ldots, \xi^d)\!\in \R^d$:
%\[
%\min_{1\leq i \leq n} \biggl\{ \sum_{j=1}^d \abs{\xi^j - x_i^j}^p + u^j(\xi_j -
%x_i^j) \biggr\} = \sum_{j=1}^d \min_{1\leq i \leq n_j}  \bigl\{ \abs{\xi^j -
%a_i^j}^p + u^j(\xi^j - a_i^j) \bigr\}.
%\]
%
%We then get from Proposition \ref{prop:dualF} \textcolor{red}{[[Not defined  in this paper, Iguess...]]}
%\begin{eqnarray*}
%\begin{split}
%\Fp(\xi; \Gamma) & = \max_{u\in\R^d} \min_{1\leq i \leq n} \biggl\{ \sum_{j=1}^d
%\abs{\xi^j - x_i^j}^p + u_j(\xi^j - x_i^j) \biggr\} \\
%& = \max_{u\in\R^d} \sum_{j=1}^d \min_{1\leq i \leq n_j}  \bigl\{ \abs{\xi^j -
%a_i^j}^p + u^j(\xi_j - a_i^j) \bigr\}\\
%& = \sum_{j=1}^d \max_{u_j\in\R} \min_{1\leq i \leq n_j}  \bigl\{ \abs{\xi^j -
%a_i^j}^p + u^j(\xi_j - a_i^j) \bigr\}\\
%& = \sum_{j=1}^d \Fp(\xi^j; \alpha_j).
%\end{split}
%\end{eqnarray*}
%\end{proof}
%\begin{prop}

$(c)$ {\em Product Quantization:}
%\label{prop:productConstruction}
Let $C=a+L\,[0,1]^d$, $a=(a_1,\ldots,a_d)\!\in \R^d$, $L>0$, be a hypercube,
with edges parallel to the coordinate axis with common edge-length $L$. Let $\Gamma$ be the product quantizer  of size $(m+1)^d$ defined by
$$
\Gamma=\Prod_{k=1}^d\Big\{a_j+\frac{i L}{m},\, i=0,\ldots,m\Big \}.
$$
%If we divide each edge of $C$ into $m$
%intervals of equal length $l/m$, the interval endpoints define $m+1$ grid
%points on each edge.
%
%For the p which is made up by this procedure,  
There exists a positive real  constant  $C_{\norm{.},p} = \sup_{|x|_{\ell^p}=1}\norm{x}^p>0$ such that 
\begin{equation}\label{eq:prodFp}
  \forall \,\xi\!\in C,\quad F^p(\xi; \Gamma)  \leq  d\, C_{\norm{\cdot},p} \cdot
  \Bigl(\frac{L}{2} \Bigr)^p \cdot m^{-p}.
\end{equation}
%where $C_{\norm{.},p} = \sup_{|x|_{\ell^p}=1}\norm{x}$.
%with some constant. 
%Moreover, for any compactly supported r.v. $X$
%\[
%%\dqp(X;\Gamma) \leq C_{\norm{\cdot}, p, d, X} \cdot m^{-p}
%\Dqpn(X) = \mathcal{O}(n^{-1/d}).
%\]
\end{prop}
%\begin{proof}
%The first claim follows directly from Propositions~\ref{prop:product} and
%\ref{prop:scalarBoundF}.
%For the second assertion let $n\geq 2^d$ and set $m = \lfloor n^{1/d} \rfloor -
%1$. If we choose the hypercube $C$ such $\supp(\ProbX) \subset C$ we arrive at
%\[
%%\dqp(X;\Gamma) \leq C_{\norm{\cdot}, p, d, X} \cdot m^{-p}
%\Dqpn(X) \leq C_1  \frac{1}{\lfloor n^{1/d} \rfloor-1}  \leq
%C_2 \Bigl( \frac{1}{n} \Bigr)^{1/d}
%\] 
%for some constants $C_1, C_2 > 0$, which yields the desired upper bound.
%\end{proof}

% As a consequence of the above Propositions we also can derive an upper bound
% for the asymptotics of the dual quantization constant $\Qpn$ from Theorem
% \ref{thm:DQRate} as $d\to\infty$ and when $\norm{\cdot} = \abs{\cdot}_{\ell^p}$.
% \begin{prop}\label{prop:asymptQ}
% Let $p\in[1,\infty)$ and $\norm{\cdot} = \abs{\cdot}_{\ell^p}$. Then it holds
% \[
% 	Q^{\text{dq}}_{\abs{\cdot}_{\ell^p}, p, d} = \mathcal{O}(d^\frac{1}{p})
% 	\,\text{ as }\, d \to\infty.
% \]
% \end{prop}
% \begin{proof}
% For $m\in\N$ set $n = m^d$ and let $\Gamma'$ be an optimal quantizer for
% $d_{m,p}(\Unif)$ (or at least $(1+\varepsilon)$-optimal for $\varepsilon > 0$).
% Denoting $\Gamma = \prod_{i=1}^d \Gamma'$, it then follows from Proposition
% \ref{prop:product} that
% \[
% 	n^\frac{p}{d}\, d^p_n(\Unifd) \leq n^\frac{p}{d}\, d^p(\Unifd;\Gamma) = m^p
% 	\sum_{i=1}^d d^p(\Unif; \Gamma')  = d\, m^p \,d^p_m(\Unif),
% \]
% which yields the upper bound.
% \end{proof}
