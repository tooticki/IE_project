\begin{abstract}
In this paper we establish the sharp rate of the optimal dual quantization problem.
The notion of dual quantization was recently introduced in %the paper
\cite{dualStat}, where it has been shown that, at least in a Euclidean setting, dual
quantizers are based on a Delaunay triangulation, the dual counterpart of the
Voronoi tessellation on which ``regular'' quantization relies.
Moreover, this new approach shares an intrinsic stationarity property, which
makes it very valuable for numerical applications.

We establish in this paper the counterpart for dual quantization of the
celebrated Zador theorem, which describes the sharp asymptotics for the
quantization error when the quantizer size tends to infinity.  On the way we establish an extension of the so-called Pierce Lemma by a random quantization argument. Numerical results confirm our choices. 

%  
% This new approach to quantization which shares an intrinsic stationarity
% property has been introduced in a companion paper~\cite{dualStat}. This
% sharpness requires  a random quantization argument which appears as an extension of the
% so-called Pierce Lemma established for the same purpose in the framework of
% ``regular" Voronoi quantization.
\end{abstract}


\bigskip
\noindent {\em Keywords: quantization, quantization rate, Zador's Theorem, Pierce's Lemma,
dual quantization, Delaunay triangulation, random quantization.}

