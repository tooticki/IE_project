% latex
\documentclass[reqno]{amsart}

\usepackage{amsmath,amssymb,amscd,amsthm,accents} \usepackage{graphicx}
\DeclareGraphicsExtensions{.eps} \usepackage{mathrsfs}
\usepackage[mathcal]{eucal}
\usepackage{enumitem}

% \usepackage{amsrefs}
%

%Side Remarks
\def\sideremark#1{\ifvmode\leavevmode\fi\vadjust{\vbox to0pt{\vss 
			\hbox to 0pt{\hskip\hsize\hskip1em          
				\vbox{\hsize3cm\tiny\raggedright\pretolerance10000%             
				%   
					\noindent #1\hfill}\hss}\vbox to8pt{\vfil}\vss}}}%
\newcommand{\edz}[1]{\sideremark{#1}}
%End Side Remark



\newtheorem{Thm}{Theorem}{\bfseries}{\itshape}
\newtheorem*{Thm*}{Theorem}{\bfseries}{\itshape}
\newtheorem{Cor}{Corollary}{\bfseries}{\itshape}
\newtheorem{Prop}[Cor]{Proposition}{\bfseries}{\itshape}
\newtheorem{Lem}[Cor]{Lemma}{\bfseries}{\itshape}
\newtheorem*{Lem*}{Lemma}{\bfseries}{\itshape}
\newtheorem{Fact}[Cor]{Fact}{\bfseries}{\itshape}
\newtheorem{Conj}[Cor]{Conjecture}{\bfseries}{\itshape}
% \theoremstyle{definition}
\newtheorem{Def}[Cor]{Definition}{\bfseries}{\rmfamily}
\newtheorem{Ex}[Cor]{Example}{\scshape}{\rmfamily}
% \theoremstyle{remark}
\newtheorem{Rem}[Cor]{Remark}{\scshape}{\rmfamily}
\newtheorem*{Claim}{Claim}{\bfseries}{\itshape}

\renewcommand\ge{\geqslant} \renewcommand\le{\leqslant}
\let\tildeaccent=\~ \let\hataccent=\^
\renewcommand\~[1]{\widetilde{#1}}
%\renewcommand\^[1]{\widehat{#1}}
\def\o{\accentset{\circ}}

\def\<{\left<} \def\>{\right>} \def\({\left(} \def\){\right)}


\def\abs#1{\left\vert #1 \right\vert} \def\norm#1{\left\Vert #1
  \right\Vert} \def\size#1{\mathbf S\left(#1 \right)}

\let\parasymbol=\S \def\secref#1{\parasymbol\ref{#1}}
\def\ssm{{\,\text{--}\,}}
\def\pd#1#2{\frac{\partial#1}{\partial#2}}

\let\genuinesimeq=\simeq \def\arg{\operatorname{Arg}}
\let\subsetneq=\subsetneqq \let\supsetneq=\supsetneqq

\let\polishL=l \def\Zoladek.{\.Zol\c adek}

\def\Var{\operatorname{Var}} \def\const{\operatorname{const}}
\def\codim{\operatorname{codim}} \def\Mat{\operatorname{Mat}}
\def\crit{\operatorname{crit}} \def\dom{\operatorname{dom}}
\def\Re{\operatorname{Re}} \def\Im{\operatorname{Im}}
\def\Arg{\operatorname{Arg}} \def\dist{\operatorname{dist}}
\def\length{\operatorname{length}}
\def\diam{\operatorname{diam}} \def\ord{\operatorname{ord}}
\def\GL{\operatorname{GL}} \def\SL{\operatorname{SL}}
\def\etc.{\emph{etc}.}
\def\ind{\operatorname{ind}} \def\Sing{\operatorname{Sing}}
\def\sign{\operatorname{sign}}

\def\iclo#1{\overline{#1}}

\def\:{\colon} \def\R{{\mathbb R}} \def\C{{\mathbb C}} \def\Z{{\mathbb
    Z}} \def\N{{\mathbb N}} \def\Q{{\mathbb Q}} \def\P{{\mathbb P}}
\def\H{{\mathbb H}}
\def\D{{\mathbb D}}

\def\A{{\mathbb A}}
\def\K{{\mathbb K}}

\let\PolishL=\L % remember polish L
\def\L{{\mathbb L}}

% \def\l{\lambda}

\def\I{{\mathbb I}} \def\e{\varepsilon} \def\S{\varSigma}
\let\ol=\overline \def\f{\varphi} \def\F{{\varPhi}}
\def\iu{\mathrm{i}} \def\diag{\operatorname{diag}}
\def\conf{\asymp}%{\buildrel\text{cnf}\over\genuinesimeq}
\def\poly{\operatorname{poly}} \def\J{\operatorname{J}}


\def\res{\operatornamewithlimits{Res}} \def\d{\,\mathrm d}
% \def\i{\mathrm i\typeout{^^J ___________________}}
\long\def\NB#1{%
  % \typeout{^^JNB: occurs on input line \the\inputlineno \space page
  % \arabic{page}^^J}
  \marginpar{$\bigstar$\raggedright #1}}

\def\G{\varGamma} \def\Lojas.{\PolishL ojasiewicz}
\def\cN{{\mathcal N}} \def\cH{{\mathcal H}}
\def\cP{{\mathcal P}} \def\cR{{\mathcal R}}
\def\scS{{\mathscr S}} \def\scC{{\mathscr C}}
\def\scP{{\mathscr P}} \def\scD{{\mathscr D}}
\def\scM{{\mathscr M}} \def\scR{{\mathscr R}}
\def\cF{{\mathcal F}} \def\cL{{\mathcal L}} \def\cR{{\mathcal R}}
\def\cI{{\mathcal I}} \def\cJ{{\mathcal J}} \def\cK{{\mathcal K}}
\def\cT{{\mathcal T}} \def\cD{{\mathcal D}} \def\cC{{\mathcal C}}
\def\cS{{\mathcal S}} \def\cSc{{\mathcal S \mathcal C}}
\def\cA{{\mathcal A}}
\def\cO{{\mathcal O}}
\def\cX{{\mathcal X}}
\def\cY{{\mathcal Y}}
\def\vU{{\varUpsilon}}

\def\Aut{\operatorname{Aut}} \def\cdiam#1{\operatorname{cdiam}(#1)}
\def\slope{\sphericalangle} \let\stbullet=\bullet
% \def\bullet{{\,\boldsymbol\cdot}\,}
\def\lbar{{\hat\l}} \def\mult{\operatorname{mult}}

\def\V{V} \def\trans{\pitchfork}

\def\rest#1{{\vert_{#1}}} \def\onL{\rest{\cL}}

\def\st#1{{\widetilde{#1}}}

\def\clo{\operatorname{Clo}}
\def\vol{\operatorname{Vol}}
\def\ivol{\#}
\def\supp{\operatorname{supp}}
\def\Gal{\operatorname{Gal}}
\def\vell{{\boldsymbol\ell}}
\def\w{\omega}


\def\id{\operatorname{id}}
\def\spec{\operatorname{Spec}}

\def\Qa{\Q^{\text{alg}}}
\def\O{\mathcal{O}}
\def\Ob{\overline{\O}}

\def\LT{\operatorname{LT}}
\def\height{\operatorname{ht}}
\def\fp{{\mathfrak p}}
\def\fq{{\mathfrak q}}

\def\vd{{\mathbf d}}
\def\vf{{\mathbf f}}
\def\vg{{\mathbf g}}
\def\vx{{\mathbf x}}
\def\vz{{\mathbf z}}
\def\vw{{\mathbf w}}
\def\vy{{\mathbf y}}
\def\vu{{\mathbf u}}
\def\vp{{\mathbf p}}
\def\vsigma{{\boldsymbol\sigma}}
\def\valpha{{\boldsymbol\alpha}}
\def\vgamma{{\boldsymbol\gamma}}
\def\vzeta{{\boldsymbol\zeta}}
\def\vdelta{{\boldsymbol\delta}}
\def\vepsilon{{\boldsymbol\epsilon}}
\def\ve{{\boldsymbol\e}}
\def\vrho{{\boldsymbol\rho}}
\def\vsigma{{\boldsymbol\sigma}}
\def\vnu{{\boldsymbol\nu}}

\def\CF{C_{\mathrm F}}
\def\hCF{{{\hat C}_{\mathrm F}}}
\def\LcC{L\cC}\def\RLcC{\R L\cC}
\def\Exp{\operatorname{Exp}}
\def\Qua{\mathcal{Q}}
\def\an{{\text{an}}}

\def\he#1{{\{#1\}}}
\def\hrho{{\he\rho}}
\def\hvrho{{\he\vrho}}
\def\hsigma{{\he\sigma}}
\def\hgamma{{\he\gamma}}
\def\hvsigma{{\he\vsigma}}

\def\alg{\mathrm{alg}}
\def\trans{\mathrm{trans}}
\def\RE{\mathrm{RE}}


\DeclareMathOperator{\lcm}{lcm}
\DeclareMathOperator{\pos}{pos}


\begin{document}

% +Title
\title{Complex Cellular Structures}

\author{Gal Binyamini}
\address{Weizmann Institute of Science, Rehovot, Israel}
\email{gal.binyamini@weizmann.ac.il}
\thanks{This research was supported by the ISRAEL SCIENCE FOUNDATION
  (grant No. 1167/17) and by funding received from the MINERVA
  Stiftung with the funds from the BMBF of the Federal Republic of
  Germany}

\author{Dmitry Novikov}
\address{Weizmann Institute of Science, Rehovot, Israel}
\email{dmitry.novikov@weizmann.ac.il}

\subjclass[2010]{14P10,37B40,03C64,30C99}
\keywords{Gromov-Yomdin reparametrization, Cellular decomposition,
  Topological entropy, Diophantine geometry}

\date{\today}

\begin{abstract}
  We introduce the notion of a \emph{complex cell}, a complexification
  of the cells/cylinders used in real tame geometry. Complex cells are
  equipped with a natural notion of holomorphic extension, and the
  hyperbolic geometry of a cell within its extension provides the
  class of complex cells with a rich geometric function theory absent
  in the real case. We use this to prove a complex analog of the
  cellular decomposition theorem of real tame geometry. In the
  algebraic case we show that the complexity of such decompositions
  depends polynomially on the degrees of the equations involved.

  Using this theory, we sharpen the Yomdin-Gromov algebraic lemma on
  $C^r$-smooth parametrizations of semialgebraic sets: we show that
  the number of $C^r$ charts can be taken to be polynomial in the
  smoothness order $r$ and in the complexity of the set. The algebraic
  lemma was initially invented in the work of Yomdin and Gromov to
  produce estimates for the topological entropy of $C^\infty$
  maps. Combined with work of Burguet, Liao and Yang, our refined
  version establishes an optimal sharpening of these estimates for
  \emph{analytic} maps, in the form of tight bounds on the tail
  entropy and volume growth. This settles a conjecture of Yomdin who
  proved the same result in dimension two in 1991. A self-contained
  proof of these estimates using the refined algebraic lemma is given
  in an appendix by Yomdin.

  The algebraic lemma has more recently been used in the study of
  rational points on algebraic and transcendental varieties. We use
  the theory of complex cells in these two directions. In the
  algebraic context we prove a sharpening of a result of Heath-Brown
  on interpolating rational points in algebraic varieties. In the
  transcendental context we prove an interpolation result for
  (unrestricted) logarithmic images of subanalytic sets.
\end{abstract}
%% -Title
\maketitle
\date{\today}

\setcounter{tocdepth}{1}
{\small \tableofcontents}

\section{Introduction}
\label{sec:intro}

In this section we aim to provide an intuitive motivation for the
formal notion of complex cells introduced
in~\secref{sec:complex-cells}. We discuss the smooth parametrization
problem, particularly the Yomdin-Gromov algebraic lemma and its
sharpenings established in this paper. We then discuss the hyperbolic
obstruction to a Gromov-Yomdin type lemma in the holomorphic category
and motivate the notion of a complex cell as a way of overcoming this
obstruction. Finally we briefly discuss the applications of our
parametrization results in smooth dynamics and in diophantine
geometry.

\subsection{Smooth parametrizations}
\label{sec:intro-smooth}

\subsubsection{The Yomdin-Gromov Algebraic Lemma}

In \cite{yomdin:entropy,yomdin:gy} Yomdin gave a proof of Shub's
entropy conjecture for $C^\infty$-maps. A key step in this proof was
the construction of $C^r$-smooth parametrizations of semialgebraic
sets with the number of charts depending only on the combinatorial
data. More precisely, for a $C^r$-smooth function
$f:U\to\R^n$ on a domain $U\subset\R^m$ we denote by
$\norm{f}$ the maximum norm on $U$ and
\begin{equation}
  \norm{f}_r := \max_{|\valpha|\le r} \frac{\norm{D^\valpha f}}{\valpha!}.
\end{equation}
The original parametrization theorem of
\cite{yomdin:entropy,yomdin:gy} involved a technical condition on the
removal of a small piece from the semialgebraic set being
parameterized. The formulation was further sharpened by Gromov
\cite{gromov:gy}, who proved the result in the following clean
formulation known as the Yomdin-Gromov Algebraic Lemma (see also
\cite{burguet:alg-lemma} for a detailed proof).

\begin{Thm*}[\protect{\cite[3.3.~Algebraic Lemma]{gromov:gy}}]
  Let $X\subset[0,1]^n$ be a semialgebraic set of dimension $\mu$
  defined by conditions $p_j(\vx)=0$ or $p_j(\vx)<0$ where $p_j$ are
  polynomials and $\sum\deg p_j=\beta$. Let $r\in\N$. There exists a
  constant $C=C(n,\mu,r,\beta)$ and semialgebraic maps
  $\phi_1,\ldots,\phi_C:(0,1)^\mu\to X$ of such that their images
  cover $X$ and $\norm{p_j}_r\le 1$ for $j=1,\ldots,C$.
\end{Thm*}

Understanding the behavior of the constant $C(n,\mu,r,\beta)$ has been
the key difficulty in establishing a conjectural sharpening of
Yomdin's results for analytic maps, as we discuss
in~\secref{sec:intro-dynamics}. More precisely, what is needed is an
estimate $C(n,\mu,r,\beta)=\poly_n(r,\beta)$. Such an estimate is our
first main result.

\begin{Thm}
  In the Yomdin-Gromov algebraic lemma one may take
  $C(n,\mu,r,\beta)=\poly_n(\beta)\cdot r^\mu$. Moreover the maps
  $\phi_j$ can be chosen to be semialgebraic of complexity
  $\poly_\ell(\beta,r)$.
\end{Thm}

\subsubsection{Pila-Wilkie's generalization to o-minimal structures}

Parametrizations by $C^r$-smooth functions have also been used to
great effect in the seemingly unrelated study of rational points on
algebraic and transcendental sets. The algebraic lemma was first
introduced to this subject by Pila and Wilkie \cite{pila-wilkie}, who
proved its far-reaching generalization for arbitrary o-minimal
structures.

\begin{Thm*}[\protect{\cite[Theorem~2.3]{pila-wilkie}}]\label{thm:yg-pw}
  Let $\{X_p\subset[0,1]^n\}$ be a family of sets definable in an
  o-minimal structure, with $\dim X_p\le\mu$. There exists a constant
  $C=C(X,r)$ such that for any $p$ there exist
  definable maps $\phi_1,\ldots,\phi_C:(0,1)^\mu\to X_p$ such that their
  images cover $X$ and $\norm{\phi_j}_r\le 1$ for $j=1,\ldots,C$.
\end{Thm*}

In the diophantine direction, understanding the behavior of $C(X,r)$
with respect to the geometry of the family $X$ and the smoothness
order $r$ is also a question of great importance as we discuss
in~\secref{sec:intro-rational}. In the recent paper \cite{cpw:params}
Cluckers, Pila and Wilkie made significant progress in this direction
by proving that for subanalytic sets (and also sets in the larger
structure $\R_\an^{\text{pow}}$) one has $C(X,r)=\poly_X(r)$. For
subanalytic sets we obtain a similar result with a precise control
over the degree.

\begin{Thm}
  In the Pila-Wilkie algebraic lemma for $\R_\an$ one may take
  $C(X,r)=O_X(r^\mu)$ where $\mu:=\max_p(\dim X_p)$.
\end{Thm}


\subsubsection{Mild parametrizations}

It is natural to inquire whether one can in fact replace the
$C^r$-charts in the algebraic lemma with $C^\infty$ charts with
appropriate control over the derivatives. In \cite{pila:mild} Pila
introduced a notion of this type called \emph{mild parametrization}
and investigated its diophantine applications related to the Wilkie
conjecture.

\begin{Def}[Mild parametrization]
  A smooth map $f:U\to[0,1]^n$ on a domain $U\subset\R^m$ is said
  to be $(A,C)$-mild if
  \begin{equation}\label{eq:mild-def}
    \norm{ D^\valpha f } \le \valpha! (A|\valpha|^C)^{|\valpha|}, \qquad \forall\valpha\in\N^m.
  \end{equation}
\end{Def}

In \cite{jmt:mild} it is shown that every subanalytic set
$X\subset [0,1]^n$ admits $(A,0)$-mild (i.e. analytic)
parametrizations, but this result is not uniform over families and its
diophantine applications are restricted to the case of curves
\cite{jmt:mild} and surfaces \cite{jt:pfaff-surfaces}. We prove the
following uniform version.

\begin{Thm}
  Let $\{X_p\subset[0,1]^n\}$ be a subanalytic family of sets, with
  $\dim X_p\le\mu$. There exist constants $A=A(X)$ and $C=C(X)$ such
  that for any $p$ there exist $(A,2)$-mild maps
  $\phi_1,\ldots,\phi_C:(0,1)^\mu\to X_p$ whose images cover $X$.  If
  $\{X_p\}$ is semialgebraic as in the Yomdin-Gromov algebraic lemma
  then one may take $A,C=\poly_n(\beta)$.
\end{Thm}

The construction of a mild parametrization is a-priori more delicate
than its $C^r$ counterpart since it requires one to control
derivatives of all orders at once. In fact, Thomas \cite{thomas:mild}
has shown that there exist o-minimal structures without definable mild
parametrizations. In our results there is also a key difference
between the $C^r$ algebraic lemma and its mild counterpart. Namely, in
the $C^r$ version the parameterizing maps are themselves subanalytic,
and can be chosen to depend subanalytically on the parameters of the
family. We make no such guarantee in the mild version. In fact, we show
that any parameterization of a family of hyperbolas by a definable
family of $\R_\an$ maps must have unbounded $C^r$-norms for some
finite $r$. More precisely we have the following sharp estimate.

\begin{Prop}\label{prop:hyp-param}
  Let $F:=(x,y):[0,1]^2\to[0,1]^2$ be an $\R_\an$-definable map
  satisfying $x(e,t)\cdot y(e,t)=e$. Suppose that for each $r$ there
  exists a constant $M_r$ such that
  \begin{align}\label{eq:derivatives bounded}
    \left|\partial_t^r x(e,t)\right|&<M_r & \left|\partial_t^r y(e,t)\right|&<M_r
  \end{align}
  whenever the derivatives are defined. Then the Euclidean length of
  $\log F(e_0,[0,1])$ is bounded by a constant independent of
  $e_0\neq0$ (here $\log$ is applied coordinate-wise).
\end{Prop}

This result is easily seen to be essentially tight: one can indeed
construct parameterizations $F$ as in Proposition~\ref{prop:hyp-param}
to cover any definable family of intervals of constant logarithmic
length in the hyperbolas. Somewhat surprisingly, our proof goes via
reduction using complex cells to the following simple statement from
geometric function theory, which is proved using a $p$-valent version
of the Koebe 1/4-theorem.

\begin{Lem}\label{lem:log-length}
  Let $\xi:D(2)\to\C\setminus\{0\}$ be holomorphic and suppose that
  $\xi$ is $p$-valent. Then the length of $\log\xi([-1,1])$ is bounded
  by a constant $8\pi p(p+1)$.
\end{Lem}

The proof of Proposition~\ref{prop:hyp-param} and
Lemma~\ref{lem:log-length} is given
in~\secref{sec:sec:uniform-param-Ran}.

\subsection{Holomorphic parametrizations}

In the preceding section we discussed the problem of constructing
smooth parametrizations for semialgebraic and subanalytic sets. It is
natural to wonder whether such constructions can be analytically
continued to give holomorphic parametrizations in some suitable
sense. In this section we demonstrate an obstruction of a
hyperbolic-geometric nature to a naive formulation, and motivate the
notion of a complex cell as a possible way of overcoming this
obstruction.

\subsubsection{Parametrizations of analytic curves and questions of
  uniformity}
\label{sec:parametrization-uniformity}

Let $D\subset\C$ denote the unit disc and $D^{1/2}\subset\C$ denote
the disc of radius two. If $X$ is a holomorphic curve and $K\subset X$
is compact then one can always cover $K$ by images of analytic charts
$f_j:D\to X$. To avoid bad behavior near the boundary, we can also
require that each $f_j$ extend as a holomorphic map to $D^{1/2}$ (this
is an arbitrary choice which could be replaced by any fixed
neighborhood of $\bar D$). Parameterizations of this type are useful
in analysis: they allow one to study holomorphic structures on $X$ in
terms of their pullback to $D\subset D^{1/2}$, where a rich geometric
function theory is available. Moreover, the theory of resolution of
singularities shows that this type of parametrization can be extended
to parametrizations of analytic sets of any dimension. A notion of
this type was introduced by Yomdin \cite{yomdin:entropy-analytic}
under the name \emph{analytic complexity unit (acu)}. A similar notion
was considered under the name ``doubling coverings'' in
\cite{yf:doubling}.

Ideally one would like to prove that holomorphic parameterizations of
the type described above can be made with a uniform number of charts
when $X$ is replaced by an analytic family $X_\e$ of curves (or higher
dimensional sets) and $K$ by its compact subfamily. Indeed, this would
be an appropriate analog of the Yomdin-Gromov algebraic lemma in the
holomorphic category. Unfortunately this is impossible, even for
simple families of algebraic curves. Consider the family $X_\e$ of
hyperbolas restricted to the polydisc of radius 2 and $K_\e$ their
restriction to a polydisc of radius 1,
\begin{align}
  X_\e &:= \{(x,y): xy=\e,\ x,y\in D^{1/2}\} & K_\e &:= X_\e\cap (D\times D).
\end{align}
The projection to the $x$-axis gives biholomorphisms
$K_\e\simeq A_\e:=A(\e,1)$ and $X_\e\simeq A_\e^{1/2}:=A(\e/2,2)$
where $A(r_1,r_2):=\{r_1<|z|<r_2\}$. Let
$f:D^{1/2}\to X_\e\simeq A_\e^{1/2}$. If we equip $D^{1/2}$ and
$A_\e^{1/2}$ with their respective hyperbolic metric then the
Schwarz-Pick lemma implies that $f$ is distance-contracting
(see~\secref{sec:fund-lemmas} for a brief reminder on this
subject). In particular the diameter of $f(D)$ in $A_\e^{1/2}$ is
bounded by the diameter of $D$ in $D^{1/2}$, which is a constant
independent on $\e$. On the other hand, the diameter of
$K_\e\simeq A_\e$ in $A_\e^{1/2}$ tends to infinity as $\e\to0$ (in
fact with order $\log|\log\e|$, see
Remark~\ref{rem:Adelta-diameter}). It follows that to cover $K_\e$ by
images of maps as above at least $\log|\log\e|$ charts are required,
and this is easily seen to be a tight asymptotic. We note that Yomdin
\cite{yomdin:param-dim2} obtains a similar result with $|\log\e|$
which holds under the assumption that the maps are $p$-valent for some
fixed $p$.

\subsubsection{A uniform parameterization result with discs and annuli}

It turns out that in the one-dimensional case, the hyperbolic
restriction described above is essentially the only obstacle to
uniform parametrization. Namely, suppose that in addition to covering
$K_\e$ by images $f(D)$ of holomorphic maps $f:D^{1/2}\to X_\e$ we
allow also images $f(A)$ of holomorphic maps $f:A^{1/2}\to X_\e$ for
any annulus $A$. Note that the diameter $A$ in $A^\delta$ is not
uniformly bounded: it depends on the conformal modulus. With this
added freedom one can indeed construct a parameterization with a
uniformly bounded number of charts for a family of curves. We proceed
to explain the elementary geometric considerations that lead to this
result.

To simplify our presentation we suppose that
$X_\e\subset\C^2\times\C_\e$ is a family of algebraic plane curves
which project properly under $\pi:(x,y)\to x$ (although a similar local
argument works in the analytic case without the assumption of
properness as well). We will parameterize the fibers of
$K_\e:=X_\e\cap(D\times D)$. Fix $\e\in\C$. Let $\Sigma\subset\C$
denote the finite set of critical values of $\pi\rest{X_\e}$. Denote by $\nu$ the
order of the monodromy group of $\pi$. Note that $\#\Sigma$ and $\nu$
are bounded in terms of $\deg X_\e$ and in particular uniformly in
$\e$. Suppose $\cC\subset\C$ is one of the following three types:
\begin{enumerate}
\item A disc $p+D(r)$ such that $p+D(2r)$ does not meet $\Sigma$.
\item A disc $p+D(r)$ such that $p\in\Sigma$ and $p+D(2^\nu r)$ does
  not meet any additional points in $\Sigma$.
\item An annulus $p+A(r_1,r_2)$ such that $p+A(r_1/2^\nu,2^\nu r_2)$ does not
  meet $\Sigma$.
\end{enumerate}
In case 1 one can construct a map $f:D\to p+D(r)\to X_\e$ by lifting
$\pi$. In cases 2,3 the lift might be multivalued with a monodromy of
finite order $\nu$ (uniformly bounded in $\e$), but precomposing with
the map $z\to z^\nu$ we similarly obtain univalued charts
$f:D\to X_\e$ or $f:A_r\to X_\e$. Moreover the assumption on $\cC$
ensure that $f$ extends holomorphically to $D^{1/2}$ or
$A_r^{1/2}$. Taking the $\deg X_\e$ different lifts of $\pi$ we thus
obtain charts covering $X_\e\cap\pi^{-1}(\cC)$.

It remains to show that $D$ can be covered by finitely many domains
$\cC$ as above, with their number depending only on the number of
points in $\Sigma$ (but not on their positions). This is a simple
exercise in plane geometry which we urge the reader to attempt for
themselves. It is also instructive to check that a similar statement
would not hold had we not allowed annuli as in item 3 above.  In
Figure~\ref{fig:diphyp} we illustrate such a decomposition for the
critical points $\Sigma=\{\pm\sqrt\e\}$ of the hyperbola $y^2=x^2-\e$
(this is just our original example $xy=\e$ rotated to satisfy our
assumption of proper projection to the $x$-axis).
\begin{figure}
  \centering
  \includegraphics[width=\textwidth]{diphyp.pdf}
  \caption{Covering of the hyperbola with discs and annuli, where $*$
    corresponds to points of $\Sigma=\{\pm\sqrt{\e}\}$.}
  \label{fig:diphyp}
\end{figure}

Decompositions of a similar type appeared under the name
``Swiss-cheese decompositions'' in \cite{hs:variations} where they
were used in the study of Green functions on Riemann surfaces. They
also appeared in \cite{me:inf16} where they were used in the study of
collisions of singular points of Fuchsian differential equations.

\subsubsection{Parametrizing higher dimensional sets}

It is natural to ask whether a similar ``uniform parametrization''
statement might hold in higher dimensions. A naive attempt might be to
use products of discs and annuli as the domains of
definition. However, since complex annuli (unlike discs) have
conformal moduli, it turns out to be more natural to allow domains
consisting of families of annuli with varying moduli. As an
illustrative example, in the two-dimensional case we allow domains of
the form
\begin{equation}
  \cC=D_\circ(1)\odot A(\vz_1,2) := \{(\vz_1,\vz_2) : 0<|\vz_1|<1,\ |\vz_1|<|\vz_2|<2\}.
\end{equation}
The notation $\odot$ above is an example of a more general notation
introduced in~\secref{sec:complex-cells}, and the domain above is an
example of a \emph{complex cell}. The definition closely resembles the
notion of cells/cylinders in semialgebraic/subanalytic geometry, with
the order relation $x<y$ replaced by the complex $|x|<|y|$.

To generalize the parametrization result from curves to higher
dimensional sets we require not only an analog of the domains $D$ or
$A$, but also an analog of their extensions
$D^{1/2},A^{1/2}$. Correspondingly our complex cells $\cC$ are
endowed with a natural notion of $\delta$-extension
$\cC\subset\cC^\delta$. In the example above
\begin{multline}
  \cC^\delta = D^\delta(1)\odot A^\delta(\vz_1,2) = D(\delta^{-1})\odot A(\delta\vz_1,2\delta^{-1}) :=\\
  \{(\vz_1,\vz_2) : |\vz_1|<\delta^{-1},\ \delta|\vz_1|<|\vz_2|<2\delta^{-1}\}.
\end{multline}
This is defined for any $1/2<\delta<1$: we require the original fiber
$A(\vz_1,2)$ to remain an annulus over the extension $D^\delta(1)$,
and for $\delta<1/2$ it would become an empty set.

Our main result, Theorem~\ref{thm:cpt}, shows in particular that one
can uniformly uniformize any family of analytic sets if one is
permitted to use general complex cells as the domains for the charts.
More generally, Theorem~\ref{thm:cpt} provides a cellular analog of
the constructions of local resolution of singularities (LRS), see
e.g. \cite{bm:subanalytic}. Loosely speaking it allows one to
transform a collection of analytic functions into normal crossings
uniformly over families - using complex cells as the domains of the
charts. In the algebraic category, Theorem~\ref{thm:cpt} gives
effective control (polynomial in the degrees) on the number and
complexity of the charts in terms of the complexity of the functions
being transformed into normal crossings. This is where the most
important step toward improving the asymptotics of the Yomdin-Gromov
algebraic lemma takes place, and the proof makes extensive use of the
hyperbolic properties of a cell $\cC$ viewed as a subset of it
extension $\cC^\delta$.

\subsubsection{Sectorial parametrizations}

In the preceding sections we discussed the problem of establishing an
analog of the algebraic lemma with holomorphic functions admitting
extensions to whole discs, and showed that (unless one also allows
annuli and more general complex cells) such an analog is impossible
for hyperbolic reasons. A more modest goal might be to establish an
analog of the algebraic lemma where the parametrizing maps admits
analytic continuation to some suitable large domain (though not a full
disc). If the domain is sufficiently large, one could then hope to
control the derivatives using complex-analytic methods, for instance
the Cauchy estimates.

It turns out that the correct domains for analytic continuation are
complex sectors. More specifically, let $S(\e)$ denote the sector
$S(\e)=\{\Arg z<\e,|z|<2\}$, and for $B=(0,1)^\mu$ let $B(\e)$ denote
the direct product of $\mu$ copies of $S(\e)$. All of our refinements
of the algebraic lemma follow from the following statement about
parametrizations with complex-analytic continuation to sectors.

\begin{Thm}
  Let $\{X_p\subset[0,1]^n\}$ be a subanalytic family of sets, with
  $\dim X_p\le\mu$. Set $B=(0,1)^\mu$. There constants $C=C(X)$ and
  $\e=\e(X)$ such that for any $p$ there exist maps
  $\phi_1,\ldots,\phi_C:B\to X_p$ whose images cover $X_p$. Moreover
  each $\phi_j$ extends holomorphically to $B(\e)$ and has unit
  $C^1$-norm there. If $\{X_p\}$ is semialgebraic as in the
  Yomdin-Gromov algebraic lemma then one may take
  $C,\e^{-1}=\poly_n(\beta)$.
\end{Thm}

This theorem is proved (in a more general form)
in~\secref{sec:sectorial-param}. We start by constructing a
complex-cellular parameterization for $X_p$, and then show how to
construct maps from $B$ into a complex cell which extend to $B(\e)$
with bounded $C^1$-norms. The $C^r$ and mild statements of the
algebraic lemma are deduced from the sectorial version by a simple
(and self-contained) argument based on the Cauchy estimates
in~\secref{sec:param-proofs}.

\subsection{Applications in dynamics}
\label{sec:intro-dynamics}

The Yomdin-Gromov algebraic lemma was first invented for the purpose
of studying the properties of the topological entropy of smooth maps.
We recall the basic notions of topological entropy theory, Shub's
entropy conjecture and Yomdin's result for $C^\infty$-maps. We then
discuss a refinement of these results in the analytic category
culminating in Theorem~\ref{thm:analytic-entropy} confirming in
arbitrary dimension a conjecture of Yomdin that was previously known
only in dimension two. Theorem~\ref{thm:analytic-entropy} follows
immediately from the refined algebraic lemma in combination with the
work of Burguet, Liao and Yang \cite{bly}. A self-contained proof of
Theorem~\ref{thm:analytic-entropy} using the refined algebraic lemma
is given in~\secref{appendix:yomdin} (by Yomdin). Below we mostly
follow the notations and presentation of \cite{bly} in our survey.

\subsubsection{Topological entropy}

Let $M$ be a compact metric space with metric $d:M^2\to\R_{\ge0}$ and
let $f:M\to M$ be a continuous map. For $n\in\N$ define the $n$-th
iterated metric by
\begin{equation}
  d_n:M\times M\to\R_{\ge0}, \qquad d_n(x,y)=\max_{i=0,\ldots,n-1} d(f^{\circ i}(x),f^{\circ i}(y)).
\end{equation}
Recall if $X$ is a metric space and $\Lambda\subset X$ we say that a
set $K\subset X$ is $\e$-spanning for $\Lambda$ if the
$\e$-neighborhood of $K$ contains $\Lambda$. For any subset
$\Lambda\subset M$ and $\e>0$ and $n\in\N$ we let $r_n(f,\Lambda,\e)$
denote the minimal cardinality of an $\e$-spanning set of $\Lambda$
with respect to the $d_n$-metric.

The \emph{$\e$-topological} and \emph{topological} entropies of
$\Lambda$ are defined by
\begin{align}
  h(f,\Lambda,\e) &= \limsup_{n\to\infty} \frac1n \log r_n(f,\Lambda,\e), & h(f,\Lambda) &= \lim_{\e\to0} f(f,\Lambda,\e).
\end{align}
We set $h(f,\e):=h(f,M,\e)$ and $h(f):=h(f,M)$. This latter quantity
is called the \emph{topological entropy} of $f$. As its name implies,
the topological entropy is independent of the choice of metric $d$. It
was first defined by Adler, Konheim and McAndrew \cite{akm:entropy} in
a purely topological fashion, and later shown by Bowen
\cite{bowen:entropy-group} to be equivalent to the metric definition
presented above.

\subsubsection{Tail entropy}

For $x\in M$ we define the \emph{infinite dynamical ball}
$B_\infty(x,\e)$ by
\begin{equation}
  B_\infty(x,\e) := \{y\in M: d_n(x,y)<\e\quad \forall n\in\N\}.
\end{equation}
Following Bowen \cite{bowen:entropy-expansive}, the \emph{$\e$-tail}
and \emph{tail} entropies of $f$ are defined by
\begin{align}
  h^*(f,\e) &= \sup_{x\in M} h(f,B_\infty(x,\e)), & h^*(f)=\lim_{\e\to0} h^*(f,\e).
\end{align}
Bowen \cite{bowen:entropy-expansive} has shown that the $\e$-tail
entropy bounds the difference between the $\e$-entropy and the
entropy, that is
\begin{equation}\label{eq:entropy-v-tail}
  |h(f)-h(f,\e)|<h^*(f,\e).
\end{equation}
If $h^*(f,\e)=0$ for some $\e>0$ then the system $(f,M)$ is called
entropy-expansive ($h$-expansive); if $h^*(f)=0$ then it is called
\emph{asymptotically entropy expansive} (asymptotically
$h$-expansive). This notion of tail entropy was first introduced (with
a different definition) by Misiurewicz \cite{misiurewicz:cond-entropy}
under the name \emph{conditional entropy}. Misiurewicz showed that for
asymptotically $h$-expansive systems the measure-theoretical entropy
is upper-semicontinuous, and such systems therefore always admit an
invariant measure of maximal entropy.

\subsubsection{Yomdin's results for smooth maps}
\label{sec:intro-yomdin-thm}

Assume from now on that $M$ is a compact $C^\infty$-smooth manifold
equipped with a Riemannian metric. In \cite{yomdin:entropy} Yomdin
proved Shub's entropy conjecture for $C^\infty$ maps. This conjecture
states that the logarithm of the spectral radius of $\spec f$ of
$f_*:H_*(M,\R)\to H_*(M,\R)$ is a lower bound for $h(f)$. In fact it
is relatively easy to show (by comparing volumes) that
\begin{equation}\label{eq:spec-vs-h}
  \log\spec f<h(f)+\max_{k=1,\ldots,\dim M} v_k^*(f)
\end{equation}
where $v_k^*(f)$ is the $k$-dimensional \emph{local volume growth}, a
quantity related to $h^*(f)$ whose precise definition we postpone
to~\secref{appendix:yomdin}. Yomdin's fundamental result was that if
$f$ is a $C^r$-map then
\begin{equation}
  v_k^*(f) \le \frac{k R(f)}{r}, \qquad
  R(f) = \lim_{n\to\infty} \frac1n \sup_{x\in M} \norm{D_x f^{\circ n}}
\end{equation}
For $C^\infty$-maps this implies $v_k^*(f)=0$, and in combination
with~\eqref{eq:spec-vs-h} yields Shub's entropy conjecture. Buzzi
\cite{buzzi} later observed that Yomdin's argument also implied the
same estimate $h^*(f)\le (\dim M/r) R(f)$ for the tail entropy. For
$C^\infty$-maps this implies $h^*(f)=0$, i.e. that $C^\infty$-maps are
asymptotically $h$-expansive.

\subsubsection{Controlling $h^*(f,\e)$ for analytic maps}

Bounding $h^*(f,\e)$ explicitly as a function of $\e$ is an important
problem with consequences for the computation of $h(f)$ (for instance
using~\eqref{eq:entropy-v-tail}) and for the study of the
semicontinuity properties of the entropy. However, for arbitrary
$C^\infty$ maps $f:M\to M$ the rate of convergence of $h^*(f,\e)$ to
zero as $\e\to0$ can be arbitrarily slow, as shown by the examples of
Burguet, Liao and Yang \cite[Theorem~L]{bly}.

In \cite{yomdin:entropy-analytic} Yomdin considered the case of a
real-analytic map $f:M\to M$ of a compact analytic surface $M$. In
this context, using a holomorphic variant of the algebraic lemma based
on the Bernstein inequality for polynomials, he was able to prove that
\begin{equation}
  v_1^*(f,\e)\le C(f)\cdot \frac{\log|\log\e|}{|\log\e|}.
\end{equation}
Yomdin conjectured \cite[Conjecture~6.1]{yomdin:entropy-analytic} that
a similar estimate should hold (for $v^*_k$) for $M$ of arbitrary
dimension, but the limitation of the holomorphic parametrization
technique to one dimension prevented such a generalization (see
\cite{yomdin:param-dim2} for some results in this direction in
dimension two). In \cite[Theorem~N]{bly} it is shown that the bound in
Yomdin's conjecture is essentially sharp (for class slightly larger
than the analytic class).

In \cite{bly} Burguet, Liao and Yang revisited the problem of the rate
of convergence of the tail entropy for analytic, and more general,
maps. The precise statement of their main result is technical and
depends on the behavior of the algebraic lemma's constant
$C(n,\mu,r,\beta)$. However, under the hypothesis
$C(n,\mu,r,\beta)=\poly_n(r,\beta)$ the results of
\cite[Corollary~B]{bly} take a particularly simple form: namely, they
imply the generalization of Yomdin's result (both for $v^*_k$ and
$h^*$) to arbitrary dimension (and also for a class of functions
somewhat larger than the analytic class). It is also shown in
\cite{bly} that $C(n,1,r,\beta)=\poly_n(r,\beta)$ and this recovers
Yomdin's result for analytic surfaces.

To summarize, following the work of \cite{bly} it has been clear that
proving the $C(n,\mu,r,\beta)=\poly_n(r,\beta)$ is the missing step to
establishing Yomdin's conjecture. Combined with the refined algebraic
lemma proved in this paper this is now a theorem.

\begin{Thm}\label{thm:analytic-entropy}
  Let $M$ be a compact analytic manifold and $f:M\to M$ be an analytic
  map. Then
  \begin{align}\label{eq:analytic-entropy}
    h^*(f,\e) &\le C(f)\cdot\frac{\log|\log\e|}{|\log\e|}, &
    v_k^*(f,\e)&\le C(f)\cdot\frac{\log|\log\e|}{|\log\e|}
  \end{align}
  for $k=1,\ldots,\dim M$.
\end{Thm}
\begin{proof}
  We indicate the proof with the notations of \cite{bly}. Applying the
  refined algebraic lemma to the graph of the polynomial map
  $P:[0,1]^l\to[0,1]^l$ in \cite[Section~2.1.2]{bly} one obtains the
  estimate $C_{r,l,m}=\poly_{m,l}(r)$. This implies that the sequence
  $M_k:=k^{k^2}$ is $(l,m)$-admissible for any $l,m\in\N$. Then
  \cite[Corollary~B]{bly} applies to any map $f$ which satisfies
  $\norm{f}_r < C r^{r^2}$ for some constant $C$ and every $r\ge0$. In
  particular, it applies to analytic maps. We also have
  \begin{equation}
    G_M(t) = \sup_k \{ k\log k<t \} = \Theta\left(\frac{t}{\log t}\right).
  \end{equation}
  Thus~\eqref{eq:analytic-entropy} follows from the conclusion of
  \cite[Corollary~B]{bly}.
\end{proof}

\subsection{Applications in diophantine geometry}
\label{sec:intro-rational}

For $p\in\P^\ell(\Q)$ we define $H(p)$ to be $\max_i |\vp_i|$ where
$\vp\in\Z^{\ell+1}$ is a projective representative of $p$ with
\begin{equation}
  \gcd(\vp_0,\ldots,\vp_\ell)=1.
\end{equation}
For $\vx\in\A^\ell(\Q)$ we define its height to be the height of
$\iota(\vx)$ for the standard embedding
\begin{equation}
  \iota:\A^\ell\to\P^\ell,\qquad \iota(\vx_{1..\ell}) = (1:\vx_1:\cdots:\vx_\ell).
\end{equation}
For a set $X\subset\P^\ell(\R)$ we denote
\begin{equation}
 X(\Q,H) := \{ \vx\in X\cap \P(\Q)^\ell : H(x)\le H\},
\end{equation}
and similarly for $X\subset\R^\ell$.

In \cite{bombieri-pila} Bombieri and Pila introduced the interpolation
determinant method for estimating the quantity $\#X(\Q,H)$ as a
function of $H$ when $X$ is the graph of a $C^r$ (or $C^\infty$)
smooth function $f:[0,1]\to\R$. It turns out that two very different
asymptotic behaviors are obtained depending on whether the graph of
$f$ belongs to an algebraic plane curve. The algebraic lemma has been
used in both of these directions, to generalize from graphs of
functions to more general sets. In~\secref{sec:bp-complex} we give a
complex-cellular analog of the Bombieri-Pila determinant method and
use it to deduce some applications for both the algebraic and
transcendental contexts. We briefly describe the main results below.

\subsubsection{A result for algebraic varieties}

We prove the following result.

\begin{Thm}\label{thm:improved-marmon}
  Let $X\subset\P(\C)^\ell$ be an irreducible algebraic variety of
  dimension $m$ and degree $d$. Then $X(\Q,H)$ is contained in $N$
  hypersurfaces of degree $k$, none of which contain $X$, where
  \begin{equation}
    N = \poly_\ell(d,k)\cdot H^{(m+1)d^{-1/m}(1+\poly_\ell(d)/k)}.
  \end{equation}
\end{Thm}

We make note of two particular choices for $k$, namely
\begin{align}
  k&=\poly_\ell(d)/\e & &\implies & N &= \poly_\ell(d,1/\e)\cdot H^{(m+1)d^{-1/m}+\e} \\
  k&=\poly_\ell(d)\cdot\log H & &\implies & N &= \poly_\ell(d,\log H)\cdot H^{(m+1)d^{-1/m}}.
\end{align}
The first choice improves the dependence on the degree $d$ to
polynomial in Heath-Brown's result \cite{heath-brown:density} for
hypersurfaces and Broberg's result \cite[Theorem~1]{broberg:note} (see
also Marmon \cite{marmon}). The second choice replaces an $H^\e$
factor by a power of $\log H$, similar to the various results
established for curves by Pila \cite{pila:pems} and Salberger
\cite{salberger:density}. We remark that in the case of curves, Walsh
\cite{walsh:boundd-rational} recently proved a result eliminating the
$\log H$ factor altogether and it would be interesting to study
whether this can be generalized to arbitrary dimension. We briefly
survey the history of these results
in~\secref{sec:alg-density-history} and prove
Theorem~\ref{thm:improved-marmon} in~\secref{sec:alg-density-proof}.

\subsubsection{The Pila-Wilkie theorem}

Let $f:I\to\R^2$ and suppose that $X_f:=f(I)$ is
transcendental. Bombieri and Pila \cite{bombieri-pila} and Pila
\cite{pila:density-Q} used the determinant method to show that
$\#X(\Q,H)=O_{X,\e}(H^\e)$ for any $\e>0$. To generalize this result
to higher dimensions, let $X\subset\R^\ell$ and denote by $X^\alg$ the
union of all positive dimensional connected semialgebraic sets
contained in $X$ and by $X^\trans:=X\setminus X^\alg$. Pila and Wilkie
\cite{pila-wilkie} proved that for any $X$ definable in an o-minimal
expansion of $\R$ we have
\begin{equation}
  \#X^\trans(\Q,H)=O_{X,\e}(H^\e) \qquad \forall\e>0.
\end{equation}
Moreover, if $X$ varies over a definable family then the asymptotic
constants can be taken uniform over the family. The o-minimal version
of the Yomdin-Gromov algebraic lemma plays the central role in the
proof of the Pila-Wilkie theorem.

\subsubsection{The Wilkie conjecture}
Wilkie \cite{pila-wilkie} has conjectured that for sets definable in
$\R_{\exp}$ the asymptotic in the Pila-Wilkie theorem can be improved
to $\poly_X(\log H)$, and it is natural to make similar conjectures
for other ``natural'' geometric structures (although such a result
does not hold for general subanalytic curves, see
\cite[Example~7.5]{pila:subanalytic-dilation}). One of the key
obstacles to proving the Wilkie conjecture, at least following the
Pila-Wilkie strategy, is to obtain an estimate on $C(X,r)$ which is
polynomial in $r$ and the complexity of $X$. Some one-dimensional and
some restricted two-dimensional cases of the Wilkie conjecture have
been established using methods involving Pfaffian functions and mild
parametrizations
\cite{pila:pfaff,jt:pfaff-surfaces,pila:exp-alg-surface,butler}.  In
\cite{me:analytic-interpolation} we developed a complex-analytic
approach to the Pila-Wilkie theorem in the subanalytic case, and in
\cite{me:rest-wilkie} we used this approach to prove the Wilkie
conjecture for the structure $\R^\RE$ generated by the exponential and
trigonometric functions \emph{restricted} to some finite interval.

\subsubsection{Applications to unlikely intersections in diophantine geometry}
The Pila-Wilkie theorem has been applied to great effect in the study
of unlikely intersections in diophantine geometry. We briefly mention
Pila-Zannier's proof the the Manin-Mumford conjecture
\cite{pila-zannier}, Pila's proof of the Andr\'e-Oort conjecture for
modular curves \cite{pila:andre-oort} and Masser-Zannier's work on
simultaneous torsion points in elliptic families \cite{mz:torsion}. We
refer the reader to \cite{zannier:book,scanlon:survey} for a
survey. Some of the most striking diophantine applications,
particularly around the study of modular curves and Shimura varieties,
require the generality of the Pila-Wilkie theorem for $\R_{\an,\exp}$,
i.e. beyond the subanalytic category. Obtaining polylogarithmic
asymptotics for such unrestricted sets is the natural challenge going
beyond the scope of \cite{me:rest-wilkie}.

\subsubsection{Interpolating rational points on log-sets}

For definiteness let $\log_e:\C\setminus\{0\}\to\C$ denote the
principal branch of the standard complex logarithm. Below $\log$ can
denote $\lambda\log_e(\cdot)$ for any $\lambda\in\C\setminus\{0\}$,
i.e.  logarithm with respect to any (fixed) base. We extend $\log$ as
a function of several variables coordinate-wise.

\begin{Def}[Log-set]
  If $A\subset(\C\setminus\{0\})^\ell$ is a bounded subanalytic set
  (where $\C^\ell$ is identified with $\R^{2\ell}$) then we call
  $\log A$ a log-set.
\end{Def}

Any bounded subanalytic set is a log-set, but log-sets are of course
more general as they involve application of an \emph{unrestricted}
logarithm. In~\secref{sec:log-sets-pila} we show that log-sets appear
naturally in the application of the Pila-Wilkie theorem to the
Andr\`e-Oort conjecture for modular curves in Pila's work
\cite{pila:andre-oort}. They are therefore perhaps the most natural
candidate to consider looking for results in the direction of the
Wilkie conjecture for unrestricted sets (with an eye to the
diophantine applications). We prove the following result in this
direction.

\begin{Prop}\label{prop:log-set-pw}
  Let $\{X_\lambda\subset[0,1]^\ell\}$ be an $\R_\an$-definable family let
  $n$ denote the maximal dimension of the fibers of $X$. Fix
  $k\in\N$. There exists an $\R_\an$-definable family
  $\{Y_{\lambda,\mu}\subset X_\lambda\}$ with maximal fiber dimension
  strictly smaller than $n$, such that for any parameter $\lambda$
  there is a collection of $N=H^{O_X(k^{-1/n})}$ parameters
  $\{\mu_j\}$ satisfying
  \begin{equation}
    (\log X_\lambda)^\trans(\Q,H)\subset \bigcup_{j=1}^N \log Y_{\lambda,\mu_j}
  \end{equation}
  Each $Y_{\lambda,\mu}$ is defined by intersecting $X_\lambda$ with a
  collection of polynomial equations of degree $k$ in the variables
  $\log \vx_1,\ldots,\log \vx_\ell$.
\end{Prop}

Proposition~\ref{prop:log-set-pw} is reminiscent of the main inductive
step in the proof of the Pila-Wilkie theorem. In particular, fixing
$k$ such that $N=H^\e$ and using induction over the fiber dimension,
the proposition immediately yields a proof of the Pila-Wilkie theorem
for log-sets. On the other hand, setting $k=(\log H)^n$ yields an
interpolation result using a constant number of hypersurfaces of
polylogarithmic degree and can be seen as the first inductive step
toward the Wilkie conjecture. This is similar to the main result of
\cite{cpw:params} for sets definable in $\R_{\an}^{\text{pow}}$.

The proof of Proposition~\ref{prop:log-set-pw} is given
in~\secref{sec:log-set-pw-proof}. It relies heavily on the theory of
complex cells, and is effective in the following sense: \emph{a
  generalization of the polynomial complexity of the algebraic CPT to
  a reduct of $\R_\an$ which includes the \emph{restricted} logarithm
  would automatically yield a proof of the Wilkie conjecture for
  log-sets $\log A$ with $A$ definable in the reduct}. While we do not
prove such a version of the CPT in this paper (and there appear to be
some technical obstacles to doing so), this seems to offer a plausible
approach to proving the Wilkie conjecture for a large class of
unrestricted sets that are important in applications.

Even for the case that $k$ is a constant independent of $H$,
Proposition~\ref{prop:log-set-pw} yields additional information
compared to the standard proof of the Pila-Wilkie theorem. Namely, in
the standard proof the sets $Y_{\lambda,\mu}$ are also defined by
intersecting $X_\lambda$ with a collection of polynomial equations of
degree $k$ in the variables $\log \vx_1,\ldots,\log \vx_\ell$. In
general such sets would not be subanalytic. In our version certain
cancellations in the logarithmic terms allow one to find suitable
equations which are \emph{subanalytic} on $X_\lambda$, thus proving
that $Y_{\lambda,\mu}$ are also subanalytic. This eventually implies
that the part of $X_\lambda^\alg$ responsible for the presence of many
points of height $H$ is also a union of log-sets.


\section{Complex cellular structures}
\label{sec:complex-cells}

\subsection{Discs and annuli in the complex plane}

For $r\in\C$ (resp. $r_1,r_2\in\C$) with $|r|>0$
(resp. $|r_2|>|r_1|>0$) we denote
\begin{align}
  D(r)&:=\{|z|<|r|\} & D_\circ(r)&:=\{0<|z|<|r|\} \\
  A(r_1,r_2)&:=\{|r_1|<|z|<|r_2|\} & *&:=\{0\}.
\end{align}
We also set $S(r):=\partial D(r)$. For any $0<\delta<1$ we define
the $\delta$-extensions, denoted by superscript $\delta$, by
\begin{align}
  D^\delta(r)&:=D(\delta^{-1}r) & D^\delta_\circ(r)&:=D_\circ(\delta^{-1}r) \\
  A^\delta(r_1,r_2)&:=A(\delta r_1,\delta^{-1}r_2) & *^\delta&:=*.
\end{align}
We also set $S^\delta(r)=A(\delta r,\delta^{-1}r)$. The notion of
$\delta$-extension is naturally associated with the Euclidean geometry
of the complex plane. However, in many cases it is more convenient to
use a different normalization associated with the hyperbolic geometry
of our domains. For any $0<\rho<\infty$ we define the
$\hrho$-extension $\cF^\hrho$ of $\cF$ to be $\cF^\delta$ where
$\delta$ satisfies the equations
\begin{equation}
  \begin{aligned}
    \rho &= \frac{2\pi\delta}{1-\delta^2} && \text{for $\cF$ of type
      $D$,} \\
    \rho &= \frac{\pi^2}{2|\log\delta|} && \text{for $\cF$ of type
      $D_\circ,A$}.
  \end{aligned}
\end{equation}

The motivation for this notation comes from the following fact,
describing the hyperbolic-metric properties of a fiber $\cF$ within
its $\hrho$-extension. For a reminder on the hyperbolic metric
associated to a planar domain see~\secref{sec:fund-lemmas}.
Fact~\ref{fact:boundary-length} is proved by explicit computation
in~\secref{sec:explicit-consts}.

\begin{Fact}\label{fact:boundary-length}
  Let $\cF$ be a fiber of type $A,D,D_\circ$ and let $S$ be a component
  of the boundary of $\cF$ in $\cF^\hrho$. Then the length of $S$ in
  $\cF^\hrho$ is at most $\rho$.
\end{Fact}


\subsection{Complex cells}

\subsubsection{The general setting}

We introduce a notation that will be used throughout the paper. Let
$\cX,\cY$ be sets and $\cF:\cX\to2^\cY$ be a map taking points of
$\cX$ to subsets of $\cY$. Then we denote
\begin{equation}
  \cX\odot\cF := \{(x,y) : x\in\cX, y\in\cF(x)\}.
\end{equation}
In this paper $\cX$ will be taken to be a subset of $\C^n$ and $\cY$
will be $\C$. If $r:\cX\to\C\setminus\{0\}$ then for the purpose of
this notation we understand $D(r)$ to denote the map assigning to each
$x\in\cX$ the disc $D(r(x))$, and similarly for $D_\circ,A$.

We now introduce the central notion of this paper, namely the notion
of a complex cell of \emph{length} $\ell\in\Z_{\ge0}$ and \emph{type}
$\cT(\cC)\subset\{*,D,D_\circ,A\}^\ell$. If $U$ is a complex manifold
we denote by $\cO(U)$ the space of holomorphic functions on $U$, and
by $\cO_b(U)\subset\cO(U)$ the subspace of bounded functions.

\begin{Def}[Complex cells]\label{def:cells}
  To base our induction, a complex cell $\cC$ of \emph{length} zero is
  the singleton $\C^0$. The \emph{type} of $\cC$ is the empty word.  A
  complex cell of length $\ell+1$ has the form $\cC_{1..\ell}\odot\cF$
  where the \emph{base} $\cC_{1..\ell}$ is a cell of length $\ell$,
  and the \emph{fiber} $\cF$ is one of $*,D(r),D_\circ(r),A(r_1,r_2)$
  where $r\in\cO_b(\cC_{1..\ell})$ satisfies $|r(\vz_{1..\ell})|>0$
  for $\vz_{1..\ell}\in\cC_{1..\ell}$; and
  $r_1,r_2\in\cO_b(\cC_{1..\ell})$ satisfy
  $0<|r_1(\vz_{1..\ell})|<|r_2(\vz_{1..\ell})|$ for
  $\vz_{1..\ell}\in\cC_{1..\ell}$. The type $\cT(\cC)$ is
  $\cT(\cC_{1..\ell})$ followed by the type of the fiber.
\end{Def}
Next, we define the notion of a $\vdelta$-extension (resp.
$\hvrho$-extension) of a cell of length $\ell$ where
$\vdelta\in(0,1)^\ell$ (resp. $\vrho\in(0,\infty)^\ell$).
\begin{Def}
  The cell of length zero is defined to be its own
  $\vdelta$-extension. A cell $\cC$ of length $\ell+1$ admits a
  $\vdelta$-extension
  $\cC^\vdelta:=\cC_{1..\ell}^{\vdelta_{1..\ell}}\odot\cF^{\vdelta_{\ell+1}}$
  if $\cC_{1..\ell}$ admits a $\vdelta_{1..\ell}$-extension, and if
  the function $r$ (resp. $r_1,r_2$) involved in $\cF$ admits
  holomorphic continuation to $\cC_{1..\ell}^{\vdelta_{1..\ell}}$ and
  satisfies $|r(\vz_{1..\ell})|>0$
  (resp. $0<|r_1(\vz_{1..\ell})|<|r_2(\vz_{1..\ell})|$) in this larger
  domain. The $\hvrho$-extension $\cC^\hvrho$ is defined in an
  analogous manner.
\end{Def}

As a shorthand, when we speak of a complex cell $\cC^\vdelta$
(resp. $\cC^\hvrho$) we imply that $\cC$ is a complex cell admitting a
$\vdelta$ (resp $\hvrho$) extension. We will usually speak about
$\delta$-extensions where $\delta\in(0,1)$ by identifying $\delta$
with $\vdelta:=(\delta,\ldots,\delta)$ and similarly with
$\rho\in(0,\infty)$.

\begin{Rem}\label{rem:repeated-ext}
  It is sometimes convenient to consider repeated extensions. We
  denote
  $\cC^{\he{\rho_1}\he{\rho_2}}:=(\cC^{\he{\rho_1}})^{\he{\rho_2}}$. A
  simple computation shows that
  $\cC^{\he{\rho_1}\he{\rho_2}}\subset\cC^\rho$ where
  $\rho^{-1}=\poly(\rho_1^{-1},\rho_2^{-1})$.
\end{Rem}

The \emph{dimension} of a cell denoted $\dim\cC$ is its length minus
the number of $*$s in its type. This clearly agrees with the dimension
of $\cC$ as a complex manifold.


\subsubsection{The real setting}

We introduce the notions of a \emph{real} (complex) cell $\cC$, the
\emph{real part} of a real cell $\cC$, and a \emph{real} holomorphic
function on a complex cell. Below we let $\R_+$ denote the set of
positive real numbers.

\begin{Def}[Real structure on complex cells]
  The cell of length zero is real and equals its real part. A cell
  $\cC:=\cC_{1..\ell}\odot\cF$ is real if $\cC_{1..\ell}$ is real and
  the radii involved in $\cF$ can be chosen to be real on
  $\cC_{1..\ell}$; The real part $\R\cC$ (resp. positive real part
  $\R_+\cC$) of $\cC$ is defined to be $\cC\cap\R^\ell$
  (resp. $\cC\cap\R_+^\ell$); A holomorphic function on $\cC$ is said
  to be real if it is real on $\R\cC$.
\end{Def}
A simple induction shows that a real cell is invariant under the
conjugation $\vz\to\bar\vz$. Note that he positive real part of a real
cell is connected, while the real part is disconnected if $\cC$ has
fibers of type $D_\circ,A$. We remark that for a function $f:\cC\to\R$
to be real it is enough to require that it is real on
$\R_+\cC$. Indeed, in this case $f(z)=\overline{f(\bar z)}$ on
$\R_+\cC$ and it follows by holomorphicity the equality holds over
$\cC$.

\subsubsection{Algebraicity}

For $X\subset\C^\ell$ a pure-dimensional algebraic variety we define
the \emph{degree} $\deg X$ to be the number of intersections between
$X$ and a generic affine-linear hyperplane of complementary dimension
(this is the same as the degree of the projective closure of $X$ in
$\C P^\ell$). We extend this by linearity to arbitrary varieties. We
remind the reader that $\deg(X\times Y)=\deg X\cdot\deg Y$ and by the
Bezout theorem $\deg(X\cap Y)\le \deg X\cdot \deg Y$.

For a domain $U\subset\C^\ell$, we say that a holomorphic function
$f:U\to\C$ is \emph{algebraic} if its graph
$G_f\subset\C^\ell\times\C$ is an analytic component of
$(U\times\C)\cap X$, where $X\subset\C^\ell\times\C$ is an algebraic
variety. The minimal degree $\beta$ of a variety $X$ satisfying this
condition is called the \emph{degree}, or \emph{complexity}, of
$f$. For $F:U\to\C^k$ we say that $F$ is algebraic if each of its
components is, and we define its complexity to be the maximum among
the complexities of the components. As an easy consequence of the
Bezout theorem we have for any pair of composable algebraic maps $F,G$
the estimate
\begin{equation}
  \deg F\circ G \le \poly_{\ell,k}(\deg F,\deg G) .
\end{equation}

We define the notion of an \emph{algebraic} complex cell of complexity
$\beta$ by induction as follows: a cell of length $0$ is algebraic of
complexity $1$; A cell $\cC=\cC_{1..\ell}\odot\cF$ is algebraic if
$\cC_{1..\ell}$ is algebraic and the radii involved in $\cF$ are
algebraic, and the complexity of $\cC$ is the maximum among the
complexity of $\cC_{1..\ell}$ and the complexities of the radii defining
$\cF$.


\subsection{Cellular maps}

We equip the category of complex cells with \emph{cellular maps}
defined as follows.

\begin{Def}[Cellular map]\label{def:cell-maps}
  Let $\cC,\hat\cC$ be two cells of length $\ell$. We say that a
  holomorphic map $f:\cC\to\hat\cC$ is \emph{cellular} if it takes the
  form $\vw_{j}=\phi_j(\vz_{1..j})$ where $\phi_j\in\cO_b(\cC_{1..j})$
  for $j=1,\ldots,\ell$ and moreover $\phi_j$ is a monic polynomial in
  $\vz_j$. We say that a cellular map $f$ is \emph{prepared}
  (resp. \emph{a translate}) in $\vz_j$ if
  $\phi_j(\vz_{1..j})=\vz_j^q+\tilde\phi_j(\vz_{1..j-1})$ for some
  $q\in\N_{\ge1}$ (resp. $q=1$) and $\tilde\phi_j$ is holomorphic on
  $\cC_{1..j-1}$. We say that $f$ is \emph{prepared} (resp. a
  translate) if it is prepared (resp. a translate) in
  $\vz_1,\ldots,\vz_\ell$. We say that $f$ is \emph{real} if
  $\cC,\hat\cC$ are real and the components of $f$ are real.
\end{Def}

Cellular maps preserve the length and dimension of cells. The
composition of two cellular maps is cellular, but note that the
composition of two prepared cellular maps is not necessarily
prepared. We will often be interested in covering a cell by cellular
images of other cells. Toward this end we introduce the following
definition.

\begin{Def}\label{def:cell-cover}
  Let $\cC^\vdelta$ be a cell and
  $\{f_j:\cC_j^{\vdelta'}\to\cC^\vdelta\}$ be a finite collection of
  cellular maps. We say that this collection is a
  \emph{$(\vdelta',\vdelta)$-cellular cover} of $\cC$ if
  $\cC\subset\cup_j(f_j(\cC_j))$. If $(\vdelta',\vdelta)$ are clear
  from the context we will speak simply of cellular covers.
\end{Def}

The number of maps $f_j$ in a cellular cover is called the \emph{size}
of the cover. If the maps $f_j$ are all algebraic of complexity
$\beta$ we say that the cover has complexity $\beta$.

The real analog of a cellular cover is defined as follows.

\begin{Def}\label{def:real-cell-cover}
  Let $\cC^\vdelta$ be a real cell and
  $\{f_j:\cC_j^{\vdelta'}\to\cC^\vdelta\}$ be a finite collection of
  real cellular maps. We say that this collection is a \emph{real
    $(\vdelta',\vdelta)$-cellular cover} of $\cC$ if
  $\R_+\cC\subset\cup_j(f_j(\R_+\cC_j))$. If $(\vdelta',\vdelta)$ are
  clear from the context we will speak simply of real cellular covers.
\end{Def}

The restriction to positive real parts in the definition of real
cellular covers is a notational convenience. One can cover the
remaining components of $\R\cC$, for instance using the signed
covering maps introduced in~\secref{sec:nu-cover}.

\begin{Rem}
  We remark that if $\{f_j:\cC_j^{\vdelta'}\to\cC^{\vdelta}\}$ is a
  cellular cover of $\cC$ and
  $\{f_{jk}:\cC_{jk}^{\vdelta''}\to\cC_j^{\vdelta'}\}$ is a cellular cover
  of $\cC_j$ then $\{f_j\circ f_{jk}\}$ is a $(\vdelta'',\vdelta)$-cover
  of $\cC$. We will often use this basic principle without further
  reference.
\end{Rem}

The following theorem implies in particular, by the remark above, that
a cellular cover can always be replaced by a cellular cover consisting
of prepared maps.

\begin{Thm}[Cellular Preparation Theorem, CPrT]\label{thm:cprt}
  Let $f:\cC^\hrho\to\hat\cC$ be a (real) cellular map. Then there
  exists a (real) cellular cover $\{g_j:\cC_j^\hrho\to\cC^\hrho\}$ of
  size $\poly_f(\rho)$ such that each $f\circ g_j$ is prepared.

  If $\cC,\hat\cC,f$ vary in a definable family then the size of the
  cover is $\poly(\rho)$ uniformly over the family, and the maps $g_j$
  can be chosen from a single definable family.  If $\cC,\hat\cC,f$
  are algebraic of complexity $\beta$ then the cover has size
  $\poly(\beta,\rho)$ and complexity $\poly(\beta)$.
\end{Thm}

In this paper when we say that a family of sets or functions is
\emph{definable} we mean that it is definable in $\R_\an$. The reader
unfamiliar with this terminology can think instead of a family whose
total space is a bounded subanalytic set.

\subsection{The Cellular Parameterization Theorem}

Recall that in semialgebraic geometry, a cell is said to be
\emph{compatible} with a function if the function vanishes either
identically or nowhere on the cell. We introduce a complex analog
below.

\begin{Def}
  For $\cC$ a complex cell and $F\in\cO_b(\cC)$ we say that $F$ is
  \emph{compatible} with $\cC$ if $F$ vanishes either identically or
  nowhere on $\cC$. For $f:\hat\cC\to\cC$ a cellular map we say that
  $f$ is compatible with $F$ if $f^*F$ is compatible with $\hat\cC$.
\end{Def}

The following is our main result.

\begin{Thm}[Cellular Parameterization Theorem, CPT]\label{thm:cpt}
  Let $\rho,\sigma\in(0,\infty)$. Let $\cC^\hrho$ be a (real) cell and
  $F_1,\ldots,F_M\in\cO_b(\cC^\hrho)$ (real) holomorphic
  functions. Then there exists a (real) cellular cover
  $\{f_j:\cC^\hsigma_j\to\cC^\hrho\}$ of size
  $\poly_{\cC,\{F_j\}}(\rho,1/\sigma)$, such that each $f_j$ is
  prepared and compatible with each $F_k$.

  If $\cC,F_1,\ldots,F_M$ vary in a definable family then the cover
  has size $\poly(\rho,1/\sigma)$ uniformly over the family and the
  $f_j$ can be chosen from a single definable family. If
  $\cC,F_1,\ldots,F_M$ are algebraic of complexity $\beta$ then the
  cover has size $\poly_\ell(\beta,M,\rho,1/\sigma)$ and complexity
  $\poly(M,\beta)$.
\end{Thm}

In~\secref{sec:uniform-families} we show that the cellular structure
of the maps essentially implies automatic uniformity over families in
the statements of the CPT and CPrT. We state the family versions of
these theorems for convenience of use, to avoid having to make such
reductions on numerous occasions.


\subsection{Topology and hyperbolic geometry of complex cells}

A complex cell $\cC$ is homotopically equivalent to a product of
points (for fibers $*,D$) and circles (for fibers $D_\circ,A$). Thus
$\pi_1(\cC)\simeq\prod G_i$ where $G_i$ is trivial for $*,D$ and $\Z$
for $D_\circ,A$. We let $\gamma_i$ denote the generator of $G_i$
chosen with positive complex orientation for $G_i$ non-trivial and
$\gamma_i=e$ otherwise. 

\begin{Def}
  Let $f:\cC\to\C\setminus\{0\}$ be continuous. We define the
  monomial associated to $f$ to be $\vz^{\valpha(f)}$ where
  \begin{equation}
    \valpha_i(f) = f_*\gamma_i \in \Z\simeq \pi_1(\C\setminus\{0\}).
  \end{equation}
\end{Def}

It is easy to verify that $f\mapsto\valpha(f)$ is a group homomorphism
from the multiplicative group of continuous maps
$f:\cC\to\C\setminus\{0\}$ to the multiplicative group of monomials,
which sends each monomial to itself.

For any hyperbolic Riemann surface $X$ we denote by
$\dist(\cdot,\cdot;X)$ hyperbolic distance on $X$
(see~\secref{sec:fund-lemmas} for a reminder on this topic). We use
the same notation when $X=\C$ to denote the usual Euclidean distance,
and when $X=\C P^1$ to denote the Fubini-Study metric. For $x\in X$
and $r>0$ we denote by $B(x,r;X)$ the $r$-ball around $x$ in $X$. For
$A\subset X$ we denote by $B(A,r;X)$ the union of $r$-balls around all
points of $A$.

The following lemma shows that a holomorphic function with a
non-vanishing bounded $\hrho$-extension is equivalent to its
associated monomial up to a unit in a strong sense.

\begin{Lem}[Monomialization lemma]\label{lem:monomial}
  Let $0<\rho<\infty$ and let $f:\cC^\hrho\to\C\setminus\{0\}$ be a
  bounded holomorphic map. Then $f=\vz^{\valpha(f)}\cdot U(\vz)$ and
  \begin{align}\label{eq:monom-lem-analytic}
    \diam(\Re\log U(\cC);\R) &< O_f(1)\cdot\rho, & \diam(\Im\log U(\cC);\R) &< O_f(1).
  \end{align}
  If $\cC$ and $f$ vary in a definable family then $|\valpha(f)|$ and
  the asymptotic constants in~\eqref{eq:monom-lem-analytic} are
  uniformly bounded over the family.  If $\cC,f$ are algebraic of
  complexity $\beta$ then $|\valpha(f)|=\poly_\ell(\beta)$ and
  \begin{align}
    \diam(\Re\log U(\cC);\R) &< \poly_\ell(\beta)\cdot\rho, & \diam(\Im\log U(\cC);\R) &< \poly_\ell(\beta).
  \end{align}
\end{Lem}

In light of the monomialization lemma, the CPT can be viewed as a
cellular analog of the monomialization of functions/ideals in the
theory of resolution of singularities. Indeed, if $\cC^\hrho$ is a
cell compatible with $F$ then either $F$ vanishes identically or
$F:\cC^\delta\to\C\setminus\{0\}$, in which case $F$ is equivalent to
$\vz^{\valpha(f)}$ up to a unit on $\cC$; hence the cells constructed
in the CPT may be viewed as ``cellular charts'' where $F_1,\ldots,F_M$
are monomialized.

We now give several results on the geometry of holomorphic maps from
cells to hyperbolic Riemann surfaces. We begin with the following
\emph{domination lemma}, which is valid for arbitrary cellular
extensions. In this paper we never use this lemma directly: instead,
we use the finer \emph{fundamental lemmas} stated later, which are
valid only for extensions with a sufficiently small $\rho$. However we
still state and prove the domination lemma to stress another line of
close analogy between complex cells and resolution of singularities.

\begin{Lem}[Domination Lemma]
  Let $\cC^\hrho$ be a complex cell and suppose\footnote{Note that if
    $\cC$ admits a $\hrho$-extension it also admits a
    $\rho'$-extension for any $\rho'>\rho$, so the restriction on
    $\rho$ is only relevant for the asymptotics
    in~\eqref{eq:dom-lemma}.} that $\rho>2e$. Let
  $f:\cC^\hrho\to\C\setminus\{0,1\}$ be holomorphic. Then on $\cC$ one
  of the following holds:
  \begin{align}\label{eq:dom-lemma}
    |f| &= O_\ell(\log\log\rho), & |1/f| &= O_\ell(\log\log\rho).
  \end{align}
\end{Lem}

The domination lemma can be seen as a cellular analog of a standard
argument from the theory of resolution of singularities
\cite[Lemma~4.7]{bm:subanalytic}: \emph{suppose $f,g$ and $f-g$ are
  monomials (up to a unit). Then either $f$ divides $g$ or $g$ divides
  $f$}. This allows one to principalize an ideal by monomializing its
generators and their pairwise differences. To see the analogy with the
domination lemma, suppose $\cC^\hrho$ is a cell compatible with
$f,g,f-g$. Then $f/g:\cC^\hrho\to\C\setminus\{0,1\}$ and the
domination lemma applies to show that either $f/g$ or $g/f$ is bounded
from above (i.e. one divides the other in $\cO_b(\cC)$). Moreover, the
bound depends only on $\ell$ and $\rho$.

\begin{Rem}
  The domination lemma for one-dimensional discs immediately implies
  the Little Picard Theorem. Indeed, suppose
  $f:\C\to\C\setminus\{0,1\}$ is entire. Applying the domination lemma
  to $f\rest{D(r)}$ for every $r>0$ we see that $f$ is bounded away
  from $0$ or $\infty$ uniformly in $\C$ and therefore constant by
  Liouville's theorem.

  Similarly, the domination lemma for one-dimensional annuli implies
  the Great Picard Theorem. Indeed, suppose
  $f:D_\circ(1)\to\C\setminus\{0,1\}$ has an essential singularity at
  $0$.  Applying the domination lemma to $f\rest{A(\e,1/2)}$ for every
  $0<\e<1/2$ we see that $f$ is bounded away from $0$ or $\infty$
  uniformly in $D_\circ(1/2)$ which is impossible, for instance by the
  removable singularity theorem.
\end{Rem}

Next, we state the more refined \emph{fundamental lemmas} on maps
$f:\cC^\hrho\to X$ into a hyperbolic Riemann surface $X$. By contrast
with the domination lemma, these lemma yield more precise asymptotic
estimates on the image of a complex cell as the extension $\rho$ tends
to zero. The proofs of the fundamental lemmas rely on the (hyperbolic)
geometry and topology of $X$. Rather than formulate the most general
possible form, we prefer to give three separate formulations for the
most useful cases $X=\D,\D\setminus\{0\},\C\setminus\{0,1\}$.

\begin{Lem*}[Fundamental Lemma for $\D$]
  Let $\cC^\hrho$ be a complex cell. Let $f:\cC^\hrho\to\D$ be
  holomorphic. Then
  \begin{equation}
    \diam(f(\cC);\D)=O_\ell(\rho).
  \end{equation}
\end{Lem*}

\begin{Lem*}[Fundamental Lemma for $\D\setminus\{0\}$]
  Let $\cC^\hrho$ be a complex cell. Let
  $f:\cC^\hrho\to\D\setminus\{0\}$ be holomorphic. Then one of the
  following holds:
  \begin{align}
    f(\cC)&\subset B(0,e^{-\Omega_\ell(1/\rho)};\C), & \diam(f(\cC);\D\setminus\{0\})&=O_\ell(\rho).
  \end{align}
  In particular, one of the following holds:
  \begin{align}
    \log |f(\cC)| &\subset (-\infty,-\Omega_\ell(1/\rho)), & \diam(\log|\log|f(\cC)||;\R)&=O_\ell(\rho).
  \end{align}
\end{Lem*}

\begin{Lem*}[Fundamental Lemma for $\C\setminus\{0,1\}$]
  Let $\cC^\hrho$ be a complex cell and $0<\rho<1$. Let
  $f:\cC^\hrho\to\C\setminus\{0,1\}$ be holomorphic. Then one
  of the following holds:
  \begin{align}
    f(\cC)&\subset B(\{0,1,\infty\},e^{-\Omega_\ell(1/\rho)};\C P^1), & \diam(f(\cC);\C\setminus\{0,1\})&=O_\ell(\rho).
  \end{align}
\end{Lem*}

The proofs of the monomialization, domination and fundamental lemmas
are given in~\secref{sec:fund-lemmas}. 


\subsection{Overview of the proof of the CPT}

In this section we aim to give an intuitive overview of the proof of
the CPT. We will consider only the algebraic case, which requires the
most delicate arguments to control the complexity with respect to
$\beta$. We will construct a covering using general cellular maps
instead of prepared maps (one can later use the CPrT to obtain a
prepared cover).

\subsubsection{Reduction to a single $F$}

We first note that the CPT can be easily reduced to the case of one
function $F$. Indeed, suppose we wish to perform a cellular
decomposition for the functions $F_1,\ldots,F_M$. Let $F$ be the
product of all the $F_j$, except those that vanish identically on
$\cC$. We construct a cellular decomposition
$f_j:\cC_j^\hsigma\to\cC^\hrho$ compatible with $F$. Each of the maps
$f_j$ whose image is disjoint from the zeros of $F$ is already
compatible with each $F_k$. Each of the remaining cells $\cC_j$ has
dimension strictly smaller than $\cC$ since cellular maps preserve
dimension. Thus we can proceed by induction over the dimension.

\subsubsection{Reduction to arbitrary $\rho,\sigma$}

We will allow ourselves to assume that $\rho$ is as small as we wish
as long as $1/\rho=\poly_\ell(\beta)$. We will also allow ourselves to
assume that $\sigma$ is as large as we wish as long as
$\sigma=\poly_\ell(\beta)$. Both of these assumptions are justified by
Theorem~\ref{thm:cell-refinement} which shows that cells with a given
extension can be ``refined'' into cells with a finer extension. This
theorem generalizes the idea of covering a disc or annulus by several
smaller discs or annuli. The proof of this theorem is independent of
the CPT and the reader can safely skip the details for now.


\subsubsection{Reduction to a proper covering map}
We proceed with the case of a single $F\in\C[\vz_{1..\ell+1}]$. Let
$\cC:=\cC_{1..\ell}\odot\cF$. We begin the proof by induction over
$\ell$. Let $D\in\C[\vz_{1..\ell}]$ denote the discriminant of $F$ and
construct a cellular decomposition
$f_j:\cC_j^\hrho\to\cC_{1..\ell}^\hrho$ compatible with $D$. Any of the cells $f_j$ with
$f_j(\cC_j)$ contained in the zeros of $D$ has dimension strictly
smaller than $\cC_{1..\ell-1}$, and we can therefore handle the cell
$\cC_j\odot(f_j^*\cF)$ with the function $f_j^*F$ by induction over
dimension. The images of these cells under $(f_j,\id)$ cover the part
of $\cC$ lying over the zeros of $D$.

It remains to cover the part lying over the images $f_j(\cC_j)$ that
are disjoint from the zeros of $D$. For each such cell we can reduce
again to the cell $\cC_j\odot(f_j^*\cF)$ and the function $f_j^*F$. We
return to the original notation, replacing $\cC\odot\cF,F$ by
$\cC_j\odot(f_j^*\cF),f_j^*F$. What we have gained is that the
projection
\begin{equation}
  \pi: (\cC\odot\cF)^{\hrho}\cap\{F=0\}\to\cC, \qquad \pi(\vz_{1..\ell+1})=\vz_{1..\ell}
\end{equation}
is now a proper covering map. We denote the degree of $\pi$ by $\nu$,
and note that $\nu=\poly_\ell(\beta)$.

\subsubsection{Covering the zeros}

The projection $\pi$ admits $\nu$ possibly multivalued sections
$y_j:\cC\to\C$. If we set $\hat\cC:=\cC_{\times\nu!}$ then $y_j$ can
be lifted to a univalued section $y_j:\hat\cC\to\C$. However, the
covering of order $\nu!$ has complexity exponential in $\nu$ and
therefore in $\beta$, and thus cannot be used explicitly in our
construction.  In fact, one can show that $y_j$ already lifts to a
univalued function on $y_j:\cC_{\times\nu_j}\to\C$ where $\nu_j\le\nu$
is the size of the $\pi_1(\cC)$-orbit of $y_j$, see
Lemma~\ref{lem:y_j-monodromy}. This is a simple statement about
permutation groups, using crucially the fact that $\pi_1(\cC)$ is
abelian. We denote $\hat\cC_j:=\cC_{\times\nu_j}$.

Using the maps $y_j$ we construct cellular maps
$(R_{\nu_j},\id):\hat\cC_j\odot*\to\cC\odot\cF$ whose images are
contained in the zeros of $F$ and hence compatible with $F$. Moreover
Proposition~\ref{prop:ext-v-cover} shows that these maps admit
$\he{\rho\cdot\nu_j}$-extensions, and choosing
$\rho\cdot\nu_j<\sigma$ gives the required extension property.

\subsubsection{Covering the complement of the zeros}

We now make a simplifying assumption, namely we replace $\cF$ by
$\C$. This simplifies the presentation by allowing us to avoid minor
technicalities around $\partial\cF$. Since $\C$ is unbounded we will
formally have to allow unbounded cells in the covering that we
construct, but the difference will be minor. The goal is therefore to
cover each fiber $\C\setminus\{y_1,\ldots,y_\nu\}$ by a finite
collection of discs, punctured discs and annuli admitting
$\hsigma$-extensions. The challenge is that this covering should
depend holomorphically on the base point $\vz_{1..\ell}\in\cC$, and
should be of size $\poly_\ell(\beta)$.

\subsubsection{The affine invariants $s_{i,j,k}$}

To achieve our goal we will study the relative positions of the points
$y_j\in\C$. Toward this end we introduce the following notation. Let
$y_i,y_j,y_k$ be three distinct sections and denote
$\hat\cC_{i,j,k}:=\cC_{\times\nu_i\nu_j\nu_k}$. We define a map
$s_{i,j,k}$ as follows,
\begin{equation}
  s_{i,j,k}:\hat\cC_{i,j,k}\to\C\setminus\{0,1\}, \qquad s_{i,j,k}=\frac{y_i-y_j}{y_i-y_k}.
\end{equation}
By the fundamental lemma for maps into $\C\setminus\{0,1\}$ one of the
following holds:
\begin{equation}\label{eq:intro-sijk-fund}
  \begin{gathered}
    s_{i,j,k}(\hat\cC_{i,j,k})\subset B(\{0,1,\infty\},e^{-\Omega_\ell(1/(\nu^3\rho))};\C P^1), \\
    \diam(s_{i,j,k}(\hat\cC_{i,j,k});\C\setminus\{0,1\})=O_\ell(\nu^3\rho).
  \end{gathered}
\end{equation}
We remark that equivalently~\eqref{eq:intro-sijk-fund} holds if we
replace $\hat\cC_{i,j,k}$ by $\hat\cC$, since $s_{i,j,k}$ on $\hat\cC$
factors through $\hat\cC_{i,j,k}$. We note that $s_{i,j,k}$ is
invariant under affine transformations of $\C$, and is in fact the
only affine invariant of $(y_i,y_j,y_k)$. We think
of~\eqref{eq:intro-sijk-fund} as implying that the relative positions
of any three sections move very little for $\rho\ll1$. This is the
crucial ingredient from hyperbolic geometry that will enable us to
carry out the decomposition into discs and annuli uniformly over the
base $\hat\cC_{i,j,k}$.

\begin{Rem}[Fulton-MacPherson compactification]
  In \cite{fm:compact} Fulton and MacPherson construct a
  compactification $X[\nu]$ for the configuration space
  $X^\nu\setminus\Delta$ of $\nu$ distinct labeled points in an
  algebraic variety $X$ (where $\Delta$ is the union of all
  diagonals). In the case $X=\C$, the construction using
  \emph{screens} is as follows. Write $y_1,\ldots,y_\nu$ for the
  coordinates on $\C^\nu\setminus\Delta$. For every subset
  $S\subset\{1,\ldots,\nu\}$ define the map
  \begin{equation}
    \iota_S:\C^\nu\setminus\Delta\to\P(\C^{|S|}/\C), \qquad i_S(y_1,\ldots,y_\nu)=(y_\alpha)_{\alpha\in S}
  \end{equation}
  where $\C\subset\C^{|S|}$ is the diagonally embedded subspace. The
  map $\iota_S$ remembers the relative positions of
  $\{y_\alpha\}_{\alpha\in S}$ up to translation (corresponding to the
  diagonally embedded $\C$) and homothety (corresponding to the
  projectivization). Consider the map
  \begin{equation}\label{eq:iota-def}
    \iota:\C^\nu\setminus\Delta\to(\P^1)^{\binom\nu3}, \qquad \iota = \prod_{|S|=3} \iota_S.
  \end{equation}
  Then $\C[\nu]$ is the Zariski closure of the graph of $\iota$. It is
  an important fact that only the screens $\iota_S$ for $S$ of size
  three are required to fully determine the compactification: if one
  defines $\iota$ by a product over all $S\subset\{1,\ldots,\nu\}$ one
  obtains an isomorphic compactification.

  In our setting, we have a natural map
  $\vy=(y_1,\ldots,y_\nu):\hat\cC\to\C^\nu\setminus\Delta$. For an
  appropriate choice of the a rational coordinate on each $\P^1$
  factor in~\eqref{eq:iota-def}, the coordinates of $\iota$ can be
  identified with our maps $s_{i,j,k}$ which are indeed invariant
  under translation and homothety. The
  equations~\eqref{eq:intro-sijk-fund} imply in particular that the
  image $\vy(\hat\cC)$ is of small diameter, not only in the metric
  induced from $\C^\nu$, but also in the finer metric induced from the
  compactification $\C[\nu]$. It is crucial here that one only needs to
  use the screens $S$ of size three. If we were to write equations
  similar to~\eqref{eq:intro-sijk-fund} for screens of arbitrary size
  the factor $\nu^3$ would be replaced by $\nu!$, which is exponential
  in $\beta$ and hence too large for our purposes.

  As we proceed to describe the cellular cover for
  $\C\setminus\{y_1,\ldots,y_\nu\}$ we will make no explicit reference
  to the Fulton-MacPherson compactification, working instead directly
  with~\eqref{eq:intro-sijk-fund}. However the reader may find it
  helpful to interpret these inequalities as estimates on the metric
  induced from $\C[\nu]$.
\end{Rem}

\subsubsection{Clustering around a center $y_i$}

We fix a section $y_i$. We will cluster the remaining sections into
annuli based on their relative distance from $y_i$. Since our
constructions are invariant under affine transformations we may assume
after an appropriate transformation that $y_i=0$.

Pick an arbitrary base point $p\in\hat\cC$. We will say that $y_j$ is
close to $y_k$ if they satisfy $\log|y_k(p)/y_j(p)|<1/\nu$, and define
the \emph{clusters} around $y_i$ to be the equivalence classes of the
transitive closure of this relation. Then for any $y_j,y_k$, if they
are in the same cluster we have $\log|y_k(p)/y_j(p)|<1$ and otherwise
we have $\log|y_k(p)/y_j(p)|\ge1/\nu$.
  
\subsubsection{The Voronoi cells associated to $y_i$}

We will use the clusters around $y_i$ to construct a collection of
\emph{Voronoi cells} mapping into $\cC\odot\C$. Their defining
property, which motivates our choice of naming, is the following: the
Voronoi cells associated to $y_i$ cover every point
$z\in\C\setminus\{y_1,\ldots,y_\nu\}$ such that
$\dist(z,y_i)\le 2\dist(z,y_j)$ for any $j\neq i$. In other words, for
every section $y_i$ we will construct cells that cover every point in
$\C$ except those that are (twice) closer to some other section
$y_j$. Clearly then the union of these cells for every $y_i$ will
cover $\C\setminus\{y_1,\ldots,y_\nu\}$, which is our goal.

We begin by covering the empty area between two clusters. This is
fairly straightforward. Suppose that $y_j$ is a section with
$|y_j(p)|$ maximal within its cluster, and $y_k$ is a section with
$|y_k(p)|$ minimal within the next cluster. Then the annulus
$A(y_je^\e,y_ke^{-\e})$ for say $\e=1/\nu^2$ does not meet any of the
sections $y_j$ over $p$, and assuming $\rho$ is sufficiently small
this remains true uniformly over $\hat\cC$
by~\eqref{eq:intro-sijk-fund}. We similarly cover the area between
$y_i=0$ and the first cluster by a punctured disc, and outside the
last cluster by $A(\cdot,\infty)$.

\begin{figure}
  \centering
  \includegraphics[width=\textwidth]{CPTIntro268.pdf}
  \caption{Clustering around $y_i$; distances are in $\log$-scale.}
  \label{fig:clusters-yi}
\end{figure}

  
\subsubsection{Voronoi cells in a cluster}

It remains to cover the part of $\C$ that lives outside the annuli
above, i.e. near one of the clusters. Choose one of the clusters and
let $y_j$ be a section in the cluster. Since our constructions are
invariant under affine transformations we may assume after an
appropriate transformation that $y_i=0$ and also $y_j=1$. With this
choice of coordinates $s_{i,k,j}=y_k$ and~\eqref{eq:intro-sijk-fund}
implies that $y_k$ does not move much over the cell $\hat\cC$. We will
construct our cells over the base
$\hat\cC_{i,j}:=\cC_{\times\nu_i\nu_j}$. Note that the complexity of
this cell is $\poly_\ell(\beta)$ as required.

Suppose $y_{j_1}$ (resp. $y_{j_2}$) is the section with
$r_1:=|y_{j_1}(p)|$ minimal (resp. $r_2:=|y_{j_2}(p)|$ maximal) within
its cluster. Then $1/e<r_1<r_2<e$. Moreover the sections in the
cluster uniformly remain within $\cA:=A(r_1e^{-\e},r_2e^\e)$ while the
sections belonging to other clusters remain outside
$\cA^{\Theta(1-1/\nu)}$.

Let $U$ be the set obtained from $\cA$ by removing discs of radius
$1/10$ around each of the points $y_k(p)$ for $y_k$ belonging to the
cluster. Note that any point in these discs is much closer to $y_k(p)$
than to $y_i(p)=0$, so we not need to cover them with the Voronoi
cells of $y_i$. Moreover, the same remains true uniformly over
$\hat\cC$ since by~\eqref{eq:intro-sijk-fund} the points $y_k$ move
very little. To finish the construction we must therefore cover $U$
using discs whose $\hsigma$-extensions remain inside
$\cA^{\Theta(1-1/\nu)}$ and away from the $1/20$-balls around the
points $y_k(p)$ (this way even as $p$ varies over $\hat\cC$ the discs
and their extensions will not meet them). This is clearly possible to
achieve with $\poly_\ell(\nu)=\poly_\ell(\beta)$ discs as required.

\begin{figure}
  \centering
  \includegraphics[width=0.6\textwidth]{CPTIntro269.pdf}
  \caption{Voronoi cells in a cluster}
  \label{fig:clusters-voronoi}
\end{figure}


\section{Generalities on complex cells}

In this section we cover some generalities on complex cells, their
subanalytic structure and some uniformity results for families.

\subsection{The $\nu$-cover of a cell}
\label{sec:nu-cover}

A key difference between the classical notion of real cellular
decompositions and the complex counterpart is the presence of a
non-trivial fundamental group, and with it the existence of
non-trivial covering maps. Recall that for a cell $\cC$ of length
$\ell$ we identify $\pi_1(\cC)\simeq\prod G_i$ where $G_i$ is trivial
for $*,D$ and $\Z$ for $D_\circ,A$.

\begin{Def}[The $\vnu$-cover of a cell]\label{def:nu-cover}
  Let $\cC$ be a cell of length $\ell$ and let
  $\vnu=(\vnu_1,\ldots,\vnu_\ell)\in\pi_1(\cC)$ be such that
  $\vnu_j\vert\vnu_k$ whenever $j>k$ and $G_j=G_k=\Z$. We define the
  \emph{$\vnu$-cover} $\cC_{\times\vnu}$ of $\cC$ and the associated
  \emph{cellular map} $R_\vnu:\cC_{\times\vnu}\to\cC$ by induction on
  $\ell$. For $\ell=0$ we let $\cC_{\times\vnu}:=\cC$ and
  $R_\vnu:=\id$. For $\cC=\cC_{1..\ell-1}\odot\cF$ we let
  \begin{equation}
    \cC_{\times\vnu_{\ell}}:=(\cC_{1..\ell-1})_{\times\vnu_{1..\ell-1}}
    \odot (R_\vnu^*\cF_{\times\vnu_{\ell}})
  \end{equation}
  where $D(r)_{\times\vnu_\ell}:=D(r^{1/{\vnu_\ell}})$ and similarly for $D_\circ,A$.
  We set $R_\vnu(\vz_{1..\ell})_k:=\vz^\vnu$.
\end{Def}

Note that the $D_\circ(r)_{\times\vnu_\ell}$ fiber (and similarly with
$A$) does not conform, a-priori, with the definition of a complex cell
since $r^{1/\vnu_\ell}$ may in general be multivalued (with cyclic
monodromy of order dividing $\vnu_\ell$). However the divisibility
conditions on $\vnu$ guarantee that
$R_\vnu:(\cC_{1..\ell-1})_{\times\vnu_{1..\ell-1}}\to\cC_{1..\ell-1}$
maps $\pi_1(\cC_{1..\ell-1})_{\times\vnu}$ into
$\vnu_\ell\pi_1(\cC_{1..\ell-1})$ and the pullback
$R_{\vnu_\ell}^*(D_\circ(r))_{\times\vnu_\ell}$ is indeed univalued and well
defined up to a root of unity. We will usually consider the
$\nu$-cover with $\nu\in\N$, meaning that we take $\vnu$ with $\vnu_i=\nu$
when $G_i=\Z$ and $\vnu_i=1$ otherwise.

A minor technicality arises in the real setting. Namely, the real part
$\R\cC_{\times\vnu}$ does always not cover $\R\cC$: if $\vnu$ is even
then only the positive real part is covered. To cover the remaining
components of the real part we introduce the notion of a \emph{signed}
cover. Namely, for a sequence $\vsigma\in\{\pm1\}^\ell$ we define
$\cC_{\times\vnu,\vsigma}$ and $R_{\vnu,\vsigma}$ by induction as above
but taking
$R_{\vnu,\vsigma}(\vz_{1..\ell+1})_k:=\vsigma_k\cdot\vz_{k}^\vnu$. It is
then clear that $\R\cC_{\times\vnu,\vsigma}$ cover $\R\cC$ when
$\vsigma$ ranges over all possible signs.

The pullback to a $\vnu$-cover will be used in our treatment to
resolve the ramification of multivalued cellular maps. We record a
simple proposition concerning the interaction between extensions and
$\nu$-covers. 

\begin{Prop}\label{prop:ext-v-cover}
  Let $\cC$ be a complex cell and $\nu\in\N$.
  \begin{enumerate}
  \item If $\cC$ admits a $\delta$-extension then $\cC_{\times\nu}$
    admits a $\delta^{1/\nu}$-extension and the covering map $R_\nu$
    extends to
    $R_\nu:(\cC_{\times\nu})^{\delta^{1/\nu}}\to\cC^\delta$.
  \item If $\cC$ admits a $\hrho$-extension then $\cC_{\times\nu}$
    admits an $\he{\nu\rho}$-extension and the covering map $R_\nu$
    extends to $R_\nu:(\cC_{\times\nu})^{\he{\nu\rho}}\to\cC^\hrho$.
  \end{enumerate}
  If $\cC$ is algebraic of complexity $\beta$ then $C_{\times\nu}$ is
  algebraic of complexity $\poly_\ell(\beta,\nu)$.
\end{Prop}

We leave the simple inductive proof to the reader. We remark that for
the part 2 it is crucial that covering maps are defined to be
identical on fibers of type $D$.

\subsection{Uniformity in families}
\label{sec:uniform-families}

In this section we show how the cellular structure of our maps implies
automatic uniformity over families, for instance in the statement of
the CPT. 

For $p\in\cC_{1..j}$ we denote
\begin{equation}
  \cC_p = \{\vz_{j+1..\ell} : (p,\vz_{j+1..\ell})\in\cC\}.
\end{equation}
If $\cC$ has type $\cF_1\odot\cdots\odot\cF_\ell$ then $\cC_p$ is a
cell of type $\cF_{j+1}\odot\cdots\odot\cF_\ell$. If $\cC$ admits a
$\delta$-extension then so does $\cC_p$.

By definition cellular map $f:\cC\to\hat\cC$ induces a cellular map
$f_{1..j}:\cC_{1..j}\to\hat\cC_{1..j}$ for $j=1,\ldots,\ell$ by
restriction to the first $j$ coordinates. The following proposition
follows directly from the definitions.

\begin{Prop}\label{prop:map-fiber-bound}
  Let $f:\cC\to\hat\cC$ be a cellular map between cells of length
  $\ell$ and $j=1,\ldots,\ell$. Then the number of points in
  $f_{1..j}^{-1}(p)$ for $p\in\hat\cC_{1..j}$ is bounded by a constant
  $\nu(f,j)$ independent of $p$. More explicitly, one may take
  \begin{equation}
    v(f,j) = \prod_{k=1}^j \deg_{\vz_k}\phi_k
  \end{equation}
  in the notations of Definition~\ref{def:cell-maps}.
\end{Prop}

The following remark illustrates how the cellular structure of the
maps in the CPT automatically implies uniformity over families.

\begin{Rem}\label{rem:uniform-cpt}
  Suppose $\ell=n+m$ and we view $\cC$ in the statement of the CPT as
  a family of $m$-dimensional cells $\cC_p$ parameterized over
  $p\in\cC_{1..n}$. Let $f_j:\cC_j^\hsigma\to\cC^\hrho$ be the maps
  constructed in the CPT. By Proposition~\ref{prop:map-fiber-bound}
  the sets
  \begin{equation}
    P_j = \{p_{j,k}\} := (f_j)^{-1}(p)
  \end{equation}
  are finite with the number of points uniformly bounded over $p$. For
  each $p_{j,k}$ restriction of $f_j$ to the fiber gives a prepared
  cellular map $f_{j,k}:(\cC_j)^\hsigma_{p_{j,k}}\to\cC^\hrho_p$ which
  is compatible with the restrictions of $F_1,\ldots,F_m$ to $\cC_p$,
  and such that $f_{j,k}((\cC_j)_{p_{j,k}})$ cover $\cC_p$. In other words
  we obtain a cellular decomposition as in the CPT for the fibers
  $\cC_p$ with the number of cells uniformly bounded over $p$.
\end{Rem}


\subsection{Laurent expansion in a complex cell}

Let $\cC=\cF_1\odot\dots\odot\cF_\ell$ be a complex cell. In this
section we show that any holomorphic function on $\cC$ can be expanded
into a series analogous to the classical Laurent series. We begin by
introducing the notion of \emph{normalized monomials}. Let
$\valpha\in\Z^\ell$. We define the $\cC$-normalized monomial
$\vz^{[\valpha]}:=\vz_1^{[\valpha_1]}\cdots\vz_\ell^{[\valpha_\ell]}$
where: if $\cF_j=D(r_j)$ or $\cF_j=D_\circ(r_j)$ then
\begin{equation}
  \vz_j^{[\valpha_j]} = \
  \begin{cases}
    (\vz_j/r_j)^{\valpha_j} & \valpha_j\ge 0 \\
    0 & \text{otherwise};
  \end{cases}
\end{equation}
if $\cF_j=A(r_{1,j},r_{2,j})$ then
\begin{equation}
  \vz_j^{[\valpha_j]} = \
  \begin{cases}
    (\vz_j/r_{1,j})^{\valpha_j} & \valpha_j\ge 0 \\
    (\vz_j/r_{2,j})^{\valpha_j} & \text{otherwise};
  \end{cases}
\end{equation}
and if $\cF_j=*$ then
\begin{equation}
  \vz_j^{[\valpha_j]} = \
  \begin{cases}
    1 & \valpha_j=0 \\
    0 & \text{otherwise}.
  \end{cases}
\end{equation}
The normalization is such that $\vz^{[\valpha]}$ is bounded by $1$ in
$\cC$ and achieves the bound only when $\valpha=0$.

Below we write $\pos\vdelta$ for the vector whose $i$-th coordinate is
$|\vdelta_i|$.
\begin{Prop}\label{prop:norm-Laurent}
  Let $f\in\cO_b(\cC)$. Then $f$ has a series expansion, absolutely
  convergent on compacts in $\cC$, as follows:
  \begin{equation}\label{eq:norm-Laurent}
    f(\vz) = \sum_{\valpha\in\Z^\ell} c_\valpha \vz^{[\valpha]}
  \end{equation}
  where $|c_\valpha|\le\norm{f}$. If $\cC$ admits a $\vdelta$-extension
  and $f\in\cO_b(\cC^\vdelta)$ then also
  $|c_\valpha|<\vdelta^{\pos\valpha} \norm{f}_{\cC^\delta}$.
\end{Prop}
\begin{proof}
  We proceed by induction on $\ell$. The case $\cF_\ell=*$ reduces
  immediately to the claim for $\cC_{1..\ell-1}$. Otherwise the
  standard formula for Laurent expansions in $\cF_\ell$ gives
  \begin{equation}
    f(\vz) = \sum_{\valpha_\ell=-\infty}^\infty c_{\valpha_\ell}(\vz_{1..\ell-1}) \vz_\ell^{[\valpha_\ell]}
  \end{equation}
  where
  \begin{equation}
    c_{\valpha_\ell}(\vz_{1..\ell-1}) = \frac{\vz_\ell^{\valpha_\ell}}{\vz_\ell^{[\valpha_\ell]}}
    (2\pi i)^{-1} \oint \frac{f(\vz_{1..\ell-1},\zeta)}{\zeta^{\valpha_\ell+1}}\d\zeta
  \end{equation}
  and the integral is over a simple positively oriented curve in
  $\cF_\ell$. The Cauchy estimate now implies
  $\norm{c_{\valpha_\ell}}_{\cC_{1.\ell-1}}\le\norm{f}$ and we proceed
  to expand $c_{\valpha_\ell}$ by induction on $\ell$ to obtain a
  series~\eqref{eq:norm-Laurent} with $|c_\valpha|<\norm{f}$. To see
  that this multivariate series is absolutely convergent on compacts
  note that for every $\vz\in\cC$ and $\valpha\in\Z^\ell$ we have
  \begin{equation}
     |\vz^{[\valpha]}|\le\rho^{|\valpha|}\quad \text{where}\quad \rho:= \max_{\vsigma\in\{-1,1\}^\ell} \vz^{[\vsigma]} < 1
  \end{equation}
  and~\eqref{eq:norm-Laurent} is thus majorated by
  $\sum_{\valpha}(\rho+\e)^{|\valpha|}$ in a neighborhood of
  $\vz$. For the final claim we write a Laurent expansion in
  $\cC^\vdelta$ and rewrite the $\cC^\vdelta$-normalized monomials as
  $\cC$-normalized monomials, gaining an extra factor of
  $\vdelta^{\pos\valpha}$.
\end{proof}

Let $\cP_\ell:=D(1)^{\times\ell}$ denote the standard unit
polydisc. We write simply $\cP$ when $\ell$ is clear from the context.

\begin{Cor}\label{cor:laurent-disc-decmp}
  Let $f\in\cO_b(\cC)$. Then there is a decomposition
  \begin{equation}
    f(\vz) = \sum_{\vsigma\in\{-1,1\}^\ell} f_\vsigma(\vz^{[\vsigma]})
  \end{equation}
  where $f_\vsigma\in\cO(\cP)$. If $\cC$ admits a $\delta$-extension
  and $f\in\cO_b(\cC^\delta)$ then $f_\sigma\in\cO(\cP^\delta)$, and
  moreover for any $\delta<\e<1$ we have
  \begin{equation}
    \norm{f_\vsigma}_{\cP^\e}\le(1-\delta/\e)^{-\ell}\norm{f}_{\cC^\delta}.
  \end{equation}
\end{Cor}
\begin{proof}
  For the first statement we simply collect all summands with index
  $\valpha$ of sign $\vsigma$ in~\eqref{eq:norm-Laurent} into
  $f_\vsigma$ (if $\valpha_j=0$ we arbitrarily treat it as positive
  for this purpose). For the second part, we have
  \begin{equation}
    \norm{f_{1,\ldots,1}(\vw)}_{\cP^\e}  \le \sum_{\valpha\in(\Z_{\ge0})^\ell} |c_\valpha| \norm{\vw^\valpha}_{\cP^\e}\le
    \sum_{\valpha\in(\Z_{\ge0})^\ell} (\delta/\e)^{|\valpha|}=(1-\delta/\e)^{-\ell},
  \end{equation}
  and similarly for the other choices of the signs.
\end{proof}


\subsection{Subanalyticity}

We say that a function is subanalytic on a domain $U\subset\C^n$ if it
is defined there and its graph over $U$ forms a subanalytic set. Our
goal in this section is to prove the following proposition.

\begin{Prop}\label{prop:cell-subanalytic}
  Let $0<\delta<1$. A complex cell $\cC$ admitting a
  $\delta$-extension is a subanalytic set. If $f\in\cO_b(\cC^\delta)$
  then $f$ is subanalytic on $\cC$.
\end{Prop}

\begin{proof}
  We prove both claims by induction on the length of $\cC$, the case
  of length zero being vacuous. Let $\cC=\cC_{1..\ell}\odot\cF$.  Then
  the radii $r(z)$ or $r_1(z),r_2(z)$ of $\cF$ are subanalytic by
  induction and it easily follows that $\cC$ is subanalytic. Now let
  $f\in\cO_b(\cC^\delta)$. By Corollary~\ref{cor:laurent-disc-decmp}
  we may write $f$ as a sum of $2^{\ell+1}$ summands
  $f_\vsigma(\vz^{[\vsigma]})$. In the definition of $\vz^{[\vsigma]}$
  each of the radii involved are subanalytic on $\cC_{1..\ell}$ by
  induction, and each of the divisions involved are ``restricted'' in
  the sense of \cite{dvdd:subanalytic}, i.e. we always have
  $|\vz^{[\vsigma]}|\le 1$ for $\vz\in\cC$. It then follows, for
  instance using the subanalytic language of \cite{dvdd:subanalytic},
  that the graph of each $f_\vsigma$ and hence of $f$ is indeed
  subanalytic over $\cC$.
\end{proof}


\section{Semialgebraic and subanalytic sets}

In our terminology, a cell in an o-minimal structure (below
\emph{classical cell}) is the $\odot$-product of a sequence of
intervals (either closed or open on each side) with the two endpoints
given by continuous definable functions. The image $f(\R_+\cC)$ of a
real prepared cellular map is itself a cell in this classical
sense. More explicitly if
$f(\vz_{1..\ell})_j=\vz_j^{q_j}+\phi_j(\vz_{1..j-1})$ and
$\cC=\cF_1\odot\cdots\odot\cF_\ell$ then
\begin{equation}
  f(\R_+\cC) = \cI_1\odot\cdots\odot\cI_\ell, \quad I_j :=
  \begin{cases}
    \{\phi_j\} & \cF_j = * \\
    (\phi_j,\phi_j+|r^{q_j}|) & \cF_j = D(r),D_\circ(r) \\
    (\phi_j+|r_1^{q_j}|,\phi_j+|r_2^{q_j}|) & \cF_j = A(r_1,r_2).
  \end{cases}
\end{equation}
Furthermore, we note that $f$ restricts to a real-analytic
diffeomorphism from $\R_+\cC$ onto its image.

A classical cell is called compatible with a continuous function $F$
if $\sign F$ is constant on the cell. Since $\R_+\cC$ is connected, we
see that $f(\R_+\cC)$ is compatible with $F$ in this classical sense
if and only if $f$ is compatible with $F$ in our sense.

\subsection{Cellular parametrizations for semialgebraic and
  subanalytic sets}

\begin{Def}\label{def:semialg-complexity}
  A semialgebraic set $S\subset\R^n$ has \emph{complexity} $(\ell,\beta)$
  if $S=\pi_{1..n}(\tilde S)$ where $\tilde S\subset\R^\ell$ is given
  by
  \begin{equation}
    \tilde S = \{ \sign P_1(\vx_{1..\ell})=\sigma_1,\ldots,\sign P_N=\sigma_N \}, \qquad \sigma_1,\ldots,\sigma_N\in\{-1,0,1\}
  \end{equation}
  where $P_1,\ldots,P_N\in\R[\vx_{1..\ell}]$ have degrees at most
  $\beta$ and $N\le\beta$. If $\ell=n$ we will say simply that $S$ has
  complexity $\beta$.
\end{Def}

The following semialgebraic parametrization result is a simple
consequence of the CPT.

\begin{Cor}\label{cor:cpt-semialg}
  Let $\rho,\sigma\in\R_+$ and let $S\subset(0,1)^n$ be semialgebraic
  of complexity $(\ell,\beta)$. Then there exist
  $\poly_\ell(\beta,\rho,1/\sigma)$ real cellular maps
  $f_j:\cC^\hsigma_j\to\cP_n^\hrho$, each of complexity
  $\poly_\ell(\beta)$, such that $f_j(\R_+\cC_j^\hsigma)\subset S$ and
  $\cup_j f_j(\R_+\cC_j)=S$.
\end{Cor}

\begin{proof}
  We first reduce to the case $\tilde S\subset(0,1)^\ell$ in the
  notations of Definition~\ref{def:semialg-complexity}. We may split
  $\tilde S$ to a union of sets and prove the result for each of their
  projections separately, so we reduce to the case
  $\tilde S\subset(0,1)^n\times I_{n+1}\times\cdots\times I_\ell$
  where each $I_j$ is one of $\{0\},(0,\infty),(-\infty,0)$. In the
  case of $\{0\}$ we can forget the $x_j$ variable, and in the case
  $(-\infty,0)$ we can replace $x_j$ by $-x_j$, reducing to the case
  $\tilde S\subset(0,1)^n\times(0,\infty)^m$. In the same manner but
  using $x_j\to1/(1+x_j)$ (which increases the degrees at most
  polynomially) we reduce to the case $\tilde S\subset(0,1)^\ell$.

  Now apply the real CPT to the cell $\cP_\ell^\hrho$ with the
  collection of polynomials defining $\tilde S$. We obtain a
  collection of real prepared maps
  $\tilde f_j:\tilde \cC^\hsigma_j\to\cP_\ell^\hrho$ with the required
  complexity estimates such that
  $\tilde f_j(\R_+\tilde\cC_j^\hsigma)\subset \tilde S$ and
  $\cup_j \tilde f_j(\R_+\tilde\cC_j)=\tilde S$. The cellular
  structure of $\tilde f_j$ implies that if we now take
  $\cC_j:=(\tilde C_j)_{1..n}$ and $f_j:=(\tilde f_j)_{1..n}$ we have
  indeed $f_j(\R_+\cC_j^\hsigma)\subset S$ and
  $\cup_j f_j(\R_+\cC_j)=S$.
\end{proof}

In an analogous manner one obtains the following subanalytic version.

\begin{Cor}\label{cor:cpt-subanalytic}
  Let $\rho,\sigma\in\R_+$ and let $S\subset(0,1)^n$ be
  subanalytic. Then there exist $\poly_S(\rho,1/\sigma)$ real
  cellular maps $f_j:\cC^\hsigma_j\to\cP_n^\hrho$ such that
  $f_j(\R_+\cC_j^\hsigma)\subset S$ and $\cup_j f_j(\R_+\cC_j)=S$.
\end{Cor}

\begin{Rem}\label{rem:cpt-semi-extra}
  We remark that from the proof it is clear that in
  Corollaries~\ref{cor:cpt-semialg} and~\ref{cor:cpt-subanalytic} one
  can also require the maps $f_j$ to be compatible with an additional
  collection of functions $F_j\in\cO_b(\cP_n^\hrho)$. We also remark
  that by rescaling one can clearly replace the domain $(0,1)^n$ in
  Corollaries~\ref{cor:cpt-semialg} and~\ref{cor:cpt-subanalytic} by
  any other bounded semialgebraic/subanalytic ambient set. We will
  sometimes use $[0,1]^n$.
\end{Rem}


\subsection{Preparation theorems}

The real CPT implies the preparation theorem for subanalytic functions
of Parusinski \cite{parusinski:preparation} and Lion-Rolin
\cite{lion-rolin}, as we illustrate below. We illustrate the algebraic
case here (where we get more effective information) but the
subanalytic case follows in a similar manner. Let $F:(-1,1)^n\to\R$ be
a bounded semialgebraic function and let $G_F\subset\R^n_x\times\R_y$
be its graph. We aim to cover $(0,1)^n$ by cylinders where $F$ admits
a simple expansion.

We apply Corollary~\ref{cor:cpt-semialg} for the set $G_F$, requiring
also that the maps $f_j$ be compatible with $y$ (see
Remark~\ref{rem:cpt-semi-extra}). We obtain maps
$f_j:\cC_j^\hsigma\to \cP_n^{1/2}\times\cP^{1/2}$ admitting such that
$f_j(\R_+\cC_j^\hsigma)\subset G_F$ and $f_j(\R_+\cC_j)$ cover $G_F$,
and moreover each map is compatible with $y$.

Let $f:\cC^\hsigma\to\cP_n^{1/2}\times\cP^{1/2}$ be one of the maps
$f_j$. Since $G_F$ is a graph the type of $\cC$ must end with $*$. It
follows from the monomialization lemma that on each $\cC^\hsigma$ we have
either $f^*y\equiv0$ or
\begin{equation}
  f^* y = \vz^{\valpha(j)} U_j(\vz)
\end{equation}
where $U$ is a holomorphic map bounded away from zero and infinity on
$\cC$. To rewrite this expansion in the $\vx$-coordinates recall that
\begin{equation}\label{eq:vz-v-vx}
  \vz_j = (\vx_j - \phi_j(\vz_{1..j-1}))^{1/\nu_j}
\end{equation}
where we restrict $\vz_j$ to the positive real part $\R_+\cC_j$ and
take the positive branch. Since $F(x)\equiv y$ on $G_F$ we have on
the cylinder $f(\R_+\cC_j)$ the expansion
\begin{equation}
  F(\vx) = \vz^\valpha U(\vz)
\end{equation}
where $\vz$ is given by~\eqref{eq:vz-v-vx}. In other words we have
obtained cylinders where $F$ expands as a monomial with fractional
powers times a unit. This implies the preparation theorem of
\cite{lion-rolin} (in the bounded semialgebraic case). Indeed, in
\cite{lion-rolin} the unit is required to be bounded away from zero
and infinity and satisfy an analytic expansion of the form
$U(\vz) = V(\psi(\vx_{1..n-1},\vx_n))$ where
\begin{multline}\label{eq:lion-rolin-expansion}
  \psi(\vx_{1..n-1},\vx_n) = (\psi_1(\vx_{1..n-1}),\ldots,\psi_s(\vx_{1..n-1}),\\
  \vx_n^{1/p}/a_1(\vx_{1..n-1}),b_1(\vx_{1..n-1})/\vx_n^{1/p}),
\end{multline}
with $p$ a positive integer and $\psi_i,a_1,b_1$ are bounded
subanalytic functions, and $V$ is a non-zero analytic function on the
compact closure of the image of $\psi$. In our case this expansion is
the Laurent expansion of $U_j(\vz)$ with respect to $\vz_n$. We of
course obtain a wealth of additional information on the number of
cylinders, their complexity, and effective estimates on the monomial
$\vz^\alpha$ and the unit $U(\vz)$ from the algebraic monomialization
lemma.

We remark that, as usual in the passage from the real analytic to the
holomorphic category, the definitions involving existence of
converging expansions are replaced by purely local counterparts. For
instance, while the notion of a function \emph{reducible} in a
cylinder requires a careful inductive definition in \cite{lion-rolin},
any holomorphic function compatible with a complex cell is
\emph{automatically} reducible there. Similarly, while the definition
of of a unit involves a delicate analytic
expansion~\eqref{eq:lion-rolin-expansion} in \cite{lion-rolin}, a unit
in a complex cell is a holomorphic function satisfying a purely
topological definition (the associated monomial being equal to zero),
which in turn implies the real condition.


\section{The principal lemmas}
\label{sec:fund-lemmas}

\subsection{Hyperbolic geometry of complex cells}

For the proofs of the domination, fundamental and monomialization
lemmas we will require some basic notions of hyperbolic
geometry. Recall that the upper half-plane $\H$ admits a unique
hyperbolic metric of constant curvature $-4$ given by $|\d z|/2y$. A
Riemann surface $X$ is called \emph{hyperbolic} if its universal cover
is the upper half-plane $\H$. In this case $U$ inherits from $\H$ a
unique metric of constant curvature $-4$ which we denote by
$\dist(\cdot,\cdot;U)$ (we sometimes omit $U$ from this notation if it
is clear from the context). In particular a domain $U\subset\C$ is
hyperbolic if its complement contains at least two points. The
Schwarz-Pick lemma states that any holomorphic map between hyperbolic
Riemann surfaces is non-expanding in their respective hyperbolic
metrics.

\subsection{Maps from cells into hyperbolic Riemann surfaces}

We will require the following notion of \emph{skeleton} of a cell.

\begin{Def}[Skeleton of a cell]
  If $\cC$ is a cell whose type does not include $D_\circ$ then we
  define its \emph{skeleton} $\cS(\cC)$ as follows: the skeleton of
  the cell of length zero is the singleton $\C^0$; The skeleton of
  $\cC_{1..\ell}\odot \cF$ is $\cS(\cC_{1..\ell})\odot \partial\cF$
  where
  \begin{align}
    \partial*&:=* & \partial D(r) &:= S(r) & \partial A(r_1,r_2)&:=
                                                                  S(r_1)\cup S(r_2).
  \end{align}
  Each connected component of $\cS(\cC)$ is a product of $\ell$ circles
  and points, and the number of connection components is equal to
  $2^\alpha$, where $\alpha$ is the number of symbols $A$ in the type of
  $\cC$.
\end{Def}

Let $\cC$ be a complex cell whose type does not contain $D_\circ$. For
the results discussed in this section any coordinate with type $*$ can
be removed without loss of generality, so we assume that the type does
not contain $*$. Then $\dim\cC=\ell(\cC)$ and we denote this number by
$n$. We assume that $\cC$ admits a $\hrho$ extension for some
$\hrho>0$ and that $f:\cC^\hrho\to X$ is a holomorphic map to a
hyperbolic Riemann surface $X$. We begin by studying the hyperbolic
behavior of $f$ on the skeleton $\cS(\cC)$.


\begin{Lem}\label{lem:fS-hyperbolic-diam}
  Let $S$ a component of the skeleton $\cS(\cC)$. Then
  \begin{equation}
    \diam(f(S);X) < n\rho.
  \end{equation}
\end{Lem}
\begin{proof}
  We proceed by induction and suppose the claim is proved for cells of
  dimension smaller than $n$. Let $\vz,\vz''\in S$. We will construct
  a point $\vz'\in S$ such that $\dist(f(\vz),f(\vz'))<\rho$ and
  such that $\vz'_1=\vz''_1$. Then $\vz'_{2..n},\vz''_{2..n}$ both
  belong to the $n-1$ dimensional cell $\cC_{\vz_1'}$ and applying the
  inductive hypothesis to the restriction of $f$ to this fiber we
  conclude that $\dist(f(\vz'),f(\vz''))<(n-1)\rho$ thus finishing the
  proof.

  By definition of the skeleton, we have
  \begin{align}
    \vz_1&=\e_1r_1 & \vz_2&=\e_2r_2(\vz_1) & &\dots & \vz_n&=\e_n 
    r_n(\vz_{1..n-1}), \quad \abs{\e_i}=1.
  \end{align}
  where $r_1>0$ and $r_2,\ldots,r_n$ are holomorphic functions on
  $\cC^\hrho$. Let $\cF$ denote the first fiber in $\cC$. We define a map
  $\gamma(\vz_1):\cF^\hrho\to\cC^\hrho$ by $\gamma(\vz_1)=(\vz_{1..n})$ where
  \begin{align}
    \vz_2 &= \e_2 r_2(\vz_1) & &\dots & \vz_n =\e_n r_n(\vz_{1..n-1}).
  \end{align}
  Then $\gamma$ is holomorphic in $\cF^\hrho$ and satisfies
  $\gamma(\vz_1)=\vz$ and $\gamma(\{|\vz_1|=r_1\})\subset S$. We take
  $\vz':=\gamma(\vz''_1)\in S$ and note that
  \begin{equation}
    \dist(f(\vz),f(\vz');X) =
    \dist(f\circ\gamma(\vz_1),f\circ\gamma(\vz''_1);X)\le \dist(\vz_1,\vz''_1;\cF^\hrho)\le\rho
  \end{equation}
  where the first inequality follows from the Schwarz-Pick lemma for
  $f\circ\gamma$ and the second inequality follows since
  $\vz_1,\vz_1''$ belong to the boundary of $\cF$ in $\cF^\hrho$.
\end{proof}

Next, we show (essentially by the open mapping theorem) that the
boundary of $f(\cC)$ is controlled by the skeleton.

\begin{Lem}\label{lem:boundary-vs-skeleton}
  We have the inclusion $\partial f(\cC) \subset f(\cS(\cC))$.
\end{Lem}
\begin{proof}
  We assume for simplicity that the type of $\cC$ is $A\cdots A$ (the
  difference with disc fibers is purely notational). Let
  $p\in\partial f(\cC)$ and fix a point $\vz\in\cC$ such that
  $f(\vz)=p$. If $\vz\in S$ then we are done. Otherwise one of the
  following holds
  \begin{equation}\label{eq:k-index-def}
    \begin{aligned}
      r_{1,1} < &|\vz_1|< r_{1,2}  \\
      |r_{2,1}(\vz_1)| < &|\vz_2| <|r_{2,2}(\vz_1)| \\
      &\vdots \\
      |r_{n,1}(\vz_{1..n-1})| < &|\vz_n| < |r_{n,2}(\vz_{1..n-1})|.
    \end{aligned}
  \end{equation}
  Let $k$ be the largest index for which such inequalities hold. Let
  \begin{equation}
    A := A(r_{j,1}(\vz_{1..j-1}),r_{j,2}(\vz_{1..j-1})).
  \end{equation}
  We suppose further that $\vz$ is chosen such that $k$ is the minimal
  possible. We define a map $\gamma(z_k):A^\delta\to\cC$ by
  $\gamma(z_j) =(z_{1..n})$ where $z_{1..k-1}=\vz_{1..k-1}$ and
  \begin{equation}
    z_{j+1} = \frac{\vz_{j+1}}{r_{j+1,i(j+1)}(\vz_{1..j})} r_{j+1,i(j+1)}(z_{1..j}),  \qquad
    j\ge k
  \end{equation}
  where $i(j)$ is the index, either 1 or 2, such that equality instead
  of inequality holds in the $j$th line
  of~\eqref{eq:k-index-def}. By definition we have $\gamma(\vz_j)=\vz$.

  If $f\circ\gamma$ is non-constant then by the open mapping theorem
  $p=f\circ\gamma(\vz_j)$ is an interior point of
  $f\circ\gamma(A)\subset f(\cC)$ contrary to our
  assumption. Otherwise we have $p=f(\vz')$ where $\vz'=\gamma(z_j')$
  and $z_j'$ is any point on $\partial A$. The index $k$ obtained for
  $\vz'$ is by definition smaller than that obtained for $\vz$, which
  yields a contradiction to our choice of $\vz$.
\end{proof}

Finally, we show that under a suitable topological condition the
hyperbolic diameter of $f(\cC)$ itself can be bounded.

\begin{Lem}\label{lem:fC-hyperbolic-diam}
  Suppose that $f_*(\pi_1(\cC))=\{e\}\subset\pi_1(X)$. Then
  \begin{equation}
    \diam(f(\cC);X)<2^nn\rho.
  \end{equation}
\end{Lem}
\begin{proof}
  Denote by $\pi:\D\to X$ the universal covering map. By our
  assumption we may lift $f$ to a map $F:\cC^\hrho\to\D$ satisfying
  $f=\pi\circ F$. Since $\pi$ is non-expanding it is enough to prove
  the claim for the hyperbolic diameter of $F(\cC)$. By
  Lemma~\ref{lem:fS-hyperbolic-diam} the hyperbolic diameter of
  $F(S)$, where $S$ is any component of the skeleton $\cS(\cC)$, is
  bounded by $n\rho$. By Lemma~\ref{lem:boundary-vs-skeleton}
  we also have the inclusion $\partial F(\cC) \subset F(\cS(\cC))$.

  Recall that we assume that the type of $\cC$ does not contain
  $D_\circ$ and hence $\bar\cC\subset\cC^\delta$. In particular it
  follows that $Z:=F(\cC)$ is relatively compact, and hence bounded,
  in $\D$. Let $U$ denote the unbounded component of
  $\D\setminus\bar Z$ and set $U^c:=\D\setminus U$ and $\Gamma:=\partial U$. To avoid pathologies
  of plane topology we remark that $Z$ is subanalytic, hence $Z,U$ and
  their boundaries can be triangulated and the Mayer-Vietoris sequence
  \begin{equation}
    0=H_1(\D)\to H_0(\Gamma)\to H_0(\bar U)\oplus H_0(\overline{U^c})\to H_0(\D)\to0
  \end{equation}
  is exact. Since $H_0(\D)=H_0(\bar U)=\Z$ we have
  $H_0(\Gamma)\simeq H_0(\overline{U^c})$. We claim that $U^c$ and
  hence $\overline{U^c}$ is connected. Assume otherwise and write
  $\overline{U^c}=V_1\cup V_2$. Since $\bar Z$ is connected we have without loss of
  generality $\bar Z$ is contained in $V_1$ and therefore disjoint
  from $V_2$. In particular $\partial V_2$ is disjoint from
  $\bar Z$, contradicting
  \begin{equation}
    \partial V_2\subset\partial U^c=\partial U\subset\partial Z.
  \end{equation}
  In conclusion we have $H_0(\Gamma)\simeq H_0(\overline{U^c})=\Z$,
  i.e. $\Gamma$ is connected.

  We claim that the hyperbolic diameter of $F(\cC)$ is bounded by the
  hyperbolic diameter of $\Gamma$. Indeed, for any two points
  $p,q\in F(\cC)$ let $\ell$ denote the geodesic line connecting them
  and denote by $p'\in\ell\cap\Gamma$ some point on $\ell$ before $p$
  and by $q'\in\ell\cap\Gamma$ some point on $\ell$ after $q$. Then
  $\dist(p,q)\le\dist(p',q')$ which is bounded by the diameter of
  $\Gamma$. Finally, recall that $\Gamma$ is connected and contained
  in the union of $F(S_j)$ where $S_j$ run over the components of
  $\cS(\cC)$ (whose number is at most $2^n$), and the diameter of each
  $F(S_j)$ is bounded by $n\rho$. From this it follows easily
  that the diameter of $\Gamma$ is at most $2^nn\rho$.
  \begin{figure}
    \centering
    \includegraphics[width=0.6\textwidth]{DominationAnn2.pdf}
    \caption{Proof of Lemma~\ref{lem:fC-hyperbolic-diam}.}
    \label{fig:DominationAnn}
  \end{figure}
\end{proof}


\subsection{Proof of the domination lemma}
\label{sec:domination-proof}

We begin by assuming that the type of $\cC$ does not contain
$D_\circ$. Let
\begin{align}\label{eq:U01infty}
  U_0 &:= \{|z|<\tfrac12\} & U_1&:= \{|z-1|<\tfrac12\} & U_\infty = \{|z|>2\}.
\end{align}
We choose $s>1$ such that the hyperbolic distance between $U_q^s$ and
$\partial U_q$ is greater than $n\rho$ for $q=0,1,\infty$. From the
explicit computations in~\secref{sec:explicit-consts} we can take
$s=O(\log|\log(n\rho)|)$. If $f(\cC)$ does not meet $U_0^s$ or
$U_\infty^s$ then the proof of the domination lemma is
completed. Henceforth we assume that $f(\cC)$ meets both $U_0^s$ and
$U_\infty^s$.

\begin{Lem}\label{lem:trivial-homotopy}
  Suppose $f(\cC)$ meets both $U_0^s$ and $U_\infty^s$. Then
  \begin{equation}
    f_*\pi_1(\cC)=\{e\}\subset\pi_1(\C\setminus\{0,1\}).
  \end{equation}
\end{Lem}
\begin{proof}
  Recall that we assume that the type of $\cC$ does not contain
  $D_\circ$, and in particular $\bar\cC\subset\cC^\delta$. Thus $f$
  extends to a continuous function $\bar\cC$, so
  $f(\bar\cC)\subset\C\setminus\{0,1\}$ is compact and it follows that
  $\partial f(\cC)$ meets both $U_0^s$ and $U_\infty^s$. By
  Lemma~\ref{lem:boundary-vs-skeleton} we conclude that there exist
  two components $S_0,S_\infty$ of the skeleton $\cS(\cC)$ such that
  $f(S_0)$ meets $U_0^s$ and $f(S_\infty)$ meets
  $U_\infty^s$. From Lemma~\ref{lem:fS-hyperbolic-diam} and the
  choice of $s$ we conclude that $f(S_0)$ does not meet the
  boundary of $U_0$, i.e. $f(S_0)\subset U_0$ and similarly
  $f(S_\infty)\subset U_\infty$.

  It is easy to verify (it is enough to check this for the standard
  polyannulus) that $\pi_1(S)\to\pi_1(\cC)$ is epimorphic. In
  particular, if $\gamma\in\pi_1(\cC)$ denotes any loop then this loop
  is free-homotopy equivalent to a loop contained in $S_0$ and to a
  loop contained in $S_\infty$. Consequently $f_*(\gamma)$ is
  free-homotopy equivalent to a loop contained in $U_0$ and to a loop
  contained in $U_\infty$. However two such loops cannot be
  homotopically equivalent in $\C\setminus\{0,1\}$ unless they are
  both contractible, hence proving the claim.
\end{proof}

Lemma~\ref{lem:trivial-homotopy} implies the condition of
Lemma~\ref{lem:fC-hyperbolic-diam}, and we conclude that the
hyperbolic diameter of $f(\cC)$ is bounded by $2^nn\rho$. We now
choose $\hat s>1$ such that the hyperbolic distance between
$U_0^{\hat s}$ and $U^{\hat s}_\infty$ is greater than
$2^nn\rho$. By~\secref{sec:explicit-consts} we may take
$\hat s=O(\log(n+\log\rho))$. Then $f(\cC)$ cannot meet both
$U_0^{\hat s},U^{\hat s}_\infty$ and the domination lemma is
proved.
\begin{figure}
  \centering
  \includegraphics[width=\textwidth]{DominationDisc.pdf}
  \caption{Proof of the domination lemma.}
  \label{fig:DominationDisc}
\end{figure}
  
It remains to consider the case that the type of $\cC$ contains
$D_\circ$. Let $0<\e<1$ and let $\cC_\e$ be the cell obtained from
$\cC$ by replacing each occurrence of a fiber $D_\circ(r)$ by
$A(\e r,r)$. It is clear that $\cC_\e$ admits a $\hrho$-extension and
$\cup_{\e>0}\cC_\e=\cC$. The domination lemma for $\cC$ thus follows
immediately from the domination lemma for $\cC_\e$, which was already
established.

\subsection{Proofs of the fundamental lemmas}

In this section we assume that the type of $\cC$ does not contain
$D_\circ$. The general case can be reduced to this case as in the end
of~\secref{sec:domination-proof}. The fundamental lemma for $\D$ is
already proved as a consequence of
Lemma~\ref{lem:fC-hyperbolic-diam}. The proof in the case of
$\D\setminus\{0\}$ is based on the following simple geometric lemma.

\begin{Lem}
  Let $z\in\D\setminus\{0\}$ and let $\gamma_z$ denote the shortest
  non-contractible loop passing through $z$. Then
  $\length(\gamma_z)\sim1/|\log|z||$.
\end{Lem}
\begin{proof}
  Let $\zeta=i^{-1}\log z$. Lifting to the universal cover
  $\exp(i\zeta):\H\to\D\setminus\{0\}$ we must calculate
  $\dist(\zeta,\zeta+2\pi;\H)$. By a standard formula for the
  hyperbolic distance we have
  \begin{equation}
    \dist(\zeta,\zeta+2\pi;\H) = 2\ln\left(\frac{\pi+\sqrt{\pi^2+\Im^2\zeta}}{\Im\zeta}\right)
    \sim 1/\Im\zeta.
  \end{equation}
\end{proof}

Let $f:\cC^\hrho\to\D\setminus\{0\}$ be holomorphic and let $z$ be the
maximum value of $|f|$ on $\cC$, which by
Lemma~\ref{lem:boundary-vs-skeleton} is obtained on a component $S$ of
the skeleton of $\cC$. By Lemma~\ref{lem:fS-hyperbolic-diam} we have
$\diam(f(S);D\setminus\{0\})=O_\ell(\rho)$. In particular, if
$1/|\log|z||=\Omega_\ell(\rho)$ then $f(S)$ cannot contain a
non-contractible loop in $D\setminus\{0\}$. Then, since
$\pi_1(S)\to\pi_1(\cC)$ is epimorphic we have
\begin{equation}
  f_*(\pi_1(\cC))=f_*(\pi_1(S))=\{e\}\subset\pi_1(\D\setminus\{0\}).
\end{equation}
In this case by Lemma~\ref{lem:fC-hyperbolic-diam} we have
$\diam(f(\cC);D\setminus\{0\})=O_\ell(\rho)$. Otherwise we have
$\log|z|=-\Omega_\ell(1/\rho)$, which concludes the proof of the first
statement. The second statement follows from the simple fact
\begin{equation}
  \dist(\zeta_1,\zeta_2;\H) \ge \dist(\log\Im\zeta_1,\log\Im\zeta_2;\R).
\end{equation}

We now pass to the proof of the fundamental lemma for
$\C\setminus\{0,1\}$. Let $S$ be a component of the skeleton of
$\cC$. By Lemma~\ref{lem:fS-hyperbolic-diam} we have
$\diam(f(S);C\setminus\{0,1\})=O_\ell(\rho)$. In particular, for
$\rho=O(1)$ we see that $f(S)$ cannot meet more than one of
$U_0,U_1,U_\infty$ in the notation~\eqref{eq:U01infty}. Suppose first
the $f(S)$ meets none of these sets. Let $\rho_0$ denote the length of
the shortest non-contractible loop in the compact set
$(U_0\cup U_1\cup U_\infty)^c\subset\C\setminus\{0,1\}$. Then for
$\rho=O(\rho_0)=O(1)$ we get
$\diam(f(\cC);\C\setminus\{0,1\})=O_\ell(\rho)$ as in the proof of the
case $\D\setminus\{0\}$.

Next, suppose that for two different components $S,S'$ the images
$f(S),f(S')$ meet two of the sets $U_0,U_1,U_\infty$, say $U_0$ and
$U_1$ respectively. Then for $\rho=O(1)$ the images $f_*(\pi_1(S))$
(resp. $f_*(\pi_1(S'))$ can only be a power of the fundamental loop
around $0$ (resp. $1$) and as in the proof of the domination lemma we
conclude that $f_*(\pi_1(\cC))=\{e\}$ and
$\diam(f(\cC);\C\setminus\{0,1\})=O_\ell(\rho)$.

Finally, suppose all skeleton components $S$ meet one of the sets
$U_0,U_1,U_\infty$, say $U_0$. Then for $\rho=O(1)$ we see that
$f(S)\subset\D\setminus\{0\}$ for each skeleton component. In this
case we can finish the proof as in the case of $\D\setminus\{0\}$. We
need only the estimate for the length of the shortest geodesic in
$\C\setminus\{0,1\}$ passing through a given point $z$ close to $0$:
this is asymptotically the same as in the metric of
$\D\setminus\{0\}$, for instance by the estimate~\eqref{eq:hyperb} of
\cite{bp:poincare}, given explicitly in~\secref{sec:explicit-consts}.

\subsection{Proof of the monomialization lemma}

The \emph{Voorhoeve index} \cite{ky:rolle} of a holomorphic function
$f:U\to\C$ along a subanalytic curve $\Gamma\subset U$ is defined by
\begin{equation}
  V_{\Gamma}(f):= \frac1{2\pi} \int_{\Gamma}|\d\Arg f(z)|.
\end{equation}
We will need the following basic fact about Voorhoeve indices.
\begin{Lem}\label{lem:voorhoeve-estimate}
  Fix $\rho>0$. Let $\cF_\lambda$ be a definable family of
  one-dimensional cells, let $f_\lambda:U^\rho\to\C$ a definable
  family of holomorphic functions, and let
  $\Gamma_\lambda\subset\cF_\lambda$ be a definable family of curves.
  Then $V_{\Gamma_\lambda}(f_\lambda)$ is uniformly bounded over
  $\lambda$.

  If $f,\Gamma$ are algebraic of complexity $\beta$ then
  $V_\Gamma(f)=\poly(\beta)$.
\end{Lem}
\begin{proof}
  Note that $2\pi V_\Gamma(f)$ is the total length of the curve
  $(\frac f{|f|})(\Gamma)\subset S(1)$, which by standard integral
  geometry is given by the average number of intersections between the
  curve and a ray through the origin, i.e. the average number of
  (isolated) solutions of the equation $\arg f(z)=\alpha$ for
  $z\in\Gamma$ $\alpha\in[0,2\pi)$. By o-minimality the \emph{maximal}
  number (and in particular the average number) of (isolated)
  solutions for the pair $f_\lambda,\Gamma_\lambda$ is bounded by a
  constant independent of $\lambda$. In the algebraic case the number
  of solutions is bounded by $O(\beta^2)$ by the Bezout theorem.
\end{proof}

The basic ingredient in the proof of the monomialization lemma is the
following one-dimensional version, proved using the Voorhoeve index.

\begin{Lem}\label{lem:monom-dim1}
  Let $\cF$ be a one-dimensional cell and
  $f:\cF^\hrho\to\C\setminus\{0\}$. Then $f=z^{\alpha(f)}\cdot U(z)$ where
  \begin{align*}
    \diam(\Re\log U(\cF);\R) &< O_f(\rho), & \diam(\Im\log U(\cF);\R) &< O_f(1).
  \end{align*}
  Moreover if $\cF,f$ vary in a definable family then $|\alpha(f)|$
  and the asymptotic constants in $O_f(\cdot)$ can be taken to be
  uniform over the family. If $f$ is algebraic of complexity $\beta$ then
  $|\valpha(f)|=\poly(\beta)$ and
  \begin{align*}
    \diam(\Re\log U(\cF);\R) &< \poly(\beta)\cdot\rho, & \diam(\Im\log U(\cF);\R) &< \poly(\beta).
  \end{align*}
\end{Lem}
\begin{proof}
  We follow the idea of \cite[Section 4.5]{ky:rolle}. By definition
  $\alpha(f)$ is equal to the total winding number of $f$ along a
  concentric circle contained in $\cF$ for types $D_\circ,A$ and zero
  in type $D$. In particular it is bounded by $V_\Gamma(f)$ where
  $\Gamma$ is such a circle, and the statements about $\alpha(f)$ then
  follow from Lemma~\ref{lem:voorhoeve-estimate}. In the algebraic
  case we see also that the complexity of $U$ is $\poly(\beta)$.

  Any two points in $\cF$ can be joined by two consecutive algebraic
  curve: a radial ray and a circle concentric with $\cF$. By
  Lemma~\ref{lem:voorhoeve-estimate} the Voorhoeve index of $f$ along
  these curves is uniformly bounded (resp.  bounded by $\poly(\beta)$
  in the algebraic case). Having already established that
  $|\alpha(f)|$ is uniformly bounded (resp. bounded by $\poly_\ell(\beta)$
  in the algebraic case) we conclude that the Voorhoeve index of $U$
  along any two such curves is similarly bounded by a uniform constant
  $v$ (and $v=\poly_\ell(\beta)$ in the algebraic case).

  Let $p\in\cF$ be an arbitrary point. Since the claim is invariant
  under scalar multiplication of $f$ we may assume without loss of
  generality that $U(p)=1$. Since $U_*\pi_1(\cF)=\{e\}$ we have a
  well-defined root $W:\cF^\hrho\to\C\setminus\{0\}$ satisfying
  $W=iU^{1/(4v)}$ and $W(p)=i$. Connecting $p$ to any other point
  $q\in\cF$ by two curves as above, the Voorhoeve index of $W$ along
  the curves is bounded by $1/4$, hence the total variation of
  argument is bounded by $\pi/2$. Since $\arg W(p)=\pi/2$ we conclude that
  in fact $W:\cF^\hrho\to\H$. By Lemma~\ref{lem:fS-hyperbolic-diam} we
  then have $\diam(W(\cF);\H)=O(\rho)$. In particular this implies
  \begin{align}
    \diam(\Re\log W(\cF);\R) &= O(\rho) & \diam(\Im\log W(\cF);\R) &< \pi
  \end{align}
  and the claim for $U$ follows immediately (using the fact that
  $v=\poly_\ell(\beta)$ in the algebraic case).
\end{proof}

We are now ready to finish the proof of the monomialization lemma by
induction on $\ell$. Let $\cC=\cC_{1..\ell}\odot\cF$. The case $\cF=*$
reduces trivially to the claim for $\cC_{1..\ell-1}$ and the case
$\ell=0$ is proved in Lemma~\ref{lem:monom-dim1}. Up to
renormalization we may assume that the outer radius of $\cF$ is $1$.
Let $\hat f:\cC_{1..\ell}^\hrho\to\C\setminus\{0\}$ be defined by
$\hat f:=f(\vz_{1..\ell},1)$. If $\cC,f$ are algebraic of complexity
$\beta$ then $\hat f$ is algebraic of complexity $\poly_\ell(\beta)$.

By definition we have $\valpha(\hat f)=\valpha_{1..\ell}(f)$, so
$\hat U:=U(\vz_{1..\ell},1)$ is equal to
$\hat f/\vz^{\alpha(\hat f)}$. By the inductive hypothesis we have
\begin{align}
  \diam(\Re\log \hat U(\cC_{1..\ell});\R) &< O_{\hat f}(1)\cdot\rho, & \diam(\Im\log \hat U(\cC_{1..\ell});\R) &< O_{\hat f}(1)
\end{align}
with uniform constants if $\cC,f$ vary in a definable family. In the
algebraic case $|\alpha(\hat f)|=\poly_\ell(\beta)$ and
\begin{align}
  \diam(\Re\log \hat U(\cC_{1..\ell});\R) &< \poly_\ell(\beta)\cdot\rho, & \diam(\Im\log \hat U(\cC_{1..\ell});\R) &< \poly_\ell(\beta).
\end{align}
Also, for each fixed $p\in\cC_{1..\ell}$ we have by
Lemma~\ref{lem:monom-dim1} for the fiber $\cC_p$
\begin{align}
  \diam(\Re\log U(\cC_p);\R) &< O_f(\rho), & \diam(\Im\log U(\cC_p);\R) &< O_f(1)
\end{align}
with uniform constants if $\cC,f$ vary in a definable family. In the
algebraic case $|\alpha_{\ell+1}(f)|=\poly(\beta)$ and
\begin{align}
  \diam(\Re\log U(\cC_p);\R) &< \poly_\ell(\beta)\cdot\rho, & \diam(\Im\log U(\cC_p);\R) &< \poly_\ell(\beta).
\end{align}
The triangle inequality now finishes the proof.



\subsection{Hyperbolic lengths}
\label{sec:explicit-consts}

We fix the hyperbolic metric, $\lambda_D(z) |dz|$ with
$\lambda_D(z)=(1-|z|^2)^{-1}$, of constant curvature $-4$ on the unit
disc $D$. An explicit lower bound for the hyperbolic metric
$\lambda_{0,1}(z)|dz|$ on $\C\setminus\{0,1\}$ in $U_0$ is given in
\cite{bp:poincare},
\begin{equation}\label{eq:hyperb}
  \lambda_{0,1}(z)\ge 
  \frac{1}{2|z|\sqrt{2}\left[4+\log(3+2\sqrt{2})-\log|z|\right]},\quad |z|\le  
  \frac{1}{2}.
\end{equation}
Integrating \eqref{eq:hyperb}, we see that the hyperbolic distance from 
$\{|z|=r\}$ to $\{|z|=1/2\}$ is greater than $s$ for $r<\tilde{\rho}(s)<\frac 1 
2$, where
\begin{equation*}
\log\tilde{\rho}(s)=-e^{\Theta(s)}+O(1).
\end{equation*}
As $z\to z^{-1}$ is an isometry of $\C\setminus\{0,1\}$, the 
hyperbolic distance from $\{|z|=r\}$ to $\{|z|=2\}$ is greater than $s$ for 
$r>\tilde{\rho}(s)^{-1}$.


\subsubsection{Proof of Fact~\ref{fact:boundary-length}.}
First, consider the case of $\mathcal{F}$ of type $D$. Up to rescaling, 
$\mathcal{F}^\delta = D(1)$ and $\partial \mathcal{F}=\{\abs{z}=\delta\}$. Then 
$\lambda_D(z)\abs{dz}\equiv(1-\delta^2)^{-1}\abs{dz}$ on $\partial 
\mathcal{F}$, so 
the length of $\partial \mathcal{F}$ in $D$ is equal to 
$\frac{2\pi\delta}{1-\delta^2}$.

Second, consider the cases $\mathcal{F}$ of type $D_\circ, A$. If $S(r)$ is (a 
component of)  $\partial\mathcal{F}$, then $S(r)\subset 
S^\delta(r)\subset\cF^\delta$. Therefore, by Schwartz-Pick 
lemma, it is enough to prove the following 
\begin{Lem}\label{lem:hyperbcircle}
  The length  of $S(1)$ in the hyperbolic metric of the annulus
  $S^\delta(1)$ is equal to $\frac{\pi^2}{2\abs{\log\delta}}$.
\end{Lem}
\begin{proof}
  Indeed, the mapping $\phi(z)=\frac{\pi i}{2\abs{\log\delta}}\log z$ maps
  the universal cover of $S^\delta(1)$ to the strip
  $\Pi=\{|\Im w|\le \frac{\pi}{2}\}$, with
  $\phi^{-1}([0,\frac{\pi^2}{\abs{\log\delta}}])=S^1$. This map preserves
  hyperbolic distances, so it is enough to find the hyperbolic length
  of $[0, \frac{\pi^2}{\abs{\log\delta}}]$ in the hyperbolic metric 
  $\lambda_{\Pi}(w)\abs{dw}$ of $\Pi$. As $\Pi$ is invariant
  under shifts by reals, it is enough to find the $\lambda_{\Pi}(0)$.  The
  map $\psi(w)=\frac{e^w-1}{e^w+1}$ send $\Pi$ isometrically to the
  unit disc, with $\psi(0)=0$, so $\lambda_{\Pi}(0)=\psi'(0)=1/2$ and the
  length of $[0, \frac{\pi^2}{\abs{\log\delta}}]$ in $\Pi$ is equal to
  $\frac{\pi^2}{2\abs{\log\delta}}$.
\end{proof}

%By Lemma~\ref{lem:hyperbcircle} and
%Lemma~\ref{lem:fS-hyperbolic-diam}, we see that
%$C_1(n,\delta)=n\frac{\pi^2}{2\abs{\log\delta}}$ is an upper bound for
%the hyperbolic diameters of images of connected components of
%$\partial\cC$.
%
%Assume that neither $\log|f|<-\log\CF$ nor $\log|f|>\log\CF$ holds on
%$\cC$, where
%$$
%\log\CF=\log\tilde{\rho}(s)=-(4+\log(6+4\sqrt{2}))
%e^{n2^{n+\tfrac 1 2}\frac{\pi^2}{\abs{\log\delta}}}+\log(3+2\sqrt(2))+4
%$$ 
%and 
%$$s=C_2(n,\delta)=2^nC_1(n,\delta)=n2^{n-1}\frac{\pi^2}{\abs{\log\delta}}.$$
%Then $|f(z_{\min})|\le\tilde{\rho}(s)$ at some point
%$z_{\min}\in\partial\cC$, so the image of the connected component of
%$\partial\cC$ containing $z_{\min}$ lies in $U_0$, and, similarly, an
%image of another connected component of $\partial\cC$ lies in
%$U_{\infty}$. Then, by Lemma~\ref{lem:fC-hyperbolic-diam}, the
%hyperbolic diameter of $f(\cC)$ is at most $s$. This contradicts the
%fact that $f(\cC)$ intersects both
%$\{\log|z|=\pm\log\tilde{\rho}(s)\}$, as the hyperbolic distance
%between these two circles is more than $2s$.

\begin{Rem}\label{rem:Adelta-diameter}
  The same chain of conformal mappings allows to compute explicitly
  the distance between $r>0$ and $1$ in the hyperbolic metric of
  $A^\delta=A(\delta r, \delta^{-1})$:
  \begin{equation}
    \dist_{A^\delta}(r,1)=\log\frac{1+\tan\phi}{1-\tan\phi}, \quad 
    \text{where}\quad
    \phi=\frac\pi4\left(1-\frac{\log \delta}{\log(\delta \sqrt{r})}\right).
  \end{equation}
  As $r\to 0$, we get
  \begin{equation}
    \dist_{A^\delta}(r,1)=\log\left(\frac{16}{\pi}\frac{\log(\delta 
        \sqrt{r})}{\log 
        \delta}\right)+O(1)=\log\abs{\log r} +O(1).
  \end{equation}
  From the inequalities
  $\dist_{A^\delta}(r,1)<\diam_{A^\delta}A< \dist_{A^\delta}(r,1)+\frac{\pi^2}{2\abs{\log\delta}}$,
  we see that
  \begin{equation}
    \diam_{A^\delta}A=\log\abs{\log r} +O(1)\quad \text{as} \quad r\to0.
  \end{equation}
\end{Rem}

\section{Geometric constructions with cells}

In this section we develop the two key geometric constructions used in
the proofs of our main theorems: cellular refinement and clustering in
fibers of proper covers.

\subsection{Refinement of cells}

In this section we show how cells with a $\hrho$-extension can be
\emph{refined} into cells with a $\hsigma$-extension for
$0<\sigma<\rho$. We remark that while the statement appears innocuous,
the proof actually requires the full strength of the fundamental lemma
for $\D\setminus\{0\}$. The key difficulty is to construct the
refinement of an annulus fiber in a manner that depends
holomorphically on the base.

\begin{Thm}[Refinement theorem]\label{thm:cell-refinement}
  Let $\cC^\hrho$ be a (real) cell and $0<\sigma<\rho$. Then there
  exists a (real) cellular cover $\{f_j:\cC_j^\hsigma\to\cC^\hrho\}$
  of size $\poly_\ell(\rho,1/\sigma)$ where each $f_j$ is a cellular
  translate map.

  If $\cC$ varies in a definable family (and $\sigma,\rho$ vary under
  the condition $0<\sigma<\rho<\infty$) then the cells $\cC_j$ and
  maps $f_j$ can also be chosen from a single definable family. If
  $\cC$ is algebraic of complexity $\beta$ then $\cC_j,f_j$ are
  algebraic of complexity $\poly_\ell(\beta)$.
\end{Thm}

\begin{Rem}[Refinement and monomialization]\label{rem:refinement-vs-monom}
  The type of any cell $\cC_j$ obtained by refinement of $\cC^\hrho$
  is obtained from the type of $\cC$ by possibly replacing some
  $D_\circ,A$ fibers by $D$ fibers. Moreover
  $(f_j)_*:\pi_1(\cC_j)\to\pi_1(\cC)$ is the natural injection. In
  particular if $F:\cC\to\C\setminus\{0\}$ then $\valpha(f_j^*F)$ is
  obtained from $\valpha(F)$ by eliminating those indices that
  correspond to fibers that were replaced by $D$ in $\cC_j$.
\end{Rem}

The proof of the refinement theorem will occupy the remainder of this
subsection. We begin with the complex case, and indicate the necessary
modifications for the real case at the end. We proceed by induction on
$\ell$ (the case $\ell=0$ is vacuous). Consider a cell
$\cC\odot\cF$. By applying the inductive hypothesis to $\cC$ and
replacing $\cC\odot\cF$ by each $\cC_j\odot f_j^*\cF$ we may assume
without loss of generality that the base $\cC$ already admits
$E$-extension, where $E$ will be chosen later. The case $\cF=*$
reduces to the inductive hypothesis directly with $E=\hsigma$.

As a notational convenience we allow ourselves to rescale the fiber
$\cF$ by a non-vanishing holomorphic function
$s\in\cO_b(\cC^\hsigma)$, where it is understood that the covering
cells $\cC_j$ that we construct will eventually be rescaled to cover
the original $\cF$. When we describe the covering of $\cF$ we allow
ourselves to use discs centered at a point $p\in\cO_b(\cC^\hsigma)$
where it is understood that such discs will be centered at the origin
in the covering cells $\cC_j$ that we construct, and the maps $f_j$
will translate the origin to $p$. We note that rescaling has the
effect of eventually rescaling the centers of the covering discs that
we construct below by a factor of $s$. In the algebraic case we will
always choose $s$ and $p$ to be algebraic of complexity $\poly(\beta)$
so that the maps that we eventually construct will indeed be of
complexity $\poly(\beta)$.

We will consider two separate cases: $\rho>1,\sigma=1$ and
$\rho=1,\sigma<1$. The general case can be reduced to a composition of
these two cases, first refining $\hrho$-extensions into
$\he1$-extensions and then refining $\he1$-extensions into
$\hsigma$-extensions

\subsubsection{The case $\rho>1,\sigma=1$}
In this case regardless of the type of $\cF$ we have
$\cF^\hrho=\cF^\delta$ where $\delta=1-\Theta(\rho^{-1})$. We will
consider the different fiber types separately.

Suppose $\cF$ is of type $D$. Up to rescaling $\cF=D(1)$.  Then we
take $E=\he1$. Up to rescaling the fiber, our goal to cover $D(1)$ by
discs whose $\Omega(1)$-extensions remain in
$D(\delta^{-1})=D(1+\Omega(\rho^{-1}))$. One can use for example any
disc of radius $O(\rho^{-1})$ centered in $D(1)$; clearly
$\poly(\rho)$ such discs suffice to cover $D(1)$.

Suppose $\cF$ is of type $D_\circ$. Up to rescaling
$\cF=D_\circ(1)$. Again we take $E=\he1$. We first embed
$D_\circ(O(1))$ in $D_\circ(1)$ choosing the radius in such a way that
the $\he1$-extension remains in $D_\circ(1)$. It remains to cover the
annulus $A(\Omega(1),1)$ using $\poly(\rho)$ discs whose
$\Omega(1)$-extensions remain in $D_\circ(1)$. This can be done as in
the case $\cF=D(r)$.

Suppose $\cF$ is of type $A$. Up to rescaling $\cF=A(r,1)$. We take
$E=\he1\he{\hat\rho}$ where $\hat\rho$ will be chosen later. By the
fundamental lemma for $\D\setminus\{0\}$ applied to $r$ on the cell
$\cC^{\he1}$ one of the following holds
\begin{align}\label{eq:fund-r-v1}
  \log |r(\cC^{\he1})| &\subset (-\infty,-\Omega_\ell(1/\hat\rho)), & \diam(\log|\log|r(\cC^{\he1})||;\R)&=O_\ell(\hat\rho).
\end{align}
We will use the following simple lemma.
\begin{Lem}\label{lem:annulus-cover-v1}
  Let $\alpha\in(0,1]$ (resp. $\beta\in[1,\infty)$) and suppose
  $r<\alpha$ (resp. $r\beta<1$) uniformly in $\cC^{\he1}$. Then one
  can cover $\cC\odot S^{1-\Omega(1/\rho)}(\alpha)$
  (resp. $\cC\odot S^{1-\Omega(1/\rho)}(\beta r)$) by $\poly(\rho)$
  cells whose $\he1$-extensions remain in $(\cC\odot\cF)^\hrho$.
\end{Lem}
\begin{proof}
  The $\he1$-extension of a disc of radius $O(\alpha/\rho)$ around a
  point of $S(\alpha)$ remains in
  $S^{1-\Omega(1/\rho)}(\alpha)$. Moreover for any point in
  $\cC^{\he1}$ this remains in $A^\hrho(r,1)$ since $r<\alpha$. We
  choose a collection of $\poly(\rho)$ such discs to cover
  $S^{1-\Omega(1/\rho)}(\alpha)$. Then $\cC\odot D_i$ gives the
  required covering. The respective case is similar, with $S(\alpha)$
  replaced by $S(\beta r)$.
\end{proof}
  
Suppose that we are in the first case of~\eqref{eq:fund-r-v1}.  Inside
$A(r,1)$ we choose the annulus $A$ such that $A^{\he1}=A(r,1)$: this
is possible if we choose $\hat\rho=\Omega(1)$. It remains to cover the
two annuli components of $A(r,1)\setminus A$. Each of these has width
$O(1)$ in the logarithmic scale. Lemma~\ref{lem:annulus-cover-v1}
allows us to cover subannuli of width $\Omega(1/\rho)$: with $\alpha$
for the outer component and with $\beta$ for the inner
component. Clearly $\poly(\rho)$ applications of the lemma suffice to
cover the two components.

Suppose that we are in the second case of~\eqref{eq:fund-r-v1}.  If we
choose $\hat\rho=\Omega(1)$ we may assume that the ratio of $\log |r|$
between any two points of $\cC^{\he1}$ is in $(9/10,10/9)$. We again
distinguish two cases: first, suppose that we can uniformly over
$\cC^{\he1}$ choose inside $A(r,1)$ the annulus $A$ such that
$A^{\he1}=A(r,1)$. In this case we can proceed as above. Otherwise,
over some point in $\cC^{\he1}$ we have $r(\vz_{1..\ell})=r_0$ with
$\log|r_0|=-O(1)$. We now apply Lemma~\ref{lem:annulus-cover-v1} to
cover the annulus given in the logarithmic scale by $(2/3\log|r_0|,0)$
(using $\alpha$) and the annulus given in the logarithmic scale by
$(\log|r|,\log|r|-2/3\log|r_0|)$ (using $\beta$). Crucially, the ratio
condition on $\log|r|$ in $\cC^{\he1}$ ensures that these two annuli
remain in $A(r,1)$ and cover it uniformly over $\cC^{\he1}$. Since the
width of each of these annuli is $O(1)$ we see as above that
$\poly(\rho)$ applications of the lemma suffice to cover them.

\subsubsection{The case $\rho=1,\sigma<1$}
In this case we have $\cF^\hsigma=\cF^\e$ where
$\e\sim\sigma$ in type $D$ and $\e= e^{-\Theta(1/\sigma)}$ in types
$D_\circ,A$. We will consider the different fiber types separately.

Suppose $\cF$ is of type $D$. Up to rescaling $\cF=D(1)$. Then we take
$E=\hsigma$. Up to rescaling the fiber, our goal to cover $D(1)$ by
discs whose $\Omega(\sigma)$-extensions remain in $D(1)^{\he1}$. One
can use for example any disc of radius $O(\sigma)$ centered in $D(1)$;
clearly $\poly(1/\sigma)$ such discs suffice to cover $D(1)$.

Suppose now that $\cF$ is of type $D_\circ$ or $A$. Up to rescaling
$\cF=D_\circ(1)$ or $\cF=A(r,1)$. We will use the following analog of
Lemma~\ref{lem:annulus-cover-v1}. The $\beta$-case of the lemma is
valid only for type $A$.
\begin{Lem}\label{lem:annulus-cover-v2}
  Let $\alpha\in(0,1]$ (resp. $\beta\in[1,\infty)$) and suppose
  $r<\alpha$ (resp. $r\beta<1$) uniformly in $\cC^\hsigma$. Then one
  can cover $\cC\odot S^{\Omega(1)}(\alpha)$
  (resp. $\cC\odot S^{\Omega(1)}(\beta r)$) by $\poly(1/\sigma)$
  cells whose $\hsigma$-extensions remain in $(\cC\odot\cF)^{\he1}$.
\end{Lem}
\begin{proof}
  The $\he{\Omega(1)}$-extension of a disc of radius $\Omega(\alpha)$
  around a point of $S(\alpha)$ remain in $S^{\Omega(1)}(\alpha)$. We
  choose a collection of $O(1)$ such discs to cover
  $S^{\Omega(1)}(\alpha)$. By what was already proved for discs, we
  can further cover each of the discs by $\poly(1/\sigma)$ discs whose
  $\hsigma$-extensions remain in $S^{\Omega(1)}(\alpha)$. Moreover for
  any point in $\cC^\hsigma$ this remains in $\cF^{\he1}$ (in type $A$
  we use that $r<\alpha$). If $\{D_i\}$ denotes the collection of all
  discs obtained in this process then $\cC\odot D_i$ gives the
  required covering. The respective case is similar, with $S(\alpha)$
  replaced by $S(\beta r)$.
\end{proof}

Suppose $\cF=D_\circ(1)$. We take $E=\hsigma$. We first embed
$D_\circ(e^{-\Theta(1/\sigma)})$ in $D_\circ(1)$ so that the
$\hsigma$-extensions remains in $D_\circ(1)$. It remains to cover the
annulus $A(e^{-\Theta(1/\sigma)},1)$, or in the logarithmic scale
$(-\Theta(1/\sigma),0)$. Lemma~\ref{lem:annulus-cover-v2} allows us to
cover subannuli of logarithmic width $\Omega(1)$, and indeed
$\poly(1/\sigma)$ applications suffice to cover the annulus.

Finally suppose $\cF=A(r,1)$. We take $E=\hsigma\he{\hat\rho}$ where
$\hat\rho$ will be chosen later. By the fundamental lemma for
$\D\setminus\{0\}$ applied to $r$ on the cell $\cC^\hsigma$ one of the
following holds
\begin{align}\label{eq:fund-r-v2}
  \log |r(\cC^\hsigma)| &\subset (-\infty,-\Omega_\ell(1/\hat\rho)), & \diam(\log|\log|r(\cC^\hsigma)||;\R)&=O_\ell(\hat\rho).
\end{align}
  
Suppose that we are in the first case of~\eqref{eq:fund-r-v2}.  Inside
$A(r,1)$ we choose the annulus $A$ (uniformly over $\cC^\hsigma$) such
that $A^\hsigma=A(r,1)$: this is possible if
$\log|r|=-\Omega(1/\sigma)$, i.e. if we choose
$\hat\rho=\Omega(\sigma)$. It remains to cover the two annuli
components of $A(r,1)\setminus A$. Each of these has width
$O(1/\sigma)$ in the logarithmic
scale. Lemma~\ref{lem:annulus-cover-v2} allows us to cover subannuli
of width $\Omega(1)$: with $\alpha$ for the outer component and with
$\beta$ for the inner component. Clearly $\poly(1/\sigma)$
applications of the lemma suffice to cover the two components.

Suppose that we are in the second case of~\eqref{eq:fund-r-v2}.  If we
choose $\hat\rho=\Omega(1)$ we may assume that the ratio of $\log |r|$
between any two points of $\cC^{\he1}$ is in $(9/10,10/9)$. We again
distinguish two cases: first, suppose that we can uniformly over
$\cC^\hsigma$ choose inside $A(r,1)$ the annulus $A$ such that
$A^\hsigma=A(r,1)$. In this case we can proceed as above. Otherwise,
over some point in $\cC^\hsigma$ we have $r(\vz_{1..\ell})=r_0$ with
$\log|r_0|=-O(1/\sigma)$. We now apply
Lemma~\ref{lem:annulus-cover-v2} to cover the annulus given in the
logarithmic scale by $(2/3\log|r_0|,0)$ (using $\alpha$) and the
annulus given in the logarithmic scale by
$(\log|r|,\log|r|-2/3\log|r_0|)$ (using $\beta$). Crucially, the ratio
condition on $\log|r|$ in $\cC^\hsigma$ ensures that these two annuli
remain in $A(r,1)$ and cover it uniformly over $\cC^\hsigma$. Since
the logarithmic width of each of these annuli is $O(1/\sigma)$ we see
as above that $\poly(1/\sigma)$ applications of the lemma suffice to
cover them.

\subsubsection{The real case}

Only a very minor modification is needed in order to treat the real
case. If $\cC$ is real then all the rescalings performed during the
proof are also real. Wherever we construct a collection of discs to
cover a domain $\cF$ by discs in the complex case, we now choose a
collection of discs with real centers to cover $\R_+\cF$. This ensures
that the cells that we construct are real, and the rest of the proof
remains unchanged.

\subsubsection{Uniformity over families and Remark~\ref{rem:refinement-vs-monom}}

The statement regarding uniformity over families follows by inspection
of the proof. By induction the refinement maps chosen for the base can
all be chosen from a single definable family. The covering constructed
for the fiber, after a uniform rescaling, consists of discs or annuli
with a constant center and radius. The family of all such discs or
annuli is certainly definable.

Remark~\ref{rem:refinement-vs-monom} follows immediately from the
proof: it suffices to note that $D$ fibers are always covered by $D$
fibers, and $D_\circ,A$ fibers are covered by a collection of $D$
fibers and possibly one additional fibers of type $D_\circ,A$; and
this additional fiber is always centered at zero, and hence homotopy
equivalent to the original fiber by the injection map.

\subsection{Monomial cells}

Let $\cC$ be a cell of length $\ell$. An \emph{admissible monomial} on
$\cC$ is a function of the form $c\cdot\vz^\valpha$ where $c\in\C$ and
$\valpha=\valpha(f)$ is the associated monomial of some function
$f\in\cO_b(\cC)$.

\begin{Def}[Monomial cell]
  We will say that a cell $\cC$ is \emph{monomial} if $\cC=*$; or if
  $\cC=\cC_{1..\ell}\odot\cF$ and the radii involved in $\cF$ are
  admissible monomials on $\cC_{1..\ell}$.
\end{Def}

Using the refinement theorem and the monomialization lemma we prove
the following proposition.

\begin{Prop}\label{prop:monomial-cover}
  In the conclusion of the refinement theorem one can further require
  that each $\cC_j$ is a monomial cell.
\end{Prop}

Proposition~\ref{prop:monomial-cover} implies that we could use
monomial cells, rather than general complex cells, as our standard
models for cellular covers. In some cases this is more convenient, as
the monomial cells have a more transparent combinatorial and algebraic
structure. However for the most part we have chosen in this paper to
state our constructions for general complex cells. 

The proof of Proposition~\ref{prop:monomial-cover} will occupy the
remainder of this section. We start by applying the refinement theorem
to $\cC$ with some
$\he{\hat\sigma}=\he{\hat\sigma_1}\he{\hat\sigma_2}$ to be chosen
later. Recall that all of the cells constructed in the refinement
theorem can be chosen from one definable family independent of
$\hat\sigma$. By the monomialization lemma, if $r(\vz)$ is any of the
radii involved in the definition of one of these cells $\cC_j$ then we
have $r(\vz)=\vz^{\valpha(r)}U(\vz)$ where
\begin{equation}\label{eq:monomial-cell-U}
  \begin{aligned}
    \diam(\Re\log U(\cC_j^{\he{\hat\sigma_2}});\R)&< O_\cC(\hat\sigma_1) \\
    \diam(\Re\log U(\cC_j^{\he{\hat\sigma_2}});\R)&< \poly_\ell(\beta)\cdot\hat\sigma_1  
  \end{aligned}
\end{equation}
in the subanalytic and algebraic cases respectively. We will now
construct a monomial cell $\tilde\cC_j^\hsigma$ such that
$\cC_j\subset\tilde\cC_j\subset\tilde\cC_j^\hsigma\subset\cC^{\he{\hat\sigma}}$,
and the identity map then gives the covering of $\cC_j$ by a monomial
cell as required.

We construct $\tilde\cC_j$ by induction. If $\cC_j$ is of length zero
or one then it is already monomial so we may take
$\tilde\cC_j:=\cC_j$. If $\cC_j:=(\cC_j)_{1..\ell}\odot\cF$ then we
set $\tilde\cC_j:=\widetilde{(\cC_j)_{1..\ell}}\odot\tilde\cF$ where
$\tilde\cF$ is defined as follows. Suppose $\cF=A(r_1,r_2)$ and write
$r_1=\vz^{\valpha_1}U_1$ and $r_2=\vz^{\valpha_2}U_2$. Then
$\tilde\cF=A(\tilde r_1,\tilde r_2)$ where
\begin{align}
  \tilde r_1&=\vz^{\valpha_1} \min_{\vz_{1..\ell}\in\widetilde{(\cC_j)_{1..\ell}^\hsigma}}|r_1(z)| &
  \tilde r_2&=\vz^{\valpha_1} \max_{\vz_{1..\ell}\in\widetilde{(\cC_j)_{1..\ell}^\hsigma}}|r_2(z)|.
\end{align}
It is clear that $\cC_j\subset\tilde\cC_j$, and what remains to be
verified is that
$\tilde\cC_j^\hsigma\subset\cC_j^{\he{\hat\sigma}}$. By~\eqref{eq:monomial-cell-U},
for an appropriate choice of $\hat\sigma_1$ we can make
$\diam(\Re\log U(\cC_j^{\he{\hat\sigma_2}});\R)<1$, which implies that
$\tilde r_1/r_1\ge1/e$ and $\tilde r_2/r_1<e$ on
$\cC_j^{\he{\hat\sigma_2}}$. Now choosing $\hat\sigma_2<\sigma$ and
also $\he{\hat\sigma}<\hsigma/e$ we have indeed
$\tilde\cC_j^\hsigma\subset\cC_j^{\he{\hat\sigma}}$.


\subsection{Clustering in fibers of proper covering maps}

\subsubsection{The general (complex) setting}
\label{sec:cover-clustering}

Let $\cC^\hrho$ be a cell and let $Z\subset\cC^\hrho\times\C$ be an
analytic set such that the natural projection $\pi:Z\to\cC^\hrho$ is a
proper covering map. Let $\nu$ denote the degree of $\pi$ and set
$\hat\cC:=\cC_{\times\nu!}$. Let $\hat Z\subset\hat\cC\times\C$ be the
pullback $(R_{\nu!},\id)^*Z$. Then the sections $y_j:\hat\cC\to\C$ of
$\hat Z$ over $\hat\cC$ are univalued, and we denote their collection
by $\Sigma$. Each section $y_j$ in fact extends holomorphically to the
cell $\hat\cC^{\he{\nu!\cdot\rho}}$ by
Proposition~\ref{prop:ext-v-cover}, but this exponential (in $\nu$)
loss in the size of the extension is too coarse for our
purposes. Instead, we denote by $\nu_j\le\nu$ the size of the
$\pi_1(\cC)$-orbit of $y_j$ thought of as a multivalued section over
$\cC$.

\begin{Lem}\label{lem:y_j-monodromy}
  Let $y_j\in\Sigma$. Then $y_j$ is already univalued as a section
  over the cover $\hat\cC_j:=\cC_{\times\nu_j}$, and extends
  holomorphically to $\hat\cC_j^{\he{\nu_j\cdot\rho}}$.
\end{Lem}
\begin{proof}
  The group $\pi_1(\cC)$ is abelian.  It is enough to check that the
  for any $g\in\pi_1(\cC)$ we have $g^{\nu_j}(y_j)=y_j$. This is
  elementary: the $\<g\>$-action induces a partition of the orbit
  $\pi_1(\cC)\cdot y_j$ of size $\nu_j$ into $\<g\>$-orbits, and
  $\pi_1(\cC)$ acts transitively on these orbits. Thus they are all of
  the same size, which in particular divides $\nu_j$, and this is true
  in particular for the orbit $\<g\>\cdot y_j$.
\end{proof}

Let $y_i,y_j,y_k\in\Sigma$ be three distinct sections. Since $\pi$ is
unramified, the sections are pairwise distinct over any point of
$\hat\cC$. We let $\nu_{i,j,k}:=\lcm(\nu_i,\nu_j,\nu_k)$ and
$\hat\cC_{i,j,k}:=\cC_{\times\nu_{i,j,k}}$. We define a map
$s_{i,j,k}$ as follows,
\begin{equation}
  s_{i,j,k}:\hat\cC_{i,j,k}^{\he{\nu_{i,j,k}\cdot\rho}}\to\C\setminus\{0,1\}, \qquad s_{i,j,k}=\frac{y_i-y_j}{y_i-y_k}.
\end{equation}
By the fundamental lemma for maps into $\C\setminus\{0,1\}$ one of the
following holds:
\begin{equation}\label{eq:sijk-fund}
  \begin{gathered}
    s_{i,j,k}(\hat\cC_{i,j,k})\subset B(\{0,1,\infty\},e^{-\Omega_\ell(1/(\nu^3\rho))};\C P^1), \\
    \diam(s_{i,j,k}(\hat\cC_{i,j,k});\C\setminus\{0,1\})=O_\ell(\nu^3\rho).
  \end{gathered}
\end{equation}
We remark that equivalently~\eqref{eq:sijk-fund} holds if we replace
$\hat\cC_{i,j,k}$ by $\hat\cC$, since $s_{i,j,k}$ on $\hat\cC$ factors
through $\hat\cC_{i,j,k}$.

Fix $y_i\in\Sigma$. We will cluster the remaining sections into annuli
according to their relative distances from $y_i$, which are expressed
by the quantities $s_{i,j,k}$. Since these quantities are invariant
under affine transformations of $\C$, we may assume for simplicity of
the notation that $y_i=0$. We record a useful corollary
of~\eqref{eq:sijk-fund} in this normalization.

\begin{Lem}\label{lem:yj-v-yk}
  Suppose $\rho=O_\ell(1/\nu^3)$. Let $j,k\neq i$ and write
  $R:=\log|y_j/y_k|$. One of the following holds:
  \begin{equation}\label{eq:yj-v-yk}
    \begin{gathered}
      R\rest{\hat\cC} < -\Omega_\ell(1/\nu^3\rho) \quad\text{or}\quad  R\rest{\hat\cC} > \Omega_\ell(1/\nu^3\rho), \\
    \diam(R(\hat\cC),\R) = O_\ell(\nu^3\rho) \\
    \frac{\max_{\vz\in\hat\cC} R(z)}{\min_{\vz\in\hat\cC} R(z)} < 1+O_\ell(\nu^3\rho).
    \end{gathered}
  \end{equation}
\end{Lem}
\begin{proof}
  If at some point in $\hat\cC$ we have $y_j/y_k\in A(1/2,2)$ then for
  $\rho=O(1)$ we have by~\eqref{eq:sijk-fund} that
  $y_j/y_k(\hat\cC)\subset A(1/4,4)$. In this domain the
  $\log|\cdot|$-distance is bounded up to constant by the
  $\C\setminus\{0,1\}$ distance so
  $\diam(R(\hat\cC);\R)=O_\ell(\nu^3\rho)$. It remains to consider the
  case $y_j/y_k(\hat\cC)\subset D_\circ(1/2)$ (or $A(2,\infty)$ which
  is the same up to inversion). In the first case
  of~\eqref{eq:sijk-fund} we have $R=-\Omega_\ell(1/(\nu^3\rho))$. In
  the second case of~\eqref{eq:sijk-fund}, since the
  $\C\setminus\{0,1\}$ metric is equivalent to the $D_\circ(1)$ metric
  in $D_\circ(1/2)$ we have (as in the fundamental lemma for
  $D_\circ$) the estimate
  \begin{equation}
    \diam(\log|R(\hat\cC)|;\R) = O_\ell(\nu^3\rho).
  \end{equation}
  With our assumption on $\rho$ this means that $R(\hat\cC)$ varies
  multiplicatively by a factor of size at most $1+O_\ell(\nu^3\rho)$.
\end{proof}

We fix a quantity $0<\gamma<1$ which we call the \emph{gap}. Pick an
arbitrary point $p\in\hat\cC$ and let
\begin{equation}
  S_i:=\{\log|y_j(p)|:y_j\in\Sigma, y_j\neq y_i\}\subset\R.
\end{equation}
We say that two points $s,s'$ in $S_i$ belong to the same cluster if
they are connected in the transitive closure of the relation
$|s-s'|<5|\log\gamma|$. We order the clusters with respect to $<$ on
$\R$. For cluster number $q=1,\ldots,m_i$ we let $I_{i,q}$ denote the
minimal closed interval containing the $2|\log\gamma|$-neighborhood of
the cluster. For each $q$ we arbitrarily choose
$\hat y_{i,q}\in\Sigma$ to be one of the sections with
$\log|\hat y_{i,q}(p)|\in I_{i,q}$, and call it the \emph{center} of
the cluster. We define
\begin{align}\label{eq:cluster-edge-bounds}
  \gamma^{5\nu}<&l_{i,q}<\gamma^2, & \gamma^{-2}<&r_{i,q}<\gamma^{-5\nu}
\end{align}
by the condition
\begin{equation}
  e^{I_{i,q}} = [l_{i,q}|\hat y_{i,q}(p)|,r_{i,q}|\hat y_{i,q}(p)|].
\end{equation}
We also fix some $\delta<1$ arbitrarily close to $1$ (merely to ensure
that the boundary circles below are covered).

We define three types of fibers over $\hat\cC$ as follows:
\begin{gather}
  \cF_{i,q}=A^\delta(l_{i,q}\hat y_{i,q},r_{i,q}\hat y_{i,q}) \qquad q=1,\ldots,m_i \\
  \cF_{i,q+}= A^\delta(r_{i,q}\hat y_{i,q},l_{i,q+1}\hat y_{i,q+1}), \qquad q=1,\ldots,m_i-1 \\
  \cF_{i,0+}=D_\circ^\delta(l_{i,1}\hat y_{i,1}), \qquad \cF_{i,m+} = A^\delta(r_{i,q}\hat y_{i,q},\infty).
\end{gather}
Note that each of these fibers actually depends only on
$y_i,\hat y_{i,q}$ for $\cF_{i,q}$ and on
$y_i,\hat y_{i,q},\hat y_{i,q+1}$ for $\cF_{i,q+}$. Thus they actually
factor as domains defined over a cover $\hat\cC_{i,j,k}$ for an
appropriate choice of the indices. We denote this cover by
$\hat\cC_{i,q}$ or $\hat\cC_{i,q+}$.

The key properties of these domains are summarized in the following
proposition.
\begin{Prop}\label{prop:clusters}
  Suppose $1/\rho>\poly_\ell(\nu,|\log\gamma|)$. Then the following hold
  uniformly over $\hat\cC$:
  \begin{enumerate}
  \item The fibers $\cF_{i,q},\cF_{i,q+}$ are well-defined and cover
    $\C\setminus\{0\}$.
  \item The domains $\cF_{i,q}^\gamma\setminus\cF_{i,q}$ do not contain any
    of the points $y_j$ for $q=1,\ldots,m_i$.
  \item The domains $\cF_{i,q+}^\gamma$ do not contain any of the
    points $y_j$ for $q=0,\ldots,m_i$.
  \end{enumerate}
\end{Prop}
\begin{proof}
  The only non-trivial assertion in the first statement is that
  $\cF_{i,q+}$ is well-defined, i.e. that
  $r_{i,q}\hat y_{i,q}<l_{i,q+1}\hat y_{i,q+1}$ uniformly over
  $\hat\cC$. At $p$ we have by construction
  \begin{equation}
    \log |\hat y_{i,q+1}(p)/\hat y_{i,q}(p)| > \log(r_{i,q}/l_{i,q+1})+|\log\gamma|.
  \end{equation}
  We need to prove that
  \begin{equation}
    \log |\hat y_{i,q+1}/\hat y_{i,q}| > \log(r_{i,q}/l_{i,q+1})
  \end{equation}
  uniformly over $\hat\cC$. This follows easily from
  Lemma~\ref{lem:yj-v-yk} for an appropriate choice of $\rho$. The
  only non-trivial case is the last one, which follows if one recalls
  that $\log(r_{i,q}/l_{i,q+1})<10\nu|\log\gamma|$.
  
  The third statement follows from the second: over $p$ all the point
  $y_j$ except $y_i=0$ lie in the domains $\cF_{i,q}$, and assuming
  that they never cross into the boundaries
  $\cF_{i,q}^\gamma\setminus\cF_{i,q}$ this remains true uniformly
  over $\hat\cC$. We proceed to the proof of the second statement. We
  will show that $y_j$ does not belong to the outer boundary of
  $\cF_{i,q}^\gamma\setminus\cF_{i,q}$ (the case of the inner boundary
  is similar). By construction over $p$ one of the following holds:
  \begin{align}
     \log|y_j(p)/\hat y_q(p)|&< \log r_{i,q}-2|\log\gamma|, &  \log|y_j(p)/\hat y_q(p)|&> \log r_{i,q}+3|\log\gamma|.
  \end{align}
  We must prove that one of
  \begin{align}
     \log|y_j/\hat y_q|&< \log r_{i,q}, &  \log|y_j/\hat y_q|&> \log r_{i,q}+|\log\gamma|,
  \end{align}
  holds uniformly over $\hat\cC$. This follows in the same manner as
  the first statement.
\end{proof}

Since the sections $y_j$ do not meet the $\cF_{i,q+}$ and
$\partial\cF_{i,q}$, it follows that each section $y_j\neq y_i$
uniformly lies in a single $\cF_{i,q}$. We say that such sections
belong to the cluster $\cF_{i,q}$.

\begin{Rem}\label{rem:single-cluster}
  If we focus our attention on a single cluster $\cF_{i,q}$ then it is
  convenient, up to an affine transformation over $\hat\cC_{i,q}$, to
  assume that both $y_i=0$ and $\hat y_{i,q}=1$. With $\gamma,\rho$ as
  in Proposition~\ref{prop:clusters} we then have
  $\cF_{i,q}=A^\delta(l_{i,q},r_{i,q})$. For any $y_j$ in the
  $\cF_{i,q}$ cluster our choice of $\rho$ ensures that the second
  option of~\eqref{eq:sijk-fund} holds, and we have
  $\diam(y_j(\hat\cC_{i,q}),\C\setminus\{0,1\})=O(\nu^3\rho)$.
\end{Rem}

\subsection{The real setting}
\label{sec:cover-clustering-real}

Suppose that $\cC^\hrho$ is a real cell and
$Z\subset\cC^\hrho\times\C$ is real, i.e. invariant under
$\vz\to\bar\vz$. In this case we would like to construct the fibers
$\cF_{i,q},\cF_{i,q+}$ to be real as well. However, the construction
above produces fibers whose centers and radii are given in terms of
the sections $y_j$, which are generally not real. We now modify this
construction to produce real fibers. We fix $p\in\R\hat\cC$.

The pullbacks $\hat\cC,\hat Z$ of $\cC,Z$ by $R_{\nu!}$ are also
real. Since $\hat Z$ is a cover, each section $y_j$ is
either always real or always non-real on $\R\hat\cC$. We denote the
former sections by $\Sigma_\R$ and the latter by $\Sigma_\C$. If
$y_j\in\Sigma$ then by the symmetry of $\hat Z$ there exists a
symmetric section $y_{\bar j}\in\Sigma$ defined by
$y_{\bar j}(\vz):=\overline{y_j(\bar\vz)}$.

\begin{Def}
  Let $y_j\in\Sigma_\C$. Making a real affine transformation we pass
  to a chart where $y_j(p)=i,y_{\bar j}(p)=-i$. Then the pair
  $y_j,y_{\bar j}$ is called \emph{admissible} if $D(1/10)$ contains
  none of the points $y_k(p)$ for $y_k\in\Sigma$.
\end{Def}
An \emph{admissible center} is a map of the form $(y_j+y_{\bar j})/2$
where $y_j\in\Sigma_\R$ or $y_j,y_{\bar j}\in\Sigma_\C$ is an
admissible pair. Every admissible center is real on $\R\hat\cC$. We
denote the set of admissible centers by $\{c_1,\ldots,c_t\}$.
\begin{Lem}\label{lem:select-admissible}
  Let $y_j\in\Sigma_\C$ and by a real affine transformation pass to a
  chart where $y_j(p)=i,y_{\bar j}(p)=-i$. Then $D(1/5)$ contains
  $c_l(p)$ for an admissible center $c_l$.
\end{Lem}
\begin{proof}
  If $y_j,y_{\bar j}$ is admissible then the claim is
  obvious. Otherwise there exists a section
  $y_k(p)\in D(1/10)$. Rescaling to make $y_k(p)=i$ we see that it
  will be enough to prove the claim for $y_k$. Clearly this process
  must terminate after no more than $\nu$ steps.
\end{proof}

The motivation for the notion of admissibility comes from the
following lemma.

\begin{Lem}\label{lem:admissible-collision}
  Suppose $1/\rho=\poly_\ell(\nu)$. For every section $y_j$ and
  admissible center $c_l$ we have $c_l\neq y_j$ uniformly over
  $\hat\cC$, unless $c_l=y_j\in\Sigma_\R$.
\end{Lem}
\begin{proof}
  If $c_l=y_k\in\Sigma_\R$ for some $k$ then the claim follows from
  the fact that $\hat Z$ is a cover whose sections are pairwise
  distinct over any point of $\hat\cC$. Assume therefore that
  $c_l=(y_k+y_{\bar k})/2$ for $y_k\in\Sigma_\C$. We make a real
  affine transform to pass to a chart where $y_k(p)=i$. Suppose toward
  contradiction that at some point $\tilde p\in\hat\cC$ we have
  $y_j=(y_k+y_{\bar k})/2$, i.e. $s_{j,k,\bar k}(\tilde p)=-1$. Then
  by~\eqref{eq:sijk-fund} we have
  $|s_{j,k,\bar k}(p)+1|=O_\ell(\nu^3\rho)$. For a suitable choice of
  $\rho$ this readily implies that $|y_j(p)|<1/10$, contradicting the
  admissibility of $c_l$.
\end{proof}

Lemma~\ref{lem:admissible-collision} implies that we can define maps
$\tilde s_{l,j,k}$ as follows
\begin{equation}
  \tilde s_{l,j,k}:\hat\cC_{l,j,k}^{\he{\nu_{l,j,k}\cdot\rho}}\to\C\setminus\{0,1\}, \qquad s_{i,j,k}=\frac{c_l-y_j}{c_l-y_k}.
\end{equation}
Fix an admissible center $c_l$. We will cluster the sections $y_j$
into annuli according to their relative distances from $c_l$ in
analogy with the complex construction. As before, we make a real
affine transformation and assume that $c_l=0$.

We define the intervals $I_{l,q}$ in the same way as in the complex
setting. However, a small variation is needed in the choice of
$\hat y_{l,q}$, which must be real on $\R\hat\cC$ to maintain the real
structure of our construction. Let $y_j$ be one of the sections with
$y_j(p)\in I_{l,q}$. We define
$\hat y_{l,q}:=\sqrt{y_jy_{\bar j}}$. Clearly $\hat y_{l,q}$ is real
on $\R\hat\cC$. To see that it is univalued on $\hat\cC$ note that
$y_j$ and $y_{\bar j}$ have the same associated monomial by symmetry,
hence $y_jy_{\bar j}$ has an even associated monomial and admits a
univalued square root.

With $\hat y_{l,q}$ chosen as above we continue the construction as in
the complex case. To verify that all the arguments remain valid we
need an analog of Lemma~\ref{lem:yj-v-yk} where either $y_j$ or $y_k$
(or both) are replaced by a cluster center
$\hat y_{l,q}:=\sqrt{y_hy_{\bar h}}$. This follows essentially from
the same lemma applied to $y_h$ and to $y_{\bar h}$, and we leave the
details for the reader. As a consequence we see that
Proposition~\ref{prop:clusters} and Remark~\ref{rem:single-cluster}
continue to hold, with the fibers $\cF_{i,q}$ and $\cF_{i,q+}$ now
real on $\R\hat\cC$.


\section{Cellular Weierstrass Preparation Theorem}
\label{sec:weierstrass}

In this section we state and prove a cellular analog of the
Weierstrass preparation theorem. We stress that unlike the classical
theorem, the cellular version does not involve a change in the order
of coordinates. 

\begin{Def}
  Let $\gamma>0$ and $\cC\odot\cF^\gamma$ be a cell and let
  $F\in\cO_b(\cC\odot\cF^\gamma)$. We say that $\cC\odot\cF^\gamma$ is
  a \emph{Weierstrass cell} with \emph{gap} $\gamma$ for $F$ if:
  \begin{itemize}
  \item $F$ vanishes identically on $\cC\odot\cF^\gamma$, or
  \item $F$ is non-vanishing on $\cC\odot*$ if $\cF=*$, or
  \item $F$ is non-vanishing on $\cC\odot(\cF^\gamma\setminus\cF)$ if
    $\cF=D,D_\circ,A$.
  \end{itemize}

  If $\hat\cC$ is a cell and $F\in\cO_b(\hat\cC)$ we say that a
  cellular map $f:\cC\odot\cF^\gamma\to\hat\cC$ is Weierstrass with
  gap $\gamma$ for $F$ if $\cC\odot\cF^\gamma$ is a Weierstrass cell
  with gap $\gamma$ for $f^*F$.
\end{Def}

\begin{Thm}[Cellular Weierstrass Preparation Theorem (WPT)]\label{thm:wpt}
  Let $\rho,\sigma>0$. Let $\cC^\hrho$ be a (real) cell and
  $F\in\cO_b(\cC^\hrho)$ a (real) function. Then there exist
  $N=\poly_{\cF,F}(\rho,1/\sigma)$ (real) Weierstrass maps
  $f_j:\cC_j^\hsigma\odot\cF_j^\gamma\to\cC^\hrho$ for $F$ with gap
  $\gamma<1$ such that $\cC\subset\cup_j f_j(\cC_j\odot\cF_j)$.

  If $\cC,F$ vary in a definable family then $N,\gamma$ can be taken
  uniform over the family and the maps $f_j$ can be chosen from a
  single definable family. If $\cC,F$ are algebraic of complexity
  $\beta$ then one may take $N=\poly_\ell(\beta,\rho,1/\sigma)$ and
  $\gamma=1-1/\poly_\ell(\beta)$.
\end{Thm}

We will prove the WPT and CPT by induction on $\ell:=\ell(\cC)$ and
$\dim\cC$. The cases $\ell=0$ and $\dim\cC=0$ are vacuous. We now
prove the WPT for a cell $\cC\odot\cF$ of length $\ell+1$ assuming
that the CPT and WPT are true for cells of smaller length or equal
length and smaller dimension. The case $\cF=*$ reduces to the CPT for
$\cC$ so we assume $\cF$ is $D,D_\circ,A$.

We will give two separate proofs: one in the algebraic case, and one
in the analytic case. This is the only part of the proof of the main
theorems where our arguments significantly diverge for these two
cases. 

\subsection{Proof in the algebraic case}
\label{sec:weierstrass-alg-proof}

\subsubsection{Algebraic discriminants}
We will require the following simple lemma on discriminants.

\begin{Lem}\label{lem:discriminant-alg}
  Let $\cC$ be a cell of length $\ell$ and $F\in\cO_b(\cC)$, both
  algebraic of complexity $\beta$. Suppose that $F$ does not vanish
  identically on $\cC$. Then there exist:
  \begin{itemize}
  \item A polynomial $P\in\C[\vz_{1..\ell}]$ of complexity $\poly(\beta)$, not
    identically vanishing on $\cC$, and satisfying
    $\{F=0\}\subset\{P=0\}$.
  \item A polynomial $D\in\C[\vz_{1..\ell-1}]$ of complexity
    $\poly(\beta)$, not identically vanishing on $\cC_{1..\ell-1}$,
    such that the projection 
    \begin{equation}
      \pi:(\cC_{1..\ell}\times\C)\cap\{P=0\}\to\cC_{1..\ell-1}, \qquad \pi(\vz_{1..\ell})=\vz_{1..\ell-1}
    \end{equation}
    is proper covering map outside $\{D=0\}$.
  \end{itemize}
  If $F$ is real then $P,D$ can be chosen real as well.
\end{Lem}
\begin{proof}
  We may assume without loss of generality that the type of $\cC$ does
  not contain $*$, since any such coordinate can be ignored for the
  statement of the lemma. Then $\cC\subset\C^\ell$ is open.

  By definition the graph of $F$ is contained in some irreducible
  algebraic variety $G_F\subset\C^\ell\times\C_w$. Since $F$ is not
  identically vanishing on $\cC$, the equation $w=0$ cuts $G_F$
  properly in a subvariety of dimension $\ell-1$, and the projection
  of this variety to $\C^\ell$ contains $\{F=0\}$ and is contained in
  a proper algebraic hypersurface of degree $\poly(\beta)$, i.e. in a
  set $\{P=0\}$ where $P\in\C[\vz_1..\ell]$ is not identically
  vanishing (on $\cC$, since $\cC$ is open).  We may also assume that
  $P$ is square-free as a polynomial in
  $\C(\vz_{1..\ell-1})[\vz_\ell]$, and in particular has no multiple
  roots for a generic value of $\vz_{1..\ell-1}$. Then the classical
  discriminant $D$ of $P$ satisfies the conditions of the lemma.

  For the final statement, if $F$ is real then its graph is invariant
  under conjugation, and the same is then also true for $G_F$. The
  polynomial $P$ is then also real by construction.
\end{proof}

\subsubsection{Proof of the algebraic WPT}

We now proceed to the proof of the WPT. Suppose $\cF=A(r_1,r_2)$ (the
cases $D(r),D_\circ(r)$ are similar). Applying the refinement theorem,
we may assume that $\rho$ is already as small as we wish as long as
$1/\rho=\poly_\ell(1/\sigma,\beta)$.

There is no harm in replacing $F$ by the polynomial $P$ obtained from
Lemma~\ref{lem:discriminant-alg} applied to the function
$F\cdot(\vz_{\ell+1}-r_1)\cdot(\vz_{\ell+1}-r_2)$. In other words we
may assume without loss of generality that $F$ is a polynomial in the
$\vz_\ell$ variable and that it vanishes when $\vz_\ell$ is either
$0$, $r_1(\vz_{1..\ell})$ or $r_2(\vz_{1..\ell})$. We let $D$ denote
the corresponding discriminant.

We apply the CPT to $\cC$ with $D$ and let
$f_j:\cC^\hrho_j\to\cC^\hrho$ denote the resulting cellular cover. If
$f_j(\cC_j^\hrho)$ is contained in $\{D=0\}$ then since cellular maps
preserve dimension $\dim\cC_j\le\dim\{D=0\}\le\dim\cC-2$. In this case
we set
\begin{align}
  \hat\cC_j&:=\cC_j\odot f_j^*\cF & \hat f_j&:=(f_j,\id):\hat\cC_j^\hrho\to\cC^\hrho
\end{align}
and inductively apply the WPT to $\hat\cC_j$ and $\hat f_j^*F$. We
obtain Weierstrass maps
$f_{j,k}:\cC^\hsigma_{j,k}\odot\cF^\gamma\to\hat\cC_j^\hrho$ for $F$,
and the compositions
$f_{j,k}\circ\hat f_j:\cC^\hsigma_{j,k}\odot\cF^\gamma\to\cC^\hrho$
are Weierstrass maps for $F$ which cover $f(\cC_j)\odot\cF$.

It remains to consider the case that $f_j(\cC_j^\hrho)$ is disjoint
from $\{D=0\}$. In the same way as before, it will suffice to prove
the WPT for $\hat\cC_j$ and $\hat f_j^*F$. We return now to the
original notation replacing this pair by $\cC,F$.  We note that $F$ is
still polynomial in $\vz_\ell$ (though perhaps not in the other
coordinates) and the projection
\begin{equation}
  \pi:(\cC^\hrho\times\C)\cap\{P=0\}\to\cC^\hrho, \qquad \pi(\vz_{1..\ell+1})=\vz_{1..\ell}
\end{equation}
is a proper covering map. It will suffice to prove the WPT under these
conditions. We are now in a position to use the constructions
of~\secref{sec:cover-clustering}. Note that
$\nu=\poly_\ell(\beta)$. We will choose $\gamma=1-1/\poly_\ell(\beta)$
(the precise choice will be determined later).

Recall that we may choose $\rho$ small as long as
$1/\rho=\poly_\ell(1/\sigma,\beta)$. We take
$\hrho<\hsigma\cdot\he{\hat\rho}$ where $\hat\rho$ is chosen in such a
way that Proposition~\ref{prop:clusters} holds over $\cC^\hsigma$. The
zero map is a section of $\pi$ which we denote by $y_0$. Since
$r_1,r_2$ are sections of $\pi$ they belong to certain clusters around
$y_0=0$, say with indices $q_1\le q_2$. We may also assume that
$r_1=\hat y_{q_1}$ and if $q_1\neq q_2$ then $r_2=\hat y_{q_2}$.

If $q_1=q_2=q$ then we define $\tilde\cF=\cF_{0,q}$. If $q_1<q_2$ then we define
\begin{equation}\label{eq:wpt-final-fiber-alg}
  \tilde\cF:= A^\delta(l_{q_1}r_1,r_{q_2}r_2),
\end{equation}
i.e. we take the left endpoint of the $\cF_{0,q_1}$ cluster and the
right endpoint of the $\cF_{0,q_2}$ cluster. Since $r_1,r_2$ are
univalued over $\cC$ this is actually a fiber over $\cC$ and the
Weierstrass condition is provided by Proposition~\ref{prop:clusters}.
The inclusion $\tilde\cF^\gamma\subset\cF^\hrho$ follows
from~\eqref{eq:cluster-edge-bounds} for an appropriate choice of
$\gamma$.

\subsubsection{The real setting}

If $\cC$ and $F$ are real then $P$ is also real, and consequently the
coverings constructed by inductive applications of the CPT can be
taken to be real. After these reductions, the fiber constructed
in~\eqref{eq:wpt-final-fiber-alg} is clearly real as well.

\subsection{Proof in the analytic case}

Before giving the proof of the WPT in the analytic case we develop
some general results concerning the Laurent coefficients of definable
families of holomorphic functions.

\subsection{Laurent domination in definable families}
\label{sec:laurent-domination}

We will study the following property of the Taylor/Laurent
coefficients of a holomorphic function.

\begin{Def}[Taylor domination]
  A holomorphic function $f:D(r)\to\C$ is said to posses the $(p,M)$
  Taylor domination property\footnote{note that we use a slightly
    simplified form of the definition given in
    \protect{\cite{by:domination}}.} \cite{by:domination} if its Taylor
  expansion $f(z)=\sum a_k(z-z_0)^k$ satisfies the estimate
  \begin{equation}
    |a_k| r^k < M \max_{j=0,\ldots,p} |a_j| r^j, \qquad k=p+1,p+2,\ldots 
  \end{equation}
  Similarly, for $r_2>r_1>0$ a holomorphic function $f:A(r_1,r_2)\to\C$
  is said to posses the $(p,M)$ Laurent domination property if its
  Laurent expansion $f(z)=\sum a_k(z-z_0)^k$ satisfies the estimates
  \begin{equation}
    \begin{aligned}
      |a_k| r_2^k &< M \max_{j=-p,\ldots,p} |a_j| r_2^j, \qquad
      k=p+1,p+2,\ldots \\
      |a_k| r_1^k &< M \max_{j=-p,\ldots,p} |a_j| r_1^j, \qquad
      k=-p-1,-p-2,\ldots
    \end{aligned}  
  \end{equation}
\end{Def}

We will need the following lemma on Laurent expansions.

\begin{Lem}\label{lem:fourier-bounds}
  Let $S:=S(1)$ and $f_\lambda:S^{\delta^2}\to\C$ a definable family of holomorphic
  functions for some $0<\delta<1$. Then there exists $B$ such that for
  every $\lambda\in\Lambda$ we have
  \begin{equation}
    \norm{f_\lambda}_{S^\delta} \le B\norm{f_\lambda}_S.
  \end{equation}
  If we write $f_\lambda(z)=\sum a_k(\lambda)z^k$ then there exists
  $p\in\N$ and $m>0$ such that for all $\lambda$ we have
  \begin{equation}
    |a_j(\lambda)| > m \norm{f_\lambda}_S\qquad \text{for some } j=j(\lambda)\in\{-p,\ldots,p\}.
  \end{equation}
\end{Lem}
\begin{proof}
  As we have seen in the proof of Lemma~\ref{lem:monom-dim1}, the
  Voorhoeve index of $f_\lambda$ along any circle in $S^{\delta^2}$ is
  uniformly bounded over $\lambda$ (and over the circle). Fix
  $\lambda$ and assume without loss of generality (up to rotation)
  that the maximum of $f_\lambda$ on $S^\delta$ is attained at
  $\delta i$ or $\delta^{-1} i$. Let $C$ be the circle intersecting
  $i\R$ at $\delta^{\pm 1}i$ and fix some disc $D$ with $C\subset D$
  and $\bar D\subset S^{\delta^2}$. The Voorhoeve index of $f_\lambda$
  over $\partial D$ is uniformly bounded over $\lambda$, and it
  follows from \cite[Theorem~3]{ky:rolle}\footnote{The result there is
    stated for $K$ with nonempty interior, but actually holds for $K$
    of the form $S\cap D$, see \protect{\cite[Section~4.1]{ky:rolle}}.}
  that $B_{S\cap D,D}(f_\lambda)$ is uniformly bounded by some
  constant $B_1$, where
  \begin{equation}
    B_{K,U}(f) = \log\max_{z\in\bar U}|f(z)| - \log\max_{z\in K}|f(z)|.
  \end{equation}
  Thus
  \begin{equation}
    \norm{f_\lambda}_{S^\delta} = \max_{z\in\bar D} |f(z)| \le e^{B_1} \norm{f_\lambda}_S
  \end{equation}
  proving the first part.

  From the first part and the Cauchy estimate it follows that
  \begin{equation}
    |a_k(\lambda)| \le B \delta^{|k|} \norm{f_\lambda}_S.
  \end{equation}
  Then
  \begin{align}
    \norm{f_\lambda}_S &\le \sum_{k\in\Z} |a_k(\lambda)| =
    \sum_{k\in\{-p,\ldots,p\}} |a_k(\lambda)|+\sum_{k\not\in\{-p,\ldots,p\}} |a_k(\lambda)| \le \\
    &\sum_{k\in\{-p,\ldots,p\}} |a_k(\lambda)|+\frac{2B\delta^{p+1}}{1-\delta} \norm{f_\lambda}_S
  \end{align}
  and the result now follows with $p$ such that
  $\frac{2B\delta^{p+1}}{1-\delta}<1/2$ and $m=1/(4p+2)$.
\end{proof}

As a direct corollary of Lemma~\ref{lem:fourier-bounds} we obtain the
following result. We remark that a similar statement for families of
discs appeared (in a slightly different form and with a different
proof) in \cite{cpw:params}. The disc case also follows fairly
directly from a classical theorem of Biernacki
\cite{biernacki}. However the annulus case does not seem to follow in
a similar manner.

\begin{Cor}\label{cor:uniform-indices}
  Let $\cF_\lambda$ be a definable family of discs or annuli and let
  $f_\lambda:\cF_\lambda^\e\to\C$ be a definable family of
  holomorphic functions, for some $0<\e<1$. Then the functions
  $f_\lambda$ have the $(p,M)$ Laurent domination property in $\cF_\lambda$
  for some uniformly bounded $p,M$.
\end{Cor}
\begin{proof}
  We prove the annuli case, the disc case being simpler. Since the
  claim is invariant under rescaling, we may assume without loss of
  generality that $\cF_\lambda$ is of the form $A(r_1,1)$. We need to
  prove
  \begin{equation}
    |a_k| < M \max_{j=-p,\ldots,p} |a_j| , \qquad k=p+1,p+2,\ldots
  \end{equation}
  And indeed by the Cauchy estimates and
  Lemma~\ref{lem:fourier-bounds} with $\delta^2=\e$ we have
  \begin{equation}
    |a_k(\lambda)| \le B \norm{f_\lambda}_s \delta^{k} \le \frac B m |a_j(\lambda)| \qquad \text{for
      some } j\in\{-p,\ldots,p\}.
  \end{equation}
  For $k=-p-1,\ldots$ we proceed similarly, assuming now that
  $\cF_\lambda$ is of the form $A(1,r_2)$.
\end{proof}

We record a simple consequence of the Taylor domination property.

\begin{Prop}\label{prop:residue-domination}
  Let $A:=A(r_1,r_2)$ and $0<\delta<1$. Let $f:A^\delta\to\C$ be a
  holomorphic function with $(p,M)$ Taylor domination. Write
  \begin{equation}
    f(z)=\sum_{j=-p}^p a_j z^j + R_p(z).
  \end{equation}
  Then for $z\in A$ we have
  \begin{equation}\label{eq:residue-ineq}
    |R_p(z)| < \frac{2\delta}{1-\delta} M \max_{j=-p,\ldots,p} |a_j z^j|.
  \end{equation}
\end{Prop}
\begin{proof}
  The Taylor domination property in $A^\delta$ gives
  \begin{equation}
    \begin{aligned}
      |a_k| r_2^k &< \delta^{k-p} M \max_{j=-p,\ldots,p} |a_j| r_2^j , \qquad
      k=p+1,p+2,\ldots \\
      |a_k| r_1^k  &< \delta^{-k-p} M \max_{j=-p,\ldots,p} |a_j| r_1^j, \qquad
      k=-p-1,-p-2,\ldots
    \end{aligned}  
  \end{equation}
  and the same clearly holds with $r_1,r_2$ replaced by $z\in A$.
  Summing over $k=p+1,\ldots,\infty$ and $k=-p-1,\ldots,-\infty$ we
  obtain~\eqref{eq:residue-ineq}.
\end{proof}

\subsubsection{Proof of the analytic WPT}

We may assume without loss of generality that $\cC\odot\cF$ admits an
extension slightly larger than $\hrho$. Then $(\cC\odot\cF)^\hrho$ is
subanalytic, and in particular the family of all discs or annuli of
the form $\{\vz_{1..\ell}\}\times\tilde\cF$ with
$\{\vz_{1..\ell}\}\times\tilde\cF^{1/2}\subset(\cC\odot\cF)^\hrho$ is
definable. Applying Lemma~\ref{cor:uniform-indices} we find $p,M$
such that $F$ satisfies the $(p,M)$ Laurent domination property on
every fiber $\{\vz_{1..\ell}\}\times\tilde\cF$ as above.

Applying the refinement theorem, we may assume that $\rho$ is already
as small as we wish as long as $1/\rho=\poly_\ell(1/\sigma)$. Since
the maps constructed in the refinement theorem are cellular translate
maps, every disc/annulus
$\{\vz_{1..\ell}\}\times\tilde\cF^{1/2}\subset(\cC\odot\cF)^\hrho$ in
a refined cell maps to a disc/annulus satisfying the same requirement
in our original cell $\cC$. Thus we may assume that after refinement
our function $F$ still satisfies the $(p,M)$ Laurent domination
property on every fiber $\{\vz_{1..\ell}\}\times\tilde\cF$ as
above. Note that crucially $(p,M)$ does not depend on our choice of $\hrho$.

Suppose $\cF=A(r_1,r_2)$ (the cases $D(r),D_\circ(r)$ are similar).
If $F$ vanishes identically on $\cC\odot\cF$ then there is nothing to
prove, so suppose otherwise. Rescaling $\cF$ by $r_2$ we may assume
that $r_2=1$ and $|r_1|<1$. We write a Laurent expansion
\begin{equation}\label{eq:F-laurent}
  F(\vz_{1..\ell+1}) = \sum_{k=-\infty}^\infty a_k(\vz_{1..\ell}) \vz_{\ell+1}^k
\end{equation}
where $a_k\in\cO_b(\cC^\hrho)$. We apply the CPT to the collection
$a_{-p},\ldots,a_p$ on $\cC^\hrho$. In the same way as
in~\secref{sec:weierstrass-alg-proof} we may reduce to the case where
every $a_k$ is either identically or non-vanishing on
$\cC^\hrho$. Note also that since this step only reparametrizes the
base without affecting the fiber, the $(p,M)$ Laurent domination
property still holds uniformly for every $\vz_{1..\ell}\in\cC^\hrho$.

Let $\Pi\subset\{-p,\ldots,p\}$ denote the indices $k$ such that
$a_k\neq0$. For $j\neq k\in\Pi$ we define
\begin{equation}
  r_{jk}(\vz_{1..\ell}) = \sqrt[k-j]{\frac{a_j(\vz_{1..\ell})}{a_k(\vz_{1..\ell})}},
\end{equation}
that is, $S(|r_{jk}|)$ is the circle in $\vz_{\ell+1}$ where the
$j$-th and $k$-th terms of~\eqref{eq:F-laurent} are of the same
modulus. Note that $r_{jk}$ is multivalued. We let $\Sigma$ denote the
set consisting of $1,2$ and the pairs $(j,k)\in\Pi^2$, $j\not=k$, and
set $\mu:=2+(2p+1)p\ge\#\Sigma$.  

We show that the zeros of $F$ can only occur in concentric annuli of
bounded width around the radii $r_{jk}$. It is more convenient to
state this in the logarithmic scale. We set $s_\alpha:=\log|r_\alpha|$
for every $\alpha\in\Sigma$ and note that $s_\alpha$ is single valued
on $\cC_{1..\ell}$.

\begin{Lem}\label{lem:norootsfaraway}  
  There exists a constant $B=B(M,p)$ such that for any
  $\vz_{1..\ell+1}\in\cC$ satisfying
  \begin{equation}\label{eq:noroots-condition}
    |\log |\vz_{\ell+1}|-s_\alpha(\vz_{1..\ell})| > B \qquad\text{for }\alpha\in\Sigma.
  \end{equation}
  we have $F(\vz_{1..\ell+1})\neq0$.
\end{Lem}
\begin{proof}
  Assume that~\eqref{eq:noroots-condition} holds for some unspecified
  constant $B$ which will be chosen later. Set
  $r:=|\vz_{\ell+1}|$. For $k,j\in\Pi^2$ with $k\neq j$ we have
  \begin{equation}
    \left\vert\log \frac{|a_j \vz_{\ell+1}^j|}{|a_k 
        \vz_{\ell+1}^k|}\right\vert=
    \left|(j-k)[-\log(|r_{jk}|+\log r]\right| > B.
  \end{equation}
  In particular there exist one term $j_0\in\Pi$ such that
  $|a_{j_0} \vz_{\ell+1}^{j_0}|$ is maximal and we have
  \begin{equation}
    |a_{k} \vz_{\ell+1}^{k}|< e^{-B}|a_{j_0} \vz_{\ell+1}^{j_0}|,
    \qquad k\in\{-p,\ldots,p\}\setminus\{j_0\}.
  \end{equation}
  Set
  \begin{equation}
    R_p(\vz_{1..\ell+1}) := F(\vz_{1..\ell+1})-\sum_{k=-p}^p a_j(\vz_{1..\ell})\vz_{\ell+1}^k.
  \end{equation}
  Since $1,2\in\Sigma$ we have
  $e^B r_1(\vz_{1..\ell})<r<e^{-B}r_2(\vz_{1..\ell})$. Then
  Proposition~\ref{prop:residue-domination} implies (with
  $\delta=e^{-B}$) that
  \begin{equation}
    |R_p(\vz_{1..\ell})| < \frac{2M}{e^B-1}|a_{j_0} \vz_{\ell+1}^{j_0}|.
  \end{equation}
  In particular choosing $B$ such that
  $2pe^{-B}+\frac{2M}{e^B-1}<1$ we see that
  \begin{equation}
    |R_p(\vz_{1..\ell})|+\sum_{k\in\Pi\setminus\{j_0\}} |a_{k}
    \vz_{\ell+1}^{k}| < |a_{j_0} \vz_{\ell+1}^{j_0}| 
  \end{equation}
  so $F(\vz_{1..\ell+1})\neq0$ as claimed.
\end{proof}

In light of Lemma~\ref{lem:norootsfaraway} our goal will be to
construct an annulus $\cA\subset\cF^\hrho$ which contains $\cF$ and
such that $\cA^\gamma\setminus\cA$ remains at logarithmic distance at
least $B$ from each of the $|r_\alpha|$, uniformly over
$\cC^\hsigma$.

Apply the CPT to the cell $\cC^\hrho$ and the hypersurfaces
$\{r_\alpha=r_\beta\}$ for $\alpha\neq\beta\in\Sigma$ (formally we
raise both sides to a power and clear denominators to obtain a bounded
analytic equation). Once again we may reduce to the case that each of
these equations is satisfied either identically or nowhere in
$\cC^\hrho$.

We let $Z$ denote the union of the graphs of the zero function and of
$r_\alpha$ over $\cC^\hrho$ for $\alpha\in\Sigma$. By the condition
above $\pi:Z\to\cC^\hrho$ is a proper covering map. We are now in a
position to use the constructions of~\secref{sec:cover-clustering}. We
choose $\gamma=e^B$.

Recall that we may choose $\rho$ small as long as
$1/\rho=\poly_\ell(1/\sigma)$. We take
$\hrho<\hsigma\cdot\he{\hat\rho}$ where $\hat\rho$ is chosen in such a
way that Proposition~\ref{prop:clusters} holds over $\cC^\hsigma$. The
zero map is a section of $\pi$ which we denote by $y_0$. Since
$r_1,r_2$ are sections of $\pi$ they belong to certain clusters around
$y_0=0$, say with indices $q_1\le q_2$. We may also assume that
$r_1=\hat y_{q_1}$ and if $q_1\neq q_2$ then $r_2=\hat y_{q_2}$.

If $q_1=q_2$ then we define $\tilde\cF=\cF_{0,q}$. If $q_1<q_2$ then we define
\begin{equation}\label{eq:wpt-final-fiber-analytic}
  \tilde\cF:= A^\delta(r_{q_1}r_1,l_{q_2}r_2),
\end{equation}
i.e. we take the left endpoint of the $\cF_{0,q_1}$ cluster and the
right endpoint of the $\cF_{0,q_2}$ cluster. Since $r_1,r_2$ are
univalued over $\cC$ this is actually a fiber over $\cC$ and the
Weierstrass condition is provided by Proposition~\ref{prop:clusters}.
The inclusion $\tilde\cF^\gamma\subset\cF^\hrho$ follows
from~\eqref{eq:cluster-edge-bounds} provided that we choose $\hrho$
large enough (note that $B$ and hence $\gamma$ did not depend on our
choice of $\hrho$).

\subsubsection{The real setting}

If $\cC$ and $F$ are real then the Laurent coefficients $a_j$ are also
real, and consequently the coverings constructed by inductive
applications of the CPT can be taken to be real. After these
reductions, the fiber constructed in~\eqref{eq:wpt-final-fiber-analytic}
is clearly real as well.

\subsubsection{Uniformity over families}

Uniformity of the number of cells over definable families follows
readily from the proof. Indeed the indices $p,M$ have already been
shown to be uniformly bounded over families. The fact that all
Weierstrass cells can be chosen from a single definable family follows
exactly as in the case of the refinement theorem.

\subsubsection{Analytic discriminants}

Later we will also require the following lemma on analytic
discriminants.

\begin{Lem}\label{lem:discriminant-analytic}
  Let $\cC^\hrho$ be a cell and let $Z\subset\cC^\hrho\times\C$ such
  that the projection $\pi:Z\to\cC^\hrho$ is proper. Then there exists
  $0\neq D\in\cO_b(\cC)$ such that $\pi$ is a covering map outside
  $\{D=0\}$. If $\cC,Z$ are real then $D$ can be chosen real as
  well. If $\cC,Z$ vary in a definable family then $D$ can also be
  chosen to vary from a definable family.
\end{Lem}
\begin{proof}
  According to \cite[Theorem~III.21]{gr:analytic} the map $\pi$ is an
  \emph{analytic cover} and in particular it is a $\lambda$-sheeted
  covering map, for some $\lambda\in\N$, outside a negligible set
  $A\subset\cC^\hrho$. Then letting
  \begin{equation}
    P_Z(\vz,w) := \prod_{\eta\in\pi^{-1}(\vz)} (w-\eta)
  \end{equation}
  we obtain a monic polynomial of degree $\lambda$ with holomorphic
  locally bounded coefficients in $\cC^\hrho\setminus A$, which by the
  removable singularity theorem extend to holomorphic locally bounded
  coefficients in $\cC^\hrho$. Then $D$ can be taken to be the
  classical discriminant of $P_Z$ with respect to $w$. If $\cC,Z$ are
  real then $P_Z$ and hence $D$ are also real. It is clear that this
  construction can be made uniform over definable families.
\end{proof}


\section{Proofs of the CPT and CPrT}

In this section we finish the proof of the CPT and CPrT for a cell
$\cC\odot\cF$. We proceed by induction on the length and dimension. By
the note following Theorem~\ref{thm:wpt}, we may already assume that
the WPT holds for $\cC\odot\cF$.

\subsection{Proof of the CPT}

We will prove the CPT in a slightly weakened form, replacing the
prepared maps $f_j$ by arbitrary cellular maps. We later prove that
this weaker form implies the CPrT, which in turn directly implies the
stronger form of the CPT. We describe the proof for the complex
version of the CPT, and at the end indicate the changes required for
the real version. To avoid cluttering the text, we state our proof for
the algebraic version of the CPT with polynomial estimates in the
complexity $\beta$. The subanalytic version where these polynomial
estimates are replaced by uniformity over definable families is
obtained in a completely analogous manner.

\subsubsection{Reduction to the case of a single function}
\label{sec:cpt-single-F}

We claim that it is enough to prove the CPT for a single function
$F$. We suppose that this is already proved and prove the result for
an arbitrary collection $F_1,\ldots,F_M$. We may suppose that none of
the $F_j$ vanish identically on $\cC$. Let $F:=F_1\cdots F_M$ and
apply the CPT to $\cC,F$ to obtain a cellular cover
$f_j:\cC_j^\hsigma\to(\cC\odot\cF)^\hrho$ compatible with $F$. Fix
some $f_j$. If $f_j(\cC_j^\hsigma)$ lies outside the zeros of $F$ then
it is already compatible with $F_1,\ldots,F_M$.  Otherwise it lies in
the zeros of $F$, and since cellular maps preserve dimension we have
$\dim\cC_j<\dim\cC\odot\cF$. By induction we obtain cellular maps
$f_{jk}:\cC_{jk}^\hsigma\to\cC_j^\hsigma$ which are compatible with
$f_j^*F_1,\ldots,f_j^*F_M$. Then the compositions $f_j\circ f_{jk}$
are compatible with $F_1,\ldots,F_M$ and cover $f_j(\cC_j)$, and taken
together this gives a cellular cover of $\cC\odot\cF$ with
$\poly_\ell(\beta,1/\sigma,\rho)$ maps as required.

\subsubsection{Reduction to large $\sigma$ and small $\rho$}
\label{sec:cpt-refine-sigma-rho}

Applying the refinement theorem to $\cC$ we may suppose that it
already admits a $\he{\hat\rho}$-extension for
$\hat\rho^{-1}=\poly_\ell(\rho,\beta)$. Below we will assume that
$\rho$ is already as small as we wish subject to this
asymptotic. Similarly, it is enough to prove the CPT with any
$\hat\sigma=\poly_\ell(\beta)$ since we may afterwards apply the refinement
theorem to the resulting cells to obtain cells with
$\sigma$-extensions. Below we will assume that $\sigma$ is already as
large as we wish subject to this asymptotic.

\subsubsection{Reduction to a Weierstrass cell for $F$}

We apply the WPT to cover $\cC\odot\cF$ by cells of the form
$f_j:\cC_j^\hrho\odot\cF_j^\gamma\to(\cC\odot\cF)^\hrho$ which are Weierstrass
for $F$ with gap $\gamma=1-1/\poly_\ell(\beta)$. It is enough to prove
the CPT for each of these cells separately, i.e. we may assume that
$F\in\cO_b(\cC^\hrho\odot\cF^\gamma)$ does not vanish in
$\cC^\hrho\odot(\cF^\gamma\setminus\cF)$. If $\cF$ is of type $*$ the
CPT reduces to the CPT for $\cC_{1..\ell-1}$. Suppose $\cF=A(r_1,r_2)$
(the cases $D(r),D_\circ(r)$ are similar).

\subsubsection{Reduction to a proper covering map}

Since $\cC^\hrho\odot\cF$ is a Weierstrass cell for $F$ the zero locus
of $F$ in this cell is proper cover of $\cC$ under the projection
$\pi:\cC^\hrho\times\C\to\cC^\hrho$. We define $Z$ to be the union of
this zero locus with the graphs of the functions
$0,r_1,r_2:\cC^\hrho\to\C$. Then by
Lemma~\ref{lem:discriminant-analytic} there exists
$D\in\cO_b(\cC^\hrho)$ not identically vanishing on $\cC$ such that
the projection $\pi\rest Z$ is a proper covering map outside
$\{D=0\}$. In the algebraic case, applying
Lemma~\ref{lem:discriminant-alg} to the function
$F\cdot\vz_{\ell+1}(\vz_{\ell+1}-r_1)(\vz_{\ell+1}-r_2)$ we see that
$D$ can be taken to be a polynomial of complexity $\poly(\beta)$.

We apply the CPT to $\cC$ with $D$ and let
$f_j:\cC^\hrho_j\to\cC^\hrho$ denote the resulting cellular cover. If
$f_j(\cC_j^\hrho)$ is contained in $\{D=0\}$ then since cellular maps
preserve dimension $\dim\cC_j\le\dim\{D=0\}\le\dim\cC-2$. In this case
we set
\begin{align}
  \hat\cC_j&:=\cC_j\odot f_j^*\cF & \hat f_j&:=(f_j,\id):\hat\cC_j^\hrho\to\cC^\hrho
\end{align}
and note that $f_j^*F\in\cO_b(\hat\cC_j^{\he{\hat\rho}})$ for
$\hat\rho=\poly(\rho,\beta)$. We inductively apply the CPT to
$\hat\cC_j$ with its $\hat\rho$-extension and $\hat f_j^*F$. We obtain
a cellular cover
$f_{j,k}:\cC^\hsigma_{j,k}\to\hat\cC_j^{\he{\hat\rho}}$ compatible
with $F$, and the compositions
$f_{j,k}\circ\hat f_j:\cC^\hsigma_{j,k}\to\cC^\hrho$ are compatible
with $F$ and cover $f(\cC_j)\odot\cF$.

It remains to consider the case that $f_j(\cC_j^\hrho)$ is disjoint
from $\{D=0\}$. In the same way as before, it will suffice to prove
the CPT for $\hat\cC_j$ and $\hat f_j^*F$. We return now to the
original notation replacing this pair by $\cC,F$ so that
\begin{equation}
  \pi:(\cC^\hrho\odot\cF^\gamma)\cap\{F=0\}\to\cC, \qquad \pi(\vz_{1..\ell+1})=\vz_{1..\ell}
\end{equation}
is a proper covering map. We are now in a position to use the
constructions of~\secref{sec:cover-clustering}. We denote the degree
of $\pi$ by $\nu$, and note that $\nu=\poly_\ell(\beta)$.

\subsubsection{Covering the zeros}
\label{sec:cpt-zero-cover}

By Lemma~\ref{lem:y_j-monodromy}, each section $y_j$ of $\pi$ lifts to
a univalued map $y_j:\hat\cC_j^{\nu_j\cdot\rho}\to\C$. By definition
the image of this maps lies in
$(\cC^\hrho\odot\cF^\gamma)\cap\{F=0\}$, i.e. it gives a cell
compatible with $F$. The collections of all of these cells cover the
zeros of $F$ in $\cC\odot\cF$. The remaining (and far more
non-trivial) task is to cover the complement of this set of zeros.

\subsubsection{The Voronoi cells associated to $y_0=0$}

For the remainder of the proof we fix a point $p\in\hat\cC$.  We may
assume that $\rho$ is small enough that
Proposition~\ref{prop:clusters} holds, not only over $\cC$ but over
$\cC^\hsigma$. The zero map is a section of $\pi$ which we denote by
$y_0$. We will construct a collection of cells which we call the
\emph{Voronoi cells} of $y_0$.

Since $r_1,r_2$ are sections of $\pi$ they belong to certain clusters
around $y_0=0$, say with indices $q_1\le q_2$. For
$q=q_1,\ldots,q_2-1$ the fibers $\cF^\gamma_{0,q+}$ over
$\hat\cC_{0,q}$ contain none of the zeros $y_j$ and are contained in
$\cF^\gamma$. Therefore the maps
$\hat\cC_{0,q+}\odot\cF_{0,q+}\to\cC\odot\cF$ admit $\gamma$-extensions
compatible with $F$. These give our first Voronoi cells.

Now fix $q=q_1,\ldots,q_2$ and consider the cluster $\cF_{0,q}$. Up to
an affine transformation as in Remark~\ref{rem:single-cluster} we may
assume that $\hat y_{0,q}=1$ and
$\cF_{0,q}=A^\delta(l_{0,q},r_{0,q})$. Recall that
$\gamma=1-1/\poly_\ell(\beta)$, and therefore we have
$\cF_{0,q}\subset A(1/2,2)$. Recall also that
\begin{equation}\label{eq:0-cluster-y_j}
  \diam(y_j(\hat\cC),\C\setminus\{0,1\})=O(\nu^3\rho)
\end{equation}
for any of the $y_j$ belonging to $\cF_{0,q}$.

Let $\alpha>0$ be such that the balls of radius $\alpha$ around
$y_j(p)$ belongs to $\cF_{0,q}^\gamma(p)$. Clearly we can choose
$\alpha^{-1}=\poly_\ell(\beta)$. Let $U_p(\alpha)$ be the set obtained
from $\cF_{0,q}(p)$ by removing all of these balls. We can cover
$U_p(\alpha)$ be $\poly_\ell(\beta)$ discs $\cD_k$ centered at
$U_p(\alpha)$ such that $\cD_k^{1/2}$ does not meet the balls of
radius $\alpha/2$ around $y_j(p)$. These give our remaining Voronoi
cells. We claim that $\hat\cC_{0,q}\odot\cD^{1/2}_k$ is compatible
with $F$. Indeed, over $p$ the discs $\cD_k$ remain at distance
$\alpha/2$ from the zeros, and by~\eqref{eq:0-cluster-y_j} the points
$y_j$ do not move enough to meet $\cD_i$ for $\rho$ sufficiently
small. By the same reasoning we obtain the following fundamental
property of the Voronoi cells.

\begin{Lem}\label{lem:voronoi-y0}
  The Voronoi cells associated to $y_0$ cover every point
  $\vz_{1..\ell+1}\in\cC\odot\cF$ such that
  \begin{equation}
    \dist(\vz_{\ell+1},y_0) < (2\alpha)^{-1}\dist(\vz_{\ell+1},y_j)
  \end{equation}
  for every $y_j$.
\end{Lem}

\subsubsection{The Voronoi cells associated to $y_i$}

Let $y_i$ be one of the non-zero sections of $\pi$. We will construct
a collection of \emph{Voronoi cells} for $y_i$ by analogy with $y_0$:
the only additional difficulty is making sure that all cells remain in
$\cF^\gamma$. Our goal will be to construct cells with the following
property.

\begin{Lem}\label{lem:voronoi-yi}
  The Voronoi cells associated to $y_i$ cover every point
  $\vz_{1..\ell+1}\in\cC\odot\cF$ such that
  \begin{equation}\label{eq:voronoi-yi-1}
    \dist(\vz_{\ell+1},y_i) \le 2\alpha\dist(\vz_{\ell+1},y_0)
  \end{equation}
  and
  \begin{equation}\label{eq:voronoi-yi-2}
    \dist(\vz_{\ell+1},y_i) \le (2\alpha)^{-1}\dist(\vz_{\ell+1},y_j)
  \end{equation}
  for every $y_j\neq y_0,y_i$.
\end{Lem}

We make an affine transformation such that $y_i=0$. Since $y_i\in\cF$,
we know that in these coordinates $D(r y_0)\subset\cF^\gamma$ for
$r\sim1-\gamma=1/\poly(\beta)$. In the notations
of~\secref{sec:cover-clustering}, let $\cF_{i,q_0}$ be the last
cluster whose outer boundary at $p$ is smaller than $ry_0(p)$.  With
our choice of $\gamma$ the logarithmic width of each cluster can be
assumed to be bounded by $\log 2$. Therefore the inner boundary of
$\cF_{i,q_0+1}$ at $p$ is at least $ry_0(p)/2$. If the outer boundary
of $\cF_{i,q_0}$ is smaller than $\gamma^2ry_0(p)/2$ then we also set
\begin{equation}
  \tilde\cF_{i,q_0+}=A^\delta(r_{i,q_0}\hat y_{i,q_0},\gamma r y_0/2).
\end{equation}
One can check in the same way as in Proposition~\ref{prop:clusters}
that when defined, $\tilde\cF_{i,q_0+}$ is a well defined annulus over
$\hat\cC_{i,q_0+}$ and $\tilde\cF_{i,q_0+}^\gamma\subset\cF^\gamma$
contains no zeros.

\begin{figure}
  \centering
  \includegraphics[width=0.6\textwidth]{CPTProofLast.pdf}
  \caption{Construction of the $\tilde\cF_{i,q_0+}$ fiber near the boundary.}
  \label{fig:cpt-proof-last}
\end{figure}

The fibers $\cF_{i,q},\cF_{i,q+}$ for $q=1,\ldots,q_0-1$, the fiber
$\cF_{i,q_0}$, and $\tilde\cF_{i,q_0+}$ if it is defined cover
$D(\gamma^2 r y_0/2)$ uniformly over $\hat\cC$. We can choose our
$\alpha=1/\poly_\ell(\beta)$ in such a way that this disc contains all
points satisfying~\eqref{eq:voronoi-yi-1}. It will be enough to
construct the Voronoi cells covering the points in these fibers which
also satisfy~\eqref{eq:voronoi-yi-2}. This is done in a manner
completely analogous to the Voronoi cells of $y_0$, except that in
place of $\cF_{0,q_1+},\ldots,\cF_{0,(q_2-1)+}$ we use
$\cF_{i,0+},\ldots,\cF_{i,(q_0-1)+}$ and $\tilde\cF_{i,q_0+}$ if it is
defined; and instead of $\cF_{0,q_1},\ldots,\cF_{0,q_2}$ we use
$\cF_{i,1},\ldots,\cF_{i,q_0-1}$.


\subsubsection{Conclusion}

We claim that the Voronoi cells for $y_0$ and the roots $y_j$ cover
$\cC\odot\cF\setminus\{F=0\}$. Indeed, let $\vz\in\cC\odot\cF$. If
$\dist(\vz_{\ell+1},y_0) < (2\alpha)^{-1}\dist(\vz_{\ell+1},y_j)$ for
every $y_j$ then $\vz$ is covered by the Voronoi cells of
$y_0$. Otherwise it is covered by the Voronoi cell of $y_j$ for the
root $y_j$ closest to $\vz_{\ell+1}$. Since the roots of $\{F=0\}$
were already covered in~\secref{sec:cpt-zero-cover}, this concludes
the proof of the CPT in the complex case.

\subsubsection{The real case of the CPT}

The proof of the CPT in the real case proceeds in the same manner up
to~\secref{sec:cpt-zero-cover}. At this point one should cover only
the sections $y_j$ that are real, ensuring that we indeed obtain real
cells. Since the sections $y_j$ are locally unramified, each of them
is either purely real or purely non-real on $\R_+\cC$ so we can indeed
choose those $y_j$ that are real to obtain a covering of the zeros of
$F$ in $\R_+(\cC\odot\cF)$.

The next step is the construction of the Voronoi cells. To ensure that
the constructed cells are real, one should replace the construction
of~\secref{sec:cover-clustering} by that
of~\secref{sec:cover-clustering-real} as we explain below.

The section $y_0=0$ is real and therefore forms an admissible
center. The Voronoi cells for $y_0$ are constructed as in the complex
case, except that the discs $\cD_k$ are now chosen with real centers
such that their positive parts cover $\R_+U_p(\alpha)$. The points
$\vz_{\ell+1}$ that are left uncovered over $p$ are those that have
distance at most $\alpha$ to some root $y_k$. If $\vz_{\ell+1}\in\R$
then by Lemma~\ref{lem:select-admissible} such points also have
distance at most $\alpha$ to some admissible center
$c_l=(y_j+y_{\bar j})/2$. By analogy with Lemma~\ref{lem:voronoi-y0}
we obtain the following lemma.

\begin{Lem}\label{lem:voronoi-y0-real}
  The Voronoi cells associated to $y_0$ cover every point
  $\vz_{1..\ell+1}\in\R_+(\cC\odot\cF)$ such that
  \begin{equation}
    \dist(\vz_{\ell+1},y_0) < (2\alpha)^{-1}\dist(\vz_{\ell+1},c_l)
  \end{equation}
  for every admissible center $c_l$.
\end{Lem}

For the remaining Voronoi cells, we construct them centered around the
admissible centers $c_l$ rather than the sections $y_j$. The
construction is analogous, replacing again the covering of
$U_p(\alpha)$ by a real covering of $\R_+U_p(\alpha)$. We similarly
obtain the following lemma.

\begin{Lem}\label{lem:voronoi-yi-real}
  The Voronoi cells associated to $c_l$ cover every point
  $\vz_{1..\ell+1}\in\R_+(\cC\odot\cF)$ such that
  \begin{equation}\label{eq:voronoi-yi-1-real}
    \dist(\vz_{\ell+1},c_l) \le 2\alpha\dist(\vz_{\ell+1},y_0)
  \end{equation}
  and
  \begin{equation}\label{eq:voronoi-yi-2-real}
    \dist(\vz_{\ell+1},c_l) \le (2\alpha)^{-1}\dist(\vz_{\ell+1},c_m)
  \end{equation}
  for every $c_m\neq y_0,y_l$.
\end{Lem}

The proof is then concluded in the same manner.

\subsection{Proof of the CPrT}

In this section we will prove the CPrT using the WPT and CPT. We will
need a simple remark on the structure of the maps constructed in these
two theorems.

\begin{Rem}\label{rem:wpt-cpt-translates}
  The maps constructed in the CPT can be assumed to be translates in
  the final variable, as the reader may easily verify by examining the
  inductive proof. Similarly, the maps constructed in the WPT can be
  assumed to be translates in the final two variables: the CPT and
  refinement theorems are applied in the base to give a translate in
  the next-to-last variable, whereas in the last variable the map is
  the identity.
\end{Rem}

The main inductive step for the proof of the CPrT is contained in the
following lemma.

\begin{Lem}\label{lem:cprt-step}
  Let $f:\cC^\hrho\to\hat\cC$ be a (real) cellular map. Then there
  exists a (real) cellular cover $\{g_j:\cC_j^\hrho\to\cC^\hrho\}$ of
  size $\poly_f(\rho)$ such that each $f\circ g_j$ is prepared
  \emph{in the last variable}.

  If $\cC,\hat\cC,f$ vary in a definable family then the size of the
  cover is $\poly(\rho)$ uniformly over the family, and the maps $g_j$
  can be chosen to vary definably over the family. If $\cC,\hat\cC,f$
  are algebraic of complexity $\beta$ then the cover has size
  $\poly(\beta,\rho)$ and complexity $\poly(\beta)$.
\end{Lem}
\begin{proof}
  We describe the proof for the complex case, and at the end indicate
  the changes required for the real version. We treat the case of one
  cell and briefly indicate the minor points related to uniformity
  over families.

  Let $\cC:=\cC_{1..\ell}\odot\cF$. Recall that the last coordinate of
  $f$ has the form $P(\vz_{1..\ell};\vz_{\ell+1})$ where $P$ is a
  monic polynomial in $\vz_{\ell+1}$ (say of degree $\mu$) with
  coefficients holomorphic in $\cC_{1..\ell}^\delta$. If $\cF=*$ then
  we can just replace $\vz_{\ell+1}$ by zero in the expression above,
  so $f$ itself is already prepared. So assume $\cF$ is of type
  $D,D_\circ,A$.

  Let $\Sigma\subset\cC$ denote the set critical points of $P$ with respect to
  $\vz_{\ell+1}$, i.e.
  \begin{equation}
    \Sigma := \left\{ \pd{P}{\vz_{\ell+1}}(\vz_{1..\ell+1})=0 \right\}.
  \end{equation}
  Note that since $P$ is a monic polynomial of positive degree in
  $\vz_{\ell+1}$ the hypersurface $\Sigma$ has zero dimensional fibers
  over $\cC_{1..\ell}$.
  
  We apply the CPT to $\cC$ and $\Sigma$ to obtain a cellular cover
  $\{g_j:\cC_j^\hrho\to\cC^\hrho\}$ compatible with $\Sigma$. If the
  image of $g_j$ is contained in $\Sigma$ then, since the fibers of
  $\Sigma$ over $\cC_{1..\ell}$ are zero dimensional, $\cC_j$ has type
  ending with $*$ and in this case $f\circ g_j$ is already prepared as
  noted above. It remains to consider the case that the image of $g_j$
  is disjoint from $\Sigma$. Recall from
  Remark~\ref{rem:wpt-cpt-translates} that $g_j$ is a translate in its
  last variable $\vw_{\ell+1}$. Then we have
  \begin{equation}
    \pd{}{\vw_{\ell+1}}(P\circ g_j) = \pd{P}{\vz_{\ell+1}}(g_j(\vw_{1..\ell+1}))\neq0
  \end{equation}
  since the image of $g_j$ is disjoint from $\Sigma$. It will now be
  enough to prove the lemma with $f$ replaced by each $f\circ g_j$ as
  above. In other words we may assume without loss of generality that
  $\Sigma=\emptyset$.

  Recall that $P$ is bounded in absolute value by some constant $N$ on
  $\cC^\hrho$. Write $D:=D(N)$ and consider the cell
  \begin{equation}
    \tilde\cC:=\cC_{1..\ell}\odot D\odot\cF
  \end{equation}
  with coordinates $\vz_{1..\ell},w,\vz_{\ell+1}$, which also admits a
  $\hrho$-extension. Let $\Gamma\subset\tilde\cC^\hrho$ be the hypersurface
  given by
  \begin{equation}
    \Gamma := \{ w=P(\vz_{1..\ell};\vz_{\ell+1}) \}.
  \end{equation}
  We apply the WPT to $\tilde\cC^\hrho,\Gamma$ to obtain a cellular
  covering
  $\{g_\alpha:\cC_\alpha^{\he{\rho/\mu}}\odot\cF_\alpha^\gamma\to\cC^\hrho\}$
  of $\tilde\cC$ by Weierstrass cells for $\Gamma$.

  Suppose first that $g_\alpha(\cC_\alpha)\subset\Gamma$. Then
  since $\Gamma$ has zero-dimensional fibers over
  $\cC_{1..\ell}\odot D$ (because $P$ is monic) the type of $\cC_\alpha$
  ends with $*$, and we have
  \begin{equation}\label{eq:psi-subord}
    \psi_\alpha(\vz_{1..\ell},w) := g_\alpha(\vz_{1..\ell},w,*) = (\cdots, w+\phi_\alpha(\vz_{1..\ell}),\zeta_\alpha(\vz_{1..\ell},w))
  \end{equation}
  where $\psi_\alpha:(\cC_\alpha)_{1..\ell}\to\Gamma$ is a cellular
  map admitting a $\he{\rho/\mu}$-extension. Here we used the fact that
  $g_\alpha$ is a translate in its next to last variable. For later
  purposes we set $\nu(\alpha)=1$.

  Suppose now that $g_\alpha$ is not compatible with $\Gamma$. Then by
  the definition of a Weierstrass cell it follows that
  $g_\alpha^*\Gamma$ forms a cover of
  $(\cC_\alpha)_{1..\ell+1}$. Since we are assuming that $\Sigma=0$,
  this is a covering map of degree $\nu(\alpha)\le\mu$. Then
  $g_\alpha^*\Gamma$ admits $\nu(\alpha)$ multivalued sections. Each
  section $s_{\alpha,j}:(\cC_\alpha)_{1..\ell+1}\to g_\alpha^*\Gamma$
  takes the form
  \begin{equation}
    s_{\alpha,j}(\vz_{1..\ell},w) = (\vz_{1..\ell},w,\hat\zeta_{\alpha,j}(\vz_{1..\ell},w))
  \end{equation}
  where $\hat\zeta$ is ramified of order at most
  $\nu(\alpha,j)\le\nu(\alpha)$ (see
  Lemma~\ref{lem:y_j-monodromy}). We note that in the algebraic case,
  $s_{\alpha,j}$ is algebraic of degree $\poly_\ell(\beta)$ since it
  is the section of a hypersurface of complexity
  $\poly_\ell(\beta)$. In the family case $\nu(\alpha)$ and the
  sections $s_{\alpha,j}$ vary definably.

  We set
  \begin{equation}
    \cC_{\alpha,j} := ((\cC_\alpha)_{1..\ell+1})_{\times\nu(\alpha,j)}.
  \end{equation}
  Precomposing with
  $R_{\nu(\alpha,j)}:\cC_{\alpha,j}\to(\cC_\alpha)_{1..\ell+1}$ we
  obtain a univalued map
  $s_{\alpha,j}\circ R_{\nu(\alpha,j)}:\cC_{\alpha,j}\to
  g_\alpha^*\Gamma$ and finally composing with $g_\alpha$ we obtain a
  map
  $\psi_{\alpha,j}:=g_\alpha\circ s_{\alpha,j}\circ
  R_{\nu(\alpha,j)}:\cC_{\alpha,j}\to\Gamma$ of the form
  \begin{equation}\label{eq:psi-cover}
    \psi_{\alpha,j}(\vz_{1..\ell},w) = (\cdots,w^{\nu(\alpha,j)}+\phi_{\alpha,j}(\vz_{1..\ell}),\zeta_{\alpha,j}(\vz_{1..\ell},w)).
  \end{equation}
  Here again we used the fact that $g$ is a translate in its next to
  last variable. By construction $\psi$ admits a $\hrho$-extension.

  Let $\beta$ be one of the indices $\alpha$ or $(\alpha,j)$ for the
  maps $\psi$ constructed
  in~\eqref{eq:psi-subord},\eqref{eq:psi-cover}. Set
  $f_\beta:=((\psi_\beta)_{1..\ell},\zeta_\beta):\cC_\beta^\hrho\to\cC$. Then
  $f\circ f_\beta$ is prepared in its final variable: indeed, since
  $\psi$ maps into $\Gamma$ we have
  \begin{equation}
    w^{\nu(\alpha)}+\phi_\beta(\vz_{1..\ell})=P((\psi_\beta)_{1..\ell}(\vz_{1..\ell}),\zeta_\beta(z_{1..\ell},w))
    = (f\circ f_\beta)_{\ell+1}
  \end{equation}
  for every $(\vz_{1..\ell},w)\in\cC_\beta^\hrho$. It remains to show
  that $f_\beta(\cC_\beta)$ cover $\cC$. But this is clear, since
  $\psi_\beta(\cC_\beta)$ covers $\tilde\cC\cap\Gamma$ by construction
  and the projection of $\tilde\cC\cap\Gamma$ to $\vz_{1..\ell+1}$
  equals $\cC$ by our choice of $D$.

  We now consider the real case. We proceed in the same manner up to
  the construction of the sections $\psi_{\alpha,j}$. Note that some
  of these sections may not be real. However, since the critical locus
  $\Sigma$ is empty it follows that on $\R_+\cC_{\alpha,j}$ the
  section $\psi_{\alpha,j}$ is either always real or always non-real
  (real roots of a real holomorphic function can only leave the real
  line if they collide). Since we are only interested in covering
  $\R_+\Gamma$ we may replace the full set of sections
  $\psi_{\alpha,j}$ by those sections that are real on
  $\R_+\cC_{\alpha,j}$. With this modification we obtain a real cover
  of $\cC$ as required.
\end{proof}

We are now ready to finish the proof of the CPrT by induction: suppose
that the CPrT is already proved for cells of length $\ell$, and we
will prove it for a cell $\cC:=\cC_{1..\ell}\odot\cF$. By
Lemma~\ref{lem:cprt-step} we may assume that the map $f$ is already
prepared in its final variable. By the inductive hypothesis we can
find a cellular cover $g_j:\cC_j^\hrho\to\cC_{1..\ell}^\hrho$ such
that $f_{1..\ell}\circ g_j$ is prepared. Then
$\hat g_j:=g_j\odot\id:(\cC_j\odot(g_j^*\cF))^\hrho\to\cC^\hrho$ is a
cellular cover for $\cC$ with $f\circ\hat g_j$ is prepared as
required.



\section{Analysis on complex cells}

In this section we prove some basic estimates on holomorphic functions
in complex cells. We also introduce the notion of a \emph{quadric}
cell which allows for some finer estimates and is used in the
construction of smooth parametrizations in the sequel.

We fix the notation for the remainder of this section. We let
$\cC=\cF_1\odot\cdots\odot\cF_\ell$ denote a complex cell. As a matter
of normalization, we assume that $\cF_j$ is is one of the forms
$*,D(1),D_\circ(1)$ or $A(r_j,1)$. Every cell is isomorphic to a cell
of this form by an appropriate rescaling map. We say that such a cell
$\cC$ is \emph{normalized}. We fix $0<\delta<1$ and assume that $\cC$
admits a $\delta$-extension.

\subsection{Logarithmic derivatives}

The following proposition is a cellular analog of the classical fact
that a logarithmic derivative of a holomorphic function admits at most
first order poles.

\begin{Prop}\label{prop:log-derivative-bound}
  Let $0<\delta<1/8$ and let $f\in\cO_b(\cC^\delta)$ be
  non-vanishing. Then for $\vz\in\cC$ we have
  \begin{equation}
    \abs{\frac{1}{f}\frac{\partial f}{\partial \vz_i}} \le O_\ell(\abs{\vz_i}^{-1})
    \qquad \text{for } i=1,\ldots,\ell.
  \end{equation}
  If $\cC,f$ are algebraic of complexity $\beta$ then
  \begin{equation}
    \abs{\frac{1}{f}\frac{\partial f}{\partial \vz_i}}\le \poly_\ell(\beta)\cdot \abs{\vz_i}^{-1}
    \qquad \text{for } i=1,\ldots,\ell.
  \end{equation}
\end{Prop}
\begin{proof}
  By the monomialization Lemma~\ref{lem:monomial} we have
  $\log f=\sum_k m_k\log \vz_k+\log U$ where $g=\log U$ is holomorphic
  and bounded in $\cC^{1/2}$. The derivative of the first term is
  bounded by $|m_i\vz_i|^{-1}$, and in the algebraic case
  $|m_i|=\poly_\ell(\beta)$. It remains to give a similar estimate for
  $\pd{g}{\vz_i}$. Since we only care about the derivatives we may,
  after a translate, assume that the image of $g$ contains $0$. 
  Then by the monomialization lemma in the algebraic case
  $\norm{g}_{C^{1/4}}\le\poly_\ell(\beta)$.
  
  Write a decomposition $g=\sum_{\vsigma} g_\vsigma(\vz^{[\vsigma]})$
  as in Corollary~\ref{cor:laurent-disc-decmp},
  \begin{equation}
    g_\vsigma\in\cO_b(\cP^\delta), \qquad \norm{g_\vsigma}_{\cP^{1/2}} = O_\ell(\norm{g}_{C^{1/4}}).
  \end{equation}
  We estimate the derivative by  
  \begin{equation}
    \abs{\pd{g_\vsigma}{\vz_i}}\le
    \sum_{j=1}^\ell \abs{(g_\vsigma)'_j(\vz^{[\vsigma]}) \pd{(\vz_j^{[\vsigma_j]})}{\vz_i}}.
  \end{equation} 
  By the Cauchy formula for $g_\vsigma$ in $\cP^{1/2}$ we see that
  $\norm{(g_\vsigma)'_j}_\cP=O_\ell(\norm{g_\vsigma}_{\cP^{1/2}})$,
  which is $\poly_\ell(\beta)$ in the algebraic case.

  The derivatives $\pd{(\vz_j^{[\vsigma_j]})}{\vz_i}$ are either $0$
  or $1$ for $\vsigma_j=1$; and either $0$ or $-r_i/\vz_i^2$ or
  $\vz_j^{-1}\pd{r_j}{\vz_i}$ for $\vsigma=-1$. But
  \begin{align}
    \abs{-\frac{r_i}{\vz_i^2}}&=\abs{\frac{r_i}{\vz_i}\cdot\frac1{\vz_i}}\le|\vz_i|^{-1}, &
    \abs{\vz_j^{-1}\pd{r_j}{\vz_i}}&=\abs{\frac{r_j}{\vz_j}\cdot\frac1{r_j} \pd{r_j}{\vz_i}}\le K|\vz_i|^{-1}
  \end{align}
  where and the final inequality follows by induction on $\ell$ and
  $K=\poly_\ell(\beta)$ in the algebraic case. From the above we
  conclude that $\pd{g}{\vz_i}<K'|\vz_i|^{-1}$ for an appropriate
  constant $K'$ as claimed.
\end{proof}

\subsection{Quadric cells}

We say that the $\cC$ cell is \emph{quadric} if the radii $r_j$ (for
$\cF_j$ of annulus type) have a univalued square root, $r_j=\rho_j^2$
on $\cC_{1..j-1}$. This can also be stated by saying that the
associated monomial of $r_j$ has even degrees. We note if $\cC$ is
quadric then $\cC^\delta$ is quadric as well. We denote by $\Qua\cC$
the \emph{positive quadrant} defined by replacing each fiber
$\cF_j=A(r_j,1)$ in the definition of $\cC$ by $A(\rho_j,1)$. We
denote by $\Qua^\delta\cC$ the cell obtained by taking
$\tilde\cF_j=A(r_j,\delta^{-1})$. Note that this is the same as
$\Qua(\cC^\delta)$ up to renormalizing all outer radii to $1$.

\begin{Prop}\label{prop:quadric-nu-cover}
  Let $\cC$ be a cell admitting a $\delta$-extension.  Set
  $\vnu:=(2^{\ell-1},\ldots,1)$. Then $\cC_{\times\vnu}$ is quadric
  and admits a $\delta^{1/2^{\ell-1}}$-extension.
\end{Prop}
\begin{proof}
  If $(\alpha_1,\ldots,\alpha_{j-1})$ are the degrees of the
  associated monomial of $r_j$ then the associated monomial of the
  $j$-th fiber of $\cC_{\times\vnu}$ has degrees
  $2^{j-\ell}(2^{\ell-1}\alpha_1,\ldots,2^{\ell-j+1}\alpha_{j-1})$,
  all of which are even as claimed. The existence of a
  $\delta^{1/2^{\ell-1}}$-extension is a simple exercise.
\end{proof}

Suppose $\cC$ is a quadric cell and $\vsigma\in\{-1,1\}^\ell$, with
negative entries allowed only for indices $j$ such that $\cF^j$ is an
annulus. We define the inversion map $I_\vsigma:\cC\to\cC$ by the
identity on coordinates $\vz_j$ with $\vsigma=1$ and by $r_j/\vz_j$ on
coordinates $\vz_j$ with $\vsigma=-1$. Using inversions we can prove
the following.

\begin{Prop}\label{prop:quadric-cover}
  Let $\cC$ be a real cell admitting a $\delta$-extension. There
  exists a collection of quadric normalized cells $\cC_j$ and real
  cellular maps $f_j:\cC_j\to\cC^\delta$ admitting $\delta$-extensions
  such that $f_j(\Qua\cC_j)$ covers $\cC$ and $f_j(\R_+\Qua\cC_j)$
  covers $\R_+\cC$. If $\cC$ is algebraic of complexity $\beta$ then
  $\cC_j,f_j$ are algebraic of complexity $\poly_\ell(\beta)$.
\end{Prop}
\begin{proof}
  First, by subdividing $\cC$ using the real CPT we may assume that it
  admits a $\delta^{2^\ell}$-extension. Applying
  Proposition~\ref{prop:quadric-nu-cover} we see that $\cC_{\times\vnu}$
  admits a $\delta^2$-extension, is quadric, and after rescaling may
  also be assumed to be normalized. It will suffice to prove the claim
  for $\cC_{\times\vnu}$, so henceforth we replace $\cC$ by
  $\cC_{\times\vnu}$.

  We essentially want to use the collection of all the inversion maps
  $I_\vsigma$ to cover $\cC$ by $I_\vsigma(\Qua\cC)$ and $\R_+\cC$ by
  $I_\vsigma(\R_+\Qua\cC)$. The minor technical issue is that this
  does not cover the equators $\{\vz_j=\rho_j\}$. To avoid this
  problem, in the case $\sigma=(1,\ldots,1)$ we use a slightly larger
  cell $\tilde\cC$. Namely, we replace each fiber $\cF_j=A(r_j,1)$ by
  $\tilde\cF_j=A(\delta r_j,1)$. Since $\cC$ admits a
  $\delta^2$-extension $\tilde\cC$ admits a $\delta$-extension, and
  now the equator of $\cC$ is covered by $\Qua\tilde\cC$.
\end{proof}

The following proposition is our principal motivation for introducing
the notion of a quadric cell.

\begin{Lem}\label{lem:QC-derivative-bound}
  Let $\delta<1/4$ and let $\cC^\delta$ be a quadric cell and
  $f\in\O_b(\cC^\delta)$. Then
  \begin{equation}
    \norm{\pd f{\vz_j}}_{\Qua\cC} \le O_\ell(\norm{f}_{\cC^\delta}\cdot\delta)
    \qquad \text{for }j=1,\ldots,\ell.
  \end{equation} 
\end{Lem}
\begin{proof}
  We assume without loss of generality that $\norm{f}_{\cC^\delta}=1$.
  By Corollary~\ref{cor:laurent-disc-decmp},
  $f=\sum_\vsigma f_\vsigma(\vz^{[\vsigma]})$ with
  $\norm{f_\sigma}_{\cP^{2\delta}}=O_\ell(1)$. It will suffice to prove
  the claim for each of these summands, so fix
  $\vsigma\in\{-1,1\}^\ell$.

  If $\vsigma_j=1$ then
  \begin{equation}
    \pd{}{\vz_j} f_\vsigma(\vz^{[\vsigma]}) = (f_\sigma)'_j(\vz^{[\vsigma]}) +
    \sum_{k>j,\vsigma_k=-1} (f_\sigma)'_k(\vz^{[\vsigma]})\cdot \frac{1}{\vz_k}\cdot\pd{r_k}{\vz_j}.
  \end{equation}
  By the Cauchy formula
  \begin{equation}
    \norm{(f_\vsigma)'_j}_\cP=O_\ell(\delta \norm{f_\sigma}_{\cP^{2\delta}})=O_\ell(\delta).
  \end{equation}
  Also, since $\rho_k=\sqrt{r_k}$ is holomorphic of norm at most $1$
  in $\cC_{1..k-1}^\delta$ we have by induction
  \begin{equation}
    \norm{\frac{1}{z_k} \pd{r_k}{\vz_j}}_{\Qua\cC} = \norm{ 2 \pd{\sqrt{r_k}}{\vz_j} \frac{\sqrt{r_k}}{\vz_k}}_{\Qua\cC}\le
    O_\ell(\delta)
  \end{equation}
  where we used the fact that $\sqrt{r_k}/\vz_k<1$ in $\Qua\cC$.
  Combining these estimates we see that
  $\norm{\pd{}{\vz_j} f_\vsigma(\vz^{[\vsigma]})}_{\Qua\cC}=O_\ell(\delta)$.

  The case $\vsigma_j=-1$ is similar. We have
  \begin{equation}
    \pd{}{\vz_j} f_\vsigma(\vz^{[\vsigma]}) = -(f_\sigma)'_j(\vz^{[\vsigma]})\cdot\frac{r_j}{\vz_j^2} +
    \sum_{k>j,\vsigma_k=-1} (f_\sigma)'_k(\vz^{[\vsigma]})\cdot \frac{1}{\vz_k}\cdot\pd{r_k}{\vz_j},
  \end{equation}
  which can be estimated in the same manner noting that
  $\norm{\frac{r_j}{\vz_j^2}}_{\Qua\cC}\le 1$ by definition.
\end{proof}

\subsection{Straightening the positive quadrant}
\label{sec:quadrant-straighten}

Let $\cC$ be a normalized quadric cell admitting a $\delta=1/100$
extension. To simplify the notations we assume that $\cC$ contains no
fibers of type $D$ (we can replace each fiber of this type by fibers
of type $D_\circ$ and $*$). To further simplify the notation we
further assume that $\cC$ contains no fibers of type $*$ (if it does
then one should add additional single coordinates to $B$ defined below
in order to preserve the cellular structure of the map).

Denote $n:=\dim\cC$ and let $B:=(0,1)^{\times n}$.  We define a map
$h:B\to\R_+\cC^\delta$ inductively by
\begin{equation}\label{eq:h-def}
  h_i(\vu) = \vu_i+R_i(\vu_{1..i-1}), \qquad R_i:=\sqrt{r_i\circ h_{1..i-1}}.
\end{equation}
Here $r_i$ denotes the inner radius of $\cF_i$ if it is of type $A$,
and $0$ if it is of type $D,D_\circ$. Since $\cC$ is normalized it
follows that $R_+\Qua\cC\subset h(B)\subset \cC^\delta$. We think of
$h$ as a straightening of the positive quadrant. Our goal will be to
show that $h$ can be analytically continued to a sector
\begin{equation}
  S_\ell(\e) := \{|\Arg u_i|<\e, |u_i|<2 : i=1,\ldots,\ell\}
\end{equation}
for some positive $\e>0$. This will later be used to construct
parametrizations of $\R_+\cC$ with control on derivatives.

\begin{Prop}\label{prop:h-extension}
  There exists $\e>0$ such that $h$ can be analytically extended to
  $S_\ell(\e)$ and we have
  \begin{equation}\label{eq:h-domains}
    h(S_\ell(\e))\subset\Qua^{1/4}\cC\cap\{\Arg\vz_j\le\pi/4:j=1,\ldots,\ell\}.
  \end{equation}
  Moreover for any $j$ such that $\cF_j$ is an annulus we have
  \begin{equation}\label{eq:R_j-C1}
    \norm{\frac{\vu_i}{R_j} \pd{R_j}{\vu_i}}_{S_\ell(\e)} \le M_\ell,\quad i=1,\ldots,\ell
  \end{equation}
  for some $M_\ell>0$, i.e. $\log R_j$ is a $C^1$-bounded function of
  $\log\vu$. If $\cC$ is algebraic of complexity $\beta$ then
  $\e^{-1},M_\ell=\poly_\ell(\beta)$.
\end{Prop}
\begin{proof}
  By induction on $\ell$ we may assume that $h_{1..\ell-1}$ extends to
  $S_{\ell-1}(\e)$ for a sufficiently small $\e$, that
  \begin{equation}\label{eq:h-domains-ind}
    h_{1..\ell-1}(S_{\ell-1}(\e))\subset\Qua^{1/4}\cC_{1..\ell-1}\cap\{\Arg\vz_j\le\pi/4:j=1,\ldots,\ell\},
  \end{equation}
  and that \eqref{eq:R_j-C1} already holds for $R_{1..\ell-1}$ with an
  appropriate constant $M_{\ell-1}$ (as a shorthand notation we write
  $R_{1..\ell-1}$ to mean only those $R_j$ which are defined,
  i.e. such that $\cF_j$ is an annulus). Our first goal is to
  prove~\eqref{eq:R_j-C1} for $R_\ell$ assuming $\cF_\ell$ is an
  annulus.

  Since $\log R_{1..\ell-1}$ is a $C^1$-bounded function of
  $\log\vu_{1..\ell-2}$ we may, taking $\e<\pi/(4M_{\ell-1})$,
  suppose that on $S_{\ell-1}(\e)$ we have
  \begin{equation}\label{eq:R_i-arg-bound}
    |\Arg \vu_i|, |\Arg R_i| \le \pi/4, \qquad i=1,\ldots,\ell-1.
  \end{equation}
  Indeed $R_i$ is real on $\R S_{i-1}(\e)$, and since every point of
  $S_{i-1}(\e)$ lies at logarithmic distance at most $\e$ from this
  real part, it follows that $\log R_i$ and in particular $\Arg R_i$
  is at distance at most $M_{\ell-1}\cdot\e$ from zero. We note that
  this implies
  \begin{equation}\label{eq:u_i-vs-z_i}
    |\vu_i|,|R_i| \le |\vu_i+R_i|=|\vz_i|, \qquad i=1,\ldots,\ell-1.
  \end{equation}
  Now computing the left hand side of~\eqref{eq:R_j-C1} with $j=\ell$
  we have
  \begin{equation}\label{eq:R_ell-der}
    \frac{\vu_i}{R_\ell} \pd{R_\ell}{\vu_i} =
    \vu_i \left(\frac1{\sqrt{r_\ell}}\pd{\sqrt r_\ell}{\vz_i}\circ h_{1..\ell-1}\right)
    + \vu_i\sum_{i<j<\ell} \pd{R_j}{\vu_i} \left(\frac1{\sqrt{r_\ell}}\pd{\sqrt{r_\ell}}{\vz_j}\circ h_{1..\ell-1}\right).
  \end{equation}
  By~\eqref{eq:h-domains-ind} the image
  $h_{1..\ell-1}(S_{\ell-1}(\e))$ is contained in
  $\cC_{1..\ell-1}^{1/4}$. Applying
  Proposition~\ref{prop:log-derivative-bound} and
  using~\eqref{eq:u_i-vs-z_i} we have for some $K>0$
  \begin{gather}
    \abs{\vu_i \big(\frac1{\sqrt{r_\ell}}\pd{\sqrt r_\ell}{\vz_i}\circ h_{1..\ell-1}\big)}\le \abs{\frac{K\vu_i}{\vz_i}}\le K \\
    \abs{\frac1{\sqrt{r_\ell}}\pd{\sqrt{r_\ell}}{\vz_j}\circ h_{1..\ell-1}}\le \frac{K}{|\vz_j|} \le \frac{K}{|R_j|}.
  \end{gather}
  Plugging into~\eqref{eq:R_ell-der} and using the
  inductive~\eqref{eq:R_j-C1} we get
  \begin{equation}
    \norm{\frac{\vu_i}{R_\ell} \pd{R_\ell}{\vu_i}}_{S_{\ell-2}(\e)}\le K\left(1+\sum_{i<j<\ell} \abs{\frac{\vu_i}{R_j}\pd{R_j}{\vu_i}}\right)
    \le K\big(1+(\ell-i-1) M_{\ell-1}\big)
  \end{equation}
  as claimed. In the algebraic case $K=\poly_\ell(\beta)$ and by
  induction we have indeed $M_\ell=\poly_\ell(\beta)$.

  We now pass to proving that $h$ can be analytically extended to
  $S_\ell(\e)$, satisfying~\eqref{eq:h-domains}. By assumption
  $h_{1..\ell-1}$ is already extended and
  satisfies~\eqref{eq:h-domains-ind}. If $\cF_\ell$ is an annulus (the
  other cases are similar but easier) what remains to be verified is
  that
  \begin{equation}\label{eq:R_l-domain-cond}
    R_l(\vu_{1..\ell-1}) +\{ |\Arg\vu_\ell|<\e,|\vu_\ell|<2 \} \subset \{ |R_\ell|<|\vz_\ell|<4 \}
  \end{equation}
  for $\vu_{1..\ell-1}\in S_{\ell-1}(\e)$. Having
  established~\eqref{eq:R_j-C1} for $R_\ell$, we can now assume
  that~\eqref{eq:R_i-arg-bound} holds for $R_\ell$ as well (for a
  sufficiently small $\e$). Then~\eqref{eq:R_l-domain-cond} follows by
  an elementary geometric argument illustrated in
  Figure~\ref{fig:SellEpsilon}.
  \begin{figure}
    \centering
    \includegraphics[width=0.6\textwidth]{SellEpsilon.pdf}
    \caption{Analytic continuation of $h$ to $S_\ell(\e)$.}
    \label{fig:SellEpsilon}
  \end{figure}
\end{proof}

\begin{Cor}\label{cor:h-jacobian-bound}
  Let $S_\ell(\e)$ be as in Proposition~\ref{prop:h-extension}. Then
  the Jacobian of $h$ satisfies
  \begin{equation}
    \norm{\left[\pd{h_i}{\vu_j}\right] - I_{\ell\times\ell}}_{S_\ell(\e)} \le O_\ell(\delta).
  \end{equation}
\end{Cor}

\begin{proof}
  By induction on $\ell$ we may assume that the Jacobian of
  $h_{1..\ell-1}$ already satisfies the condition. It remains to
  produce a bound for each $\pd{h_\ell}{\vu_j}$ with $j<\ell$. We have
  \begin{equation}
    \pd{h_\ell}{\vu_j} = \pd{\big(\sqrt{R_\ell}\circ h_{1..\ell-1})}{\vu_j},
  \end{equation}
  and since the norm of the Jacobian of $h_{1..\ell-1}$ is $O_\ell(1)$
  it remains to show that the derivatives
  $\pd{\sqrt R_\ell}{\vz_k}\circ h$ are bounded by $O_\ell(\delta)$
  for $k=1,\ldots,\ell-1$. Since
  $h(S_\ell(\e))\subset\Qua^{1/4}\cC$ and $\sqrt{R_\ell}$ is
  bounded by $1$ in $\cC^\delta$, this follows from
  Lemma~\ref{lem:QC-derivative-bound}.
\end{proof}

The following is a direct corollary of
Lemma~\ref{lem:QC-derivative-bound} and
Corollary~\ref{cor:h-jacobian-bound}.

\begin{Cor}\label{cor:Phi-C1-norm}
  Let $F\in\cO_b(\cC^\delta)$ and let $\Phi:=F\circ h$. Then
  \begin{equation}
    \norm{\pd{\Phi}{\vu_i}}_{S_\ell(\e)}< O_\ell(\norm{F}\cdot\delta) \qquad i=1,\ldots,\ell.
  \end{equation}
\end{Cor}


\section{Smooth parametrization results}

In this section we use complex cells to prove the various refinements
of the Yomdin-Gromov algebraic lemma described
in~\secref{sec:intro-smooth}. All of these results follow fairly
directly from a parametrization result involving functions with a
holomorphic continuation to a complex sector, which we describe first.


\subsection{Sectorial parametrizations of subanalytic sets}
\label{sec:sectorial-param}

We say that $B\subset\R^\ell$ is a cube if it is a direct product of
intervals $(0,1)$ and singletons $\{0\}$. We will say that a map
$h:B\to\R^\ell$ is \emph{cellular} if $h_j$ depends only on
$\vx_{1..j}$. The \emph{sectorial cube} $B(\e)\subset\C^\ell$
corresponding to $B$ is the direct product where we replace each
interval $(0,1)$ by the sector $S(\e):=\{\Arg z<\e,|z|<2\}$. We call
$\e$ the \emph{angle} of $B(\e)$.

\begin{Thm}\label{thm:sectorial-param}
  Let $S\subset[0,1]^\ell$ be subanalytic. There exists a collection
  of cubes $B_1,\ldots,B_N$ and injective cellular maps
  $\phi_j:B_j\to S$ such that $S=\cup_j\phi_j(B_j)$. Moreover, there
  exists $\e>0$ such that each $\phi_j$ extends to a holomorphic map
  $\phi_j:B_j(\e)\to\cP_\ell^{1/2}$ with unit $C^1$-norm.

  If $S$ is semialgebraic of complexity $\beta$ then we have
  $N,1/\e=\poly_\ell(\beta)$.
\end{Thm}

\begin{proof}
  Set $\delta=1/100$ as in~\secref{sec:quadrant-straighten}.  We begin
  by applying Corollary~\ref{cor:cpt-semialg} in the semialgebraic
  case or Corollary~\ref{cor:cpt-subanalytic} in the subanalytic case
  with $\hrho<1/2$ and $\hsigma<\delta$, followed by
  Proposition~\ref{prop:quadric-cover} to construct a collection
  quadric normalized cells $\cC_j$ and real cellular maps
  $f_j:\cC_j^\delta\to\cP_\ell^{1/2}$ such that
  $f_j(\R_+\cC^\delta)\subset A$ and $f_j(\R_+\Qua\cC_j)$ covers
  $\R_+\cC$. By construction each $f_j$ is a composition of a prepared
  cellular map, a covering map and an inversion maps. Since each of
  these maps is injective on the positive real part we see that $f_j$
  is injective on $\R_+\cC_j$.
  
  Let $h_j:B_j(\e)\to\R_+\cC^\delta$ be the straightening map
  constructed for $\cC_j$ in~\secref{sec:quadrant-straighten} and set
  $\phi_j:=f_j\circ h_j$. The map $h_j$ is cellular and injective by
  construction, so $\phi_j$ is cellular and injective on $B_j$.  We
  have
  \begin{equation}
    A\subset \bigcup_j f_j(\R_+\Qua\cC_j) \subset \bigcup_j \phi_j(B_j) \subset A
  \end{equation}
  Finally, by Corollary~\ref{cor:Phi-C1-norm} the $C^1$-norm of
  $\phi_j$ is bounded by $O_\ell(1)$. Making an additional subdivision
  of $B_j$ into $O_\ell(1)$ parts and rescaling one can make this norm
  bounded by $1$, thus proving the claim.
\end{proof}

Later we will use the Cauchy estimate to pass from the sectorial
parametrizations of Theorem~\ref{thm:sectorial-param} to
parametrizations with bounded higher-order derivatives. Toward this
end the following simple combinatorial lemma will be useful.

\begin{Prop}\label{prop:Phi-decomp}
  Let $B(\e)$ be a sectorial cube and let $\Phi:B(\e)\to\C$ have
  $C^1$-norm $1$. There is a decomposition
  \begin{equation}
    \Phi(\vu) = \sum_{I\subset\{1,\ldots,\ell\}} \Phi_I(\vu)
  \end{equation}
  where $\Phi_I:B(\e)\to\C$ depends on $\vu_i:i\in I$ only, vanishes on
  $\cup_{i\in I}\{\vu_i=0\}$ and has $C^1$-norm bounded by
  $O_\ell(1)$. In particular
  \begin{equation}
    |\Phi_I(\vu)|\le O_\ell(\min_{i\in I}|\vu_i|), \qquad \text{for
    }\vu\in B(\e).
  \end{equation}
\end{Prop}
\begin{proof}
  We proceed by reverse induction on $j$, defined as the minimal
  $i\in\{1,\ldots,\ell\}$ such that $\Phi$ depends on $\vu_j$ and does
  not vanish identically on $\{\vu_j=0\}$. If the condition defining
  $j$ is empty then we may take $\Phi=\Phi_I$ where $I$ denotes the
  set of indices that $\Phi$ depends on. Otherwise write
  \begin{equation}
    \Phi=(\Phi-R)+R, \qquad R(\vu) = \Phi(\vu_1,\ldots,\vu_{j-1},0,\vu_{j+1},\ldots,\vu_\ell).
  \end{equation}
  Since the first summand vanishes on $\{\vu_j=0\}$ and the second
  summand does not depend on $\vu_j$ we may finish for each of them by
  induction (the $C^1$-norm of the summands are majorated by $2,1$
  respectively).
\end{proof}

\subsection{The algebraic reparametrization lemmas}

We give more detailed statements of the algebraic lemmas in the
$C^r$-smooth and mild contexts, and show that these statements imply
the statements appearing in~\secref{sec:intro-smooth}.  Our result in
the $C^r$ case is as follows.

\begin{Thm}\label{thm:Cr}
  Let $X\subset[0,1]^\ell$ be subanalytic. There exists
  $A>0$, cubes $B_1,\ldots,B_C$ and subanalytic injective
  cellular maps $\phi_j:B_j\to X$ whose images cover $X$, with
  \begin{equation}\label{eq:Cr-norm-bound}
    \norm{D^\valpha \phi_j} \le \valpha!(A r)^{|\valpha|}, \quad
    \text{for } |\valpha|\le r.
  \end{equation}
  If $X$ is semialgebraic of complexity $\beta$ then
  $A,C=\poly_\ell(\beta)$ and the maps $\phi_j$ are semialgebraic of
  complexity $\poly_\ell(\beta,r)$.
\end{Thm}

The cellular structure and injectivity of the parametrizing maps
implies that the result of Theorem~\ref{thm:Cr} is automatically
uniform over families. Indeed, suppose $X\subset\R^{n+m}$ is a bounded
subanalytic set viewed as a family $X_p$ of subsets of
$\R^m$. Construct $\phi_1,\ldots,\phi_N$ as above. For any $p\in\R^n$
and any $j=1,\ldots,N$, the fiber of $(\phi_j)_{1..n}^{-1}(p)$ is
either empty or a cube $B_{p,j}\subset\R^m$. The $\phi_j$ of the
latter type restrict to parameterizing maps for $X_p$ with the same
$C^r$-norm bound~\eqref{eq:Cr-norm-bound} (now in $m$ variables). One
can subdivide the domains of the $\phi_j$ into smaller cubes to obtain
a parametrization with unit $C^r$ norms. Let
$\mu:=\dim X_p$. Subdividing $B_j$ into $N^\mu$ subcubes of side length
$1/N$ and rescaling back to the unit cube gives maps
$\phi_{j,k}:B_{j,k}\to\R^\ell$ with
\begin{equation}
  \frac{\norm{D^\valpha \phi_{j,k}}}{\valpha!} \le \left(\frac{A r}N\right)^{|\valpha|}, \quad
  \text{for } |\valpha|\le r.
\end{equation}
Taking $N=Ar$ one obtains unit norms above. This gives the $C^r$
statements of~\secref{sec:intro-smooth}.

Our result in the mild case is as follows.

\begin{Thm}\label{thm:mild}
  Let $X\subset[0,1]^\ell$ be subanalytic. There exists $A<\infty$,
  cubes $B_1,\ldots,B_C$ and $(A,2)$-mild injective cellular maps
  $\phi_j:B_j\to X$ whose images cover $X$.

  If $X$ is semialgebraic of complexity $\beta$ then one may take
  $A,C=\poly_\ell(\beta)$.
\end{Thm}

As in the case of Theorem~\ref{thm:Cr}, the result of
Theorem~\ref{thm:mild} is automatically uniform over families due to
the cellular structure and the injectivity of the maps.

\subsection{Proofs of the smooth parametrization results}
\label{sec:param-proofs}

Let $X$ be as in Theorem~\ref{thm:Cr} or Theorem~\ref{thm:mild}. By
Theorem~\ref{thm:sectorial-param} there exists a collection of cubes
$B_1,\ldots,B_N$ and injective cellular maps $\phi_j:B_j\to X$ such
that $X=\cup_j\phi_j(B_j)$, and each $\phi_j$ extends to a holomorphic
map $\phi_j:B_j(\e)\to\C^\ell$ with unit $C^1$-norm. Moreover in the
algebraic case $N,1/\e=\poly_\ell(\beta)$. It will be enough to prove
the parametrization results for each $\phi_j(B_j)$ separately, so we
fix a pair $(B,\phi):=(B_j,\phi_j)$ and proceed to consider the
problem of parametrizing $\phi(B)$.

Let $\Phi$ denote one of the coordinates of $\phi$ and recall that
$\Phi$ has unit $C^1$-norm. To control higher derivatives we will
consider the map $\tilde\phi:=\phi\circ S$ where $S$ is a
``flattening'' bijection $S:B\to B$. We will see that an appropriate
choice of $S$ (one for the $C^r$ version and another for the mild
version) allows one to control the higher derivatives of $\Phi\circ S$
using the Cauchy estimate.

To simplify the notation, we will suppose below that $B$ is of the
form $(0,1)^\ell$ (in general the direct product may also contain
copies of $*$, but these obviously have no impact on parametrization
questions). We will take $\Phi$ to be any function of unit $C^1$-norm
in $B(\e)$.

\subsubsection{Proof of the $C^r$-parametrization result}
\label{sec:params-proofs-cR}

Consider the bijection
\begin{equation}
P_r:B\to B, \qquad P_r(\vw_{1..\ell})=(\vw_1^r,\ldots,\vw_\ell^r).
\end{equation}
In the notation of~\secref{sec:param-proofs}, Theorem~\ref{thm:Cr}
follows from the following lemma.

\begin{Lem}
  The function $\Phi\circ P_r$ is has bounded $C^r$-norm,
  \begin{equation}
    \abs{(\Phi\circ P_r)^{(\valpha)}(\vw)}\le  \valpha! \left(O_\ell(r/\e)\right)^{|\valpha|}
    \qquad\text{for }|\valpha|\le r.
  \end{equation}
\end{Lem}
\begin{proof}
  Since $\Phi$ extends holomorphically to $S_\ell(\e)$, the
  composition $\Phi\circ P_r$ extends holomorphically to
  \begin{equation}
    \Omega:=S_\ell(\e/r)\cap\{|\zeta_i|<2^{1/r}, i=1,\ldots,\ell\}.
  \end{equation}
  In particular for every $\vw\in B$ it contains the polydisc
  $D_\vw=\{\abs{\xi_i-\vw_i}\le (\sin\frac{\e}{r}) \vw_i\}$. Denote by
  $\cSc (D_\vw)$ its skeleton.

  Using Proposition~\ref{prop:Phi-decomp} we see that 
  \begin{equation}
    (\Phi\circ P_r)^{(\valpha)}(\vw)
    =\sum_I (\Phi_I\circ P_r)^{(\valpha)}(\vw)
    =\sum_{\supp{\valpha}\subset I} (\Phi_I\circ P_r)^{(\valpha)}(\vw),
  \end{equation}
  so it is enough to consider $(\Phi_I\circ P_r)^{(\valpha)}(\vw)$
  only for those summands with $\supp{\valpha}\subset I$. Applying the
  Cauchy formula we have
  \begin{multline}\label{eq:Cauchy-Cr}
    \abs{(\Phi_I\circ P_r)^{(\valpha)}(\vw)} =
    \abs{(2\pi i)^{-\ell}\valpha!\int_{\cSc(D_\vw)}
      \frac{\Phi_I\circ P_r(\xi)}{\prod (\xi_i-\vw_i)^{\valpha_i+1} }d\xi_1\dots d\xi_\ell}\\
    \le\valpha!\prod 
       \left(\vw_i\sin\frac{\e}{r}\right)^{-\valpha_i}\max_{\xi\in\cSc(D_\vw)}\abs{\Phi_I\circ
       P_r(\xi)}.
  \end{multline}
  Denote $\vw_{\min}:=\min_{i\in I} \vw_i$. By
  Proposition~\ref{prop:Phi-decomp} we have
  \begin{equation}
    \abs{\Phi_I\circ P_r(\xi)}\le O_\ell(1)\cdot \min_{i\in I}\abs{\xi_i^r}
    \le O_\ell(1)\cdot \abs{\vw_{\min}^r},
  \end{equation}
  where in the last inequality we used
  $\left(1+\sin\tfrac{\epsilon}{r}\right)^r<e$. Then~\eqref{eq:Cauchy-Cr}
  implies
  \begin{equation}
    \abs{(\Phi_I\circ P_r)^{(\valpha)}(\vw)}\le 
    O_\ell(1) \left(\sin\frac{\epsilon}{r}\right)^{-\abs{\valpha}}\valpha!
    \vw_{\min}^{r-\abs{\valpha}}\le \valpha! \left(O_\ell(r/\e) \)^{|\valpha|}
  \end{equation}
  as long as $|\valpha|\le r$.
\end{proof}

\begin{figure}
  \centering
  \includegraphics[width=\textwidth]{Ck.pdf}
  \caption{Reparametrization of $S_\ell(\e)$ in the $C^r$ case.}
  \label{fig:Ck}
\end{figure}

\subsubsection{Proof of the mild parametrization result}

Consider the bijection
\begin{equation}
  \Exp:B\to B, \qquad
  \Exp(\vw_{1..\ell})=\left(\exp\left(1-\frac1{\vw_1}\right),\dots,\exp\left(1-\frac1{\vw_\ell}\right)\right).
\end{equation}
In the notation of~\secref{sec:param-proofs}, Theorem~\ref{thm:mild}
follows from the following lemma.

\begin{Lem}
  The function $\Phi\circ\Exp$ is $(A,2)$-mild,
  \begin{equation}
    \abs{(\Phi\circ\Exp)^{(\valpha)}(\vw)}\le  \valpha!(A|\valpha|^2)^{|\valpha|}
    \qquad\text{for } \valpha\in\N^\ell
  \end{equation}
  with $A=O_\ell(1/\e)$.
\end{Lem}
\begin{proof}
  Since $\Phi$ extends holomorphically to $S_\ell(\e)$, the
  composition $\Phi\circ\Exp$ extends holomorphically to the domain
  $\Omega:=\Exp^{-1}S_\ell(\e)$ and it is easy to check that for any $\vw\in B$,
  \begin{equation}
    D_\vw:=\{\abs{\xi_i-\vw_i}\le \frac 1 2\e \vw_i^2\}\subset\Omega.
  \end{equation}
  Denote by $\cSc(D_\vw):=\{\abs{\xi_i-\vw_i}= \frac 1 2\e \vw_i^2\}$
  the skeleton of $D_\vw$.

  Using Proposition~\ref{prop:Phi-decomp} in the same way as
  in~\secref{sec:params-proofs-cR} we see that it is enough to
  consider $(\Phi_I\circ\Exp)^{(\valpha)}(\vw)$ only for those
  summands with $\supp{\valpha}\subset I$. Applying the Cauchy formula
  we have
  \begin{multline}\label{eq:Cauchy-mild}
    \abs{(\Phi_I\circ\Exp)^{(\valpha)}(\vw)} =
    \abs{(2\pi i)^{-\ell}\valpha!\int_{\cSc(D_\vw)}
      \frac{\Phi_I\circ\Exp(\xi)}{\prod (\xi_i-\vw_i)^{\valpha_i+1} }d\xi_1\dots d\xi_\ell}\\
    \le\valpha!\prod 
    \left(\vw_i^2 \frac\e2\right)^{-\valpha_i}\max_{\xi\in\cSc(D_\vw)}\abs{\Phi_I\circ
      \Exp(\xi)}.
  \end{multline}
  By Proposition~\ref{prop:Phi-decomp} we have 
  \begin{multline}
    \abs{\Phi_I\circ\Exp(\xi)}\le
    O_\ell(1) \min_{i\in I}\abs{\exp\left(1-\xi_i^{-1}\right)} \\
    \le O_\ell(1)/ \exp(\max_{i\in I}\Re \xi_i^{-1}) 
    \le O_\ell(1)/ \exp(\max_{i\in I}\Re \vw_i^{-1}),
  \end{multline}
  where in the last inequality we used the fact that for $w<\e^{-1}$,
  the mapping $\xi\mapsto1/\xi$ maps the circle
  $\{\abs{\xi-w}\le \frac 1 2\e w^2\}$ inside the circle
  $\{\abs{\xi-w^{-1}}<2\}$.
  
  Denote $\vw_{\min}=\min_{i\in I} \vw_i$. Then~\eqref{eq:Cauchy-mild}
  implies
  \begin{equation}
    \begin{aligned}
      \abs{(\Phi_I\circ\Exp)^{(\valpha)}(\vw)}&\le 
      \valpha! O_\ell(1) (2/\e)^{|\valpha|} \vw_{\min}^{-2\abs{\valpha}}e^{-\frac{1}{\vw_{\min}}} \\
      &\le \valpha! O_\ell\left((2/\e)^{|\valpha|}\right) {\abs{\valpha}}^{2\abs{\valpha}}e^{-2\abs{\valpha}}
    \end{aligned}
  \end{equation}
  where the final estimate is a simple maximization exercise (with the
  maximum attained at $\vw_{\min}=-\frac{1}{2\abs{\valpha}}$).
\end{proof}

\begin{figure}
  \centering
  \includegraphics[width=\textwidth]{mild.pdf}
  \caption{Reparametrization of $S_\ell(\e)$ in the mild case.}
  \label{fig:mild}
\end{figure}


\subsection{Uniform parametrizations in $\R_\an$}
\label{sec:sec:uniform-param-Ran}

In this section we give a proof of
Proposition~\ref{prop:hyp-param}. We begin by proving
Lemma~\ref{lem:log-length}.

\begin{proof}[Proof of Lemma~\ref{lem:log-length}]
  It is enough to prove that $\log\xi$ has derivative bounded by
  $4\pi p(p+1)$ in $[-1,1]$. Note that $\log\xi$ is well defined and
  $p$-valent in $D(2)$. Let $r:=|(\log\xi)'(z_0)|$ for some
  $z_0\in D(1)$. By the $p$-valent analog of the Koebe 1/4-theorem
  $\log\xi(z_0+D(1))$ contains a disc of radius at least
  $1/4p\cdot r$, see \cite[Theorem~5.1]{hayman:book}. On the
  other hand, it cannot contain a disc of radius larger than
  $\pi (p+1)$ since then $\xi=e^{\log\xi}$ would not be
  $p$-valent. Thus $r\le 4\pi p(p+1)$ as claimed.
\end{proof}

We now proceed to the proof of Proposition~\ref{prop:hyp-param}. As a
first step we will reduce to the case of a single complex cell. Let
$G_F\subset[0,1]^2\times[0,1]^2$ denote the graph of $F(e,t)$.
According to Theorem~\ref{cor:cpt-subanalytic} and
Remark~\ref{rem:cpt-semi-extra} there exists a collection of prepared
cellular maps $f_j:\cC_j^{1/4}\to\cP_4^{1/2}$ such that
$f_j(\R_+\cC_j)$ covers $G_F$ and
$f_j(\R_+\cC_j^{1/2})\subset G_F$. By Remark~\ref{rem:cpt-semi-extra}
we may also assume that $f_j$ is compatible with the function $e$. Let
$\cC$ denote one of the cells $\cC_j$ and $f:=f_j$. It will be enough
to prove that
\begin{equation}\label{eq:log-length}
  \log [f(\R_+\cC)\cap\{e=e_0\}]
\end{equation}
has bounded Euclidean length independent of $e_0\neq0$.

We first note that since $\cC$ maps into a graph, its type must end
with $**$. We proceed to consider the first two fiber types. If the
type of $\cC$ begins with $*$ then $f(\R_+\cC)$ meets only one line
$e=e_0$ and the length of~\eqref{eq:log-length} is bounded by the
(finite) log-length of the hyperbola $xy=e_0$ in $[0,1]^2$. Similarly
if the type of $\cC$ begins with $D$ or $A$ then since $f_j$ is
compatible with $e$ we conclude that $f_j(\R_+\cC)$ meets only lines
$e=e_0$ with $e_0$ bounded away from $0$, and the same argument
holds. The only non-trivial case is therefore when the type of $\cC$
begins with $D_\circ$ and $f$ maps this punctured disc to a disc
centered around $e=0$.

Let $\cC$ be of the form $\cC=D_\circ(R)\odot \cF\odot*\odot *$. If
$\cF=*$, then $f(\R_+\cC)$ meets, for every $e_0$, only one point in
$xy=e_0$. Otherwise we may write $f$ in the form
\begin{equation}
  (e,t,x,y) = f(\e,\tau,*,*) = 
  (\e^m,\tau^n+\phi(\e),\xi(\e,\tau),\eta(\epsilon,\tau)).
\end{equation}
If $\cF=D(r)$, then the functions $\xi(\e,\tau),\eta(\epsilon,\tau)$
are $p$-valent in $\tau\in D^{1/2}(r)$, for some $p$ uniform over
$\e\in D_\circ(R)$ by subanalyticity of $f$ restricted to $\cC^{1/2}$
(see Proposition~\ref{prop:cell-subanalytic}). The claim then follows
from Lemma~\ref{lem:log-length}. If $\cF=D_\circ(r)$, then by
removable singularity theorem both $\xi(\e,\tau)$ and
$\eta(\epsilon,\tau)$ can be extended to $D_\circ(R)\odot D(r)$, with
extensions satisfying the same relation $\xi\cdot\eta=\e^m$ for
$\e\in D_\circ(R)$. In particular, both $\xi(\e,\tau)$ and
$\eta(\epsilon,\tau)$ do not vanish, and we can proceed as above.

The remaining case is $\cF=A(r_1,r_2)$. We now make two
reductions. First, we replace $f$ with
\begin{equation}
(e,t,x,y) = g(\e,\tau,*,*) =\left(\e^m, \tau^n/r_2^n(\e),\xi(\e,\tau), 
\eta(\e,\tau)\right)
\end{equation}
defined on the same cell. This amounts to an affine transformation of
the time variable. Evidently, the image of $\xi,\tau$ is
unchanged, and the constant $M_r$ in \eqref{eq:derivatives bounded} is
replaced by $\abs{r_2^{nr}(\e)}M_r$, which is again uniformly bounded
in $\e$. Second, we replace $\cC$ with
$\tilde{\cC}=D_\circ(R)\odot A(r,1)\odot*\odot*$ for
$r:=r_1(\e)/r_2(\e)$, and replace $g$ with the composition
\begin{equation}
(e,t,x,y) = \tilde{g}(\e,\tau,*,*)=g\circ\pi 
=\left(\e^m, \tau^n,\xi(\e,\tau r_2(\e)), \eta(\e,\tau r_2(\e))\right),
\end{equation}
where $\pi:\tilde{\cC} \to\cC$ is the mapping
$(\e,\tau,*,*)\to (\e,\tau r_2(\e),*,*)$. This only amounts to a
reparametrization of the set $g(\cC)$ leaving both the image of
$\xi,\eta$ and the bounds \eqref{eq:derivatives bounded} unchanged.
Finally by a similar rescaling of $\e$ we may assume that $R=1$.
Without loss of generality we return to our original notation,
assuming now that $\cC=D_\circ(1)\odot A(r,1)\odot*\odot*$ and
$f(\e,\tau,*,*)=\left(\e^m, \tau^n,\xi(\e,\tau), \eta(\e,\tau)\right)$.

Decompose $\eta$ as
\begin{equation}
\eta(\e,\tau)=a(\e,\tau)+b(\e,\tfrac{r}{\tau}),
\end{equation}
where both $a,b$ are holomorphic and bounded in a neighborhood of
$\cP_2^{1/2}$. We have
$\partial_t =\frac{\tau^{1-n}}{n} \partial_\tau$ and a straightforward
induction shows that if $\partial_t^r\eta$ is bounded then
$\partial_\tau^r\eta$ is bounded as well. Since $\partial_\tau^r a$ is
bounded uniformly in $\e$ by the Cauchy estimates for $a$, we conclude
that $\partial_\tau^r b(\e, \tfrac{r}{\tau})$ is also bounded
uniformly in $\e$. We claim that this is impossible unless
$b\equiv b(\e)$ or $r\sim 1$. In the former case $\eta$ itself extends
as a holomorphic function to $\cP_2^{1/2}$ and we return to the case
of functions on discs treated above using
Lemma~\ref{lem:log-length}. In the latter case, we may (for instance
following the proof of the refinement theorem) cover
$D_\circ(1)\odot A(r,1)$ by finitely many cells of the type
$D_\circ(1)\odot D(1)$ return again to the disc case.

It remains to prove our claim above. Let $v=\tfrac{r}{\tau}$, so
$b(\e,v)$ is holomorphic and bounded in a neighborhood of
$\cP^{1/2}$. Suppose
\begin{align}
  r(\e)&=\e^m(c+\dots),& &m>0,c\neq0 \\
  b(\e, v)&=\sum_{\ell\ge0} \e^\ell b_\ell(v),& &b_\ell\in\cO_b(\D^{1/2}),
\end{align}
and let $\ell_0$ be the first index such that $b_{\ell_0}$ is non-constant.
A simple computation using $\partial_\tau = -\frac{v^2}{r}\partial_v$ gives 
\begin{equation}
  \partial_\tau^k b(\e, v)=\sum_{\ell\ge\ell_0-mk} \e^\ell \tilde{b}_{\ell,k}(v),
\end{equation}
where
\begin{equation}
  \tilde b_{\ell_0-mk,k} = \left(-\tfrac{v^2}{c}\partial_v\right)^k b_{\ell_0}(v)\not\equiv0
\end{equation}
and the final inequality follows since $b_{\ell_0}$ is non-constant,
and the operator $(v^2\partial_v)^k$ does not kill any non-constant
holomorphic functions as can be seen by examining its action on Taylor
expansions. For $k$ such that $\ell_0-mk<0$ we see that
$\partial_\tau^kb(\e, v)$ has a pole in $\e=0$ for any $v$ such that
$\tilde b_{\ell_0-mk,k}(v)\neq0$ and is therefore unbounded as
$\e\to0$, contradicting our assumption.


\appendix

\section{Applications in dynamics}
\begin{center}
  \footnotesize YOSEF YOMDIN
\end{center}
\label{appendix:yomdin}

\subsection{Non-dynamical parametrization results}
\label{Sec:Non.Dyn}

In this section we present a non-dynamical result, based on the
refined algebraic lemma. This result reduces the problem of bounding
dynamical local entropy (considered in~\secref{sec:Dyn.Appl} below) to
a ``combinatorial'' analysis of the complexity of long compositions of
smooth functions. Set $Q:=[0,1]$. Our basic unit of complexity is
defined as follows.

\begin{Def}[$k$-complexity unit ($k$-cu)]
  A $k$-complexity unit is a map $\phi:Q^l\to\R^m$ with
  $\norm{\phi}_k\le1$. We say that $\{\phi_q:Q^l\to\R^m\}$ is a $k$-cu
  cover of a set $A\subset\R^m$ if $A$ is contained in the union of
  the images of $\phi_q$.
\end{Def}

Theorem~\ref{thm:non.dyn} below can be considered as a kind of
``Taylor formula'' for $C^k$-parametrizations: it measures the
complexity of the graph of a $C^k$-map in terms of a covering by
$k$-cu's; exactly as in the usual Taylor formula, the estimates depend
only on the $k$-th derivative. A result of this type appears (in a
weaker form) in \cite[Theorem~2.1]{yomdin:entropy}. In
\cite{gromov:gy} it is improved and split into ``Main lemma'' and
``Main corollary''. Following \cite{gromov:gy}, we split below
Theorem~2.1 of \cite{yomdin:entropy} into Theorem~\ref{thm:non.dyn}
and Corollary~\ref{cor:non.dyn}, and incorporate other improvements
due to Gromov. Below we denote by $G_g$ the graph of a map $g$.

\begin{Thm}\label{thm:non.dyn}
  Let $g:Q^l\to\R^m$ be a $C^k$-mapping such that
  $\max_{x\in Q^l}\norm{d^kg}\le M.$ Then $G_g\cap Q^{l+m}$ admits a
  $k$-cu cover of the form $\{(\phi_q,g\circ\phi_q)\}$ of size
  $\poly_{l,m}(k)M^{l/k}$.
\end{Thm}

\begin{proof}
  Let $S:=\{x\in Q^l:\norm{g(x)}\le1\}$. We subdivide $Q^l$ into
  sub-cubes $Q^l_j$ of size $\gamma=(kM^{1/k})^{-1}$ and parametrize
  each $Q^l_j$ by an affine mapping $\eta_j:Q^l\to Q^l_j$. The number
  $N_1$ of the sub-cubes is $k^lM^{l/k}$. Since the $k$-th derivative
  under this parametrization is multiplied by $\gamma^k$, we get for
  $\theta_j:=g\circ \eta_j$ the bound $\norm{d^k\theta_j}\le k^{-k}$.

  Let $P_j$ be the Taylor polynomial of $\theta_j$ of degree $k-1$ at
  the center of $Q^l$. By the Taylor remainder formula we have
  \begin{equation}\label{eq:Taylor.appr.norm}
    \norm{d^l\theta_j-d^lP_j}\leq \frac{1}{(k-l)!k^k}, \qquad l=0,\ldots,k.
  \end{equation}
  Put $S_j=\{x\in Q^l: \ ||P_j(x)||\le \frac{3}{2}\}.$
  By~\eqref{eq:Taylor.appr.norm} we have $|\theta_j-P_j|< \frac{1}{2},$
  and hence the sets $\eta_j(S_j)$ for $j=1,\ldots,N_1$ cover $S$.

  In the next step, which is central for all the construction, we
  apply the refined algebraic lemma to the graph of $P_j$ over $S_j$
  with smoothness order $k$. We get $k$-cu's $\psi_{ij}:Q^l\to S_j$
  for $i=1,\ldots,N_2$ with the images of $\psi_{ij}$ covering $S_j$,
  such that $P_j\circ \psi_{ij}$ are also $k$-cu's. Moreover
  $N_2\le\poly_{l,m}(k)$. Now we estimate the derivatives
  $(\theta_j\circ \psi_{ij})^{(s)}$ for $s=1,\ldots,k$, comparing them
  with the derivatives of $P_j\circ \psi_{ij}$ via the Fa\`a di Bruno
  formula:
  \begin{equation}\label{eq:Faa.di.Bruno1}
    (F\circ G)^{(s)}= \sum_{l=1}^s B_s^l(G',G'',\ldots,G^{(s-l+1)})\cdot F^{(l)}\circ G,
  \end{equation}
  with $B^l_s$ the so-called Bell polynomials, satisfying in
  particular
  \begin{equation}
    B_s:=\sum_{l=1}^s B^l_s(1,\ldots,1)<s^s.
  \end{equation}
  This formula is linear with respect to $F$, and applying it to
  $F=\theta_j-P_j$ and $G=\psi_{ij}$ we get for $s=1,\ldots,k$,
  \begin{multline}
    |(\theta_j\circ \psi_{ij})^{(s)}- (P_j\circ \psi_{ij})^{(s)}|= |((\theta_j-P_j)\circ \psi_{ij})^{(s)}|\leq \\
    \sum_{l=1}^s B_s^l(\psi_{ij}',\psi_{ij}'',\ldots,\psi_{ij}^{(k-l+1)})\cdot |(\theta_j-P_j)^{(l)}\circ \psi_{ij}| \le 
    \frac{B_s}{k^k}\le 1,
  \end{multline}
  by the bounds~\eqref{eq:Taylor.appr.norm}, and because $\psi_{ij}$
  are $k$-cu's. Since $P_j\circ \psi_{ij}$ is also a $k$-cu we
  conclude that $|(\theta_j\circ \psi_{ij})^{(s)}|\le2$. Therefore,
  another subdivision of $Q^l$ into sub-cubes of size $\frac{1}{2}$
  and the corresponding affine parametrizations of these sub-cubes
  reduces the derivative bounds to $1$. We denote by
  $\{\psi_{i,j,p}\}$ the collection of
  $N_1N_2 2^l=\poly_{l,m}(k)M^{l/k}$ maps obtained in this way. Set
  $\phi_{i,j,p}:=\eta_j\circ\psi_{i,j,p}$. Then $\phi_{i,j,p}$ is a
  $k$-cu since $\psi_{i,j,p}$ is, and
  $g\circ\phi_{i,j,p}=\theta_j\circ\psi_{i,j,k}$ is a $k$-cu as shown
  above.
\end{proof}

Next we modify Theorem \ref{thm:non.dyn} to make it suitable to apply
to $n$-fold iterations
$f^{\circ n}=f\circ f \circ \ldots \circ f$. The main reason to
separate Theorem~\ref{thm:non.dyn} from Corollary~\ref{cor:non.dyn}
below is that in the latter it is not enough to assume that
$||d^kg||\le M$ only for the highest derivative: we need
$||d^sg||\le M$ for all $s=1,\ldots,k$.

\begin{Cor}\label{cor:non.dyn}
  Let $g:Q^l\to\R^m$ be a $C^k$-mapping such that
  $\max_{x\in B'}\norm{d^sg}\le M$ for $s=1,\ldots,k$. Let
  $\sigma:Q^l\to Q^l$ be a $k$-cu and put $h=g\circ\sigma$.  Then
  $G_h\cap Q^{l+m}$ admits a $k$-cu cover of the form
  $\{(\phi_q,h\circ\phi_q)\}$ of size $\poly_{l,m}(k)M^{l/k}$.
\end{Cor}
\begin{proof}
  By the Fa\`a di Bruno formula~\eqref{eq:Faa.di.Bruno1} we have for
  the $k$-th derivative of $h$
  \begin{equation}\label{eq:Faa.di.Bruno2}
    |h^{(k)}|=|(g\circ \sigma)^{(k)}|= |\sum_{l=1}^k B_k^l(\sigma',\sigma'',\ldots,\sigma^{(k-l+1)})\cdot g^{(l)}\circ \sigma| < k^kM.
  \end{equation}
  Now we apply Theorem~\ref{thm:non.dyn} to $h$, and get the required
  covering by $k$-cu's.
\end{proof}

\subsection{Applications to volume growth and entropy}
\label{sec:Dyn.Appl}

Let $Y$ be a compact $m$-dimensional $C^\infty$-smooth Riemannian
manifold, and let $f:Y\to Y$ be a smooth (at least $C^{1+\mu}$, with
$\mu>0$) mapping. The topological entropy $h(f)$ was defined
in~\secref{sec:intro-dynamics}. Below we consider another
``entropy-type'' dynamical invariant called the \emph{volume growth}.

\subsubsection{Volume growth}

For $\sigma^l:Q^l\to Y$ a $C^k$-mapping, and for $\cS\subset Q^l$ a
measurable subset, let $v(\sigma^l,\cS)$ denote the $l$-dimensional
volume of the image of $\sigma^l(\cS)$ in $Y$. Put
$v(f,\sigma^l,\cS,n)=v(f^{\circ n}\circ \sigma^l,\cS)$, and put
\begin{align}\label{eq:vol.growth}
  v(f,\sigma^l,\cS)&= \limsup_{n\to \infty} \ \frac{1}{n} \log v(f,\sigma^l,\cS,n), &
  v(f,\sigma^l)&=v(f,\sigma^l, Q^l).
\end{align}
Finally, we put
\begin{align}
  v_{l,k}(f)&=\sup_{\sigma^l}v(f,\sigma^l), & v_{k}(f)&=\max_{l} v_{l,k}(f), & v(f)&=v_\infty
\end{align}
where the supremum is taken over all $C^k$ maps
$\sigma^l:Q^l\to Y$. An important inequality obtained by Newhouse in
\cite{newhouse:entropy-volume} connects the volume growth to the
topological entropy,
\begin{equation}\label{eq:Newhouse}
  h(f)\le v(f).
\end{equation}
The inverse inequality is not always true, but the question of its
validity is important in many dynamical applications. In particular,
as it was explained in~\secref{sec:intro-yomdin-thm}, this inverse
inequality implies Shub's entropy conjecture. In order to analyze its
validity we define a $\delta$-local volume growth at $x\in Y$ as
\begin{equation}
  v_{l,k}(f,\delta,x)=\sup_{\sigma^l} \ v(f,\sigma^l,B^n_{\delta}(x))
  = \sup_{\sigma^l} \limsup_{n\to \infty} \ \frac{1}{n} \log v(f,\sigma^l,B^n_{\delta}(x),n)
\end{equation}
where $B^n_{\delta}(x)$ is the $\delta$-ball around $x$ in the $n$-th
iterated metric $d_n$. The quantity $v_{l,k}(f,\delta,x)$ measures the
exponential rate of volume growth of the part of the images of
$f^{\circ n}\circ \sigma^l$ which always stay at a distance at most
$\delta$ from the orbit of $f^{\circ n}(x)$ for $n\to\infty$. Finally,
we put
\begin{align}
  v_{l,k}(f,\delta)&=\sup_{x\in Y} v_{l,k}(f,\delta,x), & v^0_{l,k}(f)&=\lim_{\delta\to 0} v_{l,k}(f,\delta).
\end{align}
The $\delta$-tail entropy can be also be defined in an analogous
manner setting $h^*(f,\delta,x)=h(f,B^n_{\delta}(x))$ and
$h^*(f,\delta):=\sup_{x\in Y} h^*(f,\delta,x)$. There is a version of
Newhouse's inequality~\eqref{eq:Newhouse} for these invariants.

\subsubsection{Local volume growth and entropy}

The following partial inverse to~\eqref{eq:Newhouse} is
almost immediate:
\begin{equation}
  v_{l,k}(f)\le h(f,\delta)+v_{l,k}(f,\delta).  
\end{equation}
Indeed, $h(f,\delta)$ counts the minimal number of the balls
$B^n_{\delta}(x)$ covering $Y$, while $v_{l,k}(f,\delta)$ bounds the
volume growth on each of these balls. As $\delta\to 0$ we obtain the
bound
\begin{equation}
  v_{l,k}(f)\le h(f)+v^0_{l,k}(f).  
\end{equation}
Thus the inverse to the Newhouse inequality is satisfied if
$v^0_{l,k}(f)=0$. Our first goal is to establish the following result.

\begin{Thm}[\protect{\cite{yomdin:entropy}}]\label{thm:yomdin-Cinf-volume}
  If $f:Y\to Y$ is $C^\infty$-smooth then $v_{l,k}^0(f)=0$.
\end{Thm}

% Local volume growth and entropy allow one also to compare the entropy $h(f)$ and its ``before limit'' approximations. In particular, as it was mentioned in Section 1.3 that 
% $$
% |h(f)-h(f,\e)|\le h^*(f,\e).
% $$
% Similar bound can be given for $|h(f)-h(f,n,\e)|$ (compare, e.g, \cite{Bur.Lia.Yan,Yom2}). Notice that the ``before limit'' $(n,\e)$-entropy $h(f,n,\e)$ of $f$ should be considered as a ``computable'' quantity, although the complexity of the required computations grows exponentially in $n$.

% Further applications of the local entropy and volume growth concern the semi-continuity modulus of the entropy (compare \cite{Bur.Lia.Yan,New2,Yom3}).

\subsubsection{Local $C^k$-complexity growth}

From now on we concentrate on bounding from above the local volume
growth $v_{k,l}(f,\delta,x)$ for a given $x\in Y$. For this purpose we
define yet another entropy-like invariant $h_{l,k}(f,\delta,x)$, which
measures the ``local $C^k$-complexity growth'' in a
$\delta$-neighborhood of the orbit of $x$
(cf. \cite{gromov:gy,yomdin:entropy-analytic}):
\begin{equation}\label{Ck.local.comp}
  h_{l,k}(f,\delta,x)=\sup_{\sigma^l}\limsup_{n\to \infty}\frac{1}{n}\log N_k(f,\sigma^l,\delta,x,n),
\end{equation}
where $\sigma^l:Q^l\to Y$ varies over all $C^k$-smooth mappings and
$N_k(f,\sigma^l,\delta,x,n)$ is the minimal number of $k$-cu's
$\psi_j:Q^l\to Q^l$, whose images cover
$(\sigma^l)^{-1}(B^n_{\delta}(x))$, and for which
$f^{\circ n}\circ \sigma^l \circ \psi_j$ are $k$-cu's.

An almost immediate fact is that
\begin{equation}\label{eq:vol.Ck.ineq}
  v_{l,k}(f,\delta,x)\le h_{l,k}(f,\delta,x).
\end{equation}
Indeed, since $f^{\circ n}\circ \sigma^l \circ \psi_j$ are $k$-cu's,
the $l$-volume of their image does not exceed a certain constant
depending only on $l,m$ and hence
\begin{equation}
  v_{l,k}(f,\sigma^l,B^n_{\delta}(x),n)\le O_m(N_k(f,\sigma^l,\delta,x,n))
\end{equation}
As $n\to \infty$ in~\eqref{eq:vol.growth} and~\eqref{Ck.local.comp}
the asymptotic constant disappears. Thus it remains to bound from
above $h_{l,k}(f,\delta,x)$.

Below the derivatives of $f:Y\to Y$, and their norms, are defined via
the local coordinate charts on $Y$.

\begin{Prop}\label{prop:bound.hk}
  Assume that for certain positive constants $L,M$ the inequalities
  \begin{align}\label{eq:deriv.bd}
    \norm{df}&\le L,& \max_{s=2,\ldots,k}\norm{d^sf}^{\frac{1}{s-1}}&\le M
  \end{align}
  are satisfied. Then for each $x\in Y$ and for each
  $\delta\le M^{-1}$ we have
  \begin{equation}\label{eq:bound.hk1}
    h_{l,k}(f,\delta,x)\le \frac{l\log L}{k} + \log\poly_m(k).
  \end{equation}
\end{Prop}

\begin{proof}
  Using the local coordinate charts at the points
  $x,f(x),f^{\circ 2}(x),\ldots$ of the orbit of $x$ under $f$, we
  replace iterations of $f$ with compositions of a non-autonomous
  sequence $F$ of mappings $f_i:B^m_1\to\R^m$ with $f_i(0)=0$, where
  $B^m_1$ is the unit ball in $\R^m$, centered at the origin.

  Fix $\delta\le M^{-1}$, and consider the concentric ball
  $B^m_\delta\subset B^m_1.$ Let $\eta: B^m_1\to B^m_\delta$ be a
  linear contraction $\eta(x)=\delta x$. We consider instead of the
  sequence $F$ of mappings $f_i$ a sequence $\bar F$ of
  $\delta$-rescaled mappings
  $\bar f_i=\eta^{-1}\circ f_i\circ \eta:B^m_1\to {\mathbb R}^m.$ For
  the derivatives we have $d^s\bar f_i=\delta^{s-1}d^s f_i$, and by
  the assumptions we get
  \begin{equation}
    ||d\bar f_i||\le L, \ \max_{s=2,\ldots,k}||d^s \bar f_i||\le 1.
  \end{equation}
  Thus $\delta$-rescaling ``kills'' all the derivatives of $f_i$,
  starting with the second one. The first derivative (having a
  dynamical meaning) does not change.

  We have to estimate the number
  $\bar N_{k,n}:=N_k(\bar F,\sigma^l,1,0,n),$ which, by an evident
  change of notations, is the minimal number of $k$-cu's
  $\psi_j:Q^l\to Q^l$, whose images cover
  $(\sigma^l)^{-1}(\bar B^n_{1}(0))$, and for which
  $\bar f_n\circ \bar f_{n-1}\circ \ldots \circ \bar f_1 \circ \sigma^l \circ \psi_j$
  are $k$-cu's.

  We will prove via induction by $n$ that
  \begin{equation}\label{eq:induction.ineq}
    \bar N_{k,n}\leq L^{\frac{nl}{k}}C^n
  \end{equation}
  where $C=\poly_m(k)$ is the constant from
  Corollary~\ref{cor:non.dyn}.  Assume that~\eqref{eq:induction.ineq}
  is valid for $n$. To prove it for $n+1$ we use
  Corollary~\ref{cor:non.dyn}.  We apply it to $g=\bar f_{n+1}$ and to
  each $k$-cu $\nu$ of the form
  \begin{equation}
    \nu=\bar f_n\circ \bar f_{n-1}\circ \ldots \circ \bar f_1 \circ \sigma^l \circ \psi_j,
  \end{equation}
  obtained in the $n$-th step. For each $\nu$ we get at most
  $N=CL^{\frac{l}{k}}$ new $k$-cu's $\phi_q: Q^l\to Q^l$ whose images
  cover $(\sigma^l)^{-1}(\bar B^{n+1}_{1}(0))$, and for which the
  compositions $\bar f_{n+1}\circ \nu\circ \phi_q$ are also
  $k$-cu's. By the induction assumption, the number $\bar N_{k,n}$ of
  the $k$-cu's $\nu$ is at most $L^{\frac{nl}{k}}C^n$. Therefore
  \begin{equation}
    \bar N_{k,n+1}\le \bar N_{k,n}N\le \bar N_{k,n}CL^{\frac{l}{k}}\le L^{\frac{(n+1)l}{k}}C^{n+1}.  
  \end{equation}
  This completes the induction, proving~\eqref{eq:induction.ineq} for
  $n\in\N$. Taking $\log$ and dividing by $n$ finishes the proof.
\end{proof}

To finish the proof of Theorem~\ref{thm:yomdin-Cinf-volume} we should
eliminate the extra term $\log\poly_m(k)$
in~\eqref{eq:bound.hk1}. We do this by replacing $f$ by its iterate
$f^{\circ q}$ for some sufficiently large $q$.

Let $f$ be as in Proposition~\ref{prop:bound.hk}. For each $q\in\N$
the first derivative of $f^{\circ q}$ is bounded by $L^q$. Denote by
$M_q$ a constant satisfying
\begin{equation}
  \max_{s=2,\ldots,k}||d^sf^q||^{\frac{1}{s-1}}\le M_q.
\end{equation}
The constants $M_q$ can be estimated from $L,M$ using Fa\`a di Bruno
formula (see, e.g. \cite{bly}). We do not provide here an explicit
expression, since below, for analytic $f$, we get it in a much easier
way. Using the equality
$h_{l,k}(f^{\circ q},\delta,x)=qh_{l,k}(f,\delta,x)$ and applying
Proposition~\ref{prop:bound.hk} to $f^{\circ q}$ we obtain the
following result, which contains Theorem~\ref{thm:yomdin-Cinf-volume}.

\begin{Cor}\label{cor:bound.hk1}
  For each $\delta\le1/M_q$ we have
  \begin{equation}\label{eq:bound.hk2}
    h_{l,k}(f,\delta,x)\le \frac{l\log L}{k} + \frac{\log\poly_m(k)}{q}.    
  \end{equation}
  In particular, for $q\to\infty$ and $\delta\to0$,
  \begin{align}\label{eq:lim.delta}
    v^0_{l,k}(x)\le h^0_{l,k}(x)&\le \frac{l\log L}{k}, & v^0_{l,\infty}(x)=h^0_{l,\infty}(x)&=0.
  \end{align}
\end{Cor}

\subsubsection{Volume growth for analytic maps}

Now we show, following \cite{bly} how the
Corollary~\ref{cor:bound.hk1} can be used to provide an explicit bound
for the local volume growth and local entropy. Setting
$q(k)=O_m(1)\cdot \frac{k \log k}{l\log L}$ we immediately obtain the
following.

\begin{Cor}\label{cor:bound.hk2}
  For each $\delta \leq \delta(k)= \frac{1}{M_{q(k)}}$ we have
  \begin{equation}\label{eq:bound.hk3}
    h_{l,k}(f,\delta,x)\le \frac{2l\log L}{k}.
  \end{equation}
\end{Cor}

For $C^\infty$ functions $f$ Corollary~\ref{cor:bound.hk2} reduces the
problem of estimating the asymptotic behavior of $h_{l,k}(f,\delta)$
as $\delta\to0$ to the problem of estimating the high-order
derivatives of $f^{\circ q(k)}$. We find it explicitly for analytic
$f$, referring the reader to \cite{bly}, where a much more general
setting is presented.

Let $Y\subset {\mathbb R}^p$ be a compact $m$-dimensional real
analytic submanifold, and let $f:Y\to Y$ be a real analytic
mapping. We assume that $f$ is extendible to a complex analytic
mapping $\tilde f:U\to {\mathbb C}^p$ in a neighborhood $U$ of $Y$ in
${\mathbb C}^p$ of size $\rho$. More accurately, for each point
$x\in Y$ the complex polydisc at $x$ of radius $\rho$ is contained in
$U$. We assume that the norm $||\tilde f||$ is bounded by $D$ in $U$,
and the norm of its first derivative $||d\tilde f||$ is bounded there
by $L.$ The following theorem immediately implies the statement about
volume growth in Theorem~\ref{thm:analytic-entropy} since the volume
of the image of a $k$-cu is universally bounded; the statement about
tail entropy follows similarly but we omit the details.

\begin{Thm}\label{thm:anal.tail.ent}
  For $f$ as above, for each $x\in Y$, and for all $k$ sufficiently
  big we have
  \begin{equation}
    h_{l,k}(f,\delta,x)\le O_m(\log L)\cdot \frac{\log|\log\delta|}{|\log \delta|}.
  \end{equation}
\end{Thm}

\begin{proof}
  Since, by the assumptions, $||d\tilde f||\le L$, this complex
  mapping can expand distances at most $L$ times. Therefore, the
  $q$-th iterate $f^{\circ q}$ is extendible to at least a complex
  neighborhood $U_q$ of $Y$ of the size $\frac{\rho}{L^q}$, and it is
  bounded there by $D$. Applying Cauchy formula, we conclude that for
  each $x\in Y$ we have
  \begin{equation}\label{eq:compl.bound}
    \norm{d^s\tilde f^{\circ q}} \le D s! \left(\frac{L^q}{\rho}\right)^s, \qquad s=0,1,2,\ldots    
  \end{equation}
  Therefore
  \begin{equation}
    \norm{d^s\tilde f^{\circ q}}^{1/s}\le D^{1/s}\cdot s\cdot \frac{L^q}{\rho}, \qquad s=0,1,2,\ldots
  \end{equation}
  In particular, for $s=0,1,2,\ldots,k$ we have
  \begin{equation}
    \norm{d^s\tilde f^{\circ q}}^{1/s} \le \frac{DkL^q}{\rho}
    = 2^{O_m(k\log k)}
  \end{equation}
  for $k$ sufficiently big. Therefore $M_{q(k)}= 2^{O_m(k\log k)}$
  (note that the difference between the $1/s$ and $1/(s-1)$ power is
  absorbed into the asymptotic constant), and
  Corollary~\ref{cor:bound.hk2} applies with
  $\delta=2^{-O_m(k\log k)}$. Solving for fixed $\delta$, we see
  that Corollary~\ref{cor:bound.hk2} applies with
  \begin{equation}
    k=\Omega_m\left(\frac{|\log\delta|}{\log|\log\delta|}\right).
  \end{equation}
  With this $k$, Corollary~\ref{cor:bound.hk2} gives
  \begin{equation}
    h_{l,k}(f,\delta,x)\le \frac{2l\log L}{k}=
    O_m(\log L)\cdot \frac{\log|\log\delta|}{|\log \delta|}.
  \end{equation}
  as claimed.
\end{proof}


\section{Applications in diophantine geometry}

For $p\in\P^\ell(\Q)$ we define $H(p)$ to be $\max_i |\vp_i|$ where
$\vp\in\Z^{\ell+1}$ is a projective representative of $p$ with
$\gcd(\vp_0,\ldots,\vp_\ell)=1$. For $\vx\in\A^\ell(\Q)$ we define its
height to be the height of $\iota(\vx)$ for the standard embedding
\begin{equation}
  \iota:\A^\ell\to\P^\ell,\qquad \iota(\vx_{1..\ell}) = (1:\vx_1:\cdots:\vx_\ell).
\end{equation}
For a set $X\subset\P^\ell(\R)$ we denote
\begin{equation}
 X(\Q,H) := \{ \vx\in X\cap \P(\Q)^\ell : H(x)\le H\},
\end{equation}
and similarly for $X\subset\R^\ell$.

In \cite{bombieri-pila} Bombieri and Pila introduced a method for
studying the quantity $\#X(\Q,H)$ as a function of $H$ when $X$ is the
graph of a $C^r$ (or $C^\infty$) smooth function $f:[0,1]\to\R$. It
turns out that two very different asymptotic behaviors are obtained
depending on whether the graph of $f$ belongs to an algebraic plane
curve. The Yomdin-Gromov algebraic lemma has been used in both of
these directions, to generalize from graphs of functions to more
general sets. In~\secref{sec:bp-complex} we give a complex-cellular
analog of the Bombieri-Pila determinant method. We then present an
application in the algebraic context in~\secref{sec:alg-density} and
in the transcendental context in~\secref{sec:trans-density}.

\subsection{The Bombieri-Pila method for complex cells}
\label{sec:bp-complex}

The Bombieri-Pila method is based on a clever estimate for certain
\emph{interpolation determinants} of a collection of functions, which
is obtained using Taylor expansions. In this section we develop a
parallel theory over complex cells, replacing Taylor expansions by
Laurent expansions.

\subsubsection{Interpolation determinants}

Let $\mu\in\N$. Let $\vf:=(f_1,\ldots,f_\mu)$ be a collection of
functions with a common domain $U$ and
$\vp:=(p_1,\ldots,p_\mu)\in U^\mu$ a collection of points. For minor
technical simplifications we will assume that $f_1\equiv 1$.

We define the \emph{interpolation determinant}
\begin{equation}
  \Delta(\vf;\vp) := \det(f_i(p_j))_{1\le i,j\le\mu}.
\end{equation}
The key to the Bombieri-Pila \cite{bombieri-pila} is an estimate for
the interpolation determinant assuming that $\vf$ are sufficiently
smooth and $\vp$ are contained in a sufficiently small ball. In our
complex analytic analog the small ball will be replaced by a cell
$\cC$ with a $\vdelta$-extension and the smoothness assumption will be
replaced by boundedness in $\cC^\vdelta$.

Let $\cC$ be a complex cell and let $m\le\dim\cC$ denote the number of
$D$-fibers. Set
\begin{equation}
  E=E(m):=\frac{m}{m+1}(m!)^{1/m}.
\end{equation}
Fix $0<\delta<1/2$, and let $\vdelta$ be given by $\delta$ in the
$D$-coordinates and $\delta^{E\mu^{1+1/m}}$ in the $A,D_\circ$
coordinates.
  
\begin{Lem}\label{lem:bp-upper-complex}
  With $\cC^\vdelta$ as above, suppose
  $f_1,\ldots,f_\mu\in\cO_b(\cC^\vdelta)$ with
  $\diam(f_i(\cC^\vdelta),\C)\le M$ and $p_1,\ldots,p_\mu\in\cC$. Then
  \begin{equation}
    |\Delta(\vf;\vp)| \le M^\mu \mu^{O_\ell(\mu)} \delta^{E\mu^{1+1/m}+O_\ell(\mu)}.
  \end{equation}
  When $m=0$ the bound above holds\footnote{In fact one can easily
    derive better estimates in this case but this is not needed in
    this paper.} if we replace $m=0$ by $m=1$.
\end{Lem}
\begin{proof}
  Since $f_1\equiv1$ the interpolation determinant is unchanged if we
  replace each $f_j$ by $f_j-\const$ for $j\ge2$. In this way we may
  assume without loss of generality that
  $\norm{f_i}_{\cC^\vdelta}\le M$. We may also assume by rescaling
  that $M=1$.

  Write a Laurent expansion
  \begin{equation}\label{eq:bp-f_i-exp}
    f_i(\vz) = \left(\sum_{\valpha\in\Z^\ell,|\valpha|<E \mu^{1+1/m}} c_{i,\valpha} \vz^{[\valpha]}\right)+R_i(\vz)
  \end{equation}
  and note that by Proposition~\ref{prop:norm-Laurent} we have
  $|c_{i,\valpha}|<\vdelta^{\pos\valpha}$ and
  \begin{equation}\label{eq:bp-R_i-estimate}
    \norm{R_i}_{\cC} \le \sum_{|\valpha|\ge E\mu^{1+1/m}} \vdelta^{\pos\valpha} \le 
    \sum_{k\ge E\mu^{1+1/m}} (2k+1)^\ell \delta^k = O_\ell(\mu^{2\ell}) \delta^{E \mu^{1+1/m}}
  \end{equation}
  where the final two estimates are elementary and left to the
  reader. We expand the determinant $\Delta(\vf,\vp)$ multilinearly
  into $\mu^{O_\ell(\mu)}$ determinants $\Delta_I$ where each $f_i$ is
  replaced by one of the summands in~\eqref{eq:bp-f_i-exp}. We may
  consider only those $\Delta_I$ where each normalized monomial
  appears at most once: otherwise the determinant has two identical
  columns. We claim that each such $\Delta_I$ is majorated in $\cC$ by
  $\mu^{O_\ell(\mu)}\delta^{E \mu^{1+1/m}}$.

  Consider first the case that $\Delta_I$ contains a residue $R_i$ or
  one of the coefficients $c_{i,\valpha}$ where $\valpha$ contains a
  non-vanishing $A,D_\circ$ coordinate. Expand $\Delta_I$ into $\mu!$
  summands. Each summand contains either a term $R_i(p_j)$ or
  $c_{i,|\valpha|}$ as above, and the estimate then follows
  from~\eqref{eq:bp-R_i-estimate} or from our assumption on
  $\vdelta$.

  The remaining determinants $\Delta_I$ involve only the $m$ variables
  of type $D$, which we assume for simplicity of the notation are
  given by $\vz_{1..m}$. We expand $\Delta_I$ into a $\mu!$
  summands. Each summand is a product
  \begin{equation}
    c_{1,\valpha^1}\vz^{[\valpha^1]}\cdots c_{\mu,\valpha^\mu}\vz^{[\valpha^\mu]}, \qquad \valpha^1,\ldots,\valpha^\mu\in\N^m
  \end{equation}
  with distinct $\valpha^j$. By the estimate on the Laurent
  coefficients this summand is bounded by
  $\delta^{\sum_{i,j}\valpha_i^j}$. Let $L_m(k)$ denote the number of
  monomials of degree $k$ in $m$ variables. The minimal term
  $\delta^q$ will clearly be obtained if we choose $L_m(0)$ of the
  $\valpha^j$s of degree 0, $L_m(1)$ of degree $1$ and so on. Let
  $\nu$ be the largest integer satisfying
  $\sum_{k=0}^\nu L_m(k) \le \mu$. Then
  $q\ge\sum_{k=0}^\nu L_m(k)\cdot k$. Simple computations give
  \begin{equation}
    \begin{gathered}
      L_m(k) = \frac{k^{m-1}}{(m-1)!}+O_m(k^{m-2}) \qquad  \sum_{j=0}^k L_m(j) = \frac{k^m}{m!} + O_m(k^{m-1}) \\
      \mu =\frac{\nu^m}{m!}+O(\nu^{m-1}) \qquad  \nu = (m!\mu)^{1/m}+O(1)        
    \end{gathered}
  \end{equation}
  and finally $q=E\mu^{1+1/m}+O(\mu)$ as claimed.
\end{proof}

Lemma~\ref{lem:bp-upper-complex} gives essentially the same estimate
as the classical Bombieri-Pila method, \emph{but with dimension $m$
  instead of $\dim\cC$}. The key point is that for $\vdelta$
corresponding to $\hrho$-extensions with $\rho\ll1$ the
$D_\circ,A$-coordinates have order $2^{-1/\rho}$ (compared with order
$\rho$ for the $D$ coordinates), which allows us to produce cellular
covers satisfying the conditions of
Lemma~\ref{lem:bp-upper-complex}. Roughly one may say that the
punctured discs and annuli that we produce are so thin that they may
be ignored for the purposes of the Bombieri-Pila method. More
accurately we have the following version of the refinement theorem

\begin{Lem}\label{lem:refinement-special}
  Let $\cC^{1/2}$ be a (real) complex cell with $m$ coordinates of
  type $D$. Let $\mu\in\N$ and $0<\delta<1$, and define $\vdelta$ as
  in Lemma~\ref{lem:bp-upper-complex}. Then there exists a (real)
  cellular cover $\{f_j:\cC^\vdelta_j\to\cC^{1/2}\}$ of size
  \begin{equation}
    N =
    \begin{cases}
      \poly_\ell(\mu\log(1/\delta))\cdot\delta^{-2m} & \text{for complex covers} \\
      \poly_\ell(\mu\log(1/\delta))\cdot\delta^{-m} & \text{for real covers}
    \end{cases}
  \end{equation}
  where each $f_j$ is a cellular translate map.

  If $\cC$ varies in a definable family (and $\sigma,\rho$ vary under
  the condition $0<\sigma<\rho<\infty$) then the cells $\cC_j$ and
  maps $f_j$ can also be chosen from a single definable family. If
  $\cC$ is algebraic of complexity $\beta$ then $\cC_j,f_j$ are
  algebraic of complexity $\poly_\ell(\beta)$.
\end{Lem}
\begin{proof}
  In the $D_\circ,A$ coordinates $\vdelta$ has value
  $\delta^{E\mu^{1+1/m}}$. We begin by constructing, using the
  refinement theorem, a cellular cover
  $\{g_j:\cC^\hrho_j\to\cC^{1/2}\}$ with
  $2^{-1/\rho}=O_\ell(\delta^{E\mu^{1+1/m}})$. The size of this cover
  is $\poly_\ell(\mu\log\delta)$. We now fix some $g=g_j$. It will be
  enough to construct a cellular cover
  $\{f_j:\cC^\vdelta_j\to\cC_j^{\hrho}\}$ of size
  $O_\ell(\delta^{-m})$ where each $f_j$ is as in the conclusion of
  the lemma. Since each $D_\circ,A$ coordinate already has the desired
  extension, we need only refine the $D$ fibers to achieve a
  $\delta$-extension. This is elementary: after rescaling each
  $D$-fiber to $D(1)$ we simply cover $D(1)$ (resp. $(-1,1)$) by
  $\delta^2$ (resp. $\delta$) discs (resp. real discs) of radius
  $\delta$.
\end{proof}

\subsubsection{Polynomial interpolation determinants}

Let $\Lambda\subset\N^\ell$ be a finite set and set
$d:=\max_{\valpha\in\Lambda}|\valpha|$ and $\mu:=\#\Lambda$. For minor
technical simplifications we assume ${\mathbf 0}\in\Delta$. Let
$\vf:=(f_1,\ldots,f_\ell)$ be a collection of functions with a common
domain $U$ and $\vp:=(p_1,\ldots,p_\mu)\in U^\mu$ a collection of
points. We define the \emph{polynomial interpolation determinant}
\begin{equation}
  \Delta^\Lambda(\vf;\vp) := \Delta(\vg;\vp), \qquad \vg:=(\vf^\valpha:\valpha\in\Lambda).
\end{equation}
Basic linear algebra shows that for $S\subset U$ the set
$\vf(S)\subset\C^\ell$ belongs to a hypersurface $\{P=0\}$ with
$\supp P\subset\Lambda$ if and only if $\Delta^\Lambda(\vf;\vp)$
vanishes for any $\vp\subset S$ of size $\mu$.

The following is due to Bombieri and Pila \cite{bombieri-pila}.

\begin{Lem}\label{lem:bp-lower}
  Let $H\in\N$ and suppose $H(\vf(\vp_j))\le H$ for any
  $i=1,\ldots,\ell$ and $j=1,\ldots,\mu$. Then $\Delta^\Lambda(\vf;\vp)$
  either vanishes or satisfies
  \begin{equation}
    |\Delta^\vd(\vf;\vp)| \ge H^{-\mu d}.
  \end{equation}
\end{Lem}
\begin{proof}
  Let $Q_j\le H$ denote the common denominator of $\vf(\vp_j)$. The
  row corresponding to $\vp_j$ in $\Delta^\Lambda(\vf;\vp)$ consists
  of rational numbers with common denominator dividing
  $Q_j^d$. Factoring this common denominator from each row we obtain a
  matrix with integer entries whose determinant is either vanishing or
  an integer. In the latter case
  \begin{equation}
    |\Delta^\Lambda(\vf;\vp)| \ge \prod_{j=1}^\mu Q_j^{-d} \ge H^{-\mu d}.
  \end{equation}
\end{proof}

\subsection{Rational points on algebraic hypersurfaces}
\label{sec:alg-density}

\subsubsection{Previous work}
\label{sec:alg-density-history}

Let $X\subset\P^2(\C)$ be an irreducible curve of degree $d$. Let
$f:I\to X$ be a $C^r$-smooth function with $\norm{f}_r\le1$ and write
$X_f:=f(I)$. Pila~\cite{pila:density-Q} proves, using a generalization
of the method of \cite{bombieri-pila}, that for a sufficiently large
$r=r(d)$ one has $\#X_f(\Q,H)=O_{\e,d}(H^{2/d+\e})$ for any $\e>0$.

If one allows $r=r(d,H)$, for instance if $f$ is $C^\infty$ smooth,
then the Bombieri-Pila method can be pushed further to replace
$H^{2/d+\e}$ by $H^{2/d}\log^kH$ for some $k>0$. The study of
$\#X(\Q,H)$ is therefore directly related to the $C^r$ parameterization
complexity of $X$ in the sense of the Yomdin-Gromov algebraic
lemma. Using an explicit parameterization construction Pila
\cite{pila:pems} proved that
$\#X(\Q,H)\le(6d)^{10}4^dH^{2/d}(\log H)^5$.

In \cite{heath-brown:density} Heath-Brown has developed a
non-Archimedean version of the Bombieri-Pila method and used it to
derive the estimate $\#X(\Q,H)=O_{\e,d}(H^{2/d+\e})$. This was later
improved (in arbitrary dimension $\ell$) by Salberger
\cite{salberger:density} to $O_{\ell,d}(H^{2/d}\log H)$ and very
recently by Walsh \cite{walsh:boundd-rational} to
$O_{\ell,d}(H^{2/d})$.

For a general irreducible variety $X\subset\P(\C)^\ell$ of dimension
$m$, Broberg \cite[Theorem~1]{broberg:note} (generalizing a result of
Heath-Brown \cite{heath-brown:density}) has proved that $X(\Q,H)$ is
contained in $O_{\ell,d,\e}(H^{(m+1)d^{-1/m}+\e})$ hypersurfaces
$\{P_i=0\}$, which have degrees $O_{\ell,d,\e}(1)$ and which do not
contain $X$. Intersecting $X$ with each of the hypersurfaces one
obtains varieties of dimension $m-1$, and could potentially proceed by
induction to eventually obtain estimates on $\#X(\Q,H)$. Marmon
\cite{marmon} has recovered this result using the Bombieri-Pila method
by appealing to the Yomdin-Gromov algebraic lemma.

\subsubsection{Proof of Theorem~\ref{thm:improved-marmon}}
\label{sec:alg-density-proof}

In this section we will prove Theorem~\ref{thm:improved-marmon} using
the complex-cellular version of the Bombieri-Pila method. We remark
that a result similar to Theorem~\ref{thm:improved-marmon} could
probably also be proved by following the argument of \cite{marmon} and
replacing the Yomdin-Gromov algebraic lemma by our refined
version.

We will work in the affine setting. Let $\iota_j$ denote the $j$-th
standard chart $\vx_{1..\ell}\to(\cdots:\vx_j:1:\vx_{j+1}:\cdots)$. If
$\vz=(\vz_0:\cdots:\vz_{\ell})\in\P^\ell(\C)$ with $|\vz_j|$ maximal
then in the $\iota_j$-chart $\vz$ corresponds to a point
$\vx\in\bar\cP_\ell$, and moreover $H(\vz)=H(\vx)$. We assume without
loss of generality that $j=0$. Below $\vx$ denotes the affine
coordinates in this chart. It will suffice to count the points of
height $H$ in $\bar\cP_\ell\cap\iota_0^{-1}(X)$.

We fix some degree-lexicographic monomial ordering on
$\C[\vx_{1..\ell}]$ and let $\Lambda:=\N^\ell\setminus\LT(I)$ where
$I\subset\C[\vx_{1..\ell}]$ denotes the ideal of $X$ and $\LT(I)$ the
set of leading terms of $I$ with respect to the fixed ordering. Set
$\Lambda(k):=\Lambda\cap\{|\valpha|\le k\}$. Then
$\mu(k):=\#\Lambda(k)$ is the Hilbert function of $X$. This function
eventually equals the Hilbert polynomial, and by Nesterenko's estimate
\cite{nesterenko:hilbert} we have
\begin{equation}
  \mu(k) \ge \binom{k+m+1}{m+1} - \binom{k+m+1-d}{m+1} = dk^m/m!+\poly_m(d)k^{m-1}.
\end{equation}
We will apply the Bombieri-Pila method with $\Lambda(k)$ interpolation
determinants.

Let $\cC\subset\cP_\ell^{1/2}$ be a complex cell as in
Lemma~\ref{lem:bp-upper-complex} and let $\vp\subset\cC(\Q,H)$. Then
by Lemmas~\ref{lem:bp-upper-complex} and~\ref{lem:bp-lower} we have
\begin{equation}\label{eq:bp-comparison-alg}
  H^{-\mu k} \le |\Delta^{\Lambda(k)}(\vx;\vp)| \le 2^\mu \mu^{O_\ell(\mu)} \delta^{E\mu^{1+1/m}+O_\ell(\mu)}
\end{equation}
unless $\Delta^{\Lambda(k)}(\vx;\vp)=0$. Note that in the right hand
side of~\eqref{eq:bp-comparison-alg}, Lemma~\ref{lem:bp-upper-complex}
gives the estimate with $m$ equal to the number of disc fibers in
$\cC$ (which is at most $\dim\cC$), and this implies the same estimate
with $m=\dim X$. Thus unless $\Delta^{\Lambda(k)}(\vx;\vp)=0$ we have,
for $k>\poly_m(d)$,
\begin{multline}
  \log\delta >
  \frac{- O_\ell(\log\mu) - k\log H }{E\mu^{1/m}+O_\ell(1)} =
  \frac{-k\log H - O_\ell(1)}{E\mu^{1/m}+O_\ell(1)} = \\
  \frac{m+1}{m} \cdot\frac{-k\log H-O_\ell(1)}{kd^{1/m} (1+\poly_m(d)/k)^{1/m}} =
  -\frac{m+1}{m}\cdot\frac{\log H}{d^{1/m}} (1+\poly_\ell(d)/k).
\end{multline}
Therefore, for
\begin{equation}
  \delta = H^{-\frac{m+1}m d^{-1/m}(1+\poly_\ell(d)/k)}
\end{equation}
we have $\Delta^{\Lambda(k)}(\vx;\vp)=0$ for any $\vp$ as above, and
$\cC(\Q,H)$ is therefore contained in a hypersurface $\{P=0\}$
satisfying $\supp P\subset\Lambda(k)$. In particular $P$ is of degree
at most $k$ and does not vanish identically on $X$. It remains to show
that $\R\cP_\ell^{1/2}\cap\iota_0^{-1}(X)$ can be covered by
$\delta^{-m}\poly(\mu\log(1/\delta))$ real cells satisfying the conditions
of Lemma~\ref{lem:bp-upper-complex}. For this we first use the CPT to
construct a real cover of $\R(\cP_\ell^{1/2}\cap\iota_0^{-1}(X))$ of
size $\poly_\ell(\beta)$ admitting $1/2$-extensions, and then apply
Lemma~\ref{lem:refinement-special} to further refine each of the cells
to satisfy the conditions of Lemma~\ref{lem:bp-upper-complex}.

\subsection{Rational points on transcendental sets}
\label{sec:trans-density}

\subsubsection{Log-sets in diophantine applications of the Pila-Wilkie
theorem}
\label{sec:log-sets-pila}

In some of the most remarkable applications of the Pila-Wilkie
theorem, particularly those related to modular curves and more general
Shimura varieties, it is necessary to count rational points on sets
that are definable in $\R_{\an,\exp}$ but \emph{not} in $\R_\an$. We
briefly recall the most standard example coming from the universal
covers of modular curves.

Recall that the universal covering map
$j:\H\to\SL_2(\Z)\backslash \H\simeq\C$ can be factored as a
composition $j(\tau)=J(q)$ where $q=e^{2\pi i\tau}$ and
$J(q):D_\circ(1)\to\C$ is meromorphic in a neighborhood of $q=0$. We
extend $j,J$ as a function of several variables coordinate-wise, and
denote by $\Omega\subset\H^n$ the (coordinate-wise) standard
fundamental domain for $j$. In \cite{pila:andre-oort} a key step is
the application of the Pila-Wilkie theorem to sets of the form
$Y:=\Omega\cap j^{-1}(X)$ where $X\subset\C^n$ is an algebraic
variety. This set is not (in general) subanalytic: it is of the form
$Y=\log A$ where $\log(\cdot)=(2\pi i)^{-1}\log_e(\cdot)$ and
$A:=J^{-1}(X)$. Note that since $J$ is meromorphic the set $A$ is
subanalytic, so that $Y$ is a log-set.

\subsubsection{An interpolation result for logarithms of complex cells}

Let $\cC^{\vdelta/2}$ be a complex cell as in
Lemma~\ref{lem:bp-upper-complex}, and let $n:=\dim\cC$ and $m$ denote
the number of fibers of type $D$. Let $\vf:=(f_0,\ldots,f_n)$ be a
tuple of non-vanishing holomorphic functions on $\cC^{\vdelta/2}$. We
write $\vx_{0..n}=\log\vf_{0..n}$ and denote $X:=\vx(\cC)$.

Recall that $\pi_1(\cC)\simeq\Z^{n-m}$, so we may take
$\valpha(\vf_j)\in\Z^{n-m}$ for $j=0,\ldots,n$. Then there exist $m+1$
vectors $\vgamma^0,\ldots,\vgamma^m\in\Z^{n+1}$ linearly independent
over $\Z$ and satisfying $\sum \vgamma^k_j\valpha(\vf_j)=0$. We write
$\vg_k = \prod_j \vf_j^{\vgamma^k_j}$. Then $\vg_{0..m}$ are
non-vanishing holomorphic functions on $\cC^{\vdelta/2}$ with trivial
associated monomials. Then the logarithms
$\vy_{0..m}:=\log\vg_{0..m}$, which are $\Z$-linear combinations of
the $\vx$-variables, are holomorphic univalued functions in
$\cC^{\vdelta/2}$.

\begin{Prop}\label{prop:bp-logs}
  Let $H\in\N$. If
  \begin{equation}\label{eq:bp-logs-cond}
    \log\delta < -O_\vf(k^{-1/m}\log H)
  \end{equation}
  then $X(\Q,H)$ is contained in an algebraic hypersurface
  $\{P(\vy)=0\}$ of degree at most $k$ in $\C^{n+1}$.

  If $\cC,\vf$ vary over a definable family such that the type of
  $\cC$ and the associated monomials of $\vf_{0..n}$ are fixed then
  the asymptotic constants can be taken uniformly over the family.
\end{Prop}
\begin{proof}
  Since $\vy$ are fixed $\Z$-linear combinations of $\vx$ we have
  $H(\vy)=O_\vf(H(\vx))$. Thus it will be enough to prove the
  statement for $Y(\Q,H)$ instead of $X(\Q,H)$ where $Y:=\vy(\cC)$.
  
  By the monomialization lemma the diameter of $\vy_j(\cC^\vdelta)$ is
  bounded in $\C$ for $j=0,\ldots,m$ by some quantity $M=O_\vf(1)$
  which can be taken uniform over definable families. We are thus in
  position to apply Lemma~\ref{lem:bp-upper-complex} with
  \begin{equation}
    \Lambda(k)=\{\valpha\in\N^{m+1}:|\valpha|\le k\}, \qquad \mu(k)\sim k^{m+1}.
  \end{equation}
  Let $\vp\subset Y(\Q,H)$. Then by Lemmas~\ref{lem:bp-upper-complex}
  and~\ref{lem:bp-lower} we have
  \begin{equation}
     H^{-\mu k} \le |\Delta^{\Lambda(k)}(\vy;\vp)| \le M^\mu \mu^{O_\ell(\mu)} \delta^{E\mu^{1+1/m}+O_\ell(\mu)}
  \end{equation}
  unless $\Delta^{\Lambda(k)}(\vx;\vp)=0$. In the former case we have
  \begin{equation}
    \log\delta > \frac{-O_\vg(k\log H)}{k^{1+1/m}+O_\ell(1)}=-O_\vg(k^{-1/m}\log H).
  \end{equation}
  Therefore, for $\delta$ satisfying~\eqref{eq:bp-logs-cond} we have
  $\Delta^{\Lambda(k)}(\vy;\vp)=0$ for any $\vp$ as above, and
  $Y(\Q,H)$ is indeed contained in a hypersurface of degree at most
  $k$ in the $\vy$-variables.
\end{proof}


\begin{Rem}[Logarithms in families]\label{rem:logs-in-family}
  We remark that if $g_\lambda:\cC_\lambda^{1/2}\to\C\setminus\{0\}$
  is a definable family of functions with trivial associated monomials
  then $\log g_\lambda:\cC_\lambda\to\C$ is not necessarily definable
  (in $\R_\an$) as illustrated by the example $g_\lambda\equiv\lambda$
  for $\lambda\in(0,1]$. However, if we define
  $\tilde g_\lambda=g_\lambda/\norm{g_\lambda}_{\cC_\lambda}$ then
  $\log\tilde g_\lambda$ is definable. Indeed by the monomialization
  lemma we know that $w=\log \tilde g_\lambda(\cC)\subset D(M+2\pi)$
  for some uniformly bounded $M$. Then the graph of a (univalued)
  branch of $\log\tilde g_\lambda$ is definable, being one of the
  components of the definable set $e^w=\tilde g_\lambda(\vz)$ with
  $w\in D(M)$ and $\vz\in\cC_\lambda$ (note that the exponential here
  is \emph{restricted} to a compact set).
\end{Rem}

\subsubsection{Proof of Proposition~\ref{prop:log-set-pw}}
\label{sec:log-set-pw-proof}

We apply Corollary~\ref{cor:cpt-subanalytic} to the subanalytic set
given by the total space $X\subset\R^K$ of the family $\{X_\lambda\}$,
where we order the parameter variables before the fiber variables. We
obtain real cellular maps $f_j:\cC^{1/2}_j\to\cP_K^{1/2}$ such that
$f_j(\R_+\cC_j^{1/2})\subset X$ and $\cup_j f_j(\R_+\cC_j)=X$. We may
also require by Remark~\ref{rem:cpt-semi-extra} that each $f_j$ is
compatible with the fiber variables $\vx_{1..\ell}$. It will be enough
to prove the statement for each $f_j(\R_+\cC_j)$ separately, so fix
$\cC,f=\cC_j,f_j$ and assume $X=f(\R_+\cC)$.

We view $\cC$ as a subanalytic family of cells $\cC_\lambda$ of length
$\ell$. If the type of $\cC_\lambda$ has dimension strictly less then
$n$ then we can take $Y=X$, so assume it has dimension $n$. Below we
let $\vf_{0..n}$ denote the $f$-pullback of some fixed $n+1$-tuple of
the variables $\vx_{1..\ell}$. Note that the associated monomials
$\valpha(\vf_j)$ of $\vf_j$ on $\cC_\lambda$ are independent of
$\lambda$ (it is obtained from the associated monomial of $\vf_j$ on
$\cC$ by eliminating the $\lambda$ variables).

Let $\cR=\{f_{\lambda,\theta}:\cC_{\lambda,\theta}\to\cC_\lambda\}$
denote the refinement family consisting of the refinement cells of
$\cC_\lambda$ as in Lemma~\ref{lem:refinement-special}. We split $\cR$
into subfamilies $\cR(\cT)$ consisting of the cells
$\cC_{\lambda,\theta}$ of type $\cT$. According to
Remark~\ref{rem:refinement-vs-monom} the types $\cT$ are those
obtained from the type of $\cC_\lambda$ by possibly replacing some
$D_\circ,A$ fibers by $D$. Fix some $\cT$ and let $m$ denote the
number of fibers of type $D$. The associated monomial of
$f^*_{\lambda,\theta}\vf_j$ is constant over $\cR(\cT)$: it is
obtained from $\valpha(\vf_j)$ by eliminating those indices that
correspond to fibers that were replaced by $D$ in $\cT$.

We can thus apply Proposition~\ref{prop:bp-logs} to
$f_{\lambda,\theta}^*\vf_{0..n}$ to deduce that if the cell
$\cC_{\lambda,\theta}$ admits $\vdelta/2$ extension and if
$\delta = H^{-O_X(k^{-1/m})}$ then
\begin{equation}
  [\vx_{\lambda,\theta}(\cC_{\lambda,\theta})](\Q,H)\subset\{P(\vy_{\lambda,\theta})=0\}
\end{equation}
for some polynomial $P(\vy_{\lambda,\theta})$ of degree $k$, where
$\vy_{\lambda,\theta}$ are some fixed $\Z$-linear combinations of
$\vx_{\lambda,\theta}:=\log (f^*_{\lambda,\theta}\vf_{0..n})$ which
are holomorphic in $\cC_{\lambda,\theta}^{\vdelta/2}$. Note that
$\vy_{\lambda,\theta}$ does \emph{not} necessarily depend definably on
the parameters. However, the normalized $\tilde\vy_{\lambda,\theta}$
defined by replacing $\vx_{\lambda,\theta}$ with their normalized
versions $\tilde\vx_{\lambda,\theta}$ is definable according to
Remark~\ref{rem:logs-in-family}. For each fixed value of the
parameters we have
$\tilde\vy_{\lambda,\theta}=\vy_{\lambda,\theta}+\const$, and in
particular the hypersurface $\{P(\vy_{\lambda,\theta})=0\}$ can be
rewritten as a hypersurface
$\{\tilde P(\tilde\vy_{\lambda,\theta})=0\}$ in the
$\tilde\vy_{\lambda,\theta}$-variables.

We now define a family $\tilde Y_{\lambda,\theta,c}$ with the
parameter $\theta$ of $\cR_{\lambda,\theta}$ and the parameter $c$
encoding the coefficients of an arbitrary non-zero polynomial $P_c$ of
degree $k$,
\begin{equation}
  \tilde Y_{\lambda,\theta,c} := \R (f_{\lambda,\theta}(\cC_{\lambda,\theta})\cap \{P_c(\tilde\vy_{\lambda,\theta})=0\}).
\end{equation}
For any $\lambda$ we can choose, according to
Lemma~\ref{lem:refinement-special} a collection of
\begin{equation}
  N=\poly_\ell(\delta)=H^{-O_X(k^{-1/n})}
\end{equation}
cells $\cC_{\lambda,\theta_j}$ which admit $\delta$-extensions,
satisfy the conditions of Lemma~\ref{lem:bp-upper-complex}, and cover
$\cC_\lambda$. For each of them there exists a polynomial
$P_{c_j}(\tilde\vy_{\lambda,\theta_j})$ whose zeros contain
$[\vx_{\lambda,\theta}(\cC_{\lambda,\theta})](\Q,H)$. The union over
$j=1,\ldots,N$ of $\log \tilde Y_{\lambda,\theta_j,c_j}$ thus contains
$(\log X_\lambda)(\Q,H)$.

Recall that $\tilde Y$ was constructed for some choice of $n+1$ of
the coordinates $\vx_{1..\ell}$. We now repeat this construction for
each such choice $S$. We now define a family $\hat Y_{\lambda,\mu}$
with the parameters $\mu$ encoding the pair $\theta_S,c_S$ for each of
the choices $S$,
\begin{equation}
  \hat Y_{\lambda,\mu} = \bigcap_S \tilde Y_{\lambda,\theta_S,c_S}.
\end{equation}
As before, for every $\lambda$ there is a choice of
$N=H^{-O_X(k^{-1/n})}$ parameters $\mu_j$ such that the union
$(\log \hat Y_{\lambda,\mu_j})(\Q,H)$ contains $(\log X_\lambda)(\Q,H)$.

If $\hat Y_{\lambda,\mu}$ has dimension $n$ then $X_\lambda$ satisfies
a polynomial equation of degree $k$ in $\log\vf_{0..n}$ for every
$n+1$ tuple of coordinates $\vf_{0..n}$ among $\vx_{1..\ell}$ and in
this case $\log X_\lambda$ is contained in an algebraic variety of
dimension $n$. Since $X_\lambda$ has pure dimension $n$ this implies
that $(\log X_\lambda)^\trans=\emptyset$. Thus if we define $Y$ by
removing from $\hat Y$ any fibers of dimension $n$, the conditions of
the proposition are satisfied.

\bibliographystyle{plain}
\bibliography{nrefs}

\end{document}
