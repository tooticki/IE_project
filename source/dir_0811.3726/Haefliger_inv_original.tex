\section{The Haefliger invariant}\label{sec_Haefliger}


Here we assume $n-j \ge 3$ is odd, $j \ge 2$ and $2n-3j-3 \ge 0$.
Then at present we cannot prove that $I$ is a cochain map, but in this section we describe a $(2n-3j-3)$-cocycle
$\calH :=I(H)+c$ of $\emb{n}{j}$ using a graph cocycle $H \in \D^{2,0}_0$ and some `correction term' $c$.


The graph cocycle $H$ is shown in Figure \ref{fig_H}.
We call the first graph (with four i-vertices) $H_1$ and the second $H_2$; $H =H_1 /2 +H_2 /6$.
By using the rules in Proposition \ref{def_signs}, we can see that $H$ is indeed a cocycle.
\begin{figure}[htb]%cocycle_H
\[
 \begin{xy}
  (-6,5)*{H=\dfrac{1}{2}},(24,5)*{+\dfrac{1}{6}},
  (5,0)*{\bullet}="A", (5,10)*{\bullet}="B", (15,10)*{\bullet}="C", (15,0)*{\bullet}="D",
  {\ar@{.} "A";"B"^<{1}^(.5){(1)}^>{2}},{\ar "B";"C"},{\ar@{.} "C";"D"^<{3}^(.5){(2)}^>{4}},
  (35,3.33)*{\circ}="E",(35,10)*{\bullet}="F",(29.227,0)*{\bullet}="G",(40.773,0)*{\bullet}="H",
  {\ar@{.} "E";"F"_(.5){(1)}^>{1}},{\ar@{.} "E";"G"^<{4}_(.5){(2)}^>{2}},{\ar@{.} "E";"H"^(.65){(3)}_>{3}},
  (18,-5)*{n \text{ even, } j \text{ odd}},
  (56,5)*{H=\dfrac{1}{2}},(84,5)*{+\dfrac{1}{6}},
  (65,0)*{\bullet}="a", (65,10)*{\bullet}="b", (75,10)*{\bullet}="c", (75,0)*{\bullet}="d",
  {\ar@{.>} "a";"b"^<{1}^>{2}},{\ar@{-} "b";"c"_(.5){(1)}},{\ar@{.>} "c";"d"^<{3}^>{4}},
  (95,3.33)*{\circ}="e",(95,10)*{\bullet}="f",(89.227,0)*{\bullet}="g",(100.773,0)*{\bullet}="h",
  {\ar@{.>} "f";"e"_<{1}},{\ar@{.>} "g";"e"^<{2}_>{4}},{\ar@{.>} "h";"e"_<{3}},
  (78,-5)*{n \text{ odd, } j \text{ even}}
 \end{xy}
\]
\caption{Graph cocycle $H$}\label{fig_H}
\end{figure}


\begin{thm}\label{thm_closed}
Suppose $n-j \ge 3$ is odd, $j \ge 2$ and $2n-3j-3 \ge 0$.
There exists a differential form $c\in \Omega^{2n-3j-3}_{DR}(\emb{n}{j})$ such that $\calH := I(H)+c$ is closed.
\end{thm}


Though rigorous definition of $c$ and the proof of Theorem \ref{thm_closed} can be found in \S \ref{subsec_proof_closed},
we give a rough explanation here.
Let $\I_j (\R^n )$ be the space of linear injections $\R^j \to \R^n$.
Then $c:=-p_* D^* \mu$, where $D:\emb{n}{j} \times \R^j \to \I_j (\R^n )$ is the derivation map,
$p:\emb{n}{j}\times \R^j \to \emb{n}{j}$ is the first projection and the form
$\mu \in \Omega^{2n-2j-3}_{DR}(\I_j (\R^n ))$ is given so that $d\mu$ describes the contribution of a boundary stratum
of $C_{H_2}$ corresponding to the collision of all the four points.
See Definition \ref{def_c} for details.


Below we prove Theorem \ref{thm_main1}, which states that $\calH$ gives a non-zero cohomology class and is the
Haefliger invariant when $2n-3j-3 = 0$.





\subsection{Additivity}\label{subsec_additive}


Here we assume $2n-3j-3=0$, which implies $n=6k$ and $j=4k-1$ for some $k \ge 1$.
Then $\calH =I(H)+c$ is an isotopy invariant for long $(4k-1)$-knots in $\R^{6k}$.
What we will show in this subsection is the `additivity' of the invariant $\calH$.


\begin{prop}\label{prop_additive}
The invariant $\calH$ is additive under the connect-sum;
for any $f_+ ,f_- \in \emb{6k}{4k-1}$, the Kronecker pairing satisfies
\[
 \pair{\calH}{f_+ \sharp f_-} = \pair{\calH}{f_+} + \pair{\calH}{f_-}.
\]
\end{prop}


\noindent
{\bf Notation.}
We will show in Lemma \ref{lem_indep_H_vol} that the invariant $\calH$ is independent of the choice of (anti-)symmetric
volume forms.
So we may choose (anti-)symmetric volume forms $vol_{S^{N-1}}$, $N=4k-1$ or $6k$ such that
\begin{itemize}
\item
 their supports are contained in the sufficiently small neighborhoods of the poles
 $p^{N-1}_{\pm} := (0,\dots ,0,\pm 1) \in S^{N-1}$, and
\item
 they are ``$O(N-1)$-invariant,'' that is, $A^* vol_{S^{N-1}}=(\det A)vol_{S^{N-1}}$ for any $A\in O(N-1)$ regarded as
 in $O(N-1)\oplus 1 \subset O(N)$ (in other words $A\in O(N)$ fixes $x_N$-axis).
\end{itemize}

\smallskip

For any $f \in \emb{n}{j}$, the {\em support} of $f$, denoted by $\text{supp}(f) \subset [-1,1]^j$, is defined by
\[
 \text{supp}(f) := \overline{\{ x \in \R^j \, | \, f(x) \ne (x,0) \in \R^j \times \{ 0 \}^{n-j} \}} .
\]
Since we are considering an isotopy invariant, we may suppose
$\text{supp}(f_{\pm}) \subset B^{4k-1}_{\pm}(\varepsilon )$ and
$f_{\pm}(\text{supp}(f_{\pm})) \subset B^{6k}_{\pm}(\varepsilon /2) \cup (\R^{4k-1}\times \{ 0\}^{2k+1})$, where
\[
 B^m_{\pm} (\varepsilon ) := \{ (x_1 ,\dots ,x_m ) \in \R^m \, | \, ( x_1 \pm 1/2 )^2 +x^2_2 +\dots +x^2_m
 < \varepsilon^2 \} ,
\]
and $\varepsilon >0$ is a sufficiently small number.


First we compute $\pair{\calH}{f_+} = \pair{I(H_1 )}{f_+}/2 +\pair{I(H_2 )}{f_+}/6 + \pair{c}{f_+}$.
The first term $\pair{I(H_1 )}{f_+}$ is equal to
\[
 \int_{C_{H_1}(f_+ )}\omega_{H_1} ,
\]
where $C_{H_1}(f) \approx C^o_4 (\R^{4k-1})$ is the fiber of $\pi_{H_1}$ over $f \in \emb{6k}{4k-1}$.
Recall that the integrand is $\theta_{12}\eta_{23}\theta_{34}$ restricted to the fiber.
Since we choose $vol_{S^{N-1}}$ with support `localized' around $p^{N-1}_{\pm}$, we must know for which configurations
$x=(x_1 ,\dots ,x_4 ) \in C_{H_1}(f_+ )= C_4 (\R^{4k-1})$ all the three vectors $\varphi^{\theta}_{12}(x)$,
$\varphi^{\theta}_{34}(x)$ and $\varphi^{\eta}_{23}(x)$ are near $p^{6k-1}_{\pm}$ or $p^{4k-2}_{\pm}$;
$\omega_{H_1}$ does not vanish only on the subspace of such configurations.


\begin{lem}\label{lem_localize1}
Define $X_+ := C^o_4 (B^{4k-1}_+ (\varepsilon )) \subset C_{H_1}(f_+ )$.
Then
\[
 \pair{I(H_1 )}{f_+} = \int_{X_+}\omega_{H_1} .
\]
\end{lem}


\begin{proof}
For example, consider $(x_1 ,\dots ,x_4 ) \in C_{H_1}(f_+ ) \setminus X_+ $ with $x_1 \not\in B^{4k-1}_+ (\varepsilon )$.
Then by our choice of $f_+$, the vector $\varphi^{\theta}_{12} (x_1 ,\dots ,x_4 )$ cannot be in the support of
$vol_{S^{6k-1}}$ (a neighborhood of the pole $p^{6k-1}$).
So the form $\theta_{12}$ vanishes on such configurations.
In this way we can see that the integrand does not vanish only on $X_+$, since outside of $X_+$ the images of
$\varphi^{\theta}_{12}$ and $\varphi^{\theta}_{34}$ cannot intersect with the supports of $vol_{S^{6k-1}}$
simultaneously.
\end{proof}


Next we compute $\pair{I(H_2 )}{f_+}$ in similar fashion.
This equals
\[
 \int_{C_{H_2}(f_+ )}\omega_{H_2},
\]
where $\omega_{H_2} = \theta_{14}\theta_{24}\theta_{34}$.
Recall $C_{H_2}(f_+ ) \subset C^o_3 (\R^{4k-1}) \times \R^{6k}$.


\begin{lem}\label{lem_localize2}
Define a subspace $Y_+ \subset C_{H_2}(f_+ )$ by
\[
 Y_+ := \{ (x_1 ,x_2 ,x_3 ;y) \in C_{H_2}(f_+ ) \, | \, \text{at least two }x_p \in B^{4k-1}_+ (\varepsilon )\} .
\]
Then
\[
 \pair{I(H_2 )}{f_+} = \int_{Y_+} \omega_{H_2}.
\]
\end{lem}


\begin{proof}
This is because the image of
\[
 \varphi_{H_2}:= \varphi^{\theta}_{14} \times \varphi^{\theta}_{24} \times \varphi^{\theta}_{34}
 : C_{H_2}(f_+ ) \setminus Y_+ \longrightarrow (S^{6k-1})^3
\]
is of positive codimension, and hence the pullback of the top form of $(S^{6k-1})^3$ must vanish on
$C_{H_2}(f_+ ) \setminus Y_+$.
Indeed, outside $Y_+$ at least two points $f_+ (x_p )$, $f_+ (x_q )$ are on the standardly embedded $\R^{4k-1}$, hence
the image of $C_{H_2}\setminus Y_+$ is the set of $(u_1 ,u_2 ,u_3 ) \in (S^{6k-1})^3$ such that one vector
$u_p$ has to be in the linear subspace $(\R^{4k-1}\times \{ 0 \}^{2k+1})+\R u_q$ of $\R^{6k}$, hence is of
codimension $\ge 2k$.
\end{proof}


Now consider the third term $\pair{c}{f_+}$.
This is an integration of $D(f_+ )^* \mu$ over $\R^{4k-1}$, where $D(f_+ ):\R^{4k-1} \to \I_{4k-1}(\R^{6k})$ is given by
$x\mapsto (df_+ )_x$ (for the definition of $\mu$, see Definition \ref{def_c} and the explanation after
Theorem \ref{thm_closed}).
Since $D(f_+ )$ is constant outside $\text{supp}(f_+ )$, we obtain the following.


\begin{lem}\label{lem_localize3}
$\displaystyle \pair{c}{f_+} = \int_{B^{4k-1}_+ (\varepsilon)}D(f_+ )^* \mu$.
\end{lem}


Finally we can compute $\pair{\calH}{f_+ \sharp f_-}$.
The long knot $f_+ \sharp f_-$ is isotopic to
\[
 f_+ \sharp f_- (x) :=
 \begin{cases}
  f_{\pm}(x)  & \text{if } x \in B^{4k-1}_{\pm}(\varepsilon ) , \\
  (x,0) & \text{if } x \not\in B^{4k-1}_+ (\varepsilon ) \sqcup B^{4k-1}_- (\varepsilon ).
 \end{cases}
\]
First, by completely similar arguments to Lemma \ref{lem_localize1}, only the configurations $x=(x_1 ,\dots ,x_4 )$
such that $x_1$ and $x_2$ are in the same $B^{4k-1}_+ (\varepsilon )$ or $B^{4k-1}_- (\varepsilon )$, and so is
$(x_3 ,x_4 )$, can contribute to $\pair{I(H_1)}{f_+ \sharp f_-}$.
Moreover, if $x_2$ and $x_3$ are in the different $B^{4k-1}(\varepsilon )$'s, then the image of the direction map
$\varphi^{\eta}_{23}(x)$ cannot be in $\text{supp}(vol_{S^{4k-2}})$.
Thus only the configurations with all four $x_p$'s being in the same $B^{4k-1}_+ (\varepsilon )$ or
$B^{4k-1}_- (\varepsilon )$ can contribute to the pairing, and hence
\begin{equation}\label{eq_sum_1}
 \pair{I(H_1 )}{f_+ \sharp f_-} = \int_{X_+ \sqcup X_-}\omega_{H_1} =\int_{X_+}\omega_{H_1} + \int_{X_-}\omega_{H_1},
\end{equation}
where $X_- :=C^o_4 (B^{4k-1}_- (\varepsilon ))$.


Next, similarly to Lemma \ref{lem_localize3},
\begin{equation}\label{eq_sum_2}
 \pair{c}{f_+ \sharp f_-} = \int_{B^{4k-1}_+ (\varepsilon ) \sqcup B^{4k-1}_- (\varepsilon )}D(f_+ \sharp f_- )^* \mu
 =\sum_{i=\pm}\int_{B^{4k-1}_i (\varepsilon )}D(f_i )^* \mu .
\end{equation}


As for $I(H_2 )$, similar argument to Lemma \ref{lem_localize2} shows
\begin{equation}\label{eq_sum_3}
 \pair{I(H_2 )}{f_+ \sharp f_-} = \int_{Y_+ (f_+ \sharp f_- )}\omega_{H_2} + \int_{Y_- (f_+ \sharp f_- )}\omega_{H_2}
 + \int_{Y_0 (f_+ \sharp f_- )}\omega_{H_2},
\end{equation}
where $Y_- (f_+ \sharp f_- ) \subset C_{H_2}(f_+ \sharp f_- )$ is defined in similar way as $Y_+$ with $B_+$'s replaced
by $B_-$'s, and $Y_0 (f_+ \sharp f_- ) \subset C_{H_2}(f_+ \sharp f_- )$ consists of $(x_1 ,x_2 ,x_3 ;y)$ with exactly
one $x_p$ in each $B^{4k-1}_{\pm}(\varepsilon )$.
By our choice of volume forms, the integrand $\omega_{H_2}$ does not vanish only on the subspace consisting of
$(x_1 ,x_2 ,x_3 ,y)$ with
\begin{enumerate}
\item
 $y$ is very near `$(0,\dots ,0,\pm \infty )$,' or
\item
 the first $6k-1$ factors of $f_+ (x_1 )$, $f_+ (x_2 )$, $f_+ (x_3 )$ and $y$ are close to each other (see Figure
 \ref{fig_contribution}).
\end{enumerate}
\begin{figure}[htb]%config_contribution
\[
 \begin{xy}
 (0,15)*{\R^n},(10,3)*{\R^j},
 {\ar@{-} (0,0);(50,0)},{\ar@{-}(0,0);(10,7)},{\ar@{-}(10,7);(60,7)},{\ar@{-}(50,0);(60,7)},
 (29,6)*{\bullet}="A",(30,2)*{\bullet}="B",(33,4)*{\bullet}="C", (31,20)*{\circ}="D",
 {\ar@{.>}"A";"D"^(.15){x_1}^>{y}},{\ar@{.>}"B";"D"^<{x_2}},{\ar@{.>}"C";"D"_<{x_3}}
 \end{xy}
\]
\caption{Configurations which contribute to $\pair{I(H_2 )}{f_+ \sharp f_-}$}
\label{fig_contribution}
\end{figure}
On $Y_0 (f_+ \sharp f_- )$ the condition (2) cannot hold, so we can restrict the integral over $Y_0 (f_+ \sharp f_- )$
to that over the subspace of $Y_0 (f_+ \sharp f_- )$ with $\abs{y}$ large.
Then since the image of $f_+ \sharp f_-$ is very close to that of the trivial knot $f_0$
(if $\varepsilon$ is very small relative to $\abs{y}$),
\[
 \int_{Y_0 (f_+ \sharp f_- )}\omega_{H_2} - \int_{C_{H_2}(f_0 )}\omega_{H_2} = O(\varepsilon ).
\]
But since $\text{supp}(f_0 )=\emptyset$, Lemma \ref{lem_localize2} shows that $\int_{C_{H_2}(f_0 )}\omega_{H_2}=0$.
Hence
\begin{equation}\label{eq_sum_4}
 \int_{Y_0 (f_+ \sharp f_- )}\omega_{H_2} = O(\varepsilon ).
\end{equation}
$Y_{\pm} (f_+ \sharp f_- )$ is decomposed into two subspaces; in one space, two $x_p$, $x_q$ are in
$B^{4k-1}_{\pm}(\varepsilon )$ and remaining $x_r$ is in $B^{4k-1}_{\mp}(\varepsilon )$, and in the other space $x_r$
is not in $B^{4k-1}_{\mp} (\varepsilon )$.
In the former space, the condition (2) above cannot hold, so $\abs{y}$ has to be large.
So replacing $f_+ \sharp f_-$ with $f_{\pm}$ changes the integral by $O(\varepsilon )$;
\begin{equation}\label{eq_sum_5}
 \int_{Y_{\pm}(f_+ \sharp f_- )}\omega_{H_2} = \int_{Y_{\pm}(f_{\pm})}\omega_{H_2} + O(\varepsilon ).
\end{equation}
Summing up \eqref{eq_sum_1}, \eqref{eq_sum_2} and \eqref{eq_sum_3} and substituting \eqref{eq_sum_4} and
\eqref{eq_sum_5}, we have
\begin{align*}
 \pair{I(H)}{f_+ \sharp f_-}
  &= \sum_{i=\pm} \left(\frac{1}{2}\int_{X_i}\omega_{H_1} + \frac{1}{6}\int_{Y_i (f_i )}\omega_{H_2}
   + \int_{B^{4k-1}_i (\varepsilon)}D(f_i )^* \mu  \right) + O(\varepsilon ) \\
 &= \pair{I(H)}{f_+} + \pair{I(H)}{f_-} + O(\varepsilon ).
\end{align*}
This completes the proof of Proposition \ref{prop_additive}, since $\pair{\calH}{f_+ \sharp f_-}$ is independent of
$\varepsilon$.


\begin{rem}
By similar argument, the order two invariant \cite{Rossi_thesis, Watanabe07} for $\emb{m+2}{m}$ ($m \ge 3$ is odd) is
proved to be additive under the connect-sum.
\end{rem}





\subsection{Evaluation}\label{subsec_evaluation}
Here we prove the last part of Theorem \ref{thm_main1}.


We know by \cite{Haefliger66} that $\pi_0 (\emb{6k}{4k-1})$ forms a group under the connect-sum and is isomorphic to
$\Z$.
Our goal here is the following.

\begin{thm}\label{thm_Haefliger}
When $n=6k$ and $j=4k-1$ for some $k \ge 1$, then $\pair{\calH}{S}$ is equal to $\pm 1$ for the generator $S$ of
$\pi_0 (\emb{6k}{4k-1})$.
\end{thm}


Since both the Haefliger invariant and our $\calH$ are additive under the connect-sum (see \cite{Haefliger62} and
Proposition \ref{prop_additive}), Theorem \ref{thm_Haefliger} says that $\calH$ is the Haefliger invariant.


We will use the generator of $\pi_0 (\emb{6k}{4k-1})$ given by Budney \cite{Budney08} and Roseman-Takase
\cite{RosemanTakase07}.





\subsubsection{Deform-spinning}\label{subsubsec_deform_spinning}
Given an $N$-fold based loop $\gamma \in \Omega^N \emb{n}{j}$ (for any $n,j$) represented by a smooth map
$\gamma : \R^N \to \emb{n}{j}$ with $\gamma (x)\equiv \iota$ (the trivial long $j$-knot) for any $x \not\in [-1,1]^j$,
we have $S_{\gamma}\in \emb{N+n}{N+j}$ defined by
\[
 S_{\gamma}(x,t) :=(x,\gamma (x)(t)) \in \R^N \times \R^n , \quad x\in \R^N ,\ t\in \R^j.
\]
This is a special case of `deform-spinning' construction \cite{Roseman89}.
Putting $N=1$, we obtain the {\em graphing map} $\gr : \Omega \emb{n}{j} \to \emb{n+1}{j+1}$ given in \cite{Budney08}.
It has been shown in \cite{Budney08} that all the following maps
\[
 \pi_{4k-2}(\emb{2k+2}{1}) \xrightarrow{\gr} \pi_{4k-3}(\emb{2k+3}{2}) \xrightarrow{\gr} \dots \xrightarrow{\gr}
 \pi_0 (\emb{6k}{4k-1}) \cong \Z
\]
are isomorphisms, so $\pi_0 (\emb{6k}{4k-1})\cong \Z$ is generated by $S_{\psi}:=\gr^{4k-2}(\psi )$;
\begin{equation}\label{eq_deform_spinning}
 S_{\psi}(z) := (x,\psi(x)(t)) \in \R^{4k-2}\times\R^{2k+2}, \quad z=(x,t) \in \R^{4k-2} \times \R^1 ,
\end{equation}
where $\psi : \R^{4k-2} \to \emb{2k+2}{1}$ is a map which gives the generator of $\pi_{4k-2}(\emb{2k+2}{1})$.
In fact $\pi_{4k-2}(\emb{2k+2}{1})$ is the first non-vanishing homotopy group of $\emb{2k+2}{1}$ (for example see
\cite{Budney08, Tourtchine04_2}), and the image of its generator $\psi$ via Hurewicz isomorphism is given as follows
(probably the dual cocycle to $\psi$ first appeared in \cite{Vassiliev01}; see also
\cite{Budney08, CCL02, Longoni04, K07}).


Consider a `long immersion' $f:\R^1 \to \R^{2k+2}$ which has only two transversal doublepoints
$z_i = f(\xi_i )=f(\xi_{i+2} )$, $i=1,2$, $\xi_1 < \xi_2 < \xi_3 < \xi_4$ (see Figure \ref{fig_f}).
Consider the unit sphere $S^{2k-1}_i$ in the complementary $2k$-space to $\R f' (\xi_i ) \oplus \R f' (\xi_{i+2})$ in
$T_{z_i}\R^{2k+2}$.
At each doublepoint $z_i$, we can `lift' the segments $f(\xi_{i+2} -\varepsilon , \xi_{i+2} +\varepsilon )$ to the
directions of complementary $S^{2k-1}_i$ to obtain non-singular embeddings (see Figure \ref{fig_blowup}).
More precisely, for $(w_1 ,w_2 ) \in (S^{2k-1})^2$, define $\psi (w_1 ,w_2 )\in \emb{2k+2}{1}$ by
\[
 \psi (w_1 ,w_2 )(t) :=
 \begin{cases}
  f(t)+\left( \delta \exp \left[ \frac{1}{(t-\xi_{i+2} )^2 -\varepsilon^2}\right] \right) w_i
   & \abs{t-\xi_{i+2}} < \varepsilon ,\ i=1,2 \\
  f(t) & \text{otherwise}
 \end{cases}
\]
where $\delta$ and $\varepsilon$ are positive small numbers.
Thus a map $\psi =\psi_{\delta}: (S^{2k-1})^2 \to \emb{2k+2}{1}$ is defined, and it is known that $\psi$ induces an
isomorphism $\psi_* : H_{4k-2}((S^{2k-1})^2) \xrightarrow{\cong} H_{4k-2}(\emb{2k+2}{1})$.



\begin{figure}[htb]%lambda
\[
 \begin{xy}
 {\ar@{.>}(0,0);(85,0)^>{1}},{\ar@{.>}(40,-10);(40,12)^>{3,\dots ,2k+2}},{\ar@{.>}(20,-10);(60,10)^>{2}},
 {\ar@{-}(5,0);(50,0)},{\ar@{-}(30,5);(39,5)},{\ar@{-}(41,5);(60,5)},
 {\ar@{-}@(r,r)(50,0);(60,5)},{\ar@{-}@(l,l)(30,5);(15,-5)},
 {\ar@{-}(15,-5);(40,-5)},
 {\ar@{-}@(r,l)(40,-5);(63,3)},
 {\ar@(r,l)(67,3);(80,0)},
 (12,2)*{z_1}, (55,-2)*{z_2}, (68,5)*{c}
 \end{xy}
\]
\caption{Immersion $f$ in $\R^{2k+2}$}\label{fig_f}
\end{figure}
\begin{figure}[htb]
\[
 \begin{xy}%resolution
 {\ar (0,0);(40,0)}, {\ar (4,-8);(36,8)}, {\ar@{.>}(20,-6);(20,12)},
 (0,3)*{f}, (22,-2)*{z_i}, (20,9)*{\bullet}, (17,9)*{w_i},
 (50,0)*{\leadsto},
 {(60,0);(100,0) **\crv{(65,0)&(70,3)&(80,12)&(90,3)&(95,0)}?>*\dir{>}},
 {(60,0);(100,0) **\crv{~*{.}(65,0)&(70,-3)&(80,-12)&(90,-3)&(95,0)}?>*\dir{>}},
 (60,3)*{\psi (w_i )},
 {\ar@{-}(64,-8);(88,4)}, {\ar (90,5);(96,8)}, {\ar@{.>}(80,-6);(80,12)},
 (70,-12)*{I_i}="I", {\ar@{.}"I";(78,-1)},
 (95,-10)*{W_{i+2}}="W", {\ar@{.}"W";(85,7.2)},{\ar@{.}"W";(85,-7.2)}
 \end{xy}
\]
\caption{Definition of $\psi$}\label{fig_blowup}
\end{figure}


The following is proved by inspection (see Figure \ref{fig_blowup}).


\begin{lem}\label{lem_linking_number}
The union
\[
 W_3 := \bigcup_{w_1 \in S^{2k-1}}\psi (w_1 ,w_2 ) (\xi_3 -\varepsilon ,\xi_3 + \varepsilon ) \subset \R^{2k+2}
\]
is homeomorphic to $S^{2k}$.
Consider the map
\[
 \varphi : W_3 \times I_1 \longrightarrow S^{2k+1}, \quad (x,y) \longmapsto \frac{y-x}{\abs{y-x}},
\]
where $I_1 := f(\xi_1 -\varepsilon ,\xi_1 +\varepsilon )$.
The linking number of $W_3$ with $I_1$ is one; more precisely, the limit $\delta \to 0$ of the integration of
$\varphi^* vol_{S^{2k+1}}$ over $W_3 \times I_1$ is $\pm 1$.
Similar statement holds for
\[
 W_4 := \bigcup_{w_2 \in S^{2k-1}}\psi (w_1 ,w_2 ) (\xi_4 -\varepsilon ,\xi_4 + \varepsilon ) \quad \text{and}\quad
 I_2 := f(\xi_2 -\varepsilon ,\xi_2 +\varepsilon ).
\]
\end{lem}


Fixing the natural projection $\pi : [-1/2,1/2]^{2k-1} \to S^{2k-1}$ such that
\begin{itemize}
\item
 $q^{2k-1}_{\pm}:=\pi^{-1}(p^{2k-1}_{\pm}) \in (-1/2,1/2)^{2k-1}$ (recall
 $p^{2k-1}_{\pm}:=(0,\dots ,0,\pm 1) \in S^{2k-1}$),
\item
 $\pi (\partial [-1/2,1/2]^{2k-1})$ consists of a single point $u_0 := (1,0,\dots ,0)$,
\end{itemize}
we can think of $\psi$ as $\psi :[-1/2,1/2]^{4k-2} \to \emb{2k+2}{1}$.
We can extend $\psi$ to whole $\R^{4k-2}$ so that $\psi (x)\equiv \iota$ if $x \not\in [-1,1]^{4k-2}$
(see also \S \ref{sssec_choice_psi}), since $\emb{2k+2}{1}$ is $(4k-3)$-connected as mentioned above.
Thus we have $S_{\psi}\in \emb{6k}{4k-1}$ defined as (\ref{eq_deform_spinning}).


\begin{thm}[\cite{Budney08, RosemanTakase07}]
If $\psi$ is as above, then $S_{\psi}$ generates $\pi_0 (\emb{6k}{4k-1})$.
\end{thm}


\begin{rem}
In \cite{RosemanTakase07} a generator of $\pi_0 (\text{Emb}\, (S^{4k-1},S^{6k}))$ was defined by the
deform-spinning $\psi :(S^{2k-1})^2 \to \emb{2k+2}{1}$ along the torus.
But it is not difficult to see that such a spinning also gives an element of $\emb{6k}{4k-1}$ which is isotopic to our
$S_{\psi}$ given by the graphing map \cite{Budney08}.
\end{rem}


The above construction can be done for the generator of $\pi_{2d-2}(\emb{d+2}{1})$ for any $d>1$ to obtain
$S_{\psi}\in \emb{3d}{2d-1}$.
Since we assumed that $n-j=d+1$ is odd, we put $d=2k$ here (otherwise there exist orientation reversing automorphisms
of graphs $H_i$, and hence they vanish).
When $d$ is odd, any generator of $\pi_0 (\emb{3d}{2d-1})\cong \Z /2$ would not be detected by any de Rham cohomology
classes.





\subsubsection{A suitable choice of $\psi : \R^{4k-2}\to \emb{2k+2}{1}$}\label{sssec_choice_psi}
Below we will compute
\begin{equation}\label{eq_evaluation}
 \pair{\calH}{S_{\psi}} = \frac{1}{2}\int_{C_{H_1}(S_{\psi})}\omega_{H_1}
 + \frac{1}{6}\int_{C_{H_2}(S_{\psi})}\omega_{H_2} + \int_{\R^{4k-1}} D(S_{\psi})^* \mu ,
\end{equation}
where $D(S_{\psi}):\R^{4k-1} \to \I_{4k-1}(\R^{6k})$ is defined by $D(S_{\psi})(x) := (dS_{\psi})_x$, and $\mu$ is
defined in the sentence after the proof of Lemma~\ref{lem_hatomega_closed}.
To simplify the computation of \eqref{eq_evaluation}, we need to choose a favorable extension
$\psi : \R^{4k-2} \to \emb{2k+2}{1}$.


First, we take the immersion $f$ as in Figure \ref{fig_f};
almost all the image of $\psi (x)$ is contained in the 2-plane $\R^2 \times \{ 0\}^{2k}\subset \R^{2k+2}$,
except for the neighborhoods of $z_1$, $z_2$ and another crossing $c$ which corresponds to $f(\zeta_1 )$ and
$f(\zeta_2 )$, $\xi_2 <\zeta_1 <\xi_3$, $\xi_4 <\zeta_2 <1$.
We suppose that the arc $\psi (x)(\zeta_1 -\varepsilon ,\zeta_1 +\varepsilon)$ is in the 2-plane,
$\psi (x)(\zeta_2 -\varepsilon ,\zeta_2 +\varepsilon)$ is in the 3-plane $\R^3 \times \{ 0\}^{2k-1}\subset \R^{2k+2}$,
and that the resolutions of segments $\psi (x)(\xi_{i+2} -\varepsilon ,\, \xi_{i+2} +\varepsilon )$ occur in
$\{ 0\}^2 \times \R^{2k}$.


Next we suppose that the image of $\psi (x)$ is `almost' in $\R^2$;
\begin{equation}\label{eq_thin}
 \psi (x)(\R^1 ) \subset \R^2 \times (-\delta ,\delta )^{2k}\quad \text{for any }x\in \R^{4k-2}.
\end{equation}
Lastly we suppose $\psi$ is `symmetric' in the following sense; define
\begin{align*}
 X&:= \{ (s,t,u) \in \R^{2k-1}\times \R^{2k-1}\times \R^1 \, | \,
  s \not\in [-1/2 ,1/2]^{2k-1}, t \in [-1/2 ,1/2]^{2k-1}\} ,\\
 Y&:= \{ (s,t,u) \in \R^{2k-1}\times \R^{2k-1}\times \R^1 \, | \, 
  s \in [-1/2 ,1/2]^{2k-1}, t \not\in [-1/2 ,1/2]^{2k-1}\} ,\\
 Z&:= \{ (s,t,u) \in \R^{2k-1}\times \R^{2k-1}\times \R^1 \, | \,  s,t \not\in [-1/2 ,1/2]^{2k-1} \}
\end{align*}
and $X',Y',Z'$ as the images of $X,Y,Z$ respectively via the first $4k-2$ projection $pr_{4k-2}:\R^{4k-1}\to \R^{4k-2}$
(see Figure \ref{fig_outside_zero}).
Define $\psi : X' \to \emb{2k+2}{1}$ by
\[
 \psi (s,t)=h_{\abs{s}/\abs{s'}}(\pi (t))
\]
where $h_{\alpha} : S^{2k-1}\to \emb{2k+2}{1}$ ($\alpha \ge 1$) is a one parameter family with $h_1 =\psi (u_0 ,-)$ and
$h_{\alpha} \equiv \iota$ (the trivial long knot) for $\alpha \ge 2$, and $s' \in \partial [-1/2,1/2]^{2k-1}$ is the
unique element with $s=\alpha s'$ for some $\alpha \ge 1$.
Using similar one parameter family connecting $\psi (-,u_0 )$ (resp.\ $\psi (u_0 ,u_0 )$) and $\iota$, we can define
$\psi$ on $Y'$ (resp.\ on $Z'$) (see Example \ref{ex_outside_zero} and Figure \ref{fig_outside_zero}).
We can modify $\psi$ so that it is smooth.
The following properties of $\psi$ will be important below;
\begin{align}
 \psi (s,t) &= \psi (-s,t) \quad \text{if }(s,t)\in X'\cup Z', \label{eq_psi_symmetry_1} \\
 \psi (s,t) &= \psi (s,-t) \quad \text{if }(s,t)\in Y'\cup Z'. \label{eq_psi_symmetry_2}
\end{align}


\begin{ex}\label{ex_outside_zero}
Consider the case $k=1$ (see Figure \ref{fig_outside_zero}).
\begin{figure}[htb]
\[
 \begin{xy}%outside_zero
%axis
 {\ar@{.>}(-25,0);(25,0)^(.38){-1/2}^>{s}_(.53){O}_(.65){1/2}},{\ar@{.>}(0,-20);(0,20)^(.3){-1/2}_(.7){1/2}_>{t}},
%upper left
 {\ar@{-}(-20,10);(-10,10)},{\ar@{-}(-10,10);(-10,15)},
 {\ar@{.}(-10,12);(-12,10)},{\ar@{.}(-10,14);(-14,10)},{\ar@{.}(-12,14);(-16,10)},{\ar@{.}(-14,14);(-18,10)},
  {\ar@{.}(-16,14);(-18,12)},
%upper right
 {\ar@{-}(20,10);(10,10)},{\ar@{-}(10,10);(10,15)},
 {\ar@{.}(10,12);(12,10)},{\ar@{.}(10,14);(14,10)},{\ar@{.}(12,14);(16,10)},{\ar@{.}(14,14);(18,10)},
  {\ar@{.}(16,14);(18,12)},
%lower left
 {\ar@{-}(-20,-10);(-10,-10)},{\ar@{-}(-10,-10);(-10,-15)},
 {\ar@{.}(-10,-12);(-12,-10)},{\ar@{.}(-10,-14);(-14,-10)},{\ar@{.}(-12,-14);(-16,-10)},{\ar@{.}(-14,-14);(-18,-10)},
  {\ar@{.}(-16,-14);(-18,-12)},
%lower right
 {\ar@{-}(20,-10);(10,-10)},{\ar@{-}(10,-10);(10,-15)},
 {\ar@{.}(10,-12);(12,-10)},{\ar@{.}(10,-14);(14,-10)},{\ar@{.}(12,-14);(16,-10)},{\ar@{.}(14,-14);(18,-10)},
  {\ar@{.}(16,-14);(18,-12)},
%square
 {\ar@{.}(-10,10);(-10,-10)}, {\ar@{.}(-10,-10);(10,-10)}, {\ar@{.}(10,-10);(10,10)}, {\ar@{.}(10,10);(-10,10)},
%left
 {\ar@{.}(-12,10);(-12,-10)}, {\ar@{.}(-14,10);(-14,-10)}, {\ar@{.}(-16,10);(-16,-10)}, {\ar@{.}(-18,10);(-18,-10)},
%right
 {\ar@{.}(12,10);(12,-10)}, {\ar@{.}(14,10);(14,-10)}, {\ar@{.}(16,10);(16,-10)}, {\ar@{.}(18,10);(18,-10)},
%upper
 {\ar@{.}(10,12);(-10,12)}, {\ar@{.}(10,14);(-10,14)},
%lower
 {\ar@{.}(10,-12);(-10,-12)}, {\ar@{.}(10,-14);(-10,-14)},
 (23,5)*{X'},(-23,5)*{X'}, (-5,17)*{Y'},(-5,-17)*{Y'},
 (21,14)*{Z'},(-21,14)*{Z'},(21,-14)*{Z'},(-21,-14)*{Z'}
 \end{xy}
\]
\caption{Extension $\psi : \R^2 \to \emb{4}{1}$}
\label{fig_outside_zero}
\end{figure}
$X'\to \emb{4}{1}$ is given as a null homotopy $\psi(s,-) :S^1 \to \emb{4}{1}$ ($\abs{s}\ge 1/2$) from
$\psi (\{ 1/2\} \times S^1 )$ to $\iota$.
Since $\psi (\{ 1/2 \} \times [-1/2,1/2])$ and $\psi (\{ -1/2 \} \times [-1/2,1/2])$ are the same cycle of $\emb{4}{1}$,
two homotopies corresponding to $X' \cap \{ s\ge 1/2\}$ and  $X' \cap \{ s\le -1/2\}$ can be taken so that
$\psi (s,t)=\psi (-s,t)$.
This is \eqref{eq_psi_symmetry_1}.
Similarly $Y'\to \emb{4}{1}$ is a homotopy from $\psi (S^1 \times \{ 1/2\} )$ to $\iota$, and $\psi (s,t)=\psi (s,-t)$
on $Y'$.


Eight thick segments $\{ \pm 1/2\} \times \{ \abs{t}\ge 1/2\}$, $\{ \abs{s}\ge 1/2\} \times \{ \pm 1/2\}$ in
Figure \ref{fig_outside_zero} represent the same homotopy from $\psi (1/2, 1/2 )$ to $\iota$.
Thus we can define $\psi$ by
\[
 \psi (r\cos \beta +1/2 ,r\sin \beta +1/2):= \psi (r+1/2 ,1/2) \quad (0 \le \beta \le \pi /4)
\]
on $Z' \cap \{ s,t \ge 1/2\}$, and similarly on other components.
Then it is easy to see $\psi (s,t)=\psi (-s,t)=\psi (s,-t)=\psi (-s,-t)$ on $Z'$.
\end{ex}


Using such an extension $\psi$, we will compute \eqref{eq_evaluation}.





\subsubsection{The first term $\pair{I(H_1 )}{S_{\psi}}$}\label{sssec_first_term}
First let us study on which configurations $x=(x_1 ,\dots ,x_4 ) \in C_{H_1}(S_{\psi})$ the integrand $\omega_{H_1}$
does not vanish, as was done in \S \ref{subsec_additive}.


Let us write $x_p := (s^{(p)},t^{(p)},u^{(p)}) \in (\R^{2k-1})^2 \times \R^1$ ($p=1,\dots ,4$).
Then by \eqref{eq_deform_spinning},
\begin{equation}\label{eq_direction}
\begin{split}
 &S_{\psi}(x_2 ) - S_{\psi}(x_1 )= \\
 &\quad (s^{(2)}-s^{(1)},t^{(2)}-t^{(1)},\psi (s^{(2)},t^{(2)})(u^{(2)})-\psi (s^{(1)},t^{(1)})(u^{(1)})) \\
 &\hskip200pt \in ( \R^{2k-1})^2 \times \R^{2k+2}.
\end{split}
\end{equation}
By \eqref{eq_thin}, the length of the last $2k$ factors of (\ref{eq_direction}) is at most $2\sqrt{2k}\, \delta$.
So $\varphi^{\theta}_{12}(x_1 ,x_2 ) = (S_{\psi}(x_2 ) - S_{\psi}(x_1 ))/\abs{S_{\psi}(x_2 ) - S_{\psi}(x_1 )}$
is near $p^{6k-1}_{\pm}$ only if at least the first $4k-2$ factors of (\ref{eq_direction}) are nearly zero,
that is, $(s^{(1)},t^{(1)})$ is near $(s^{(2)},t^{(2)})$.
Similarly $(s^{(3)},t^{(3)})$ must be near $(s^{(4)},t^{(4)})$.
Moreover $(s^{(2)},t^{(2)})$ must be near $(s^{(3)},t^{(3)})$, since $w=(x_3 -x_2)/\abs{x_3 -x_2}$ is near
$p^{4k-2}_{\pm}$.
Thus we need to consider only $x=(x_1 ,\dots ,x_4 )$ with all $(s^{(p)},t^{(p)})$ ($p=1,\dots ,4$) close to each other.
Notice that they must become closer and closer if we choose $vol_{S^{N-1}}$ with smaller support.


Let $L:=(-1/2-a ,1/2+a)^{4k-2}\times \R^1$ ($a>0$ is a fixed small number) be a small neighborhood of
$[-1/2,1/2]^{4k-2}\times \R^1$, and consider $x \in C_4 (\R^{4k-1}\setminus L)$.


\begin{lem}\label{lem_outside_zero}
If we choose $\psi$ as in \S \ref{sssec_choice_psi} and $vol_{S^{N-1}}$ with sufficiently small support, then
the integration of $\omega_{H_1}$ over $C_4 (\R^{4k-1}\setminus L)$ vanishes.
\end{lem}


\begin{proof}
If $\text{supp}(vol_{S^{N-1}})$ is sufficiently small relative to $\delta >0$, then
$(x_1 ,\dots ,x_4 ) \in C_4 (\R^{4k-1}\setminus L)$ must be such that all $(s^{(p)},t^{(p)})$ ($p=1,\dots ,4$) are
close to each other so that at most one of
$\{ x_1 ,\dots ,x_4 \} \cap (X\setminus L)$ and $\{ x_1 ,\dots ,x_4 \} \cap (Y\setminus L)$ is not empty.


Consider the integration of $\omega_{H_1}$ over $\{ x \in C_4 (\R^{4k-1}\setminus L)\, |\, x \cap X \ne \emptyset \}$.
Notice that in this case $x\in C_4 (X \cup Z)$.
Define an involution $F_1$ of this subspace by
\[
 F_1 (x_1 ,\dots ,x_4 ):=(i_1 x_1 ,\dots ,i_1 x_4 ),
\]
where $i_1 :\R^{4k-1} \to \R^{4k-1}$ is given by $i_1 (s,t,u):=(-s,t,u)$ ($s,t \in \R^{2k-1}$, $u\in \R^1$).
The map $F_1$ preserves the orientation, but $F^*_1 \omega_{H_1}=-\omega_{H_1}$ because
\begin{itemize}
\item
 $\varphi^{\theta}_{12}\circ F_1 =i_1 \circ \varphi^{\theta}_{12}$ by \eqref{eq_psi_symmetry_1} and \eqref{eq_direction}
 (here we abbreviate $i_1 \times \id_{\R^{2k+1}} :\R^{6k}\to \R^{6k}$ to $i_1$), and hence
 $F^*_1 \theta_{12}=(-1)^{2k-1}\theta_{12}=-\theta_{12}$ (since we have assumed
 $i^*_1 vol_{S^{6k-1}}=(-1)^{2k-1}vol_{S^{6k-1}}$; see the remark after Proposition \ref{prop_additive}),
\item
 similarly $F^*_1 \theta_{34}=-\theta_{34}$, and
\item
 $\varphi^{\eta}_{23}\circ F_1 =i_1 \circ \varphi^{\eta}_{23}$ and hence $F^*_1 \eta_{23}=-\eta_{23}$.
\end{itemize}
Hence the integration of $\omega_{H_1}$ over $\{ x \in C_4 (\R^{4k-1}\setminus L)\, |\, x \cap X \ne \emptyset \}$ must
vanish.
Similarly, we can show the vanishing of the integration of $\omega_{H_1}$ over $\{ x\cap Y\ne \emptyset \}$ by considering
an involution $F_2$ given by $F_2 (x_1 ,\dots ,x_4 ):=(i_2 x_1 ,\dots ,i_2 x_4)$, where $i_2 :\R^{4k-1} \to \R^{4k-1}$
is given by $i_2 (s,t,u):=(s,-t,u)$.
The vanishing of the integration of $\omega_{H_1}$ over $\{ x \cap X = x\cap Y = \emptyset \}$ is proved in similar way;
in this case $x \in C_4 (Z\setminus L)$, and hence either $F_1$ or $F_2$ can work.
\end{proof}


By Lemma \ref{lem_outside_zero}, only $x=(x_1 ,\dots ,x_4 )$ with at least one $x_p$ in $L$ (and other $x_q$'s near $L$)
contributes to $\pair{I(H_1 )}{S_{\psi}}$.


Consider the tubular neighborhood $N=\R^{4k-2} \times [-1/2,\, 1/2]$ of $\R^{4k-2}$ in $\R^{4k-1}$.
Define four `fat planes' by
\[
 N_i := \R^{4k-2} \times (\xi_i -\varepsilon ,\, \xi_i + \varepsilon ) \subset N \subset \R^{4k-1}, \quad 1 \le i \le 4
\]
($\xi_i$ and $\varepsilon$ have appeared in the definition of $\psi$).
Recall that we write $q^{2k-1}_{\pm} :=\pi^{-1}(p^{2k-1}_{\pm})$ (we choose $\pi$ so that
$q^{2k-1}_{\pm} \in (-1/2,1/2)^{2k-1}$).


\begin{lem}\label{lem_local_N1}
Let $x_1 , x_2 \in \R^{4k-1}$ be two distinct points with at least one $x_p \in L$.
Then the direction $\varphi^{\theta}_{12}(x_1 ,x_2 )$ is near $p^{6k-1}_{\pm}$ only if
\begin{itemize}
\item $(s^{(1)},t^{(1)})$ is near $(s^{(2)},t^{(2)})$,
\item $x_1 \in N_i$ and $x_2 \in N_{i+2}$ for some $i =1,\dots ,4$ (here the suffixes are understood modulo $4$), and
\item $s^{(2)}$ (resp.\ $t^{(2)}$, $s^{(1)}$, $t^{(1)}$) is near $q_{\pm}$ when $(x_1 ,x_2 )\in N_1 \times N_3$
 (resp.\ $N_2 \times N_4$, $N_3 \times N_1$, $N_4 \times N_2$).
\end{itemize}
\end{lem}


\begin{proof}
We have already seen that $(s^{(1)},t^{(1)})$ must be near $(s^{(2)},t^{(2)})$.
But it is not enough; the direction determined by the last $2k+2$ factors
$\psi (s^{(2)},t^{(2)})(u^{(2)})-\psi (s^{(1)},t^{(1)})(u^{(1)})$ of \eqref{eq_direction} must be near $p^{2k+1}_{\pm}$,
and it is the case only if the two points $\psi (s^{(p)},t^{(p)})(u^{(p)})$ ($p=1,2$) are around a self-intersection
$z_i$ of $f$ (see Figure \ref{fig_blowup}) which is resolved in a direction near $p^{2k+1}_{\pm}$.
Such a situation is realized, for example, if
\begin{itemize}
\item $u^{(1)} \in (\xi_1 -\varepsilon ,\xi_1 +\varepsilon )$ and
 $u^{(2)} \in (\xi_3 -\varepsilon ,\xi_3 +\varepsilon )$, and
\item $\pi (s^{(2)})$ is near $p^{2k+1}_{\pm}$ (recall $\pi : [-1/2,1/2]^{2k-1} \twoheadrightarrow S^{2k-1}$).
\end{itemize}
This is the case of $(x_1 ,x_2 ) \in N_1 \times N_3$ in the Lemma.
\end{proof}


Thus it suffices to consider $x\in C_{H_1}(S_{\psi})$ such that both $(x_1 ,x_2 )$ and $(x_3 ,x_4 )$ satisfy the
condition of Lemma \ref{lem_local_N1}, and all the $(s^{(p)},t^{(p)})\in \R^{4k-2}$ are close to each other.
We divide such configurations into two types.


\noindent{\bf Type I.}
All the four points $S_{\psi}(x_p )$ ($1\le p\le 4$) are near the resolution of a single $z_i$ ($i=1$ or $2$).


Let us write $N_{p,q,r,s}:=N_p \times N_q \times N_r \times N_s \subset C_4 (\R^{4k-1})$ ($p,q,r,s=1,2,3,4$).
Type I configuration $x=(x_1 ,\dots ,x_4 )$ is in $N_{i, i+2,i,i+2}$ or $N_{i,i+2,i+2,i}$ for some $i$ (modulo $4$).
Figure \ref{fig_unlink} shows an example of $x \in N_{1,3,1,3}$ such that $\varphi_{H_1}(x) =(v_1 ,v_2 ,w)$
($\varphi_{H_1}:=\varphi^{\theta}_{12}\times \varphi^{\theta}_{34}\times \varphi^{\eta}_{23}$) is in the support of
the volume forms.


\begin{figure}[htb]%inverse_image_1
\[
 \begin{xy}
 {\ar (0,0);(80,0)},{\ar (0,0);(20,30)},{\ar (0,0);(0,25)}, (0,30)*{\R^1},(12,30)*{\R^{2k-1}},(85,3)*{\R^{2k-1}},
 {\ar@{-}(25,10);(37,28)},{\ar@{-}(65,10);(77,28)},{\ar@{.}(25,10);(65,10)},{\ar@{.}(37,28);(77,28)},
 {\ar@{-}(25,6);(27,9)},{\ar@{.}(28,10.5);(37,24)},{\ar@{-}(65,6);(77,24)}, {\ar@{.}(25,6);(65,6)},
  {\ar@{.}(37,24);(77,24)},
 (33,28)*{N_3},(69,6)*{N_1},
 (33,22)*{\bullet},(29.6,12.9)*{\bullet},(30,22)*{x_4},(28,19)*{x_3},(34,14)*{x_2},(32,11.4)*{x_1},
 {\ar (31,19);(31,14)},
 {\ar@{.}(21,0);(25,6)},{\ar@{.}(61,0);(65,6)},
 (20,2)*{q_+},(60,2)*{q_-}
 \end{xy}
\]
\caption{$x=\varphi^{-1}_{H_1}(v_1 ,v_2 ,w)\in N_{1,3,1,3}$}\label{fig_unlink}
\end{figure}


\noindent{\bf Type II.}
Two points $S_{\psi}(x_1 )$ and $S_{\psi}(x_2 )$ are near the resolution of $z_i$ of $f$, and
$S_{\psi}(x_3 )$ and $S_{\psi}(x_4 )$ are near the resolution of $z_{i+1}$, $i=1$ or $2$ (here $z_3 :=z_1$).


Such a configuration is in $N_{i,i+2,i+1,i+3}$ or $N_{i,i+2,i+3,i+1}$ for some $i$.
Type II inverse image of $(v_1 ,v_2 ,w) \in (S^{6k-1})^2 \times S^{4k-2}$ via $\varphi_{H_1}$ looks like
Figure \ref{fig_link}, because of the following.


\begin{lem}\label{lem_local_N2}
Let $v \in S^{6k-1}$ near $p^{6k-1}_{\pm}$ be given.
If the pair $(x_1 ,x_2 ) \in N_1 \times N_3$ satisfies $\varphi^{\theta}_{12}(x_1 ,x_2 )=v$, then $s^{(2)}$ and
$u^{(2)}$ are uniquely determined by $v$.
If moreover we give any $t^{(2)}$ to fix $x_2$, then accordingly $x_1$ is determined.


Similarly, if $(x_1 ,x_2 ) \in N_2 \times N_4$ satisfies $\varphi^{\theta}_{12}(x_1 ,x_2 )=v$, then
$(x_1 ,x_2 )$ is uniquely determined according to given $s^{(2)}$.
\end{lem}


\begin{proof}
Consider the case $(x_1 ,x_2 ) \in N_1 \times N_3$.
The last $2k+2$ factors of $\varphi^{\theta}_{12}(x_1 ,x_2 )$ are determined by
$\psi (s^{(2)}, t^{(2)})(u^{(2)})-\psi (s^{(1)}, t^{(1)})(u^{(1)})$ as we see in (\ref{eq_direction}).
Comparing them with those of $v$, we can see that $s^{(2)}$, $u^{(1)}$ and $u^{(2)}$ are uniquely determined
so that $s^{(2)}$ is near $q_+$ (see also Lemma \ref{lem_linking_number}).


Thus we can fix $x_2$ if we give any $t^{(2)}$.
Then $x_1$ is also uniquely determined, since $u^{(1)}$ is already determined as above, and $(s^{(1)},t^{(1)})$ is
determined by comparing the first $4k-1$ factors of $v$ with those of $\varphi^{\theta}_{12}(x_1 ,x_2 )$.
\end{proof}


Figure \ref{fig_link} shows an example of $x=(x_1 ,\dots ,x_4) \in N_{1,3,4,2}$.
In this case $x_1,\dots ,x_4$ are uniquely determined by $v_1 =\varphi^{\theta}_{12}(x)$ and
$v_2 =\varphi^{\theta}_{34}(x)$ up to $t^{(2)}$ and $s^{(4)}$ by Lemma \ref{lem_local_N2}, and $t^{(2)}$ and $s^{(4)}$
are determined by $w=(x_3 -x_2 )/\abs{x_3 -x_2}$.
\begin{figure}[htb]%inverse_image_3
\[
 \begin{xy}
 {\ar (0,0);(80,0)},{\ar (0,0);(20,30)},{\ar (0,0);(0,25)}, (0,30)*{\R^1},(12,30)*{\R^{2k-1}},(85,3)*{\R^{2k-1}},
 {\ar@{-}(23,7);(37,28)},(32,26)*{N_4},
 {\ar@{-}(16,15);(27,15)},{\ar@{-}(29.5,15);(70,15)},(68,18)*{N_3},
 {\ar@{-}(23,3);(30.6,14.4)},{\ar@{-}(31.6,15.9);(37,24)},(39,22)*{N_2},
 {\ar@{-}(16,9);(23,9)},{\ar@{.}(25,9);(27,9)},{\ar@{-}(28,9);(70,9)},(68,6)*{N_1},
 (31,19)*{\bullet},(33,15)*{\bullet},{\ar (33,15);(31,19)},(28,19)*{x_3}, (36,17)*{x_2},
 (29,12)*{\bullet},(33,9)*{\bullet}, (35,7)*{x_1},
 (29,4)*{x_4}="A", {\ar@{-}@/_1mm/"A";(29,12)},
 {\ar@{.}(6,9);(16,9)}, (3,9)*{q_{\pm}},
 {\ar@{.}(21,0);(23,3)}, (19,2)*{q_{\pm}},
 \end{xy}
\]
\caption{$x=\varphi^{-1}_{H_1}(v_1 ,v_2 ,w)\in N_{1,3,4,2}$}\label{fig_link}
\end{figure}


Firstly we compute the contribution of Type II configurations.
Choose neighborhoods $U^{N-1}_{\pm}$ of $p^{N-1}_{\pm}$ ($N=6k$ or $4k-1$) so that the support of $vol_{S^{N-1}}$ is
contained in $U^{N-1}_+ \sqcup U^{N-1}_-$.
Also choose neighborhoods $V^{2k-1}_{\pm}$ of $q_{\pm}$ so that all the four points $(x_1 ,\dots ,x_4 )$ of Type II
configurations are mapped into the same $V^{2k-1}_l \times V^{2k-1}_m$ ($l,m=\pm$) by $pr_{4k-2}:\R^{4k-1}\to \R^{4k-2}$.
Define
\[
 N^{l,m}_i := V^{2k-1}_{l} \times V^{2k-1}_{m} \times (\xi_i -\varepsilon ,\, \xi_i +\varepsilon )
 \subset N_i ,\quad l,m =\pm .
\]
Define $N^{l,m}_{p,q,r,s}:=N^{l,m}_p \times N^{l,m}_q \times N^{l,m}_r \times N^{l,m}_s$ ($l,m=\pm$,
$\{ p,q,r,s \} =\{ 1,2,3,4\}$).
Type II configurations are in $N^{l,m}_{i,i+2,i+1,i+3}$ or $N^{l,m}_{i,i+2,i+3,i+1}$ for some $i=1,\dots ,4$ and
$l,m=\pm$.
There are $4\times 4\times 2=32$ such components.
Via the direction map $\varphi_{H_1}$, each component is mapped to some
$U^{6k-1}_l \times U^{6k-1}_{l'} \times U^{4k-2}_{l''}$ homeomorphically.
Hence each component contributes to $\pair{I(H_1 )}{S_{\psi}}$ by $\pm (1/2)^3 =\pm 1/8$.


The signs are given as follows.


\begin{lem}\label{lem_signs}
\begin{enumerate}
\item
 All the signs for $N^{l,m}_{p,q,r,s}$, $l,m=\pm$, are same for fixed $p,q,r,s$.
\item
 If the orientations are chosen so that the sign for $N^{l,m}_{1,3,2,4}$ is $+1$, then the signs for $N^{l,m}_{2,4,1,3}$
 and $N^{l,m}_{3,1,4,2}$ are $-1$, and those for other five components $N^{l,m}_{2,4,3,1}$, $N^{l,m}_{3,1,4,2}$,
 $N^{l,m}_{4,2,1,3}$, $N^{l,m}_{1,3,4,2}$ and $N^{l,m}_{4,2,3,1}$ are $+1$.
\end{enumerate}
\end{lem}


\begin{proof}
(1) is because the difference between the integrations of $\omega_{H_1}$ over $N^{l,m}_{p,q,r,s}$ and
$N^{-l,m}_{p,q,r,s}$ comes from the antipodal map of $S^{2k-1}$, which preserves the orientation.


For (2), first we show that the signs for $N^{l,m}_{1,3,2,4}$ and $N^{l,m}_{3,1,4,2}$ are different.
This follows from the diagram which is commutative up to isotopy
\[
 \xymatrix{
  N^{l,m}_{1,3,2,4} \ar[d]_-G \ar[r]^-{\varphi_{H_1}} & U^{6k-1}_l \times U^{6k-1}_m \times U^{4k-2}_-
   \ar[d]^-{\iota_{S^{6k-1}} \times \iota_{S^{6k-1}} \times g} \\
  N^{l,m}_{2,4,1,3} \ar[r]^-{\varphi_{H_1}} & U^{6k-1}_l \times U^{6k-1}_m \times U^{4k-2}_-
 }
\]
where
\begin{multline*}
 G(x_1 ,\dots ,x_4 ) := (t^{(1)},s^{(1)},u^{(1)}-\xi_1 +\xi_2 \, ; \, t^{(2)},s^{(2)},u^{(2)}-\xi_3 +\xi_4 \, ; \\
 t^{(3)},s^{(3)},u^{(3)}-\xi_3 +\xi_1 \, ; \, t^{(4)},s^{(4)},u^{(4)}-\xi_4 +\xi_3 ),
\end{multline*}
and $g$ is induced from $(s,t,u)\mapsto (s,t,-u)\in (\R^{2k-1})^2 \times \R^1$.
$G$ preserves the orientation but $\iota_{S^{6k-1}} \times g \times \iota_{S^{6k-1}}$ does not, so the signs for
$N^{l,m}_{1,3,2,4}$ and $N^{l,m}_{3,1,4,2}$ are different.
The signs for $N^{l,m}_{2,4,1,3}$ and $N^{l,m}_{3,1,4,2}$ are same, since their difference comes from an automorphism
of $H_1$, which always preserves the orientation.


Next we show that all the signs for other five components are same as that for $N^{l,m}_{1,3,2,4}$.


\noindent
\underline{(i) $N^{l,m}_{1,3,2,4}$ and $N^{l,m}_{4,2,3,1}$};
the signs for them are same, since the difference arises from an automorphism of $H_1$.


\noindent
\underline{(ii) $N^{l,m}_{1,3,2,4}$ and $N^{l,m}_{2,4,3,1}$};
their signs are same because of the following commutative diagram (up to isotopy);
\[
 \xymatrix{
  N^{l,m}_{1,3,2,4} \ar[d]_-{G'} \ar[r]^-{\varphi_{H_1}} & U^{6k-1}_l \times U^{6k-1}_m \times U^{4k-2}_-
   \ar[d]^-{i_1 \times \id \times g} \\
  N^{l,m}_{2,4,3,1} \ar[r]^-{\varphi_{H_1}} & U^{6k-1}_l \times U^{6k-1}_m \times U^{4k-2}_+
 }
\]
where
\begin{multline*}
 G'(x_1 ,\dots ,x_4 ) := (s^{(2)},t^{(1)},u^{(2)} \, ; \, s^{(1)},t^{(2)},u^{(1)} ; \\
 s^{(3)}-s^{(2)}+s^{(1)},t^{(3)},u^{(3)} \, ; \, s^{(4)}-s^{(2)}+s^{(1)},t^{(4)},u^{(4)}),
\end{multline*}
and $i_1$ was given in Lemma \ref{lem_outside_zero}.
The vertical arrows preserve the orientations, so the signs are same.
Similar diagrams shows that the signs for $N^{l,m}_{1,3,2,4}$ and $N^{l,m}_{3,1,2,4}$ are same.


\noindent
\underline{(iii) $N^{l,m}_{4,2,1,3}$ and $N^{l,m}_{1,3,4,2}$};
their signs are same as those of $N^{l,m}_{3,1,2,4}$ and $N^{l,m}_{1,3,2,4}$ respectively, since their differences
come from automorphisms of $H_1$.
\end{proof}


By Lemma \ref{lem_signs}, 24 components contribute to $\pair{I(H_1 )}{S_{\psi}}$ by $\pm 1$ and other 8 components
by $\mp 1$.
Thus their contribution is $\pm (1/8)\times (24-8) =\pm 2$.


Next consider Type I configurations.


\begin{lem}\label{lem_dim_reason}
The contribution of Type I configurations is zero.
\end{lem}


\begin{proof}
Consider the case when all the four points $S_{\psi}(x_p )$ are near $z_1$.
Then $x\in C_4 ((N_1 \sqcup N_3 )\cap L)$.
But on this space, the direction map $\varphi_{H_1}$ is invariant under the translation
\[
 (x_1 ,\dots ,x_4 ) \mapsto (\tau_v x_1 ,\dots ,\tau_v x_4 ),\quad v\in \R^{2k-1}
\]
where $\tau_v (s,t,u):=(s,t+v,u)\in \R^{2k-1}\times \R^{2k-1}\times \R^1$.
This is because of \eqref{eq_direction} and $\psi (s,t)(u)=\psi (s,t+v)(u)$ if $(s,t,u)\in (N_1 \sqcup N_3 )\cap L$.
Hence the image of $\varphi_{H_1}$ must be of positive codimension and the integrand
$\omega_{H_1}=\varphi^*_{H_1} (vol_{S^{6k-1}}\times vol_{S^{6k-1}}\times vol_{S^{4k-2}})$ must vanish on
$C_4 ((N_1 \sqcup N_3 )\cap L)$.


In the case when four points $S_{\psi}(x_p )$ are near $z_2$, a similar translation $\tau'_v (s,t,u):=(s+v,t,u)$
would prove the vanishing of $\omega_{H_1}$ on $C_4 ((N_2 \sqcup N_4 )\cap L)$.
\end{proof}


Thus the first term of \eqref{eq_evaluation} is equal to $\pm 2 \times (1/2)=\pm 1$.





\subsubsection{Remaining terms}\label{sssec_remaining}
To complete the proof of Theorem \ref{thm_Haefliger}, we will prove that the second and the third terms of
\eqref{eq_evaluation} do not contribute to $\pair{\calH}{S_{\psi_{\delta}}}$.


\noindent
{\bf Contribution of $H_2$.}
Now we compute the second term of \eqref{eq_evaluation}.
Recall $C_{H_2}(S_{\psi}) \subset C^o_3 (\R^{4k-1})\times \R^{6k}$.


\begin{lem}\label{lem_ev_H2}
$\pair{I(H_2 )}{S_{\psi_{\delta}}}=O(\delta )$.
\end{lem}


\begin{proof}
Divide $\pair{I(H_2 )}{S_{\psi}}$ into two integrations;
the integration over
\[
 R_{\ge 1}:=\{ (x;y) \in C_{H_2}(S_{\psi}) \, | \, y \not\in \R^{4k}\times (-1,1)^{2k}\}
\]
($x=(x_1 ,x_2 ,x_3 )$) and that over $R_{<1}:=C_{H_2}(S_{\psi}) \setminus R_{\ge 1}$.


The integration over $R_{\ge 1}$ is well defined and continuous at $\delta =0$, since $y\in \R^{6k}$ is
far from the image of $S_{\psi_{\delta}}$ (see \eqref{eq_thin}).
But when $\delta =0$, this integral is zero, since all the three points $S_{\psi}(x_p )$ ($p=1,2,3$) are in
$\R^{4k}\times \{ 0 \}^{2k}$ and hence the image of the direction map
$\varphi_{H_2}:=\varphi^{\theta}_{14}\times \varphi^{\theta}_{24}\times \varphi^{\theta}_{34}$ is of positive codimension
$\ge 2k-1$ in $(S^{6k-1})^3$ (see Lemma \ref{lem_localize2}).
Hence the integration over $R_{\ge 1}$ is $O(\delta )$.


Next consider the integration over $R_{<1}$.
We have only to consider $(x;y)$ with the first $6k-1$ factors of $S_{\psi}(x_1 ),S_{\psi}(x_2 ),S_{\psi}(x_3 )$ and $y$
close to each other; otherwise the image of $\varphi_{H_2}$ cannot be in $(\text{supp}(vol))^3$ (see condition (2) just
before Figure \ref{fig_contribution}).
In particular all $(s^{(p)},t^{(p)})\in \R^{4k-2}$ ($p=1,2,3$) must be close to each other.


Consider the case that $x=(x_1 ,x_2 ,x_3 )\in C_3 (\R^{4k-1}\setminus L)$.
Similarly as in Lemma \ref{lem_outside_zero}, if we choose $vol_{S^{6k-1}}$ with sufficiently small support, then we
may assume that at most one of $x\cap (X\setminus L)$ and $x\cap (Y\setminus L)$ is non-empty.
We can prove the vanishing of the integration of $\omega_{H_2}$ over $\{ (x;y) \,| \, x \cap X \ne \emptyset \}$ in a
similar way as in Lemma \ref{lem_outside_zero} by considering an involution
$F_1 : (x_1 ,x_2 ,x_3 ;y)\mapsto (i_1 x_1 ,i_1 x_2 ,i_1 x_3 ;i_1 y)$ ($i_1$ was defined in Lemma
\ref{lem_outside_zero}), which preserves the orientation but satisfies $F^*_1 \omega_{H_2}=-\omega_{H_2}$ because
$\varphi^{\theta}_{p4} \circ F_1 = i_1 \circ \varphi^{\theta}_{p4}$ ($p=1,2,3$) on $X$ by \eqref{eq_psi_symmetry_1} and
$i^*_1 vol_{S^{6k-1}}=-vol_{S^{6k-1}}$.
The vanishing of the integrations over $\{ (x;y) \,| \, x \cap Y \ne \emptyset \}$ and
$\{ x\cap X = x\cap Y=\emptyset \}$ can also be proved by similar involutions $F_i$, $i=1,2$.


So we may assume that one of $x_p$ is in $L$ (and other two points are near $L$).
Since the first $6k-1$ factors of $S_{\psi}(x_p )$ ($p=1,2,3$) are close to each other, an analogous argument to the
proof of Lemma \ref{lem_localize2} shows that only the integration over the subspace of $(x;y)$ with $x$ in
$C_3 (N_1 \sqcup N_3 )$ or $C_3 (N_2 \sqcup N_4 )$ or $C_3 (N'_1 \sqcup N'_2 )$, where
$N'_i :=\R^{4k-2}\times (\zeta_i -\varepsilon ,\zeta_i +\varepsilon )$ ($i=1,2$) correspond to the crossing $c$ of $f$
(see \S \ref{sssec_choice_psi} and Figure \ref{fig_f}); otherwise two or more $S_{\psi}(x_p )$'s are in
$\R^{4k}\times \{ 0\}^{2k}$ and hence the image of the map $\varphi_{H_2}$ is of positive codimension $\ge 2k-1$ in
$(S^{6k-1})^3$.
But similarly as in Lemma \ref{lem_dim_reason}, on these spaces we can define translations $\tau$, $\tau'$ under which
$\varphi_{H_2}$ is invariant, and hence the integrand $\omega_{H_2}$ must vanish by dimensional reason.
\end{proof}


\noindent
{\bf Contribution of the correction term $c$.}
Lastly we compute the third term of \eqref{eq_evaluation}.
This is an integration of $D(S_{\psi})^* \mu$ over $\R^{4k-1}$ (cf.\ Lemma \ref{lem_localize3}).
See Lemma \ref{lem_hatomega_closed} and Definition \ref{def_c} for the definition of $c$.


\begin{lem}\label{lem_ev_c}
$\pair{c}{S_{\psi}}=0$.
\end{lem}


\begin{proof}
First we show that integrations of $D(S_{\psi})^* \mu$ over $X$, $Y$ and $Z$ (see \S \ref{sssec_choice_psi}) vanish.
For $X$, this is because
\begin{itemize}
\item
 $i^*_1 D(S_{\psi})^* \mu =D(S_{\psi})^* \mu$ on $X$, since $D(S_{\psi})\circ i_1 =i_1 \circ D(S_{\psi})$ on $X$ by
 \eqref{eq_psi_symmetry_1} (where $i_1 : \I_{4k-1} (\R^{6k})\to \I_{4k-1} (\R^{6k})$ is given by $f\mapsto i_1 \circ f$;
 see Remark \ref{rem_mu_symmetric}) and we can choose $\mu$ so that $i^*_1 \mu =\mu$ (see Remark \ref{rem_mu_symmetric}),
 and
\item
 $i_1$ is an orientation reversing diffeomorphism of $X$.
\end{itemize}
Similar arguments hold for $Y$ and $Z$.


So we may restrict the integration to $[-1/2,1/2]^{4k-2}\times \R^1$.
If $(s,t,u)\not\in N_3 \sqcup N_4$, then $\psi (s,t)(u)$ does not depend on $(s,t)\in \R^{4k-2}$, so $D(S_{\psi})$ is
invariant under the translations $\tau$ and $\tau'$ defined in the proof of Lemma \ref{lem_dim_reason}.
If $(s,t,u)\in N_3$ (resp.\ $N_4$), then $\psi (s,t)(u)$ does not depend on $t\in \R^{2k-1}$ (resp.\ $s$), so
$D(S_{\psi})$ is invariant under the translations $\tau$ (resp.\ $\tau'$).
Thus  the image of $[-1/2,1/2]^{4k-2}\times \R^1$ via $D(S_{\psi})$ must be of dimension $<4k-1$ and hence
a ($4k-1$)-form $D(S_{\psi})^* \mu$ must vanish on $[-1/2,1/2]^{4k-2}\times \R^1$.
\end{proof}


Thus we have completed the proof of Theorem \ref{thm_Haefliger};
only Type II configurations for $H_1$ contribute to $\pair{\calH}{S_{\psi}}$ by $\pm 1$, and hence
$\pair{\calH}{S_{\psi}}=\pm 1$.





\subsection{Non-triviality of $\calH$ in general dimensions}\label{subsec_H_general}


Here we complete the proof of Theorem \ref{thm_main1}.
Suppose that $n>j\ge 2$, $n-j\ge 3$ is odd and $m:=2n-3j-3>0$.
Put $n-j=2k+1$ ($k\ge 1$) and consider $S_{\psi} \in \emb{6k}{4k-1}$ as above.
Notice that $n=6k-m$ and $j=4k-m-1$, and in particular $4k-m-1>0$.


Since $S_{\psi}$ is of the form (\ref{eq_deform_spinning}), we can find
$l_m \in \Omega^m \emb{6k-m}{4k-m-1}=\Omega^m \emb{n}{j}$ such that $S_{\psi}=\gr^m (l_m )$.
Explicitly we can define $l_m :\R^m \to \emb{n}{j}$ by
\begin{equation}\label{eq_graphing}
\begin{split}
 &l_m (t_1 ,\dots ,t_m)(x_1 ,\dots ,x_j ) := \\
 &\quad ((x_1 ,\dots ,x_{j-1}), \psi(t_1 ,\dots ,t_m , x_1 ,\dots ,x_{j-1})(x_j)) \in \R^{j-1} \times \R^{2k+2}=\R^n
\end{split}
\end{equation}
and regard it as in $\Omega^m \emb{n}{j}$.
We think of $[l_m ]$ as the generator of $H_m (\emb{n}{j})$ via the Hurewicz isomorphism ($\emb{n}{j}$ is
$(2n-3j-4)$-connected; see \cite{Budney08}).


Consider $\calH =[I(H)+c]\in H^m_{DR}(\emb{n}{j})$.
The following completes the proof of Theorem \ref{thm_main1}.


\begin{thm}\label{thm_Haefliger_general}
The Kronecker pairing $\pair{\calH}{l_m}$ is equal to $\pm 1$.
\end{thm}


\begin{proof}
Define the spaces $\hat{C}_{H_i}$, $i=1,2$, by $\hat{C}_{H_1} := \R^m \times C^o_4 (\R^j )$ and
\[
 \hat{C}_{H_2} :=
 \{ (t,(x_1 ,x_2 ,x_3 ),y)\in \R^m \times C^o_3 (\R^j ) \times \R^n \, | \, l_m (t)(x_p )\ne y,\ p=1,2,3 \} .
\]
These spaces are also defined as the following pullback square;
\[
 \xymatrix{
  \hat{C}_{H_i} \ar[r]^-{\hat{l}_m} \ar[d] & C_{H_i} \ar[d]^-{\pi_{H_i}} \\
  \R^m \ar[r]^-{l_m}                       & \emb{n}{j}
 }
\]
Then $\pair{\calH}{l_m}$ is equal to
\begin{equation}\label{eq_ev_general_dim}
 \frac{1}{4}\int_{\hat{C}_{H_1}} \hat{l}^*_m \omega_{H_1} + \frac{1}{12} \int_{\hat{C}_{H_2}} \hat{l}^*_m \omega_{H_2}
 +\int_{\R^m \times \R^j}D(l_m )^* \mu ,
\end{equation}
where $D(l_m ):\R^m \times \R^j \to \I_j (\R^n )$ is defined by $(t,x)\mapsto d(l_m (t))_x$.


Recall $C_{H_1}(S_{\psi})=C^o_4 (\R^{4k-1})$ and $C_{H_2}(S_{\psi})\subset C^o_3 (\R^{4k-1})\times \R^{6k}$.
We regard $\hat{C}_{H_i} \subset C_{H_i}(S_{\psi})$ ($i=1,2$);
\begin{align*}
 \hat{C}_{H_1} &\cong \{ (x_1 ,\dots ,x_4 ) \in C_{H_1}(S_{\psi}) \, |\, pr_m (x_1 )=\dots =pr_m (x_4 )\} , \\
 \hat{C}_{H_2} &\cong \{ (x_1 ,x_2 ,x_3 ;y) \in C_{H_2}(S_{\psi}) \, |\, pr_m (x_1 )=pr_m (x_2 )=pr_m (x_3 )=pr_m (y) \}
\end{align*}
($pr_m :\R^N \to \R^m$ ($N=4k-1$ or $6k$) is the first $m$ projection), via diffeomorphisms given by respectively
\[
 (t,x)   \longmapsto ((t,x_1 ),\dots ,(t,x_4 )), \quad
 (t,x,y) \longmapsto ((t,x_1),(t,x_2 ),(t,x_3 ),(t,y)).
\]
The direction maps $C_{H_i} \to S^{N-1}$, $N=n$ or $j$, composed by $\hat{l}_m$ are regarded as
\begin{equation}\label{eq_direction_proj}
 \begin{split}
 &\varphi^{\theta}_{12} : \hat{C}_{H_1}\longrightarrow S^{n-1}=S^{6k-1}\cap \{ x\in \R^{6k} \, ; \, pr_m (x) =0 \}, \\
 &\varphi^{\theta}_{12}((t,x_1 ),(t,x_2 ))=
 \frac{(x_2 -x_1 ,S_{\psi}(t, x_2 )-S_{\psi}(t, x_1 ))}{\abs{(x_2 -x_1 ,S_{\psi}(t, x_2 )-S_{\psi}(t, x_1 ))}},
 \end{split}
\end{equation}
and so on.
Then the integrations relating to $L\subset \R^m \times \R^j =\R^{4k-1}$ (a neighborhood of
$[-1/2 ,1/2]^m \times [-1/2,1/2]^{j-1}\times \R^1$) can be computed in similar ways as in the previous subsection;


\noindent{\bf The first term.}
Type I contribution vanishes by the translations $\tau$ or $\tau'$ as in Lemma \ref{lem_dim_reason}.
Type II configurations contribute by $\pm 2$; each component
\begin{align*}
 &(N^{l,m}_{i,i+2,i+1,i+3}) \cap \{ pr_m (x_1 )=\dots =pr_m (x_4 )\} \ \text{ or} \\
 &(N^{l,m}_{i,i+2,i+3,i+1}) \cap \{ pr_m (x_1 )=\dots =pr_m (x_4 )\}
\end{align*}
for some $i=1,\dots ,4$ and $l,m=\pm$ is mapped via the direction map to some
\[
 U^{6k-1}_l \times U^{6k-1}_{l'} \times U^{4k-2}_{l''}\cap \{ \text{first } m \text{ projections}=0\} ,
\]
which is $U^{n-1}_l \times U^{n-1}_{l'} \times U^{j-1}_{l''}$ (see \S \ref{sssec_first_term}).
The sign arguments are slightly different, but the result is same as Lemma \ref{lem_signs};
in the first diagram in the proof of Lemma \ref{lem_signs}, the map $G$ restricted to $\hat{C}_{H_1}$ preserve the
orientation but the left vertical map does not.
Both the vertical maps in the second diagram restricted to $\hat{C}_{H_1}$ have the same orientation sign $(-1)^a$,
where $a=\min \{ m,2k-1\}$.



\noindent{\bf The second term.}
Similarly as in Lemma \ref{lem_ev_H2}, the integration over
$R_{\ge 1}:=\{ (t,x,y)\in \hat{C}_{H_2}\, | \, y\not\in \R^{j+1}\times (-1,1)^{2k}\}$ (notice $j+1+2k=n$) is
$O(\delta )$.
In $R_{<1}:=\hat{C}_{H_2}\setminus R_{\ge 1}$, only $(t,x,y)$ with $pr_{j-1}(x_p )$ ($p=1,2,3$) close to each other and
$x \in C_3 ((N_i \sqcup N_{i+2})\cap L)$ ($i=1,2$) or $C_3((N'_1 \sqcup N'_2 )\cap L)$ may contribute to the integral
(where $N_i := \R^m \times \R^{j-1}\times (\xi_i -\varepsilon ,\xi_i +\varepsilon )$,
$N'_i := \R^m \times \R^{j-1}\times (\zeta_i -\varepsilon ,\zeta_i +\varepsilon )$); otherwise the image of
$\varphi_{H_2}$ is of positive codimension.
But on these spaces the direction map is invariant under the similar translations $\tau$ or $\tau'$ to those in
Lemma \ref{lem_dim_reason} and the integrand vanishes by dimensional reason.


\noindent{\bf The third term.}
Similarly as in Lemma \ref{lem_ev_c}, since the derivation map $D(l_m )$ is invariant on $L$ under the translations
given by using $\tau$ or $\tau'$, the integration over $L$ vanishes by dimensional reason.


The signs appearing in the proof of vanishing of integrations over $\R^{4k-1}\setminus L$ are slightly different from
those in the previous subsection.
Recall the involutions $F_1$ and $F_2$ on ${C}_{H_i}(S_{\psi})$.
They clearly preserve $\hat{C}_{H_i}$, and are given by
\begin{alignat*}{2}
 F_l ((t,x_1 ),\dots ,(t,x_4 ))&=(i_l (t,x_1 ),\dots ,i_l (t,x_4 )) &\quad &\text{on } \hat{C}_{H_1}, \\
 F_l ((t,x_1 ),(t,x_2 ),(t,x_3 );(t,y))&=(i_l (t,x_1 ),i_l (t,x_2 ),i_l (t,x_3 );i_l (t,y))&\quad &\text{on }\hat{C}_{H_2}
\end{alignat*}
($l=1,2$), here we regard $\hat{C}_{H_i}$ as a subspace of $C_{H_i}(S_{\psi})$ as above.
The maps $i_1 ,i_2:\R^N\to \R^N$ ($N=4k-1$ or $6k$) are as given in Lemma \ref{lem_outside_zero}.
For $N=4k-1$, they are written explicitly as
\begin{align*}
 i_1 (t,x) &=
  \begin{cases}
   (-t_1 ,\dots ,-t_m ;-x_1 ,\dots ,-x_{2k-1-m},x_{2k-m},\dots ,x_j ) & m\le 2k-1, \\
   (-t_1 ,\dots ,-t_{2k-1}, t_{2k},\dots ,t_m ;x_1 ,\dots ,x_j )      & m>2k-1,
  \end{cases} \\
 i_2 (t,x) &=
  \begin{cases}
   (t_1 ,\dots ,t_m ;x_1 ,\dots ,x_{2k-1-m},-x_{2k-m},\dots ,-x_{j-1}, x_j ) & m\le 2k-1, \\
   (t_1 ,\dots ,t_{2k-1}, -t_{2k},\dots ,-t_m ;-x_1 ,\dots -x_{j-1}, x_j )   & m>2k-1.
  \end{cases}
\end{align*}
First consider the first and the second terms of \eqref{eq_ev_general_dim}.
Let $X,Y,Z\subset \R^m \times \R^j =\R^{4k-1}$ be subsets defined similarly as in \S \ref{sssec_choice_psi}, and set
\begin{align*}
 \hat{C}_{H_1}(X) &:= \{ (t,(x_1 ,\dots ,x_4 ))\in \hat{C}_{H_1} \, | \, (t,x_p )\in X,\ \forall p \} , \\
 \hat{C}_{H_2}(X) &:= \{ (t,(x_1 ,x_2 ,x_3 ),y)\in \hat{C}_{H_2} \, | \, (t,x_p )\in X,\ \forall p \}
\end{align*}
and so on.
Then the actions of $F_1$ and $F_2$ on the forms $\omega_{H_i}$ and orientations of the spaces are described as in
Table \ref{table:signs}, which is a consequence of the equations
\begin{align*}
 \varphi \circ F_1 &=
  \begin{cases}
   i_{1,2k-m-1} \circ \varphi & m\le 2k-1 \\
   \varphi                    & m>2k-1
  \end{cases}
 \quad \text{on } \hat{C}_{H_i}(X), \\
 \varphi \circ F_2 &=
  \begin{cases}
   i_{2k-m,j-1} \circ \varphi & m\le 2k-1 \\
   i_{1,j-1}    \circ \varphi & m>2k-1
  \end{cases}
 \quad \text{on } \hat{C}_{H_i}(Y),
\end{align*}
where $\varphi$ is one of the direction maps, and $i_{p,q}:\R^n \to \R^n$ ($p<q$) is given by
\[
 i_{p,q} (a_1 ,\dots ,a_n ) :=(a_1 ,\dots ,a_{p-1},-a_p ,\dots ,-a_q , a_{q+1},\dots ,a_n )
\]
(in particular $i_1$ and $i_2$ we have used can be written as $i_1 =i_{1,2k-1}$, $i_2 =i_{2k,4k-2}$).
\begin{table}[ht]
%\tbl{Signs of $F_l$}
{\begin{tabular}{|c||c|c|}\hline
                                           & $m\le 2k-1$                 & $m>2k-1$ \\
\hline\hline
orientation sign of $F_1$                  & $(-1)^m$                    & $-1$ \\
\hline
$F^*_1 \omega_{H_i}$ on $\hat{C}_{H_i}(X)$ & $(-1)^{2k-1-m}\omega_{H_i}$ & $+\omega_{H_i}$ \\
\hline
orientation sign of $F_2$                  & $+1$                        & $(-1)^{m-2k+1}$ \\
\hline
$F^*_2 \omega_{H_i}$ on $\hat{C}_{H_i}(Y)$ & $-\omega_{H_i}$             & $(-1)^{j-1}\omega_{H_i}$ \\ \hline
\end{tabular}}
\caption{Signs of $F_l$}
\label{table:signs}
\end{table}
Thus the integrations of $\omega_{H_i}$ over $\hat{C}_{H_i}(X)$ and $\hat{C}_{H_i}(Y)$ vanish (when $m>2k-1$, we use
$j=4k-1-m$ and hence $m-2k+j$ is odd).


For the third term of \eqref{eq_ev_general_dim}, we have to study the signs arising from the involutions $i_1$ and $i_2$.
The orientation signs of $i_1 ,i_2 :\R^m \times \R^j \to \R^m \times \R^j$ are always $-1$.
But $i_1,i_2$ always preserve the integrand, because
\begin{align*}
 D(l_m )\circ i_1 &=
 \begin{cases}
  i_{1,2k-m-1} \circ D(l_m ) & \text{if }m\le 2k-1, \\
  D(l_m )                    & \text{if }m>2k-1,
 \end{cases}\quad \text{on }X\cup Z, \\
 D(l_m )\circ i_2 &=
 \begin{cases}
  i_{2k-m,j-1} \circ D(l_m )  & \text{if }m\le 2k-1, \\
  i_{1,j-1} \circ D(l_m ) & \text{if }m>2k-1,
 \end{cases}\quad \text{on }Y\cup Z,
\end{align*}
and we can choose $\mu$ so that $i'_1$, $i'_2$ and $i''_2$ preserve $\mu$ (see Remark \ref{rem_mu_symmetric}).
Hence the integrations of $\mu$ over $X$, $Y$ and $Z$ vanish.
\end{proof}