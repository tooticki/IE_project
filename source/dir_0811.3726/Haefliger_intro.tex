\section{Introduction}\label{sec_intro}


The aim of this research is to study the topology of the space of long knots.


\begin{defn}\label{def_knotspace}
A {\em long $j$-knot} in $\R^n$ is an embedding $f: \R^j \hookrightarrow \R^n$ which is standard at infinity:
\[
 f(x) = (x,0) \in \R^j \times \{ 0 \}^{n-j} , \quad x \not\in [-1,1]^j .
\]
Denote by $\emb{n}{j}$ the space of long $j$-knots in $\R^n$, equipped with $C^{\infty}$-topology.
\end{defn}


In this and the forthcoming papers \cite{KWatanabe08} we will develop the methods to study $\emb{n}{j}$ originated in
perturbative Chern-Simons theory.
This method was used by various authors \cite{AltschulerFreidel97, BottTaubes94, Kohno94} to give some integral
expressions of the finite type invariants for knots in $\R^3$.
Cattaneo, Cotta-Ramusino and Longoni \cite{CCL02} generalized their constructions to define a cochain map from a
certain graph complex to the de Rham complex of the space of (long) $1$-knots in $\R^n$, $n>3$, via fiber-integrations
over configuration spaces.
Not only the trivalent graphs correspond to finite type invariants, but non-trivalent graphs can also work in their
framework.
Indeed, some non-trivalent graph cocycles produce non-trivial cohomology classes \cite{Longoni04, K07}.


Another generalization was done by Rossi \cite{Rossi_thesis} and Cattaneo-Rossi \cite{CattaneoRossi05}, who proved the
invariance of (order two) Bott invariant and order three invariant for long $m$-knots in $\R^{m+2}$ for $m \ge 2$.
Following their work, Watanabe \cite{Watanabe07} proved that there is one finite type invariant for long ``ribbon''
$m$-knots \cite{HabiroKanenobuShima99} in $\R^{m+2}$, $m \ge 3$ odd, at each even order.
These invariants also come from trivalent graphs via the perturbative method.


In this paper we introduce graph complexes in the same manner as \cite{CCL02} generated by more general graphs
than those in \cite{CattaneoRossi05, Rossi_thesis, Watanabe07}.
We prove that some graph cocycles produce cohomology classes of $\emb{n}{j}$ via configuration space integrals.


\begin{thm}\label{thm_main2}
There exist graph complexes\footnote
{
The combinatorial meaning of the number $k$ will be specified in Definition \ref{definition_space_graphs}.
}
 $\D^{k,*}_g$, $k \ge 1$ and $g\ge 0$, spanned by graphs of first Betti number $g$ (after its
`small loops' are removed) and linear maps $I : \D^{k,l}_g \to \Omega^{k(n-j-2)+(g-1)(j-1)+l}_{DR}(\emb{n}{j})$ given by
configuration space integral.
The map $I$ is a cochain map when $n-j \ge 2$ is even and $g=0$, or both $n>j\ge 3$ are odd and $g=1$.
\end{thm}


Unfortunately it is not known how `big' the graph cohomology $H^* (\D^*_0 )$ is.
But in \cite{KWatanabe08} we will prove that $H^* (\D^*_1 )$ is not trivial, and the map $I$ produces non-trivial
cohomology classes of $\emb{n}{j}$ and $\overline{\mathcal{K}}_{n,j}$ for various $n$ and $j$, which generalize the
`finite type invariants' for `long ribbon knots' \cite{HabiroKanenobuShima99, Watanabe07}.
Here $\overline{\mathcal{K}}_{n,j}$ is the space of long $j$-knots `modulo immersions' (see for example
\cite{KWatanabe08, Sinha04}).
When $n-j$ is odd, we have no idea to prove that $I$ is a cochain map, since we cannot ignore the contributions of some
`hidden faces' of the boundary of configuration spaces.
So we will need some correction terms (different from that given in Theorem \ref{thm_main1}).
See \cite{KWatanabe08}.


The graph complex introduced in \cite{CCL02} is conjectured to give all the real cohomology classes of $\emb{n}{1}$,
while it is not clear whether the graphs appeared in \cite{CattaneoRossi05, Rossi_thesis, Watanabe07} are enough for
$H^*_{DR}(\emb{n}{j})$.
But our graph complex $\D^* =\bigoplus_g \D^*_g$ contains more general graphs, so we might be able to conjecture that
our graph complex would describe whole $H^*_{DR}(\emb{n}{j})$.
The next Theorem \ref{thm_main1} confirms the conjecture; via perturbative method we can give a new formulation of the
{\em Haefliger invariant} \cite{Haefliger62,Haefliger66}.



\begin{thm}\label{thm_main1}
Suppose $n-j \ge 3$ is odd and $2n-3j-3 \ge 0$.
Then a graph cocycle $H \in \D^{2,0}_0$ produces a non-trivial cohomology class
$\calH :=[I(H)+c]\in H^{2n-3j-3}_{DR}(\emb{n}{j})$, where $c$ is some correction term.
Moreover when $n=6k$ and $j=4k-1$, $\calH \in H^0_{DR}(\emb{6k}{4k-1})$ is nothing but the Haefliger invariant
(up to sign).
\end{thm}


The correction term $c$ is added to kill some contribution of the `anomalous face' of a compactified configuration space.
This contribution is an obstruction for $I$ to be a cochain map.
See \S \ref{sec_Haefliger} and \S \ref{sec_vanish} for details.


The Haefliger invariant, an isotopy invariant for (long) $(4k-1)$-knots in $6k$-space, was originally defined by
using  a $4k$-manifold bounded by the knot in $(6k+1)$-space.
Instead of such additional data, we use the generators of cohomology of configuration space.


This paper is organized as follows.
We define the graph complexes in \S \ref{section_graph} and describe the integration map $I$ in detail in \S
\ref{sec_conf}.
Several vanishing results are proved in \S \ref{sec_vanish} to prove Theorem \ref{thm_main2}.
The class $\calH \in H^{2n-3j-3}_{DR}(\emb{n}{j})$ is studied in \S \ref{sec_Haefliger}.
To prove $d\calH =0$ we need some of the results in \S \ref{sec_vanish}, which hold even if $n-j$ is odd.


\subsection*{Acknowledgment}
The author expresses his great appreciation to Professor Toshitake Kohno for his encouragement, to Tadayuki Watanabe
for many useful suggestions and comments, and to Masamichi Takase for his idea to improve Theorem \ref{thm_main1}.
The author is partially supported by the Grant-in-Aid for Young Scientists (B), MEXT, Japan, by The Sumitomo Foundation,
and by The Iwanami Fujukai Foundation.