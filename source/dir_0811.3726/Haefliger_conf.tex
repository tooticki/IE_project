\section{Configuration space integral}\label{sec_conf}


\subsection{Configuration spaces}\label{subsec_conf}


For any space $M$, denote the space of configurations in $M$ by
\[
 C^o_k (M) := \{ (x_1 ,\dots ,x_k )\in M^k \, |\, x_p \ne x_q \text{ if } p \ne q \} .
\]
Let $\Gamma$ be a (possibly non-admissible) decorated graph with $s$ i-vertices, $t$ e-vertices and $u$ loops.
Define the configuration space associated with (vertices of) $\Gamma$ by
\[
 C^o_{\Gamma} := \left\{
 \genfrac{}{}{0pt}{}
  {(f;x_1 ,\dots ,x_s ; y_{s+1} ,\dots ,y_{s+t})}{\in \emb{n}{j} \times C^o_s (\R^j ) \times C^o_t (\R^n )}
 \right. \left|
 \genfrac{}{}{0pt}{}{f(x_p ) \ne y_q ,}{\forall p,q}
 \right\}  \times (S^{j-1})^u .
\]
We think of i-vertex $p$ (resp.\ e-vertex $q$) as corresponding to $x_p \in \R^j$ (resp.~$y_q \in \R^n$)
for all $1 \le p \le s$ (resp.\ $s+1 \le q \le s+t$).
The $S^{j-1}$-factors will be used to define a differential form $\omega_e$ for a loop $e$ (see below).
There is a projection
\[
 \pi_{\Gamma} : C^o_{\Gamma} \longrightarrow \emb{n}{j}.
\]
We will denote its fiber over $f$ by $C^o_{\Gamma}(f)$.





\subsection{Differential forms associated to graphs}\label{subsection_form}
Let $e=\overrightarrow{pq}$ be an (oriented) edge or a loop of $\Gamma$.
To $e$ we will assign a differential form $\omega_e \in \Omega^*_{DR}(C^o_{\Gamma})$ as follows.


First consider the case that $e$ is not a loop (thus $p \ne q$).
When $e$ is an $\eta$-edge (then $p,q \le s$), define the `direction map' $\varphi^{\eta}_e : C^o_{\Gamma} \to S^{j-1}$
by
\[
 \varphi^{\eta}_e (f,x,y,v) := \frac{x_q -x_p}{\abs{x_q -x_p}} \in S^{j-1}.
\]
When $e$ is a $\theta$-edge, define $\varphi^{\theta}_e : C^o_{\Gamma} \to S^{n-1}$ by
\[
 \varphi^{\theta}_e (f,x,y,v) := \frac{z_q -z_p}{\abs{z_q -z_p}} \in S^{n-1},
\]
where
\[
 z_p =
 \begin{cases}
  f(x_p ) & \text{if the vertex $p$ is internal (thus $p \le s$)}, \\
  y_p     & \text{if the vertex $p$ is external (thus $p>s$)}.
 \end{cases}
\]


\noindent
{\bf Notation.}
We denote by $vol_{S^{N-1}}$ the {\em (anti-)symmetric} volume form of $S^{N-1}$ ($N=j$ or $n$).
Namely, $vol_{S^{N-1}}$ is an $(N-1)$-form of $S^{N-1}$ with total integral one and
$\iota^* vol_{S^{N-1}} = (-1)^N vol_{S^{N-1}}$ for the antipodal map $\iota : S^{N-1} \to S^{N-1}$.

\smallskip

Define the differential forms $\omega_e \in \Omega^*_{DR}(C^o_{\Gamma})$ as
$\eta_e := (\varphi^{\eta}_e )^* vol_{S^{j-1}}$ or $\theta_e := (\varphi^{\theta}_e )^* vol_{S^{n-1}}$,
according to whether $e$ is an $\eta$-edge or a $\theta$-edge.
Notice that, if $e$ is not oriented, then the map $\varphi_e$ has ambiguity of signs, but in such a case the
corresponding volume form is of odd degree and is invariant under the antipodal map of spheres.
Hence the form $\omega_e$ is well defined.


When $n-j$ is even, we also assign differential forms to small loops.
For the $a$-th small loop $e$ with sign $\varepsilon$ (which is always $+1$ when $n$ is even) at the i-vertex $p$,
define $D_a : C^o_{\Gamma} \to S^{n-1}$ by
\[
 D_a (f,x,y,v) := \epsilon \cdot \frac{df_{x_p}(v_a )}{\abs{df_{x_p}(v_a )}},
\]
here $df_{x_p} : T_{x_p}\R^j \to T_{f(x_p )}\R^n$ is the derivation map.
The differential form associated with $e$ is $\omega_e :=D^*_a vol_{S^{n-1}} \in \Omega^{n-1}_{DR}(C^o_{\Gamma})$.
Such a form is not needed when $n-j$ is odd (see Definition \ref{definition_space_graphs}).


When $n$ is even, to the $a$-th double loop at the i-vertex $p$ with sign $\epsilon$ (which is $+1$ when $j$ is even),
we assign a map $\tilde{D}_a : C^o_{\Gamma} \to S^{j-1} \times S^{n-1}$ defined by
\[
 \tilde{D}_a (f,x,y,v):= \left( \epsilon v_a , \frac{df_{x_p}(v_a )}{\abs{df_{x_p}(v_a )}}\right)
\]
and a differential form
$\omega_e :=\tilde{D}_a^* (vol_{S^{j-1}} \times vol_{S^{n-1}})\in \Omega^{n+j-2}_{DR}(C^o_{\Gamma})$.


Define the differential form $\omega_{\Gamma}$ by
\[
 \omega_{\Gamma} := \bigwedge_{e \in E(\Gamma )}\omega_e \in \Omega^*_{DR} (C^o_{\Gamma}),
\]
here $\omega_e$'s for labeled edges must be ordered according to the labels, since by definition they are odd forms.
Since non-labeled edges correspond to even forms, we need not to mention the order of corresponding forms.





\subsection{Fiber integration and compactified configuration spaces}\label{subsection_integral}
We would like to define $I(\Gamma ) \in \Omega^*_{DR} (\emb{n}{j})$ by integrating $\omega_{\Gamma}$ along the fiber
of $\pi_{\Gamma} : C^o_{\Gamma} \to \emb{n}{j}$.
It is not clear whether such an integral converges, because the fiber of $\pi_{\Gamma}$ is not compact.
This difficulty has been resolved in \cite{AxelrodSinger94,BottTaubes94}; for any manifold $M$, we can construct a
compact manifold $C_k (M)$ with corners so that its interior is $C^o_k (M)$, by `blowing up' all the diagonals of $M^k$.
We can smoothly extend the direction maps like $\varphi$'s, the projections $C^o_k (M) \to C^o_{k-l}(M)$ ($l>0$)
and the evaluation map
\[
 ev : \emb{n}{j} \times C^o_k (\R^j ) \longrightarrow C^o_k (\R^n ), \quad (f;x_1 ,\dots ,x_k ) \longmapsto
 (f(x_1 ), \dots ,f(x_k ))
\]
onto $C_k (M)$.


Let $s$ and $t$ be the numbers of i- and e-vertices of $\Gamma$ respectively.
Define $C'_{\Gamma}$ by the pull-back square
\[
 \xymatrix{
  C'_{\Gamma} \ar[r] \ar[d] & C_{s+t}(\R^n ) \ar[d]^-{pr_s} \\
  \emb{n}{j} \times C_s (\R^j ) \ar[r]^-{ev} & C_s (\R^n )
 }
\]
where $pr_s$ is the first $s$ projection, and define $C_{\Gamma} := C'_{\Gamma} \times (S^{j-1})^u$ ($u$ is the number of
loops of $\Gamma$).
The differential form $\omega_{\Gamma}$ is defined on $C_{\Gamma}$ since the direction maps $\varphi$'s are well defined
on $C_{\Gamma}$.
The natural projection $\pi_{\Gamma} : C_{\Gamma} \to \emb{n}{j}$ is defined and is a fibration with compact fibers.
Thus the integration
\[
 I(\Gamma ) :=(\pi_{\Gamma})_* \omega_{\Gamma} \in \Omega^*_{DR}(\emb{n}{j})
\]
along the fiber is well defined.
The degree of $I(\Gamma )$ is given by
\begin{align*}
 &\deg \omega_{\Gamma} -\dim \text{fib}(\pi_{\Gamma}) \\
 &\ =(n-1)\sharp \{ \theta \} + (j-1)\sharp \{ \eta \} -j\sharp \{ \bullet \} - n\sharp \{ \circ \} -(j-1)u \\
 &\ =(n-3)(\sharp \{ \theta \} - \sharp \{ \circ \}) + (j-1)(\sharp \{ \eta \} - \sharp \{ \bullet \}-u)
  + 2 \sharp \{ \theta \} - 3\sharp \{ \circ \} -\sharp \{ \bullet \} \\
 &\ =(n-3)\ord \Gamma + (j-1)(\sharp \{ \eta \} - \sharp \{ \bullet \}-u) + \deg \Gamma .
\end{align*}
Suppose that a one-dimensional CW complex $\Gamma \setminus \{ \text{small loops}\}$ has $g$ loop components (that is,
the first Betti number of it is $g$).
Then
\[
 \sharp \{ \eta \} - \sharp \{ \bullet \} -u = -\sharp \{ \theta \} + \sharp \{ \circ \}+(g-1) = -\ord \Gamma +(g-1),
\]
and hence
\[
 \deg I(\Gamma ) = (n-j-2)\ord \Gamma +(g-1)(j-1) + \deg \Gamma .
\]


\begin{prop}[\cite{CCL02}]
The form $I(\Gamma )$ depends only on the equivalence class of $\Gamma \in \D^*$.
\end{prop}


\begin{proof}
This is because the space $\D^*$ is arranged so that the map $I$ is compatible with the permutations of coordinates
and the antipodal map of spheres (see \cite{CCL02} for details).
But there are two point we should stress here; one is that we have to temporarily gave the labels to loops when $j$ is
odd.
But the choices of the labeling do not change the form $I(\Gamma )$ since the permutation on the $(S^{j-1})^u$-factor
of the fiber $C_{\Gamma}(f)$ is an orientation preserving diffeomorphism and preserves the form $\omega_{\Gamma}$.


Another is that we ruled out the graphs with small loops when $n-j \ge 3$ is odd, and those with double loops when $n$ is
odd (see Definition \ref{definition_space_graphs}).
This is because the differential forms arising from such graphs are zero (see Lemmas \ref{lem_small_loop_vanish},
\ref{lem_double_loop_vanish}).
\end{proof}


\begin{lem}\label{lem_small_loop_vanish}
When $n-j$ is odd, then the differential form $I(\Gamma )$ is zero for any graph $\Gamma$ with at least one small loop.
\end{lem}


\begin{proof}
Let $e$ be a small loop of $\Gamma$.
Define a fiberwise involution $F:C_{\Gamma} \to C_{\Gamma}$ as the antipodal map on $S^{j-1}$ corresponding to $e$,
and the identity on the other factors.
The orientation sign of the involution on the fiber of $\pi_{\Gamma}$ is $(-1)^j$, while
$F^* \omega_{\Gamma}=(-1)^n \omega_{\Gamma}$ since
\begin{align*}
 F^* \omega_e &= \left( \frac{df}{\abs{df}} \circ \iota_{S^{j-1}} \right)^* vol_{S^{n-1}}
  = \left( \iota_{S^{n-1}} \circ \frac{df}{\abs{df}} \right)^* vol_{S^{n-1}} \\
 &= (-1)^n \left( \frac{df}{\abs{df}} \right)^* vol_{S^{n-1}} =(-1)^n \omega_e
\end{align*}
and other $\omega_e$'s do not change.
Thus
\[
 (\pi_{\Gamma})_* \omega_{\Gamma} = (\pi_{\Gamma} \circ F)_* F^* \omega_{\Gamma}
 = (-1)^{j+n}(\pi_{\Gamma})_* \omega_{\Gamma} = -(\pi_{\Gamma})_* \omega_{\Gamma}
\]
since $n+j$ is odd, and hence $(\pi_{\Gamma})_* \omega_{\Gamma}=0$.
\end{proof}


\begin{lem}\label{lem_double_loop_vanish}
When $n$ is odd, then the differential form $I(\Gamma )$ is zero for any graph $\Gamma$ with at least one double loop.
\end{lem}


\begin{proof}
The fiberwise involution $F$ in the proof of Lemma \ref{lem_small_loop_vanish} has an orientation sign $(-1)^j$ on each
fiber, but in the case here $F^* \omega_{\Gamma}=(-1)^{n+j}\omega_{\Gamma}$ since $F^*$ affects $vol_{S^{j-1}}$ and
$vol_{S^{n-1}}$ as the antipodal map.
Thus
\[
 (\pi_{\Gamma})_* \omega_{\Gamma} = (-1)^{n+2j}(\pi_{\Gamma})_* \omega_{\Gamma}
 = -(\pi_{\Gamma})_* \omega_{\Gamma}.
\]
\end{proof}


Thus we obtain linear maps
\[
 I : \D^{k,l}_g \longrightarrow \Omega^{(n-j-2)k+(g-1)(j-1)+l}_{DR}(\emb{n}{j}),
\]
where $\D^{k,l}_g$ is the subspace of $\D^{k,l}$ spanned by the graphs $\Gamma$ whose first Betti numbers are $g$
after its small loops (not double loops) are removed.
It can be easily seen that $\D^{k,*}_g$ forms a subcomplex of $\D^{k,*}$ for any $g$ (we regard the contraction (4)
in Figure \ref{edge_contraction} as preserving the first Betti number).


Here we restate the last half of Theorem \ref{thm_main2}.


\begin{thm}\label{cochain}
Suppose $n-j\ge 2$ is even and $j \ge 2$.
The integration map $I : \D^{k,*}_g \to \Omega^{(n-j-2)k-(j-1)+*}_{DR}(\emb{n}{j})$ is a cochain map if
(1) both $n>j \ge 2$ are even and $g=0$, or (2) both $n > j \ge 3$ are odd and $g=0,1$.
\end{thm}


Theorem \ref{cochain} is a direct consequence of Theorems \ref{3.1_first_half}, \ref{3.1_second_half}, which are
proved in similar ways as the results in \cite{CCL02, Rossi_thesis, Watanabe07}.
The key step is the generalized Stokes theorem (see \cite{CCL02});
\[
 dI(\Gamma ) = (\pi_{\Gamma})_* (d\omega_{\Gamma})
 +(-1)^{\deg \omega_{\Gamma }+1} (\pi^{\partial}_{\Gamma})_* \omega_{\Gamma} 
 = (-1)^{\deg \omega_{\Gamma }+1} (\pi^{\partial}_{\Gamma})_* \omega_{\Gamma}
\]
where $\pi^{\partial}_{\Gamma}$ is the restriction of $\pi_{\Gamma}$ onto the boundaries of fibers.
The second equality holds since $\omega_{\Gamma}$ is a product of closed forms.
So we need to study the boundaries of fibers to prove Theorem \ref{cochain}.
The proof will be given in \S \ref{sec_vanish}.


Here we state one more result, which concerns the choices of volume forms.



\begin{prop}[\cite{CCL02}]\label{prop_indep_vol}
Suppose $g$, $n$ and $j$ satisfy (1) or (2) in Theorem \ref{cochain} and $n-j>2$.
Let $v_0$ and $v_1$ be two (anti-)symmetric volume forms of $S^{n-1}$ with total integral one, and $I_0$, $I_1$ the
corresponding integration maps.
Then $I_1 (\Gamma ) - I_0 (\Gamma )$ is an exact form for any  graph cocycle $\Gamma = \sum a_i \Gamma_i$.
\end{prop}


\begin{proof}
Choose any $w' \in \Omega^{n-2}_{DR}(S^{n-1})$ such that $v_1 -v_0 =dw'$, and put $w=(w' +(-1)^n \iota^*_{S^{n-1}}w')/2$.
Then $dw=v_1 -v_0$ and $\iota^*_{S^{n-1}}w = (-1)^n w$.
Define
\[
 \tilde{v} := v_0 +d(tw) \in \Omega^{n-1}_{DR}(S^{n-1}\times [0,1]),
\]
where $t$ is the coordinate of $[0,1]$.
Then $\tilde{v}$ is a (anti-)symmetric closed form on $S^{n-1}\times [0,1]$.
Moreover the restriction of $\tilde{v}$ onto $S^{n-1}\times \{ \varepsilon \}$, $\varepsilon =0,1$, is $v_{\varepsilon}$.


Assigning to $\theta$-edges $e$ of $\Gamma_i$ the differential forms
\[
 \tilde{\theta}_e := (\varphi^{\theta}_e \times \id )^* \tilde{v} \in \Omega^{n-1}_{DR}(C_{\Gamma_i} \times [0,1])
\]
instead of $\theta_e$, and integrating their product along the fiber of $\pi_{\Gamma_i}\times \id_{[0,1]}$,
we obtain a differential form of $\emb{n}{j} \times [0,1]$, which we denote by
$\tilde{I}(\Gamma_i )$.
Its restriction onto $\emb{n}{j} \times \{ \varepsilon \}$, $\varepsilon =0,1$, is $I_{\varepsilon}(\Gamma_i )$.


In Lemma \ref{tilde_I_closed} we will prove that $\tilde{I} (\Gamma ) \in \Omega^*_{DR}(\emb{n}{j} \times [0,1])$
is a closed form if $n-j >2$.
This completes the proof; if we denote the first projection by $p: \emb{n}{j} \times [0,1] \to \emb{n}{j}$, then
by the generalized Stokes theorem
\[
 d_{\emb{n}{j}}p_* \tilde{I}(\Gamma )
 = p_* d_{\emb{n}{j} \times [0,1]} \tilde{I}(\Gamma ) \pm p^{\partial}_* \tilde{I}(\Gamma ) 
 = \pm (\tilde{I}(\Gamma )|_{\emb{n}{j}\times \{1 \}} - \tilde{I}(\Gamma )|_{\emb{n}{j}\times \{0 \}})
\]
and thus $I_1 (\Gamma ) - I_0 (\Gamma ) = \pm d p_* \tilde{I}(\Gamma )$.
\end{proof}


In completely similar way we can prove the following.


\begin{prop}\label{prop_indep_vol2}
Suppose $g$, $n$ and $j$ are as in Proposition \ref{prop_indep_vol}.
Then the cohomology class $[I(\Gamma )]$ for a graph cocycle $\Gamma$ does not depend on the choice of $vol_{S^{j-1}}$.
\end{prop}