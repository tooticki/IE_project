\section{Vanishing results}\label{sec_vanish}


In \S\S \ref{subsection_boundary_survey}, ..., \ref{subsection_hidden_infty} we prove Theorem \ref{cochain} assuming
$n-j \ge 2$ is even, by studying the boundary strata of compactified configuration spaces.
Some results here hold even if $n-j\ge 3$ is odd and can be used to prove Theorem \ref{thm_closed}
(see \S \ref{subsec_proof_closed}).
In \S \ref{subsec_indep_vol} we complete the proof of Proposition \ref{prop_indep_vol}.





\subsection{Boundary strata}\label{subsection_boundary_survey}


Let $\Gamma$ be a graph with $s$ i-vertices and $t$ e-vertices.
We denote by $C_{s,t}$ the fiber of $\pi_{\Gamma}$.
The compactified configuration space $C_{s,t}$ is a manifold with corners.
The boundary $\partial C_{s,t}$ consists of configurations where some points in the configuration are allowed to
`collide together.'
Moreover $\partial C_{s,t}$ is stratified via `complexities' of collisions.


But here we do not need the complete description of all strata.
For our purpose only the most `generic' part, the codimension one strata, are needed.
Such strata correspond to `coinstantaneous collisions' of points, and are parametrized by subsets of the set
$V(\Gamma )$ of vertices of $\Gamma$, as we will explain in \S \ref{subsection_boundary_strata}.





\subsection{Codimension one strata}\label{subsection_boundary_strata}


To any subset $A \subset V(\Gamma )$ with $\sharp A \ge 2$, a codimension one stratum $C_A \subset \partial C_{s,t}$
is assigned.
Namely $C_A$ consists of configurations where the points labeled by $A$ `simultaneously collide together.'
More precisely, any point in $C_A$ can be written as a limit point
\begin{equation}\label{first_limit}
 \lim_{\tau \to 0} (x_1 (\tau ),\dots ,x_s (\tau ),y_{s+1} (\tau ),\dots ,y_{s+t} (\tau ))
\end{equation}
such that $(x_1 (\tau ),\dots ,y_{s+t} (\tau )) \in C^o_{s,t}$ for $\tau >0$, and $x_p (\tau )$, $y_q (\tau )$ can be
written as
\[
 x_p (\tau ) =
 \begin{cases}
  x_p \quad (\text{constant}) & \text{if } p \not\in A, \\
  z + \tau v_p & \text{if } p \in A,
 \end{cases}\quad
 y_q (\tau ) =
 \begin{cases}
  y_q \quad (\text{constant}) & \text{if } q \not\in A, \\
  z + \tau w_q & \text{if } q \in A,
 \end{cases}
\]
for some $v_p \in \R^j \setminus \{ 0\}$, $w_q \in \R^n \setminus \{ 0\}$ and $z \in \R^n$.


There are other types of codimension one strata, denoted by $C^{\infty}_A$, parametrized by all the
non-empty subsets $A \subset V(\Gamma )$.
The stratum $C^{\infty}_A \subset \partial C_{s,t}$ consists of configurations where the points labeled by $A$
`escape to infinity.'
More precisely $C^{\infty}_A$ is the set of limit points
\begin{equation}\label{infty_limit}
 \lim_{\tau \to \infty} (x_1 (\tau ),\dots ,x_s (\tau ),y_{s+1} (\tau ),\dots ,y_{s+t} (\tau ))
\end{equation}
where $(x_1 (\tau ),\dots ,y_{s+t} (\tau )) \in C^o_{s,t}$ ($0 <\tau <\infty$) is of the form
\[
 x_p (\tau ) =
 \begin{cases}
  x_p \quad (\text{constant}) & \text{if } p \not\in A, \\
  x_p + \tau v_p              & \text{if } p \in A,
 \end{cases}\quad
 y_q (\tau ) =
 \begin{cases}
  y_q \quad (\text{constant}) & \text{if } q \not\in A, \\
  y_q + \tau w_q              & \text{if } q \in A,
 \end{cases}
\]
for some $v_p \in \R^j \setminus \{ 0\}$, $w_q \in \R^n \setminus \{ 0\}$, $x_p \in \R^j$ and $y_q \in \R^n$.


All the codimension one strata is of the form $C_A$ or $C^{\infty}_A$, hence we have
\[
 \partial C_{s,t}= \left( \bigcup_{A \subset V(\Gamma ), \ \sharp A \ge 2}\overline{C_A} \right) \cup
 \left( \bigcup_{A' \subset V(\Gamma ), \ \sharp A' \ge 1}\overline{C^{\infty}_{A'}} \right) .
\]
We will call $C_A$ ($\sharp A \ge 2$) a {\it stratum of non-infinity type}, and $C^{\infty}_A$
($A \ne \emptyset$) a {\it stratum at infinity}.


Define the subsets $\Sigma_A$ and $\Sigma^{\infty}_A$ of $C_{\Gamma}$ by
\[
 \Sigma_A :=\bigcup_{f \in \emb{n}{j}}C_A (f) , \quad \Sigma^{\infty}_A :=\bigcup_{f \in \emb{n}{j}}C^{\infty}_A (f)
\]
where $C_A (f)$ and $C^{\infty}_A (f)$ are codimension one strata of $\pi_{\Gamma}^{-1}(f)$ described as above.
Then $\Sigma_A$ and $\Sigma^{\infty}_A$ fiber over $\emb{n}{j}$.


Notice that it is enough to describe $\Int \Sigma^{(\infty )}_A$ for the proof of Theorem \ref{cochain}.
In \S \ref{subsection_strata_1} and \S \ref{subsection_strata_infinity} we will describe these strata explicitly,
following \cite{BottTaubes94, CCL02, Rossi_thesis, Watanabe07}.





\subsection{Explicit description of non-infinity type strata}\label{subsection_strata_1}


Let $A \subset V(\Gamma )$ be a subset with $\sharp A \ge 2$
(recall that $V(\Gamma )$ denotes the set of vertices of a graph $\Gamma$).
Here we study the strata $\Sigma_A$ of non-infinity type.
Denote by $E(\Gamma )$ the set of edges of $\Gamma$.



\begin{defn}\label{subgraph}
The {\em subgraph} $\Gamma_A$ of $\Gamma$ associated with $A$ is a (possibly non-admissible) graph with
$V(\Gamma_A )=A$ and $E(\Gamma_A ) = \{ pq \in E(\Gamma )\, |\, p,q \in A,\ p \ne q \}$ (hence small loops are ignored).
If $A=\{ p_1 <\dots <p_k \}$, then the vertex of $\Gamma_A$ which was labeled by $p_a$ in $\Gamma$ is re-labeled
by $a$.
The labels of edges are suitably decreased.


The {\it quotient graph} $\Gamma / \Gamma_A$ is a graph obtained by `collapsing $\Gamma_A$ to a point $v_A$.'
More precisely,
\begin{align*}
 V(\Gamma / \Gamma_A ) &:= (V(\Gamma ) \setminus A) \sqcup \{ v_A \} ,\\
 E(\Gamma / \Gamma_A ) &:= \{ pq \in E(\Gamma ) \, |\, p,q \not \in A \} \sqcup
 \{ pv_A \,|\, p \not\in A,\ pq \in E(\Gamma ) \text{ for }\exists q \in A \} .
\end{align*}
The vertex $v_A$ is internal if there is an i-vertex in $A$, and is external otherwise.
We label $v_A$ by $\min \{ p\, ;\, p\in A\}$, and the labels of other vertices and edges are suitably decreased
(see Figure \ref{fig_subgraph_noninfty} for an example).
\end{defn}


\begin{figure}[htb]%subgraph_noninfty
\[
 \begin{xy}
 (0,0)*{\circ}="O",(0,8)*{\bullet}="A",(-6.928,-4)*{\bullet}="B",(6.928,-4)*{\bullet}="C",
 {\ar@{.>}"A";"O"_<{1}^>{4}},{\ar@{.>}"B";"O"^<{2}},{\ar@{.>}"C";"O"_<{3}},(-12,2)*{\Gamma =},
 (35,0)*{\circ}="o",(35,8)*{\bullet}="a", {\ar@{.>}"a";"o"_<{1}^>{2}},(23,2)*{\Gamma_A =},
 (70,0)*{\bullet}="p",(63.072,-4)*{\bullet}="q",(76.928,-4)*{\bullet}="r",
 {\ar@{.>}"q";"p"^<{2}_>{1}}, {\ar@{.>}"r";"p"_<{3}},(53,2)*{\Gamma /\Gamma_A =},
 \end{xy}
\]
\caption{Examples of $\Gamma_A$ and $\Gamma /\Gamma_A$ for $A=\{ 1,4 \}$ ($n$ odd)}\label{fig_subgraph_noninfty}
\end{figure}


There is a projection
\[
 p_A : \Int \Sigma_A \longrightarrow C^o_{\Gamma / \Gamma_A}
\]
which maps the limit point (\ref{first_limit}) to $(\dots ,x_p , \dots ,z,\dots ,y_q ,\dots )$, $p,q \not\in A$.
The point $z$ corresponds to the vertex $v_A$, hence $z \in \R^n$ if $A$ contains no i-vertex, and $z \in \R^j$ if
there is an i-vertex in $A$.


The fiber of $p_A$ is thought of as the space of `infinitesimal configurations' at the colliding point.
We will define a fibration $\rho_A :\hat{B}_A \to B_A$ which describes such infinitesimal configurations.


\begin{defn}\label{def_I_j}
Define $\I_j (\R^n )$ as the set of all $j$-frames in $\R^n$.
In other words
\[
 \I_j (\R^n ) := \{ \text{linear injective maps } \R^j \hookrightarrow \R^n \} .
\]
We give $\I_j (\R^n )$ a natural structure of an open submanifold of $(\R^n \setminus \{ 0\})^j$.
\end{defn}


Let $a$ and $b$ be the numbers of i- and e-vertices in $A$.
Define a manifold $B_A$ by
\[
 B_A :=
 \begin{cases}
  \{ * \}     & a=0, \\
  \I_j (\R^n ) & a>0.
 \end{cases}
\]
When $a=0$, we define $\hat{B}_A$ to be $C^o_{\sharp A}(\R^n )$ modulo scaling and translation;
\[
 \hat{B}_A := C^o_b (\R^n ) / \R^n \rtimes \R_{>0} ,
\]
where the action of $\R^n \rtimes \R_{>0}$ is defined by
\[
 (y_p )_{p \in A} \longmapsto (\alpha (y_p -\beta ))_{p \in A},\quad \forall \alpha >0,\ \forall \beta \in \R^n .
\]
The map $\rho_A : \hat{B}_A \to B_A =\{ *\}$ is defined as the canonical one.


When $a>0$, $\hat{B}_A$ is $\I_j (\R^n ) \times C^o_{a,b}$ modulo scaling and translation in the directions of $j$-planes;
\[
  \hat{B}_A := (\I_j (\R^n ) \times C^o_{a,b}) / \R^j \rtimes \R_{>0}, \\
\]
where the action of $\R^j \rtimes \R_{>0}$ is defined by
\[
 (\iota ;(x_p ,y_q )_{p,q \in A}) \longmapsto (\iota ;(\alpha (x_p -\beta ), \alpha (y_q -\iota (\beta )))_{p,q \in A}),
 \quad \forall \alpha >0,\ \forall \beta \in \R^j .
\]
The map $\rho_A : \hat{B}_A \to B_A =\I_j (\R^n )$ is defined as the natural projection.


Finally define $D_A : C^o_{\Gamma / \Gamma_A} \to B_A$ to be the canonical map if $a=0$, and
\[
 D_A (f;\dots ,x_p ,\dots ,z,\dots ,y_q ,\dots )_{p,q \not\in A}) := (df_z : T_z \R^j \longrightarrow T_{f(z)}\R^n )
 \in \I_j (\R^n )
\]
if $a>0$ (each tangent spaces are naturally identified with $\R^j$ and $\R^n$).


\begin{prop}[\cite{BottTaubes94, CCL02, Rossi_thesis, Watanabe07}]\label{Interior_strata}
The fibration $p_A : \Int \Sigma_A \to C^o_{\Gamma / \Gamma_A}$ is the pull-back of $\rho_A$ via $D_A$;
\[
 \xymatrix{
  \Int \Sigma_A \ar[r]^-{\hat{D}_A} \ar[d]_-{p_A} & \hat{B}_A \ar[d]^-{\rho_A} \\
  C^o_{\Gamma / \Gamma_A} \ar[r]^-{D_A} & B_A
 }
\]
In particular $\Int \Sigma_A \approx C^o_{\Gamma /\Gamma_A} \times \hat{B}_A$ if $A$ has no i-vertex.
\end{prop}



Notice that the differential form $\omega_{\Gamma / \Gamma_A} \in \Omega^*_{DR}(C^o_{\Gamma / \Gamma_A} )$ can be
defined similarly to \S \ref{subsection_form}, by using the direction maps $\varphi$ corresponding to the
edges of $\Gamma / \Gamma_A$.
Similarly, the maps $\varphi_e$ for any edges of $\Gamma_A$ are well defined on $\hat{B}_A$;
if $e=\overrightarrow{pq}$ is an $\eta$-edge,
\[
 \hat{\varphi}^{\eta}_e (\iota ; (x_r ,y_s )_{r,s \in A}) := \frac{x_q -x_p}{\abs{x_q -x_p}} \in S^{j-1},
\]
and if $e$ is a $\theta$-edge,
\[
 \hat{\varphi}^{\theta}_e (\iota ; (x_r ,y_s )_{r,s \in A}) := \frac{z_q -z_p}{\abs{z_q -z_p}} \in S^{n-1},
\]
where $z_p =x_p$ or $\iota (y_p )$ according to whether $p$ is internal or external.
Hence
\[
 \hat{\omega}_{\Gamma_A}:=\bigwedge_{e \in E(\Gamma_A )}\varphi^*_e vol \in \Omega^*_{DR} (\hat{B}_A)
\]
can be defined.
Then we have
\[
 \omega_{\Gamma}|_{\Int \Sigma_A} =\pm (p^*_A \omega_{\Gamma / \Gamma_A}) \wedge (\hat{D}^*_A \hat{\omega}_{\Gamma_A})
\]
and hence
\begin{equation}\label{p_A}
 (p_A )_* \omega_{\Gamma}|_{\Int \Sigma_A}=\pm \omega_{\Gamma / \Gamma_A} \wedge D^*_A (\rho_A )_* \hat{\omega}_{\Gamma_A}
\end{equation}
by the compatibility of fiber-integrations with pullbacks (for signs see \S\S \ref{subsec_orientation},
\ref{subsec_principal_cancel}).


Denote $\pi^{\partial_A}_{\Gamma} :=\pi_{\Gamma}|_{\Int \Sigma_A} : \Int \Sigma_A \to \emb{n}{j}$.
Notice that $\pi^{\partial_A}_{\Gamma}=\pi_{\Gamma / \Gamma_A} \circ p_A$.
We will often use the following criterion to show the vanishing of an integration along $\Int \Sigma_A$.


\begin{lem}[\cite{CCL02}]\label{vanish_1}
Let $a$ and $b$ be the numbers of i- and e-vertices in $A$ respectively.
Then the fiber integration $(\pi^{\partial_A}_{\Gamma})_* \omega_{\Gamma}|_{\Int \Sigma_A}$ vanishes unless
\begin{alignat*}{2}
 &\deg \hat{\omega}_{\Gamma_A} = nb-(n+1) &\quad &\text{if } a=0, \\
 &0 \le \deg \hat{\omega}_{\Gamma_A} - (ja+nb-(j+1)) \le nj &\quad &\text{if } a>0.
\end{alignat*}
\end{lem}


\begin{proof}
The form $(\rho_A )_* \hat{\omega}_{\Gamma_A}$ vanishes unless
$0 \le \deg (\rho_A )_* \hat{\omega}_{\Gamma_A} \le \dim B_A$.
We have
\[
 \deg (\rho_A )_* \hat{\omega}_{\Gamma_A} = \deg \hat{\omega}_{\Gamma_A}
 - \dim (\text{fiber of }\rho_A : \hat{B}_A \to B_A )
\]
and, by definition of the fibration $\rho_A : \hat{B}_A \to B_A$,
\[
 (\dim B_A ,\, \dim (\text{fiber of }\rho_A )) =
 \begin{cases}
  (0,\, nb -(n+1))      & \text{if } a=0, \\
  (nj,\, ja +nb -(j+1)) & \text{if } a>0.
 \end{cases}
\]
Hence $(\rho_A )_* \hat{\omega}_{\Gamma_A}$ vanishes unless the conditions of the Lemma are satisfied.
Then the formulas $(\pi^{\partial_A}_{\Gamma})_* = \pm (\pi_{\Gamma / \Gamma_A})_* \circ (p_A )_*$ and (\ref{p_A})
complete the proof.
\end{proof}





\subsection{Explicit description of strata at infinity}\label{subsection_strata_infinity}


Let $\Gamma$ be a graph and $A$ a non-empty subset of $V(\Gamma )$.
Below we describe $\Int \Sigma^{\infty}_A$ following \cite{BottTaubes94, CCL02, Rossi_thesis, Watanabe07}.


\begin{defn}
Define the {\it complementary graph} $\Gamma^c_A$ by letting
\[
 V(\Gamma^c_A ) := V(\Gamma ) \setminus A, \quad
 E(\Gamma^c_A ) := \{ \overrightarrow{pq} \in E(\Gamma ) \, | \, p,q \not\in A \}
\]
($\Gamma^c_A :=\emptyset$ if $A=V(\Gamma )$).
A {\it graph with infinity} is a graph with a specified vertex $v^{\infty}$ (called {\it vertex at infinity}),
which is not regarded as being internal nor external.


Define $\Gamma^{\infty}_A$, a graph with infinity, by `shrinking $\Gamma^c_A$ to a point.'
Namely $\Gamma^{\infty}_A$ is defined similarly as a quotient graph $\Gamma / \Gamma^c_A$ (see Definition
\ref{subgraph}), but its vertex $v_A$ is replaced by $v^{\infty}$, a vertex at infinity (see Figure
\ref{fig_subgraph_infty} for an example).
By definition $\Gamma^{\infty}_A :=\Gamma$ if $A=V(\Gamma )$.
\end{defn}
\begin{figure}[htb]%subgraph_infty
\[
 \begin{xy}
 (0,0)*{\circ}="O",(0,8)*{\bullet}="A",(-6.928,-4)*{\bullet}="B",(6.928,-4)*{\bullet}="C",
 {\ar@{.}"A";"O"_<{1}^(.45){(1)}},{\ar@{.}"B";"O"^<{2}^(.6){(2)}_>{4}},{\ar@{.}"C";"O"_<{3}_(.6){(3)}},(-12,2)*{\Gamma =},
 (35,8)*{\bullet},(33,8)*{\sb 1},(28.072,-4)*{\bullet},(28.072,-2)*{\sb 2},(21,2)*{\Gamma^c_A =},
 (70,2)*{\circ}="o",(64,8)*{*}="a",(76,-4)*{\bullet}="b",
 {\ar@{.>}@/^/"a";"o"^(.5){(1)}^>{2}}, {\ar@{.>}@/_/"a";"o"_<{v^{\infty}}_(.5){(2)}},{\ar@{.>}"b";"o"_<{1}^(.5){(3)}},
 (51,2)*{\Gamma /\Gamma^c_A =},
 \end{xy}
\]
\caption{Examples of $\Gamma^c_A$ and $\Gamma^{\infty}_A$ for $A= \{ 3,4\}$ ($n$ even)}\label{fig_subgraph_infty}
\end{figure}



$\Int \Sigma^{\infty}_A$ fibers over $C^o_{\Gamma^c_A}$;
\[
 p^{\infty}_A : \Int \Sigma^{\infty}_A \longrightarrow C^o_{\Gamma^c_A}
\]
which maps the limit point (\ref{infty_limit}) in \S \ref{subsection_boundary_strata} to $(x_p ,y_q )_{p,q \not\in A}$.
In other words $p^{\infty}_A$ forgets the points escaping to infinity.


As in \S \ref{subsection_strata_1}, we will define a space $\hat{B}^{\infty}_A$ which describes infinitesimal
configurations around infinity.
The space $\hat{B}^{\infty}_A$ is a subquotient of $C^o_{a,b}$ ($a,b$ are the numbers of i- and e-vertices in $A$)
modulo scaling;
\[
 \hat{B}^{\infty}_A :=
 \{ (x_p ,y_q )_{p,q\in A} \in C^o_{a+b}(\R^n \setminus 0) \, |\, x_p \in \R^j \times \{ 0\}^{n-j}\}/\R_{>0} .
\]
The origin $0 \in \R^n$ corresponds to $v^{\infty}$, which we use to fix the coordinates.
So in this case translation is not needed (compare it with the definition of $\hat{B}_A$).



\begin{prop}[\cite{BottTaubes94, CCL02, Rossi_thesis, Watanabe07}]\label{face_at_infinity}
$\Int \Sigma^{\infty}_A$ is homeomorphic to $C^o_{\Gamma^c_A} \times \hat{B}^{\infty}_A$.
\end{prop}


The form $\omega_{\Gamma^{\infty}_A} \in \Omega^*_{DR}(\hat{B}^{\infty}_A)$ can be defined as in \S \ref{subsection_form}
since the direction maps $\hat{\varphi}$ are invariant under scaling.
The vertex $v^{\infty}$ corresponds to $0 \in \R^n$.
More precisely, for each $e \in E(\Gamma^{\infty}_A)$, define the maps $\varphi_e : \hat{B}^{\infty}_A \to S^{N-1}$
($N=j$ or $n$) by
\[
 \varphi_e (x_p ,y_q )_{p,q \in A} :=
 \begin{cases}
  (z_q -z_p ) / \abs{z_q -z_p} & e = \overrightarrow{pq},\ p,q \in A, \\
  -z_p / \abs{z_p}             & e = \overrightarrow{pv^{\infty}} ,\ p \in A,
 \end{cases}
\]
where $z_p$ denotes $x_p$ or $y_p$ according to whether $p$ is an i-vertex or not.
Then
\[
 \omega_{\Gamma^{\infty}_A} := \bigwedge_{e \in E(\Gamma^{\infty}_A )} \omega_e .
\]
Under the identification in Proposition \ref{face_at_infinity},
\[
 \omega_{\Gamma}|_{\Int \Sigma^{\infty}_A}=\pm pr^*_1 \omega_{\Gamma^c_A}\wedge pr^*_2 \omega_{\Gamma^{\infty}_A}.
\]
Define $\pi^{\partial^{\infty}_A}_{\Gamma} : \Int \Sigma^{\infty}_A \to \emb{n}{j}$ as the restriction of $\pi_{\Gamma}$
onto the stratum $\Int \Sigma^{\infty}_A$ of fibers.
Then we have the following (which should be compared with Lemma \ref{vanish_1}).


\begin{lem}[\cite{CCL02}]\label{vanish_infty}
Let $a$ and $b$ be the numbers of i- and e-vertices of $\Gamma^{\infty}_A$ respectively (other than $v^{\infty}$).
Then the integration $(\pi^{\partial^{\infty}_A}_{\Gamma})_* \omega_{\Gamma}|_{\Int \Sigma^{\infty}_A}$ vanishes unless
$\deg \omega_{\Gamma^{\infty}_A} = \dim \hat{B}^{\infty}_A$, or equivalently, unless
\[
 \deg \omega_{\Gamma^{\infty}_A} = ja+nb-1.
\]
\end{lem}





\subsection{Orientations of boundary strata}\label{subsec_orientation}


Let $\Gamma$ be a graph and $A$ a non-empty subset of $V(\Gamma )$.

\begin{defn}
The boundary face $\Sigma_A$ (or the subgraph $\Gamma_A$) is said to be {\it principal} if $A$ consists of exactly
two vertices.
Similarly the boundary face $\Sigma^{\infty}_A$ is said to be {\it principal} if $A$ consists of exactly
one vertex.


All the other boundary strata are said to be {\it hidden}.
\end{defn}


Here we study the induced orientations of the non-infinity type principal strata $C_A$ from that of
$\text{fib}(\pi_{\Gamma}) \approx C_{s,t}$ (see \S \ref{subsection_boundary_strata}).
We are not interested in the orientations of strata at infinity and hidden strata, since the integrations along these
strata will be proved to vanish (see below).

Let $s$ and $t$ be the numbers of i- and e-vertices of $\Gamma$ respectively.
The fiber $C_{s,t}$ is equipped with the natural orientation as the subspace of $(\R^j )^s \times (\R^n )^t$.


Let $A \subset V(\Gamma )$ be a subset with $\sharp A=2$ (thus $\Sigma_A$ is principal), and $a$ and $b$ (with $a+b=2$)
the numbers of i- and e-vertices in $A$.


\noindent
\underline{\bf Case 1}: $a=2$, $b=0$.


Let $A=\{ x_p , x_q \}$ ($p<q \le s$).
In this case $\hat{B}_A = \I_j (\R^n ) \times S^{j-1}$ and hence
$\Int \Sigma_A \approx C_{\Gamma / \Gamma_A} \times S^{j-1}$ (see Proposition \ref{Interior_strata}). 
A neighborhood of $\Int \Sigma_A$ in $C_{\Gamma}$ is identified with $[0,1) \times C_{\Gamma / \Gamma_A} \times S^{j-1}$
by the homeomorphism onto the image
\begin{equation}\label{eq_coord1}
\begin{split}
 &(\varepsilon ;f;x_1 , \dots ,x_{q-1}, x_{q+1}, \dots ,x_s ;y_{s+1},\dots ,y_{s+t};v) \\
 &\hskip100pt \longmapsto (f; x_1 ,\dots ,x_{q-1}, x_p +\varepsilon v ,x_{q+1}, \dots ;y_{s+1},\dots )
\end{split}
\end{equation}
where $f\in \emb{n}{j}$ and $v \in S^{j-1}$.
It is not hard to see that the local coordinate of $C_{\Gamma}$ given by (\ref{eq_coord1}) has the orientation sign
$(-1)^{j(q-1)+(j-1)nt}$.
Putting $\varepsilon =0$ in the local coordinate (\ref{eq_coord1}), we obtain the natural orientation of
$C_{\Gamma / \Gamma_A} \times S^{j-1}$.
Thus the induced orientation of $C_{\Gamma /\Gamma_A} \times S^{j-1}$ as a boundary face of $C_{\Gamma}$ has the sign
$(-1)^{j(q-1)+(j-1)nt+1}$.


\noindent
\underline{\bf Case 2}: $a=b=1$ or $a=0$, $b=2$.


Let $A=\{ x_p , y_q \}$ ($p \le s < q$ or $s<p<q$).
In this case $\hat{B}_A =\I_j (\R^n ) \times S^{n-1}$ and hence
$\Int \Sigma_A \approx C_{\Gamma / \Gamma_A} \times S^{n-1}$ (see Proposition \ref{Interior_strata}). 
A neighborhood of $\Int \Sigma_A$ in $C_{\Gamma}$ is identified with $[0,1) \times C_{\Gamma / \Gamma_A} \times S^{n-1}$
by the homeomorphism onto the image
\begin{equation}\label{eq_coord2}
\begin{split}
 &(\varepsilon ;f;x_1 , \dots ,x_s ;y_{s+1},\dots y_{q-1}, y_{q+1}, \dots ,y_{s+t};w) \\
 &\hskip100pt \longmapsto (f; x_1 \dots ;y_{s+1},\dots ,y_{q-1}, z_p +\varepsilon w ,y_{q+1},\dots )
\end{split}
\end{equation}
where $w \in S^{n-1}$ and $z_p =f(x_p )$ or $y_p$ according to whether $p$ is internal or not.
The local coordinate of $C_{\Gamma}$ given by (\ref{eq_coord2}) has the orientation sign
$(-1)^{n(s+q-1)+js}$.
Putting $\varepsilon =0$ in the local coordinate (\ref{eq_coord2}), we obtain the natural orientation of
$C_{\Gamma / \Gamma_A} \times S^{n-1}$.
Thus the induced orientation of $C_{\Gamma /\Gamma_A}\times S^{n-1}$ as a boundary face of $C_{\Gamma}$ has the sign
$(-1)^{n(s+q-1)+js+1}$.





\subsection{Principal faces}\label{subsec_principal_cancel}
In this subsection we compute the fiber integration along the principal faces.


\begin{thm}[\cite{CCL02, Rossi_thesis, Watanabe07}]\label{principal}
The integration of $\omega_{\Gamma}$ along the principal face $\Sigma_A$ of non-infinity type (thus $\sharp A =2$)
vanishes unless the two vertices are joined by an edge in $\Gamma$.
\end{thm}


\begin{proof}
Let $a$ and $b$ be the numbers of i- and e-vertices in $A$ respectively ($a+b=2$ since $A$ is principal).
If two vertices in $A$ are not connected by an edge, then we have
$\hat{\omega}_{\Gamma_A}=1 \in \Omega^0_{DR}(\hat{B}_A)$.
Thus, if $A$ has no i-vertex ($a=0$, $b=2$), then the first equality of Lemma \ref{vanish_1} does not hold;
\[
 \deg \hat{\omega}_{\Gamma_A} = 0 \ne 0\cdot j + 2n - (n+1) =n-1.
\]
If $A$ has an i-vertex ($(a,b)=(1,1)$ or $(2,0)$), then the second inequality of Lemma \ref{vanish_1} does not
hold since
\[
 \deg \hat{\omega}_{\Gamma_A} - (ja+nb-(j+1)) =
 \begin{cases}
 -(j-1) & \text{if }(a,b)=(2,0) \\
 -(n-1) & \text{if }(a,b)=(1,1).
 \end{cases}
\]
\end{proof}


\begin{thm}[\cite{CCL02, Rossi_thesis, Watanabe07}]\label{infty_principal}
Let $\Gamma$ be an admissible graph.
Then the integration of $\omega_{\Gamma}$ along the principal face $\Sigma^{\infty}_A$ (thus $\sharp A =1$) always
vanishes.
\end{thm}


\begin{proof}
Let $p$ be the only vertex in $A$.
The graph $\Gamma^{\infty}_A =\Gamma / \Gamma^c_A$ has two vertices; one is $p$ and the other is $v^{\infty}$.
$\sharp E(\Gamma^{\infty}_A )$ is equal to the valency of $p$ in $\Gamma$.


By Lemma \ref{vanish_infty}, $(\pi^{\partial^{\infty}_A}_{\Gamma})_* \omega_{\Gamma}|_{\Int \Sigma^{\infty}_A}$
vanishes unless
\[
 \deg \omega_{\Gamma^{\infty}_A}=
 \begin{cases}
  j-1 & \text{if } p \text{ is internal}, \\
  n-1 & \text{if } p \text{ is external},
 \end{cases}
\]
hence $(\pi^{\partial^{\infty}_A}_{\Gamma})_* \omega_{\Gamma}|_{\Int \Sigma^{\infty}_A}$ does not vanish only if,
in $\Gamma$,
\begin{itemize}
\item $p$ is a uni-valent i-vertex with exactly one adjacent $\eta$-edge, or
\item $p$ is a uni-valent e-vertex with exactly one adjacent $\theta$-edge.
\end{itemize}
But neither case occurs since $\Gamma$ is admissible (see Definition \ref{def_admissible}).
\end{proof}


\begin{thm}[\cite{CCL02, Rossi_thesis, Watanabe07}]\label{3.1_first_half}
The sum of integrations of $\omega_{\Gamma}$ along all the principal faces $\Sigma_A$ of non-infinity type is equal to
$I(\delta \Gamma )$.
\end{thm}


\begin{proof}
By the above Theorems \ref{principal} and \ref{infty_principal}, we only need to consider the principal faces
$\Sigma_A$ such that the two vertices of $\Gamma_A$ are joined by an edge $e$.
We will show
\[
 (\pi^{\partial_A}_{\Gamma})_* \omega_{\Gamma}|_{\Int \Sigma_A} = (-1)^{\tau (e)}(\pi_{\Gamma /e})_* \omega_{\Gamma /e}
\]
for any principal strata $\Sigma_A$, where $e\in E(\Gamma )$ is the only edge of the subgraph $\Gamma_A$ and $\tau (e)$
is the sign given in Proposition \ref{def_signs}.
Then we will obtain
\[
 \sum_{\genfrac{}{}{0pt}{}{A \subset V(\Gamma )}{\sharp A=2}}
 (\pi^{\partial_A}_{\Gamma})_* \omega_{\Gamma}|_{\Int \Sigma_A}
 = \sum_{e \in E(\Gamma ) \setminus \{ \text{loops}\}} (-1)^{\tau (e)}(\pi_{\Gamma /e})_* \omega_{\Gamma /e}
 =I(\delta \Gamma ).
\]
First we consider the case when $n$ and $j$ are odd.
We divide the proof into four cases.


\noindent
\underline{\bf Case (a)}.
$A=\{ p,q \}$ consists of two e-vertices (so we can assume $s < p < q$ and the edge $e = \overrightarrow{pq}$
is a $\theta$-edge).


In this case $\hat{B}_A = S^{n-1}$ and $B_A =\{ * \}$, so $\Int \Sigma_A \approx C^o_{\Gamma /e} \times S^{n-1}$.
The induced orientation of $C^o_{\Gamma /e} \times S^{n-1}$ from $C_{\Gamma}$ is $(-1)^{n(s+q-1)+js+1}$ by Case 2 in
\S \ref{subsec_orientation}, and is equal to $(-1)^q$ since $n$ and $j$ are odd.
This sign is $(-1)^{\tau (e)}$ (see Proposition \ref{def_signs}).


Under the identification $\Int \Sigma_A \approx C^o_{\Gamma /e} \times S^{n-1}$, the map
$\varphi_e : C^o_{\Gamma} \to S^{n-1}$ restricts to the projection $pr_2 : C^o_{\Gamma /e} \times S^{n-1} \to S^{n-1}$.
Hence via the diffeomorphism $\Int \Sigma_A \approx S^{n-1} \times C^o_{\Gamma /e}$, the form
$\omega_{\Gamma_A}|_{\Int \Sigma_A}$ corresponds to
\[
  (-1)^{\tau (e)} p^*_A \omega_{\Gamma /e} \wedge pr^*_2 vol_{S^{n-1}}
 \in \Omega^*_{DR}(S^{n-1} \times C^o_{\Gamma /e})
\]
and hence we have
\begin{align*}
 (\pi^{\partial_A}_{\Gamma})_* \omega_{\Gamma}|_{\Int \Sigma_A}
 &= (-1)^{\tau (e)} (\pi_{\Gamma / e})_* \circ (p_A )_* (p^*_A \omega_{\Gamma / e} \wedge pr^*_2 vol_{S^{n-1}}) \\
 &= (-1)^{\tau (e)} (\pi_{\Gamma / e})_* \left( \int_{S^{n-1}} vol_{S^{n-1}} \right) \omega_{\Gamma / e} \\
 &= (-1)^{\tau (e)} (\pi_{\Gamma / e})_* \omega_{\Gamma / e}.
\end{align*}


\noindent
\underline{\bf Case (b)}.
$A= \{ p,q\}$ contains both an e- and an i-vertex (thus we can assume $p \le s <q$, and
$e=\overrightarrow{pq}$ is a $\theta$-edge).


In this case $\Int \Sigma_A \approx C^o_{\Gamma / e} \times S^{n-1}$.
Similarly as in Case (a) above, the induced orientation of $C^o_{\Gamma / e} \times S^{n-1}$ from $C_{\Gamma}$
is $(-1)^q =(-1)^{\tau (e)}$ and $\varphi_e : C^o_{\Gamma} \to S^{n-1}$ restricts to the projection
$pr_2 : C^o_{\Gamma /e} \times S^{n-1} \to S^{n-1}$.
Thus, as in the Case (a),
\[
 (\pi_{\Gamma})_* \omega_{\Gamma}|_{\Int \Sigma_A}  = (-1)^{\tau (e)} (\pi_{\Gamma /e})_* \omega_{\Gamma /e}.
\]


\noindent
\underline{\bf Case (c)}.
Both two points $p,q$ of $A$ are internal and $e=\overrightarrow{pq}$ is an $\eta$-edge.


In this case $\Int \Sigma_A \approx C^o_{\Gamma /e} \times S^{j-1}$.
Proof is the same as the above cases, since by Case 1 in \S \ref{subsec_orientation}, the induced orientation of
$C^o_{\Gamma / e} \times S^{j-1}$ from $C_{\Gamma}$ is $(-1)^q =(-1)^{\tau (e)}$.



\noindent
\underline{\bf Case (d)}.
$A=\{ p,q\}$ consists of two i-vertices (thus we can assume $p<q\le s$) and the edge $e=\overrightarrow{pq}$ is a
$\theta$-edge.


In this case $\Int \Sigma_A \approx C^o_{\tilde{\Gamma}} \times S^{j-1}$, where $\tilde{\Gamma}$ is $\Gamma /e$
with its small loop $e$ removed.
The right hand side is nothing but the space $C^o_{\Gamma /e}$ ($\Gamma /e$ is a graph with small loop; see \S
\ref{subsection_integral}), up to the orientation.
Again by Case 1 in \S \ref{subsec_orientation}, the induced orientation of $\Int \Sigma_A \subset C_{\Gamma}$ has the
sign $(-1)^{\tau (e)}$.
Hence
\[
 (\pi_{\Gamma})_* \omega_{\Gamma}|_{\Int \Sigma_A}  = (-1)^{\tau (e)} (\pi_{\Gamma /e})_* \omega_{\Gamma /e}
\]
as desired.


The proof of the case when $n$ and $j$ are even is similar.
In Case (a), the induced orientation of $\Sigma_A \subset C_{\Gamma}$ has the sign $-1$.
To integrate the form $\theta_e$ first, we must put $\theta_e$ at the top of $\omega_{\Gamma}$.
Such a re-ordering yields the sign $(-1)^{i-1}$ if $e$ is the $i$-th edge.
Hence
\[
 (\pi^{\partial_A}_{\Gamma})_* \omega_{\Gamma}|_{\Int \Sigma_A} = (-1)^i (\pi_{\Gamma / e})_* \omega_{\Gamma / e}
 = (-1)^{\tau (e)}(\pi_{\Gamma / e})_* \omega_{\Gamma / e}.
\]
The remaining three cases are proved in similar ways.
In Case (d), we have to put the $S^{j-1}$-factor at the end of the $(S^{j-1})^u$-part, so the sign $(-1)^u$ appears.


When $n$ is even, we have to consider one more case;


\noindent
\underline{\bf Case (e)}.
$A=\{ p,q\}$ consists of two internal vertices (thus we can assume $p<q\le s$) which are joined by an $\eta$-edge
$\overrightarrow{pq}_{\eta}$ and a $\theta$-edge $\overrightarrow{pq}_{\theta}$.


In this case, the induced orientation of $\Sigma_A \subset C_{\Gamma}$ has the sign $-1$ as in Case (a).
But we need to put $S^{j-1}$ at the end of $S^{j-1}$-factor, and not to move the forms $\theta$ and $\eta$.
Hence
\[
 (\pi^{\partial_A}_{\Gamma})_* \omega_{\Gamma}|_{\Int \Sigma_A} = (-1)^{u+1}(\pi_{\Gamma / e})_* \omega_{\Gamma / e}
 = (-1)^{\tau (e)}(\pi_{\Gamma / e})_* \omega_{\Gamma / e}.
\]
\end{proof}



By the above Theorem \ref{3.1_first_half}, the proof of Theorem \ref{cochain} is reduced to showing that
hidden faces do not contribute to the fiber integration.
The following, whose proof will be given in \S \ref{subsection_hidden_first} and \S \ref{subsection_hidden_infty},
will complete the proof of Theorem \ref{cochain}.


\begin{thm}\label{3.1_second_half}
Let $\Gamma$ be an admissible graph.
Then all the integrations of $\omega_{\Gamma}$ along hidden boundary faces of the fiber of
$\pi_{\Gamma} : C_{\Gamma} \to \emb{n}{j}$ vanish if
(1) $n-j \ge 2$ is even and $\Gamma$ is a tree, or
(2) both $n>j\ge 3$ are odd and $\Gamma$ has at most one loop component.
\end{thm}





\subsection{Hidden faces; the non-infinity type}\label{subsection_hidden_first}


Here we show that all the hidden faces $\Sigma_A$ of non-infinity type, $\sharp A \ge 3$, do not contribute
to the fiber integration.


\begin{lem}[\cite{Rossi_thesis, Watanabe07}]\label{disconnected}
Suppose $A \subset V(\Gamma )$ is such that the subgraph $\Gamma_A$ is not connected.
Then $(\pi^{\partial_A}_{\Gamma})_* \omega_{\Gamma}|_{\Int \Sigma_A}$ vanishes.
\end{lem}


\begin{proof}
If $\Sigma_A$ is principal (so $\sharp A =2$), then the claim of this Lemma is exactly that of Theorem \ref{principal}.
So we assume $\sharp A \ge 3$.

Suppose $\Gamma_A = \Gamma_{A_1} \sqcup \Gamma_{A_2}$ for non-empty subsets $A_1 ,A_2 \subset A$ (thus we can assume
$\sharp A_1 \ge 2$).
Let $a_i$ and $b_i$ be the numbers of i- and e-vertices of $A_i$, $i=1,2$.

We define the space $\tilde{B}_A$, which contains $\hat{B}_A$ as an open subset, by
\[
 \tilde{B}_A := (C^0_{a_1 ,b_1} \times C^o_{a_2 ,b_2}) / \sim
\]
here $\sim$ is defined by using the translation and the scaling.
In other words, `a point in $A_1$ may collide with a point in $A_2$.'
Since there are no edges joining a point in $A_1$ to that of $A_2$, the direction maps $\varphi$ corresponding to
the edges of $\Gamma_A$ are well-defined on $\tilde{B}_A$, and so is the associated differential form (we denote it by
$\tilde{\omega}_{\Gamma_A}$).
The restriction of $\tilde{\omega}_{\Gamma_A}$ onto $\hat{B}_A$ is $\hat{\omega}_{\Gamma_A}$.


Consider a free action of $\R^N$ on $\tilde{B}_A$ defined by the translations of points in $A_1$ (points in $A_2$ are
fixed).
Here $N=j$ or $n$ according to whether $A_1$ contains an i-vertex or not.
Let $p:\tilde{B}_A \to \tilde{B}_A / \R^N$ be the quotient map.
Since the direction maps $\tilde{\varphi}:\tilde{B}_A \to S^{j-1}$ or $S^{n-1}$ factor through $p$, there exists a
form $\omega'_{\Gamma_A} \in \Omega^*_{DR}(\tilde{B}_A /\R^N )$ such that
$p^* \omega'_{\Gamma_A} = \tilde{\omega}_{\Gamma_A}$.
We have a map of fibrations
\[
 \xymatrix{
 \hat{B}_A \ar@{^{(}->}[r] \ar[rd]_-{\rho_A} & \tilde{B}_A \ar[d]^-{\tilde{\rho}_A} \ar@{->>}[r]^-p
  & \tilde{B}_A / \R^N \ar[ld]^-{\rho'_A} \\
  & B_A &
 }
\]
(for definition of $\rho_A$ see Proposition \ref{Interior_strata})
and it holds that
\[
 (\rho_A )_* \omega_{\Gamma_A} =(\tilde{\rho}_A )_* \tilde{\omega}_{\Gamma_A} = (\rho'_A )_* \omega'_{\Gamma_A}.
\]
This implies $(\rho_A )_* \omega_{\Gamma_A}=0$, since the fiber of $\rho'_A$ is of strictly less dimension than those of
$\rho_A$ and $\tilde{\rho}_A$.
This together with the formula (\ref{p_A}) completes the proof.
\end{proof}


Thanks to Lemma \ref{disconnected}, below we can assume that $\Gamma_A$ is connected.


In Theorem \ref{cochain} we assumed $g \le 1$, that is, our graph has at most one loop component (other than small
loops), and so does its connected subgraph $\Gamma_A$.


\begin{prop}\label{vanish_tree}
Suppose $n-j$ is even.
If $\sharp A \ge 3$ and $\Gamma_A$ is a tree, then
$(\pi^{\partial_A}_{\Gamma})_* \omega_{\Gamma}|_{\Int \Sigma_A}$ vanishes.
\end{prop}


\begin{proof}
Since $\Gamma_A$ is a tree, there are at least two uni-valent vertices in $\Gamma_A$.
All the possibilities of uni-valent vertices are listed in Figure \ref{types_ends}.
\begin{figure}[htb]%end_types
\[
 \begin{xy}
 (-5,5)*{\text{(a)}},
 (0,0)*{\bullet}="A",
 {\ar@{-}"A";(10,0)^<{p}},
 (15,5)*{\text{(b)}},
 (20,0)*{\bullet}="B",
 {\ar@{.}"B";(30,0)^<{p}},
 (35,5)*{\text{(c)}},
 (40,0)*{\circ}="C",
 {\ar@{.}"C";(50,0)^<{p}}
 \end{xy}
\]
\caption{Uni-valent vertices}\label{types_ends}
\end{figure}
We will prove the vanishing of integration along $\Sigma_A$ in the successive Lemmas;
types (a) and (c) in Lemma \ref{hidden_1} and type (b) in Lemma \ref{hidden_2}.
The assumption that $n-j$ is even will be used in the proof of Lemma \ref{hidden_2}.
\end{proof}


\begin{lem}[\cite{CCL02}]\label{hidden_1}
If $\sharp A \ge 3$ and $A$ has a uni-valent vertex $p$ of types (a) or (c) in Figure \ref{types_ends}, then the
integration $(\pi^{\partial_A}_{\Gamma})_* \omega_{\Gamma}|_{\Int \Sigma_A}$ vanishes.
\end{lem}


\begin{proof}
Let $q \in A$ be the vertex joined to $p$ in $\Gamma_A$, which must be internal in the case (a), while in the case (c)
it may be both internal or external.
There is a fiberwise free action $\R_{>0}$ on $\hat{B}_A$ defined on each fiber by
\[
 (\dots ,z_p , \dots )\longmapsto (\dots ,az_p +(1-a)z_q ,\dots ),\quad a\in \R_{>0}.
\]
Then $\hat{\omega}_{\Gamma_A} \in \Omega^*_{DR}(\hat{B}_A)$ is basic with respect to the quotient
$\hat{B}_A \to \hat{B}_A /\R_{>0}$.
Since $\sharp A \ge 3$, the fiber of $\hat{B}_A / \R_{>0} \to B_A$ is of strictly less dimension than that of
$\rho_A$.
Hence the similar argument as in Lemma \ref{disconnected} completes the proof.
\end{proof}



\begin{lem}\label{hidden_2}
Suppose $n-j$ is even.
If $\sharp A \ge 3$ and $\Gamma_A$ is a tree all of whose uni-valent vertices are of type (b) in Figure \ref{types_ends},
then $(\pi_{\Gamma})_* \omega_{\Gamma}|_{\Int \Sigma_A}$ vanishes.
\end{lem}


\begin{proof}
The vertex $q$ of $A$ which is joined to $p$ may be both internal or external.
Since valency of $q$ is greater than one, there exist vertices $r_1 ,\dots ,r_a$ ($r_i \ne p$, $a \ge 1$) which are
joined to $q$.


Suppose one of them, say $r_1$, is uni-valent.
By our assumption $r_1$ is also of type (b).
Consider a fiberwise involution $\chi_1 : \hat{B}_A \to \hat{B}_A$ defined by
\[
 \chi_1 (\iota ; \dots ,x_p , \dots ,x_{r_1},\dots ) := (\iota ; \dots ,x_{r_1}, \dots ,x_p ,\dots )
\]
(other coordinates are not changed).
This involution changes the orientation of the fiber by $(-1)^j$, while
$\chi_1^* \hat{\omega}_{\Gamma_A}=(-1)^{n-1}\hat{\omega}_{\Gamma_A}$ since $\chi^*_1 \theta_{pq}=\theta_{r_1 q}$,
$\chi^*_1 \theta_{r_1 q}=\theta_{pq}$ and $\theta$'s are of degree $n-1$.
Thus
\[
 (\rho_A )_* \hat{\omega}_{\Gamma_A} =(-1)^{j+n-1}(\rho_A )_* \hat{\omega}_{\Gamma_A}
 =-(\rho_A )_* \hat{\omega}_{\Gamma_A}
\]
since $n-j$ is even, and it must vanish.


Thus we may assume all the $r_i$'s are at least bi-valent.
That is, we can assume that all the uni-valent vertices $p$ (of type (b)) and adjacent $q$ are such that no other
uni-valent vertex is joined to $q$.
Then we can find at least two pairs $(p,q)$ of vertices such that $p$ is uni-valent vertex joined to exactly one
bi-valent vertex $q$; since otherwise $\Gamma_A$ cannot be a tree.
Such pairs, say $(p,q)$ and $(p' ,q' )$, are of types (b-1) or (b-2) or (b-3) in Figure \ref{fig_b1}, where an asterisk
can be both internal or external.
\begin{figure}[htb]%b1
\[
 \begin{xy}
 (-4,5)*{\text{(b-1)}},
 (0,0)*{\bullet}="A",(6,0)*{\circ}="B",(12,0)*{*}="C",
 {\ar@{.}"A";"B"^<{p}},{\ar@{.}"B";"C"^<{q}},{\ar@{.}"C";(15,0)^<{r}},
 (21,5)*{\text{(b-2)}},
 (25,0)*{\bullet}="D",(31,0)*{\circ}="E",(37,0)*{*}="F",
 {\ar@{.}"D";"E"^<{p}},{\ar@{.}"E";"F"^<{q}},{\ar@{.}"F";(40,0)^<{r}},
 (46,5)*{\text{(b-3)}},
 (50,0)*{\bullet}="G",(56,0)*{\bullet}="H",(62,0)*{\bullet}="I",
 {\ar@{.}"G";"H"^<{p}},{\ar@{-}"H";"I"^<{q}},{\ar@{.}"I";(65,0)^<{r}}
 \end{xy}
\]
\caption{Types (b-1), (b-2), (b-3)}\label{fig_b1}
\end{figure}


If $(p,q)$ is of type (b-1), define a fiberwise involution $\chi_2 :\hat{B}_A \to \hat{B}_A$ by
\[
 \chi_2 (\iota ; \dots ,y_q , \dots ) := (\iota ; \dots ,\iota(x_p )+ z_r -y_q ,\dots ),
\]
where $z_r$ is $\iota(x_r )$ or $y_r$ according to whether $r$ is internal or not.
This involution has orientation sign $(-1)^n$, while $\chi^*_2 \hat{\omega}_{\Gamma_A}=(-1)^{n-1}\hat{\omega}_{\Gamma_A}$
similarly to $\chi_1$.
Thus
\[
 (\rho_A )_* \hat{\omega}_{\Gamma_A} =(-1)^{n+n-1}(\rho_A )_* \hat{\omega}_{\Gamma_A}
 =-(\rho_A )_* \hat{\omega}_{\Gamma_A}
\]
and it must vanish.


So finally we can assume that both pairs $(p,q)$ and $(p' ,q' )$ are of types (b-2) or (b-3).
The proof of this case appeared in \cite{Rossi_thesis, Watanabe07}; consider a fiberwise involution
$\chi_3 :\hat{B}_A \to \hat{B}_A$ defined by
\[
 \chi_3 (\iota ; \dots ,x_p , \dots ,x_{p'},\dots )
 := (\iota ; \dots , x_q -x_{q'} + x_{p'}, \dots ,x_q - x_{q'} + x_p , \dots ).
\]
The orientation sign of $\chi_3$ is $(-1)^j$, and it satisfies
$\chi^*_3 \hat{\omega}_{\Gamma_A}=(-1)^{n-1}\hat{\omega}_{\Gamma_A}$ since $\chi^*_3 \theta_{pq}=\theta_{p' q'}$,
$\chi^*_3 \theta_{p'q'}=\theta_{pq}$.
Since $n-j$ is even,
\[
 (\rho_A )_* \hat{\omega}_{\Gamma_A} =(-1)^{n+j-1}(\rho_A )_* \hat{\omega}_{\Gamma_A}
 =-(\rho_A )_* \hat{\omega}_{\Gamma_A}
\]
as in the case of $\chi_1$, and it must vanish.
\end{proof}


When both $n>j\ge 3$ are odd, we have to consider the case that $\Gamma_A$ has one loop component to prove Theorem
\ref{cochain}.


\begin{prop}[\cite{Rossi_thesis, Watanabe07}]
Suppose $n>j\ge 3$ are odd.
If $\sharp A \ge 3$ and $\Gamma_A$ is connected with one loop component, then
$(\pi_{\Gamma})_* \omega_{\Gamma}|_{\Int \Sigma_A}$ vanishes.
\end{prop}


\begin{proof}
Let $s$ and $t$ be the numbers of i- and e-vertices in $A$ respectively.
Define an involution $F$ of $\hat{B}_A$ by
\begin{multline*}
 F(\iota ; x_1 ,\dots ,x_s ; y_{s+1} ,\dots ,y_{s+t}) := \\
 (\iota ; x_1 ,2x_1 -x_2 ,\dots ,2x_1 -x_s ;2\iota (x_1 )-y_{s+1},\dots ,2\iota (x_1 )-y_{s+t}),
\end{multline*}
whose orientation sign is $(-1)^{j(s-1)+nt}=(-1)^{s+t-1}$.


Let $\alpha$ and $\beta$ be the numbers of $\eta$- and $\theta$-edges of $\Gamma_A$ respectively.
Since $\hat{\varphi}_e \circ F =\iota_{S^{N-1}}\circ \hat{\varphi}_e$ for any edge $e$ ($N=j$ or $n$ according to whether
$e$ is an $\eta$-edge or a $\theta$-edge), we have
\[
 F^* \omega_{\Gamma_A} =(-1)^{j\alpha + n\beta} \omega_{\Gamma_A} =(-1)^{\alpha + \beta}\omega_{\Gamma_A}.
\]
But by our assumption, $\Gamma_A$ has exactly one loop component, so $\alpha+ \beta$ is equal to the number of
vertices of $\Gamma_A$, that is, $s+t$.
Thus
\[
 (\rho_A )_* \omega_{\Gamma_A} = (-1)^{s+t-1}(-1)^{s+t}(\rho_A )_* \omega_{\Gamma_A} =-(\rho_A )_* \omega_{\Gamma_A}
\]
and hence $(\rho_A )_* \omega_{\Gamma_A} = 0$.
The formula (\ref{p_A}) in \S \ref{subsection_strata_1} completes the proof.
\end{proof}





\subsection{Hidden faces; strata at infinity}\label{subsection_hidden_infty}


In this subsection we will prove that the hidden strata $\Sigma^{\infty}_A$ at infinity (thus $\sharp A \ge 2$)
do not contribute to integrals.


\begin{lem}[\cite{CCL02, Rossi_thesis, Watanabe07}]\label{lem_tree_infty}
If $\sharp A \ge 2$, then the integration along $\Sigma^{\infty}_A$ vanishes.
\end{lem}


\begin{proof}
First we will show that $\omega_{\Gamma^{\infty}_A}$ cannot satisfy the equation in Lemma \ref{vanish_infty} when
$A \subset V(\Gamma )$ is a proper subset.


If $\Gamma$ is an admissible graph, then each e-vertex of $\Gamma^{\infty}_A$ ($=\Gamma / \Gamma^c_A$) is of at least
tri-valent, and each i-vertex is an endpoint of some $\theta$-edge.
This implies
\begin{equation}\label{eq_deg_positive}
 2\sharp \{ \theta\text{-edges of }\Gamma^{\infty}_A \} \ge a+3b,
\end{equation}
where $a$ and $b$ are numbers of i- and e-vertices of $\Gamma_A$.


Since $\Gamma^{\infty}_A \setminus \{ v^{\infty}\}$ is a one-dimensional open object with at most one loop component and
with at least one open edge,
\begin{equation}\label{number_edge}
 \sharp \{ \eta \text{-edges of } \Gamma^{\infty}_A \} + \sharp \{ \theta \text{-edges of } \Gamma^{\infty}_A \} \ge a+b.
\end{equation}
By using estimations \eqref{eq_deg_positive}, \eqref{number_edge} and $n-j-2\ge 0$, we see that
$\omega_{\Gamma^{\infty}_A}$ does not satisfy the criterion of Lemma \ref{vanish_infty};
\begin{align*}
 \deg \omega_{\Gamma^{\infty}_A}
 & \ge (n-1)\sharp \{ \theta \text{-edges of } \Gamma^{\infty}_A \}
  +(j-1)(a+b-\sharp \{ \theta \text{-edges of } \Gamma^{\infty}_A \} ) \\
 & = (n-j)\sharp \{ \theta \text{-edges of } \Gamma^{\infty}_A \} + (j-1)(a+b) \\
 & \ge \frac{1}{2}(n-j)(a+3b) + (j-1)(a+b) \\
 & = (ja+nb-1) +\frac{1}{2}(n-j-2)(a+b)+1 \\
 & > ja+nb-1 \\
 & = \dim \hat{B}^{\infty}_A .
\end{align*}
Next consider the case $A=V(\Gamma )$ (thus $\Gamma^{\infty}_A =\Gamma$).
Define a fiberwise free action of $\R_{>0}$ on $\hat{B}^{\infty}_A$ by
\[
 (x_1 ,\dots ,y_{a+1},\dots ) \longmapsto
 (x_1 , \alpha x_2 +(1-\alpha )x_1 , \dots ,\alpha y_{a+1} + (1-\alpha )f_0 (x_1 ), \dots ),
\]
where $f_0 : \R^j \hookrightarrow \R^n$ is the standard inclusion given by $x \mapsto (x,0,\dots ,0)$.
This action is non-trivial since $\sharp A \ge 2$.
The differential form $\omega_{\Gamma^{\infty}_{V(\Gamma)}}$ is basic with respect to this action, hence similar argument
to Lemma \ref{disconnected} completes the proof.
\end{proof}


\begin{rem}
In the end of the proof we used that the long knots are standard near infinity, so $\omega_{\Gamma^{\infty}_{V(\Gamma )}}$
contains no information about the base space $\emb{n}{j}$.
\end{rem}


Thus we have shown that all of hidden and infinity contributions vanish, and completed the proof of
Theorem \ref{3.1_second_half}.





\subsection{Proof of Theorem \ref{thm_closed}}\label{subsec_proof_closed}
In the above proofs, we have used
\begin{itemize}
\item symmetry of the fiber, and
\item dimension counting.
\end{itemize}
Below we check that some of the arguments are valid even if $n-j \ge 3$ is odd (in particular $n=6k$, $j=4k-1$), and
can be used to prove Theorem \ref{thm_closed}.


\begin{lem}\label{lem_dH=0}
$d\calH =I(\delta H)$ modulo the contributions of hidden faces.
\end{lem}


\begin{proof}
Exactly similar to Theorem \ref{3.1_first_half}.
We used Theorems \ref{principal} and \ref{infty_principal} to prove Theorem \ref{3.1_first_half}, which are proved by
only dimension counting.
\end{proof}


\begin{lem}\label{lem_dH_hidden}
The hidden faces except for the face $\Sigma_A$ corresponding to $A=V(H_2 )$ do not contribute to the integral.
\end{lem}


\begin{proof}
The strata at infinity do not contribute; the proof of Lemma \ref{lem_tree_infty} does not use the parities of $n,j$.
By Lemma \ref{disconnected}, which uses only dimension counting, we have only to consider the faces $\Sigma_A$ for $A$
such that the subgraph $\Gamma_A$ is connected.

For $A \subset V(H_1 )$ with $\sharp A =3$, use Lemma \ref{hidden_1}.

For $A=V(H_1 )$, consider a fiberwise involution $F:\hat{B}_A \to \hat{B}_A$ defined by
\[
 F(\iota ; (x_1 ,\dots ,x_4 )) := (\iota ; (2x_2 -x_1 ,x_2,x_3 ,x_4)).
\]
The orientation sign of $F$ is $(-1)^j$, while $F^* \hat{\omega}_{H_1}=(-1)^n \hat{\omega}_{H_1}$ because
$F^* \theta_{12} = (-1)^n \theta_{12}$ and $F^*$ preserves $\eta_{23}$, $\theta_{34}$.
Since $n-j$ is odd,
\[
 (\rho_A )_* \hat{\omega}_{H_1} =(-1)^{n+j}(\rho_A )_* \hat{\omega}_{H_1} = -(\rho_A )_* \hat{\omega}_{H_1}
\]
and hence the integration along $\Sigma_{V(H_1)}$ must vanish.

For $A \subset V(H_2 )$ with $\sharp A=3$, then $A$ contains two i-vertices joined to the e-vertex labeled by $4$.
Then using the involution $\chi_2$ appeared in the proof of Lemma \ref{hidden_2}, vanishing for type (b-1),
we can complete the proof.
\end{proof}


Next consider the contribution of `anomalous face' $\Sigma_A$, $A=V(H_2 )$ (hence $\Gamma_A =H_2$).
Recall from \S \ref{subsection_strata_1} that the face $\Sigma_A$ is described by the pullback square in Proposition
\ref{Interior_strata} with $B_A =\I_j (\R^n )$ and $C_{\Gamma / \Gamma_A}=\emb{n}{j}\times \R^j$.
The contribution of $\Sigma_A$ is given by $p_* D^*_A (\rho_A )_* \hat{\omega}_{H_2}$, where
$p:\emb{n}{j} \times \R^j \to \emb{n}{j}$ is the first projection.


At present we cannot determine whether this contribution vanishes or not, so
\[
 dI(H) = \frac{1}{6}p_* D^*_A (\rho_A )_* \hat{\omega}_{\Gamma_A}.
\]
But we can define a correction term $c$ which kills this contribution as follows.


\begin{lem}\label{lem_hatomega_closed}
The form $(\rho_A )_* \hat{\omega}_{H_2} \in \Omega^{2n-2j-2}_{DR}(\I_j (\R^n ))$ is closed.
\end{lem}


\begin{proof}
By the generalized Stokes theorem,
\[
 d(\rho_A )_* \hat{\omega}_{H_2} = \pm (\rho^{\partial}_A )_* \hat{\omega}_{H_2},
\]
where $\rho^{\partial}_A$ is the restriction of $\rho_A$ onto the boundary of the fiber.
The principal faces correspond to the graph (up to labeling)
\[%principal
 \begin{xy}
 (0,0)*{\bullet}="A", (10,0)*{\bullet}="B", (20,0)*{\bullet}="C",
 {\ar@{.}"A";"B"^<{1}},{\ar@{.}"B";"C"^>{3}},
 (10,2)*{\sb 2}
 \end{xy}
\]
obtained by contracting the edge $i4$ ($i=1,2,3$) of $H_2$.
Consider the involution
\[
 F : (x_1 ,x_2 ,x_3 ) \longmapsto (x_1 ,x_2 ,2x_2 -x_3)
\]
of the principal face.
The orientation sign of $F$ is $(-1)^j$, while $F^* \hat{\omega}_{H_2}=(-1)^n \hat{\omega}_{H_2}$
since $F^* \theta_{12} = \theta_{12}$ and $F^*_1 \theta_{23} = \theta_{32}=(-1)^n \theta_{23}$.
Since $j+n$ is odd, the integration along the principal faces vanish.


The hidden (but non-anomalous) contributions are proved to vanish in completely similar way as in
Lemma \ref{lem_dH_hidden}.
Since there is no anomalous face of the fiber of $\hat{B}_A \to B_A$, we have proved that
$d(\rho_A )_* \hat{\omega}_{H_2} = \pm (\rho^{\partial}_A )_* \hat{\omega}_{H_2}=0$.
\end{proof}


Thus we have a cohomology class $[(\rho_A )_* \hat{\omega}_{H_2}] \in H^{2n-2j-2}_{DR}(\I_j (\R^n ))$.
But in fact $\I_j (\R^n )$ is homotopy equivalent to Stiefel manifold of $j$-frames in $\R^n$ \cite{Rossi_thesis}, and
it is known that, when $n-j$ is odd, its cohomology ring with coefficients in $\R$ is given by
\[
 H^* (\I_j (\R^n );\R ) \cong
 \begin{cases}
  H^* (S^{2n-5}\times S^{2n-9} \times \dots \times S^{2(n-j)+1} \times S^{n-1};\R ) & n \text{ is even}, \\
  H^* (S^{2n-7}\times S^{2n-11} \times \dots \times S^{2(n-j)+1};\R ) & n \text{ is odd}.
 \end{cases}
\]
Hence we can find a form $\mu \in \Omega^{2n-2j-3}_{DR}(\I_j (\R^n ))$ such that $d\mu =(\rho_A )_* \hat{\omega}_{H_2}/6$
(the factor $S^{n-1}$ in the right hand side does not cause any trouble; if $2n-2j-2=n-1$ then $n=2j+1$ and $n$ becomes
odd).


\begin{defn}\label{def_c}
Define $c:=-p_* D^*_A \mu \in \Omega^{2n-3j-3}_{DR}(\emb{n}{j})$ ($p:\emb{n}{j} \times \R^j \to \emb{n}{j}$ is the first
projection).
\end{defn}


Then Theorem \ref{thm_closed} is easily proved;
\begin{alignat*}{2}
 &d(I(H)+c) = p_* D^*_A (\rho_A )_* \hat{\omega}_{H_2}/6 -dp_* D^*_A \mu & \quad & \\
 &\ = p_* D^*_A (\rho_A )_* \hat{\omega}_{H_2}/6 -p_* dD^*_A \mu \pm p^{\partial}_* D^*_A \mu & \quad &
  \text{(Stokes theorem)} \\
 &\ = p_* D^*_A (\rho_A )_* \hat{\omega}_{H_2}/6 -p_* D^*_A d\mu & \quad & (D_A \text{ is constant near }
  \partial C_1 (\R^j ))\\
 &\ =0 & \quad & \text{(by definition of }\mu ).
\end{alignat*}


\begin{rem}\label{rem_mu_symmetric}
We considered involutions $i_{p,q} : \R^n \to \R^n$ in the proof of Theorem \ref{thm_Haefliger_general}.
They induce involutions $i_{p,q}$ on $\I_j (\R^n )$ given by $\iota \mapsto i_{p,q} \circ \iota$.
This lifts to $\hat{i}_{p,q} : \hat{B}_A \to \hat{B}_A$ ($A=V(H_2 )$) defined by
$ \hat{i}_1 (\iota ;(x_1 ,x_2 ,x_3 );y) := (i_{p,q} \circ \iota ;(x_1 ,x_2 ,x_3 );i_{p,q} (y))$.
This has the orientation sign $(-1)^{p-q+1}$ on the fiber, and $\hat{i}^*_1 \hat{\omega}_A =(-1)^{p-q+1}\hat{\omega}_A$
since $i^*_{p,q} \theta_{*4}=(-1)^{p-q+1}\theta_{*4}$ (see the remark after Proposition \ref{prop_additive}).
Hence $i^*_{p,q} (\rho_A )_* \hat{\omega}_A =(\rho_A )_* \hat{\omega}_A$.
This implies that, replacing $\mu$ with $(\mu +i^*_{p,q} \mu )/2$, we may assume that $i^*_{p,q} \mu =\mu$.
Since $i_{p,q}$'s for different $p,q$ commute with each other, we can arrange $\mu$ so that it is preserved by all of these
involutions by repeating the same procedure as above.
\end{rem}





\subsection{Independency on volume forms}\label{subsec_indep_vol}


There is another vanishing result, which is needed in the proof of Propositions \ref{prop_indep_vol},
\ref{prop_indep_vol2} (independency on the choices of volume forms of the map $I$ on cohomology).
Recall that we assume that $g$, $n$ and $j$ are such that the integration map $I$ is a cochain map and that $n-j>2$.



\begin{lem}\label{tilde_I_closed}
The differential form $\tilde{I} (\Gamma ) \in \Omega^*_{DR}(\emb{n}{j} \times [0,1])$ from the proof of Proposition
\ref{prop_indep_vol} is closed if $n-j>2$.
\end{lem}


\begin{proof}
As already done for $I(\Gamma)$ in \S \ref{subsec_principal_cancel}, \S \ref{subsection_hidden_first} and \S
\ref{subsection_hidden_infty}, we must show
\begin{itemize}
\item
 $d_{\emb{n}{j}\times [0,1]}\tilde{I}(\Gamma ) = \tilde{I}(\delta \Gamma )$ modulo the contributions of hidden faces
 of the fibers of $\pi_{\Gamma_i}$, and hence it vanishes since $\Gamma$ is a cocycle, and
\item
 the contributions of hidden faces also vanish.
\end{itemize}
In this section we have proved the vanishing results by using symmetry and dimension counting.
We have to repeat these proofs for $\tilde{I}$.
The symmetry arguments can be applied to the cases here, since the factor $[0,1]$ does not cause any trouble.


We can check that the dimension-counting arguments also work.
The key ingredients in the proofs are the equation (\ref{p_A}) and Lemmas \ref{vanish_1}, \ref{vanish_infty}.
For $\tilde{I}$, we need to replace the criteria in Lemmas \ref{vanish_1}, \ref{vanish_infty} with
\begin{alignat*}{2}
 &\deg \omega_{\Gamma_A} = nb-n \ \text{ or } \ nb-(n+1) &\quad &\text{if } a=0, \\
 &0 \le \deg \omega_{\Gamma_A} - (ja+nb-(j+1)) \le nj+1  &\quad &\text{if } a>0.
\end{alignat*}
and
\[
 \deg \omega_{\Gamma^{\infty}_A} = ja+nb \ \text{ or } \ ja+nb-1
\]
respectively, since these criteria are consequences of the following pullback square
\[
\xymatrix{
  \Int \Sigma_A \times [0,1] \ar[r]^-{\hat{D}_A \times \id} \ar[d]_-{p_A \times \id}
   & \hat{B}_A \times [0,1] \ar[d]^-{\rho_A \times \id} \\
  C^o_{\Gamma / \Gamma_A} \times [0,1] \ar[r]^-{D_A \times \id} & B_A \times [0,1]
 }
\]
and of $\Int \Sigma^{\infty}_A \times [0,1] \approx C_{\Gamma^c_A} \times \hat{B}^{\infty}_A \times [0,1]$.
These replacements do not affect all the arguments in \S \ref{subsection_hidden_first} and
\S \ref{subsection_hidden_infty}, except for Lemma \ref{lem_tree_infty};
a problem may occur when $n-j=2$ and $A=\{ p,q\}$ consists of two i-vertices, since in such a case
$\deg \omega_{\Gamma^{\infty}_A}=(j-1)+(n-1)=2j=\dim (\hat{B}^{\infty}_A \times [0,1])$ can happen.
\end{proof}


Similar arguments show the following.


\begin{lem}\label{lem_indep_H_vol}
Suppose $n-j \ge 3$ is odd.
Then the cohomology class $\calH \in H^{2n-3j-3}_{DR}(\emb{n}{j})$ is independent of the choices of (anti-)symmetric
volume forms.
\end{lem}


\begin{proof}
First let $vol_{S^{j-1}}$ be fixed, and let $v_0$, $v_1$ be two symmetric volume forms of $S^{n-1}$ with total integral
one.
Define the maps $I_0$, $I_1$, $\tilde{I}$ and the form $\tilde{v}$ as in the proof of Proposition \ref{prop_indep_vol}.
Then we have
\begin{itemize}
\item $d_{\emb{n}{j}\times [0,1]}\tilde{I}(H) = \tilde{I}(\delta H)$ modulo the contributions of hidden
faces of the fibers of $\pi_{H_i}$, and hence vanishes since $H$ is a cocycle, and
\item the contributions of hidden faces except for $\Sigma_{V(H_2 )}$ also vanish.
\end{itemize}
Let $A:=V(H_2 )$.
Then $(\rho_A \times \id )_* \hat{\omega}_A \in \Omega^{2n-2j-2}_{DR}(\I_j (\R^n )\times [0,1])$ is a closed form;
the proof is same as Lemma \ref{lem_hatomega_closed}.
But since $H^{2n-2j-2}_{DR}(\I_j (\R^n )\times [0,1])=0$ when $n-j$ is odd, we have a form $\tilde{\mu}$ which satisfies
$(\rho_A \times \id )_* \hat{\omega}_A /6=d\tilde{\mu}$.
Using $\tilde{c}:=-\tilde{p}_* (D_A \times \id )^* \tilde{\mu}$ (where
$\tilde{p}:\R^j \times \emb{n}{j}\times [0,1]\to \emb{n}{j}\times [0,1]$ is the projection), we have a closed form
\[
 \tilde{\calH}:=\tilde{I}(H)+\tilde{c} \in \Omega^{2n-3j-3}_{DR}(\emb{n}{j} \times [0,1]).
\]
By the generalized Stokes theorem,
\[
 (I_1 (H)+\tilde{c}|_{\emb{n}{j}\times \{ 1\}}) - (I_0 (H)+\tilde{c}|_{\emb{n}{j}\times \{ 0\}}) =\pm d p_* \tilde{\calH}.
\]
But $\tilde{c}|_{\emb{n}{j}\times \{ \varepsilon \}}$ ($\varepsilon =0,1$) comes from
$\tilde{\mu}|_{\I_j (\R^n )\times \{ \varepsilon \}}$ which satisfies
$d\tilde{\mu}|_{\I_j (\R^n )\times \{ \varepsilon \}}=(\rho_A \times \id_{\{ \varepsilon\}})_* \hat{\omega}_A /6$.
Hence $\tilde{c}|_{\emb{n}{j}\times \{ \varepsilon \}}$ works as a correction term for $I_{\varepsilon}(H)$.
Thus we see that $\calH$ is independent of the choices of symmetric $vol_{S^{n-1}}$.


Similar arguments work when we fix $vol_{S^{n-1}}$ and use two different $vol_{S^{j-1}}$'s
(in this case we can choose $\tilde{\mu}$ of the form $q^* \mu$ where $q:\I_j (\R^n )\times [0,1] \to \I_j (\R^n )$ is
the projection, because the graph $H_2$ contains no $\eta$-edges and hence we do not use $vol_{S^{j-1}}$ to define the
correction term).
\end{proof}