\section{Graph complexes}\label{section_graph}


This and the next sections provide preliminaries for not only the subsequent sections but the forthcoming paper
\cite{KWatanabe08}.
In this section we introduce more general graphs than those in \cite{CattaneoRossi05, Rossi_thesis, Watanabe07},
including `degenerate graphs.'





\subsection{Graphs}\label{subsection_graph}


The {\em graphs} appearing here have two types of vertices.
One is the {\it external} vertex (or shortly {\it e-vertex}), which is depicted by $\circ$, while the other is the
{\it internal} vertex (shortly {\it i-vertex}), depicted by $\bullet$.


Similarly there are two types of edges.
One is $\theta${\em -edge}, which is depicted by a dotted line, and another is $\eta${\it -edge}, depicted by a
solid line.
We suppose that all the endpoints of $\eta$-edges are i-vertices.

A $\theta$-edge $e$ can form a {\it small loop} at an i-vertex $p$, that is, $e$ may have exactly one i-vertex $p$ as
its endpoint.
A single $\eta$-edge is not allowed to be a small loop, but an $\eta$-edge together with a $\theta$-edge can form
a loop at an i-vertex, called a {\em double loop}.
When we count the edges of a graph, we count a double loop twice, regarding it as consisting of an $\eta$-loop and a
$\theta$-loop.
But a double loop raises the first Betti number of the graph by one.


\begin{defn}\label{def_admissible}
A vertex $v$ of a graph is {\em admissible} if it is
\begin{enumerate}
\item an i-vertex of valence $\ge 1$ with at least one $\theta$-edge (possibly a loop) emanating from $v$, or
\item an e-vertex of valence $\ge 3$, with only $\theta$-edges (which are not loops) emanating from $v$
\end{enumerate}
(see Figure \ref{fig_admissible}).
A graph is called {\em admissible} if all its vertices are admissible.
\end{defn}


\begin{figure}[htb]%admissible
\[
 \begin{xy}
  (0,6)*{(1)},(75,6)*{(2)},
  (0,0)*{\bullet}, (30,0)*{\bullet},(55,0)*{\bullet},(80,0)*{\circ}="A",
  {\ar@{.}(0,0);(10,0)},
  {\ar@{-}(20,0);(30,0)},{\ar@{.}(30,0);(40,0)},
  {\ar@{.}(55,6);(55,0)},{\ar@{-}(55,0);(49.8038,-3)},{\ar@{-}(55,0);(60.1961,-3)},
  {\ar@{.}(80,6);"A"},{\ar@{.}"A";(74.8038,-3)},{\ar@{.}"A";(85.1961,-3)}
 \end{xy}
\]
\caption{Admissible vertices}\label{fig_admissible}
\end{figure}


Figure \ref{example_1} shows an example of an admissible graph.
There might be an i-vertex which is adjacent to more than one $\theta$-edges, an e-vertex of valency $\ge 4$, and so on.
Such vertices did not appear in \cite{CattaneoRossi05,Rossi_thesis,Watanabe07}.


\begin{figure}[htb]%graph_example
\[
 \begin{xy}
                        (15,20)*{\bullet}="A", (30,20)*{\bullet}="B",
  (0,14)*{\bullet}="C",                                               (45,14)*{\circ}="D", (55,14)*{\bullet}="E",
                        (15, 8)*{\bullet}="F", (30, 8)*{\bullet}="G",
                        (15, 0)*{\bullet}="H", (30, 0)*{\bullet}="I",
  {\ar@{.>}"A";"C"^<{1}_>{4}}, {\ar "A";"B"_>{5}},{\ar@{.}@(ru,lu)"B";"B"}, {\ar@{.>}"B";"D"^>{9}},
  {\ar@{.>}"D";"E"^>{7}},
  {\ar@{.>}"C";"F"^>{8}},{\ar "G";"F"_<{3}},{\ar@{.>}"G";"D"},
  {\ar@{.>}"F";"H"^>{6}},{\ar@{.>}"G";"I"^>{2}}
 \end{xy}
\]
\caption{An example of an admissible graph for odd $n$ and $j$}\label{example_1}
\end{figure}


In \S \ref{sec_conf} the graphs will be regarded as in $\R^n$ (Figure \ref{example_2}); the set of vertices will become
a configuration of points in $\R^n$, where the i-vertices are on $\R^j$ (or a long $j$-knot) embedded in $\R^n$.
Each edge will correspond to a `direction map' determined by its endpoints, or to the volume form of sphere pulled back
by the direction map.
The $\eta$-edges correspond to directions in $\R^j$, while $\theta$-edges to those in $\R^n$.
\begin{figure}[htb]
\[
 \begin{xy}
  {\ar@{-}(0,20);(3,3)},{\ar@{-}(3,3);(63,3)},{\ar@{-}(0,20);(60,20)},{\ar@{-}(60,20);(63,3)},
  (5,17)*{\R^j},
  (50,23)*{\circ}="A",
  (20,18)*{\bullet}="B", (40,18)*{\bullet}="C",
  (10,12)*{\bullet}="D", (18,7)*{\bullet}="E", (34,7)*{\bullet}="F",(50,9)*{\bullet}="G",
  (10,6)*{\bullet}="H",  (40,5)*{\bullet}="I",
  {\ar@{.>}@/_/"B";"D"_>{4}},{\ar "B";"C"_<{1}_>{5}},{\ar@{.}@(ru,lu)"C";"C"}, {\ar@{.>}"C";"A"},
  {\ar@{.>}@/^/"D";"E"},{\ar "F";"E"_<{3}_>{8}}, {\ar@{.>}"F";"A"},{\ar@{.>}@/_/"I";"F"_<{2}},
  {\ar@{.>}"A";"G"^<{9}_>{7}},{\ar@{.>}@/_/"E";"H"^>{6}}
 \end{xy}
\]
\caption{The graph from Figure \ref{example_1} in $\R^n$}\label{example_2}
\end{figure}


Afterward we will need to fix the orientations of configuration spaces and the signs of the volume forms.
For these purposes we `decorate' the graphs as follows (see Figure \ref{fig_decoration}).
\begin{enumerate}
\item
 The i-vertices are labeled by $1,\dots ,s$ and the e-vertices are labeled by $s+1,\dots ,s+t$ for some suitable $s$
 and $t$.
\item
 When both $n$ and $j$ are odd, all the edges are oriented, all the loops are ordered and each loop is given the sign
 $\pm 1$.
\item
 When $n$ is odd and $j$ is even, all the $\theta$-edges are oriented, while all the $\eta$-edges are labeled.
\item
 When $n$ is even and $j$ is odd, all the $\theta$-edges (including those in double loops) are labeled, while all the
 $\eta$-edges are oriented.
 The double loops are given other labels $1,2,\dots$ than those for all the $\theta$-edges.
 A double loop is given a sign $\pm 1$.
\item
 When both $n$ and $j$ are even, all the edges (including those in double loops) are labeled.
 The small / double loops are given other labels.
\end{enumerate}
\begin{figure}[htb]
\[
 \begin{xy}
  (0,9)*{(2)},(0,5)*{\circ}="A", (8,0)*{\bullet}="B", (16,5)*{\bullet}="C",
  {\ar@{.>}"A";"B"_<{p}_>{q}},{\ar "B";"C"_>{r}},
  (25,9)*{(3)},(25,5)*{\circ}="D", (33,0)*{\bullet}="E", (41,5)*{\bullet}="F",
  {\ar@{.>}"D";"E"_<{p}_>{q}},{\ar@{-} "E";"F"^(.5){(a)}_>{r}},
  (50,9)*{(4)},(50,5)*{\circ}="G", (58,0)*{\bullet}="H", (66,5)*{\bullet}="I",
  {\ar@{.}"G";"H"_<{p}^(.5){(a)}_>{q}},{\ar "H";"I"_>{r}},
  (75,9)*{(4)},(75,5)*{\circ}="J", (83,0)*{\bullet}="K", (91,5)*{\bullet}="L",
  {\ar@{.}"J";"K"_<{p}^(.4){(a)}_>{q}},{\ar@{-} "K";"L"^(.6){(b)}_>{r}},
 \end{xy}
\]
\caption{Decorations of graphs}\label{fig_decoration}
\end{figure}





\subsection{Space of graphs}\label{subsection_space_of_graph}


Below we assume that all the graphs are admissible and decorated unless otherwise stated.


\begin{defn}\label{definition_space_graphs}
For a graph $\Gamma$, define
\begin{align*}
 \ord \Gamma &:= \sharp \{ \theta \text{-edges of } \Gamma \} - \sharp \{ \text{e-vertices of } \Gamma \} , \\
 \deg \Gamma &:= 2 \sharp \{ \theta \text{-edges of } \Gamma \} - 3 \sharp \{ \text{e-vertices of } \Gamma \}
 - \sharp \{ \text{i-vertices of } \Gamma \}
\end{align*}
(for example, $\ord \Gamma =7$ and $\deg \Gamma =5$ for the graph $\Gamma$ in Figure \ref{example_1}; see below for
more explanations).
We denote by $\D^{k,l}$ the vector space spanned by admissible decorated graphs with $\ord =k$, $\deg =l$ modulo the
subspace spanned by
\[
 \Gamma' - (-1)^{j\text{sign}\, \sigma +n\text{sign}\, \tau +a+b+\text{sign}\, \rho}\Gamma
 \quad \text{and} \quad \Gamma'' ,
\]
where $\Gamma'$ is obtained from $\Gamma$ by
\begin{itemize}
\item
 permuting the labels of i- and e-vertices of $\Gamma$ by $\sigma \in \mathfrak{S}_s$ and $\tau \in \mathfrak{S}_t$
 respectively ($s$ and $t$ are the numbers of i- and e-vertices of $\Gamma$ respectively),
\item
 reversing $a$ oriented edges of $\Gamma$, and
\item
 switching $b$ signs and permuting the labels of small / double loops by $\rho$,
\end{itemize}
and $\Gamma''$ is a graph with `multiple $\eta$- (or $\theta$-) edges,' i.e., there are two vertices $p,q$ which are
joined by two or more $\eta$- (or $\theta$-) edges (we allow $p,q$ joined by two edges, one $\eta$-edge and one
$\theta$-edge).
Moreover we introduce one more relation; $\Gamma \sim 0$ in $\D^*$ if $n-j$ is odd and $\Gamma$ is a graph with
at least one small loop, or if $n$ is odd and $\Gamma$ has a double loop.
\end{defn}


\begin{rem}\label{rem_graph}
The sign $j\text{sign}\, \sigma +n\text{sign}\, \tau$ will correspond to the orientation sign of the configuration
space.
In \S \ref{sec_conf} we will associate the volume forms of spheres of even dimensions with the oriented edges.
Reversing an oriented edge corresponds to pull-back via the antipodal map, hence yields a sign $-1$.
Unoriented edges correspond to differential forms of odd degrees, so permuting the labels yields a sign.
\end{rem}


The meaning of $\ord \Gamma$ is as follows.
In Figure \ref{example_2}, if we contract $\R^j$ together with $\eta$-edges regarded as in $\R^j$, then we obtain a
one dimensional CW complex whose edges are $\theta$-edges. Its first Betti number is equal to $\ord \Gamma$.


To explain the meaning of $\deg \Gamma$, we need some terminologies.



\begin{defn}\label{non_degenerte_graphs}
An admissible vertex is said to be {\it non-degenerate} if it is
\begin{itemize}
\item an i-vertex with exactly one $\theta$-edge (and possibly many $\eta$-edges) emanating from it, or
\item a tri-valent e-vertex.
\end{itemize}
All other vertices are said to be {\it degenerate}.
\end{defn}



For example, all the vertices in Figure \ref{fig_admissible}, and the vertices $1$, $2$, $6$, $7$ and $9$ of the graph
in Figure \ref{example_1} are non-degenerate.


\begin{rem}
It can be easily understood that ``non-degenerate vertex'' is the same notion as ``trivalent vertex'' in \cite{CCL02}.
All the vertices of the graphs appeared in \cite{CattaneoRossi05, Rossi_thesis, Watanabe07} are non-degenerate.
\end{rem}


\begin{lem}
We have $\deg \Gamma \ge 0$ for any admissible graph $\Gamma$, and $\deg \Gamma =0$ if and only if all the vertices
of $\Gamma$ are non-degenerate.
\end{lem}



\begin{proof}
This Lemma is obvious by the definition of admissible vertices;
at least one $\theta$-edge emanates from any i-vertex of a graph and at least three $\theta$-edges emanate from
any i-vertex.
This implies
\[
 2 \sharp \{ \theta \text{-edges}\} \ge 3 \sharp \{ \text{e-vertices}\} + \sharp \{ \text{i-vertices}\}
\]
and the equality holds if and only if exactly one (resp.\ three) $\theta$-edge emanates from any i-vertices
(resp.\ e-vertices), that is, all the vertices are non-degenerate.
\end{proof}



\begin{rem}\label{number_of_vertices}
Let $\Gamma$ be an admissible graph of $\ord \Gamma =k$, $\deg \Gamma =l$.
Then $\Gamma$ has $2k-l$ vertices; it is a direct consequence of the definition.
In particular, if $\Gamma$ is non-degenerate ($l=0$), then the number of all vertices is $2k$.
In \cite{Watanabe07} the half of the number of the vertices of a (non-degenerate) graph is called its `degree.'
Thus our terminology `$\ord$' is a generalization of the `degree' in \cite{Watanabe07}.
\end{rem}





\subsection{Coboundary operation}\label{subsection_coboundary}





\begin{defn}
Let $\Gamma$ be a graph and $e = \overrightarrow{pq}$ its (possibly oriented) edge (but not a loop).
Define a new graph $\Gamma / e$ as follows (see Figure \ref{edge_contraction}).


{\bf (1)} When $e$ is an $\eta$-edge (then endpoints $p,q$ are both internal), define $\Gamma / e$ by contracting $e$,
that is, identifying the endpoints $p,q$ of $e$ and removing the edge $e$.
The decoration of $\Gamma /e$ is derived from that of $\Gamma$; the vertex of $\Gamma / e$ where the contraction
occurred is re-labeled by $\min \{ p,q \}$, and all the labels of the vertices of $\Gamma$ bigger than $\max \{ p,q\}$
are decreased by one.
The labels of the other vertices remain unchanged.
When $j$ is even and $e$ is the $i$-th edge, then the labels of other edges bigger than $i$ is decreased by one.


{\bf (2)} When $e$ is a $\theta$-edge and at least one of $p,q$ is an e-vertex, then $\Gamma / e$ is defined in the same
way as above.
If both $p,q$ are external, then the vertex where the contraction occurred is also external.
If one of $p,q$ is internal, then the resulting vertex is internal.


{\bf (3)} When $e$ is a $\theta$-edge with both $p,q$ being i-vertices, then $\Gamma / e$ is obtained from $\Gamma$ by
identifying the vertices $p$ and $q$, but not removing the edge $e$.
The edge $e$ becomes a small loop at the i-vertex $\min \{ p,q\}$.
The labeling of $\Gamma / e$ is determined similarly as above.
When $n$ and $j$ are odd, its sign is $+1$ (resp.\ $-1$) if $p<q$ (resp.\ $p>q$).
This small loop is labeled by $a$ if $\Gamma$ has $(a-1)$ small loops.


{\bf (4)} When $e$ is the $\eta$-edge of the multiple edges joining two i-vertices, then $\Gamma / e$ is obtained from
$\Gamma$ by identifying the vertices $p$ and $q$, and attaching a double loop at $p$.
This double loop is labeled by $a$ if $\Gamma$ has $(a-1)$ loops.
When $j$ is odd, the sign $\pm 1$ is given similarly as in (3).
We do not define $\Gamma /e$ for a $\theta$-edge of the multiple edges.
\end{defn}



\begin{figure}[htb]
\[
 \begin{xy}
  (0,20)*{(1)}, (3,15)*{\bullet}="A", (15,15)*{\bullet}="B", {\ar@{-}"A";"B"^<{p}^>{q}},(20,15)*{\mapsto},
  (25,15)*{\bullet},(25,17)*{{\sb p}},
  (35,20)*{(2)}, (38,15)*{\circ}="C", (50,15)*{\circ}="D", {\ar@{.}"C";"D"^<{p}^>{q}},(55,15)*{\mapsto},
  (60,15)*{\circ},(60,17)*{{\sb p}}, (70,15)*{,},
  (78,15)*{\bullet}="E", (90,15)*{\circ}="F", {\ar@{.}"E";"F"^<{p}^>{q}},(95,15)*{\mapsto},
  (100,15)*{\bullet},(100,17)*{{\sb p}},
  (0,10)*{(3)}, (3,5)*{\bullet}="G", (15,5)*{\bullet}="H", {\ar@{.}"G";"H"^<{p}^>{q}},(20,5)*{\mapsto},
  (25,5)*{\bullet},(25,3)*{{\sb p}},{\ar@{.}@(ru,lu)(25,5);(25,5)},
  (35,10)*{(4)}, (38,5)*{\bullet}="I", (50,5)*{\bullet}="J", {\ar@{.}@/_/"I";"J"}, {\ar@{-}@/^/"I";"J"^<{p}^>{q}},
  (55,5)*{\mapsto}, (60,5)*{\bullet},(60,3)*{{\sb p}},{\ar@{.}@(ru,lu)(60,5);(60,5)},
  (60,7)*+[Fo]{\phantom{a}},
  (80,6)*{(p<q)}
  \end{xy}
\]
\caption{Contractions of edges}\label{edge_contraction}
\end{figure}



We should notice that a graph $\Gamma /e$ of type (3) in Figure \ref{edge_contraction} is ruled out in $\D^*$
when $n-j$ is odd, and similarly a graph $\Gamma /e$ of type (4) is ruled out when $n$ is odd (see Definition
\ref{definition_space_graphs}).


We would like to define the operator $\delta$ by
\[
 \delta \Gamma = \sum_{e \in E(\Gamma ) \setminus \{ \text{loops} \}} (-1)^{\tau (e)} \Gamma / e,
\]
by giving some suitable signs $\tau (e)$, where $E(\Gamma )$ is the set of edges of $\Gamma$.


\begin{prop}\label{def_signs}
If we define the signs $\tau (e)$ as in (1)-(5) below, then the operator $\delta$ is well-defined and determines
a coboundary operation
\[
 \delta : \D^{k,l} \longrightarrow \D^{k,l+1},
\]
that is, $\delta \circ \delta =0$.
Thus $\{ \D^{k,*}, \delta \}$ is a cochain complex for any $k$.


{\bf (1)} Let both $n$ and $j$ be odd.
For any oriented edge $e=\overrightarrow{pq}$, define $\tau (e)$ by
\begin{equation}\label{eq_sign_odd_odd}
 \tau (e) :=
 \begin{cases}
  q   & p<q, \\
  p+1 & p>q.
 \end{cases}
\end{equation}


{\bf (2)} Let both $n$ and $j$ be even.
For the $i$-th edge $e$, define $\tau (e)$ by
\[
 \tau (e) :=
 \begin{cases}
  i   & e \text{ is of type (1), (2) in Figure \ref{edge_contraction}}, \\
  u+1 & e \text{ is of type (3) in Figure \ref{edge_contraction}},
 \end{cases}
\]
where $u$ is the number of small / double loops of $\Gamma$.


{\bf (3)} Let $n$ be even and $j$ be odd.
For any oriented $\eta$-edge $e=\overrightarrow{pq}$, define $\tau (e)$ by \eqref{eq_sign_odd_odd}.
For the $i$-th $\theta$-edge $e=pq$, $p<q$ with $q$ being an e-vertex, define $\tau (e):=i+s+1$, where $s$ is the number
of i-vertices of $\Gamma$.


{\bf (4)} Let $n$ be odd and $j$ be even.
For the $i$-th $\eta$-edge $e$, define $\tau (e):=i+t+1$ where $t$ is the number of e-vertices of $\Gamma$.
For any oriented $\theta$-edge $e=\overrightarrow{pq}$, define $\tau (e)$ by \eqref{eq_sign_odd_odd}.


{\bf (5)} Consider the case that $n$ is even, and $p$ and $q$ are i-vertices joined by `multiple edges,' one $\eta$-edge
and one $\theta$-edge.
If $j$ is odd and the $\eta$-edge is oriented from $p$ to $q$, then $\tau$ is given by \eqref{eq_sign_odd_odd}.
If $j$ is even, then $\tau =u+1$.
\end{prop}


\begin{rem}
The signs in Proposition \ref{def_signs} correspond to those of induced orientations of the boundary strata of
configuration spaces; see \S \ref{subsec_orientation}.
\end{rem}


The proof is completely similar to \cite[Theorem 4.2]{CCL02}; choose two edges $e_1$ and $e_2$, and contract them
in two different orders, then we obtain the same graph with opposite signs.
Notice that if $n-j$ is odd (resp.\ $n$ is odd) the case (3) (resp.\ (4)) in Figure \ref{edge_contraction} does not occur.