\documentclass[10pt]{amsart}

\usepackage{latexsym}
\usepackage{amssymb}
\usepackage{amsmath}

\newtheorem{theorem}{Theorem}[section]
\newtheorem{lemma}[theorem]{Lemma}
\newtheorem{proposition}[theorem]{Proposition}
\newtheorem{corollary}[theorem]{Corollary}

\theoremstyle{definition}
\newtheorem{definition}[theorem]{Definition}

\newtheorem{example}[theorem]{Example}

\newtheorem{exercise}[theorem]{Exercise}

\theoremstyle{remark}
\newtheorem{remark}[theorem]{Remark}

\def\N{\mathbb{N}}
\def\Z{\mathbb{Z}}
\def\R{\mathbb{R}}

\def\x{\mathbf{x}}

\def\hN{{{}^*\N}}
\def\hZ{{{}^*\Z}}
\def\hR{{{}^*\R}}
\def\hA{{{}^*\!A}}
\def\hB{{{}^*\!B}}
\def\hX{{{}^*X}}
\def\hY{{{}^*Y}}

\def\B{\mathcal{B}}
\def\D{\mathcal{D}}
\def\F{\mathcal{F}}
\def\G{\mathcal{G}}
\def\U{\mathcal{U}}
\def\V{\mathcal{V}}
\def\W{\mathcal{W}}

\def\orb{{\textit{orb}\,}}


\begin{document}


\title[Translation invariant filters
and van der Waerden's Theorem]{Translation invariant filters
\\
and van der Waerden's Theorem}

\author{Mauro Di Nasso}

\address{Dipartimento di Matematica\\
Universit\`a di Pisa, Italy}

\email{mauro.di.nasso@unipi.it}

\subjclass[2000]
{Primary 05D10; Secondary 03E05, 54D80.}

\keywords{Arithmetic progressions, Piecewise syndetic sets, 
Translation invariant filters, Algebra on $\beta\N$.}

%\date{}

\begin{abstract}
We present a self-contained proof of a strong version of 
van der Waerden's Theorem. By using
translation invariant filters that are maximal with respect to inclusion,
a simple inductive argument shows
the existence of ``piecewise syndetically"-many
monochromatic arithmetic progressions
of any length $k$ in every finite coloring of the natural numbers.
All the constructions presented are constructive in nature, in the sense that
the involved maximal filters are defined by recurrence on suitable 
countable algebras of sets. No use of the axiom of choice or 
of Zorn's Lemma is needed.
\end{abstract}

\maketitle



\section*{Introduction}

The importance of maximal objects in mathematics
is well-known, starting from the fundamental examples of
maximal ideals in algebra, and of ultrafilters in 
certain areas of topology and of Ramsey theory.
In this paper we focus on maximal filters on suitable 
countable algebras of sets
which are stable under translations.
By using such maximal objects, along with ultrafilters extending it,
we give a proof of a strong version of the following classical
result in Ramsey theory:

\smallskip
\noindent
\textbf{Theorem} (van der Waerden - 1927)
\emph{In every finite partition $\N=C_1\cup\ldots\cup C_r$
there exists a piece $C=C_i$ that contains
arbitrarily long arithmetic progressions,
that is, for every $k$ there exists a progression
$x+y, x+2y,\ldots,x+ky\in C$.}

\smallskip
In fact, we will prove the 
existence of ``piecewise syndetically"-many
monochromatic arithmetic progressions
of any length $k$.
% in every finite coloring of the natural numbers.

Usually, van der Waerden's Theorem is proved either by double
induction using elaborated, but elementary, combinatorial arguments
in the style of the original proof \cite{vdW},
or by using properties of the smallest ideal 
$K(\beta\N,\oplus)$ in the algebra
of ultrafilters (see \cite[Ch.14]{hs}; see also
\cite{fg,bh} for stronger versions).
In our proof, for any given piecewise set,
we restrict to a suitable countable algebra
of sets, and explicitly construct by recursion a maximal translation invariant
filter, and then an ultrafilter extending it.
The desired result is finally obtained by a short proof by induction,
that is essentially a simplified version of an argument
that was used in \cite{h} in the framework of
the compact right-topological semigroup $(\beta\N,\oplus)$.
It is worth remarking that,
contrarily to the usual ultrafilter proof, we make no explicit use
of the algebra in the space of ultrafilters; in fact,
we make no use of the axiom of choice nor of
Zorn's Lemma.
%\footnote{
%~However, maximal TIFs turn out to be closely related to the 
%smallest ideal $K(\beta\N,\oplus)$; this will be pointed out in the last section.}


\section{Preliminary notions}

$\N$ denotes the set of \emph{positive integers}, 
and $\N_0=\N\cup\{0\}$ the set
of non-negative integers.
For $A\subseteq\N$ and $n\in\N_0$, 
the \emph{leftward shift} of $A$ by $n$ is the set:
$$A-n:=\{m\in\N\mid n+m\in A\}$$

Elemental notions in combinatorics of numbers that we will use
in this paper are those of thick set, syndetic set, and piecewise syndetic set.
For completeness, let us recall them here.

A set $A\subseteq\N$ is \emph{thick} if it includes arbitrarily long intervals.
Equivalently, $A$ is thick if every finite set $F=\{n_1,\ldots,n_k\}\subset\N$
has a \emph{rightward shift} included in $A$, that is, there exists $x$ such that
$$x+F:=\{x+n_1,\ldots,x+n_k\}\subseteq A.$$ 
Notice that such an $x$ can
be picked in $A$. In terms of intersections, the property
of thickness of $A$ can be rephrased by saying that
the family $\{A-n\mid n\in\N_0\}$ has the \emph{finite intersection property}
(FIP for short), that is, $\bigcap_{i=1}^k(A-n_i)\ne\emptyset$
for any $n_1,\ldots,n_k$. 

A set $A\subseteq\N$ is \emph{syndetic} if it has ``bounded gaps",
that is, there exists $k\in\N$ such that $A$ meets every
interval of length $k$. Equivalently, $A$ is syndetic if
a finite number of leftward shifts of $A$ covers all the natural numbers,
that is, $\N=\bigcup_{i=1}^k(A-n_i)$ for suitable $n_1,\ldots,n_k\in\N_0$.

A set is \emph{piecewise syndetic} if it is the intersection of a thick set with
a syndetic set. Equivalently, $A$ is piecewise syndetic
if a finite number of leftward shifts cover a thick set, that is,
$\bigcup_{i=1}^k(A-n_i)$ is thick for suitable $n_1,\ldots,n_k\in\N_0$.

Notice that the families of thick, syndetic, and piecewise syndetic sets
are all invariant with respect to shifts.
A well-known relevant property of piecewise syndetic sets that is neither satisfied
by thick sets nor by syndetic sets, is the \emph{Ramsey property} below.
For the sake of completeness, we include here a proof.

\begin{proposition}\label{psRamsey} 
In every finite partition $A=C_1\cup\ldots\cup C_r$ of a piecewise
syndetic set $A$, one of the pieces $C_i$ is piecewise syndetic.
\end{proposition}

\begin{proof}
For simplicity, let us say that an interval $I$ is $k$-good for the set $B$
if for every sub-interval $J\subseteq I$ of lenght $k$ one has $J\cap B\ne\emptyset$.
By the hypothesis of piecewise syndeticity of $A$,
there exists $k\in\N$ and a sequence of
intervals $\langle I_n\mid n\in\N\rangle$ with
increasing length such that every $I_n$ is $k$-good for $A$.
It is enough to consider the case when $A=C_1\cup C_2$ is partitioned into two pieces,
because the general case $r\ge 2$ will then follow by induction.
We distinguish two cases. 

Case \# 1: There exists $h$ and there exist
infinitely many intervals $I_n$ that are $h$-good for $C_1$.
In this case $C_1$ is piecewise syndetic.

Case \# 2: For every $h$, there are only finitely many intervals $I_n$ that
are $h$-good for $C_1$. So, for every $h$ we can pick an interval 
$I_{n_h}$ of length $\ge h$ that is not $h$-good.
Let $J_h\subseteq I_{n_h}$ be a sub-interval of length $h$ such
that $J_h\cap C_1=\emptyset$.
The sequence of intervals $\langle J_h\mid h\in\N\rangle$ shows
that $C_2$ is piecewise syndetic.
Indeed, given $h$, for every sub-interval $J\subseteq J_h$ of length $k$
we have that $J\cap C_1\subseteq J_h\cap C_1=\emptyset$; and so
$J\cap C_2=J\cap A\ne\emptyset$, since $J\subseteq I_{n_h}$
and $I_{n_h}$ is $k$-good for $A$.
\end{proof}


\section{Maximal filters}

An \emph{algebra of sets} on $\N$ is a family 
of subsets of $\N$ that contains $\N$ and is closed under
finite unions, finite intersections, and complements.
Given a countable family $\mathcal{A}$,
one can explicitly construct the (countable)
algebra of sets generated by $\mathcal{A}$, that is, 
the smallest algebra that contains $\mathcal{A}$.\footnote
{~Let $\langle A_n\mid n\in\N\rangle$ be an enumeration of the elements of $\mathcal{A}$,
and let $\langle F_n\mid n\in\N\rangle$ be an enumeration 
of the nonempty finite sets of natural numbers
(\emph{e.g.}, if 
$n=\sum_{k=1}^\infty{a_{nk}}{2^{k-1}}$ is written
in binary expansion
where $a_{nk}\in\{0,1\}$, let $F_n:=\{k\mid a_{nk}=1\}$).
Then it is easily verified that
the following family $\B$ is the algebra generated by $\mathcal{A}$, 
where we denoted $A^{1}=A$ and $A^{-1}=A^c$:
$$\B:=\left\{
\bigcup_{i=1}^t(\bigcap_{k\in F_{n_i}}A_k^{\sigma_i(k)})\,\,\Big|\,\,
n_1,\ldots,n_t\in\N,\ \sigma_i:F_{n_i}\to\{-1,1\}\right\}.$$}

A \emph{filter} on an algebra of sets $\mathcal{B}$
is a nonempty family $\F\subseteq\mathcal{B}$ such that:
\begin {itemize}
\item
$\F$ is closed under finite intersections, that is, 
$A,B\in\F\Rightarrow A\cap B\in\F$;
\item
$\F$ is closed under supersets, that is, 
if $B\in\mathcal{B}$ and $B\supseteq A\in\F$ then $B\in\F$.
\end{itemize}

Every family $\G\subseteq\mathcal{B}$
with the \emph{finite intersection property} (FIP for short) 
generates a filter $\langle\G\rangle$, namely 
$$\langle\G\rangle:=
\{B\in\mathcal{B}\mid B\supseteq A_1\cap\ldots\cap A_k\ \text{where}\ 
A_i\in\G\}.$$


An \emph{ultrafilter} $\U$ on the algebra of sets $\B$ is a filter with the additional 
property that $A\in\U$ whenever $A\in\B$ and the complement $A^c\notin\U$.
Equivalently, an ultrafilter is a filter $\U$ 
with the \emph{Ramsey property}:
If $A_1\cup\ldots\cup A_k\in\U$ where all sets $A_i\in\B$,
then $A_j\in\U\ \text{for some}\ j$.
Ultrafilters can also be characterized as the filters
that are maximal under inclusion and so, by a straight
application of \emph{Zorn's Lemma}, 
it is proved that every filter can be extended to an ultrafilter.

The main ingredient in our proof of van der Waerden's Theorem
are those filters that are stable under shifts.

\begin{definition}
$\F$ is a \emph{translation invariant filter} (TIF for short)
if $\F$ is a filter with the additional property
that if $A\in\F$ then also $A-1\in\F$ (and hence, $A-n\in\F$ for all $n\in\N_0$).
\end{definition}

The notions of TIF and thick set are closely related.

\begin{proposition}\label{thickTIF}
A set $A$ is thick if and only if it belongs to a TIF $\F$.
\end{proposition}

\begin{proof}
Recall that $A$ is thick if and only if the family $\G=\{A-n\mid n\in\N_0\}$
has the FIP. It is easily verified that the generated
filter $\langle\G\rangle$ is translation invariant.

Conversely, assume that $A\in\F$ for some TIF $\F$.
Then trivially the family $\G=\{A-n\mid n\in\N_0\}$ has the FIP because $\G\subseteq\F$.
\end{proof}

Similarly to ultrafilters,
by a straightforward application of \emph{Zorn's Lemma}
it can be shown that every TIF can be extended to a maximal TIF.
However, in the countable case, recursive constructions suffice
to produce both ultrafilters and maximal TIFs, which are thus obtained
in a constructive manner, without any use of the axiom of choice.

\begin{proposition}\label{maximal}
Let $\B=\{B_n\mid n\in\N\}$ be a countable algebra of sets.
\begin{enumerate}
\item
Given a family $\G\subseteq\B$ with the FIP, define 
$\G_0=\G$; $\G_{n+1}=\G_n\cup\{B_n\}$
in case $B_n\cap A\ne\emptyset$ for every $A\in\G_n$; and
$\G_{n+1}=\G_n$ otherwise. Then 
$\U:=\bigcup_n\G_n$ is an ultrafilter that extends $\G$.
\item
Given a family $\G\subseteq\B$ with the FIP
and such that $A\in\G\Rightarrow A-1\in\G$, define 
$\G_0=\G$; $\G_{n+1}=\G_n\cup\{B_n-k\mid k\in\N_0\}$
in case that union has the FIP; and
$\G_{n+1}=\G_n$ otherwise. Then 
$\mathcal{M}:=\bigcup_n\G_n$ is a maximal TIF
that extends $\G$.
\end{enumerate}
\end{proposition}

\begin{proof}
(1). By the definition, it is clear that all families $\G_n$ have the FIP, and 
so also $\U$ has the FIP. Now assume by contradiction that $A\in\B$ is such that
both $A,A^c\notin\U$. If $A=B_n$ and $A^c=B_m$, then, by the definition
of $\U$, there exist $U\in\G_m$ and $U'\in\G_m$ such that
$A\cap U=A^c\cap U'=\emptyset$. 
But then $U\cap U'\subseteq(A\cap U)\cup(A^c\cap U')=\emptyset$,
against the FIP of $\U$.
Finally, if $B\supseteq A$ where $\B\in\B$ and $A\in\U$ then
$B\in\U$, as otherwise, by what just proved, $B^c\in\U$,
and hence $\emptyset=B^c\cap A\in\U$, a contradiction.

(2). By induction, it directly follows from the definition that all families 
$\G_n$ have the FIP and satisfy the property 
``$A\in\G_n\Rightarrow A-1\in\G_n$";
so, the same properties hold for $\mathcal{M}$.
If $B\supseteq A$ where $\B\in\B$ and $A\in\mathcal{M}$, say $A\in\G_n$,
then $B\in\G_{n+1}$ because trivially the FIP of $\G_n\cup\{A-n\mid n\in\N_0\}$
implies the FIP of $\G_n\cup\{B-n\mid n\in\N_0\}$. This shows that $\mathcal{M}$ is a TIF.
As for the maximality, let $\mathcal{M}'\supseteq\mathcal{M}$ be a TIF. 
Given $A\in\mathcal{M}'$, pick $n$ with $A=B_n$. The family 
$\G_n\cup\{A-n\mid n\in\N_0\}$ has the FIP, since it
is included in the filter $\mathcal{M}'$, and so $A\in\G_{n+1}$.
This shows that $\mathcal{M}'\subseteq\mathcal{M}$, and hence the
two TIFs are equal.
\end{proof}


Two properties of maximal TIFs that will be relevant to our
purposes are the following.

\begin{proposition}\label{syndetic}
Let $\U$ be an ultrafilter that includes a maximal TIF $\mathcal{M}$.
Then:
\begin{enumerate}
\item
Every $B\in\U$ is piecewise syndetic.
\item
For every $B\in\U$, the set $B_\U:=\{n\in\N\mid B-n\in\U\}$ is syndetic.\footnote
{~We remark that in general the set $B_\U$ does not belong to the algebra 
of sets $\B$.}

\end{enumerate}
\end{proposition}

\begin{proof}
Notice first that for every $B\in\U$ there exist $n_1,\ldots,n_k$ such that 
the union $\bigcup_{i=1}^k(B-n_i)\in\mathcal{M}$. Indeed, if
$\Lambda:=\{B^c-n\mid n\in\N_0\}$ then the union $\mathcal{M}\cup\Lambda$
does not have the FIP, as otherwise $\mathcal{M}\cup\Lambda$ would generate
a TIF that properly extends $\mathcal{M}$, against the maximality. 
So, there exist $A\in\mathcal{M}$ and
$n_1,\ldots,n_k$ such that $A\cap\bigcap_{i=1}^k(B^c-n_i)=\emptyset$.
But then $\bigcup_{i=1}^k(B-n_i)\in\mathcal{M}$, as a superset of $A\in\mathcal{M}$.

(1). Pick a finite union of shifts 
$\bigcup_{i=1}^k(B-n_i)\in\mathcal{M}$. By Proposition \ref{thickTIF},
that union is thick as an element of a TIF, and hence $B$ is piecewise syndetic.

(2). As above, pick a finite union of shifts 
$\bigcup_{i=1}^k(B-n_i)\in\mathcal{M}$.
By translation invariance, for every $m\in\N$ one has that
$\bigcup_{i=1}^k(B-n_i-m)\in\mathcal{M}\subseteq\U$
and so, by the Ramsey property of ultrafilters, there exists $i$ such that
$B-n_i-m\in\U$, that is, $m\in B_\U-n_i$.
This shows that $\N=\bigcup_{i=1}^k(B_\U-n_i)$ is a finite union of shifts of $B_\U$,
and hence $B_\U$ is syndetic.
\end{proof}


\section{A strong version of van der Waerden's Theorem}

The following property of piecewise syndetic sets 
was proved by exploiting the properties
of ultrafilters in the smallest ideal of the
right-topological semigroup $(\beta\N,\oplus)$
(see \cite{fg,bh}).

\begin{theorem}\label{main}
Let $A$ be a piecewise syndetic set. Then for every $k\in\N$, the set 
$\text{AP}_k(A):=
\{x\in A\mid \exists y\in\N\ \text{s.t.}\ x+i y\in A\ \text{for}\ i=1,\ldots,k\}$
is piecewise syndetic.
\end{theorem}

Notice that, as a straight consequence, one obtains the following strong version
of van der Waerden's Theorem.

\begin{corollary}
In every finite partition $\N=C_1\cup\ldots\cup C_r$
there exists a piece $C=C_i$ such that,
for every $k\in\N$, the set 
$\text{AP}_k(C)$ is piecewise syndetic.
\end{corollary}

\begin{proof}
By the Ramsey property of piecewise syndetic sets 
(see Proposition \ref{psRamsey}),
we can pick a color $C_i$ which is piecewise syndetic.
\end{proof}

In this section we will give a new proof of the above theorem
which only relies on the existence of a maximal
TIF $\mathcal{M}$ that contains $A$, and of an ultrafilter $\U\supseteq\mathcal{M}$.

\begin{proof}[Proof of Theorem \ref{main}]
Let $\mathcal{B}$ be the countable algebra of sets
generated by the family $\{A-n\mid n\in\N_0\}$.
By the property of piecewise syndeticity, 
a finite union of shifts $T=\bigcup_{i=1}^k(A-n_i)$ is thick.
Then the family $\G:=\{T-n\mid n\in\N_0\}\subseteq\B$ has the FIP,
and since $\G$ trivially satisfies the shift invariance
property ``$A\in\G\Rightarrow A-1\in\G$", by Proposition \ref{maximal}
we can pick a maximal TIF $\mathcal{M}$ on $\B$ with $\mathcal{M}\supseteq\G$,
and an ultrafilter $\U$ on $\B$ with $\U\supseteq\mathcal{M}$.
The desired result is a consequence of the following general property.

\smallskip
\textbf{Claim.}
\emph{Let $\U$ be an ultrafilter that extends a maximal TIF.
If a shift $B-\ell\in\U$ for some $\ell\in\N_0$,
then $B_\U-\ell$ contains arbitrarily long arithmetic progressions.}

\smallskip
Indeed, since $T=\bigcup_{j=1}^k(A-n_j)\in\F\subseteq\U$, there
exists $n_j$ such that $A-n_j\in\U$.
Then for every $k\in\N$ there
exists an arithmetic progression $\ell+i y\in A_\U-n_j$ for $i=0,1,\ldots,k$.
But then $\bigcap_{i=0}^k(A-n_j-\ell-iy)\subseteq\text{AP}_k(A)-n_j-\ell\in\U$.
In consequence, $\text{AP}_k(A)$ is piecewise syndetic by Proposition \ref{syndetic}, 
as a shift of a member of $\U$.

\smallskip
We are left to prove the Claim.
We proceed by induction on $k$, and prove that if $B-\ell\in\U$ for some 
$\ell\in\N_0$, then $B_\U-\ell$ contains a $k$-term arithmetic progression.\footnote
{~This inductive construction uses a simplified version of an argument 
in \cite{h}.}

If $B-\ell\in\U$, then the set $(B-\ell)_\U=B_\U-\ell$ is syndetic by Proposition \ref{syndetic}.
In particular, $B_\U-\ell\ne\emptyset$, and this proves the induction base $k=1$.

Let us turn to the inductive step $k+1$, and assume that $B-\ell_0\in\U$.
By syndeticity of $B_\U-\ell_0$, there exists a finite $F\subset\N_0$ such that
for every $n\in\N$ there exists $x\in F$ with $\ell_0+n+x\in B_\U$.
For convenience, let us assume that $0\in F$.
By the inductive hypothesis, there exist $\ell_1\in\N_0$ and 
$y_1\in\N$ such that $\ell_1+i y_1\in B_\U-\ell_0$ for $i=1,\ldots,k$, that is,
$\ell_0+\ell_1+x_0+i y_1\in B_\U$ where $x_0=0\in F$.
Pick $x_1\in F$ with $\ell_0+\ell_1+x_1\in B_\U$.
If $x_1=x_0$ then we already
found a $(k+1)$-term arithmetic progression in $B_\U$,
as desired. Otherwise, let us consider the intersection
$$B_1=(B-x_1)\cap\bigcap_{i=1}^k(B-x_0-i y_1).$$
Since $\ell_0+\ell_1+x_1\in B_\U$ and
$\ell_0+\ell_1+x_0+i y_1\in B_\U$, the shift $B_1-\ell_0-\ell_1\in\U$
and so, by the inductive hypothesis, there exist $\ell_2\in\N_0$
and $y_2\in\N$ such that $\ell_2+i y_2\in (B_1)_\U-\ell_0-\ell_1$ for $i=1,\ldots,k$.
In consequence, $\ell_0+\ell_1+\ell_2+x_0+i(y_1+y_2)\in B_\U$
and $\ell_0+\ell_1+\ell_2+x_1+i y_2\in B_\U$ for every $i=1,\ldots,k$.
Pick $x_2\in F$ with $\ell_0+\ell_1+\ell_2+x_2\in B_\U$.
Notice that if $x_2=x_0$ or $x_2=x_1$ then we have
a $(k+1)$-term arithmetic progression in $B_\U$. Otherwise, let
us consider the intersection
$$B_2=(B-x_2)\cap\bigcap_{i=1}^k(B-x_1-iy_2)\cap
\bigcap_{i=1}^k(B-x_0-i(y_1+y_2)).$$
Since $B_2-\ell_0-\ell_1-\ell_2\in\U$, by the inductive hypothesis
we can find an arithmetic progression in $B_\U-\ell_0-\ell_1-\ell_2$
of lenght $k$. We iterate the procedure.
As the set $F$ is finite,
after finitely many steps we will find elements $x_n=x_m$ where $n>m$,
and finally obtain the following arithmetic progression of length $k+1$:
$$\ell_0+\ell_1+\ldots+\ell_n+x_n+i(y_{m+1}+\ldots+y_n)\quad
i=0,1,\ldots,k.$$
\end{proof}

\section{TIFs and left ideals in the space of ultrafilters}

The usual ultrafilter proof of van der Waerden's Theorem
(see \cite[\S 14.1]{hs})
is grounded on the existence of minimal ultrafilters, that is, 
on ultrafilters that belong to minimal left ideals of the
compact right-topological semigroup $(\beta\N,\oplus)$.
In this final section, we show how
(maximal) translation invariant filters are in fact related
to the closed (minimal) left ideals of $(\beta\N,\oplus)$. 
Let us recall here the involved notions.

The \emph{space $\beta\N$} is the topological space of all
ultrafilters $\U$ over the full algebra of sets $\B=\mathcal{P}(\N)$ where
a base of (cl)open sets is given by the family
$\{\mathcal{O}_A\mid A\subseteq\N\}$, with
$\mathcal{O}_A:=\{\U\in\beta\N\mid A\in\U\}$.
The space $\beta\N$ is Hausdorff and compact, and coincides with
the \emph{Stone-C\u{e}ch compactification} of the discrete space $\N$.

The \emph{pseudosum} $\U\oplus\V$ of ultrafilters $\U,\V\in\beta\N$ is defined by letting:
$$A\in\U\oplus\V\ \Longleftrightarrow\ \{n\in\N\mid A-n\in\V\}\in\U.$$
The operation $\oplus$ is associative (but not commutative),
and for every $\V$, the map $\U\mapsto\U\oplus\V$ is continuous.
This makes $(\beta\N,\oplus)$ a \emph{right-topological semigroup}.

A \emph{left ideal} $L\subseteq\beta\N$ is a nonempty set
such that $\V\in L$ implies $\U\oplus\V\in L$ for all $\U\in\beta\N$.
Right ideals are defined similarly.
Left ideals that are minimal with respect to inclusion are particularly
relevant objects, as they satisfy special properties. 
For instance, their union $K(\beta\N,\oplus)$
is shown to be the smallest bilater ideal (\emph{i.e.} both a left and a right ideal).
Moreover, all ultrafilters $\U$ in $K(\beta\N,\oplus)$, named \emph{minimal ultrafilters},
have the property
that every set $A\in\U$ includes arbitrarily long arithmetic progressions.\footnote
{~For all notions and basic results on the space of ultrafilters
and on its algebraic structure, including properties of the smallest
ideal $K(\beta\N,\oplus)$, we refer the reader to the book \cite{hs}.}

It is well-known that there are natural correspondences 
between families with the finite intersection
property on the full algebra $\mathcal{P}(\N)$, 
and closed nonempty subsets of $\beta\N$.
Indeed, the following properties are directly verified from the definitions.

\begin{itemize}
\item
If $\G\subseteq\mathcal{P}(\N)$ is a family with the FIP 
then $\mathfrak{C}(\G):=\{\V\in\beta\N\mid\V\supseteq\G\}$
is a nonempty closed subspace.
\item
If $X\subseteq\beta\N$ is nonempty then
$\mathfrak{F}(X):=\bigcap\{\V\mid \V\in X\}$
is a filter on $\mathcal{P}(\N)$.
\item
$\mathfrak{C}(\mathfrak{F}(X))=\overline{X}$ for every
nonempty $X\subseteq\beta\N$. 
\item
$\mathfrak{F}(\mathfrak{C}(\F))=\F$ for every filter $\F$ on $\mathcal{P}(\N)$.
\end{itemize}


\begin{proposition}
Let $\B=\mathcal{P}(\N)$ be the full algebra of all subsets of $\N$. 
If $\F$ is a TIF then 
$\mathfrak{C}(\F)$ is a closed left ideal of $(\beta\N,\oplus)$; and conversely, 
if $L$ is a left ideal of $(\beta\N,\oplus)$ then
$\mathfrak{F}(L)$ is a TIF. 
Moreover, $\mathcal{M}$ is a maximal TIF if and only if $\mathfrak{C}(\mathcal{M})$
is a minimal left ideal of $(\beta\N,\oplus)$; and
$L$ is a minimal left ideal of $(\beta\N,\oplus)$
if and only if $\mathfrak{F}(L)$ is a maximal TIF.
\end{proposition}

\begin{proof}
Let $\V\in\mathfrak{C}(\F)$ and let $\U\in\beta\N$ be any ultrafilter.
For every $A\in\F$, by translation invariance we know that $A-n\in\F$ for all $n$,
and so $\{n\mid A-n\in \V\}=\N\in\U$.
This shows that $A\in\U\oplus\V$. As this is true for every $A\in\F$,
we conclude that $\U\oplus\V\in\mathfrak{C}(\F)$, and so $\mathfrak{C}(\F)$
is a closed left ideal.

Now let $A\in\mathfrak{F}(L)$ where $L$ is a left ideal. For every $\V\in L$,
we have that $\mathfrak{U}_1\oplus\V\in L$,
where $\mathfrak{U}_1:=\{B\subseteq\N\mid 1\in B\}$ is the principal
ultrafilter generated by $1$. Then
$A\in\mathfrak{U}_1\oplus\V$, which is equivalent to $A-1\in\V$.
As this holds for every $\V\in L$, we have proved that
$A-1\in\mathfrak{F}(L)$, as desired.

Let $\F$ be a TIF. If the left ideal $\mathfrak{C}(\F)$ is not minimal,
pick a minimal $L\subsetneq\mathfrak{C}(\F)$.
Then $\F\subsetneq \mathfrak{F}(L)$, and hence $\F$ is not maximal.
Indeed, $L\subseteq \mathfrak{C}(\F)\Rightarrow \mathfrak{F}(L)\supseteq
\mathfrak{F}(\mathfrak{C}(\F))=\F$;
moreover, $\F\ne\mathfrak{F}(L)$, as otherwise
$\mathfrak{C}(\F)=\mathfrak{C}(\mathfrak{F}(L))=\overline{L}=L$,
against our assumptions.
(Recall that a minimal left ideal
$L$ is necessarily closed because,
by minimality, $L=\beta\N\oplus\V:=\{\U\oplus\V\mid \U\in\beta\N\}$ 
for every given $\V\in L$, and $\beta\N\oplus\V$ is closed as the image 
of the compact Hausdorff space $\beta\N$ under 
the continuous function $\U\mapsto\U\oplus\V$.)
In a similar way, one shows the converse implication:
If the TIF $\F$ is not maximal then the left ideal $\mathfrak{C}(\F)$ is not minimal.
In consequence, $L=\mathfrak{C}(\mathfrak{F}(L))$ is minimal
if and only if $\mathfrak{F}(L)$ is maximal, and also
the last equivalence follows.
\end{proof}


As a straight consequence, we obtain the desired characterization.

\begin{corollary}
Let $\B=\mathcal{P}(\N)$ be the full algebra of all subsets of $\N$. 
An ultrafilter $\U$ includes a maximal TIF if and only if
$\U$ belongs to the smallest ideal $K(\beta\N,\oplus)$.
\end{corollary}

\begin{proof}
Recall that $\U\in K(\beta\N,\oplus)$ if and only if
$\U$ belongs to some minimal left ideal.
Now let $\U\supseteq\mathcal{M}$ where $\mathcal{M}$ is
a maximal TIF. Since $\mathcal{M}=\mathfrak{F}(\mathfrak{C}(\mathcal{M}))$, 
we have that $\U\in\mathfrak{C}(\mathcal{M})$,
where $\mathfrak{C}(\mathcal{M})$ is a minimal left ideal.
Conversely, let $\U\in L$ where
$L$ is a minimal left ideal. Then $\mathfrak{F}(L)$ is a maximal TIF
and $\U\supseteq\mathfrak{F}(L)$, since
$\U\in L=\mathfrak{C}(\mathfrak{F}(L))$.
\end{proof}

\begin{remark}
One can generalize the contents of this paper
from the natural numbers to arbitrary countable semigroups $(S,\cdot)$.
Indeed, the notion of translation invariant filter also
makes sense in that more general framework.\footnote
{~Actually, our techniques also apply for uncountable semigroups,
but in that case one needs Zorn's Lemma
to prove the existence of maximal TIFs and of ultrafilters.}
Precisely, if $\B$ is an algebra of subsets of $S$,
one defines a TIF on $\B$ as a filter $\F$ with the
additional property that 
$A\in\F\Rightarrow s^{-1}A:=\{t\in S\mid s\cdot t\in A\}\in\F$ for every $s\in S$.
Then, by the same arguments as used in this paper,
one can prove that Theorem \ref{main} holds,
provided one adopts a suitable generalization of the
notion of piecewise syndetic set.\footnote
{~In an arbitrary semigroup $(S,\cdot)$, one
defines a subset $T\subseteq S$ to be \emph{thick} if
for every finite $F$ there exists $x\in S$ with 
$F\cdot x:=\{y\cdot x\mid y\in F\}\subseteq T$;
a set $A\subseteq S$ is \emph{syndetic} if a suitable finite union 
$\bigcup_{i=1}^k s_i^{-1}A=S$ 
covers the whole semigroup; and finally a set $A\subseteq S$ is
\emph{piecewise syndetic}
if a suitable finite union $\bigcup_{i=1}^k s_i^{-1}A$ is thick
(see \cite[\S 4.4 and \S 4.5]{hs}).
%See XXX
}
\end{remark}


\begin{thebibliography}{}


\bibitem{bh}
V. Bergelson and N. Hindman,
Partition regular structures contained in large
sets are abundant, \emph{J. Combin. Theory Ser. A} \textbf{93} (2001), 18--36.

\bibitem{fg}
H. Furstenberg and E. Glasner, Subset dynamics and van der Waerden's Theorem,
in \emph{Topological Dynamics and Applications}
(M.G. Nerurkar, D.P. Dokken, and D.B. Ellis, eds.),  
Contemp. Math. \textbf{215}, Amer. Math. Soc., 1998, 197--203. 
%Canad. J. Math. 32 (1998), 197?203.


\bibitem{h}
N. Hindman, Problems and new results in the algebra of $\beta S$
and Ramsey Theory, in
\emph{Unsolved Problems in Mathematics for the 21st Century}
(J. Abe and S. Tanaka, eds.), IOS Press, 2001, 295--305. 


\bibitem{hs}
N. Hindman and D. Strauss, \emph{Algebra in the Stone-\v{C}ech
Compactification, Theory and Applications} (2nd edition),
W. de Gruyter, 2012.


\bibitem{vdW}
B.L. van der Waerden, Beweis einer baudetschen vermutung, 
\emph{Nieuw. Arch. Wisk.} \textbf{15} (1927), 212--216.

\end{thebibliography}

\end{document}
