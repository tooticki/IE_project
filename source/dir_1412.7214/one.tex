\hyperdef\chapteRi{ChapterSection}{chapteRi}{}
\Chapter Introduction



\hyperdef\chapteRiSECi{ChapterSection}{chapteRiSECi}{}
\Section Hypergeometric terms and term ratios

Throughout this work $K$ denotes a field of characteristic zero. We write $K[z]$ for the ring
of polynomials in one variable over $K$, and $K[
\z]=K[z_1,\ldots,z_k]$ for the ring of polynomials in $k$ variables over
$K$. Similarly, we write $K(z)$ for field of rational functions in one
variable over $K$, and $K(\z)=K(z_1,\ldots,z_k)$ for the field of
rational functions in $k$ variables over $K$. 

\Definition[def:shift]  \hyperdef\defshift{Definition}{shift}{} For any function $f\colon\Z^k\to K$ and any
vector $\v \in\Z^k$, the function $f^{\v}\colon\Z^k\to K$ is defined by
$$f^{\v}(\z)=f(\z+
\v)=f(z_{1}+v_{1},\ldots,z_{k}+v_{k}).$$
Similarly, for any rational function $R\in K(\z)$ and any
vector $\v \in\Q^k$, the rational
function $R^{\v} \in K(\z)$ is defined by
$$R^{\v}(\z)=R(\z+
\v)=R(z_{1}+v_{1},\ldots,z_{k}+v_{k}).$$
\DefinitionStop

\Definition[def:B1]  \hyperdef\defBone{Definition}{B1}{} A \defword{hypergeometric term} on $\Z^k$
over a field $K$ is
a function $f\colon\Z^k\to K$ such that for $i\in\{1,\ldots,k\}$ there
exist nonzero polynomials $A_1,\ldots,A_k$ and $B_1,\ldots,B_k\in K[\z]$
such that
$$A_if=B_if^{\e_i}.$$
Of course, this last equation is equivalent to $A_i(\z)f(
\z)=B_i(\z)f(\z+\e_i)$ for all $\z\in\Z^k$.
\DefinitionStop

For any \hgt\ $f$ on $\Z^k$ and any $\v\in\Z^k$, a rational
function $R_{\v}\in K(\z)$ is a \defword{term ratio in the
direction} $\v$ if
$$R_{\v}=\frac{A_{\v}}{B_{\v}}$$
for some nonzero polynomials $A_{\v},B_{\v}\in K[\v]$ such that
$A_{\v}f=B_{\v}f^{\v}$.

\Definition[def:B2]  \hyperdef\defBtwo{Definition}{B2}{} A function $f$ on $\Z^k$ is a \defword{zero
divisor} if there exists a nonzero polynomial $p\in K[\z]$ such that
$pf=0$.
\DefinitionStop

We show in Lemma~\hyperref\lemBsix{\ref[lem:B6]} that if a \hgt\ $f$ is not a \zd, then
for each $\v\in\Z^k$ there exists a unique term ratio $R_{\v}$
in the direction $\v$. Furthermore, we show in Lemma~\hyperref\lemBseven{\ref[lem:B7]}
that if $f$ is not a \zd\ then the term ratios satisfy
$$R_{\v}R_{\w}^{\v}=R_{\w}R_{\v}^{\w}$$
for all $\w,\v\in\Z^k$. In particular, letting $R_i=R_{
\e_i}$, we have $R_iR_j^{\e_i}=R_jR_i^{\e_j}$ for all
$i,j\in\{1,\ldots,k\}$
(cf.\ \hyperref\WZninety{\bibref[WZ90]}).
 This is the relation for the term ratios of a
\hgt. The first step in understanding the structure of a \hgt\ is to
understand the structure of rational solutions of this system of
equations.

\hyperdef\chapteRiSECii{ChapterSection}{chapteRiSECii}{}
\Section The rational solutions of $R_iR_j^{\e_i}=R_jR_i^{\e_j}$

It will be useful to define
$$\gp iabA_i=\cases
\prod_{i=a}^{b-1}A_i&\text{if $b>a$}\\
1&\text{if $b=a$}\\
1\big/\prod_{i=b}^{a-1}A_i&\text{if $a>b$.}
\endcases$$
The usefulness of the notation stems in part from the fact that
$$\gp iabA_i\,\gp ibcA_i=\gp iacA_i$$
and
$$\gp iabA_i=\gp iba\frac1{A_i},$$
which is easily verified.

In Chapter~2, we prove Theorem~\hyperref\thmfifteena{\ref[thm:15a]} \hyperref\oresato{\OreSato} which completely describes
the rational solutions of the system of equations
$$R_iR_j^{\e_i}=R_jR_i^{\e_j} \quad i,j \in \{1,\ldots,k\}.$$ 

\smallskip
\Statement[thm:15a] 
\noindent{\bf{}Theorem~\hyperref\thmfifteena{\ref[thm:15a]}} \ 
Let $R_i\in K(\z)$, $i=1,\ldots,k$, be rational functions such that
$$R_iR_j^{\e_i}=R_jR_i^{\e_j}
\text{\quad 
for all $i,j\in\{1,\ldots,k\}$.
}
$$
Then there exist polynomials $C$ and
$D\in K[\z]$, a finite set $V\subset\Z^k$, and univariate
polynomials $a_{\v}$ and $b_{\v}\in K[z]$ for each $\v\in
V$ such that for all $i\in\{1,\ldots,k\}$,
$$R_i(\z)=\frac{C(\z+\e_i)}{C(\z)}\frac{D(
\z)}{D(\z+\e_i)}\prod_{\v\in V}\gp j0{\v_i}\frac{a_{
\v}(\z\cdot\v+j)}{b_{\v}(\z\cdot\v+j)}.$$
\StatementStop

The proof of Theorem~\hyperref\thmfifteena{\ref[thm:15a]} is quite involved.
The first part of the proof is showing that the solution can be
expressed in the form
$$R_i=\frac{C^{\e_i}}{C}\frac{D}{D^{\e_i}}\frac{A_i}{B_i},$$
$i=1,\ldots,k$, where $C$ and $D$ are polynomials (not depending on
$i$) and $A_i$ and $B_i$ are products of {\sl simple} polynomials. 
A simple
polynomial is a composition of a univariate polynomial and a linear
polynomial $\v\cdot\z$. The one-dimensional case is
trivial. By using Gosper's Lemma~\hyperref\Gospseventyeight{\bibref[Gosp78]}, we reduce the
two-dimensional case to the following lemma. Let $A_1$, $A_2$, $B_1$,
and $B_2\in K[z_1,z_2]$. If $A_1$ and $B_1^{n\e_1}$ are
relatively prime for all $n\in\Z$, $A_2$ and $B_2$ are relatively
prime, and $R_1=A_1/B_1$ and $R_2=A_2/B_2$ satisfy $R_1R_2^{
\e_1}=R_2R_1^{\e_2}$, then the irreducible divisors of $A_1$, $B_1$,
$A_2$, and $B_2$ are simple. An interesting feature of the proof of the
lemma is the association of a lattice path with an irreducible divisor
of $A_1$, $B_1$, $A_2$, or $B_2$. 
(
A \defword{lattice path in $\Z^k$} is a sequence 
$\{T_i\}_{i\ge1}$ in $\Z^k$ such that
$T_i-T_{i-1}\in \{\pm \e_1,\ldots,\pm\e_k\}$ for all
$i>1$.
)
The primeness conditions of the lemma
impose restrictions of the shape of the lattice path, from which we
deduce that the path is unbounded. From the unboundedness of the
path, it follows that $d$ satisfies a nontrivial relation $d=d^{\v}$,
and from this it follows that $d$ is simple.
We derive the higher-dimensional cases from the two-dimensional case
with the aid of the notions of the {\sl rational Galois space} of a rational
function and the {\sl fixed factor} of a rational function for a subspace. 

The second part of the 
proof of Theorem~\hyperref\thmfifteena{\ref[thm:15a]} is deriving the formula for $R_i$
given the simplicity of the divisors of the $A_i$ and $B_i$.
A crucial step is observing that the
multiplicative components of the $R_i$ determined by rational Galois space
still satisfy the relation for the term ratios: if $s$ is a subspace of
$\Q^k$ and $R_{i,s}=\fix_sR_i$, then
$$R_{i,s}R_{j,s}^{\e_i}=R_{j,s}R_{i,s}^{\e_j}.$$

\hyperdef\chapteRiSECiii{ChapterSection}{chapteRiSECiii}{}
\Section The multiplicative structure of a hypergeometric term

Define a \defword{hyperplane} in $\Z^k$ to be a set of the form $\{
\z\colon\v\cdot\z=n\}$, where $n\in\Z$ and $\v$ is some vector in $\Z^k$.
Define a \defword{half-space} in $\Z^k$ to be a set of the form $\{
\z\colon\v\cdot\z>n\}$, where $n\in\Z$ and $\v$ is some vector in
$\Z^k$.
A \defword{set of measure zero} in $\Z^k$ is a set that can be covered
by a finite number of hyperplanes. Two functions $f,g\colon\Z^k\to K$
are \defword{equal almost everywhere} (written $f=g$~a.e.) if there
exists a set $S$ of measure zero such that $f(\z)=g(\z)$ for
all $\z\in\Z^k\setminus S$.
A region $\R\subset\Z^k$ is \defword{polyhedral} if 
$R=\Z^k$ or $R$ is the
intersection of a finite number of half-spaces. 


It is easily seen
(Lemma~\hyperref\lemBthirteen{\ref[lem:B13]}) that a function $f$ is a \hgt\ on $\Z^k$ if and only if,
for all $\v \in \Z^k$, there exist nonzero polynomials $A_{\v} $ and
$B_{\v}\in K[\z]$ such that
$A_{\v}f=B_{\v}f^{\v}$~a.e.
A \hgt\ $f$ on $\Z^k$ is \defword{honest}
if for all $\v\in\Z^k$ there exist {\sl relatively prime}
polynomials $A_{\v}$ and $B_{\v}\in K[\z]$ such that
$A_{\v}f=B_{\v}f^{\v}$~a.e.

The main results of Chapter 3  are Theorem~\hyperref\thmBtwentynine{\ref[thm:B29]} and its
corollary for
algebraically closed fields, Corollary~\hyperref\corBthirtyone{\ref[cor:B31]}.

\smallskip

\Statement[thm:B29] 
\noindent{\bf Theorem~\hyperref\thmBtwentynine{\ref[thm:B29]}} \ Let $f$ be an \hyperref\defBthirteena{honest} \hgt\ on
$\Z^k$. There exist relatively prime polynomials $C$ and $D\in K[\z]$, a finite
set $V\subset\Z^k$, univariate polynomials $a_{\v},b_{\v}\in
K[z]$, for each $\v\in V$, and a finite number of \hyperref\defBeighteen{polyhedral regions}
$\R_1,\ldots,\R_m$ such that
\Enumerate
\item $\Z^k$ is the disjoint union of the $\R_i$ and a \hyperref\defBeleven{set of measure
zero};
\item for each $i\in\{1,\ldots,m\}$ there exists $\z_0\in \R_i$ such
that $C(\z_0)\ne0$, and for all $\z\in \R_i$ for which $D(
\z)\ne0$,
$$f(\z)=f(\z_0)\frac{C(\z)}{C(\z_0)}\frac{D(
\z_0)}{D(\z)}\prod_{\v\in V}\gp j{\z_0\cdot\v}{
\z\cdot\v}\frac{a_{\v}(j)}{b_{\v}(j)}.$$
\item  all the terms $a_{\v}(j)$ and $b_{\v}(j)$
occurring in the product are nonzero.
\EnumerateStop
\StatementStop

The \hgt s that occur in practice can be expressed as products of
Pochhammer symbols, so the question arises: Is this true in general?
Corollary~\hyperref\corBthirtyone{\ref[cor:B31]} show that if the field $K$ is algebraically
closed and the \hgt\ is honest, then the answer is yes, at least
piecewise.

\smallskip

\Statement[cor:B31] 
\noindent{\bf Corollary~\hyperref\corBthirtyone{\ref[cor:B31]}} \ Let $f$ be an \hyperref\defBthirteena{honest}
\hgt\ on $\Z^k$ over a field $K$ that is algebraically closed. Then there exist
relatively prime polynomials $C$ and $D\in K[\z]$ and a finite number of \hyperref\defBeighteen{polyhedral regions}
$\R_1,\ldots,\R_L$ such that $\Z^k$ is the union of the $\R_{\ell}$ and a
\hyperref\defBeleven{set of measure zero}, and for each region $\R_{\ell}$ there exist a
vector $\vecgamma\in K^k$, constants $m_1,\ldots,m_p,n_1,\ldots,n_q\in
K$, vectors $\v_1,\ldots,\v_p,\w_1,\ldots,\w_q\in\Z^k$,
and integers $r_1,\ldots,r_p,s_1,\ldots,s_q$ such that
\Enumerate
\item for all $\z\in \R_{\ell}$ such that $D(\z)\ne0$,
$$f(\z)=
\gamma_{1}^{z_{1}}\cdots \gamma_{k}^{z_{k}}
\frac{C(\z)}{D(\z)}
\frac{\prod_{i=1}^p(m_i)_{\v_i\cdot
\z+r_i}}{\prod_{j=1}^q(n_j)_{\w_j\cdot\z+s_j}};$$
\item for all $i$ and $j$ and all $\z\in \R_{\ell}$, $\v_i\cdot
\z+r_i$ and $\w_j\cdot\z+s_j$ are positive;
\item the Pochhammer symbols occurring
in the products are nonzero.
\EnumerateStop
\StatementStop

If we assume that everything in sight is nonzero,
then the expression for $f$ in Theorem~\hyperref\thmBtwentynine{\ref[thm:B29]} follows by
induction from the expression for the term ratios in
Theorem~\hyperref\thmfifteena{\ref[thm:15a]}.
Unfortunately, zeros of $C$, $D$ and the $a_{\v}$ and $b_{\v}$
are
deadly to the induction.
The polyhedral regions $\R_i$
arise as regions for which the nonzero points of these polynomials
have a certain connectedness property.


\hyperdef\chapteRiSECiv{ChapterSection}{chapteRiSECiv}{}
\Section A holonomic \hgt\ is piecewise proper

We prove the following result which settles the discrete part of
Wilf and Zeilberger's conjecture that a \hgt\  is holonomic if and only
if
it is proper.

\smallskip

\Statement[thm:C11] 
\noindent{\bf Theorem~\hyperref\thmCeleven{\ref[thm:C11]}} \ A holonomic \hgt\  $f$ on $\Z^k$ over an
algebraically closed field $K$ is \hyperref\defpwp{piecewise proper};
there exist a polynomial $C\in K[\z]$ 
and a finite number of \hyperref\defBeighteen{polyhedral
regions} $\R_1,\ldots,\R_L$ such that $\Z^k$ is the union of the
$\R_{\ell}$ and a \hyperref\defBeleven{set of measure zero}, and for each region $\R_{\ell}$
there exist a vector $\vecgamma\in K^k$, constants
$m_1,\ldots,m_p,n_1,\ldots,n_q\in K$, vectors $\v_1,\ldots,
\v_p,\w_1,\ldots,\w_q\in\Z^k$, and integers
$r_1,\ldots,r_p,s_1,\ldots,s_q$ such that
\Enumerate
\item for all $\z\in \R_{\ell}$,
$$f(\z)=
\gamma_{1}^{z_{1}}\cdots \gamma_{k}^{z_{k}}
C(\z)
\frac{\prod_{i=1}^p(m_i)_{\v_i\cdot
\z+r_i}}{\prod_{j=1}^q(n_j)_{\w_j\cdot\z+s_j}};$$
\item for all
$i$ and $j$ and all $\z\in \R_{\ell}$, $\v_i\cdot\z+r_i$ and
$\w_j\cdot\z+s_j$ are positive;
\item the Pochhammer symbols occurring in the products are nonzero.
\EnumerateStop
\StatementStop

Theorem~\hyperref\thmCeleven{\ref[thm:C11]} applies to \hgt s on polyhedral regions other than $\Z^k$ via Lemma~\hyperref\lemBtwentyoneb{\ref[lem:B21b]}.




The proof of Theorem~\hyperref\thmCeleven{\ref[thm:C11]} uses the results of Chapter 3 and
only elementary facts about holonomic functions
that can be found in \hyperref\Zeilninety{\bibref[Zeil90]} and \hyperref\Bjoone{\bibref[Bjo1]}.

\hyperdef\chapteRiSECv{ChapterSection}{chapteRiSECv}{}
\Section A Solution to a Problem of Cameron
On Sum-free Complete Sets

For any subsets $A$ and $B$ of an additive group $G$, define
$A+B=\{a+b:a \in A \text{ and } b \in B \}$ and $-A= \{-a:a
\in A \}$.  A subset $S$ of $G$ is said to be sum-free,
complete, and symmetric respectively if $S+S \subset S^\c$,
$S+S \supset S^\c$, and $S=-S$.  Hence, $S$ is sum-free and
complete if and only if $S+S=S^\c$.


Cameron observed that for any sufficiently small modulus $m$,
every sum-free complete set in $\Z/m\Z$ is also symmetric.
In fact, Calkin found that $m=36$ is the smallest modulus  for which
there is a sum-free complete set that is not
symmetric \hyperref\CamPortrait{\bibref[CamPortrait]}.  Cameron asked if there exists
such a set for all sufficiently large moduli \hyperref\CamPortrait{\bibref[CamPortrait]}.
We answer Cameron's question by showing
there exists such a set for all moduli greater than or
equal to $890626$.

We also show that every sum-free complete set in $\Z/m\Z$ that
is not symmetric can be used to construct a counterexample
to a conjecture of J.H.~Conway.
 Conway conjectured that for any finite set
$S$ of integers, $|S+S| \le |S-S|$.
Conway's conjecture
was disproved by Marica \hyperref\Marica{\bibref[Marica]}.  Later Stein showed
how to make the ratio ${|S+S|}/{|S-S|}$ arbitrarily large
\hyperref\Stein{\bibref[Stein]}.  We show that if $S$ is sum-free and complete
modulo $m$ but not symmetric, then $|S+S|>|S-S|$; hence, $S$
is a counterexample to a modular version of Conway's conjecture.
Further, we show that if $S' \subset \Z$ is a certain set
derived from $S$, then $|S'+S'| > |S'-S'|$; hence, $S'$ is a
counterexample to Conway's conjecture proper.

The history of sum-free sets begins with Schur who
showed that the positive integers can not be partitioned
into finitely many sum-free sets \hyperref\Schur{\bibref[Schur]}.  Sum-free sets
have been used to find lower bounds for Ramsey numbers
(see pp.~28, 128, 264 in~\hyperref\Wallis{\bibref[Wallis]}).  Cameron describes  some
applications of sum-free sets and poses several problems
\hyperref\CamGraph{\bibref[CamGraph]},\hyperref\CamStructure{\bibref[CamStructure]},\hyperref\CamPortrait{\bibref[CamPortrait]}.  George Andrews
observed that sum-free complete
sets play a role in partition identities (personal communication).
For example, the
set $\{1,4\} \in \Z/5\Z$, which arises in the Rogers-Ramanujan
Identities (p.~109 in~\hyperref\Andseventysix{\bibref[And76]}), is sum-free, complete,
and symmetric.  Calkin showed that the number of sum-free
sets contained within the first $n$ integers is $o(2^{n(1/2
+\epsilon)})$ for every $\epsilon > 0$ \hyperref\Calkin{\bibref[Calkin]}.



