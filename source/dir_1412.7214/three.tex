
\hyperdef\chapteRiii{ChapterSection}{chapteRiii}{}
\Chapter The Multiplicative Structure\\ 
of a Hypergeometric Term

The main results of  this chapter are Theorem~\hyperref\thmBtwentynine{\ref[thm:B29]} and its
corollary for
algebraically closed fields, Corollary~\hyperref\corBthirtyone{\ref[cor:B31]}.

\smallskip

\Statement[thm:B29] 
\noindent{\bf Theorem~\hyperref\thmBtwentynine{\ref[thm:B29]}} \ Let $f$ be an \hyperref\defBthirteena{honest} \hgt\ on
$\Z^k$. There exist relatively prime polynomials $C$ and $D\in K[\z]$, a finite
set $V\subset\Z^k$, univariate polynomials $a_{\v},b_{\v}\in
K[z]$, $\v\in V$ (all of which can be determined as in
Theorem~\hyperref\thmBnine{\ref[thm:B9]}), and a finite number of \hyperref\defBeighteen{polyhedral regions}
$\R_1,\ldots,\R_m$ such that
\Enumerate
\item $\Z^k$ is the disjoint union of the $\R_i$ and a \hyperref\defBeleven{set of measure
zero};
\item for each $i\in\{1,\ldots,m\}$ there exists $\z_0\in \R_i$ such
that $C(\z_0)\ne0$, and for all $\z\in \R_i$ for which $D(
\z)\ne0$,
$$f(\z)=f(\z_0)\frac{C(\z)}{C(\z_0)}\frac{D(
\z_0)}{D(\z)}\prod_{\v\in V}\gp j{\z_0\cdot\v}{
\z\cdot\v}\frac{a_{\v}(j)}{b_{\v}(j)}.$$
\item  all the terms $a_{\v}(j)$ and $b_{\v}(j)$
occurring in the product are nonzero.
\EnumerateStop
\StatementStop

The \hgt s that occur in practice can be expressed as products of
Pochhammer symbols, so the question arises: Is this true in general?
Corollary~\hyperref\corBthirtyone{\ref[cor:B31]} show that if the field $K$ is algebraically
closed and the \hgt\ is honest, then the answer is yes, at least
piecewise. \hyperdef\Honestpw{Remark}{honesthtgpiecewise}{}

\smallskip

\Statement[cor:B31] 
\noindent{\bf Corollary~\hyperref\corBthirtyone{\ref[cor:B31]}} \ Let $f$ be an \hyperref\defBthirteena{honest}
\hgt\ on $\Z^k$ over a field $K$ that is algebraically closed. Then there exist
relatively prime polynomials $C$ and $D\in K[\z]$ (as in
Theorem~\hyperref\thmBtwentynine{\ref[thm:B29]}) and a finite number of \hyperref\defBeighteen{polyhedral regions}
$\R_1,\ldots,\R_L$ such that $\Z^k$ is the union of the $\R_{\ell}$ and a
\hyperref\defBeleven{set of measure zero}, and for each region $\R_{\ell}$ there exist a
vector $\vecgamma\in K^k$, constants $m_1,\ldots,m_p,n_1,\ldots,n_q\in
K$, vectors $\v_1,\ldots,\v_p,\w_1,\ldots,\w_q\in\Z^k$,
and integers $r_1,\ldots,r_p,s_1,\ldots,s_q$ such that
\Enumerate
\item for all $\z\in \R_{\ell}$ such that $D(\z)\ne0$,
$$f(\z)=
\gamma_{1}^{z_{1}}\cdots \gamma_{k}^{z_{k}}
\frac{C(\z)}{D(\z)}
\frac{\prod_{i=1}^p(m_i)_{\v_i\cdot
\z+r_i}}{\prod_{j=1}^q(n_j)_{\w_j\cdot\z+s_j}};$$
\item for all $i$ and $j$ and $\z\in \R_{\ell}$, $\v_i\cdot
\z+r_i$ and $\w_j\cdot\z+s_j$ are positive;
\item the Pochhammer symbols occurring
in the products are nonzero.
\EnumerateStop
\StatementStop


\hyperdef\chapteRiiiSECi{ChapterSection}{chapteRiiiSECi}{}
\Section Hypergeometric terms and term ratios

\Definition[def:B1]  \hyperdef\defBone{Definition}{B1}{} A \defword{hypergeometric term} on $\Z^k$ over a
field $K$ is a function $f\colon\Z^k\to K$ such that for
$i\in\{1,\ldots,k\}$ there exist
nonzero polynomials $A_1,\ldots,A_k$ and $B_1,\ldots,B_k\in K[\z]$
such that
$$A_i(\z)f(\z)=B_i(\z)f(\z+\e_i)$$
for every $\z\in\Z^k$.
\DefinitionStop

Unfortunately, the definition of a \hgt\ includes some pathological
functions. For example, if $f$ is any function $f\colon\Z^k\to K$ that
is supported by the set of zeros of a nonzero polynomial $p$, then
$p(\z)f(\z)=0$ for all $\z$, that is, $pf=0$.  It follows
that $p^{\e_i}f^{\e_i}=0$ for $i\in\{1,\ldots,k\}$, and hence
$pf=p^{\e_i}f^{\e_i}$ for $i\in\{1,\ldots,k\}$. By
Definition~\hyperref\defBone{\ref[def:B1]}, $f(\z)$ is a \hgt. Such a \hgt\ is called
a \defword{zero divisor}.


\Definition[def:B2]  \hyperdef\defBtwo{Definition}{B2}{} A function $f$ on $\Z^k$ is a \defword{zero
divisor} if there exists a nonzero polynomial $p\in K[\z]$ such
that $pf=0$.
\DefinitionStop

\Lemma[lem:B3]  \hyperdef\lemBthree{Lemma}{B3}{} If $f$ is a \hgt\ on $\Z^k$, then for all $
\v\in\Z^k$, there exist nonzero polynomials $A_{\v}$ and $B_{
\v}\in K[\z]$ such
$$A_{\v}f=B_{\v}f^{\v}.$$
\LemmaStop

\Proof
We prove the lemma by induction. By the definition (\hyperref\defBone{\ref[def:B1]}) of
\hgt, the statement of the lemma is true if $\v=\e_i$,
$i\in\{1,\ldots,k\}$, and clearly the statement is true if $\v=\vec0$.
We complete the induction by showing that
if the statement is true for $\v=\u$ and $
\v=\w$, then it is true for $\v=\u+\w$ and $\v=
\u-\w$. Taking $A_{\u+\w}=A_{\u}A_{\w}^{\u}$ and
$B_{\u+\w}=B_{\u}B_{\w}^{\u}$, we have $A_{
\u+\w}f=A_{\u}A_{\w}^{\u}f=A_{\w}^{\u}B_{
\v}f^{\u}=B_{\u}(A_{\w}f)^{\u}=B_{\u}(B_{
\w}f^{\w})^{\u}=B_{\u}B_{\w}^{\u}f^{\w+
\u}=B_{\u+\w}f^{\u+\w}$. 
Similarly, taking $A_{\u-\w}=A_{\u}B_{\w}^{
\u-\w}$ and $B_{\u-\w}=B_{\u}A_{\w}^{\u-
\w}$, $A_{\u-\w}f=A_{\u}B_{\w}^{\u-\v}f=B_{
\w}^{\u-\w}B_{\u}f^{\u}=B_{\u}(B_{\w}f^{
\w})^{\u-\w}=B_{\u}(A_{\w}f)^{\u-\w}=B_{
\u}A_{\w}^{\u-\w}f^{\u-\w}=B_{\u-\w}f^{
\u-\w}$.
\ProofStop




\Definition[def:B4]  \hyperdef\defBfour{Definition}{B4}{} For any \hgt\ $f$ on $\Z^k$ and any $
\v\in\Z^k$, a rational function $R_{\v}\in K(\z)$ is a
\defword{term ratio in the direction} $\v$ if
$$R_{\v}=\frac{A_{\v}}{B_{\v}}$$
for some nonzero polynomials $A_{\v},B_{\v}\in K[\z]$ such
that $A_{\v}f=B_{\v}f^{\v}$.
\DefinitionStop

Thus, by Lemma~\hyperref\lemBthree{\ref[lem:B3]}, a term ratio exists for each $\v$. Of
course, we use the term {\sl term ratio} because
$$\frac{f^{\v}(\z)}{f(\z)}=\frac{A_{\v}(
\z)}{B_{\v}(\z)}=R_{\v}(\z)$$
for all $\z$ such that $f(\z)\ne0$ and $B_{\v}(
\z)\ne0$. The following lemma shows that if $f$ is not a \zd\ then the
term ratio for $f$ in the direction $\v$ is unique.

\Lemma[lem:B6]  \hyperdef\lemBsix{Lemma}{B6}{} If a \hgt\ $f$ on $\Z^k$ is not a \hyperref\defBtwo{\zd}, then for each
$\v\in\Z^k$ there is a unique \hyperref\defBfour{term ratio in the direction $\v$.}
\LemmaStop

\Proof Let $R_{\v}$ and $\bar R_{\v}$ be two term ratios for
$f$ in the direction $\v$. Then
$$R_{\v}=\frac{A_{\v}}{B_{\v}}\text{ and }\bar R_{
\v}=\frac{\bar A_{\v}}{\bar B_{\v}}$$
where $A_{\v}f=B_{\v}f^{\v}$ and $\bar A_{\v}f=\bar
B_{\v}f^{\v}$. From the last two equations it follows that
$(\bar B_{\v}A_{\v}-B_{\v}\bar A_{\v})f=0$. Since $f$
is not a \zd, this implies that $\bar B_{\v}A_{\v}-B_{
\v}\bar A_{\v}=0$, which implies that $R_{\v}=\bar R_{\v}$.
\ProofStop

\Lemma[lem:B7]  \hyperdef\lemBseven{Lemma}{B7}{} Let $f$ be a \hgt\ on $\Z^k$ that is not a \hyperref\defBtwo{\zd}\ and let $R_{
\v}$ be the \hyperref\defBfour{term ratio} for $f$ in the direction $\v$ (which is
unique by Lemma~\hyperref\lemBsix{\ref[lem:B6]}). Then
$$R_{\v+\w}=R_{\v}R_{\w}^{\v}=R_{\w}R_{
\v}^{\w}$$
for all $\v,\w\in\Z^k$.
\LemmaStop

\Proof For each $\v\in\Z^k$, let $A_{\v}$ and $B_{\v}$ by
polynomials such that $A_{\v}f=B_{\v}f^{\v}$ as in
Lemma~\hyperref\lemBthree{\ref[lem:B3]}. Then $A_{\v}A_{\w}^{\v}f=A_{
\w}B_{\v}f^{\v}=B_{\v}(A_{\w}f)^{\v}=B_{
\v}(B_{\w}f^{\v})=B_{\v}B_{\w}^{\v}f^{\v+
\w}$. Thus, we have
$$A_{\v+\w}f
=B_{\v+\w}f^{\v+\w}$$
and
$$A_{\v}A_{\w}^{\v}f
=B_{\v}B_{\w}^{\v}f^{\v+\w}.$$
It follows from these equations that
$$(A_{\v+\w}B_{\v}B_{\w}^{\v}-B_{\v+
\w}A_{\v}A_{\w}^{\v})f=0.$$
Since $f$ is not a \zd, this implies that $A_{\v+\w}B_{
\v}B_{\w}^{\v}-B_{\v+\w}A_{\v}A_{\w}^{
\v}=0$, from which it follows that $R_{\v+\w}=R_{\v}R_{
\w}^{\v}$. By symmetry, $R_{\v+\w}=R_{\w}R_{
\v}^{\w}$.
\ProofStop

\Proposition[prop:B8]  \hyperdef\propBeight{Proposition}{B8}{}
If a \hgt\ $f$ on $\Z^k$ is a \hyperref\defBtwo{\zd}, then 
for each 
$\v\in \Z^k$
there exist nonzero 
$A_{\v}$
and $B_{\v}\in K[\z]$  such that $A_{\v}=B_{
\v}f^{\v}$ and $R_{\v}=A_{\v}/B_{
\v}$ satisfies $R_{\v}R_{\w}^{\v}=R_{\w}R_{\v}^{
\w}$ for $\v,\w\in\Z^k$.
\PropositionStop

\Proof Since $pf=0$ for some polynomial $p\in K[\z]$, $p^{
\v}f^{\v}=0$ for all $\v\in\Z^k$. Thus, $pf=p^{\v}f^{\v}$
for all $\v\in\Z^k$. Taking $A_{\v}=p$ and $B_{\v}=p^{
\v}$, it's easily verified that $R_{\v}R_{\w}^{\v}=R_{
\w}R_{\v}^{\w}$ for all $\v,\w\in\Z^k$.
\ProofStop

\Theorem[thm:B9]  \hyperdef\thmBnine{Theorem}{B9}{} Let $f$ be a \hgt\ on $\Z^k$ that is not a \hyperref\defBtwo{\zd}.
For all $\w\in\Z^k$, let $R_{\w}$ be the \hyperref\defBfour{term ratio} of $f$ in
the direction $\w$. Then there exist polynomials $C$ and $D$ in
$K[\z]$, a finite set $V\subset\Z^k$, and univariate polynomials
$a_{\v}$ and $b_{\v}\in K[z]$ for each $\v\in V$ such
that
$$R_{\w}(\z)=\frac{C(\z+\w)}{C(\z)}\frac{D(
\z)}{D(\z+\w)}\prod_{\v\in V}\gp j0{\v\cdot
\w}\frac{a_{\v}(\z\cdot\v+j)}{b_{\v}(\z\cdot
\v+j)}\text{for all $w\in \Z^k$.}$$
\TheoremStop

\Proof By Lemma~\hyperref\lemBseven{\ref[lem:B7]}, $R_{\v + \w}=R_{\v}R_{
\w}^{\v}$ for all $\v,\w\in\Z^k$. The result follows
immediately by Corollary~\hyperref\corsixteen{\ref[cor:16]}.
\ProofStop

\hyperdef\chapteRiiiSECii{ChapterSection}{chapteRiiiSECii}{}
\Section Sets of measure zero

When considering sequences of one variable, it is useful to
identify sequences that are equal except at a finite number of points.
Thus, we identify sequences that are equal `almost everywhere' and
think of finite sets as `sets of measure zero'. When considering
sequences of several variables, that is, functions from $\Z^k$ to a
field $K$, it is useful to consider the `sets of measure zero' to
be not finite sets, but sets that can be covered by a finite number of
hyperplanes. Thus, in dimension~2, a `set of measure zero' can be covered
by a finite number of lines, and in three dimensions a `set of measure
zero' can be covered by a finite number of planes.

\Definition[def:B10]  \hyperdef\defBten{Definition}{B10}{} Define a \defword{hyperplane} in $\Z^k$ to be a
set of the form $\{\z\colon\v\cdot\z=n\}$, where $n\in\Z$ and $\v$
is some vector in $\Z^k$. Define a \defword{half-space} in $\Z^k$ to be
a set of the form $\{\z\colon\v\cdot\z>n\}$, where $n\in\Z$ and $\v$
is some vector in $\Z^k$. The boundary of the half-space $\{
\z\colon\v\cdot\z>n\}$ is the hyperplane $\{\z\colon
\v\cdot\z=n\}$. Since $\{\z\colon\v\cdot\z<n\}=\{
\z\colon-\v\cdot\z>-n\}$ is a half-space with boundary $\{
\z\colon\v\cdot\z=n\}$, each hyperplane is the boundary of two
half-spaces, and $\Z^k$ is the disjoint union of these two half-spaces
and the boundary.
\DefinitionStop

\Definition[def:B11]  \hyperdef\defBeleven{Definition}{B11}{} A \defword{set of measure zero} in $\Z^k$ is a set
that can be covered by a finite number of \hyperref\defBten{hyperplanes}. Two functions
$f,g\colon\Z^k\to K$ are \defword{equal almost everywhere} (written
$f=g$~a.e.) if there exists a set $S$ of measure zero such that $f(
\z)=g(\z)$ for all $\z\in\Z^k\setminus S$. We say $f$ is
\defword{degenerate} if $f=0$~a.e.\ and $f$ is \defword{nondegenerate}
if it is not degenerate.
\DefinitionStop

Clearly, a finite union of sets of measure zero is a set of measure zero.
Note that although we have defined a set of measure zero, we have not
defined a measure. We use the phrase only because it is familiar and
suggestive.

\Lemma[lem:B12]  \hyperdef\lemBtwelve{Lemma}{B12}{} A nonzero \hyperref\defAtwo{simple} polynomial $p\in K[\z]$ is nonzero
except on a \hyperref\defBeleven{set of measure zero}.
\LemmaStop

\Proof Let $p(\z)=\bar p(\z\cdot\v)$, where $\v\in\Z^k$
and $\bar p\in K[z]$ is a univariate polynomial. Let $n_1,\ldots,n_m$
be the integer roots of $\bar p$. If $p(\z_0)=0$, then $
\z_0\cdot\v$ is a root of $\bar p$, and hence $\z_0\cdot
\v=n_i$ for some $i$. Hence $\z_0$ lies in one of the $m$
hyperplanes $\{\z\colon\z\cdot\v=n_j\}$, $j=1,\ldots,m$,
the union of which is a set of measure zero.
\ProofStop

\Lemma[lem:B13]  \hyperdef\lemBthirteen{Lemma}{B13}{} A function $f$ is a \hgt\ on $\Z^k$ if and only if, for all $
\v \in \Z^k$, there exist nonzero polynomials $A_{\v} $ and $B_{
\v}\in K[\z]$ such that
$A_{\v}f=B_{\v}f^{\v}\text{ \hyperref\defBeleven{a.e.}}$
\LemmaStop

\Proof If $f$ is a \hgt, 
$A_{\v}f=B_{\v}f^{\v}\text{ a.e.}$
automatically. Conversely, if
$A_{\v}f=B_{\v}f^{\v}\text{ a.e.}$,
then
$p_{\v}A_{\v}f=p_{\v}B_{\v}f^{\v}$,
where $p_{\v}\in K[\z]$ is a product of nonzero linear polynomials.
Hence $f$ is a \hgt.
\ProofStop

\Definition[def:B13a]  \hyperdef\defBthirteena{Definition}{B13a}{} A \hgt\ $f$ on $\Z^k$ is \defword{honest} if for
all $\v\in\Z^k$ there exist relatively prime polynomials $A_{
\v}$ and $B_{\v}\in K[\z]$ such that $A_{\v}f=B_{
\v}f^{\v}$~\hyperref\defBeleven{a.e.}
\DefinitionStop

\hyperdef\chapteRiiiSECiii{ChapterSection}{chapteRiiiSECiii}{}
\Section Boxes

\Definition[def:box]  \hyperdef\defbox{Definition}{box}{} A \defword{$k$-dimensional box of size $n$} is a set of the
form $\{\z \in \Z^k\colon c_i\le z_i\le c_i+n\}$, where
$c_1,\ldots,c_k\in\Z$. (A 
$k$-dimensional box is the set of integer points in
a 
$k$-dimensional hypercube.)
\DefinitionStop

\Lemma[lem:B15]  \hyperdef\lemBfifteen{Lemma}{B15}{} If a $k$-variable polynomial $p\in K[\z]$ is $0$ on
a \hyperref\defbox{$k$-dimensional box of size $n$,} where $n$ is the total degree of
$p$, then $p$ is identically $0$.
\LemmaStop

\Proof The lemma is clearly true if the degree of $p$ is $0$. We assume
that it is true if the degree of $p$ is $n-1$, and show that it is true
if the degree of $p$ is $n$. Suppose $p$ is of degree $n$ and is zero
on a
$k$-dimensional box of size $n$. For $i\in\{1,\ldots,k\}$, $p-p^{
\e_i}$ is $0$ on a
$k$-dimensional box of size $n-1$. By Lemma~\hyperref\lemzero{\ref[lem:zero]} $p-p^{
\e_i}$ is of degree at most $n-1$, so $p-p^{\e_i}$ is identically
$0$ by assumption. Thus, $p=p^{\e_i}$ for all $i\in\{1,\ldots,k\}$.
By Lemma~\hyperref\lemthree{\ref[lem:three]} $p$ is free of $z_i$ for $i\in\{1,\ldots,k\}$.
Thus $p$ is constant, and hence $p$ is identically zero.
\ProofStop

\Lemma[lem:B16]  \hyperdef\lemBsixteen{Lemma}{B16}{} If a \hgt\  on $\Z^k$ 
is not a \hyperref\defBtwo{\zd}, then for any sequence $
\w_1,\ldots,\w_n$ of vectors in $\Z^k$ the \hgt\ $f^{
\w_1}f^{\w_2}\cdots f^{\w_n}$ is not a \hyperref\defBtwo{\zd}.
\LemmaStop

\Proof By Lemma~\hyperref\lemBthree{\ref[lem:B3]}, for any $\v\in\Z^k$ there exist
nonzero polynomials $A_{\v},B_{\v}\in K[\z]$ such that
$A_{\v}f=B_{\v}f^{\v}$. If $f^{\w_1}f^{\w_2}\cdots
f^{\w_n}$ is a \zd, then for some nonzero polynomial $p$, $pf^{
\w_1}f^{\w_2}\cdots f^{\w_n}=0$.
Hence, $$pB_{\w_1}\cdots B_{\w_n}f^{\w_1}f^{\w_2}\cdots
f^{\w_n}=0,$$ hence, $pA_{\w_1}\cdots A_{\w_n}f^n=0$,
hence, $pA_{\w_1}\cdots A_{\w_n}f=0$, and hence, $f$ is a \zd,
contrary to assumption.
\ProofStop

\Lemma[lem:B17]  \hyperdef\lemBseventeen{Lemma}{B17}{} If $f$ is a \hgt\ on $\Z^k$, then either $f$ is a
\hyperref\defBtwo{\zd}\ or $f$ is nonzero on arbitrarily large
$k$-dimensional \hyperref\defbox{boxes}.
\LemmaStop

\Proof Let $V$ be the $k$-dimensional box
$\{\z\in\Z^k\colon 0\le z_i\le n\}$,
and let $V=\{\v_1,\ldots,\v_m\}$. If $f$ is not a \zd, by
Lemma~\hyperref\lemBsixteen{\ref[lem:B16]} $g=f^{\v_1}\cdots f^{\v_m}$ is not a \zd,
hence for some $\w$ (infinitely many), $g(\w)\ne0$, and hence
$f(\w+\v_j)\ne0$ for $j\in\{1,\ldots,m\}$. Thus $f$ is nonzero
on the
$k$-dimensional box
$$\{\z+\w\colon 0\le z_i\le n,\ i=1,\ldots,k\},$$
which is of size $n$.
\ProofStop

\hyperdef\chapteRiiiSECiv{ChapterSection}{chapteRiiiSECiv}{}
\Section Polyhedral regions

\Definition[def:B18]  \hyperdef\defBeighteen{Definition}{B18}{} A region $\R\subset\Z^k$ is
\defword{polyhedral} if $R=\Z^k$ or $R$
is the intersection of a finite number
of \hyperref\defBten{half-spaces} of $\Z^k$. 
\DefinitionStop

\Lemma[lem:B19]  \hyperdef\lemBnineteen{Lemma}{B19}{} The characteristic function of a \hyperref\defBten{half-space} in $\Z^k$ is a \hgt;
in fact, for all $w\in \Z^k$, $f=f^{\w}$ \hyperref\defBeleven{a.e.}
\LemmaStop

\Proof Let the half-space be $\{\z\colon\v\cdot\z>n\}$ for
some $\v\in\Z^k$ and $n\in\Z$. Then $f(\z)=f(\z+\w)$ unless $
\z\cdot\v>n$ and $(\z+\w)\cdot\v\le n$, or
unless $\z\cdot\v\le n$ and $(\z+\w)\cdot\v> n$;
that is, $n<\z\cdot\v\le n-\v\cdot\w$ or $n-
\v\cdot\w<\z\cdot\v\le n$. In either case, the set of
exceptions is just the union of $|\v\cdot\w|$ hyperplanes. Thus
$f(\z)=f(\z+\w)$ a.e.
\ProofStop

\Lemma[lem:B20]  \hyperdef\lemBtwenty{Lemma}{B20}{} The characteristic function of a \hyperref\defBeighteen{polyhedral region} is a
\hgt.
\LemmaStop

\Proof By definition, the region is the intersection of a finite number
of half-spaces, so the characteristic function of the region is just
the product of the characteristic functions of the half-spaces, which
are all hypergeometric terms
by Lemma~\hyperref\lemBnineteen{\ref[lem:B19]}. Since the product of two \hgt s
is a \hgt, the characteristic function of the region is a \hgt.
\ProofStop

\Lemma[lem:B21]  \hyperdef\lemBtwentyone{Lemma}{B21}{} If $\R$ is a \hyperref\defBeighteen{polyhedral region} in $\Z^k$, then either
$\R$ contains arbitrarily large
$k$-dimensional \hyperref\defbox{boxes}, or $\R$ is a \hyperref\defBeleven{set of measure zero}.
\LemmaStop

Lemma~\hyperref\lemBtwentyone{\ref[lem:B21]} can be proved by elementary means, but we can't
resist this short proof based on a result from Chapter~4.

\Proof Let $f$ be the characteristic function of the region $\R$. By
Lemma~\hyperref\lemBtwenty{\ref[lem:B20]}, $f$ is a \hgt. The function $f$ is the product of
characteristic functions of half-spaces, so $f$ is holonomic. If $f$ is
a \zd, then by Lemma~\hyperref\lemCthree{\ref[lem:C3]} $f=0$~a.e., so $\R$ is a set of
measure $0$. If $f$ is not a \zd, then by Lemma~\hyperref\lemBseventeen{\ref[lem:B17]}, $f$ is
nonzero on arbitrarily large
$k$-dimensional boxes, and hence $\R$ contains arbitrarily large 
$k$-dimensional boxes.
\ProofStop

\Definition[def:B21a]  \hyperdef\defBtwentyonea{Definition}{B21a}{} Let $\R\subset\Z^k$ be a \hyperref\defBeighteen{polyhedral region}. A
\defword{hypergeometric term} on $\R$ over $K$ is a function $f\colon
\R\to K$ such that for $i\in\{1,\ldots,k\}$ there exist nonzero
polynomials $A_i,B_i\in K[\z]$ such that
$$A_i(\z)f(\z)=B_i(\z)f(\z+\e_i)$$ 
for all $\z$ such that $\z$ and $\z+\e_i$ are in $\R$.
\DefinitionStop

\Lemma[lem:B21b]  \hyperdef\lemBtwentyoneb{Lemma}{B21b}{} If $f$ is a \hgt\ on a \hyperref\defBeighteen{polyhedral region}
$\R\subset\Z^k$, then the function
$$g(\z)=\cases
f(\z)&\z\in \R\\
0&\z\notin \R\endcases$$
is a \hgt\ on $\Z^k$.
\LemmaStop

\Proof For $i\in\{1,\ldots,k\}$ there exist nonzero polynomials $A_i$
and $B_i$ such that
$$A_i(\z)f(\z)=B_i(\z)f^{\e_i}(\z)$$
for all $\z$ such that $\z\in \R$ and $\z+\e_i\in \R$.
Let $h$ be the characteristic function for $\R$. By
Lemma~\hyperref\lemBtwentyone{\ref[lem:B21]}, $h$ is a \hgt\ on $\Z^k$, so for
$i\in\{1,\ldots,k\}$ there exist nonzero polynomials $\bar A_i$ and
$\bar B_i$ such that
$$\bar A_i(\z)h(\z)=\bar B_i(\z)h(\z+\e_i)$$
for all $\z\in\Z^k$. By considering separately the cases
\Enumerate
\item $\z\in \R$, $\z+\e_i\in \R$,
\item $\z\in \R$, $\z+\e_i\notin \R$,
\item $\z\notin \R$, $\z+\e_i\in \R$, and
\item $\z\notin \R$, $\z+\e_i\notin \R$,
\EnumerateStop
it is easily seen that 
$$A_i\bar A_ig=B_i\bar B_ig^{\e_i}$$
for $i\in\{1,\ldots,k\}$.
\ProofStop



\hyperdef\chapteRiiiSECv{ChapterSection}{chapteRiiiSECv}{}
\Section Factorial hypergeometric terms

\Definition[def:B22]  \hyperdef\defBtwentytwo{Definition}{B22}{} A \hgt\ on $\Z^k$ is \defword{factorial} on a region
$\R\subset\Z^k$ if there exist a finite set $V\subset\Z^k$, univariate
polynomials $a_{\v}$ and $b_{\v}\in K[z]$ and an integer
$n_{\v}$ for each $\v\in V$ such that, for all $\z\in \R$,
\Enumerate
\item $\ds{f(\z)=\prod_{\v\in V}\prod_{j=1}^{\v\cdot
\z+n_{\v}}\frac{a_{\v}(j)}{b_{\v}(j)}}$,
\item for all $\v\in V$, $\v\cdot\z+n_{\v}$ is a
positive integer, and
\item for all $\v\in V$ and $j$, $1\le j\le\v\cdot
\z+n_{\v}$, $a_{\v}(j)\ne0$ and $b_{\v}(j)\ne0$.
\EnumerateStop
\DefinitionStop

Recall that $\vecgamma^{\z}=\gamma_1^{z_1}\cdots\gamma_k^{z_k}$ and
$(m)_r=m(m+1)\cdots(m+r-1)$.

\Lemma[lem:B23]  \hyperdef\lemBtwentythree{Lemma}{B23}{} If $f$ is a \hgt\ on $\Z^k$ over a field that is
algebraically closed and $f$ is \hyperref\defBtwentytwo{factorial} on a region $\R$, then there
exist a vector $\vecgamma\in K^k$, constants $m_1,\ldots,m_p$ and
$n_1,\ldots,n_q\in K$, vectors $\v_1,\ldots,\v_p$ and $
\w_1,\ldots,\w_q\in\Z^k$, and integers $r_1,\ldots,r_p$ and
$s_1,\ldots,s_q\in\Z$ such that, for all $\z\in \R$,
\Enumerate
\item $\ds{f(\z)=
\vecgamma^{\z}
\frac{\prod_{i=1}^p(m_i)_{\v_i\cdot
\z+r_i}}{\prod_{j=1}^q(n_j)_{\w_j\cdot\z+s_j}}}$;
\item $\v_i\cdot\z+r_i$ and $\w_j\cdot\z+s_j$ are
positive integers for each $i\in\{1,\ldots,p\}$ and $\{1,\ldots,q\}$;
\item all the terms appearing in the numerator and denominator are nonzero.
\EnumerateStop
\LemmaStop

\Proof Since $f$ is factorial on $\R$,
$$f(\z)=\prod_{\v\in V}\prod_{j=1}^{\v\cdot\z+n_{
\v}}\frac{a_{\v}(j)}{b_{\v}(j)}$$
for all $\z\in \R$ where $V$, $n_{\v}$, $a_{\v}$, and
$b_{\v}$ are as in Definition~\hyperref\defBtwentytwo{\ref[def:B22]}. Since a product of
functions of the form~(1) satisfying conditions~(2) and~(3) is of the
same form (and likewise for reciprocals), we need only prove the lemma
for
$$f(\z)=\prod_{j=1}^{\u\cdot\z+\ell}a(j).$$
Since $K$ is algebraically closed, $a(j)=\alpha(j-a_1)\cdots(j-a_t)$,
where $\alpha,a_i\in K$. By the same reasoning as before, we need only
prove the lemma for
$$\prod_{j=1}^{\u\cdot\z+\ell}\alpha$$
and
$$\prod_{j=1}^{\u\cdot\z+\ell}(j-a_1).$$
The first is $\alpha^{\u\cdot
\z+\ell}=\alpha^{\ell}\alpha^{u_1z_1+\cdots
u_kz_k}=(\alpha^{\ell})_1\boldsymbol\gamma^{\z}$, where
$\boldsymbol\gamma=(\alpha^{u_1},\ldots,\alpha^{u_k})$, and the second
is $(1-a_1)_{\u\cdot\z+\ell}$. We're given that $
\u\cdot\z+\ell$ is positive on $\R$ and $a(j)$ is nonzero for $1\le
j\le\u\cdot\z+\ell$, so conditions~(2) and~(3) follow.
\ProofStop

\Definition[def:B24]  \hyperdef\defBtwentyfour{Definition}{B24}{} A \hgt\ $f$ on $\Z^k$ is \defword{weakly factorial} on a
region $\R\subset\Z^k$ if there exist a finite set
$V\subset\Z^k$, univariate polynomials $a_{\v}$ and $b_{\v} \in
K[z]$ for each $\v\in V$, and $\z_0\in \R$ such that, for all
$\z\in \R$,
\Enumerate
\item $\ds{f(\z)=\prod_{\v\in V}\gp j{\z_0\cdot\v}{
\z\cdot\v}\frac{a_{\v}(j)}{b_{\v}(j)}}$, and
\item for all $\v\in V$ and all $j$ occurring in the product
$\ds{\gp j{\z_0\cdot\v}{\z\cdot\v}\frac{a_{
\v}(j)}{b_{\v}(j)}}$, $a_{\v}(j)\ne0$ and $b_{\v}(j)\ne0$.
\EnumerateStop
\DefinitionStop

An example of a \hgt\ on $\Z$ that is weakly factorial but 
not factorial is 
$$f(z_1)=\gp j0{z_1}(2j+1)=\cases
\ds{\prod_{j=0}^{z_1-1}(2j+1)},&z_1\ge0\\
\ds{\prod_{j=1}^{-z_1}\frac1{1-2j}},&z_1<0.
\endcases$$
Though $f$ is not factorial, the domain of $f$ can be split into two
regions, $\{z_1\ge0\}$ and $\{z_1<0\}$, such that it is factorial on
each region. The following lemma shows that this is true in general.


\Lemma[lem:B25]  \hyperdef\lemBtwentyfive{Lemma}{B25}{} If a \hgt\ is \hyperref\defBtwentyfour{weakly factorial} on a \hyperref\defBeighteen{polyhedral region}
$\R$, then there exist a finite number of \hyperref\defBeighteen{polyhedral regions}
$\R_1,\ldots,\R_m$ such that $\R=\R_1\cup\cdots\cup \R_m$ and $f$ is
\hyperref\defBtwentytwo{factorial} on each of the $\R_i$.
\LemmaStop

\Proof Since $f$ is weakly factorial on $\R$,
$$f(\z)=\prod_{\v\in V}\gp j{\v\cdot\z_0}{
\v\cdot\z}\frac{a_{\v}(j)}{b_{\v}(j)},$$
where $a_{\v},b_{\v},V$, and $\z_0$ are as in
Definition~\hyperref\defBtwentyfour{\ref[def:B24]}. Let $\Cal G$ be the set of functions $g\colon
V\to\{-1,1\}$. We show that $\R$ can be divided into at most $2^{|V|}$
regions $\{\R_g\colon g\in\Cal G\}$, such that for each $\v\in V$,
$g\in\Cal G$, and region $\R_g$, there exist $\w=\pm\v$,
$n=\pm\v\cdot\z_0$, univariate polynomials $a_{\v}$ and
$b_{\v}$ which are nonzero for $1\le j\le\w\cdot\v+n$, such
that
$$\gp j{\v\cdot\z_0}{\v\cdot\z}\frac{a_{
\v}(j)}{b_{\v}(j)}=\prod_{j=1}^{\w\cdot\z+n}\frac{a_{
\v}(j)}{b_{\v}(j)}.\tag{*}$$

We construct the region $\R_g$ so that for every $\v\in V$, the
sign of $\v\cdot(\z-\z_0)$ is constant over $\z\in
\R_g$. For each $\v\in V$ define the half-spaces
$${\Cal H}_{\v}(1)=\{\z\colon\v\cdot(\z-\z_0)\ge0\}$$
and
$${\Cal H}_{\v}(-1)=\{\z\colon\v\cdot(\z-\z_0)<0\}.$$
For each $g\in\Cal G$, the region $\R_g$ is defined by
$$\R_g=\R\cap\bigcap_{\v\in V}{\Cal H}_{\v}(g(\v)).$$
Clearly $\R=\bigcup_g\R_g$ and $\R_g$ is polyhedral. Note $|{\Cal
G}|=2^{|V|}$, so $|\{\R_g\colon g\in{\Cal G}\}|\le2^{|V|}$. By
definition, the sign of $\v\cdot(\z-\z_0)$ is constant on
${\Cal H}_{\v}(g(\v))$, so, for each $\v$, the sign of
$\v\cdot(\z-\z_0)$ is constant
for $\z \in \R_g$.

We need to show that for a given region (*) is true. Let a region
$\R_g$ and a vector $\v$ be given. If $\v\cdot\z-
\v\cdot\z_0=\v\cdot(\z-\z_0)$ is nonnegative on $\R_g$,
then
$$\gp j{\v\cdot\z_0}{\v\cdot\z}\frac{a_{
\v}(j)}{b_{\v}(j)}
=\prod_{j=\v\cdot\z_0}^{\v\cdot
\z-1}\frac{a_{\v}(j)}{b_{\v}(j)}
=\prod_{j=1}^{\v\cdot
\z-\v\cdot\z_0}\frac{a_{\v}(j+\v\cdot
\z_0-1)}{b_{\v}(j+\v\cdot\z_0-1)}
=\prod_{j=1}^{\w\cdot
\z +n_{\v}}\frac{\bar a_{\v}(j)}{\bar b_{\v}(j)}
,$$
where $\bar a_{\v}(j)=a_{\v}(j+\v\cdot\z_0-1)$, $\bar b_{
\v}(j)=b_{\v}(j+\v\cdot\z_0-1)$,
$\w=\v$, and
$n_{\v}=-\v\cdot\z_0$. 
If $\v\cdot\z-\v\cdot\z_0=
\v\cdot(\z-\z_0)$ is negative on $\R_g$,
$$\align
\gp j{\v\cdot\z_0}{\v\cdot\z}\frac{a_{
\v}(j)}{b_{\v}(j)}
&=\prod_{j=\v\cdot\z_0}^{\v\cdot\z-1}\frac{b_{
\v}(j)}{a_{\v}(j)}\\
&=\prod_{j=1-\v\cdot\z_0}^{-\v\cdot\z}\frac{b_{
\v}(-j)}{a_{\v}(-j)}\\
&=\prod_{j=1}^{\v\cdot\z_0-\v\cdot\z}\frac{b_{
\v}(-j-\v\cdot\z_0)}{a_{\v}(-j-\v\cdot\z)}\\
&=\prod_{j=1}^{\w\cdot\z+n_{\v}}\frac{\bar a_{
\v}(j)}{\bar b_{\v}(j)},
\endalign$$ 
where $\bar a_{\v}(j)=b_{\v}(-j-\v\cdot\z_0)$ and $\bar
b_{\v}=a_{\v}(-j-\v\cdot\z_0)$,
$\w=-\v$, and $n_{\v}= \v \cdot \z_0$.
\ProofStop

\hyperdef\chapteRiiiSECvi{ChapterSection}{chapteRiiiSECvi}{}
\Section Path connected regions

Recall that a lattice path in $\Z^k$ is a sequence 
$\{T_i\}_{i\ge1}$ in $\Z^k$ such that
$T_i-T_{i-1}\in \{\pm \e_1,\ldots,\pm \e_k\}$ for all
$i>1$.
\Definition[def:B26]  \hyperdef\defBtwentysix{Definition}{B26}{} 
Let  $\R$ and $\bar \R\in \Z^k$. 
The region $\R$ is
\defword{lattice path connected in $\bar \R$} 
if for all $\z_1$ and $\z_2\in \R$ there exists a lattice path
$\{T_i\}_{i\ge1}$ contained in $\bar \R$ such that $T_1=\z_1$ and
$T_i=\z_2$ for some positive integer~$i$.
The region $\R$ is
\defword{lattice path connected } if it is lattice path connected in itself.
\DefinitionStop

\Definition[def:B27]  \hyperdef\defBtwentyseven{Definition}{B27}{} For any subsets $S$, $\R$,
and $\bar \R \subset\Z^k$, the region $\R$ is
\defword{$S$-path connected in $\bar R$} if for all $\z_1$
and $\z_2\in \R$ there
exists a sequence $\{T_i\}_{i\ge1}$ contained in $\bar \R$ such that
$T_1=\z_1$
and  $T_i=\z_2$ for some positive integer~$i$, and $T_i-T_{i-1}\in S$ for all
$i>1$.
\DefinitionStop

Thus a region is lattice path connected if it is $S$-path connected in itself
for $S=\{\pm\e_1,\ldots,\e_k\}$.

\Lemma[lem:B28a]  \hyperdef\lemBtwentyeighta{Lemma}{B28a}{} Let $\R\subset\Z^k$ be a \hyperref\defBeighteen{polyhedral region}. For any
positive integer $n$, $\R$ can be written as the union of a \hyperref\defBeleven{set of
measure zero} and a \hyperref\defBeighteen{polyhedral region} $\R'$ such that
every $\z\in \R'$ lies in a \hyperref\defbox{$k$-dimensional box of size $n$}
contained entirely in $\R$.
\LemmaStop

\Proof
Let $\R'=\cap_{\v\in B} (R-\v)$ where 
$B=\{\b\in \Z^k: \text{\ for each\ }
 i\in \{1,\ldots, k\}, 0\le b_i \le n\}$.
Clearly $\R'$ is a polyhedral region and for every $\z \in \R'$
the box $\z + B=\{\z + \b:\b\in B\}$
is contained in $\R$, so we need to show 
that $S=\R\setminus \R'$ is a set of measure zero.

Since $\R$ is the intersection of a finite number of half-spaces, 
there exists a finite set $T\subset \Z^k \times \Z$ such that 
$\z\in\R$ if and only if $\z \in \Z^k$ and $\v\cdot\z\le n$
for all $(\v,n)\in T$.
Thus, $\z \in S$ implies there exists $(\v, n) \in T$ and 
$\b\in B$ such that $\v \cdot \z\le n$ and $\v \cdot
(\z+\b) > n$, which in turn implies $\v \cdot \z = m$
for some integer
$m$ such that $n-\v \cdot \b < m\le n$.
It follows that $S$ can be covered by hyperplanes of the form 
$\{\z:\v \cdot\z =m\}$, where $\b \in B$,
$(\v, n) \in T$ and $n-\v \cdot \b < m\le n$. Clearly, there are
a finite number of such hyperplanes, so $S$ is a set of measure zero.
\ProofStop


\Lemma[lem:B28b]  \hyperdef\lemBtwentyeightb{Lemma}{B28b}{}
Let $B_0$ and $B_1$ be \hyperref\defbox{$k$-dimensional boxes of size $1$,} 
let $\bar S$ be the convex hull of $B_0$ and $B_1$ 
in $\Q^k$, and let $S$ be the set of integer points
in $\bar S$. Then $S$ is \hyperref\defBtwentysix{lattice path connected}.
\LemmaStop

\Proof
Let $M$ be a $k\times k$ matrix with entries equal to
$\pm 1$ on the diagonal and $0$ off the diagonal.  It is easily seen
that the lemma is true for the boxes $B_0$ and $B_1$ 
if and only if it is true for the reflections 
$M B_0 = \{Mb:b\in B_0\}$ and 
$M B_1 = \{Mb:b\in B_1\}$ of $B_0$ and $B_1$. Similarly, letting
$\v \in \Z^k$, the lemma is true for $B_0$ and $B_1$ if and 
only if it is true for the translations 
$B_0+\v=\{\b+\v:\b \in B_0\}$ 
and 
$B_1+\v=\{\b+\v:\b \in B_1\}$ 
of $B_0$ and $B_1$.  By applying a sequence of such transformations,
we may assume that $B_0$ is the unit 
$k$-dimensional box at the origin in the first
orthant, $B_0=\{\z \in \Z^k: \text {\ for each\ } i \in
\{1,\ldots,k\}, 0\le z_i \le 1\}$, and that $B_1=B_0+\w$ where
$\w \in \Z^k$ and $w_i\ge0$ for each $i\in\{1,\ldots,k\}$.

Further, if $w_i=0$ for some $i\in\{1,\ldots,k\}$, then the lemma
follows easily by induction from the case of dimension $k-1$.
Since the case of dimension $1$ is trivial, we may assume that 
$w_i > 0$ for each $i\in\{1,\ldots,k\}$. 

Let $\bar B_0$
be the convex hull of $B_0$ in $\Q^k$.  
Thus, $\bar B_0
=\{\z \in \Q^k: \text {\ for each\ } i \in
\{1,\ldots,k\}, 0\le z_i \le 1\}$.

We claim that any $\z\in S$ 
can be written in the form
$\z=\z'+ s \w$,
where
$\z'\in \bar B_0$,
$s\in\Q$, and $s\ge 0$.
Since $\z \in S$, 
$$ \z=
\sum_{\b\in B_0} \lambda_{\b} \b
+ \sum_{\b\in B_1} \lambda_{\b} \b \quad,
$$
where 
$$ \sum_{\b\in B_0} \lambda_{\b} +
\sum_{\b\in B_1} \lambda_{\b} =1\quad.
$$
Hence,
$$\align \z &=
\sum_{\b\in B_0} \lambda_{\b} \b +
\sum_{\b\in B_0} \lambda_{\b+\w} (\b +\w)\\
&= \sum_{\b\in B_0} (\lambda_{\b}+\lambda_{\b +\w}) \b +
\sum_{\b\in B_0} \lambda_{\b+\w} \w\\
&=\z' + s \w 
\quad,\\
\endalign
$$
where
$$ \z'
= \sum_{\b\in B_0} (\lambda_{\b}+\lambda_{\b +\w}) \b 
$$
and
$$ s= \sum_{\b\in B_0} \lambda_{\b+\w} \ge 0
\quad.$$
Since
$$
\sum_{\b\in B_0} (\lambda_{\b}+\lambda_{\b +\w}) 
= \sum_{\b\in B_0} \lambda_{\b}+
\sum_{\b\in B_1} \lambda_{\b } 
=1 \quad,
$$
it follows that 
$\z'\in \bar B_0$.

We claim that for any $\z\in S\setminus B_1$, there exists $j \in
\{1,\ldots,k\}$ such that $\z+\e_j\in S$.
Let $\z=\z'+s\w$ where
$\z'\in \bar B_0$, $s\in \Q$ and $s\ge 0$.  
For each $i\in \{1,\ldots,k\}$,
the line $\{\z'-r\w:r\in\Q\}$ intersects the hyperplane
$\z\cdot \e_i=0$ at a point $\v_i=\z'-r_i \w$ where
$r_i\in\Q$ and $r_i\ge 0$.  
Thus, $v_{i,i}=0$ for each $i\in\{1,\ldots,k\}$.  Let $j\in\{1,\ldots,k\}$
be such that $r_j$ is minimal.  By the minimality of $r_j$
and the positivity of $w_i$, it follows that $v_{j,i}\ge v_{i,i}=0$
for $i\ne j$.  Thus, $v_{j,i}\ge 0$ for all $i\in\{1,\ldots,k\}$.
Further, since $0\le z_{i}'\le 1$, and $r_i\ge 0$, it follows that
$0\le v_{j,i}\le 1$ for all $i\in\{1,\ldots,k\}$ and $v_{j,j}=0$.
Letting $t=r+s$, it follows that $\z=\v_j+t \w$,
and, hence,
$$\z+\e_j=(1-t)(\v_j+\e_j)+t(\v_j+\e_j+\w)
\quad.$$
It is easily seen
that $t<1$, otherwise $\z \in B_1$.
Since $v_{j,j}=0$ and $\v_j \in \bar B_0$, it follows that $
\v_j+\e_j\in \bar B_0$, 
and, hence, $\v_j+\e_j+\w\in \bar B_1$. 
Thus, $\z+\e_j\in \bar S$ as claimed.

We show that every $\z\in S$ is lattice path connected to
a point in $B_1$.  
Suppose to the contrary some $\z \in S\setminus B_1$ is not
lattice path connected to a point in $B_1$.  Assume that 
$\sum_{i=1}^k z_i$ is maximal.  Such a $\z$ exists since 
$\sum_{i=1}^k z_i<\sum_{i=1}^k w_i$ for all $\z\in S\setminus B_1$.
For some $j\in \{1,\ldots,k\}$, we have $\z+\e_j \in S$.
By the maximality of $\sum_{i=1}^k z_i$, it follows that 
$\z+\e_j \in B_1$.
Thus, $\z$ is lattice path connected to $B_1$ contrary to 
assumption. 
Clearly $B_1$ is lattice path connected, so
$S$ is lattice path connected.

\ProofStop

\Corollary[cor:B28c]  \hyperdef\corBtwentyeightc{Corollary}{B28c}{} Let $\R\subset\Z^k$ be a \hyperref\defBeighteen{polyhedral region} 
and let $\z_1$ and $\z_2\in R$.
If $\z_1$ and $\z_2$ are contained in 
\hyperref\defbox{$k$-dimensional boxes of size $1$} contained entirely
in $R$, then
there exists a lattice path from $\z_1$ to $\z_2$ contained
entirely in $R$.  
\CorollaryStop

\Proof
Let $B_1$ and $B_2\subset \R$ be $k$-dimensional boxes
containing $\z_1$ and $\z_2$ respectively.
Let $\bar S$ be the convex hull of $B_0$ and $B_1$ in $\Q^k$
and let $S$ be the set of integer points in $\bar S$. By the definition
of a polyhedral region there exists a convex subset $\bar R$ of $\Q^k$
such that $R$ is the set of integer points in $\bar R$.  Since $\bar R$
is convex, it contains the convex hull of any of its subsets.
Thus, 
$\bar S\subset \bar R$ and, hence, $S\subset R$.  But $S$ is lattice path
connected by Lemma~\hyperref\lemBtwentyeightb{\ref[lem:B28b]}.
\ProofStop

\hyperdef\chapteRiiiSECvii{ChapterSection}{chapteRiiiSECvii}{}
\Section Structure theorem for honest hypergeometric terms

\Theorem[thm:B29]  \hyperdef\thmBtwentynine{Theorem}{B29}{} Let $f$ be an \hyperref\defBthirteena{honest} \hgt\ on $\Z^k$. There exist
relatively prime
polynomials $C$ and $D\in K[\z]$, a finite set $V\subset\Z^k$,
univariate polynomials $a_{\v},b_{\v}\in K[z]$, $\v\in V$
(all of which can be determined as in Theorem~\hyperref\thmBnine{\ref[thm:B9]}), and a
finite number of \hyperref\defBeighteen{polyhedral regions} $\R_1,\ldots,\R_m$ such that
\Enumerate
\item $\Z^k$ is the disjoint union of the $\R_i$ and a \hyperref\defBeleven{set of measure zero};
\item for each $i\in\{1,\ldots,m\}$ there exists $\z_0\in \R_i$
such that $C(\z_0)\ne0$, and for all $\z\in \R_i$ for which
$D(\z)\ne0$,
$$f(\z)=f(\z_0)\frac{C(\z)}{C(\z_0)}\frac{D(
\z_0)}{D(\z)}\prod_{\v\in V}\gp j{\z_0\cdot\v}{
\z\cdot\v}\frac{a_{\v}(j)}{b_{\v}(j)}.$$
\item  all the terms $a_{\v}(j)$ and $b_{\v}(j)$
occurring in the product are nonzero.
\EnumerateStop
\TheoremStop

We conjecture that Theorem~\hyperref\thmBtwentynine{\ref[thm:B29]} is true for \hgt s that
are not honest if the condition that $C$ and $D$ are relatively prime
is dropped.

Theorem~\hyperref\thmBtwentynine{\ref[thm:B29]} applies to \hgt s on polyhedral regions
other than $\Z^k$ via Lemma~\hyperref\lemBtwentyoneb{\ref[lem:B21b]}.

\Proof 
If $f$ is a \zd\  and $pf=0$, then the lemma is true with $D=p$, $C=1$,
$V$ empty, and one region $\R_1=\Z^k$.
We assume hence forth that $f$ is not a \zd\.
Let $R_{\w}$ be the term ratio of $f$ in the direction $\w$.
By Theorem~\hyperref\thmBnine{\ref[thm:B9]},
$$R_{\w}(\z)=\frac{C(\z+\w)}{C(\z)}\frac{D(
\z)}{D(\z+\w)}\prod_{\v\in V}\gp j0{\v\cdot
\w}\frac{a_{\v}(\z\cdot\v+j)}{b_{\v}(\z\cdot
\v+j)}$$
for all $\w\in\Z^k$.  We may assume $C$ and $D$ are relatively prime.
Let $\bar A_{\w}$ and $\bar B_{\w} \in K[\z]$ 
be relatively prime polynomials such that
\hyperdef\Btwentyninezero{item}{zero}{
$$
\frac{\bar A_{\w}}{\bar B_{\w}}=\prod_{\v\in V}\gp
j0{\v\cdot\w}\frac{a_{\w}(\z\cdot\v+j)}{b_{
\w}(\z\cdot\v+j)}.\tag 1$$}
Thus $\bar A_{\w}$ and $\bar B_{\w}$ are products of simple
polynomials, and by Lemma~\hyperref\lemBtwelve{\ref[lem:B12]} $\bar A_{\w}$ and $\bar
B_{\w}$ are nonzero except on a set of measure zero.

Since $f$ is honest, for each $\w\in\Z$ there exist relatively
prime polynomials $A_{\w}$ and $B_{\w}\in K[\z]$ such
that $A_{\w}f=B_{\w}f^{\w}$~a.e. Since $f$ is not a \zd, by
Lemma~\hyperref\lemBsix{\ref[lem:B6]} the term ratio $R_{\w}$ is unique. Hence,
$$\frac{A_{\w}}{B_{\w}}=R_{\w}=\frac{C^{\w}D\bar
A_{\w}}{CD^{\w}\bar B_{\w}}$$
for all $\w\in\Z^k$ and, hence,
there exist relatively prime polynomials 
$p_{\w} $
and 
$q_{\w} \in K[\z]$ such that
$$p_{\w}A_{\w}
=q_{\w}C^{\w}D\bar A_{\w}
\text{ and }
p_{\w}B_{\w}
=q_{\w}CD^{\w}\bar B_{\w}\quad.$$
Since $p_{\w}$ and $q_{\w}$ are relatively prime,
it follows that 
$q_{\w} \mid A_{\w}$
and 
$q_{\w} \mid B_{\w}$, and 
since $A_{\w}$ and $B_{\w}$ are relatively prime, it follows that 
$q_{\w}$ must be trivial.
Hence, for each $\w\in\Z^k$,
$$C^{\w}D\bar A_{\w}f
=CD^{\w}\bar B_{\w}f^{\w}\text{ a.e.}$$

Let $d$ be the total degree of $CD$. By Lemma~\hyperref\lemBfifteen{\ref[lem:B15]}, $CD$ is
nonzero for at least one point in any 
$k$-dimensional box of size $d$. Let
$S_0\subset\Z^k$ be the union of all 
$k$-dimensional boxes of size $d$ containing $0$.
Let $S_1=\{\vec s\pm\e_i\colon\vec s\in S_0\text{ and
}i\in\{1,\ldots,k\}\}$. Thus, $S_1$ is the union of all 
$k$-dimensional boxes of size
$d$ containing a point $\pm\e_i$, $i\in\{1,\ldots,k\}$. Finally,
let $S=S_0-S_1=\{\vec s_0-\vec s_1\colon\vec s_0\in S_0,\vec s_1\in
S_1\}$. 
For each $\w\in\Z^k$, the polynomials $\bar A_{\w}$ and
$\bar B_{\w}$ are nonzero except on a set of measure zero, and
$\bar A_{\w}C^{\w}Df
=\bar B_{\w}CD^{\w}f^{\w}$
except on a set of measure zero.
Call the union of these two sets of measure zero $H_1(\w)$
and let $H_1=\cup_{\w\in S}H_1(\w)$.  Since $S$ is finite,
it follows that $H_1$ is a set of measure zero.   
Thus, for all $\w\in S$ and all $
\z\notin H_1$,
\Enumerate \itemno=1
\item[B29:one] \hyperdef\Btwentynineone{item}{one}{} $\bar A_{\w}(\z)\ne0$,
$\bar B_{\w}(\z)\ne0$, and
\item[B29:three] \hyperdef\Btwentyninethree{item}{three}{} $\bar A_{\w}(\z)C(\z+\w)D(
\z)f(\z)=\bar B_{\w}(\z)C(\z)D(\z+\w)f(\z+
\w)$.
\EnumerateStop

\def\RR{{\bar {\R}}}
\accentedsymbol\RRR{{\Bar {\Bar {\Cal R}}}}
Let $H_2$ be a finite set of hyperplanes covering $H_1$. The
hyperplanes in $H_2$ divide space into a finite number of polyhedral
regions $\RRR_i$ such that $\Z^k$ is the disjoint union of the regions and the
union of the hyperplanes. (The hyperplanes are not necessarily disjoint,
but the regions are disjoint from each other and from the hyperplanes.)
Further, by Lemma~\hyperref\lemBtwentyeighta{\ref[lem:B28a]} each region
$\RRR_i$ can be written as a union of a polyhedral region $\RR_i$ and a set of
measure zero such that 
each $\z\in\RR_i$ is
contained in a
$k$-dimensional box of size $d$ contained entirely in $\RRR_i$.
By Lemma~\hyperref\lemBtwentyeightb{\ref[lem:B28b]} each $\RR_i$ can be written as a union of
a polyhedral region $\R_i$ and a set of measure zero such that 
$\R_i$ is lattice path connected in $\RR_i$. 
Thus, 
\Enumerate \itemno=3
\item[B29:four] \hyperdef\Btwentyninefour{item}{four}{} $\Z^k=\R_1\cup\cdots\cup \R_m\cup H$, where $H$ is
a set of measure zero;
\item[B29:five] \hyperdef\Btwentyninefive{item}{five}{} $\bar A_{\w}C^{\w}Df=\bar B_{\w}CD^{
\w}f^{\w}$ for every $\z\in \RRR_i$, $i\in\{1,\ldots,m\}$, and
every $\w\in S$;
\item[B29:six] \hyperdef\Btwentyninesix{item}{six}{} $\bar A_{\w}$ and $\bar B_{\w}$ are nonzero on
$\RRR_i$, $i\in\{1,\ldots,m\}$, for every $\w\in S$;
\item[B29:seven] \hyperdef\Btwentynineseven{item}{seven}{} $\R_i$ is lattice path connected in $\RR_i$ for $i\in\{1,\ldots,m\}$;
\item[B29:eight] \hyperdef\Btwentynineeight{item}{eight}{} for every $\z\in \RR_i$, $\z$ is contained in a
$k$-dimensional box of size $d$ contained in $\RRR_i$, $i\in\{1,\ldots,m\}$.
\EnumerateStop
From (\hyperref\Btwentyninezero{1}), (\hyperref\Btwentyninefive{\ref[B29:five]}) and (\hyperref\Btwentyninesix{\ref[B29:six]}) it follows that
\Enumerate \itemno=8
\item[B29:nine] \hyperdef\Btwentyninenine{item}{nine}{} $\ds{f(\z_2)C(\z_1)D(\z_2)=f(
\z_1)C(\z_2)D(\z_1)\prod_{\v\in V}\gp j{\z_1\cdot
\v}{\z_2\cdot\z}\frac{a_{\v}(j)}{b_{\v}(j)}}$ if $
\z_1\in \RRR_i$ and $\z_2-\z_1\in S$, $i\in\{1,\ldots,m\}$.
\EnumerateStop
Since any $\R_i$ of measure zero can be absorbed into~$H$ we may assume
$\R_i$ is not a set a measure zero for $i\in\{1,\ldots,m\}$.
For each $i\in\{1,\ldots,m\}$ let 
$\R_i'=\{\z\in \R_i\colon C(\z)D(\z)\ne0\}$, and let
$\RRR_i'=\{\z\in \RRR_i\colon C(\z)D(\z)\ne0\}$.
 Since $\R_i$ is not a set of measure zero, $\R_i$
contains a 
$k$-dimensional box of size~$d$ by Lemma~\hyperref\lemBtwentyone{\ref[lem:B21]}.
Hence, by Lemma~\hyperref\lemBfifteen{\ref[lem:B15]}, $C(
\z)D(\z)$ is nonzero for some $\z\in \R_i$, and hence $\R_i'$ is
not empty.

We claim $\R_i'$ is $S$-path connected in $\RRR_i'$ for each
$i\in\{1,\ldots,m\}$. Let $\z_1$ and $\z_2\in \R_i'$ be given.
We construct a sequence $\v_1',\ldots,\v_\ell'$ in $\RR_i'$
such that 
$\v_1'=\z_1$, $\v_\ell'=\z_2$, and $\v_j'-\v_{j-1}'\in S$
for $j=2,\ldots,\ell$.
Since $\R_i$ is lattice path connected in $\RR_i$, there exists a sequence
$\v_1,\ldots,\v_\ell$ in $\RR_i$ such that
$\v_1=\z_1$, $\v_{\ell}= \z_2$, and $\v_j-\v_{j-1}\in
\{\pm\e_1,\ldots,\pm\e_k\}$. For each $j\in\{1,\ldots,\ell\}$ there is a 
$k$-dimensional box of size $d$
containing $\v_j$ contained in $\RRR_i$. This box contains a point
$\v_j'$ such that $C(\v_j')D(\v_j')\ne0$, since the total
degree of $CD$ is~$d$. Obviously, we can take $\v_1'=\z_1$ and $\v_{\ell}'=
\z_2$. By the construction of $S$, it's clear that $\v_j'-\v_{j-1}'\in S$ for
$j\in\{2,\ldots,\ell\}$. Thus, $\R_i'$ is $S$-path connected in $\RRR_i'$ as
claimed.


For all $\z_1$ and $\z_2\in\RRR_i'$, 
define 
$$g(\z_1,\z_2)
=\frac{C(\z_2)}{C(\z_1)}
\frac{D(\z_1)}{D(\z_2)}
\prod_{\v\in V}\gp j{\z_1\cdot\v}{\z_2\cdot\v}
\frac{a_{\v}(j)}{b_{\v}(j)}.$$
We claim that for all $\z_1$ and $\z_2\in \R_i'$ ,
$$ f(\z_2) =f(\z_1)g(\z_1,\z_2). $$
We show by induction on $j$ that if $\{\z_\nu\}_{\nu\ge 1}$ is a sequence
in $\RRR_i$ such that $\z_\nu-\z_{\nu-1}\in S$ for all $\nu\ge 2$,
then 
$$ f(\z_j)=f(\z_1)g(\z_1,\z_j) $$
for all $j\ge 1$.
The statement is clearly true if $j=1$.  Suppose inductively that 
$f(\z_{j-1})=f(\z_1)g(\z_1,\z_{j-1})$.  Since $\z_j-\z_{j-1}\in S$,
it follows by 
(\hyperref\Btwentyninenine{\ref[B29:nine]})
that $f(\z_j)=f(\z_{j-1})g(\z_{j-1},\z_j)$.
Hence, 
$f(\z_j)=
f(\z_1)g(\z_1,\z_{j-1})
g(\z_{j-1},\z_j)$.
Using the fact that 
$\gp iab\,\gp ibc=\gp iac$,
it is easily seen that 
$ f(\z_j)=f(\z_1)g(\z_1,\z_j) $,
as was to be shown.
Since $\R_i'$ is $S$-path connected in $\RRR_i'$, it follows that
$ f(\z_2) =f(\z_1)g(\z_1,\z_2)$ for all $\z_1$ and $\z_2\in \R_i'$.


Let $\z_0\in \R_i'$. 
We claim that for all 
$\z\in \R_i$ 
such that $D(\z)\ne0$,
$$f(\z)=f(\z_0)g(\z_0,\z).$$
If $\z\in\R_i'$, we're done.
If $\z\notin \R_i'$, then, since $D(\z)\ne0$,
we must have $C(\z)=0$.
In this case~(\hyperref\Btwentyninenine{\ref[B29:nine]}) implies $f(\z)=0$.
But $g(\z_0,\z)=0$, 
so $f(\z)=f(\z_0)g(\z_0,\z)$.

\ProofStop

\Corollary[cor:B30]  \hyperdef\corBthirty{Corollary}{B30}{} Let $f$ be an \hyperref\defBthirteena{honest} \hgt\ on $\Z^k$. Then there
exist relatively prime polynomials $C$ and $D\in K[\z]$ (as in
Theorem~\hyperref\thmBtwentynine{\ref[thm:B29]}) and  a finite number of \hyperref\defBeighteen{polyhedral regions}
$\R_1,\ldots,\R_L$ such that
$\Z^k$ is the union of the $\R_\ell$ and a \hyperref\defBeleven{set of measure zero} and for each
region $\R_\ell$
there exist a finite set $V\subset\Z^k$ and
univariate polynomials $a_{\v}$ and $b_{\v}\in K[z]$ and an
integer $n_{\v}$ for each $\v\in V$ such that, for all $
\z\in \R_\ell$,
\Enumerate
\item $\ds{f(\z)=\frac{C(\z)}{D(\z)}\prod_{\v\in
V}\prod_{j=1}^{\v\cdot\z+n_{\v}}\frac{a_{
\v}(j)}{b_{\v}(j)}}$
for all $\z\in \R_{\ell}$ for which $D(\z)\ne 0$,
\item for all $\v\in V$, $\v\cdot\z+n_{\v}$ is a
positive integer, and
\item for all $\v\in V$ and $j$, $1\le j\le\v\cdot
\z+n_{\v}$, $a_{\v}(j)\ne0$ and $b_{\v}(j)\ne0$.
\EnumerateStop
\CorollaryStop


Corollary~\hyperref\corBthirty{\ref[cor:B30]} applies to \hgt s on polyhedral regions
other than $\Z^k$ via Lemma~\hyperref\lemBtwentyoneb{\ref[lem:B21b]}.

\Proof
This follows from Theorem~\hyperref\thmBtwentynine{\ref[thm:B29]} and Lemma~\hyperref\lemBtwentyfive{\ref[lem:B25]}.
The product 
$$
\prod_{\v\in V}\gp j{\z_0\cdot\v}{\z\cdot
\v}\frac{a_{\v}(j)}{b_{\v}(j)}$$
that occurs in the expression for $f(\z)$ in Theorem \hyperref\thmBtwentynine{\ref[thm:B29]}
is a weakly factorial \hgt\ on the region $\R_i$.
By lemma \hyperref\lemBtwentyfive{\ref[lem:B25]}, the region $\R_i$ can be divided into 
polyhedral subregions such that the weakly factorial
\hgt\ is factorial on
each subregion.
\ProofStop

\Corollary[cor:B31]  \hyperdef\corBthirtyone{Corollary}{B31}{} Let $f$ be an \hyperref\defBthirteena{honest} \hgt\ on $\Z^k$
over a field $K$ that is algebraically closed. Then there exist
relatively prime polynomials $C$ and $D\in K[\z]$ (as in
Theorem~\hyperref\thmBtwentynine{\ref[thm:B29]}) and a finite number of \hyperref\defBeighteen{polyhedral regions}
$\R_1,\ldots,\R_L$ such that $\Z^k$ is the union of the $\R_{\ell}$ and
a \hyperref\defBeleven{set of measure zero}, and for each region $\R_{\ell}$ there exist a
vector $\vecgamma\in K^k$, constants $m_1,\ldots,m_p,n_1,\ldots,n_q\in
K$, vectors $\v_1,\ldots,\v_p,\w_1,\ldots,\w_q\in\Z^k$,
and integers $r_1,\ldots,r_p,s_1,\ldots,s_q$ such that
\Enumerate
\item for all $\z\in \R_{\ell}$ such that $D(\z)\ne0$,
$$f(\z)=
\gamma_1^{z_1}\cdots\gamma_k^{z_k}
\frac{C(\z)}{D(\z)}
\frac{\prod_{i=1}^p(m_i)_{\v_i\cdot
\z+r_i}}{\prod_{j=1}^q(n_j)_{\w_j\cdot\z+s_j}};$$
\item for all $i$ and $j$ and $\z\in \R_{\ell}$, $\v_i\cdot
\z+r_i$ and $\w_j\cdot\z+s_j$ are positive;
\item the Pochhammer symbols occurring
in the products are nonzero.
\EnumerateStop
\CorollaryStop


Corollary~\hyperref\corBthirtyone{\ref[cor:B31]} applies to \hgt s on polyhedral regions
other than $\Z^k$ via Lemma~\hyperref\lemBtwentyoneb{\ref[lem:B21b]}.

\Proof This corollary is an immediate consequence of
Corollary~\hyperref\corBthirty{\ref[cor:B30]} and Lemma~\hyperref\lemBtwentythree{\ref[lem:B23]}.
\ProofStop


