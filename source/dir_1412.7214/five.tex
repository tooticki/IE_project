\hyperdef\chapteRv{ChapterSection}{chapteRv}{}
\Chapter A Solution to a Problem of Cameron\\ 
On Sum-free Complete Sets


\hyperdef\chapteRvSECi{ChapterSection}{chapteRvSECi}{}
\Section Introduction

For any subsets $A$ and $B$ of an additive group $G$, define
$A+B=\{a+b:a \in A \text{ and } b \in B \}$ and $-A= \{-a:a
\in A \}$.  A subset $S$ of $G$ is said to be sum-free,
complete, and symmetric respectively if $S+S \subset S^\c$,
$S+S \supset S^\c$, and $S=-S$.  Hence, $S$ is sum-free and
complete if and only if $S+S=S^\c$.


Cameron observed that for any sufficiently small modulus $m$, 
every sum-free complete set in $\Z/m\Z$ is also symmetric.
In fact, Calkin found that $m=36$ is the smallest modulus  for which
there is a sum-free complete set that is not
symmetric \hyperref\CamPortrait{\bibref[CamPortrait]}.  Cameron asked if there exists
such a set for all sufficiently large moduli \hyperref\CamPortrait{\bibref[CamPortrait]}.
We answer Cameron's question by showing
there exists such a set for all moduli greater than or 
equal to $890626$.

We also show that every sum-free complete set in $\Z/m\Z$ that
is not symmetric can be used to construct a counterexample
to a conjecture of J.H.~Conway.  
 Conway conjectured that for any finite set
$S$ of integers, $|S+S| \le |S-S|$.  
Conway's conjecture
was disproved by Marica \hyperref\Marica{\bibref[Marica]}.  Later Stein showed
how to make the ratio ${|S+S|}/{|S-S|}$ arbitrarily large
\hyperref\Stein{\bibref[Stein]}.  We show that if $S$ is sum-free and complete
modulo $m$ but not symmetric, then $|S+S|>|S-S|$; hence, $S$
is a counterexample to a modular version of Conway's conjecture.
Further, we show that if $S' \subset \Z$ is a certain set
derived from $S$, then $|S'+S'| > |S'-S'|$; hence, $S'$ is a
counterexample to Conway's conjecture proper.

The history of sum-free sets begins with Schur who 
showed that the positive integers can not be partitioned
into finitely many sum-free sets \hyperref\Schur{\bibref[Schur]}.  Sum-free sets
have been used to find lower bounds for Ramsey numbers
(see pp.~28, 128, 264 in~\hyperref\Wallis{\bibref[Wallis]}).  Cameron describes  some
applications of sum-free sets and poses several problems
\hyperref\CamGraph{\bibref[CamGraph]},\hyperref\CamStructure{\bibref[CamStructure]},\hyperref\CamPortrait{\bibref[CamPortrait]}.  George Andrews
observed that sum-free complete
sets play a role in partition identities (personal communication). 
For example, the
set $\{1,4\} \in \Z/5\Z$, which arises in the Rogers-Ramanujan
Identities (p.~109 in~\hyperref\Andseventysix{\bibref[And76]}), is sum-free, complete,
and symmetric.  Calkin showed that the number of sum-free
sets contained within the first $n$ integers is $o(2^{n(1/2
+\epsilon)})$ for every $\epsilon > 0$ \hyperref\Calkin{\bibref[Calkin]}.  



\hyperdef\chapteRvSECii{ChapterSection}{chapteRvSECii}{}
\Section Cameron's problem

For any $S \subset \Z$ and $a$, $b \in \Z \cup \{-\infty,\infty\}$
define $S_a^b=\{s \in S : a \le s \le b \}$.
The following sets $S_1, \dots, S_5$ are used as building
blocks in the construction of infinite families of modular
sum-free complete sets that are not symmetric.
Define 
$$ 
\align
S_1&= 
   (-3+5\Z)_{-\infty}^{-354} \cup 
   (F_1)_{-353}^{353} \cup 
   (3+5\Z)_{354}^{\infty}, \\
S_2&= 
   (-1+5\Z)_{-\infty}^{-192} \cup 
   (F_2)_{-191}^{191} \cup 
   (1+5\Z)_{192}^{\infty}, \\
S_3&= 
   (-4+5\Z)_{-\infty}^{-185} \cup 
   (F_3)_{-184}^{184} \cup 
   (4+5\Z)_{185}^{\infty}, \\
S_4&= 
   (-2+5\Z)_{-\infty}^{-253} \cup 
   (F_4)_{-252}^{252} \cup 
   (2+5\Z)_{253}^{\infty}, 
	\quad \text{and} \\ 
S_5&= 
   (-1+3\Z)_{-\infty}^{-95} \cup 
   (F_5)_{-94}^{94} \cup 
   (1+3\Z)_{95}^{\infty}, \\
\endalign
$$ where 
$$ 
\allowdisplaybreaks
\align 
F_1=&  \{-6, 3\} \cup  \pm \{1, 8, 13, 17, 22, 27, 38, 42, 53, 58, 
62, 67, 72, 74, 86, 88, 93, 98, 107, \\ 
& \quad 117, 119, 121, 133,137, 142, 147, 152, 168, 173, 178, 182, 
187, 192, 197, 208, \\ 
& \quad 213, 218, 222, 227,232, 243, 248, 253,
288, 293, 298, 323, 328, 333, 338\},  \\ 
F_2=&  \{-6, 3\} \cup
\pm \{1, 8, 13, 17, 22, 27, 38, 42, 53, 58, 62, 72, 74, 86, 88,
93, 109, 119, \\ 
&\quad   121, 156, 166\},  \\ 
F_3=&  \{-6, 3\}
\cup  \pm \{1, 8, 13, 17, 22, 27, 38, 42, 53, 58, 62, 67, 72, 74,
86, 88, 93, 98, 119, \\ 
&\quad   149, 154,159,164,169\},  \\ 
F_4=&
\{-6, 3\} \cup  \pm \{1, 8, 13, 17, 22, 27, 38, 42, 53, 58, 62,
67, 72, 74, 86, 88, 93, 98, 107,  \\ 
&\quad  117, 119, 121, 133,
137,142, 147, 152, 182, 187, 192, 197, 202, 222, 227, 232, \\ 
&\quad
237, 247\}, \quad \text{and}   \\ 
F_5=&  \{ -1, 2 \} \cup  \pm \{5,
8, 14, 17, 29, 40, 44, 47, 67, 70, 79, 82, 85\}. \\ 
\endalign 
$$


Lemma~\hyperref\lemfivedashone{\ref[lem:5-1]} reduces to a computation the problem of determining
if a set of the form of $S_1, \dots , S_5$ is sum-free and
complete.

\Lemma[lem:5-1]  \hyperdef\lemfivedashone{Lemma}{5-1}{} Let $S = (A+m\Z)_{-\infty}^{-a} \cup
F_{-a+1}^{b-1} \cup (B+m \Z)_b^{\infty}$ where $a$, $b$, and
$m$ are positive integers and  $A$, $F$, and $B \subset
\Z$.  The set $S$ is sum-free and complete in $\Z$  if and only if
$$(S^\c)_{-2a-2m}^{2b+2m} = (S_{-(2a+b+3m)}^{2b+a+3m}
+ S_{-(2a+b+3m)}^{2b+a+3m})_{-2a-2m}^{2b+2m}.$$
\LemmaStop

\Proof Let $i=-(2a+2m)$, $j=2b+2m$,
$k=-(2a+2b+3m)$, and $l=2b+a+3m$.  We need to show that $S$ 
is sum-free and complete if and only if $(S^\c)_{i}^{j}
= (S_{k}^{l}+S_{k}^{l})_{i}^{j}$.  We first show that $S $ is
sum-free and complete if and only if $(S^\c)_{i}^{j} = 
(S+S)_{i}^{j}$, and then show that $(S+S)_{i}^{j} =
(S_{k}^{l}+S_{k}^{l})_{i}^{j} $.  
To prove the first assertion
it suffices to show that if $(S^\c)_{i}^{j} = (S+S)_{i}^{j}$,
then $S$ is sum-free and complete; the implication in the
other direction is trivial. 

We show that if $(S^\c)_{i}^{j} = (S+S)_{i}^{j}$, then $S$
is sum-free.  Suppose the contrary and let $s_1$, $s_2$, and
$s_3 \in S$ be such that $s_1+s_2=s_3$ and $|s_3|$ is minimal.
By assumption $(S+S)_{i}^{j} \subset
S^\c$, so $s_3 > j$ or $s_3 < i$.  We consider the case $s_3
> j$. Let $s_3'=s_3-m$. Since $s_3 > j > b+m$, it follows that $s_3 \in
(B+m\Z)_b^{\infty}$ and $s_3' \in (B+m\Z)_b^{\infty}$.
Since $s_3 \ge 2b+2m$, it follows that $s_1 \ge b+m$ or $s_2
\ge b+m$.  We may assume without loss of generality that $s_1 \ge
b+m$. Let $s_1'=s_1-m$.  
Since $s_1 \ge b+m$, it follows that $s_1 \in (B+m \Z)_b^{\infty}$ and
$s_1' \in (B+m \Z)_b^{\infty}$.  But $s_1'+s_2=s_3'$ and $|s_3'|<|s_3|$,
contradicting the minimality of $|s_3|$.  The case $s_3 < i$ is
similar.

We show that if $(S^\c)_{i}^{j} = (S+S)_{i}^{j}$, then $S$
is complete.  Suppose the contrary and let $c \in S^\c$ be
such that $c \notin S+S$ and $|c|$ is minimal.
By assumption, $(S^\c)_{i}^{j} \subset S+S$, so $c > j$ or
$c < i$.  We consider the case $c > j$.  
Let $T=(B+m\Z)^\c$. 
Since 
$c > j > b +m$, 
it follows that 
$c \in T_{b+m}^{\infty}$. 
Let $c'=c-m$.
Since $T=T+m\Z$, 
it follows that
$c' \in T_b^{\infty} \subset S^\c$.  
By the minimality of $|c|$, it follows that  $c'=s_1+s_2$ where $s_1$ and $s_2
\in S$.  Since $c' > j-m > 2b$, it follows that $s_1 \ge b$
or $s_2 \ge b$.  We may assume without loss of generality that
$s_1 \ge b$.  Let $s_1'=s_1+m$. Since $s_1 \ge b$, it follows that 
$s_1 \in (B+m\Z)_b^{\infty}$ and hence 
$s_1' \in (B+m\Z)_b^{\infty}$.
But $c=s_1'+s_2$ contradicting the assumption $c \notin S+S$.
The case $c < i$ is similar.

We complete the proof by showing that $(S+S)_{i}^{j} =
(S_{k}^{l}+S_{k}^{l})_{i}^{j}$.  It suffices to show that
$(S+S)_{i}^{j} \subset S_{k}^{l}+S_{k}^{l}$.  Let $c \in
(S+S)_{i}^{j}$ and let $s_1$ and $s_2 \in S$ be such that $s_1
+s_2=c$ and $|s_1| + |s_2|$ is minimal.  We claim that $s_1$,
$s_2 \in S_{k}^{l}$ and, hence, $c \in (S_{k}^{l}+S_{k}^{l})_{i}^{j}$.
Suppose to the contrary $s_1 > l$ or $s_1 < k$.  We consider
the case $s_1 > l$.  Let $s_1'=s_1-m$ and $s_2'=s_2+m$. Since
$s_1 > l > b+m$, it follows that $s_1 \in (B+m \Z)_b^{\infty}$
and $s_1' \in (B+m\Z)_b^{\infty}$.  Since $s_1 >l= 2b+a +3m$
and $c \le 2b+2m$, it follows that $s_2=c-s_1 \le -a -m$.
Since $s_2 \le -a-m$, it follows that $s_2 \in (A+m
\Z)_{-\infty}^{-a}$ and $s_2' \in (A+m \Z)_{-\infty}^{-a}$.
But $s_1'+s_2'=c$ and $|s_1'|+|s_2'|<|s_1|+|s_2|$, 
contradicting the minimality of $|s_1|+|s_2|$.
Hence, the claim is true in the case $s_1 > l$.  The case $s_1
< k$ is similar.
\ProofStop

Using Lemma~\hyperref\lemfivedashone{\ref[lem:5-1]}, it is easily shown that the sets $S_1,\dots,S_5$
are sum free and complete.  Taking $F=F_1$, $a=b=354$, $m=5$,
and $-A=B=\{3\}$ and verifying that $((S_1)^\c)_{-718}^{718}
= ((S_1)_{-1077}^{1077}+(S_1)_{-1077}^{1077})_{-718}^{718}$,
it follows from Lemma~\hyperref\lemfivedashone{\ref[lem:5-1]} that $S_1$ is sum-free and complete.
The cases corresponding to $S_2,\dots,S_5$ are proved similarly.




Let $T\subset \Z$ and let $n$ be a positive integer.
A sum-free complete set of the form $T + n\Z$ can
be identified with a sum-free complete set modulo $n$; 
the subset $T+n\Z$ of $\Z$ is sum-free and complete if and only 
if the subset $\{t+n\Z:t \in T\}$ of $\Z/n\Z$ is sum-free and 
complete.
Lemma~\hyperref\lemfivedashtwo{\ref[lem:5-2]} shows that a sum-free complete set of the form of
$S_1,\dots,S_5$ can be used to construct a sum-free complete set modulo
$n$ for all sufficiently large $n$ in an arithmetic progression.

\Lemma[lem:5-2]  \hyperdef\lemfivedashtwo{Lemma}{5-2}{} Let $S=(A+m\Z)_{-\infty}^{-a} \cup
F_{-a+1}^{b-1} \cup (B+m\Z)_b^{\infty}$ where $a$, $b$ and
$m$ are positive integers, $A$, $F$ and $B \subset \Z$, and
for some integer $d$, $d+A+m \Z=B+m \Z$.  If $S $ is sum-free
and complete in $\Z$, then $S_{-n/2}^{n/2} +n \Z$ is sum-free and
complete in $\Z$ for all $n$ such that $n \ge 2m+ 4\max(a,b)-2\ $ and
$\ n\cong d \pmod m$.
\LemmaStop

\Proof Let $n$ be such that $n \ge 2m+
4\max(a,b)-2$ and $n\cong d \pmod m$.

We show that $S_{-n/2}^{n/2} +n \Z$ is sum-free. Suppose the
contrary and let $s_1$, $s_2$, and $s_3 \in S_{-n/2}^{n/2} $
be such that $s_1+s_2\cong s_3 \pmod n$.  Since $|s_1+s_2-s_3| \le
|s_1|+|s_2|+|s_3| \le 3n/2 $ and $s_1+s_2-s_3 \neq 0$, it
follows that $s_1+s_2-s_3= \pm n$.  We consider the case
$s_1+s_2-s_3=n$.  Let $\sg(1)=\sg(2)=-\sg(3)=1$. 
If for all $i\in \{1,2,3\}$
either $s_i \in F_{-a+1}^{b-1}$ or $s_i \sg(i) \le
0$, then $s_1+s_2-s_3 \le 3b < n$, so we may assume there
exists $j=1$, $2$, or $3$ such that $s_j \notin F_{-a+1}^{b-1}$
and $s_j \sg(j) >0$.  In fact, we may assume without loss of
generality that either $j=1$ or $j=3$.

We consider the case $j=1$. Let $s_1'=s_1-n$. 
Since $s_1 \notin F_{-a+1}^{b-1}$ and $s_1 >0$,  
it follows that $s_1 \in B+m\Z$
and $s_1' \in B+m\Z-n$.  Since $n\cong d \pmod m$, it follows
that $m\Z-n=m\Z-d$. Hence, $s_1' \in B+m\Z-d=A+m\Z$.  Since
$s_1 \le n/2$, it follows that $s_1' \le -n/2 \le -a$.  Hence,
$s_1' \in (A+m\Z)_{-\infty}^{-a} \subset S$.  But $s_1'+s_2=s_3$,
contradicting the assumption that $S$ is sum-free.

The case $j=3$ is similar.  This completes the proof in the
case $s_1+s_2-s_3=n$.  The case $s_1+s_2-s_3=-n$ is similar.
We have shown that $S_{-n/2}^{n/2} +n \Z$ is sum-free.



We show $S_{-n/2}^{n/2} +n \Z$ is complete. Let $c \in S^\c$
be such that $|c| \leq n/2$.  We need to show that $c \cong t_1
+t_2 \pmod n$ for some $t_1$ and $t_2 \in S_{-n/2}^{n/2}$.
Since $S$ is complete, $c= s_1 + s_2$ for some $s_1$ and  $s_2
\in S$. Let $s_1$ and $s_2$ be such that $|s_1-s_2|$ is minimal.
If $s_1$ and $s_2$ have the same sign, $|s_1|+|s_2|=|c|\le
n/2$. Hence, $s_1$ and $s_2 \in S_{-n/2}^{n/2}$ and we are
done, so assume $s_1 >0$ and $s_2 <0$. 

We claim that  $s_1 <b+m$ or $s_2 >-a-m$. 
Suppose the contrary and let $s_1'=s_1-m$ and $s_2'=s_2+m$.  
Since $s_1 \ge b+m$, 
it follows that $s_1$ and $s_1' \in (B+m\Z)_b^\infty$. 
Hence,  $s_1' \in S$. Similarly,  $s_2' \in S$. 
But $s_1' +s_2'=c$ and $|s_1'-s_2'|<|s_1-s_2|$,
contradicting the minimality of $|s_1-s_2|$. Hence,
$s_1 <b+m$ or $s_2 >-a-m$ as claimed. 
 
We consider the case $s_2 >-a-m$.  If $s_1 \leq n/2$ we are 
done, so assume
$s_1 >n/2$ and let $s_1'=s_1-n$.  Since $s_1 > n/2 > b$, it
follows that $s_1 \in B+m\Z$.  Arguing as in the case $j=1$
above, $s_1' \in A+m\Z$.  Since $c \le n/2$ and $s_2 > -a-m$,
it follows that $s_1' = c-s_2-n < -n/2 +a+m $. Since $n \ge
2m +4a-2$, it follows that $ -n/2+ m +a \le -a +1$.  Hence,
$s_1' < -a+1$. Since $s_1' \le -a$ and $s_1' \in A+m\Z$, it
follows that $s_1' \in S$.  Since $s_1 >n/2$, it follows that
$s_1' > -n/2 $, and hence $s_1' \in S_{-n/2}^{n/2}$.  But
$s_1' +s_2 \cong c \pmod n$.  The case $s_1 <b+m$ is similar.
\ProofStop

\Theorem[thm:5-1]  \hyperdef\thmfivedashone{Theorem}{5-1}{} For all $n\ge 890626$,  there exists a sum-free complete set 
in $\Z/n\Z$ that is not symmetric.  
\TheoremStop


\Proof Applying Lemma~\hyperref\lemfivedashtwo{\ref[lem:5-2]} with $S=S_1$,
$A=\{-3\}$, $B=\{3\}$, $F=F_1$, $a=b=354$, $m=5$, and $d=1$,
it follows that for every $n$ such that $n \ge 1424 $ and $n
\cong 1 \pmod 5 $, the set $(S_1)_{-n/2}^{n/2} +n\Z$ is sum-free 
and complete.
Equivalently, for every $n$ such that $n \ge 1426 $ and $n \cong
1 \pmod 5 $, the set 
$T_n=\{s + n\Z : s \in (S_1)_{-n/2}^{n/2}  \} \subset
\Z/n\Z$ is sum-free and complete. Further, it is clear from
the form of $S_1$ that $3 \in T_n $ and $-3 \notin T_n$.
Hence, $T_n$ is not symmetric.  Thus, we have shown that
there is a sum-free complete set that is not symmetric for
all moduli in the set $R_1=\{m \geq 1426 : m \cong 1 \pmod 5
\}$.  Similar arguments using the sets $S_2, \dots S_5$ show
that there is a sum-free complete set that is not symmetric
for all moduli in sets
$R_2=\{m \geq 777 : m \cong 2 \pmod 5\} $, 
$R_3=\{m \geq 748 : m \cong 3 \pmod 5\} $, 
$R_4=\{m \geq 1024 : m \cong 4 \pmod 5\} $, and
$R_5=\{m \geq 386 : m \cong 2 \pmod 3\}$.

It is easily seen that if there is a sum-free complete set
that is not symmetric for the modulus $m$, then there is such
a set for any modulus that is a multiple of $m$.  Hence, there
is a sum-free complete set that is not symmetric for any
modulus with a divisor in the set $R= R_1 \cup R_2 \dots \cup
R_5$.

It remains only to show that for all $n \geq 890626$, $n$
has a divisor in the set $R$.  Let $n$ be greater than or
equal to $890626$ and write $n$ in the form $n=5^a b$ where
$b$ is not divisible by $5$.  Since $890626 = 5^4 \ 1425 +1$,
either $a \geq 5$ or $b \geq 1426$.  If $a \geq 5$, then
$5^5=3125$ divides $n$.  Since $3125 \in R_5$, it follows that
$3125 \in R$.  If $b \geq 1426$, then $b \in R$ since the set
$R_1 \cup R_2 \dots \cup R_4$ contains every number greater
than or equal to $1426$ that is not a multiple of $5$.  In
either case $n$ has a divisor in $R$.  We have shown that for
every $n$ greater than or equal to $890626$ there is a sum-free
complete set in $\Z/n\Z$ that is not symmetric.  
\ProofStop

The condition that $n\ge 890626$ in Theorem~\hyperref\thmfivedashone{\ref[thm:5-1]} is not sharp and can
undoubtedly be considerably reduced.






\hyperdef\chapteRvSECiii{ChapterSection}{chapteRvSECiii}{}
\Section Conway's conjecture

The following theorem shows that if $S$ is sum-free and complete
but not symmetric modulo $m$, then $S$ must be a counterexample
to a modular version of Conway's conjecture.

\Theorem[thm:5-2]  \hyperdef\thmfivedashtwo{Theorem}{5-2}{} If a set $S$ is sum-free and complete but
not symmetric modulo $m$, then $|S+S| > |S-S|$.
\TheoremStop

Before beginning the proof of Theorem~\hyperref\thmfivedashtwo{\ref[thm:5-2]}, it is useful to
observe that $S+S \subset S^\c$ if and only if $S-S \subset
S^\c$; there are no solutions $s_1,s_2,s_3 \in S$ to
$s_1+s_2=s_3$ if and only if there are no solutions to
$s_3-s_2=s_1$

\Proof Suppose to the contrary $S$ is
sum-free and complete but not symmetric modulo $m$ and $|S+S| \le
|S-S|$.  Since $S$ is sum-free, $S+S \subset S^\c$ and hence
$S-S \subset S^\c$.  Since $S^\c=S+S$, it follows that
$S-S \subset S+S$.  But $|S+S| \le |S-S|$, so $S-S = S+S$ and
hence $S-S = S^\c$.  The set $S-S$ is symmetric, hence
$S^\c$ is symmetric, hence $S$ is symmetric, contrary to
assumption.
\ProofStop

Theorem~\hyperref\thmfivedashthree{\ref[thm:5-3]} shows that if $S$ is sum-free and complete but not
symmetric modulo $m$, then $S$ can be used to construct a
counterexample to  Conway's conjecture.

\Theorem[thm:5-3]  \hyperdef\thmfivedashthree{Theorem}{5-3}{} Let $A \subset \Z/ m \Z$ be sum-free and
complete but not symmetric, let $r={|A-A|}/ {|A+A|} $, and
let $S$ be the set of integers congruent modulo $m$ to a member
of $A$.  For all $n \ge 2m/{(1-r)}$, we have ${|S_0^n+S_0^n|}
> {|S_0^n-S_0^n|}$.
\TheoremStop

We require the following lemma for the proof of Theorem~\hyperref\thmfivedashthree{\ref[thm:5-3]}.

\Lemma[lem:5-3]  \hyperdef\lemfivedashthree{Lemma}{5-3}{} Let $A$, $B \subset \Z$, $S=A+m \Z$, and
$T=B+m \Z$.  For any integers $i$, $j$, $k$, and $l$ such 
that $j-i\ge m-1$
and $k-l \ge m-1$, $$(S_i^j+T_k^l)_{i+k+m-1}^{j+l-m+1} =
(S+T)_{i+k+m-1}^{j+l-m+1}.  $$ 
\LemmaStop

\Proof To prove Lemma~\hyperref\lemfivedashthree{\ref[lem:5-3]}, we need only show $$(S_i^j+T_k^l)_{i+k+m-1}^{j+l-m+1}
\supset (S+T)_{i+k+m-1}^{j+l-m+1}, \tag 1$$ since the containment
in the other direction is trivial.  We first consider the case
$j-i = k-l = m-1$. Let $s=j+l-m+1=i+k+m-1$. Since $j-i \ge
m-1$, $S=S_i^j+ m \Z$.  Similarly, $T=T_k^l+ m \Z$.  Hence,
$$ 
\aligned
S+T &= (S_i^j+ m \Z)+(T_k^l+ m \Z)= S_i^j+T_k^l+ m \Z \\ 
&= S_i^j+T_k^l+ (m \Z_{-\infty}^{-1} 
	\cup \{0\} \cup m \Z_1^{\infty}) \\ 
&= (S_i^j+T_k^l+ m \Z_{-\infty}^{-1}) \cup (S_i^j+T_k^l)
      \cup (S_i^j+T_k^l+ m \Z_1^{\infty}) \\ 
&\subset \Z_{-\infty}^{j+l-m} 
	\cup (S_i^j+T_k^l) \cup \Z^{\infty}_{i+k+m}\\
&\subset \Z_{-\infty}^{s-1} \cup (S_i^j+T_k^l) 
	\cup \Z^{\infty}_{s+1}.  
\endaligned 
$$ 
It follows that $(S_i^j+T_k^l)_s^s
\supset (S+T)_s^s$ which is that statment of (1) for the
case $j-i = k-l = m-1$.

We derive the general case from the previous one.  Let $j-i
\ge m-1$, $k-l \ge m-1$ and define $ I=\{(a,b,c,d) :\Z_a^b
\subset \Z_i^j,\ \Z_c^d \subset \Z_k^l, \text{ and }b-a=d-c=m-1
\}$.  
$$ \split (S_i^j+T_k^l)_{i+k+m-1}^{j+l-m+1} &\supset
\bigcup_{(a,b,c,d) \in  I} (S_a^b+T_c^d)_{a+c+m-1}^{b+d-m+1}\\ 
&\supset \bigcup_{(a,b,c,d) \in I} (S+T)_{a+c+m-1}^{b+d-m+1}
\text{\quad by the previous case} \\ &\supset 
(S+T)_{i+k+m-1}^{j+l-m+1}, \endsplit $$ so (1) is proved.
\ProofStop

\Proof We now prove Theorem~\hyperref\thmfivedashthree{\ref[thm:5-3]}. Let $n \ge 2m/{(1-r)}$. By Theorem~\hyperref\thmfivedashtwo{\ref[thm:5-2]}, $r <1$
so $n \ge m-1$. Hence, by Lemma~\hyperref\lemfivedashthree{\ref[lem:5-3]}, $$(S_0^n+S_0^n)_{m-1}^{2n-m+1}
= (S+S)_{m-1}^{2n-m+1} = (S+S)_{m-1}^{m-1+km-1} \supset
(S+S)_{m-1}^{m-1+\floor k m-1}\quad , $$ where $k = (2n-2m+3)/m$ and
$\floor k$ is the greatest integer less than or equal to $k$.
Since $S+S$ is the set of integers congruent modulo $m$ to a
member of $A+A$, it follows that $|(S+S)_{a}^{a+im-1}| =
i|A+A|$ for any $a$ and $i \ge 0$.  In particular,
$|(S+S)_{m-1}^{m-1+\floor k m-1}| = \floor k|A+A|$.  Hence
$|S_0^n+S_0^n| >\floor k |A+A| > (k-1) |A+A|.$

Since $S_0^n-S_0^n \subset \Z_{-n}^{n}$, it follows that
$S_0^n-S_0^n \subset (S-S)_{-n}^{n} \subset (S-S)_{-n}^{-n+\ceil j 
m-1}$, where $j= (2n+1)/m$ and $\ceil j$ is the least integer
greater than or equal to $j$.  Arguing as before, 
$|(S-S)_{-n}^{-n+\ceil j m-1}| = 
\ceil j |A-A|$.  Hence, $|S_0^n-S_0^n|< \ceil j |A-A|
< (j+1) |A-A|$.

It follows that $$\frac{ |S_0^n+S_0^n|}{|S_0^n-S_0^n|} >
\frac{(k-1)|A+A|}{(j+1)|A-A|} = \frac{2n-3m+3}{(2n+m+1)r}.$$ 
The right side is greater than or equal to 1 if
and only if $$n \ge \dfrac{(3+r)m+r-3}{2(1-r)}.$$
Since $r<1$, a sufficient condition is that $n\ge 2m/{(1-r)}$.  
\ProofStop
