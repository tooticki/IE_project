
\hyperdef\chapteRii{ChapterSection}{chapteRii}{}
\Chapter The Rational Solutions of \\
$R_iR_j^{\e_i}=R_jR_i^{\e_j}$

The main result of this chapter is Theorem~\hyperref\thmfifteena{\ref[thm:15a]} \hyperref\oresato{\OreSato}, which reveals
the multiplicative structure of solutions to the relation for the term ratios.

\smallskip
\Statement[thm:15a] 
\noindent{\bf{}Theorem~\hyperref\thmfifteena{\ref[thm:15a]}} \ 
Let $R_i\in K(\z)$, $i=1,\ldots,k$, be rational functions such that
$$R_iR_j^{\e_i}=R_jR_i^{\e_j}$$
for all $i,j\in\{1,\ldots,k\}$. Then there exist polynomials $C$ and $D\in K[\z]$, a finite set $V\subset\Z^k$, and univariate polynomials $a_{\v}$ and $b_{\v}\in K[z]$ for each $\v\in V$ such that for all $i\in\{1,\ldots,k\}$,
$$R_i(\z)=\frac{C(\z+\e_i)}{C(\z)}\frac{D(\z)}{D(\z+\e_i)}\prod_{\v\in V}\gp j0{\v_i}\frac{a_{\v}(\z\cdot\v+j)}{b_{\v}(\z\cdot\v+j)}.$$
\StatementStop

Recall that the symbol $\gp jab{}$ denotes 
$\prod_{j=a}^{b-1}$ if $b>a$,
$1$ if $a=b$,
and
$1/\prod_{j=b}^{a-1}$ if $a>b$.



We require the following definitions.

\Definition[def:A1]  \hyperdef\defAone{Definition}{A1}{} Define the \defword{lead term} of a nonzero polynomial $p\in 
K[z_{1},\ldots,z_{k}]$ to be the nonzero term $cz_{1}^{n_{1}}\cdots 
z_{k}^{n_{k}}$ of~$p$ for which the degree vector $(n_{1},\ldots,n_{k})$ is 
maximal in the standard ordering. The polynomial $p$ is 
\defword{monic} if the coefficient of the lead term is~$1$.
\DefinitionStop

For any $\vecgamma=(\gamma_1,\ldots,\gamma_k)\in (K(\z))^k$ such
that $\gamma_i\ne 0$ for each $i\in \{1,\ldots,k\}$ and
$\v=(v_1,\ldots,v_k)\in \Z^k$, we denote
$\gamma_{1}^{v_{1}}\cdots \gamma_{k}^{v_{k}}$
by $\vecgamma^{\v}$.

\Lemma[lem:zero]  \hyperdef\lemzero{Lemma}{zero}{} Let $\z=(z_{1},\ldots,z_{k})$, let $\vec m\in\Z^{k}$, 
and 
let $p=\sum_{\vec n\in\Z^{k}}c_{\vec n}\z^{\vec n}\in 
K[z_{1},\ldots,z_{k}]$ be a polynomial. If the term $c_{\vec m}
\z^{\vec m}$ 
is maximal in the sense that $\z^{\vec m}\mid \z^{\vec n}$ and 
$\vec m\ne\vec n$ 
imply $c_{\vec n}=0$, then for any $\v\in\Q^{k}$, the coefficient of 
$\z^{\vec m}$ in $p^{\v}$ is equal to the coefficient $c_{\vec m}$ of 
$\z^{\vec m}$ in $p$.
\LemmaStop

\Proof By the binomial theorem, if $\z^{\vec m}$ is not a proper 
divisor of $\z^{\vec n}$, then the coefficient of $\z^{\vec m}$ in 
$(\z+\v)^{\vec n}-\z^{\vec n}$ is~$0$. If $\z^{\vec m}$ is a proper 
divisor of $\z^{\vec n}$, then $c_{\vec n}=0$. Hence, in either case, the 
coefficient of $\z^{\vec m}$ in $c_{\vec n}((\z+\v)^{
\vec n}-\z^{\vec n})$ 
is~$0$, and, hence, the coefficient of $\z^{\vec m}$ in $p^{
\v}-p=\sum c_{\vec n}((\z+\v)^{\vec n}-\z^{\vec n})$ is~$0$.
\ProofStop

\Corollary If $p\in K[z_{1},\ldots,z_{k}]$ is monic, then so is 
$p^{\v}$ for any $\v\in\Q^{k}$.
\CorollaryStop

\Corollary[cor:second]  \hyperdef\corsecond{Corollary}{second}{} If $p\in K(z_{1},\ldots,z_{k})$ is nonzero, $
\v\in\Q^{k}$, $c\in K$, and $p^{\v}=cp$, then $c=1$.
\CorollaryStop

\hyperdef\chapteRiiSECi{ChapterSection}{chapteRiiSECi}{}
\Section Gosper's lemma

We require the following adaptation of 
Petkov\hacek sek's refinement~\hyperref\Petninetytwo{\bibref[Pet92]} of 
Gosper's lemma~\hyperref\Gospseventyeight{\bibref[Gosp78]}.

\Lemma[lem:one]  \hyperdef\lemone{Lemma}{one}{} 
Let $K$ be a field of characteristic~$0$, and let $R\in 
K(z_1,\ldots,z_k)$ be a nonzero rational function. There exist polynomials $A$, $B$, 
$C$, and~$D\in K[z_1,\ldots,z_k]$ such that
$$R=\frac AB\frac{C^{\e_{1}}}C\frac D{D^{\e_{1}}}$$
and
\Enumerate
\item $A$ and $B^{n\e_1}$ are relatively prime for all $n\in\Z$,
\item $A$ and $CD^{\e_{1}}$ are relatively prime,
\item $B$ and $C^{\e_{1}}D$ are relatively prime, and
\item $C$ and $D$ are relatively prime.
\EnumerateStop
\LemmaStop

\Proof The lemma is proved by a double application 
of Gosper's 
lemma. The gcds below are with respect to $z_1$. By Gosper's lemma there exist $a$, $b$, 
and~$c\in K(z_2,\ldots,z_k)[z_1]$ such that $\ds{R=\frac ab\frac{c^{\e_1}}c}$ where 
$\gcd(a,b^{n\e_1})=1$ for all nonnegative integers~$n$ and by Petkov\hacek 
sek's refinement we may assume in addition that $\gcd(a,c)=1$ and 
$\gcd(b,c^{\e_1})=1$. Applying this principle again, this time to 
$\ds{\frac ba}$, we can write
$$\frac ba=\frac BA\frac {d^{\e_1}}d\leqno(1)$$
where $A,B,d\in K(z_2,\ldots,z_k)[z_1]$, $\gcd(B,A^{n\e_1})=1$ for all integers~$n\ge0$, 
$\gcd(B,d)=1$, and $\gcd(A,d^{\e_1})=1$. Since $\gcd(B,Ad)=1$, it 
follows from~$(1)$ that $B\mid b$. Similarly, $A\mid a$. Hence, 
$\gcd(A,B^{n\e_1})=1$ for all nonnegative integers~$n$. On the other 
hand, since $\gcd(B,A^{n\e_1})=1$ for all nonnegative integers~$n$, it 
follows that $\gcd(A,B^{n\e_1})=1$ for all nonpositive integers~$n$. 
Hence $\gcd(A,B^{n\e_1})=1$ for all integers~$n$, positive, negative, or 
zero. Let $g=\gcd(c,d)$, let $C=c/g$, and let $D=d/g$. It's easily 
seen that $\ds{R=\frac AB\frac{C^{\e_1}}C\frac D{D^{\e_1}}}$, 
$\gcd(A,CD^{\e_1})=1$, $\gcd(B,C^{\e_1})=1$, and $\gcd(C,D)=1$.

For some $\alpha$ and $\beta\in K[z_2,\ldots,z_k]$, $\alpha A$ and
$\beta B$ are in $K[\z]$. Replace $A$ with $\alpha\beta A$ and $B$
with $\alpha\beta B$; the ratio $A/B$ is unchanged, hence, $R$ is unchanged.
 Similarly, adjust $C$ and $D$ so that $A$, $B$,
$C$, and $D$ are in $K[\z]$. If $A$ and $B$ have a common factor
$d$ (which must be free of $z_1$), we replace $A$ and $B$ by $A/d$
and $B/d$. Since $d=d^{n\e_1}$ it follows that $A$ and
$B^{n\e_1}$ are relatively prime for $n\in\Z$. If $C$ has any
divisor $d$ that is free of $z_1$, replace $C$ with $C/d$. Since
$d^{\e_1}=d$, $C^{\e_1}/C$ is unchanged. Similarly, adjust $D$.
If follows that $A_1$ and $CD^{\e_1}$ are relatively prime, $B_1$
and $C^{\e_1}D$ are relatively prime, $C$ and $D$ are
relatively prime, and
$\ds{R=\frac AB\frac{C^{\e_1}}C\frac D{D^{\e_1}}}$. 
\ProofStop

\Definition[def:A2]  \hyperdef\defAtwo{Definition}{A2}{} Let $R\in K(z_{1},\ldots,z_{k})$ be 
a rational function. The rational function~$R$ is said to be 
\defword{simple} over $K$ if it can be written in the form
$$R(\z)=\bar R(\v\cdot\z)=\bar R(v_1z_1+\cdots+v_kz_k)$$
where $\bar R\in K(z)$ is a univariate rational function and $\v\in\Q^k$.
\DefinitionStop

Hence, $(z_{1}^{2}+z_{2}+3z_{3})^{2}$ is simple over $\Q(z_{1})$ but 
(apparently) not over $\Q$. It's easily proved that it's not simple 
over $\Q$ using Corollary~\hyperref\coroldlemmafour{\ref[cor:oldlemmafour]}.

\Lemma[lem:two]  \hyperdef\lemtwo{Lemma}{two}{} Let $K$ be a field of characteristic~$0$, and let 
$R_{1},\ldots,R_{k}\in K(z_{1},\ldots,z_{k})$ be nonzero rational 
functions satisfying the relation
$$R_{i}R_{j}^{\e_{i}}=R_{j}R_{i}^{\e_{j}}$$
for all $i,j\in\{1,\ldots,k\}$. Then there exist polynomials 
$A_{1},\ldots A_{k}$, $B_{1},\ldots,B_{k}$, $C$, and $D$ such that
$$R_{i}=\frac {A_{i}}{B_{i}}\frac{C^{\e_{i}}}C\frac D{D^{
\e_{i}}}$$
for $i\in\{1,\ldots,k\}$, and the irreducible divisors of $A_{i}$ and 
$B_{i}$ are all \hyperref\defAtwo{simple} for $i\in\{1,\ldots,k\}$.
\LemmaStop

\Proof The one-variable case of Lemma~\hyperref\lemtwo{\ref[lem:two]} is trivial. 
The proof of the two-variable case continues through
section
2.4 and the proof of the higher dimensional cases occupies
sections 2.5 through 2.7.

\def\AB{1}
We prove the two-variable case.
In fact, we prove a stronger version of the two-variable
case that is needed for the proof of the higher dimensional cases:
if 
$$R_{1}=\frac{\bar A_{1}}{\bar B_{1}}\frac{\bar C^{\e_{1}}}{\bar C}
\frac {\bar D}{\bar D^{ \e_{1}}}$$
and
$\bar A_{1}$ and $\bar B_{1}^{n\e_{1}}$ are relatively prime
for all $n\in\Z$,
then we may take
$A_1=\bar A_1$,
$B_1=\bar B_1$,
$C=\bar C$,
and
$D=\bar D$.
By Lemma~\hyperref\lemone{\ref[lem:one]}, such
$\bar A_1$,
$\bar B_1$,
$\bar C$,
and
$\bar D$ exist.
Thus, we assume that
$$R_{1}=\frac{A_{1}}{B_{1}}\frac{C^{\e_{1}}}C\frac D{D^{
\e_{1}}}$$
where 
$$A_{1}\text{ and }B_{1}^{n\e_{1}}\text{ are relatively prime
for all $n\in\Z$. 
}\tag{\AB}$$

Let $\ds{r_{i}=R_{i}\frac C{C^{\e_{1}}}\frac{D^{\e_{1}}}D}$.
Thus, $\ds{r_{1}=\frac{A_{1}}{B_{1}}}$. Let $A_{2}$ and $B_{2}$ be 
relatively prime polynomials such that $\ds{r_{2}=\frac{A_{2}}{B_{2}}}$. 
Using $R_{1}R_{2}^{\e_{1}}=R_{2}R_{1}^{\e_{1}}$ and the 
definition of~$r_i$, it's easily verified that $r_{1}r_{2}^{
\e_{1}}=r_{2}r_{1}^{\e_{2}}$ and, hence,
$$A_{1}A_{2}^{\e_{1}}B_{1}^{\e_{2}}B_{2}=A_{1}^{
\e_{2}}A_{2}B_{1}B_{2}^{\e_{1}}.\leqno(2)$$
We will show that the irreducible divisors of the left side of~$(2)$ 
are simple.

\hyperdef\chapteRiiSECii{ChapterSection}{chapteRiiSECii}{}
\Section The divisibility lattice path

For any irreducible divisor $d$ we construct a lattice path 
$\{T_{i}\}$ in $\Z^{2}$ such that for all $i\ge0$, $d^{T_{i}}$ divides 
one of the four factors on the left side of~$(2)$. Further, the 
factor $T_{i}$ divides is determined by the value of 
$S_{i}=T_{i}-T_{i-1}$, which, since $T_{i}$ is a lattice path, is one 
of the four directions $\e_{1}$, $-\e_{1}$, $\e_{2}$, 
and $-\e_{2}$.

To be precise, define
$$X(-\e_{2})=A_{1},\quad X(\e_{1})=A_{2}^{\e_{1}},\quad 
X(\e_{2})=B_{1}^{\e_{2}},\quad X(-\e_{1})=B_{2}.$$
We will construct $T_{i}$ so that
$$d^{T_{i}}\mid X(S_{i}),\quad\text{where 
}S_{i}=T_{i}-T_{i-1}.\leqno(3)$$
The product of the values of $X$ is the left side of~$(2)$. It will be 
useful to also define
$$Y(-\e_{2})=A_{2}^{\e_{1}},\quad Y(\e_{1})=A_{2},\quad 
Y(\e_{2})=B_{1},\quad Y(-\e_{1})=B_{2}^{\e_{1}}$$
so that the product of the values of~$Y$ is the right side of~$(2)$. 
The relationship between terms on the left side of~$(2)$ and terms on 
the right side is captured by the shift relation
$$(Y(\v))^{\v}=X(\v)\leqno(4)$$
and coprimeness condition
$$X(\v)\text{ and }Y(-\v))\text{ are relatively prime}\leqno(5)$$
for $\v\in\{\pm\e_{1},\pm\e_{2}\}$, which are easily 
verified.

We construct the lattice path $\{T_{i}\}$ by induction. Let an 
irreducible divisor~$d$ of the left side of~$(2)$ be given. Since~$d$ 
is irreducible, $d$ must divide $X(\v)$ for some $
\v\in\{\pm\e_{1},\pm\e_{2}\}$. Define $T_{0}=0$ and 
$S_{0}=\v$. Thus, $d^{T_{0}}\mid X(S_{0})$. Now assume inductively 
that $T_{i}$ and $S_{i}$ have been defined for $0\le i\le j$. Since 
$d^{T_{j}}$ divides the left side of~$(2)$, it must also divide the 
right side, and since $d^{T_{j}}$ is irreducible, $d^{T_{j}}\mid 
Y(\v)$ for some $\v\in\{\pm\e_{1},\pm\e_{2}\}$. Let 
$S_{j+1}=\v$ and let $T_{j+1}=T_{j}+S_{j+1}$. Since 
$d^{T_{j}}\mid Y(S_{j+1})$, it follows that 
$d^{T_{j}+S_{j+1}}\mid(Y(S_{j+1}))^{S_{j}+1}$, and hence 
$d^{T_{j+1}}\mid X(S_{j+1})$ by~$(4)$. This completes the induction 
step. We have constructed $T_{i}$ and $S_{i}$ satisfying~$(3)$.


\hyperdef\chapteRiiSECiii{ChapterSection}{chapteRiiSECiii}{}
\Section What goes up must not come down

Having constructed the path~$T$, we explore what restrictions the 
coprimeness conditions~$(5)$ and~(\AB) impose on the shape of~$T$.

From the condition~$(5)$ we deduce path~$T$ does not traverse the same 
segment of the lattice path in opposite directions. By the path~$T$ 
traversing a segment of the lattice in the direction~$\v$, we 
mean for some $i\ge1$, the end points of the segment are $T_{i-1}$ 
and $T_{i}$, and $T_{i}-T_{i-1}=\v$.

Suppose to the contrary the path traverses the same segment in 
opposite directions. Then for some $i$ and $j>0$ we have 
$S_{i}=-S_{j}$ and $T_{i}=T_{j-1}$. By~$(3)$, $d^{T_{j}}\mid 
X(S_{j})$, hence, by~$(4)$, $d^{T_{j-1}}\mid Y(S_{j})$, hence 
$d^{T_{i}}\mid Y(S_{j})$, and hence $d^{T_{i}}\mid Y(-S_{i})$. But 
$d^{T_{i}}\mid X(S_{i})$ which contradicts condition~$(5)$.

Taking $j=i+1$ we see that the path can't reverse directions 
without going in a perpendicular direction first ($S_{i}\ne -S_{i+1}$).

From the condition~(\AB) we deduce that 
the path~$T$ doesn't go up a 
segment and come down one of its horizontal translates. (By go 
up and come down, we mean traverse in the directions $\e_{2}$ 
and $-\e_{2}$, respectively.) Suppose the contrary. Then for 
some $i$ and $j>0$ and $n\in\Z$ we have $S_{i}=-\e_{2}$, 
$S_{j}=\e_{2}$, and $T_{i}=T_{j-1}+n\e_{1}$. By~$(3)$, 
$d^{T_{j}}\mid X(S_{j})$, hence by~$(4)$ $d^{T_{j-1}}\mid Y(S_{j})$, 
hence $d^{T_{i}-n\e_{1}}\mid Y(S_{j})$, hence by the definition 
of~$Y$, $d^{T_{i}-n\e_{1}}\mid B_{1}$, and hence $d^{T_{i}}\mid 
B_{1}^{n\e_{1}}$. But $d^{T_{i}}\mid X(S_{i})=A_{1}$, which 
contradicts~(\AB).

The condition that the path~$T$ doesn't go up a 
segment and come down one of its horizontal translates
immediately implies a stronger condition:
$$
\text{If $T$ goes up any segment, it must not come down any segment.
}\leqno(6)$$
Suppose to the contrary $T$ goes up to $T_{i}$ 
and down to $T_{j}$ ($S_{i}=\e_{2}$ and $S_{j}=-\e_{2}$), and 
assume that $|i-j|$ is minimal. By the minimality of $|i-j|$, 
$S_{k}\ne\pm\e_{2}$ for any $k$ between $i$ and $j$. It follows 
that the segments $(T_{i-1},T_{i})$ and $(T_{j-1},T_{j})$ are 
horizontal translates of each other, contradicting the original condition.

Finally, from ~$(6)$
it's easy to deduce that the path $T$ is unbounded. 
If the path is unbounded vertically, we are done. Thus, we may assume that for 
some $k\ge0$, $|T_{i}\cdot\e_{2}|\le k$ for all~$i$. It follows 
from~$(6)$ that the path can go in a vertical direction at most~$k$ 
times. Since the path can't reverse direction without going in a 
perpendicular direction first, the path can change horizontal 
direction at most $k$ times. Hence the path is unbounded horizontally.

\hyperdef\chapteRiiSECiv{ChapterSection}{chapteRiiSECiv}{}
\Section The divisor $d$ is simple.

Using the fact that the path $T$ is unbounded, we show that it 
satisfies a nontrivial relation $d=d^{\v}$ for some $
\v\in\Z^{2}$. By the fact that $T$ is unbounded, the set ${\Cal 
T}=\{T_{i}:i\ge0\}$ of values assumed by $T$ is infinite. Let $P$ be 
the left side of~$(2)$. For every $t\in\Cal T$, $d^{t}\mid P$. Since 
$\Cal T$ is infinite and $P$ has only finitely many monic divisors, 
$cd^{t_{1}}=d^{t_{2}}$ for some $t_{1}$ and $t_{2}\in\Cal T$ and 
$c\in K$. By Corollary~\hyperref\corsecond{\ref[cor:second]}, $c=1$. Letting $\v=t_{1}-t_{2}$, we have 
$d^{\v}=d$.

Using the equation $d=d^{\v}$, we show that $d$ is the composition 
of a univariate polynomial and a linear polynomial:
$$d(z_{1},z_{2})=p(v_{2}z_{1}-v_{1}z_{2})\quad\text{where }p\in 
K[z].\leqno(7)$$
We may assume without loss of generality that $v_{2}\ne0$. Define 
$$q(z_{1},z_{2})=d(z_{1}+v_{1}z_{2},v_{2}z_{2}).$$ 
Thus, $q=q^{
\e_{2}}$. Iterating the last equation, $q=q^{n\e_{2}}$ for all 
positive integers $n$. Thus,
$q(z_{1},z_{2})=q(z_{1},z_{2}+n)$ identically.
The left side of the last equation 
is free of~$n$ and the right side is symmetric in $n$ and $z_{2}$; thus,  
the left side must also be free of $z_{2}$. Hence $q$ is free of its 
second argument. It's easily verified that 
$d(z_{1},z_{2})=q(z_{1}-(v_{1}/v_{2})z_{2},z_{2}/v_{2})$. Letting 
$p(z)=q(z/v_{2},0)$, $(7)$ is proved and, hence, $d$ is simple.

This completes the proof of the two-variable case.

\hyperdef\chapteRiiSECv{ChapterSection}{chapteRiiSECv}{}
\Section The rational Galois space of a rational function

We require the following definitions and lemmas for the proof of the 
$k>2$ case of Lemma~\hyperref\lemtwo{\ref[lem:two]}.

\Definition[def:A3]  \hyperdef\defAthree{Definition}{A3}{}
Let $S=\{x_1,\ldots,x_k\}$ be a finite set
that is algebraically independent
over $K$, and 
let $R\in K(S)$ be a rational function in
$\vec x=(x_1,\ldots,x_k)$. 
We define the \defword{rational Galois 
space} of $R$ over $K$ with respect to $\vec x$ to be the set 
$$\rgal(R,\vec x,K)=
\{ \v\in \Q^k: R(\vec x + \v)=R(\vec x) \}.$$
\DefinitionStop
We write $\rgal(R)$ for $\rgal(R,\z,K)$;
thus, 
$\rgal(R)=\{ \v\in\Q^{k}:R^{\v}=R\}$.


Let $G$ be the Galois group of $K(\vec x)$ over $K$.
It's easily seen that
$\rgal(R,\vec x,K)$
is isomorphic to the subgroup 
$
\{ g\in G:g(R)=R \text {\ and $g(x_i)-x_i\in \Q$
for $i\in \{1,\ldots,k\}$}\}$ of $G$.


We show that the rational Galois space is a subspace of $\Q^{k}$. It's 
clearly closed under vector addition. By iterating $R^{\v}=R$, it 
follows that $R^{n\v}=R$ for any positive integer~$n$. Writing 
$R=A/B$ where $A$ and $B$ are polynomials, it follows that $AB^{n
\v}=BA^{n\v}$ for all positive integers~$n$. Each side of the last 
equation is a polynomial in~$n$. Thus, $AB^{n\v}=BA^{n\v}$ 
identically. Hence $R^{n\v}=R$ for all $n\in\Q$ and the rational Galois 
space is closed under multiplication by scalars.

\Lemma[lem:three]  \hyperdef\lemthree{Lemma}{three}{} Let $R\in K(z_{1},\ldots,z_{k})$ be a rational 
function such that $\e_{j}\in\rgal(R)$. Then $R$ is free of $z_{j}$.
\LemmaStop

\Proof Since $R=R^{n\e_{j}}$ for all $n\in\Q$, it follows that 
$$R(z_{1},\ldots,z_{k})=R(z_{1},\ldots,z_{j}+n,\ldots,z_{k})$$ 
identically. The left side is free of $n$ and the right side is 
symmetric in $z_{j}$ and $n$, so the left side must be free of $z_{j}$.
\ProofStop

\Lemma[lem:four]  \hyperdef\lemfour{Lemma}{four}{} Let $R\in K(z_1,\ldots,z_k)$ be a rational function
and let $M$ be an $r\times k$ matrix over $\Q$. If the kernel of $M$ is
equal to the \hyperref\defAthree{rational Galois space} of $R$, then there exists a rational
function $\bar R\in K(z_1,\ldots,z_r)$ such that $R(\z)=\bar
R(M\z)$.
\LemmaStop

\Proof Let $A$ and $B$ be invertible $k\times k$ 
and $r\times r$ matrices
respectively such that 
the $r\times k$ matrix $AMB$ is of the form
 $$AMB=\left[\matrix I_{j,j} & {Z}_{j,k-j}\\
{Z}_{r-j,j} & {Z}_{r-j,k-j} \endmatrix\right]$$
where $j$ is the rank of $M$, $I_{j,j}$ is the $j\times j$ identity
matrix, and ${Z}_{\ell,m}$ is the $\ell\times m$ zero matrix. Since $j$
is the rank of $M$, it follows that
 $j\le r$. Let $R_1(\z)=R(B\z)$. Then $\rgal R_1=B^{-1}\rgal
R=B^{-1}\ker M=\ker MB=\ker AMB=\spn(\e_{j+1},\ldots,\e_k)$. By
Lemma~\hyperref\lemthree{\ref[lem:three]}, $R_1(\z)$ is free of $z_{j+1},\ldots,z_k$.
It follows that $R_1(\z)=\bar R_1(AMB\z)$, where $\bar R_1\in
K(z_1,\ldots, z_r)$ is defined by
$$\bar R_1(z_1,\ldots,z_r)=R_1(z_1,\ldots,z_k)=R_1(z_1,\ldots,z_j,0,\ldots,0).$$
Letting $\bar R(\z)=\bar R_1(A\z)$, we have $R_1(\z)=\bar R(MB\z)$. Hence $R(\z)=R_1(B^{-1}\z)=\bar R(M\z)$ as claimed.
\ProofStop

\Corollary[cor:oldlemmafour]  \hyperdef\coroldlemmafour{Corollary}{oldlemmafour}{} A nontrivial rational function $R\in 
K(z_{1},\ldots,z_{k})$ is \hyperref\defAtwo{simple} if and only if the dimension of $\rgal 
R$ is $k-1$.
\CorollaryStop

\Proof If $R$ is simple, then there exists $\v\in\Q^{k}$ and 
a univariate rational function $r\in K(\z)$ such that 
$R(\z)=r(\z\cdot\v)$. Clearly $\rgal R$ contains the space orthogonal to $\v$, and thus
$\dim\rgal R\ge k-1$. If the dimension were $k$, then $R$ would be
trivial; therefore, the dimension must be $k-1$.

Conversely, if the dimension of $\rgal(R)$ is $k-1$, there exists a 
$1\times k$ matrix $\v=[v_1,\ldots,v_k]$ with kernel equal to $\rgal R$. By Lemma~\hyperref\lemfour{\ref[lem:four]}, $R(\z)=\bar R(\v\cdot\z)=\bar R(v_1z_1+\cdots+v_kz_k)$, so $R$ is simple.
\ProofStop

\Lemma[lem:five]  \hyperdef\lemfive{Lemma}{five}{} Let $d$ and $A\in K[z_{1},\ldots,z_{k}]$ be 
polynomials. If $d\mid A$, then $\rgal d\supseteq\rgal A$.
\LemmaStop

\Proof Suppose the contrary. Then the quotient space $\rgal A/(\rgal 
A\cap\rgal d)$ is infinite. Let $S$ be a set of representatives of the 
cosets.

For any $s\in S$, $A^{-s}=A$, hence $d\mid A^{-s}$, hence $d^{s}\mid A$. 
Since $S$ is infinite and $A$ has only finitely many monic divisors, 
$d^{s_{1}}=cd^{s_{2}}$ for some $s_{1}$ and $s_{2}\in S$ and $c\in K$. 
Letting $s=s_{1}-s_{2}$, $d^{s}=cd$ and, by a corollary of 
Lemma~\hyperref\lemzero{\ref[lem:zero]}, $c=1$. But $d^{s_{1}}=d^{s_{2}}$, contradicting 
the definition of~$s$.
\ProofStop

\Corollary If $A$ and $B\in K[z_{1},\ldots,z_{k}]$ are polynomials 
such that $\rgal A=\rgal B=s$, then $\rgal AB=s$.
\CorollaryStop

\Proof By Lemma~\hyperref\lemfive{\ref[lem:five]}, $\rgal AB\subseteq s$. But clearly 
$\rgal AB\supseteq s$, so $\rgal AB=s$.
\ProofStop

\Lemma[lem:fiveprime]  \hyperdef\lemfiveprime{Lemma}{fiveprime}{} If $A$ and $B\in K[z_{1},\ldots,z_{k}]$ are 
relatively prime polynomials, then
$$\rgal\frac AB\subseteq\rgal A\cap\rgal B.$$
\LemmaStop

\Proof Suppose the contrary. Then the quotient space
$$\rgal\frac AB\bigg/(\rgal\frac AB\cap\rgal A\cap\rgal B)$$
is infinite. Let $S$ be a set of representatives of the cosets, and 
let $s_{1}$ and $s_{2}\in S$.

Since $(A/B)^{s_{1}}=(A/B)^{s_{2}}$, letting $s=s_{1}-s_{2}$ it 
follows that $(A^{s}/B^{s})=(A/B)$. Hence $AB^{s}=BA^{s}$. Since $A$ 
and $B$ are relatively prime, $A\mid A^{s}$. Similarly, $A^{s}\mid 
A$. Hence $A=cA^{s}$, and hence, by corollary~\hyperref\corsecond{\ref[cor:second]} of 
Lemma~\hyperref\lemzero{\ref[lem:zero]}, $c=1$. Hence $A=A^{s}$. Similarly, $B=B^{s}$. 
Hence $s_{1}-s_{2}=s\in\rgal(A/B)\cap\rgal A\cap\rgal B$, contradicting 
the definition of $s_1$ and $s_2$.
\ProofStop

\Lemma[lem:six]  \hyperdef\lemsix{Lemma}{six}{} Every rational function $R\in K(z_{1},\ldots,z_{k})$ 
can be written as a product
$$R=g\prod_{s\in S}\frac{A_{s}}{B_{s}},$$
where $S$ is the collection of all proper subspaces of $\Q^{k}$, 
$g\in K$, and, for all $s\in S$, $A_{s}$ and $B_{s}\in 
K[z_{1},\ldots,z_{k}]$ are relatively prime \hyperref\defAone{monic} polynomials such 
that if $d$ is a nontrivial divisor of $A_{s}$ or $B_{s}$, then 
$\rgal(d)=s$. This expression is unique.
\LemmaStop

Of course, $A_{s}=B_{s}=1$ for all but a finite number of $s\in S$.

\Proof We first show that such a factorization exists. Let $R=gA/B$,
where $A$ and $B$ are relatively prime monic polynomials and $g\in
K$. Write $A$ and $B$ as products of nontrivial irreducible monic
factors
$$A=a_1a_2\cdots a_m\text{ and }B=b_1\cdots b_n,\ m,n\ge0.$$

For all $s\in S$ let $A_s$ be the product of all $a_i$ such that
$\rgal(a_i)=s$, with the understanding that empty products
are~$1$. Similarly define $B_s$. Clearly $A=\prod_{s\in S}A_s$ and
$B=\prod_{s\in S}B_s$. Hence 
$$R=g\prod_{s\in S}\frac{A_s}{B_s}$$
and $A_s$ and $B_s$ are relatively prime.

Let $d$ be a nontrivial monic divisor of $A_s$ or $B_s$. Then $d$ can
be expressed as a product of a subset of the $a_i$ for which $\rgal
a_i=s$ and the $b_i$ for which $\rgal b_i=s$. Hence, by the corollary
of Lemma~\hyperref\lemfive{\ref[lem:five]}, $\rgal d=s$.

We show that the factorization is unique. Let
$$R=g\prod_{s\in S}\frac{A_s}{B_s}\text{ and }R=h\prod_{s\in
S}\frac{C_s}{D_s}\leqno(1)$$
be two factorizations satisfying the conditions of the lemma. We will
show that $g=h$ and $A_s=C_s$ and $B_s=D_s$ for all $s\in
S$. By~$(1)$, 
$$g\prod A_s\prod D_s=h\prod B_s\prod C_s.\leqno(2)$$
For any $t\in S$, $A_t$ is relatively prime to $\prod_{s\ne
t}B_s\prod_{s\ne t}C_s$ since any nontrivial irreducible divisor
$d$ of the former has $\rgal(d)=s$ and any nontrivial divisor $d$
of the latter has $\rgal(d)\ne s$. Since $A_t$ and $B_t$ are
relatively prime by assumption, it follows that $A_t\mid
C_t$. Similarly $C_t\mid A_t$. Hence $C_t=A_t$. Similarly
$B_t=D_t$. It follows that $g=h$.
\ProofStop

\hyperdef\chapteRiiSECvi{ChapterSection}{chapteRiiSECvi}{}
\Section Fixed factors

The complementary idea to the rational Galois space is the fixed factor.

\Definition[def:A4]  \hyperdef\defAfour{Definition}{A4}{}
Using Lemma~\hyperref\lemsix{\ref[lem:six]}, we define the \defword{fixed
factor} $\fix_s(R)$ of a rational function $R\in K(z_1,\ldots,z_k)$
for a subspace $s$ of $\Q^k$.
Let $S$ be the set of all proper subspaces of $\Q^k$.
Let $R=g\prod_{s\in S}(A_s/B_s)$ be the unique expression guaranteed
by Lemma~\hyperref\lemsix{\ref[lem:six]}. 
Define
$$
\fix_s(R)=
\cases
\frac{A_s}{B_s}&\text{ for all }s\in S\\
g&\text{ for }s = \Q^k.
\endcases
$$
\DefinitionStop

\Lemma[lem:seven]  \hyperdef\lemseven{Lemma}{seven}{} Let $R\in K(z_1,\ldots,z_k)$ be a rational function
and let $s$ be a subspace of $\Q^k$. Then
$$\rgal\fix_sR=s$$
unless $\fix_sR=1$.
\LemmaStop

\Proof If $\fix_sR\ne1$ then either $\fix_sR$ has a nontrivial divisor
or $\fix_sR=c$ for some $c\in K$, $c\ne1$. In the first case,
$\rgal\fix_sR\subseteq\rgal d=s$ by Lemma~\hyperref\lemfive{\ref[lem:five]}. But clearly
$\rgal\fix_sR\supseteq s$, so $\rgal\fix_sR=s$. In the second case,
$s=\Q^k$ and $\rgal c=\Q^k$.
\ProofStop

\Lemma[lem:nine]  \hyperdef\lemnine{Lemma}{nine}{} For any nonzero rational functions $R_1$ and $R_2\in
K(z_1,\ldots,z_k)$, any subspace $s$ of $\Q^k$, and any $\v\in\Q^k$
\Enumerate
\item[nine:one] \hyperdef\nineone{item}{one}{} $\fix_sR_1R_2=\fix_sR_1\fix_sR_2$
\item[nine:two] \hyperdef\ninetwo{item}{two}{} $\ds{\fix_s\frac1{R_1}=\frac1{\fix_sR_1}}$
\item[nine:three] \hyperdef\ninethree{item}{three}{} $\fix_sR_1^{\v}=(\fix_sR_1)^{\v}$
\EnumerateStop
\LemmaStop

\Proof Let $S$ be the set of proper subspaces of $\Q^k$.
Let $R_3=R_1R_2$, $R_4=1/R_1$, and $R_5=R_1^{\v}$. For
$i=1,2,3,4$, and $5$ let 
$$R_i=g_i\prod_{s\in S}\frac{A_{i,s}}{B_{i,s}}$$
be the unique expression guaranteed by Lemma~\hyperref\lemsix{\ref[lem:six]}.

(\hyperref\nineone{\ref[nine:one]}): Since $R_3=R_1R_2$, it follows that
$$g_3\prod_{s\in S}\frac{A_{3,s}}{B_{3,s}}=g_1g_2\prod_{s\in
S}\frac{A_{1,s}A_{2,s}}{B_{1,s}B_{2,s}}.$$
Let $C_s$ and $D_s$ be relatively prime polynomials such that
$$\frac{C_s}{D_s}=\frac{A_{1,s}A_{2,s}}{B_{1,s}B_{2,s}}.$$
Since $C_s\mid A_{1,s}A_{2,s}$ and every nontrivial divisor of
$A_{1,s}$ or $A_{2,s}$ has $\rgal d=s$, it follows by the corollary of
Lemma~\hyperref\lemfive{\ref[lem:five]} that every nontrivial divisor $d$ of $C_s$ has
$\rgal d=s$. Similarly, every nontrivial divisor $d$ of $D_s$ has $\rgal
d=s$. It follows that 
$$R_3=g_1g_2\prod_{s\in S}\frac{C_s}{D_s}$$
is the unique expression guaranteed by Lemma~\hyperref\lemsix{\ref[lem:six]}, hence
$$\frac{A_{3,s}}{B_{3,s}}=\frac{C_s}{D_s}=\frac{A_{1,s}}{B_{1,s}}
\cdot\frac{A_{2,s}}{B_{2,s}}$$
for all $s\in S$. Hence $\fix_sR_3=\fix_sR_1\fix_sR_2$ for all $s\in
S$. Clearly $g_3=g_1g_2$ since the polynomials in~(\hyperref\nineone{\ref[nine:one]}) are
monic, so $\fix_sR_3=\fix_sR_1\fix_sR_2$ for $s=\Q^k$.

(\hyperref\ninetwo{\ref[nine:two]}): We have
$$g_4\prod_{s\in
S}\frac{A_{4,s}}{B_{4,s}}=R_4=\frac1{R_1}=\frac1{g_1}\prod_{s\in
S}\frac{B_{1,s}}{A_{1,s}}.$$
It follows immediately that the factorization
$$R_4=\frac1{g_1}\prod_{s\in S}\frac{B_{1,s}}{A_{1,s}}$$
satisfies the conditions of the lemma, hence $g_4=1/g_1$ and 
$$\frac{A_{4,s}}{B_{4,s}}=\frac{B_{1,s}}{A_{1,s}}.$$
Hence $\fix_sR_4=1/(\fix_sR_1)$ for all $s\in S$ and for $s=\Q^k$.

(\hyperref\ninethree{\ref[nine:three]}): We have
$$g_5\prod_{s\in S}\frac{A_{5,s}}{B_{5,s}}=R_5=R_1^{\v}=g_1^{
\v}\prod_{s\in S}\frac{A_{1,s}^{\v}}{B_{1,s}^{\v}}.$$
Since $\rgal d^{\v}=\rgal d$ for any polynomial $d$ and any $
\v\in\Q^k$, 
$$R_5=g_1^{\v}\prod_{s\in S}\frac{A_{1,s}^{\v}}{B_{1,s}^{
\v}}$$ 
satisfies the conditions of the lemma. Hence
$$g_5=g_1^{\v}\text{ and
}\frac{A_{5,s}}{B_{5,s}}=\frac{A_{1,s}^{\v}}{B_{1,s}^{\v}}$$
for all $s\in S$. Hence $\fix_sR_5=(\fix_sR_1)^{\v}$ for all $s\in
S$ and for $s=\Q^k$.
\ProofStop

\Lemma[lem:ten]  \hyperdef\lemten{Lemma}{ten}{} Let $R_1,\ldots,R_k\in K(z_1,\ldots,z_k)$ be rational
functions satisfying the relation
$$R_iR_j^{\e_i}=R_jR_i^{\e_j}$$
for all $i,j\in\{1,\ldots,k\}$. For $i=1,\ldots,k$ and all subspaces
$s$ of $\Q^k$ define $R_{i,s}=\fix_s(R_i)$. Then for all subspaces $s$
of $\Q^k$ and all $i,j\in\{1,\ldots,k\}$,
$$R_{i,s}R_{j,s}^{\e_i}=R_{j,s}R_{i,s}^{\e_j}.$$
\LemmaStop

\Proof By Lemma~\hyperref\lemnine{\ref[lem:nine]},
$$\align
\fix_sR_i(\fix_sR_j)^{\e_i}
&=\fix_sR_iR_j^{\e_i}\\
&=\fix_sR_jR_i^{\e_j}\\
&=\fix_sR_j(\fix_sR_i)^{\e_j}.
\endalign$$
\ProofStop

\Lemma[lem:eleven]  \hyperdef\lemeleven{Lemma}{eleven}{} Let $R_1,\ldots,R_k\in K(z_1,\ldots,z_k)$ be
rational functions satisfying the relation
$$R_iR_j^{\e_i}=R_jR_i^{\e_j}$$
for all $i,j\in\{1,\ldots,k\}$. Let $s$ be a proper subspace of $\Q^k$
such that $\e_i\in s$. Then $\fix_sR_i=1$.
\LemmaStop

\Proof By Lemma~\hyperref\lemten{\ref[lem:ten]}, for $j\in\{1,\ldots,k\}$,
$$\align
\frac{\fix_sR_i}{(\fix_sR_i)^{\e_j}}
&=\frac{\fix_sR_j}{(\fix_sR_j)^{\e_i}}\\
&=\frac{\fix_sR_j}{\fix_sR_j}\\
&=1.
\endalign$$
Hence $\e_j\in\rgal\fix_sR_i$. By Lemma~\hyperref\lemseven{\ref[lem:seven]}, either
$\fix_sR_i=1$ or $\rgal\fix_sR_i=s$. In the first case we are done. In
the second case $\e_j\in s$ for all $j\in\{1,\ldots,k\}$. Hence
$s=\Q^k$, contradicting the assumption that $s$ is a proper subspace.
\ProofStop
\secno=6

\hyperdef\chapteRiiSECvii{ChapterSection}{chapteRiiSECvii}{}
\Section Proof of Lemma~\ref[lem:two] for $k>2$

Now we are in a position to prove the case $k>2$ of Lemma~\hyperref\lemtwo{\ref[lem:two]}. 
We proceed by induction. Assume the case of $k-1$ variables is proved. 
Let
$$R_{1}=\frac{A_{1}}{B_{1}}\frac{C_{1}}{C_{1}^{
\e_{1}}}\frac{D_{1}^{\e_{1}}}{D_{1}}$$
as in Lemma~\hyperref\lemone{\ref[lem:one]}. Let 
$K_{i,j}=K(\{z_{1},\ldots,z_{k}\}\setminus\{z_{i},z_{j}\})$ and let 
$\Q_{i,j}$ be the subspace of $\Q^{k}$ generated by $\e_{i}$ and 
$\e_{j}$. Let $r_{1}=A_{1}/B_{1}$. 

Let $i\in\{2,\ldots,k\}$.
Applying the strong version
of the two-variable case of Lemma~\hyperref\lemtwo{\ref[lem:two]} to $R_{1}$ and 
$R_{i}$ over the field $K_{1,i}$, it follows that we can write
$$R_{i}=\frac{A_{i}}{B_{i}}\frac{C_{1}}{C_{1}^{
\e_{i}}}\frac{D_{1}^{\e_{i}}}{D_{1}},$$
where the irreducible divisors of $A_{1}$, $A_{i}$, $B_{1}$, and 
$B_{i}$ are all simple over $K_{1,i}$. 
The point of using the strong version of
the two-variable case is that the same
$C_1$ and $D_1$ can be used for all 
$i\in\{2,\ldots,k\}$.

Let $d$ be a nontrivial 
irreducible divisor of $A_{1}$ or $B_{1}$. The subspace 
$\rgal(d)\cap\Q_{1,j}$ of $\Q_{1,j}$ is isomorphic to the subspace 
$\rgal(d,(z_1,z_j),K_{1,j})$ of $\Q^{2}$.
Hence $\rgal(d)\cap\Q_{1,j}$ is a one-dimensional subspace 
of $\Q_{1,j}$ by Lemma~\hyperref\lemfour{\ref[lem:four]}. Let $s=\rgal(d)$. Let $r_i=A_i/B_i$ for $i\in\{2,\ldots,k\}$. It's easily 
verified that $r_{i}$ satisfies
$$r_{i}r_{j}^{\e_{i}}=r_{j}r_{i}^{\e_{j}}$$
for $i,j\in\{1,\ldots,k\}$, hence, by Lemma~\hyperref\lemeleven{\ref[lem:eleven]}, if $
\e_{1}\in s$ then $\fix_{s}R_{1}=1$. But $\fix_{s}R_{1}=1$ implies 
$\fix_{s}A_{1}=\fix_{s}B_{1}=1$, which implies $d=1$. Since $d$ is 
nontrivial, $\e_{1}\notin s$.

Let $\v_{i}$ be a nonzero vector in $\rgal d\cap\Q_{1,i}$. Since 
$\e_{1}\notin s$, it follows that $\e_{i}\cdot\v_{i}\ne0$. 
Let $ \vec t=\sum_{i=2}^{k}c_{i}\v_{i}$.
It is easily seen that the vectors 
$\v_{2},\ldots,\v_{k}$ are linearly independent:
if $c_{j}\ne0$, then $
\e_{j}\cdot\vec t=c_{j}(\e_{j}\cdot\v_{j})$ is nonzero, hence 
$\vec t$ is nonzero.

It follows that the dimension of $\rgal(d)$ is at least $k-1$.  
By Lemma~\hyperref\lemfour{\ref[lem:four]}, $d$ is simple over $K$. We have shown all 
the nontrivial irreducible divisors of $A_{1}$ and $B_{1}$ are simple 
over $K$.

Define $\bar R_{i}$ and $\tilde R_{i}$ by
$$\bar R_{i}=\prod\fix_{s}r_{i}$$
and
$$\tilde R_{i}=\mathop{{\prod}'}\fix_{s}r_{i}$$
where the first product is taken over all subspaces $s$ of $\Q^{k}$ 
such that $\e_{1}\notin s$, and the second is taken over all 
subspaces such that $\e_{1}\in s$. Hence $r_{i}=\bar R_{i}\tilde 
R_{i}$. It follows from Lemma~\hyperref\lemten{\ref[lem:ten]} that
$$\bar R_{i}\bar R_{j}^{\e_{i}}=\bar R_{j}\bar R_{i}^{
\e_{j}}$$
and
$$\tilde R_{i}\tilde R_{j}^{\e_{i}}=\tilde R_{j}\tilde R_{i}^{
\e_{j}}$$
for all $i,j\in\{1,\ldots,k\}$.

Let $\bar A_{i}$ and $\bar B_{i}$ be relatively prime polynomials such 
that $\bar R_{i}=\bar A_{i}/\bar B_{i}$. We claim that the irreducible 
divisors of $\bar A_{i}$ and $\bar B_{i}$ are simple. By 
Lemma~\hyperref\lemeleven{\ref[lem:eleven]}, $\tilde R_{1}=1$. Hence $\bar R_{1}=r_{1}$, 
and hence the irreducible divisors of $\bar A_{1}=A_{1}$ and $\bar 
B_{1}=B_{1}$ are simple. Let $d$ be an irreducible divisor of $\bar 
A_{i}$ or $\bar B_{i}$ for $i\in\{2,\ldots,k\}$, and let $s=\rgal(d)$. 
If the dimension of $s$ is less than $k-1$, then $\fix_{s}\bar R_{1}=1$ 
since all the irreducible divisors of $\bar A_{1}$ and $\bar B_{1}$ 
are simple and, by Lemma~\hyperref\lemfour{\ref[lem:four]}, have rational Galois spaces of 
dimension $k$ or $k-1$. By Lemma~\hyperref\lemten{\ref[lem:ten]},
$$\frac{\fix_{s}\bar R_{i}}{\fix_{s}\bar R_{i}^{
\e_{1}}}=\frac{\fix_{s}\bar R_{1}}{\fix_{s}\bar R_{1}^{\e_{i}}}=1,$$
so $\e_{1}\in\rgal(\fix_{s}\bar R_{i})$. Hence $\e_{1}\in s$ 
by Lemma~\hyperref\lemseven{\ref[lem:seven]}, which contradicts the definition of $\bar 
R_{i}$, so the dimension of $\rgal d$ is $k-1$ or $k$ and $d$ is 
simple by Lemma~\hyperref\lemfour{\ref[lem:four]}.

By Lemma~\hyperref\lemthree{\ref[lem:three]}, $\tilde R_{i}$ is free of $z_{i}$ for all 
$i\in\{1,\ldots,k\}$. Inductively applying the $k-1$ variable case of 
Lemma~\hyperref\lemtwo{\ref[lem:two]} to $\tilde R_{2},\ldots,\tilde R_{k}\in 
K(z_{2},\ldots,z_{k})$, there exist $\tilde C$ and $\tilde D$ such 
that
$$\tilde R_{i}=\frac{\tilde A_{i}}{\tilde B_{i}}\frac{\tilde C}{\tilde 
C^{\e_{i}}}\frac{\tilde D^{\e_{i}}}{\tilde D}\leqno(1)$$
for $i\in\{2,\ldots,k\}$, and the irreducible divisors of $\tilde 
A_{i}$ and $\tilde B_{i}$ are simple over $K$. Let $\tilde 
A_{1}=\tilde B_{1}=1$. Then $(1)$ is true for $i=1$.

Letting $C=C_{1}\tilde C$, $D=D_{1}\tilde D$, $a_{i}=\bar A_{i}\tilde 
A_{i}$, and $b_{i}=\bar B_{i}\tilde B_{i}$,
$$R_{i}=\frac{a_{i}}{b_{i}}\frac{C}{C^{\e_{i}}}\frac{D^{
\e_{i}}}D$$
where the irreducible divisors of $a_{i}$ and $b_{i}$ are all simple.
This completes the proof of Lemma~\hyperref\lemtwo{\ref[lem:two]}.
\ProofStop

\secno=7
\hyperdef\chapteRiiSECviii{ChapterSection}{chapteRiiSECviii}{}
\Section The multiplicative structure of $R_i$


\Lemma[lem:thirteen]  \hyperdef\lemthirteen{Lemma}{thirteen}{} Let $R_1,\ldots,R_k\in K(z_1,\ldots,z_k)$ be \hyperref\defAtwo{simple} rational functions with the same \hyperref\defAthree{rational Galois space} such that
$$R_iR_j^{\e_i}=R_jR_i^{\e_j}\leqno(1)$$
for all $i,j\in\{1,\ldots,k\}$. Then there exist $\v\in\Z^k$, a univariate rational function $t\in K(z)$, and $c_1,\ldots,c_k\in K$ such that
$$R_i(z_1,\ldots,z_k)=c_i\gp j0{v_i}t(\v\cdot\z+j)$$
for $i\in\{1,\ldots,k\}$.
\LemmaStop


\Proof Let $s=\rgal R_i$. The case $s=\Q^k$ is trivial: if $s=\Q^k$, then $R_i\in K$ for all $i\in\{1,\ldots,k\}$, so we may take $c_i=R_i$, $t=1$, and $\v=\vec 0$.

We consider the case $s\ne\Q^k$. By the corollary of Lemma~\hyperref\lemfour{\ref[lem:four]} the dimension of $s$ is $k-1$. Hence, there exists a vector $\v\in\Z^k$ with $\gcd(\v)=1$ such that the kernel of the $1\times k$ matrix $\v$ is $s$. By Lemma~\hyperref\lemfour{\ref[lem:four]}, for all $i\in\{1,\ldots,k\}$ there exist univariate $r_i\in K(z)$ such that $R_i(\z)=r_i(\v\cdot\z)$.

Since $\gcd(\v)=1$, there exists $\w\in\Z^k$ such that $\w\cdot\v=w_1v_1+\cdots w_kv_k=1$. 
For any univariate rational function $r \in K(z)$
and any
$n\in \Z$,
the rational 
function $r^{(n)} \in K(z)$ is defined by
$r^{(n)}(z)=r(z+n)$.
Define 
$$t=\gp i0{w_1}r_1^{(v_1i)}
\gp i0{w_2}r_2^{(v_1w_1+v_2i)}\cdots
\gp i0{w_k}r_k^{(v_1w_1+\cdots v_{k-1}w_{k-1}+v_ki)}.$$
We claim that $R_i(z_1,\ldots,z_k)=\gp j0{v_i}t^{(j)}(\v\cdot\z)$.
We first show that $r_i/r_i^{(1)}=t/t^{(v_i)}$. Since
$$\align
R_i^{\e_j}(z_1,\ldots,z_k)
&=(r(\v\cdot\z))^{\e_j}\\
&=r(\v\cdot\z+\v\cdot\e_j)\\
&=r(\v\cdot\z+v_j)\\
&=r^{(v_j)}(\v\cdot\z),
\endalign$$
it follows from $(1)$ that
$$\frac{r_i}{r_i^{(v_j)}}=\frac{r_j}{r_j^{(v_i)}}$$
for all $i,j\in\{1,\ldots,k\}$. Hence
$$\align
\frac t{t^{(v_j)}}
&=\gp i0{w_1}\left(\frac{r_1}{r_1^{(v_j)}}\right)^{(v_1i)}
\gp i0{w_2}\left(\frac{r_2}{r_2^{(v_j)}}\right)^{(v_1w_1+v_2i)}\cdots\\
&\qquad\qquad\qquad\qquad\qquad\qquad\qquad\cdots
\gp i0{w_k}\left(\frac{r_k}{r_k^{(v_j)}}\right)^{(v_1w_1+
\cdots+v_{k-1}w_{k-1}+v_ki)}\\
&=\gp i0{w_1}\left(\frac{r_j}{r_j^{(v_1)}}\right)^{(v_1i)}
\gp i0{w_2}\left(\frac{r_j}{r_j^{(v_2)}}\right)^{(v_1w_1+v_2i)}\cdots\\
&\qquad\qquad\qquad\qquad\qquad\qquad\qquad\cdots
\gp i0{w_k}\left(\frac{r_j}{r_j^{(v_k)}}\right)^{(v_1w_1+
\cdots+v_{k-1}w_{k-1}+v_ki)}\\
&=\frac{r_j^{(0)}}{r_j^{(v_1w_1)}}\cdot
\frac{r_j^{(v_1w_1)}}{r_j^{(v_1w_1+v_2w_2)}}\cdots
\frac{r_j^{(v_1w_1+\cdots+v_{k-1}w_{k-1})}}{r_j^{(v_1w_1+\cdots+v_kw_k)}}\\
&=\frac{r_j}{r_j^{(v_1w_1+\cdots+v_kw_k)}}\\
&=\frac{r_j}{r_j^{(1)}}
\endalign$$
since $v_1w_1+\cdots+v_kw_k=1$.

We show that $r_j=c_j\gp i0{v_j}t^{(i)}$. Since $r_j/r_j^{(1)}=t/t^{(v_j)}$, letting $T_j=\gp i0{v_j}t^{(i)}$ it follows that $r_j/r_j^{(1)}=T_j/T_j^{(1)}$; hence $(r_j/T_j)=(r_j/T_j)^{(1)}$. It follows by Lemma~\hyperref\lemthree{\ref[lem:three]} that $r_j/T_j=c_j$ for some $c_j\in K$. Thus, $R_j(z_1,\ldots,z_k)=r_j(\v\cdot\z)=c_i\gp j0{v_j}t^{(j)}(\v\cdot\z)=c_i\gp j0{v_j}t(\v\cdot\z+j)$.
\ProofStop

Lemma~\hyperref\lemfourteen{\ref[lem:fourteen]} summarizes the results of this section thus far.

\Lemma[lem:fourteen]  \hyperdef\lemfourteen{Lemma}{fourteen}{} Let $R_1,\ldots,R_k\in K(z_1,\ldots,z_k)$ be rational functions satisfying the relation
$$R_iR_j^{\e_i}=R_jR_i^{\e_j}$$
for all $i,j\in\{1,\ldots,k\}$. For each subspace $s$ of $\Q^k$ let $R_{i,s}=\fix_sR_i$.
\Enumerate
\item[fourteen:one] \hyperdef\fourteenone{item}{one}{} If the dimension of $s$ is less than $k-1$, then there exists a rational function $r\in K(z_1,\ldots,z_k)$ such that for all $i\in\{1,\ldots,k\}$, 
$$R_{i,s}=r/r^{\e_i}.$$
\item[fourteen:two] \hyperdef\fourteentwo{item}{two}{} If the dimension of $s$ is $k-1$ then there exists a univariate rational function $t\in K(z)$, $c_1,\ldots,c_k\in K$,
 and a vector $\v\in\Z^k$ such that for all $i\in\{1,\ldots,k\}$
$$R_{i,s}=c_i\gp j0{\v_i}t(\v\cdot\z+j).$$
\item[fourteen:three] \hyperdef\fourteenthree{item}{three}{} If the dimension of $s$ is $k$ then $R_{i,s}\in K$ for all $i\in\{1,\ldots,k\}$.
\item[fourteen:four] \hyperdef\fourteenfour{item}{four}{} $R_i=\prod_sR_{i,s}$ for all $i\in\{1,\ldots,k\}$, where the product is taken over all subspaces $s$ of $\Q^k$.
\EnumerateStop
\LemmaStop

\Proof Write
$$R_i=\frac C{C^{\e_i}}\frac{D^{\e_i}}D\frac{A_i}{B_i}$$
as in Lemma~\hyperref\lemtwo{\ref[lem:two]} and let $C_s=\fix_sC$, $D_s=\fix_sD$, $A_{i,s}=\fix_sA_i$, and $B_{i,s}=\fix_sB_i$. By Lemma~\hyperref\lemnine{\ref[lem:nine]}
$$R_{i,s}=\frac{C_s}{C_s^{\e_i}}\frac{D_s^{\e_i}}{D_s}\frac{A_{i,s}}{B_{i,s}}.$$

(1) If the dimension of $s$ is less than $k-1$, then $A_{i,s}=B_{i,s}=1$ since the irreducible divisors of $A_i$ and $B_i$ are simple. Hence $R_{i,s}=r/r^{\e_i}$ where $r=C_s/D_s$.

(2) If the dimension of $s$ is $k-1$ then $R_{1,s},\ldots,R_{k,s}$ are simple by the corollary of Lemma~\hyperref\lemfour{\ref[lem:four]}. By Lemma~\hyperref\lemten{\ref[lem:ten]}
$$R_{i,s}R_{j,s}^{\e_i}=R_{j,s}R_{i,s}^{\e_j},$$
hence, by Lemma~\hyperref\lemthirteen{\ref[lem:thirteen]}, there exist $c_1,\ldots,c_k\in K$, $\v\in\Z^k$, and $t\in K(z)$ such that
$$R_{i,s}=c_i\gp j0{v_i}t(\v\cdot\z+j)$$
for $i\in\{1,\ldots,k\}$. 

(3) By Lemma~\hyperref\lemthree{\ref[lem:three]}, $r_{i,s}$ is free of $z_1,\ldots,z_k$, and thus constant.

(4) This is immediate from Lemma~\hyperref\lemsix{\ref[lem:six]}.
\ProofStop

\Lemma[lem:fifteen]  \hyperdef\lemfifteen{Lemma}{fifteen}{} Let $R_1,\ldots,R_k\in K(z_1,\ldots,z_k)$ be rational functions satisfying the relation
$$R_iR_j^{\e_i}=R_jR_i^{\e_j}$$
for all $i$ and $j\in\{1,\ldots,k\}$. There exist a rational function $r\in K(z_1,\ldots,z_k)$, constants $c_1,\ldots,c_k\in K$, finitely many vectors $\v_1,\ldots,\v_m\in\Z^k$, and corresponding univariate rational functions $t_1,\ldots,t_m\in K(z)$ such that for all $i\in\{1,\ldots,k\}$,
$$R_i(\z)=c_i\frac{r(\z)}{r^{\e_i}(\z)}\prod_{\ell=1}^m\gp j0{v_{\ell,i}}t_\ell(\v_{\ell}\cdot\z+j).$$
\LemmaStop

\Proof For each subspace of $s$ of $\Q^k$ and $i\in\{1,\ldots,k\}$ let $R_{i,s}=\fix_sR_i$. Let $s_1,\ldots,s_m$ be the subspaces $s$ of $\Q^k$ of dimension $k-1$ such that $R_{i,s}\ne1$ for some $i\in\{1,\ldots,k\}$. Let $t_1,\ldots,t_n$ be the subspaces of $\Q^k$ of dimension less than $k-1$ such that $R_{i,s}\ne1$ for some $i\in\{1,\ldots,k\}$.

By Lemma~\hyperref\lemfourteen{\ref[lem:fourteen]} (2), for each $\ell$, $1\le\ell\le m$, $R_{i,s_\ell}=\gp j0{v_{\ell,i}}t_\ell(\v_\ell\cdot\z+j)$ for some $\v_\ell\in\Z^k$ and $t_\ell\in K(z)$. By Lemma~\hyperref\lemfourteen{\ref[lem:fourteen]} (1), for each $p$, $1\le p\le n$, there exists $r_p\in K(z_1,\ldots,z_k)$ such that for $i\in\{1,\ldots,k\}$, $R_{i,t_p}=r_p/r_p^{\e_i}$. Let $r=\prod_{p=1}^nr_p$. 
Let $c_i=R_{i,\Q^k}$ for $i\in\{1,\ldots,k\}$. By Lemma~\hyperref\lemfourteen{\ref[lem:fourteen]} (4),
$$\align
R_{i,s}
&=c_iR_{i,t_1}\cdots R_{i,t_n}R_{i,s_1}\cdots R_{i,s_m}\\
&=c_i\frac{r_{t_1}}{r_{t_1}^{\e_i}}\cdots\frac{r_{t_n}}{r_{t_n}^{\e_i}}\prod_{\ell=1}^mR_{i,s\ell}\\
&=c_i\frac r{r^{\e_i}}\prod_{\ell=1}^m\gp j0{v_{\ell,i}}t_\ell(\v_\ell\cdot\z+j).
\endalign$$
\ProofStop



\Theorem[thm:15a]  \hyperdef\thmfifteena{Theorem}{15a}{} \hyperref\oresato{\OreSato}
Let $R_1,\ldots,R_k\in K(\z)$ be rational functions such that
$$R_iR_j^{\e_i}=R_jR_i^{\e_j}$$
for all $i,j\in\{1,\ldots,k\}$. Then there exist polynomials $C$ and $D\in K[\z]$, a finite set $V\subset\Z^k$, and univariate polynomials $a_{\v}$ and $b_{\v}\in K[z]$ for each $\v\in V$ such that for all $i\in\{1,\ldots,k\}$,
$$R_i(\z)=\frac{C(\z+\e_i)}{C(\z)}\frac{D(\z)}{D(\z+\e_i)}\prod_{\v\in V}\gp j0{\v_i}\frac{a_{\v}(\z\cdot\v+j)}{b_{\v}(\z\cdot\v+j)}.$$
\TheoremStop

\Proof The $c_i$ from Lemma~\hyperref\lemfifteen{\ref[lem:fifteen]} can be absorbed into the product by noting that
$$\gp j0{\e_i\cdot\w}c_i=\cases
c_i&\text{if }\w=\e_i,\\
1&\text{if }\w\in\{\e_1,\ldots,\e_k\}\setminus\{\e_i\}.
\endcases$$
The rest is just a change of notation.
\ProofStop

\Corollary[cor:16]  \hyperdef\corsixteen{Corollary}{16}{} For all $\w\in\Z^k$, let $R_{\w}\in K(\z)$ be rational functions such that
$$ R_{\v +\w}=R_{\v}R_{\w}^{\v}$$
for all $\w$ and $\v\in\Z^k$. Then there exist polynomials $C$ and $D\in K[\z]$, a finite set $V\subseteq\Z^k$, and univariate polynomials $a_{\v}$ and $b_{\v}\in K[z]$ for each $\v\in V$ such that for all $\w\in\Z^k$,
$$R_{\w}(\z)=\frac{C(\z+\w)}{C(\z)}\frac{D(\z)}{D(\z+\w)}\prod_{\v\in V}\gp j0{\v\cdot\w}\frac{a_{\v}(\z\cdot\v+j)}{b_{\v}(\z\cdot\v+j)}.$$
\CorollaryStop


\Proof
By symmetry,
$R_{\v +\w}=R_{\w}R_{\v}^{\w}$,
hence,
$R_{\w}R_{\v}^{\w}=R_{\v}R_{\w}^{\v}$.
Letting $R_i=R_{\e_i}$ for $i\in\{1,\ldots,k\}$, we have $R_iR_j^{\e_i}=R_jR_i^{\e_j}$. It follows by Theorem~\hyperref\thmfifteena{\ref[thm:15a]} that there exist polynomials $C,D\in K[\z]$, a finite set $V\in\Z$, and univariate polynomials $a_{\v}$ and $b_{\v}$ for $v\in V$ such that
$$R_{\w}(\z)=\frac{C(\z+\w)}{C(\z)}\frac{D(\z)}{D(\z+\w)}\prod_{\v\in V}\gp j0{\v\cdot\w}\frac{a_{\v}(\z\cdot\v+j)}{b_{\v}(\z\cdot\v+j)}$$
for $\w\in\{\e_1,\ldots,\e_k\}$. Let $\bar R_{\w}(\z)$ be the right side of the last equation. Thus $R_{\w}=\bar R_{\w}$ for $\w\in\{\e_1,\ldots,\e_k\}$.

We show that $R_{\w}=\bar R_{\w}$ for all $\w\in\Z^k$. It's clear that $R_{\vec 0}=1$ and $\bar R_{\vec 0}=1$ so $R_{\vec 0}=\bar R_{\vec 0}$. We show that $\bar R_{\u+\w}=\bar R_{\u}\bar R_{\w}^{\u}$.

$$\align
\bar R_{\u}\bar R_{\w}^{\u}
&=\vec c^{\u}\vec c^{\w}\frac{C^{\u}}C\frac D{D^{\u}}\left(\frac{C^{\w}}C\frac D{D^{\w}}\right)^{\u}\prod_{\v\in V}\gp j0{\v\cdot\u}\frac{a_{\v}(\v\cdot\z+j)}{b_{\v}(\v\cdot\z+j)}\left(\gp j0{\v\cdot\w}\frac{a_{\v}(\v\cdot\z+j)}{b_{\v}(\v\cdot\z+j)}\right)^{\u}\\
&=\vec c^{\u+\w}\frac{C^{\u}}C\frac D{D^{\u}}\frac{C^{\w+\u}}{C^{\u}}\frac{D^{\u}}{D^{\w+\u}}\prod_{\v\in V}\gp j0{\v\cdot\u}\frac{a_{\v}(\v\cdot\z+j)}{b_{\v}(\v\cdot\z+j)}\gp j0{\v\cdot\w}\frac{a_{\v}(\v\cdot(\z+\u)+j)}{b_{\v}(\v\cdot(\z+\u)+j)}\\
&=\vec c^{\u+\w}\frac{C^{\u+\w}}C\frac D{D^{\u+\w}}\prod_{\v\in V}\gp j0{\v\cdot\u}\frac{a_{\v}(\v\cdot\z+j)}{b_{\v}(\v\cdot\z+j)}\gp j{\v\cdot\u}{\v\cdot\w+\v\cdot\u}\frac{a_{\v}(\z\cdot\v+j)}{b_{\v}(\z\cdot\v+j)}\\
&=\vec c^{\u+\w}\frac{C^{\u+\w}}C\frac D{D^{\u+\w}}\prod_{\v\in V}\gp j0{\v\cdot(\u+\w)}\frac{a_{\v}(\z\cdot\v+j)}{b_{\v}(\z\cdot\v+j)}\\
&=\bar R_{\u+\w}.
\endalign$$
Substituting $\u-\w$ for $\u$ in $R_{\u+\w}=R_{\u}R_{\w}^{\u}$, it follows that $R_{\u-\w}=R_{\u}/R_{\w}^{\u-\w}$. Similarly $\bar R_{\u-\w}=\bar R_{\u}/\bar R_{\w}^{\u-\w}$. 
It follows that if $R_{\u}=\bar R_{\u}$ and $R_{\w}=\bar R_{\w}$ then $R_{\u+\w}=\bar R_{\u+\w}$ and $R_{\u-\w}=\bar R_{\u-\w}$. 
Thus, it follows by induction that $R_{\u}=\bar R_{\u}$ for all $\u\in\Z^k$.
\ProofStop

