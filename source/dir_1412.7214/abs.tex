
In \hyperref\WZninetytwo{\bibref[WZ92]}, Wilf and Zeilberger developed a proof theory for
hypergeometric multisums centered around the notion of a multivariate
\hgt. A multivariate function $f(z_1,\ldots,z_k)=f(\z)$ from $\Z^k$
to a field $K$ is a {\it\hgt} if for each $i\in\{1,\ldots,k\}$ there
exist nonzero polynomials $A_i(\z)$ and $B_i(\z)$ in $K[\z]$ such
that
$$A_i(\z)f(\z)=B_i(\z)f(\z+\e_i)$$
for all $\z\in\Z^k$. Here $\e_1,\ldots,\e_k$ is the
standard basis for $\Z^k$. If $f$ is not a {\it\hyperref\defBtwo{\zd}}, then the {\it \hyperref\defBfour{term
ratios}} $R_1=A_1/B_1,\ldots,R_k=A_k/B_k$ are unique and satisfy the
relation
$$R_i(\z)R_j(\z+\e_i)=R_j(\z)R_i(\z+\e_j)\quad
\text{for each }i,j\in
\{1,\ldots,k\}$$
(cf.\ \hyperref\WZninety{\bibref[WZ90]}).

We introduce the concepts of divisibility lattice paths, \hyperref\defAthree{rational Galois  
spaces}, and \hyperref\defAfour{fixed factors} of rational functions to the study of the
relation for the \hyperref\defBfour{term ratios}. We prove that a solution $R_1,\ldots,R_k$
must be of the form \hyperref\oresato{\OreSato}
$$
R_i(\z)=\frac{C(\z+\e_i)}{C(\z)}\frac{D(
\z)}{D(\z+\e_i)}\prod_{\v\in V}\gp j0{\e_i \cdot \v}\frac{a_{
\v}(\v\cdot\z+j)}{b_{\v}(\v\cdot\z+j)}
\text{\ \ for $i\in\{1,\ldots,k\}$,}
$$
where $C$ and $D$ are polynomials, $V$ is a
finite subset of $\Z^k$, and, for each $\v\in V$, $a_{\v}$ and
$b_{\v}$ are {\sl univariate} polynomials all independent of
$i$.
The symbol $\gp jab{}$ denotes 
$\prod_{i=a}^{b-1}$ if $b>a$,
$1$ if $a=b$,
and
$1/\prod_{i=b}^{a-1}$ if $a>b$.

We use this factorization of $R_i$ to answer an obvious question about
multivariate \hgt s. Recall the Pochhammer symbol $(m)_r=m(m+1)\cdots(m+r-1)$. The
multivariate \hgt s that arise in practice have the form $$f(
\z)=\gamma_1^{z_1}\cdots\gamma_k^{z_k}\frac{C(\z)}{D(
\z)}\frac{\prod_{i=1}^p(m_i)_{\v_i\cdot
\z+r_i}}{\prod_{j=1}^q(n_j)_{\w_j\cdot\z+s_i}}$$ where
$\gamma_1,\ldots,\gamma_k\in K$, $C$ and $D$ are polynomials in $K[
\z]$, the $\v_i$ and $\w_j$ are in $\Z^k$, the $r_i$ and $s_j$
are in $\Z$, and the $m_i$ and $n_j$ are in $K$. The question is: do
all \hgt s have this form? We prove that if $K$ is algebraically
closed, and the \hgt\ is \hyperref\defBthirteena{honest}, then such an expression for $f$
exists \Mark\hyperref\Honestpw{{\it piecewise}}\StopMark. This is trivial in the case of one variable,
but {\sl not} in the case of several variables. We use this result to
settle the discrete part of a conjecture of Wilf and Zeilberger
\hyperref\WZninetytwo{\bibref[WZ92]} by showing that a holonomic \hgt\ is \hyperref\defpwp{piecewise proper},
which means roughly that we can take $D(\z)=1$ in the expression for 
$f$ above.

For any subsets $A$ and $B$ of an additive group $G$, define
$A+B=\{a+b:a \in A \text{ and } b \in B \}$ and $-A= \{-a:a
\in A \}$.  A subset $S$ of $G$ is said to be sum-free,
complete, and symmetric respectively if $S+S \subset S^\c$,
$S+S \supset S^\c$, and $S=-S$.  Cameron asked if for all
sufficiently large moduli $m$ there exists a sum-free complete
set in $\Z/m\Z$ that is not symmetric \hyperref\CamPortrait{\bibref[CamPortrait]}.
We answer Cameron's question by showing there
exists such a set for all moduli greater than or equal to
$890626$.  We also show that every sum-free complete set in
$\Z/m\Z$ that is not symmetric can be used to construct a
counterexample to a conjecture of Conway disproved by Marica
\hyperref\Marica{\bibref[Marica]}.  Conway conjectured that for any finite set
$S$ of integers, $|S+S| \le |S-S|$.


