\hyperdef\chapteRiv{ChapterSection}{chapteRiv}{}
\Chapter A Holonomic Hypergeometric Term\\
Is Piecewise Proper


The theory of holonomic systems was developed by 
Bernstien~\hyperref\Berone{\bibref[Ber1]} and applied to the
the theory of hypergeometric identities by
Zeilberger~\hyperref\Zeilninety{\bibref[Zeil90]}.
In~\hyperref\WZninetytwo{\bibref[WZ92]}
Wilf and Zeilberger
developed a proof theory
for multisums
that applies
only to {\sl proper} \hgt s.
Wilf and Zeilberger conjectured \bibref[WZ92%
] that a \hgt\ is holonomic  if and only if it is proper. 
We consider the discrete case of their conjecture and
interpret it to
mean roughly that a \hgt\  on $\Z^k$ is holonomic if and only if it is
{\sl piecewise proper}.
We use only elementary facts about holonomic functions that can be
found in~\hyperref\Zeilninety{\bibref[Zeil90]}
and~\hyperref\Bjoone{\bibref[Bjo1]}.

\Definition[def:pwp]  \hyperdef\defpwp{Definition}{pwp}{} 
A \hgt\  $f$ on $\Z^k$ over an
field $K$ is \defword{piecewise proper} if
there exist a polynomial $C\in K[\z]$ 
and a finite number of \hyperref\defBeighteen{polyhedral
regions} $\R_1,\ldots,\R_L$ such that $\Z^k$ is the union of the
$\R_{\ell}$ and a \hyperref\defBeleven{set of measure zero}, and for each region $\R_{\ell}$
there exist a vector $\vecgamma\in K^k$, constants
$m_1,\ldots,m_p,n_1,\ldots,n_q\in K$, vectors $\v_1,\ldots,
\v_p,\w_1,\ldots,\w_q\in\Z^k$, and integers
$r_1,\ldots,r_p,s_1,\ldots,s_q$ such that 
\Enumerate
\item for all $\z\in \R_{\ell}$,
$$f(\z)=
\gamma_1^{z_1}\cdots\gamma_k^{z_k}
C(\z)
\frac{\prod_{i=1}^p(m_i)_{\v_i\cdot
\z+r_i}}{\prod_{j=1}^q(n_j)_{\w_j\cdot\z+s_j}};$$ 
\item for all
$i$ and $j$ and $\z\in \R_{\ell}$, $\v_i\cdot\z+r_i$ and
$\w_j\cdot\z+s_j$ are positive; 
\item the Pochhammer symbols occurring
in the products are nonzero.
\EnumerateStop
\DefinitionStop

Our interpretation of Wilf and Zeilberger's conjecture is
Theorem~\hyperref\thmCtwelve{\ref[thm:C12]}.
\Theorem[thm:C12]  \hyperdef\thmCtwelve{Theorem}{C12}{}
A holonomic \hgt\  $f$ on $\Z^k$ over an
algebraically closed field $K$ is \hyperref\defpwp{piecewise proper}.
Conversely,
if
$f $ is \hyperref\defpwp{piecewise proper} then there exists a
holonomic function $g$ such that
$f=g$~\hyperref\defBeleven{a.e.}
\TheoremStop

The following example shows 
the necessity of introducing
sets of measure zero
and polyhedral regions
to the conjecture.
Let
$$f(z_1,z_2)=
\cases
(z_1-z_2+1)(z_1-z_2-1)&\text{if $z_1\ne z_2$}\\
g(z_1,z_2)&\text{if $z_1=z_2$,}
\endcases$$
where $g$ is an arbitrary holonomic function. Letting 
$$p(z_1,z_2)=(z_1-z_2+1)(z_1-z_2-1),$$
it's easily verified that
$$p^{\e_1}f=pf^{\e_1}\text{ and }p^{\e_2}f=pf^{\e_2}$$
regardless of the choice of $g$, so $f$ is a \hgt. Furthermore, using the fact that the characteristic function of a half-space is holonomic, it's easily seen that $f$ is holonomic:
$$f=\chi(z_1>z_2)p+\chi(z_1<z_2)p+\chi(z_1\ge z_2)\chi(z_1\le z_2)g,$$
where $\chi(\relation(z_1,z_2))$ is the characteristic function of $\{(z_1,z_2)\colon\relation(z_1,z_2)\}$. Each of the relations in the equation defines a half-plane, so each characteristic function is holonomic. Sums and products of holonomic functions are holonomic, so $f$ is holonomic. But $f$ is clearly not proper hypergeometric for arbitrary~$g$.

The function $f$ is, however, {\sl piecewise} proper. The set of measure zero is the line $z_1=z_2$, and the polyhedral regions are the half-spaces $z_1>z_2$ and $z_1<z_2$.


The `conversely'
part of  Theorem~\hyperref\thmCtwelve{\ref[thm:C12]} follows by the arguments of~\hyperref\Zeilninety{\bibref[Zeil90]}.
Thus to prove the conjecture, we need only prove
the first part, which  is the content of Theorem~\hyperref\thmCeleven{\ref[thm:C11]}.

\smallskip

\Statement[thm:C11] 
\noindent{\bf Theorem~\hyperref\thmCeleven{\ref[thm:C11]}} \ 
A holonomic \hgt\  $f$ on $\Z^k$ over an
algebraically closed field $K$ is \hyperref\defpwp{piecewise proper}:
there exist a polynomial $C\in K[\z]$ (as
in Theorem~\hyperref\thmCnine{\ref[thm:C9]}) and a finite number of \hyperref\defBeighteen{polyhedral
regions} $\R_1,\ldots,\R_L$ such that $\Z^k$ is the union of the
$\R_{\ell}$ and a \hyperref\defBeleven{set of measure zero}, and for each region $\R_{\ell}$
there exist a vector $\vecgamma\in K^k$, constants
$m_1,\ldots,m_p,n_1,\ldots,n_q\in K$, vectors $\v_1,\ldots,
\v_p,\w_1,\ldots,\w_q\in\Z^k$, and integers
$r_1,\ldots,r_p,s_1,\ldots,s_q$ such that 
\Enumerate
\item for all $\z\in \R_{\ell}$,
$$f(\z)=
\gamma_1^{z_1}\cdots\gamma_k^{z_k}
C(\z)
\frac{\prod_{i=1}^p(m_i)_{\v_i\cdot
\z+r_i}}{\prod_{j=1}^q(n_j)_{\w_j\cdot\z+s_j}};$$ 
\item for all
$i$ and $j$ and $\z\in \R_{\ell}$, $\v_i\cdot\z+r_i$ and
$\w_j\cdot\z+s_j$ are positive; 
\item the Pochhammer symbols occurring
in the products are nonzero.
\EnumerateStop
\StatementStop

Theorem~\hyperref\thmCeleven{\ref[thm:C11]} 
and
Theorem~\hyperref\thmCtwelve{\ref[thm:C12]}
apply to \hgt s on polyhedral regions
other than $\Z^k$ via Lemma~\hyperref\lemBtwentyoneb{\ref[lem:B21b]}.
We define a function $f$ on a 
polyhedral region $\R\subset \Z^k$
to be holonomic if the function
$g$ on $\Z^k$  is holonomic,
where
$$g(\z)=
\cases
f(\z)&\text{if $\z \in \R$}\\
0&\text{otherwise.}
\endcases$$


\smallskip

\hyperdef\chapteRivSECi{ChapterSection}{chapteRivSECi}{}
\Section Proof of Theorem~\ref[thm:C11]

We require the following lemmas.

\Lemma[lem:C1]  \hyperdef\lemCone{Lemma}{C1}{} If $f$ is holonomic and \hyperref\defBeleven{nondegenerate} then there exists a finite set $V\subset\Z^k$ such that $f(\z)\ne0$ implies $f(\z+\v)\ne0$ for some $\v\in V$.
\LemmaStop

Let $E_i$ be the shift
operator for $z_i$. Thus, $\vec E^{\v}f(\z)=E_1^{v_1}\cdots
E_k^{v_k}f(\z)=f(\z+\v)$.

\Proof
Since $f$ is holonomic, by
Lemma~4.1 of~\hyperref\Zeilninety{\bibref[Zeil90]} or~1.5 of~\hyperref\WZninetytwo{\bibref[WZ92]}, for each
$i\in\{1,\ldots,k\}$ there exists a nonzero operator $L_i(z_i,\vec E)$
that annihilates~$f$; $L_i(z_i,\vec E)=\sum_{\v\in V_i}c_{i,
\v}(z_i)\vec E^{\v}$ for some finite set $V_i\subset\Z^k$ and nonzero
univariate polynomials $c_{i,\v}\in K[z]$. The point is that the
operator~$L_i$ is free of all the variables $z_1,\ldots,z_k$ except
$z_i$. Since $L_if=0$ implies $\vec E^{\v}L_if=0$, we may assume
that $\vec0\in V_i$.

Let $b_i=c_{i,\vec0}$. Thus $b_i$ is nonzero and
$$b_i(z_i)f(\z)=-\sum_{\v\in V_i\atop\v\ne\vec0}c_{i,
\v}(z_i)f(\z+\v).$$
Hence $b_i(z_i)f(\z)\ne0$ implies $f(\z+\v)\ne0$ for some
$\v\in V_i$. Letting $\bar V=V_1\cup\cdots\cup V_k$ and $S=\{
\z\colon b_i(z_i)=0\text{ for }1\le i\le k\}$, it follows that if
$f(\z)\ne0$ and $\z\notin S$, then $f(\z+\v)\ne0$ for
some $\v\in\bar V$. Of course, the set $S$ is finite since each of
the $b_i$ has a finite number of roots. Since $f$ is nondegenerate,
there exists $\z_0\in\Z^k$ such that $f(\z_0)\ne0$ and $
\z_0\notin S$. Finally, let $V=\bar V\cup\{\z_0-\z\colon
\z\in S\}$.
\ProofStop

\Lemma[lem:C3]  \hyperdef\lemCthree{Lemma}{C3}{} If $f$ is holonomic on $\Z^k$ and $f$ is a \hyperref\defBtwo{\zd}, then $f=0$~\hyperref\defBeleven{a.e.}
\LemmaStop

\Proof We prove by induction on the dimension $k$ and the total degree $d$ of $p$ that if $f$ is holonomic on $\Z^k$ and $pf=0$~a.e. then $f=0$~a.e. The statement is clearly true if either $k=1$ or $d=1$ since in either case $p$ is simple: by Lemma~\hyperref\lemBtwelve{\ref[lem:B12]}, $p(\z)\ne0$ except on a set of measure zero, so $f=0$~a.e.

We assume that the lemma is true if either the dimension is less than $k$ or the degree is less than $d$ and prove that it is true if the dimension is $k$ and the degree is $d$. Suppose, to the contrary, there exists a polynomial $p\in K[\z]$ of degree $d$ and a holonomic function $f$ on $\Z^k$ such that $pf=0$ but $f$ is nondegenerate. By Lemma~\hyperref\lemfour{\ref[lem:four]}, there exists a finite set $V$ such that $f(\z)\ne0$ implies $f(\z+\v)\ne0$ for some $\v\in V$. Let $\vec x=(x_1,\ldots,x_k)$ where the $x_i$ are indeterminates. Let $g=\sum_{\v\in V}ff^{\v}\vec x^{\v}$. If $f(\z)\ne0$, then $g(\z)\ne0$, so $g$ is nondegenerate. Since $V$ is finite, $ff^{\v}=0$~a.e.\ for each $\v\in V$ implies $g=0$~a.e. Thus for at least one $\v\in V$, $ff^{\v}$ is nondegenerate. If $p\ne p^{\v}$, then $q=p-p^{\v}$ is of lower degree and $qff^{\v}=f^{\v}(pf)-f(pf)^{\v}=0$~a.e. But $ff^{\v}$ is holonomic, contradicting the assumption that the statement is true if the degree is less than $d$ and the dimension is $k$.

If $p=p^{\v}$ we reduce the dimension. If $h$ is holonomic on
$\Z^k$, 
and $M$ is an invertible $k\times k$ integer matrix,
then $g(\z)=h(M\z)$
is
also holonomic. Thus, by a change of variable we may assume that $p=p^{\e_k}$, and hence $p$
is free of $z_k$.

Let $x$ and $y$ be indeterminates,
and let $K[[x,y]]$ be the ring of formal power series in
$x$ and $y$.
Define the function $F\colon \Z^{k-1} \to K[[x,y]]$ by
$$F(z_1,\ldots,z_{k-1})=\sum_{z_k\ge0}f(z_1,\ldots,z_k)x^{z_k}
+\sum_{z_k<0}f(z_1,\ldots,z_k)y^{-z_k}.$$
By Proposition~3.4 of~\hyperref\Zeilninety{\bibref[Zeil90]}, $F$ is holonomic in
$z_1,\ldots,z_{k-1}$. But $pF=0$, and therefore $F=0$~a.e.\ by the assumption
that the statement is true for dimension $k-1$ and degree~$d$. The
function
$F$ is constructed so that 
the formal power series
$F(z_1,\ldots,z_{k-1})=0$ only if
$f(z_1,\ldots,z_k)=0$ for all $z_k\in\Z$.
Using the fact that 
$F=0$~a.e.\ in $\Z^{k-1}$,
it is easily seen then that 
$f=0$~a.e. in $\Z^k$, contrary to assumption.
\ProofStop

\Corollary[cor:C4]  \hyperdef\corCfour{Corollary}{C4}{} A holonomic \hgt\ is \hyperref\defBthirteena{honest}.
\CorollaryStop

\Proof By Lemma~\hyperref\lemBthree{\ref[lem:B3]} there exist $\bar A_{\v}$ and $\bar
B_{\v}$ such that $\bar A_{\v}f=\bar B_{\v}f^{\v}$. Let
$\bar A_{\v}=pA_{\v}$ and $\bar B_{\v}=pB_{\v}$ where
$p$, $A_{\v}$, and $B_{\v}$ are polynomials and $A_{\v}$
and $B_{\v}$ are relatively prime. Then $p(A_{\v}f-B_{
\v}f^{\v})=0$. But $g=A_{\v}f-B_{\v}f^{\v}$ is
holonomic and a \zd, so by Lemma~\hyperref\lemCthree{\ref[lem:C3]}, $g=0$~a.e.
Hence, $A_{\v}f=B_{\v}f^{\v}$~a.e.
\ProofStop

\Lemma[lem:C5]  \hyperdef\lemCfive{Lemma}{C5}{} Let $f$ be a \hgt\  on $\Z^k$ that is not a \hyperref\defBtwo{\zd}, and let $R_{
\v}$ be the \hyperref\defBfour{term ratio} of $f$ in the direction $\v$. Let
$V\subset\Z^k$ be a finite set and let $c_{\v}\in K[\z]$ be
given for each $\v\in V$. If $\sum_{\v\in V}c_{\v}f^{
\v}=0$~\hyperref\defBeleven{a.e.}, then $\sum_{\v\in V}c_{\v}R_{\v}=0$.
If $f$ is also holonomic, then the converse is true.
\LemmaStop

\Proof Since $f$ is a \hgt, by Lemma~\hyperref\lemBthree{\ref[lem:B3]} there exist polynomials $A_{\v}$ and $B_{\v}$ for each $\v\in\Z^k$ such that $A_{\v}f=B_{\v}f^{\v}$. Since $\sum_{\v\in V}c_{\v}f^{\v}=0$~a.e., $(\prod_{\w\in V}B_{\w})\sum_{\v\in V}c_{\v}f^{\v}=0$~a.e., hence $\sum_{\v\in V}c_{\v}(\prod_{\w\in V\setminus\{\v\}}B_{\w})B_{\v}f^{\v}=0$~a.e., hence 
$$\left(\sum_{\v\in V}c_{\v}\left(\prod_{\w\in V\setminus\{\v\}}B_{\w}\right)A_{\v}\right)f=0 \text{\ a.e.\ .}$$
Since $f$ is not a \zd, 
$$\sum_{\v\in V}c_{\v}\left(\prod_{\w\in V\setminus\{\v\}}B_{\w}\right)A_{\v}=0.$$ 
Dividing by $\prod_{\w\in V}B_{\w}$ it follows that $\sum_{\v\in V}c_{\v}A_{\v}/B_{\v}=0$. By Lemma~\hyperref\lemBsix{\ref[lem:B6]}, $R_{\v}=A_{\v}/B_{\v}$.
Hence $\sum_{\v\in V}c_{\v}R_{\v}=0$. 

Conversely, assuming $f$ is holonomic,
if $\sum_{\v\in V}c_{\v}R_{\v}=0$,
then multiplying by
$\left(\prod_{\w\in V}B_{\w}\right)f$,
it follows that
$\sum_{\v\in V}c_{\v}\left(\prod_{\w\in V\setminus\{\v\}}B_{\w}\right)A_{\v}f=0$. 
Hence,
$$\sum_{\v\in V}c_{\v}\left(\prod_{\w\in V\setminus\{\v\}}B_{\w}\right)B_{\v}f^{\v}=0,$$ 
and hence,
$$\left(\prod_{\w\in V}B_{\w}\right)\sum_{\v\in V}c_{\v}f^{\v}=0.$$ 
Since $f$ is holonomic, 
$\sum_{\v\in V}c_{\v}f^{\v}$
is holonomic.
But $\sum_{\v\in V}c_{\v}f^{\v}$ is a \zd ,
 so by Lemma~\hyperref\lemCthree{\ref[lem:C3]},
$\sum_{\v\in V}c_{\v}f^{\v}=0$~a.e.
\ProofStop

\Lemma[lem:C6]  \hyperdef\lemCsix{Lemma}{C6}{} Let $V\subset\Z^2$ be finite and let $A_{\v}$ and $B_{\v}\in K[z_1,z_2]$ be products of nonzero \hyperref\defAtwo{simple} polynomials for each $\v\in V$. Let $R\in K(z_1,z_2)$ be a rational function such that
$$\sum_{\v\in V}\frac{A_{\v}}{B_{\v}}R^{\v}=0.$$
Then the denominator of $R$ is a product of \hyperref\defAtwo{simple} polynomials.
\LemmaStop

\Proof Let $R=P/Q$, where $P$ and $Q$ are relatively prime. Let $d$ be
an irreducible divisor of $Q$. We will show that $d$ is simple. Suppose
the contrary. Then the dimension of $\rgal d$ is less than~$1$ by the
corollary of Lemma~\hyperref\lemfour{\ref[lem:four]}, that is, the dimension is~$0$. Thus
$d\ne d^{\w}$ for any nonzero $\w\in\Z^2$. By a corollary of
Lemma~\hyperref\lemzero{\ref[lem:zero]}, $d\ne cd^{\w}$ for any nonzero $
\w\in\Z^2$ and $c\in K$. Thus $d^{\w}$ divides $Q$ for only finitely
many $\w\in\Z^2$. Let $\w_0$ be the leftmost of the lowest of
the $\w$ such that $d^{\w}\mid Q$. In other words,
$\w_0=(x_0,y_0)$ where
$$y_0=\min\{y\in\Z\colon d^{(x,y)}\mid Q\text{ for some }x\in\Z\}$$
and 
$$x_0=\min\{x\in\Z\colon d^{(x,y_0)}\mid Q\}.$$
Similarly, let $\v_0$ be the leftmost of the lowest vectors in $V$.

Multiplying the equation
$$\sum_{\v\in V}\frac{A_{\v}}{B_{\v}}R^{\v}=0$$
by $\prod_{\u\in V}B_{\u}Q^{\u}$, we have
$$\sum_{\v\in V}A_{\v}P^{\v}\prod_{\u\in V\setminus\{\v\}}B_{\u}Q^{\u}=0.\tag{*}$$
Since $d^{\w_0}\mid Q$, it follows that 
$d^{\v_0+\w_0}\mid Q^{\v_0}$. Since $Q^{\v_0}$ appears
in every term of the sum except the term for $\v=\v_0$, it
follows that $d^{\v_0+\w_0}\mid A_{\v_0}P^{
\v_0}\prod_{\u\in V\setminus\{\v_0\}}B_{\u}Q^{\u}$.
Furthermore, since $A_{\v}$ and $B_{\v}$ are simple and $d$ is
not and $Q^{\v_0}$ and $P^{\v_0}$ are relatively prime, it
follows that $d^{\v_0+\w_0}\mid Q^{\u}$ for some $
\u\in V\setminus\{\v_0\}$. Thus $d^{\w_0+\v_0-\u}\mid
Q$ for some $\u\in V\setminus\{\v_0\}$, contradicting the
definition of $\w_0$.
\ProofStop

\Lemma[lem:C7]  \hyperdef\lemCseven{Lemma}{C7}{} Let $f$ be a \hyperref\defBeleven{nondegenerate} 
holonomic \hgt\ on $\Z^k$ and let $R_{\v}$ be the \hyperref\defBfour{term ratio} of $f$ 
in the direction $\v$ for each $\v\in\Z^k$. Let
$$R_{\w}(\z)=\frac{C(\z+\w)}{C(\z)}\frac{D(\z)}{D(\z+\w)}\prod_{\v\in V}\gp j0{\v\cdot\w}\frac{a_{\v}(\z\cdot\v+j)}{b_{\v}(\z\cdot\v+j)}$$
as in Theorem~\hyperref\thmBnine{\ref[thm:B9]}. Then $D$ is a product of \hyperref\defAtwo{simple} polynomials.
\LemmaStop

\Proof By Lemma~4.1 of~\hyperref\Zeilninety{\bibref[Zeil90]}, for each $i,j\in\{1,\ldots,k\}$
there exists a nonzero operator
$$L(E_i,E_j,z_1,\ldots,z_{j-1},z_{j+1},\ldots,z_k)$$
such that $L$ is free of all $E$s except $E_i$ and $E_j$, and $L$ is
free of $z_j$. Thus
$$\sum_{\w\in W}\alpha_{\w}f^{\w}=0,$$
where $W$ is a finite subset of the subspace of $\Z^k$ generated by
$\e_i$ and $\e_j$. Furthermore, since each polynomial $\alpha_{\w}$ is free of
$z_j$,
each $\alpha_{\w}$ is simple
over the field
$K_{i,j}=K(\{z_1,\ldots,z_k\}\setminus\{z_i,z_j\})$.

By Lemma~\hyperref\lemCfive{\ref[lem:C5]}, $\sum_{\v\in W}\alpha_{\w}R_{\w}=0$. Hence 
$$\sum_{\w\in W}\alpha_{\w}\frac{C^{\w}}C\frac D{D^{\w}}\frac{\bar
A_{\w}}{\bar B_{\w}}=0,$$
where
$$\frac{\bar A_{\w}}{\bar B_{\w}}=\prod_{\v\in V}\gp
j0{\v\cdot\w}\frac{a_{\v}(\z\cdot\v+j)}{b_{\v}(\z\cdot\v+j)}.$$
Hence
$$\sum_{\w\in W}\alpha_{\w}\frac{\bar A_{\w}}{\bar
B_{\w}}\frac{C^{\w}}{D^{\w}}=0,$$
and hence
$$\sum_{\w\in W}\frac{A_{\w}}{B_{\w}}\frac{C^{\w}}{D^{\w}}=0,$$
where $A_{\w}$ and $B_{\w}$ are the products of polynomials
that are simple over $K_{i,j}$.


It follows by Lemma~\hyperref\lemCsix{\ref[lem:C6]} that any irreducible divisor $d$ of
$D$ is simple over $K_{i,j}$.
Hence,
by Corollary~\hyperref\coroldlemmafour{\ref[cor:oldlemmafour]},
the dimension of
$\rgal(d,(z_i,z_j),K_{i,j})$ is at least~$1$.
Let $\Q_{i,j}$ be the subspace of $\Q^k$ generated by $\e_i$ and
$\e_j$.
Since
$\rgal(d,(z_i,z_j),K_{i,j})$ is 
isomorphic to
$\Q_{i,j} \cap \rgal d$,
the dimension of $\Q_{i,j}\cap\rgal d$ is at
least~$1$.

We show that $d$ is simple in $z_1,\ldots,z_k$, that is, the dimension
of $\rgal d$ is at least $k-1$. Let $\Q_i$ be the subspace of $\Q^k$
generated by $\e_i$. If $\Q_i\subset\rgal d$ for each
$i\in\{1,\ldots,k\}$, then $\rgal d=\Q^k$. Hence the dimension is $k$.
Otherwise, say for concreteness that $\Q_1\not\subset\rgal d$. Then for
each $j\in\{2,\ldots,k\}$ there exists $\u_j\in\Q_{1,j}\cap\ \rgal d$ such that
$\u_j\notin\Q_1$. It's easily seen that $\u_2,\ldots,\u_k$ are
independent, so the dimension of $\rgal d$ is at least $k-1$.
\ProofStop



\Lemma[lem:C8]  \hyperdef\lemCeight{Lemma}{C8}{} Let $f$ be a holonomic \hgt\ on $\Z^k$ and let $R_{\v}$ be the \hyperref\defBfour{term ratio} of $f$ in the direction $\v$ for each $\v\in\Z^k$. There exist a polynomial $C$, a finite set $V\subset\Z^k$, and univariate polynomials $a_{\v}$ and $b_{\v}$ for each $\v\in V$ such that
$$R_{\w}(\z)=\frac{C(\z+\w)}{C(\z)}\prod_{\v\in V}\gp j0{\v\cdot\w}\frac{a_{\v}(\z\cdot\v+j)}{b_{\v}(\z\cdot\v+j)}.$$
\LemmaStop

\Proof By Theorem~\hyperref\thmBnine{\ref[thm:B9]} we can express $R_{\w}(\z)$ as
$$R_{\w}(\z)=\frac{C(\z+\w)}{C(\z)}\frac{D(\z)}{D(\z+\w)}\prod_{\v\in V}\gp j0{\v\cdot\w}\frac{a_{\v}(\z\cdot\v+j)}{b_{\v}(\z\cdot\v+j)},$$
and by Lemma~\hyperref\lemCseven{\ref[lem:C7]} the irreducible divisors of $D$ are simple. The result follows by noting that
$$\frac{d(\v\cdot\z)}{d(\v\cdot(\z+\w))}=\gp j0{\v\cdot\w}\frac{d(\v\cdot\z+j)}{d(\v\cdot\z+1+j)},$$
and hence the divisors of $D$ can be absorbed into the `factorial part'.
\ProofStop


\Theorem[thm:C9]  \hyperdef\thmCnine{Theorem}{C9}{} Let $f$ be a holonomic \hgt\ on $\Z^k$. Then there exist a polynomial $C$, a finite set $V\subset\Z^k$, and univariate polynomials $a_{\v}$ and $b_{\v}\in K[z]$ for each $\v\in V$ (as in Lemma~\hyperref\lemCeight{\ref[lem:C8]}) and a finite number of \hyperref\defBeighteen{polyhedral regions} $\R_1,\ldots,\R_m$ such that
\Enumerate
\item $\Z^k$ is the union of the $\R_i$ and a \hyperref\defBeleven{set of measure zero};
\item for each $i\in\{1,\ldots,m\}$ there exists $\z_0\in \R_i$ such that $C(\z_0)\ne0$, and for all $\z\in \R_i$,
$$f(\z)=f(\z_0)\frac{C(\z)}{C(\z_0)}\prod_{\v\in V}\gp j{\z_0\cdot\v}{\z\cdot\v}\frac{a_{\v}(j)}{b_{\v}(j)}.$$
\item  all the terms $a_{\v}(j)$ and $b_{\v}(j)$
occurring in the product are nonzero.
\EnumerateStop
\TheoremStop

Theorem~\hyperref\thmCnine{\ref[thm:C9]} applies to \hgt s on polyhedral regions
other than $\Z^k$ via Lemma~\hyperref\lemBtwentyoneb{\ref[lem:B21b]}.



\Proof The theorem is an immediate consequence of Theorem~\hyperref\thmBtwentynine{\ref[thm:B29]} and Lem\-ma~\hyperref\lemCeight{\ref[lem:C8]}.
\ProofStop

\Corollary[cor:C10]  \hyperdef\corCten{Corollary}{C10}{} Let $f$ be an holonomic \hgt\ on $\Z^k$. Then there
exist a polynomials $C\in K[\z]$  and  a finite number of
\hyperref\defBeighteen{polyhedral regions}
$\R_1,\ldots,\R_L$ such that
$\Z^k$ is the union of the $\R_\ell$ and a \hyperref\defBeleven{set of measure zero} and for each
region $\R_\ell$
there exist a finite set $V\subset\Z^k$ and
univariate polynomials $a_{\v}$ and $b_{\v}\in K[z]$ and an
integer $n_{\v}$ for each $\v\in V$ such that, for all $
\z\in \R_\ell$,
\Enumerate
\item $\ds{f(\z)=C(\z)\prod_{\v\in V}\prod_{j=1}^{\v\cdot\z+n_{\v}}\frac{a_{\v}(j)}{b_{\v}(j)}}$,
\item for all $\v\in V$, $\v\cdot\z+n_{\v}$ is a positive integer, and
\item for all $\v\in V$ and $j$, $1\le j\le\v\cdot\z+n_{\v}$, $a_{\v}(j)\ne0$ and $b_{\v}(j)\ne0$.
\EnumerateStop
\CorollaryStop

Corollary~\hyperref\corCten{\ref[cor:C10]} applies to \hgt s on polyhedral regions
other than $\Z^k$ via Lemma~\hyperref\lemBtwentyoneb{\ref[lem:B21b]}.


\Proof This corollary is an immediate consequence of  Corollary~\hyperref\corBthirty{\ref[cor:B30]} and Lemma~\hyperref\lemCeight{\ref[lem:C8]}.
\ProofStop

\Theorem[thm:C11]  \hyperdef\thmCeleven{Theorem}{C11}{} A holonomic \hgt\  $f$ on $\Z^k$ over an
algebraically closed field $K$ is \hyperref\defpwp{piecewise proper}:
there exist a polynomial $C\in K[\z]$ (as
in Theorem~\hyperref\thmCnine{\ref[thm:C9]}) and a finite number of \hyperref\defBeighteen{polyhedral
regions} $\R_1,\ldots,\R_L$ such that $\Z^k$ is the union of the
$\R_{\ell}$ and a \hyperref\defBeleven{set of measure zero}, and for each region $\R_{\ell}$
there exist a vector $\vecgamma\in K^k$, constants
$m_1,\ldots,m_p,n_1,\ldots,n_q\in K$, vectors $\v_1,\ldots,
\v_p,\w_1,\ldots,\w_q\in\Z^k$, and integers
$r_1,\ldots,r_p,s_1,\ldots,s_q$ such that 
\Enumerate
\item for all $\z\in \R_{\ell}$,
$$f(\z)=
\gamma_1^{z_1}\cdots\gamma_k^{z_k}
C(\z)
\frac{\prod_{i=1}^p(m_i)_{\v_i\cdot
\z+r_i}}{\prod_{j=1}^q(n_j)_{\w_j\cdot\z+s_j}};$$ 
\item for all
$i$ and $j$ and $\z\in \R_{\ell}$, $\v_i\cdot\z+r_i$ and
$\w_j\cdot\z+s_j$ are positive; 
\item the Pochhammer symbols occurring
in the products are nonzero.
\EnumerateStop
\TheoremStop

Theorem~\hyperref\thmCeleven{\ref[thm:C11]} applies to \hgt s on polyhedral regions
other than $\Z^k$ via Lemma~\hyperref\lemBtwentyoneb{\ref[lem:B21b]}.


\Proof This theorem is an immediate consequence of  Corollary~\hyperref\corCten{\ref[cor:C10]} and Lemma~\hyperref\lemBtwentythree{\ref[lem:B23]}.
\ProofStop


