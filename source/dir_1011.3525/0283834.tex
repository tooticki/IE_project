\documentclass[11pt]{amsart}
\usepackage{graphicx}
\usepackage{amssymb}
\usepackage{amsmath}

\textwidth = 6.5 in \textheight = 9 in \oddsidemargin = 0.0 in
\evensidemargin = 0.0 in \topmargin = -0.2 in \headheight = 0.2 in
\headsep = 0.2 in
\parskip = 0.2in
\parindent = 0.0in

\theoremstyle{theorem}
\newtheorem{theorem}{Theorem}[section]
\theoremstyle{lemma}
\newtheorem{lemma}{Lemma}
\newtheorem{claim}{Claim}
\theoremstyle{corollary}
\newtheorem{corollary}[theorem]{Corollary}
\theoremstyle{proposition}
\newtheorem{proposition}[theorem]{Proposition}
\theoremstyle{definition}
\newtheorem{definition}{Definition}
\theoremstyle{remark}
\newtheorem*{remark}{Remark}

\def\d{\displaystyle}
\def\z{\hat{\zeta}}
\def\tz{\zeta}
\newcommand\gl{\mathfrak{gl}}


\begin{document}

\normalsize

\title{Calculation of local formal Fourier transforms}
\author{Adam Graham-Squire}
\date{}





 \begin{abstract} We calculate the local Fourier transforms for connections on the formal punctured disk, corroborating the results of J. Fang \cite{fang} and C. Sabbah \cite{sabbah} using a different method. Our method is similar to Fang's, but more direct.
 \end{abstract}

\maketitle

\section{Introduction}

In \cite{bloch}, S.~Bloch and H.~Esnault introduced the local Fourier transforms for connections on the formal punctured disk.
Explicit formulas for it were proved by J.~Fang \cite{fang} and C.~Sabbah \cite{sabbah}. Interestingly, the calculations rely
on different ideas: the proof of \cite{fang} is more algebraic, while \cite{sabbah} uses geometric methods.

In this paper, we provide yet another proof of these formulas. Our approach is closer to Fang's, but more straightforward. In order to
calculate a particular local Fourier transform, one must ascertain the `canonical form' of the local Fourier transform of a given connection.
This amounts to constructing an isomorphism between two connections (on a punctured formal disk). In \cite{fang}, this is done
by writing matrices of the connections with respect to certain bases. We work with operators directly, using techniques
described by D. Arinkin in \cite[Section 7]{dima}.

\subsection*{Acknowledgements} I am very grateful to my advisor Dima Arinkin for many helpful discussions and his consistent encouragement of this work.

\section{Definitions and Conventions}

We fix a ground field $\Bbbk$, which is assumed to be algebraically closed of characteristic zero.

\subsection{Connections on formal disks}
Consider the field of formal Laurent series $K=\Bbbk((z))$.

\begin{definition} \label{connection} Let $V$ be a finite-dimensional vector space over $K$. A \emph{connection} on $V$ is a $\Bbbk$-linear operator
$\nabla:V\to V$ satisfying the Leibniz identity:
\[\nabla(fv)=f\nabla(v)+\frac{df}{dz}v\]
for all $f\in K$ and $v\in V$. A choice of basis in $V$ gives an isomorphism $V\simeq K^n$; we can then write $\nabla=\nabla_z$ as $\frac{d}{dz}+A$, where
$A=A(z)\in\gl_n(K)$ is the \emph{matrix} of $\nabla$ with respect to this basis.
\end{definition}

We write $\mathcal{C}$ for the category of vector spaces with connections over $K$. Its objects are pairs
$(V, \nabla)$, where $V$ is a finite-dimensional $K$-vector space and $\nabla:V\to V$ is a connection. Morphisms between
$(V_1,\nabla_1)$ and $(V_2,\nabla_2)$ are $K$-linear maps $\phi:V_1\to V_2$ that are \emph{horizontal} in the sense that
$\phi\nabla_1=\nabla_2\phi$.

 We summarize below some well-known
properties of connections on formal disks. The results go back to Turritin \cite{turritin} and Levelt \cite{levelt};
a more recent and concise reference is \cite[Sections 5.9 and 5.10]{beilinson}.

Let $q$ be a positive integer and consider the field $K_q=\Bbbk((z^{1/q}))$. (Note: $K_q$ is the unique extension of $K$ of degree $q$.)
For every $f\in K_q$, we define an object $E_f\in\mathcal{C}$ by
\[E_f=E_{f,q}=\left(K_q,\frac{d}{dz}+z^{-1}f\right).\]

In terms of the isomorphism class of an object $E_f$, the reduction procedures of \cite{levelt} and \cite{turritin} illustrate that we need only consider $f$ in the quotient
\[\Bbbk((z^{1/q}))\Big/\left(z^{1/q}\Bbbk[[z^{1/q}]]+\frac{1}{q}\mathbb{Z}\right).\]

Let $R_q$ (we write $R_q(z)$ when we wish to emphasize the local coordinate) be the set of orbits for the action of the Galois group $\mathrm{Gal}(K_q/K)$ on this quotient. Explicitly, the
Galois group is identified with the group of degree $q$ roots of unity $\eta\in \Bbbk$; the action on $f\in R_q$ is by
$f(z^{1/q})\mapsto f(\eta z^{1/q})$. Finally, denote by $R^\circ_q\subset R_q$ the set of $f\in R_q$ that cannot be represented
by elements of $K_r$ for any $0<r<q$.

The following proposition lists some well-known facts about the objects $E_f$.  The proofs of the different parts of the proposition are either straightforward or common in the literature, and are thus omitted.

\begin{proposition}\label{prop} With notation as above:
\begin{enumerate}
\item \label{prop1} The isomorphism class of $E_f$ depends only on the orbit of the image of $f$ in $R_q$.
\item \label{prop2} $E_f$ is irreducible if and only if the image of $f$ in $R_q$ belongs to $R^\circ_q$. As $q$ and $f$ vary,
we obtain a complete list of irreducible objects of $\mathcal{C}$.
\item \label{prop3}Every $E\in\mathcal{C}$ can be written as
\[E\simeq\bigoplus_i(E_{f_i,q_i}\otimes J_{m_i}),\]
where the $E_{f,q}$ are irreducible and $J_m=(K^m,\frac{d}{dz}+z^{-1}N_m)$, with $N_m$ representing the nilpotent Jordan block of size $m$.
\end{enumerate}
\end{proposition}

\begin{remark}
Proposition \label{prop} \label{prop3} is particularly useful because it allows us to reduce the calculation of the local Frouier transform of $E\in\mathcal{C}$ to looking at the calculation on $E_f$.  A precise statement is found in Corollary \label{corollary}.
\end{remark}
%\begin{proof}[Sketch of Proofs]
 %   \begin{enumerate}
  %      \item The morphism $E_{f(z^{1/q})}\mapsto E_{f(\eta z^{1/q})}$ is an isomorphism.
   %     \item Let $f\in K_q$ such that $f\in K_r$ for $r<q$.  Then $r|q$ and $K_r$ is a proper subspace of $K_q$ which is preserved by the action of multiplication by $f$.
    %    \item See \cite[Section 5.9]{bloch}
    %\end{enumerate}
%\end{proof}

\subsection{Local Fourier transforms}


 Sometimes it is useful to keep track of the choice of local coordinate for $\mathcal{C}$.  To stress the coordinate, we write $\mathcal{C}_0$ to indicate the coordinate $z$ at the point zero and $\mathcal{C}_{\infty}$ to indicate the coordinate $\zeta=\frac{1}{z}$ at the point at infinity.  Note that $\mathcal{C}_0$ and $\mathcal{C}_{\infty}$ are both isomorphic to $\mathcal{C}$, but not canonically.  We also denote by  $\mathcal{C}_{\infty}^{<1}$ (respectively $\mathcal{C}_{\infty}^{>1}$) the full subcategory of $\mathcal{C}_{\infty}$ of connections whose irreducible components all have slopes less than one (respectively greater than one); that is, $E_f$ such that $-1< \text{ord}(f)$ (respectively $-1>\text{ord}(f)$).

We define the local Fourier transforms $\mathcal{F}^{(0,\infty)}$, $\mathcal{F}^{(\infty,0)}$ and $\mathcal{F}^{(\infty,\infty)}$ using \cite[Propositions 3.7, 3.9 and 3.12]{bloch} while following the convention of \cite[Section 2.2]{dima}.  The Fourier transform coordinate of $z$ is $\hat{z}$, with $\z=\frac{1}{\hat{z}}$. Let $E=(V,\nabla)\in\mathcal{C}_0$ such that $\nabla$ has no horizontal sections, thus $\nabla$ is invertible.  The following is a precise definition for $\mathcal{F}^{(0,\infty)}E$, the other local Fourier transforms can be defined analogously.  Consider on $V$ the $\Bbbk$-linear operators
\begin{equation}\label{fourierdef}
	\z=-\nabla_z^{-1}: V\to V \text{ and } \hat{\nabla}_{\z}=-\z^{-2}z: V \to V.
\end{equation}
As in \cite{dima}, $\z$ extends to define an action of $\hat{K}=\Bbbk((\z))$ on $V$ and dim$_{\hat{K}}V<\infty$.  Then the $\hat{K}$-vector space $V$ with connection $\hat{\nabla}_{\z}$ is denoted by
\[ \mathcal{F}^{(0,\infty)}(E)\in \mathcal{C}_{\infty}^{<1},\]
which defines the functor $\mathcal{F}^{(0,\infty)}:\mathcal{C}_0 \to \mathcal{C}_{\infty}^{<1}$.

Given the conventions above, we can express the other local Fourier transforms by the functors
$$\mathcal{F}^{(\infty,0)}:\mathcal{C}_{\infty}^{<1} \to \mathcal{C}_0 \text{   and   }\mathcal{F}^{(\infty,\infty)}:\mathcal{C}_{\infty}^{>1}\to \mathcal{C}_{\infty}^{>1}.$$

If one considers only the full subcategories of $\mathcal{C}_0$ and $\mathcal{C}_{\infty}^{<1}$ of connections with no horizontal sections, the functors $\mathcal{F}^{(0,\infty)}$ and $\mathcal{F}^{(\infty,0)}$ define an equivalence of categories.  Similarly, $\mathcal{F}^{(\infty,\infty)}$ is an auto-equivalence of the subcategory $\mathcal{C}_{\infty}^{>1}$ \cite[Propositions 3.10 and 3.12]{bloch}.

\section{Statement of Theorems}
Let $s$ be a nonnegative integer and $r$ a positive integer.

\begin{theorem}\label{thm1} \label{firstthm}Let $f\in R^{\circ}_r(z)$ with ord$(f)=-s/r$ and $f\neq 0$.  Then
\[\mathcal{F}^{(0,\infty)}E_f\simeq E_g,\]
where $g\in {R^{\circ}_{r+s}(\z)}$ is determined by the following system of equations:
\begin{equation}\label{zisyseq1}f=-z\hat{z}
\end{equation}
\begin{equation}\label{zisyseq2} g=f+\frac{s}{2(r+s)}
\end{equation}
\end{theorem}

\begin{remark} Recall that $\z=\frac{1}{\hat{z}}$.
We determine $g$ using \eqref{zisyseq1} and \eqref{zisyseq2} as follows.
First, using \eqref{zisyseq1} we express $z$ in terms of $\z^{1/(r+s)}$.  We then substitute this expression
into \eqref{zisyseq2} and solve to get an expression for $g(\z)$ in terms of $\z^{1/(r+s)}$.\\

When we use \eqref{zisyseq1} to write an expression for $z$ in terms of $\z^{1/(r+s)}$, the expression is not unique since we must make a choice of a root of unity.  More concretely, let $\eta$ be a primitive $(r+s)^{\text{th}}$ root of unity.  Then replacing $\z^{1/(r+s)}$ with $\eta\z^{1/(r+s)}$ in our equation for $z$ will yield another possible expression for $z$.  This choice will not affect the overall result, however, since all such possible expressions will lie in the same Galois orbit.  Thus by Proposition \ref{prop} (1), they will all correspond to the same connection.
\end{remark}

\begin{corollary}\label{corollary}

Let $E$ be an object in $\mathcal{C}$.  By Proposition \ref{prop} (3), let $E$ have decomposition
$\d{E\simeq \bigoplus_i \bigg(E_{f_i}\otimes J_{m_i}\bigg).}$
Then
$$\mathcal{F}^{(0, \infty)}E\simeq \bigoplus_i \bigg(E_{g_i}\otimes J_{m_i}\bigg)$$for $E_g$ as defined in Theorem \ref{firstthm}.
\end{corollary}

\begin{proof}[Sketch of Proof]
  $E_f\otimes J_m$ is the unique indecomposable object in $\mathcal{C}$ formed by $m$ successive extensions of $E_f$, thus we only need to know how $\mathcal{F}^{(0, \infty)}$ acts on $E_f$.  This is given by Theorem \ref{thm1}.
\end{proof}

\begin{theorem}\label{thm2}
	 Let $f\in R^{\circ}_r(\zeta)$ with ord$(f)=-s/r$, $s<r$, and $f\neq 0$. Then
\[\mathcal{F}^{(\infty, 0)}E_{f}\simeq E_{g},\]
where $g\in R^{\circ}_{r-s}(\hat{z})$ is determined by the following system of equations:
\begin{equation}\label{izsyseq1}f=z\hat{z}
\end{equation}
\begin{equation}\label{izsyseq2} g=-f+\frac{s}{2(r-s)}
\end{equation}

\end{theorem}

\begin{remark}
We determine $g$ from \eqref{izsyseq1} and \eqref{izsyseq2} as follows.  First, we use \eqref{izsyseq1} to express $\zeta$ in terms of $\hat{z}^{1/(r-s)}$.  We then substitute this expression into \eqref{izsyseq2} to get an expression for $g(\hat{z})$ in terms of $\hat{z}^{1/(r-s)}$.
\end{remark}

\begin{theorem}\label{thm3}
    Let $f\in  R^{\circ}_r(\zeta)$ with ord$(f)=-s/r$ and $s>r$. Then
\[\mathcal{F}^{(\infty, \infty)}E_{f}\simeq E_{g},\]
where $g\in R^{\circ}_{s-r}(\z)$ is determined by the following system of equations:
\begin{equation}\label{iisyseq1}f=z\hat{z}
\end{equation}
\begin{equation}\label{iisyseq2} g=-f+\frac{s}{2(s-r)}
\end{equation}

\end{theorem}

\begin{remark}
  We determine $g$ from \eqref{iisyseq1} and \eqref{iisyseq2} as follows.  First, we use \eqref{iisyseq1} to express $\zeta$ in terms of $\z^{1/(s-r)}$.  We then substitute this expression into \eqref{iisyseq2} to get an expression for $g(\z)$ in terms of $\z^{1/(s-r)}$.
\end{remark}

\section{Proof of Theorems}

\subsection{Outline of Proof of Theorem \ref{thm1}} We start with the operators given in \eqref{fourierdef}, viewing them as equivalent operators over $K_r$.  We wish to understand how the operator $\hat{\nabla}_{\z}$ acts in terms of the operator $\z$.  To do so, we need to define a fractional power of an operator, which is done in Lemma \ref{abratlemma}.  Lemma \ref{abratlemma} is the heavy lifting of the proof; the remaining portion is just calculation to extract the appropriate constant term (see remark below) from the expression given by Lemma \ref{abratlemma}.

\begin{remark} We give a brief explanation regarding the origin of the system of equations found in Theorem \ref{thm1}.  Consider the expressions given in \eqref{fourierdef}.  Suppose we were to make a \lq\lq naive" local Fourier transform over $K_r$ by defining $\nabla_z=z^{-1}f(z)$ and $\hat{\nabla}_{\z}=\z^{-1}g(\z)$; in other words, as in Definition \ref{connection} but without the differential parts.  Then from the equation $-(z^{-1}f)^{-1}=\z$ we conclude
\begin{equation}\label{syseq1nd}
f=-z\hat{z}.
\end{equation}Similarly, from $-\z^{-2}z=\z^{-1}g$ we find
$-\hat{z} z=g$, which when combined with \eqref{syseq1nd} gives
\begin{equation}\label{syseq2nd}
f=g.
\end{equation}When one incorporates the differential parts into the expressions for $\nabla_{z}$ and $\hat{\nabla}_{\z}$, one sees that the system of equations \eqref{syseq1nd} and \eqref{syseq2nd} nearly suffices to find the correct expression for $g(\z)$, only a constant term is missing.  This constant term arises from the interplay between the differential and linear parts of $\nabla_{z}$, and we wish to derive what the value of it is.  Similar calculations can be carried out to justify the systems of equations for Theorems \ref{thm2} and \ref{thm3}.
\end{remark}

\subsection{Lemmas}

 \begin{definition} Let $A$ and $B$ be $\Bbbk$-linear operators from $K_q$ to $K_q$.
We define Ord($A$) to be $$\d{\text{Ord}(A)=\text{inf}_{f\in K_q}\big(\text{ord}(Af)-\text{ord}(f)\big), \text{ with Ord}(0):=\infty}$$ and define $\underline{o}(z^k)$ by $$A=B+\underline{o}(z^k) \text{  if and only if  } \text{Ord}(A-B)\geq k.$$
\end{definition}

\begin{lemma}\label{abintlemma}
    Let $A$ and $ B$ be operators on $K_q$, with Ord$(A)=a$, Ord$(A^{-1})=-a$, Ord$(B)=b$, $a\leq b$, $A$ invertible, and $[A,[B,A]]=0$.  Then
    \begin{equation}\label{abint}
        (A+B)^{m}=A^{m}+mA^{(m-1)}B+\frac{m(m-1)}{2}A^{m-2}[B,A] + \underline{o}(z^{a(m-1)+b})
    \end{equation}for all $m\in\mathbb{Z}$.
\end{lemma}
\begin{proof} We prove that \eqref{abint} holds for $m\geq 1$ using induction. When one uses the expansion $(A+B)^{-1}=A^{-1}-A^{-1}BA^{-1}+\dots$ the proof for $m\leq -1$ is similar.  The case $m=0$ is trivial.  Our base case is $m=1$, where the result clearly holds.  Assuming the equation holds for $(A+B)^m$, we have
    \begin{equation*}
    \begin{split}
        (A+B)^{m}(A+B) & = A^{m+1}+mA^{m-1}BA+\frac{m(m-1)}{2}A^{m-2}[B,A]A+A^mB + \underline{o}(z^{a(m-1)+b+a})\\
                    & = A^{m+1}+mA^{m}B+mA^{m-1}[B,A]+\frac{m(m-1)}{2}A^{m-1}[B,A]+A^mB + \underline{o}(z^{am+b})\\
                    & = A^{m+1}+(m+1)A^{m}B+\frac{m(m+1)}{2}A^{m-1}[B,A] + \underline{o}(z^{am+b})
    \end{split}
    \end{equation*}which completes the induction.
\end{proof}

We now wish to use \eqref{abint} to define fractional powers of the operator $(A+B)$, given certain operators $A$ and $B$.  We follow the method of \cite[Section 7.1]{dima} to extend the definition, though our goal is more narrow; Arinkin defines powers for all $\alpha\in \Bbbk$, but we only need to define fractional powers $m\in\frac{1}{p}\mathbb{Z}$ for a given nonzero integer $p$.
\begin{lemma}\label{abratlemma}
    Let $A$ and $B$ be operators on $K_q$. Let $A=\text{ multiplication by } f= jz^{p/q}+ \underline{o}(z^{p/q})$, $j\neq 0$, and $B=z^n\frac{d}{dz}$ with $n$, $p\neq 0$ and $q\neq 0$ integers, Ord$(A)=a=\frac{p}{q}$, Ord$(B)=b=n-1$, and $a\leq b$. Then we can choose a $p^{\text{th}}$ root of $(A+B)$, $(A+B)^{1/p}$, such that \eqref{abint} holds for all $m\in \frac{1}{p}\mathbb{Z}$ where $(A+B)^{m}=((A+B)^{1/p})^{pm}$.
\end{lemma}

\begin{proof}
    We use notation as in \cite{dima}.  Letting $P=(1/j)(A+B)$ we have $P:K_q\to K_q$ is $\Bbbk$-linear of the form
    $$P\left(\sum_{\beta}c_{\beta}z^{\beta/q} \right)=\sum_{\beta}c_{\beta}\sum_{i\geq 0}p_i(\beta)z^{\frac{\beta+i+p}{q}}.$$
    Thus $p_0(\beta)=1$ and all $p_i$ are constants or have the form $\beta/q+$constant, so the necessary conditions \cite[Section 7.1, conditions (1) and (2)]{dima} are satisfied.  We can now define $P^m$, and likewise $(A+B)^m=j^mP^m$, for $m=\frac{1}{p}$.
\end{proof}


\subsection{Proof of Theorems}
\begin{proof}[Proof of Theorem \ref{thm1}] From \cite[Proposition 3.7]{bloch} we have the following equations for the local Fourier transform $\mathcal{F}^{(0,\infty)}$:
\begin{equation}\label{benotationzi}
    z=-\z^2(\partial_{\z}) \text{ and } \partial_z=-\z^{-1}.
\end{equation}
Writing $\partial_{\z}=\hat{\nabla}_{\z}=\frac{d}{d\z}+\z^{-1}g(\z)$ and $\partial_z=\nabla_{z}=\frac{d}{dz}+z^{-1}f(z)$, \eqref{benotationzi} becomes
\begin{equation}\label{mynotationzi}
    z=-\z^2\frac{d}{d\z}-\z g(\z)
\end{equation}
 and
\begin{equation}\label{firsteq}
    \frac{d}{dz}+z^{-1}f(z)=-\z^{-1}.
\end{equation}
Our goal is to use \eqref{firsteq} to write an expression for the operator $z$ in terms of $\z$, at which point we can substitute into \eqref{mynotationzi} to find an expression for $g(\z)$.  Since the leading term of $z^{-1}f(z)$ is $az^{-(r+s)/r}$, \eqref{firsteq} implies the operator $z$ can be written as
\begin{equation}\label{zdefzi}
    z=a^{r/(r+s)}(-\z)^{r/(r+s)}+\dots+*(-\z)+\underline{o}(\z).
\end{equation}
Here the ellipsis refers to higher order terms coming from the algebraic calculation of taking the $(\frac{-r}{r+s})^{\text{th}}$ power of $z^{-1}f(z)$, and the * represents the coefficient that will arise from the interplay between the differential and linear parts of $(-\z)$.  As explained in the outline, we wish to find the value of *.  Let $A=z^{-1}f(z)$ and $B=\frac{d}{dz}$, then $[B,A]=A'$. From \eqref{firsteq} we have $-\z=(A+B)^{-1}$, and we apply Lemma \ref{abratlemma} to find
\begin{equation*}
    (-\z)^{r/(r+s)}=a^{-r/(r+s)}\left(z+\dots+a^{-1}\left[\frac{-r}{r+s}\left(\frac{\mathbb{Z}}{r}\right)+\frac{-r}{r+s} + \frac{-s}{2(r+s)}\right]z^{1+(s/r)}+\underline{o}(z^{1+(s/r)})\right).
\end{equation*}
\begin{remark}  We use the notation $\frac{\mathbb{Z}}{r}$ to represent the operator $z\frac{d}{dz}$.  This notation makes sense, because $z\frac{d}{dz}:K_r\to K_r$ acts as $z\frac{d}{dz}(z^{n/r})=\frac{n}{r}(z^{n/r})$ for any $n\in\mathbb{Z}$.
\end{remark}

Also from Lemma \ref{abratlemma} we have
\begin{equation*}
    (-\z)=a^{-1}z^{1+(s/r)}+\underline{o}(z^{1+(s/r)}).
\end{equation*}
The appropriate value for * in \eqref{zdefzi} is the expression that will make the leading term of $*(-\z)$ cancel with $a^{-1}\left[\frac{-\mathbb{Z}}{r+s}+\frac{-r}{r+s} + \frac{-s}{2(r+s)}\right]z^{1+(s/r)}$, thus we find that
\begin{equation}\label{starzi}
    *=\frac{\mathbb{Z}+r}{r+s}+\frac{s}{2(s+r)}.
\end{equation}

Applying the equivalent operators of \eqref{mynotationzi} to $1\in K_r$, and using the fact that $\frac{d}{d\z }(1)=0$, we see that $z(1)=-\z g(\z)(1)$.
 Thus to find the expression for $g$ we simply need to compute the Laurent series in $\z$ given by $(-\z^{-1})z$.  Substituting the expressions from \eqref{zdefzi} and \eqref{starzi} into $(-\z^{-1})z$, we have
\begin{equation*}
    g(\z)=a^{r/(r+s)}(-\z)^{-s/(r+s)}+\dots + \left(\frac{\mathbb{Z}+r}{r+s} +\frac{s}{2(r+s)}\right)+\underline{o}(1).
\end{equation*}  By Proposition \ref{prop}, (1), $E_{g,r+s}$ will be isomorphic to $E_{\dot{g},r+s}$ where
$$\dot{g}(\z)=a^{r/(r+s)}(-\z)^{-s/(r+s)}+\dots + \frac{s}{2(r+s)},$$
since $g$ and $\dot{g}$ differ only by $\frac{\mathbb{Z}+r}{r+s}\in \frac{1}{r+s}\mathbb{Z}$.  The $a^{r/(r+s)}(-\z)^{-s/(r+s)}+\dots$ portion of $\dot{g}$ comes from the purely algebraic calculation as described in the remark following the statement of Theorem \ref{thm1}, so this completes the proof.
  \end{proof}

\begin{proof}[Proof of Theorem \ref{thm2}]
This proof is much the same as the proof of Theorem \ref{thm1}, so we will only sketch the pertinent details.  From \cite[Proposition 3.9]{bloch}, in our notation we have
\begin{equation*}
    \tz^2\nabla_{\tz}=\hat{z} \text{ and } \tz^{-1}=-\hat{\nabla}_{\hat{z}}
\end{equation*}
We wish to write $\zeta^{-1}=z$ in terms of $\hat{z}^{r-s}$. Letting $A=\tz f(\tz)$, $B=\tz^2\frac{d}{d\tz}$ and $\hat{z}=A+B$, we have $[B,A]=\zeta^2A'$ and Lemma \ref{abratlemma} gives
$$\hat{z}^{r/(r-s)}=a^{r/(r-s)}\left(\tz+\dots +a^{-1}\left[\frac{r}{r-s}\left(\frac{\mathbb{Z}}{r}\right)+\frac{s}{2(r-s)}\right]\tz^{1+(s/r)}+ \underline{o}(\tz^{1+(s/r)})\right)$$
and
$$\hat{z}^{(r+s)/(r-s)}=a^{(r+s)/(r-s)}\tz^{1+(s/r)}+ \underline{o}(\tz^{1+(s/r)}).$$
We conclude that
$$\tz=a^{-r/(r-s)}\hat{z}^{r/(r-s)}+\dots+a^{-2r/(r-s)}\left[\frac{-\mathbb{Z}}{r-s}+ \frac{-s}{2(r-s)}\right]\hat{z}^{(r+s)/(r-s)}+ \underline{o}(\hat{z}^{(r+s)/(r-s)}).$$
Inverting the operator, we find $$z=a^{r/(r-s)}\hat{z}^{-r/(r-s)}+\dots+\left(\frac{\mathbb{Z}}{r-s}+ \frac{s}{2(r-s)}\right)\hat{z}^{-1}+ \underline{o}(\hat{z})$$
and it follows that $$g(\hat{z})=-\hat{z} z=-a^{r/(r-s)}\hat{z}^{-s/(r-s)}+\dots+\frac{-\mathbb{Z}}{r-s}+\frac{-s}{2(r-s)}+\underline{o}(1).$$
As in the proof of Theorem \ref{thm1}, we use Proposition \ref{prop} \eqref{prop1} to find an object isomorphic to $E_g$ which matches the object given in the theorem, completing the proof of Theorem \ref{thm2}.
\end{proof}

\begin{proof}[Proof of Theorem \ref{thm3}]
The calculations are virtually identical to the proof of Theorem \ref{thm2}, but the expressions are written in terms of $\z$ instead of $\hat{z}$ and $s-r$ instead of $r-s$.  Starting with \cite[Proposition 3.12]{bloch}, in our notation we have
\begin{equation*}
    \tz^2\nabla_{\tz}=\hat{z} \text{ and } \tz^{-1}=-\z^2\hat{\nabla}_{\z}.
\end{equation*}
Repeating the calculations of Theorem \ref{thm2} we conclude that $$g(\z)=-\z^{-1}z = -a^{\frac{-r}{s-r}}\z^{-s/(s-r)} + \dots+\frac{\mathbb{Z}}{s-r} + \frac{s}{2(s-r)} + \underline{o}(1).$$
As before, by considering an appropriate isomorphic object we eliminate the term with $\mathbb{Z}$, thus completing the proof of Theorem \ref{thm3}.
\end{proof}

\section{Comparison with previous results}

One notes that in \cite{fang}, Fang's Theorems 1, 2, and 3 look slightly different from those given in (respectively) our Theorems \ref{thm1}, \ref{thm2}, and \ref{thm3}.  We shall present a brief synopsis of how to see the equivalence of Fang's Theorem 1 and our Theorem \ref{thm1}.  One difference in our methods is that Fang's calculations are split into a regular and irregular part, whereas we calculate both parts simultaneously.  We first verify the equivalence for the irregular part.

Suppose $f$ in Theorem \ref{thm1} has zero regular part, in other words $f$ has no scalar term.  Then with Fang's notation on the left and our notation on the right, we have the following:
\[ t \text{  corresponds to  } z\]
\[ t' \text{  corresponds to  } \hat{z}\]
\[ t\partial_t(\alpha) \text{  corresponds to  } f\]
\[ (1/t')\partial_{(1/t')}(\beta)+\frac{s}{2(r+s)} \text{  corresponds to  } g\]

Using the correspondences above and equation (2.1) from Fang's paper, one can manipulate the systems of equations to see that the theorems coincide on the irregular part.

 To verify that the regular portion of our calculation matches up with the results from \cite{fang}, it suffices to prove the claim below.  We note that one can calculate the regular part by using the global Fourier transform and meromorphic Katz extension; our proof is independent of that method.

\begin{claim}\label{claim}
	Given $f(z)=az^{-s/r}+\dots+b+\underline{o}(1)$ as in Theorem \ref{thm1}, if $\mathcal{F}^{(0, \infty)}E_f = E_g$ then $g$ will have constant term $\d{\left(\frac{r}{r+s}\right)b+\frac{s}{2(r+s)}}$.
\end{claim}
Before we prove Claim 1, we first prove two lemmas regarding general facts about formal Laurent series and their compositional inverses.

\begin{lemma}\label{Laurinv}
	Every formal Laurent series $j(z)=az^{p/q}+\dots$ with $a, p \neq0$ has an expression for a compositional inverse $j^{\langle -1 \rangle}(z)$.
\end{lemma}

\begin{proof}
	Let  $h(z) = (z^{1/p}\circ j\circ z^q)(z)$.  Then $h(z)$ is a formal power series with no constant term and a nonzero coefficient for the $z$ term.  Such a power series will have a compositional inverse, call it $h^{\langle -1 \rangle}(z)$.  Then $j^{\langle -1 \rangle}(z)=(z^q\circ h^{\langle -1 \rangle}\circ z^{1/p})(z)$ will be a compositional inverse for $j$.
\end{proof}

\textbf{Remarks:}  If $p$ is negative then $j^{\langle -1 \rangle}$ is not a Laurent series unless it is written in terms of the variable $z^{-1}$.  Also note that $h$ (and $h^{\langle -1 \rangle}$ as well) is not unique  since a choice of root of unity is made.  This will not affect our result, though, since $h^p$ and $(h^{\langle -1 \rangle})^q$ will be unique.

\begin{lemma}\label{fgcoeffslemma}
	For $\d{j(z)=az^{-(r+s)/r}+\dots+bz^{-1}+\underline{o}(z^{-1})}$, $j(z)\in K_r$, with $s$ a nonnegative integer and $r\in\mathbb{Z}^+$, we have $$j^{\langle -1 \rangle}(z)=a^{-1}z^{-r/(r+s)}+\dots+\frac{br}{r+s}z^{-1}+\dots$$
\end{lemma}

\begin{proof}
	Given the construction of $j^{\langle -1 \rangle}$ as described in the proof of Lemma \ref{Laurinv}, the only part of the proof that is not straightforward is the calculation of the coefficient for the $z^{-1}$ term of $j^{\langle -1 \rangle}$.  Let $h(z) = (z^{-1/(r+s)}\circ j\circ z^r)(z)$.  Then from the proof of Lemma \ref{Laurinv} we have
\begin{equation}\label{inverseeqs}
	j(z^r)=h^{-(r+s)} \text{ and } j^{\langle -1 \rangle}(z^{-(r+s)})=(h^{\langle -1 \rangle})^r.
\end{equation}
According to the Lagrange inversion formula, the coefficients of $h$ and $h^{\langle -1 \rangle}$ are related by
\begin{equation}\label{lif}
	(r+s)[z^{r+s}](h^{\langle -1 \rangle})^r=r[z^{-r}]h^{-(r+s)}
\end{equation}
where $[z^{r+s}](h^{\langle -1 \rangle})^r$ denotes the coefficient of the $z^{r+s}$ term in the expansion of $(h^{\langle -1 \rangle})^r$.  Substituting \eqref{inverseeqs} into \eqref{lif} we conclude that
\begin{equation}\label{lifconc}
	[z^{r+s}]j^{\langle -1 \rangle}(z^{-(r+s)})=\frac{r}{r+s}[z^{-r}]j(z^r)
\end{equation}Since $[z^{-r}]j(z^r)=b$, the conclusion follows.
\end{proof}

\begin{proof}[Proof of Claim \ref{claim}]
    Given the notation used above for the Lagrange inversion formula, we can restate the claim as follows: if $[z^0]f=b$, then $[\z^0]g=\frac{br}{r+s}+\frac{s}{2(r+s)}$.\\
      Let $j(z)=-z^{-1}f$.  Then
    \begin{equation*}
    [z^{-1}]j=-[z^0]f=-b.
    \end{equation*}
      By \eqref{zisyseq1} we conclude that $\hat{z}=j(z)$, and by Lemma \ref{Laurinv} let $j^{\langle -1 \rangle}$ be the compositional inverse.  Then $j^{\langle -1 \rangle}(\hat{z})=z$.  From \eqref{zisyseq2} we have $g=-z\hat{z}+\frac{s}{2(r+s)}$, which implies that $-\hat{z}^{-1}(g-\frac{s}{2(r+s)})=j^{\langle -1 \rangle}(\hat{z})$.  This gives
      \begin{equation}\label{last}
    [\hat{z}^{-1}]j^{\langle -1 \rangle}=-[\hat{z}^0]g+\frac{s}{2(r+s)}.
    \end{equation}
    By Lemma \ref{fgcoeffslemma}, $[z^{-1}]j=-b$ implies that $[\hat{z}^{-1}]j^{\langle -1 \rangle}=\frac{-br}{r+s}$.  The result then follows from \eqref{last} and noting that $[\hat{z}^0]g=[\z^0]g$.
\end{proof}


\begin{thebibliography}{99}
\bibitem{dima} D.~Arinkin: Fourier transform and middle convolution for irregular $\mathcal{D}$-modules, \textbf{arXiv:08080699}.
\bibitem{beilinson} A.~Beilinson; S.~Bloch; H.~Esnault: $\epsilon$-factors for Gauss-Manin determinants, \textit{Moscow Mathematical Journal} 2(3):477-532, 2004
\bibitem{bloch} S.~Bloch and H.~Esnault: Local Fourier transforms and rigidity for  $\mathcal{D}$-modules, \textit{Asian Journal of Mathematics} 8(4):587-605, 2004
\bibitem{fang} J.~Fang: Calculation of local Fourier transforms for formal connections, \textbf{arXiv.0707.0090}.
\bibitem{levelt} A.~Levelt: Jordan decomposition for a class of singular differential operators, \textit{Arkiv f\"{o}r Matematik} 13(1-2): 1-27, 1975
\bibitem{sabbah} C.~Sabbah: An explicit stationary phase formula for the local formal Fourier-Laplace transform, \textbf{arXiv:0706.3570}.
\bibitem{turritin}H.L.~Turritin. Convergent solutions of ordinary linear homogeneous differential equations in the neighborhood of an irregular singular point, \emph{Acta Math.} 93:27-66, 1955.
\end{thebibliography}

\end{document}
