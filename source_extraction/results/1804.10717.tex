\documentclass[11pt]{article}
%\documentclass[11pt]{lms}

\usepackage{fullpage}
%\usepackage{framed}


%\usepackage{amsmath, amsthm, amssymb, bbm}
\usepackage{amsmath, amsthm, amssymb}

%\textwidth160mm
%\textheight225mm
%\oddsidemargin10mm
%\evensidemargin10mm
%\def\baselinestretch{1.2}
%\topmargin20mm
%\headsep0mm
%\headheight0mm
%\topskip0mm

% make proof environment use bold font, including alternative header
\makeatletter
\renewenvironment{proof}[1][\proofname]
{\par\pushQED{\qed}
	\normalfont\topsep6\p@\@plus6\p@\relax\trivlist
	\item[\hskip\labelsep\bfseries#1\@addpunct{.}]
	\ignorespaces}
{\popQED\endtrivlist\@endpefalse}
\makeatother

%%EXTRACTING
 \usepackage{./extract}
%ENDEXTRACTING
\newtheorem{theo}{Theorem}[section]
\newtheorem{theo}{Theorem}[section]
\newtheorem{prop}[theo]{Proposition}
\newtheorem{lemma}[theo]{Lemma}
\newtheorem{coro}[theo]{Corollary}
\newtheorem{conj}[theo]{Conjecture}
\newtheorem{claim}[theo]{Claim}
\newtheorem{remark}[theo]{Remark}
%\newtheorem{remark}{Remark}[subsection]
\newtheorem*{remark*}{Remark}
\newtheorem{fact}[theo]{Fact}
%\usepackage{thmtools}
%\usepackage{thm-restate}

\newtheorem{definition}{Definition}[section]

%\usepackage{cleveref}

%\declaretheorem[name=Theorem,numberwithin=section]{thm}

\newcommand{\CC}{{\cal C}}
\newcommand{\TT}{{\bf T}}
\newcommand{\FF}{{\cal F}}
\newcommand{\GG}{{\cal G}}
\newcommand{\PP}{{\cal P}}
\renewcommand{\SS}{{\cal S}}
\newcommand{\UU}{{\cal U}}
\newcommand{\R}{{\mathbb R}}

\newcommand{\floor}[1]{\left\lfloor{#1}\right\rfloor}
\newcommand{\ceil}[1]{\left\lceil #1 \right\rceil}

%\renewcommand{\wp}{\mathrm{ind}}
\renewcommand{\wp}{\mathrm{span}}
\renewcommand{\a}{\alpha}
\renewcommand{\d}{\delta}
\newcommand{\g}{\gamma}
\newcommand{\D}{\Delta}
 
\newcommand{\sub}{\subseteq}
\newcommand{\Ex}{\mathbb{E}}

\newcommand{\N}{\mathbb{N}}
\newcommand{\HH}{{\cal H}}

\newcommand{\C}{\mu}

%\newcommand{\trace}{Tr}
\DeclareMathOperator{\trace}{Tr}
%\DeclareMathOperator{\Exp}{\mathbb{E}}

\DeclareMathOperator{\poly}{poly}

\date{}

\title{Traces of Hypergraphs}
\author{
Noga Alon\thanks{Department of Mathematics, Princeton University, 
Princeton, New Jersey, USA and
Center for Mathematical Sciences and Applications, 
Harvard University, Cambridge, Massachusetts, USA.
Email: \texttt{nalon@math.princeton.edu}.}
\and
Guy Moshkovitz\thanks{Center for Mathematical Sciences and Applications, Harvard University, Cambridge, Massachusetts, USA. Email: \texttt{guymoshkov@gmail.com}.}
\and
Noam Solomon\thanks{Center for Mathematical Sciences and Applications, Harvard University, Cambridge, Massachusetts, USA. Email: \texttt{noam.solom@gmail.com}.}
}
\begin{document}
\maketitle

%\input{traces-intro}

\begin {abstract} 
Let $\trace(n,m,k)$ denote the largest number of distinct projections onto $k$ coordinates guaranteed in any family of $m$ binary vectors of length $n$.
%Let $\trace(n,m,k)$ denote the largest number $t$ such that for any family of $m$ binary vectors of length $n$ there is a set of $k$ coordinates so that the number of distinct projections of the vectors in the family on these coordinates is at least $t$. 
%Fix $r \ge 1$. 
The classical Sauer-Perles-Shelah Lemma implies that $\trace(n, n^r, k) = 2^k$ for $k \le r$.
While determining $\trace(n,m,k)$ precisely for general $k$ and $m$ seems
hopeless, estimating it remains a widely open problem with connections to 
important questions in computer science and combinatorics.
For example, an influential result of Kahn-Kalai-Linial gives non-trivial bounds on	 $\trace(n, m, k)$ for $k=\Theta(n)$ and $m = \Theta(2^n)$.
%exponential $m$.
Here we prove that, for $r,\a^{-1} \le n^{o(1)}$, it holds that
$\trace(n,n^r,\a n) = n^{\C(1+o(1))}$ with $$\C=\frac{r+1-\log(1+\a)}{2-\log(1+\a)}.$$
Thus, we (essentially) determine $\trace(n,m,k)$ for $k=\Theta(n)$ and all $m$ up to $2^{n^{o(1)}}$.

%Let $\trace(n,m,k)$ denote the largest number of distinct projections onto $k$ coordinates guaranteed in any family of $m$ binary vectors of length $n$.
%%Let $\trace(n,m,k)$ denote the largest number $t$ such that for any family of $m$ binary vectors of length $n$ there is a set of $k$ coordinates so that the number of distinct projections of the vectors in the family on these coordinates is at least $t$. 
%%Fix $r \ge 1$. 
%The classical Sauer-Perles-Shelah Lemma implies that $\trace(n, n^r, k) = 2^k$ for $k \le r$.
%%$k=\floor{r}$.
%While determining $\trace(n,n^r,k)$ precisely for general $k$ seems
%hopeless even for constant $r$, estimating it, and more generally estimating 
%the function $\trace(n,m,k)$ for all range of the parameters,
%remains a widely open problem with connections to 
%important questions in computer science and combinatorics.
%Here we essentially resolve this problem when $k$ is linear and $m=n^r$ where
%$r$ is constant,
%proving that, for any constant $\a>0$, 
%$\trace(n,n^r,\a n) = \tilde\Theta(n^\C)$ 
%with $\C=\C(r,\a)=\frac{r+1-\log(1+\a)}{2-\log(1+\a)}$. 

For the proof we establish a ``sparse'' version of another 
classical result, the Kruskal-Katona Theorem, 
which gives a stronger guarantee when the hypergraph does not 
induce dense sub-hypergraphs.
Furthermore, we prove that the parameters in our sparse Kruskal-Katona 
theorem are essentially best possible. 
Finally, we mention two simple applications which may be 
of independent interest. 
\end {abstract}

\section{Introduction}\label{se:intro}

%TODO: define $k$-grpah

For a hypergraph (or a set system) $\FF$, the \emph{trace} of $\FF$ on a vertex subset $I$ is defined as the set of projections of the edges of $\FF$ onto $I$, namely,
$\FF_I = \{e \cap I \,\colon\, e \in \FF \}$.
The \emph{shatter function}, or \emph{trace function}, of $\FF$ is 
$\trace(\FF,k) =\max_{I} \big|\FF_I\big|$ with $I$ a set $k$ vertices.
The focus of this paper is the following important extremal function; for integers $n \ge k$ and $0 \leq m \leq 2^n$, let $\trace(n,m,k)$ denote 
the largest number of distinct projections onto $k$ vertices guaranteed in any $n$-vertex $m$-edge hypergraph:
%the largest $t$ such that for every hypergraph on $n$ vertices with $m$ hyperedges there is a set of $k$ vertices on which the hypergraph has at least $t$ distinct projections. % on these coordinates. 
%That is,
$$\trace(n,m,k) = \min_{\substack{\FF \sub 2^{[n]}\\|\FF|=m}} \trace(\FF,k) = \min_{\substack{\FF \sub 2^{[n]}\\|\FF|=m}} \, \max_{\substack{I \sub [n]\\|I|=k}} \big|\FF_I\big| .$$ 

There is a considerable number of results, in various areas of discrete mathematics, 
%, in both computer science and combinatorics, 
determining or estimating this function for certain values
of the parameters. The most famous result is arguably the Sauer-Perles-Shelah Lemma (\cite{Sa}, \cite{Sh}, see also
Vapnik and Chervonenkis~\cite{VC} for a slightly weaker estimate).
% can be expressed in terms of $\trace()$ as follows.
\begin {theo}[Sauer-Perles-Shelah]
\label{th:sps}
%We have $
%$\trace\big(n,\,\sum_{i=0}^{k-1} {n \choose i}+1,\, k\big)=2^k$.
$\trace(n,m,k)=2^k$ for $m > \sum_{i=0}^{k-1} \binom{n}{i}$.
\end {theo}
The \emph{VC-dimension} of a hypergraph $\FF$ is the largest $k$ so that $\trace(\FF, k) = 2^k$, i.e., it is the largest number $k$ so that $\FF$ has a full projection on some $k$ vertices. The VC-dimension is a basic combinatorial measure of the \emph{complexity} of a hypergraph; understanding the shatter function beyond the case of full projections is a very natural direction. Shatter functions and the VC-dimension are extensively studied in combinatorial and computational geometry, as well as in machine learning (see the survey of Matou\v sek~\cite{Mat} for several geometric and algorithmic applications of shatter functions, and the survey of Angulin~\cite{An} for the role VC-dimension is playing in computational learning theory).

In~\cite{Bo}, Bondy proved that $\trace(n,n,n-1)=n$, and a
remark in \cite{Al} and in \cite{Fr} is that
$$
\trace(n,m,3)=7\, \text{ for } \,m=1+n+[n^2/4]+1,
%\trace\big(n,\,1+n+[n^2/4]+1,\,3\big)=7,
$$
and the same argument implies that
determining the smallest $m$ for which $\trace(n,m,4)=15$ is equivalent
to determining the maximum possible number of edges of a
$3$-uniform hypergraph  on $n$ vertices with no complete hypergraph
on $4$ vertices---a well-known open problem of Tur\'an. Additional results
that can all be formulated in terms of the function $\trace(n,m,k)$ 
appear in~\cite{BR}, \cite{KKL}, \cite{BKK}, \cite{CGN} and more.

Recently, Bukh and Goaoc 
were able to estimate $\trace(n,n^{O(1)},k)$ for constant values of $k$ that are not too small (and also improved an earlier lower bound of \cite{CGN}).
%studied how the growth rate of the shatter function can be controlled by fixing one of its values~\cite{BG}. When $|\FF|$ is polynomial in $n$, they estimated the shatter function $\trace(\FF,k)$ for constant values of $k$ that are not too small (and also improved an earlier lower bound of \cite{CGN}).
%
%In the other extreme regime, Kahn et al.~\cite{KKL} proved that for every $0<\a<1$
In the other extreme regime, a classical paper of Kahn et al.~\cite{KKL} proves that for every $0<\a, \beta<1$,
$$\trace(n, \beta 2^{n}, \a n) \ge (1-n^{-c})2^{\a n},$$
where $c=c(\a,\beta)>0$ depends only on $\a,\beta$.
Benny Chor conjectured in the 80s that one can in fact make the error term exponentially rather than polynomially small in $n$.
%Benny Chor conjectured in the 80s that one can in fact achieve
%$$\trace(n,2^{n-1}, \a n) \ge (1-c(\a)^n)2^{\a n},$$
%where again, $0<c(\a)<1$ depends on $\a$.
This conjecture was recently disproved by Bourgain et al.~\cite{BKK}. In fact, Bourgain et al.\ prove several additional
results, in particular strengthening those of~\cite{KKL}.
%Moreover, Bourgain et al.\ improved the above result of~\cite{KKL} for values of $\a$ that are sufficiently close to $1/2$.\footnote{By Theorem~\ref{th:sps}, $\trace(n,2^{n-1}, k)=2^{k}$ for $k \le n/2$, so when $|\FF|=2^{n-1}$, values of $\alpha$ below $1/2$ are trivial.}
%\footnote{By Theorem~\ref{th:sps}, $\trace(n,2^{n-1}, \frac n 2)=2^{n/2}$, so when $|\FF|=2^{n-1}$, values of $\alpha$ below $1/2$ are trivial.}
%Bourgain et al.~\cite{BKK} disproved this conjecture by proving that for any fixed $0<\a, \d <1$, 
%$$\trace(n, \alpha 2^n, (1/2+\d)n)\le (1-n^{-C(\a,\d)})2^{(1/2+\d)n}.$$
%
%Kahn et al.~\cite{KKL} also proved that there are fixed positive $\d <1/2$ and $c(\d) > 0$ such that
%$$\trace(n, 2^n/n^c, (1/2 +\d) n)\ge 0.9 \cdot 2^{(1/2+\d) n}.$$
%
%Bourgain et al.~\cite{BKK} improved it, by showing that for every $c>0$,
%$$\trace(n, 2^n/n^c, (1/2+\d)n) \ge 0.9\cdot  2^{(1/2+\d)n},
%$$ 
%for a suitable $\d=\d(c)>0$.
%By Theorem~\ref{th:sps}, $\trace(n,2^{n-1}, \frac n 2)=2^{n/2}$, so one cannot expect to go much below $(1/2+\delta)n$. 

%They also disproved the above conjecture of Chor and another conjecture of Kalai, by proving certain (non-matching) upper %bounds.\footnote{They proved that for any fixed $0<\alpha, \delta <1$, $\trace(n, \alpha 2^n, (1/2+\delta)n)\le (1-n^{-%C(\alpha,\delta)})2^{(1/2+\delta)n}$, and for any fixed $\epsilon, \delta>0$, $\trace(n, \left((1-\epsilon) 2\right)^n, %(1/2+\delta)n)\le \exp(-n^{C(\epsilon,\delta)})2^{(1/2+\delta)n}$.}

In~\cite{BR}, Bollob\'as and Radcliffe considered the case where $m$ is polynomial and $k$ is linear. For the lower bound they were able to prove the following.
\begin{theo} [Bollob\'as and Radcliffe{~\cite[Theorem 7]{BR}}]
	\label{th:brlb}
	For constants $r\ge 2$ and $0 < \a \le 1$ it holds that $\trace(n,\,n^r,\,\a n) \ge \Omega(n^{\lambda r})$
	with\footnote{Here $H(x) = -x\log(x) - (1-x)\log(1-x)$ is the binary entropy function, and the logarithms are in base $2$.}
	%\begin{align*} 
	$$\lambda =
	\begin{cases}
	\log(1+\alpha) & \alpha \in [\sqrt 2 -1, 1]\\
	\log(1+\alpha)/H(\log(1+\alpha)) & \alpha \in (0, \sqrt 2 -1)
	\end{cases}$$
	%\end{align*}
\end{theo}

As for the upper bound, it would seem that among hypergraphs on $n$ vertices with a given number of edges $m$, a hypergraph $\FF$ with $\trace(\FF,k) = \trace(n,m,k)$ should be very symmetric, when $k$ is not too small or too large. A natural candidate for such an extremal hypergraph is thus the hypergraph containing all edges up to the appropriate size.
% Hamming ball of appropriate radius.
%around $0$. 
Bollob\'as and Radcliffe were able 
to show that this is in fact not the case. 
%Concretely, they proved
\begin {theo} [Bollob\'as-Radcliffe{~\cite[Theorem 11]{BR}}]
\label{th:brup}
For every constant integer $r \ge 2$, 
$$\trace\bigg(n,\, \sum_{i=0}^r \binom n i,\, n/2\bigg) \le o(n^r) .$$
\end {theo}

%Fix an integer $r \ge 2$ and $0  < \epsilon < \frac 1 2$. There is $n_0 =n_0(r,\epsilon)$ such that for all $n\ge  n_0$, 
%there exists a hypergraph $\FF$ with $|\FF|=\sum_{i=0}^r \binom n i$ and
%$\trace(\FF, n/2) \le  \sum_{i=0}^r \binom {n/2} i - (1-\epsilon)2^{-r} \binom n r$.
%$$\trace\bigg(n,\, \sum_{i=0}^r \binom n i,\, n/2\bigg) \le \sum_{i=0}^r \binom {n/2} i - (1-\epsilon)2^{-r} \binom n r = o(n^r) .$$


%\paragraph{Our results.}

\subsection{Our results}
Our main result in this paper determines the value of $\trace(n,n^r,\a n)$, for constant $r$ and $\a$, up to logarithmic
factors, thus closing the gap between the lower and upper bounds in Theorems~\ref{th:brlb} and~\ref{th:brup}. 
We henceforth use the following standard notation: for two functions $f(n)$ and $g(n)$, by $f=\tilde{O}(g)$ we mean
$f = O(g\log^c(g))$ for some absolute constant $c>0$;
%$f(n) = O(g(n) \log^c(g(n)))$ for some absolute constant $c>0$.
%$f(n) \le g(n)\cdot c\log^{c} n$ for some absolute constant $c>0$; 
$f=\tilde{\Omega}(g)$ and
$f=\tilde{\Theta}(g)$ are defined analogously. 
The main result of this paper is as follows.
\begin{theo}[Main result]\label{t12}
	Let $r \ge 1$, $\a \in (0,1]$. If $r, \a^{-1} \le n^{o(1)}$ then %it holds that
	$\trace(n,\,n^r,\,\a n) = n^{\C(1-o(1))}$
	%For constants $r>1$ and $0 < \a \le 1$
	%we have $\trace(n,\,n^r,\,\a n) = \tilde{\Theta}(n^{\C})$,
	where
	\begin{equation}\label{eq:C}
	\C=\C(r,\a)=\frac{r+1-\log(1+\a)}{2-\log(1+\a)}.
	\end{equation}
	Moreover, if $r=O(1)$, $\a^{-1} \le (\log n)^{O(1)}$ then $\trace(n,\,n^r,\,\a n) = \tilde{\Theta}(n^{\C})$.
\end{theo}

It is perhaps instructive to consider one representative special 
case: $r=2$, $\a = 1/2$. 
In this case, 
the proofs in~\cite{BR} 
%Theorems~\ref{th:brlb} and~\ref{th:brup} %(and its proof)
bound $\trace(n,n^2,n/2)$ as follows;
$$
\Omega(n^{1.169925..}) =
\Omega(n^{2 \log_2 {3/2}}) \leq \trace(n,n^2,n/2) 
\le \frac{n^2 (\log \log n)^{O(1)}}{\log n} = o(n^2),
$$
whereas Theorem~\ref{t12} in particular implies that
$$\trace(n,n^2,n/2)=\tilde{\Theta}(n^{1+1/(3-\log_2 3)})
=\tilde{\Theta}(n^{1.706695..}).
$$
%In fact, for fixed $r$ it is easy to see that the gap between Theorems~\ref{th:brlb} and~\ref{th:brup} grows as $\a$ decreases.

\paragraph{A new Kruskal-Katona-type theorem.}

As it turns out, our main result can be readily deduced from a new version of the well-known Kruskal-Katona Theorem.
Recall that the Kruskal-Katona Theorem gives a lower bound on the number of $i$-sets contained within the edges of a uniform hypergraph. Formally, for a hypergraph $\FF$ and $i \in \N$ we denote 
$$\binom{\FF}{i} = \big\{S \,\big\vert\, |S|=i \text{ and } \exists e \in \FF \colon S \sub e \big\}.$$
The following classical version of the Kruskal-Katona Theorem 
was given by Lov\'asz~\cite{Lovasz}.
Henceforth, for real $y > 0$ we use the standard notation 
$$\binom{y}{i}=\frac{y(y-1)\cdots(y-i+1)}{i!} .$$ 
We use the abbreviation that $\FF$ is a \emph{$k$-graph} 
to mean that $\FF$ is a $k$-uniform hypergraph.
%We denote by $|\FF|$ the number of edges of $\FF$.

%\begin{restatable}
\begin{theo}[Kruskal-Katona Theorem, Lov\'asz~\cite{Lovasz}]%{thm}{kk}
	\label{th:kk}
	Let $\FF$ be a $k$-graph. 
	If $|\FF| = \binom{y}{k}$ with real $y > 0$
	then for every $0 \le i \le k$ we have $\big|\binom{\FF}{i}
\big| \ge \binom{y}{i}$.
\end{theo}

Our new version of the Kruskal-Katona Theorem gives a stronger lower bound depending on the sparsity %\footnote{The reciprocal notion to density.} 
of the hypergraph $\FF$.
As is standard, we denote the sub-hypergraph of a hypergraph $\FF$ induced on a vertex subset $I$ by
$\FF[I] =(\,I, \,\,\{e\,\vert\, e \in \FF \text{ and } e\sub I\}\,)$.
We denote the largest number of edges in an induced sub-hypergraph on $i$ vertices by
$$\wp(\FF,i) := \max_{\substack{I \sub [n]\\|I|=i}} \big|\FF[I]\big|.$$
%TODO
%\footnote{It is worth noting that $\trace()$ is a special case of $\wp()$, in the sense that if $\FF^{\downarrow}$ is the down-closed hypergrap generated by $\FF$ then $\wp(\FF^{\downarrow},k)=\trace(\FF,k)$.}$$ 
%%%}$$
%\footnote{Notice that $\wp(\FF,i) =\mathrm{dens}(\FF,i) \cdot  \binom i k $, with $\mathrm{dens}(\FF,i) =\underset {\substack{I \sub [n]\\|I|=i}} {\max} \mathrm{dens}(\FF,I)$ is the maximal density over all induced sub-hypergraphs on sets $I \subset [n]$ of $i$ elements.}.$$
%
%Theorem for ``sparse'' hypergraphs. Here by sparse we roughly mean that there is no dense induced sub-hypergraph. 
%More formally, TODO...
%Henceforth, for real $x>0$ we denote $\binom{x}{x/2}=\max_{1 \le i \le x} \binom{x}{i}$ (alternatively, we could also use the extended definition of binomial coefficients over real numbers).
%We are now ready to state 
We next state our new version of the Kruskal-Katona Theorem. 
(See Theorem~\ref{th:skk} for a slightly stronger form.)
For the parameters relevant to our applications here, it provides
a significantly stronger estimate than the classical theorem,
using an appropriate sparseness assumption.
%, and Remark~\ref{re:weak} for deriving this form from it.

%We also denote $\FF[I]$ to denote the induced sub-hypergraph on the vertices of $I$, that is, 
%$$\FF(I)=(\,V(F),\,\{e \in E(\FF) \,\vert\, I \sub e\}\,).$$
%
%the \emph{link} of $I$ is the subgraph $L(I)=(V(\FF) \setminus L,\, \{ e \setminus I \,\vert\, I \sub e \in \FF \})$, 
%while the \emph{star} at $I$ is the subgraph $S(I)=(V(\FF),\, \{ e \,\vert\, I \sub e \in \FF\})$.
%\footnote{This motion is different from the \emph{link} of $S$ in that we do not remove the vertices in $I$.}
%

%{restatable}

%We are now ready to state the following theorem, which is a ``sparse version'' of the Kruskal-Katona Theorem.
%, and applies to uniform hypergraphs.
%\begin{restatable}[``Sparse Kruskal-Katona Theorem'']{thm}{skk}
\begin{theo}[``Sparse Kruskal-Katona Theorem'']\label{th:skk-intro}
	Let $\FF$ be a $k$-graph with $n$ vertices and $|\FF|=n^r$ edges, $r \ge 1$. 
	If 
	$$\wp(\FF,\,\a n) \le \min\bigg\{\binom{x}{k-\ceil{r}}n, \,\, \frac{1}{2}|\FF|\bigg\}$$
	with real $x \ge 2k$ then for every $r+1 \le i \le k$ we have 
	$$\bigg|\binom{\FF}{i}\bigg| \ge \frac{1}{C}\cdot \frac{\binom{x}{i}}{\binom{x}{k}}|\FF|,$$
	with $C = (8k/\a)^{\ceil{5r}}\log n$.
	%(16k/\a)^{5r+4} \log n$.
\end{theo}
%\end{restatable}
%Note that for $|\FF|=\binom{x}{k}$, Theorem~\ref{th:skk-intro} recovers Theorem~\ref{th:kk}, up to the error term $C$. 

%Note that if the condition in Theorem~\ref{th:skk-intro} is satisfies then the guarantee is at least as strong 

It is of course natural to ask whether the bound in Theorem~\ref{th:skk-intro} is essentially best possible.
Our third result in this paper proves that this is indeed the case.
%We moreover prove that the bound in Theorem~\ref{th:skk-intro} is essentially best possible.

\begin{theo}[Upper bound for Sparse Kruskal-Katona]\label{theo:sKK-UB-intro}
	Let $n,k,x \in \N^+$, $r \ge 1$ and $0 < \a \le 1$ with 
	$3r \le k \le x \le n^{1/6}$
	%$k \ge 3r$, $x \le n^{1/6}$, 
	and $n \le \a^k n^r \le \binom{x}{k}n$.
	%$\a \ge n^{-(r-1)/k}$, 
	%and $\binom{x}{k}n \ge \a^k n^r$.
	There exists a $k$-graph $\FF$ with $n$ vertices, $|\FF|=n^r$ edges, 
	and 
	$\wp(\FF,\,\a n) \le O(\binom{x}{k}n)$
	such that for every $0 \le i \le k$ we have 
	$\big|\binom{\FF}{i}\big| \le \frac{\binom{x}{i}}{\binom{x}{k}}|\FF|$.
\end{theo} 


\paragraph{Applications.}
We end the paper with two simple 
applications of our main result, in geometry and in graph theory. 
We first describe the geometric application.
%which does not appear to be previously known.
%The fourth and final result in this paper is an application of our main result, which we now describe.
Let $\HH$ be a family of halfspaces\footnote{A halfspace consist of all points above a hyperplane.}  in $\R^d$, and let $P$ be a set of points in $\R^d$. We say that $P$ \emph{separates} $\HH$ if for every pair of distinct halfspaces $H_1 \neq H_2 \in \HH$ there is a point in $P$
that lies in one and outside the other.
% distinguishing them, meaning it lies in one of them and outside the other. 

\begin{prop}\label{prop:application-intro}
	Let $P \subset \R^d$ be a set of $n$ points and let $\HH$ be a family of $n^r$ halfspaces in $\R^d$, with $1 \le r \le n^\d$, such that $P$ separates $\HH$.  
	Then there exists a subset $P' \subseteq P$ of at most $n^{1-\d}$ points and
	a subset $\HH' \subseteq \HH$ of at least $n^{\frac{r+1}{2}(1-O(\d))}$ halfspaces such that $P'$ separates $\HH'$.
%	
%	Fix $r > 1$, let $\HH$ be a family of $n^r$ halfspaces in $\R^d$ and let $P \subset \R^d$ be a set of $n$ points such that $P$ separates $\HH$. 
%	Then there is a subset $P' \subseteq P$ of at most $\frac n {\log n}$ points and
%	a subset $\HH' \subseteq \HH$ of $\tilde{\Omega}(n^{\frac {r+1} 2})$ halfspaces such that $P'$ separates $\HH'$.
\end{prop}

The graph-theoretic application shows that in any graph with not too many independent sets, one can always find an induced subgraph on a vanishingly small number of vertices that nevertheless retains significantly more than square root of the total number of independent sets.

\begin{prop}%\label{prop:app-graphs}
	Let $G = (V,E)$ be an $n$-vertex graph, and assume that the number of independent sets in $G$ is $n^r$ with $1 \le r \le n^\d$.
	Then there exists a subset $V' \subseteq V$ of at most $n^{1-\d}$ vertices such that the number of independent sets in the induced subgraph $G[V']$ is at least $n^{\frac{r+1}{2}(1-O(\d))}$.
	%$\tilde{\Omega}(n^{\frac {r+1} 2})$.
\end{prop}

\paragraph{Organization.}
In Section~\ref{sec:sKK-LB} we prove Theorem~\ref{th:skk-intro} using 
an appropriate 
hypergraph decomposition method. 
We use it in Subsection~\ref{sec:traces} to deduce the lower bound in Theorem~\ref{t12}. 
In Subsection~\ref{sec:sKK-UB} we prove that the parameters of Theorem~\ref{th:skk-intro} are essentially best possible, and  in Subsection~\ref{subsec:UB-traces} we prove a matching upper 
bound for Theorem~\ref{t12}, 
%\ref{theo:sKK-UB-intro}, 
%we show that the parameters in Theorem~\ref{th:skk} are essentially best possible, 
using a probabilistic construction.
% whose trace is very close to the expected size of a projection. 
%%%We then use this to deduce the upper bound in Theorem~\ref{th:main}.
%The proof of Theorem~\ref{t12} consists of two parts. In Section~\ref{se:upb}, we prove the first part of Theorem~\ref{t12}, which is the upper bound, stated as Theorem~\ref{th:ubt}. We give a probabilistic construction for the upper bound, and the fact that the maximal trace is very close to the expected trace is based on standard probabilistic arguments. 
Our applications, Proposition~\ref{pr:tr} and Proposition~\ref{prop:app-graphs}, are given in Section~\ref{sec:applications}.
%we give  a geometric application of Theorem~\ref{th:main}.

%The second part of Theorem~\ref{t12}, stated as Theorem~\ref{th:main}, and proved in Section~\ref{se:lob}, is based on a new hypergraph decomposition method. This method applies when the hypergraph is sparse enough, that is, when $\trace(\FF,\alpha n)$ is small enough. 
%In fact, when the hypergraph is sparse, we prove a new, stronger version, of the Kruskal-Katona theorem for uniform hypergraphs~\cite{Ka},~\cite{Kr},~\cite{Lovasz}, which lower bounds the number of $i$-sets contained in the hypergraph, for any integer $i$, as a function of the number of hyperedges and the sparsity of the hypergraph. The sparsity of the hypergraph is a measure of how well-spread the edges of the graphs are among the vertices. To state  it, we use the following notation: here $i \in \N$ is a nonnegative integer and $\FF$ is a hypergraph.

\paragraph{Proofs overview.}
For the proof of the sparse Kruskal-Katona Theorem (Theorem~\ref{th:skk-intro}, see also Theorem~\ref{th:skk} below) we proceed as follows. In the first part of the proof we apply a new approximate hypergraph decomposition method, relying on the sparseness of the input hypergraph, into  links. The decomposition is performed iteratively, in each step finding many vertices of high degree within the current link and restricting the next links to them. The final step of these iterations consists of ``cleaning'' each link by iteratively removing vertices of low degree. We then prove, using the sparseness of the hypergraph, that in fact most of the parts in our decomposition have few edges.  
In the second part of the proof we find, by applying the classical Kruskal-Katona Theorem, $i$-subsets within the edges of each (sub)link separately. We then argue that, since the links approximately decompose the hypergraph, we may essentially collect the $i$-subsets from all links without much overcounting.
The proof of our lower bound for traces (in Theorem~\ref{t12}) follows quite easily from the sparse Kruskal-Katona Theorem by applying it on the most ``popular layer'' of the hypergraph (i.e., the uniform hypergraph with the most edges contained in our hypergraph, which we may assume is down-closed) and projecting onto a random subset of $\a n$ vertices.

For the proof that the parameters in the sparse Kruskal-Katona Theorem are essentially best possible (Theorem~\ref{theo:sKK-UB} below)
we give a randomized construction of a uniform hypergraph whose edges contain few $i$-subsets. The construction is fairly simple: the union of a carefully chosen number of cliques on random subsets, such that it simultaneously holds that there are many cliques and yet they are nearly edge disjoint. We show in particular that the expected number of edges induced on subsets of $\a n$ vertices is sufficiently small so as to allow taking a union bound over all cliques. 
%crucially, sufficiently large so that one may take a union bound over all these subsets.
%so as to allow for concentration of measure. 
The proof of the upper bound for traces (in Theorem~\ref{t12}) follows by taking the down-closed hypergraph generated by the uniform hypergraph above, and then upper bounding the expected trace on a random subset of $\a n$ vertices.
\vspace{0.2cm}



%\paragraph{Preliminaries.}

%Throughout the paper we identify subsets of $[n]$ with their characteristic vectors. 
%A family of subsets is called
%monotone if it is closed under taking subsets. As proved in \cite{Al},
%\cite{Fr}, for every family $\FF$ of subsets of $[n]$ there is a
%monotone family with the same cardinality whose projection on every
%set of coordinates is at most as large as the projection of $\FF$
%on these coordinates. Thus, for the proof of Theorem~\ref{t12}, we may consider only monotone families.
%%%
%For nonnegative integers $a \ge b$ we write ${a \choose \le b} = \sum_{i=0}^{b} {a \choose i}$, and for a set $A$ we write ${A \choose b}=\{A' \sub A \colon |A'|=b\}$ and ${A \choose \le b} = \{A' \sub A \colon |A'|\le b\}$.
%\bigcup_{i=0}^b {A \choose i}$. 

\noindent
Throughout the paper we 
assume, whenever needed, that $n$ is sufficiently large. 
%We use the notation $a \sim b$ for $a = (1+o(1))b$ where $o(1)$ tends to $0$ with $n$.
All logarithms are in base $2$ unless otherwise specified. 
To simplify the presentation we omit all floor and ceiling signs whenever these are not crucial. 

%%%%%%%%%%%%%%%%%%%%%%%%%%%%%%%%%%%%%%%%%%%%%%%%%%%%%%%%%%%%%%%%%%%%%%%%%%%%%%%%%%%
%\input{traces-LB}


\section{Sparse Kruskal-Katona and Traces Lower Bound}\label{sec:sKK-LB}

In this section we prove Theorem~\ref{th:skk-intro}.
%the lower bound in Theorem~\ref{t12}. 
Henceforth, for a hypergraph $\FF$ on $V$ and for a vertex subset $I \sub V$ we denote by $\FF(I)$ the \emph{link} of $I$ in $\FF$, that is, 
$$\FF(I)=(\,V \setminus I,\,\,\{e \setminus I \,\vert\, I \sub e \in \FF \}\,).$$
Note that if $\FF$ is a $k$-graph then $\FF(I)$ is
a $(k-|I|)$-graph.
For a tuple $U$ of vertices in $V$ we denote by $|U|$ the number of 
distinct vertices in $U$, and by $\FF(U)$ the link $\FF(I)$ where 
$I$ is the set of (distinct) vertices in $U$ (and so $\FF(U)$ is a $(k-|U|)$-graph).
%We moreover denote $\deg_\FF(U) :=|\FF(U)|$.

%This proof is more complicated than that of the upper bound. 
For the proof we will need several lemmas which we state and prove below.
We begin with the following simple ``hypergraph regularization'' lemma.
\begin{lemma}
	\label{l31}
	Every hypergraph $\FF=(V,E)$ has an induced sub-hypergraph $\FF'=(V',E')$ satisfying:
	\begin{enumerate}
		\item\label{item:reg-1}
		$|E'|/|V'| \geq |E|/|V|.$
		\item\label{item:reg-2}
		The degree of each vertex $v \in V'$ in $\FF'$ is at least
		%\begin{equation}
		%\label{e31}
		$\frac{|E'|}{2|V'| \log|V|}$.
		\item\label{item:reg-3}
		$|E'| > |E|/2.$
	\end{enumerate}
\end{lemma}
%\vspace{0.1cm}
%\noindent
%{\bf Proof:}\,
\begin{proof} 
	Put $n=|V|$, $V_0=V$ and $E_0=E$. Starting with $i=0$, as long as the
	hypergraph $\FF_i=(V_i,E_i)$ in which $|V_i|=n-i$ does not satisfy~(\ref{item:reg-2}), let $V_{i+1}$ be the set obtained from $V_i$ by
	removing a vertex of minimum degree in $\FF_i$, and let $\FF_{i+1}$ be
	the induced subhypergraph on this set. 
	It is easy to see that $|E_{i+1}|/|V_{i+1}| > |E_i|/|V_i|$ and hence this process must terminate with a nonempty hypergraph $\FF_j=(V_j,E_j)$. Define $V'=V_j$, $E'=E_j$. Then~(\ref{item:reg-1}) holds as the quantity $|E_i|/|V_i|$ keeps increasing during the process, 
	(\ref{item:reg-2}) holds by the definition of $j$, and (\ref{item:reg-3}) holds since 
	\begin{align*}
	|E'| &= |E_j| \ge |E_0| \prod_{i=0}^{j-1} \Big(1-\frac{1}{2(n-i) \log n}\Big) \\
	&\geq |E_0| \Big(1-\sum_{i=0}^{j-1} \frac{1}{2(n-i) \log n}\Big)
	\geq |E|\Big(1-\frac{\ln n}{2\log n}\Big ) > \frac{|E|}{2}.
	\end{align*}
	%$\Box$
	%$\qed$
\end{proof}

%\begin{lemma}
%	\label{min-deg}
%	Let $H=(V,E)$ be an arbitrary hypergraph with $n$ vertices. 
%	Then there is an induced
%	subhypergraph $H'=(V',E')$ of $H$ satisfying the following:
%	\begin{enumerate}
%		\item
%		$|E'|/|V'| \geq |E|/|V|.$
%		\item
%		The degree of each vertex $v \in V'$ in $H'$ is at least
%		\begin{equation}
%		\label{e31}
%		\frac{|E'|}{2 k}
%		\end{equation}
%		\item
%		$|E'| \geq |E|/2.$
%	\end{enumerate}
%\end{lemma}
%%\vspace{0.1cm}
%%\noindent
%%{\bf Proof:}\,
%\begin{proof} 
%	Put $V_0=V$ and $E_0=E$. Starting with $i=0$, as long as the
%	hypergraph $H_i=(V_i,E_i)$ in which $|V_i|=n-i$ does not satisfy
%	Condition~(2), let $V_{i+1}$ be the set obtained from $V_i$ by
%	deleting a vertex of minimum degree in $H_i$ and let $H_{i+1}$ be
%	the induced subhypergraph on this set. 
%	It is easy to see that $|E_{i+1}|/|V_{i+1}| > |E_i|/|V_i|$ and hence this process must terminate with a nonempty hypergraph $H_j=(V_j,E_j)$. Define $V'=V_j$, $E'=E_j$. Then Condition~(1) holds as the quantity $|E_i|/|V_i|$ keeps increasing during the process. 
%	Condition~(2) holds by the definition of $j$. Condition~(3) holds since 
%	$$
%	|E'| =|E_j| \ge |E_0| \prod_{i=0}^{j-1} (1-\frac{1}{2k}) \ge |E_0| e^{-j/2k}
%	\geq |E| (1-\frac{\ln n}{2\log n}) > \frac{|E|}{2}.
%	$$
%\end{proof}


%We proceed with the proof of the lower bound in Theorem \ref{t12}.
%Let $C=C(r,\a)=\frac{r+1-\log(1+\a)}{2-\log(1+\a)}$ be as in the statement of Theorem \ref{t12}. 

%Let $\FF$ be a collection of subsets of $[n]$, and denote the set of vertices $[n]$ by $V$. 
%By assumption, for any subset $V' \subset V$ of size at most $\alpha n$, the induced hypergraph on $V'$ is uniformly bounded above by $n^C$; we refer to this assumption as the \emph{$\alpha$-sparseness} assumption on $\FF$.
%
%We recall that for a subset $U$ of $V$, its degree is the number of hyperedges of $\FF$ containing $U$, and the link of $U$ inside $V$ is the hypergraph on the vertices $V\setminus U$ containing all subsets $W\subset V\setminus U$ such that $U \cup W$ is a hyperedge in $\FF$.
%
%We have the following simple observation.


%Call a hypergraph $H$ \emph{$(\a,B)$-sparse} if for every vertex subset $I \sub V(H)$ of size $\a|V(H)|$, the induced hypergraph $H[I]$ has at most $B$ edges.
%The \emph{weight} of a hypergraph $\FF$ is 


%===



\begin {lemma}
\label{lelink}
Let $\FF$ be a hypergraph on $n$ vertices with $\wp(\FF,i) \le \frac{|\FF|}{2}$, for some integer $0<i<n$.
Then $F$ has at least $i$ vertices of degree at least $\frac{|\FF|}{2n}$. 
%
%Let $\FF$ be a hypergraph on $n$ vertices such that for every $S \sub V(F)$ with $|S| \le \a n$ we have $|\FF[S]| \le \frac{|\FF|}{2}$. Then $F$ has at least $\alpha n$ vertices of degree at least $\frac{|\FF|}{2n}$. 
%
%Let $\FF$ be a hypergraph on $n$ vertices such that $|\FF_I| \le \frac{|\FF|}{2}$ for every $I \sub V(F)$ with $|I|=\alpha n$. 
%
%Let $\FF$ be a hypergraph on $n$ vertices, such that the restriction of $\FF$ to any subset of at most $\alpha n$ vertices has at most $\frac {|\FF|} 2$ hyperedges. Then, there are least $\alpha n$ vertices of $V$ of degree at least $\frac {|\FF|} {2n}$. 
\end {lemma}

%\noindent 
%{\bf Proof:}\,

\begin{proof}
	Let $I \sub V(F)$ denote the set of vertices of $\FF$ of degree at least $\frac{|\FF|}{2n}$.
	%The number of edges in the induced hypergraph 
	%Then 
	%$F[S]$ is smaller than $|\FF|-n \cdot \frac{|\FF|}{2n} = 
	Then
	$|F[I]| > |\FF|-n \cdot \frac{|\FF|}{2n} =
	\frac{|\FF|}{2}.$
	By the assumption on $\wp(\FF,i)$ we thus have $|I| > i$.
	%
	%n, along with any edge containing any of them. The total number of edges thus removed is at most 
	%
	%Assume to the contrary that the set of vertices of degree at most $\frac {\FF} {2n}$ contains at least $(1-\alpha) n$ vertices of $V$, and, by throwing these vertices away, along with their adjacent hyperedges, we remain with at most $\alpha n$ vertices containing at least $\frac {|\FF|} 2$ edges of $\FF$, contradicting our hypothesis on $\FF$. 
	%$\Box$
	%$\qed$
\end{proof}

By an iterative application of Lemma~\ref{lelink} we obtain the following. % useful lemma.
%We will use the trivial observation that one can apply Lemma~\ref{lelink} on any hypergraph $\FF$ satisfying $\trace(\FF,\,\a n) \le \frac{|\FF|}{2}$.
%; this weaker form is all that we would need in the proof below.
\begin {lemma}
\label{lelinkcor}
Let $s \in \N^+$ and let $\FF$ be a hypergraph on $V$ with 
%$\wp(\FF,k) \le \frac12\frac{|\FF|}{(2n)^{s-1}}$.
$\wp(\FF,i) \le \frac{|\FF|}{2^{s} |V|^{s-1}}$. %, for some $0<i<n$.
Then there are at least $i^s$ $s$-tuples $U \in V^s$ with $|\FF(U)| \ge \frac{|\FF|}{(2|V|)^s}$.
\end{lemma}

%TODO: in fact, these are $s$-tuples of distinct vertices

%\noindent
%{\bf Proof:}\,
\begin{proof}
Put $n=|V|$. For a tuple $U$ of vertices in $V$ we denote by $\FF^U$ the sub-hypergraph of $\FF$ on $V$ with edge set $\{e \in \FF \,\colon\, U \sub e\}$. Note that $|\FF^U|=|\FF(U)|$.
We proceed by induction on $s$, noting that the induction basis $s=1$ is Lemma~\ref{lelink}. 
For the induction step, let $\FF$ be as in the statement, and note that by the induction hypothesis there are at least $i^{s-1}$ $(s-1)$-tuples $U \in V^{s-1}$ with $|\FF^U|=|\FF(U)| \ge \frac{|\FF|}{(2n)^{s-1}}$.
Fix one such $U=(v_1,\ldots,v_{s-1})$ 
and apply Lemma~\ref{lelink} on the hypergraph $\FF^U$, noting that, as required,
$$\wp(\FF^U,i) \le \wp(\FF,i) \le \frac{|\FF|}{2^{s} n^{s-1}} \le \frac{|\FF^U|}{2},$$
where the first inequality uses the fact that $\FF^U$ is a sub-hypergraph of $\FF$ on $V$.
Thus, $\FF^U$ has at least $i$ vertices $v$ of degree at least $\frac{|\FF^U|}{2n} \ge \frac{|\FF|}{(2n)^{s}}$. 
This means that for each such $v$, the $s$-tuple $U'=(v_1,\ldots,v_{s-1},v)$ satisfies $|\FF(U')| = |\FF^{U'}| \ge \frac{|\FF|}{(2n)^{s}}$.
Going over all $i^{s-1}$ $(s-1)$-tuples $U$ in a similar fashion, we deduce that the total number of $s$-tuples $U'$ as above is at least $i^{s-1} \cdot i$.
This completes the induction step and the proof. 	
\end {proof}

%We will need the following %technical 
%lemma, which gives 
The following lemma gives
a unified lower bound for the summation $\sum_{i=0}^k \binom{x}{i}\g^i$ that is independent of the ratio between $k$ and $x$.
See Section~\ref{sec:aux} in the Appendix for a proof of this lemma.
\begin{lemma}\label{lemma:113}
	For every $k \in \N^+$ and real $0 \le \g \le 1$, $x \ge k$ we have
	$$\sum_{i=0}^k \binom{x}{i}\g^i 
	\ge \frac14\bigg(\sum_{i=0}^k \binom{x}{i}\bigg)^{\log(1+\g)} .$$
	%
	%$$\sum_{i=0}^{\log \sum_{i=1}^k \binom{x}{i}} \binom{\log \sum_{i=1}^k \binom{x}{i}}{i} \a^i$$
\end{lemma}


Finally, we have the following well-known bounds.
\begin{claim}\label{claim:e}
	We have $e^{-2x} \le 1-x \le e^{-x}$, where the upper bound holds for every real $x$ and the lower bound holds for every $0 \le x \le 1/2$.
\end{claim}
\begin{proof}
	The upper bound is well known, and the lower bound follows from it since we have $1-x = \big(1+\frac{x}{1-x}\big)^{-1} \ge e^{-\frac{x}{1-x}} \ge e^{-2x}$, where the last inequality uses $0 \le x \le 1/2$.
\end{proof}

%
%\begin{lemma}\label{lemma:113}
%	$$\sum_{i=0}^k \binom{x}{i}\a^i 
%	%\ge (1+\a)^{\log \sum_{i=1}^k \binom{x}{i}} = 
%	\ge \bigg(\sum_{i=0}^k \binom{x}{i}\bigg)^{\log(1+\a)} .$$
%	%
%	%$$\sum_{i=0}^{\log \sum_{i=1}^k \binom{x}{i}} \binom{\log \sum_{i=1}^k \binom{x}{i}}{i} \a^i$$
%\end{lemma}
%\begin{proof}
%	Remark113:
%	$$\sum_{i=0}^k \binom{x}{i}\a^i \ge (1+\a)^{\log\big(\sum_{i=0}^k \binom{x}{i}\big)}$$
%	TODO
%\end{proof}
%%\begin{proof}
%%	By induction on $k$.
%%	$$\sum_{i=0}^k \binom{x}{i}\a^i \ge \Big(\sum_{i=0}^{k-1} \binom{x}{i}\Big)^{\log(1+\a)} + \binom{x}{k}\a^k
%%	\ge \Big(\sum_{i=0}^{k} \binom{x}{i}\Big)^{\log(1+\a)}$$
%%	$$A^c - B^c = (B^{c-1}+AB^{c-2}+\cdots+A^{c-1})(A-B)$$
%%\end{proof}


\subsection{Sparse Kruskal-Katona Theorem}
%We recall from Section~\ref{se:intro} that the Kruskal-Katona Theorem, given here by a (slightly weaker) version due to Lov\'asz~\cite{Lovasz}, is as follows. 
%\kk*

%\newcommand{\s}{\sigma}


In this subsection we prove Theorem~\ref{th:skk}, which is a more precise version of Theorem~\ref{th:skk-intro}.
First, we will need a lemma which extends the classical Kruskal-Katona Theorem~\ref{th:kk} by collecting $i$-subsets from the hypergraph's links.

\newcommand{\II}{\mathcal{I}}

\begin{lemma}\label{lemma:neat}
	Let $\FF$ be a $k$-graph on $V$, and let $t \in \N$.
	For every $t \le i \le k$ we have
	$$\bigg|\binom{\FF}{i}\bigg| \ge i^{-t} \sum_{U \in V^t} \binom{x_U}{i-|U|} $$
	where $x_U$ is given by $|F(U)| = \binom{x_U}{k-|U|}$.
\end{lemma}
\begin{proof}
	Put $\II=V^t$, and let $t \le i \le k$.
	Apply the Kruskal-Katona Theorem~(Theorem~\ref{th:kk}) on each link $\FF(U)$ with $U \in \II$. Since $\FF(U)$ is a $(k-|U|)$-graph and $0 \le i-|U|  \le k-|U|$ (using $|U|\le t \le i$ for the lower bound), Theorem~\ref{th:kk} implies that $\big|\binom{\FF(U)}{i-|U|}\big| \ge \binom{x_U}{i-|U|}$.
	Now, for every $U \in \II$ denote 
	$$\binom{\FF(U)}{i}^* = \bigg\{ f \cup U \,\bigg\vert\, f \in \binom{\FF(U)}{i-|U|} \bigg\}.$$ 
	We have that
	$$\binom{\FF(U)}{i}^* \sub \binom{\FF}{i} \quad\text{ and }\quad \bigg|\binom{\FF(U)}{i}^*\bigg| = \bigg|\binom{\FF(U)}{i-|U|}\bigg| \ge \binom{x_U}{i-|U|}.$$
	We therefore deduce that
	$$\bigg|\binom{\FF}{i}\bigg| \ge \bigg|\bigcup_{U \in \II} \binom{\FF(U)}{i}^*\bigg| \ge i^{-t}\sum_{U \in \II} \bigg|\binom{\FF(U)}{i}^*\bigg| \ge i^{-t}\sum_{U \in \II} \binom{x_U}{i-|U|} ,$$
	where, crucially, the penultimate inequality uses the fact that if an $i$-set $g$ appears in $\binom{\FF(U)}{i}^*$ then 
	$U \in g^t$, 
	%$U$ is a $t$-tuple of elements from $f'$, 
	implying that $g$ appears in at most $i^t$ families $\binom{\FF(U)}{i}^*$.
	This completes the proof.
	%	
	%	where, crucially, the penultimate inequality uses the fact that if an $i$-set $f'$ appears in $\binom{\FF(U)}{i}^*$ then $U \sub f'$ (i.e., all vertices in $U$ lie in $f'$), meaning that $f'$ appears in at most 
	%	$$\sum_{j=0}^t \binom{i}{j} \le \sum_{j=0}^t \frac {i^j} {j!} \le \sum_{j=1}^t \frac {i^j} {2^{j-1}} \le i^t$$ 
	%	families $\binom{\FF(U)}{i}^*$ (here $j$ represents all possible values of $|U| \le t$).
	%	This completes the proof.
\end{proof}

%
%\begin{lemma}\label{lemma:neat}
%	Let $\FF$ be a $k$-graph and $t \in \N$.
%	For every $t \le i \le k$ we have
%	$$\bigg|\binom{\FF}{i}\bigg| \ge i^{-t} \sum_{U \in V(F)^t} \binom{x_U}{i-|U|} $$
%	where $x_U$ is given by $|F[U]| = \binom{x_U}{k-|U|}$.
%\end{lemma}
%\begin{proof}
%	Put $\II=V(F)^t$.
%	Let $t \le i \le k$.
%	Apply the Kruskal-Katona Theorem~(Theorem~\ref{th:kk}) on each link $\FF(U)$ with $U \in \II$. Since $\FF(U)$ is a $(k-|U|)$-graph and $0 \le i-|U|  \le k-|U|$ (using $|U|\le t \le i$ for the lower bound), this implies that $\big|\binom{\FF(I)}{i-|U|}\big| \ge \binom{x_U}{i-|U|}$.
%	Now, for every $U \in \II$ denote 
%	$$\binom{\FF(U)}{i}^* = \bigg\{ f \cup U \,\bigg\vert\, f \in \binom{\FF(U)}{i-|U|} \bigg\}.$$ 
%	We have that
%	$$\binom{\FF(U)}{i}^* \sub \binom{\FF}{i} \quad\text{ and }\quad \bigg|\binom{\FF(U)}{i}^*\bigg| = \bigg|\binom{\FF(U)}{i-|U|}\bigg| \ge \binom{x_U}{i-|U|}.$$
%	We therefore deduce that
%	$$\bigg|\binom{\FF}{i}\bigg| \ge \bigg|\bigcup_{I \in \II} \binom{\FF(U)}{i}^*\bigg| \ge i^{-t}\sum_{I \in \II} \bigg|\binom{\FF(U)}{i}^*\bigg| \ge i^{-t}\sum_{I \in \II} \binom{x_U}{i-|U|} ,$$
%	TODO 
%	where, crucially, the penultimate inequality uses the fact that if an $i$-set $f'$ appears in $\binom{\FF(U)}{i}^*$ then $U \sub f'$, meaning that $f'$ appears in at most $\binom{i}{t} \le i^t$ families $\binom{\FF(I)}{i}^*$.
%	This completes the proof.
%\end{proof}
%	
%\begin {remark}
Note that Lemma~\ref{lemma:neat} recovers Theorem~\ref{th:kk} by taking $t=0$.
%\end {remark}




%\begin{definition}[$\s(\FF,p)$]\label{def:s}
%For a hypergraph $\FF$ on $V$ and $0 \le p \le 1$ we denote by \emph{$\s(\FF,p)$} the largest integer $\s \ge 0$ such that for every $s < \s$,
%$$\Pr_{U \sim V^{s}} \bigg( \deg_{\FF}(U) \ge \frac{|\FF|}{(2|V|)^{s}} \bigg) \ge p^{s},$$
%with the vertex $s$-tuple $U$ chosen uniformly at random. 
%\end{definition}

%By an iterative application of Lemma~\ref{lelink} we obtain the following. % useful lemma.
%%We will use the trivial observation that one can apply Lemma~\ref{lelink} on any hypergraph $\FF$ satisfying $\trace(\FF,\,\a n) \le \frac{|\FF|}{2}$.
%%; this weaker form is all that we would need in the proof below.
%\begin {lemma}
%\label{lelinkcor}
%Let $\FF$ be a hypergraph on $n$ vertices satisfying 
%%$\wp(\FF,k) \le \frac12\frac{|\FF|}{(2n)^{s-1}}$.
%$\wp(\FF,\, \a n) \le \frac{|\FF|}{2^{s} n^{s-1}}$ with $s \in \N^+$.
%Then $\s(\FF,\a) > s$.
%\end{lemma}


We prove the following stronger form of Theorem~\ref{th:skk-intro}, our sparse Kruskal-Katona Theorem.

\begin{theo}\label{th:skk}
	Let $\FF$ be a $k$-graph with $n$ vertices and $|\FF|=n^r$ edges, $r \ge 1$. Let $s \in \N$ be the smallest satisfying $\frac{|\FF|}{(2n)^s} < c \cdot \wp(\FF,\, \a n)$ with 
	%$c=(16\a k)^{2r+1}$, 
	$c=(8k)^{\ceil{2r}}/\a^{\ceil{r}}$
	%$c=(8k)^{\ceil{r}}/\a^{\ceil{r}-1}$
	,\footnote{One may think of $|\FF|/(2n)^s$ as an approximation (from below) to the average degree of a vertex $s$-tuple in $\FF$. Alternatively, $s$ can be defined as $s=\big\lceil\log\big(\frac{|\FF|}{\sigma}\big)/\log(2n)\big\rceil$ with $\sigma=c \cdot \wp(\FF,\a n)$. 
			%In fact, we can assume without loss of generality that $s \le r-1$. Indeed, if $s = r$ then $\frac n {2^{r-1}} = \frac{|\FF|}{(2n)^{r-1}} > c \cdot \wp(\FF,\, \a n)$, or, equivalently, $\wp(\FF,\,\a n) < \frac n {2^{r-1}c}$. Then we may apply Theorem~\ref{th:skk} with $x=k-t$ and deduce that the number of $t$-sets is at least $\frac 1 C |\FF|$, implying that $|\FF| \le C\binom{n}{t}$ since the number of $t$-sets is upper bounded by $\binom{n}{t}$. In this case, life is good.
	} 
	and put $t=s+1$.
	If 
	$$\wp(\FF,\,\a n) \le \min\bigg\{\binom{x}{k-t}n, \,\, \frac{1}{2}|\FF|\bigg\}$$
	with real $x>0$ then for every $t \le i \le k$ we  have 
	$$\bigg|\binom{\FF}{i}\bigg| \ge \frac{1}{C}\cdot \frac{\binom{x}{i-t}}{\binom{x}{k-t}}|\FF|,$$
	with $C = (8k/\a)^{\ceil{4r}} \log n$.
	%$C = (16k/\a)^{4r+3} \log n$.
	%TODO should we add that $r \le \alpha n 2$ and remove the footnote in the proof?.
	%with $c' = \frac{1}{r\log n}\big(\frac{\a\g}{512 k}\big)^{2r+2}$ provided $r \le \frac12\a n$.
	%$C = \frac12c^2\log n
\end{theo}

\begin{remark}
	The parameters in Theorem~\ref{th:skk} satisfy the following relations:
	\begin{equation}\label{eq:sKK-relations}
		t \le \ceil{r} \le k.
	\end{equation}
	%$$t \le \ceil{r} \quad\text{ and }\quad r \le k.$$
	For the first inequality, note that otherwise $s=\ceil{r}$ and so, by the definition of $s$, $c \cdot \wp(\FF,\a n) \le |\FF|/(2n)^{s-1} \le n$, implying that $\wp(\FF,\a n) < \a n$ and thus, by averaging, $|\FF| < n$, contradicting the statement's assumption $r \ge 1$. 
	For the second inequality, notice $n^r = |\FF| \le \binom{n}{k} \le n^k$.
%	\footnote{For the last inequality, notice $n^r = |\FF| \le \binom{n}{k} \le n^k$.}
\end{remark}

%\begin{remark}
%	The classical Kruskal-Katona Theorem (Theorem~\ref{th:kk}) essentially follows from Threorem~\ref{th:skk} (up to the error term). To see this take $\a=1$, in which case $s=0$, and so if $|\FF|=\binom{x}{k}$ then $\wp(\FF,n)=\binom{x}{k} \le \binom{x}{k-1}n = \binom{x}{k-t}n$. (One can ignore the condition  ) TODO
%\end{remark}

%Note that the error term $C$ becomes large, the larger the quotient $k/\a$ is and the larger $r$ is. 
Note that the error term $C$ increases with the quotient $k/\a$ and with $r$.
This precludes us from taking hypergraphs of large uniformity, with many edges, or from inducing on too few vertices.
More formally, we have the following corollary.

%We can lower bound the error term $C$ in the statement of Theorem~\ref{th:skk}.
%\begin{coro}%[Corollary of Theorem~\ref{th:skk}]
%	\label{co:al}
\begin{remark}
	Under the assumptions of Theorem~\ref{th:skk}: %assuming $x \le \log |\FF|$,  the following holds: 
	\begin{enumerate}
		\item If $(k/\a)^r \le (\log n)^{O(1)}$ then $|\FF|/C \ge \tilde{\Omega}(|\FF|)$.
		\item If $k/\a \le n^{o(1)}$ then $|\FF|/C \ge |\FF|^{1-o(1)}$.
		%\item If $(k/\a)^r \le n^{o(1)}$ then $|\FF|/C \ge |\FF|^{1-o(1)}$.
	\end{enumerate}
\end{remark}

%projecting onto too small number of coordinates.

%\begin{theo}[``Sparse'' Kruskal-Katona Theorem]
%	\label{th:s}
%	Let $\FF$ be a $k$-graph with $n$ vertices and $|\FF|=n^r$ edges. 
%	If 
%	$$\wp(\FF,\,\a n) \le \min\bigg\{\binom{x}{k-r}n, \,\, \frac{1}{2}|\FF|\bigg\}$$
%	with real $x>0$ then for every $r \le i \le k$ we  have 
%	$$\bigg|\binom{\FF}{i}\bigg| \ge \frac{1}{C'}\cdot \frac{\binom{x}{i}}{\binom{x}{k}}|\FF|,$$
%	with $C' = (16ek/\a)^{5r+4} \log n$.
%\end{theo}

We now show how to deduce the sparse Kruskal-Katona Theorem from Theorem~\ref{th:skk}. 
\begin{proof}[Proof of Theorem~\ref{th:skk-intro}]
	%For the condition in Theorem~\ref{th:skk} 
	We have that $\binom{x}{k-\ceil{r}} \le \binom{x}{k-t}$ using~(\ref{eq:sKK-relations}) and since, by assumption, $x \ge 2k$. Thus, the condition here implies the condition in Theorem~\ref{th:skk}.
	As for the guarantee in Theorem~\ref{th:skk}, note that
	\begin{align*}
	\frac{\binom{x}{i-t}}{\binom{x}{k-t}} &= \frac{(k-t)!}{(i-t)!} \bigg(\prod_{j=i}^{k-1}(x-j+t)\bigg)^{-1}\\
	&= \frac{(k-t)!}{(i-t)!} \bigg(\prod_{j=i}^{k-1}(x-j) \left(1+\frac{t}{x-j}\right) \bigg)^{-1} 
	\ge \frac{k^{-t}k!}{i!} \bigg(\bigg(1+\frac{t}{x-(k-1)}\bigg)^{k-i} \cdot \prod_{j=i}^{k-1} (x-j) \bigg)^{-1}\\
	&\ge \frac{k^{-t}k!}{i!} \bigg(\Big(1+\frac{t}{k}\Big)^{k} \cdot \prod_{j=i}^{k-1} (x-j) \bigg)^{-1}
	\ge (ek)^{-t}\frac{k!}{i!} \bigg(\prod_{j=i}^{k-1} (x-j) \bigg)^{-1}
	= (ek)^{-t}\frac{\binom{x}{i}}{\binom{x}{k}} ,
	%\ge \frac{k^{-t}k!}{i!} \bigg(\prod_{j=i}^{k-1}(x-j) \left(1+\frac {t} {x-k+1}\right)^{k-1-i},
	\end{align*}
	where the second inequality uses the statement's assumption $x \ge 2k$, and the third inequality uses the upper bound in Claim~\ref{claim:e}.
	%bound $1+t/k \le e^{t/k}$.
	Thus, multiplying $C$ from Theorem~\ref{th:skk} by $(8k)^{\ceil{r}} %\ge (ek)^{r+1} 
	\ge (ek)^t$ (recall~(\ref{eq:sKK-relations})) completes the proof.
	%
	%TODO:
	%Here we show how to remove the restriction that $r \le \alpha n/2$. 
	%To see this, observe that $r \le k$, since $\FF$ is a $k$-graph, and thus $n^r \le \binom n k \le n^k$. If $r > \alpha n 2$, then also $k > \alpha n 2$, and thus $C > |\FF|^4.$ In particular, Theorem~\ref{th:skk} only asserts that $\binom {\FF} i \ge \binom x i$. 
	%Also note that $x \le y$ where $\binom y k = |\FF|$, for otherwise Theorem~\ref{th:kk} implies that $\binom {\FF} i \ge \binom y i = \frac{\binom{y}{i}}{\binom{y}{k}}|\FF| \ge \frac{\binom{x}{i}}{\binom{x}{k}}|\FF|,$ by the decreasing monotonicity of the functions $z \mapsto \binom{z}{a}/\binom{z}{b}$ (with $a \le b \le z$ and $a, b \in \N$).
	%By applying Theorem~\ref{th:kk} to $\FF$, we obtain that
	%$$\binom {\FF} i \ge \binom y i \ge \binom x i,$$ and so in this case our theorem is weaker than Theorem~\ref{th:kk} anyway.
\end{proof}


%Below is the proof of our sparse Kruskal-Katona theorem.

\newcommand{\reg}{_\text{reg}}

\begin{proof}[Proof of Theorem~\ref{th:skk}]
	We will prove the implication that if the stronger condition
	\begin{equation}\label{as:st}
	\wp(\FF,\,\a n) \le \min\bigg\{\frac{1}{c}\binom{x}{k-t}\a n,\,\, 
\frac12|\FF|\bigg\},
	\end{equation} 
	(where $c$ is as in the statement of the theorem) holds then for every $t \le i \le k$ we in fact have
	\begin{equation}\label{eq:sKK-first-goal}
	\bigg|\binom{\FF}{i}\bigg| \ge \frac{1}{c\log n}|\FF|\frac{\binom{x}{i-t}}{\binom{x}{k-t}} .
	\end{equation}
	To see why this would complete the proof, let $\tilde x$ satisfy 
	\begin {equation}
	\label {eq:tildex}
	\binom{\tilde x}{k-t} = \frac c \a \binom{x}{k-t},
	\end {equation} 
	so that if $\FF$ satisfies the statement's original assumption that $\wp(\FF,\,\a n) \le \min\big\{\binom{x}{k-t}n,\,\, \frac{1}{2}|\FF|\big\}$ then it satisfies~(\ref{as:st}) with $\tilde x$ replacing $x$.
	Thus, from~(\ref{eq:sKK-first-goal}),
	$$\bigg|\binom{\FF}{i}\bigg| \ge \frac{1}{c\log n}|\FF|\frac{\binom{\tilde x}{i-t}}{\binom{\tilde x}{k-t}} 
	\ge \frac{\a}{c^2\log n}|\FF|\frac{\binom{x}{i-t}}{\binom{x}{k-t}} 
	\ge \frac{1}{C}|\FF|\frac{\binom{x}{i-t}}{\binom{x}{k-t}} ,$$
	%= \frac{1}{512\cdot (2k)^{2r}\log n}|\FF|\frac{\binom{x}{i-t}}{\binom{x}{k-t}},$$
	where the second inequality uses~(\ref{eq:tildex}) for both the denominator and the
numerator, as~(\ref{eq:tildex}) 
implies $\tilde x \ge x$ since $c/\a \ge 1$.
%, which would complete the proof.
	
	We henceforth assume~(\ref{as:st}), and our goal is to prove~(\ref{eq:sKK-first-goal}).
	By the definition of $s$ we have 
	\begin{equation}\label{eq:sKK-s}
	\frac{|\FF|}{(2n)^s} < c \cdot \wp(\FF,\, \a n) \le \frac{|\FF|}{(2n)^{s-1}}.
	\end{equation}
	The upper bound in~(\ref{eq:sKK-s}) implies in particular that $\wp(\FF,\, \a n) \le \frac{|\FF|}{2^{s} n^{s-1}}$.
	Thus, by Lemma~\ref{lelinkcor}, there is a family $\UU \sub V(F)^s$ of $s$-tuples 
	$U$ of vertices of $\FF$ satisfying 
	\begin{equation}\label{eq:link}
	|\FF(U)| \ge \frac{|\FF|}{(2n)^s} =: b ,
	\end{equation}
	such that $|\UU| \ge \a^s n^s$.
	%$$|\UU| \ge \a^s n^s - sn^{s-1} = \a^s n^{s-1}(n - s(1/\a)^{s}) \ge \a^s n^{s-1}(n - r(1/\a)^{r}) \ge \frac12(\a n)^s,$$
	%where the penultimate inequality uses the fact that the function $z \mapsto z(1/\a)^z$ is monotone increasing for $z \ge 0$, and the last inequality uses the statement's assumption that $r < \frac {\log n} {2\log{\frac 1 \a}}$.
	For each $U \in \UU$ apply Lemma~\ref{l31} on the $(k-|U|)$-graph $\FF(U)$ to obtain an induced subgraph $\FF(U)\reg$ 
	with
	\begin{equation}\label{eq:link-edges}
	|\FF(U)\reg| > \frac12|\FF(U)| \ge \frac12 b,
	\end{equation}
	such that $\FF(U)\reg$ has $n_U$ vertices and minimum degree at least
	%\begin{equation}\label{eq:F_U}
	%\frac{|\FF(U)|}{4\log n \cdot t_U } \ge 
	%\d_U \ge \frac{1}{4 \log n} \cdot \frac{b}{t_U}
	%=: \binom{x_U}{k-(s+1)}
	%\end{equation}
	\begin{equation}\label{eq:F_U}
	%\frac{|\FF(U)|}{4\log n \cdot t_U } \ge 
	%\binom{x_U}{k-(s+1)} := \d_U \ge \frac{1}{4 \log n} \cdot \frac{b}{t_U} .
	\frac{1}{4 \log n} \cdot \frac{b}{n_U} =: \binom{x_U}{k-(|U|+1)}
	\end{equation}
	with $x_U>0$.
	Let $t \le i \le k$.
	Applying Lemma~\ref{lemma:neat} with our $t$ to the links of the vertices of each $\FF(U)\reg$, we deduce that
	\begin{align}\label{eq:sKK-main}
	%\begin{split}
	%\begin{equation}
	\bigg|\binom{\FF}{i}\bigg| &\ge \frac{1}{i^t}\sum_{U \in \UU} n_U\binom{x_U}{i-(|U|+1)} 
	\ge \frac{1}{4(2k)^t\log n} \frac{1}{n^s}\sum_{U \in \UU}\frac{\binom{x_U}{i-(|U|+1)}}{\binom{x_U}{k-(|U|+1)}} |\FF| ,%\\
	%&\ge \frac{1}{4i^t\log n} b\sum_{U \in \UU} \frac{\binom{x_U}{i-(|U|+1)}}{\binom{x_U}{k-(|U|+1)}}
	%\ge \frac{1}{4(2k)^t\log n} \frac{1}{n^s}\sum_{U \in \UU}\frac{\binom{x_U}{i-t}}{\binom{x_U}{k-t}} |\FF|
	%\end{split}
	\end{align}
	%\end{equation}
	where the second inequality uses~(\ref{eq:F_U}) together with~(\ref{eq:link}) and the fact that $i \le k$.
	%
	%where the second inequality uses~(\ref{eq:F_U}), the third relies (crucially) on the decreasing monotonicity of the function $z \mapsto \binom{y}{a-z}/\binom{y}{b-z}$ (with $z \le a \le b \le y$ and $a,b,z \in \N)$ as well as~(\ref{eq:link}) and the fact that $i \le k$.
	%Now, if $s=0$ then $|\UU|=1$, in which case~(\ref{eq:sKK-main}) immediately implies~(\ref{eq:sKK-first-goal}), and so we are done. We thus henceforth assume that $s \ge 1$.
	%
	Let 
	%$$\UU' = \bigg\{ U' \in V(\FF)^{s+1} \,\colon\, |\FF[U']| \le \binom{x}{k-(s+1)} \bigg\}.$$
	%$$\UU^+ = \bigg\{ U \in V(\FF)^s \,\colon\, \d_U > \binom{x}{k-(s+1)} \bigg\} \,\,\text{ and }\,\, 
	%$$\UU' = \bigg\{ U \in V(\FF)^s \,\colon\, \d_U \le \binom{x}{k-(s+1)} \bigg\}.$$
	$$\UU' = \bigg\{ U \in \UU \,\colon\, \binom{x_U}{k-(|U|+1)} \le \binom{x}{k-t} \bigg\}.$$
	%
	%
	%If $\binom{y}{k-u} \le \binom{x}{k-t}$ then $\frac{\binom{y}{i-u}}{\binom{y}{k-u}} \ge k^{-t} \frac{\binom{x}{i-t}}{\binom{x}{k-t}}$
%
	For every $U \in \UU'$ we have $\frac{\binom{x_U}{i-(|U|+1)}}{\binom{x_U}{k-(|U|+1)}} \ge k^{-t} \frac{\binom{x}{i-t}}{\binom{x}{k-t}}$,
	which follows from Claim~\ref{claim:binom-ratio}.\footnote{Indeed, take $x$, $y$, $k$, $i$, $\D$ there to be, respectively, $x$, $x_U$, $k-(|U|+1)$, $i-(|U|+1)$,  $t-(|U|+1)$, and bound $(i-(|U|+1))^{-(t-(|U|+1))}$ from below by $k^{-t}$.}
%
	%
%	We claim that for every $U \in \UU'$ we have $\frac{\binom{x_U}{i-(|U|+1)}}{\binom{x_U}{k-(|U|+1)}} \ge k^{-t} \frac{\binom{x}{i-t}}{\binom{x}{k-t}}$. 
%	Indeed, if $x_U \le x$ then $\frac{\binom{x_U}{i-(|U|+1)}}{\binom{x_U}{k-(|U|+1)}} \ge \frac{\binom{x_U}{i-t}}{\binom{x_U}{k-t}} \ge \frac{\binom{x}{i-t}}{\binom{x}{k-t}}$ using, respectively, the decreasing monotonicity of the functions $z \mapsto \binom{y}{a-z}/\binom{y}{b-z}$ (with $z \le a \le b \le y$) and $z \mapsto \frac{\binom{z}{a}}{\binom{z}{b}}$ (with $a \le b \le z$),
%	whereas if $x_U \ge x$ then, using the definition of $\UU'$, 
%	$$\frac{\binom{x_U}{i-(|U|+1)}}{\binom{x_U}{k-(|U|+1)}} 
%	\ge \frac{\binom{x}{i-(|U|+1)}}{\binom{x}{k-t}} 
%	= \frac{\binom{x}{i-(|U|+1)}}{\binom{x}{i-t}} \cdot \frac{\binom{x}{i-t}}{\binom{x}{k-t}}
%	\ge \frac{1}{i^{t}}\cdot \frac{\binom{x}{i-t}}{\binom{x}{k-t}} 
%	\ge \frac{1}{k^t}\cdot \frac{\binom{x}{i-t}}{\binom{x}{k-t}},$$
%	where the penultimate inequality uses the bound
%	$\frac{\binom{z}{a}}{\binom{z}{b}} = 
%	\frac{b!}{a!}\prod_{j=a}^{b-1}\frac{1}{z-j}
%	\le b^{b-a}$ for all $a \le b \le z$.
	%
	We will show that
	\begin{equation}\label{eq:goal}
	|\UU'| \ge \frac12 \a^s n^s .
	\end{equation}
%	
%	
%	We claim that for every $U \in \UU'$ we have $\frac{\binom{x_U}{i-t}}{\binom{x_U}{k-t}} \ge \frac{1}{?}\frac{\binom{x}{i-t}}{\binom{x}{k-t}}$. Using the identity 
%	$\frac{\binom{z}{a}}{\binom{z}{b}} = \frac{b!}{a!}\prod_{j=a}^{b-1} \frac{1}{z-j}$  
%	%$\binom{z}{a}/\binom{z}{b} = \frac{b!}{a!}\frac{1}{(z-a)\cdots(z-b+1)}$ 
%	(with $a \le b \le z$ and $a, b \in \N$), the claim is equivalent to the inequality 
%	$\prod_{j=i-t}^{k-t-1} \frac{x_U-j}{x-j} \le ?$.  
%	%$\frac{(z'-a)\cdots(z'-b+1)}{(z-a)\cdots(z-b+1)} \le ...$.
%	
%	Indeed, if $x_U \le x$ then this trivially follows from the decreasing monotonicity of the function $z \mapsto \binom{z}{a}/\binom{z}{b}$ (with $a \le b \le z$ and $a, b \in \N$), whereas if $x_U \ge x$ then this follows from the fact that, by the definition of $\UU'$, $\binom{x}{k-(|U|+1)} \le \binom{x_U}{k-(|U|+1)} \le \binom{x}{k-t}$, hence $k-t \ge x/2$, so the ratio of the right hand side to the left hand side in the claimed inequality is at most 
	%
	By~(\ref{eq:sKK-main}), this would imply that
	$$\bigg|\binom{\FF}{i}\bigg| \ge \frac{\a^s}{8(2k)^{2t}\log n} \frac{\binom{x}{i-t}}{\binom{x}{k-t}} |\FF|
	\ge \frac{1}{c\log n}\frac{\binom{x}{i-t}}{\binom{x}{k-t}} |\FF| ,$$
	where the last inequality uses $8(2k)^{2t}/\a^s \le (8k)^{\ceil{2r}}/\a^{\ceil{r}} = c$
	%$c = (8k)^{2\ceil{r}}/\a^{\ceil{r}-1} \ge 8(2k)^{2t}/\a^s$ 
	(recall~(\ref{eq:sKK-relations})).
	Thus, proving~(\ref{eq:goal}) would imply~(\ref{eq:sKK-first-goal}) and complete the proof.
	
	%Applying Lemma~\ref{lemma:neat} with our $t$, we would deduce from~(\ref{eq:goal}) that for every $t \le i \le k$ we have
	%\begin{align}\label{eq:sKK-main}
	%\bigg|\binom{\FF}{i}\bigg| &\ge i^{-t}\sum_{U \in \UU} n_U\binom{x_U}{i-(|U|+1)} 
	%\ge \frac{1}{c'} b\sum_{U \in \UU'} \frac{\binom{x_U}{i-(|U|+1)}}{\binom{x_U}{k-(|U|+1)}} 
	%\ge \frac{1}{c'} b |\UU'|\frac{\binom{x}{i-t}}{\binom{x}{k-t}} 
	%\ge \frac{1}{c\log n} |\FF|\frac{\binom{x}{i-t}}{\binom{x}{k-t}}
	%\end{align}
	%with $c'=4\log n \cdot k^{r+1}$ (as $i \le k$ and $t \le r+1$) and, we recall, $c =  (16\a k)^{2r+1}$; here, the second inequality uses~(\ref{eq:F_U}), the third inequality relies (crucially) on the decreasing monotonicity of the functions $z \mapsto \binom{z}{a}/\binom{z}{b}$ (with $a \le b \le z$ and $a, b \in \N$) and $z \mapsto \binom{c}{a-z}/\binom{c}{b-z}$ (with $z \le a \le b \le c$ and $a,b,z \in \N)$, and the fourth inequality uses~(\ref{eq:link}) and our assumption~(\ref{eq:goal}).
	
	Put $S = {\binom{x}{k-t}} \a n$. 
	It remains to prove~(\ref{eq:goal}). 
	Assume for contradiction that $|\UU \setminus \UU'| \ge \frac12(\a n)^s$.
	Note that by definition of $\UU'$ together with~(\ref{eq:F_U}) 
we deduce that for every $U \in \UU \setminus \UU'$ we have 
	%\begin{equation}\label{eq:t0}
	%t_U \le (4\log n\cdot 2^s)^{-1}\frac{|\FF|}{n^s\binom{x}{k-t}} =  (4\log n\cdot 2^s)^{-1}\frac{|\FF|}{n^s B}\a n  =: t_0.
	%n_U \le (4\log n)^{-1}\frac{b}{\binom{x}{k-t}} = (4\log n)^{-1}\frac{b}{S} \a n =: n_0,
	$$
n_U \le \frac{1}{4 \log n} \cdot \frac{b}{\binom{x}{k-t}} \leq
\frac12\cdot\frac{b}{\binom{x}{k-t}} = \frac{b}{S} \cdot \frac12\a n =: n_0.
$$
	%\end{equation}
	Note that from the lower bound in~(\ref{eq:sKK-s}) together with~(\ref{as:st}) we deduce that $b \le S$.
	%$s \ge 1$ and that $b \le 2S$.
	Since $n_0 \ge n_U \ge 1$, we have that $n_0$ satisfies
	\begin{equation}\label{eq:sKK-n0-bounds}
	1 \le n_0 \le \frac12 \a n .
	\end{equation}
	Put
	\begin{equation}\label{eq:sKK-ell}
	\ell = \frac12\cdot\begin{cases}
	\frac{\a n}{n_0+s} 	& \text{ if } s \ge 1\\
	1					& \text{ if } s = 0
	\end{cases}
	\end{equation}
	%
	%where the lower bound uses the fact that $n_0 \ge n_U \ge 1$, and the upper bound uses the fact that $b \le 2S$ by TODO.
	%\begin{equation}\label{eq:t0}
	%%t_U \le (4\log n\cdot 2^s)^{-1}\frac{|\FF|}{n^s\binom{x}{k-t}} =  (4\log n\cdot 2^s)^{-1}\frac{|\FF|}{n^s B}\a n  =: t_0.
	%t_U \le (4\log n)^{-1}\frac{b}{\binom{x}{k-t}} 
	%\le (4\log n)^{-1}\frac{b}{\binom{x}{k-t}} 
	%= (4\log n)^{-1}\frac{b}{B}\cdot \a n 
	%= (16k^{s+1})^{-1}\frac{b}{S} \cdot \a n =: t_0.
	%\end{equation}
	%\begin{equation}
	%t_U \le 16k^{s+1} \cdot \frac{b}{\binom{x}{k-t}} 
	%=  16k^{s+1} \frac{b}{S} \cdot \a n =: t_0.
	%\end{equation}
	Note that~(\ref{eq:sKK-n0-bounds}) implies 
	%Put $\ell := \frac{\a n}{2(n_0+s)}$, and note that we have the bounds
	\begin{equation}\label{eq:sKK-l-bounds}
	1 \le \frac{\a n}{n_0+s} \le \a n ,
	\end{equation}
	where the lower bound further uses the fact that $s \le r$ and the fact that we may assume $r \le \a n/2$ as otherwise there is nothing to prove\footnote{Otherwise $C \ge |\FF|$, and so the statement's lower bound on $\big|\binom{\FF}{i}\big|$ is trivially true since $\frac{\binom{x}{i-t}}{\binom{x}{k-t}} \le \frac{\binom{k-t}{i-t}}{\binom{k-t}{k-t}} \le \binom{k}{i}$, where the first inequality uses the decreasing monotonicity of the function $z \mapsto \binom{z}{a}/\binom{z}{b}$ with $a \le b \le z$.}.
	%	 holds in general is left to the reader, but it clearly holds when $r/\alpha$ does not exceed $n/2$, which is the most interesting case anyway.}
	Let $\UU^* \sub \UU \setminus \UU'$ be an arbitrary subset with $|\UU^*|=\ceil{\ell}$, 
	%%|\UU^*|= \frac{\a n}{2(t_0+s)} =: \ell,
	%%|\UU^*|=(2k)^{-s} \cdot \a n\binom{x}{k-t} =: \ell.
	%\ell := \frac{\a n}{2(n_0+s)} .
	%\end{equation}
	which is well defined as $\ell \le \frac12 (\a n)^s \le |\UU \setminus \UU'|$; here, the first inequality is immediate for $s=0$ by~(\ref{eq:sKK-ell}), and for $s\ge 1$ follows from the upper bound in~(\ref{eq:sKK-l-bounds}) together with the bound $\a n \le (\a n)^s$.
	For each $U \in \UU$ denote $I_U = V\big(\FF(U)\reg\big) \cup U$, and note that
	\begin{equation}\label{eq:liin}
	\FF[I_U] = \big\{e \cup U \mid e  \in \FF(U)\reg\big\}.
	\end{equation}
	Let $I = \bigcup_{U \in \UU^*} I_U$ denote the union of these sets of vertices. Then $I$ satisfies that
	$$|I| \le \sum_{U \in \UU^*} (n_U+s) \le \ceil{\ell} (n_0+s) \le 2\ell(n_0+s) \le \a n ,$$
	where the penultimate inequality uses the lower bound $\ell \ge \frac12$ from~(\ref{eq:sKK-l-bounds}).
	%for $\ell \ge 1/2$ the fact that $\ceil{\ell} \le 2\ell$, and for $\ell < 1/2$ the fact that 
	Moreover, $I$ satisfies that
	$$|\FF[I]| \ge \Big|\bigcup_{U \in \UU^*} \FF[I_U] \Big|
	\ge k^{-s} \sum_{U \in \UU^*} \big|\FF[I_U]\big|
	= k^{-s} \sum_{U \in \UU^*} \big|\FF(U)\reg\big|
	> k^{-s} \cdot \frac12 b \ceil{\ell},$$
	where the second inequality uses the fact that
	$e \in \FF[I_U]$ for at most $k^s$ $s$-tuples $U$ 
	(by~(\ref{eq:liin}), $e \in \FF[I_U]$ implies $U \sub e$), and the third inequality uses~(\ref{eq:link-edges}).
	Now, if $s=0$ then $\ceil{\ell}=1$ and $b=|\FF|$, hence we get $|\FF[I]| > |\FF|/2 \ge \wp(\FF,\a n)$ using~(\ref{as:st}), a contradiction.
	Otherwise, we get
	%Moreover, we have 
	$$|\FF[I]| > \frac{1}{8 k^{s+1}} \cdot b\frac{\a n}{n_0}
	%= (64\log n \cdot k^{s+1})^{-1} \cdot B = 
	= \frac{1}{4k^{s+1}} S \ge \frac{1}{c}S \ge \wp(\FF,\,\a n),$$
	%= \frac{\log n}{k^{s+1}} S ,
	%\ge \a n\binom{x}{k-t}
	where the first inequality uses~(\ref{eq:sKK-ell}) and bounds $n_0+s \le 2n_0k$ (as $n_0 \ge 1$ by~(\ref{eq:sKK-n0-bounds}) and $s \le k$ by~(\ref{eq:sKK-main})), 
	the equality uses~(\ref{eq:sKK-n0-bounds}), 
	and the last inequality uses~(\ref{as:st}).
	We thus again obtain a contradiction.
	%
	%This contradicts the statement's assumption that $\wp(\FF,\a n) \le (16 k^t)^{-1}\binom{x}{k-t}\a n$, and completes the proof under the stronger assumption~(\ref{as:st}).
	This completes the proof. %assuming~(\ref{as:st}).
\end{proof}

%\begin{remark}\label{re:log}
%Under the assumptions of Theorem~\ref{th:skk}, if $x>\log |\FF|$, Theorem~\ref{th:kk} implies Theorem~\ref{th:skk}.
%%\begin {proof}
%To see this, assume to the contrary that $x> \log |\FF|$.  Then, Theorem~\ref{th:skk} lower bounds $\big|\binom{\FF}{i}\big|$ by $c\frac{\binom{x}{i-s}}{\binom{x}{k-s}}|\FF| \le c\frac{\binom{\log|\FF|}{i-s}}{\binom{\log|\FF|}{k-s}}|\FF| = O(\binom{\log|\FF|}{i-s})$ (one can check that indeed the function $\frac{\binom{a}{i}}{\binom{a}{\frac a 2}}$ is monotone decreasing in $a$).
%	On the other hand, Theorem~\ref{th:kk} lower bounds $\big|\binom{\FF}{i}\big|$ by $\binom{y}{i}$ where $|\FF| \le \binom{y}{k} \le \binom{y}{y/2}$, and thus by $\Omega(\binom{\log|\FF|}{i})$.
%	%
%%	\ge \binom{\log|\FF|}{i}$ with as $|\FF| \le \binom{x'}{k} \le \binom{x'}{x'/2}$.
%\end{remark}



We have the following important corollary of Theorem~\ref{th:skk}.

\begin{coro}%[Corollary of ]
	\label{co:exptr}
	Let $\FF$ be a $k$-uniform hypergraph with $n$ vertices and $|\FF|=n^r$ edges, $r \ge 1$. 
	If 
	$$\wp(\FF,\,\a n) \le \min\Big\{B n,\,\, \frac12|\FF|\Big\}$$
	with real $B>0$, then for every $0 \le \gamma \le 1$ the expected trace of $\FF$ on a 
	uniformly random subset
	%random\footnote{I.e., each vertex is chosen independently with probability $\g$.} set 
	of $\g n$ vertices is at least
	$$\frac{1}{C'} \cdot \frac{|\FF|}{B^{1-\log(1+\gamma)}},$$
	with $C' = (8k/\a\g)^{\ceil{5r}}\log n$, provided $k \le \sqrt{\g n}$.
	\end{coro}
	
	\begin {proof}
	Write $B = \binom{x}{k-t}$ with $x>0$ with $t \in \N$ as in Theorem~\ref{th:skk}, so that 
	$\wp(\FF,\,\a n) \le \min\big\{\binom{x}{k-t}n,\,\, \frac{1}{2}|\FF|\big\}$.
	The expected trace of $\FF$ on a uniformly random set of $q = \g n$ vertices is, by Theorem~\ref{th:skk},
	\begin{align*}
	\sum_{i=0}^k \bigg| \binom{\FF}{i}\bigg| \frac{\binom{n-i}{q-i}}{\binom{n}{q}} &\ge \sum_{i=0}^k \bigg| \binom{\FF}{i}\bigg| \g^i e^{-1}
	\ge \frac{|\FF|}{eC\binom{x}{k-t}}\sum_{i=t}^k \binom{x}{i-t} \g^i \\
	&= \frac{|\FF|}{eC\binom{x}{k-t}}\g^t\sum_{j=0}^{k-t} \binom{x}{j} \g^{j} 
	\ge \frac{\g^t |\FF|}{4eC\binom{x}{k-t}}\binom{x}{k-t}^{\log(1+\g)}
	= \Big(\frac{\g^t}{4eC}\Big)\frac{|\FF|}{B^{1-\log(1+\g)}}
	\end{align*} 
	with $C= (8k/\a)^{\ceil{4r}}\log n$ as in Theorem~\ref{th:skk},
	where the first inequality uses (recall the lower bound in Claim~\ref{claim:e} and the statement's assumption $k \le \sqrt{q}$)
	$$\frac{\binom{n-i}{q-i}}{\binom{n}{q}} = \frac{(n-i)!}{n!}\frac{q!}{(q-i)!} = \prod_{j=0}^{i-1} \frac{q-j}{n-j}
	\ge \Big(\frac{q}{n}\Big)^i \prod_{j=0}^{i-1} (1-j/q)
	\ge \g^i \prod_{j=0}^{i-1} e^{-2j/q} = \g^i e^{-i^2/q} \ge \g^i e^{-1},$$
	%	
	%$C=(16k/\a)^{4r+3}\log n$. 
	%with $c = \frac{1}{\log n}\bigl(\frac \a{kr}\bigr)\big(\frac{\a}{256k}\big)^{2s+1}$, 
	and the last inequality uses Lemma~\ref{lemma:113}. 
	%Since $B = \binom x {\frac x 2} \ge 2^{\frac x 2}$, it follows that $\log B \ge \frac x 2$, implying that $c \ge c'$, and the proof is complete.
	Using~(\ref{eq:sKK-relations}) to bound $t$, the proof follows.
	%we are done.
	\end {proof}
	
	%\begin {remark}
	Note that the statement in Corollary~\ref{co:exptr} does not depend on $k$, the uniformity of the hypergraph, except in the error term $C'$. In fact, in order for this statement to be meaningful, $C'$ should be negligible, so $k$ should be relatively small, e.g., poly-logarithmic in $n$.
	%\end {remark}
	
	\begin{remark}
	If the bound in Corollary~\ref{co:exptr} is tight, then, in the proof of Theorem~\ref{th:skk}, all the inequalities become (essentially) equalities. In the proof of Theorem~\ref{th:skk}, we showed that the number of $s$-tuples in $\UU'$ is at least $\frac12(\a n)^s$.
	It is then easy to verify that for half of these $s$-tuples $U$, we have $t_U \approx t_0, x_U \approx x$ and then $x \approx 2k$. (Quantifying this statement precisely requires more details, which we omit here.)
	Note that this remark applies to Corollary~\ref{co:exptr} but not to Theorem~\ref{th:skk} itself, as evident from the matching upper bound in Subsection~\ref{sec:sKK-UB}.
	\end{remark}
	

	
%%%%%%%%%%%%%%%%%%%%%%%%%%%%%%%%%%%%%%%%%%%
%\input{traces-coro}

\subsection{Tight lower bound for traces}\label{sec:traces}

Recalling the definition of $\C$ in~(\ref{eq:C}), we now prove the lower bound in our main result Theorem~\ref{th:main}.

\begin{theo} 
	\label{th:main}
	Let $r \ge 1$, $\a \in (0,1]$. If $r, \a^{-1} \le n^{\d}$ then
	$\trace(n,\,n^r,\,\a n) \ge n^{\C(1-O(\d))}$.
	Moreover, if $r=O(1)$, $\a^{-1} \le (\log n)^{O(1)}$ then $\trace(n,\,n^r,\,\a n) \ge \tilde\Omega(n^\C)$.
\end{theo}
\begin{proof}
	Let $\HH$ be a down-closed hypergraph\footnote{Also called monotone. A hypergraph $\HH$ is down-closed if for every edge $\HH$, all its subsets are also edges of $\HH$.} on $n$ vertices with $|\HH|=n^r$.
	We will prove lower bounds on $\trace(\HH,\a n)$, which would complete the proof since we may assume the given hypergraph is down-closed (this is standard, see, e.g.,~\cite{Al}).
	Put $\HH_i = \big\{e \in \HH \,\big\vert\, |e|=i\big\}$, and let $\FF=\HH_k$ where $k$ maximizes $|\HH_k|$.
	Observe that since $\HH$ is down-closed we have $|\HH| \ge 2^k$, which implies that $k \le \log|\HH|$, and therefore $|\FF| \ge |\HH|/(\log|\HH|+1)$.
	We assume $\trace(\FF,\,\a n) \le \frac12|\FF|$, as otherwise we are done since
	$$\trace(\HH,\a n) \ge \trace(\FF,\a n) \ge \frac12|\HH|/(\log|\HH|+1) = \tilde{\Omega}(n^r).$$ 
	
	Now, write $\trace(\FF,\,\a n) =B n$ with $B > 0$.
	Since $\wp(\FF,\,\a n) \le \trace(\FF,\,\a n)$ and since $k \le \sqrt{\a n}$ (i.e., $k^2/\a \le n$) for all large enough $n$, we apply Corollary~\ref{co:exptr} with $\g=\a$ to obtain that
	$$B n = \trace(\FF,\,\a n) 
	%\ge \underset{I} {\mathbb E} \big|\FF_I\big|
	%\ge \underset{\substack{I \sub [n]\\|I|=\alpha n}} {\mathbb E} \big|\FF_I\big|
	\ge \frac{1}{C'}\frac{|\FF|}{B^{1-\log(1+\a)}},$$
	%with $I$ a random subset of $V(\FF)$ of expectation $\a n$, where
	with $C' =  (8k/\a^2)^{\ceil{5r}}\log n$, where the inequality uses the fact that the maximal trace on $\alpha n$ vertices is at least as large as the expected trace on a random subset 
	of $\a n$ vertices. 
	%where each vertex is independently chosen with probability $\a$.
	%, and the second inequality uses Corollary~\ref{co:exptr} with $\gamma = \alpha$.
	%	Therefore,
	%	$$\binom{x}{k} \ge \Big(\frac{c|\FF|}{n}\Big)^{\frac{1}{2-\log(1+\a)}},$$
	%	which in turn implies that
	%	$$\trace(\FF,\,\a n) = \binom{x}{k}n \ge \Big(c|\FF|n^{1-\log(1+\a)}\Big)^{\frac{1}{2-\log(1+\a)}} .$$
	%	We deduce that
	%	$$\trace(H,\,\a n) \ge \trace(\FF,\,\a n) 
	%	\ge \Big(\frac{c}{\log|H|}|H|n^{1-\log(1+\a)}\Big)^{\frac{1}{2-\log(1+\a)}} = \tilde\Omega \Big(|H|n^{1-\log(1+\a)}\Big)^{\frac{1}{2-\log(1+\a)}}.$$
	%	This completes the proof.
	%%	
	This gives a lower bound on $B$; we therefore deduce
	\begin{align*}
	\trace(\HH,\,\a n) &\ge \trace(\FF,\,\a n) = B n 
	\ge \Big(\frac{1}{C'}|\FF|\cdot \frac{1}{n}\Big)^{\frac{1}{2-\log(1+\a)}} \cdot n\\
	&= \Big(\frac{1}{C'}|\FF| n^{1-\log(1+\a)}\Big)^{\frac{1}{2-\log(1+\a)}}
	\ge \Big(\frac{1}{C''}|\HH|n^{1-\log(1+\a)}\Big)^{\frac{1}{2-\log(1+\a)}} 
	\ge \frac{1}{C''} \cdot n^{\C},
	\end{align*}
	with $C''=(\log|\HH|+1)C' \le (8\log|\HH|/\a^2)^{\ceil{5r}}\log n \le (8r\log n/\a^2)^{\ceil{6r}}$. 
	Now, if $r,\a^{-1} \le n^{\d}$ then $C'' \le n^{O(\d r)}$, which implies that $\trace(\HH,\,\a n) \ge n^{\C(1-O(\d))}$.
	%Now, if $r,\a^{-1} \le n^{o(1)}$ then $C'' \le n^{o(r)}$, which implies that $\trace(\HH,\,\a n) \ge n^{\C(1-o(1))}$.
	Moreover, if $r \le O(1)$, $\a^{-1} \le (\log n)^{O(1)}$ then $C'' \le (\log n)^{O(1)}$, which implies that 
	$\trace(\HH,\,\a n) \ge \tilde\Omega(n^{\C})$.
	This completes the proof.
	%Since $|H|=n^r$, $r$ and $\a$ are constants, and $k \le \log|H|$, we deduce that $C' \le (\log n)^{O(1)}$ and thus $C'' \le (\log n)^{O(1)}$. 
	%This completes the proof.
	%Recall that 
	%Note that $|\FF| \ge \trace(\FF,\,\a n) = Bn$, %= 2^{x-o(1)}\frac{\a n}{k}$. 
	%implying that $\log B \le \log|\FF| \le \log|H| = r\log n$.
	%This completes the proof.
\end{proof}

\begin{remark*}
	In the proof of Theorem~\ref{th:main} it seems tempting to write $\trace(\FF,\alpha_0 n) = B_0 n$ with $\alpha_0 \ll \alpha$, so that $B_0 \ll B$. Then, Corollary~\ref{co:exptr} would mean that the expected trace on $\alpha n$ random vertices would be roughly $\frac{|\FF|}{B_0^{1-\log(1+\a)}} \gg \frac{|\FF|}{B^{1-\log(1+\a)}},$ which seemingly contradicts the fact that our lower bound is tight!
	%
	The reason this cannot happen is that, for any $\alpha_0$ that is not too small, though the \emph{average} trace on $\alpha_0 n$ vertices is substantially smaller than that on $\alpha n$ vertices, the \emph{maximal} traces can actually be the same. Indeed, in the upper bound construction one can show that $\trace(\FF,\alpha_0 n) \approx \trace(\FF,\alpha n)$ for any $\alpha_0 \le \alpha$ that is not too small.
\end{remark*}

%%%%%%%%%%%%%%%%%%%%%%%%%%%%%%%%%%%%%%%%%%%%%%%%%%%%%%%%%%%%%%%%%%%%%%%%%%%%
%\input{traces-UB}


\section{Upper Bounds}%\label{sec:sKK-UB}

\subsection{Upper Bound for the Sparse Kruskal-Katona Theorem}\label{sec:sKK-UB}

In this subsection we show that the parameters in Theorem~\ref{th:skk} are best possible up to the error term. Formally, we prove the following. %upper bound.

\begin{theo}[Upper bound for sparse Kruskal-Katona]\label{theo:sKK-UB}
	Let $n,k,x \in \N^+$, $r \ge 1$ and $0 < \a \le 1$ with 
	$3r \le k \le x \le n^{1/6}$
	%$k \ge 3r$, $x \le n^{1/6}$, 
	and $n \le \a^k n^r \le \binom{x}{k}n$.
	%$\a \ge n^{-(r-1)/k}$, 
	%and $\binom{x}{k}n \ge \a^k n^r$.
	There exists a $k$-graph $\FF$ with $n$ vertices, $|\FF|=n^r$ edges, 
	and 
	$\wp(\FF,\,\a n) \le 6\binom{x}{k}n$
	such that for every $0 \le i \le k$ we have 
	$\big|\binom{\FF}{i}\big| \le \frac{\binom{x}{i}}{\binom{x}{k}}|\FF|$.
\end{theo} 

We will need the following lemma relating hypergeometric and binomial random variables.
\begin{lemma}\label{lemma:HG}
	Let $H$ be a hypergeometric random variable with parameters 
$(n,x,y)$,\footnote{I.e., $H=|X \cap Y|$ where $X \sub [n]$ is a 
uniformly random subset of size $x$ and $Y \sub [n]$ is a fixed 
subset of size $y$.} 
	and let $B$ be a binomial random variable with parameters $(x,\, y/n)$.
	If $x \le \sqrt{n}$ then for every $0 \le h \le x$ we have $\Pr[H=h] \le 2\Pr[B=h]$.
\end{lemma}
\begin{proof}
	We have
	\begin{align*}
	\Pr[H=h] &= 
	\frac{\binom{y}{h}\binom{n-y}{x-h}}{\binom{n}{x}} 
	=\frac{\binom{x}{h}\binom{n-x}{y-h}}{\binom{n}{y}}
	= \binom{x}{h}\frac{(n-x)!}{n!}\cdot\frac{y!}{(y-h)!}\cdot\frac{(n-y)!}{(n-y-(x-h))!}\\
	&= \binom{x}{h}\frac{\prod_{t=0}^{h-1} y-t}{\prod_{t=0}^{h-1} n-t} 
	\cdot \frac{\prod_{t=0}^{x-h-1} n-y-t}{\prod_{t=0}^{x-h-1} n-h-t}
	\le \binom{x}{h}\bigg(\frac{y}{n}\bigg)^h\bigg(\frac{n-y}{n-h}\bigg)^{x-h} \\
	&\le \binom{x}{h}\bigg(\frac{y}{n}\bigg)^h\bigg
	(\frac{n-y}{n}\bigg)^{x-h} e^{(2h/n)\cdot(x-h)}
	%&\le \binom{x}{h}\bigg(\frac{q}{n}\bigg)^h\bigg(\frac{n-q}{n}\bigg)^{x-h}e^{h(x-h)/(n-h)} 
	\le 2\binom{x}{h} \bigg(\frac{y}{n}\bigg)^h\bigg
	(\frac{n-y}{n}\bigg)^{x-h}.
	\end{align*}
	where the first inequality uses that $h \le y$ (as otherwise 
$\Pr[H=h]=0$ and there is nothing to prove), the second inequality 
%uses the upper bound in Fact~\ref{fact:e},
uses the lower bound in Claim~\ref{claim:e}
as $h \le x \le \sqrt{n} \le n/2$ (the last inequality assumes $\sqrt{n} \ge 2$, for otherwise $x \le 1$ in which case $H=B$ so there is nothing to prove),
and the third inequality uses the fact that $h(x-h) \le x^2/4 \le n/4$
	as $x \le \sqrt{n}$. 
	%and that $h \leq x \leq \sqrt n$.
%by the statement's assumption.
	Recalling that $B$ is a binomial random variable with parameters $(x,\a)$ with $\a=y/n$, we deduce
	$$\Pr[H=h] \le 2\binom{x}{h} \a^h(1-\a)^{x-h} = 2\Pr[B=h] ,$$
	as desired.
\end{proof}

We will make use of the following version of Chernoff's bound
(c.f., e.g., \cite{AS}, Appendix A).
%multiplicative version of Chernoff's inequality.
%\begin{theo}[Chernoff's large-deviation bound]\label{fact:LD}
\begin{claim}\label{claim:Chernoff}
	Let $X_1,\ldots,X_n$ be mutually independent random variables with $X_i \in [0,1]$,
	and put $X=\sum_{i=1}^n X_i$, $\mu = \Ex[X]$.
	Then $\Pr(X \ge 6x) \le \exp(-x)$ for every $x \ge \mu/3$.
	%For every $y \ge \mu$ we have
	%$$\Pr(X-\mu \ge y) \le \exp(-y/3) \quad\text{ and in particular }\quad \Pr(X \ge 2y) \le \exp(-y/3)\;.$$
	%$$\Pr(X-\mu \ge y) \le \exp(-y/3) \quad\text{ and in particular }\quad \Pr(X \ge 6z) \le \exp(-z)\;.$$
	%In particular,
	%$$\Pr(X \ge 4\mu) \le \exp(-\mu) \;.$$
\end{claim}
%\end{theo}
\begin{proof}
	Put $y = 3x \ge \mu$. Then
	$\Pr(X \ge 6x) = \Pr(X \ge 2y) \le \Pr(X-\mu \ge y) \le \exp(-y/3) = \exp(-x),$
	where the second inequality is Chernoff's large-deviation bound (again using $y \ge \mu$).
\end{proof}

\begin{proof}[Proof of Theorem~\ref{theo:sKK-UB}]
	Let $\ell \in \N$ satisfy
	\begin{equation}\label{eq:UB-identity-KK}
	\frac{n^r}{\binom{x}{k}} \le \ell \le \frac{n}{\a^k},
	%\ell = \frac{n^r}{\binom{x}{k}} \cdot c
	%\ell := \frac{n}{\a^k} ,
	%\ell := \frac{n^r}{\binom{x}{k}} = \frac{n}{\a^k} ,
	%\frac{n}{k^2 \a^{k-1}}.
	%\ell := \frac{n^r}{2^s} = \frac{n^{C}}{(1+\a)^s}.
	\end{equation}
	which is well defined by the statement's upper bound on $\a^k n^r$.
	%$x$, using $\binom{x}{k} \ge (x/k)^k$.
	%
	%	Note that we may assume $\binom{x}{k}n \ge \binom{\a n}{k}$. 
	%	
	%Let $n,r,k,x \in \N$ with 
	%Let $k \in \N$ with $k \ge 3$, $r \ge 1$, $0 < \a \le 1$ and 
	%Let $x \in \N$ with $x \le \a^{k/3}n^{1/6}$.
	%$x \le (\a^{2k}n)^{1/6}$.
	%$x \le \a^k n^{\min\{\frac{r-1}{k},\, \frac16\}}$.
	%\a^{k/3}n^{1/6}$.
	% n^{1/6}$.
	Let $S_1,\ldots,S_\ell$ be $\ell$ independent uniformly random size-$x$ subsets of $[n]$.
	%Let $S_1,\ldots,S_\ell$ be $\ell$ independent random subsets of $[n]$, each obtained by independently retaining each element of $[n]$ with probability $x/n$,
	We let $\FF$ be the $k$-graph on $[n]$ consisting of a complete $k$-graph on each $S_j$, that is, $\FF = \big([n],\,\bigcup_{j=1}^\ell \binom{S_j}{k}\big)$.
	%
	
	We next analyze the random $k$-graph $\FF$ constructed above.
	Let $E_1$ be the event that every two sets $S_j,S_{j'}$ $(1 \le j\neq j' \le \ell)$ intersect in fewer than $t:=3r$ elements, 
	%between any two distinct sets $S_j,S_{j'}$ $(1 \le j\neq j' \le \ell)$ is of size at most  
	%Let $E_1$ be the event that the cliques $\binom{S_j}{k}$ $(1 \le j \le \ell)$  are (edge-)disjoint, 
	and let $E_2$ be the event that $\wp(\FF,\, \a n) \le 6\binom{x}{k}n$. % with $c$ an appropriate absolute constant.
	We first show that the proof would follow by proving that %with positive probability we have
	\begin{equation}\label{eq:KK-goal}
	%\Pr\Bigg(|\FF| = n^r \,\text{ and }\, F[I] \le O\bigg(\binom{x}{k}n\bigg) \Bigg) > 0.
	%\Pr\bigg(\text{the cliques } \binom{S_j}{k} \,(1 \le j \le \ell)\, \text{ are edge-disjoint,} \text{ and }\, \wp(\FF,\, \a n) \le O\bigg(\binom{x}{k}n\bigg) \bigg) > 0.
	\Pr(E_1 \text{ and } E_2) > 0 .
	\end{equation}
	To see this first note that, by construction, $\big|\binom{\FF}{i}\big| \le \sum_{j=1}^\ell \binom{|S_j|}{i} = \ell\binom{x}{i}$ for every $0 \le i \le k$.
	Furthermore, note that the event $E_1$ implies that the cliques $\binom{S_j}{k}$ are (edge-)disjoint by the statement's assumption $k\ge t$, and so
	\begin{equation}\label{eq:UB-disj}
	E_1 \text{ implies } |\FF|=\ell \binom{x}{k}.
	\end{equation}
	Therefore, (\ref{eq:KK-goal}) implies the existence of an $n$-vertex $k$-graph $\FF$ satisfying:
	\begin{itemize}
		\item $|\FF| = \ell \binom{x}{k} \ge n^r$, using~(\ref{eq:UB-disj}) and the lower bound in~(\ref{eq:UB-identity-KK}),
		\item $\wp(\FF,\, \a n) \le 6\binom{x}{k}n$, and
		\item $\big|\binom{\FF}{i}\big| \le \ell\binom{x}{i} = \frac{\binom{x}{i}}{\binom{x}{k}}|\FF|$, using~(\ref{eq:UB-disj}),
	\end{itemize}
	from which the proof immediately follows by taking an arbitrary subgraph of $\FF$ with $n^r$ edges.
	
	%	Indeed,~(\ref{eq:KK-goal}) would imply that there exists an $n$-vertex $k$-graph $\FF$ satisfying the following two properties:
	%	%the existence of an $n$-vertex $k$-graph $\FF$ with $\binom{\FF}{i} \le \ell\binom{x}{i} = \frac{\binom{x}{i}}{\binom{x}{k}}|\FF|$ for every $0 \le i \le k$ such that:
	%	\begin{itemize}
	%		\item $|\FF|=\ell \binom{x}{k}$, as the cliques $\binom{S_j}{k}$ are (edge-)disjoint by the statement's assumption $k\ge 3$,
	%		\item $\wp(\FF,\, \a n) \le O(\binom{x}{k}n)$.
	%	%	\item for every $0 \le i \le k$, 
	%	\end{itemize}
	%	%Setting $x=2k$, as we may by  the statement's assumed upper bound on $k$, would thus complete the proof.
	%	Since, by construction, $\big|\binom{\FF}{i}\big| \le \sum_{j=1}^\ell \binom{|S_j|}{i} = \ell\binom{x}{i}$ for every $0 \le i \le k$, 
	%	and using~(\ref{eq:UB-identity-KK}), we deduce th that~(\ref{eq:KK-goal}) would indeed imply the existence of a hypergraph as required by the statement.
	%	%This would indeed immediately completes the proof.
	
	To prove~(\ref{eq:KK-goal}) we first claim that
	\begin{equation}\label{eq:F-size-KK}
	\Pr(E_1) \ge \frac12 .
	%\Pr\bigg(\text{the cliques } \binom{S_j}{k} \,(1 \le j \le \ell)\, \text{ are edge-disjoint}\bigg) \ge \frac12 .
	%\Pr\big[|\FF| = n^r\big] \ge \frac12 .
	\end{equation}
	Denote by $B$ the binomial random variable with parameters $(x,p)$ where $p=x/n$. %, and put $t=3r$.
	For every $j \neq j'$ we have
	%$1 \le j \neq j' \le \ell$ we have
	%
	%For $j \neq j'$, note that the size of the intersection between $S_j,S_{j'}$ is a hypergeometric random variable of expectation $x^2/n$; 
	%Note that the size of the intersection between any two sets $S_j,S_{j'}$ with $j \neq j'$ is a binomial random variable where each element of $[n]$ is retained with probability $p=x^2/n^2$;
	%thus, denoting by $B$ the binomial random variable with parameters $(x,p:=x/n)$,
	%counting the number of elements of $S_j$ where each is independently retained with probability $p=x/n$, 
	%we have
	\begin{align*}
	\Pr\big(|S_j \cap S_{j'}| \ge t\big) &\le 2\Pr\big(B \ge t\big) \le 2\binom{x}{t}p^{t} \\
	&\le (xp)^{t} = x^{2t}/n^t = x^{6r}/n^{3r} \le n^{- 2r} \le \ell^{-2},
	\end{align*}
	%$$\Pr(|S_j \cap S_{j'}| \ge 4) \le \binom{n}{4}p^{4} \le (np)^{4} = (x^2/n)^{4} \le n^{-2}.$$
	%$$\Pr(|S_j \cap S_{j'}| \ge 10) \le \binom{n}{10}\Big(\frac{x^2}{n^2}\Big)^{10} \le $$
	where the first inequality uses Lemma~\ref{lemma:HG} using the statement's upper bound on $x$,
	%the fact that $|S_j \cap S_{j'}|$ is a sum of $|S_j|=x$ Bernoulli random variables with parameter $|S_{j'}|/n=p$ which are pairwise negatively correlated, 
	the penultimate inequality again uses the statement's upper bound on $x$,
	and the last inequality uses the upper bound in~(\ref{eq:UB-identity-KK}) together with the statement's lower bound on $\a^k n^r$.
	This implies, by taking the union bound over all $\binom{\ell}{2}$ unordered pairs $1 \le j\neq j' \le \ell$, that with probability at least $\frac12$ all set pairs $S_j,S_{j'}$ with $j \neq j'$ intersect at fewer than $3r$ elements. This proves~(\ref{eq:F-size-KK}).
	%Conditioned on the above, and recalling $k \ge 3$, we have that the cliques $\binom{S_j}{k}$ are (edge-)disjoint, proving~(\ref{eq:F-size-KK}).
	%meaning that
	%$$|\FF| = \ell\binom{x}{k} = n^r .$$
	%This proves~(\ref{eq:F-size-KK}).
	
	
	We now show that $\Pr(E_2) \ge 1/2$ (that is, that $\wp(\FF,\,\a n) \le 6\binom{x}{k}n$ except with probability smaller than $1/2$), which would prove~(\ref{eq:KK-goal}). %and completing the proof TODO:explain.
	Fix $I \sub [n]$ of size $q=\a n$.
	Note that $\FF[I] = \bigcup_{j=1}^\ell \binom{S_j \cap I}{k}$.
	For each $1 \le j \le \ell$ consider the random variable $X_j = \binom{|S_j \cap I|}{k}$, let $X=\sum_{j=1}^\ell X_j$, and note that $|\FF[I]| \le X$.
	We have
	%For every $0 \le h \le x$ 
	%we claim that $\Pr[|S_j \cap I|=h] \le 2\binom{x}{h} \a^h(1-\a)^{x-h}$; indeed,
%	
%	\begin{align*}
%	\Pr\big[|S_j \cap I|=h\big] &= 
%	\frac{\binom{q}{h}\binom{n-q}{x-h}}{\binom{n}{x}} 
%	=\frac{\binom{x}{h}\binom{n-x}{q-h}}{\binom{n}{q}}
%	= \binom{x}{h}\frac{(n-x)!}{n!}\cdot\frac{q!}{(q-h)!}\cdot\frac{(n-q)!}{(n-q-(x-h))!}\\
%	&= \binom{x}{h}\frac{\prod_{t=0}^{h-1} q-t}{\prod_{t=0}^{h-1} n-t} 
%	\cdot \frac{\prod_{t=0}^{x-h-1} n-q-t}{\prod_{t=0}^{x-h-1} n-h-t}
%	\le \binom{x}{h}\bigg(\frac{q}{n}\bigg)^h\bigg(\frac{n-q}{n-h}\bigg)^{x-h} \\
%	&\le \binom{x}{h}\bigg(\frac{q}{n}\bigg)^h\bigg(\frac{n-q}{n}\bigg)^{x-h}e^{2h(x-h)/n} \le 2\binom{x}{h} \a^h(1-\a)^{x-h},
%	\end{align*}
%	where the first inequality uses that $h \le q$ (as otherwise $\Pr[|S_j \cap I|=h]=0$ and there is nothing to prove), the second inequality uses the lower bound in Fact~\ref{fact:e}, and the third inequality uses the fact that $2h(x-h) \le x^2/2 \le n/2$
%	as $x \le \sqrt{n}$ by the statement's assumption.
	%
	%	\begin{align*}
	%	\Pr\big[|S_j \cap I|=h\big] &= 
	%	\frac{\binom{q}{h}\binom{n-q}{x-h}}{\binom{n}{x}} 
	%	=\frac{\binom{x}{h}\binom{n-x}{q-h}}{\binom{n}{q}}
	%	= \binom{x}{h}\frac{(n-x)!}{n!}\frac{q!}{(q-h)!}\frac{(n-q)!}{(n-q-(x-h))!}\\
	%	&\le \binom{x}{h}n^{-x}q^h(n-q)^{x-h} = \binom{x}{h} \a^h(1-\a)^{x-h},
	%	\end{align*}
	%	where the estimate uses the fact that for every $a,b$ with $b=o(\sqrt{a})$ we have $a!/(a-b)! \sim a^b$.
	%Therefore,
	%	\begin{align*}
	%	\Ex(X_j) &= \sum_{h=k}^x \binom{h}{k}\Pr\big[|S_j \cap I| = h\big]  = \sum_{h=k}^x \binom{h}{k}\Pr\big[|S_j \cap I| \ge h\big] - \sum_{h=k}^x \binom{h}{k}\Pr\big[|S_j \cap I| \ge h+1\big]
	%	\end{align*}
	%	\begin{align*}
	%	\Ex(X_j) &= \sum_{h=k}^x \Pr[|S_j \cap I|=h] \cdot \binom{h}{k}
	%	\sim \sum_{h=k}^x\binom{x}{h}\binom{h}{k} \a^h(1-\a)^{x-h}\\
	%	&= \sum_{h=k}^x\binom{x}{k}\binom{x-k}{h-k} \a^h(1-\a)^{x-h}
	%	= \a^k\binom{x}{k}\sum_{j=0}^{x-k}\binom{x-k}{j} \a^j(1-\a)^{(x-k)-j}
	%	= \a^k\binom{x}{k}.
	%	\end{align*}
	\begin{align*}
	\Ex(X_j) &= \sum_{h=k}^x \binom{h}{k}\Pr[|S_j \cap I|=h] 
	\le 2\sum_{h=k}^x\binom{x}{h}\binom{h}{k} \a^h(1-\a)^{x-h}\\
	&= 2\sum_{h=k}^x\binom{x}{k}\binom{x-k}{h-k} \a^h(1-\a)^{x-h}
	= 2\a^k\binom{x}{k}\sum_{j=0}^{x-k}\binom{x-k}{j} \a^j(1-\a)^{(x-k)-j}
	= 2\a^k\binom{x}{k},
	\end{align*}
	where the first inequality follows from Lemma~\ref{lemma:HG} using the statement's upper bound on $x$.
	Thus, by linearity of expectation,
	\begin{equation}\label{eq:Ex-KK}
	\Ex(X) \le \ell \cdot 2\a^k \binom{x}{k} \le 2\binom{x}{k}n,
	\end{equation}
	where the second inequality uses the upper bound in~(\ref{eq:UB-identity-KK}).
	%
	%
	%$$\Ex(X) \sim \binom{x}{k}\ell \a^k = \binom{x}{k}\frac{\a n}{k} .$$
	Since $X_j \le \binom{x}{k}$ for every $1 \le j \le \ell$, 
	we have that $X/\binom{x}{k}$ is a sum of mutually independent random variables each in $[0,1]$.
	%we apply Chernoff's large-deviation bound (Fact~\ref{fact:LD}) 
	We thus apply Claim~\ref{claim:Chernoff} on $X/\binom{x}{k}$, using the fact that $n \ge \frac13\Ex(X/\binom{x}{k})$ by~(\ref{eq:Ex-KK}), to deduce that
	%$$\Pr[X \ge 4\Ex(X)] \le \exp\bigg(-\Ex(X)/\binom{x}{k}\bigg) 
	%= \exp\big(-(1+o(1))n\big) \le o(2^{-n}),$$
	$$\Pr\Big[X \ge 6\binom{x}{k}n\Big] = \Pr\Big[X/\binom{x}{k} \ge 6n\Big] \le \exp(-n) < 2^{-n}.$$
	%where we used the fact that, by~(\ref{eq:Ex-KK}), $\Ex(X/\binom{x}{k}) \le 3n$.
	%$$\Pr[X \ge (k+1)\Ex(X)] \le \exp\Big(-k\Ex(X)/\binom{x}{k}\Big) \le \exp(-\a n) \le 2^{-H(\a)n},$$
	%where in the last inequality we used the fact that $\Ex(X)/2^{s} \sim n^C/2^{(C-1)\log n}=n$.
	Using the union bound over all $\binom{n}{\a n} \le \frac12 2^n$ choices of $I \sub [n]$ with $|I|=\a n$ we deduce 
	that, except with probability smaller than $1/2$, for every $I \sub [n]$ with $|I|=\a n$ it holds that $|\FF[I]| \le 6\binom{x}{k}n$. As mentioned before, together with~(\ref{eq:F-size-KK}) this proves~(\ref{eq:KK-goal}) and so we are done.
\end{proof}


\subsection{Traces upper bound}\label{subsec:UB-traces}

In this subsection we complete the proof of Theorem~\ref{t12} by proving the upper bound
$\trace(n,\,n^r,\,\a n) \le O(n^{\C})$, with $\C$ as in~(\ref{eq:C}), for every $r \le \sqrt{n}/\log n$ and $0 \le \a \le 1$.
A proof can be obtained from the proof of Theorem~\ref{theo:sKK-UB} with some effort. For completeness, we give here a self-contained proof.
%for every $1 \le x \le n^{1/6}$ and $0 \le a \le 1$. 

Put $x=(\C-1)\log n$.
Let $S_1,\ldots,S_\ell$ be $\ell$ independent uniformly random subsets of $[n]$, each of size $x$,
%Set, %with foresight,
%$$s = \frac{r-1}{2-\log(1+\a)}\log n ,$$%
%Note that we have the identity 
where
\begin{equation}\label{eq:UB-identity}
\frac12\ell := \frac{n^r}{2^x} = \frac{n^{\C}}{(1+\a)^x}.
%\ell := \frac{n^r}{2^s} = \frac{n^{C+1}}{(1+\a)^s}.
%\ell := \frac{n^r}{2^s} = n\Big(\frac{2}{1+\a}\Big)^s \quad \Big(= \frac{n^{C+1}}{(1+\a)^s} \Big).
\end{equation}
%$$\frac{n^C}{2^s}=n$$
%$$n^{r-1} = (\frac{4}{1+\a})^s$$
%$$2^s\ell/n = n^{r-1} = (4/1+\a)^s$$
%
Let the family $\FF \sub 2^{[n]}$ consist of the union over $j$ of 
all subsets of the set $S_j$; that is, $\FF = \bigcup_{j=1}^\ell 2^{S_j}$.
Let $E_1$ be the event that $|\FF| \ge n^r$, and let $E_2$ be the event that  $\trace(\FF,\,\a n) \le 8n^\C$.
Note that the proof would follow by showing that %with positive probability we have
%, and assuming $n$ is larger than some absolute constant, we have 
%both that $|\FF| \ge n^r$ and $\trace(\FF,\,\a n) \le O(n^C)$, which would complete the proof.
\begin{equation}\label{eq:traces-goal}
%|\FF| \ge n^r \quad\text{ and }\quad \trace(\FF,\,\a n) \le O(n^C) .
%\Pr\Big(|\FF| \ge n^r \,\text{ and }\, \trace(\FF,\,\a n) \le O(n^C) \Big) > 0.
\Pr(E_1 \text{ and } E_2) > 0.
\end{equation}
%We will henceforth assume, as we may, that $n$ is larger than some absolute constant.
First, we claim that
\begin{equation}\label{eq:F-size}
\Pr(E_1) \ge \frac12 .
%\Pr\big[|\FF| \ge n^r\big] \ge \frac12 .
\end{equation}
Put $t=3r$.
Denote by $B$ the binomial random variable with parameters $(x,p)$ where $p=x/n$. For every $j \neq j'$ we have
%Indeed, the size of the intersection between any two distinct sets $S_j,S_{j'}$ is a hypergeometric random variable of expectation $s^2/n$; 
%thus, denoting by $B$ the binomial random variable obtained by retaining each element of $S_j$ independently with probability $p=x/n$, we have, denoting %$t=3r$, that
\begin{align*}
\Pr(|S_j \cap S_{j'}| \ge t) &\le 2\Pr(B \ge t) \le 2\binom{x}{t}p^{t} \\
&\le (xp)^{t} = x^{6r}/n^{3r} \le n^{-2r} \le \ell^{-2}, 
\end{align*}
%where the first inequality uses the fact that $|S_j \cap S_{j'}|$ is a sum of $|S_j|=x$ Bernoulli random variables with parameter $|S_{j'}|/n=p$ which are pairwise negatively correlated, 
where the first inequality uses Lemma~\ref{lemma:HG} together with the fact that
\begin{equation}\label{eq:UB-x-sqrtn}
x \le r\log n \le \sqrt{n} %\quad \text{ for all large enough $n$}
\end{equation}
by the assumed upper bound on $r$, 
%the penultimate inequality uses the assumed upper bound on $x$, 
and the last inequality uses~(\ref{eq:UB-identity}).
%HERE
%$$\Pr(|S_j \cap S_{j'}| \ge 3) \le \Pr(B \ge 3) \le \binom{s}{3}p^{3} \le (sp)^{3} = s^6/n^3 \le (\a^k/n)^{2} \le \ell^{-2}.$$
%
%thus, by Chernoff's inequality together with the union bound, all pairs intersect in at most $2$ elements except with probability at most $\ell^2\exp(-4(s^2/n)^{-1}) < 1/3$.
Conditioned on the above we have, by taking the union bound over all $\binom{\ell}{2}$ pairs of sets, that
$$|\FF| \ge \ell \bigg(2^x-\binom{x}{\le t}\bigg) \ge \ell \cdot \frac12 2^x = n^r ,$$
where the second inequality uses $t = 3r \le x/2$ (for all $n$ large enough), and the equality uses~(\ref{eq:UB-identity}). This proves~(\ref{eq:F-size}).
%which assumes (as we may) that $n$ is larger than some absolute constant.

% $|\FF| \ge \ell 2^s - (1+n+\binom{n}{2}) \sim  n^r$. 
%since ..., we have $|\FF| \approx \ell 2^s = n^r$. 
%$|\FF| \approx \ell\sum_{i=1}^b {a \choose i}$.

%$$\overline{\FF} = \bigcap_{j=1}^\ell \overline{2^{S_j}}= \overline{2^{\bigcup S_j}} $$

We now show that $\Pr(E_2) \ge 1/2$ (that is, $\trace(\FF,\,\a n) \le 8n^\C$ except with probability smaller than $1/2$), thus proving~(\ref{eq:traces-goal}).
Fix $I \sub [n]$ of size $q=\a n$.
Note that 
$$\FF_I = \bigcup_{j=1}^\ell \{ S \cap I \,\vert\, S \sub S_j \} = \bigcup_{j=1}^\ell 2^{S_j \cap I}.$$
For each $1 \le j \le \ell$ consider the random variable $X_j = |2^{S_j \cap I}|$, 
%$X_j = \big|(2^{S_j})_I\big|$, 
let $X=\sum_{j=1}^\ell X_j$ and note that $|\FF_I| \le X$.
%$|\FF_I| \le \sum_{j=1}^\ell X_j$.
%Consider the random variable $\sum_{j=1}^\ell \big|{A_j \choose \le b}_I\big|$
%For every $1 \le h \le x$ 
We have
%we claim that $\Pr[|S_j \cap I|=h] \le 2\binom{x}{h} \a^h(1-\a)^{x-h}$; indeed,
%\begin{align*}
%\Pr\big[|S_j \cap I|=h\big] &= 
%\frac{\binom{q}{h}\binom{n-q}{x-h}}{\binom{n}{x}} 
%=\frac{\binom{x}{h}\binom{n-x}{q-h}}{\binom{n}{q}}
%= \binom{x}{h}\frac{(n-x)!}{n!}\cdot\frac{q!}{(q-h)!}\cdot\frac{(n-q)!}{(n-q-(x-h))!}\\
%&\le \binom{x}{h}\bigg(\frac{q}{n}\bigg)^h\bigg(\frac{n-q}{n-h}\bigg)^{x-h} \le 2\binom{x}{h}\bigg(\frac{q}{n}\bigg)^h\bigg(\frac{n-q}{n}\bigg)^{x-h} = 2\binom{x}{h} \a^h(1-\a)^{x-h},
%\end{align*}
%where for the last inequality recall $x = O(\log n)$.
%
%For every $1 \le i \le s$ we have
%$$\Pr\big[|S_j \cap I|=i\big] = 
%\frac{\binom{q}{i}\binom{n-q}{s-i}}{\binom{n}{s}} 
%=\frac{\binom{s}{i}\binom{n-s}{q-i}}{\binom{n}{q}}
%\sim \binom{s}{i} \a^i(1-\a)^{s-i}.$$
%Therefore,
\begin{align*}
\Ex(X_j) &= \sum_{h=0}^x 2^h\Pr[|S_j \cap I|=h]
\le 2\sum_{h=0}^x 2^h \binom{x}{h} \a^h (1-\a)^{x-h}\\
&= 2(1-\a)^x\sum_{h=0}^x\binom{x}{h} \Big(\frac{2\a}{1-\a}\Big)^h
= 2(1-\a)^x \Big(1 + \frac{2\a}{1-\a}\Big)^x
= 2(1+\a)^x,
\end{align*}
%Using our choice of $s$ we thus have
where the first inequality uses Lemma~\ref{lemma:HG} together with~(\ref{eq:UB-x-sqrtn}).
Thus, by linearity of expectation and by~(\ref{eq:UB-identity}),
\begin{equation}\label{eq:Expectation}
\Ex(X) \le \ell \cdot 2(1+\a)^x  = 4n^{\C} .
\end{equation}
Note that, by our choice of $x$ at the beginning of the proof,
\begin{equation}\label{eq:UB-traces-normalized}
n^\C/2^x = n^\C/2^{(\C-1)\log n} = n.
%\Ex(X)/2^x \le 4 \cdot n^C/2^{(C-1)\log n} = 4n.
\end{equation}
Since $X_j \le 2^x$ for every $1 \le j \le \ell$, 
we have that $X/2^x$ is a sum of mutually independent random variables each in $[0,1]$.
We thus apply Claim~\ref{claim:Chernoff} on $X/2^x$, using the fact that $\Ex(X/2^x) \le 4n$ by~(\ref{eq:Expectation}) and~(\ref{eq:UB-traces-normalized}), to deduce that
$$\Pr\big[X \ge 8 n^\C \big] = \Pr\big[X/2^x \ge 8n \big] \le \exp\big(-(4/3)n\big) < 2^{-n}.$$
%$$\Pr\Big[X \ge 12 \cdot  2^x n\Big] = \Pr\Big[X/2^x \ge 12n\Big] \le \exp(-2n) < 2^{-n}.$$
%where we used the fact that, by~(\ref{eq:Ex-KK}), $\Ex(X/\binom{x}{k}) \le 3n$.
%$$\Pr[X \ge (k+1)\Ex(X)] \le \exp\Big(-k\Ex(X)/\binom{x}{k}\Big) \le \exp(-\a n) \le 2^{-H(\a)n},$$
%where in the last inequality we used the fact that $\Ex(X)/2^{s} \sim n^C/2^{(C-1)\log n}=n$.
Using the union bound over all $\binom{n}{\a n} \le \frac12 2^n$ choices of $I \sub [n]$ with $|I|=\a n$ we deduce 
that, except with probability smaller than $1/2$, for every $I \sub [n]$ with $|I|=\a n$ it holds that $|\FF_I| \le 8n^\C$. As mentioned before, together with~(\ref{eq:F-size-KK}) this proves~(\ref{eq:KK-goal}) and so we are done.

%Note that $X_j \le 2^s$ for each $1 \le j \le s$. 
%Applying Chernoff's large-deviation inequality (Fact~\ref{fact:LD}) on $X/2^s$, which is a sum of mutually independent random variables each in $[0,1]$, we get
%$$\Pr[X \ge 4\Ex(X)] \le \exp(-\Ex(X)/2^s) = o(2^{-n}),$$
%%$$\Pr[X \ge c\Ex(X)] \le \exp(-\frac13(c-1)\Ex(X)/2^s) \le 2^{-n},$$
%where in the last inequality we used  the fact that, by~(\ref{eq:Expectation}),  $\Ex(X)/2^{s} \sim n^C/2^{(C-1)\log n}=n$.
%Using the union bound over all $\binom{n}{\a n} \le 2^n$ choices of $I \sub [n]$ with $|I|=\a n$ as well as~(\ref{eq:F-size}), we deduce 
%that with positive probability both $|\FF| \ge n^r$ and for every $I \sub [n]$ with $|I|=\a n$ it holds that $|\FF_I| \le 4\Ex(X) \le O(n^{C})$ (the last inequality again uses~(\ref{eq:Expectation})). This proves~(\ref{eq:traces-goal}) and so we are done.

%%%%%%%%%%%%%%%%%%%%%%%%%%%%%%%%%%%%%%%%%%%%%%%%%%%%%%%%%%%%%%%%%%%%
%\input{traces-application}



\section{Applications}\label{sec:applications}

In this section we give two easy applications of our results on traces, in geometry and in graph theory.
%, and algebra.

\subsection{Separating halfspaces using few points}
We recall the necessary definitions and the statement of this application.
Let $\HH$ be a family of halfspaces in $\R^d$, and let $P$ be a set of points in $\R^d$. We say that $P$ \emph{separates} $\HH$ if for every pair of distinct halfspaces $H_1 \neq H_2 \in \HH$ there is a point in $P$ that lies in one and outside the other. 
Given $P$ and $\HH$ such that $P$ separates $\HH$, it is interesting to ask how few points in $P$ can we choose while still separating many of the hyperplanes in $\HH$.
%we are interested in finding a small subset of $P$ that still separates a large number of halfspaces in $\HH$. 
% distinguishing them, meaning it lies in one of them and outside the other. 

\begin{prop}\label{pr:tr}
	Let $P \subset \R^d$ be a set of $n$ points and let $\HH$ be a family of $n^r$ halfspaces in $\R^d$, with $1 \le r \le n^\d$, such that $P$ separates $\HH$.  
	Then there exists a subset $P' \subseteq P$ of at most $n^{1-\d}$ points and
	a subset $\HH' \subseteq \HH$ of at least $n^{\frac{r+1}{2}(1-O(\d))}$ halfspaces such that $P'$ separates $\HH'$.
%	
%	Fix $r > 1$, let $\HH$ be a family of $n^r$ halfspaces in $\R^d$ and let $P \subset \R^d$ be a set of $n$ points such that $P$ separates $\HH$. 
%	Then there is a subset $P' \subseteq P$ of at most $\frac{n}{\log n}$\footnote{In fact, by part 2 of Corollary~\ref{co:al}, we can find a set of $n^{1-\epsilon}$ points and slightly increase the error term in the $\tilde\Omega(n^{\frac {r+1} 2})$ notation.} points and
%	a subset $\HH' \subseteq \HH$ of $\tilde{\Omega}(n^{\frac {r+1} 2})$ halfspaces such that $P'$ separates $\HH'$.
\end{prop}
\begin {proof}
Let $\FF$ be the hypergraph on $P$ with edge set $\{H \cap P \,\vert\, H \in \HH\}$. 
%Consider the $n$-vertex hypergraph $\FF = \,(P,\, \HH_{P})$. 
By assumption, for any pair of distinct halfspaces $H_1 \neq H_2 \in \mathcal H$ there is a point $p \in P$ such that $p\in H_1$ and $p\not\in H_2$, or $p \in H_2$ and $p \notin H_1$. In particular, $H_1\cap P \ne H_2 \cap P$, and therefore $|\FF| = |\HH| = n^r$. 
Applying Theorem~\ref{th:main} with $r,\a^{-1} = n^{\d}$, there exists a subset $P' \subseteq P$ of size $n^{1-\d}$ such that $|\FF_{P'}| \ge n^{\C(1-O(\d))} \ge n^{\frac{r+1}{2}(1-O(\d))}$. 
%Applying Theorem~\ref{th:main} with $\alpha = \frac 1 {\log n}$, there exists a subset $P' \subseteq P$ of size $\frac n {\log n}$ such that $|\FF_{P'}| = \tilde{\Omega}(n^{\frac {r+1} 2})$ (the choice of $\alpha$ so small is allowed by Corollary~\ref{co:al} TODO). 
Note that $\FF_{P'}=\{H \cap P' \,\vert\, H \in \HH\}$. Let $\HH' \subseteq \HH$ be obtained by assigning to each member $Q$ of $\FF_{P'}$ an arbitrary halfspace $H' \in \HH'$ with $H' \cap P' = Q$. 
By construction, $|\HH'|=|\FF_{P'}|$, hence it remains to show that $P'$ separates $\HH'$.
Again by construction, for every pair of distinct halfspaces $H_1 \neq H_2 \in \HH'$ we have $H_1 \cap P' \neq H_2 \cap P'$. 
%Indeed, otherwise, there is one member in $\FF_{P'}$ that was assigned to both $H_1$ and $H_2$, contradicting the uniqueness in the assignment. 
This means that there exists a point $p' \in P'$ such that either $p' \in H_1$ and $p' \notin H_2$, or $p' \in H_2$ and $p' \notin H_1$, thus completing the proof. 
%
%For each hyperedge $e' \in \FF_{P'}$, pick some halfspace $H_{e'}\in\mathcal H$ with $H_{e'} \cap P' = e'$, and put $\mathcal H':= \{H_{e'} \mid e'\in \FF_{P'}\}$ . It follows that the set of points in $P'$ separates the family of halfspaces $\mathcal H'$. Indeed, given $H'_1, H'_2 \in \mathcal H'$, let $e'_1, e'_2$ be the corresponding edges in $\FF_{P'}$. Clearly, $e'_1 \ne e'_2$, as these are two distinct hyperedges in $\FF_{P'}$. This implies that one $p\in P'$ belongs to $e'_1$ and not to $e'_2$ or vice versa. This implies that $p_i \in P'$ is in $H'_1$ and outside $H'_2$ or vice versa, as claimed, thus completing the proof
\end {proof}

In fact, halfspaces can be replaced in Proposition~\ref{pr:tr} by any family of subsets of $\R^d$, as long as the set of points $P$ separates them.
Indeed, in the proof of Proposition~\ref{pr:tr} we considered for each halfspace only the subset of points in $P$ that are contained in it. The condition that $P$ separates $\mathcal H$ implies that all the corresponding subsets of $P$ are distinct, and that is all that is really needed.
%
However, it does seem reasonable to expect that for ``favorable'' geometric objects, such as halfspaces, better bounds hold---as we believe is the case.

\begin{remark}
	It is worth noting that the number of halfspaces in 
Proposition~\ref{pr:tr} is necessarily polynomial 
if the dimension $d$ is constant; in fact, an exact formula is known~\cite{Ha}, 
which in particular implies that the number of halfspaces in $\R^d$ 
separated by $n$ points is at most $O(n^d)$. As mentioned above, 
the assertion in Proposition~\ref{pr:tr} holds also when we replace 
halfspaces by any family of $n^r$ subsets, for example, general 
convex sets. 
\end{remark}


%\begin{remark}
%	It is worth noting that the number of halfspaces in Proposition~\ref{pr:tr} is indeed necessarily polynomial if the dimension $d$ is constant; in fact, it can never be more than $n^{d+1}$.
%	To see this, first recall Radon's theorem on convex sets (see, e.g.,~\cite{Mat02}) which states that any set of $d + 2$ points in $\R^d$ can be partitioned into two disjoint sets whose convex hulls intersect. This easily implies (and is well known) that the set of halfspaces in $\R^d$ has VC-dimension at most $d+1$. On the other hand, by Theorem~\ref{th:sps}, if $P$ of size $n$ separates $\HH$ of size $|\HH| > \sum_{i=0}^{d+1} \binom n d$ then there exist $d+2$ points of $P$ separating a subset of size $2^{d+2}$, contradicting the fact that the VC-dimension of halfspaces in $\R^d$ is at most $d+1$. We deduce that a set of $n$ points in $\R^d$ can separate a family of at most $\sum_{i=0}^{d+1} \binom n d \le n^{d+1}$ hyperplanes, as claimed. Furthermore, it is easy to prove via induction that the bound $\sum_{i=0}^{d+1} \binom n i$ is in fact sharp. %for hyperplanes. 
%	
%	We recall that the VC-dimension of general convex bodies in $\R^d$ is unbounded, and yet the bound in Proposition~\ref{pr:tr} holds in this case too.
%	We refer the reader to~\cite{MV} for a nice exposition of known results for VC-dimension and other related geometric notions.  
%\end{remark}


\subsection{Retaining independent sets in induced subgraphs}\label{subsec:app-graphs}

An \emph{independent set} in a graph is a vertex subset that spans no edges.
%such that the induced subgraph $G[I]$ has no edges.
Given a graph $G$, one can ask how many independent sets are retained in small subgraphs of $G$. 
Using our result for traces, we easily obtain that if $G$ has $2^{n^{o(1)}}$ independent sets then it must have a subset of at most $n^{1-\d}$ vertices, with $\d$ a sufficiently small constant, retaining asymptotically more than square root of the number of independent sets.

\begin{prop}\label{prop:app-graphs}
	Let $G = (V,E)$ be an $n$-vertex graph, and assume that the number of independent sets in $G$ is $n^r$ with $1 \le r \le n^\d$.
	Then there exists a subset $V' \subseteq V$ of at most $n^{1-\d}$ vertices such that the number of independent sets in the induced subgraph $G[V']$ is at least $n^{\frac{r+1}{2}(1-O(\d))}$.
	%$\tilde{\Omega}(n^{\frac {r+1} 2})$.
\end{prop}
\begin{proof}
	Let $\FF$ be the hypergraph on $V$ whose edges are the independent sets in $G$.
	Apply Theorem~\ref{th:main} with $r,\a^{-1} = n^\d$ to deduce that there is a subset $S$ of $V(\FF)=V$ with $n^{1-\d}$ vertices such that $|\FF_S|$, the number of projections of the edges of $\FF$ onto $S$, is at least 
	$n^{\mu(1-O(\d))} \ge n^{\frac{r+1}{2}(1-O(\d))}$.
	%$\tilde{\Omega}(n^{\frac {r+1} 2})$.
	The proof follows by observing that, by definition and construction, 
	\begin{align*}
	\FF_S &=\{I \cap S \,\colon\, I \in \FF \} \\ 
	&=\{ I \cap S \,\colon\, I \text{ is an independent set of } G \}
	= \{ I \,\colon\, I \text{ is an independent set of } G[S] \} .
	\end{align*}
\end{proof}

We note that, since Theorem~\ref{th:main} is such an abstract statement, 
we could have replaced the notion of independent sets in a graph 
by any other monotone family (i.e., 
a family closed under taking subsets) of vertex subsets. 
Thus, for example, a statement analogous to 
Proposition~\ref{prop:app-graphs} holds for independent sets in 
hypergraphs, for subsets inducing a subgraph not containing some 
fixed graph $H$, and more.

%We note that, since Theorem~\ref{th:main} is such an abstract statement, we could have replaced the graph $G$ in Proposition~\ref{prop:app-graphs} by a hypergraph, and replaced the notion of independent sets by any other family of vertex subsets of $G$, such as cliques.


%\subsection{Restricting polynomials of small degree}
%
%TODO

%%%%%%%%%%%%%%%%%%%%%%%%%%%%%%%%
%\input{traces-bib}
\subsection*{Acknowledgement} We would like to thank Michael Langberg for useful discussions.

\begin{thebibliography}{99}
	\bibitem{Al}
	N. Alon, On the density of sets of vectors, {\em Discrete Math.} 46
	(1983), 199-202.
	\bibitem{AS}
	N. Alon and J. H. Spencer, {\em The Probabilistic Method, 
		Fourth Edition}, Wiley, 2016, xiv+375 pp.
	\bibitem{An}
	D. Angluin, Computational learning theory: survey and selected bibliography, 
	{\it Proc. 24th Annual ACM Symposium on Theory of Computing} 1992.
	\bibitem{BKK}
	J. Bourgain, J. Kahn and G. Kalai,
	Influential coalitions for Boolean Functions,
	arXiv 1409.3033.
	\bibitem{BR}
	B. Bollob\'as and A. J. Radcliffe,
	Defect Sauer results,
	{\it J. Combinatorial Theory Ser. A} 72 (1995), 189-208.
	\bibitem{Bo}
	J. A. Bondy, Induced subsets, {\it J. Combin. Theory Ser. B} 12
	(1972), 201-202.
	\bibitem{BG}
	B. Bukh and X. Goaoc,
	Shatter functions with polynomial growth rates,
	arXiv 1701.06632.
	\bibitem{CGN}
	O. Cheong, X. Goaoc, and C. Nicaud, Set systems and families of permutations with small traces, {\it European Journal of Combinatorics 34} (2013), 229-239.
	\bibitem{Fr}
	P. Frankl, On the trace of finite sets, {\it J. Combin. Theory Ser.
		A} 34 (1983), 41-45.
\bibitem{Ha}
E. F.  Harding, 
The number of partitions of a set of $N$ points in $k$ dimensions 
induced by hyperplanes, 
{\it Proc. Edinburgh Math. Soc.} (2) 15 1966/1967, 285--289. 

	\bibitem{Ka}
	G. O. H. Katona, A theorem on finite sets, in: {\em Theory of Graphs}
	(Erd\H{o}s, P. and Katona, G. O. H., eds.), Akad\'emiai Kiad\'o,
	Budapest.
	\bibitem{KKL}
	J. Kahn, G. Kalai and N. Linial , The influence of variables on
	Boolean functions, {\it Proc. 29th Annual Symposium on Foundations
		of Computer Science}, 1988.
	\bibitem{Kr} J. B. Kruskal, The number of simplices in a complex,
	in: {\em Mathematical Optimization Techniques}, Univ. California
	Press, Berkeley (1963), 251-278.
	\bibitem{Lovasz} L. Lov\'asz, Combinatorial Problems and Exercises,
	13.31, North-Holland, Amsterdam, 1979.
	\bibitem{Mat}
	J.\ Matou\v sek, Geometric Set Systems, {\it European Congress of
		Mathematics}, Springer (1998) 1–27.
%	\bibitem{Mat02}
%	J. Matou\v sek,
%	Lectures on Discrete Geometry, Springer Verlag, Heidelberg, 2002.
%	\bibitem{MV}
%	N.H. Mustafa, Nabil and K. Varadarajan, 
%      Epsilon-approximations and epsilon-nets, 
%      in  arXiv:1702.03676 (2017).  
\bibitem{Sa}
	N. Sauer, On the density of families of sets, 
	{\it J. Combin. Theory Ser. A} 13 (1972), 145-147.
	\bibitem{Sh}
	S. Shelah, A combinatorial problem; stability and order for
	models and theories in infinitary languages, {\it Pacific or. Math.} 41 (1972), 271- 276.
	\bibitem{VC}
	V. N. Vapnik and A. Ya. Chervonenkis,
	On the uniform convergence of relative frequencies of events to their probabilities, {\it Theory Probab. Appl.} 16 (1971), 264-280.
\end{thebibliography}

%%%%%%%%%%%%%%%%%%%%%%%%%%%%%%%%%%%%%%%%%%%%%%%%%%%%
\appendix
%\input{traces-appendix}

\section{Proof of Auxiliary Lemmas}\label{sec:aux}

For the proof of Theorem~\ref{th:skk} we use the following bound.

\begin{claim}\label{claim:binom-ratio}
	Let $x,y>0$ and $k,i,\Delta \in \N$ with $\D \le i \le k$.
	If $\binom{y}{k} \le \binom{x}{k-\D}$ then $\frac{\binom{y}{i}}{\binom{y}{k}} \ge i^{-\D} \frac{\binom{x}{i-\D}}{\binom{x}{k-\D}}$.
\end{claim}
\begin{proof}
	First, note that
	\begin{equation}\label{eq:binomial-frac-bound}
	\frac{\binom{z}{a}}{\binom{z}{b}} = 
	%\frac{b!}{a!}\prod_{j=a}^{b-1}\frac{1}{z-j}
	\prod_{j=a+1}^{b} \frac{j}{x-j+1} 
	\le b^{b-a}
	\end{equation}
	for all $a \le b \le z$ with $a,b \in \N$. 
	Second, observe that $\frac{j}{x-j+1}$ is:
	\begin{enumerate}
		\item monotone decreasing in $x$,
		\item monotone increasing in $j$.
	\end{enumerate}
	Now, if $y \le x$ then $\frac{\binom{y}{i}}{\binom{y}{k}} \ge \frac{\binom{y}{i-\D}}{\binom{y}{k-\D}} \ge \frac{\binom{x}{i-\D}}{\binom{x}{k-\D}}$ using~(\ref{eq:binomial-frac-bound}) together with items~1 and~2, respectively,
	%the decreasing monotonicity of the functions $z \mapsto \binom{y}{a-z}/\binom{y}{b-z}$ (for $z \le a \le b \le y$ with $z,a,b \in \N$) and $z \mapsto \frac{\binom{z}{a}}{\binom{z}{b}}$ (for $a \le b \le z$ with $a,b \in \N$),
	whereas if $y \ge x$ then
	$$\frac{\binom{y}{i}}{\binom{y}{k}} 
	\ge \frac{\binom{x}{i}}{\binom{x}{k-\D}} 
	= \frac{\binom{x}{i}}{\binom{x}{i-\D}} \cdot \frac{\binom{x}{i-\D}}{\binom{x}{k-\D}}
	\ge \frac{1}{i^{\D}}\cdot \frac{\binom{x}{i-\D}}{\binom{x}{k-\D}},$$
	where the first inequality follows from the statement's assumption,
	and the last inequality uses~(\ref{eq:binomial-frac-bound}).
%	fact that
%	$\frac{\binom{x}{a}}{\binom{x}{b}} = 
%	%\frac{b!}{a!}\prod_{j=a}^{b-1}\frac{1}{z-j}
%	\prod_{j=a+1}^{b} \frac{j}{x-j+1} 
%	\le b^{b-a}$ for all $a \le b \le x$.
\end{proof}

%====
%
%
%\begin{claim}\label{claim:binom-ratio}
%	Let $x,y>0$ and $k,i,\Delta \in \N$.
%	If $\binom{y}{k+\D} \le \binom{x}{k}$ then $\frac{\binom{y}{i+\D}}{\binom{y}{k+\D}} \ge (i+\D)^{-\D} \frac{\binom{x}{i}}{\binom{x}{k}}$.
%\end{claim}
%\begin{proof}
%	If $y \le x$ then $\frac{\binom{y}{i+\D}}{\binom{y}{k+\D}} \ge \frac{\binom{y}{i}}{\binom{y}{k}} \ge \frac{\binom{x}{i}}{\binom{x}{k}}$ using, respectively, the decreasing monotonicity of the functions $z \mapsto \binom{y}{a-z}/\binom{y}{b-z}$ (with $z \le a \le b \le y$) and $z \mapsto \frac{\binom{z}{a}}{\binom{z}{b}}$ (with $a \le b \le z$),
%	whereas if $y \ge x$ then, by the statement's assumption,
%	$$\frac{\binom{y}{i+\D}}{\binom{y}{k+\D}} 
%	\ge \frac{\binom{x}{i+\D}}{\binom{x}{k}} 
%	= \frac{\binom{x}{i+\D}}{\binom{x}{i}} \cdot \frac{\binom{x}{i}}{\binom{x}{k}}
%	\ge \frac{1}{(i+\D)^{\D}}\cdot \frac{\binom{x}{i}}{\binom{x}{k}},$$
%	where the penultimate inequality uses the bound
%	$\frac{\binom{z}{a}}{\binom{z}{b}} = 
%	\frac{b!}{a!}\prod_{j=a}^{b-1}\frac{1}{z-j}
%	%\prod_{j=a+1}^{b} \frac{j}{z-j+1} 
%	\le b^{b-a}$ for all $a \le b \le z$.
%\end{proof}


For the proof of Lemma~\ref{lemma:113} we use the following claim, which follows easily from Newton's generalized binomial theorem.
\begin{claim}\label{claim:Newton}
	For every real $x > 0$ we have $2^{x-1} < \sum_{i=0}^{\floor{x}} \binom{x}{i} \le 2^x$.%\footnote{Newton's generalized binomial formula gives $2^x=\sum_{i=0}^\infty \binom{x}{i} $, and it is easy to verify that the sum of every two consecutive terms (one positive and the other negative) after the first $\floor x$ terms is positive.}
	%\footnote{In Newton's genearlized binomial formula for $2^x$, it is easy to verify that the sum of every two consecutive terms (one positive and the other negative) after the first $\floor x$ terms is positive.}
\end{claim}
\begin{proof}
	We bound the summation in the statement from above and from below as follows;
	$$2^{x-1} < 2^{\floor{x}} \le \sum_{i=0}^{\floor{x}} \binom{\floor{x}}{i} 
	\le \sum_{i=0}^{\floor{x}} \binom{x}{i} 
	\le \sum_{i=0}^\infty \binom{x}{i} = (1+1)^x = 2^x ,$$
	where the last inequality follows from
	\begin{align*}
	\sum_{i=0}^\infty \binom{x}{i} - \sum_{i=0}^{\floor{x}} \binom{x}{i} 
	&= \sum_{j} \bigg(\binom{x}{j}+\binom{x}{j+1}\bigg) 
	%= \sum_{i=n+1,n+3,\ldots}^{\infty} \binom{x}{i+1}\bigg(1+\frac{i+1}{x-i}\bigg)$$
	= \sum_{j} \binom{x+1}{j+1}\\
	&= \sum_{j} \bigg(\frac{(x+1)x\cdots(x-\floor{x})}{(j+1)!} \prod_{k=\floor{x}+1}^{j-1} (x-k) \bigg) \ge 0
	\end{align*}
	with $j$ running over the values $\floor{x}+1,\floor{x}+3,\ldots$, 
	%= \sum_{i=n+1,n+3,\ldots}^{\infty} \binom{x}{i}\bigg(\frac{x+1}{i+1}\bigg)$$
	%= \frac{x(x-1)\cdots(x-i+1)}{i!}$$
	and where the first equality follows from Newton's generalized binomial formula.
\end{proof}

\begin{proof}[Proof of Lemma~\ref{lemma:113}]
	For real $z \ge 0$ we henceforth abbreviate $\binom{x}{\le z} = \sum_{i=0}^{\floor{z}} \binom{x}{i}$.
	If $k \ge \lceil \frac x 2 \rceil$ we are done since
	\begin{align*}
	2\sum_{i=0}^k \binom{x}{i}\g^i &\ge 2\sum_{i=0}^{\floor{\frac{x}{2}}} \binom{\floor{x}}{i}\g^i
	= \sum_{i=0}^{\floor{\frac{x}{2}}} \bigg(\binom{\floor{x}}{i}+\binom{\floor{x}}{\floor{x}-i}\bigg)\g^i
	\ge \sum_{i=0}^{\floor{\frac{x}{2}}} \bigg( \binom{\floor{x}}{i}\g^i+\binom{\floor{x}}{\floor{x}-i}\g^{\floor{x}-i} \bigg)\\
	&\ge \sum_{i=0}^{\floor{x}} \binom{\floor{x}}{i}\g^i = (1+\g)^{\floor{x}}
	\ge \frac12 (1+\g)^x \ge \frac12\binom{x}{\le x}^{\log(1+\g)}  \ge \frac12\binom{x}{\le k}^{\log(1+\g)} ,
	\end{align*}
	where the penultimate inequality follows from Claim~\ref{claim:Newton},
	and dividing over by $2$ gives the desired result.
	%	which implies that
	%	$$\sum_{i=0}^k \binom{x}{i}\g^i \ge \frac12 (1+\g)^x \ge \frac12\binom{x}{\le x}^{\log(1+\g)}  \ge \frac12\binom{x}{\le k}^{\log(1+\g)},$$
	%	where the second inequality follows from Fact~\ref{fact:Newton}.
	
	We thus assume $k \le \ceil{\frac{x}{2}}-1$ for the remainder of the proof.
	Put
	$$y=\log \binom{x}{\le k} \quad\text{ and }\quad K=\floor{y}+1.$$
	We have $0\le y \le x-1$, which follows using the fact that for $i \le \floor{\frac{x}{2}}$ we have $\binom{x}{i} \le \binom{x}{\floor{x}-i}$ as follows;
	\begin{align*}
	1\le 2\binom{x}{\le k} \le 2\binom{x}{\le\ceil{\frac{x}{2}}-1} \le \sum_{i=0}^{\ceil{\frac{x}{2}}-1} \bigg( \binom{x}{i} + \binom{x}{\floor{x}-i} \bigg) \le
	\binom{x}{\le x} \le 2^x,
	\end{align*}
	where the first inequality follows since $1 \le x$, the second inequality uses our assumption $k \le \ceil{\frac{x}{2}}-1$, and the last inequality uses Claim~\ref{claim:Newton}. 
	Furthermore, we have $K \ge k$, since otherwise $k \ge K+1$ and thus
	$$\binom{x}{\le k} \ge \binom{x}{\le K+1} \ge \binom{y+1}{\le K+1} = \binom{y+1}{\le \floor{y}+2} \ge \binom{y+1}{\le \floor{y}+1} > 2^y,$$
	a contradiction, where the second and third inequalities used the fact that for a positive number $t$, $\binom {t} {\lfloor t \rfloor + 1}$ is non-negative (and $K = \lfloor y \rfloor + 1$), and the last inequality uses Claim~\ref{claim:Newton}. 
	
	Let $a_0,\ldots,a_k$ and $b_0,\ldots,b_{K}$ be given by %$K=\floor{y}+1$ and
	$$a_i = \binom{x}{i}\text{ for $0 \le i \le k$,}\qquad b_i = \binom{y}{i} \text{ for $0 \le i \le K-1$, and $b_K=\sum_{i=0}^k a_i - \sum_{i=0}^{K-1} b_i$}.$$
	We have that $b_K \ge 0$ since
	$$\sum_{i=0}^{K-1} b_i = \binom{y}{\le y} \le 2^y = \binom{x}{\le k} = \sum_{i=0}^k a_i,$$
	where the inequality follows from Claim~\ref{claim:Newton}.
	%Moreover, again using Fact~\ref{fact:Newton}, we have $x \ge \log\binom{x}{\le x} \ge y$.
	Summarizing the properties of the sequences $(a_i)_{i=0}^k$ and $(b_i)_{i=0}^K$, we have $\sum_{i=0}^k a_i = \sum_{i=0}^{K} b_i$, and for every $0 \le i \le k-1$ we have $a_i \ge b_i$ (recall $x > y \ge 0$). 
	Denote $\D_i = a_i-b_i$ $(\ge 0)$ for every $0 \le i \le k-1$.
	We have 
	\begin{align*}
	\sum_{i=0}^{\floor y} \binom{y}{i}\g^i &\le \sum_{i=0}^K b_i \g^i = \sum_{i=0}^{k-1} (a_i-\D_i)\g^i + \sum_{i=k}^K b_i\g^i\\ 
	&\le \sum_{i=0}^{k-1} (a_i-\D_i)\g^i + \Big(\sum_{i=k}^K b_i\Big)\g^k
	= \sum_{i=0}^{k-1} (a_i-\D_i)\g^i + \Big(\sum_{i=0}^{k-1} \D_i + a_k\Big)\g^k\\
	&\le \sum_{i=0}^{k-1} a_i\g^i -\sum_{i=0}^{k-1}\D_i\g^k + \Big(\sum_{i=0}^{k-1} \D_i + a_k\Big)\g^k
	= \sum_{i=0}^k a_i \g^i = \sum_{i=0}^k \binom{x}{i} \g^i.
	\end{align*}
	We thus showed that
	$$\sum_{i=0}^k \binom{x}{i}\g^i \ge 
\sum_{i=0}^{\floor y} \binom{y}{i} \g^i \geq
	\sum_{i=0}^{\floor{y}} \binom{\floor{y}}{i} \g^i = (1+\g)^{\floor{y}}
	\geq  \frac12(1+\g)^{y} = \frac12\binom{x}{\le k}^{\log(1+\g)},$$
	which completes the proof.
\end{proof}


\end{document}
