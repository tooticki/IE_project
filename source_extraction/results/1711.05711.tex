%%%%%%%%%%%%%%%%%%%%%%%%%%%%%%%%%%%%%%%%%%%%%%%%%%%%%%%%%%%%%%%%%%%%%%%%%
%
% Author: Jaroslaw Mederski
%
%   Title: Nonradial solutions of nonlinear scalar field equations
%
%          Version of August 20, 2018
%
%%%%%%%%%%%%%%%%%%%%%%%%%%%%%%%%%%%%%%%%%%%%%%%%%%%%%%%%%%%%%%%%%%%%%%%%%
\documentclass[12pt,leqno,twoside]{amsart}
\usepackage{amssymb,amsmath,amsthm,soul,color}
\usepackage{t1enc}
\usepackage[cp1250]{inputenc}
\usepackage{a4,indentfirst,latexsym}
\usepackage{graphics}
%\usepackage[mathscr]{eucal}
\usepackage{mathrsfs}
\usepackage{cite,enumitem,graphicx}
\usepackage[colorlinks=true,urlcolor=blue,
citecolor=red,linkcolor=blue,linktocpage,pdfpagelabels,
\usepackage[english]{babel}
%\usepackage[scrtime]{prelim2e}
\usepackage[left=2.61cm,right=2.61cm,top=2.72cm,bottom=2.72cm]{geometry}
%\usepackage{refcheck}
%\usepackage{showkeys}
\usepackage[metapost]{mfpic}
%\opengraphsfile{myfigs}
%\usepackage[hyperpageref]{backref}
%\usepackage[colorinlistoftodos]{todonotes}
\usepackage[normalem]{ulem}

%\usepackage[small,nohug,heads=curlyvee]{diagrams}
%\diagramstyle[labelstyle=\scriptstyle]
%\newarrow {Multi} ----o
%\newarrow {To} ----{->}


\linespread{1.1}
%\baselineskip=20pt


\parskip 4mm
\parindent 7mm \voffset -7mm
\hoffset -2mm \textwidth 168mm \textheight 225mm \oddsidemargin
0mm\evensidemargin 0mm\footnotesep 3mm \hbadness 10000



%EXTRACTING
 \usepackage{./extract}
%ENDEXTRACTING
\newtheorem{Th}{Theorem}[section]
\newtheorem{Prop}[Th]{Proposition}
\newtheorem{Lem}[Th]{Lemma}
\newtheorem{Cor}[Th]{Corollary}
\newtheorem{Def}[Th]{Definition}
\newtheorem{Rem}[Th]{Remark}
\newtheorem{Ex}[Th]{Example}
\newtheorem{Ass}[Th]{Assumption}
\newtheorem{Que}[Th]{Question}

\newenvironment{altproof}[1]
{\noindent%\addvspace{0.3cm}
	{\em Proof of {#1}}.}
{\nopagebreak\mbox{}\hfill $\Box$\par\addvspace{0.5cm}}

\newcommand{\wt}{\widetilde}
\newcommand{\mf}{\mathfrak}
\newcommand{\vp}{\varphi}
\newcommand{\Vp}{\Phi}
\newcommand{\eps}{\varepsilon}
\newcommand{\ov}{\overline}
\def\li{\mathop{\mathrm{liminf}\,}}
\def\div{\mathop{\mathrm{div}\,}}
\def\ls{\mathop{\mathrm{limsup}\,}}
\def\Li{\mathop{\mathrm{Liminf}\,}}
\def\Ls{\mathop{\mathrm{Limsup}\,}}
\def\deg{\mathrm{deg}}
\def\span{\mathrm{span}}
\def\ind{\mathrm{ind}}
\def\supp{\mathrm{supp}}
\def\lan{\langle}
\def\ran{\rangle}
\def\conv{\mathrm{conv}\,}
\def\Dom{\mathrm{Dom}}
\def\id{\mathrm{id}}
%\def\H{\check{H}}
\def\O{\mathrm{O}}
\def\Q{\mathbb{Q}}
\def\Z{\mathbb{Z}}
%\def\longmulti{-\!\!\!\multi}
%\def\longto{-\!\!\!\to}
\def\d{\diamond}
\def\bd{\mathrm{bd}\,}
\def\multi{\multimap}
\def\multito{\mapstochar\multimap}
\def\N{\mathbb{N}}
\def\R{\mathbb{R}}
\def\Gr{\mathrm{Gr}\,}
\def\Ind{\mathrm{Ind}}
\def\ind{\mathrm{ind}}
\def\inr{\mathrm{int}}
\def\curl{\mathrm{curl}}
\def\dim{\mathrm{dim}}
\def\codim{\mathrm{codim}}
\def\defdim{\mathrm{defdim}}
\def\diam{\mathrm{diam}}
\def\Gr{\mathrm{Gr}}
\def\st{\mathrm{st}}
\def\cl{\mathrm{cl\,}}
\def\Fix{\mathrm{Fix}}
\def\F{{\cal F}}

\def\P{\mathcal P} %P family
\def\Uc{\mathscr{U}} %U cover
\def\U{\mathcal{U}} %U neighborhood of graph
\def\Vc{\mathscr{V}}
\def\V{\mathcal{V}}
\def\E{\mathcal{E}}
\def\J{\mathcal{J}}
\def\G{\mathcal{G}}
\def\Wc{\mathscr{W}}
\def\W{\mathcal{W}}
\def\D{\mathcal{D}}
\def\C{\mathbb{C}}
\def\T{\mathcal{T}}
\def\Dc{\mathscr{Dc}}
\def\A{\mathcal{A}}

\newcommand{\cA}{{\mathcal A}}
\newcommand{\cB}{{\mathcal B}}
\newcommand{\cC}{{\mathcal C}}
\newcommand{\cD}{{\mathcal D}}
\newcommand{\cE}{{\mathcal E}}
\newcommand{\cF}{{\mathcal F}}
\newcommand{\cG}{{\mathcal G}}
\newcommand{\cH}{{\mathcal H}}
\newcommand{\cI}{{\mathcal I}}
\newcommand{\cJ}{{\mathcal J}}
\newcommand{\cK}{{\mathcal K}}
\newcommand{\cL}{{\mathcal L}}
\newcommand{\cM}{{\mathcal M}}
\newcommand{\cN}{{\mathcal N}}
\newcommand{\cO}{{\mathcal O}}
\newcommand{\cP}{{\mathcal P}}
\newcommand{\cQ}{{\mathcal Q}}
\newcommand{\cR}{{\mathcal R}}
\newcommand{\cS}{{\mathcal S}}
\newcommand{\cT}{{\mathcal T}}
\newcommand{\cU}{{\mathcal U}}
\newcommand{\cV}{{\mathcal V}}
\newcommand{\cW}{{\mathcal W}}
\newcommand{\cX}{{\mathcal X}}
\newcommand{\cY}{{\mathcal Y}}
\newcommand{\cZ}{{\mathcal Z}}

\newcommand{\fA}{{\mathfrak A}}
\newcommand{\fB}{{\mathfrak B}}
\newcommand{\fC}{{\mathfrak C}}
\newcommand{\fD}{{\mathfrak D}}
\newcommand{\fE}{{\mathfrak E}}
\newcommand{\fF}{{\mathfrak F}}
\newcommand{\fG}{{\mathfrak G}}
\newcommand{\fH}{{\mathfrak H}}
\newcommand{\fI}{{\mathfrak I}}
\newcommand{\fJ}{{\mathfrak J}}
\newcommand{\fK}{{\mathfrak K}}
\newcommand{\fL}{{\mathfrak L}}
\newcommand{\fM}{{\mathfrak M}}
\newcommand{\fN}{{\mathfrak N}}
\newcommand{\fO}{{\mathfrak O}}
\newcommand{\fP}{{\mathfrak P}}
\newcommand{\fQ}{{\mathfrak Q}}
\newcommand{\fR}{{\mathfrak R}}
\newcommand{\fS}{{\mathfrak S}}
\newcommand{\fT}{{\mathfrak T}}
\newcommand{\fU}{{\mathfrak U}}
\newcommand{\fV}{{\mathfrak V}}
\newcommand{\fW}{{\mathfrak W}}
\newcommand{\fX}{{\mathfrak X}}
\newcommand{\fY}{{\mathfrak Y}}
\newcommand{\fZ}{{\mathfrak Z}}
\renewcommand{\dim}{{\rm dim}\,}
\newcommand{\vphi}{\varphi}
%\newcommand{\eps}{\varepsilon}
\newcommand{\al}{\alpha}
\newcommand{\be}{\beta}
\newcommand{\ga}{\gamma}
\newcommand{\de}{\delta}
\newcommand{\la}{\lambda}
\newcommand{\De}{\Delta}
\newcommand{\Ga}{\Gamma}
\newcommand{\Om}{\Omega}

\def\curlop{\nabla\times}
\newcommand{\weakto}{\rightharpoonup}
\newcommand{\pa}{\partial}
\def\id{\mathrm{id}}
\def\ran{\mathrm{range}}
\newcommand{\tX}{\widetilde{X}}
\newcommand{\tY}{\widetilde{Y}}
\newcommand{\tu}{\widetilde{u}}
\newcommand{\tv}{\widetilde{v}}
\newcommand{\tcV}{\widetilde{\cV}}

\newcommand{\wh}{\widehat}
\newcommand{\wti}{\widetilde}
\newcommand{\cTto}{\stackrel{\cT}{\longrightarrow}}

\numberwithin{equation}{section}


\begin{document}
	\title{Nonradial solutions of nonlinear scalar field equations}
	\author[J. Mederski]{Jaros\l aw Mederski}
	\address[J. Mederski]{\newline\indent 
		Institute of Mathematics,
		\newline\indent
		Polish Academy of Sciences,
		\newline\indent 
		ul. \'Sniadeckich 8, 00-956
		Warszawa, Poland
		\newline\indent
		and
		\newline\indent
		Faculty of Mathematics and Computer Science,
		\newline\indent 
		Nicolaus Copernicus University,
		\newline\indent
		ul. Chopina 12/18, 87-100 Toru\'n, Poland}
	\email{\href{mailto:jmederski@impan.pl}{jmederski@impan.pl}}
	%\date{}
	%\date{\today}
	\maketitle
	
	\pagestyle{myheadings} \markboth{\underline{J. Mederski}}{
		\underline{Nonradial solutions of nonlinear scalar field equations}}




\begin{abstract}
We prove new results concerning the nonlinear scalar field equation
\begin{equation*}
\left\{
\begin{array}{ll}
-\Delta u  = g(u)&\quad \hbox{in }\mathbb{R}^N,\; N\geq 3,\\
u\in H^1(\mathbb{R}^N)&
\end{array}
\right.
\end{equation*}
with a nonlinearity $g$ satisfying the general assumptions due to Berestycki and Lions. In particular, we find at least one nonradial solution for any $N\geq 4$ minimizing the energy functional on the Pohozaev constraint. If in addition $N\neq 5$, then there are infinitely many nonradial solutions. The results give a partial answer to an open question posed by Berestycki and Lions in \cite{BerLions,BerLionsII}. 
Moreover, we build a critical point theory on a topological manifold, which enables us to solve the above equation as well as to treat new elliptic problems.
\end{abstract}

{\bf MSC 2010:} Primary: 35J20, 58E05

{\bf Key words:} Nonlinear scalar field equations, critical point theory, nonradial solutions, concentration compactness, profile decomposition, Lions lemma, Pohozaev manifold.





\section*{Introduction}
\setcounter{section}{1}


%\section{Main results}\label{sec:results}


We investigate the nonlinear scalar field equation
\begin{equation}
\label{eq}
\left\{
\begin{array}{ll}
-\Delta u  = g(u)&\quad \hbox{in }\R^N,\; N\geq 3,\\
u\in H^1(\R^N)&%\quad\hbox{as }|x|\to\infty.
\end{array}
\right.
\end{equation}
under the following general assumptions introduced by Berestycki and Lions  in their fundamental papers \cite{BerLions,BerLionsII}:
\begin{itemize}
	\item[(g0)] $g:\R \to \R$ is continuous and odd,
	\item[(g1)] $-\infty<\liminf_{s\to 0}g(s)/s\leq\limsup_{s\to 0}g(s)/s=-m<0$,
	\item[(g2)] $-\infty\leq \limsup_{s\to \infty}g(s)/s^{2^*-1}\leq 0$, where $2^*=\frac{2N}{N-2}$,
	\item[(g3)] There exists  $\xi_0>0$ such that $G(\xi_0)>0$, where 
	$$G(s)=\int_0^s g(t)\, dt\quad\hbox{for }s\in\R.$$
\end{itemize}

\newpage
Recall that the existence of a least energy solution $u\in H^1(\R^N)$, which  is positive, spheri\-cally symmetric (radial) and decreasing in $r=|x|$ is established in \cite{BerLions} and the existence of infinitely many radial solutions but not necessarily positive are provided in \cite{BerLionsII}.
Moreover Jeanjean and Tanaka \cite{JeanjeanTanaka} showed that $J(u)=\inf_{\cM} J$, where $\cM$ stands for the {\em Pohozaev manifold} defined below and $J$ is the energy functional associated with \eqref{eq}; see  \eqref{def:Poh} and \eqref{eq:action}.\\
\indent Firstly, the aim of this paper is to answer to the open problem \cite{BerLionsII}[Section 10.8] concerning the existence and multiplicity of nonradial solutions of \eqref{eq} for dimensions $N\geq 4$ under the almost optimal assumptions (g0)--(g3). Secondly, we present a new variational approach based on a critical point theory built on the Pohozaev manifold. Without the radial symmetry one has to deal with the lack of compactness issues and we present a concentration-compactness approach in the spirit of Lions \cite{Lions82,Lions84} together with profile decompositions in the spirit of G\'erard \cite{Gerard} and Nawa \cite{Nawa} adopted to a general nonlinearity satisfying (g0)--(g3); see Theorem \ref{ThGerard}. Using these techniques we provide a new proof of the following result. 

\begin{Th}[\hspace{-0.1mm}\cite{BerLions,JeanjeanTanaka}]\label{ThMain1}
	There is a solution  $u\in \cM$ of \eqref{eq} such that 
	$J(u)=\inf_{\cM} J>0$,
	where 
	\begin{equation}\label{def:Poh}
	\cM=\Big\{u\in H^1(\R^N)\setminus\{0\}: \int_{\R^N}|\nabla u|^2\,dx=2^*\int_{\R^N}G(u)\, dx\Big\}.
	\end{equation}
\end{Th}

Moreover we find nonradial solutions of  \eqref{eq} provided that $N\geq 4$. Indeed, let us fix $\tau\in\cO(N)$ such that $\tau(x_1,x_2,x_3)=(x_2,x_1,x_3)$ for $x_1,x_2\in\R^m$ and $x_3\in\R^{N-2m}$, where $x=(x_1,x_2,x_3)\in\R^N=\R^m\times\R^m\times \R^{N-2m}$ and $2\leq m\leq N/2$.
We define
\begin{equation}\label{eq:DefOfX}
X_{\tau}:=\big\{u\in H^1(\R^N): u(x)=-u(\tau x)\;\hbox{ for all }x\in\R^N\big\}.
\end{equation}
Clearly, if $u\in X_\tau$ is radial, i.e. $u(x)=u(\rho x)$ for any $\rho \in\cO(N)$, then $u=0$. Hence $X_\tau$ does not contain nontrivial radial functions. Then $\cO_1:=\cO(m)\times \cO(m)\times\id \subset \cO(N)$ acts isometrically on $H^1(\R^N)$ and let $H^1_{\cO_1}(\R^N)$ denote the subspace of invariant functions with respect to $\cO_1$.\\
\indent Our first main result reads as follows.
\begin{Th}\label{ThMain2}
If $N\geq 4$, then there is a solution  $u\in \cM\cap X_\tau\cap H^1_{\cO_1}(\R^N)$ of \eqref{eq} such that 
	\begin{equation}\label{eq:thmain2}
	J(u)=\inf_{\cM\cap X_\tau\cap H^1_{\cO_1}(\R^N)}J\geq 2\inf_{\cM} J.
	\end{equation}
\end{Th}

\indent If in addition $N\neq 5$, then we may assume that $N-2m\neq 1$ and let us consider $\cO_2:=\cO(m)\times \cO(m)\times\cO(N-2m)\subset \cO(N)$ acting isometrically on $H^1(\R^N)$ with the subspace of invariant function denoted by $H^1_{\cO_2}(\R^N)$.
\begin{Th}\label{ThMain3}
If $N\geq 4$ and $N\neq 5$,  then the following statements hold.\\
(a) There is a solution  $u\in \cM\cap X_\tau\cap H^1_{\cO_2}(\R^N)$ of \eqref{eq} such that 
\begin{equation}\label{eq:thmain3}
J(u)=\inf_{\cM\cap X_\tau\cap H^1_{\cO_2}(\R^N)}J\geq \inf_{\cM\cap X_\tau\cap H^1_{\cO_1}(\R^N)}J.
\end{equation}
(b) 
%If in addition the following condition is satisfied
%\begin{itemize}
%	\item[$(g1)'$] $-\infty<\lim_{s\to 0}g(s)/s=-m<0$,
%\end{itemize}
%then 
There is an infinite sequence of distinct solutions $(u_n)\subset \cM\cap X_\tau\cap H^1_{\cO_2}(\R^N)$ of \eqref{eq}.
\end{Th}


Note that the associated energy functional $J:H^1(\R^N)\to\R$ is given by
\begin{equation}\label{eq:action}
J(u)=\frac12\int_{\R^N} |\nabla u|^2\,dx - \int_{\R^N} G(u)\, dx,
\end{equation}
is of class $\cC^1$ and has the mountain pass geometry \cite{JeanjeanTanaka}. Our problem is modelled in $\R^N$, so that we have deal with the lack of compactness of Palais-Smale sequences. In the classi\-cal approach \cite{BerLions,BerLionsII} the compactness properties can be obtained by considering only radial functions $H^1_{\cO(N)}(\R^N)$ in the spirit of Strauss \cite{Strauss} due to $\cO(N)$-invariance of $J$. In a nonradial case, however, for instance in $H^1(\R^N)$, $X_\tau\cap H^1_{\cO_1}(\R^N)$ or in $X_\tau\cap H^1_{\cO_2}(\R^N)$, the crucial Radial Lemma \cite{BerLions}[Lemma A.II]  is no longer available and an application of the compactness lemma of Strauss \cite{BerLions}[Lemma A.I] is impossible. As usual, one needs to analyse the lack of compactness of Palais-Smale sequences by means of  a concentration-compactness argument of Lions \cite{Lions84}.  %To the best of our knowledge there is no such concentration-compactness analysis for variational problems involving general nonlinearities satisfying (g0)-(g3) in the literature. 
The main difficulty concerning the concentration-compactness analysis is that, in general, $g(s)$ has not subcritical growth of order $s^{p-1}$ for large $s$ with $2<p<2^*$ and $g$ does not satisfy an Ambrosetti-Rabinowitz-type condition \cite{AR}, or any monotonicity assumption. In the present paper we show how to deal with the lack of compactness having the general nonlinearity $g$ and our argument requires a deeper analysis of profiles of bounded sequences in $H^1(\R^N)$; see Theorem \ref{ThGerard} below.\\
\indent Beside the lack of compactness difficulties, it is not clear how to treat \eqref{eq} by means of the standard variational methods. Although $J$ has the classical mountain pass geometry \cite{JeanjeanTanaka}, we do not know whether Palais-Smale sequences of $J$ are bounded. To overcome this difficulty in the radial case in \cite{BerLions,BerLionsII}, the authors considered the following constrained problems: the minimization of $u\mapsto \int_{\R^N}|\nabla u|^2\, dx$ on 
$$\Big\{u\in H^1_{\cO(N)}(\R^N): \int_{\R^N}G(u)\, dx=1\Big\}$$ and a critical point theory of the functional $u\mapsto \int_{\R^N} G(u)\,dx$ on 
$$\Big\{u\in H^1_{\cO(N)}(\R^N): \int_{\R^N}|\nabla u|^2\, dx=1\Big\}.$$ Both approaches require the compactness properties and the scaling invariance of the equation \eqref{eq} with application of Lagrange multipliers. Another method in the radial case in \cite{Hirata} is based on the Mountain Pass Theorem for an extended functional in the spirit of Jeanjean \cite{Jeanjean}.  Let us mention that a direct minimization method on the Pohozaev manifold in $H^1_{\cO(N)}(\R^N)$ is due to  Shatah \cite{Shatah}, who studied a nonlinear Klein-Gordon equation with a general nonlinearity. Again, the radial symmetry and the Strauss lemma played an important role in these works.\\
\indent In this paper we provide a new constrained approach which allows to deal with noncompact problems and can be described in an abstract and transparent way for future applications; see Section \ref{sec:criticaltheory} for details. Let us briefly sketch our approach. Recall that if $u\in H^1(\R^N)$ is a critical point of $J$, then
$u\in W^{2,q}_{loc}(\R^N)$ for any $q<\infty$ and $u$ satisfies the {\em Pohozaev identity}, i.e. $M(u)=0$, where
$$M(u):=\int_{\R^N}|\nabla u|^2\,dx-2^*\int_{\R^N}G(u)\, dx.$$
Observe that $M:H^1(\R^N)\to\R$ is of class $\cC^1$ and but, in general, $M'$ is not locally Lipschitz and 
$$\cM=\{u\in H^1(\R^N)\setminus\{0\}: M(u)=0\}$$
need not be of class $\cC^{1,1}$. Hence it seems to be impossible to use any critical point theory based on the deformation lemma involving a Cauchy problem directly on $\cM$. Our crucial observation is that $\cM$ is a topological manifold and there is a homeomorphism $m:\cU\to\cM$ such that
$$\cU:=\Big\{u\in H^1(\R^N): \int_{\R^N} |\nabla u|^2\, dx=1 \hbox{ and }\int_{\R^N}G(u)\, dx>0\Big\}$$
is a manifold of class $\cC^{1,1}$.
Moreover $J\circ m :\cU\to \R$ is still of class $\cC^1$ and $u\in\cU$ is a critical point of $J\circ m$ if and only if $m(u)$ is a critical point of the unconstrained functional $J$. The main difficulty is the fact that it is not clear whether a Palais-Smale sequence $(u_n)\subset \cU$ of $J\circ m$ can be mapped into a Palais-Smale sequence $m(u_n)\subset \cM$ of the unconstrained functional $J$. Moreover, we do not know and if a nontrivial weak limit point of $(m(u_n))$ is a critical point of $J$ and stays in $\cM$.\\ 
\indent In order to overcome these obstacles we introduce a new variant of the Palais-Smale condition at level $\beta\in\R$ denoted by $(M)_\beta\; (i)$ (see Section \ref{sec:criticaltheory}), which roughly says that, for
every Palais-Smale sequence $(u_n)\subset \cU$ at level  $\beta$, $(m(u_n))$
contains a subsequence converging weakly towards a point $u\in H^1(\R^N)$ up to the $\R^N$-translations, which can be projected on a critical point $m_{\cP}(u)\in \cM$. Moreover we may choose a proper $\R^N$-translation such that
$$J(m_{\cP}(u))\leq \beta=\lim_{n\to\infty} J(m(u_n)).$$
The selection of the proper translation plays a crucial role and requires the following profile decompositions of bounded sequences in $H^1(\R^N)$ in the spirit of \cite{Gerard,HmidiKeraani,SoliminiDev}.

\begin{Th}\label{ThGerard}
	Suppose that $(u_n)\subset H^{1}(\R^N)$ is bounded.
	Then there are sequences
	$(\tu_i)_{i=0}^\infty\subset H^1(\R^N)$, $(y_n^i)_{i=0}^\infty\subset \R^N$ for any $n\geq 1$, such that $y_n^0=0$,
	$|y_n^i-y_n^j|\rightarrow \infty$ as $n\to\infty$ for $i\neq j$, and passing to a subsequence, the following conditions hold for any $i\geq 0$:
	\begin{eqnarray}%\label{EqSplit1a}
	\nonumber
	&& u_n(\cdot+y_n^i)\weakto \tu_i\; \hbox{ in } H^1(\R^N)\text{ as }n\to\infty,\\
	%&&\tu_i\neq 0\text{ for }  i \geq 1,\\
	\label{EqSplit2a}
	&& \lim_{n\to\infty}\int_{\R^N}|\nabla u_n|^2\, dx=\sum_{j=0}^i \int_{\R^N}|\nabla\tu_j|^2\, dx+\lim_{n\to\infty}\int_{\R^N}|\nabla v_n^i|^2\, dx,
	\end{eqnarray}
	where $v_n^i:=u_n-\sum_{j=0}^i\tu_j(\cdot-y_n^j)$ and
	\begin{eqnarray}
	&& \limsup_{n\to\infty}\int_{\R^N}\Psi(u_n)\, dx= \sum_{j=0}^i
	\int_{\R^N}\Psi(\tu_j)\, dx+\limsup_{n\to\infty}\int_{\R^N}\Psi(v_n^i)\, dx	\label{EqSplit3a}
	\end{eqnarray}
for any function $\Psi:\R\to[0,\infty)$ of class $\cC^1$ such that $\Psi'(s)\leq C(|s|+|s|^{2^*-1})$ for any $s\in\R$ and some constant $C>0$.
	Moreover, if in addition $\Psi$ satisfies
		\begin{equation}
		\label{eq:Psi2}\lim_{s\to 0} \frac{\Psi(s)}{s^{2}}=\lim_{|s|\to\infty} \frac{\Psi(s)}{s^{2^*}}=0,
		\end{equation}
		then
	\begin{equation}\label{EqSplit4a}
	\lim_{i\to\infty}\Big(\limsup_{n\to\infty}\int_{\R^N}\Psi(v_n^i)\, dx\Big)=0.
	\end{equation}
\end{Th}
In particular, taking $\Psi(s)=|s|^p$ with $p=2$ and with $2<p<2^*$ we obtain \cite{HmidiKeraani}[Proposition 2.1]. 
%Note that the methods of \cite{HmidiKeraani} are based on the Fourier transform, which seem to be difficult to deal with the general function $\Psi$. 
Our argument relies only on new variants of Lions lemma; see Section \ref{sec:Lions} and variants of Theorem \ref{ThGerard} in $H^1_{\cO_1}(\R^N)$ as well as in $H^1_{\cO_2}(\R^N)$.\\
\indent Having a minimizing sequence of $J\circ m$, we find a proper translation such that a weak limit point can be projected on a critical point of $J$ in $\cM$ and we prove Theorem \ref{ThMain1}. The same procedure works in the subspace $X_\tau\cap H^1_{\cO_1}(\R^N)\subset H^1(\R^N)$, however we have to ensure that we choose a proper translation along $\R^{N-2m}$-variable and we get Theorem \ref{ThMain2}.\\
\indent In order to get multiplicity of critical points, we show that
$J\circ m$ satisfies the Palais-Smale condition in  $\cU\cap X_\tau\cap H^1_{\cO_2}(\R^N)$ and in view of the critical point Theorem \ref{Th:CrticMulti} of Section \ref{sec:criticaltheory}, $J$ has infinitely many critical points and we prove Theorem \ref{ThMain3}.\\
\indent Note that the existence and the multiplicity results concerning similar problems to \eqref{eq} in the noncompact case present in the literature require strong growth conditions imposed on the nonlinear term, e.g. $f$ has to be of subcritical growth and, in addition, must satisfy an Ambrosetti-Rabinowitz-type condition \cite{AR,CotiZelatiRab},  or a monotonicity-type assumption \cite{SzulkinWeth}; see also  references therein.
If a nonlinear equation like \eqref{eq} exhibits radial symmetry, then the problem of existence of nonradial solutions is particularly challenging and there are only few results in this direction. The first paper \cite{BartschWillem} due to Bartsch and Willem dealt with semili\-near elliptic problems in dimension $N=4$ and $N\geq 6$ under subcritical growth conditions and an Ambrosetti-Rabinowitz-type condition. In fact, from \cite{BartschWillem} we borrowed an idea of the decomposition of $\R^N$ and  $\cO_2$-action on $H^1(\R^N)$ given in Theorem \ref{ThMain2}. Further analysis of decompositions of $\R^N$ in this spirit has been recently studied in \cite{Marzantowicz} and in the references therein.  Next, Lorca and Ubilla \cite{Lorca} solved the similar problem in dimension $N=5$ by considering  $\cO_1$-action on $H^1(\R^N)$, and recently Musso, Pacard and Wei \cite{Musso} obtained nonradial solutions in any dimension $N\geq 2$; see also \cite{AoWei}. 
In these works, again, strong assumptions needed to be imposed on nonlinear terms, for instance a nondegeneracy condition in \cite{Musso,AoWei}, which  allows to apply a Liapunov-Schmidt-type reduction argument.\\
\indent The paper is organized as follows. In Section 2 we build a critical point theory on a general topological manifold $\cM$ in the setting of abstract assumptions (A1)--(A3). Having our variant of Palais-Smale condition $(M)_\beta$, in Theorem \ref{Th:CrticMulti} we prove the existence of minimizers on $\cM$ and the multiplicity result. The general theorem can be useful in the study of strongly indefinite problems as well, like \cite{SzulkinWeth,MederskiENZ}, where the classical linking approach due to Benci and Rabinowitz \cite{BenciRabinowitz} does not apply and the classical Palais-Smale condition is not satisfied; see Remark \ref{rem:CrtiticalPointTheory}. Moreover, in a subsequent work \cite{MederskiZeroMass} 
these techniques will be used to obtain nonradial solutions in the zero mass case problem \eqref{eq}, which has been studied in the radial case so far in \cite{BerLions,BerLionsInfZero}. In Section \ref{sec:Lions} we prove three variants of Lions lemma in $H^1(\R^N)$, $H^1_{\cO_1}(\R^N)$ and in $H^1_{\cO_2}(\R^N)$. These allow us to prove the profile decomposition Theorem \ref{ThGerard} and its variant Corollary \ref{CorGerard} in order to analyse Palais-Smale sequences in Proposition \ref{prop:PSanaysis} and in Corollary \ref{cor:PSanalysis}. We complete proofs of Theorems \ref{ThMain1}, \ref{ThMain2} and \ref{ThMain3} in the last Section \ref{sec:proof}.




\section{Critical point theory on a topological manifold}\label{sec:criticaltheory}
Let $G$ be an isometric group action on a reflexive Banach space $X$ with norm $\|\cdot\|$ and $J:X\to \R$ is a $\cC^1$-functional. Assume that
\begin{itemize}
	\item[(A1)] $J$ is $G$-invariant, i.e. if $u\in X$ and $g\in G$ then $J(gu)=J(u)$. If $gu=-u$ for some $u\in X\setminus\{0\}$, then $g=\id$. Moreover if $g_n\in G$, $u\in X$ and $g_nu\weakto  v$, then $v=gu$ for some $g\in G$ or $v=0$. 
\end{itemize}
Let $\cM\subset X\setminus\{0\}$ be a closed and nonempty subset of $X$ such that
\begin{itemize}
	\item[(A2)] $\cM$ is $G$-invariant and $\inf_{\cM} J>0$.
\end{itemize}
Since, in general, $\cM$ has not the $\cC^{1,1}$-structure, we introduce a manifold
$$\cS=\{u\in Y: \psi(u)=1\}$$%\subset X\setminus\{0\}$$
in a closed $G$-invariant subspace $Y\subset X$, where $\psi\in\cC^{1,1}(Y,\R)$ is $G$-invariant and such that $\psi'(u)\neq 0$ for $u\in\cS$. Clearly, from the implicit function theorem,  $\cS$ is a $G$-invariant manifold of class $\cC^{1,1}$ and of codimension $1$ in $Y$ with the following tangent space at $u\in\cS$
$$T_u \cS=\{v\in Y: \psi'(u)(v)=0\}.$$
\begin{itemize}
	\item[(A3)] There are a $G$-invariant open neighbourhood $\cP\subset X\setminus\{0\}$ of $\cM$ and $G$-equivariant map $m_{\cP}:\cP\to \cM$ such that $m_{\cP}(u)=u$ for $u\in\cM$ and the restriction $m:=m_{\cP}|_{\cU}:\cU\to\cM$ for $\cU:=\cS\cap\cP$  is a homeomorphism. Moreover
	$J\circ m=J|_{\cM}\circ m$ is of class $\cC^1$ and $(J\circ m)(u_n)\to\infty$ as $u_n\to u\in\partial\cU$, $u_n\in\cU$, where the boundary of $\cU$ is taken in $\cS$.
\end{itemize}
\indent As usual, we say that $(u_n)\subset \cU$ is a {\em $(PS)_\beta$-sequence} of $J\circ m:\cU\to\R$ 
provided that
$$(J\circ m)(u_n)\to \beta\hbox{ and }(J\circ m)'(u_n) \to 0.$$
Let $\cK$ be the set of all critical points of $J\circ m$, i.e.
$$\cK:=\big\{u\in\cU: (J\circ m)'(u)(v)=0\hbox{ for any }v\in T_u\cS\big\}.$$
 For $u\in X$, $G\ast u$ denotes the orbit of $u$
$$G\ast u:=\{gu: g\in G\}.$$ We introduce the following variant of the {\em Palais-Smale condition} at level $\beta\in\R$.


\begin{itemize}
	\item[$(M)_\beta\;(i)$]  For every $(PS)_{\beta}$-sequence $(u_n)\subset \cU$ of $J\circ m$, there are a sequence $(g_n)\subset G$ and $u\in\cP$ such that
	$g_nm(u_n) \weakto u$ along a subsequence, $J'(m_{\cP}(u))=0$ and $J(m_{\cP}(u))\leq \beta$.
	\item[$\hspace{1cm}(ii)$]
	If $\cK$ has a finite number of distinct orbits $G\ast u$ for $u\in \cK$,  then there is $m_\beta>0$ such that for every $(u_n)\subset \cU$ such that $(J\circ m)'(u_n)\to 0$ as $n\to\infty$, $(J\circ m)(u_n)\leq\beta$ and $\|u_n-u_{n+1}\|<m_\beta$ for $n\geq 1$, there holds
	$\liminf_{n\to\infty}\|u_n-u_{n+1}\|=0$.
\end{itemize}

Note that  $(M)_\beta(i)$ implies that if $(J\circ m)'(u)=0$, then $J'(m(u))=0$ for $u\in\cU$.
Indeed, taking a sequence $u_n=u$, observe that $g_n m(u)\weakto \tu$ along a subsequence, $\tu\in \cP\subset X\setminus\{0\}$ and by (A1), $\tu=gm(u)$ for some $g\in G$. Then $\tu\in \cM$ and by (A3), $m_{\cP}(\tu)=m_{\cP}(gm(u))=gm(u)$ is a critical point of $J\circ m$, hence by (A1), we conclude $J'(m(u))=0$.
Therefore critical points of $J\circ m$ are mapped by $m$ into nontrivial critical points of the unconstrained functional $J$. Observe that, however, $m(u_n)$ need not to be a Palais-Smale sequence of the unconstrained functional $J$ if $(u_n)\subset \cU$ is a $(PS)_\beta$-sequence of $J\circ m$.\\
\indent In what follows, for $A\subset X$ and $r>0$, $B(A,r):=\{u\in X: \|u-v\|<r\hbox{ for some }v\in A\}$.

\begin{Lem}\label{lem:M_betacond}
	Suppose that (A1)--(A3), $(M)_\beta\; (ii)$ hold for some $\beta\in\R$
	and $\cK$ has finite number of distinct orbits $G\ast u$ for $u\in \cK$. If $(u_n)\subset\cU$ is a $(PS)_\alpha$-sequence for some $\alpha<\beta$ and
	\begin{equation}\label{eq:closetoK}
	u_n\in B(\cK\cap (J\circ m)^{-1}((-\infty,\beta]),m_\beta),
	\end{equation}
	then passing to a subsequence $g_nu_n\to u$ for some $u\in \cK$ and $g_n\in G$.
\end{Lem}
\begin{proof}
	Let $(u_n)\subset\cU$ be a $(PS)_\alpha$-sequence such that \eqref{eq:closetoK} holds. Passing to a subsequence $(J\circ m)(u_n)\leq \beta$. Then 
	we put $w_{2n-1}:=u_n$ and take any $w_{2n}\in \cK$ such that 
	$$\|w_{2n}-u_n\|<m_\beta$$
	and  $(J\circ m)(w_{2n})\leq \beta$ 
	for any $n\geq 1$. Take $\tilde{\cK}\subset \cK$ such that each orbit $G\ast u$ has a unique representative in $\tilde{\cK}$ for $u\in \cK$, so that $\tilde{\cK}\cap (G\ast u)$ is a singleton. Since $\tilde{\cK}$ is finite, 
	passing to a subsequence we may assume that $g_{2n}w_{2n}=u\in \tilde{\cK}$ for some $g_{2n}\in G$ and $u\in \tilde{\cK}$. Take $g_{2n-1}=g_{2n}$ for $n\geq 1$ and observe that by (A1), $(J\circ m)'$ is $G$-equivariant, hence $(J\circ m)'(g_nw_n)\to 0$, $(J\circ m)(g_nw_n)\leq \beta$ and
	$$\|g_nw_n-g_{n+1}w_{n+1}\|<m_{\beta}$$
	for $n\geq 1$.
	Then, in view of $(M)_{\beta}\; (ii)$ we obtain
	$$\liminf_{n\to\infty}\|g_{2n}u_n-u\|=\liminf_{n\to\infty}\|g_{2n-1}w_{2n-1}-g_{2n}w_{2n}\|=0,$$
	and $g_{2n}u_n\to u\in \cK$ passing to a subsequence.
\end{proof}
Hence, roughly speaking, Lemma \ref{lem:M_betacond} says that if $(M)_\beta\; (ii)$ holds, then a sufficiently close Palais-Smale sequence of $J\circ m$ to the set of critical points with finite number of distinct orbits contains a convergent subsequence up to the $G$-action.\\
\indent Now our main result of this section reads as follows.


\begin{Th}\label{Th:CrticMulti}
	Suppose that $J:X\to\R$ is of class $\cC^1$ and satisfies (A1)--(A3).\\
	(a) If  $(M)_{\beta}\; (i)$ holds for $\beta=\inf_{\cM} J$, then
	$J$ has a critical point $u\in\cM$ such that $$J(u)=\inf_{\cM}J.$$
	(b) Assume that $(M)_{\beta}$ holds for every $\beta\geq \inf_{\cM} J$, $J$ is even, $m_{\cP}$ is odd and $\cU$, $\cM$ are symmetric, i.e. $\cU=-\cU$, $\cM=-\cM$.
	Then $J$ has infinitely many $G$-distinct critical points in $\cM$, i.e. there is a sequence of critical points $(u_n)\subset\cM$ such that $(G\ast u_n)\cap (G\ast u_m)=\emptyset$ for $n\neq m$, provided that
	\begin{equation}\label{eq:LSvaluesnonempty}
	\hbox{ for any } k\geq 1,
	\hbox{ there exists a continuous and odd map from }S^{k-1}\hbox{ to }\cP,
	\end{equation}
	where $S^{k-1}$ is the unit sphere in $\R^k$.\\
	(c) Assume that $G=\id$, $J$ is even, $m_{\cP}$ is odd and $\cU$, $\cM$ are symmetric, and for every $(PS)_{\beta}$-sequence $(u_n)\subset \cU$ of $J\circ m$ with $\beta\geq \inf_{\cM} J$, there is $u\in\cU$ such that $J'(m(u))=0$ and $u_n \to u$ along a subsequence. 
	Then $J$ has infinitely many critical points in $\cM$ provided that \eqref{eq:LSvaluesnonempty} holds.
\end{Th}


\begin{proof}
	Let  $\Phi:=J\circ m:\cU\to\R$. Similarly as in \cite{Willem}[Lemma 5.14] we find
	an odd and  locally Lipschitz pseudo-gradient vector field $v:\cU\setminus \cK\to Y$ such that  $v(u)\in T_u\cS$ and
	\begin{eqnarray}
	%\nonumber
	%v(u)&\in& X_L,\\
	\|v(u)\|&<&2\| \Phi'(u)\|,\label{eq:flow2}\\
	\Phi'(u)(v(u)) &>& \|\Phi'(u)\|^2\label{eq:flow3}
	\end{eqnarray}
	for any $u\in \cU\setminus \cK$. The obtained pseudo-gradient vector field allows to prove a variant of deformation lemma \cite{Willem}[Lemma 5.15] in 
$\cU$ and arguing as in  \cite{Willem}[Theorem 8.5], we find a minimizing sequence $(u_n)\subset \cU$ such that 
	$$(J\circ m)(u_n)\to c:=\inf_{\cU}J\circ m=\inf_{\cM}J$$
	and $(J\circ m)'(u_n)\to 0$ as $n\to\infty$. In view of $(M)_{\beta}\;(i)$ we find a nontrivial critical point $m_{\cP}(u)\in \cM$ of $J$ such that passing to a subsequence $g_nu_n\weakto u$ for some $g_n\in G$. Since 
	$$c\geq J(m_{\cP}(u))\geq\inf_{\cM}J,$$
	we get $J(m_{\cP}(u))=c$, which completes proof of (a). 
	For any $\alpha<\beta$ let us denote 
	\begin{eqnarray*}
		\Phi^{\beta}_\alpha&:=&\{u\in\cU: \alpha\leq \Phi(u)\leq \beta\},\\
		\Phi^{\beta}&:=&\{u\in\cU: \Phi(u)\leq \beta\}.
	\end{eqnarray*}	
	Proof of (b) and (c) is based on the fact that the Lusternik-Schnirelman values
		\begin{equation}\label{eq:LSvalue}
			%\Sigma_k&=&\{A\in\Sigma:  \gamma(A)\geq k\}\\
			\beta_k:= \inf\{\beta\in\R:  \gamma(\Phi^{\beta})\geq k\}.
		\end{equation}
	are increasing critical values for $k\geq 1$, where $\gamma$ stands for the Krasnoselskii genus for closed and symmetric subsets of $X$. Observe that \eqref{eq:LSvaluesnonempty} implies that for any $k\geq 1$ there is $\beta>0$ such that 
	$$\gamma(\Phi^{\beta})\geq \gamma\Big(m^{-1}\big(m_{\cP}(\tau(S^{k-1}))\big)\Big)\geq k,$$
	hence $\beta_k<\infty$.
	To prove (c) one can argue as in  \cite{Rabinowitz:1986}[Theorem 8.10]. For the reader's convenience we provide details, in particular we demonstrate how $(M)_\beta\; (ii)$ works here, cf. \cite{SzulkinWeth}.
Observe that we find the unique flow
	$\eta:\cG\to  \cU\setminus \cK$ such that
	\begin{equation*}
	\left\{
	\begin{aligned}
	&\partial_t \eta(t,u)=-v(\eta(t,u))\\
	&\eta(0,u)=u
	\end{aligned}
	\right.
	\end{equation*}
	where $\cG:=\{(t,u)\in [0,\infty)\times (\cU\setminus \cK):\; t<T(u)\}$ and $T(u)$ is the maximal time of the existence of $\eta(\cdot,u)$. 
Suppose that there is a finite number of distinct orbits $G\ast u$ for $u\in \cK$.
	Take $\beta\geq c$ and let
	\begin{eqnarray*}
		\cK^\beta&:=&\{u\in \cK:  \Phi(u)=\beta\}.
	\end{eqnarray*}
	Then there is $\eps_0>0$ such that 
	\begin{equation}\label{eq:KL}
	\cK\cap \Phi_{\beta-\eps_0}^{\beta+\eps_0}=\cK^\beta.
	\end{equation}
	Indeed, suppose  that there is a sequence $(u_n)\subset \cK\cap \Phi_{\beta-\eps_n }^{\beta+\eps_n}$ such that $\Phi(u_n)\neq \beta$ and $\eps_n\to 0$.  Passing to a subsequence we may assume that $\Phi(u_n)$ is strictly increasing or decreasing. In view of (A1), the orbits $G\ast u_n$ consist of critical points on different levels $\Phi(u_n)$, which contradicts the finiteness of distinct orbits $G\ast u$ for $u\in \cK$.
	Now we show that for every $\delta\in (0,m_{\beta+\eps_0})$ there is $\eps\in (0,\eps_0]$ such that
	\begin{equation}\label{eq:entrancetime1}
	\lim_{t\to T(u)} \Phi(\eta(t,u)) < \beta -\eps \quad\hbox{for } u\in \Phi^{\beta+\eps}_{\beta-\eps_0}\setminus B(\cK^\beta,\delta).
	\end{equation}
	Take $u\in \Phi^{\beta+\eps}_{\beta-\eps_0}\setminus \cK^\beta$ and
	observe that by (A2),
	$\Phi(\eta(t,u))=J(m(\eta(t,u)))$  is bounded from below by $c$, and by  \eqref{eq:flow3} it is decreasing in $t\in [0,T(u))$. Hence $\lim_{t\to T(u)} \Phi(\eta(t,u))$ exists.
	Suppose that 
	\eqref{eq:entrancetime1} does not hold, i.e. there is $\delta\in (0,m_{\beta+\eps_0})$ such that for any $\eps\in (0,\eps_0]$ 
	$$A_\eps:=\{u\in \Phi^{\beta+\eps}_{\beta+\eps_0}\setminus B(\cK^\beta,\delta):  
	\lim_{t\to T(u)} \Phi(\eta(t,u)) \geq \beta -\eps
	\}\neq \emptyset.$$
	We show that for any $u\in A_{\eps_0}$
	\begin{equation}\label{eq:Claim1}
	\lim_{t\to\infty} \inf_{v\in \cK^\beta}\|\eta(t,u)-v\| =0.
	\end{equation}
%Of course, if $\Phi$ satisfies the Palais-Smale condition as in (c), then taking into account \eqref{eq:KL} we easy see that \eqref{eq:Claim1}  holds. Suppose that $(M)_\beta\; (ii)$ is satisfied.
We show that for $u\in A_{\eps_0}$, $\lim_{t\to T(u)}\eta(t,u)$ exists.
	Suppose that, on the contrary, there is $0<\eta_0<m_{\beta+\eps_0}$ and there is an increasing sequence $(t_n)\subset [0,T(u))$ such that $t_n\to T(u)$ and 
	\begin{eqnarray}\label{eq:eps_0}
	\eta_0&<&\|\eta(t_{n+1},u)-\eta(t_n,u)\|<m_{\beta+\eps_0}
	\end{eqnarray} 
	for $n\geq 1$.
	Note that by \eqref{eq:flow2} and \eqref{eq:flow3}
	\begin{eqnarray*}
		\eta_0<\|\eta (t_{n+1},u)-\eta(t_n,u)\|&\leq& \int_{t_n}^{t_{n+1}}\|v(\eta(s,u))\|\, ds\leq 2 \int_{t_n}^{t_{n+1}}\|\Phi'(\eta(s,u))\|\, ds\\
		&\leq& 2\int_{t_n}^{t_{n+1}}\big(\Phi'(\eta(s,u))(v(\eta(s,u)))\big)^{1/2}\, ds\\
		&\leq& 2\sqrt{t_{n+1}-t_n} \Big(\int_{t_n}^{t_{n+1}} \Phi'(\eta(s,u))(v(\eta(s,u)))\,ds\Big)^{1/2}\\
		&=& 2\sqrt{t_{n+1}-t_n} \big(\Phi(\eta(t_{n},u))-\Phi(\eta(t_{n+1},u))\big)^{1/2}\\
		&\leq & 2\sqrt{t_{n+1}-t_n} (\beta +\eps)^{1/2}.
	\end{eqnarray*}
	Hence $|t_{n+1}-t_n|\geq \frac{\eta_0^2}{4(\beta+\eps)}$ and $T(u)=\infty$.
	Again, by \eqref{eq:flow3}
	\begin{equation}\label{eq:PSetha}
	\int_{t_n}^{t_{n+1}}\|\Phi'(\eta(s,u))\|^2\, ds\leq \big(\Phi(\eta(t_n,u))-\Phi(\eta(t_{n+1},u))\big)\to 0
	\end{equation}
	as $n\to\infty$, we may assume that $\Phi'(\eta(t_n,u))\to 0$. Then by \eqref{eq:eps_0} we get a contradiction with $(M)_{\beta+\eps_0}(ii)$. Hence $u_0=\lim_{t\to T(u)}\eta(t,u)$ exists and since $J(\eta(t,u))\leq J(u)$ is bounded as $t\to T(u)$, by (A3) we get $u_0\notin \partial\cU$. From
	the definition of $T(u)$, we infer that $u_0\in \cK$.  
Moreover, by \eqref{eq:KL}
	$$u_0\in \cK\cap \Phi^{\beta+\eps_0}_{\beta-\eps_0}=\cK^\beta,$$
which completes the proof of \eqref{eq:Claim1}.\\
Now observe that in view of \eqref{eq:Claim1}, for $u\in A_{\eps_0}$ we may define
	\begin{eqnarray*}
		t_0(u)&:=&\inf\big\{t\in [0,T(u)):  \eta(s,u)\in B(\cK^\beta,m_{\beta+\eps_0})\hbox{ for all }s> t\big\}\\
		t(u)&:=&\inf\big\{t\in [t_0(u),T(u)):  \eta(t,u)\in B(\cK^\beta,\delta/2)\big\}
	\end{eqnarray*}
	Note that $0\leq t_0(u)< t(u)<T(u)$ and we show that
	\begin{equation}\label{eq:inft(u)}
	\inf_{u\in A_{\eps_0}}
	t(u)-t_0(u)\geq \frac{\delta^2}{16(\beta+\eps_0)}.
	\end{equation}
	Indeed, if $u\in A_{\eps_0}$, then by \eqref{eq:flow2} and \eqref{eq:flow3} we have
	\begin{eqnarray*}
		\frac{\delta}{2}&\leq&
		\|\eta(t_0(u),u)-\eta(t(u),u)\| \leq \int_{t_0(u)}^{t(u)}\|v(\eta(s,u))\|\, ds
		\leq 2\int_{t_0(u)}^{t(u)}\|\Phi'(\eta(s,u))\|\, ds\\
		&\leq& 2\int_{t_0(u)}^{t(u)}\big(\Phi'(\eta(s,u))(v(\eta(s,u_n)))\big)^{1/2}\, ds\\
		&\leq& 
		2\sqrt{t(u)-t_0(u)}\Big(\int_{t_0(u)}^{t(u)}\Phi'(\eta(s,u))(v(\eta(s,u)))\, ds\Big)^{1/2}\\
		&=& 
		2\sqrt{t(u)-t_0(u)}\big(\Phi(\eta(t_0(u)u)-\Phi(\eta(t(u),u))\big)^{1/2}\\
		&\leq& 2\sqrt{t(u)-t_0(u)}(\beta+\eps_0)^{1/2}
	\end{eqnarray*}
	and we get \eqref{eq:inft(u)}. Note that $A_{\eps_0/2}\subset A_{\eps_0}$ and
	let
	$$\rho:=\inf_{u\in A_{\eps_0/2}}\int_{t_0(u)}^{t(u)}\|\Phi'(\eta(s,u)\|^2\, ds.$$
	If $\rho=0$ then by \eqref{eq:inft(u)} we find $u_n\in A_{\eps_0/2} $ and $t_n\in (t_0(u_n),t(u_n))$ such that $$\Phi'(\eta(t_n,u_n))\to 0 \hbox{ as } n\to\infty.$$
	Since $t_n> t_0(u_n)$ we have $\eta(t_n,u_n)\in B(\cK^\beta,m_{\beta+\eps_0})$ and passing to a subsequence $$\Phi(\eta(t_n,u_n))\to\alpha\leq \beta+\eps_0/2 < \beta+\eps_0.$$ In view of Lemma \ref{lem:M_betacond}, or by the Palais-Smale condition assumed in (c), passing to a subsequence, we obtain
	$$g_n\eta(t_n,u_n)\to u$$
	for some $u\in \cK$ and $g_n\in G$. By \eqref{eq:KL} we get $u\in \cK^\beta$. Since $t_n<t(u_n)$ we obtain
	$$g_{n}\eta(t_n,u_n)\notin B(\cK^\beta,\delta/2),$$
	which is a contradiction. Therefore $\rho>0$ and we take 
	$$\eps < \min\Big\{\frac12\eps_0,\frac14\rho\Big\}.$$
	Let $u\in A_\eps\subset A_{\eps_0/2}$ and 
	since
	\begin{eqnarray*}
		\Phi(\eta(t(u),u))-\Phi(\eta(t_0(u),u)) &=&
		-\int_{t_0(u)}^{t(u)}\Phi'(\eta(s,u))(v(\eta(s,u)))\,ds\\
		&\leq& 
		-\frac12\int_{t_0(u)}^{t(u)}\|\Phi'(\eta(s,u)\|^2\, ds,
	\end{eqnarray*}
	we obtain
	\begin{eqnarray*}
		\beta-\eps&\leq& \lim_{t\to T(u)} \Phi(\eta(t,u))\leq \Phi(\eta(t(u),u))\\
		&\leq&\beta +\eps -\frac12\int_{t_0(u)}^{t(u)}\|\Phi'(\eta(s,u)\|^2\, ds
		\leq \beta+\eps -\frac12\rho\\
		&<&\beta-\eps,
	\end{eqnarray*}
	which gives again a contradiction.
	Thus we have finally proved that \eqref{eq:entrancetime1} holds and now take any $\delta<m_{\beta+\eps_0}$ such that
	$$\gamma (\cl B(\cK^\beta,\delta))=\gamma (\cK^\beta).$$
	%and we may assume that $B(\cK^\beta,\delta)\subset $
	Let us define the entrance time map $e:\Phi^{\beta+\eps}_{\beta-\eps_0}\setminus B(\cK^\beta,\delta)\to [0,\infty)$ such that
	$$e(u):=\inf\{t\in [0,T(u)):  \Phi(\eta(s,u))\leq \beta -\eps\}.$$
	It is standard to show that $e$ is continuous and even.
	Moreover we may define a continuous and odd map 
	$h:\Phi^{\beta+\eps}\setminus B(\cK^\beta,\delta)\to \Phi^{\beta-\eps}$
	such that 
	\begin{equation*}
	%\label{eq}
	h(u)=\left\{
	\begin{array}{ll}
	\eta(e(u),u)&\quad \hbox{for }u\in \Phi^{\beta+\eps}_{\beta-\eps_0}\setminus B(\cK^\beta,\delta),\\
	u &\quad \hbox{for }u\in \Phi^{\beta-\eps_0}.
	\end{array}
	\right.
	\end{equation*}
	Let us take $\beta=\beta_k$ defined by \eqref{eq:LSvalue} for some $k\geq 1$. Then
	$$\gamma(\Phi^{\beta+\eps}\setminus B(\cK^\beta,\delta))\leq 
	\gamma(\Phi^{\beta-\eps})\leq k-1$$
	and
	\begin{equation}\label{eq:LSvaluse}
	k\leq \gamma(\Phi^{\beta+\eps})\leq \gamma (\cl B(\cK^\beta,\delta))+
	\gamma(\Phi^{\beta+\eps}\setminus B(\cK^\beta,\delta))\leq \gamma(\cK^\beta)+k-1.
	\end{equation}
	Thus $\cK^\beta\neq \emptyset$, and since it has finite number of orbits and (G) holds, we easy show that there is a continuous and odd map from $\cK^\beta$ with values in $\{-1,1\}$. Thus $\gamma(\cK^\beta)=1$. Note that if $\beta_k=\beta_{k+1}$ for some
	$k\geq 1$, then by \eqref{eq:LSvaluse} we get $\gamma(\cK_{\beta_k})\geq 2$, which is a contradiction. Hence we get an infinite sequence
	$\beta_1<\beta_2<...$
	of critical values, which contradicts that $\cK$ consists of a finite number of distinct orbits. 
%	Now suppose that $\cK$ contains a compact subset of representatives, i.e. there is $\tilde{\cK}\subset \cK$ such that $\tilde{\cK}\cap (G\ast u)$ is an singleton for any $u\in\cK$. Since $0\not\in \tilde{\cK}$, we find a continuous and odd map $\tilde{j}:\tilde{\cK}\to \R^l\setminus\{0\}$ for some $0\leq l <\infty$;
%	see e.g. \cite{Struwe}[Proposition 5.4]. Then we may define an extension $j:\cK\to \R^l\setminus\{0\}$ of $\tilde{j}$
%	such that $j$ is constant on $G\ast j(u)$. Then we easy check that $j$ is well-defined, odd, and by (A1), $j$ is continuous. Hence $\gamma(\cK)\leq l<\infty$.
%	Let $l:=\gamma(\cK)$ and 
%$$\beta_l:= \inf\{\beta\in\R:  \gamma(\Phi^{\beta})\geq l\}.$$
%Since $\Phi(\cK)=\Phi(\tilde{\cK})$ is bounded, then $\beta_l <\infty$.
%Similarly as above we show that
% for every $\delta\in (0,m_{l+\eps_0})$ there is $\eps\in (0,\eps_0]$ such that
%\begin{equation}\label{eq:entrancetime2}
%\lim_{t\to T(u)} \Phi(\eta(t,u)) < \beta_l -\eps \quad\hbox{for } u\in \Phi^{\beta_l+\eps}\setminus B(\cK,\delta).
%\end{equation}
This completes the proof of (b) and (c).
%
%In view of $(M)_\beta\; (i)$ we know that critical points of $J\circ m$ are mapped into critical points of $J$ in $\cM$. 
\end{proof}


	










\begin{Rem}\label{rem:CrtiticalPointTheory}
In this paper we consider the problem \eqref{eq} having the mountain pass geometry, hence we assume that $Y=X$, and we show that $(M)_\beta\;(i)$ holds with translations $G=\R^N$ or $G=\{0\}\times\{0\}\times\R^{N-2m}$;  see Lemma \ref{lem:Mcond} and Lemma \ref{lem:Mcond2} below. Moreover, we apply the multiplicity result Theorem \ref{ThMain3} (c) and we show that $J\circ m$ satisfies the Palais-Smale condition in $\cU\cap H^{1}_{\cO_2}(\R^N)\cap X_\tau$; see Lemma \ref{lem:Mcond3}.
However,  in other applications, for an indefinite functional $J$ one needs to find a proper subspace $Y\subset X$ such that (A3) holds and the Palais-Smale condition may not be satisfied. For instance, one can consider a generalized Nehari manifold $\cM$, and the approaches considered in \cite{SzulkinWeth,MederskiENZ} fit into the abstract setting of this section. The discreteness of Palais-Smale sequences obtained in \cite{SzulkinWeth}[Lemma 2.14] implies $(M)_\beta\;(ii)$ with $G=\Z^N$ and we can reprove the results of \cite{SzulkinWeth}.
Hence, Theorem \ref{Th:CrticMulti} may be applied to Pohozev as well as Nehari-type topological constraints.
\end{Rem}


\section{Concentration compactness and profile decompositions}\label{sec:Lions}

We need the following variant of Lions lemma.
\begin{Lem}\label{lem:Lions}
	Suppose that $(u_n)\subset H^{1}(\R^N)$ is bounded and for some $r>0$ 	\begin{equation}\label{eq:LionsCond11}
	\lim_{n\to\infty}\sup_{y\in \R^N} \int_{B(y,r)} |u_n|^2\,dx=0.
	\end{equation}
	Then 
	$$\int_{\R^N} \Psi(u_n)\, dx\to 0\quad\hbox{as } n\to\infty$$
	for any continuous function $\Psi:\R\to [0,\infty)$ such that \eqref{eq:Psi2} holds.
%	Moreover, if in addition $(v_n)\subset H^1(\R^N)$ is bounded, then 
%	$$\int_{\R^N} \Psi(|v_n|) |u_n|\, dx\to 0\quad\hbox{as } n\to\infty$$
%	for any continuous function $\Psi:[0,\infty)\to [0,\infty)$ such that $s\mapsto\Psi(s)s$ satisfies \eqref{eq:Psi2}.
%	%If $r>1$, then we may take supremum over $y\in\Z^N$ in \eqref{eq:LionsCond11}.
\end{Lem}
\begin{proof}
	Take any $\eps>0$ and $2<p<2^*$ and suppose that $\Psi$ satisfies \eqref{eq:Psi2}. Then we find $0<\delta<M$ and $c_\eps>0$ such that 
	\begin{eqnarray*}
	\Psi(s)&\leq& \eps |s|^{2}\quad\hbox{ if }|s|\in [0,\delta],\\
	%		\Psi(s)&\leq& c s^{p}\quad\hspace{1.5mm}\hbox{ for }s\in (\delta ,M],\\
	\Psi(s)&\leq& \eps |s|^{2^*}\quad\hbox{ if }|s|>M,\\
	\Psi(s)&\leq& c_\eps |s|^{p}\quad\hbox{ if }|s|\in (\delta,M].
	\end{eqnarray*}
Hence, in view of Lions lemma \cite{Willem}[Lemma 1.21] we get
$$\limsup_{n\to\infty}\int_{\R^N}\Psi(u_n)\, dx\leq \eps \limsup_{n\to\infty}\int_{\R^N}|u_n|^2+|u_n|^{2^*}\, dx.$$
Since $(u_n)$ is bounded in $L^2(\R^N)$ and in $L^{2^*}(\R^N)$, we conclude by letting $\eps\to0$.
\end{proof}





Let us consider $x=(x^1,x^2,x^3)\in\R^N=\R^m\times\R^m\times \R^{N-2m}$ with $2\leq m\leq N/2$ 
such that $x^1,x^2\in\R^m$ and $x^3\in\R^{N-2m}$. Then for
$\cO_1:=\cO(m)\times \cO(m)\times\id\subset \cO(N)$ invariant functions with respect to $\cO_1$ we get the following corollary.

\begin{Cor}\label{CorLions1}
	Suppose that $(u_n)\subset H^1_{\cO_1}(\R^N)$ is bounded, $r_0>0$ is such that for all $r\geq r_0$
	\begin{equation}\label{eq:LionsCond12}
	\lim_{n\to\infty}\sup_{z\in \R^{N-2m}} \int_{B((0,0,z),r)} |u_n|^2\,dx=0.
	\end{equation}
	Then 
	$$\int_{\R^N} \Psi(u_n)\, dx\to 0\quad\hbox{as } n\to\infty$$
	for any continuous function $\Psi:\R\to [0,\infty)$ such that \eqref{eq:Psi2} holds.
\end{Cor}
\begin{proof}
	Suppose that
	\begin{equation}\label{eq:LionsCond12proof1}
	\int_{B(y_n,1)} |u_n|^2\,dx\geq c>0
	\end{equation}
%%	for some sequence $(y_n)\subset \R^N$ and a constant $c$.
	Observe that in the family $\{B(gy_n,1)\}_{g\in\cO_1}$ we find an increasing number of disjoint balls provided that $|(y^1_n,y^2_n)|\to\infty$. Since $(u_n)$ is bounded in $L^2(\R^N)$ and invariant with respect to $\cO_1$, by \eqref{eq:LionsCond12proof1} $|(y^1_n,y^2_n)|$ must be bounded. Then for sufficiently large $r\geq r_0$ one obtains
	$$\int_{B((0,0,y_n^3),r)} |u_n|^2\,dx\geq \int_{B(y_n,1)} |u_n|^2\,dx\geq  c>0,$$
	and we get a contradiction with \eqref{eq:LionsCond12}. Therefore  \eqref{eq:LionsCond11} is satisfied with $r=1$ and by Lemma \ref{lem:Lions} we conclude.
\end{proof}

\begin{Rem} Instead of $\cO_1$ in Corollary \ref{CorLions1} one can consider any subgroup $G=\cO'\times\id\subset \cO(N)$ such that $\cO'\subset \cO(M)$ and $\R^M$ is compatible with $\cO'$ for some $0\leq M\leq N$, cf. \cite{Willem}[Theorem 1.24].
\end{Rem}

Now let us assume in addition that  $N-2m\neq 1$ and
$$\cO_2:=\cO(m)\times \cO(m)\times\cO(N-2m)\subset \cO(N).$$
In view of \cite{Lions82}, $H^1_{\cO_2}(\R^N)$ embeds compactly into $L^p(\R^N)$ for $2<p<2^*$. In order to deal with the general nonlinearity we need the following result.

\begin{Cor}\label{CorLions2}
	Suppose that $(u_n)\subset H^1_{\cO_2}(\R^N)$ is bounded and $u_n\to 0$ in $L^2_{loc}(\R^N)$.
	Then
	$$\int_{\R^N} \Psi(u_n)\, dx\to 0\quad\hbox{as } n\to\infty$$
	for any continuous function $\Psi:\R\to [0,\infty)$ such that \eqref{eq:Psi2} holds.
\end{Cor}
\begin{proof}
	Observe that for all $r>0$
	$$\lim_{n\to\infty}\int_{B(0,r)} |u_n|^2\,dx=0$$
	and similarly as in proof of Corollary \ref{CorLions1} we complete the proof.
\end{proof}


\begin{altproof}{Theorem \ref{ThGerard}}
	Let $(u_n)\subset H^1(\R^N)$ be a bounded sequence and $\Psi$ as in Theorem \ref{ThGerard}.
	We claim that there is $K\in \N\cup \{\infty\}$ and there is a sequence
	$(\tu_i)_{i=0}^K\subset H^1(\R^N)$, for $0\leq i <K+1$$(\footnote{If $K=\infty$ then $K+1=\infty$ as well.})$  there are sequences $(v_n^i)\subset H^1(\R^N)$, $(y_n^i)\subset \R^N$ and positive numbers $(c_i)_{i=0}^{K}, (r_i)_{i=0}^{K}$ such that $y_n^0=0$, $r_0=0$ and, up to a subsequence, for any $n$ and $0\leq i<K+1$ one has
	\begin{eqnarray}	\label{Eqxnxm1}
	&&u_n(\cdot+y_n^i)\weakto\tu_i\hbox{ in }H^1(\R^N)\hbox{ and }u_n(\cdot+y_n^i)\chi_{B(0,n)}\to\tu_i\hbox{ in }L^{2}(\R^N),\\	\label{Eqxnxm2}
	&&\tu_i\neq 0\hbox{ for }1\leq i <K+1,\\
	\label{Eqxnxm}
	&&|y_n^i-y_n^j|\geq n-r_i-r_j\hbox{ for } j\neq i, 0\leq j< K+1,\\	\label{Eqxnxm3}
	&& v_n^i:=v_n^{i-1}-\tu_i(\cdot-y_n^i),\\\label{EqIntegralunSumci}
	&&\int_{B(y_n^{i},r_i)}|v_n^{i-1}|^2\, dx \geq c_{i}\geq\frac{1}{2}\sup_{y\in\R^N}\int_{B(y,r_i)}|v_n^{i-1}|^2\, dx\\
	&&\hspace{4.1cm}	\geq \frac{1}{4}\sup_{r>0,y\in\R^N} \int_{B(y,r)}|v_n^{i-1}|^2\, dx\nonumber
	>0, r_i\geq \max\{i,r_{i-1}\} \hbox{ for }i\geq 1,
	\end{eqnarray}
	and \eqref{EqSplit2a} is satisfied.
%	\begin{eqnarray}
%	&&\hspace{-12mm}\lim_{n\to\infty}\int_{\mathbb{R}^N}\Psi(u_n)\, dx=\lim_{n\to\infty}\int_{\R^N}\Psi(v_n^i)\, dx+\sum_{j=0}^{i}\int_{\mathbb{R}^N}\Psi(\tu_i)\, dx,
%	\label{eq:GiDecomp}\\
%	&&\hspace{-12mm}\lim_{n\to\infty}\int_{\R^N}|\nabla u_n|\,dx=\lim_{n\to\infty}\int_{\R^N}|\nabla v_n^i|^2\,dx+\sum_{j=0}^{i}\int_{\R^N}|\nabla \tu_i|^2\,dx.
%	\label{eq:GiDecompPsi}
%	\end{eqnarray}
	Since $(u_n)$ is bounded, 
	passing to a subsequence we may assume that
	\begin{eqnarray*}
		u_n &\weakto& \tu_0\quad \hbox{ in }H^1(\R^N)\\
		u_n\chi_{B(0,n)} &\to& \tu_0\quad \hbox{ in }L^2(\R^N),
	\end{eqnarray*}
	and $\lim_{n\to\infty}\int_{\R^N}|\nabla u_n|^2\,dx$ exists. Take $v_n^0:=u_n-\tu_0$ and
	if
	\begin{equation*}%\label{eq:CondL122}
	\lim_{n\to\infty}\sup_{y\in\R^N}\int_{B(y,r)}|v_n^0|^2\, dx=0
	\end{equation*}
	for every $r\geq 1$,
	then we finish the proof of our claim with $K=0$.
	Otherwise, there is $r_1\geq 1$ such that, passing to a subsequence, we find $(y_n^1)\subset\R^N$ and a constant $c_1>0$ such that
	\begin{equation}\label{eq:LemProofLions1}
	\int_{B(y_n^{1},r_1)}|v_n^0|^2\, dx \geq c_{1}\geq \frac{1}{2}\sup_{y\in\R^N} \int_{B(y,r_1)}|v_n^0|^2\, dx
	\geq \frac{1}{4}\sup_{r>0,y\in\R^N} \int_{B(y,r)}|v_n^0|^2\, dx>0.
	\end{equation}
	Note that $(y_n^1)$ is unbounded and we may assume that $|y_n^1|\geq n-r_1$.  Since $(u_n(\cdot+y_n^1))$ is bounded in $H^1(\R^N)$, we find $\tu_1\in H^1(\R^N)$ such that up to a subsequence
	$$u_n(\cdot+y_n^1)\weakto \tu_1.$$ In view of \eqref{eq:LemProofLions1},  we get $\tu_1\neq 0$, and again we may assume that $u_n(\cdot+y_n^1)\chi_{B(0,n)}\to \tu_1$ in $L^2(\R^N)$.  
	Since
	$$\lim_{n\to\infty}\Big(\int_{\R^N}|\nabla (u_n-\tu_0)(\cdot +y_n^1)|^2\, dx-\int_{\R^N}|\nabla v_n^1(\cdot +y_n^1)|^2\,dx\Big)=\int_{\R^N}|\nabla \tu_1|^2\, dx,$$
	where  $v_n^1:=v_n^0-\tu_1(\cdot -y_n^1)=u_n-\tu_0-\tu_1(\cdot -y_n^1)$ ,
	then
	\begin{eqnarray*}%\label{eq:eqthetaPsi2}
	\lim_{n\to\infty}\int_{\R^N}|\nabla u_n|^2\, dx &=&\int_{\R^N}|\nabla \tu_0|^2\,dx+\int_{\R^N}|\nabla \tu_1|^2\, dx+\lim_{n\to\infty}\int_{\R^N}|\nabla v_n^1|^2\, dx
	\end{eqnarray*}
	If 
	\begin{equation*}%\label{eq:LionscondTheta}
	\lim_{n\to\infty}\sup_{y\in\R^N}\int_{B(y,r)}|v_n^1|^2\, dx=0
	\end{equation*}
	for every $r\geq \max\{2,r_1\}$,
	then we finish the proof of our claim with $K=1$. Otherwise, there is $r_2\geq \max\{2,r_1\}$ such that, passing to a subsequence, we find $(y_n^2)\subset\R^N$ and a constant $c_2>0$ such that
	\begin{equation}\label{eq:LemProofLions2}
	\int_{B(y_n^{2},r_2)}|v_n^1|^2\, dx \geq c_{2}\geq \frac{1}{2}\sup_{y\in\R^N} \int_{B(y,r_2)}|v_n^1|^2\, dx
	\geq \frac{1}{4}\sup_{r>0,y\in\R^N} \int_{B(y,r)}|v_n^1|^2\, dx>0
	\end{equation}
	and $|y_n^2|\geq n-r_2$. Moreover $|y_n^2-y_n^1|\geq n-r_2-r_1$. Otherwise $B(y_n^2,r_2)\subset B(y_n^1,n)$ and the convergence $u_n(\cdot+y_n^1)\chi_{B(0,n)}\to \tu_1$ in $L^2(\R^N)$ contradicts \eqref{eq:LemProofLions2}.
	Then we find $\tu_2\neq 0$ such that passing to a subsequence 
	$$v_n^1(\cdot+y_n^2),\;u_n(\cdot +y_n^2)\weakto \tu_2\text{ in }H^1(\R^N)\hbox{ and }u_n(\cdot+y_n^2)\chi_{B(0,n)}\to \tu_2\hbox{ in } L^2(\R^N).$$ Again, if 
	\begin{equation*}
	\lim_{n\to\infty}\sup_{y\in\R^N}\int_{B(y,r)}|v_n^2|^2\, dx=0,
	\end{equation*}
	for every $r\geq \max\{3,r_2\}$, where $v_n^2:=v_n^1-\tu_2(\cdot-y_n^2)$,
	then we finish proof with $K=2$.
	Continuing the above procedure we finally find $K\in \N\cup \{\infty\}$ such that for $0\leq i<K+1$, \eqref{Eqxnxm1}--\eqref{EqIntegralunSumci} and \eqref{EqSplit2a} are satisfied. Now we show that \eqref{EqSplit3a} holds.
	Observe that
	\begin{eqnarray*}
	\lim_{n\to\infty}\int_{\mathbb{R}^N}\Psi(u_n)-\Psi(v_n^0)\, dx &=&\int_{\mathbb{R}^N}\Psi(\tu_0)\, dx.
	\end{eqnarray*}
	Indeed, by Vitali's convergence theorem 
	\begin{eqnarray*}
		\int_{\mathbb{R}^N}\Psi(u_n)-\Psi(v_n^0)\, dx
		&=&\int_{\mathbb{R}^N}\int_0^1 -\frac{d}{ds}\Psi(u_n-s\tu_0)\, ds\,dx\\\nonumber
		&=&\int_{\mathbb{R}^N}\int_0^1 \Psi'(u_n-s\tu_0)\tu_0\,ds\, dx\\\nonumber
		&\rightarrow& \int_0^1 \int_{\mathbb{R}^N} \Psi'(\tu_0-s\tu_0)\tu_0\,dx\,ds\\\nonumber
		&=&\int_{\mathbb{R}^N}\int_0^1 -\frac{d}{ds}\Psi(\tu_0-s\tu_0)\, ds\, dx\\
		&=&\int_{\mathbb{R}^N}\Psi(\tu_0)\, dx\nonumber
	\end{eqnarray*}
	as $n\to\infty$. 
	Then 
	\begin{equation}\label{eq:Vit1}
	\limsup_{n\to\infty}\int_{\mathbb{R}^N}\Psi(u_n)\, dx =\int_{\mathbb{R}^N}\Psi(\tu_0)\, dx+\limsup_{n\to\infty}\int_{\R^N}\Psi(v_n^0)\, dx
	\end{equation}
	and \eqref{EqSplit3a} holds for $i=0$.
	Similarly as above 
	we show that
	$$\lim_{n\to\infty}\int_{\mathbb{R}^N}\Psi((u_n-\tu_0)(\cdot +y_n^1))-\Psi(v_n^1(\cdot +y_n^1))\, dx=\int_{\mathbb{R}^N}\Psi(\tu_1)\, dx.$$
	In view of \eqref{eq:Vit1} we obtain
	\begin{eqnarray*}
		\limsup_{n\to\infty}\int_{\mathbb{R}^N}\Psi(u_n)\, dx &=&\int_{\mathbb{R}^N}\Psi(\tu_0)\, dx+\limsup_{n\to\infty}\int_{\R^N}\Psi(u_n-\tu_0)\, dx\\	
		&=&\int_{\mathbb{R}^N}\Psi(\tu_0)\, dx+\int_{\mathbb{R}^N}\Psi(\tu_1)\, dx+\limsup_{n\to\infty}\int_{\R^N}\Psi(v_n^1)\, dx.
	\end{eqnarray*}
	Continuing the above procedure we prove that \eqref{EqSplit3a} holds for every $i\geq 0$. Now observe that, if there is $i\geq 0$ such that
	\begin{equation*}%\label{eq:Lionsinfty}
	\lim_{n\to\infty}\sup_{y\in\R^N}\int_{B(y,r)}|v_n^i|^2\, dx=0
	\end{equation*}
	for every $r\geq \max\{i,r_i\}$,
	then $K=i$. If in addition \eqref{eq:Psi2} holds, then in view of Lemma \ref{lem:Lions} we obtain that
	$$\lim_{n\to\infty}\int_{\R^N}\Psi(v_n^i)\, dx=0$$
	and we finish the proof by setting $\tu_j=0$ for $j>i$. 
	Otherwise we have $K=\infty$. It remains to prove \eqref{EqSplit4a} in this case. Note that by \eqref{EqIntegralunSumci} we have
	\begin{eqnarray*}
		c_{k+1}&\leq& \int_{B(y_n^{k+1},r_{k+1})}
		|v_n^k|^2\, dx\\
		&\leq&2\int_{B(y_n^{k+1},r_{k+1})}|v_n^i|^2\, dx+2\int_{B(y_n^{k+1},r_{k+1})}\Big|\sum_{j=i+1}^{k}\tu_j(\cdot -y_n^j)\Big|^2\, dx\\
		&\leq& 8 c_{i+1} + 2(k-i)\sum_{j=i+1}^{k}\int_{B(y_n^{k+1}-y_n^j,r_{k+1})}|\tu_j|^2\, dx
	\end{eqnarray*}
	for any $0\leq  i<k$.
	Taking into account \eqref{Eqxnxm} and letting $n\to\infty$ we get $c_{k+1}\leq 8 c_{i+1}$. Take $k\geq 1$ and   $n>4r_{k}$. Again by \eqref{EqIntegralunSumci} and \eqref{Eqxnxm}  we obtain
	\begin{equation*}%\label{eq:Step7_1}
	\begin{aligned}
	\frac{1}{32}\sup_{y\in\R^N}\int_{B(y,r_{k+1})}|v_n^k|^2\,dx
	&\leq \frac{1}{16} c_{k+1}\leq \frac{1}{2k} \sum_{i=0}^{k-1} c_{i+1}\leq\frac{1}{2k} \sum_{i=0}^{k-1}
	\int_{B(y_n^{i+1},r_{i+1})}|v_n^i|^2\, dx\\
	&\leq\frac{1}{k} \sum_{i=0}^{k-1}
	\int_{B(y_n^{i+1},r_{i+1})}|u_n|^2 + \Big|\sum_{j=0}^{i}\tu_j(\cdot -y_n^j)\Big|^2\, dx\\
	&= \frac{1}{k}
	\int_{\bigcup_{i=0}^{k-1}B(y_n^{i+1},r_{i+1})}|u_n|^2\,dx +\frac1k\int_{\R^N} \Big|\sum_{i=0}^{k-1}\sum_{j=0}^{i}\tu_j(\cdot -y_n^j)\chi_{B(y_n^{i+1},r_{i+1})}\Big|^2\, dx\\
	&\leq \frac{1}{k}
	|u_n|_2^2+\frac1k\Big|\sum_{i=0}^{k-1}\sum_{j=0}^{i}\tu_j(\cdot -y_n^j)\chi_{B(y_n^{i+1},r_{i+1})}\Big|^2_2,
	\end{aligned}
	\end{equation*}
	where $|\cdot|_p$ denotes the $L^p$-norm for $p\geq 1$.
	Observe that by \eqref{Eqxnxm} and since $n>4r_{k}$ we have
	$$B(y_n^{i+1}-y_n^j,r_{i+1})\subset \R^N\setminus B(0,n-3r_k)\hbox{ for }0\leq j< i<k$$
	%$$\|x\|\geq \|y_n^{i}-y_n^j\|-\|x-(y_n^{i}-y_n^j)\|\geq n-3r$$
	and
	\begin{eqnarray*}
		\Big|\sum_{i=0}^{k-1}\sum_{j=0}^{i}\bar{u}_j(\cdot -y_n^j)\chi_{B(y_n^{i+1},r_{i+1})}\Big|_{2}&\leq& \nonumber
		\sum_{i=0}^{k-1}\sum_{j=0}^{i}\big|\bar{u}_j\chi_{B(y_n^{i+1}-y_n^j,r_{i+1})}\big|_{2}\leq \sum_{i=0}^{k-1}\sum_{j=0}^{i}\big|\bar{u}_j\chi_{\R^N\setminus B(0,n-3r_k)}\big|_{2}\\
		&\leq& k\sum_{j=0}^{k-1}\big|\bar{u}_j\chi_{\R^N\setminus B(0,n-3r_k)}\big|_{2}\to 0
	\end{eqnarray*}
	as $n\to\infty$. Hence
	\begin{equation}\label{eq:ineqGer1}
	\limsup_{n\to\infty}\Big(\sup_{y\in\R^N}\int_{B(y,r_{k+1})}|v_n^k|^2\,dx\Big)\leq  \frac{32}{k}
	\limsup_{n\to\infty}|u_n|_2^2,
	\end{equation}
and
	suppose that \eqref{EqSplit4a} does not holds, that is
	\begin{equation}\label{eq:ineqGer2}
	\limsup_{i\to\infty} \Big(\limsup_{n\to\infty} \int_{\R^N}\Psi(v_n^i)\, dx\Big)>\delta
	\end{equation}
	for some $\delta>0$. Then we find increasing sequences $(i_k), (n_k)\subset \N$ such that
	$$\int_{\R^N}\Psi(v_{n_k}^{i_k})\, dx>\delta$$
	and 
	$$\sup_{y\in\R^N}\int_{B(y,r_{k+1})}|v_{n_k}^{i_k}|^2\,dx\leq \limsup_{n\to\infty}\Big(\sup_{y\in\R^N}\int_{B(y,r_{k+1})}|v_n^{i_k}|^2\,dx\Big)+\frac{1}{i_k}.$$
	Since \eqref{eq:ineqGer1} holds, we get
	$$\lim_{k\to\infty} \Big(\sup_{y\in\R^N}\int_{B(y,r_{k+1})}|v_{n_k}^{i_k}|^2\,dx\Big)=0,$$
	and in view of Lemma \ref{lem:Lions} we obtain that
	$$\lim_{k\to\infty} \int_{\R^N}\Psi(v_{n_k}^{i_k})\, dx=0,$$
	which is a contradiction. Hence \eqref{EqSplit4a}  is satisfied.			
\end{altproof}

Now we observe that in Theorem \ref{ThGerard} we may find translations $(y_n^i)_{i=0}^\infty\subset  \{0\}\times\{0\}\times\R^{N-2m}$ provided that
$(u_n)\subset H^{1}_{\cO_1}(\R^N)$ and $2\leq m< N/2$ .
\begin{Cor}\label{CorGerard}
	Suppose that $(u_n)\subset H^{1}_{\cO_1}(\R^N)$ is bounded and $2\leq m< N/2$.
	Then there are sequences
	$(\tu_i)_{i=0}^\infty\subset H^1_{\cO_1}(\R^N)$, $(y_n^i)_{i=0}^\infty\subset  \{0\}\times\{0\}\times\R^{N-2m}$ for any $n\geq 1$, such that the statements of Theorem \ref{ThGerard} are satisfied.
\end{Cor}
\begin{proof}
A careful inspection of proof of Theorem \ref{ThGerard} leads to the following claim: there is $K\in \N\cup \{\infty\}$ and there is a sequence
$(\tu_i)_{i=0}^K\subset H^1(\R^N)$, for $0\leq i <K+1$ there are sequences $(v_n^i)\subset H^1_{\cO_1}(\R^N)$, $(y_n^i)\subset \{0\}\times\{0\}\times\R^{N-2m}$ and positive numbers $(c_i)_{i=0}^{K}, (r_i)_{i=0}^{K}$ such that $y_n^0=0$, $r_0=0$ and, up to a subsequence, for any $n$ and $0\leq i<K+1$ one has
\eqref{Eqxnxm1}-\eqref{Eqxnxm3},
\begin{eqnarray*}
&&\int_{B(y_n^{i},r_i)}|v_n^{i-1}|^2\, dx \geq c_{i}\geq\frac{1}{2}\sup_{y\in\R^{N-2m}}\int_{B((0,0,y),r_i)}|v_n^{i-1}|^2\, dx\\
&&\hspace{4.1cm}	\geq \frac{1}{4}\sup_{r>0,y\in\R^{N-2m}} \int_{B((0,0,y),r)}|v_n^{i-1}|^2\, dx\nonumber
>0, r_i\geq \max\{i,r_{i-1}\} \hbox{ for }i\geq 1,
\end{eqnarray*}
and \eqref{EqSplit2a}, \eqref{EqSplit3a} are satisfied. In order to prove \eqref{EqSplit4a} we use Corollary \ref{CorLions1} instead of Lemma \ref{lem:Lions}. 
\end{proof}

Observe that if $m=N/2$, then we consider $\cO_2$-invariant sequences. In general we assume that  $N-2m\neq 1$ and we have the following result.

\begin{Cor}\label{cor:LionsO2}
Suppose that $(u_n)\subset H^{1}_{\cO_2}(\R^N)$ is bounded.
Then passing to a subsequence we find $\tu_0\in H^{1}_{\cO_2}(\R^N)$ such that
\begin{eqnarray*}%\label{EqSplit1a}
\nonumber
&& u_n\weakto \tu_0\; \hbox{ in } H^1(\R^N)\text{ as }n\to\infty,\\
%&&\tu_i\neq 0\text{ for }  i \geq 1,\\
%\label{EqSplit2a}
&& \lim_{n\to\infty}\int_{\R^N}|\nabla u_n|^2\, dx= \int_{\R^N}|\nabla\tu_0|^2\, dx+\lim_{n\to\infty}\int_{\R^N}|\nabla (u_n-\tu_0)|^2\, dx,\\
&& \limsup_{n\to\infty}\int_{\R^N}\Psi(u_n)\, dx= 
\int_{\R^N}\Psi(\tu_0)\, dx+\limsup_{n\to\infty}\int_{\R^N}\Psi(u_n-\tu_0)\, dx	%\label{EqSplit3a}
\end{eqnarray*}
for any function $\Psi:\R\to[0,\infty)$ of class $\cC^1$ such that $\Psi'(s)\leq C(|s|+|s|^{2^*-1})$ for any $s\in\R$ and some constant $C>0$.
Moreover, if $\Psi$ satisfies \eqref{eq:Psi2},
then
\begin{equation*}%\label{EqSplit4a}
\lim_{n\to\infty}\int_{\R^N}\Psi(u_n-\tu_0)\, dx=0
\end{equation*}
and if $s\mapsto |\Psi'(s)s|$ satisfies \eqref{eq:Psi2}, then
\begin{equation}\label{EqPsiLast}
\lim_{n\to\infty}\int_{\R^N}\Psi'(u_n)u_n\, dx=\int_{\R^N}\Psi'(u_0)u_0\, dx.
\end{equation}
\end{Cor}
\begin{proof}
Similarly as in proof of Theorem \ref{ThGerard} we show that passing to a subsequence\\ $\lim_{n\to\infty}\int_{\R^N}|\nabla u_n|^2\,dx$ exists and 
\eqref{eq:Vit1} holds. Then we apply Corollary \ref{CorLions2} instead of Lemma \ref{lem:Lions}. In order to prove \eqref{EqPsiLast} observe that 
\begin{eqnarray*}
\int_{\R^N}|\Psi'(u_n)u_n-\Psi'(u_0)u_0|\, dx
&\leq& \int_{\R^N}|\Psi'(u_n)||u_n-u_0|\, dx+\int_{\R^N}|\Psi'(u_n)-\Psi'(u_0)||u_0|\, dx.
%&\leq& \int_{\R^N}|\Psi'(u_n)-\Psi'(u_0)||u_0|\, dx+\int_{\R^N}|\Psi'(u_n)||u_n-u_0|\, dx
\end{eqnarray*}
Take any $\eps>0$, $2<p<2^*$ and we find $0<\delta<M$ and $c_\eps>0$ such that 
\begin{eqnarray*}
	|\Psi'(s)|&\leq& \eps (|s|+|s|^{2^*-1})\quad\hbox{ if }|s|\in [0,\delta]\hbox{ or }|s|>M,\\
	|\Psi'(s)|&\leq& c_\eps |s|^{2^*(1-\frac1p)}\quad\hbox{ if }|s|\in (\delta,M].
\end{eqnarray*}
Then, passing to a subsequence, $u_n\to u_0$ in $L^p(\R^N)$ and
we infer that 
$\int_{\R^N}|\Psi'(u_n)||u_n-u_0|\, dx\to 0$
 and
by the Vitali's convergence theorem  $ \int_{\R^N}|\Psi'(u_n)-\Psi'(u_0)||u_0|\, dx\to 0$ as $n\to\infty$.
\end{proof}


\section{Proofs of Theorems \ref{ThMain1},  \ref{ThMain2} and  \ref{ThMain3}}\label{sec:proof}

Similarly as in \cite{BerLions} we modify $g$ in the following way. If $g(s)\geq 0$ for all $s\geq \xi_0$, then $\tilde{g}=g$. Otherwise we set
$\xi_1:=\inf\{\xi\geq \xi_0: g(\xi)=0\}$, 
\begin{equation*}
%\label{eq}
\tilde{g}(s)=\left\{
\begin{array}{ll}
g(s)&\quad\hbox{if }0\leq s \leq \xi_1,\\
0 &\quad\hbox{if }s>\xi_1,
\end{array}
\right.
\end{equation*}
and $\tilde{g}(s)=-\tilde{g}(-s)$ for $s<0$. Hence $\tilde{g}$ satisfies the same assumptions (g0)--(g3) and by the strong maximum principle if $u\in H^1(\R^N)$ solves $-\Delta u=\tilde{g}(u)$, then $|u(x)|\leq \xi_1$ and $u$ is a solution of \eqref{eq}. Hence, from now on we replace $g$ by $\tilde{g}$ and we use the same notation $g$ for the modified function $\tilde{g}$. Then observe that instead of (g2) we may assume
\begin{itemize}
	\item[(g2)'] $\lim_{s\to \infty}g(s)/s^{2^*-1}=0$.
\end{itemize}
Let $g_1(s)=\max\{g(s)+ms,0\}$ and $g_2(s)=g_1(s)-g(s)$ for $s\geq 0$ and $g_i(s)=-g_i(-s)$ for $s<0$. Then $g_1(s),g_2(s)\geq 0$ for $s\geq 0$,
\begin{eqnarray}\label{eq:NewCond1}
\lim_{s\to 0} g_1(s)/s&=&\lim_{s\to \infty}g_1(s)/s^{2^*-1}=0\\
g_2(s)&\geq& m s \quad \hbox{ for }s\geq 0,\label{eq:NewCond2}
\end{eqnarray}
 and let
$$G_i(s)=\int_0^s g_i(t)\, dt\quad \hbox{ for }i=1,2.$$ 
%The above decomposition $g=g_1-g_2$ has been used in \cite{BerLions,BerLionsII}. In order to prove the multiplicity result in Theorem \ref{ThMain3} (b), we introduce a new function $g_3$
%$$g_3(s):=g_2(s)-ms=\max\{0,-g(s)-ms\}=-\min\{0,g(s)+ms\}\geq 0,$$
%and observe that if the following condition holds
%\begin{itemize}
%	\item[(g1)'] $-\infty<\lim_{s\to 0}g(s)/s=-m<0$,
%\end{itemize}
%then
%\begin{eqnarray}\label{eq:NewCond3}
%\lim_{s\to 0} g_3(s)/s&=&\lim_{s\to \infty}g_3(s)/s^{2^*-1}=0.
%\end{eqnarray}
Now let us consider the standard norm of $u$ in $H^1(\R^N)$ given by
$$\|u\|^2=\int_{\R^N}|\nabla u|^2+|u|^2\, dx.$$
In view of \cite{BerLions}[Theorem A.VI], $J:H^1(\R^N)\to \R$ given by \eqref{eq:action} is of class $\cC^1$. In the next subsections we build the variational setting according to Section \ref{sec:criticaltheory}.

\subsection{Critical point theory setting}
Let $X=Y=H^1(\R^N)$ and let $M,\psi:H^1(\R^N)\to \R$ be given by 
$$M(u)=\int_{\R^N}|\nabla u|^2\, dx-2^*\int_{\R^N}G(u)\, dx,\hbox{ and }\psi(u)=\int_{\R^N}|\nabla u|^2\, dx\quad\hbox{ for }u\in H^1(\R^N).$$ 
\begin{Prop}\label{prop:defMPSU}
	Let us denote
	\begin{eqnarray*}
		\cM&:=&\Big\{u\in H^1(\R^N): M(u)=0\Big\},\\
		\cS&:=&\Big\{u\in H^1(\R^N): \psi(u)=1\Big\},\\
		\cP&:=&\Big\{u\in H^1(\R^N): \int_{\R^N}G(u)\, dx>0\Big\},\\
		\cU&:=&\cS\cap\cP.
	\end{eqnarray*}
	Then the following holds.\\
	(i) There is a continuous map $m_\cP:\cP\to \cM$ such that $m_\cP(u)=u(r\cdot)\in\cM$ with 
	\begin{equation}\label{eq:defOfR}
	r=r(u)=\Big(\frac{2^*\int_{\R^N}G(u)\, dx}{\psi(u)}\Big)^{1/2}>0.
	\end{equation}
	(ii)  $m:=m_\cP|_{\cU}:\cU\to\cM$ is a homeomorphism with the inverse $m^{-1}(u)=u(\psi(u)^{\frac{1}{N-2}} \cdot)$, $J\circ m_{\cP}:\cP\to\R$ is of class $\cC^1$ with
	\begin{eqnarray*}
	(J\circ m_\cP)'(u)(v)&=&J'(m_\cP(u))(v(r(u)\cdot))\\
	&=&r(u)^{2-N}\int_{\R^N}\langle \nabla u,  \nabla v\rangle\,dx-r(u)^{-N}\int_{\R^N}g(u)v\, dx
	\end{eqnarray*}
	for $u\in\cP$ and $v\in H^1(\R^N)$.\\
	(iii) $J$ is coercive on $\cM$, i.e. for $(u_n)\subset \cM$, $J(u_n)\to\infty$ as $\|u_n\|\to\infty$,  and
	\begin{equation}\label{eq:infJM}
	c:=\inf_{\cM} J>0.
	\end{equation}
	(iv) If $u_n\to u$, $u_n\in \cU$ and $u\in\partial\cU$, where the boundary of $\cU$ is taken in $\cS$, then $(J\circ m)(u)\to\infty$ as $n\to\infty$.
\end{Prop}
\begin{proof}
(i)	If $u\in \P$ then 
	\begin{eqnarray*}
		M(u(r\cdot)) &=& r^{-N}\Big(r^{2}\int_{\R^N} |\nabla u|^2\, dx - 2^*\int_{\R^N} G(u)\,dx\Big)=0
	\end{eqnarray*}
	for $r=r(u)$ given by \eqref{eq:defOfR}.
	Let $m_{\cP}:\P\to\cM$ be a map such that 
	$$m_{\cP}(u):=u(r(u)\cdot).$$
	Let $u_n\to u_0$, $u_n\in \cP$ for $n\geq 0$. 
	Observe that $r(u_n)\to r(u_0)$ and
	\begin{eqnarray*}
		\psi(m_\cP(u_n)-m_\cP(u_0)) &=& \int_{\R^N} \big| \nabla \big(u_n(r(u_n)\cdot )-u_0(r(u_0)\cdot)\big)\big|^2\, dx\\
		&\leq & 2r(u_n)^{2-N}\psi(u_n-u_0)+ 2\int_{\R^N} \big| \nabla \big(u_0(r(u_n)\cdot)-u_0(r(u_0)\cdot)\big)\big|^2\, dx\\
	%	&= & r(u_n)^{2-N}\psi(u_n-u_0) + \int_{\R^N} \big| \nabla \big(u_0(r(u_n)x)-u_0(r(u_n)x)\big)\big|^2\, dx\\
		&\leq & 2r(u_n)^{2-N}\psi(u_n-u_0) + 2\big(r(u_n)^{2-N}-r(u_0)^{2-N}\big)\psi(u_0)\\
		&&+4r(u_0)^{2-N}
		\int_{\R^N}\Big\langle \nabla u_0-\nabla u_0\Big(\frac{r(u_n)}{r(u_0)}\cdot\Big),\nabla u_0\Big\rangle\, dx\\
		&\to& 0
	\end{eqnarray*}
	passing to a subsequence. Similarly we show that $m_{\cP}(u_n)\to m_{\cP}(u_0)$ in $L^2(\R^N)$, hence 
	$m_\cP$ is continuous.\\ 
(ii) Observe that
	$$m_\cP^{-1}(u):=\{v\in \P: m_\cP(v)=u\}=\{u_\lambda: u_\lambda=u(\lambda \cdot),\; \lambda>0\}.$$
Then
$1=\psi(u_\lambda)=\lambda^{2-N}\psi(u)$
if and only if $\lambda = \psi(u)^{\frac{1}{N-2}}$. Therefore $m^{-1}(u)=u( \psi(u)^{\frac{1}{N-2}}\cdot)\in \cU$. Similarly as in (i) we show the continuity of $m^{-1}:\cM\to\cU$.
	Moreover for $u\in\cP$ and $v\in X$ one obtains
	$$
	 \begin{aligned}
		(J\circ m_{\cP})&'(u)(v)
		=\lim_{t\to 0}\frac{J(m_{\cP}(u+tv))-J(m_{\cP}(u))}{t}\\
		&=\lim_{t\to 0}\frac{(r(u+tv)^{2-N}-r(u)^{2-N})\int_{\R^N} |\nabla u|^2\, dx
			+r(u+tv)^{2-N}t\int_{\R^N} \langle \nabla(2u+tv),\nabla v\rangle\, dx}{2t}\\
		&\hspace{5mm}- \lim_{t\to 0}\frac{(r(u+tv)^{-N}-r(u)^{-N})\int_{\R^N} G(u)\,dx+r(u+tv)^{-N}\int_{\R^N}G(u+tv)-G(u)\, dx}{t}
			 \end{aligned}
				$$
			 		$$
			 		\begin{aligned}
		&\hspace{-15mm}=\frac{2-N}{2}r(u)^{1-N}r'(u)(v)\int_{\R^N}|\nabla u|^2\, dx+r(u)^{2-N}\int_{\R^N}\langle \nabla u, \nabla v\rangle\,dx\\
		&\hspace{-10mm}-\Big((-N)r(u)^{-N-1}r'(u)(v)\int_{\R^N}G(u)\, dx+r(u)^{-N}\int_{\R^N}g(u)v\, dx\Big)\\
%	 \end{aligned}
%	$$
%		$$
%		\begin{aligned}
		&\hspace{-15mm}=
		\frac{2-N}{2}r(u)^{-N-1}r'(u)(v)\Big( r(u)^2\int_{\R^N}|\nabla u|^2\, dx-
		2^*\int_{\R^N}G(u)\, dx\Big)\\
		&\hspace{-10mm}+ r(u)^{-N}\Big(r(u)^2\int_{\R^N}\langle \nabla u,  \nabla v\rangle\,dx-\int_{\R^N}g(u)v\, dx\Big)\\
		&\hspace{-15mm}=\frac{2-N}{2}r(u)^{-1} r'(u)(v) M(m_{\cP}(u)) + J'(m_\cP(u))(v(r(u)\cdot)\\
		&\hspace{-15mm}=J'(m_{\cP}(u))(v(r(u)\cdot).
		\end{aligned}
	$$
(iii) Suppose that for some $(u_n)\subset\cU$
$$J(m(u_n))=\Big(\frac12-\frac{1}{2^*}\Big)\int_{\R^N}|\nabla m(u_n)|^2\, dx$$
is bounded.  Then we obtain that $m(u_n)$ is bounded in $L^{2^*}(\R^N)$ and by \eqref{eq:NewCond1}, $\int_{\R^N}G_1(m(u_n))\, dx$ is bounded as well. By \eqref{eq:NewCond2} and since $m(u_n)\in\cM$, we infer that $m(u_n)$ is bounded in $H^1(\R^N)$. Thus $J$ is coercive on $\cM$. Observe that 
for some constants $0<C_1<C_2$ one has
\begin{eqnarray*}
|m(u_n)|_{2^*}^2+|m(u_n)|_{2}^2&\leq& C_1 \int_{\R^N}|\nabla m(u_n)|^2+2^*G_2(m(u_n))\, dx
=C_1 2^*\int_{\R^N}G_1(m(u_n))\, dx\\
& \leq& |m(u_n)|_{2}^2+C_2 |m(u_n)|_{2^*}^{2^*}
\end{eqnarray*}
and 
we conclude that $|m(u_n)|_{2^*}\geq C_2^{-1/(2^*-2)}>0$. Hence $c=\inf_{\cM} J>0$.\\
(iv) Note that if $u_n\to u\in\partial \cU$ and $u_n\in \cU$, then $r(u_n)\to 0$ and 
$$\|m(u_n)\|^2=r(u_n)^{2-N}+r(u_n)^{-N}|u_n|_2^2\to \infty$$
as $n\to\infty$. Hence by the coercivity,
$J(m(u_n))\to\infty$ as $n\to\infty$.
\end{proof}


Now observe that we may consider the group of translations $G=\R^N$ acting on $X=H^1(\R^N)$, i.e.  
$$(gu)(x)=u(x+g)$$ for $g\in\R^N$, $u\in X$, $x\in\R^N$,
and in view of Proposition \ref{prop:defMPSU} conditions (A1)--(A3) are satisfied.\\
\indent In the similar way we may consider the following subgroup of translations $G=\{0\}\times\{0\}\times\R^{N-2m}$ acting on $X=X_\tau\cap H^1_{\cO_1}(\R^N)$ and conditions (A1)--(A3) are satisfied provided that instead of  $\cM$, $\cS$, $Y=X=H^1(\R^N)$, $\cU$, $m_\cP$ and $m$, we consider $\cM\cap X$, $\cS\cap X$, $Y=X=X_\tau\cap H^1_{\cO_1}(\R^N)$, $\cU\cap X$, $m|_{\cP\cap X}:\cP\cap X\to \cM\cap X$ and $m|_{\cU\cap X}:\cU\cap X\to \cM\cap X$ respectively.\\
\indent Finally, in case of $X=X_\tau\cap H^1_{\cO_2}(\R^N)$ we consider the trivial group $G=\{(0,0,0)\}$ acting on  $X$.

\begin{Rem}\label{remTau}
We show how to easy construct functions in $X_\tau\cap H^1_{\cO_2}(\R^N)$. Let 
$u\in H^1_0(B(0,R))\cap L^{\infty}(B(0,R))$ be $\cO(N)$-invariant (radial) function, $R>1$ and take any odd and smooth function $\vp:\R\to [0,1]$ such that $\vp(x)=1$ for $x\geq 1$ and $\vp(x)=-1$ for $x\leq -1$. Note that, defining
\begin{equation*}
\tu(x_1,x_2,x_3):=u\big(\sqrt{|x_1|^2+|x_2|^2+|x_3|^2}\big)\vp(|x_1|-|x_2|)\;\hbox{ for }x_1,x_2\in\R^m,x_3\in\R^{N-2m},
\end{equation*}
we get $\tu\in X_\tau\cap H^1_{\cO_2}(\R^N)$. Take $A:=\mathrm{ess \sup }\;|u|$ and $B:=\max_{s\in [0,A]}|G(s)|$. Let us denote $r=|x|$ and $r_i=|x_i|$ for $i=1,2,3$. Observe that
\begin{eqnarray*}
\int_{\R^N} G(\tu)\, d x&=&\int_{0}^\infty\int_{0}^\infty\int_0^\infty  G(\tu)r_1^{m-1}r_2^{m-1}r_3^{N-2m-1}\, d r_1dr_2 dr_3\\
&=&2\int_{0}^R\int_{0}^R\int_{r_2}^{r_2+R}G(\tu)r_1^{m-1}r_2^{m-1}r_3^{N-2m-1}\, d r_1dr_2 dr_3
%&=&2\int_{0}^R\int_{0}^R\int_{r_2}^{r_2+R}G\Big(v\big(\sqrt{|r_1-r_2|^2+r_3^2}\big)\vp(r_1-r_2)\Big)r_1^{m-1}r_2^{m-1}r_3^{N-2m-1}\, d r_1dr_2 dr_3\\
\end{eqnarray*}
\begin{eqnarray*}
&=&2\int_{0}^R\int_{0}^R\int_{r_2}^{r_2+R}G(u(r))r_1^{m-1}r_2^{m-1}r_3^{N-2m-1}\, d r_1dr_2 dr_3\\
&&-2\int_{0}^R\int_{0}^R\int_{r_2}^{r_2+1}G(u(r))r_1^{m-1}r_2^{m-1}r_3^{N-2m-1}\, d r_1dr_2 dr_3\\
&&+2\int_{0}^R\int_{0}^R\int_{r_2}^{r_2+1}G(u(r)\vp(r_1-r_2))r_1^{m-1}r_2^{m-1}r_3^{N-2m-1}\, d r_1dr_2 dr_3\\
&\geq& \int_{\R^N} G(u)\, dx-c_1B\Big(\sum_{i=N-m}^{N-1}R^i\Big)
\end{eqnarray*}
for some constant $c_1>0$ dependent only on $N$. In \cite{BerLions}[page 325], for any $R>0$ one can find a radial function $u\in H^1_0(B(0,R))\cap L^{\infty}(B(0,R))$ such that $\int_{\R^N}G(u)\, dx\geq c_2 R^N-c_3 R^{N-1}$ for some constants $c_2,c_3>0$. Therefore we get $\int_{\R^N}G(\tu)\, dx>0$ for sufficiently large $R$, hence $\cP\cap X_\tau\cap H^1_{\cO_2}(\R^N)\neq\emptyset$ and $\cM\cap X_\tau\cap H^1_{\cO_2}(\R^N)\neq\emptyset$.
\end{Rem}
 
\subsection{$\theta$-analysis of Palais-Smale sequences}

Below we explain the role of $\theta$ in the analysis of Palais-Smale sequences of $J\circ m$.

\begin{Lem}\label{lem:theta}
Suppose that $(u_n)\subset \cU$ is a $(PS)_\beta$-sequence of $J\circ m$ such that 
	$$m(u_n)(\cdot +y_n)\weakto\tu\neq 0\hbox{ in } H^1(\R^N)$$ for some sequence $(y_n)\subset \R^N$ and $\tu\in H^1(\R^N)$. Then $\tu$ solves
	\begin{equation}\label{eq:thetaEQ}
	-\theta \Delta u = g(u),\hbox{ where }\theta:=\psi(\tu)^{-1}\int_{\R^N}g(\tu)\tu\,dx,
	\end{equation}
	and  passing to a subsequence 
	\begin{equation}\label{eq:theta2}
	\theta=\lim_{n\to\infty}\psi(m(u_n))^{-1}\int_{\R^N} g(m(u_n))m(u_n)\,dx.
	\end{equation}
	Moreover $\theta\neq 0$ and
	\begin{equation}\label{eq:theta}
	\theta=2^*\psi(\tu)^{-1}\int_{\R^N}G(\tu)\,dx.
	\end{equation}
If $\theta>0$, then $m_{\cP}(\tu)\in\cM$ is a critical point of $J$. If $\theta\geq 1$, then $J(m_{\cP}(\tu))\leq \beta$.
	\end{Lem}
\begin{proof}
For $v\in X$ we set $v_n(x)=v(r(u_n)^{-1}x-y_n)$ and observe that passing to a subsequence $m(u_n)(x +y_n)\to\tu(x)$  for a.e. $x\in\R^N$ and by Vitali's convergence theorem
\begin{eqnarray}\nonumber
(J\circ m)'(u_n)(v_n)
&=& \int_{\R^N}\langle \nabla m(u_n)(\cdot+y_n),\nabla v\rangle\, dx -
\int_{\R^N}g(m(u_n)(\cdot+y_n))v\,dx\\\label{eq:conv22q}
&\to& \int_{\R^N}\langle \nabla \tu,\nabla v\rangle\, dx -\int_{\R^N}g(\tu)v\,dx.
\end{eqnarray}
We find the following decomposition
$$v_n=\Big(\int_{\R^N}\langle \nabla u_n,\nabla v_n\rangle\,dx\Big)u_n+\tv_n$$
with $\tv_n\in T_{u_n}\cS$.  In view of Proposition \ref{prop:defMPSU} (iii) we get that $r(u_n)$ is bounded from above, bounded away from $0$ and  passing to a subsequence $r(u_n)\to r_0>0$. Note that $(v_n)$ is bounded, hence $(\tv_n)$ is bounded and $(J\circ m)'(u_n)(\tv_n)\to 0$. Moreover
\begin{eqnarray*}
\int_{\R^N}\langle \nabla u_n,\nabla v_n\rangle\,dx&=&
r(u_n)^{N-2}
\int_{\R^N}\langle \nabla m(u_n), \nabla v_n(r(u_n)\cdot)\rangle\,dx\\
&=&r(u_n)^{N-2}\int_{\R^N}\langle \nabla m(u_n)(\cdot+y_n), \nabla v\rangle\,dx\\
&\to& r_0^{N-2} \int_{\R^N}\langle \nabla \tu,\nabla v\rangle\,dx=0
\end{eqnarray*}
provided that $\int_{\R^N}\langle\nabla \tu,\nabla v \rangle\, dx=0$.
Hence
$$(J\circ m)'(u_n)(v_n)=\Big(\int_{\R^N}\langle \nabla u_n,\nabla v_n\rangle\,dx\Big)(J\circ m)'(u_n)(u_n)+(J\circ m)'(u_n)(\tv_n)\to 0$$
and by \eqref{eq:conv22q} we obtain
$$\int_{\R^N}\langle \nabla \tu,\nabla v\rangle\, dx -\int_{\R^N}g(\tu)v\,dx=0$$
for any $v$ such that $\int_{\R^N}\langle\nabla \tu,\nabla v \rangle\, dx=0$. We define
$\xi:H^1(\R^N)\to \R$ by the following formula
\begin{eqnarray*}
\xi(v)&=&\int_{\R^N}\langle \nabla \tu,\nabla v\rangle\, dx -\int_{\R^N}g(\tu)v\,dx\\
&&-\Big(\int_{\R^N}|\nabla \tu|^2\, dx -\int_{\R^N}g(\tu)\tu\,dx\Big)
\psi(\tu)^{-1}\int_{\R^N}\langle\nabla \tu,\nabla v\rangle\, dx.
\end{eqnarray*}
Observe that any $v\in H^1(\R^N)$ has the following decomposition
$$v=\Big(\int_{\R^N}\langle\nabla \tu,\nabla v\rangle\, dx \Big)\tu+\tv$$
such that $\int_{\R^N}\langle \nabla \tu,\nabla \tv\rangle\,dx=0$. Note that $\xi(\tu)=0$ and
\begin{eqnarray*}
\xi(v)&=&\Big(\int_{\R^N}\langle\nabla \tu,\nabla v\rangle\, dx \Big)\xi(\tu)+\xi(\tv)\\
&=& \xi(\tv)=0
\end{eqnarray*}
for any $v\in H^1(\R^N)$.
Then
\begin{eqnarray}\label{eq:xieq}
0=\xi(v)&=& \int_{\R^N}\Big(1-\Big(\int_{\R^N}|\nabla \tu|^2\, dx -\int_{\R^N}g(\tu)\tu\,dx\Big)
\psi(\tu)^{-1}\Big)\langle\nabla  \tu,  \nabla v\rangle\,dx\\
&&-\int_{\R^N}g(\tu)v\, dx\nonumber
\end{eqnarray}
and $\tu$ is a weak solution to the problem \eqref{eq:thetaEQ}
with 
$$\theta=1-\Big(\int_{\R^N}|\nabla \tu|^2\, dx -\int_{\R^N}g(\tu)\tu\,dx\Big)\psi(\tu)^{-1}
=\psi(\tu)^{-1}\int_{\R^N}g(\tu)\tu\,dx.$$
Now we show \eqref{eq:theta2}. Let us define a map $\eta:\cP\to (H^1(\R^N))^*$ by the following formula
\begin{eqnarray*}
	\eta(u)(v)&=& (J\circ m_\cP)'(u)(v)-(J\circ m_\cP)'(u)(u)\int_{\R^N}\langle \nabla u,\nabla v\rangle\,dx
	%&=&	\int_{\R^N}\langle \nabla \tu,\nabla v\rangle\, dx -\int_{\R^N}g(\tu)v\,dx\\
	%	&&-\Big(\int_{\R^N}|\nabla \tu|^2\, dx -\int_{\R^N}g(\tu)\tu\,dx\Big)
	%	\psi(\tu)^{-1}\int_{\R^N}\langle\nabla \tu,\nabla v\rangle\, dx.
\end{eqnarray*}
for $u\in\cP$ and $v\in H^1(\R^N)$.
Observe that any $v\in H^1(\R^N)$ has the unique decomposition
$$v=\Big(\int_{\R^N}\langle\nabla u_n,\nabla v\rangle\, dx \Big)u_n+\tv_n$$
such that $\tv_n\in T_{u_n}\cS$. Note that
\begin{eqnarray*}
	\eta(u_n)(v)&=&\Big(\int_{\R^N}\langle\nabla u_n,\nabla v\rangle\, dx \Big)\eta(u_n)(u_n)+\eta(u_n)(\tv_n)\\
	&=& \eta(u_n)(\tv_n)=(J\circ m)'(u_n)(\tv_n).
\end{eqnarray*}
Since $(u_n)$ is a $(PS)_\beta$-sequence of $J\circ m$,  we obtain $\eta(u_n)\to 0$ in $(H^1(\R^N))^*$. On the other hand, in view of Proposition \ref{prop:defMPSU} (ii)
\begin{eqnarray*}
	\eta(u_n)(v(r(u_n)^{-1}x-y_n))
	%&=& \int_{\R^N}\big(r(u_n)^{2-N}-(J\circ m_\cP)(u_n)(u_n)\big)
	%	\langle\nabla  u_n,  \nabla v\rangle\,dx-r(u_n)^{-N}\int_{\R^N}g(u_n)v\, dx\\
	&=& \int_{\R^N}\big(1-r(u_n)^{N-2}(J\circ m_\cP)'(u_n)(u_n)\big)
	\langle\nabla  m(u_n)(\cdot+y_n),  \nabla v\rangle\,dx\\
	&&-\int_{\R^N}g(m(u_n)(\cdot+y_n))v\, dx\\
	&=& \int_{\R^N}\theta_n
	\langle\nabla  m(u_n)(\cdot+y_n),  \nabla v\rangle\,dx-\int_{\R^N}g(m(u_n)(\cdot+y_n))v\, dx,
\end{eqnarray*}
where
$$\theta_n=r(u_n)^{N-2}\int_{\R^N}g(m(u_n))m(u_n)\,dx=\psi(m(u_n))^{-1}\int_{\R^N}g(m(u_n))m(u_n)\,dx.$$
Passing to a subsequence $\theta_n\to\tilde{\theta}$ and
\begin{eqnarray*}
	0=\lim_{n\to\infty}\eta(u_n)(v(r(u_n)^{-1}x-y_n))
	%&=& \int_{\R^N}\big(r(u_n)^{2-N}-(J\circ m_\cP)(u_n)(u_n)\big)
	%	\langle\nabla  u_n,  \nabla v\rangle\,dx-r(u_n)^{-N}\int_{\R^N}g(u_n)v\, dx\\
	&=& \int_{\R^N}\tilde{\theta}
	\langle\nabla  \tu,  \nabla v\rangle\,dx-\int_{\R^N}g(\tu)v\, dx
\end{eqnarray*}
for any $v\in H^1(\R^N)$. Taking into account \eqref{eq:xieq} we obtain that $\tilde{\theta}=\theta$ and
\eqref{eq:theta2} is satisfied.
Now we show that $\theta\neq 0$.  Suppose that $\theta=0$, hence $g(\tu(x))=0$ for a.e. $x\in\R^N$. 
%Since $\tu\in H^1(\R^N)\setminus\{0\}$, in view of Nikodym theorem, we find $\tu'\in H^1(\R^N)\setminus\{0\}$ such that $\tu'$ is absolutely continuous on almost every line segments parallel to the coordinate directions in $\R^N$.
 Take 
$\Sigma:=\{x\in\R^N: g(\tu(x))=0\}$ 
and clearly $\R^N\setminus \Sigma$ has measure zero and let
$\Om:=\{x\in \Sigma: \tu(x)\neq 0\}$.
Suppose that
$\eps:=\inf_{x\in\Om}|\tu(x)|>0$. Since $\tu\in L^2(\R^N)\setminus\{0\}$, we infer that $\Om$ has finite positive measure 
and note that
$$\int_{\R^N}|\tu(x+h)-\tu(x)|^2\,dx\geq \eps \int_{\R^N}|\chi_\Om(x+h)-\chi_\Om(x)|^2\,dx,$$
where $\chi_\Om$ is the characteristic function of $\Om$. In view of \cite{Ziemer}[Theorem 2.1.6] we infer that $\chi_\Om\in H^1(\R^N)$ and we get the contradiction.
Therefore we find a sequence $(x_n)\subset\R^N$ such that $\tu(x_n)\to 0$, $\tu(x_n)\neq 0$ and $g(\tu(x_n))=0$. Thus, by \eqref{eq:NewCond1} and \eqref{eq:NewCond2} we obtain the next contradiction
$$0=\lim_{n\to\infty}\frac{g(\tu(x_n))}{\tu(x_n)}=\lim_{n\to\infty}\frac{g_1(\tu(x_n))}{\tu(x_n)}-\lim_{n\to\infty}\frac{g_2(\tu(x_n))}{\tu(x_n)}\leq -m<0.$$
Therefore $\theta\neq 0$ and
by the elliptic regularity we infer that
$\tu\in W^{2,q}_{loc}(\R^N)$ for any $q<\infty$. In view of the Pohozaev identity
$$\theta \int_{\R^N}|\nabla \tu|^2\, dx=2^*\int_{\R^N}G(\tu)\,dx,$$
hence \eqref{eq:theta} holds.
Now suppose that $\theta>0$. Then 
$$r(\tu)=\Big(\frac{2^*\int_{\R^N}G(\tu)\, dx}{\psi(\tu)}\Big)^{1/2}=\Big(\frac{\int_{\R^N}g(\tu)\tu\, dx}{\psi(\tu)}\Big)^{1/2}=\theta^{1/2}.$$
Observe that for $v\in X$ and $v_r=v(r(\tu)^{-1}\cdot)$ one has
\begin{eqnarray*}
J'(m_{\cP}(\tu))(v)&=&r(\tu)^{2-N}\int_{\R^N}\langle \nabla \tu,\nabla v_r\rangle\, dx -r(\tu)^{-N}\int_{\R^N}g(\tu)v_r\,dx\\
&=&r(\tu)^{-N}\Big(\int_{\R^N}\langle \nabla \theta \tu,\nabla v_r\rangle\, dx -\int_{\R^N}g(\tu)v_r\,dx\Big)=0,
\end{eqnarray*}
which finally shows that $m_{\cP}(\tu)$ is a critical point of $J$. If $\theta\geq 1$, then
\begin{eqnarray*}
\beta=\lim_{n\to\infty}J(m(u_n))&\geq& \Big(\frac{1}{2}-\frac{1}{2^*}\Big)\int_{\R^N}|\nabla \tu|^2\, dx
\geq r(\tu)^{2-N}\Big(\frac{1}{2}-\frac{1}{2^*}\Big)\int_{\R^N}|\nabla \tu|^2\, dx\\
&=&J(m_{\cP}(\tu)).
\end{eqnarray*}
\end{proof}

%\begin{Rem}\label{rem:On}
%Observe that in the radial case if $\tu\in H^1_{\cO(N)}(\R^N)\setminus\{0\}$ solves \eqref{eq:thetaEQ}, then the proof of $\theta\neq 0$ is simpler and follows from the Radial lemma \cite{BerLions}[Lemma A.II].
%Indeed, suppose that $\theta=0$. Then, $\tu$ is almost everywhere equal to a continuous function, hence we find a sequence $(x_n)\subset\R^N$ such that $\tu(x_n)\to 0$, $\tu(x_n)\neq 0$ and $g(\tu(x_n))=0$. Thus, by \eqref{eq:NewCond1} and \eqref{eq:NewCond2} we obtain the following contradiction
%$$0=\lim_{n\to\infty}\frac{g(\tu(x_n))}{\tu(x_n)}=\lim_{n\to\infty}\frac{g_1(\tu(x_n))}{\tu(x_n)}-\lim_{n\to\infty}\frac{g_2(\tu(x_n))}{\tu(x_n)}\leq -m<0.$$
%\end{Rem}
				
								
The main difficulty in the analysis of Palais-Smale sequences of $J\circ m$ is to find proper translations $(y_n)\subset \R^N$ such that $\theta>0$ in Lemma \ref{lem:theta}. In order to check $(M)_\beta (i)$ condition one needs to ensure that even $\theta\geq 1$. This can be performed with the help of 
the following result providing decompositions for Palais-Smale sequences of $J\circ m$, which is based on the profile decomposition Theorem \ref{ThGerard}. Observe that in the usual variational approach e.g. due to Struwe \cite{StruweSplitting} or  Coti Zelati and Rabinowitz \cite{CotiZelatiRab}, such decompositions of Palais-Smale sequences are finite. In our case, however, a finite procedure cannot be performed in general, since we do not know whether a weak limit point of a Palais-Smale sequence of $J\circ m$ is a critical point. Therefore we need to employ the profile decompositions from Theorem \ref{ThGerard} and Corollary \ref{CorGerard}. 

\begin{Prop}\label{prop:PSanaysis}
Let $(u_n)\subset \cU$ be a Palais-Smale sequence of $J\circ m$ at level $\beta=c$.
Then there is $K\in \N\cup \{\infty\}$ and there are sequences
$(\tu_i)_{i=0}^K\subset H^1(\R^N)$, $(\theta_i)_{i=0}^K\subset \R$, for any $n\geq 1$, $(y_n^i)_{i=0}^K\subset \R^N$ is such that $y_n^0=0$,
$|y_n^i-y_n^j|\rightarrow \infty$ as $n\to\infty$ for $i\neq j$, and passing to a subsequence, the following conditions hold:
\begin{eqnarray}\label{EqSplit1}
&& m(u_n)(\cdot+y_n^i)\weakto \tu_i\; \hbox{ in } H^1(\R^N)\text{ as }n\to\infty\text{ for } 0\leq i <K+1,\\
\label{EqSplit2}
&& 
\tu_i\text{ solves }\eqref{eq:thetaEQ}\text{ with }\theta_i\text{ for } 0\leq i <K+1,\tu_i\neq 0\text{ and }\eqref{eq:theta}\text{ holds  for } 1\leq i <K+1,\\\nonumber
&&\text{if }\tu_0\neq 0,\text{ then }\theta_0\neq 0\text{ and satisfies }\eqref{eq:theta},\\
%&&\nonumber
%\tu_0\text{ solves }\eqref{eq:thetaEQ}\text{ with }\theta_0\neq 0\text{ if } \tu_0\neq 0,\\
\label{EqSplit3}
&& \lim_{n\to\infty}\int_{\R^N}G_1(m(u_n))\, dx= \sum_{i=0}^K
\int_{\R^N}G_1(\tu_i)\, dx,
\\\label{EqSplit4}
&& \lim_{n\to\infty}\int_{\R^N}G_2(m(u_n))\, dx\geq \sum_{i=0}^K
\int_{\R^N}G_2(\tu_i)\, dx,\\\label{EqSplit5}
&&\lim_{n\to\infty}\psi(m(u_n))\geq \sum_{i=0}^K \psi(\tu_i).
\end{eqnarray}
\end{Prop}
\begin{proof}
Since $J$ is coercive on $\cM$, we know that $m(u_n)$ is bounded and passing to a subsequence we may assume that $\lim_{n\to\infty}\int_{\R^N}G_1(m(u_n))\, dx$, $\lim_{n\to\infty}\int_{\R^N}G_2(m(u_n))\, dx$ exist. In view of Theorem \ref{ThGerard} we obtain  sequences
$(\tu_i)_{i=0}^\infty\subset H^1(\R^N)$ and $(y_n^i)_{i=0}^\infty\subset \R^N$ for  $n\geq 1$, such that \eqref{EqSplit2a}--\eqref{EqSplit3a} are satisfied. If $(\tu_i)_{i=1}^\infty$ contains exactly $K$ nontrivial functions, then we may assume that $\tu_i\neq 0$ for $i=1,...,K$. Otherwise we set $K=\infty$. In view of Lemma \ref{lem:theta}, $\tu_i$ solves \eqref{eq:thetaEQ} with $\theta_i$ for $1\leq i< K+1$. If $\tu_0=0$, then $\theta_0=0$, and if $\tu_0\neq 0$, then $\theta_0$ is given by \eqref{eq:thetaEQ}, so that \eqref{EqSplit1}-\eqref{EqSplit2} hold. Since \eqref{eq:NewCond1} holds, then \eqref{EqSplit3} follows from \eqref{EqSplit3a} and \eqref{EqSplit4a}. Moreover \eqref{EqSplit4} and \eqref{EqSplit5} follow from \eqref{EqSplit3a} and \eqref{EqSplit2a} respectively.
\end{proof}



\begin{Cor}\label{cor:PSanalysis}
	Suppose that $X=H^{1}_{\cO_1}(\R^N)\cap X_\tau$  and $2\leq m< N/2$.
	Let $(u_n)\subset \cU\cap X$ be a Palais-Smale sequence of $J|_X\circ m|_{\cU\cap X}$ at level $\beta=\inf_{\cM\cap X}J$.
	Then there is $K\in \N\cup \{\infty\}$ and there are sequences
	$(\tu_i)_{i=0}^K\subset X$, $(\theta_i)_{i=0}^K\subset \R$, $(y_n^i)_{i=0}^K\subset  \{0\}\times\{0\}\times\R^{N-2m}$ for any $n\geq 1$ such that $y_n^0=0$,
	$|y_n^i-y_n^j|\rightarrow \infty$ as $n\to\infty$ for $i\neq j$, and passing to a subsequence, \eqref{EqSplit1}--\eqref{EqSplit5} are satisfied. 
\end{Cor}
\begin{proof} We argue as in proof of Proposition \ref{prop:PSanaysis}, but instead of Theorem \ref{ThGerard} we use Corollary \ref{CorGerard}.
Arguing as in Lemma \ref{lem:theta} we obtain that $\tu_i$ solves \eqref{eq:thetaEQ} with $\theta_i$ in $X$ for $0\leq i <K+1$, i.e.
$$\theta_i\int_{\R^N}\langle\nabla \tu_i, \nabla v\rangle\,dx=\int_{\R^N} g(\tu_i)v\,dx\quad\hbox{for every }v\in X,$$
and $\tu_i\neq 0$ for $1\leq i <K+1$. By the Palais principle of symmetric criticality \cite{Palais}, $\tu_i$ solves \eqref{eq:thetaEQ} with $\theta_i$. As in Lemma \ref{lem:theta} we show that $\theta_i\neq 0$ and by the Pohozaev identity \eqref{eq:theta} holds for $\tu_i$ and $\theta_i$ for $1\leq i <K+1$. If $\tu_0=0$, then $\theta_0=0$, otherwise $\theta_0$ is given by \eqref{eq:thetaEQ} and by the Pohozaev identity satisfies also \eqref{eq:theta}.
\end{proof}

\begin{Rem}\label{Rem:theta}
An important consequence of Proposition \ref{prop:PSanaysis} and Corollary \ref{cor:PSanalysis} is the existence of a sequence of translations $(y^i_n)$ such that $\theta_i\geq 1$ for some $i\geq 0$. Indeed, in view of \eqref{EqSplit2} we get
$$\theta_i\psi(\tu_i)=2^*\Big(\int_{\R^N}G_1(\tu_i)\, dx-\int_{\R^N}G_2(\tu_i)\, dx\Big)$$
for $0\leq i < K+1$. Then by \eqref{EqSplit3}--\eqref{EqSplit5} we obtain
\begin{eqnarray*}
\sum_{i=0}^K\theta_i\psi(\tu_i)&=&2^*\Big(\sum_{i=0}^K\int_{\R^N}G_1(\tu_i)\, dx-\sum_{i=0}^K\int_{\R^N}G_2(\tu_i)\, dx\Big)\\
&\geq& 2^*\Big(\lim_{n\to\infty}\int_{\R^N}G_1(m(u_n))\, dx - \lim_{n\to\infty}\int_{\R^N}G_2(m(u_n))\, dx\Big)\\
&=& \lim_{n\to\infty}\psi(m(u_n))\geq \sum_{i=0}^K \psi(\tu_i).
\end{eqnarray*}
Therefore there is $\theta_i\geq 1$ for some $0\leq i<K+1$.
\end{Rem}

\subsection{Proof of Theorems \ref{ThMain1}, \ref{ThMain2} and \ref{ThMain3}}
\begin{Lem}\label{lem:Mcond}
$J\circ m$ satisfies $(M_\beta)\; (i)$ for $\beta=c$. 
\end{Lem}
\begin{proof}
	Let $(u_n)\subset \cU$ be a $(PS)_\beta$-sequence of $J\circ m$.  Since $J$ is coercive on $\cM$, $(m(u_n))$ is bounded and  in view of Proposition \ref{prop:PSanaysis} and Remark \ref{Rem:theta} we find a sequence $(y_n)\subset \R^N$ such that $m(u_n)(\cdot+y_n)\weakto \tu$ in $H^1(\R^N)$ for some $\tu\neq 0$ and $\theta\geq 1$ given by \eqref{eq:theta}. Observe that $\tu\in\cP$ and by Lemma \ref{lem:theta} we conclude.
\end{proof}

Now, let us consider $\cO_1$-invariant functions.

\begin{Lem}\label{lem:Mcond2}
Suppose that $X=H^{1}_{\cO_1}(\R^N)\cap X_\tau$ and $2\leq m< N/2$. Then $J|_X\circ m|_{\U\cap X}$ satisfies  $(M_\beta)\; (i)$ for $\beta=\inf_{\cM\cap X}J$.
\end{Lem}
\begin{proof}
Let $(u_n)\subset \cU\cap X$ be a $(PS)_\beta$-sequence of $J|_X\circ m|_{\U\cap X}$.  Similarly as in  proof of Lemma \ref{lem:Mcond}, in view of Corollary \ref{cor:PSanalysis} and and Remark \ref{Rem:theta} we find a sequence $(y_n)\subset \{0\}\times\{0\}\times\R^{N-2m}$ such that $m(u_n)(\cdot+y_n)\weakto \tu$ in $X$ for some $\tu\neq 0$ and $\theta\geq 1$ given by \eqref{eq:theta}. Observe that $\tu\in\cP\cap X$ and as in Lemma \ref{lem:theta}, $J(m_\cP(\tu))\leq \beta$.

\end{proof}

More can be said for $\cO_2$-invariant functions.
\begin{Lem}\label{lem:Mcond3}
	Suppose that $X=H^{1}_{\cO_2}(\R^N)\cap X_\tau$. If $(u_n)\subset\cU\cap X$ is a $(PS)_\beta$-sequence of $(J|_X\circ m|_{\U\cap X})$, then passing to a subsequence $u_n\to u_0$ for some $u_0\in\cU\cap X$ such that $J|'_X(m(u_0))=0$.
\end{Lem}
\begin{proof}
Let $(u_n)\subset \cU\cap X$ be a sequence such that $(J|_X\circ m|_{\U\cap X})'(u_n)\to 0$ and $(J|_X\circ m|_{\U\cap X})(u_n)\to\beta$.  Since $J$ is coercive on $\cM$, $(m(u_n))$ is bounded and  in view of Corollary \ref{cor:LionsO2} we find $\tu\in X$ such that 
\begin{equation}\label{eq:G_1conv}
\int_{\R^N}G_1(m(u_n))\, dx\to \int_{\R^N}G_1(\tu)\, dx
\end{equation}
as $n\to\infty$. If $\tu=0$, then by \eqref{eq:NewCond2}
$$\min\big\{1,\frac{m}{2}\big\}\|m(u_n)\|^2\leq\int_{\R^N}|\nabla m(u_n)|^2\,dx+2^*\int_{\R^N}G_2(m(u_n))\,dx=2^*\int_{\R^N}G_1(m(u_n))\,dx\to 0,$$
which contradicts the fact that $\inf_{\cM\cap X}J>0$. Therefore $\tu\neq 0$.
Now, observe that applying Corollary \ref{cor:LionsO2}  with \eqref{EqPsiLast} for $\Psi(s)=G_1(s)$ and passing to subsequence we obtain 
\begin{eqnarray*}
	\lim_{n\to\infty}\int_{\R^N} g_1(m(u_n))m(u_n)\,dx&=&\int_{\R^N} g_1(\tu)\tu\,dx,\\
	\lim_{n\to\infty}\int_{\R^N} g_2(m(u_n))m(u_n)\,dx&\geq&\int_{\R^N} g_2(\tu)\tu\,dx.
\end{eqnarray*}
Similarly as in Lemma \ref{lem:theta} we infer that \eqref{eq:theta2}, \eqref{eq:theta} hold and 
\begin{eqnarray*}
	\psi(\tu)&\leq&\lim_{n\to\infty}\psi(m(u_n))=\lim_{n\to\infty}\int_{\R^N}g_1(m(u_n))m(u_n)\,dx-\lim_{n\to\infty}\int_{\R^N}g_2(m(u_n))m(u_n)\,dx\\
	&\leq&\int_{\R^N}g_1(\tu)\tu\,dx-\int_{\R^N}g_2(\tu)\tu\,dx= \theta \psi(\tu).
\end{eqnarray*}	
Hence $\theta\geq 1$ and $m_{\cP}(u)\in \cM\cap X$ is a critical point of $J|_X\circ m|_{\U\cap X}$.
 In view of \eqref{eq:theta2} we get
 $$\theta=\lim_{n\to\infty}\psi(m(u_n))^{-1}\int_{\R^N} g(m(u_n))m(u_n)\,dx\leq 
 \psi(\tu)^{-1}\int_{\R^N} g(\tu)\tu\,dx=\theta,$$
hence 
 $$\lim_{n\to\infty}\psi(m(u_n))=\psi(\tu)$$
 and 
 $$\lim_{n\to\infty}\int_{\R^N} g_2(m(u_n))m(u_n)\,dx=\int_{\R^N} g_2(\tu)\tu\,dx.$$
 Note that $g_2(s)=ms+g_3(s)$, where $g_3(s):=\max\{0,-g(s)-ms\}\geq 0$ for $s\geq 0$ and $g_3(s):=-\max\{0,-g(s)-ms\}$ for $s\leq 0$. Then $g_3(s)s\geq 0$ for $s\in\R$ and we easy infer that
 $$\lim_{n\to\infty}\int_{\R^N}  m|m(u_n)|^2\,dx=\int_{\R^N} m|\tu|^2\,dx.$$
 Therefore $m(u_n)\to \tu$ in $H^1(\R^N)$, $\tu\in\cM\cap X$ and $u_n\to u_0:=m^{-1}(\tu)$ in $\cU\cap X$. Since $\theta=1$, $J|'_X(m(u_0))=J|'_X(\tu)=0$.
\end{proof}




\begin{altproof}{Theorem \ref{ThMain1}}
Since $J$  satisfies (A1)--(A3) and $(M_\beta)\;(i)$, proof follows from Theorem \ref{Th:CrticMulti} (a).
\end{altproof}



\begin{altproof}{Theorem \ref{ThMain2}}
If $X=H^{1}_{\cO_1}(\R^N)\cap X_\tau$ and $2\leq m< N/2$, then $J|_X\circ m|_{\U\cap X}$ satisfies
 (A1)--(A3) and $(M_\beta)\;(i)$. Then, in view of Theorem \ref{Th:CrticMulti} (a) there is a critical point $u\in \cM\cap X$ of $J|_X$ such that 
 $$J(u)=\inf_{\cM\cap X}J.$$
 In view of the Palais principle of symmetric criticality \cite{Palais}, $u$ solves \eqref{eq}.
Let
\begin{eqnarray*}
\Omega_1&:=&\{x\in\R^N: |x_1|>|x_2|\},\\
\Omega_2&:=&\{x\in\R^N: |x_1|<|x_2|\}.\\
\end{eqnarray*}
Since $u\in X_\tau\cap H^1_{\cO_1}(\R^N)$, we get $\chi_{\Omega_1} u\in H^1_0(\Omega_1)\subset H^1(\R^N)$ and 
 $\chi_{\Omega_2} u\in H^1_0(\Omega_2)\subset H^1(\R^N)$. Moreover $\chi_{\Omega_1} u\in \cM$ and
$$J(u)=J(\chi_{\Omega_1} u)+J(\chi_{\Omega_2} u)=2J(\chi_{\Omega_1} u)\geq 2 \inf_{\cM} J,$$
which completes the proof of \eqref{eq:thmain2}. The remaining case $2\leq m=N/2$ is contained in Theorem \ref{ThMain3}.
\end{altproof}



\begin{altproof}{Theorem \ref{ThMain3}}
If $X=H^{1}_{\cO_2}(\R^N)\cap X_\tau$, then $J|_X\circ m|_{\U\cap X}$ satisfies
(A1)--(A3), and note that Lemma \ref{lem:Mcond3} holds.  In view of \cite{BerLionsII}[Theorem 10], for any $k\geq 1$ we find an odd continuous map $\tau:S^{k-1}\to H_0^1(B(0,R))\cap L^{\infty}(B(0,R))$ such that $\tau(\sigma)$ is a radial function and $\tau(\sigma)\neq 0$ for all $\sigma\in S^{k-1}$. Moreover 
$$\int_{B(0,R)}G(\tau(\sigma))\,dx\geq c_2R^N-c_3R^{N-1}$$
for any $\sigma\in S^{k-1}$ and some constants $c_2,c_3>0$. As in Remark \ref{remTau} we define a map $\tilde{\tau}:S^{k-1}\to H_0^1(B(0,R))\cap L^{\infty}(B(0,R))$ such that $\tilde{\tau}(\sigma)(x_1,x_2,x_3)=\tau(\sigma)(x_1,x_2,x_3)\vp(|x_1|-|x_2|)$. Observe that $\tilde{\tau}(\sigma)\in X$ and
$\int_{B(0,R)}G(\tilde{\tau}(\sigma))\,dx>0$ for $\sigma\in S^{k-1}$ and sufficiently large $R$. Therefore \eqref{eq:LSvaluesnonempty} is satisfied and
 proof follows from Theorem \ref{Th:CrticMulti} (c) and from the Palais principle of symmetric criticality \cite{Palais}.
\end{altproof}

{\bf Acknowledgements.}
I would like to thank Louis Jeanjean and Sheng-Sen Lu, who observed that one has to ensure that the Lusternik-Schnirelmann values are finite in proof of Theorem 2.2. Very recently they also recovered the results of this paper in preprint \cite{LuJeanjean}.  Moreover I am  grateful to  Rupert Frank for pointing out the reference \cite{Nawa}.
The author was partially supported by the National Science Centre, Poland (Grant No. 2014/15/D/ST1/03638). 
%and he would like to thank the referee for many valuable
%comments helping to improve the paper.

%{\bf Compliance with Ethical Standards.}
%The author declares that he has no conflict of
%interest, also confirms that the manuscript complies to the Ethical Rules applicable for this journal.


%%\appendix
%\section{Other application of the critical point theory}
%
%We investigate the nonlinear scalar field equation
%\begin{equation}
%\label{eq}
%\left\{
%\begin{array}{ll}
%-\Delta u  +V(x) = f(x,u)&\quad \hbox{in }\R^N,\; N\geq 1,\\
%u\in H^1(\R^N)&%\quad\hbox{as }|x|\to\infty.
%\end{array}
%\right.
%\end{equation}
%under the following general assumptions introduced by Berestycki and Lions  in their fundamental papers \cite{BerLions,BerLionsII}:
%\begin{itemize}
%	\item[(f0)] $f:\R\times\R \to \R$ is continuous and $\Z^N$-periodic in $x$.
%	\item[(f1)] $\lim_{s\to 0}f(x,s)/s=0$ uniformly in $x$.
%	\item[(f2)] $\lim_{s\to \infty}f(x,s)/s^{2^*-1}=0$ uniformly in $x$.
%%	\item[(f3)] $\liminf_{\|u\|\to \infty}\frac{1}{\|u\|^2}\int_{\R^N}F(x,u)\, dx>\frac12$, where $u\in H^1(\R^N)$.
%	\item[(f4)] $s\mapsto f(x,s)/|s|$ is strictly increasing on $(-\infty,0)$ and on $(0,\infty)$.
%\end{itemize}
%
%Maybe $F(x,s)=s^2 arctg(s^2)$ satisfies these conditions.

\begin{thebibliography}{99}
\baselineskip 2 mm
%\bibitem{Adams} R. Adams, {\em Sobolev Spaces}, Boston, Academic Press (1975).

%\bibitem{Akhmediev} N. Akhmediev, J.-M. Soto-Crespo, A. Ankiewicz: {\em Does the nonlinear Schr\"odinger equation correctly describe beam propagation?}, Opt. Lett. {\bf 18}, (1993), 411--413.  

\bibitem{AR} A. Ambrosetti, P.H. Rabinowitz, {\em Dual variational methods in critical point theory and applications}, 
J. Funct.  Anal. {\bf 14} (1973), 349--381. 

\bibitem{AoWei} W. Ao, M. Musso, F. Pacard, J. Wei: {\em Solutions without any symmetry for semilinear elliptic problems}, J. Funct. Anal. {\bf 270} (2016), no. 3, 884--956. 

%\bibitem{Ball} J. M. Ball, Y. Capdeboscq, B. T. Xiao: {\em On uniqueness for time harmonic anisotropic Maxwell's equations with piecewise regular coefficients}, Math. Models Methods Appl. Sci. {\bf 22} (2012), 1250036.

%\bibitem{BenForAzzAprile} A. Azzollini, V. Benci, T. D'Aprile, D. Fortunato: {\em Existence of Static Solutions of the Semilinear Maxwell Equations}, Ric. Mat. {\bf 55} (2006), no. 2, 283--297.

%\bibitem{AzzolliniPomponio} A. Azzollini, A. Pomponio: {\em On a ''zero mass'' nonlinear Schr\"odinger equation}, Adv. Nonlinear Stud. {\bf 7} (2007), no. 4, 599--627. 

%\bibitem{BoostingPhysRevB} Ch. Argyropoulos, P.-Y. Chen, G. D'Aguanno, N. Engheta, A. Al\'u: {\em Boosting optical nonlinearities in $\eps$-near-zero plasmonic channels}, Phys. Rev. B {\bf 85} (4) (2012), 045129-5

%\bibitem{AmbrosettiFellMal} A. Ambrosetti, V. Felli, A. Malchiodi: {\em Ground states of nonlinear Schr\"odinger equations with potentials vanishing at infinity}, J. Eur. Math. Soc. {\bf 7} (2005), no. 1, 117--144.

%\bibitem{BadialePisaniRolando} M. Badiale, L. Pisani, S. Rolando: {\em Sum of weighted Lebesgue spaces and nonlinear elliptic equations}, Nonlinear Differ. Equ. Appl. {\bf 18} (2011), 369--405.

%\bibitem{BartschDing} T. Bartsch, Y. Ding: {\em On a nonlinear Schr\"odinger equation with periodic potential}, Math. Ann. {\bf 313}, (1999), no. 1, 15--37. 

%\bibitem{BartschDing} T. Bartsch, Y. Ding : {\em Deformation theorems on non-metrizable vector spaces and applications to critical point theory}, Math. Nach. {\bf 279} (12), (2006), 1267--1288.

%\bibitem{BartschMederski} T. Bartsch, J. Mederski: {\em Ground and bound state solutions of semilinear time-harmonic Maxwell equations in a bounded domain}, Arch. Rational Mech. Anal. {\bf 215} (1), (2015), 283--306.

\bibitem{BartschWillem} T. Bartsch, M. Willem: {\em Infinitely many nonradial solutions of a Euclidean scalar field equation}, J. Funct. Anal. {\bf 117} (1993), 447--460.

\bibitem{BenciRabinowitz} V. Benci, P. H. Rabinowitz: {\em Critical point theorems for indefinite functionals}, Invent. Math. {\bf 52} (1979), no. 3, 241--273.

%\bibitem{BenFor} V. Benci, D. Fortunato: 
%{\em Towards a Unified Field Theory for Classical Electrodynamics}, 
%Arch. Rat. Mech. Anal. {\bf 173} (2004), 379--414.

%\bibitem{BenGrisantiMich} V. Benci, C. Grisanti, A. M. Micheletti: {\em Existence and non existence of the ground state solution for the nonlinear Schroedinger equations with V($\infty$)=0}, Topol. Meth. Non. Anal. {\bf 26} (2005), no. 2, 203--219.

%\bibitem{BenGrisantiMich2} V. Benci, C. Grisanti, A. M. Micheletti: {\em Existence of solutions for the nonlinear Schr\"odinger equation with V($\infty$)=0}, Contributions to nonlinear analysis, 53--65, Progr. Nonlinear Differential Equations Appl., {\bf 66}, Birkh\"auser, Basel, (2006)

\bibitem{BerLions} H. Berestycki, P.L. Lions: {\em Nonlinear scalar field equations. I - existence of a ground state}, Arch. Ration. Mech. Anal. {\bf 82} (1983), 313--345.

\bibitem{BerLionsII} H. Berestycki, P.L. Lions: {\em Nonlinear scalar field equations. II. Existence of infinitely many solutions}, Arch. Ration. Mech. Anal. {\bf 82} (1983), 347--375.


\bibitem{BerLionsInfZero} H. Berestycki, P.L. Lions:  {\em Existence d'\'etats multiples dans des \'equations de champs scalaires non lin\'eaires dans le cas de masse nulle}, C. R. Acad. Sci. Paris S\'er. I Math. {\bf 297}, (1983), 267--270.

%\bibitem{BornInfeld} M. Born, L. Infeld: {\em Foundations of the new field theory}, Proc. Roy. Soc. Lond. A {\bf 144} (1934), 425--451. 

%\bibitem{BrezisKato} H. Brezis, T. Kato: {\em Remarks on the Schr\"odinger operator with singular complex potentials}. J. Math. Pures Appl. {\bf 58} (1979) 137--151.


%\bibitem{BrezisLieb} H. Br\'ezis, E. Lieb: {\em A relation between pointwise convergence of functions and convergence of functionals}, Proc. Amer. Math. Soc. {\bf 88} (1983), no. 3, 486--490.



%\bibitem{Linearpermittivity} A. Ciattoni, C. Rizza, E. Palange: {\em Transmissivity directional hysteresis of a nonlinear metamaterial slab with very small linear permittivity}, Optics Letters {\bf 35} (13) (2010), 2130--2132.

\bibitem{CotiZelatiRab} V. Coti Zelati, P.H. Rabinowitz: {\em Homoclinic type solutions for a semilinear elliptic PDE on $\R^N$}, Comm. Pure and Applied Math. {\bf 45}, no. 10, (1992), 1217--1269.


%\bibitem{DAprileSiciliano} T. D'Aprile, G. Siciliano:{\em Magnetostatic solutions for a semilinear perturbation of the Maxwell equations}, Adv. Differential Equations {\bf 16} (2011), no. 5--6, 435--466.

\bibitem{SoliminiDev} G. Devillanova, S. Solimini, {\em Some remarks on profile decomposition theorems},
Adv. Nonlinear Stud. {\bf 16} (2016), no. 4, 795--805. 

%\bibitem{DingBook} Y. Ding 
%{\em Variational methods for strongly indefinite problems}. 
%Interdisciplinary Mathematical Sciences, 7. World Scientific Publishing, 2007
%\bibitem{Doerfler} W. D\"orfler, A. Lechleiter, M. Plum, G. Schneider, C. Wieners: {\em Photonic Crystals: Mathematical Analysis and Numerical Approximation}, Springer Basel (2012).

\bibitem{Gerard} G\'erard: {\em Description du d\'efaut de compacit\'e de l'injection de Sobolev}, ESAIM: Control, Optimisation and Calculus of Variations {\bf 3} (1998), 213--233. 

%\bibitem{Hile} C.V. Hile: {\em Comparisons between Maxwell's equations and an extended nonlinear Schr\"odinger equation}, Wave Motion {\bf 24} (1), (1996), 1--12.

%\bibitem{HmidiKeraani2005} T. Hmidi, S. Keraani: {\em Blowup theory for critical nonlinear Schr\"odinger equations revisited}, International Mathematical Research Notices (46) (2005), 2815--2828.

\bibitem{HmidiKeraani} T. Hmidi, S. Keraani: {\em Remarks on the blow-up for the $L^2$-critical nonlinear Schr\"odinger equations}, SIAM J. Math. Anal. {\bf 38} (2006), no. 4, 1035--1047.

\bibitem{Hirata} J. Hirata, N. Ikoma, K. Tanaka: {\em Nonlinear scalar field equations in $\R^N$: mountain pass and symmetric mountain pass approaches},  Topol. Methods Nonlinear Anal. {\bf 35}, (2010), 253--276.

\bibitem{Jeanjean} L. Jeanjean: {\em Existence of solutions with prescribed norm for semilinear elliptic equations}, Nonlinear Anal. {\bf 28} (1997), 1633--1659.

\bibitem{JeanjeanTanaka} L. Jeanjean, K. Tanaka: {\em A remark on least energy solutions in $\R^N$}, Proc. Amer. Math. Soc. {\bf 131} (2003), 2399--2408. 

\bibitem{LuJeanjean} L. Jeanjean, S.-S. Lu: {\em Nonlinear scalar field equations with general nonlinearity}, arXiv:1807.07350.

%\bibitem{Killip} R. Killip, T. Oh, O. Pocovnicu, M. Vi�an: {\em Solitons and Scattering for the Cubic�Quintic Nonlinear Schr\"odinger Equation on  $\R^3$}, Arch. Rational Mech. Anal. {\bf 225} (1), (2017) 469--548.

\bibitem{Lions82} P.-L. Lions: {\em Sym\'etrie et compacit\'e dans les espaces de Sobolev}, J. Funct. Anal. {\bf 49} (1982), no. 3, 315--334.

\bibitem{Lions84} P.-L. Lions: {\em The concentration-compactness principle in the calculus of variations. The locally compact case. Part I and II}, Ann. Inst. H. Poincar�, Anal. Non Lin�are., {\bf 1}, (1984), 109--145; and 223--283.

%\bibitem{Lions85} P.L. Lions: {\em The concentration-compactness principle in the calculus of variations. The limit case. Part I and II.}, Rev. Mat. Iberoamer. {\bf 1} (1985), no. 1, 145--201: and no. 2, 45--121.

\bibitem{Lorca} S. Lorca, P. Ubilla: {\em Symmetric and nonsymmetric solutions for an elliptic equation on $\R^N$}, Nonlinear Anal. {\bf 58}, 961--968. 

%\bibitem{Kwong} M. K. Kwong: {\em Uniqueness of positive solutions of $\Delta u-u+u^p=0$ in $\R^N$}, Arch. Rat. Mech. Anal. {\bf 105} (1989), 243--266.


\bibitem{Marzantowicz} W. Marzantowicz: {\em 
Geometrically distinct solutions given by symmetries of variational problems with the $O(N)$-symmetry}, arXiv:1711.08425.

\bibitem{MederskiENZ} J. Mederski: {\em Ground states of time-harmonic semilinear Maxwell equations in $\R^3$ with vanishing permittivity}, Arch. Rational Mech. Anal. {\bf 218} (2), (2015), 825--861.

%\bibitem{MederskiBNcurl} J. Mederski: {\em The Brezis-Nirenberg problem for the curl-curl operator}, preprint arXiv:1609.03989.
%\bibitem{Nie} W. Nie: {\em Optical Nonlinearity: Phenomena, Applications, and Materials}, Advanced Materials {\bf 5}, (1993), 520--545.

\bibitem{MederskiZeroMass} J. Mederski: {\em Nonlinear scalar field equations involving the zero mass case}, in preparation.

\bibitem{Musso} M. Musso, F.  Pacard, J.  Wei: {\em Finite-energy sign-changing solutions with dihedral symmetry for the stationary nonlinear Schr\"odinger equation}, J. Eur. Math. Soc. {\bf 14} (2012), no. 6, 1923--1953. 

\bibitem{Nawa} H. Nawa: {\em  "Mass concentration'' phenomenon for the nonlinear Schr\"odinger equation with the critical power nonlinearity. II } Kodai Math. J. {\bf 13} (1990), no. 3, 333--348.

\bibitem{Palais} R.S. Palais: {\em The principle of symmetric criticality}, Commun. Math. Phys. {\bf 69} (1979), 19--30. 


\bibitem{Rabinowitz:1986} P. Rabinowitz:
{\em Minimax Methods in Critical Point Theory with Applications to Differential Equations},
CBMS Regional Conference Series in Mathematics, Vol. {\bf 65}, Amer. Math. Soc., Providence, Rhode Island 1986.

\bibitem{Shatah} J. Shatah {\em Unstable ground state of nonlinear Klein�Gordon equations}, Trans. Amer. Math. Soc.
{\bf 290} (2) (1985) 701--710.

\bibitem{Strauss} W.A. Strauss: {\em Existence of solitary waves in higher dimensions}, Commun. Math. Phys. {\bf 55}, (1977), 149--162.


\bibitem{StruweSplitting} M. Struwe: {\em A global compactness result for elliptic boundary value problems involving limiting nonlinearities}, Math. Z. {\bf 187} (1984), no. 4, 511--517.

%\bibitem{Struwe} M. Struwe: {\em Variational Methods}, Springer 2008.

\bibitem{SzulkinWeth} A. Szulkin, T. Weth: {\em Ground state solutions for some indefinite variational problems}, J. Funct. Anal. {\bf 257} (2009), no. 12, 3802--3822. 

%\bibitem{SzulkinWethHandbook} A. Szulkin, T. Weth: {\em The method of Nehari manifold. Handbook of nonconvex analysis and applications}, Int. Press (2010), 597--632.
\bibitem{Willem} M. Willem: {\em Minimax Theorems}, Birkh\"auser Verlag (1996).

\bibitem{Ziemer} W. P. Ziemer: {\em Weakly differentiable functions. Sobolev spaces and functions of bounded variation}, Graduate Texts in Mathematics, 120. Springer-Verlag, New York (1989).


\end{thebibliography}



\end{document}


%Zmiany: dodaem referencje Nawa [23] na str. 2,
%w warunku A3 dodalem nonempty P, str. 6
%Twierdzenie 2.2. dodalem (2.2) i uzasadnienie w dowodzie, ze beta_k sa skonczone
%Nowy Remark 4.2
%Str. 31. uzasadnienie w dowodzie Th. 1.3., ze (2.2) zachodzi.
%dodale w (A1) gu=-u
%Lemma 4.3 has been generlaized and the proof of \theta\neq 0 is simplified.
%
%Letter:
%
%Dear Colleagues,
%
%first of all I would like to thank you your manuscript, your valuable remarks, in particular for pointing out that one has to ensure that the Lusternik-Schnirelmann values must be finite. Please find enclosed updated version of my manuscript. In fact I have added condition (2.2), which can be verified as in your paper or by an easy construction  demonstrated in Remark 4.2 (of the enclosed version). Roughly speaking, taking into account the construction of the radial continuous and odd map of Berestycki and Lions, it is enough to multiply by a simple function $\vp(|x_1|-|x_2|)$ and we conclude as in proof of Theorem 1.3 on page 30. Note that in this version,  Lemma 4.3 has been generalized, so $\theta\neq 0$ for all levels and the proof is shorter now.  
%
%I am reading your interesting approach based on the monotonicity trick and I have one remark. Clearly the profile decomposition theorem Theorem 1.4 from my work holds for $Psi:\R\to\R$ satisfying (1.9).  Then Theorem 3.1 from your work seems to follow the profile decomposition theorem, namely taking a bounded Palais-Smale sequence, we easy observe that there is only a finite number of nontrivial weak limit points $w_k$ and from Theroem 1.4 (1.7) and (1.8) with $\Psi(s)=s^2$ we conclude that $v_n^l\to 0$ 




