\documentclass[leqno, 11pt]{article}\usepackage{latexsym}\usepackage{amsmath, amscd}\usepackage{amssymb}
\begin{document}
\title{{\sc Sperner's Lemma, the Brouwer Fixed-Point Theorem, and Cohomology}}
\date{}
\author{\sc Nikolai V. Ivanov}                                                                                                                                                                                                                                                                                        
\renewcommand{\thefootnote}{\fnsymbol{footnote}}


\maketitle

\paragraph{1. INTRODUCTION.} The proof of the Brouwer fixed-point Theorem based on Sperner's Lemma \cite{S} is often presented 
as an elementary combinatorial alternative to advanced proofs based on algebraic topology. See, for example, Section 6.3 of \cite{P1}. 
One may ask if this proof is really based on ideas completely different from the ideas of algebraic topology (and, in particular, the ideas of Brouwer's own proof, based on the degree theory, a fragment of algebraic topology developed by him)? After the author discovered \cite{I} that the famous analytic proof of the Brouwer Theorem due to Dunford and Schwartz is nothing else than the usual topological proof in disguise, he started to suspect that the same is true for the proof based on Sperner's Lemma. This suspicion turned out to be correct, and the goal of this note is to uncover the standard topology hidden in this proof.

In fact, the two situations are very similar. The Dunford-Schwartz proof can be considered as a cochain-level version of the standard proof based on the de Rham cohomology (cochains of the de Rham theory are differential forms), written in the language of elementary multivariable calculus. 
Similarly, the combinatorial proof of Sperner's Lemma can be considered as a cochain-level version, written in the combinatorial language, of a standard cohomological argument. This time one needs to use simplicial cohomology and simplicial cochains. 
It is remarkable that both alternative approaches turn out to be cohomological proofs in disguise, and not the homological ones. 

It turns out that the standard deduction of the Brouwer Theorem from Sperner's Lemma is also similar to some arguments in algebraic topology. 
Namely, most proofs of the Brouwer Theorem are based on a construction of a retraction of a disk or a simplex onto its boundary starting from its fixed point free map to itself, combined with the No-Retraction Theorem to the effect that there are no such retractions. It turns out that this construction of a retraction underlies 
the standard deduction of the Brouwer Theorem from Sperner's Lemma. The No-Retraction Theorem itself can be proved by a modification of this deduction suggested by
the notion of a simplicial approximation. The latter is a basic tool of cohomology theory.\\

The rest of the paper consists of two parts. Section 2 is devoted to Sperner's Lemma. 
We start with Sperner's proof presented in a geometrical language. Then we present a cohomological proof. 
In order to compare these two proofs, we rewrite the cohomological proof explicitly in terms of cochains. 
The resulting proof may be thought as a cochain-level realization of Sperner's arguments.

The version of cohomology theory most suitable for us is the simplicial cohomology theory. We will assume some familiarity with it, 
but will recall the ideas crucial for our chain-level proof. 
Recent books by Hatcher \cite{H} and by Prasolov \cite{P2} are good references. 

Section 3 is devoted to the Brouwer's Theorem. We present the standard deduction of the Brouwer Theorem from Sperner's Lemma, and then explain how a modification
of this argument can be used to prove the No-Retraction Theorem. The Brouwer Theorem follows.

\paragraph*{2. SPERNER'S LEMMA AND COHOMOLOGY.} Sperner's Lemma deals with the following situation. Let $\Delta$ be an $n$-dimensional simplex with the vertices $v_1, v_2,\ldots, v_{n+1}$. Let $\Delta_i$ be the face of $\Delta$ opposite to the vertex $v_i$. Its vertices are all $v_1,v_2,\ldots,v_{n+1}$ except $v_i$. The boundary $\partial\Delta$ is equal to the union of the $n+1$ faces $\Delta_i$. Suppose that the simplex $\Delta$ is subdivided into smaller simplices forming a simplicial complex $\Delta'$.

\paragraph{Sperner's Lemma.} {\em Suppose that vertices of $\Delta'$ are labeled by the numbers $1,2,\ldots, n+1$ in such a way that if a vertex $v$ belongs to a face $\Delta_i$, then the label of $v$ is not equal to $i$. Then the number of $n$-dimensional simplices of $\Delta'$ with set of labels of their vertices equal to  $\{1,2,\ldots, n+1\}$ is odd.}\\

Sperner's Lemma admits a geometric interpretation. Namely, if a vertex $v$ is labeled by $i$, let us set $\varphi(v)=v_i$. Then $\varphi$ maps the set of vertices of $\Delta'$ to the set of vertices of $\Delta$. Since $\Delta$ is a simplex, and therefore any set of vertices of $\Delta$ is a set of vertices of a subsimplex, the map $\varphi$ defines a simplicial map $\Delta'\rightarrow\Delta$, which we will also denote by $\varphi$. Clearly, an $n$-dimensional simplex has 
the set of labels of its vertices equal to  $\{1,2,\ldots, n+1\}$ if and only if $\varphi$ maps it onto $\Delta$. 

This geometric interpretation is not new (although the author rediscovered it). In their classical treatise \cite{AH} Alexandroff and Hopf state Sperner's Lemma in this language and prove it by homological means, without even mentioning the combinatorial approach.

In this language, the assumption of Sperner's Lemma means that $\varphi$ maps simplices of $\Delta'$ contained in a face $\Delta_i$ of $\Delta$ into $\Delta_i$. In particular, simplices of $\Delta'$ contained in $\partial\Delta$ and mapped onto $\Delta_{n+1}$ are contained in $\Delta_{n+1}$. The conclusion is that $\varphi$ maps an odd number of simplices onto $\Delta$.

Let us prove Sperner's Lemma following his arguments, but using the above geometric language. Along the way we will introduce various notations which will be used later. The numbers $e,f,g,h$ defined in the proof have exactly the same meaning as in Sperner \cite{S} and play a crucial role in what follows.\\

\noindent
{\em Combinatorial proof of Sperner's Lemma.} We use an induction by $n$. The result is trivially true for $n=0$. Suppose that $n>0$. 
Clearly, we can restrict $\varphi$ to $\Delta_{n+1}$, which is an $(n-1)$-dimensional simplex. 
The assumption of Sperner's Lemma obviously holds for this restriction, 
and the inductive assumption allows us to conclude that $\varphi$ maps an odd number of simplices of $\Delta'$ contained in 
$\Delta_{n+1}$ onto $\Delta_{n+1}$. It remains to deduce from this that $\varphi$ maps an odd number of simplices of $\Delta'$ onto $\Delta$. 
This deduction constitutes the main part of the proof.

Let $\sigma_1,\ldots,\sigma_h$ be the simplices of $\Delta'$ contained in $\Delta_{n+1}$ and mapped by $\varphi$ onto $\Delta_{n+1}$. As we noted above, no other simplices in $\partial\Delta'$ are mapped onto $\Delta_{n+1}$. Let $\tau_1,\ldots,\tau_g$ be the remaining simplices of $\Delta'$ mapped by $\varphi$ onto $\Delta_{n+1}$. 
They are contained in the interior of $\Delta$.

Consider an arbitrary $n$-dimensional simplex $\sigma$ of $\Delta'$. There are three types of such simplices, depending on the image of simplex under $\varphi$. 
The image of a simplex of the first type is neither $\Delta$, nor $\Delta_{n+1}$. The image of a simplex of the second type is $\Delta$. 
Let $\rho_1,\ldots,\rho_e$ be the simplices of this type. For every $\rho_j$, exactly one face of it is mapped onto $\Delta_{n+1}$. 
The image of a simplex of the third type is $\Delta_{n+1}$. In this case, there is a face $\sigma'$ of $\sigma$ mapped by $\varphi$ onto $\Delta_{n+1}$. 
If $v$ is the vertex of $\sigma$ not belonging to this face, then $\varphi(v)=v_i$ for some $i\leqslant n$. 
Let $w$ be the vertex of $\sigma'$ mapped by $\varphi$ to $v_i$. Clearly, if we replace $w$ by $v$ in $\sigma'$, 
we will get another face $\sigma''$ of $\sigma$ mapped by $\varphi$ onto $\Delta_{n+1}$,
and no other face of $\sigma$ is mapped onto $\Delta_{n+1}$. So, a simplex of third type 
has exactly 2 faces mapped by $\varphi$ onto $\Delta_{n+1}$. Let $f$ be the number of such simplices.

Let us count in two ways the number of pairs $(\sigma, \sigma')$ such that $\sigma'$ is a 
$(n-1)$-dimensional face of an $n$-dimensional simplex $\sigma$ and $\varphi$ maps $\sigma'$ onto $\Delta_{n+1}$. 
If we count simplices $\sigma'$ first, we see that the number of such pairs is equal to $h+2g$, 
because every $(n-1)$-dimensional simplex in the interior is a face of exactly two $n$-dimensional simplices, 
and every $(n-1)$-dimensional simplex contained in the boundary is a face of exactly one  $n$-dimensional simplex. 
If we count simplices $\sigma$ first, then only simplices of the second and third type matter, and the we see that the number of pairs is $e+2f$. 
Therefore,
\begin{equation}
\label{sperner}
h+2g=e+2f,
\end{equation}
and $e$ is odd if and only if $h$ is. But $h$ is odd by the inductive assumption. Therefore, $e$ is also odd. Since $e$ is the number of  
simplices mapped onto $\Delta$, this completes the proof. $\Box$ \\

Now, let us present a cohomological proof of Sperner's Lemma. We will use simplicial cohomology with coefficients in the 
field ${\mathbb F}_2$. This choice of the coefficients is the most convenient one since 
we are interested only in the distinction between even and odd numbers. Also, using ${\mathbb F}_2$ 
as the group of coefficients allows us to ignore all orientation issues. 

Let $\partial\Delta$ be the boundary of $\Delta$ considered as a simplicial complex with the top-dimensional simplices $\Delta_1,\ldots,\Delta_{n+1}$, and let 
$\partial\Delta'$ be the simplicial complex consisting of simplices of $\Delta'$ contained in the boundary of $\Delta$. Clearly, $\varphi: \Delta'\rightarrow\Delta$ induces a simplicial map $\partial\Delta'\rightarrow\partial\Delta$, which we will denote by $\varphi_{\partial}$.\\

\noindent
{\em Cohomological proof of Sperner's Lemma.} As in the combinatorial proof, we use an induction by $n$, the case $n=0$ being trivial. Let $n>0$. As in the combinatorial proof, $\varphi_{\partial}$ maps onto $\Delta_{n+1}$ exactly $h$ simplices of $\Delta'$ contained in  $\partial\Delta'$. 

Let us consider the following commutative diagram.
\begin{equation}
\label{hdiag}
\begin{CD}
H^{n-1}(\partial\Delta') @>{\delta}>> H^n(\Delta',\partial\Delta')\\
@A{\varphi_{\partial}^*}AA @A{\varphi^*}AA\\
H^{n-1}(\partial\Delta) @>{\delta}>> H^n(\Delta,\partial\Delta)
\end{CD}
\end{equation}
Here the horizontal maps $\delta$ are the connecting homomorphisms in the cohomological sequences of pairs $(\Delta',\partial\Delta')$, $(\Delta,\partial\Delta)$ respectively. Since cohomology of $\Delta'$ and $\Delta$ are trivial, the exactness of the cohomological sequences of pairs $(\Delta',\partial\Delta')$ and $(\Delta,\partial\Delta)$ imply that $\delta$'s are isomorphisms. As is well known, all cohomology groups in this diagram are isomorphic to the coefficient group ${\mathbb F}_2$.
 
The map $\varphi_{\partial}^*$ is equal to the multiplication by the degree $({\rm mod}\, 2)$ of $\varphi_{\partial}$. This degree is equal to the number of 
$(n-1)$-simplices mapped by $\varphi_{\partial}$ onto any $(n-1)$-simplex of $\partial\Delta$, for example, onto $\Delta_{n+1}$. Therefore, the degree 
$({\rm mod}\, 2)$ of $\varphi_{\partial}$ is equal to $h\, ({\rm mod}\, 2)$, and $\varphi_{\partial}^*$ is equal to the multiplication by $h$. Similarly, 
$\varphi^*$ is equal to the multiplication by $e$, the number of simplices of $\Delta'$ mapped onto $\Delta$.

Since the horizontal maps are isomorphisms, the commutativity of the diagram (\ref{hdiag}) implies that $e\equiv h\, ({\rm mod}\, 2)$. Since $h$ is assumed to be odd, 
it follows that $e$ is odd also. This completes the proof. $\Box$\\
 
Our next goal is to present a cochain-level version of this proof. In order to do this, we need to spell out some basic constructions related to cochains in a somewhat old-fashioned manner.

Let $X$ be a finite simplicial complex and let $A$ be a subcomplex of $X$. 
A {\em $n$-dimensional cochain} of $X$ is a function on the set of simplices of $X$ of dimension $n$ with values in ${\mathbb F}_2$.
Such cochains form a vector space over ${\mathbb F}_2$, denoted by $C^n(X)$. A cochain belongs to the subspace of {\em relative cochains} $C^n(X,A)$ if it is equal to $0$ on all simplices of $A$.  

To every $n$-simplex $\beta$ corresponds a cochain taking the value $1$ on $\beta$ and the value $0$ on all other simplices. By an abuse of notations, 
we will denote this cochain also by $\beta$. Such cochains form a basis of $C^n(X)$, 
and the cochains corresponding to $n$-simplices not contained in $A$ form a basis of $C^n(X,A)$.
Our abuse of notations allows us to consider cochains as a formal sums of simplices. Considered as a formal sum of simplices, 
a cochain belongs to $C^n(X,A)$ if it does not involve simplices of $A$. 

The {\em coboundary map} $\partial^* : C^n(X)\rightarrow C^{n+1}(X)$ is defined as follows. If $\beta$ is an $n$-simplex considered as a cochain, 
then $\partial^*(\beta)$ 
is defined as the sum of simplices of $X$ having $\beta$ as a top-dimensional face. 
The map $\partial^*$ is extended to the whole space $C^n(X)$ by linearity. 
Clearly, $\partial^*$ maps $C^n(X,A)$ to $C^{n+1}(X,A)$. If $\alpha\in C^n (X)$ belongs to the the kernel of $\partial^*$, then it is called a {\em cocycle}. 
Every cocycle $\alpha\in C^n (X)$ defines an element of $H^n (X)$, its cohomology class. 

If $\psi: X \rightarrow Y$ is a simplicial map, then the {\em induced map} $\psi^*: C^n (Y)\rightarrow C^n (X)$ is defined  as follows. 
If $\beta$ is an $n$-simplex, then $\psi^*(\beta)$ is defined as the (formal) sum of all $n$-simplices of $X$ mapped by $\psi$ onto $\beta$. 
The map $\psi^*$ is extended to the whole space $C^n (Y)$ by linearity. One immediately checks that our $\psi^*$ agrees 
with the usual one on simplices (considered as cochains), and, therefore, is equal to it.
The map $\psi^*$ induces a map of cohomology groups $H^n (Y)\rightarrow H^n (X)$, denoted also by $\psi^*$. 

Recall the construction of the {\em connecting map} $\delta : H^{n-1}(A)\rightarrow H^n(X,A)$. Given a cohomology class $a\in H^{n-1}(A)$ represented
by a cocycle $\alpha$, extend  $\alpha$ as a function on simplices in an arbitrary way to a cochain $\overline{\alpha}\in C^{n-1}(X)$, 
and then take coboundary $\partial^*(\overline{\alpha})$. This coboundary is a cocycle in $C^n(X,A)$, and its cohomology class $\delta(a)$
does not depend on the choices $\alpha$ and of the extension $\overline{\alpha}$ of $\alpha$.

The commutativity of the diagram (\ref{hdiag}) was a key step in our cohomological proof. In order to present a cochain-level version of this proof, 
we need to outline a proof of its commutativity. We will do this in the general situation.
So, let $\varphi : (X',A')\rightarrow (X,A)$ be a simplicial map of pairs, i.e. the map $\varphi: X'\rightarrow X$ such that $\varphi(A')\subset A$. 
Let us denote by $\varphi_A$ the induced map $A'\rightarrow A$. We claim that the diagram
\[
\begin{CD}
H^{n-1}(A') @>{\delta}>> H^n(X',A')\\
@A{\varphi_{A}^*}AA @A{\varphi^*}AA\\
H^{n-1}(A) @>{\delta}>> H^n(X,A)
\end{CD}
\]
is commutative. 
Let $a\in H^{n-1}(A)$ be a cohomology class represented by a cocycle $\alpha\in C^{n-1}(A)$, 
and let $\overline{\alpha}\in C^{n-1}(X)$ be an extension of $\alpha$ as in the definition of $\delta$. Then the cocycle $\varphi^*(\partial^*(\overline{\alpha}))$ represents
$\varphi^*(\delta(a))$. Clearly, $\varphi^*(\overline{\alpha})$
extends $\varphi_A^*(\alpha)$. 
Therefore $\partial^*(\varphi^*(\overline{\alpha}))$ represents $\delta(\varphi_A^*(a))$. 
But $\varphi^*\circ\partial^*=\partial^*\circ\varphi^*$, 
as is well known and easy to check using our definitions. Hence
\begin{equation}
\label{commut-chains}
\varphi^*(\partial^*(\overline{\alpha}))=\partial^*(\varphi^*(\overline{\alpha})),
\end{equation}
and therefore
\[
\varphi^*(\delta(a))=\delta(\varphi_A^*(a)).
\]  
This proves the commutativity of our diagram. Our cochain-level proof will be directly based on the equation (\ref{commut-chains}).\\
 
\noindent
{\em Cochain-level proof of Sperner's Lemma.} As in the previous proofs, we use an induction by $n$, the case $n=0$ being trivial. Let $n>0$.

Consider $\Delta_{n+1}$ as a cochain of $\partial\Delta$. 
Since $C^n(\partial\Delta)=0$ (there are no $n$-simplices in $\partial\Delta$), it is a cocycle.
The choice of $\Delta_{n+1}$ is motivated by two circumstances: its cohomological class is a generator of $H^{n-1}(\partial\Delta)$; 
it is the simplex from Sperner's Lemma (any $\Delta_i$ would satisfy the first condition).

We will apply (\ref{commut-chains}) to $\Delta_{n+1}$ in the role of $\alpha$. To this end, 
we need to extend $\Delta_{n+1}$ to a cochain $\overline{\Delta_{n+1}}$ in $C^{n-1}(\Delta)$.
Since all $(n-1)$-simplices of $\Delta$ are contained in $\partial\Delta$, the only possible extension of $\Delta_{n+1}$ is $\Delta_{n+1}$ itself;
$\overline{\Delta_{n+1}}=\Delta_{n+1}$. Therefore, the equation (\ref{commut-chains}) reduces in our case to
\begin{equation}
\label{rc-chains}
\varphi^*(\partial^*(\Delta_{n+1}))=\partial^*(\varphi^*(\Delta_{n+1}))
\end{equation}

Let us compute the left hand side of (\ref{rc-chains}). Clearly,  $\partial^*(\Delta_{n+1})=\Delta$, and hence $\varphi^*(\partial^*(\Delta_{n+1}))=\varphi^*(\Delta)$. 
Using the definition of $\varphi^*$, we see that $\varphi^*(\Delta)=\rho_1+\ldots+\rho_e$, where $\rho_i$'s are as in the combinatorial proof Sperner's Lemma. 
Therefore
\begin{equation}
\label{rhs}
\varphi^*(\partial^*(\Delta_{n+1}))=\rho_1+\ldots+\rho_e.
\end{equation}

Now, let us compute the right hand side of (\ref{rc-chains}). By the definition of $\varphi^*$,
\[
\varphi^*(\Delta_{n+1})= \sigma_1+\ldots+\sigma_h+\tau_1+\ldots+\tau_g,
\]
where $\sigma_j$'s and $\tau_k$'s are as in the combinatorial proof Sperner's Lemma. Indeed,
$\sigma_1,\ldots,\sigma_h,\tau_1,\ldots,\tau_g$ are, by definition, all $(n-1)$-dimensional simplices of $\Delta'$ mapped by $\varphi$ onto $\Delta_{n+1}$.
Every $\sigma_i$ is an $(n-1)$-dimensional simplex contained in $\partial\Delta'$. Therefore, it is a face of a unique $n$-dimensional simplex $\widehat{\sigma_i}$ of $\Delta'$, and $\partial^*(\sigma_i)= \widehat{\sigma_i}$. 
Note that the simplices $\widehat{\sigma_1},\ldots,\widehat{\sigma_h}$ are distinct, because they have distinct intersections with  $\partial\Delta'$.
We have
\begin{equation}
\label{lhs}
\partial^*(\varphi^*(\Delta_{n+1}))=\partial^*(\sigma_1+\ldots+\sigma_h+\tau_1+\ldots+\tau_g)=
\end{equation}
\[=
\widehat{\sigma_1}+\ldots+\widehat{\sigma_h}+\partial^*(\tau_1)+\ldots+\partial^*(\tau_g).
\]

By substituting (\ref{rhs}) and (\ref{lhs}) into (\ref{rc-chains}), we get
\begin{equation}
\label{chain}
\widehat{\sigma_1}+\ldots+\widehat{\sigma_h}+\partial^*(\tau_1)+\ldots+\partial^*(\tau_g)=\rho_1+\ldots+\rho_e.
\end{equation}
No simplex $\tau_k$ is contained in $\partial\Delta'$, and hence every $\tau_k$
is a face of exactly two $n$-dimensional simplices of $\Delta'$. 
Therefore, every $\partial^*(\tau_k)$ is a sum of two simplices. 
By summing ${\rm mod}\, 2$ the coefficients of the simplices in the both sides of this equation, we get $h+2g \equiv e\, ({\rm mod}\, 2)$. This implies that 
$e\equiv h\, ({\rm mod}\, 2)$. Note that the cohomology class of every top-dimensional simplex is a generator of $H^n(\Delta',\partial\Delta')$, and therefore
the summing ${\rm mod}\, 2$ of the coefficients of the simplices in our cochains is nothing else as the computation of their cohomology classes. 

As before, the congruence $e\equiv h\, ({\rm mod}\, 2)$ implies that $e$ is odd if and only if $h$ is odd, and allows us to complete the proof. $\Box$\\ 

In fact, one can get more from the equation (\ref{chain}). Two simplices on the left hand side of (\ref{chain}) cancel if and only if either $\tau_i$ and $\tau_j$, or $\tau_i$ and $\sigma_j$ are two faces of the same $n$-dimensional simplex (it has to be $\widehat{\sigma_j}$ in the second case). Then this simplex enters the sum twice (and cancels because $2=0$ in $\mathbb{F}_2$). These $n$-simplices are exactly the simplices of the third type in the sense of the combinatorial proof of Lemma 1. So, there are $f$ of them, and there are exactly $f$ cancellations on the left hand side of (\ref{chain}). Therefore, (\ref{chain}) implies that $h+2g=e+2f$, i.e. the basic equation (\ref{sperner}) of the combinatorial proof of Lemma 1. One can say that (\ref{chain}) is a chain-level realization of the equation (\ref{sperner}). Alternatively, we can say that the combinatorial proof of Sperner's Lemma is a down-to-earth version of the chain-level proof, which, in turn, is a chain version of the cohomological proof.

\paragraph{3. DEDUCTION OF THE BROUWER FIXED-POINT THEOREM FROM SPERNER'S LEMMA.} To begin with, 
we present the standard deduction of the Brouwer Theorem from Sperner's Lemma.

\paragraph{Brouwer Theorem.} {\em Every continuous map $f:\Delta\rightarrow\Delta$ has a fixed point.}\\ 

\noindent
{\em Proof.} Let us realize $\Delta$ as the standard $n$-simplex 
\[\{(x^1,\ldots,x^{n+1})\,|\, x^1,\ldots,x^{n+1}\geqslant 0; x^1+\ldots+x^{n+1}=1\}\]
in ${\mathbb R}^{n+1}$. Choose a sequence of subdivisions $\Delta'_0,\Delta'_1, \Delta'_2, \ldots$ of the simplex $\Delta$ in such a way that the maximal diameter of  simplices of $\Delta'_i$ tends to $0$ when $i\rightarrow\infty$. For example, one can take $\Delta'_0=\Delta$ and $\Delta'_{k+1}$ to be the barycentric subdivision of $\Delta_k$ for $k>0$. 

Suppose that $f$ has no fixed points. Let us denote by $x^j$ the $j$-th coordinate of a vector $x\in{\mathbb R}^{n+1}$. 
Suppose that $x,y\in\Delta$ and $y^j\geqslant x^j$ for all $j=1,2,\ldots,n+1$. Since the coordinates of $y$ are nonnegative and 
their sum is equal to $1$, and the same is true for $x$, these inequalities imply that $y=x$. Therefore,
for every $x\in\Delta$ there is some $j$ such that $f(x)^j<x^j$ (because $f(x)\neq x$).
Let us label a vertex $w$ of  $\Delta'_i$ by any $j$ such that $f(w)^j<w^j$. 
By Sperner's Lemma, some simplex $\sigma_i$ of $\Delta'_i$ has
$\{1,2,\ldots, n+1\}$ as the set of labels of its vertices. Pick up an arbitrary point $x_i$ in $\sigma_i$. Since $\Delta$ is compact, we can assume (after replacing the sequence $\Delta'_i$ by a subsequence, if necessary) that the sequence $x_i$ converges to a point $x\in\Delta$. Since the diameters of simplices $\sigma_i$ tend to $0$, for every $j=1,2,\ldots, n+1$ there are points $w\in\Delta$ arbitrarily close to $x$ and such that $f(w)^j<w^j$, namely, the vertices of simplices $\sigma_i$ labeled by $j$. By taking the limit, we conclude that $f(x)^j\leqslant x^j$ 
for all $j = 1,2,\ldots, n+1$. By the observation at the beginning of the proof this implies that $f(x)=x$, contrary to the assumption. $\Box$\\

At the first sight this proof completely avoids a key step in essentially all proofs of the Brouwer Theorem: the construction of a retraction $\Delta\rightarrow\partial\Delta$ from a fixed point free map and using the No-Retraction Theorem to the effect that such retractions do not exist. 
Let us recall this construction. If $f:\Delta\rightarrow\Delta$ has no fixed points, 
we can define a  map $r$ assigning to $x\in\Delta$ the point $r(x)$ of intersection with $\partial\Delta$ of the ray going from $f(x)$ to $x$. 
The map $r$ is well-defined because $f(x)\neq x$ for all $x$ by the assumption. 
Clearly, it is a retraction $\Delta\rightarrow\partial\Delta$.\\

In fact, this map $r$ is implicitly used in the above proof. 
Notice that if a vertex $w$ is labeled by $j$, then $w-f(w)$ has positive $j$-th coordinate. At the same time the vector
$w-f(w)$ is contained in the hyperplane $x^1+\ldots+x^{n+1}=1$. This implies that the ray from $f(w)$ to $w$ is contained in the hyperplane $x^1+\ldots+x^{n+1}=1$ and intersects the boundary $\partial\Delta$. Hence, $w$ may be labeled by $j$ if and only if $r(w)$ is contained in the intersection of $\partial\Delta$ with the open half-space $\{x\,|\, f(w)^j<x^j\}$. 
The last intersection is contained in $\partial\Delta\setminus\Delta_j$ (because the face $\Delta_j$ is defined by $x_j=0$). 
It is tempting to allow as a label of $w$ any $j$ such that $r(w)\in\partial\Delta\setminus\Delta_j$. 
This has the advantage of $\partial\Delta\setminus\Delta_j$ being independent of $w$. It turns out that while this condition is too weak to
prove the Brouwer theorem, we can deduce No-Retraction Theorem from Sperner's Lemma using a strengthening of it. 
We will need also to use the compactness in a less direct manner. 

\paragraph{No-Retraction Theorem.} {\em There is no retraction $\Delta\rightarrow\partial\Delta$.}\\

\noindent
{\em Proof.} Let $r$ be such a retraction. Recall that {\em the open star\/} ${\rm st}(w)$ of a vertex $w$ of a simplicial complex is the union of the interiors of all simplices having $w$ as a vertex (the interior of a simplex is the simplex minus its boundary). 
In particular, $\partial\Delta\setminus\Delta_j={\rm st}(v_j)$ (the star is taken in $\partial\Delta$). 
Note that diameter of a star of a vertex is no bigger than twice the maximal diameter of a simplex. 
Therefore, if the diameters of simplices of a subdivision $\Delta'$ of $\Delta$ are small enough, 
then for every vertex $w$ of $\Delta'$ the image $r({\rm st}(w))$ is contained in some ${\rm st}(v_j)$. 
This follows from the Lebesgue lemma applied to the open covering of the compact set $\Delta$ by the preimages $r^{-1}({\rm st} (v_j))$. 
Let us label $w$ by any $j$ such that  $r({\rm st}(w))\subset\partial\Delta\setminus\Delta_j={\rm st}(v_j)$. 
By Sperner's Lemma, some simplex $\sigma$ of $\Delta'$ is has $\{1,2,\ldots, n+1\}$ as the set of labels of its vertices. 
The interior ${\rm int}\,\sigma$ of $\sigma$ is contained ${\rm st}(w)$ for any vertex $w$ of $\sigma$. 
Since for every $j=1,2,\ldots,n+1$ some vertex $w$ of $\sigma$ is labeled by $j$, we conclude that 
$r({\rm int}\,\sigma)\subset{\rm st}(v_j)$ for every $j$. 
But the sets ${\rm st}(v_j)=\partial\Delta\setminus\Delta_j$ obviously have empty intersection. 
So, we have a contradiction, completing the proof. $\Box$\\

Note that the labeling from this proof defines a map $\varphi:\Delta'\rightarrow\partial\Delta$ which is a {\em simplicial approximation} of $r$. Indeed,
$r({\rm st}(w))\subset{\rm st}(\varphi(w))$ is a well known necessary and sufficient condition for a simplicial map $\varphi$ to be a simplicial approximation of a continuous map $r$ (see \cite{P1}, Theorem 8.6, for example). The strengthening of the condition $r(w)\in\Delta\setminus\Delta_j$ to $r({\rm st}(w))\subset \Delta\setminus\Delta_j$ is motivated exactly by this. Note that the above use of the Lebesgue lemma is the standard way to establish the existence of simplicial approximations (see \cite{H}, Section 2.C, or \cite{P1}, Section 8.4, for example). Sperner's Lemma implies that the existence of such a simplicial approximation leads to a contradiction.

\paragraph{Acknowledgments.} I would like to thank F. Petrov and A. Zelevinsky for their interest in cohomological interpretation of Sperner's Lemma, 
which stimulated me to write this note.
This work was supported in part by the NSF Grant DMS-0406946.


\begin{thebibliography}{XXX}

\bibliography{References}

\bibitem[AH]{AH} P. Alexandroff, H. Hopf, {\em Topologie}, Springer, 1935; xiii, 636 pp. 

\bibitem[H]{H} A. Hatcher, {\em Algebraic Topology}, Cambridge Uiversity Press, 2001; xii, 544 pp.

\bibitem[I]{I} N. V. Ivanov, A Topologist's View of the Dunford-Schwartz Proof of the Brouwer Fixed-Point Theorem, {\em The Mathematical Intelligencer}, V. 22, No. 3 (2000), 55--57.

\bibitem[P1]{P1} V. V. Prasolov, {\em  Elements of Combinatorial and Differential Topology}, Graduate Studies in Math., V. 74, AMS, 2006; xii, 331 pp.

\bibitem[P2]{P2} V. V. Prasolov, {\em Elements of Homology Theory}, Graduate Studies in Math., V. 81, AMS, 2007; xi, 418 pp.

\bibitem[S]{S} E. Sperner, Neuer Beweis f\"ur die Invarianz der Dimensionszahl und des Gebietes, {\em Abh. Math. Semin. Hamburg. Univ.\/} Bd. 6 (1928), 265--272. 


\end{thebibliography}


\noindent
{\it Michigan State University, Department of Mathematics, Wells Hall,
East Lansing, MI 48824-1027\\

\noindent
ivanov@math.msu.edu}

\end{document}