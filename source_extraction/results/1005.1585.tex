% uniform_boundedness_v2d.tex

% Version 1:  ADS started 22 Mar 2010 @ UCL, finished 7 May 2010
% Version 2:  ADS 14 Jun 2010 @ ENS-Paris -- revised for Amer Math Monthly
% Version 2a: ADS 1 July 2010 @ UCL -- revised again for Amer Math Monthly
% Version 2b: ADS 13-21 Sep 2010 @ NYU -- revised again, incorporating comments
%     of Jurgen Voigt and Markus Haase
% Version 2b_AMM: ADS 23 Sep 2010 @ NYU -- title page abridged for AMM
% Version 2c_AMM: ADS 12 Oct 2010 @ NYU -- inserted AMM mark-up
% Version 2d_AMM: ADS 20 Oct 2010 @ NYU -- removed \limits in in-text equations
% Version 2d: ADS 29 Dec 2010 @ UCL:  for arXiv, restored original cover page

\documentclass[12pt]{article}

%% \usepackage{drafthead} %% COMMENT THIS OUT FOR FINAL VERSION!!!
\usepackage{amsfonts,amssymb,amsthm,amsmath,latexsym}
%%\usepackage{subeqnarray,eqsection,indent}
%%\usepackage{subeqnarray,indent,eepicemu}
\usepackage{cite}


\oddsidemargin 0.25in \evensidemargin 0.25in \textwidth 6.0in
\topmargin -0.2in \headsep 0.2in
\textheight 8.9in

\footnotesep 14pt
\floatsep 28pt plus 2pt minus 4pt  % Nominal is double what is in art12.sty
\textfloatsep 40pt plus 2pt minus 4pt
\intextsep 28pt plus 4pt minus 4pt

\def\doublespace{ \renewcommand{\baselinestretch}{1.7} \large\normalsize }
\def\singlespace{ \renewcommand{\baselinestretch}{1} \large\normalsize }
  % The \large\normalsize forces the new \baselineskip to take effect.

%% \def\thefootnote{\fnsymbol{footnote}}   % To get asterisk for footnote.



\begin{document}

\title{A really simple elementary proof \\
       of the uniform boundedness theorem}

\author{
     {\small Alan D.~Sokal\thanks{Also at Department of Mathematics,
           University College London, London WC1E 6BT, England.}}  \\[-2mm]
     {\small\it Department of Physics}       \\[-2mm]
     {\small\it New York University}         \\[-2mm]
     {\small\it 4 Washington Place}          \\[-2mm]
     {\small\it New York, NY 10003 USA}      \\[-2mm]
     {\small\tt sokal@nyu.edu}               \\[-2mm]
     {\protect\makebox[5in]{\quad}}  % To force authors' names to be written
                                     %   vertically, one above another.
                                     % (\author seems to put them side-by-side
                                     %   if there is room.)
     \\
}

\date{May 7, 2010 \\[3mm]
      revised October 20, 2010 \\[1mm]
      to appear in the {\em American Mathematical Monthly}}
%% \date{\today\ {\bf Need to fix final date here!!!}}

\maketitle
\thispagestyle{empty}   % Suppress page number on front page.
\begin{abstract}
I give a proof of the uniform boundedness theorem that is elementary
(i.e., does not use any version of the Baire category theorem)
and also extremely simple.
\end{abstract}

\bigskip
\noindent
{\bf Key Words:}  Uniform boundedness;  gliding hump;  sliding hump;
Baire category.

\bigskip
\noindent
{\bf Mathematics Subject Classification (MSC 2000) codes:}
46B99 (Primary); 46B20, 46B28 (Secondary).

\clearpage



%%%EXTRACTING
 \usepackage{./extract}
%ENDEXTRACTING
\newtheorem{thm}{Theorem}
%%\newtheorem{cor}[thm]{Corollary}
%%\newtheorem{lem}[thm]{Lemma}

\newtheorem{theorem}{Theorem}[section]
%%\newtheorem{theorem}{Theorem}
\newtheorem{proposition}[theorem]{Proposition}
\newtheorem{lemma}[theorem]{Lemma}
\newtheorem{corollary}[theorem]{Corollary}
\newtheorem{definition}[theorem]{Definition}
\newtheorem{conjecture}[theorem]{Conjecture}
\newtheorem{question}[theorem]{Question}
\newtheorem{example}[theorem]{Example}
\newtheorem{remark}[theorem]{Remark}


\renewcommand{\theenumi}{\alph{enumi}}
\renewcommand{\labelenumi}{(\theenumi)}
\def\eop{\hbox{\kern1pt\vrule height6pt width4pt
depth1pt\kern1pt}\medskip}
\def\prf{\par\noindent{\bf Proof.\enspace}\rm}
\def\rmk{\par\medskip\noindent{\bf Remark\enspace}\rm}

\newcommand{\be}{\begin{equation}}
\newcommand{\ee}{\end{equation}}
\newcommand{\<}{\langle}
\renewcommand{\>}{\rangle}
\newcommand{\widebar}{\overline}
\def\reff#1{(\protect\ref{#1})}
\def\spose#1{\hbox to 0pt{#1\hss}}
\def\ltapprox{\mathrel{\spose{\lower 3pt\hbox{$\mathchar"218$}}
    \raise 2.0pt\hbox{$\mathchar"13C$}}}
\def\gtapprox{\mathrel{\spose{\lower 3pt\hbox{$\mathchar"218$}}
    \raise 2.0pt\hbox{$\mathchar"13E$}}}
\def\textprime{${}^\prime$}
\def\proof{\par\medskip\noindent{\sc Proof.\ }}
\def\firstproof{\par\medskip\noindent{\sc First Proof.\ }}
\def\secondproof{\par\medskip\noindent{\sc Second Proof.\ }}
\def\thirdproof{\par\medskip\noindent{\sc Third Proof.\ }}
%\def\qed{\hbox{\hskip 6pt\vrule width6pt height7pt depth1pt \hskip1pt}\bigskip}
\def\qed{ $\square$ \bigskip}
\def\proofof#1{\bigskip\noindent{\sc Proof of #1.\ }}
\def\firstproofof#1{\bigskip\noindent{\sc First Proof of #1.\ }}
\def\secondproofof#1{\bigskip\noindent{\sc Second Proof of #1.\ }}
\def\thirdproofof#1{\bigskip\noindent{\sc Third Proof of #1.\ }}
\def\half{ {1 \over 2} }
\def\third{ {1 \over 3} }
\def\twothird{ {2 \over 3} }
\def\smfrac#1#2{{\textstyle{#1\over #2}}}
\def\smhalf{ {\smfrac{1}{2}} }
\newcommand{\real}{\mathop{\rm Re}\nolimits}
\renewcommand{\Re}{\mathop{\rm Re}\nolimits}
\newcommand{\imag}{\mathop{\rm Im}\nolimits}
\renewcommand{\Im}{\mathop{\rm Im}\nolimits}
\newcommand{\sgn}{\mathop{\rm sgn}\nolimits}
\newcommand{\tr}{\mathop{\rm tr}\nolimits}
\newcommand{\supp}{\mathop{\rm supp}\nolimits}
\def\hboxscript#1{ {\hbox{\scriptsize\em #1}} }
\renewcommand{\emptyset}{\varnothing}

\newcommand{\restrict}{\upharpoonright}
%%\newcommand{\implies}{\;\Longrightarrow\;}

\newcommand{\scra}{{\mathcal{A}}}
\newcommand{\scrb}{{\mathcal{B}}}
\newcommand{\scrc}{{\mathcal{C}}}
\newcommand{\scre}{{\mathcal{E}}}
\newcommand{\scrf}{{\mathcal{F}}}
\newcommand{\scrg}{{\mathcal{G}}}
\newcommand{\scrh}{{\mathcal{H}}}
\newcommand{\scrk}{{\mathcal{K}}}
\newcommand{\scrl}{{\mathcal{L}}}
\newcommand{\scro}{{\mathcal{O}}}
\newcommand{\scrp}{{\mathcal{P}}}
\newcommand{\scrr}{{\mathcal{R}}}
\newcommand{\scrs}{{\mathcal{S}}}
\newcommand{\scrt}{{\mathcal{T}}}
\newcommand{\scrv}{{\mathcal{V}}}
\newcommand{\scrw}{{\mathcal{W}}}
\newcommand{\scrz}{{\mathcal{Z}}}

\newcommand{\ahat}{{\widehat{a}}}
\newcommand{\Zhat}{{\widehat{Z}}}
\renewcommand{\k}{{\mathbf{k}}}
\newcommand{\n}{{\mathbf{n}}}
\newcommand{\vv}{{\mathbf{v}}}
\newcommand{\bv}{{\mathbf{v}}}
\newcommand{\w}{{\mathbf{w}}}
\newcommand{\x}{{\mathbf{x}}}
\newcommand{\g}{{\boldsymbol{g}}}
\newcommand{\cc}{{\mathbf{c}}}
\newcommand{\zero}{{\mathbf{0}}}
\newcommand{\one}{{\mathbf{1}}}
\newcommand{\balpha}{{\boldsymbol{\alpha}}}

\newcommand{\C}{{\mathbb C}}
\newcommand{\D}{{\mathbb D}}
\newcommand{\Z}{{\mathbb Z}}
\newcommand{\N}{{\mathbb N}}
\newcommand{\Q}{{\mathbb Q}}
\newcommand{\R}{{\mathbb R}}
\newcommand{\RR}{{\mathbb R}}

\def\cbar{{\overline{C}}}
\def\ctilde{{\widetilde{C}}}
\def\zbar{{\overline{Z}}}
\def\pitilde{{\widetilde{\pi}}}


%
% Variants of \binom  (defined using the AMS "genfrac" command)
%
\newcommand{\stirlingsubset}[2]{\genfrac{\{}{\}}{0pt}{}{#1}{#2}}
\newcommand{\stirlingcycle}[2]{\genfrac{[}{]}{0pt}{}{#1}{#2}}
\newcommand{\assocstirlingsubset}[3]{{\genfrac{\{}{\}}{0pt}{}{#1}{#2}}_{\! \ge #3}}
\newcommand{\assocstirlingcycle}[3]{{\genfrac{[}{]}{0pt}{}{#1}{#2}}_{\ge #3}}
\newcommand{\genstirlingsubset}[4]{{\genfrac{\{}{\}}{0pt}{}{#1}{#2}}_{\! #3,#4}}
\newcommand{\euler}[2]{\genfrac{\langle}{\rangle}{0pt}{}{#1}{#2}}
\newcommand{\eulergen}[3]{{\genfrac{\langle}{\rangle}{0pt}{}{#1}{#2}}_{\! #3}}
\newcommand{\eulersecond}[2]{\left\langle\!\! \euler{#1}{#2} \!\!\right\rangle}
\newcommand{\eulersecondgen}[3]{{\left\langle\!\! \euler{#1}{#2} \!\!\right\rangle}_{\! #3}}
\newcommand{\binomvert}[2]{\genfrac{\vert}{\vert}{0pt}{}{#1}{#2}}


% Array for subscripts

\newenvironment{sarray}{
             \textfont0=\scriptfont0
             \scriptfont0=\scriptscriptfont0
             \textfont1=\scriptfont1
             \scriptfont1=\scriptscriptfont1
             \textfont2=\scriptfont2
             \scriptfont2=\scriptscriptfont2
             \textfont3=\scriptfont3
             \scriptfont3=\scriptscriptfont3
           \renewcommand{\arraystretch}{0.7}
           \begin{array}{l}}{\end{array}}

\newenvironment{scarray}{
             \textfont0=\scriptfont0
             \scriptfont0=\scriptscriptfont0
             \textfont1=\scriptfont1
             \scriptfont1=\scriptscriptfont1
             \textfont2=\scriptfont2
             \scriptfont2=\scriptscriptfont2
             \textfont3=\scriptfont3
             \scriptfont3=\scriptscriptfont3
           \renewcommand{\arraystretch}{0.7}
           \begin{array}{c}}{\end{array}}



\clearpage


One of the pillars of functional analysis is the
uniform boundedness theorem:

\bigskip

{\sc Uniform boundedness theorem.}
Let $\scrf$ be a family of bounded linear operators
from a Banach space $X$ to a normed linear space $Y$.
If $\scrf$ is pointwise bounded
(i.e., $\sup_{T \in \scrf} \|Tx\| < \infty$ for all $x \in X$),
then $\scrf$ is norm-bounded
(i.e., $\sup_{T \in \scrf} \|T\| < \infty$).

\bigskip

The standard textbook proof
%% of the uniform boundedness theorem
% \cite{Dunford_58,Reed_72,Royden_88,Rudin_73,Yosida_80},  %% Also Cheney_01
(e.g., \cite[p.~81]{Reed_72}),
which goes back to Stefan Banach, Hugo Steinhaus,
and Stanis\l{}aw Saks in 1927 \cite{Banach_27},
employs the Baire category theorem or some variant thereof.\footnote{
   See \cite[p.~319, note~67]{Birkhoff_84} concerning credit to Saks.
}
This proof is very simple,
but its reliance on the Baire category theorem makes it
not completely elementary.

By contrast, the original proofs
%% of the uniform boundedness theorem
given by Hans Hahn \cite{Hahn_22} and Stefan Banach \cite{Banach_22}
in 1922 were quite different:
they began from the assumption that $\sup_{T \in \scrf} \|T\| = \infty$
and used a ``gliding hump'' (also called ``sliding hump'') technique
to construct a sequence $(T_n)$ in $\scrf$ and a point $x \in X$
such that $\lim_{n \to\infty} \|T_n x \| = \infty$.\footnote{
   Hahn's proof is discussed in at least two modern textbooks:
   see \cite[Exercise~1.76, p.~49]{Megginson_98} and
   \cite[Exercise~3.15, pp.~71--72]{MacCluer_09}.
}
Variants of this proof were later given by
%% Theophil Henry Hildebrandt
T.~H.~Hildebrandt \cite{Hildebrandt_23}
and Felix Hausdorff \cite{Hausdorff_32,Hennefeld_80}.\footnote{
   See also \cite[pp.~63--64]{Riesz_55} and \cite[pp.~74--75]{Weidmann_80}
   for an elementary proof that is closely related to the standard
   ``nested ball'' proof of the Baire category theorem;
   and see \cite[Problem 27, pp.~14--15 and 184]{Halmos_67}
   and \cite{Holland_69} for elementary proofs
   in the special case of linear functionals on a Hilbert space.
   %% (i.e.\ $X=\scrh$ and $Y=\R$ or $\C$).
}
These proofs are elementary, but the details are a bit fiddly.

Here is a {\em really}\/ simple proof
%% of the uniform boundedness theorem
along similar lines:

\bigskip

{\bf Lemma.}   Let $T$ be a bounded linear operator
from a normed linear space $X$ to a normed linear space $Y$.
Then for any $x \in X$ and $r > 0$, we have
\be
   \sup\limits_{x' \in B(x,r)} \| Tx' \| \;\ge\; \|T\| r  \;,
\ee
where $B(x,r) = \{x' \in X \colon\: \|x'-x\| < r \}$.

\smallskip

\proof
For $\xi \in X$ we have
\be
   \max\bigl\{ \| T(x+\xi) \| ,\, \| T(x-\xi) \| \bigr\}
   \;\,\ge\;\,
   \smhalf \bigl[ \| T(x+\xi) \| + \| T(x-\xi) \| \bigr]
   \;\,\ge\;\,
   \| T \xi \|  \;,
   \quad
\ee
where the second $\ge$ uses the triangle inequality
in the form $\| \alpha-\beta \| \le \|\alpha\| + \|\beta\|$.
Now take the supremum over $\xi \in B(0,r)$.
\qed

\vspace*{-4mm}

\proofof{the uniform boundedness theorem}
Suppose that $\sup_{T \in \scrf} \|T\| = \infty$,
and choose $(T_n)_{n=1}^\infty$ in $\scrf$ such that $\|T_n\| \ge 4^n$.
Then set $x_0 = 0$, and for $n \ge 1$ use the lemma to
choose inductively $x_n \in X$
such that $\| x_n - x_{n-1} \| \le 3^{-n}$
and $\| T_n x_n \| \ge \smfrac{2}{3} 3^{-n} \| T_n \|$.
The sequence $(x_n)$ is Cauchy, hence convergent to some $x \in X$;
and it is easy to see that
$\| x-x_n \| \le \smfrac{1}{2} 3^{-n}$ and hence
$\| T_n x \| \ge \smfrac{1}{6} 3^{-n} \| T_n \| \ge \smfrac{1}{6} (4/3)^n
 \to \infty$.
\qed

\medskip

{\bf Remarks.}
1.  As just seen, this proof is most conveniently expressed in terms of a
{\em sequence}\/ $(x_n)$ that converges to $x$.
This contrasts with the earlier ``gliding hump'' proofs,
which used a {\em series}\/ that sums to $x$.
Of course, sequences and series are equivalent,
so each proof can be expressed in either language;
it is a question of taste which formulation one finds simpler.

2.  This proof is extremely wasteful from a quantitative point of view.
A quantitatively sharp version of the uniform boundedness theorem
%% for real Banach spaces
follows from Ball's ``plank theorem'' \cite{Ball_91}:
namely, if $\sum_{n=1}^\infty \|T_n\|^{-1} < \infty$,
then there exists $x \in X$ such that
$\lim_{n\to\infty} \| T_n x \| = \infty$
(see also \cite{Muller_09}).
%% {\bf Is this the correct statement?????
%%   And is it true for general linear maps, or only for linear functionals???}

3. A similar (but slightly more complicated) elementary proof
of the uniform boundedness theorem can be found in \cite[p.~83]{Gohberg_03}.

4.  ``Gliding hump'' proofs continue to be useful in functional analysis:
see \cite{Swartz_96} for a detailed survey.

5.  The standard Baire category method
yields a slightly stronger version of the uniform boundedness theorem
than the one stated here, namely:
if $\sup_{T \in \scrf} \|Tx\| < \infty$
for a {\em nonmeager}\/ (i.e., second category) set of $x \in X$,
then $\scrf$ is norm-bounded.

6.  The uniform boundedness theorem has generalizations
to suitable classes of non-normable
%% \cite{Dunford_58}
and even non-metrizable
%% \cite{Narici_85,Robertson_73,Rudin_73,Schaefer_99}
topological vector spaces
(see, e.g., \cite[pp.~82--87]{Schaefer_99}).
I~leave it to others to determine whether any ideas from this proof
can be carried over to these more general settings.

7.  More information on the history of the uniform boundedness theorem
can be found in
\cite[pp.~302, 319n67]{Birkhoff_84},
\cite[pp.~138--142]{Dieudonne_81}, and
\cite[pp.~21--22, 40--43]{Pietsch_07}.



\bigskip
\bigskip

\noindent
{\bf Acknowledgments.}
I wish to thank Keith Ball for reminding me of Hahn's ``gliding hump'' proof;
it~was my attempt to fill in the details of Keith's sketch
that led to the proof reported here.
Keith informs me that versions of this proof have been independently devised
by at least four or five people,
though to his knowledge none of them have bothered to publish it.
I also wish to thank David Edmunds and Bob Megginson
for helpful correspondence, and three anonymous referees
for valuable suggestions concerning the exposition.
Finally, I wish to thank J\"urgen Voigt and Markus Haase
for drawing my attention to the elementary proofs in
\cite{Gohberg_03,Riesz_55,Weidmann_80}.

This research was supported in part by
U.S.\ National Science Foundation grant PHY--0424082.

\bigskip
\bigskip





%%\addcontentsline{toc}{section}{References}
\begin{thebibliography}{199}

\bibitem{Ball_91}  K. Ball, The plank problem for symmetric bodies,
   {\em Invent. Math.}\/ {\bf 104} (1991) 535--543.

\bibitem{Banach_22}  S. Banach, Sur les op\'erations dans les ensembles
   abstraits et leur application aux \'equations int\'egrales,
  {\em  Fund. Math.}\/ {\bf 3} (1922) 133--181;
   also in {\em \OE{}uvres avec des Commentaires}\/,
   vol.~2,
   \'Editions scientifiques de Pologne, Warsaw, 1979. %% pp.~305--348.

\bibitem{Banach_27}  S. Banach and H. Steinhaus, Sur le principe de la
   condensation des singularit\'es, {\em Fund. Math.}\/ {\bf 9} (1927) 50--61.

\bibitem{Birkhoff_84}  G. Birkhoff and E. Kreyszig,
   The establishment of functional analysis,
   {\em Historia Math.}\/ {\bf 11} (1984) 258--321.
   %% see p.~302 and note 67 (p.~319).
   %% Note 67 says:  "Actually, it was S. Saks who first promoted the
   %%   Baire category method, as a referee of the Banach--Steinhaus paper,
   %%   by suggesting the original lengthy constructive proof be replaced
   %%   with the now familiar category argument."

%% \bibitem{Cheney_01}  W. Cheney, {\em Analysis for Applied Mathematics}\/
%%    (Springer-Verlag, New York, 2001), Section~1.7 (pp.~41--42).

\bibitem{Dieudonne_81}  J. Dieudonn\'e, {\em History of Functional Analysis}\/,
   %% North-Holland Mathematics Studies \#49
   North-Holland, Amsterdam, 1981.
   %% Section~VI.4 (pp.~138--142).

%% \bibitem{Dunford_58}  N. Dunford and J.T. Schwarz
%%    [with the assistance of W.G. Bade and R.G. Bartle],
%%    {\em Linear Operators.\ I.~General Theory}\/,
%%    Interscience, New York--London, 1958.
%%    %% Sections~II.1 (pp.~49--55) and II.5 (pp.~80--82).
%%    Reprinted by Wiley, New York, 1988.

\bibitem{Gohberg_03}  I. Gohberg, S. Goldberg, and M.~A. Kaashoek,
  {\em Basic Classes of Linear Operators}\/, Birkh\"auser, Basel, 2003.

\bibitem{Hahn_22}  H. Hahn, \"Uber Folgen linearer Operationen,
   {\em Monatsh. Math. Phys.}\/ {\bf 32} (1922) 3--88.

\bibitem{Halmos_67}  P.~R. Halmos, {\em A Hilbert Space Problem Book}\/,
   2nd ed., Springer-Verlag, New York, 1982.
   % Problem~27 (pp.~14--15) and Solution~27 (p.~184).
   %% IN FIRST ED:
   %% (Van Nostrand, Princeton--Toronto--London, 1967),
   %% Problem~20 (pp.~12--13) and Solution~20 (pp.~183--184).

\bibitem{Hausdorff_32}  F. Hausdorff, Zur Theorie der linearen metrischen
   R\"aume, {\em J. Reine Angew. Math.}\/ {\bf 167} (1932) 294--311.

\bibitem{Hennefeld_80}  J. Hennefeld, A nontopological proof of the
   uniform boundedness theorem,
   {\em Amer. Math. Monthly}\/ {\bf 87} (1980) 217.

\bibitem{Hildebrandt_23}  T.~H. Hildebrandt, On uniform limitedness of sets
   of functional operations, {\em Bull. Amer. Math. Soc.}\/
   {\bf 29} (1923) 309--315.

\bibitem{Holland_69}  S.~S. Holland Jr., A Hilbert space proof of the
   Banach-Steinhaus theorem,
   {\em Amer. Math. Monthly}\/ {\bf 76} (1969) 40--41.

\bibitem{MacCluer_09}  B.~D. MacCluer, {\em Elementary Functional Analysis}\/,
   Springer-Verlag, New York, 2009.

\bibitem{Megginson_98}  R.~E. Megginson, {\em An Introduction to Banach Space
   Theory}\/, Springer-Verlag, New York, 1998.

\bibitem{Muller_09}  V. M\"uller and J. Vr\v{s}ovsk\'y,
   Orbits of linear operators tending to infinity,
   {\em Rocky Mountain J. Math.}\/ {\bf 39} (2009) 219--230.

%% \bibitem{Narici_85}  L. Narici and E. Beckenstein, {\em Topological Vector
%%    Spaces}\/, Dekker, New York, 1985.
%%    %% Chapter~11.

\bibitem{Pietsch_07}  A. Pietsch, {\em History of Banach Spaces and
   Linear Operators}\/, Birkh\"auser, Boston, 2007.
   %% Sections~1.8 (pp.~21--22) and 2.4 (pp.~40--43).

\bibitem{Reed_72}  M. Reed and B. Simon, {\em Methods of Modern Mathematical
   Physics}\/, vol.~I, {\em Functional Analysis}\/,
   Academic Press, New York, 1972.
   %% Section~III.5 (p.~81).

\bibitem{Riesz_55}  F. Riesz and B. Sz.-Nagy, {\em Functional Analysis}\/,
   Ungar, New York, 1955.

%% \bibitem{Robertson_73}  A.P. Robertson and W. Robertson,
%%    {\em Topological Vector Spaces}\/, 2nd ed., 
%%    Cambridge University Press, London-New York, 1973, reprinted 1980.
%%    %% Chapter~IV.

%% \bibitem{Royden_88}  H.L. Royden, {\em Real Analysis}\/, 3rd ed.,
%%    Macmillan, New York, 1988.
%%    %% Sections~7.8 (p.~160) and 10.4 (p.~232).
%% 
%% \bibitem{Rudin_73}  W. Rudin, {\em Functional Analysis}\/, 2nd ed.,
%%    McGraw-Hill, New York, 1991.
%%    %% pp.~43--46.

\bibitem{Schaefer_99}  H.~H. Schaefer and M.~P. Wolff,
   {\em Topological Vector Spaces}\/, 2nd ed.,
   Springer-Verlag, New York, 1999.
   %% Section~III.4 (pp.~82--87).

\bibitem{Swartz_96}  C. Swartz, {\em Infinite Matrices and the Gliding Hump}\/,
   World Scientific, River Edge, NJ, 1996.
   
%% \bibitem{Taylor_58} A.E. Taylor, {\em Introduction to Functional Analysis}\/,
%%    Wiley, New York, 1958.
%%    %% Section~4.4 (pp.~201--204).
%%    %% See also A.E. Taylor and D.C. Lay,
%%    %% {\em Introduction to Functional Analysis}\/, 2nd ed.\ 
%%    %% (Wiley, New York--Chichester--Brisbane, 1980),
%%    %% Sections~III.9 (pp.~169--172) and IV.1 (p.~190).

\bibitem{Weidmann_80}  J. Weidmann, {\em Linear Operators in Hilbert Spaces}\/,
   Springer-Verlag, New York, 1980.

%% \bibitem{Yosida_80}  K. Yosida, {\em Functional Analysis}\/, 6th ed.,
%%    Springer-Verlag, Berlin--New York, 1980, reprinted 1995.
%%    %% Sections~II.1 (pp.~68--69) and II.4 (pp.~73--75).

\end{thebibliography}


%% \bigskip
%% 
%% \noindent\textit{%
%% Department of Physics, New York University, New York, NY 10003 \\
%% and \\
%% \hbox{Department of Mathematics, University College London, London WC1E 6BT, 
%%    England} \\
%% sokal@nyu.edu
%% }


\end{document}







