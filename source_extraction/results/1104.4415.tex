%notes on conjecture that r(G) is bounded above by the size of a maximal laman subgraph of G, 25 march to 2 April 2011
\documentclass[11pt]{article}

\usepackage{amsmath,amsfonts,amssymb,graphicx,a4,bm}
\usepackage{amsfonts}
\usepackage{graphicx}
\usepackage{color}
\usepackage{color}

\begin{document}

%EXTRACTING
 \usepackage{./extract}
%ENDEXTRACTING
\newtheorem{theorem}{Theorem}  [section]
\newtheorem{proposition}[theorem]{Proposition}
\newtheorem{lemma}[theorem]{Lemma}
\newtheorem{claim}[theorem]{Claim}
\newtheorem{corollary}[theorem]{Corollary}
\newtheorem{fact}[theorem]{Fact}
\newtheorem{defn}[theorem]{Definition}
\newtheorem{conj}[theorem]{Conjecture}
%%\newenvironment{proof}{\noindent{\bf Proof}:\,\,} {\qed}

\newcommand{\bproof}{\noindent{\bf Proof: }}
\newcommand{\eproof}{\hfill $\bullet$\\}
\newcommand{\de}{\mbox{def}}
\newcommand{\cee}{{\mathbb C}}
\newcommand{\real}{{\mathbb R}}
\newcommand{\rat}{{\mathbb Q}}
\newcommand{\zed}{{\mathbb Z}}
\newcommand{\calx}{{\cal X}}
\newcommand{\tran}{{{td}}}
\newcommand{\rank}{{\mbox{rank }}}
\newcommand{\sm}{\setminus}

\title{A necessary condition for generic rigidity of bar-and-joint frameworks in $d$-space}



\author{Bill Jackson\thanks{School of Mathematical Sciences,
 Queen Mary University of London,
Mile End Road, London E1 4NS, England. e-mail:
b.jackson@qmul.ac.uk. }
}

\date{May 15, 2011}

\maketitle

\begin{abstract}
A graph $G=(V,E)$ is {$d$-sparse} if each subset $X\subseteq V$ with
$|X|\geq d$ induces at most $d|X|-{{d+1}\choose{2}}$ edges in $G$.
Laman showed in 1970 that a necessary and sufficient condition for a
realisation of $G$ as a generic bar-and-joint framework in $\real^2$
to be rigid is that $G$ should have a 2-sparse subgraph with
$2|V|-3$ edges. Although Laman's theorem does not hold when $d\geq
3$, Cheng and Sitharam recently showed that if $G$ is generically
rigid in $\real^3$ then every maximal $3$-sparse subgraph of $G$
must have $3|V|-6$ edges. We extend their result to all $d\leq 5$ by
showing that if $G$ is generically rigid in $\real^d$ then every
maximal $d$-sparse subgraph of $G$ must have
$d|V|-{{d+1}\choose{2}}$ edges.
\end{abstract}

\section{Introduction}
A {\em $d$-dimensional (bar-and-joint) framework} is a pair $(G,p)$
where $G=(V,E)$ is a graph and $p:V\to \real^d$. It is a long
standing open problem to determine when a given bar-and-joint
framework is {\em rigid} i.e. every continuous motion of the points
$p(v)$ which preserves the distances $\|p(u)-p(v)\|$ for all $uv\in
E$ must also preserve the distances $\|p(u)-p(v)\|$ for all $u,v\in
V$. It is not difficult to see that a 1-dimensional framework
$(G,p)$ is rigid if and only if the graph $G$ is connected. Abbot
\cite{A} has recently shown that the problem of determining rigidity
is NP-hard for all $d\geq 2$.

This problem becomes more tractable, however, if we assume that the
framework is {\em generic} i.e. there  are no algebraic dependencies
between the coordinates of the points $p(v)$, $v\in V$. It is known
that the rigidity of a $d$-dimensional generic framework $(G,p)$
depends only on the graph $G$. Indeed we can define an $|E|\times
d|V|$ matrix, the {\em $d$-dimensional rigidity matrix $R_d(G)$},
whose entries are linear combinations of indeterminates representing
the coordinates of the points $p(v)$,  in such a way that $(G,p)$ is
rigid if and only if the rank of $R_d(G)$, $r_d(G)$, is equal to
$d|V|-{{d+1}\choose{2}}$. This naturally gives rise to a matroid on
$E$, the {\em $d$-dimensional rigidity matroid ${\cal R}_d(G)$} in
which a set of edges $F\subseteq E$ is {\em independent/dependent}
if and only if the corresponding rows of $R_d(G)$ are linearly
independent/dependent. We refer the reader to \cite{W} for a precise
definition of the rigidity matrix, the rigidity matroid, and other
information on the topic of rigidity.


Laman \cite{L} characterized when a 2-dimensional generic framework
is rigid (see also Lov\'asz and Yemini \cite{LY}). His
characterization is based on the following concept. We say that a
subgraph $H$ of $G$ is {\em $d$-sparse} if each subset $X$ of at
least $d$ vertices of $H$induces at most $d|W|-{{d+1}\choose{2}}$
edges of $H$. Maxwell \cite{M} showed that being $d$-sparse is a
necessary condition for the rows of $R_d(G)$ labeled by the edges of
$H$ to be linearly independent. Laman showed that that this
condition is also sufficient when $d=2$ and deduced that a
2-dimensional generic framework $(G,p)$ is rigid if and only if it
has a 2-sparse subgraph with $2|V|-3$ edges. Since every linearly
independent set of rows of $R_2(G)$ can be extended to a basis for
the row space of $R_2(G)$, Laman's theorem implies that every
maximal 2-sparse subgraph of $G$ has the same number of edges.

It is known that the condition that $H$ is $d$-sparse is not
sufficient for the rows of $R_d(G)$ labeled by the edges of $H$ to
be linearly independent when $d\geq 3$. Indeed it is not even true
that all maximal $d$-sparse subgraphs of $G$ have the same number of
edges when $d\geq 3$. On the other hand Cheng and Sitharam \cite{CS}
have recently shown that the number of edges in any maximal
$3$-sparse subgraph of $G$ does at least give an upper bound on
$r_3(G)$.

The purpose of this paper is to prove a result, Theorem
\ref{upperbound} below, which extends the theorem of Cheng and
Sitharam to all values of $d\leq 5$.


\section{Sparse subgraphs}
Let $G=(V,E)$ be a graph and $d\geq 1$ be an integer. For
$X\subseteq V$ we use $E_G(X)$ to denote the set, and $i_G(X)$ the
number, of edges of $G$ joining pairs of vertices of $X$. We
simplify these to $E(X)$ and $i(X)$ when it is obvious to which
graph we are referring. We may rewrite the condition for $G$ to be
{$d$-sparse} as $i(X)\leq d|X|-{{d+1}\choose{2}}$ for all
$X\subseteq V$ with $|X|\geq d$. (Note that if $|X|\in \{d,d+1\}$
then we have $i(X)\leq {{|X|}\choose{2}}=d|X|-{{d+1}\choose{2}}$ and
the inequality holds trivially.) A subgraph $H=(U,F)$ of a
$d$-sparse graph $G$ is {\em $d$-critical} if either $|U|=2$ and
$|F|=1$, or $|U|\geq d+2$ and $|F|=d|X|-{{d+1}\choose{2}}$. The
assumption that $G$ is $d$-sparse implies that every $d$-critical
subgraph of $G$ is an induced subgraph. A {\em $d$-critical
component} of $G$ is a $d$-critical subgraph which is not properly
contained in any other $d$-critical subgraph of $G$.

\begin{lemma}\label{int}
Let $G=(V,E)$ be a $d$-sparse graph and
$H_1=(U_1,F_1),H_2=(U_2,F_2)$ be distinct critical components of
$G$. Then $|U_1\cap U_2|\leq d-1$ and, if equality holds, then
$i_G(U_1\cap U_2)={{d-1}\choose{2}}$.
\end{lemma}
\bproof Suppose that $|U_1\cap U_2|\geq d-1$. When $|U_1\cap
U_2|\geq d$ we have $i(U_1\cap U_2)\leq d|U_1\cap
U_2|-{{d+1}\choose{2}}$ since $G$ is sparse. When $|U_1\cap
U_2|=d-1$, we have $i(U_1\cap U_2)\leq {{d-1}\choose{2}}=d|U_1\cap
U_2|-{{d+1}\choose{2}}+1$ trivially. The maximality of $H_1,H_2$ and
the definition of a critical component imply that $|U_1|,|U_2|\geq
d+2$, and $d(|U_1|+|U_2|)-2{{d+1}\choose{2}}= i_G(U_1)+i_G(U_2)\leq
i_G(U_1\cup U_2)+i_G(U_1\cap U_2) \leq d|U_1\cup
U_2|-{{d+1}\choose{2}}-1+d|U_1\cap
U_2|-{{d+1}\choose{2}}+1=d(|U_1|+|U_2|)-2{{d+1}\choose{2}}.$
%\begin{eqnarray*}
%3|U_1|-6+3|U_2|-6= i(U_1)+i(U_2)&\leq& i(U_1\cup U_2)+i(U_1\cap U_2)\\
%&\leq& 3|U_1\cap U_2|-7+3|U_1\cap U_2|-5.
%\end{eqnarray*}
Equality must hold throughout. In particular we have  $i_G(U_1\cap
U_2)={{d+1}\choose{2}}+1$. This implies that $|U_1\cap U_2|= d-1$
and $i_G(U_1\cap U_2)={{d-1}\choose{2}}$. \eproof

\section{Covers}
Let $k,t$ be nonnegative integers, $G=(V,E)$ be a graph and $\cal X$
be a family of subsets of $V$. We say that $\cal X$ is a {\em cover}
of $G$ if every set in $\calx$ contains at least two vertices, and
every edge of $G$ is induced by at least one set in $\cal X$. A
cover $\cal X$ is {\em $t$-thin} if every pair of sets in $\calx$
intersect in at most $t$ vertices. A {\em $k$-hinge} of $\cal X$ is
set of $k$ vertices which lie in the intersection of at least two
sets in $\cal X$. A $k$-hinge $U$ of $\calx$ is {\em closed in $G$}
if $G[U]$ is a complete graph. We use $\Theta_k(\calx)$
%,respectively $\widehat\Theta_k(\calx)$,
to denote the set of all
$k$-hinges, respectively closed $k$-hinges, of $\cal X$. For $U\in
\Theta_k(\calx)$, let $d_\calx(U)$ denote the number of sets in
$\cal X$ which contain $U$. Note that if $G$ is $t$-thin then
$\Theta_k(\calx)=\emptyset$ for all $k\geq t+1$. Note also that
$\Theta_0(\calx)=\{\emptyset\}$ and $d_\calx(\emptyset)=|\calx|$.

\iffalse
\begin{lemma}\label{covercount}
Let $\cal X$ be a cover of a graph $G=(V,E)$. Then
\begin{equation}\label{cover1}
 |V|=\sum_{X\in \calx}|X|-\sum_{U\in \Theta_1(\calx)}(d_\calx(U)-1)
\end{equation}
and
\begin{equation}\label{cover2}
 |E|=\sum_{X\in \calx}i_G(X)-\sum_{U\in
\widehat\Theta_2(\calx)}(d_\calx(U)-1).
\end{equation}
\end{lemma}
\bproof We first prove (\ref{cover1}). Choose $v\in V$. If $U=\{v\}$
is not a 1-hinge of $\cal X$ then the sum $\sum_{X\in \calx}|X|$
counts $v$ exactly once. If $U$ is a 1-hinge of $\cal X$ then this
sum counts $v$ exactly $d_\calx(U)$ times. Equation (\ref{cover2})
can be proved similarly.
%Similarly if $e=uv\in E$. If $U=\{u,v\}$ is
%not a closed 2-hinge of $\cal X$ then the sum $\sum_{X\in
%\calx}i_G(X)$ counts $e$ exactly once. If $U$ is a closed 2-hinge of
%$\cal X$ then the sum $\sum_{X\in \calx}i_G(X)$ counts $e$ exactly
%$d_\calx(U)$ times.
\eproof \fi




\begin{lemma}\label{cover}
Let $G=(V,E)$ be a graph, $H=(V,F)$ be a maximal $d$-sparse subgraph
of $G$, and  $H_1,H_2,\ldots,H_m$ be the  $d$-critical components of
$H$. Let $X_i$ be the vertex set of $H_i$ for  $1\leq i \leq m$.
Then ${\cal X}=\{X_1, X_2,\ldots, X_m\}$ is a $(d-1)$-thin cover of
$G$ and each $(d-1)$-hinge of $\calx$ is closed.
\end{lemma}
\bproof The definition of a $d$-critical subgraph implies that each
$H_i$ has at least two vertices and that every edge of $H$ belongs
to at least one $H_i$. Thus $\calx$ is a cover of $H$. To see that
$\calx$ also covers $G$ we choose $e=uv\in E\sm F$. The maximality
of $H$ implies that $H+e$ is not $d$-sparse. Hence $\{u,v\}$ is
contained in some $d$-critical subgraph of $H$. Thus $\calx$ also
covers $G$. The facts that $\cal X$ is $(d-1)$-thin and that each
$(d-1)$-hinge of $\calx$ is closed follow from Lemma \ref{int}
\eproof

We refer to the closed $(d-1)$-thin cover of $G$ described in Lemma
\ref{cover} as the {\em $H$-critical cover} of $G$. When $G$ is
$d$-sparse (and so $H=G$), we refer to this cover as the  {\em
$d$-critical cover} of $G$. Note that the definition of a
$d$-critical set implies that each set in a $d$-critical cover has
size two or has size at least $d+2$.



\begin{lemma}\label{prefixedhinge}
Let $H=(V,E)$ be a $d$-sparse graph, $\cal X$ be its $d$-critical
cover and $W\in \Theta_k(\calx)$ for some $0\leq k\leq d-1$. Suppose
that each critical component of $H$ which contains $W$ has at least
$d+2$ vertices. Then
$$
(d-k)\sum_{\substack{U\in\Theta_{k+1}(\calx)\\[0.7mm]W\subset U}}(d_\calx(U)-1)-
\sum_{\substack{U\in\Theta_{k+2}(\calx)\\[0.7mm]W\subset U}}(d_\calx(U)-1)<{{d+1-k}\choose{2}}(d_\calx(W)-1)
\,.$$
\end{lemma}
\bproof Let $d_\calx(W)=t$ and let $H_1,H_2,\ldots,H_t$ be the
critical components of $H$ which contain $W$. Put $H_i=(V_i,E_i)$
for $1\leq i\leq t$. Let $H'=\bigcup_{i=1}^tH_i$ and put
$H'=(V',E')$. Then
\begin{equation}\label{fixedhingeeq1}
 |V'|=\sum_{i=1}^t|V_i|-k(t-1)-\sum_{\substack{U\in\Theta_{k+1}(\calx)\\[0.7mm]W\subset
 U}}(d_\calx(U)-1)
\end{equation}
since, for $v\in V'$, if $v\in W$ then $v$ is counted $t$ times in
the sum $\sum_{i=1}^t|V_i|$, if $v\in U\setminus W$ for some
$U\in\Theta_{k+1}$ with $W\subset U$ then $v$ is counted
$d_\calx(U)$ times in this sum, and all other vertices of $V'$ are
counted exactly once in this sum. Similarly,
\begin{equation}\label{fixedhingeeq2}
 |E'|\geq\sum_{i=1}^t|E_i|-{{k}\choose{2}}(t-1)-k\sum_{\substack{U\in\Theta_{k+1}(\calx)\\[0.7mm]W\subset
 U}}(d_\calx(U)-1)-\sum_{\substack{U\in\Theta_{k+2}(\calx)\\[0.7mm]W\subset
 U}}(d_\calx(U)-1)
\end{equation}
since, for $e=uv\in E'$, if $u,v\in W$ then $e$ is counted $t$ times
in the sum $\sum_{i=1}^t|E_i|$ and there are at most
${{k}\choose{2}}$ such edges, if $u\in W$ and $v\in U\setminus W$
for some $U\in\Theta_{k+1}$ with $W\subset U$ then $e$ is counted
$d_\calx(U)$ times in this sum and for each such $v$ there are at
most $k$ choices for $u$, if $u,v\in U\setminus W$ for some
$U\in\Theta_{k+2}$ with $W\subset U$ then $e$ is counted
$d_\calx(U)$ times in this sum, and all other edges of $E'$ are
counted exactly once in this sum.

Since $H'\subseteq H$, $H'$ is sparse and since $W\in \Theta_k$ we
have $t\geq 2$ so $H'$ is not critical. Hence $|E'|<
d|V'|-{{d+1}\choose{2}}$. We may substitute equations
(\ref{fixedhingeeq1}) and (\ref{fixedhingeeq2}) into this inequality
and use the fact that $|E_i|=d|V_i|-{{d+1}\choose{2}}$ for all
$1\leq i\leq t$ to obtain
\begin{eqnarray*}
(d-k)\sum_{\substack{U\in\Theta_{k+1}(\calx)\\[0.7mm]W\subset
U}}(d_\calx(U)-1)&-&
\sum_{\substack{U\in\Theta_{k+2}(\calx)\\[0.7mm]W\subset
U}}(d_\calx(U)-1)\\
&<&\left[{{d+1}\choose{2}}+{{k}\choose{2}}-dk\right](t-1)\\
&=& {{d+1-k}\choose{2}}(t-1).
\end{eqnarray*}
\eproof

\begin{lemma}\label{fixedhinge}
Let $H=(V,E)$ be a $d$-sparse graph and $\cal X$ be its $d$-critical
cover. Suppose that each
%$X\in \calx$ has $|X|\geq d+2$
critical component of $H$ has at least $d+2$ vertices. Put
$a_k=\sum_{U\in\Theta_k(\calx)}(d_\calx(U)-1)$ for  $0\leq k\leq d$.
Then
for all $0\leq k\leq d-2$ we have:\\[2mm]
(a) $(d-k)(k+1)a_{k+1}-{{k+2}\choose{2}}a_{k+2}
<{{d+1-k}\choose{2}}a_{k}$;
\\[2mm]
%(b) \mbox{$(d-k)a_{k+1}- (k+2)a_{k+2}
%<{{d}\choose{k+1}}\left[(d+1)(|\calx|-1)- a_{1}\right]$}\,;
%\\[2mm]
%(c) \mbox{$(d-k)a_{k+1}
%<(d-k-1){{d}\choose{k+1}}\left[(d+1)(|\calx|-1)- a_{1}\right]$}\,;
%\\[2mm]
(b) $(d-k)a_{k+1}- (k+1)a_{k+2} <{{d+1}\choose{k+2}}(|\calx|-1)$;
\\[2mm]
(c) $d(d-k)a_{k+1} <(k+2)(d-k-1){{d+1}\choose{k+2}}(|\calx|-1)$.
\end{lemma}
\bproof Part (a) follows by summing the inequality in Lemma
\ref{prefixedhinge} over all $W\in \Theta_k$, and using the facts
that
$$\sum_{W\in \Theta_k(\calx)} \;\sum_{\substack{U\in\Theta_{k+1}(\calx)\\[0.7mm]W\subset
U}}(d_\calx(U)-1)=(k+1)\sum_{U\in\Theta_{k+1}(\calx)}(d_\calx(U)-1)=(k+1)a_{k+1}$$
and
$$\sum_{W\in \Theta_k(\calx)} \;\sum_{\substack{U\in\Theta_{k+2}(\calx)\\[0.7mm]W\subset
U}}(d_\calx(U)-1)={{k+2}\choose{2}}\sum_{U\in\Theta_{k+2}(\calx)}(d_\calx(U)-1)={{k+2}\choose{2}}a_{k+2}\,.$$

\iffalse

To prove (b) we put  $b_i=(d+1-i)a_i-(i+1)a_{i+1}$  for $0\leq i\leq
d$. Then the identity in (a) can be rewritten as
$b_{k+1}<\frac{d-k}{k+1}\,b_k$ for all $0\leq k\leq d-1$. A simple
induction on $k$ gives $b_{k+1}<{{d}\choose{k+1}}\,b_0$. We may now
replace $b_{k+1}$ and $b_0$ by $(d-k)a_{k+1}-(k+2)a_{k+2}$ and
$(d+1)a_{0}-a_{1}$, respectively, and use the fact that
$a_0=|\calx|$ to obtain (b).

We prove (c) by induction on $d-k$. When $d-k=2$, (c) follows by
putting $k=d-2$ in (b) and using the fact that $\Theta_d=\emptyset$.
Hence suppose that $d-k\geq 3$. Then (b) gives


$$(d-k)a_{k+1}<\mbox{${{d}\choose{k+1}}$}\,b_0+(k+2)a_{k+2}\,.
$$
We may now apply induction to $a_{k+2}$ to obtain
\begin{eqnarray*}
(d-k)a_{k+1} &<& \mbox{${{d}\choose{k+1}}$}\,b_0+
(k+2)\,\mbox{$\frac{d-k-2}{d-k-1}\,{{d}\choose{k+2}}$}\,b_0\\[1mm]
&=& (d-k-1)\,\mbox{${{d}\choose{k+1}}$}\,b_0.
\end{eqnarray*}

\fi


We prove (b) by induction on $k$. When $k=0$, (b) follows by putting
$k=0$ in (a). Hence suppose that $k\geq 1$. Then (a) gives
\begin{equation}\label{fixedhingeeq3}
2(d-k)a_{k+1}-2(k+1)a_{k+2}<\frac{(d-k+1)(d-k)}{k+1}\,a_k-ka_{k+2}\,.
\end{equation}
We may also use (a) to obtain
\begin{equation}\label{fixedhingeeq4}
ka_{k+2}>\frac{k(d-k)}{k+2}\left(2a_{k+1}-\frac{d-k+1}{k+1}a_{k}\right)\,.
\end{equation}
Substituting (\ref{fixedhingeeq4}) into (\ref{fixedhingeeq3}) and
using induction we obtain
\begin{eqnarray*}
(d-k)a_{k+1}-(k+1)a_{k+2}&<&\mbox{$\frac{d-k}{k+2}\,[(d-k+1)a_k-ka_{k+1}]$}\\
&<&\mbox{$\frac{d-k}{k+2}\,{{d+1}\choose{k+1}}\,(|\calx|-1)$}\\
&=&\mbox{${{d+1}\choose{k+2}}\,(|\calx|-1)$}\,.
\end{eqnarray*}

We prove (c) by induction on $d-k$. When $d-k=2$, (c) follows by
putting $k=d-2$ in (b)  and using the fact that $\Theta_d=\emptyset$
by Lemma \ref{cover}. Hence suppose that $d-k\geq 3$. Then (b) gives
%\begin{equation}\label{fixedhingeeq5}
$$
d(d-k)a_{k+1}<\mbox{$d{{d+1}\choose{k+2}}\,(|\calx|-1)+d(k+1)a_{k+2}$}\,.
$$
%\end{equation}
We may now apply induction to $a_{k+2}$ to obtain
\begin{eqnarray*}
d(d-k)a_{k+1}&<&\mbox{$[d{{d+1}\choose{k+2}}+\frac{(k+1)(k+3)(d-k-2)}{d-k-1}{{d+1}\choose{k+3}}]
\,(|\calx|-1)$}\\[1mm]
&=&\mbox{$(k+2)(d-k-1){{d+1}\choose{k+2}} \,(|\calx|-1)$}\,.
\end{eqnarray*}
\eproof


\begin{theorem}\label{boundedhinges}
Let $H=(V,E)$ be a $d$-sparse graph and $\cal X$ be its $d$-critical
cover. For each critical component $H_i$ of $H$ let $\theta_k(H_i)$
be the number of $k$-hinges of $\calx$ contained in $H_i$.
Then:\\[1mm]
(a) $\theta_{1}(H_1)\leq 2d-1$ for some critical component $H_1$ of
$H$;
\\[1mm]
(b) $\theta_{2}(H_2)\leq (d-2)(d+1)-1$ for some critical component
$H_2$ of $H$;
\\[1mm]
%(c) $d(d-2)\theta_{1}(H_3)+2\theta_{2}(H_3)\leq 2d(d-2)(d+1)-1$ for
%some critical component $H_3$ of $H$;
%\\[1mm]
(c) $\theta_{d-1}(H_3)\leq d$ for some critical component $H_3$ of
$H$.
\end{theorem}
\bproof The theorem is trivially true if some critical component of
$H$ has only two vertices. Hence we may suppose that every critical
component of $H$ has at least $d+2$ vertices.

We first prove (a). Putting $k=0$ in Lemma \ref{fixedhinge}(c) we
obtain
\begin{equation}\label{boundedhingeseq1}
d\sum_{U\in\Theta_{1}(\calx)}(d_\calx(U)-1)
<(d-1)(d+1)(|\calx|-1)\,.
\end{equation}
Since $d_\calx(U)\geq 2$ for all $U\in\Theta_{1}(\calx)$ we have
$d_\calx(U)-1\geq d_\calx(U)/2$ and hence (\ref{boundedhingeseq1})
gives
$$\sum_{U\in\Theta_{1}(\calx)}d_\calx(U)
<2d\,|\calx|\,.
$$
This tells us that the average number of $1$-hinges in a critical
component is strictly less that $2d$.


We next prove (b). Putting $k=1$ in Lemma \ref{fixedhinge}(c) we
obtain
\begin{equation}\label{boundedhingeseq1.5}
\sum_{U\in\Theta_{2}(\calx)}(d_\calx(U)-1)
<(d-2)(d+1)(|\calx|-1)/2\,.
\end{equation}
We can now proceed as in (a).

\iffalse

We next prove (c). Putting $k=1$ in Lemma \ref{fixedhinge}(d) we
obtain
\begin{equation}\label{boundedhingeseq2}
d(d-2)\sum_{U\in\Theta_{1}(\calx)}(d_\calx(U)-1)+2\sum_{U\in\Theta_{2}(\calx)}(d_\calx(U)-1)
<d(d-2)(d+1)(|\calx|-1)\,.
\end{equation}
Since $d_\calx(U)\geq 2$  we have $d_\calx(U)-1\geq d_\calx(U)/2$
for all $U\in\Theta_{1}(\calx)\cup \Theta_{2}(\calx)$ and hence
(\ref{boundedhingeseq2}) gives
$$d(d-2)\sum_{U\in\Theta_{1}(\calx)}d_\calx(U)+2\sum_{U\in\Theta_{2}(\calx)}d_\calx(U)
<2d(d-2)(d+1)|\calx|\,.
$$
This tells us that the average value of
$d(d-2)\theta_{1}(H_i)+2\theta_{2}(H_i)$ over all critical
components $H_i$ of $H$ is strictly less that $2d(d-2)(d+1)$.

\fi

Finally we prove (c). Putting $k=d-2$ in Lemma \ref{fixedhinge}(c)
gives
\begin{equation}\label{boundedhingeseq3}
2\sum_{U\in\Theta_{d-1}(\calx)}(d_\calx(U)-1) <(d+1)(|\calx|-1)\,.
\end{equation}
We can now proceed as in (a).

\section{An upper bound on the rank}
\begin{theorem}\label{upperbound}
Let $G=(V,E)$ be a graph, $d\leq 5$ be an integer and $H=(V,F)$ be a
maximal $d$-sparse subgraph of $G$.  Then $r_d(G)\leq |F|$.
\end{theorem}
\bproof We proceed by contradiction. Suppose the theorem is false
and choose a counterexample $(G,H)$ such that $|E|$ is as small as
possible. Let $H_1,H_2,\ldots,H_m$ be the $d$-critical components of
$H$ where $H_i=(V_i,E_i)$ for $1\leq i\leq m$. Then
$\calx=\{V_1,V_2,\ldots,V_m\}$ is the $H$-critical cover of $G$. For
all $1\leq i \leq m$ let $E_i^*$ be the set of all edges $uv\in E_i$
such that $\{u,v\}$ is a 2-hinge of $\calx$.
%H_i^*=(V_i^*,E_i^*)$ be
%the subgraph of $H_i$ induced by the hinges of $\calx$. Note that
%each hinge of $\calx$ induces an edge of $H$ by Lemma \ref{int}.

\begin{claim}\label{c1}
For all $1\leq i \leq m$ either $E_i^*=E_i$ or $E_i^*$ is a
dependent set of edges in the $d$-dimensional rigidity matroid
${\cal R}_d(G)$.
\end{claim}
\bproof We proceed by contradiction. Suppose that $E_i^*\neq E_i$
and $E_i^*$ is an independent set of edges in ${\cal R}(G)$ for some
$1\leq i\leq m$. Let $G'=G-(E_G(V_i)\sm E_i^*)$, $H'=H-(E_i\sm
E_i^*)$ and $F'=F\sm(E_i\sm E_i^*)$. Then $H'$ is a maximal sparse
subgraph of $G'$ ($H'$ is sparse since $H'\subseteq H$, and $H'$ is
maximal since for each edge $e=uv$ of $G'-F'$ we have
$\{u,v\}\subseteq V_j$ for some $1\leq j\leq m$ with $j\neq i$ so
$H_j+e\subseteq G'+e$ is not sparse). By the minimality of the
counterexample $(G,H)$,
\begin{equation}\label{c1eq1}
r_d(G')\leq |F'|=|F|-|E_i|+ |E_i^*|.
\end{equation}
Choose a base $B'$ for ${\cal R}_d(G')$ which contains $E_i^*$. We
may extend $B'$ to a base $B$ for ${\cal R}_d(G)$. Then
$E_i^*\subseteq B$. Since $B'$ spans $E(G')$ and since $B$ can
contain at most $|E_i|$ edges between the vertices of $V_i$ we have
\begin{equation}\label{c1eq2}
r_d(G)=|B|\leq |B'|+|E_i|-|E_i^*|=r(G')+|E_i|- |E_i^*|.
\end{equation}
We may now combine (\ref{c1eq1}) and (\ref{c1eq2}) to obtain
$$r_d(G)\leq |F'|+|E_i|- |E_i^*|= |F|.$$
This contradicts the choice of $(G,H)$ as a counterexample to the
theorem. \eproof

\begin{claim}\label{c2}
$|V_i|\geq d+2$ and $|E_i^*|\geq {{d+2}\choose{2}}-1$ for all $1\leq
i\leq m$.
\end{claim}
\bproof If $|V_i|=2$ then $E_i^*=\emptyset$. This would contradict
Claim \ref{c1} and hence $|V_i|\geq d+2$. Suppose $E_i^*=E_i$. Then
$$|E_i^*|=|E_i|=d|V_i|-{{d+1}\choose{2}}\geq  d(d+2)-{{d+1}\choose{2}}={{d+2}\choose{2}}-1.$$
Thus we may suppose that $E_i^*\neq E_i$. By Claim \ref{c1}, $E_i^*$
is a dependent set of edges in the $d$-dimensional rigidity matroid.
The claim now follows since the smallest dependent set of edges in
this matroid has size ${{d+2}\choose{2}}$.
%sparse dependent set of edges has size 18.
\eproof

Claim \ref{c2} implies that the number of 2-hinges of $\calx$ in
each $H_i$ is at least ${{d+2}\choose{2}}-1$.  We may now apply
Theorem \ref{boundedhinges}(b) to obtain the required contradiction.
\eproof


\section{Closing remarks}
\subsection*{An improved upper bound on the rank}
Given a graph $G$, let $s_d(G)$ be the minimum number of edges in a
maximal $d$-sparse subgraph of $G$. Theorem \ref{upperbound} tells
us that $r_d(G)\leq s_d(G)$ when $d\leq 5$. It is not difficult to
construct graphs for which strict inequality holds. We use the
following operation. Given two graphs $G_1=(V_1,E_1)$ and
$G_2=(V_2,E_2)$ with $V_1\cap V_2=\{u,v\}$ and $E_1\cap E_2=\{uv\}$,
we refer to the graph $G=G_1\cup G_2$ as the {\em parallel
connection of $G_1$ and $G_2$ along the edge $uv$}.

The graph $G$ obtained by taking the parallel connection of two
copies of $K_5$ along an edge $uv$ and then deleting $uv$, is
3-sparse and is not rigid in $\real^3$. Hence
$s_3(G)=|E(G)|=18>17=r_3(G)$. On the other hand we may improve the
upper bound on $r_3(G)$ in this example by considering the graph
$G^*=G+uv$. A maximal sparse subgraph of $G^*$ which contains $uv$
has 17 edges. Thus we have $17=r_3(G)\leq r_3(G^*)\leq s_3(G^*)=17$.

More generally, for any graph $G$ we have the improved upper bound
\begin{equation}\label{improvedub}
r_d(G)\leq \min\{s_d(G^*)\;:\;G\subseteq G^*\}=:s_d^*(G)
\end{equation}
for all $d\leq 5$. The following example shows that strict
inequality can also hold in (\ref{improvedub}). Let $G$ be obtained
from $K_5$ by taking parallel connections with 10 different $K_5$'s
along each of the edges of the original $K_5$. We have $r_3(G)=89$.
On the other hand,
 $s_3(G)=90$ (obtained by taking a maximal sparse
subgraph which contains 9 of the edges of the original $K_5$).
Furthermore we have $s_3(G^*)\geq r_3(G^*)> r_3(G)$ for all graphs
$G^*$ which properly contain $G$. Thus  $s_3^*(G)=90>r_3(G)$.

\subsection*{Algorithmic considerations}
For fixed $d$, we can use network flow algorithms to test whether a
graph is $d$-sparse in polynomial time, see for example \cite{BJ}.
This means we can greedily construct a maximal $d$-sparse subgraph
$H$ of a graph $G$ in polynomial time and hence obtain an upper
bound on $r_d(G)$ via Theorem \ref{upperbound}. We do not know
whether $s_d(G)$ or $s_d^*(G)$ can be determined in polynomial time.

\iffalse
%\begin{conj}\label{rank} Let $G$ be a graph. Then $r(G)=\min\{s(G^*)\;:\;G\subseteq G^*\}$.
%\end{conj}

At a conference on rigidity held in Montreal in 1987,
Dress conjectured that $r(G)$ is determined by
the following special 2-thin cover of $G=(V,E)$.
 For $u,v\in V$, the edge
$uv$ is an {\em implied edge} of $G$ if $uv\not\in E$ and
$r(G+uv)=r(G)$. The closure $\hat{G}$ of $G$ is the graph obtained by
adding all the
implied edges to $G$. A {\em rigid cluster} of $G$ is a set of vertices which
induce a maximal complete subgraph of $\hat{G}$.
It can be seen that
any two rigid clusters of $G$ intersect in at most two vertices
(see for example \cite[Lemma 4.8]{JJ1}).
Thus the set of rigid clusters of $G$ is a 2-thin cover of $G$.

\begin{conj}\label{dress}(see \cite[Conjecture 5.6.1]{GSS}, \cite{CDT}, and
\cite[Conjecture 2.3]{Taycon})
Let $G=(V,E)$ be a graph and $\cal X$ be the set of
rigid clusters of $G$. Then
$$r(G)= \sum_{X\in {\cal X}}f(X) - \sum_{\{u,v\}\in \Theta(\calx)}(d_\calx(u,v)-1).$$
where $f(X)=1$ when $|X|=2$ and $f(X)=3|X|-6$ when  $|X|\geq 3$.
\end{conj}

Maybe Conjecture \ref{dress} would imply that equality holds in (\ref{improvedub}) if and only if the set of hinges of the rigid cluster decomposition of $G$
is independent in the 3-dimensional rigidity matroid???

\fi

\subsubsection*{Acknowledgement} I would like to thank Meera Sitharam
for helpful conversations on this topic.

\begin{thebibliography}{99}

\bibitem{A}
Abbot, T. G., {\it Generalizations of Kempe's Universality Theorem}.
MSc thesis, MIT (2008).
http://web.mit.edu/tabbott/www/papers/mthesis.pdf


\bibitem{BJ} Berg A.  and Jord\'an T.,
{\it Algorithms for graph rigidity and scene analysis}, in:
Proceedings of the 11th Annual European Symposium on Algorithms
2003, Springer Lecture Notes in Computer Science, vol. 2832, 2003,
78--89.


%\bibitem{CDT} {H. Crapo, A. Dress and T.-S. Tay},
%Problem 4.2, in Matroid Theory (J.E. Bonin, J.G. Oxley and B. Servatius eds.,
%Seattle, WA, 1995),
%{\em Contemp. Math.}, 197, Amer. Math. Soc., Providence, RI, 1996, 414.

%\bibitem{GSS}
%J. Graver, B. Servatius, and H. Servatius,
%Combinatorial Rigidity,
%{\em AMS Graduate Studies in
%Mathematics} Vol. 2, 1993.



%\bibitem{JJ1} {B. Jackson and T. Jord\'an}, The Dress Conjecture on Rank in the
%3-Dimensional Rigidity Matroid,  {\em Advances in Applied Math.}, 35,
%(2005) 355-367.

%\bibitem{JJ2} {B. Jackson and T. Jord\'an}, On the Rank Function of the
%$3$-Dimensional Rigidity Matroid, {\em International Journal of
%Computational Geometry and Applications}, 16 (2006) 415-429.

\bibitem{CS} {Cheng M. J.  and Sitharam M.}, {\it Maxwell independence: a better
rank estimate for 3D rigidity matroids}, submitted.

\bibitem{L} Laman, G.,
{\it On graphs and rigidity of plane skeletal structures}. J.
Engineering Math., {\bf 4} (1970) 331-340



\bibitem{LY} Lov\'asz, L. and  Yemini, Y.,
{\it On generic rigidity in the plane}. SIAM J. Algebraic Discrete
Methods, {\bf 3}  (1982) 91-98.


\bibitem{M} Maxwell, J. C.,
{\it On the calculation of the equilibrium and stiffness of frames}.
Philosophical Magazine 27, (1864), 294 - 299.

\bibitem{W}
Whiteley, W., {\it Some matroids from discrete applied geometry}, in
Matroid Theory, Bonin, J. E. et al., (eds.), Contemp. Math., vol
197, Amer. Math. Soc., Providence, RI (1996) 171--311.


%\bibitem{Taycon} {T-S. Tay},
%On the generic rigidity of bar frameworks,
%{\it Advances in Applied Mathematics}, 23, (1999), 14-28.


\iffalse
\bibitem{AR} {\scshape
L. Asimow and B. Roth},
\newblock The rigidity of graphs,
\newblock {\itshape Trans.
Amer. Math. Soc., Ser. B.}, 245 (1978) 279-289.

\bibitem{BJ} {\scshape
A.R. Berg and T. Jord\'an},
\newblock A proof of Connelly's conjecture
on $3$-connected circuits of the rigidity matroid,
\newblock {\itshape J.
Combinatorial Theory, Ser. B.}, Vol. 88, 77-97, 2003.

%\bibitem{BJalgo}
%{\scshape
%A.R. Berg and T. Jord\'an},
%\newblock Algorithms for graph rigidity and
%scene analysis,
%\newblock Proc. 11th Annual European Symposium on
%Algorithms (ESA) 2003, (G. Di Battista, U. Zwick, eds)
%Springer Lecture Notes in
%Computer Science 2832, pp. 78-89, 2003.

\bibitem{Cunp} {\scshape R. Connelly},
\newblock
Generic global rigidity,
\newblock {\itshape Discrete Comput. Geom.} 33 (2005), no. 4,
549--563.

\bibitem{Erenetal} {\scshape T. Eren, W. Whiteley,
A.S. Morse, P.N. Belhumeur, and B.D.O. Anderson},
\newblock Sensor and network topologies of formations
with direction, bearing and angle information between
agents,
\newblock In {\it Proc. of the 42nd IEEE Conference on Decision
and Control}, pp. 3064-3069, 2003.

%\bibitem{Eren} {\scshape T. Eren},
%\newblock Using angle of arrival (bearing) information
%for localization in robot networks,
%\newblock {\itshape Turk J Elec Engin}, Vol 15, No. 2,
%2007, pp. 169-186.

\bibitem{FJ} {\scshape M. D. Fried and M. Jarden},
\newblock {\itshape Field Arithmetic}, Results in Mathematics
and Related Areas (3), 11. Springer-Verlag, Berlin, 1986.

%\bibitem{Gabow} {\scshape H.N. Gabow},
%Algorithms for graphic polymatroids and parametric $\overline s$-sets,
%{\itshape J. Algorithms} 26 (1998), no. 1, 48--86.

\bibitem{hend} {\scshape B. Hendrickson},
\newblock Conditions for unique graph realizations,
\newblock {\itshape SIAM J. Comput.} {\bfseries 21} (1992), no. 1, 65-84.

\bibitem{Hberg} {\scshape L. Henneberg}, Die Graphische Statik der
starren Systeme, 1911.

\bibitem{JJ} {\scshape B. Jackson and T. Jord\'an},
\newblock Connected rigidity matroids and unique realizations
of graphs,
\newblock {\itshape J. Combinatorial Theory Ser B}, Vol. 94, 1-29, 2005.

\bibitem{JJS} {\scshape B. Jackson, T. Jord\'an, and Z. Szabadka},
Globally linked pairs of vertices in equivalent realizations
of graphs,
{\itshape Discrete and Computational Geometry},
Vol. 35, 493-512, 2006.

\bibitem{JJmixedA} {\scshape B. Jackson and T. Jord\'an},
\newblock Globally rigid circuits of the direction-length
rigidity matroid,
\newblock {\itshape J. Combinatorial Theory Ser B}, in press.

%\bibitem{KGW} {\scshape B. Katz, M. Gaertler, and
%D. Wagner},
%\newblock Maximum rigid components as means for
%direction-based localization in sensor networks,
%\newblock Proc. SOFSEM 2007, LNCS 4362,
%Jan van Leeuwen et al. (Eds.), pp. 330-341, 2007.
%Springer, Berlin.

\bibitem{LY} {\scshape L. Lov\'asz and Y. Yemini},
\newblock On generic rigidity in the plane,
\newblock {\itshape SIAM J. Algebraic Discrete Methods}
3 (1982), no. 1, 91--98.

\bibitem{M} {\scshape J.W. Milnor},
\textit{Topology from the differentiable viewpoint},
University Press of Virginia, Charlottesville, 1965.

\bibitem{Saxe} {\scshape J.B. Saxe},
\newblock Embeddability of weighted graphs in
$k$-space is strongly NP-hard,
\newblock Tech. Report, Computer Science
Department, Carnegie-Mellon University, Pittsburgh, PA, 1979.

\bibitem{SW} {\scshape B. Servatius and W. Whiteley},
\newblock Constraining plane configurations in CAD:
Combinatorics of directions and lengths,
\newblock {\itshape SIAM J. Discrete Math.}, 12, (1999) 136--153.

\bibitem{Streinu} {\scshape I. Streinu},
\newblock Parallel-redrawing mechanisms, pseudo-triangulations
and kinetic planar graphs,
\newblock P. Healy and N.S. Nikolov (Eds.): GD 2005,
Springer LNCS 3843, pp. 421-433, 2005.
%Springer-Verlag Berlin Heidelberg 2005.

\bibitem{TW} {\scshape T-S. Tay and W. Whiteley},
\newblock Generating isostatic frameworks,
\newblock
{\itshape Structural Topology} No. 11 (1985), 21--69.

\bibitem{Whlong} {\scshape
W. Whiteley},
\newblock Some matroids from discrete applied geometry.
\newblock Matroid theory (Seattle, WA, 1995), 171--311,
Contemp. Math., 197, Amer. Math. Soc., Providence, RI, 1996.

\bibitem{Whmol}
{\scshape W. Whiteley},
\newblock Rigidity of molecular structures: geometric and generic
analysis,
\newblock in: Rigidity theory and applications (Edited by M.F. Thorpe and
P.M. Duxbury),
Kluwer 1999, pp. 21-46.
\fi
\end{thebibliography}

\end{document}
