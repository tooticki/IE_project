
\documentclass[a4paper,11pt]{amsart}
%\documentclass[11pt]{amsart}

%\documentclass[10pt]{article}


%\usepackage{showkeys}
\usepackage{amsfonts,amsmath,mathrsfs,amssymb,amsthm,amscd,latexsym,amstext,amsxtra}
%\usepackage{showkeys}
\usepackage[usenames,dvipsnames]{color}
%\usepackage[margin=0cm,paperwidth=21cm,paperheight=29.7cm,top=2.5cm, inner=3.5cm,outer=3.5cm,bottom=3.4cm]{geometry}

\usepackage[left=2cm,right=2cm, top=3cm, bottom=3cm]{geometry}
\renewcommand\arraystretch{1.2}
\linespread{1,2}



\usepackage[all,cmtip]{xy}
\usepackage{enumerate}
\xyoption{all}
\usepackage{srcltx} %per fer cerca inversa
%\usepackage{natbib}
%\usepackage[authoryear]{natbib}
%\usepackage{showkeys} %per mostrar e label/ref/cite
%\usepackage{tikzsymbols}
\usepackage{textcomp}
\usepackage{parskip}



% \setlength{\parindent}{.4 in} \setlength{\textwidth}{6.3 in}
% \setlength{\topmargin} {-.3 in} \setlength{\evensidemargin}{0 in}
% \setlength{\oddsidemargin}{0 in} \setlength{\footskip}{.3 in}
% \setlength{\headheight}{.3 in} \setlength{\textheight}{8.9 in}

%\usepackage{pgf,tikz}
%\usetikzlibrary{arrows}
\usepackage{amsthm, amsmath, amssymb,latexsym} 
\usepackage{amsfonts,epsfig,amscd}
\usepackage[]{fontenc}
\usepackage{xy}
\usepackage{enumerate}
\usepackage[]{fontenc}
\usepackage[all]{xy}
%\usepackage{makeidx}
%\usepackage{graphicx}
%\usepackage{mathrsfs}
%\usepackage{geometry}
%\usepackage{booktabs}

%\setlength{\textheight}{240mm}
%\setlength{\textwidth}{140mm}
%\setlength{\oddsidemargin}{0mm}
%\setlength{\topmargin}{-2cm}
%\newenvironment{proof}{\textbf{Proof.\ }}{\samepage
%	\begin{flushright}$\blacksquare$\end{flushright}}

%EXTRACTING
 \usepackage{./extract}
%ENDEXTRACTING
\newtheorem{theorem}{Theorem}[section]
\newtheorem{lemma}[theorem]{Lemma}
\newtheorem{corollary}[theorem]{Corollary}
\newtheorem{proposition}[theorem]{Proposition}
\newtheorem{remark}[theorem]{Remark}
\newtheorem{definition}[theorem]{Definition}
\newtheorem{example}{Example}
\newtheorem{problem}{Problem}
\newtheorem{conj}{Conjecture}
\newtheorem{hyp}[theorem]{Hypothesis} \newtheorem{ciccio}[theorem]{}


%%%%%%%%%%%%%%%%%%%%%%%%%%%%%%%%%%%%%%%%%%%%%%%%%%%%%%%%%%%%%%%%%%%%%%%%%%%%%

\newcommand{\nc}{\newcommand}
\nc{\cH}{{\mathcal H}}
\nc{\cA}{{\mathcal A}}
\nc{\cG}{{\mathcal G}}
\nc{\cC}{{\mathcal C}}
\nc{\cD}{{\mathcal D}}
\nc{\cO}{{\mathcal O}}
\nc{\cI}{{\mathcal I}}
\nc{\cB}{{\mathcal B}}
\nc{\cY}{{\mathcal Y}}
\nc{\cK}{{\mathcal K}}
\nc{\cX}{{\mathcal X}}
\nc{\cS}{{\mathcal S}}
\nc{\cE}{{\mathcal E}}
\nc{\cF}{{\mathcal F}}
\nc{\cZ}{{\mathcal Z}}
\nc{\cQ}{{\mathcal Q}}
\nc{\cN}{{\mathcal N}}
\nc{\cP}{{\mathcal P}}
\nc{\cL}{{\mathcal L}}
\nc{\cM}{{\mathcal M}}
\nc{\cT}{{\mathcal T}}
\nc{\cW}{{\mathcal W}}
\nc{\cU}{{\mathcal U}}
\nc{\cJ}{{\mathcal J}}
\nc{\cV}{{\mathcal V}}
\nc{\bH}{{\mathbb H}}
\nc{\bA}{{\mathbb A}}
\nc{\bG}{{\mathbb G}}
\nc{\bC}{{\mathbb C}}
\nc{\bO}{{\mathbb O}}
\nc{\bI}{{\mathbb I}}
\nc{\bB}{{\mathbb B}}
\nc{\bY}{{\mathbb Y}}
\nc{\bK}{{\mathbb K}}
\nc{\bX}{{\mathbb X}}
\nc{\bS}{{\mathbb S}}
\nc{\bE}{{\mathbb E}}
\nc{\bF}{{\mathbb F}}
\nc{\bZ}{{\mathbb Z}}
\nc{\bQ}{{\mathbb Q}}
\nc{\bN}{{\mathbb N}}
\nc{\bP}{{\mathbb P}}
\nc{\bL}{{\mathbb L}}
\nc{\bM}{{\mathbb M}}
\nc{\bT}{{\mathbb T}}
\nc{\bW}{{\mathbb W}}
\nc{\bU}{{\mathbb U}}
\nc{\bD}{{\mathbb D}}
\nc{\bJ}{{\mathbb J}}
\nc{\bV}{{\mathbb V}}
\nc{\bbZ}{{\mathbb Z}}
\nc{\bR}{{\mathbb R}}
\nc{\fm}{{\mathfrak m}}

\nc{\fr}{{\rightarrow}}
\nc{\co}{{\nabla}}
\newcommand{\la}{\longrightarrow}
\nc{\cu}{{\overlineline{\nabla}}}
\nc{\gmc}{\nabla}
\nc{\mtin}[1]{\mbox{{\tiny #1}}}
\nc{\rank}[1]{r_{\mbox{{\tiny #1}}}}
%\nc{\gmc}{\nabla_{\mbox{\tiny GM}}}
\newcommand{\del}{\partial}
\newcommand{\delbar}{\overline{\del}}

\DeclareMathOperator{\de}{d}
\DeclareMathOperator{\Aut}{Aut}
\DeclareMathOperator{\Grass}{Grass}
\DeclareMathOperator{\Ext}{Ext}
\DeclareMathOperator{\Hom}{Hom}
\DeclareMathOperator{\Coker}{Coker}
\DeclareMathOperator{\Log}{log}
\DeclareMathOperator{\Ima}{Im}
\DeclareMathOperator{\Sing}{Sing}
\DeclareMathOperator{\Crit}{Crit}
\DeclareMathOperator{\pr}{pr}
\DeclareMathOperator{\ev}{ev}
%\DeclareMathOperator{\rk}{rk}
\DeclareMathOperator{\rk}{rk}
\DeclareMathOperator{\degr}{deg}
\DeclareMathOperator{\spec}{Spec}


%

\DeclareMathOperator{\coker}{coker}
\newcommand*\pp{{\rlap{\('\)}}}
%%%%%%%%%%%%%%%%%%%%%%%%%%%%%%%%%%%%%%%%%%%%%%%%%%%%%%%%%%%%%%%%%%%%%%%%%%%%%
%\usepackage{showkeys}
%\documentclass [11pt]{article}
%\usepackage{bbm}

%%\newcommand{\opp}{\"{o}}

%%%%%%%%%%%%%%%%%%%%%%%%%%%%%%%%%%%%%%%%%%%%%%%%%%%%%%%%%%%%%%%%%%%%%%%%%%%%%


%\title{}
%\author{  Gian Pietro Pirola}
%\address{Dipartimento di Matematica, Universit\`a degli Studi di Pavia, via Ferrata 1, 27100 Pavia - Italy}



\title{Massey Products and Fujita decompositions } 																											
\makeatletter
\renewcommand\theequation{\thesection.\arabic{equation}}
%\renewcommand{\thefootnote}{}
\@addtoreset{equation}{section}
\makeatother
\author[P. Pirola]{Gian Pietro Pirola}
\address{Dipartimento di Matematica,
	Universit\`a di Pavia,
	Via Ferrata, 1,
	27100 Pavia, Italy}
\email{gianpietro.pirola@unipv.it }

\author[S. Torelli]{Sara Torelli}
%\author{Sara Torelli}
\address{Dipartimento di Matematica,
	Universit\`a di Pavia,
	Via Ferrata, 1,
	27100 Pavia, Italy}
\email{sara.torelli7@gmail.com }

\thanks{The authors were supported by PRIN 2015   Moduli spaces and Lie Theory, INdAM - GNSAGA and FAR 2016 (PV)  Variet\`a algebriche, calcolo algebrico,
	grafi orientati e topologici.}

%\MSC{}

\keywords{Massey products, Fujita decomposition, Fibrations of curves, Local system}
 
\subjclass[2010]{14D06, 14C30, 32G20} %%www.ams.org/msc

\date{\today}




\begin{document}	
	
	\maketitle
	
	\bigskip
	\begin{abstract} Let $f:S\to B$ be a fibred surface and $f_\ast\omega_{S/B}=\cU\oplus \cA$ be the second Fujita decomposition of $f.$ We study  a Massey product related with variation of the Hodge structure over flat sections of $\cU.$ We prove that the vanishing of the Massey product implies that the monodromy of $\cU$ is finite and  described by morphisms over a fixed curve.  The main tools are a lifting lemma of flat sections of $\cU$ to closed holomorphic forms of $S$ and two classical results due (essentially) to de Franchis.
 As applications we find a new proof of a theorem of Luo and Zuo for hyperelliptic fibrations. We  also analyze, as for the surfaces constructed by Catanese and Dettweiler, the case when $\cU$ has not finite monodromy.		
		\end{abstract}
		
	\section{Introduction}
	
%%%%%%% Statement not-true if we ask Massey triviality only on a fibre %%%%%%

%	\begin{theorem}\label{Thm-Main}Let $f:S\to B$ be a \textcolor{red}{(semistable)} fibration of genus $g\geq 2$ and $M:=\bM\otimes\cO_B\subset U$ be a unitary flat subbundle. If $M$ is \textcolor{red}{{\bf NO! serve un intorno di vector spaces } generated by a maximal Massey-trivial vector subspace $W_b\subset \bM_b$ (i.e. $\bM_b=G W_b$ and $\bigwedge^2W_b=0$),} then the monodromy of $\bM$ is finite.
%	\end{theorem}
	
%%%%%% Statement versione fibrati %%%%%%%
	
%	\begin{theorem}\label{Thm-MainP1}Let $f:S\to B$ be a \textcolor{red}{(semistable)} fibration of genus $g\geq 2$ and $M:=\bM\otimes\cO_B\subset U$ be a unitary flat subbundle. If there is $A\subset B^\circ$ open subset such that $\bM_{|A}$ is generated by a maximal Massey-trivial sub ??? $W_{|A}\subset \bM_{|A}$ (i.e. $\bM_b=G W_b$ for a fixed $b\in A$ and $\bigwedge^2W_b=0$ for any $b\in A$), then the monodromy of $\bM$ is finite.
%	\end{theorem}

We study fibrations $f:S\to B$ over smooth complex curves $B$ of smooth complex surfaces $S$ with general fibre a smooth projective curve $F$ of genus $g(F)\geq 2.$ Information on the geometry of fibred surfaces can be obtained relating the direct image sheaf $f_\ast\omega_{S/B}$ of the relative dualizing sheaf $\omega_{S/B},$ which is a vector bundle of rank $g(F)$ (see for instance  \cite{Fuj78a}), the first order deformations and the geometric variation of the Hodge structure defined by the smooth fibres  of $f.$ In  \cite{Fuj78a}, Fujita proved that $f_\ast\omega_{S/B}$ is nef (numerically effective) and admits a decomposition, the {\em first Fujita decomposition}, as the direct sum $f_\ast\omega_{S/B}=\cO^{\oplus h}\oplus \cE$ of the trivial bundle of rank $h=h^1(f_\ast\omega_S)$ (which is just the relative irregularity $q_f$ of $f$ on fibred surfaces) and a locally free sheaf $\cE$ such that $h^1(\cE(\omega_B))=0.$ Later in the paper \cite{Fuj78b}, Fujita announced the existence of a splitting on $f_\ast\omega_{S/B},$ the {\em second Fujita decomposition}, as the direct sum $f_\ast\omega_{S/B}=\cU\oplus \cA$ of a unitary flat bundle $\cU$ and an ample bundle $\cA,$ which allows to move the study of the semiampleness from $f_\ast\omega_{S/B}$ to $\cU.$ A key point lies in the fact that the geometry of a unitary flat bundle is completely determined by its monodromy group. The missing details of the proof appeared later, first in the paper \cite{CatDet_TheDirectImage_2014} (see also \cite{CD:Answer_2017} and \cite{CatDet_Vector_2016}). In these Catanese and Dettweiler were interested in the study of semiampleness of $f_\ast\omega_{S/B},$ suggested by a question posed by Fujita himself. They faced it using a criterion for semiapleness on unitary flat bundles (see \cite[Theorem 2.5]{CD:Answer_2017}), which states that semiapleness is equivalent to the finiteness of the monodromy. Then they provided counterexamples to the conjecture of Fujita, constructing fibrations with unitary flat factor $\cU$ of non finite monodromy. Our first motivation lies in the interest on the mentioned examples.
 We remark indeed that whenever the monodromy is finite, by the theorem of the fixed part proved by Deligne in \cite{Del_Theorie_1971}, $\cU$ defines (up to finite base changes) a constant Hodge substructure in the geometric variation of the Hodge structure of the fibers (that is a fixed abelian subvariety of the Jacobian of the general fiber $F$). In other words, the non-finiteness of the monodromy implies that the flat bundle\ \ $\cU\oplus \overline\cU\subset R^1f_\ast\bC\otimes \cO_B$ is not defined over the rational. This is consistent with a result contained in \cite{barja-fujita}, where the author proved that the unitary factor $\cU$ of fibrations over elliptic curves has always finite monodromy. We refer to  \cite{FGP} for details of the previous argument, where it has been developed, in a more general setting, in order to apply it to the study of the Hodge loci.
 
 In the paper we study obstructions to the non-finiteness of the monodromy using some techniques from deformation theory and Infinitesimal Variation of the Hodge Structure (IVHS). 
%  A remark is that $\cU$ turns out to lie in the kernel of the morphism describing the Infinitesimal Variation of the Hodge Structure (IVHS) \textcolor{red}{TODO:cite}. 
  In \cite{LuZuo_OnTheSlope_2017}[theorem A.1], the authors proved that the monodromy of the unitary flat bundle on hyperelliptic fibrations is finite. On the other hand it is well known that in the hyperelliptic locus the canonical normal function induced by the Ceresa cycles vanishes (see subsection \ref{SubSec-NonVanCriteria}). These facts suggested us to look at the vanishing of a second order cohomological invariant (the Griffiths infinitesimal invariant) in relation with the monodromy of $\cU.$
%   On the other hand $\cU$ turns out to lie in the kernel of the morphism describing the IVHS



The property we study is called in the paper {\em Massey-trivial property}. The {\em Massey product} has been introduced in \cite{C-P_TheGriffiths_1995} under the name of {\em  adjoint image} to compute the infinitesimal invariant of the Ceresa cycle and then studied by many authors in \cite{Gonz_OnDef_2016}, \cite{B-N-P_OnTheTopological_2007}, \cite{P-Z_Variations_2003}, \cite{RizZuc_Generalized_2017}, \cite{R_Infinitesimal_2008} for different purposes. 
We follow the terminology introduced in the last cited paper, where adjoint images have been interpreted as Massey products in the Dolbeault complex.
We now briefly recall the construction, postponing details to subsection \ref{Sec-MPOnFibrations}.
Let $F$ be a smooth fiber of $f$ over $b\in B$ ($F=f^{-1}(b)$), $\xi\in H^1(T_{F})$ be the  Kodaira-Spencer class attached to the first-order deformation of $F.$  Let $K_{\xi}$ be  the kernel of the cup product  $\cup {\xi}:H^0(\omega_{F})\to H^1(\cO_{F}),$  describing the IVHS of the family at $b,$ and assume that $\dim K_{\xi}\geq 2.$ We define  the {\em Massey-product} $\fm_{\xi}(s_1,s_2)$ of a pair $(s_1,s_2)$ of independent elements in $K_{\xi}$ (Definition \ref{Def-MTpair}) as the cohomological class $v_1c_2-v_2c_1\in H^0(\omega_F)$ modulo the $\bC$-vector space $<s_1,s_2>_\bC,$ where $c_1,c_2$ are $\cC^{\infty}$-functions on $F$ given by computing the cohomology of $\xi\cup s_i=0\in H^1(\cO_{F})$ through the Dolbeault resolution (i.e. $\xi_{\delbar}\cup s_i=\delbar c_i,$ where $\xi_{\delbar}$ is a Dolbeault representative of $\xi$ and  $c_i$ are determined modulo $H^0(\cO_F)\simeq \bC$). In particular, a pair $(s_1,s_2)$ is {\em Massey-trivial} if $\fm_{\xi}(s_1,s_2)$ lies in $<s_1,s_2>_\bC.$
Then the definition extends  in families considering sections of the kernel $\cK_\partial=\ker \partial$ of the connecting morphism $\partial:f_\ast \Omega^1_{S/B}\to R^1f_\ast\cO_F\otimes \omega_B $ defined by the pushforward of the short exact sequence  \begin{equation*}
\xymatrix@!R{
	{0}  & {f^\ast\omega_B}  & {\Omega^1_S}  & {\Omega^1_{S/B}}  & {0,}                                  & 
	% orizzontal arrows
	%%%	\ar"1,i";"j,k"
	\ar"1,1";"1,2"\ar"1,2";"1,3"\ar"1,3";"1,4"\ar"1,4";"1,5"
	\hole
}
\end{equation*}
where $\Omega^1_{S/B}$ is the sheaf of relative differentials of $f.$ The restriction of the connecting morphism over a regular value of $b$ gives exactly the cup product with the Kodaira-Spencer class of the fibre $F$ over $b$ and thus describes the above  situation. 
%
%We remark for the convenience of the reader that the common name used in literature, that is {\em vanishing adjoint images}, is different but we follow the one used in \cite{R_Infinitesimal_2008}. Such a property is a certain kind of vanishing request on families of second-order cohomological objects defined inside the kernel of the IVHS in a way we are going to clarify a little in what follow (a detailed explaination is then contained subsection \ref{Sec-MPOnFibrations}).

Looking at the natural inclusion $\cU\hookrightarrow \cK_{\partial}, $ we study families of Massey-products defined on local flat sections of the unitary flat bundle $\cU$ given by the second Fujita decomposition of $f.$ 
%This is actually possible considering the natural the inclusion $\cU\hookrightarrow \cK_{\partial}.$ \textcolor{red}{TODO:cinesi altro lift?} 
Let $\bU$ be the local system of stalk $U,$ underlying the unitary flat bundle $\cU,$ and $\rho_{\mtin{U}}$ be the associated monodromy representation (see section \ref{Sec-LocSyst}). The second Fujita decomposition defines an inclusion $U\subset H^0(\omega_F),$ where $F$ is the general fibre of $f.$  We remark that the space $\Gamma(A,\bU)$ of sections of $\bU$ over a contractible subset $A$ of $B$ is naturally isomorphic to $U$ (see Remark \ref{garibaldi}) and we
will use this isomorphism as an identification. We consider a {\em Massey-trivial} subspace
 $W\subset \Gamma(A,\bU)$ of local sections over an open contractible subset  $A$ of $B$
%where the fact that the holomorphic structure of the local system is completely determined by monodromy admits to control holomorphically the behavior of families of Massey products (that is basically getting holomorphic sections). 
and the unitary flat subbundle $\cM$ of $\cU$  {\em generated by} $W.$ Just to fix the ideas, a subspace of sections is Massey-trivial if each pair of sections is  Massey-trivial (Definition \ref{Def-MTSubspSec}) and a pair of sections is Massey-trivial if the Massey-product vanishes at the general point of $A.$ By Definition \ref{Def-GenLS}, the bundle $\cM$ is  associated to the smallest local subsystem $\bM$ of $\bU$ of stalk $M\subset U$ such that $W\subset M.$ In general, the subspace $M$ is not Massey-trivial even if $W$ is. 
%Then we study the {\em Massey-trivial property} on $W\subset \Gamma(A,\bU)\simeq U,$ extending the definition on pairs of sections in a natural way (definition \ref{Def-MTSubspSec}) . A pair
 Anyway, using a standard argument of analytic continuation it is easy to show that this condition does not depend on the open set $A$ (see Remark {\ref{rem-MTaction}) and therefore is actually a condition on $\cM$ generated by $W.$ 


%o each subspace $W \subset\Gamma(A, \bU)$ we correspond a unitary flat subbundle $\cM$ of $\cU$ called {\em the bundle generated by $W$} (definition \ref{Def-MTsubbundleA}), which is the one attached to the smallest subsystem $\bM$ containing $W.$ By definition, $\bM$ has stalk $M=G_{\mtin{$U$}}\cdot W=\sum_{g\in G}gW\subset U \subset H^0(\omega_F),$ where the quotient group $G_{\mtin U}$ of the fundamental group of the base $B$ with the kernel of the monodromy representation $\rho_{\mtin{$U$}}$ (isomorphic to the monodromy group of $\cU$) while $gW$ is the subspace of $U$ moved with $g\in G_{\mtin{$U$}}$ via the monodromy action $\rho_{\mtin{$U$}}$ of $U.$ In particular, $\cM$ has a monodromy representation $\rho_{\mtin{$M$}}$ of kernel $H_{\mtin{$M$}}$ and quotient group $G_{\mtin{$M$}}=\pi_1(B,b)/H_{\mtin{$M$}}$(again, isomorphic to the monodromy group $\Ima {\rho_{\mtin{$M$}}}$).  
 
%\textcolor{red}{  Then the development of the research showed connections with a more precise description of the bundles endowed with the interested property, moving the research from the whole $\cU$ to some suitable subbundles. Indeed, the fact that monodromy turned out to be captured by morphisms of curves, which in roughly speaking the content of the main result in the paper, says that the property forces automorphisms of the Jacobian $J(F)$ of the fibre $F$ induced by monodromy action to factorize through morphisms over the Jacobian of another fixed curve, which are moreover induced by morphisms between the curves themselves.} 
We are now able to state the main results of the paper. The first one provides a sufficient condition for the finiteness of the monodromy group, while the second one gives a more precise geometric description of the structure of unitary flat subbundles which are Massey-trivial generated.
\begin{theorem}\label{Thm-MainG}Let $f:S\to B$ be a complete fibration of genus $g(F)\geq 2$ and $\cU$ be the unitary factor in the second Fujita decomposition of $f.$ Let $\cM\subset \cU$ be a flat subbundle of $\cU$ generated by a Massey-trivial subspace. Then $\cM$ has finite monodromy.
\end{theorem}
%%%%%%OLDMainStatement
%\begin{theorem}\label{Thm-MainS} Let $f:S\to B$ be a semistable fibration of genus $g(F)\geq 2$ and $\cM\subset \cU$ be a unitary flat subbundle generated by a maximal Massey-trivial subspace $W\subset \Gamma(A, \bU).$ Then there is a finite set of distinct morphisms $k_g:F\to \Sigma$ of curves from a fixed fibre $F$ over $b$ to a smooth compact curve $\Sigma$ parametrized injectively by the monodromy group $G_{\mtin{M}}$ of $\cM$ and trivializing $\cM$ as $\sum_{g\in G_{\mtin{M}}} k_g^*H^0(\omega_{\Sigma})\otimes \cO_{B_{\mtin{M}}}$ after a suitable base change $u_{\mtin{M}}:B_{\mtin{M}}\to B.$ In particular, the monodromy of $\cM$ is finite. 
%\end{theorem}

\begin{theorem}\label{Thm-MainSbis} Let $f:S\to B$ be a semistable complete fibration of genus $g(F)\geq 2$ and $\cM\subset \cU$ be a unitary flat subbundle generated by a maximal dimensional Massey-trivial subspace. Then there is an injection of monodromy group $G_{\mtin{M}}$ of $\cM$ inside the group of bijection on a set $\mathscr{K}$ of morphisms $k_g:F\to \Sigma$ from the general fiber $F$ to a smooth compact curve $\Sigma$ of genus $g(\Sigma)\geq 2.$  Moreover, after a finite \'{e}tale base change $u_{\mtin{M}}:B_{\mtin{M}}\to B$ trivializing the monodromy, the pullback bundle of $\cM$ becomes the trivial bundle $V\otimes \cO_{B_{\mtin{M}}}$ of fibre $V=\sum_{g\in G_{\mtin{M}}} k_g^*H^0(\omega_{\Sigma})\subset H^0(\omega_F).$ 
\end{theorem}

The key examples of Massey trivial bundles are given by the trivial fibrations $f:S=F\times B\to B$. The finiteness of the monodromy in general allows (up to base changes) to define fibred maps $S\to A\times B,$ where $A$ is an abelian subvariety of the Albanese of $S,$ contained in the kernel of the map induced by $f$ between the Albanese variety of $S$ and $B.$ In the above case the abelian variety $A$ seen inside the Jacobian of the general fibre $F$ is defined by a morphism of curves.  
%%%The theorem shows that the general case is obtained as a sum of the above case,  up to finite base changes and pull-back via fibred morphisms. Thus they naturally define a fibred map $S\to J(F)\times B,$ that is into an abelian variety contained in the kernel of the map induced by $f$ between the Albanese variety of $S$ and $B.$  
%This is actually a much more stronger condition than 
%%The finiteness of the monodromy instead only admits (up to base changes) to define fibred maps $S\to A\times B$ where $A$ is an abelian variety contained in the Albanese variety of $S.$

A few comments on the assumptions on the statements. The geometric description given in Theorem \ref{Thm-MainSbis} needs that $f$ allows at most semistable singularities. This is because in this case $\cU$ has a good direct description in terms of variation of the geometric Hodge structure (we will recall details in subsection \ref{Def-LocSystBundlesMonRepr}). The reason depends on the behavior of local monodromies around the singularities of $f,$ which are trivial, and this is what one has to check in order to generalize the result.   

Theorem \ref{Thm-MainG} instead holds for fibrations with arbitrary singularities, even if it follows from Theorem \ref{Thm-MainSbis}. This is because of the semistable reduction theorem (see e.g. \cite[Theorem 2.7]{CD:Answer_2017}), which states that up to a finite possibly ramified base change we can reduce to a semistable fibration.


The heart of our proof is the description of $\bU$ given in the Lifting lemma \ref{Lem-CharFlatSecN}. We prove in fact that the local system $\bU$  is the image of the sheaf $f_\ast\Omega_{S,d}$ via the map $f_\ast\Omega_{S,d}\to R^1f_\ast\bC$ defined by the external differential (\ref{Dia-HolDeRhams}),
 where $\Omega_{S,d}$ is the sheaf of holomorphic closed forms on $S.$  

Then we use a version of Castelnuovo-de Franchis theorem generalized to the case of surfaces properly fibred over a non-compact base \cite{GonStopTor-On}.
In fact the Massey-trivial condition produces pairs of closed independent holomorphic forms $\omega_1,\omega_2$ defined in an open neighborhood of the general fiber $F$ with wedge zero, that is $\omega_1\wedge \omega_2=0.$ Then the above theorem produces morphisms $F\to \Sigma$ from the general fiber $F$ to a smooth compact curve  $\Sigma$ of genus greater than $2.$ We remark that in \cite{Cat_Moduli_1991} Catanese studied the Castelnuovo de-Franchis theorem in higher dimension. Unfortunately, there is no a natural generalization of the definition of Massey-products in higher dimension and it should be interesting to find an analogous condition. 
At this point, the proof of the finiteness of the monodromy follows from a classical theorem of de Franchis (see \cite{Martens_Obervations_1988} and also \cite{AlzatiPirola_Some_1991}), which states that the number of morphisms between two smooth compact curves of genus greater than $2$ is finite. 

In the final section we apply our results. A first application is obtained just reading our result in term of the infinitesimal Griffiths invariant of the  canonical normal function induced by the Ceresa cycle. It turns out that the non-finiteness of the monodromy is an obstruction to the vanishing of the infinitesimal Griffiths invariant. 
A second application is obtained using the triviality of the Massey product (a local condition) to get a semiampleness criterion on $\cU$ (a global condition) and in particular this applies to families of hyperelliptic curves and gives a new proof of \cite[Theorem A.1]{LuZuo_OnTheSlope_2017}.

The paper is organized as follows. Section 
\ref{Sec-LocSyst} is devoted to local systems, flat bundles, monodromy representations on curves. We fix the general setting and then we focus on some constructions given by fibrations, that is the variation of the Hodge structure and the second Fujita decomposition (subsection \ref{Sec-LocSyst}). Section \ref{Sec-MPOnFibrations} deals with infinitesimal deformation theory and contains the definition of Massey products together with some ad-hoc developments and constructions in the theory. In section \ref{Sec-LiftingsOnU} we prove the lifting lemma \ref{Lem-CharFlatSecN} and more that it gives a splitting. In section \ref{Sec-MTsubbundles} we apply the lifting lemma on Massey-trivial generated subbundles and fix the relation with on adapted version of Castelnuovo de Franchis theorem for fibrations. In section \ref{Sec-ProofMainTheorems} we give the proof of the main theorems contained in the paper. Section \ref{Sec-Applications} contains the applications.


{\bf Acknowledgements.} The authors would like to thank Fabrizio Catanese for the very helpful discussions on the subjects, during which he pointed out an inaccuracy in the proof. We also would like to thank the organizers  Elisabetta Colombo, Paola Frediani, Alessandro Ghigi, Ernesto Mistretta, Matteo Penegini, Lidia Stoppino of a series of seminaries on the topics relating Fujita decompositions. 
%\section{Preliminaries and notations}\label{Sec-Preliminaries}
%In the following, we introduce the reader to the results of the paper, giving an introduction on the main tools involved in it and fixing the language we are going to use.  In the paper we work over $\bC.$
\subsection*{Assumptions and notations.}\label{SusSec-GeneralitiesOnFibrations}
Let $Y$ be a smooth variety of dimension $r$ defined over $\bC.$ We will denote by $T_Y$ the tangent sheaf of $Y,$ with $\Omega^k_Y=\wedge^k\Omega^1_Y$  the sheaf of holomorphic $k$-forms on $Y,$ with $\omega_Y=\wedge^r\Omega^1_Y$ the canonical sheaf of $Y$ and with $\Omega^k_{Y,d}$ the subsheaf  $\Omega^k_Y$ of the closed holomorphic $k-$forms on $Y$ with respect to the de-Rham differential $d$ ($\Omega^i_{Y,d}= d\Omega^{i-1}_{Y}$). 
%$\Omega^i_{Y,d}\equiv d\Omega^{i-1}_{Y}.$
%Moreover, $q(Y)=\dim H^0(Y, \Omega^1_Y)$ denotes the irregularity of $Y$ and $p_g(Y)=\dim H^0(Y,\omega_Y)=H^{r,0}(Y)$ the geometric genus of $Y.$ \\ 
Through all the article $S$ will be a smooth surface and $B$ a smooth curve.  
A fibration $f:X\to B$ over a curve $B$ is a proper surjective and holomorphic morphism with connected fibres between a smooth analytic variety $X$ (namely, a complex manifold) of dimension $\dim X=n$ and a smooth irreducible curve $B.$ We also call $X$ a fibred space over the base $B.$ The base $B$ is not always complete and when it is we will say that the fibration $f$ is complete (that is, $B$ is projective and thus $X$ is compact). We denote by $F$ the general fibre  of $f,$ which we is a smooth variety of dimension $\dim F=n-1$.
Let $\omega_{X/B}=\omega_X\otimes f^*\omega_B^{\vee}$ be the relative dualizing sheaf and  $\Omega^1_{X/B}$ be the sheaf of relative differentials defined by the exact sequence
\begin{equation}\label{Ses-RelativeDifferentials}
\xymatrix@!R{
	{0}  & {f^*\Omega^1_B}  & {\Omega^1_X}  & {\Omega^1_{X/B}}  & {0,}                                  & 
	% orizzontal arrows
	%%%	\ar"1,i";"j,k"
	\ar"1,1";"1,2"\ar"1,2";"1,3"\ar"1,3";"1,4"\ar"1,4";"1,5"
	\hole
}
\end{equation}
obtained dualizing the morphism $d f:T_X\to f^*T_B$ induced by the differential. 
%We denote by $q_f(X)=\dim H^0(X,\Omega^1_X)-\dim H^0(B,\Omega^1_B) $ the relative irregularity of $X.$ 
%Form now on we focus on the case of fibrations over a smooth curve, that is $\dim Y=1,$ and we denote it with $B.$ 
%Finally, we consider the direct image sheaf $f_\ast\omega_{X/B},$ which is under our assumptions a locally free sheaf (i.e. a vector bundle) of rank $p_g(F),$ see for instance \cite{Fuj78a}. 
Let $Z\subset X$ be the locus of the critical points of $f,$ $B_0$ the subset of $B$ of the critical values of $f,$ $D=f^{-1}(B_0)$ the divisor of the singular fibres of $f$ and also $B^0=B\setminus B_0$ the locus of regular values and $X^0=X\setminus D.$ We recall that a fibration $f:X\to B$ is smooth when $B_0$ is empty, namely when all the fibres are smooth. We detote by $f^0:X^0\to B^0$ the restriction of $f$ over the curve $B^0$ (which is by definition a smooth fibration). 
%A fibration $f:X\to B$ is semistable when $D$ is normal crossing, namely when all the components of $D$ are reduced and  locally given by intersection of coordinate hyperplanes \textcolor{red}{TODO}. 
A fibred surface $S$ over a smooth curve $B$ is a $2-$dimensional fibred space as above and we will say that $f: S \to B$ is a fibration of genus $g(F)$ if the general fibre $F$ is a smooth irreducible complete curve of geometric genus $g(F).$ A fibration $f: S \to B$ has isolated singularities if $Z$ is supported over isolated points and it is called semistable if it is relatively minimal, that is it does not contain $(-1)-$curves on its fibres, and has at most nodes as singularities. 

%The main subject of the paper will be a fibred surface $S$ over a smooth curve $B$ and we say that $f: S \to B$ is a fibration of genus $g(F)$ if it is a fibration as before and the general fibre $F$ is a smooth irreducible complete curve of geometric genus $g(F).$ 
%\textcolor{red}{TODO:aggiungere notations fibrati} 

\section{Local systems, flat vector bundles and monodromy representations on fibrations over curves.}\label{Sec-LocSyst} In this section we set the notations and we recall some preliminaries about local systems, flat vector bundles, monodromy representations over a smooth curve $B.$ We refer to \cite{V_HodgeTheoryI_2002}, \cite{V_HodgeTheoryII_2003}, \cite{PetSteen_Mixed_2008} and also \cite{Kob_Differential_1987} for the details. 
			
			\begin{definition}\label{Def-LocSystBundlesMonRepr} Let $B$ be a smooth irreducible curve, $\pi_1(B,b)$ be the fundamental group of $B$ with base point $b.$
				\begin{itemize}
					\item[$(\mtin{LS})$] A {\em Local system} of $\bC$-vector spaces over $B$ is a sheaf $\bV$  of $\bC$-vector spaces which is locally isomorphic to the constant sheaf of stalk a $\bC$-vector space $V$.   
					\item[$(\mtin{FB})$] A {\em flat vector bundle} over $B$ of fibre a $\bC$-vector space $V$ is a pair $(\cV,\nabla)$ given by a locally free sheaf  $\cV$ of $\cO_B$-modules such that $\cV_b\otimes_{\cO_{B,b}}\cO_{B,b}/\fm_b\simeq V$ and a flat connection $\nabla:\cV\to\cV\otimes\Omega^1_B$ (i.e. such that the curvature $\Theta=\nabla^2$ is identically zero).
			      \item[$(\mtin{MS})$] A {\em Monodromy representation} of  $B$ over a $\bC$-vector space $V$ is a representation of the fundamental group $\pi_1(B,b)$, that is a homomorphism
				$$
				\rho_{\mtin{V}}: \pi_1(B,b)\to \Aut(V),
				$$ and the image $\Ima \rho_{\mtin{V}}$ is called the monodromy group.
				\end{itemize}
			\end{definition}
			Morphisms in the respective category are the natural ones: maps of sheaves of $\bC-$vector spaces on local systems, maps of vector bundles preserving the connection on flat vector bundles and maps of representations on monodromy representations. We remark that the above definitions can be generalized replacing $\bC$-vector spaces with $\bZ-$modules. 
In the sequel we assume that $V$ has finite dimension.			

\begin{proposition}\label{Prop-corrLSFBMR} There are one to one correspondences between local systems, flat vector bundles and monodromy representations modulo isomorphisms in the respective categories. More precisely, it holds 
	\begin{equation}\label{Mor-LStoFB}
	{\left\{\begin{array}{c}\mbox{Local systems $\bV$ over $B$} \\ \mbox{of } \bC-\mbox{vector spaces} \\
	\end{array}\right\}}_{/\mbox{iso}}\rightleftarrows{\left\{\begin{array}{c}\mbox{Flat vector bundles  } \\ (\cV,\nabla)
	\end{array}\right\}}_{ /\mbox{iso}} 			
	\end{equation} 
	
	\begin{equation}\label{Mor-LStoMR}
	{\left\{\begin{array}{c}\mbox{Local systems $\bV$ over $B$} \\ \mbox{of } \bC-\mbox{vector spaces} \\
	\end{array}\right\}}_{/\mbox{iso}}\rightleftarrows{\left\{\begin{array}{c}\mbox{Monodromy representations } \\ \rho_{\mtin{V}}:\pi_1(B,b)\to \Aut(V)
	\end{array}\right\}}_{ /\mbox{con}} 			
		\end{equation} 
where "$\mbox{iso}$" denotes the action given by isomorphisms in the correspondent category, while "$ \mbox{con}$" denotes the action given by conjugation.
\end{proposition}	


We shortly recall the constructions of the stated correspondences.

{\bf Correspondence \ref{Mor-LStoFB}.} Correspondence $\bV\mapsto (\cV, \nabla)$ is constructed taking $\cV$ the vector bundle $\cV:=\bV\otimes_{\bC}\cO_B$ and $\nabla$ the flat connection defined by $\ker \nabla\simeq \bV;$ the inverse $(\cV, \nabla)\mapsto \bV$ is given setting $\bV$ to be the sheaf $\ker \nabla,$ called  {\em the local system of flat sections} of $\cV.$ 

%%%
	
	{\bf Correspondence \ref{Mor-LStoMR}.}  Correspondence $\bV\mapsto \rho_{\mtin{V}}$ is constructed fixing a point $b\in B,$ considering the isomorphism $\alpha:\bV_b  \simeq V$ and defining $\rho_V(\gamma)=\alpha\circ \gamma^*\alpha^{-1},$ where $\gamma^*:\bV_b\simeq  \bV_b$ is the isomorphism induced by $\gamma\in \pi_1(B,b).$ Conversely, $\rho_{\mtin{V}}\mapsto \bV$ is given by looking  at the action of $\pi_1(B,b)$ on  $\widetilde{B}\times V,$ where $\widetilde{B}$ is the universal covering of $B,$ induced by $\rho_{\mtin{V}}.$
		
	We have the following properties about local systems and behaviour of their monoromy groups.
	\begin{proposition}\label{Prop-LocSysBC} Let $u : B'\to B$ be a morphism of curves and $\bV$ be a local system over $B.$ Then $u^{-1}\bV$ is a local system over $B'.$ Moreover, the associated monodromy representation factors through $u_\ast:\pi_1(B',b')\to \pi_1(B,b),$ where $b'\in u^{-1}(b).$ 
		\end{proposition}
		\begin{proposition}\label{Prop-SumLocSys} Let $\bV_1$ and $\bV_2$ be two local subsystems of the local system $\bV.$ If they  both have finite monodromy, then the local subsystem $\bV_1+ \bV_2$ of $\bV$ has finite monodromy.
		\end{proposition}
		\begin{proof} Let $\rho_{V}$ be the monodromy representation of $\bV,$ $H$ the kernel of $\rho_{V}$ and $H_i$ the kernel of the sub-representations induced by $\rho$ on $\bV_i,$ for $i=1,2.$  Then $H_{12}:=H_1\cap H_2$ is the kernel of the sub-representation of $\bV_1+ \bV_2.$ We prove that $H_{12}$ has finite index in $\pi_1(B,b).$ By assumption, $\bV_1$ and $\bV_2$ have both finite monodromy, which means that $\pi_1(B,b)/H_i,$ for $i=1,2,$ are finite. Consider the chain of normal extensions $H_{12}\triangleleft H_1 \triangleleft  H.$ Then $H_1\triangleleft  H$ has finite index by assumption, while $H_{12}\triangleleft H_1$ has finite index since there is a natural injective morphism $H_1/H_{12}\hookrightarrow \pi_1(B,b)/H_2$ and $\pi_1(B,b)/H_2$ is finite by assumption.  Thus $H_{12}\vartriangleleft H$ has finite index too.
		\end{proof}
		
		\begin{remark} \label{garibaldi}
			There is a natural isomorphism $\Gamma(A,\bV)\to V$ over any contractible subset $A$ of $B$ since $\bV$ is trivial over $A.$
		\end{remark}
			
			The correspondences given in proposition \ref{Prop-corrLSFBMR} generalize when some suitable metric structures  are introduced: a {\em  unitary local system} $(\bV, h),$ with $h$ an hermitian structure on $\bV;$ a {\em unitary flat vector bundle} $(\cV,\nabla, h),$ with $h$ an hermitian metric compatible with the holomorphic connection $\nabla;$ a {\em unitary monodromy representation} $(\rho_{\mtin{$V$}},(V,h)),$ with $(V,h)$ a hermitian vector space and $h$ preserved under the monodromy action. Under this assumption, there is a fundamental structure theorem  due to Narasimhan and Seshadri \cite{NarSes_Stable_1965}, which links unitarity to stability. We recall that a holomorphic vector bundle on a complete smooth curve $B$ is stable if the slope (i.e. the number given by the degree over the rank of a vector bundle) decreases on subbundles.
			
			\begin{theorem}\label{Thm-UnitaryFlatBundlesNS} Let $B$ be a smooth complete irreducible curve of genus $g(B)\geq 2.$ Then a holomorphic vector bundle $\cV$ on $B$ of degree zero is stable if and only if it is induced by a irreducible unitary representation of the fundamental group of $B$.
				\end{theorem}
			%We are interested in studying the behaviour on vector subspaces of the $\bC$-vector space $V$ in relation with the objects introduced above. 
			Let $\bV$ be a local system over a smooth curve $B$ of  stalk $V$  and let $\rho_{\mtin{V}}: \pi_1(B,b)\to \Aut(V)$ be its monodromy representation. We will always denote by $H_{\mtin{V}}=\ker \rho_{\mtin{V}}$  the kernel of $\rho_{\mtin{V}}$  and $G_{\mtin{V}}=\pi_1(B,b)/H_{\mtin{V}}$ the quotient group which is isomorphic to the monodromy group $\Ima \rho_V.$ 
			 We want to attach a local subsystem of $\bV$ to a vector subspace $W\subset V.$ Given a vector subspace $W$ of $V$ we define $$G_{\mtin{V}}\cdot W:=\sum_{g\in G_{\mtin{V}}} g\cdot W,$$ where $g\cdot W:=\rho_{\mtin{V}}(g)(W)$ (we will also write shortly $gW$). We remark that $G_{\mtin{V}}\cdot W$ is smallest subspace of $V$ containing $W$ and invariant under the action $\rho_{\mtin{V}}.$
			Thus it defines the smallest sub-representation of $V$ containing $W.$
			 
		
			\begin{definition}\label{Def-GenLS} Let $\bV$ be a local system over a smooth curve $B$ of stalk the $\bC$-vector space $V$ and  $W$ be a vector subspace of $V$. The local system $\widehat{\bW}$ {\em generated by $W$} is the local sub-system of $\bV$ of stalk $\widehat{W}=G_{\mtin{V}}\cdot W.$ 
				\end{definition}
				As usual, we denote by $\rho_{\mtin{$\widehat{W}$}}$ the monodromy representation of $\widehat{W},$ with $H_{\mtin{$\widehat{W}$}}$ the kernel and with $G_{\mtin{$\widehat{W}$}}$ the quotient. We also denote by $H_{\mtin{W}}$ the subgroup of $H_{\mtin{V}},$ which fixes pointwise $W.$ We remark that $H_{\mtin{$\widehat{W}$}}$ is the normalization of $H_{\mtin{W}}.$
				
				Let $\bV$ be a local system of stalk $V$ over a smooth curve $B$ and let $A\hookrightarrow B$ be an open contractible subset of $B$ and we allow to identify $V\simeq \Gamma(A,\bV)$ via the canonical isomorphism between the stalk $V$ and the sections over $A.$ In the following we prove some properties of generated local systems. 
				\begin{proposition}\label{Prop-SubLocSystMon} Let $W_1$ and $W_2$ be two subspaces of $ \Gamma(A, \bV)$ such that $W_1\subset W_2.$ If the local system $\widehat{\bW}_2$ generated by $W_2$ has finite monodromy, then the local system $\widehat{\bW}_1$ generated by $W_1$ has finite monodromy.
				\end{proposition}
				\begin{proof}
					Let $H=\ker\rho$ be the kernel of the unitary representation of $\bV,$ and $H_i=\ker \rho_i$ be the kernel of the sub-representations $\rho_i$ defining $\widehat{\bW}_i,$ for $i=1,2$. Then we have an inclusion $ H_2 \vartriangleleft H_1$  of subgroups which gives a surjection 
					\begin{equation}
					\xymatrix@!R{
						{G_2:=\pi_1(B,b)/H_2}  & {G_1:=\pi_1(B,b)/H_1}  & {0} 
						% orizzontal arrows
						%%%	\ar"1,i";"j,k"
						\ar"1,1";"1,2"\ar"1,2";"1,3"
						% vertical arrows
						\hole
					}
					\end{equation}
					on the quotients groups isomorphic to the monodromy groups of $\widehat{\bW}_2$ and $\widehat{\bW}_1,$ respectively. Thus whenever the monodromy of $\widehat{\bW}_2$ is finite, the monodromy of $\widehat{\bW}_1$ is finite.
				\end{proof}
				
				
				\begin{proposition}\label{Prop-LocSysAndBaseChange} Let $\bV$ be a local system over a curve $B,$ $W \subset \Gamma(A, \bV)$ a vector subspace and $\widehat{\bW}$ the local subsystem of $\bV$ generated by $W.$  Then a Galois covering of curves $u:B'\to B$ induces a isomorphism of local systems over $B'$
					\begin{equation}\label{Mor-GenLocSysBaseChange}
					\xymatrix@!R{
						{u^{-1}\widehat{\bW}}  & {\sum_{g_i\in I_u}{\widehat{\bW}}_{g_i},}        &    
						% vertical arrows
						\ar@{<->}"1,1";"1,2"  
						%%%
						\hole
					} 						
					\end{equation}
					where ${\widehat{\bW}}_{g_i}$ is the local subsystem generated by $u^*(g_i\cdot W),$ for  $g_i$ varying in a set $I_u\subset \pi_1(B,b)$ of generators of the quotient given by $u_\ast:\pi_1(B',b')\to \pi_1(B,b).$ \end{proposition}
				\begin{proof} Consider the local system $\widehat{\bW}$ generated by $W,$ which is by definition the local system on $B$ of stalk $\pi_1(B,b)\cdot W$ and monodromy representation $\rho_{\mtin{W}}.$ The inverse image $u^{-1}\widehat{\bW}$ is a local system of the same stalk (i.e. $\pi_1(B,b)\cdot W$) and monodromy representation $\rho^{-1}_{\mtin{W}}$ given by the action of $\pi_1(B',b')$ via the composition $\rho\circ u_\ast,$  where $u(b')=b$ and  $u_\ast:\pi_1(B',b')\to \pi_1(B,b)$ is the natural homomorphism induced by $u.$ Consider the local system $\widehat{\bW}_{g},$ which is a local system on $B'$ of stalk generated by $u^*gW$ (i.e. $\pi_1(B',b')\cdot u^*gW).$ Then, since the monodromy action of $g\in \pi_1(B,b)$ sends $W$ to $gW,$ it is clear that the sum over a set of generators of the cokernel of $u_\ast$ reconstructs exactly $u^{-1}\widehat{\bW}.$ 
				\end{proof}
		%%%%%%%%%%%%%%%%%%%%%%%%%%%						
					
			\subsection{Local systems on fibred surfaces: geometric variation of the Hodge structure and the second Fujita decomposition}\label{SubSec-Prel-LocSystOnFibr} In this subsection we briefly recall the construction of two local systems naturally attached to a fibration $f:S\to B$ of genus $g(F).$ The first is given by the geometric VHS of weight one and the second by the unitary flat factor in the second Fujita decomposition. For details in Hodge theory please consult \cite{Grif_PeriodsIII_1970}, \cite{V_HodgeTheoryII_2003}, \cite{CatElZFouGrif_Hodge_2014} and \cite{PetSteen_Mixed_2008}. The references for the second Fujita decomposition are \cite{Fuj78b}, \cite{CatDet_TheDirectImage_2014}, \cite{CD:Answer_2017}, \cite{CatDet_Vector_2016} and also \cite{barja-fujita}.
			
			{\bf $(1)$ Geometric variation of the Hodge structure and semistable fibrations.}\label{ES-LocSyst1}  Let $f:S\to B$ be a smooth fibration (i.e. $f$ is a submersive morphism). In this case, $(\bH_{\bZ}=R^1f_*\bZ,\,  \cF^1=f_*\omega_{S/B},\, Q )$ is a polarized variation of the Hodge structure of weight one pointwise defined by the polarized geometric Hodge structure of weight one $\{H_{\bZ,b}=H^1(F_{b},\bZ),\,H_b^{1,0}=H^0(\omega_{F_{b}}),\,Q_b(-,=)=\int_{F_b}-\wedge =\}.$  The sheaf $R^1f_\ast\bZ $ is indeed a local system of $\bZ-$modules of stalk $H^1(F,\bZ),$ where $F$ is the general fibre of $f,$ the sheaf $R^1f_\ast\bC$ is a local system of $\bC-$vector spaces of stalk $H^1(F,\bC)\simeq H^1(F,\bZ)\otimes_\bZ\bC$ and there is an injection $ f_\ast\omega_{S/B}\hookrightarrow R^1f_\ast\bC \otimes \cO_B$ of vector bundles pointwise defined by $H^0(\omega_{F_b})\hookrightarrow H^1(F_b,\bC)\simeq H^1(F,\bC).$  %(i.e.  $Q(-,-)=\int_{F_b}-\wedge-$) 
			
			
			Let us now assume that $f:S\to B$ acquires isolated singularities. The sheaf $R^1f_\ast\bC$ is no more a local system in general, since the homology of the singular fibres can differ from the one of the general fibre and this fact is related with the behaviour under the {\em local monodromies}.
			
			Let $B^0$ be the locus of regular values of $f$ and $j: B^0\hookrightarrow B$ be natural the inclusion. Then the restriction $f^0:S^0\to B^0$ of $f$ to $B^0$ is smooth and defines a polarized geometric VHS of weight one  $({\bH_{\bZ}}_0=R^1f^0_\ast\bZ,\,  \cF^1_0=f^0_*\omega_{S^0/B^0},\, Q_0 )$ as above. 
			 Consider the morphism 
			\begin{equation}\label{Mor-AdjRestr}
			\alpha : R^1f_\ast\bC_S\to j_\ast j^*R^1f_\ast\bC_S, 
			\end{equation} 
		locally given by restriction to the local system $j^*R^1f_\ast\bC\simeq R^1f^0_\ast\bC.$ We briefly recall the role played by the local monodromies around the singularities. Let $b_i$ be a singular value of $f$ and $\Delta$ be a coordinate complex disk centered in $b_i,$ which 		does not contain any other critical values of $f.$ Let $f_{\mtin{$\Delta$}}:S_{\mtin{$\Delta$}}\to \Delta,$  $S_{\mtin{$\Delta$}}=S_{|{\mtin{$\Delta$}}},$ be the morphism given by restriction of $f$ to $\Delta$ and $F_t$ be a fibre over a regular value $t\neq 0.$ Then the monodromy action of $\pi_1(\Delta\setminus \{0\})$ on $H^1(F_t)$ naturally defines a homomorphism
			\begin{equation}\label{Mor-MonOper}
			T_i:H^1(F_t)\to H^1(F_t)
			\end{equation} 
			called {\em monodromy operator } or {\em Picard-Lefschetz transformation} around $b_i.$ This describes the {\em local monodromy around $b_i$}.
			Let $r: F_t\to F_{b_i}$ be the map given by composition of the inclusion $F_t\to S_{\mtin{$\Delta$}}$ and the retraction $S_{\mtin{$\Delta$}}\to F_{b_i}.$ 
			\begin{definition}\label{Def-LocInvCycPropPoint}
				Let $f:S\to B$ be a fibration of genus $g(F)$ with isolated singularities and $b_i$ be a singular value of $f$. We say that $f$ satisfies the \emph{Local invariant cycle property} around $b_i$ if the sequence
				\begin{equation}\label{Seq-LocInvCycProp}
				H^1(F_{b_i})\stackrel{r_i^*}{\to}H^1(F_t)\stackrel{T_i-I}{\to}H^1(F_t)
				\end{equation}
				given by $f_{\mtin{$\Delta$}}:S_{\mtin{$\Delta$}}\to \Delta$ is exact. In this case, the vector space $H^1(F_t)_{\mbox{\tiny INV}_i}:= \ker (T_i-I)$ of invariants under the local monodromy $T_i:H^1(F_t)\to H^1(F_t)$ is given by the cohomology of the singular fibre $F_{b_i}.$ 
			\end{definition}
			We remark that when the property introduced above holds on each singularity of $f,$ the morphism \eqref{Mor-AdjRestr} is surjective. Moreover, from \cite[Lemma C.13, pag. 440]{PetSteen_Mixed_2008} and \cite[Theorem 5.3.4, pag. 266]{CatElZFouGrif_Hodge_2014}) we get the following result. 
			\begin{lemma} \label{Lem-LocInvIso}
				Let $f:S\to B$ be a complete fibration of genus $g(F)$ with isolated singularities (e.g. semistable). Then $f$ satisfies the local invariant cycle property near all singular values, the morphism \eqref{Mor-AdjRestr} is an isomorphism and $R^1f_\ast\bC$ is completely determined by the local system $j^\ast R^1f_\ast \bC.$
			\end{lemma}
			
			 
			{\bf $(2)$ The second Fujita decomposition of $f.$} Let $f:S\to B$ be a complete fibration of genus $g(F)$ and $f_\ast\omega_{S/B}$ the direct image of the relative dualizing sheaf. The {\em second Fujita decomposition} (\cite{Fuj78b},    \cite{CatDet_TheDirectImage_2014}) states that there exists a unitary flat bundle $\cU$  giving a splitting 
			\begin{equation}\label{Dec-IIFuj}
			f_\ast\omega_{S/B}=\cU\oplus \cA
			\end{equation}
			on $f_\ast\omega_{S/B},$ with $\cA$ an ample vector bundle.  %Even if already announced by Fujita in \cite{Fuj78a}, a complete proof is contained in \cite{CatDet_TheDirectImage_2014}.
			As explained of subsection \ref{Sec-LocSyst}, there exists a unitary local system $\bU$ of stalk the fibre $U$ of $\cU$ such that $\cU=\bU\otimes \cO_B,$ uniquely determined up to isomorphisms of sheaves, and a unitary monodromy representation $\rho_{U}:\pi_1(B,b)\to \Aut(U,h),$ determined up to conjugacy classes. We briefly denote by $\rho$ (instead of $\rho_{U}$) the monodromy representation, with $H$ the kernel and with $G$ the quotient $\pi_1(B,b)/H.$ We recall that $G$ is naturally isomorphic to the monodromy group of $\cU$ and we identify them. 
%			  Let $F$ be the general fibre of $f,$ then $U$ is a subvector space of $H^0(\omega_F).$ In fact on the general (smooth) fiber $F$ there is a natural isomorphism 
%			\begin{equation}\label{Dia-EvalKER}
%			\xymatrix@!R{
%				 {{f_*\omega_{S/B}}_{b}\otimes_{\cO_{B,b}}\cO_{B,b'}/\fm_{b}}  & {H^0(\omega_{F})} .
%				% vertical arrows
%				\ar"1,1";"1,2"^>>>>\sim
%				\hole
%			} 						
%			\end{equation}
%			Then the unitarity of $\cU$ reflects to the unitarity structure on the representation $\rho_U$ with respect to an hermitian metric on $U.$ 

			 Let $f^0:S^0\to B^0$ be the restriction of $f$ to the locus of regular values $B^0.$  The restriction $\cU_{|B^0}$ of 
$\cU$ to $B^0$ is the unitary flat sub-bundle of $j^\ast R^1f_\ast\bC\otimes \cO_{B^0}$ of fiber $(U,h),$ where  $U\hookrightarrow H^0(\omega_F)\simeq H^{1,0}(F)$ and $h$ is the hermitian form induced by the polarization $Q$ on the fibers (i.e. $h(-,-)=iQ(-, \overline{-})$ where bar is the complex conjugation). Thus it is described by the geometric variation of the Hodge structure of weight one as the sheaf of locally flat sections of $f^0_\ast\omega_{S^0/B^0},$ under the inclusion $f^0_\ast \omega_{S^0/B^0} \subset j^\ast R^1f_\ast\bC\otimes \cO_{B^0}.$ The relation between $\cU_{|B^0}$ and $\cU$ depends on the behaviour under locals monodromies. 
%			Indeed, it turns out that the restriction $Q_{|H^0(\omega_F)}$ of $Q$ to $H^0(\omega_F)$ is positive definite.
			According to previous results, in \cite{CD:Answer_2017} the authors proved that $\cU_{|B^0}$ extends trivially on $B$ when $f$ is semistable, that is $\cU$ is precisely described by variation of the Hodge structure (see also \cite{CD:Answer_2017}). This is a consequence of the unipotency of the local monodromies.
%			where it is proved that the hermitian metric giving the unitary structure to $U$ is the one given by the polarization of the Hodge structure attached to the restriction of $f$ to the set of regular values $B^\circ.$ \textcolor{red}{TODO: EXPLAIN BETTER Indeed local monodromies act trivially on unitary and unipotent monodromy representations.}
%			
%			Releasing the request of semistability, the previous canonical description of $\bU$ is not always true. 
			Let $f:S\to B$ be a fibration and assume it is not semistable. The
%			we loose in general direct informations on $\cU$ as well as it happens in general to local monodromies. Anyway 
 existence of $\cU$ has been proven \cite{CD:Answer_2017} using different techniques concerning the behaviour of quotients of $f_\ast\omega_{S/B},$ which is nef (\cite{Fuj78a}) and then applying the semistable-reduction theorem (see  \cite[Theorem 2.7 and Proposition 2.9 ]{CD:Answer_2017}, ), which allows to reduce to the semistable case. The  theorem provides a base change $u:B'\to B$ given by a ramified finite morphism of curves and a resolution of the fiber product 
			\begin{equation}\label{Dia-BaseChangeSemistab}
			\xymatrix@!R{
				{S':=\widetilde{S\times_BB'}} \,        &     {S}              &    \\
				{B'}            \,        & {B}         &    
				% vertical arrows
				\ar"1,1";"2,1"^{f'}   \ar "1,1";"1,2"^>>>>>>{\varphi}
				%%
				\ar"1,2";"2,2"^{f} \ar "2,1";"2,2"^{u}
				% diagonal arrow, with 1 hole
				%	\ar@{-}"p1"|!{"2,3";"4,5"}
				\hole,
			}
			\end{equation}     
			producing a semistable fibration $f':S'\to B'$ from a smooth complete surface  $S'$ to a smooth complete curve $B'.$ We will refer to $f':S'\to B'$ as the {\em semistable-reduceed fibration of $f$}. Let $\cU'$ be the unitary bundle given the second Fujita decomposition of $f'.$ The relation between the unitary factor $\cU$ of $f$ and the unitary factor $\cU'$ of its semistable reduction is the following. 
			\begin{lemma} There exists a short exact sequence 
				\begin{equation}\label{SES-UnitaryFacBaseChange}
				\xymatrix@!R{
				{0} &	{\cK_{\mtin{U}}}  & {\cU'}  & {u^*\cU}  &  {0,} 
					% orizzontal arrows
					%%%	\ar"1,i";"j,k"
					\ar"1,1";"1,2"\ar"1,2";"1,3"\ar"1,3";"1,4"\ar"1,4";"1,5"
					% vertical arrows
					\hole
				}
				\end{equation}
				which is split. Moreover, $\cK_{\mtin{U}}$ is unitary flat and the splitting is compatible with the underlying local systems. 				
				\end{lemma} 
				\begin{proof}
					Let $\cU'$ be the unitary factor of the semistable reducted fibration $f'$ of $f$ and $\cU$ be the unitary factor of $f.$ Recall that there is a short exact sequence (\cite[Proposition 2.9]{CatDet_TheDirectImage_2014})
				\begin{equation}\label{SES-HodgeBundlesBaseChange}
				\xymatrix@!R{
					{0}  & {f'_\ast\omega_{S'/B'}}  & {u^*f_\ast\omega_{S/B}}  & {\cG}  & {0,} 
					% orizzontal arrows
					%%%	\ar"1,i";"j,k"
					\ar"1,1";"1,2"\ar"1,2";"1,3"\ar"1,3";"1,4"\ar"1,4";"1,5"
					% vertical arrows
					\hole
				}
				\end{equation}
				where $\cG$ is a skyscraper sheaf supported on points over the singular fibers of $f.$ Comparing the second Fujita decompositions of $f$ and $f'$ we get 
					\begin{equation}\label{SES-HodgeBundlesBaseChange}
				\xymatrix@!R{
					{0}  & {\cA'\oplus \cU'}  & {u^*\cA\oplus u^*\cU}  & {\cG}  & {0}  
					% orizzontal arrows
					%%%	\ar"1,i";"j,k"
					\ar"1,1";"1,2"\ar"1,2";"1,3"^{i'}\ar"1,3";"1,4"\ar"1,4";"1,5"
					\hole
				}
				\end{equation}
				which induces by projection a morphism $\cU'\to u^*\cU.$ Using standard arguments of vector bundles, it is easy to see that a map from an ample bundle to a unitary flat bundle must be the null map. Using the characterization of unitary flat bundle over curves of genus greater that $2$ recalled in section \ref{Sec-LocSyst}, it turns out that the morphism above is surjective of vector bundles. The cases of genus $0$ and $1$ are trivial.  We refer to \cite{CatDet_TheDirectImage_2014}, \cite{CD:Answer_2017} and also \cite{CatDet_Vector_2016} for details.
					\end{proof}
%%%%		

\begin{remark}\label{Rem-AmpleIntoUnitary} As a consequence of the proof of the previous theorem, we also have the exact sequence
	\begin{equation}\label{SES-AmpleIntoUnitary}
	\xymatrix@!R{
		{0}  & {\cA'\oplus \cK_{\mtin{U}}}  & {u^*\cA}  & {\cG}  & {0.}
		% orizzontal arrows
		%%%	\ar"1,i";"j,k"
		\ar"1,1";"1,2"\ar"1,2";"1,3"\ar"1,3";"1,4"\ar"1,4";"1,5"
		\hole
	}
	\end{equation}
	
	\end{remark}

To conclude this section, we analyze a little more the structure on $\cU$ in relation to the first Fujita decomposition \cite{Fuj78a}.
\begin{remark}The injection $\cO_B^{\oplus q_f}\hookrightarrow\cU,$ where $q_f=h^1(\cO_S)-g(B)$ is the relative irregularity of $f$ 
 gives a splitting  
 	\begin{equation}\label{Split-FujiDecTOT}
	f_\ast\omega_{S/B}=\cO_B^{\oplus q_f}\oplus \cU'\oplus \cA,
	\end{equation}  
with	$h^1(\omega_B(\cU'))=0.$
We remark that this is compatible with the {\em first Fujita decomposition} proved in \cite{Fuj78a}
\begin{equation}\label{Split-FujiDecI}
f_*\omega_{S/B}=\cO_B^{\oplus q_{f}}\oplus \cE,
\end{equation}  
with $\cE$ a vector bundle. 
	\end{remark}         
        \section{Fibrations of curves and families of Massey products.}\label{Sec-MPOnFibrations} In this section we recall the construction of the  "Massey-products" on families of curves. Some references in this topics are \cite{C-P_TheGriffiths_1995}, \cite{Gonz_OnDef_2016}, \cite{P-Z_Variations_2003}, \cite{NarPirZuc_Poly_2004}, \cite{R_Infinitesimal_2008}. For details on deformation theory and variation of the Hodge structure instead we refer to  \cite{Grif_InfinitesimalVariationsIII_1983},\cite{GriffithsHarris_Infinitesimal||_1983},\cite{GriffithsHarris_Infinitesimal||i_1983},\cite{V_HodgeTheoryI_2002} and also \cite{V_HodgeTheoryII_2003}.
        
                Let $f:S\to B$ be a fibration of genus $g(F)\geq 2.$ The fibration $f$ defines an infinitesimal deformation $f_{\mtin{$\Delta_\epsilon$}}:S_{\mtin{$\Delta_\epsilon$}}\to \Delta_\epsilon$  on the general fibre $F,$ where $\Delta_\epsilon= \spec \bC[\epsilon]/(\epsilon^2)$ is the spectrum of the ring $ \bC[\epsilon]/(\epsilon^2)$ of dual numbers. Let $\xi\in \Ext^1_{\cO_F}(\omega_F,\cO_F\otimes T^{\vee}_{\Delta_\epsilon,0})\simeq H^1(T_F)\otimes T^{\vee}_{\Delta_\epsilon,0}$ be the extension class and          \begin{equation}\label{SeS-DeRhamOnSFibre}
         \xymatrix@!R{
         	{0}  & {\cO_F\otimes T^{\vee}_{\Delta_\epsilon,0}}  & {\Omega^1_{S|F}}  & {\omega_{F}}  & {0}                                  & 
         	% orizzontal arrows
         	%%%	\ar"1,i";"j,k"
         	\ar"1,1";"1,2"\ar"1,2";"1,3"\ar"1,3";"1,4"\ar"1,4";"1,5"
         	\hole
         }
         \end{equation} 
   the exact sequence   
       given by $\xi$. Up to fix a generator $\sigma$ of $T_{\Delta_\epsilon,0}^\vee,$ we get an induced isomorphism $\sigma: \cO_F\otimes  T^{\vee}_{\Delta_\epsilon,0}\simeq \cO_F,$ which we call $\sigma$ itself with a little abuse of notation, and we can look at $\xi\in H^1(T_F)\otimes T^{\vee}_{\Delta_\epsilon,0}$ as the Kodaira-Spencer class $\xi\in H^1(T_F)$ defining the short exact sequence
         \begin{equation}\label{SeS-DeRhamOnSFibrebis}
         \xymatrix@!R{
         	{0}  & {\cO_F}  & {\Omega^1_{S|F}}  & {\omega_{F}}  & {0.}                                  & 
         	% orizzontal arrows
         	%%%	\ar"1,i";"j,k"
         	\ar"1,1";"1,2"\ar"1,2";"1,3"\ar"1,3";"1,4"\ar"1,4";"1,5"
         	\hole
         }
         \end{equation} 
         The connecting homomorphism $\delta$ on the associated long exact sequence in cohomology
         \begin{equation}\label{LeS-DeRhamOnSFibre}
         \xymatrix@!R{
         	{0}  & {H^0(\cO_F)}  & {H^0(\Omega^1_{S|F})}  & {H^0(\omega_{F})}  & {H^1(\cO_F)},                                  & 
         	% orizzontal arrows
         	%%%	\ar"1,i";"j,k"
         	\ar"1,1";"1,2"\ar"1,2";"1,3"\ar"1,3";"1,4"\ar"1,4";"1,5"^{\delta=\cup\xi}
         	\hole
         }
         \end{equation}
        is given by the cup product $\cup \xi : H^0(\omega_F)\to H^1(\cO_F).$ We recall that, by the Griffiths trasversality theorem (see for example \cite{GrifTopics1984}), $\cup \xi$ defines an IVHS, which is given in this case by a VHS. We denote by $K_\xi=\ker(\cup\xi).$ Let 
         \begin{equation}\label{Mor-Mp/Aj}
         \wedge_\xi\colon \xymatrix@!R{
         	{\bigwedge^2H^0(\Omega^1_{S|F} )}  & {H^0(\bigwedge^2\Omega^1_{S|F})\simeq H^0(\omega_{F})}                  
         	% orizzontal arrows
         	%%%	\ar"1,i";"j,k"
         	\ar"1,1";"1,2"%^-{\wedge_\xi}
         	\hole
         }
         \end{equation}
         be the map defined by the composition of the wedge product with the isomorphism $\bigwedge^2H^0(\Omega^1_{S|F} )\simeq \omega_F$ induced by sequence \eqref{SeS-DeRhamOnSFibrebis}. On any pair $(s_1,s_2)$ of linearly independent elements of $K_{\xi}$, we can choose a pair of liftings $(\tilde{s_1},\tilde{s_2})$ in $H^0(\Omega^1_{S|F})$ and take the image $\tilde{s_1}\wedge_\xi \tilde{s_2}\in H^0(\omega_F)$ of the map $\wedge_\xi,$ where we have set $\tilde{s_1}\wedge_\xi \tilde{s_2}:=\wedge_\xi(\tilde{s_1}\wedge\tilde{s_2}) .$ Such image depends on the choice of both liftings but it turns out to be well defined modulo the $\bC$-vector space $< s_1,s_2>_\bC$ generated by $(s_1,s_2),$ since each lifting must differ from the previous one for an element in $H^0(\cO_F)\simeq\bC$ according to \eqref{LeS-DeRhamOnSFibre}.  
         \begin{definition}\label{Def-MPAJ}
         	The equivalence class 
         	\begin{equation}\label{Mor-Mp/Aj}
         	\fm_{\xi}(s_1,s_2):=[(\tilde{s}_1\wedge_\xi\tilde{s}_2)]\in H^0(\omega_F)/< s_1,s_2>_\bC 
         	\end{equation}
         	is called {\em Massey product of $(s_1,s_2)$} along $\xi$. 
         \end{definition}
         
         A {\em vanishing request} natural in setting of Massey-products is the following.
         \begin{definition}\label{Def-MTpair}
         	A pair $(s_1,s_2)\subset K_{\xi}$ is {\em Massey-trivial} if $\fm_\xi(s_1,s_2)=0,$ that is $\tilde{s}_1\wedge_\xi\tilde{s}_2\in  < s_1,s_2>_\bC,$ for a (hence every) pair $(\tilde{s}_1,\tilde{s}_2)$ of liftings of $(s_1,s_2)$ in $H^0(\Omega^1_{S|F})$. Equivalently, we also say that the pair $(s_1,s_2)\subset K_{\xi}$ has {\em vanishing Massey-products}.
         \end{definition}
         
         
%         
%         We recall for the convenience of the reader that Massey-products have been introduced in litterature with the name of {\em Adjoint images} in the paper \cite{C-P_TheGriffiths_1995}. Anyway, in \cite{R_Infinitesimal_2008} a possible representation of these adjoint images as Massey products has been shown using the Dolbeault resolution to compute cohomology. This provides a good description in order to analyze them as second-order cohomological objects.
%         %		The choice of the name here is done in order to move to the study of higher dimensional vector spaces.
         %	
         Consider now the exact sequence 
         \begin{equation}\label{SeS-DeRhamOnFib}
         \xymatrix@!R{
         	{0}  & {f^*\omega_B}  & {\Omega^1_S}  & {\Omega^1_{S/B}}  & {0,}                                  & 
         	% orizzontal arrows
         	%%%	\ar"1,i";"j,k"
         	\ar"1,1";"1,2"\ar"1,2";"1,3"\ar"1,3";"1,4"\ar"1,4";"1,5"
         	\hole
         }
         \end{equation}
         defined by $f.$ By push forward, we also get the exact sequence
         \begin{equation}\label{SeS-DeRhamOnFibf_*}
         \xymatrix@!R{
         	{0}  & {f_*f^*\omega_B\simeq\omega_B}  & {f_*\Omega^1_S}  & {f_*\Omega^1_{S/B}}  & {(R^1f_*\cO_S)\otimes\omega_B}  , 
         	% orizzontal arrows
         	%%%	\ar"1,i";"j,k"
         	\ar"1,1";"1,2"\ar"1,2";"1,3"\ar"1,3";"1,4"\ar"1,4";"1,5"^-{\partial}
         	\hole
         }
         \end{equation}
         where $\partial$ is the connecting morphism which describes pointwise the IVHS (see details in the references given above, or also \cite{Gonz_OnDef_2016}). We denote by $\cK_\partial=\ker \partial$ the kernel of $\partial.$ We remark that, modulo torsion, the restriction of $\cK_\partial$ outside the singular locus of $f$ is locally free with fibre equal to the kernel of the cup product with the Kodaira-Spencer class of the (smooth) fibre $F_b.$ 
         Moreover, as a subsheaf of $f_*\Omega^1_{S/B},$ the sheaf $\cK_\partial$ injects in $f_*\omega_{S/B}$ outside the singular locus of $f.$ Indeed, we recall that the relation between the direct images $f_\ast\omega_{S/B}$ and $f_*\Omega^1_{S/B}$ of the the relative dualizing sheaf and the sheaf of the relative differential, respectively, is given by the following proposition (see \cite{Gonz_PhdTs_2013} or \cite{Gonz_OnDef_2016}).
         
         \begin{proposition}
         	
         	Let $f:S\to B$ be a complete fibration over the curve $B.$ Then $f_*\omega_{S/B}$ is a locally free sheaf of $\cO_{B}$-modules (namely, a vector bundle) of $\rk f_*\omega_{S/B}=g(F).$ Moreover, $f_*\Omega^1_{S/B}$ is a sheaf of the same rank and there is injection of sheaves
         	\begin{equation}\label{Mor-InjDifferentialsDualizing}
         	\xymatrix@!R{
         		{(f_*\Omega^1_{S/B})^{\vee\vee}}  &  {f_*\omega_{S/B}}
         		% orizzontal arrows
         		%%%	\ar"1,i";"j,k"
         		\ar@{^{(}->}"1,1";"1,2"^<<<<{\nu'}
         		\hole
         	}
         	\end{equation}
         	defined by the exact sequence
         	\begin{equation}\label{SeS-DirectImages1}
         	\xymatrix@!R{
         		{0}  & {{f^*\omega_B(Z_d)}_{|Z_d}}& {\Omega^1_{S/B}}  & {\omega_{S/B}}  & {{\omega_{S/B}}_{|Z_0}}  & {0,}                                  & 
         		% orizzontal arrows
         		%%%	\ar"1,i";"j,k"
         		\ar"1,1";"1,2"\ar"1,2";"1,3"\ar"1,3";"1,4"^{\nu}\ar"1,4";"1,5"\ar"1,4";"1,5"\ar"1,5";"1,6"
         		\hole
         	}
         	\end{equation}
         	where $Z=Z_d+Z_0$ is the singular locus of $f,$ with $Z_d$ a divisor and $Z_0$ supported on isolated points, and $(f_*\Omega^1_{S/B})^{\vee\vee}$ is the double dual sheaf which is locally free. In particular, when all the fibres of $f$ are reduced, the morphism $\nu$ is injective and we get 
         	\begin{equation}\label{Mor-InjDifferentialsDualizingIsolated}
         	\xymatrix@!R{
         		{(f_*\Omega^1_{S/B})}  &  {\omega_{S/B}} 
         		% orizzontal arrows
         		%%%	\ar"1,i";"j,k"
         		\ar@{^{(}->}"1,1";"1,2"^<<<<{\nu'}
         		\hole
         	}
         	\end{equation}
         	\end{proposition}
         Thus under the assumption of isolated singularities, $f_\ast \Omega^1_{S/B}$ is torsion free over a curve (i.e, locally free) and the previous proposition provides an injection $\nu':f_*\Omega^1_{S/B}\hookrightarrow f_*\omega_{S/B}.$ Consequently, $\cK_\partial$ is locally free and injects $\nu:\cK_\partial\hookrightarrow f_*\omega_{S/B}$ by restriction. In general, we have to consider $(\cK_{\partial})^{\vee\vee}$ instead of $\cK_\partial.$
       %By assumption, the sheaf of relative differentials $\Omega^1_{S/B}$ is torsion free (since singularities are isolated, as proven in \cite{Ser_Iso_1996}) and thus a vector bundle (since defined over a smooth curve, as proved in \cite{Fuj78a}). Then the same holds for $\cK_{\partial}.$ Moreover, we get an injection of sheaves $f_*\Omega^1_{S/B}\hookrightarrow f_*\omega_{S/B}$ and consequently also $\cK_\partial\hookrightarrow f_*\omega_{S/B}.$ 
         
         We introduce local families of Massey-products around a regular value $b$ of a fibration $f.$ Let $A$ be an open contractible subset of $B^0$ around $b$ (that is all the fibres $F_{b'}$ of $f$ over $b'\in A$ are smooth). Up to shrinking $A,$ we can take $\sigma\in \Gamma(A,T_B)$ a local trivialization of $T_B$ over $A.$ Then $\sigma$ defines a generator $\sigma_{b'}$ of $T_{\Delta_\epsilon,b'}^\vee,$ together with an isomorphism $\sigma_{b'}: \cO_{F_{b'}}\otimes  T^{\vee}_{\Delta_\epsilon,b'}\simeq \cO_{F_{b'}},$ where $\Delta_\epsilon= \spec \bC[\epsilon]/(\epsilon^2)$ is the the ring of dual numbers of the infinitesimal deformation induced on $F_{b'}$ by $f.$  Following the above construction we get a short exact sequence as \eqref{SeS-DeRhamOnFib},
         \begin{equation}\label{SeS-DeRhamOnSFibreb'}
         \xymatrix@!R{
         	{0}  & {\cO_{F_{b'}}}  & {\Omega^1_{S|F_{b'}}}  & {\omega_{F_{b'}}}  & {0}                                  & 
         	% orizzontal arrows
         	%%%	\ar"1,i";"j,k"
         	\ar"1,1";"1,2"\ar"1,2";"1,3"\ar"1,3";"1,4"\ar"1,4";"1,5"
         	\hole
         }
         \end{equation}
         with extension class $\xi_{b'}\in \Ext^1_{\cO_{F_{b'}}}(\omega_{F_{b'}},\cO_{F_{b'}})\simeq H^1(T_{F_{b'}})$ and then we can pointwise repeat the previous construction on a pair of local section $(s_1,s_2)$ in $\cK_\partial$ obtaining a {\em poitwise definition} of Massey-products on each smooth fibre of $f$ around $b,$ getting a section $\fm_{\sigma}(s_1,s_2)\in \Gamma(A, f_*\omega_{S/B})$ well defined modulo the $\cO_B(A)-$submodule $<s_1,s_2>_{\cO_B(A)}$ of $\Gamma(A,f_*\omega_{S/B})$ generated by $s_1,s_2.$ We remark that the definition works since the injection $\cK_{\partial}\hookrightarrow f_*\omega_{S/B}$ is compatible withe the restriction to each fibre over $b'\in A.$
         \begin{definition}\label{Def-MPAJLoc}
         	We say that {\em a local family of Massey-products} of the pair $(s_1,s_2)$ of sections in $\cK_\partial$ along $\sigma$ in $T_B$ is a section 
         	\begin{equation}\label{Mor-Mp/AjLoc}
         	\fm_{\sigma}(s_1,s_2)\in \Gamma(A, f_*\omega_{S/B}) \end{equation}
          defined modulo the $\cO_B(A)-$submodule $<s_1,s_2>_{\cO_B(A)}$ of $\Gamma(A,f_*\omega_{S/B})$  Moreover, we say it is Massey-trivial (equivalently that the pair that the pair $(s_1,s_2)$ is trivial along $\sigma$) if it is pointwise Massey-trivial (Definition \ref{Def-MTpair}).
          
         \end{definition}
%         \begin{definition}\label{Def-MPvanishing}
%         	\end{definition}
%%         	\textcolor{red}{TODO: check.. Me la da una sezione olomorfa?}
         	
         	We remark that up to the choice of a pair of liftings of $s_1,s_2\in \Gamma(A, f_*\omega_{S/B})$ in $\Gamma(A,f_*\Omega^1_S)$ the section $\fm_{\sigma}(s_1,s_2)\in \Gamma(A, f_*\omega_{S/B})$
         	is computed by the map 
         	\begin{equation} 		 
         	\bigwedge^2 \Gamma(A,f_*\Omega^1_S)\otimes \Gamma(A,T_B) \to\Gamma(A,f_*\omega_{S/B}),
         	\end{equation} 
         	where the last isomorphism is given by the projection formula. This is indeed a local version of the adjoint map. The above construction naturally globalizes to a sheaf map under the assumption of isolated singularities, where $\cK_\partial$ is locally free. We remark that one can repeat a similar construction on $\cK_\partial$ modulo torsion in the general set-up.
         	
         	Consider the short exact sequence
         	\begin{equation}\label{SeS-Kernel}
         	\xymatrix@!R{
         		{\zeta:} &{0}  & {\omega_B}  & {f_*\Omega^1_S}  & {\cK_\partial}  & {0} 
         		% orizzontal arrows
         		%%%	\ar"1,i";"j,k"
         		\ar"1,2";"1,3"\ar"1,3";"1,4"\ar"1,4";"1,5"\ar"1,5";"1,6"
         		\hole
         	}
         	\end{equation}
         	of locally free sheaves of $\cO_B-$modules defined by \eqref{SeS-DeRhamOnFibf_*} and we set $\zeta\in \Ext^1_{\cO_B}(\cK_\partial,\omega_B)$ the extension class.
         	We are able to prove the following.
         	
         	\begin{lemma}\label{Lem-SesSPLITKER}
         		Let $f:S\to B$ be complete fibration with isolated singularities, $\cK_\partial$ be the kernel of $\partial$ in the exact sequence \eqref{SeS-DeRhamOnFibf_*}. Then the short exact sequence \eqref{SeS-Kernel}, namely
         		\begin{equation}\label{SeS-KernelSplit<}
         		\xymatrix@!R{
         			{\zeta:} & {0}  & {\omega_B}  & {f_*\Omega^1_S}  & {\cK_\partial}  & {0}  , 
         			% orizzontal arrows
         			%%%	\ar"1,i";"j,k"
         			\ar"1,2";"1,3"\ar"1,3";"1,4"\ar"1,4";"1,5"\ar"1,5";"1,6"\ar@/^-1.1pc/"1,5";"1,4"_<<<\eta
         			\hole
         		}
         		\end{equation}
         		is split. 
         	\end{lemma}
         	\begin{proof}
         		We show that the coboundary map induced in cohomology $\delta: H^0(B,\cK_\partial )\to H^1(B,\omega_B)$ is the zero map. It is indeed the dual of the classifying map $\delta^\vee:H^0(B,\cO_B)\to H^1(B,\cK_\partial^\vee\otimes \omega_B)\simeq \Ext^1_{\cO_B}(\cK_\partial,\omega_B).$ We consider the map $H^1(B,\omega_B)\to H^1(B,f_*\Omega^1_S)$ induced by the long exact sequence in cohomology and we prove that it is an injection. First, observe that the pullback map $H^1(B,\omega_B)\to H^1(S,\Omega^1_S)$ is an injection, as it sends the class of a point $b$ on $B$ (which corresponds to a K\"{a}hler form) to the class of the fibre $F$ in $S,$ which is non zero. Then, also the map $H^1(B,\omega_B)\to H^1(B,f_\ast\Omega^1_S)$ is non zero, since it must factorize through the Leray spectral sequence 
         		\begin{equation}\label{Dia-PullBackLeray}
         		\xymatrix@!R{
         			& &  {H^1(B,\omega_B)}    &  & \\
         			{0}                &  {H^1(B,f_*\Omega^1_S)}    & {H^1(S,\Omega^1_S)}  &    {H^0(B,R^1f_*\Omega^1_S).}      
         			%\ar@{-}`r[d]`[d]^\delta[d] % curved arrow 1 	
         			% 	&                          &     & {f_*\omega_{S/B}} ,
         			% 	%\ar@{}+<0.6cm,0cm>="p1"  %intersection point 1
         			% 	&   &     
         			% 	% orizontal arrows
         			%%%	\ar"1,i";"j,k"
         			\ar"2,1";"2,2"\ar"2,2";"2,3"\ar"2,3";"2,4"
         			%%%	\ar"2,i";"j,k"
         			% vertical arrows
         			%%%	\ar"1,i";"j,k"
         			\ar@{^{(}->}"1,3";"2,3"   
         			% 	\ar@/_2pc/@{^{(}->}
         			% 	"1,4";"3,4"_>>>>>>{i_{\nabla}}
         			%%%	\ar"2,i";"j,k"
         			\ar@{^{(}->}"1,3";"2,2" 
         			% diagonal arrow, with 1 hole
         			%	\ar@{-}"p1"|!{"2,3";"4,5"}
         			\hole
         		}
         		\end{equation}
         		\end{proof}
         		Let us now introduce the morphism
         		\begin{equation}\label{Mor-MPRelativeForms}
         		\xymatrix@!R{
         			{\wedge : \bigwedge^2f_*\Omega^1_S\otimes T_B}                &  {f_*\omega_{S/B},}    
         			% 	% orizontal arrows
         			%%%	\ar"1,i";"j,k"
         			\ar"1,1";"1,2"
         			\hole
         		}
         		%\wedge : \bigwedge^2f_*\Omega^1_S\otimes T_B\to f_*\omega_{S/B}
         		\end{equation} 
         		defined by the morphism  $\bigwedge^2f_*\Omega^1_S\to f_*\bigwedge^2\Omega^1_S$ and then twisted by $T_B.$ We remark that by the projection formula we have  $f_*\bigwedge^2\Omega^1_S\otimes T_B\simeq f_*\omega_{S/B}.$
         	Then by looking to the short exact sequence
         	\begin{equation}\label{SeS-wedge2KernelSPlit}
         	\xymatrix@!R{
         		{0} & {\omega_B\otimes \cK_\partial}  & {\bigwedge^2f_*\Omega^1_S}  & {\bigwedge^2\cK_\partial}  & {0}  
         		% orizzontal arrows
         		%%%	\ar"1,i";"j,k"
         		\ar"1,1";"1,2"\ar"1,2";"1,3"\ar"1,3";"1,4"\ar"1,4";"1,5"
         		\hole
         	}
         	\end{equation}
         	induced by $\zeta,$ which is split by Lemma \ref{SeS-KernelSplit<}, we get an injection $\bigwedge^2\cK_\partial \hookrightarrow\bigwedge^2f_*\Omega^1_S$
         	which factorizes through the morphism \eqref{Mor-MPRelativeForms}. Thus we obtain
         	\begin{equation}\label{Mor-MPRelativeFormsKernel}
         	\xymatrix@!R{
         		{\wedge : \bigwedge^2\cK_\partial\otimes T_B}                &  {f_*\omega_{S/B}.}    
         		% 	% orizontal arrows
         		%%%	\ar"1,i";"j,k"
         		\ar"1,1";"1,2"
         		\hole
         	}
         	%\wedge : \bigwedge^2f_*\Omega^1_S\otimes T_B\to f_*\omega_{S/B}
         	\end{equation} 
         	We remark that a direct computation shows that $\omega_B\otimes \cK_\partial$ has image via the morphism $\wedge$ in $\cK_\partial\hookrightarrow f_*\omega_{S/B}.$  
         	The Massey-product $\fm_{\sigma}(s_1,s_2)\in \Gamma(A, f_*\omega_{S/B}) $ of the pair of sections $s_1,$ $s_2$ of $\cK_\partial$ over a subset $A$ of $B$ is computed by \eqref{Mor-MPRelativeFormsKernel} modulo the the $\cO(A)-$submodule $<s_1,s_2>$ of $\Gamma(A, f_*\omega_{S/B}).$    
         	\begin{remark}\label{Rem-MTandRestrictions} Let $A$ be a connected open subset of $B.$ Then it is equivalent that a pair $(s_1,s_2)$ of sections in $\Gamma(A, \cK_\partial)$ is Massey-trivial (Definition \ref{Def-MPAJLoc}) and that the restriction of $s_1$ and $s_2$ to the general point $b$ is Massey-trivial.
         	\end{remark} 
			    	
         	\subsection{Massey-trivial products on sections of $\cK_\partial$ and liftings.}
         	Assume $\cK_\partial$ has rank greater than $2.$ Let $A$ be an open subset of $B$ and $W\subset\Gamma(A,\cK_\partial )$ be a subspace of sections over $A$ such that $\dim_\bC W\geq 2.$ We study a vanishing condition of Massey-products on subspaces of sections in $\cK_\partial$ in relation with suitable liftings in $f_*\Omega_S.$ The vanishing property is the following.
   	       	 \begin{definition} \label{Def-MTSubspSec}A subspace $W\subset\Gamma(A,\cK_\partial )$ is {\em Massey-trivial } if each pair of sections on $W$ is Massey-trivial (Definition \ref{Def-MPAJLoc}).
         	 \end{definition}
         	 It is clear that the splitting \eqref{SeS-KernelSplit<} lifts $W$ to $f_*\Omega^1_S$ and then we can apply the morphism \eqref{Mor-MPRelativeFormsKernel}, which uses the above liftings to compute the Massey-products of each pair of $W.$ We get sections of $f_*\omega_{S/B}$ and, by definition of Massey-trivial pairs, they lie in $<s_1,s_2>_{\cO_B(A)}.$ In the following we are able to prove that one can choose suitable liftings on $f_*\Omega^1_S$ with wedge zero. We remark that these can be different from the ones given by the splitting fixed  in \eqref{SeS-KernelSplit<}, which is actually far away to be unique.
         	 
         	 	
\begin{proposition} \label{Prop-MTtoIsotropic} Let $A$ be an open set of $B,$  $i_A: A\hookrightarrow B$ be the inclusion and $W\subset \Gamma(A,\cK_\partial)$ be a Massey-trivial subspace of sections of $\cK_\partial$ over $A.$ Assume that the evaluation map $W\otimes \cO_A\to i_A^*\cK_\partial$ defines an injective map of vector bundles. Then there exists a unique $\widetilde{W}\subset H^0(A,f_*\Omega^1_S)$ which lifts $W$ to $f_\ast\Omega^1_S$ and such that    $\bigwedge ^2\widetilde{W}\to \Gamma(A, f_\ast \omega_S)$ is the  zero map, that is the map \eqref{Mor-MPRelativeForms}
         	 		\begin{equation}\label{Mor-MPRlift}
         	 		\xymatrix@!R{
         	 			{\wedge_A : \bigwedge^2\widetilde{W}\otimes i^\ast_AT_B}                &  {i^\ast_Af_\ast\omega_{S/B},}    
         	 			% 	% orizontal arrows
         	 			%%%	\ar"1,i";"j,k"
         	 			\ar"1,1";"1,2"
         	 			\hole
         	 		}
         	 		%\wedge : \bigwedge^2f_*\Omega^1_S\otimes T_B\to f_*\omega_{S/B}
         	 		\end{equation} 
        	 		is the null map.          	 		
%         	 		\begin{itemize}
%         	 			\item[(M1)] the map $\widetilde{W}\to W$ given by \ref{Dia-HolDeRhams} is an isomorphism;
%         	 			\item[(M2)]  $\widetilde{W}$ is isotropic with respect to $\wedge : \bigwedge^2H^0(f^{-1}(A),\Omega^1_{S,d})\to H^0({\omega_S}_{|A}) .$
%         	 		\end{itemize} 
         	 	\end{proposition}	
         	 	\begin{proof}  We can prove the result on an open coverings, which is enough since the lifltings will glue by unicity. Assume $A$ to be an open coordinate set. Let $\tau\in i^\ast_AT_B\simeq T_A$ be a trivialization of $T_A$  and set $\beta \in i^\ast_AT^{\vee}_B\simeq i^\ast_A\omega_B\simeq \omega_B$ its dual (that is, $\tau\cdot \beta=1$). 
		By composition of $W\otimes \cO_A\to i_A^\ast\cK_\partial$ with the splitting $\cK_\partial \to f_\ast \Omega^1_S,$ we obtain a lifting map $\rho : W\to \Gamma(A,  f_\ast \Omega^1_S).$
		  Let $V=\rho(W)$ be its image and let $v_i=\rho(s_i)$ be the images of a base $\{s_1,v_2,\dots,s_n\}$ of $W.$ 
		  Since the Massey products are zero on any pairs of section of $W,$ we have
		  $$\wedge_A (v_1\wedge v_i,\sigma)= f_is_1+g_is_i$$  and also
		  $$\wedge_A ( v_1 \wedge \sum _2^nv_i,\sigma)=f_0s_1+g_0 (\sum _2^ns_i),$$
		  where the $f_i$ and the $g_i$ are holomorphic function on $A.$ Now assume $n=2$ and set
	          $\tilde v_1=v_1-g_2\beta$ and
		 $\tilde v_2= v_2-f_2\beta.$ Then $\wedge_A (\tilde v_1\wedge  \tilde v_2,\sigma )=0$ and thus 
		 $\tilde v_1\wedge  \tilde v_2=0\in\Gamma(A, f_\ast \omega_S) .$ 
		We remark that  the unicity of the liftings $ \tilde v_1$ and $\tilde v_2$ follows at once.
		 
	Now we assume $n>2$ and, by induction, that the proposition holds for $k<n.$ We apply this to the space $W'$
	generated by $\{s_1\dots s_{n-1}\}$ and we find liftings $\tilde v_i\ i=1,\dots n-1$ such that $\tilde v_i\wedge \tilde v_j=0$ for
	$i, j$ smaller than $n.$	We also have that $\wedge_A (\tilde v_1\wedge v_n,\sigma)= es_1+fs_n$ and 
	$$\wedge_A (\tilde v_1 \wedge( \sum _2^{n-1}\tilde v_i+ v_n),\sigma)=\wedge_A  (\tilde v_1\wedge v_n,\sigma)=
	ls_1+m (\sum _2^ns_i).$$
	Therefore, we get  $es_1+fs_n=ls_1+ m(\sum _2^ns_i)$ and  since the sections are independent we conclude that $e=l$ and $f=0=m.$
	As a consequence, the sections $\tilde v_n=v_n-e\beta$ are such that $\wedge_A ( \tilde v_1\wedge\tilde v_n,\sigma)=0$ and thus 
	$\tilde v_1\wedge\tilde v_n=0.$ The condition $\tilde v_i\wedge\tilde v_n=0$ and the unicity of the lifting follow immediately.
	
		          	 		 
         	 	\end{proof}         	 	

      	  
\section{The local system $\bU$ and liftings to the sheaf of closed holomorphic forms on $S.$}\label{Sec-LiftingsOnU}
 
 In this section we study the relation between the unitary flat bundle $\cU$ in the second Fujita decomposition of $f:S\to B$ and the subsheaf $\Omega^1_{S,d}\subset \Omega^1_S$ of the closed holomorphic $1$-forms on $S.$ In subsection \ref{SubSec-RelHoldeRham} we introduce a suitable short exact sequence which allows to describe $\bU$ as a local system contained in the sheaf $R^1f_*\bC$ (which is not a local system as previously recalled). Then, in Lemma \ref{Lem-CharFlatSecN} of subsection \ref{SubSec-UandTubForms}, we prove that the local system $\bU$ underlying $\cU$ can be described in terms of local sections of $f_*\Omega^1_{S,d},$ that is in terms of holomorphic forms on the fibres of $f$ which locally attach to closed local holomorphic forms on $S.$ Moreover, suggested by a lifting of $\cU$ on $\Omega^1_S$ contained in the work \cite{ChenLuZu_OnTheOort_2016}, we prove that the sequence involved in Lemma \ref{Lem-CharFlatSecN} is split (Lemma \ref{Lem-SesUSPLIT}), providing a special a lifting for $\bU.$
% As an important application for the aim of the paper, we analyze the {\em Massey-trivial} case.

% 
% Let $f:S\to B$ be a fibration according to subsection \ref{SubSec-PrelLocSyst}. Recall that $B_\circ:=f(\Crit f)\subset B$ denotes the set of critical values of $f,$  $D:=f^{-1}(B_\circ)$ is the divisor of the singular fibres and we put $Z_p:=\Crit f.$ In this section we assume $f$ to be semistable, namely $D=\sum_i F_i$ is a normal crossing divisor and we also assume that it does not contain any non-reduceed components.
  
 \subsection{ Relative holomorphic de-Rham: a useful short exact sequence.}\label{SubSec-RelHoldeRham} 
 Given a fibration $f:S\to B$ there is a suitable short exact sequence which can be constructed by comparing the holomorphic de Rham sequences of the surface $S$ and the base $B.$  
% We start with the construction of a short exact sequence naturally attached to a fibration $f:S\to B$ we will call {\em Relative holomorphic de-Rham.} 

 
 Let us consider the holomorphic de-Rham sequence on $S$
 \begin{equation}\label{Ses-HoldeRahmS}
 \xymatrix@!R{
 	0  &    {\bC_S}    &    {\cO_S}    & {\Omega^1_{S,d}}   &   {0} ,    
 	% orizzontal arrows 
 	%%%\ar"1,i";"j,k"
 	\ar"1,1";"1,2"\ar"1,2";"1,3"\ar"1,3";"1,4"^\de\ar"1,4";"1,5"
 	% vertical arrows
 	\hole
 }	 
 \end{equation}
 where $\Omega^1_{S,d}$ denotes the sheaf of $d$-closed holomorphic $1$-forms on $S.$ Then, we get the exact sequence
 \begin{equation}\label{Ex-HoldeRahmf_*S}
 \xymatrix@!R{
 	{0}  & {f_*\bC_S}  & {f_*\cO_S}  & {f_*\Omega^1_{S,d}}  & {R^1f_*\bC_S}      & {R^1f_*\cO_S.}  & 
 	% orizzontal arrows
 	%%%	\ar"1,i";"j,k"
 	\ar"1,1";"1,2"\ar"1,2";"1,3"\ar"1,3";"1,4"^{\de}\ar"1,4";"1,5"\ar"1,5";"1,6"
 	% diagonal arrow, with 1 hole
 	%	\ar@{-}"p1"|!{"2,3";"4,5"}
 	\hole
 }	 
 \end{equation}
 We compare it with the holomorphic de-Rham sequence on $B$
 \begin{equation}\label{Ses-HoldeRahmB}
 \xymatrix@!R{
 	0  &    {\bC_B}    &    {\cO_B}    & {\omega_B}   &   {0} ,  
 	% orizzontal arrows 
 	%%%\ar"1,i";"j,k"
 	\ar"1,1";"1,2"\ar"1,2";"1,3"\ar"1,3";"1,4"^\de\ar"1,4";"1,5"
 	% vertical arrows
 	\hole
 }	 
 \end{equation}
 using the natural morphisms $\bC_B\to f_*\bC_S$ and $\cO_B\to f_*\cO_S$ induced by $f,$ which are both isomorphisms in this case. We obtain a diagram
 \begin{equation}\label{Dia-HolDeRhams}
 \xymatrix@!R{
 	{0}  & {f_*\bC_S}  & {f_*\cO_S}  & {f_*\Omega^1_{S,d}}  & {R^1f_*\bC_S}      & {R^1f_*\cO_S}  & \\
 	{0}  & {\bC_B}      & {\cO_B}  & {\omega_B}  & 0,         &                                     & 
 	% orizzontal arrows
 	%%%	\ar"1,i";"j,k"
 	\ar"1,1";"1,2"\ar"1,2";"1,3"\ar"1,3";"1,4"^{\de}\ar"1,4";"1,5"\ar"1,5";"1,6"
 	%%%\ar"2,i";"j,k"
 	\ar"2,1";"2,2"\ar"2,2";"2,3"\ar"2,3";"2,4"^{\de}\ar"2,4";"2,5"
 	% vertical arrows
 	%%%	\ar"2,i";"j,k"
 	\ar"2,2";"1,2"^{\parallel} \ar"2,3";"1,3"^{\parallel} \ar@{^{(}->}"2,4";"1,4" 
 	% diagonal arrow, with 1 hole
 	%	\ar@{-}"p1"|!{"2,3";"4,5"}
 	\hole
 }
 \end{equation}
 which induces an injective morphism $\omega_B\hookrightarrow f_*\Omega^1_{S,d}$ together with the short exact sequence
 \begin{equation}\label{SeS-HolRelDeRham}
 \xymatrix@!R{
 	{0}  & {\omega_B}  & {f_*\Omega^1_{S,d}}  & {\widehat{D}}  & {0,} 
 	% orizzontal arrows
 	%%%	\ar"1,i";"j,k"
 	\ar"1,1";"1,2"\ar"1,2";"1,3"\ar"1,3";"1,4"\ar"1,4";"1,5"
 	% vertical arrows
 	\hole
 }
 \end{equation}
 where $\widehat{D}$ denotes the image of the morphism $f_*\Omega^1_{S,d} \to R^1f_*\bC_S.$ We will call the above sequence {\em Relative holomorphic de-Rham sequence}. 
 
 We analyze the subsheaf $\widehat{D}\hookrightarrow R^1f_*\bC,$ which turns out to be strictly connected with $\bU.$
\begin{lemma}\label{Lem-hatDInj} There is an injection of sheaves $i_{\mtin{$\widehat{D}$}}:\widehat{D}\hookrightarrow \cK_{\partial},$ where $\cK_{\partial}$ was defined in (\ref{SeS-DeRhamOnFibf_*}).
	\end{lemma} 
\begin{proof} Consider the natural injection of sheaves  $ \Omega^1_{S,d}\stackrel{i_d}{\hookrightarrow}\Omega^1_S$ and compare the Relative holomorphic de-Rham sequence (\ref{SeS-HolRelDeRham})
	\begin{equation*}
	\xymatrix@!R{
		{0}  & {\omega_B}  & {f_*\Omega^1_{S,d}}  & {\widehat{D}}  & {0} 
		% orizzontal arrows
		%%%	\ar"1,i";"j,k"
		\ar"1,1";"1,2"\ar"1,2";"1,3"\ar"1,3";"1,4"\ar"1,4";"1,5"
		% vertical arrows
		\hole
	}
	\end{equation*} with sequence (\ref{SeS-DeRhamOnFibf_*})
	\begin{equation*}
	\xymatrix@!R{
		{0}  & {f_*f^*\omega_B\simeq\omega_B}  & {f_*\Omega^1_S}  & {f_*\Omega^1_{S/B}}  & {(R^1f_*\cO_S)\otimes\omega_B}  , 
		% orizzontal arrows
		%%%	\ar"1,i";"j,k"
		\ar"1,1";"1,2"\ar"1,2";"1,3"\ar"1,3";"1,4"\ar"1,4";"1,5"^-{\partial}
		\hole
	}
	\end{equation*}
	using the induced morphism on the direct image sheaves.
%	the injection of sheaves  $ \Omega^1_{S,d}\stackrel{i_d}{\hookrightarrow}\Omega^1_S.$ 
	We get a diagram 
	\begin{equation}\label{Dia-LinkDiagramShort}
	\xymatrix@!R{
		0 &       {\omega_B}       &   {f_*\Omega^1_{S,d}}                &  {\widehat{D}}       &   {0}     &        \\
		0   & {f_*f^*\omega_B}    & {f_*\Omega^1_{S}}  & {f_*\Omega^1_{S/B}}  %\ar@{}[l]+<1cm,0cm>="p1"
		%\ar@{-}`r[d]`[d]^\delta[d] % curved arrow 1 	
		& {R^1f_*\cO_S\otimes \omega_B,} &       &       
		% 	&                          &     & {f_*\omega_{S/B}} ,
		% 	%\ar@{}+<0.6cm,0cm>="p1"  %intersection point 1
		% 	&   &     
		% 	% orizontal arrows
		%%%	\ar"1,i";"j,k"
		\ar"1,1";"1,2"\ar"1,2";"1,3"\ar"1,3";"1,4"\ar"1,4";"1,5"
		%%%	\ar"2,i";"j,k"
		\ar"2,1";"2,2"\ar"2,2";"2,3"\ar"2,3";"2,4"\ar"2,4";"2,5"
		% vertical arrows
		%%%	\ar"1,i";"j,k"
		\ar"1,2";"2,2"^{  \parallel}       \ar@{^{(}->}"1,3";"2,3"^{i_{\de}}   \ar@{^{(}->}"1,4";"2,4"^{i_{\mtin{$\widehat{D}$}}}   \ar"1,5";"2,5" 
		% 	\ar@/_2pc/@{^{(}->}
		% 	"1,4";"3,4"_>>>>>>{i_{\nabla}}
		%%%	\ar"2,i";"j,k"
		\ar"2,4";"2,5"^-{\partial}  
		%	\ar@{^{(}->}"2,4";"3,4"^{i_2} 
		% diagonal arrow, with 1 hole
		%	\ar@{-}"p1"|!{"2,3";"4,5"}
		\hole
	}
	\end{equation}
	defining the injection $i_{\mtin{$\widehat{D}$}}:\widehat{D}\hookrightarrow \cK_{\partial}$ as claimed. 
	
	\end{proof}	
	\begin{lemma}\label{Lem-cDLocSys} Let $f:S\to B$ be a semistable fibration. Then $\widehat{D}$ is a local system over $B$  and this case we denote it with $\bD.$  Moreover, the stalk $D$ of $\bD$ is isomorphic over the general fiber $F$ to the greatest subspace of $H^{1,0}(F)\subset H^1(F,\bC)$ defining a local system over $B.$
		\end{lemma}
%	QQ: serve?	Let $B_\circ$ denotes the singular locus of $f.$$f:S\to B$ be a semistable fibration.
	\begin{proof} Let $B_0$ be the singular locus of $f$ and $j: B^0=B\setminus B_0\to B$ be the natural injection. Let  $k_{\mtin{$\widehat{D}$}}:\widehat{D}\hookrightarrow R^1f_*\bC_S$ be a injection of sheaves defined in \eqref{Dia-HolDeRhams} and  $\alpha:R^1f_*\bC\to j_*j^* R^1f_*\bC $ be the morphism introduced in subsection \ref{SubSec-Prel-LocSystOnFibr} (simply given by restriction). The morphism $\alpha$ is an isomorphism, whenever $f$ is semistable (see Lemma \ref{Lem-LocInvIso}) and $j^* R^1f_*\bC $ is always a local system of stalk $H^1(F,\bC),$ where $H^1(F,\bC)$ is the first cohomology group of the general fiber $F$ of $f.$
%%%			\begin{equation}\label{Dia-LinkDiagramDR1}
%%%			\xymatrix@!R{
%%%				0 &       {\omega_B}       &   {f_*\Omega^1_{S,d}}                &  {\widehat{D}}       &   {0}     &        \\
%%%				0   & {\omega_B}    & {f_*\Omega^1_{S,d}}  & {R^1f_*\bC_S}  %\ar@{}[l]+<1cm,0cm>="p1"
%%%				%\ar@{-}`r[d]`[d]^\delta[d] % curved arrow 1 	
%%%				& {R^1f_*\cO_S.} &       &       
%%%				% 	&                          &     & {f_*\omega_{S/B}} ,
%%%				% 	%\ar@{}+<0.6cm,0cm>="p1"  %intersection point 1
%%%				% 	&   &     
%%%				% 	% orizontal arrows
%%%				%%%	\ar"1,i";"j,k"
%%%				\ar"1,1";"1,2"\ar"1,2";"1,3"\ar"1,3";"1,4"\ar"1,4";"1,5"
%%%				%%%	\ar"2,i";"j,k"
%%%				\ar"2,1";"2,2"\ar"2,2";"2,3"\ar"2,3";"2,4"\ar"2,4";"2,5"
%%%				% vertical arrows
%%%				%%%	\ar"1,i";"j,k"
%%%				\ar"1,2";"2,2"^{  \parallel}       \ar"1,3";"2,3"^{  \parallel}   \ar@{^{(}->}"1,4";"2,4"^{k_{\mtin{$\widehat{D}$}}}   \ar"1,5";"2,5" 
%%%				% 	\ar@/_2pc/@{^{(}->}
%%%				% 	"1,4";"3,4"_>>>>>>{i_{\nabla}}
%%%				%%%	\ar"2,i";"j,k"
%%%				\ar"2,4";"2,5"  
%%%				%	\ar@{^{(}->}"2,4";"3,4"^{i_2} 
%%%				% diagonal arrow, with 1 hole
%%%				%	\ar@{-}"p1"|!{"2,3";"4,5"}
%%%				\hole
%%%			}
%%%		\end{equation}
%			According to notations, let $f^\circ : S^\circ\to B^\circ$ be the fibration induced by restriction on $B^\circ.$ Recall that since $f^\circ$ is smooth, then $R^1f^\circ_*\bC_{S^\circ}$ is a local system over $B^\circ$ of stalk $H^1(F,\bC),$ where $F$ is the general fiber of $f.$ 
%We recall that $j^*R^1f_*\bC$ is the local system of stalk the first cohomology group $H^1(F,\bC)$ of the general fiber $F$ of $f.$ 

We consider the injective morphism  $\alpha_{\mtin{$\widehat{D}$}}: \widehat{D}\to j_*j^*R^1f_*\bC_S$  given by composition as follows
\begin{equation}\label{Dia-LinkDiagramDR12}
\xymatrix@!R{
	&   {\widehat{D}}                &  {R^1f_*\bC_{S}}       &          \\
	{}   & {}    & {j_*j^*R^1f_*\bC_S.}  & 
	%orizzontal arrows 
	\ar"1,2";"1,3"^{k_{\mtin{$\widehat{D}$}}}
	\ar"1,3";"2,3"^{\alpha}
	\ar"1,2";"2,3"^{\alpha_{\mtin{$\widehat{D}$}}}  
	%	\ar@{-}"p1"|!{"2,3";"4,5"}
	\hole
}
\end{equation}
and we identify $\widehat{D}$ with its image under the above morphism. Then $j^*\widehat{D}$ is a local subsystem of $j^*R^1f_*\bC$ with stalk over each value $b\in B^0$ a vector subspace $D_b$ of $H^{1,0}(F_b)\hookrightarrow H^1(F_b,\bC)\simeq H^1(F,\bC).$ We claim that the direct image $j_*j^*\widehat{D}\simeq \widehat{D}$ is a local system over $B.$ To prove this it is enough to look at the local monodromies around singular values and recall that under the assumption of semistable singularities they act trivially on $D_b$ (see subsection \ref{SubSec-Prel-LocSystOnFibr}). Thus $j^*\widehat{D}$ extends to a local system on $B.$ Moreover, it defines the greatest local subsystem of $j^*R^1f_*\bC$ with stalk $D$  isomorphic over the general fiber $F$ to a subspace of $H^{1,0}(F)\subset H^1(F,\bC).$ Otherwise, one should have a local subsystem $\bD'\hookrightarrow j^*R^1f_*\bC$ together with a not identically zero map $ j_*\bD' \to R^1f_*\cO_S.$  This is impossible since the stalk of $R^1f_*\cO_S$ over a regular value $b\in B$ is $H^{0,1}(F_b).$ %We prove tha $\alpha_{\mtin{$\widehat{D}$}}: \widehat{D}\to j_*j^*R^1f_*\bC_S$ is an injection. Consider the diagram
	%\begin{equation}\label{Dia-LinkDiagramDD}
	%\xymatrix@!R{
		%{f_*\Omega^1_{S,d}}                &  {\widehat{D}}       &    {0}      \\
	%	{}    & {j_*j^*R^1f_*\bC_S}  & 
		%\ar@{-}`r[d]`[d]^\delta[d] % curved arrow 1 	
		% 	&                          &     & {f_*\omega_{S/B}} ,
		% 	%\ar@{}+<0.6cm,0cm>="p1"  %intersection point 1
		% 	&   &     
		% 	% orizontal arrows
		%%%	\ar"1,i";"j,k"
		%\ar"1,1";"1,2"\ar"1,2";"1,3"
		%%%	\ar"2,i";"j,k"
		% vertical arrows
		%%%	\ar"1,i";"j,k"
    	%\ar"1,2";"2,2"^{\alpha_{\mtin{$\widehat{D}$}}}   
		% 	\ar@/_2pc/@{^{(}->}
		% 	"1,4";"3,4"_>>>>>>{i_{\nabla}}
		%%%	\ar"2,i";"j,k"
		%	\ar@{^{(}->}"2,4";"3,4"^{i_2} 
		% diagonal arrow, with 1 hole
    %	\ar"1,1";"2,2"^{\beta_{\mtin{$\widehat{D}$}}}  
		%	\ar@{-}"p1"|!{"2,3";"4,5"}
		%\hole
	%}
	%\end{equation}
		
		%where the morphism $\beta_{\mtin{$\widehat{D}$}}:f_*\Omega^1_{S,d}\to j_*j^*R^1f_*\bC_S$ is given by restriction of holomorphic forms defined over open neighborhoods around the general fiber $F,$ which gives a cohomology class in $H^1(F,\bC)\cap H^{1,0}(F).$ Note that a non zero element in the stalk ${f_*\Omega^1_{S,d}}_{b_0}$ of a singular value $b_0$ must have non zero image. Otherwise we should have a non zero holomorphic form on $F$ leading the zero class in de-Rham which is not possible. Thus $\beta_{\mtin{$\widehat{D}$}}$ is injective, and as a consequence, the same holds for $\alpha_{\mtin{$\widehat{D}$}}.$ 
		
%		Finally, we prove that $\widehat{D},$ which we allow to identify with its image via $\alpha_{\mtin{$\widehat{D}$}},$ is a local system, namely that $\widehat{D}$ is locally isomorphic to a constant sheaf. Note that the stalk of $\widehat{D}$ is isomorphic to a vector subspace of $H^{1,0}(F)$ and has an hermitian structure given by restriction of the Hodge metric on $H^{1,0}(F).$ Then, the proof follows just using the local monodromies on the stalk $H^1(F,\bC)$ of the local system $j^*R^1f_*\bC,$ which acts trivially on $H^{1,0}(F)$ under the assumption of semistability on $f$ as recalled in subsection \ref{SubSec-Prel-LocSystOnFibr}. 
%%		In fact, when $f$ is semistable, by the Picard Lefshetz' formula [CIT] local monodromy is unipotent.  Thus trivial, since it is also unitary using of the invariance of the \textcolor{red}{TODO:sistemare Hodge metric} with respect to local monodromy on $H^{1,0}(F).$ 
%		This allows to conclude that the stalk on a singular value of $f$ has the same rank of the stalk on a regular value near it and thus $\widehat{D}$ is locally trivial.
%%		This follows from the fact that $\beta_{\mtin{$\widehat{D}$}}$ is injective and then the same argument similar on holomorphic non-zero forms used before applied to non-zero section of $\widehat{D}$ shows that the stalk of $\widehat{D}$ on each point is exactly given by the stalk of the image of $\beta_{\mtin{$\widehat{D}$}},$ which remains locally constant. In fact,  around each point $b$ of the base, smooth or not, we start from holomorphic forms defined over the inverse image of a sufficiently small neighborhood of it and then we restric forms to the central fiber taking their de-Rham cohomology class which is uniquely determined because holomorphic forms are primitive and locally constant because $f$ is a fibration.
 
% \begin{equation}\label{Dia-LinkDiagramDR12}
% \xymatrix@!R{
% 	&   {f_*\Omega^1_{S,d}}                &  {R^1f_*\bC_{S}}       &          \\
% 	{}   & {}    & {j_*j^*R^1f_*\bC_S}  & 
% 	%\ar@{-}`r[d]`[d]^\delta[d] % curved arrow 1 	
% 	% 	&                          &     & {f_*\omega_{S/B}} ,
% 	% 	%\ar@{}+<0.6cm,0cm>="p1"  %intersection point 1
% 	% 	&   &     
% 	% 	% orizontal arrows
% 	%%%	\ar"1,i";"j,k"
% 	\ar"1,2";"1,3"
% 	\ar"1,3";"2,3"
% 	\ar"1,2";"2,3"^\beta  
% 	%	\ar@{-}"p1"|!{"2,3";"4,5"}
% 	\hole
% }
% \end{equation}
% which restricts to a morphism $ \beta_{\mtin{$\widehat{D}$}}:\widehat{D}\to j_*j^*R^1f_*\bC_S.$

%
%\textcolor{blue}{\Large OLD}
%Consider the local system $j^*R^1f_*\bC$ of stalk the first cohomology group $H^1(F,\bC)$ of the general fiber $F$ of $f.$ Recall that on semistable fibration it hold the local invariant property, stating that the adjuction homorphism   
%\begin{equation}
%\alpha_1:R^1f_*\bC\to j_*j^* R^1f_*\bC 
%\end{equation} 
%is surjective [CIT].
%By comparing, we get a commutative diagram
%			\begin{equation}\label{Dia-LinkDiagramDR12}
%			\xymatrix@!R{
%				    &   {f_*\Omega^1_{S,d}}                &  {R^1f_*\bC_{S}}       &          \\
%				{0}   & {\widehat{D}}    & {j_*j^*R^1f_*\bC_S}  & \\ 
%				   & {0}    & {0}  & %\ar@{}[l]+<1cm,0cm>="p1"
%				%\ar@{-}`r[d]`[d]^\delta[d] % curved arrow 1 	
%				% 	&                          &     & {f_*\omega_{S/B}} ,
%				% 	%\ar@{}+<0.6cm,0cm>="p1"  %intersection point 1
%				% 	&   &     
%				% 	% orizontal arrows
%				%%%	\ar"1,i";"j,k"
%				\ar"1,2";"1,3"
%				%%%	\ar"2,i";"j,k"
%				\ar"2,1";"2,2"\ar"2,2";"2,3"^<<<<<{dR_{\mtin{$\widehat{D}$}}}
%				% vertical arrows
%				%%%	\ar"1,i";"j,k"
%				\ar"1,2";"2,2" \ar"1,3";"2,3" 
%				\ar"2,2";"3,2"\ar"2,3";"3,3"  
%				% 	\ar@/_2pc/@{^{(}->}
%				% 	"1,4";"3,4"_>>>>>>{i_{\nabla}}
%				%%%	\ar"2,i";"j,k"
%				%	\ar@{^{(}->}"2,4";"3,4"^{i_2} 
%				% diagonal arrow, with 1 hole
%				\ar"1,2";"2,3"^{\mtin{dR}}  
%				%	\ar@{-}"p1"|!{"2,3";"4,5"}
%				\hole
%			}
%			\end{equation}
%			given an injective morphism $dR_{\mtin{$\widehat{D}$}}:\widehat{D}\to j_*j^*R^1f_*\bC_S$ induced by the natural map $dR$ sending a holomorphic form on the fiber $F$ to its de-Rham cohomology class. Injectivity follows since a non zero element in the stalk ${f_*\Omega^1_{S,d}}_{b_0}$ over $b_0$ must have non zero image. Otherwise we should have a non zero holomorphic form on $F$ leading the zero class in de-Rham.
%			Moreover, using the previous argument together with the local invariant cycle property says that $\widehat{D}$ is a local system over $B.$   
%			\textcolor{red}{TODO:check esattezza}
%			\textcolor{red}{mmmm Devo davvero restringere per mostrare che $f_*\Omega^1_{S,d}$ induce un sottosistema locale ?! non penso, penso sia il local invariant a dirmi di non farlo... to be finished}
%			
%			
%			
%		
	   
	   
	   
		
		\end{proof}
%%%			{\em Notations.}\label{Not-D} Whenever $f$ is semistable and consequently $\widehat{D}$ is a local system, we will denote such sheaf with $\bD$ and set $(\cD=\bD\otimes \cO_B,\nabla_{\mtin{$\cD$}})$ the associated flat vector bundle, according to conventions given in subsection \ref{Sec-LocSyst}.\textcolor{red}{TOASK: per la rappresentazione.. Diciamo qui che è unitaria o lo diamo per sottointeso?}
		
		\subsection{The local system $\bU$ and the sheaf $f_*\Omega^1_{S,d}$}\label{SubSec-UandTubForms}In this subsection we describe local sections of $\bU$ in terms of the sheaf $f_*\Omega^1_{S,d}.$ The result, which we will call {\em Lifting lemma}, follows immediately from the preparation done in the previous section. It is enough to remark that the local system $\bD$ in Lemma \ref{Lem-cDLocSys} is by construction the local system $\bU$ underlying the unitary factor $\cU$ (see subsection \ref{SubSec-Prel-LocSystOnFibr}, or directly \cite{CatDet_TheDirectImage_2014}).
		\begin{lemma}[Lifting lemma]\label{Lem-CharFlatSecN}
			Let $f:S\to B$ be a complete semistable fibration and $\bU$ be the local system underlying the unitary factor $\cU$ in the second Fujita decomposition of $f.$ Then, there is a short exact sequence of sheaves
				\begin{equation}\label{SeS-LiftingLemma}
				\xymatrix@!R{
					{0}  & {\omega_B}  & {f_*\Omega^1_{S,d}}  & {\bU}  & {0.} 
					% orizzontal arrows
					%%%	\ar"1,i";"j,k"
					\ar"1,1";"1,2"\ar"1,2";"1,3"\ar"1,3";"1,4"\ar"1,4";"1,5"
					% vertical arrows
					\hole
				}
				\end{equation}
			Moreover, the above sequence remains exact
			\begin{equation}\label{SeS-LiftLemmaLoc}
			\xymatrix@!R{
				{0}  & {H^0(A,\omega_B)}  & {H^0(f^{-1}(A),\Omega^1_{S,d})}  & {H^0(A,\bU)}  & {0.} 
				% orizzontal arrows
				%%%	\ar"1,i";"j,k"
				\ar"1,1";"1,2"\ar"1,2";"1,3"\ar"1,3";"1,4"\ar"1,4";"1,5"
				% vertical arrows
				\hole
			}
			\end{equation}
			over each proper open subset $A$ of $B.$
		\end{lemma}
		\begin{proof}
			 Since sequence \eqref{SeS-LiftLemmaLoc} is simply the Relative holomorphic de Rham sequence (sequence \eqref{SeS-HolRelDeRham}) in the case of semistable fibrations, where we have set $\bD\simeq\bU,$ we just have to remark that by taking the long exact sequence in cohomology on each proper open subset $A$ of $B,$ then $H^1(A,\omega_B)$ is zero since $A$ is Stein.
%			since $H^1(A,\omega_B)$ is zero according to the following remark \ref{Rem-SteinCoh}. When $f$ is semistable and then $\Ima\psi =\bU$ we get the proof.
			% 		
			% 		Idea: $\Ima \psi$ is the sheaf of flat sections when restricted to the regular valueas of $f$ with respect to $\nabla^\circ$ of the Hodge bungle $f_*\omega_{S/B}.$ Since $f$ is semistable, then it is a local system, or in other words the stalks over the regular values defines the stalks over the singular values since local monodromy is trivial. Then this sheaf is contained in $U.$ But it is $U$ itself by invariance on the local monodromy over any singular values.
			% 		
			% 		Precisely: 
			% 		
			% 		 We have two cases. First let $b\in B^\circ$ be a regular value of $f$ and take $A_b$ a neighborhood of $b$ which does not intersect $B_\circ.$ 
		\end{proof} 
%		

\begin{remark} The analogue description of the trivial trivial factor $\cO^{\oplus q_f}$ of rank the relative irregularity of $f$ in terms of $H^0(S,\Omega_S)/f^\ast H^0(B,\omega_B)$ has been given in \cite{Gonz_PhdTs_2013}.
\end{remark} 
We remark that the local liftings of $\bU$ provided by the above lemma are all closed holomorphic forms since they differs from pullback of holomorphic forms from the curve $B,$ which are automatically closed.  
Inspired by the work \cite{ChenLuZu_OnTheOort_2016}, where the authors prove a lifting of $\bU$ on $f_*\Omega_S$ (Corollary $7.2$), we provide a splitting of the kernel $\cK_\partial$ to $f_*\Omega^1_S$ and of the local system $\bU$ underlying $\cU$ to $f_*\Omega^1_{S,d}.$ 
  \begin{lemma}\label{Lem-SesUSPLIT}
% 	Let $f:S\to B$ be complete fibration with isolated singularities, $\cK_\partial$ be the kernel of $\partial$ and $\bU$ be the local system underlying the unitary factor $\cU$ in the second Fujita decomposition of $f.$ Then the short exact sequence 
% 		\begin{equation}\label{SeS-Kernel}
% 		\xymatrix@!R{
% 			{0}  & {\omega_B}  & {f_*\Omega^1_S}  & {\cK_\partial}  & {0}  , 
% 			% orizzontal arrows
% 			%%%	\ar"1,i";"j,k"
% 			\ar"1,1";"1,2"\ar"1,2";"1,3"\ar"1,3";"1,4"\ar"1,4";"1,5"
% 			\hole
% 		}
% 		\end{equation}
% 		is split.
 Let $f:S\to B$ be a complete semistable fibration. Then the short exact sequence
 	\begin{equation}\label{Ses-SplitbU}
 	\xymatrix@!R{
 		{0}  & {\omega_B}  & {f_*\Omega^1_{S,d}}  & {\bU}  & {0.} 
 		% orizzontal arrows
 		%%%	\ar"1,i";"j,k"
 		\ar"1,1";"1,2"\ar"1,2";"1,3"\ar"1,3";"1,4"\ar"1,4";"1,5"\ar@/^-1.1pc/"1,4";"1,3"_<<<{\eta'}
 		% vertical arrows
 		\hole
 	}
 	\end{equation}
 	is split.
 	\end{lemma}
 	\begin{proof} Let $\eta  \in H^0(\cK_\partial^\vee\otimes f_*\Omega^1_S)$ be the section that splits the sequence \eqref{SeS-KernelSplit<}
 		\begin{equation*}
 		\xymatrix@!R{
 			{\zeta:} & {0}  & {\omega_B}  & {f_*\Omega^1_S}  & {\cK_\partial}  & {0}  , 
 			% orizzontal arrows
 			%%%	\ar"1,i";"j,k"
 			\ar"1,2";"1,3"\ar"1,3";"1,4"\ar"1,4";"1,5"\ar"1,5";"1,6"\ar@/^-1.1pc/"1,5";"1,4"_<<<\eta
 			\hole
 		}
 		\end{equation*}
 		as proved in Lemma \ref{Lem-SesSPLITKER}. We prove that $\eta$ induces also a splitting on sequence \eqref{Ses-SplitbU}. Consider the diagram (see \eqref{Dia-LinkDiagramShort}) 
 		\begin{equation}\label{Dia-LinkDiagramShortU}
 		\xymatrix@!R{
 			0 &       {\omega_B}       &   {f_*\Omega^1_{S,d}}                &  {\bU}       &   {0}     &        \\
 			0   & {\omega_B}    & {f_*\Omega^1_{S}}  & {\cK_\partial}  %\ar@{}[l]+<1cm,0cm>="p1"
 			%\ar@{-}`r[d]`[d]^\delta[d] % curved arrow 1 	
 			& {0,} &       &       
 			% 	&                          &     & {f_*\omega_{S/B}} ,
 			% 	%\ar@{}+<0.6cm,0cm>="p1"  %intersection point 1
 			% 	&   &     
 			% 	% orizontal arrows
 			%%%	\ar"1,i";"j,k"
 			\ar"1,1";"1,2"\ar"1,2";"1,3"\ar"1,3";"1,4"\ar"1,4";"1,5"
 			%%%	\ar"2,i";"j,k"
 			\ar"2,1";"2,2"\ar"2,2";"2,3"\ar"2,3";"2,4"\ar"2,4";"2,5"
 			% vertical arrows
 			%%%	\ar"1,i";"j,k"
 			\ar"1,2";"2,2"^{  \parallel}       \ar@{^{(}->}"1,3";"2,3"^{i_{\de}}   \ar@{^{(}->}"1,4";"2,4"^{i_{\mtin{$\bU$}}}  
% 			 \ar"1,5";"2,5" 
 			% 	\ar@/_2pc/@{^{(}->}
 			% 	"1,4";"3,4"_>>>>>>{i_{\nabla}}
 			%%%	\ar"2,i";"j,k"
% 			\ar"2,4";"2,5"^-{\partial}  
% 			%	\ar@{^{(}->}"2,4";"3,4"^{i_2} 
% 			% diagonal arrow, with 1 hole
 			%	\ar@{-}"p1"|!{"2,3";"4,5"}
 			\hole
 		}
 		\end{equation}
 		Then the proof follows immediately since the kernel of the two sequences is the same. 
The morphism $\eta':\bU\to f_*\Omega^1_S$ given by composition of $\eta$ with the injection $i_{\mtin{$\bU$}}:\bU\hookrightarrow \cK_\partial$ is in fact in the image  $i_d:f_*\Omega^1_{S,d}\hookrightarrow f_\ast\Omega^1_S$ and this gives the desired splitting.
% Let $A$ be a open contractible subset of $B,$ $s\in \Gamma(A, \bU)$ a local section on $\bU$ and $\omega\in \Gamma(A,f_*\Omega^1_{S,d})$ a local lifting of $s.$ Since $\sigma_{\mtin{$\bU$}}(s)$ and $\omega$ both lifts $i_{\mtin{$\bU$}}(s)$ to $f_*\Omega^1_{S},$ their difference lies in $\Gamma(A,\omega_B)$ and thus its image lies in $\Gamma(A,f_*\Omega^1_{S,d}).$ Then also  $\sigma_{\mtin{$\bU$}}(s)$ must be an element in $\Gamma(A,f_*\Omega^1_{S,d}),$ from which we get the proof.
 		
 		
 		
 		\end{proof}
  				

 				\section{Massey-trivial families on $\bU$ and the Castelnuovo-de Franchis theorem for fibred surfaces}\label{Sec-MTsubbundles}
 				In this section we relate the geometry of Massey-trivial  subspaces $W\subset \Gamma(A,\bU)$ of flat local sections of the unitary factor $\cU$ of a fibration $f:S\to B$ of genus $g(F)\geq 2$ to the existence, up to base change, of morphisms from the surface into a smooth compact curve $\Sigma$ of genus greater than $2.$ The construction is given by a Castelnuovo de Franchis theorem for fibred surfaces proved in \cite{GonStopTor-On}, which we adapt to our setting. 
% 				to applied to a suitable lifting of $W$ to the sheaf $f_*\Omega^1_{S}.$ The above lifting is provided by the Massey-trivial condition as in \ref{Prop-MTtoIsotropic} together with the splitting of $\bU$ to $f_*\Omega^1_{S,d}.$ 
 				
 				Let us drop the assumption of compactness on the surface $S.$ Let $f:S\to B$ be a fibration of $S$ over a smooth curve $B$ (not necessarily compact or algebraic) and let $F$ be the general fibre of $f.$ Assume $g(F)\geq 2.$ 
%% 				We can give the following definition, generalizing the classical notion of irrational pencil on $S.$ 
% 				 In the paper we call a complex surface $S$ (not necessarly compact) {\em fiber-compact surface } when it is fibred over a disk $\Delta$ by a fibration $f$ as before (or more in general over a Riemann surface). We give the following definition.
% 				
%%% 				\begin{definition} An {\em irrational pencil} on a fibred surface $S$ is a holomorphic non constant map $\varphi:S\to \Sigma$ from $S$ to a smooth compact curve $\Sigma$ of genus $g(\Sigma)\geq 2.$ 	\end{definition}
%			The following is a generalization of the classical Castelnuovo de-Franchis theorem (\textcolor{red}{TODO: cit}) in the case of fibred (even if not necessarly compact) surfaces, proved in \textcolor{red}{TODO:cite}.  
%%			Note that the objects introduced even if generalizing the classical notion, give fibrations with non-compact fibres, as expected moving for example tranversely to $f:S\to \Delta.$ 
%Anyway, using the assumption that $S$ is fibred, an adapted version of the classical Castelnuovo-de Franchis theorem has been proved in \textcolor{red}{TODO:cit maledetto rango} and ensures the existence of these kind of irrational pencils attached to vector spaces isotropic with respect to the wedge-product on forms.
			Let $\Omega^1_{S,d}$ be the sheaf of closed holomorphic $1-$forms on $S$ and consider the wedge map 
			\begin{equation}
			\wedge: \bigwedge^2H^0(S,\Omega^1_{S,d})\to H^0(S,\omega_S) .
			\end{equation}
			We recall the following definition. 
			\begin{definition}A subspace $V$ of $H^0(S,\Omega^1_{S,d})$ is {\em isotropic} if the $\wedge-$map restricts to the null map on $\bigwedge^2 V.$ Moreover, it is {\em maximal} if it is not properly contained in any larger isotropic space. 
				\end{definition}
				Then the theorem is the following.
			
 				
 				\begin{theorem}[Castelnuovo-de Franchis for fibred surfaces]\label{Thm-CdFL} Let $f:S\to B$ be a fibred surface over a smooth curve $B$ and let $V\subset H^0(S,\Omega^1_{S,d})$ be a maximal isotopic subspace of dimension $r\geq 2$ such that the restriction $V \to H^0(\omega_F)$ to a general fibre $F$ is injective. Then there is a non-constant morphism $\varphi : S\to \Sigma$ from the surface $S$ to a smooth compact curve $\Sigma$ of genus $g(\Sigma)\geq 2$ such that $\varphi^*H^0(\omega_{\Sigma})=V.$
 				\end{theorem}
 				\begin{remark} The above theorem has been stated in \cite{GonStopTor-On} for $B$ equal to a complex disk. We remark that the proof generalizes to proper fibrations since the key point is that closed forms give rise to integrable foliations.
 					\end{remark}    
 				%Let us note the fibres of $\varphi$ could be not compact, defining a non proper morphism. 
% 				Anyway, the assumption of  
% 				We only remark that since the proof contained in \textcolor{red}{TODO:ref} does not care about the assumption of contractibility of the base, then it holds on more general non-compact Riemann surfaces different from a complex disk.
 				
 				We relate the above theorem with the existence of Massey-trivial subspaces $W$ of flat local sections of $\cU.$
 				Let $W\subset \Gamma(A, \bU)$ be a subspace of sections over an open subset $A$. In section \ref{Sec-LiftingsOnU}, we showed that there is an injection $\bU\hookrightarrow \cK_\partial$ and moreover each lifting of $\bU$ to  $f_*\Omega^1_S$ lies in $f_*\Omega_{S,d}$ (Lemma \ref{Lem-SesUSPLIT}). We get the following.
 				\begin{proposition} \label{Prop-UMTtoIsotropic} Let $A$ be an open set of $B,$ let $i_A: A\hookrightarrow B$ be the inclusion and let $W\subset \Gamma(A,\bU)$ be a Massey-trivial subspace of sections of $\bU$ over $A.$ Then there exists a unique $\widetilde{W}\subset H^0(A,f_*\Omega^1_{S,d})$ which lifts $W$ to $f_\ast\Omega^1_{S,d}$ and such that    $\bigwedge ^2\widetilde{W}\to \Gamma(A, f_\ast \omega_S)$ is the  zero map. 					%         	 		\begin{itemize}
 					%         	 			\item[(M1)] the map $\widetilde{W}\to W$ given by \ref{Dia-HolDeRhams} is an isomorphism;
 					%         	 			\item[(M2)]  $\widetilde{W}$ is isotropic with respect to $\wedge : \bigwedge^2H^0(f^{-1}(A),\Omega^1_{S,d})\to H^0({\omega_S}_{|A}) .$
 					%         	 		\end{itemize} 
 				\end{proposition}	
 				\begin{proof} The proof follows immediately by Proposition \ref{Prop-MTtoIsotropic}. It is enough to observe that we can choose liftings in $f_*\Omega^1_{S,d}$ by Lemma \ref{Lem-SesUSPLIT} and then all the other admissible splittings are still sections of $f_*\Omega^1_{S,d}$ since they must differs from the first ones by sections of $\omega_B.$ Moreover, the evaluation map $W\otimes \cO_A\to i_A^*\cK_\partial$ is automatically an injective map of vector bundles since $\Gamma(A,\bU)$ is the space of flat sections of $\cU.$
 					\end{proof}
 					The above proposition shows that Massey-trivial subspaces $W$ of sections on $\bU$ correspond to isotropic subspaces of sections on $f_\ast\Omega_{S,d}.$ Let $H<\pi_1(B,b)$ be the kernel of the monodromy representation of $\bU$ (see section \ref{SubSec-Prel-LocSystOnFibr}) and set $H_{\mtin{W}}$ be the subgroup of $H$ which acts trivially on $W,$ that is 
 					\begin{equation}
 					H_{\mtin{W}}=\{g\in H\, |  \, g\cdot w= w, \forall w\in W\}.
 					\end{equation}
 					
 					 As an application of the Castelnuovo-de Franchis for fibred surfaces \ref{Thm-CdFL} we obtain the following.     
 				
 				 
 					\begin{theorem}\label{Prop-MPpencils2} Let $f:S\to B$ be a complete semistable fibration of genus $g(F)\geq 2$ and let $W\subset \Gamma(A,\bU)$ be a maximal Massey-trivial subspace of sections over $A.$ Then for any subgroup $K$ of $H_{\mtin{W}},$  the fibred surface $f_{\mtin{K}}:S_{\mtin{K}}\to B_{\mtin{K}}$ defined by the \'{e}tale base change $u_{\mtin{K}}: B_{\mtin{K}}\to B$ classified by $K$ has an irrational pencil $h_{\mtin{K}}:S_{\mtin{K}}\to \Sigma$ over a smooth compact curve $\Sigma$ such that $W\simeq h_{\mtin{K}}^*H^0(\omega_{\Sigma}).$ 
 						
 					\end{theorem}
 					
 					\begin{proof}
 						Let $u_{\mtin{W}}:B_{\mtin{W}}\to B$ be the \'{e}tale covering classified by  $H_{\mtin{W}}< H.$ By construction, the pull back of $W$ extends to a subspace $\widehat{W}$ of global sections $\Gamma(B_{\mtin{W}},\bU_{\mtin{W}}),$ where  $\bU_{\mtin{W}}$ is the unitary factor of the fibration $f_{\mtin{W}}:S_{\mtin{W}}\to B_{\mtin{W}}$ defined by the base change. The proof follows for $H_{\mtin{W}}$ by applying Proposition \ref{Prop-UMTtoIsotropic} to $\widehat{W}$ and then also for each \'{e}tale covering $u_{\mtin{K}}:B_{\mtin{K}}\to B$ given by a subgroup $K$ of $H_{\mtin{W}},$ in a natural way.
 						
 					\end{proof}
 					
 					\begin{remark}\label{Rem-MPpencils} If we drop the assumption of maximality, we only get an inclusion $\widetilde{W}\subset h^*H^0(\omega_{\Sigma}).$ 
 					\end{remark}
 					We remark that a subspace $W\subset \Gamma(A, \bU)$ of local sections over a contractible open subset $A$ of $B$ is not necessarily invariant under the monodromy action of $\bU.$ This motivates the following definition, discussed in a more general setting in section \ref{SubSec-Prel-LocSystOnFibr}.
 					
 					\begin{definition}\label{Def-MTsubbundleB} 
 						Let $\bM$ be a local subsystem of $\bU$ of stalk $M.$ We say that $\bM$ is {\em Massey-trivial} if the stalk $M$ is isomorphic to a Massey-trivial subspace of  $\Gamma(A,\bU)$ of sections over an open contractible subset $A$ of $B.$ Moreover, we say that $\bM$ is {\em Massey-trivial generated} if the stalk $M$ is generated by a Massey-trivial subspace of $\Gamma(A,\bU).$  
 					\end{definition}  
 					
 					\begin{remark}\label{rem-MTaction} Massey-triviality over the general point $b$ of $A$ is a strong condition. By using a standard argument of analytic continuation, the above property on subspaces $W \subset \Gamma(A,\bU)$ of local flat sections on $\bU$ is stable under the monodromy action. 
 						\end{remark} 				

% 				
 				\section{Proof of the main theorems} \label{Sec-ProofMainTheorems}
% 				The proof of the main theorems deeply involves some classical results of the geometry of curves and surfaces due to Castelnuovo and de Franchis. We will see that a geometric description of Massey-trivial generated local systems follows form proposition \ref{Prop-MPpencils2}, which we recall is an application of an adapted version of the Castelnuovo and de Franchis theorem. The result of finitness for the monodromy instead is a consequence of a count of morphisms between curves of genus $\geq 2$  due to de Franchis. 
% 				
%% 				
%% 				The proof of the main theorems deeply involves some classical results of the geometry of curves and surfaces due to Castelnuovo and de Franchis. Indeed, a geometric description of Massey-trivial generated local systems with the existence of some kind of irrational pencils coming from an adapted version of Castelnuovo de Franchis' theorem (introduced in the previous section), while the result of finitness for the monodromy is a consequence of a count of morphisms between curves of genus $\geq 2$  due to de Franchis. 
%% 				
 				
In this section we give the proof of Theorems \ref{Thm-MainSbis} and \ref{Thm-MainG}. 
Let $f:S\to B$ be a complete fibration of genus $g(F)\geq 2$ and let $\cU$ be the unitary factor in the second Fujita decomposition of $f.$ Let $b\in B$ be a regular value and $F_b$ the (smooth) fibre over $b.$ Let $\bU$ be the underlying local system (i.e $\cU=\bU\otimes \cO_B$), $\rank{$U$}$ the rank of $\cU,$ $\rho:\pi_1(B,b)\to U(	\rank{$U$}, \bC)$ the unitary representation of $\bU,$  $H=\ker \rho$ the kernel and $G=\pi_1(B,b)/H.$ We recall that $G$ is naturally isomorphic to the monodromy group $\Ima \rho$ of $\cU$ and we identify them. 

Let $W\subset \Gamma(A,\bU)$ be a Massey-trivial subspace of sections over a contractible open subset $A$ of $B$ around $b$ and $\bM$ the local sub-system of $\bU$ generated by $W.$ We recall that $\bM$ has stalk $M=G\cdot W$ and defines a unitary flat subbundle $\cM$ of rank $\rank{$M$}=\dim M$ of $\cU$ together with a unitary sub-representation $\rho_{\tiny{M}}:\pi_1(B,b)\to U(\rank{$M$},\bC)$ of $\rho.$ We denote by $H_{\mtin{M}}$ the kernel and by $G_{\mtin{M}}=\pi_1(B,b)/H_{\mtin{M}}$ the quotient, again isomorphic to the monodromy group $\Ima \rho_{\mtin{M}}$ (see section \ref{Sec-LocSyst}).
 					 						
\subsection*{Proof of Theorem \ref{Thm-MainSbis}}
 						
% 						We also remark that as a unitary subrepresentation this gives a splitting $\bU=\bM\oplus \bM^\perp.$
%% 						giving the splitting $\bU=\bM\oplus \bM^\perp.$ In particular $M_{\mtin{M}}$ the  
						Assume $f:S\to B$ to be a semistable complete fibration. We prove the following.
 						
 						{\em {\bf Thesis.} There is a one to one correspondence between the monodromy group $G_{\mtin{M}}$ of $\cM$ and the automorphisms group of a finite set of morphisms $\mathscr{K}=\{k_g:F\to \Sigma\}_{g\in G_{\mtin{M}}}$ from the general fiber $F$ to a smooth compact curve $\Sigma$ of genus $g(\Sigma)\geq 2.$  Moreover, after a finite \'{e}tale base change $u_{\mtin{M}}:B_{\mtin{M}}\to B$ trivializing the monodromy, the pullback bundle of $\cM$ becomes the trivial bundle $V\otimes \cO_{B_{\mtin{M}}}$ of fibre $V=\sum_{g\in G_{\mtin{M}}} k_g^*H^0(\omega_{\Sigma})\subset H^0(\omega_F).$ }
%%%%%OLD Thesis
% 						{\em {\bf Thesis.} Given a maximal Massey-trivial subspace $W\subset \Gamma(A, \bU),$ there is a finite set of distinct morphisms $k_g:F\to \Sigma$ of curves from a fixed fibre $F$ over $b$ to a smooth compact curve $\Sigma$ parametrized injectively by the monodromy group $G_{\mtin{M}}$ of the unitary flat bundle $\cM$ generated by $W$ and trivializing $\cM$ as $\sum_{g\in G_{\mtin{M}}} k_g^*H^0(\omega_{\Sigma})\otimes \cO_{B_{\mtin{M}}}$ after a suitable base change $u_{\mtin{M}}:B_{\mtin{M}}\to B.$ 
% 						}
%%%%%%%%Old thesis
% 						{\em {\bf Thesis.} There is base change given by a finite normal covering $u_{\mtin{M}}:B_{\mtin{M}}\to B$ and a finite set of irrational pencils $h_g:S_{\mtin{M}}\to \Sigma$ over a fixed smooth compact curve $\Sigma$ of genus $g(\Sigma)\geq 2$ on $S_{\mtin{M}}$ \textcolor{red}{TODO: parametrized by $G_{\mtin{M}}$}  
% 						
% 							such that the pull back bundle $u_{\mtin{M}}^*\cM$ is isomorphic to the trivial bundle $\sum_{a\in G} i^*h_g^*H^0(\omega_{\Sigma})\otimes \cO_{B_{\mtin{M}}},$ where $i$ in the natural inclusion of $F$ in $S_{\mtin{M}}.$   	}					
 						
% 						{\em {\bf Thesis.} There is base change given by an unramified finite morphism $u_{\mtin{M}}:B_{\mtin{M}}\to B$ and a finite set of irrational pencils $h_g:S_{\mtin{M}}\to \Sigma$ over a fixed smooth compact curve $\Sigma$ of genus $g(\Sigma)\geq 2$ on $S_{\mtin{M}}$ 
% 							%	such that $u_{\mtin{M}}^*gW=f_g^*H^0(\omega_{\Sigma})$ and 
% 							such that the pull back bundle $u_{\mtin{M}}^*M$ is the trivial bundle $\sum_{a\in G} i^*h_g^*H^0(\omega_{\Sigma})\otimes \cO_{B_{\mtin{M}}},$ where $i$ in the natural inclusion of $F$ in $S_{\mtin{M}}.$   	}					
% 						
 						
 						The proof is developed in three steps.
 						\begin{itemize}
% 							\item[(1)] Construction of a base change trivializing the monodromy of $\cM;$
 							\item[(1)] The construction of the set $\mathscr{K}$ of morphisms of curves;
 							\item[(2)] The proof of the finitness of the monodromy group $G;$
 							\item[(3)] The geometric description of $\cM.$ 
 							\end{itemize}
 						
% 						{\bf $(1)$ Construction of a base change trivializing the monodromy of $M.$}
%We start recalling the construction of a base change trivializing the monodromy as seen in proposition \ref{Prop-MPpencils2} by fixing $M:=G\cdot W$ instead of $W$ and then we give a description of its kernel for the development of the proof. Let $H_{\mbox{{\tiny M}}}$ be the kernel of the representation of $\bM,$ which is exactly the group trivializing the monodromy of $\bM.$ Let $u_{\mbox{{\tiny M}}}:B_{\mbox{{\tiny M}}}\to B$ be the Galois covering map associated to $H_{\mtin{M}}$ and given by the action of $G_{\mbox{{\tiny M}}}=\pi_1(B,b)/ H_{\mbox{{\tiny M}}}$ on $B_{\mtin{M}}.$  
%Then  $u_{\mbox{{\tiny M}}}:B_{\mbox{{\tiny M}}}\to B$ gives the desired \'{e}tale base change 
%\begin{equation}\label{Dia-BaseChangeM}
%\xymatrix@!R{
%	{S_{\mtin{M}}:=S\times_BB_{\mtin{M}}} \,        &     {S}              &    \\
%	{B_{\mtin{M}}}            \,        & {B}         &    
%	% vertical arrows
%	\ar"1,1";"2,1"^{f_{\mtin{M}}}   \ar "1,1";"1,2"^>>>>>>{\varphi_{\mtin{M}}}
%	%%
%	\ar"1,2";"2,2"^{f} \ar "2,1";"2,2"^{u_{\mtin{M}}}
%	% diagonal arrow, with 1 hole
%	%	\ar@{-}"p1"|!{"2,3";"4,5"}
%	\hole,
%}
%\end{equation}
%where $S_{\mtin{M}}$ is a smooth surface and $\varphi_{\mtin{M}}:S_{\mtin{M}}\to S$ is a covering given by the action of $G_{\mtin{M}}\times S_{\mtin{M}}\to S_{\mtin{M}}$ sending a point $(p,t)$ in the point $g(p,t):=(p, gt),$ for $g\in G_{\mtin{M}}.$ Note that $g: S_{\mtin{M}}\to S_{\mtin{M}}$ is an automorphism of $S_{\mtin{M}}$ which preserves each fiber of $f_{\mtin{M}}.$ Moreover, it is easy to check that $H_{\mtin{M}}$ can be described as 
%\begin{equation}\label{Eq-desKer}
%H_{\mbox{{\tiny M}}}=\{g\in \pi_1(B,b)| \, gg'w=g'w, \, \forall w\in W, \, \forall g'\in G_{\mtin{M}}\}.
%\end{equation}
 
 						 {\bf $(1)$ The construction of the set $\mathscr{K}$ of morphisms of curves.} 
% 						 Let $f:S\to B$ be a semistable fibration of genus $g\geq 2$ and $U=\bU\otimes \cO_B$ be the unitary factor in the second Fujita decomposition $f_*\omega_{S/B}.$
						Let $u_{\mbox{{\tiny M}}}:B_{\mbox{{\tiny M}}}\to B$ be the Galois covering map classified by the normal group $H_{\mtin{M}}$ of $\pi_1(B,b)$ and with Galois group  $G_{\mbox{{\tiny M}}}=\pi_1(B,b)/ H_{\mbox{{\tiny M}}}$ of $\cM.$ By construction, $u_{\mbox{{\tiny M}}}:B_{\mbox{{\tiny M}}}\to B$ trivializes the monodromy of $\cM$ (that is, $u_{\mbox{{\tiny M}}}^{-1}\bM$ is a trivial local system on $B_{\mbox{{\tiny M}}}$). We consider the \'{e}tale base change 
						\begin{equation}\label{Dia-BaseChangeM}
						\xymatrix@!R{
							{S_{\mtin{M}}:=S\times_BB_{\mtin{M}}} \,        &     {S}              &    \\
							{B_{\mtin{M}}}            \,        & {B}         &    
							% vertical arrows
							\ar"1,1";"2,1"^{f_{\mtin{M}}}   \ar "1,1";"1,2"^>>>>>>{\varphi_{\mtin{M}}}
							%%
							\ar"1,2";"2,2"^{f} \ar "2,1";"2,2"^{u_{\mtin{M}}}
							% diagonal arrow, with 1 hole
							%	\ar@{-}"p1"|!{"2,3";"4,5"}
							\hole,
						}
						\end{equation}
						where $S_{\mtin{M}}$ is a smooth surface (not necessarily compact) given by the fibred product $S\times_{B}B_{\mtin{M}}$ and $\varphi_{\mtin{M}}:S_{\mtin{M}}\to S$ is an \'{e}tale Galois covering. The action of $G_{\mtin{M}}\times S_{\mtin{M}}\to S_{\mtin{M}}$ sends a point $(p,b')$ to the point $g(p,b'):=(p, gb'),$ for $g\in G_{\mtin{M}},$ where $b'\mapsto gb'$ is the automorphism of $B_{\mtin{M}}$ defined by the action of $G_{\mtin{M}}$ on $B_{\mtin{M}}.$  Note that $g: S_{\mtin{M}}\to S_{\mtin{M}}$ is an automorphism of $S_{\mtin{M}}$ compatible with the fibration $f_{\mtin{M}}.$ 
						
 						Let $W\subset \Gamma(A,\bU)$ be a maximal Massey-trivial subspace of sections over $A$ around $b$ and generating $\bM.$ From now on we identify $W$ with a subspace of $H^0(\omega_{F_b})$ (Remark \ref{garibaldi}). It is easy to check that $H_{\mtin{M}}$ can be described as 
 						\begin{equation}\label{Eq-desKer}
 						H_{\mbox{{\tiny M}}}=\{g\in \pi_1(B,b)| \, gg'w=g'w, \, \forall w\in W, \, \forall g'\in G_{\mtin{M}}\}.
 						\end{equation}
 						Applying Proposition \ref{Prop-MPpencils2} to $K=H_{\mtin{M}}<H_{\mtin{W}},$  we get an non-constant map $h:S_{\mtin{ M}}\to \Sigma$ of $S_{\mtin{ M}}$ onto a smooth compact curve $\Sigma$ such that $u_{\mtin{M}}^*W\simeq h^*H^0(\omega_{\Sigma}).$ We want to construct a family $\mathscr{H}:=\{h_g:S_{\mtin{ M}}\to \Sigma\}_{g\in G_{\mtin{M}}}$ of non-constant morphisms and a family  $\mathscr{K}:=\{k_g:F_b\to \Sigma\}_{g\in G_{\mtin{M}}}$ of morphisms of curves from the smooth fiber $F_b$ over $b$ parametrized by $G_{\mtin{M}}.$ The construction is as follows.
 						
 						Let $b_0$ a preimage of a point $b\in A$ via $u_{\mtin{M}}$ and $F_0$ be the fibre of $f_{\mtin{M}}$ over $b_0$ (which is isomorphic to the fibre of $f$ over $b$). For any $g\in G,$ we consider the automorphism $g:S_{\mtin{M}}\to S_{\mtin{M}}$ and we define $h_g$ and $k_g$ by composition
 						\begin{equation}\label{Dia-PencMor}
 						\xymatrix@!R{
 							{F_0}  & {S_{\mtin{M}}} \,        &     {S_{\mtin{M}}}       &      \\
 						           & {}                               &     {\Sigma}
 							% vertical arrows
 							\ar@{^{(}->}"1,1";"1,2"^{i}   \ar "1,2";"1,3"^{g} 
 							%%
 							\ar"1,3";"2,3"^{h} 
  							% diagonal arrow, with 1 hole
 							\ar"1,1";"2,3"_{k_g} \ar "1,2";"2,3"^{h_g}
 							%	\ar@{-}"p1"|!{"2,3";"4,5"}
 							\hole
 						} 						
 						\end{equation}
% 						Varying $g\in G,$ we get a family $h_g:=h\circ g :S_{\mtin{M}}\to \Sigma$ of irrational pencils such that $h_g^*H^0(\omega_{\Sigma})\simeq gW$ and a family  $k_g:=h_g\circ i :F\to \Sigma$ of morphisms of curves of genus $\geq2$ such that
% 						$k_g^*H^0(\omega_\Sigma)=gW\subset H^0(\omega_F),$ where we have set $gW$ the space given by the monodromy action of $g$ on $W.$
% 						
% 						\textcolor{red}{TODO: Forse qua vale la pena scrivere la formula!! perché implicitamente è la cosa scritta sopra!}

 						Moreover, we have the following formula.
 						\begin{lemma}\label{Lem-formulaFormsMonodromy} Let $e\in G_{\mtin{M}}$ be the neutral element and  $\alpha\in H^0(\omega_{\Sigma}).$ Then for each $g\in G_{\mtin{M}},$
 						\begin{equation}\label{Formula-PullBackMonod}
 							{k^*_{g}(\alpha) = g^{-1} k^*_e(\alpha)}, 
 						\end{equation}
 						 where $g^{-1}$ acts on $k^*_e(\alpha)\in W$ via the monodromy action $\rho_{\mtin{M}}$ defining $\bM.$ 
 						\end{lemma}
 						\begin{proof}
 						Let $b_0$ be the preimage of $b$ via  $u_{\mtin{M}}$ and $A_0$ an open contractible subset of $B_{\mtin{M}}$ such that $u_{\mtin{M}}(A_0)\subset A.$ We identify $W\subset H^0(\omega_{F_b})$ as a subspace $W\subset H^0(\omega_{F_0})$ via the isomorphism $H^0(\omega_{F_0})\simeq H^0(\omega_{F_b}).$   
 						Let $\widetilde{W}\in \Gamma(A_0,{f_{\mtin{M}}}_\ast \Omega^1_{S_{\mtin{M}},d})$ be the unique lifting of $W$ provided by Proposition \ref{Prop-UMTtoIsotropic}. By construction, we can lift in a natural way the monodromy action from $W$ to $\widetilde{W}.$ Then for $\eta\in \widetilde{W},$ we have $g\eta=\eta,$ for each $g\in H_{\mtin{M}}.$ This means that $\widetilde{W}$ extends to a subspace of global forms $H^0(S_{\mtin{M}}, \Omega^1_{S_{\mtin{M}},d})$ and we identify them. Let $\eta=h_e^*\alpha\in \widetilde{W}$ and let $w=\eta_{|F}.$ Then $k^*_{g}(\alpha)=(g^*h_e^*\alpha)_{|F}=(g^*\eta)_{|F}=\eta_{|F_{g^{-1}b}}$ and $g^{-1}k^*_{e}(\alpha)=g^{-1}(\eta_{|F})=g^{-1}w=\eta_{|F_{g^{-1}b}}$ are equal.
 						\end{proof}
 						
 						
 						We prove that $G_{\mtin{M}}$ is in one to one correspondence with the automorphism group $\Aut (\mathscr{K}),$ where $\Aut (\mathscr{K})$ is the group of bijections on the set $\mathscr{H}.$  Consider the actions on $\mathscr{H}$ and $\mathscr{K}$
 						\begin{equation}\label{Act-IrrPenc}
 						\xymatrix@!R{
 							{G_{\mtin{M}}\times \mathscr{H}}  & {\mathscr{H}} ,\,        &     {(g_1,h_{g_2})}       &  {g_1\cdot h_{g_2}:=h_{g_1g_2}}  							
 							% orizzontal arrows
 							\ar"1,1";"1,2"%^<<<<{\Psi_{\mtin{M}}} 
 							  \ar@{|->} "1,3";"1,4" 
 							%	\ar@{-}"p1"|!{"2,3";"4,5"}
 							\hole
 						} 	
	 					\end{equation}
	 					\begin{equation}\label{Act-Mor}
	 					\xymatrix@!R{
	 						{G_{\mtin{M}}\times \mathscr{K}}  & {\mathscr{K}} ,\,        &     {(g_1,k_{g_2})}       &  {g_1\cdot k_{g_2}:=(g_1\cdot h_{g_2})\circ i=k_{g_1g_2}}  							
	 						% orizzontal arrows
	 						\ar"1,1";"1,2"%^<<<<{\Psi_{\mtin{M}}} 
	 						\ar@{|->} "1,3";"1,4" 
	 						%	\ar@{-}"p1"|!{"2,3";"4,5"}
	 						\hole
	 					} 	
	 					\end{equation}
	 					defined in the natural way by the action of $G_{\mtin{M}}$ on $S_{\mtin{M}}.$ Then we get a homomorphism
	 					\begin{equation}\label{Map-AutK}
	 					\xymatrix@!R{
	 						{\Psi_{\mtin{M}}\,\colon\,G_{\mtin{M}}}  & {\Aut(\mathscr{K})} ,\,        &     {g_1}       &  {g_1\cdot\,:\, \mathscr{K}\to \mathscr{K}}  							
	 						% orizzontal arrows
	 						\ar"1,1";"1,2"%^<<<<{\Psi_{\mtin{M}}} 
	 						\ar@{|->} "1,3";"1,4" 
	 						%	\ar@{-}"p1"|!{"2,3";"4,5"}
	 						\hole
	 					} 	
	 					\end{equation}
%	 					
%	 					where $\mathscr{M}(F,\Sigma)$ is the set of the morphisms from $F$ to $\Sigma.$ By construction, the image of $\Psi_{\mtin{M}}$ is exactly $\mathscr{K}$ and thus we just have to prove the following.
%	 					
	 					\begin{lemma}\label{Lem-PsiInj} $\Psi_{\mtin{M}}$ is injective.
	 						\end{lemma}
	 						\begin{proof}
	 							Let $e$ be the neutral element in $G_{\mtin{M}}$ and $g_1 \in G_{\mtin{M}},$ $g_1\neq e.$ We want to prove that $\Psi_{\mtin{M}}(g_1)\neq \Psi_{\mtin{M}}(e),$ that is there exists $g_2\in G_{\mtin{M}}$ such that $g_1\cdot k_{g_2}\neq  e\cdot k_{g_2}$ (i.e. $k_{g_1g_2}\neq   k_{g_2}$) 
	 							It is enough to prove that $k_{g_1g_2}^*\neq k_{g_2}^*.$ Since $g_1\neq e,$ then $g_1\notin H_{\mtin{M}}.$ Thanks to description \eqref{Eq-desKer} of $H_{\mtin{M}},$ there exists $w\in W\subset H^0(\omega_F)$ and $g_2\in G_{\mtin{M}}$ such that $g_1g_2w\neq g_2w.$ Let $\alpha \in H^0(\omega_\Sigma)$ be such that $k_e^*\alpha=w.$ Then by formula \eqref{Formula-PullBackMonod} with $g=g_1g_2$ we get $g_1g_2w=(g_1g_2)^{-1} k_e^*(\alpha)=k^*_{g_1g_2}w,$ while applying the same formula with $g=g_2$ we get $g_2w=g_2^{-1} k^*_e(\alpha)=k^*_{g_2}(\alpha).$ By assumption, $g_1g_2w\neq g_2 w$ and thus $k^*_{g_1g_2}\neq k^*_{g_2}.$
%%%%%OLD
%Consider the map
%\begin{equation}\label{Map-GtoMor}
%\xymatrix@!R{
%	{\Psi_{\mtin{M}}\,\colon\,G_{\mtin{M}}}  & {\mathscr{M}(F,\Sigma)} ,\,        &     {g}       &  {k_g}  							
%	% orizzontal arrows
%	\ar"1,1";"1,2"%^<<<<{\Psi_{\mtin{M}}} 
%	\ar@{|->} "1,3";"1,4" 
%	%	\ar@{-}"p1"|!{"2,3";"4,5"}
%	\hole
%} 	
%\end{equation}
%where $\mathscr{M}(F,\Sigma)$ is the set of the morphisms from $F$ to $\Sigma.$ By construction, the image of $\Psi_{\mtin{M}}$ is exactly $\mathscr{K}$ and thus we just have to prove the following.
%
%\begin{lemma}\label{Lem-PsiInj} $\Psi_{\mtin{M}}$ is injective.
%\end{lemma}
%\begin{proof}
%	Let $g_1,g_2 \in G_{\mtin{M}}$ such that $g_1\neq g_2.$ We want to prove that $k_{g_1}\neq k_{g_2}.$ Consider the pullback functor 
%	$\mathscr{M}(F,\Sigma)\to \hom(H^{1,0}(\Sigma),H^{0,1}(F)).$ As remarked in \ref{Rem-MorHomDeFra}, this is injective and thus we can prove $k_{g_1}^*\neq k_{g_2}^*$ to get the thesis. Let $g\in G_{\mtin{M}}$ different from the neutral element $e\in G_{\mtin{M}}.$ Then, by definition of $H_{\mtin{M}},$ there is $w\in W\subset H^0(\omega_F)$ and $g''\in G_{\mtin{M}}$ such that $gg''w\neq g''w.$ Let $\alpha \in H^0(\omega_\Sigma)$ such that $k_e^*\alpha=w.$ Then we have $k^*_{g''}w=g''w\neq gg''w=k^*_{gg''}w$ and thus also $k^*_g\neq k^*_e.$ Applying the same argument to $gg'\in G_{\mtin{M}}$ when $gg'\neq e$ we get $k^*_{gg'}\neq k^*_e$ and thus also $k^*_g\neq k^*_{g'}.$
	 							
%%%OLD
%	 							Let $g,g' \in G_{\mtin{M}}$ such that $g\neq g'.$ We want to prove that $k_g\neq k_{g'}.$ Consider the pullback functor 
%	 							$\mathscr{M}(F,\Sigma)\to \hom(H^{1,0}(\Sigma),H^{0,1}(F)).$ As remarked in \ref{Rem-MorHomDeFra}, this is injective and thus we can prove $k_{g'}^*\neq k_{g}^*$ to get the thesis. Let $g\in G_{\mtin{M}}$ different from the neutral element $e\in G_{\mtin{M}}.$ Then, by definition of $H_{\mtin{M}},$ there is $w\in W\subset H^0(\omega_F)$ and $g''\in G_{\mtin{M}}$ such that $gg''w\neq g''w.$ Let $\alpha \in H^0(\omega_\Sigma)$ such that $k_e^*\alpha=w.$ Then we have $k^*_{g''}w=g''w\neq gg''w=k^*_{gg''}w$ and thus also $k^*_g\neq k^*_e.$ Applying the same argument to $gg'\in G_{\mtin{M}}$ when $gg'\neq e$ we get $k^*_{gg'}\neq k^*_e$ and thus also $k^*_g\neq k^*_{g'}.$
%
%	 							Note that since $G_{\mtin{M}}$ is a group and by definition $h_{g{g'}^{-1}}=h,$ then it is enough to prove that $k_g\neq k_e,$ for some $g\neq e.$ Let $g \in G_{\mtin{M}}$ such that $g\neq e,$ where $e$ is the \textcolor{red}{neutral element}. We prove that $k_{g}^*\neq k_{e}^*,$ which implies $k_g\neq k_e$ for morphisms of curves of genus greater than $2.$\textcolor{red}{TODO:referenza o spiegazione} 
								\end{proof}
%%%%%%OLD: proof injectivity
%	 					\begin{proof}
%	 						Let $g,g' \in G_{\mtin{M}}$ such that $g\neq g'.$ We want to prove that $k_g\neq k_{g'}.$ Note that since $G_{\mtin{M}}$ is a group and by definition $h_{g{g'}^{-1}}=h,$ then it is enough to prove that $k_g\neq k_e,$ for some $g\neq e.$ Let $g \in G_{\mtin{M}}$ such that $g\neq e,$ where $e$ is the \textcolor{red}{neutral element}. We prove that $k_{g}^*\neq k_{e}^*,$ which implies $k_g\neq k_e$ for morphisms of curves of genus greater than $2.$\textcolor{red}{TODO:referenza o spiegazione} Since $g\in G_{\mtin{M}},$ then by definition of $H_{\mtin{M}}$ there is $w\in W\subset H^0(\omega_F)$ and $g'\in G_{\mtin{M}}$ such that $gg'w\neq g'w.$ Let $\alpha \in H^0(\omega_\Sigma)$ such that $k_e^*\alpha=w.$ Then we have $k^*_{g'}w=g'w\neq gg'w=k^*_{gg'}w$ and thus also $k^*_g\neq k^*_e.$
%	 					\end{proof}
%	 					
%%%%%%OLD Sketch proof 
% 						\begin{proof}
% 							Let $g,g' \in G_{\mtin{M}}$ such that $g\neq g'.$ We want to prove that $k_g\neq k_{g'}.$ Note that since $G_{\mtin{M}}$ is a group and by definition $h_{g{g'}^{-1}}=h,$ then it is enough to prove that $k_g\neq k_e,$ for some $g\neq e.$ Let $g \in G_{\mtin{M}}$ such that $g\neq e,$ where $e$ is the \textcolor{red}{neutral element}. We prove that $k_{g}^*\neq k_{e}^*,$ which implies $k_g\neq k_e$ for morphisms of curves of genus greater than $2.$\textcolor{red}{TODO:referenza o spiegazione} Since $g\in G_{\mtin{M}},$ then by definition of $H_{\mtin{M}}$ there is $w\in W\subset H^0(\omega_F)$ and $g'\in G_{\mtin{M}}$ such that $gg'w\neq g'w.$ Let $\alpha \in H^0(\omega_\Sigma)$ such that $k_e^*\alpha=w.$ Then we have $k^*_{g'}w=g'w\neq gg'w=k^*_{gg'}w$ and thus also $k^*_g\neq k^*_e.$
% 						\end{proof}
% 						
 						
% 						Let $\{b_g\}_{g\in G_{\mtin{M}}}=u_{\mtin{M}}^{-1}(b)$ be the fibre of $u_{\mtin{M}}$ over $b,$  $F_{b_g}$ the fibre in $S_{\mtin{M}}=S\times_BB_{\mtin{M}}$ over $b_g.$  The restriction of $h$ to ech fiber $F_{b_g}$ gives a family of non constant morphisms $h_g:F_{b_g}\to \Sigma$ between curves of genus greater than $2$ parametrized by $g\in G_{\mtin{M}}$. By composition with the isomorphisms $F_b\to F_{b_g}$ naturally induced by the base change, we get a family $f_g:F_b\to \Sigma$ of morphisms between two fixed curves of genus greater than $2$ parametrized by $g\in G_{\mtin{M}}.$ 
% 						and we can assume the family parametrized by $\pi_1(B,b)$ just by taking $f_g=f_e$ for each $g\in H_{\mtin{M}}.$
% 						
% 						We have constructed a family $\cM_{\pi_1} := \{h_g:F\to \Sigma \,| \, g\in \pi_1(B,b)\}$ between the fixed  fibre $F$ and $\Sigma$ starting from $h:F\to \Sigma.$ Let $g\in \pi_1(B,b)$ and $gW$ be the vector subspace of $\Gamma(A,\bU)$ given by the action of $g$ on $W,$ which is Massey-trivial when $W$ is (as seen in [SEZ]), and $\widetilde{gW}$ the correspondent isotropic vector space attched to it. In particular, $g$ moves the base $\cB_{\mtin{W}}$ to a base $\cB_{\mtin{gW}}=(s_1^g,\dots,s_k^g)$ of $gW$ and the same happens on the lifted bases in $\widetilde{W}$ and $\widetilde{gW}$. Then we can construct linear maps which gives the diagram    
% 						\begin{equation}\label{Dia-ActionOnW}
% 						\xymatrix@!R{
% 							{W}  & {\widetilde{W}}  & {u_{\mtin{M}}^*\widetilde{W}} \\ 
% 							{gW}  & {\widetilde{gW}}      & {u_{\mtin{M}}^*\widetilde{gW}}  \\
% 							% orizzontal arrows
% 							%%%	\ar"1,i";"j,k"
% 							\ar"1,2";"1,1"^{\sim}\ar@{^{(}->}"1,2";"1,3"^{u_{\mtin{M}}^*}
% 							%%%\ar"2,i";"j,k"
% 							\ar"2,2";"2,1"^{\sim}\ar@{^{(}->}"2,2";"2,3"^{u_{\mtin{M}}^*}
% 							% vertical arrows
% 							%%%	\ar"2,i";"j,k"
% 							\ar"1,1";"2,1"^{g} \ar@{-->}"1,2";"2,2"^{g} \ar@{-->}^{\eta_g}"1,3";"2,3" 
% 							% diagonal arrow, with 1 hole
% 							%	\ar@{-}"p1"|!{"2,3";"4,5"}
% 							\hole
% 						}
% 						\end{equation}
% 						and the data $(gW, \cB_{\mtin{gW}})$ gives a pencil $h_g: F\to \Sigma_g.$ But now it is enough to observe that $\Sigma_g$ must be biholomorphic to $\Sigma.$ 
%This comes from an argument of analytic continuation. In fact the maps $\eta_g$ is obtained by moving analitically the base along a lifting of $g$ on $B_{\mtin{M}}$ and thus the constructed morphism and its image moves analitically too.
%% 						 By applying proposition \ref{Prop-MPpencils1} and then the restriction to the fibre $F$ we get a map $f_g:F\to \Sigma_g$ over a smooth compact curve $\Sigma_g$ of genus $g(\Sigma_g)=\dim W.$ By a standard argument of analitic continuation of the pencil over $A$ with respect to $g\in \pi_1(B,b),$ it must be $\Sigma=\Sigma_g.$ \textcolor{red}{TODO. Explain better.... La mappa associata a $gW$ la voglio su $\Sigma.$ Per farlo o trasporto parallelamente la famiglia o compongo con un biolomorfismo, o vado a sistemarmmi le forme.. CAPIRE!} 
%						This constructs the family of maps $\cM_{G_{\mtin{M}}} $ together with a natural action of $\pi_1(B,b)$ over the automorphism group of this set
% 						\begin{equation}
% 						\rho': \pi_1(B,b)\to \Aut \cM_{G_{\mtin{M}}},\quad g\mapsto m_g: f_{g'}\mapsto g\cdot f_{g'}:=f_{m(g)(g')},
% 						\end{equation}
% 						given by the monodromy action of $u_{\mtin{M}}:B_{\mtin{M}}\to B$ which moving the point parametrizing the fiber to which we restrict the pencil. To be more precise we have identified $g$ with the element $gb_0$ of the fibre of $b,$ which is possible after fixing a $b_0$ in such fibre, so we have $m(g')(g)=m(g'g)(b_0).$
% 						
% 						\begin{lemma}\label{Thm-LemKer}$\ker \rho'= H_{\mbox{{\tiny M}}},$ where $H_{\mbox{{\tiny M}}}:=\{g\in \pi_1(B,b)\,|\, gw=w, \, \forall w\in \bM_b\}.$
% 							\end{lemma}
% 							
% 							\begin{proof} By construction it is clear that $H_{\mtin{M}}< \ker \rho'.$ To prove the converse, let $g\in \ker \rho'.$ Then $m_g(f_{g'})=m_{g'},$ for every $g'\in \pi_1(B,b),$ meaning that $f_{gg'}=f_{g}.$\textcolor{red}{TODO: qua va sistemata la dim} i.e. $g'w^g=w^g$ for each $w^g\in gW,$ for each $g\in \pi_1(B,b).$ Thus $\ker \rho' \subset H_{\mtin(M)}.$      
% 								\end{proof}
% 							
% 						
% 					\textcolor{red}{{\bf NOT SURE} {\bf Claim.} $\ker \rho'= H_{\mbox{{\tiny inv}}},$ where $H_{\mbox{{\tiny inv}}}:=\{g\in \pi_1(B,b)\,|\, gw=w, \, \forall w\in W\}.${\bf Maybe} $\ker \rho'= H_{\mbox{{\tiny W}}},$ where $H_{\mbox{{\tiny W}}}:=\{g\in \pi_1(B,b)\,|\, gW=W\}$}
% 					
% 					\textcolor{red}{proof of claim}
%
% 					\textcolor{red}{{\bf SURE AND ENOUGH. \LARGE NOT ENOUGH}}
% 					
% 					{\bf Claim.} $\ker \rho'\subset H_{\mbox{{\tiny W}}},$ where $H_{\mbox{{\tiny W}}}:=\{g\in \pi_1(B,b)\,|\, gW=W\}$
% 					
% 					In fact $f_g=f,$ then also $f_g^*: H^0(\omega_\Sigma^g)\to H^0(\omega_F) $ and $f: H^0(\omega_\Sigma)\to H^0(\omega_F)$ are the same maps. Thus by maximality, $gW= f_g^* H^0(\omega_\Sigma^g)=f^*H^0(\omega_\Sigma)=W.$
% 					
%%% 					We have so proved that $\Aut (\mathscr{K}$) is one to one with $G_{\mtin{M}}.$ 
% 					As a consequence, the same holds for $\mathscr{H}.$ 
% 					

 					{\bf$(2)$  The proof of the finiteness of the monodromy group $G.$} The proof follows from a classical de-Franchis's theorem (see \cite{Martens_Obervations_1988}) applied to the set of morphisms $\mathscr{M}(F,\Sigma).$ 
 					 Let $C$ and $C'$ be smooth compact curves and let $\mathscr{M}(C,C')$ be the set of non-constant morphisms between them. 
 					% 				 The classical result due to de-Franchis \textcolor{red}{TODO:CIT} can be stated as follows.
 					\begin{theorem}[de Franchis]\label{Teo-deFranchis} Let $C,C'$ be smooth compact curves of genus $\geq 2.$ Then the set $\mathscr{M}(C,C')$ is finite. 
 					\end{theorem}
 					As a consequence, we also have the following. 
 					\begin{proposition}Let $C,C'$ be smooth projective curves of genus $\geq 2.$ Then the map
 						\begin{equation}\label{Cor-MorHomDeFra}
 						\mathscr{M}(C,C')\to \hom(H^{1,0}(C),H^{1,0}(C'))
 						\end{equation}
 						given by the pullback functor is injective. 
 					\end{proposition}
 					
 					
 					{\em Sketch of the proof.}
 					 						Let $\phi_i:C\to C'$ be two morphims between $C$ and $C',$ for $i=1,2.$  They induce linear maps $ \phi_i^{1,0}: H^{1,0}(C')\to H^{1,0}(C)$ and $ \phi_i^{0,1}: H^{0,1}(C')\to H^{0,1}(C)$ by pullback. Assume that $\phi_1^{1,0}=\phi_2^{1,0}.$ Applying complex conjugation, also $\phi_1^{0,1}=\phi_2^{0,1}.$ Using the Hodge decomposition,  the maps $\phi_{i \bC}^\ast: H^{1}(C',\bC)\to H^{1,0}(C,\bC)$ and  $\phi_{i \bZ}^\ast: H^{1}(C',\bZ)\to H^{1,0}(C,\bZ)$ must be equal. By the above argument, the same holds on the map from the Jacobians. The result follows by the standard proof of de Franchis given by Martens in \cite{Martens_Obervations_1988} (see also \cite{AlzatiPirola_Some_1991}). This concludes the argument.
% 						\textcolor{red}{TODO:CIT Martens, Henrik H.
% 							Observations on morphisms of closed Riemann surfaces.
% 							Bull. London Math. Soc. 10 (1978), no. 2, 209?212.
% 							  Alzati,  ; Pirola, G. P. 
% 							Some remarks on the de Franchis theorem. 
% 							Ann. Univ. Ferrara Sez. VII (N.S.) 36 (1990), 45?52 (1991) }  
% 						
 	
 					
 					Applying de-Franchis's theorem to the set of morphisms $\mathscr{M}(F,\Sigma),$ we have that this is a finite set. Thus also $\mathscr{K}$ and consequently $\Aut (\mathscr{K})$ are finite. By Lemma \ref{Lem-PsiInj}, $\Psi_{\mtin{M}}(G_{\mtin{M}})$ is in one to one correspondence with $G_{\mtin{M}},$ thus the monodromy $G_{\mtin{M}}$ is finite group and $u_{\mbox{{\tiny M}}}$ is a finite covering. 
% 					$\mbox{Mor}(F,\Sigma)$ of morphisms from $F$ to $\Sigma$ \textcolor{red}{TODO: inserire qua o in un altra sezione??} is finite, we immediately get that $G_{\mtin{M}}$ is finite, since isomorphic to the subgroup $\Ima_{\Psi_{\mtin{M}}}$ of $\mbox{Mor}(F,\Sigma)$ as proven in step $(2).$ Thus  $u_{\mbox{{\tiny M}}}:B_{\mbox{{\tiny M}}}\to B$ is fnite.
%%%%%%OLD
% We prove it is a finite morphism, showing that $H_{\mbox{{\tiny M}}}$ is a subgroup of finite index in $\pi_1(B,b).$ By Lemma \ref{Thm-LemKer} given in step $(2),$ it is equivalent to show that $\ker \rho'$ is of finite index. This fact is a consequence of de Franchis theorem
% 					 \ref{Thm-deFranchis}.
% 					 \textcolor{red}{TODO:sostituire con enunciato a curva fissa\begin{theorem}[de Franchis \cite{DeFranchisUnTeo1913}]\label{Thm-deFranchis}
% 					 	There are only finitely many isomorphism classes of separable, non-constant morphisms $f: C\to C'$, where $C'$ runs over all curves of genus $g'>2.$
% 					 \end{theorem}}
% 					   According to the theorem, the set $\cM_{\pi_1}$ is finite and thus the group $\Aut \cM_{\pi_1}$  of automorphisms is finite too, which implies that $\Ima \rho'$ is finite and $\ker \rho'=H_{\mtin{M}}$ is of finite index in $\pi_1(B,b),$ which concludes the proof. 
% 					
 				
 					{\bf $(3)$  The geometric description of $\cM.$}
 					Let $\bM$ be the local system generated by $W,$ with stalk the vector space $M=G\cdot W$ and recall that $G_{\mbox{{\tiny M}}}\cdot W=\sum_{g\in G_{\mtin{M}}}g\cdot W,$ since $H_{\mtin{M}}$ fixes $W.$ The inverse image $u^{-1}_{\mbox{{\tiny M}}}\bM$ of the local system $\bM$ via the base change $u_{\mtin{ M}}:B_{\mtin{ M}}\to B$ is by construction a trivial local system on $B_{\mtin{M}}$ of the same stalk on $\bM.$ 
 					Using the family $\mathscr{K}$ of morphisms $h_g$ parametrized by $G_{\mtin{M}},$ we get $k_g^*H^0(\omega_\Sigma)=i^*h_g^*H^0(\omega_\Sigma)=gW,$ for each $g\in G_{\mtin{M}}$ and then the stalk of $\bM$ is described by $\sum_{a\in G} k_g^*H^0(\omega_{\Sigma})\subset H^0(\omega_F)$
 					
 					
%% 					\textcolor{red}{BIS: versione pencils: Then by using the family $\mathscr{H}$ of irrational pencils $h_g$ parametrized by $G_{\mtin{M}}$ contructed in step $(2),$ we have $i^*h_g^*H^0(\omega_\Sigma)=gW,$ for each $g\in G_{\mtin{M}}.$ This gives a geometric description of the trivial bundle $M$ attached to $\bM$ as $\sum_{a\in G} i^*h_g^*H^0(\omega_{\Sigma})\otimes \cO_{B_{\mtin{M}}}.$
%% 					}
% 					\textcolor{blue}{TODOREDO OLD}
% 					 Let $W\subset \Gamma(A,\bU)$ be a maximal Massey trivial subspace of sections around $b\in B^\circ$ and $u_{\mbox{{\tiny M}}}:B_{\mbox{{\tiny M}}}\to B$ be the unramified Galois covering constructed in $(1).$ The local system $\bM$ generated by $W$ has fiber the vector space $\bM_b=G_{\mbox{{\tiny M}}}\cdot W=\sum_{g\in \pi_1(B,b)}g\cdot W.$ Then the pullback local system $u^*_{\mbox{{\tiny M}}}\bM$ has a natural geometric description in terms of the irrational pencils contructed in $(2).$ Precisely for each $g\in G_{\mbox{{\tiny M}}},$ let $f_g:F\to \Sigma_g$ be the map introduced in $(2)$ given by proposition proposition \ref{Prop-MPpencils1}. Then in particular holds the equality $f_g^*H^0(\omega_\Sigma)=g\cdot W$ which gives the desired description of $\bM_b$ as $\sum_{\{g|gW\neq W\}} f_g^*H^0(\omega_{\Sigma}).$
% 					
%					\begin{remark} Let $f:S\to B$ a semistable fibration of genus $g\geq 2.$ By maximality of $W$ we have only two cases: $gW=W$ or $gW\cap W=\emptyset.$  The first one should give a map on the same curve $\Sigma$ while the second should give a map onto another curve $\Sigma^g.$ 
%						Thus we chould have that after an algebraic base change given by the unramified morphism $u_{H_{\mbox{\tiny W}}}: B_{H_{\mbox{\tiny W}}}\to B$ classified by $H_{\mbox{\tiny W}}=\ker \rho,$ the vector bundle $M=\bM\otimes \cO_B$ generated by $W$ looks like the trivial bundle 
%						$u^*_{H_{\mbox{\tiny W}}}M=H^0(\omega_{\Sigma})\otimes \cO_{B_{H_{\mbox{\tiny W}}}}.$ Moreover $\dim H^0(\Sigma)=g(\Sigma)=g(\Sigma^{g})\dim H^0(\Sigma^{g})$ and \textcolor{red}{Check}$f^*f_{g*}:H^0(\Sigma)\to H^0(\Sigma^{g})$ is an automorphism.
%						$u^*_{H_{\mbox{\tiny W}}}M=\bigoplus_{\{g|gW\neq W\}}H^0(\omega_{\Sigma_g})\otimes \cO_{B_{H_{\mbox{\tiny W}}}},$ where $\oplus_{\{g|gW\neq W\}}$ is finite sum. Moreover $\dim H^0(\Sigma^g)=g(\Sigma^g)=g(\Sigma^{g'})\dim H^0(\Sigma^{g'})$ and \textcolor{red}{Check}$f_g^*f_{g'*}:H^0(\Sigma^g)\to H^0(\Sigma^{g'})$ is an automorphism.
%						\end{remark}
%
%						\textcolor{red}{ {\bf AIM:} understand precisely the description of $\bM$ for semistable fibration and non- semistable (in non semistable should I loose the maximality of $W$ after the reduction theorem?)}
%						
 					
 					\subsection{Proof of Theorem \ref{Thm-MainG}} Let $f:S\to B$ be a complete fibration and assume that $f$ is not semistable (otherwise we can apply Theorem \ref{Thm-MainSbis}). 
% 					 $W\subset \Gamma(A, \bU)$ be a Massey-trivial subspace not necessarly maximal. We want to prove the following
%                   {\em {\bf Thesis.}The monodromy of $\cM=\bM\otimes \cO_B$ is finite.} 
%                   
%                   We start with some preliminary facts related to representation theory, we will use inside the proof. The first one asserts that maximality is not an obstruction for the finitness of monodromy. 
%                   
%                   \begin{proposition}\label{Prop-SubLocSystMon} Let $W_1$ and $W_2$ be two subspaces of $ \Gamma(A, \bU)$ such that $W_1\subset W_2.$ If the local system $\bM_2$ generated by $W_2$ has finite monodromy, then the local system $\bM_1$ generated by $W_1$ has finite monodromy too.
%                   	\end{proposition}
%                   	\begin{proof}
%                   		Let $H=\ker\rho$ be the kernel of the unitary representation of $\bU,$ and $H_i=\ker \rho_i$ be the kernel of the subrepresentations $\rho_i$ defining $\bM_i,$ for $i=1,2$ (definition \ref{Def-MTsubbundleA}). Then we have an inclusion $ H_2 \vartriangleleft H_1$  of subgrups which gives a surjection 
%                   		\begin{equation}
%                   		\xymatrix@!R{
%                   			 {G_2:=\pi_1(B,b)/H_2}  & {G_1:=\pi_1(B,b)/H_1}  & {0} 
%                   			% orizzontal arrows
%                   			%%%	\ar"1,i";"j,k"
%                   			\ar"1,1";"1,2"\ar"1,2";"1,3"
%                   			% vertical arrows
%                   			\hole
%                   		}
%                   		\end{equation}
%                   		on the quotients groups respectively isomorphic to the monodromy groups of $\bM_2$ and $\bM_1.$ Thus whenever the monodromy of $\bM_2$ is finite, the monodromy of $\bM_1$ is finite too.
%                   		\end{proof}
%                   		Thus by taking $W_1$ as a Massey-trivial subspace and $W_2$ as the maximal Massey-trivial one containing it, we conclude that we can release the assumption of maximality. 
%%                   		is not obstructive to the finetness of monodromy and we can release this assumption.
%%                   		
%                   	
%                   		The second one concernes the behaviour of local subsytems generated by vector spaces and base changes given by finite morphisms. 
%                   		
%                   		\begin{proposition}\label{Prop-LocSysAndBaseChange} Let $\bV$ be a local system over a curve $B,$ $W \subset \Gamma(A, \bV)$ a vector subspace and $\bW$ the local subsystem of $\bV$ generated by $W.$  Then a finite morphism of curves $u:B'\to B$ induces a isomorphism of local systems over $B'$
%                   			\begin{equation}\label{Mor-GenLocSysBaseChange}
%                   			\xymatrix@!R{
%                   				{u^{-1}\bW}  & {\sum_{g_i\in I_u}{\widehat{\bW}}_{g_i},}        &    
%                   				% vertical arrows
%                   				\ar@{<->}"1,1";"1,2"  
%                   				%%%
%                   				\hole
%                   			} 						
%                   			\end{equation}
%                   			where ${\widehat{\bW}}_{g_i}$ is the local subsystem generated by $u^*g_iW$ via $\rho_{\mtin W}^{-1},$ for  $g_i$ varying in a set $I_u\subset \pi_1(B,b)$ of generators of the quotient given by $u_*:\pi_1(B',b')\to \pi_1(B,b).$ \end{proposition}
%                   			\begin{proof} Consider the local system $\bW$ generated by $W,$ which is by definition the local system on $B$ of stalk $\pi_1(B,b)\cdot W$ and monodromy representation $\rho_{\mtin{W}}.$ The inverse image $u^{-1}\bW$ is a local system of the same stalk (i.e. $\pi_1(B,b)\cdot W$) and monodromy representation $\rho^{-1}_{\mtin{W}}$ given by the action of $\pi_1(B',b')$ via the composition $\rho\circ u_*,$  where $u(b')=b$ and  $u_*:\pi_1(B',b')\to \pi_1(B,b)$ is the natural homomorphism induced by $u.$ Consider the local system $\widehat{\bW}_{g},$ which is a local system on $B'$ of stalk generated by $u^*gW$ (i.e. $\pi_1(B',b')\cdot u^*gW).$ Then, since the monodromy action of $g\in \pi_1(B,b)$ sends $W$ to $gW,$ it is clear that the sum over a set of generators of the kokernel of $u_*$ reconstructs exactly $u^{-1}\bM.$ 
%                   				\end{proof}
%                   			The proof is just an easy computation we skip here. Using the previous description, we get informations on the finitness of the monodromy of $u^{-1}\bW$ just by looking at the monodromy of $\widehat{\bW}_{g_i},$ for each $g_i\in I_u.$ This is indeed possible according to the following fact.
%                   			
%                   			\begin{proposition}\label{Prop-SumLocSys} Let $\bV_1$ and $\bV_2$ be two local subsystems of the local system $\bV.$ If they  both have finite monodromy, then the local subsystem $\bV_1+ \bV_2$ of $\bV$ has finite monodromy too.
%                   				\end{proposition}
%                   				\begin{proof} Let $\rho_{\bV}$ be the monodromy representation of $\bV,$ $H$ the kernel of $\rho_{\bV}$ and $H_i$ the kernel of the subrepresentations induced by $\rho$ on $\bV_i,$ for $i=1,2.$  Then $H_{12}:=H_1\cap H_2$ is the kernel of the subrepresentation of $\bV_1+ \bV_2.$ We prove that $H_{12}$ has finite index in $\pi_1(B,b).$ By assumption, $\bV_1$ and $\bV_2$ have both finite monodromy, which means that $\pi_1(B,b)/H_i$ for $i=1,2$ are finite. Consider the chain of normal extensions $H_{12}\triangleleft H_1 \triangleleft  H.$ Then $H_1\triangleleft  H$ has finite index by assumption while $H_{12}\triangleleft H_1$ has finite index since there is a natural injective morphism $H_1/H_{12}\hookrightarrow \pi_1(B,b)/H_2$ and $\pi_1(B,b)/H_2$ is finite by assumption.  Thus $H_{12}\vartriangleleft H$ has finite index too.
%                   					\end{proof}
%                   			
%                   			{\em Proof of theorem \ref{Thm-MainG}.}
%                   			
                   				Let $\cU$ be the unitary factor in the second Fujita decomposition of $f,$ let $W\subset \Gamma(A, \bU)$ be a Massey-trivial subspace and let $\bM$ be the generated local subsystem of $\bU.$ We want to prove that $\bM$ has finite monodromy.
                   				
                   			 Following \cite{CD:Answer_2017}, we apply the semistable-reduction theorem to reduce to the semistable case.  
%                   			The strategy of the proof can be summarized as follows. First we apply the semistable-reduction theorem (see  \cite{CatDet_TheDirectImage_2014}) and reread the problem after the base change.
                   			Then the proof follows using Theorem \ref{Thm-MainSbis} together with some basic facts concerning the behaviour of the monodromy on local systems. 
                   			
                   		 Indeed, by applying the semistable-reduction theorem (see e.g.  \cite[Theorem 2.7 and Proposition 2.9]{CD:Answer_2017}), we get a base change $u:B'\to B$ given by a ramified finite morphism of curves and a resolution on the fiber product 
					\begin{equation}\label{Dia-BaseChangeSemistab}
					\xymatrix@!R{
						{S':=\widetilde{S\times_BB'}} \,        &     {S}              &    \\
						{B'}            \,        & {B}         &    
						% vertical arrows
						\ar"1,1";"2,1"^{f'}   \ar "1,1";"1,2"^>>>>>>{\varphi}
						%%
						\ar"1,2";"2,2"^{f} \ar "2,1";"2,2"^{u}
						% diagonal arrow, with 1 hole
						%	\ar@{-}"p1"|!{"2,3";"4,5"}
						\hole,
					}
					\end{equation}     
					 producing a semistable fibration $f':S'\to B'$ from a smooth compact surface  $S'$ to a smooth compact curve $B'.$  The base change induces the short exact sequence \eqref{SES-UnitaryFacBaseChange} of unitary flat vector bundles
					 		\begin{equation*}
					 		\xymatrix@!R{
					 			{0} &	{\cK_{\mtin{U}}}  & {\cU'}  & {u^*\cU}  &  {0,} 
					 			% orizzontal arrows
					 			%%%	\ar"1,i";"j,k"
					 			\ar"1,1";"1,2"\ar"1,2";"1,3"\ar"1,3";"1,4"\ar"1,4";"1,5"
					 			% vertical arrows
					 			\hole
					 		}
					 		\end{equation*}
					 		which is split and compatible with the underlying structure of local systems (see section \ref{SubSec-Prel-LocSystOnFibr}). 
					 		
%					  \textcolor{red}{TODO:ripercorrere??Indeed $u^*\cU\to \cG$ must be the null map and . Moreover $u^*\cU$ lies in the image of the unitary factor $U'$ of the second Fujita decomposition of $f'.$} 
%					 
					 Thus in particular $u^{-1}\bM$ is a local subsystem of the local system $\bU'$ underlying $\cU'.$ We prove that $u^{-1}\bM$ has finite monodromy, which is enough to prove that $\bM$ has finite monodromy (by Proposition \ref{Prop-LocSysAndBaseChange}). Let ${\widehat{\bW}}_{g_i}$ be the local system generated by $u^*g_iW$ via $\Ima \rho_W^{-1},$ for  $g_i\in I_u$ and 
					 $I_u\subset \pi_1(B,b)$ be a set of generators of the quotient given by $u_*:\pi_1(B',b')\to \pi_1(B,b).$ We apply Theorem \ref{Thm-MainSbis} to the local system generated by the maximal Massey-trivial subspace of $\bU$ containing $u^*g_iW,$ which remains Massey-trivial by pullback, and we get that this one has finite monodromy. By Proposition \ref{Prop-SubLocSystMon}, we get that maximality is not not an obstruction to the finiteness of the monodromy group. Thus we can conclude that  ${\widehat{\bW}}_{g_i}$ has finite monodromy. Finally, Propositions \ref{Prop-LocSysAndBaseChange} and \ref{Prop-SumLocSys} showthat  $u^{-1}\bM$ has finite monodromy.
					 
					 
%					                   		 
%                   The proof is a consequence of theorem \ref{Thm-MainS} and can be summarized into two steps.
%                   
%                   \begin{itemize}
%                   	\item[(1)] Reduction to the semistable case;
%                  	\item[(2)] Application of theorem \ref{Thm-MainG}.
%                   \end{itemize}		
%                   
%                   {\bf$(1)$ Reduction to the semistable case.} 
%                   
%                   {\bf$(2)$ Application of theorem \ref{Thm-MainG}}				
%                   
%                   
%                   
%                   
%                   
% 						
% 						\textcolor{red}{{\bf How to contruct fibration taking the product of two curves? }Not possible to use the Cesarea cycle $Z=C-C^{-}$ because the Massey product doesn't vanish.}
% 						\
% 						
 						
% 				In this section we give the proof of theorem \ref{Thm-MainA}. Almost all the tecnical results or generalizations of well-known results we need has been proven in previous sections: precisely the description o unitary flat sections with tubular forms and the Massey triviality with its connection with Castelnuovo de-Franchis theorem. We are going to use one more classical result in order to get the proof. So first we recall this one due originally to de Franchis, concernig the number of maps form a given smooth compact curve $C$ of genus $g\geq 2$ to another non-fixed one of genus grater than $2.$ The reader who is interested in  can see in \citealp{KanBounds1986} for details and some developments.
% 				
% 				\begin{theorem}[de Franchis \cite{DeFranchisUnTeo1913}]\label{Thm-deFranchis}
% 					There are only finitely many isomorphism classes of separable, non-constant morphisms $f: C\to C'$, where $C'$ runs over all curves of genus $g'>2.$
% 					\end{theorem}
% 					
% 			  Let $f: S\to B$ be a semistable fibration of genus $g\geq 2.$ We want to apply it to a family of maps $f_g:F\to C_g$ from a general fibre $F$ of  $f$ parametrized by the monodromy group $G$ of a trivial-Massey unitary flat bundle $M\subset U.$
% 			  
% 			  {\bf Construction of the family of maps $f_g:F\to C_g.$} Let $\rho: \pi_1(B^\circ,b_\circ)\to U(k, \bC)$ be the representation of $M$ as a unitary flat vector bundle, given as a subrepresentation of $U$ [definition \ref{Def-UFMTbundle}]. Denote with $H:=\ker \rho$ and with $G:=\pi_1(B^\circ,b_\circ)/H$ the quotient group which is naturally isomorphic to the monodromy group of $\bM.$ Consider the exact sequence of groups
% 			  \begin{equation}\label{Ses-compareBBpunctured}
% 			  0\to \pi_1(B,b_\circ)\to \pi_1(B^\circ, b_\circ)\to Q \to 0.
% 			  \end{equation}
% 			  Then by semiampleness it gives 
% 			  
% 			  \textcolor{red}{to do: diagrammino che chiarisce legame tra rappresentazioni su $B$ e su $B^\circ$: semiampiezza-vanishing cycles. Q: serve davvero?}
% 			  
% 			  {bf Claim. } $H$ is a subgroup of $\pi_1(B,b_\circ)\subset \pi_1(B^\circ,b_\circ).$
% 			  
% 			  \textcolor{red}{to do: if sopra, then ok. check in this way or boh. Q: if yes, then the covering is unramified over $B$? Explained in Cata-det (penso) }
% 			  
% 			  Consider the Galois covering  $\varphi_H: B_H\to B$ given by $B_H:=\tilde{B}/H,$ where $\tilde{B}$ is the universal cover of $B,$ as a quotient by $G$ and let $f_H: S_H\to B_H$ the fibration given by the base change and the rsolution of singularities, which is still semistable \textcolor{red}{check} of genus $g\geq 2.$ Let $f_{H*}\omega_{S_H/B_H}=U_H\oplus A_H$ the Hodge bundle of $f_H$ together with its second Fujita decomposition. As proved in [CIT] \textcolor{red}{forse check: follow Cata-det formula di Hurwitz sostanzialmente} $\phi_H^*U=U_H$ for any algebraic base change over a semistable fibration (recall that $\phi_H^*U_H\subset U_H$ is always true). Let $(s_1,\dots s_k)$ the $k-$upla of trivial-Massey constant sections around $b_\circ$ as required in definition \ref{Def-UFMTbundle} generating $\bM.$ By proposition \ref{Prop-FlatMasseyProd} we get a $k-$upla of tubular forms $(\omega_1,\dots,\omega_k)$ in $H^0(f^{-1}(V),\Omega^1_{S,d}),$ with $V$ a suitable neighborhood around $b_\circ$ where constant sections are defined, such that $\omega_i\wedge \omega_j=0,$ for any $i,j.$ Moreover, as a direct consequence of our choice of $H,$ we may continue analitically the lifting of $\omega_i$ to the full $S_H$ preserving the $\wedge$-relation, getting a $k-$pla $(\hat \omega_1,\dots\hat{\omega_k})$ of sections in $H^0(\Omega^1_{S,d}).$ 
% 			  By theorem \ref{Thm-CdeFranchisB} [Castelnuovo the Franchis for tubular surfaces], we get a non-constant map $h:S_H\to C'$ such that $\hat\omega_i\in h^*H^0(\omega_{C'}),$ which naturally restricts to the general fibre $F$ to $h:F\to C'.$ By acting with $g\in G$ on the $k-$upla $(s_1,\dots,s_k),$ we can repeat the previous procedure getting a family of $f_g:F\to C^g,$ with $g\in G.$
% 			  
% 			  {\bf De Franchis theorem to get the finitness of the monodromy group} Denote by $\cF$ the set of maps from the general fibre $F$ of $f$ to a curve of genus $g\geq 2,$ according with the assumption of theorem \ref{Thm-deFranchis}.  Consider the action
% 			  \begin{equation}
% 			  \rho_F:\pi_1(B, b_\circ)\to \Aut \cF, g\mapsto g\cdot h:=h_g,
% 			  \end{equation}
% 			 \textcolor{red}{NO! Va costruita..} where $h_g;F\to C^g$ is the map given by the previous construction. By De Franchis' theorem \ref{Thm-deFranchis}, $\Aut \cF$ is a finite set, then $\ker \rho_F$ is a subgroup of finite index. Now is enough to observe that $ \ker \rho \subset H,$ then $H$ is of finite index too and we get the proof.  
% 			   			  
 					 			\section{Applications}\label{Sec-Applications}
 					 			\subsection{Non-vanishing criterion for the Griffiths infinitesimal invariant on the canonical normal function and Catanese-Dettweiler fibrations}\label{SubSec-NonVanCriteria}
 					 			
 					 			In this section we apply a formula for the Griffiths infinitesimal invariant given in \cite{C-P_TheGriffiths_1995} in relation to the unitary flat factor $\cU$ of the second Fujita decomposition of a fibration $f:S\to B$ of genus $g(F)\geq 2.$ The formula together with Theorem \ref{Thm-MainSbis} provides a non-vanishing criterion for the Griffiths infinitesimal invariant of the canonical normal function in terms of the monodromy of $\cU.$ This is the content of Corollary \ref{Cor-InfMon&NonVanGriff}.
 					 			
 					 				We briefly recall the construction of the objects involved and refer to \cite{V_HodgeTheoryII_2003}, \cite{Green_InfinitesimalMathods_1994}, \cite{Grif_InfinitesimalVariationsIII_1983} \cite{Vois_UneRemark_1988} for details and formal definitions. Let $f:S\to B$ be a complete fibration with general fiber $F.$ Let $j:B^0\hookrightarrow B$ be the natural inclusion and consider the restriction of $f$ to $B^0,$ which is a smooth fibration $f^0: S^0\to B^0.$ Then we can associate to such a family the {\em  Jacobian fibration} $j(f^0):{\cJ}(f)\to B^0$ of general fiber the Jacobian $J(F).$ This is a fibration of $g-$dimensional polarized abelian varieties over the smooth curve $B^0$ and has a variation of the Hodge structure. The $(g-1)-$Griffiths intermediate Jacobian of $J(F)$ is defined by the  Hodge structure $\{H^{2g-3}_{\bZ}=H^{2g-3}(J(F),\bZ),\,F^pH^{2g-3}(J(F),\bC), p\geq 0\}$ of $J(F)$ as
 					 				\begin{equation}
 					 				J^{g-1}(J(F))=H^{2g-3}(J(F),\bC)/(F^{g-1}H^{2g-3}(J(F),\bC)\oplus H^{2g-3}_{\bZ})\simeq F^2H^3(J(F),\bC)^*/H_3(J(C),\bZ),
 					 				\end{equation} 
 					 				where $F^pH^{2g-3}(J(F),\bC), p\geq 0$ is the Hodge filtration (see \cite{V_HodgeTheoryI_2002} for details).
 					 				The intermediate Jacobians fit into a fibration $j^{g-1}(f):\cJ^{g-1}(f^0)\to B^0$ of complex tori over the smooth curve $B^0.$
 					 				The construction is compatible with the decomposition given by the Lefschetz operator in terms of primitive cohomology and we get the {\em intermediate Primitive Jacobian family }
$p(f^0):\cP(f)\to B^0,$ defined on the general fibre $F$ of $f$ as 
\begin{equation}
P(F)=F^2P^3(J(F),\bC)^*/H_3(J(C),\bZ)_{prim}, 
\end{equation} 
where $H_3(J(F),\bZ)_{prim}$ is the image of $H_3(J(F),\bZ)$ in $F^2P^3(J(F),\bC)^*.$ Integration over the group of one dimensional algebraic cycles $Z^{g-1}(J(F))_{\mtin{hom}}$ homologically equivalent to zero in $J^{g-1}(F)=F^2H^3(J(F),\bC)^*/H_3(J(C),\bZ)$ provides a higher Abel-Jacobi map, which is independent on the choice of the base point of the projection $J^{g-1}(F)\to P(F)$ over $P(F).$ The {\em Ceresa cycle} of the general fibre $F$ is defined as the one cycle given by the image of $F-F^{-}$ in $J(F)$ via the Abel Jacobi map.
 The {\em canonical normal function} is defined to be the section $\nu: B^0\to \cP(f)$ associating to each $b\in B^0$ the image of the Ceresa cycle $[F_b-F_b^{-}]\in Z^{g-1}(J(F_b))_{\mtin{hom}}$ via the higher Abel-Jacobi map. The {\em Griffiths infinitesimal invariant }$\delta(\nu)$ of the canonical normal function contains information on flatness of local liftings of those normal functions. We refer to \cite{C-P_TheGriffiths_1995} for the explicit definition. The link with our result is given by the fact that this infinitesimal invariant induces over a point $b\in B^0$ a linear map $\ker (\gamma) \to \bC ,$ where $\gamma : T_{B,b}\otimes P^{2,1}J(F_b)\to P^{1,2}J(F_b)$ is naturally defined by the IVHS on the primitive cohomological groups $P^{1,2}J(F_b)$ and $P^{1,2}J(F_b)$ in $H^3(J(F_b),\bC).$ The infinitesimal invariant has been computed in \cite{C-P_TheGriffiths_1995},  for some special elements. Let $\{H^1(F_b,\bZ), H^{1,0}=H^0(\omega_{F_b}), Q)\}$ be the geometric Hodge structure on $F_b,$ polarized by the intersection form $Q(-,-)=\frac{i}{2}\int_{F_b}-\wedge-,$ which as usual induces a hermitian form on $H^0(\omega_{F_b})$ using the conjugation $H^{0,1}(F_b)\simeq \overline{H^{1,0}(F_b)}.$  %\ref{SubSec-Prel-LocSystOnFibr}).
 Let $\xi_b\in H^1(T_{F_b})$ be the Kodaira-Spencer class of $F_b.$ We have the following. 
 					 				\begin{lemma}(\cite{C-P_TheGriffiths_1995})\label{Lem-GrifFor} Let $\omega_1,\omega_2,\sigma\in H^0(\omega_{F_b})$ be such that $\xi_b\cdot\omega_1=\xi_b\cdot\omega_2=0$ and $Q(\omega_1,\bar \sigma)=Q(\omega_2,\bar \sigma)=0.$ Then $\omega_1\wedge\omega_2\wedge \bar \sigma$ lies, up to a natural isomorphism, in $\ker \gamma$ and we have
 					 					\begin{equation} \label{For-CompGII}
 					 					\delta(\nu)(\xi_b\otimes \omega_1\wedge\omega_2\wedge \bar \sigma)= -2Q (\fm_{\xi_b}(\omega_1,\omega_2), \bar \sigma).
 					 					\end{equation} 
 					 					\end{lemma}
 					 					
 					 					As an application of Theorem \ref{Thm-MainG}, we get the following. 
 					 					\begin{corollary}\label{Cor-InfMon&NonVanGriff}
 					 						Let $f:S\to B$ be a fibration of genus $g(F)\geq 2$ and $\cU$ be the unitary factor in the second Fujita decomposition of $f.$ If the monodromy of $\cU$ is not finite, then the Griffiths infinitesimal invariant on the canonical normal function  $\nu: B^0\to \cP(f)$ is not zero at the general point $b\in B^0.$  In particular, $\nu$  is not a torsion section.
 					 					\end{corollary}
 					 					\begin{proof}
% 					 									
% 					 				We read the previous construction inside the unitary factor $\cU$ of the second Fujita decomposition of $f.$ It is enough to observe that up to a finite base change given by the semistable reduction Theorem, we can reduct to the study of semistable fibrations where the local system attached to the unitary factor involved in the second Fujita decomposition of $f$ is completely determined by the geometric variation of the Hodge structure (as recalled in the prelimiaries and more precisely in subsection \ref{Sec-LocSyst}). 
 					 				%is completely determined by the geometric variation of the Hodge structure (details in  \ref{Sec-LocSyst}) and moreover $\cU\hookrightarrow \ker \partial $ as seen in \ref{SubSec-UandTubForms}. 
								We apply the previous formula to sections of $j^\ast\cU\subset j^\ast\cK_\partial.$  Since the monodromy of $\cU$ is not finite, then by Theorem \ref{Thm-MainSbis} is not Massey-trivial generated and we can find a pair $(\omega_1,\omega_2)\subset H^0(\omega_{F_b})$ of independent element such that $\fm_{\xi_b}(\omega_1,\omega_2)\neq0.$ Applying the formula \ref{For-CompGII} to $\omega_1,\omega_2\in H^0(\omega_{F_b})$ (which are such that $\xi_b\cdot\omega_1=\xi_b\cdot\omega_2=0$) and $\sigma=\fm_{\xi_b}(\omega_1,\omega_2)\in H^0(\omega_{F_b})$ 
 					 			 we get 
 					 					\begin{equation}
 					 					\delta(\nu)(\xi_b\otimes \omega_1\wedge\omega_2\wedge \bar \sigma)= -2Q (\fm_{\xi_b}(\omega_1,\omega_2), \bar \fm_{\xi_b}(\omega_1,\omega_2))<0
 					 					\end{equation} 
 					 					This concludes the proof, since the fact that the normal function is non-torsion when the Griffiths infinitesimal invariant in not zero has been proven in \cite{GriffithsHarris_Infinitesimal||i_1983}, \cite{Green_1989},\cite{Vois_UneRemark_1988}.
 					 				\end{proof}
 					 				
 					 				In particular, the previous result applies to the examples provided in \cite{CatDet_TheDirectImage_2014}, in \cite{CD:Answer_2017} and also in \cite{CatDet_Vector_2016}, concerning the construction of fibrations where the monodromy of $\cU$ is not finite. More precisely, one can state the following.
 					 				\begin{corollary}\label{Cor-CataDetNFnotTor}
 					 					Let $f:S\to B$ be a fibration as those constructed in \cite{CatDet_TheDirectImage_2014}, \cite{CD:Answer_2017} and \cite{CatDet_Vector_2016} with $\cU$ of not finite monodromy. Then the canonical normal function is not torsion.
 					 					\end{corollary}
 					 				
 					 				
% 					 				 \textcolor{red}{TODO: ora $A$ è la $U$ dell'aticolo Pirola-Collino e vorrei riportare velocissimamente definitione di famiglia di Jacobiane associate, famiglia di jacobiane intermedie primitive, funzione normale come sezione, definizione dell'invariante infinitesimale di Griffiths e formula che lega ai prodotti. E' troppo?? Consiglio.}
% 					 				
 					 			\subsection{Semiampleness criterion for the Hodge bundle and hyperelliptic fibrations}\label{SubSec-SemiamplenessCriteria}
 					 			In this section we state a criterion for the semiampleness of $f_*\omega_{S/B},$ where $f:S\to B$ is a fibration of genus $g(F)\geq 2$ in terms of Massey-trivial generated bundles, which is a corollary of Theorem \ref{rem-MTaction}, together with a characterization for semiampleness on unitary flat bundles ( see e.g \cite[Theorem 2.5]{CD:Answer_2017}). Then we show that hyperelliptic fibrations naturally satisfy the condition. 
 					 			
 					 			Let $f:S\to B$ be a projective fibration of genus $g(F)\geq 2$ and $f_*\omega_{S/B}$ be the Hodge bundle of $f$. According to the second Fujita decomposition, we have a splitting  $f_*\omega_{S/B}= \cU\oplus \cA$ as a direct sum of a unitary flat bundle $\cU$ and an ample bundle $\cA.$ Since $\cA$ is semiample, the semiampleness of $f_*\omega_{S/B}$ depends only on $\cU.$ We recall the following characterization of semiampleness of unitary flat bundles, referring to \cite{CD:Answer_2017} for a complete proof.
 					 			\begin{proposition} A unitary flat bundle $\cV$ over a smooth compact curve $B$ is semiample if and only if it has finite monodromy. 
 					 				\end{proposition}
 					 			As a consequence, applying Theorem \ref{Thm-MainG} we get the following criterion.
 					 			\begin{corollary}\label{Cor-SemiamplenessCriteria} Let $f:S\to B$ be a projective semistable fibration of genus $g(F)\geq 2$ and $\cU$ be the unitary factor in the second Fujita decomposition of $f.$ If $\cU$ is Massey-trivial generated, then $f_*\omega_{S/B}$ is semiample.
 					 			\end{corollary}
 					 			Now we analyze the case of hyperelliptic fibrations, where it turns out that the unitary factor $\cU$ is more than Massey-trivial generated. Indeed, the hyperelliptic involution forces it to be Massey-trivial. We recall that a fibration $f:S\to B$ is hyperelliptic of genus $g(F)\geq 2$ if the general fibre $F$ of $f$ is hyperelliptic curve of genus $g(F)\geq 2$ and we denote by $\sigma: F\to F$ the hyperelliptic involution of $F.$  
 					 			\begin{proposition}\label{Prop-HyperellipticApplication} Let $f:S\to B$ be a hyperelliptic fibration of genus $g(F)\geq 2.$ Then $f_*\omega_{S/B}$ is semiample. 
 					 				\end{proposition}
 					 				\begin{proof} We prove that $\cU$ is Massey-trivial generated whenever $f$ is hyperelliptic. Then the proof follows immediately applying Criterion \ref{Cor-SemiamplenessCriteria}. Let $F$ be the general fibre of $f$ and $\xi\in H^1(T_F)$ the extension of $F.$ Consider $s_1,s_2\in U\subset H^0(\omega_F)$ two independent vectors in the fibre $U$ of $\cU.$ Observe that since $f$ is hyperelliptic, each $s\in U$ lies in $K_\xi$ and we can compute the Massey product of the pair $(s_1,s_2).$ By formula \eqref{Mor-Mp/Aj}, $m_{\xi}(s_1,s_2 )$ is antisymmetric in $s_1,s_2.$ Applying the hyperelliptic involution, which acts on $H^0(\omega_F)$ by pullback  $\sigma^*:H^0(\omega_F)\to H^0(\omega_F)$ as the $-1$ multiplication map, 
% 					 					while as the identity map on $H^1(T_F),$ 
									we get $\sigma^*m_{\xi}(s_1,s_2 )=-m_{\xi}(s_1,s_2 ).$ On the other hand, $\sigma^*m_{\xi}(s_1,s_2 )=m_{\xi}(-s_1,-s_2 )=m_{\xi}(-s_1,-s_2 )$ and thus by antisymmetry it must be zero.     
 					 					\end{proof}
 					 					We remark that the same conclusion has been proven in \cite{LuZuo_OnTheSlope_2017} using different techniques.
 					 			
	
	\begin{thebibliography}{CEZGT14}
		
		\bibitem[AP90]{AlzatiPirola_Some_1991}
		A.~Alzati and G.~P. Pirola.
		\newblock Some remarks on the de {F}ranchis theorem.
		\newblock {\em Ann. Univ. Ferrara Sez. VII (N.S.)}, 36:45--52 (1991), 1990.
		
		\bibitem[Bar00]{barja-fujita}
		Miguel~\`Angel Barja.
		\newblock On a conjecture of {F}ujita.
		\newblock {\em Available on the ResearchGate page of the author}, 2000.
		
		\bibitem[BNP07]{B-N-P_OnTheTopological_2007}
		M.~A. Barja, J.~C. Naranjo, and G.~P. Pirola.
		\newblock On the topological index of irregular surfaces.
		\newblock {\em J. Algebraic Geom.}, 16(3):435--458, 2007.
		
		\bibitem[Cat91]{Cat_Moduli_1991}
		Fabrizio Catanese.
		\newblock Moduli and classification of irregular {K}aehler manifolds (and
		algebraic varieties) with {A}lbanese general type fibrations.
		\newblock {\em Invent. Math.}, 104(2):263--289, 1991.
		
		\bibitem[CD]{CD:Answer_2017}
		Fabrizio Catanese and Michael Dettweiler.
		\newblock Answer to a question by {F}ujita on {V}ariation of {H}odge
		{S}tructures.
		\newblock {\em Adv. Stud. in Pure Math.}, 74-04.
		
		\bibitem[CD14]{CatDet_TheDirectImage_2014}
		Fabrizio Catanese and Michael Dettweiler.
		\newblock The direct image of the relative dualizing sheaf needs not be
		semiample.
		\newblock {\em C. R. Math. Acad. Sci. Paris}, 352(3):241--244, 2014.
		
		\bibitem[CD16]{CatDet_Vector_2016}
		Fabrizio Catanese and Michael Dettweiler.
		\newblock Vector bundles on curves coming from variation of {H}odge structures.
		\newblock {\em Internat. J. Math.}, 27(7):1640001, 25, 2016.
		
		\bibitem[CEZGT14]{CatElZFouGrif_Hodge_2014}
		Eduardo Cattani, Fouad El~Zein, Phillip~A. Griffiths, and L{\^e}~D{\~u}ng
		Tr{\'a}ng, editors.
		\newblock {\em Hodge theory}, volume~49 of {\em Mathematical Notes}.
		\newblock Princeton University Press, Princeton, NJ, 2014.
		
		\bibitem[CLZ16]{ChenLuZu_OnTheOort_2016}
		Ke~Chen, Xin Lu, and Kang Zuo.
		\newblock On the {O}ort conjecture for {S}himura varieties of unitary and
		orthogonal types.
		\newblock {\em Compos. Math.}, 152(5):889--917, 2016.
		
		\bibitem[CP95]{C-P_TheGriffiths_1995}
		Alberto Collino and Gian~Pietro Pirola.
		\newblock The {G}riffiths infinitesimal invariant for a curve in its
		{J}acobian.
		\newblock {\em Duke Math. J.}, 78(1):59--88, 1995.
		
		\bibitem[Del71]{Del_Theorie_1971}
		Pierre Deligne.
		\newblock Th\'eorie de {H}odge. {II}.
		\newblock {\em Inst. Hautes \'Etudes Sci. Publ. Math.}, (40):5--57, 1971.
		
		\bibitem[FGP17]{FGP}
		Paola Frediani, Alessandro Ghigi, and Gian~Pietro Pirola.
		\newblock Fujita decomposition and hodge loci.
		\newblock {\em In preparation}, 2017.
		
		\bibitem[Fuj78a]{Fuj78a}
		Takao Fujita.
		\newblock On {K}\"ahler fiber spaces over curves.
		\newblock {\em J. Math. Soc. Japan}, 30(4):779--794, 1978.
		
		\bibitem[Fuj78b]{Fuj78b}
		Takao Fujita.
		\newblock The sheaf of relative canonical forms of a {K}\"ahler fiber space
		over a curve.
		\newblock {\em Proc. Japan Acad. Ser. A Math. Sci.}, 54(7):183--184, 1978.
		
		\bibitem[GA13]{Gonz_PhdTs_2013}
		{V}. {G}onz{\'a}lez {A}lonso.
		\newblock {\em Hodge Numers of Irregular varieties and fibrations}.
		\newblock PhD thesis, Universitat Polit{\`e}cnica de Catalunya, 2013.
		\newblock PhD Thesis.
		
		\bibitem[GA16]{Gonz_OnDef_2016}
		V{\'{\i}}ctor Gonz{\'a}lez-Alonso.
		\newblock On deformations of curves supported on rigid divisors.
		\newblock {\em Ann. Mat. Pura Appl. (4)}, 195(1):111--132, 2016.
		
		\bibitem[GH83a]{GriffithsHarris_Infinitesimal||_1983}
		Phillip Griffiths and Joe Harris.
		\newblock Infinitesimal variations of {H}odge structure. {II} an infinitesimal
		invariant of hodge classes.
		\newblock {\em Compositio Math.}, 50(2-3):207--265, 1983.
		
		\bibitem[GH83b]{GriffithsHarris_Infinitesimal||i_1983}
		Phillip Griffiths and Joe Harris.
		\newblock Infinitesimal variations of {H}odge structure. {II|} determinantal
		varieties and the infinitesimal invariant of normal functions.
		\newblock {\em Compositio Math.}, 50(2-3):267--324, 1983.
		
		\bibitem[Gre89]{Green_1989}
		Mark~L. Green.
		\newblock Griffiths' infinitesimal invariant and the {A}bel-{J}acobi map.
		\newblock {\em J. Differential Geom.}, 29(3):545--555, 1989.
		
		\bibitem[Gri70]{Grif_PeriodsIII_1970}
		Phillip~A. Griffiths.
		\newblock Periods of integrals on algebraic manifolds. {III}. {S}ome global
		differential-geometric properties of the period mapping.
		\newblock {\em Inst. Hautes \'Etudes Sci. Publ. Math.}, (38):125--180, 1970.
		
		\bibitem[Gri83]{Grif_InfinitesimalVariationsIII_1983}
		Phillip~A. Griffiths.
		\newblock Infinitesimal variations of {H}odge structure. {III}. {D}eterminantal
		varieties and the infinitesimal invariant of normal functions.
		\newblock {\em Compositio Math.}, 50(2-3):267--324, 1983.
		
		\bibitem[Gri84]{GrifTopics1984}
		Phillip Griffiths, editor.
		\newblock {\em Topics in transcendental algebraic geometry}, volume 106 of {\em
			Annals of Mathematics Studies}. Princeton University Press, Princeton, NJ,
		1984.
		
		\bibitem[GST17]{GonStopTor-On}
		V.~{Gonz{\'a}lez-Alonso}, L.~{Stoppino}, and S.~{Torelli}.
		\newblock {On the rank of the flat unitary factor of the Hodge bundle}.
		\newblock {\em ArXiv e-prints}, September 2017.
		
		\bibitem[Kob87]{Kob_Differential_1987}
		Shoshichi Kobayashi.
		\newblock {\em Differential geometry of complex vector bundles}, volume~15 of
		{\em Publications of the Mathematical Society of Japan}.
		\newblock Princeton University Press, Princeton, NJ; Princeton University
		Press, Princeton, NJ, 1987.
		\newblock Kan\^o Memorial Lectures, 5.
		
		\bibitem[LZ17]{LuZuo_OnTheSlope_2017}
		Xin Lu and Kang Zuo.
		\newblock On the slope of hyperelliptic fibrations with positive relative
		irregularity.
		\newblock {\em Trans. Amer. Math. Soc.}, 369(2):909--934, 2017.
		
		\bibitem[Mar88]{Martens_Obervations_1988}
		Henrik~H. Martens.
		\newblock Observations on morphisms of closed {R}iemann surfaces. {II}.
		\newblock {\em Bull. London Math. Soc.}, 20(3):253--254, 1988.
		
		\bibitem[NPZ04]{NarPirZuc_Poly_2004}
		J.~C. Naranjo, G.~P. Pirola, and F.~Zucconi.
		\newblock Polygonal cycles in higher {C}how groups of {J}acobians.
		\newblock {\em Ann. Mat. Pura Appl. (4)}, 183(3):387--399, 2004.
		
		\bibitem[NS65]{NarSes_Stable_1965}
		M.~S. Narasimhan and C.~S. Seshadri.
		\newblock Stable and unitary vector bundles on a compact {R}iemann surface.
		\newblock {\em Ann. of Math. (2)}, 82:540--567, 1965.
		
		\bibitem[PS08]{PetSteen_Mixed_2008}
		Chris A.~M. Peters and Joseph H.~M. Steenbrink.
		\newblock {\em Mixed {H}odge structures}, volume~52 of {\em Ergebnisse der
			Mathematik und ihrer Grenzgebiete. 3. Folge. A Series of Modern Surveys in
			Mathematics [Results in Mathematics and Related Areas. 3rd Series. A Series
			of Modern Surveys in Mathematics]}.
		\newblock Springer-Verlag, Berlin, 2008.
		
		\bibitem[PZ03]{P-Z_Variations_2003}
		Gian~Pietro Pirola and Francesco Zucconi.
		\newblock Variations of the {A}lbanese morphisms.
		\newblock {\em J. Algebraic Geom.}, 12(3):535--572, 2003.
		
		\bibitem[Riz08]{R_Infinitesimal_2008}
		Cecilia Rizzi.
		\newblock Infinitesimal invariant and {M}assey products.
		\newblock {\em Manuscripta Math.}, 127(2):235--248, 2008.
		
		\bibitem[RZ17]{RizZuc_Generalized_2017}
		Luca Rizzi and Francesco Zucconi.
		\newblock Generalized adjoint forms on algebraic varieties.
		\newblock {\em Ann. Mat. Pura Appl. (4)}, 196(3):819--836, 2017.
		
		\bibitem[Voi88]{Vois_UneRemark_1988}
		Claire Voisin.
		\newblock Une remarque sur l'invariant infinit\'esimal des fonctions normales.
		\newblock {\em C. R. Acad. Sci. Paris S\'er. I Math.}, 307(4):157--160, 1988.
		
		\bibitem[Voi02]{V_HodgeTheoryI_2002}
		Claire Voisin.
		\newblock {\em Hodge theory and complex algebraic geometry. {I}}, volume~76 of
		{\em Cambridge Studies in Advanced Mathematics}.
		\newblock Cambridge University Press, Cambridge, 2002.
		\newblock Translated from the French original by Leila Schneps.
		
		\bibitem[Voi03]{V_HodgeTheoryII_2003}
		Claire Voisin.
		\newblock {\em Hodge theory and complex algebraic geometry. {II}}, volume~77 of
		{\em Cambridge Studies in Advanced Mathematics}.
		\newblock Cambridge University Press, Cambridge, 2003.
		\newblock Translated from the French by Leila Schneps.
		
	\end{thebibliography}
%	
%	\bibliographystyle{alpha}
%	\bibliography{mybibMPFJFinal}
%	
	

\end{document}
